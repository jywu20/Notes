\documentclass[a4paper, oneside, scheme=plain, 12pt]{article}


\usepackage[T1]{fontenc}
\usepackage{libertinus}
\usepackage{xeCJK}
\usepackage{xpinyin}
\setCJKmainfont{SimSun}
\usepackage{geometry}
\usepackage{float}
\usepackage{titling}
\usepackage{titlesec}
\usepackage{paralist}
\usepackage{footnote}
\usepackage{enumerate}
\usepackage{amsmath, amsthm}
\usepackage{gb4e}
\noautomath
\usepackage{bbm}
\usepackage{textcomp}
\usepackage{soul}
\usepackage{graphicx}
\usepackage{siunitx}
\usepackage[table,xcdraw,svgnames]{xcolor}
\usepackage{tikz}
\usepackage[ruled, vlined, linesnumbered, noend]{algorithm2e}
\usepackage{xr-hyper}
\usepackage[colorlinks, citecolor = purple]{hyperref} % linkcolor=black, anchorcolor=black, citecolor=black, filecolor=black
\usepackage[most]{tcolorbox}
\usepackage{caption}
\usepackage{subcaption}
\usepackage{booktabs}
\usepackage{multirow}
\usepackage[figuresright]{rotating}
\usepackage{acro}
\usepackage[citestyle=authoryear,backend=bibtex,natbib=true,doi=false,isbn=false,url=false]{biblatex}
\addbibresource{famous-grammars.bib}
\addbibresource{controversy.bib}
\addbibresource{typology.bib}
\addbibresource{historical.bib}
\usepackage{prettyref}

\geometry{left=3.18cm,right=3.18cm,top=2.54cm,bottom=2.54cm}
\titlespacing{\paragraph}{0pt}{1pt}{10pt}[20pt]
\setlength{\droptitle}{-5em}

\DeclareMathOperator{\timeorder}{\mathcal{T}}
\DeclareMathOperator{\diag}{diag}
\DeclareMathOperator{\legpoly}{P}
\DeclareMathOperator{\primevalue}{P}
\DeclareMathOperator{\sgn}{sgn}
\newcommand*{\ii}{\mathrm{i}}
\newcommand*{\ee}{\mathrm{e}}
\newcommand*{\const}{\mathrm{const}}
\newcommand*{\suchthat}{\quad \text{s.t.} \quad}
\newcommand*{\argmin}{\arg\min}
\newcommand*{\argmax}{\arg\max}
\newcommand*{\normalorder}[1]{: #1 :}
\newcommand*{\pair}[1]{\langle #1 \rangle}
\newcommand*{\fd}[1]{\mathcal{D} #1}
\newcommand*{\textto}{$\to$}
\newcommand{\focus}[1]{\textbf{#1}}

\newcommand*{\citesec}[1]{\S~{#1}}
\newcommand*{\citechap}[1]{Ch.~{#1}}
\newcommand*{\citechaps}[1]{Chs.~{#1}}
\newcommand*{\citefig}[1]{Fig.~{#1}}
\newcommand*{\citetable}[1]{Table~{#1}}
\newcommand*{\citepage}[1]{p.~{#1}}
\newcommand*{\citepages}[1]{pp.~{#1}}
\newcommand*{\citefootnote}[1]{fn.~{#1}}

\newrefformat{sec}{\citesec{\ref{#1}}}
\newrefformat{fig}{\citefig{\ref{#1}}}
\newrefformat{tbl}{\citetable{\ref{#1}}}
\newrefformat{chap}{\citechap{\ref{#1}}}
\newrefformat{fn}{\citefootnote{\ref{#1}}}
\newrefformat{box}{Box~\ref{#1}}
\newrefformat{ex}{\ref{#1}}

% color boxes

\tcbuselibrary{skins, breakable, theorems}

\newtcbtheorem{infobox}{Box}{
    enhanced,
    boxrule=0pt,
    %colback=blue!5,
    %colframe=blue!5,
    colback=white,
    colframe=white,
    coltitle=blue!60,
    borderline west={4pt}{0pt}{blue!65},
    sharp corners,
    fonttitle=\bfseries, 
    breakable,
    before upper={\parindent15pt\noindent}}{box}
\definecolor{my-orange}{HTML}{F58123}
\newtcbtheorem[use counter from=infobox]{theorybox}{Box}{
    enhanced,
    boxrule=0pt,
    %colback=orange!5, 
    %colframe=orange!5, 
    colback=white,
    colframe=white,
    coltitle=my-orange!65,
    borderline west={4pt}{0pt}{my-orange!65},
    sharp corners,
    fonttitle=\bfseries, 
    breakable,
    before upper={\parindent15pt\noindent}}{box}
\newtcbtheorem[use counter from=infobox]{todobox}{Box}{
    enhanced,
    boxrule=0pt,
    colback=red!5,
    colframe=red!5,
    coltitle=red!50,
    borderline west={4pt}{0pt}{red!65},
    sharp corners,
    fonttitle=\bfseries, 
    breakable,
    before upper={\parindent15pt\noindent}}{box}
\newtcbtheorem[use counter from=infobox]{perspectivebox}{Box}{
    enhanced,
    boxrule=0pt,
    %colback=red!5,
    %colframe=red!5,
    colback=white,
    colframe=white,
    coltitle=red!50,
    borderline west={4pt}{0pt}{red!65},
    sharp corners,
    fonttitle=\bfseries, 
    breakable,
    before upper={\parindent15pt\noindent}}{box}


\AtBeginEnvironment{infobox}{\small}
\AtBeginEnvironment{todobox}{\small}
\AtBeginEnvironment{theorybox}{\small}

\newcommand*{\concept}[1]{\textbf{#1}}
\newcommand*{\term}[1]{\emph{#1}}
\newcommand{\form}[1]{\emph{#1}}
\newcommand{\work}[1]{\textit{#1}}
\newcommand{\species}[1]{\textit{#1}}

\newcommand{\redp}{\textasciitilde}

% To be used in \form{}, to label the function of the constituents.
\newcommand{\annotate}[1]{\text{\emph{#1}}}

\newcommand{\asis}[1]{\textsc{#1}}
\newcommand{\oneof}[1]{{#1}}
\newcommand*{\homo}[2]{#1$_{\text{#2}}$}

\newcommand*{\textgt}{$>$ }
\newcommand*{\textlt}{$<$ }

\newcommand{\ala}{à la}
\newcommand{\translate}[1]{`#1'}
\newcommand{\vP}{\textit{v}P}
\newcommand*{\category}[1]{\textsc{#1}}
\newcommand*{\specialunit}[1]{$<$\textit{#1}$>$}
\newcommand{\before}{$> \ $}

% Make subsubsection labeled
\setcounter{secnumdepth}{4}
\setcounter{tocdepth}{4}
% reset example counter every chapter (but do not include the chapter number to the label)
%\counterwithin{exx}{chapter} 

% Reference formats
\renewcommand*{\nameyeardelim}{\space} % No comma between year and name
\DeclareNameAlias{sortname}{family-given} % Putting the family name before the given name
\DeclareNameAlias{default}{family-given} 
\DeclareFieldFormat{labelnumberwidth}{} % No number label like [12] in the reference list
\setlength{\biblabelsep}{0pt} % No space for these labels

\makeindex

\title{What to compare in historical linguistics?}
\author{Jinyuan Wu}

\begin{document}

\automath

\maketitle

\section{The (lack of) synchronic foundation for diachronic studies}\label{sec:intro}

The Neogrammarian hypothesis states that language changes can be explained \emph{completely} by 
(a) regular sound change without exceptions,
(b) analogy, and (c) borrowing.
We can then use the \concept{comparative method} and \concept{internal reconstruction}
to identify cognates and layers of borrowed words.

So far we are just repeating words you can find on standard historical linguistics books.
There however is a usually unspoken caveat:
what is the unit that the comparative method runs on?
A historical linguist will immediately answer ``the word''.
But what is a word, then? And \emph{why} don't we try to determine genetic relations based on syntactic patterns, but words, whatever the term means?

A choice in methodology eventually reflects a certain underlying assumption on how things work.
Choosing to apply the comparative method and internal reconstruction to ``the word''
means that we believe that when a language is passed to the younger generation,
what are actually passed are sequences with relatively stable internal structures,
which we name \term{words}.
Now, we have to be able to identify what \emph{historical} wordhood means \emph{synchronically},
or otherwise in theory we will be unable to gather enough materials for diachronic studies.

Thus historical linguistics should ideally have a synchronic, and ultimately psycholinguistic foundation.
Ideally, the Neogrammarian hypothesis should be explained by acquisition of phonology,
and its (alleged) breakdown in dialectical continua should be explained by
e.g. the psycholinguistics of how two mutually intelligible languages are perceived in the brain.
Methodological disputes in historical linguistics should eventually be resolved \emph{experimentally},
by testing their implicit assumptions on how languages are transmitted from one generation to another.
Given the current status of theoretical linguistics and psycholinguistics,
however, we should not expect to see this in the foreseeable future.

Still what is a word in historical linguistics is too important
to be left to future biologists who will literally peak into your brain to see how language works.
It is fundamental to the everyday job of historical linguists.


\section{How grammar works}\label{sec:grammar}

Let's forget about history and focus on synchronic concepts for a while here.
We first go over modern theories of syntax (\prettyref{sec:abstract-syntax}),
and point out that syntactic structures provide no definite definition for wordhood
(\prettyref{sec:no-word-in-syntax}).
We then turn to the linearization of abstract syntax,
as well as the structure of the lexicon,
and define morphological wordhood.
Finally, we turn to phonological wordhood,
and emphasize that phonological wordhood may have subtle differences with morphological wordhood.

\subsection{Abstract syntax}\label{sec:abstract-syntax}

We aim to provide \emph{a} theory of abstract (``narrow'') syntax of language
and demonstrate that it is possible to do syntactic analysis without mentioning \term{words}.
Note that this is only half of the study:
we still need to put various abstract function items (affixes or clitics or particles)
on roots and get everything phonologically realized,
which is discussed in \prettyref{sec:vocabulary-inesrtion}
and does give us more solid definitions of wordhood.

\subsubsection{Peeling off morphophonology and focusing on abstract syntax}

If you are convinced by Distributed Morphology or theories along this line of thinking,%
\footnote{
    Note that Distributed Morphology can be formulated into a form quite close to Word-and-Paradigm theories of morphology \citep{ermolaeva2018distributed}.
}
you will know that it seems we cannot define wordhood in a completely intuitive way in \emph{abstract} or \emph{pure} syntax.%
\footnote{
    Lexicalists will push back -- but I believe they are wrong \citep{bruening2018lexicalist}.
}
Let me explain.

Consider the example \form{the two ugly blackbirds}.
Should we bracket the noun phrase as \form{the [two [ugly [blackbirds]]]}?
Not necessarily. The category of plural number, marked by \form{-s}, seems to have a scope covering 
at least the nominal \form{two ugly blackbirds}.
This can first be seen from semantic interpretation:
\form{blackbird} is a compound that denotes a certain type of birds,
and \form{ugly blackbird} is a conjunction of being ugly and being a blackbird.
Now \form{two ugly blackbirds} specifies a set of two ugly blackbirds,
and finally, \form{the two ugly blackbirds} reminds the listener to 
recall an aforementioned or at least identifiable set of two ugly blackbirds.
If we assume that the clearly hierarchical semantics has a structural origin,
then we should assume that the category of number is somehow higher than adjectival modification.
This head noun-adjective-number-determiner hierarchy can be found cross-linguistically.
In Japhug, for instance, the number marker follows coordinated head nouns and also the numeral,
highlighting its scope over the whole noun phrase
\citep[\citepage{368},(2-3)]{jacques2021grammar}.

This means we are probably to analyze \form{two ugly blackbirds} as something like 
\form{[the_{\annotate{D}} [two [-s [[ugly]_{\annotate{AP}} [blackbird]_{\annotate{N}}]_{\annotate{FP}}]_{\annotate{Num'}}]_{\annotate{NumP}}]_{\annotate{DP}}}.%
\footnote{
    We are describing the phrase structure using \term{functional heads};
    see the end of this section, ans \prettyref{sec:remove-functional-heads}.
    We are making the Cartographic assumption that adjectival modifications
    are also introduced by functional heads (FPs) and not an adjunction operation
    radically different from complementation.
    The motivation is to make the primitives of syntax more simple and flexible.
}
Does the compound \form{blackbird} get isolated from the rest of syntax
and hence have a special status (sometimes called \term{lexical integrity})
and can be seen as a word?
Not necessarily. A noun phrase is also a small world in the eyes of the clause.
This doesn't make a noun phrase a ``word'' in any proper sense.
Furthermore, derived words are indeed subject to syntactic processes.
(\ref{ex:word-coordination}) shows some attested examples.

\begin{exe}
    \ex\label{ex:word-coordination}
    \begin{xlist}
        \ex {} [pre- and post-revolutionary] France 
        \ex  back- and tooth ache (from \href{https://web.archive.org/web/20250419153415/https://www.knightpiesold.com/sites/pa/assets/File/Bakubung/App%20D4%20Terrestrial.pdf}{Internet})
    \end{xlist}
\end{exe}

Thus \form{the two ungly blackbirds} are probably to be analyzed as something like (\ref{ex:np-structure}),
where \category{definite} and \category{plural} are going to be replaced by
the particle \form{the} and the suffix \form{-s} after phonological realization
(\prettyref{sec:vocabulary-inesrtion}).
Here following the usual Distributed Morphology assumption that \form{backbird} is \emph{categorized} into what we commonly know as a noun
by virtue of referring to an abstract notion of a type of objects, etc.,
and we describe the ``categorizer phrase'' as nP.
The adjective phrase \form{ugly} has its own internal structure,
but for simplicity it is not represented here.

\begin{exe}
    \ex\label{ex:np-structure} [\category{definite} [two [\category{plural} [[ugly]_{\text{AP}} [√black √bird]_{\text{nP}}]_{\text{FP}}]_{\text{Num'}}]_{\text{NumP}}]_{\text{DP}}
\end{exe}

How (\ref{ex:np-structure}) is turned into its surface form involves post-syntactic operations that bridge syntax and phonology,
commonly known as (realizational) morphology.
This is discussed in \prettyref{sec:vocabulary-inesrtion}.

\subsubsection{Abstract syntax as hierarchies of grammatical categories around roots}\label{sec:abstract-syntax-hierarchy}

There is nothing in abstract syntax besides roots and what are known as functional heads in generative syntax,
more commonly called grammatical markers in descriptive linguistics.
And the grammatical categories, together with ``arguments'' they introduce
(including clausal arguments, adjective phrases, etc.) are wrapped around the core root of a construction (like a noun phrase or a clause)
layer by layer, forming an endocentric structure.
The endocentric structure of layered grammatical categories in noun phrases is shown in (\ref{ex:np-structure}).
In the clause, the structure is like vP-TP-CP,%
\footnote{
    See standard Chomskyan generative syntax textbooks. 
}
or in descriptive terms, a hierarchy of grammatical categories in the hierarchy of argument structure \textlt tense, aspect and modality \textlt speech force categories or complement clause types.

The hierarchy of functional heads (i.e. grammatical relations and categories)
as is exemplified in (\ref{ex:np-structure}) has real effects.
We have already seen its semantic effects in interpreting (\ref{ex:np-structure}).
In clauses, we find that the order of tense-aspect-modality adverbs and corresponding auxiliaries seem to have a regular correspondence,
which can be explained by assuming that the TP or \term{tense phrase}
actually splits into a series of functional projections,
as is what is done in Cartographic syntax \citep{cinque2009cartography}.

In the vP layer, we can find the influence of this layered structure as well:
we have several syntactic tests to show that certain arguments (usually the agentative ones)
are ``higher'' than others.
Besides commonly known phenomena of binding of reflexive pronouns (\form{she hates herself}),
we have a good example from causativization in Japhug.
We note that Japhug allows double causative,
and when this happens, the meaning is always like 
\translate{$X$ makes [[$Z$ do sth. to $W$] with $Y$]}
(we denote it by $X \to Y \to Z \to W$),
and the polypersonal direct-inverse indexation on the main verb (with the form $X \to Y$)
is determined by first comparing the prominence of $W$ and $Z$ on the empathy hierarchy,
and then comparing the prominence of the winner with that of $Y$,
and then the prominence of the final winner is compared with $X$,
the result of which determines if the inverse marker appears.
Hence a 1\textto 3\textto 2\textto 3 configuration
is morphologically the same as 1\textto 2
\citep[\citepage{848}, (67)]{jacques2021grammar}.
Similarly, both 2\textto 3\textto 1 and 2\textto 1\textto 3 
are equivalent to 2\textto 1 in argument indexation
\citep[\citepage{584}]{jacques2021grammar},
and both 3\textto 3\textto 1 and 3\textto 1\textto 3 
are equivalent to 3\textto 1 in argument indexation
\citep[\citepage{310}]{jacques2021grammar}.%
\footnote{
    On the other hand, 3\textto 1\textto 2 becomes 3\textto 1,
    and 3\textto 2\textto 1 becomes 3\textto 2.
    But this just means that when both inner arguments are speech participants,
    then agentivity leads to a higher prominence.
    Still predictable on structural basis
    once we refine the empathy hierarchy.
}
This strongly suggests a \form{[\category{causer} [\category{instrument} [\category{agent} \category{patient}]]]} hidden structure.
Actually tense and aspect can be analyzed in this way as well 
\citet[\citesec{7.4.1}]{wiltschko2014universal}.

Hierarchies like this are actually one of the best criterion that tell a grammatical marker from a root.
In English, auxiliaries and suffixes in \form{have been being consulted} 
shows a passive \textlt progressive \textlt perfect \textlt present hierarchy,
which is fixed in its semantics and in its linear order.
The arrangement of tense-aspect-modality adverbs appears to be a part of the same hierarchy \citep{cinque1999adverbs}.
Therefore we are \emph{not} observing an ordinary complement clause construction.
On the other hand, \form{want to be able to do sth.}
and \form{be able to want to do sth.} are both valid:
the latter is less frequently attested but is attested anyway%
\footnote{
    You can check yourself by searching it in COCA.
}.
Therefore \form{be able to} and \form{want} are not auxiliaries -- yet.

\subsubsection{A note for panicking descriptive linguists}\label{sec:remove-functional-heads}

In (\ref{ex:np-structure}), we have \term{determiner phrase} or \term{number phrase} or \term{nominal categorizer phrase},
but we do not have \term{noun phrase}.
This is related to how the idea of functional heads was historically developed in generative syntax.
At first we only had lexical heads, which roughly corresponded the root at the center of a construction.
Later it was found that certain phenomena are better captured
if we assume that the functional markers have their own ``phrases'' as well,
like DP, nP, TP, vP, etc.,
and finally it is found that we can keep the concept of \term{head} to functional markers only.

Still a descriptive linguist wants to avoid 
(a) explicitly mentioning functional heads,
(b) using constituency relations to representing certain grammatical information, which intuitively would be better represented by dependency relations, and 
(c) assuming a tree that is too deep, containing many layers, constituents, etc. (nP, TP-splitting, multiple functional projections for different types of adjectives in Cartography).
To be fair, we \emph{can} always do away with these.
Constituency and dependency are formally equivalent,
and we can always replace sentences like ``in a CP, \dots''
by ``in a full clause that allows information structure marking, \dots'',
i.e. avoid functional heads by focus on the grammatical environments they create.
What is being done here is quite similar to how in physics,
virtual photons are integrated out, leaving an effective Coulomb interaction.

\citet[\citepage{49}]{dixon2009basic1}%
\footnote{
    Although Dixon is strongly against formal linguistics,
    his Basic Linguistic Theory is largely in line with what we describe here.
}
complains about using constituency
to represent the relation between roots and grammatical items.
He is right: for consistency, in \form{to the fat man},
either we write \form{[fat man]} as a functional projection
as well (according to Cartographic syntax),
or we de-emphasize the status of \form{the} and \form{to} as constituents
and only treat them as markers of certain \emph{syntactic environments}
or \concept{constructions}.%
\footnote{
    We however do not endorse Construction Grammar,
    as \term{constructions} defined in this way are still subject to compositional analyses.
}
Thus (\ref{ex:np-structure}) may be replaced by something like (\ref{ex:np-structure-corrected}),
which avoids problem (a) by replacing concepts like DP or NumP
by ``a definite noun phrase with a numeral'',
and also contains the final phonological realization of grammatical categories (\category{definite} \textto \form{the}) for convenience.
Now since functional heads are eliminated, the term \term{head} can be kept
to the \emph{core} of a construction, and not grammatical markers.
We intend to use the term \term{head} loosely here:
any reasonable constituency containing the center root of a construction can be called a head,
and thus we can say that \form{blackbird} heads (\ref{ex:np-structure-corrected}),
although within \form{blackbird} we may still say that \form{-bird} is the head.
It is hard to see what should be seen as the core of a construction (e.g. when analyzing coordination)
and in this case we simply stop using the term \term{head}.%
\footnote{
    If we go back to generative syntax, the (functional) head is the coordinator.
}

\begin{exe}
    \ex\label{ex:np-structure-corrected} [the_{\text{\category{definit}}} two_{\text{\textcolor{blue}{\category{plural}}}} [ugly [black-bird]_{\text{nominal compound}}\textcolor{blue}{-s_{\text{\category{plural}}}}]_{\text{modification}}]_{\text{noun phrase}}
\end{exe}


(b) and (c) are not huge problems in (\ref{ex:np-structure-corrected}).
(c) is a problem that occurs when describing e.g. the \form{have been being consulted} split TP projections.
These split TP projections are what are \emph{newly} introduced into the clause,
when a clause is being formed:
all arguments, adverbials, etc. are first finished on their own 
and then sent to clausal syntax,
so clausal syntax doesn't have a lot to do with their internal structures.
On the other hand, things like the tense, aspect and modality markers are built up \emph{within one batch} when the clause is being built.
The clause doesn't see finished materials clearly and can only see things like person, number, etc.,
but it sees tense, aspect, modality, etc. clearly.
This intuition is related to the cyclic nature of syntax, and in particular, the \emph{phase} in generative syntax.
Thus markers of grammatical categories of valency, tense-aspect-modality, and speech forces (imperative, interrogative etc.)
are \emph{closer} to the verb in this sense.
We therefore find a way to flatten the deeply hierarchical syntax tree:
we just package everything \emph{newly introduced into the clause},
like \form{have been being consulted},
into something known as e.g. \term{verb phrase},%
\footnote{
    When the flatten-tree approach is not adopted,
    \term{verb phrase} often refers to the nucleus clause (i.e. TP) minus the subject.
    See e.g. \citet{cgel}.
    The flat-tree notion of verb phrase is related to how wordhood is defined in \prettyref{sec:inflection-flatten-tree}.
}
and then study the hierarchical relations between components of that verb phrase
within the verb phrase.
Thus problem (c) is solved.
The result will be comparable to how the clause structure is represented in \citet{quirk2010comprehensive}:
a flat syntactic tree is given first (\citepage{45}, no excessive hierarchies of grammatical categories mentioned first),
and then the authors go into the details of the hierarchy and relative scopes of auxiliaries
(\citepage{121}).

As for (b), we can replace the notion of layered functional projections
by the notion of a bunch of \emph{dependency relations} with different closeness to the head
(the root at the center of a construction, like \form{blackbird} in \form{the two ugly blackbird}, \emph{not} a functional head).
Actually the remaining constituency relations posited in (\ref{ex:np-structure-corrected})
can also be described in terms of dependency relations:
the relation between \form{ugly} and \form{blackbird} is closer than
that between \form{the} and \form{blackbird}, etc.
Without functional heads, dependency and constituency are still equivalent.
Which language to use depends on the features of the language,
like whether there are multiple topicalization and focalization
(which probably will make dependency-based analysis a good choice
as it makes starting easier).

Therefore, in the notation of typical descriptive grammars,
we may say that there is nothing in abstract syntax besides roots
and grammatical categories, relations, and constructions.
A one-to-one transform between the more tradition description
and the generative description that seems more exotic but involves less primitive concepts
is sketched in this section.

\subsection{Commonly understood wordhood cannot be defined based on abstract syntax}\label{sec:no-word-in-syntax}



\subsubsection{Wordhood as small constituency?}\label{sec:derivation-deep-tree}

The abstract syntax defined above causes a problem.
If we insist on defining a \emph{syntactic} word as a small constituent,
then \form{blackbird} \emph{is} a word -- but \form{blackbirds} isn't,
because the latter has an affix with a quite high position in the structure attached to it.
Which goes against the common notion of wordhood.

Similar problems occur in clausal syntax.
The abstract syntax of \form{he sleepwalked into this frustrative situation}
can be described as follows
(we can also write it in a more compact way as in \ref{ex:np-structure-corrected}:
see \ref{ex:clause-new-batch}):

\begin{enumerate}
    \item \form{sleep} and \form{walk} are first placed together to form a compound,
    meaning that someone is walking while sleeping,
    with a metaphoric meaning of \translate{taking action blindly}.
    \item The compound takes \form{he} and \form{into this frustrative situation},
    two already well-formed phrases, as its arguments.
    \form{he} is structurally higher in the sense that it binds the other when a reflexive appears
    (thus \form{he_i dreamwalked into this problem caused by himself_i}).
    The argument structure is formed.
    \item The clause is in the simple past \category{tense}.
    \item The agent in the argument structure, by default, is promoted to the subject position,
    as the pivot of the whole clause (which can be tested in coordination, etc.).
\end{enumerate}

So \form{sleepwalk} forms a constituent, and can be seen as a word.
Yet the past tense marker \form{-ed}, being added into the clause much later,
has a scope that covers the whole clause.
\form{sleepwalked} is \emph{not} recognized as a word based on constituency!

\subsubsection{Wordhood in flat-tree syntax}\label{sec:inflection-flatten-tree}

Now, another way to define wordhood is based on the flat-tree approach mentioned in 
\prettyref{sec:remove-functional-heads}.
This immediately solves the problem of \form{sleepwalked}:
now the verb, everything related to the voice, tense-aspect-modality are considered to form one ``constituent''
(with the definition of constituency modified a little bit to be consistent with the flattened tree),
because they all below to the new things added into the clause when it is formed 
(see the list in the last section),
and hence \form{sleepwalk-ed} is a word.
This is shown in (\ref{ex:clause-new-batch}):
the already finished materials are labeled in gray,
leaving only the verb compound and the tense marker in black.

\begin{exe}
    \ex\label{ex:clause-new-batch} [[\textcolor{gray}{he}]_{\text{subject},i} [---_i [sleep-walk]_{\text{verbal compound}} [\textcolor{gray}{into this frustrative \\ situation}]_{\text{location}}]_{\text{argument structure}}-\category{past}]_{\text{declarative clause}}
\end{exe}

But now it seems we have to recognize that \form{have been performing} is also a \term{word} under this definition,
if we define wordhood based on the flattened version of abstract syntax.
We can make it even radical by pointing out that \form{have been being annoyed} is also a word in this sense.
However, usually people will just call it a \term{verb phrase}.

In certain languages, stacked auxiliaries do have a strong ``word'' vibe,
and what is originally considered a verb phrase may really be eventually considered a word.
A good example is Modern Japanese:
so-called auxiliaries in the School Grammar system,
once closely inspected, look more like suffixes and not true auxiliaries,
as nothing can be inserted between them and the root.
The so-called verb phrase in the rGyalrongic language Jiaomuzu
is now described in a way that is not quite phrase-like \citep{prins2016grammar}.
But abstract syntax does \emph{not} guarantee this:
in English, the verb phrase (in the flattened-tree meaning)
may contain materials outside of the batch newly introduced into the clause structure as well:
we have \form{he \textcolor{blue}{has} recently \textcolor{blue}{discovered} that \dots}

In the same way, we may even want to say that \form{the} and \form{blackbirds} 
in \form{the two ugly blackbirds} form a syntactic word,
if we accept the definition of wordhood developed above:
in the same way we gray out the already finished syntactic objects in (\ref{ex:clause-new-batch}),
we should gray out \form{two} and \form{ugly} as they are well-formed phrases on their own,
while \form{the} and \form{blackbirds} are newly introduced materials
when constructing the noun phrase.
Arabic and Hebrew's nouns actually fit well in this definition of noun wordhood,
but English nouns certain do not.

A \term{word} defined in this section, when it contains a root,%
\footnote{
    The standard of being finished varies.
    In languages where the distinction between TP and CP is clear,
    we may want to collect materials in CP into one ``phrase'' or a syntactic ``word'' in the flattened-tree sense of this section,
    and collect materials in TP into another ``phrase'' or ``word''.
    In this case the CP ``word'' contains no root.
}
probably should be considered as a \term{form} in an inflection table, and not literally a word,
as an inflection paradigm maps a root with all newly added materials in the construction headed by the root,
i.e. the grammatical categories, into a concrete form.
And in an inflection paradigm, \term{wordhood} defined in this section starts to make sense.
The English \form{has \dots been \dots reading} ``word'',
which contains all \emph{newly added} materials in a \category{present progressive} clause 
with a third person subject,
is indeed a \term{inflected form} in the English verbal inflection paradigm,
although there is no guarantee that all parts of the inflected form are held together
by post-syntactic morphological operations (\prettyref{sec:realization}):
the auxiliaries are subject to movements, and adverbs may interrupt the linear continuity.

Whether the syntactic ``wordhood'' defined in this section sounds absurd is strongly language-specific.
We have argued above that \form{the} and \form{blackbirds}
in the noun phrase \form{the extremely ugly and annoying blackgirds}
contain all the materials newly added into the noun phrase
besides the already finished \form{extremely ugly and annoying},
and therefore form a syntactic word.
in Modern German, considering the article and the noun as one word
-- despite them being separable -- is beneficial for descriptive purposes,
as case inflection appears on the article,
and the article-plus-noun combo looks quite like the English \form{have \dots been \dots reading}
and fills a cell in the nominal inflection paradigm.
Besides definiteness, the nominal inflection paradigm should also include the category of case,
which marks the grammatical relation of the noun phrase.
In Japhug, case is marked by independent words \citep[\citepage{8.2.1}]{jacques2021grammar},
and although considering the case marker and the noun stem as one word sounds reasonable for
speakers of German, Latin, or Sanskrit,
it sounds absurd for Japhug speakers.
What is language-specific here is actually the definition of 
\emph{morphological} wordhood, i.e. what formatives in a syntactic ``word'' in this section
are actually brought together by post-syntactic operations (\prettyref{sec:morphological-wordhood-production}).

Finally, from the perspective of historical linguistics,
we note that the meaning of \term{already finished material} may need some deliberation.
When the internal structure of certain ``finished materials'' are simple enough,
we have reasons to include them as a part of syntactic ``words''.
For instance, consider the emergence of \form{etiam} as a single particle
(\prettyref{sec:syntactic-reanalysis}):
here if \form{iam} is analyzed as a content word,
then strictly speaking, we have no reason to consider \form{et iam} as a syntactic ``word''.
Similarly a full clause containing only pronouns and the main verb may need to be seen as a syntactic ``word'' (\prettyref{sec:phonological-word-reanalysis-condition}).
What we are seeing here looks like effects of idiomization i.e. lexicalization
(\prettyref{sec:vocabulary-inesrtion}).

\subsubsection{Comment: inflection and derivation}\label{sec:inflection-derivation}

We note that a ``word'' defined in \prettyref{sec:derivation-deep-tree}
is always a part of another ``word'' defined in \prettyref{sec:inflection-flatten-tree}.
Roughly speaking, the so-called ``wordhood'' defined in \prettyref{sec:derivation-deep-tree}
can be described as the \concept{derived stem},
while the so-called ``words'' in \prettyref{sec:inflection-flatten-tree}
are forms to be found in a inflection table possibly containing periphrastic forms.
Note, however, that certain operations commonly known as derivation
fall under the category of the latter:
nominalization (\form{his skillful playing of the nationan anthem},
cf. the non-finite gerund clause \form{his skillfully playing the national anthem}),
involves alternation of the subcategorization frame of the root \form{play},
and [\form{√play}, n] or [\form{√play}, v] both do not form constituencies
in the sense of \prettyref{sec:derivation-deep-tree}.
Furthermore, in \citet{jacques2021grammar}, valency alternation is classified as derivation,
probably because of the morphological structure of the verb
(derivational affixes appear to be a part of the extended stem,
which is then placed into a rigid template; \prettyref{sec:morphological-wordhood-production}).
The derivation/inflection also seems to be based on a variety of 
not necessarily converging criteria, just like wordhood.
And given the inherent ambiguity of the derivation/inflection distinction,
the concept of \term{step} is not always useful.

\subsubsection{Wordhood of function words}\label{sec:no-function-word-in-syntax}

In abstract syntax, roots (and structures formed around them) and grammatical markers are different.
Therefore even if we can define something like wordhood of function words,
it will be different from the wordhood definition we desire for content words.
And we do not have a clear definition of wordhood of function words, either.
We may say that a grammatical marker belonging to a larger construction is a function word.
Thus \form{the} in (\ref{ex:np-structure}) is a function word.
But in Latin, we have the \form{=que} clitic which works just like a conjunction,
and yet it has to be attached to something else and usually is not considered a word.
Therefore for grammatical markers, wordhood is still not something definable in abstract syntax.

\subsection{Lexicon, phonological realization, and morphological wordhood}\label{sec:vocabulary-inesrtion}

\subsubsection{Phonological realization guided by lexicon}\label{sec:realization}

Now we go to the second half of the story in \prettyref{sec:abstract-syntax}.
The abstract syntax (e.g. \ref{ex:np-structure} or \ref{ex:np-structure-corrected})
has to be linearized into the phonological i.e. surface representation.
The whole process of course is guided by the lexicon,
which may give us a proper definition of wordhood.
In Distributed Morphology the lexicon contains List A containing roots and grammatical items,
List B that guides phonological realization of the roots and grammatical items,
and List C that records idiomized meaning of everything.
We should note that what is discussed here is about ideal competence of a person already fluent in a language.
We can expect that the human brain always wants to find shortcuts
and does not start the whole structural building process from sketch all the time \citep{matchin2020cortical},
and tends to store finished trees in the mental lexicon,
and that Lists A, B, and C are actually kind of mixed in the actual brain. 

List A is useless in defining wordhood.
List C is also hopeless, because the constructions it contain vary wildly in size:
we have meanings of (category-less) roots,
meanings of roots plus categorizers (thus \form{buffalo} in a verbal environment means \translate{to intimidate}),
and even meanings of a whole sentence.
Lexicalization is simply idiomization or in other words semantic fossilization:%
\footnote{
    Semantic fossilization does have syntactic effects:
    they may block certain movements, like topicalization of a prepositional phrase 
    after a verb, to avoid disrupting interpretation \citep{nediger2017unifying}.
    Therefore, it can be a hotbed for \emph{syntactic} fossilization,
    i.e. graduate erosion of the internal structure of a stored structure in the lexicon.

    Note that syntactic fossilization is not always accompanied by loss of compositional semantics:
    syntactic fossilization of a prototypical clause structure 
    into a verb morphological template involves nothing non-compositional.
    But as is said above, as a shortcut in the brain,
    a completed tree -- like a simple clause -- therefore can also be stored
    and undergoes mild semantic fossilization,
    which starts its syntactic fossilization.
}
what is being lexicalized does not have to be a word \citep{harley1999distributed}.
Thus, we should place our hope in List B. 

In Distributed Morphology, phonological realization of an utterance
is done by so-called post-syntactic operations:
post-syntactic rules adjust the positions of roots and grammatical items.
Note that the term \term{syntactic} here means \emph{abstract syntax}.
It is not until post-syntactic realization that some syntactic phenomenon appear:
for instance, where the main verb eventually appears (a syntactic property) may be determined by
whether functional heads along the TP-CP hierarchy are ``strong''
and have to attract something to them for correct surface realization
(sounds morphological!),
which is how we capture English subject-auxiliary inversion and similar phenomena.
A bundle for example may look like [\form{√eat}, v, T[\category{past}]]:
root \form{eat-}, verbalized, in past tense -- basically a \term{verb phrase}
in the sense of \prettyref{sec:inflection-flatten-tree}.
Then \concept{vocabulary insertion} happens,
which gives all pieces in the syntactic structure (roots or functional items) phonological forms,
turning them into substantive \concept{exponents} (\ref{ex:vi}).%
\footnote{
    Some terms with meanings similar to \term{exponent} should be noted here.
    The term \concept{formative} is sometimes used to refer to a piece in morphology from the parsing side, not necessarily with a clearly understood grammatical function --
    it may be even totally historical and mark no synchronic grammatical category.
    A \term{exponent} on the other hand always has a clearly specified grammatical function.
    Thus we may say ``zero exponent'' -- but rarely we say ``zero formative''.

    Sometimes, in certain Distributed Morphology papers, it means things in List A,
    i.e. abstract primitive syntactic objects. 
    This meaning is completely opposite to its usual surface-oriented meaning.

    Finally we have the good old term \term{morpheme},
    which unfortunately is theoretically loaded, implying a transparent,
    one-to-one relation between form and meaning,
    and indistinguishability between roots and grammatical items.
}
This is not the final step of the syntactic operations,
because the concrete phonological forms still need to undergo certain phonological reconstructions
(a most radical example is Semitic template morphology; \citealt{tucker2011morphosyntax}).

\begin{exe}
    \ex\label{ex:vi} \begin{xlist}
        \ex √love, v \textto love-
        \ex T[\category{past}] \textto -ed
        \ex √eat, v, T[\category{past}] \textto ate
    \end{xlist}
\end{exe}

The fact that certain roots are only used as verbs or nouns
can be simply explained by stipulating that the other configurations do not have corresponding List B entries:
thus \form{√eat}, n cannot be phonologically realized,
simply because there is no such thing in the mental dictionary of English speakers.
Thus \form{*eat (n.)}

\subsubsection{Bundles in phonological realization as morphological words}\label{sec:morphological-wordhood-production}

Now we see something that looks like a good definition of wordhood.
Post-syntactic reordering of exponents treats different parts of the \term{verb phrase},
or the noun phrase minus adjectives, or whatever considered to be a word in the sense of \prettyref{sec:inflection-flatten-tree}, in different ways.
T[\category{past}] in English does not want to stay alone,
and wants to get attracted to something bigger:
once it gets attracted near the verb root,
it can no longer go away from it, besides some possible local dislocations.
This is why we call \form{loved} or \form{ate} a word.
on the other hand, things like \form{has been considering} are considered multi-word:
T[\category{present}] still has to be attached to something else,
but this time, we have a \category{perfect} aspect (or secondary tense, depending on terminology) feature in the clause as well,
which is realized as \form{have-} and the two combine into \form{has}.
The Asp[\category{progressive}] feature, having no tense marker to combine with it,
takes the \form{been} form, while the main verb is in the \form{-ing} form surrounded by the progressive aspect.
Basically, post-syntactic operations never collect T[\category{present}], T[\category{perfect}], Asp[\category{progressive}] and the the root into one bundle:
the first two are placed into one bundle, the third and fourth are left on their own.
A ``word'' defined in \prettyref{sec:inflection-flatten-tree}
is therefore realized as one or more morphological words.

In a sentence: \form{what are moved together in phonological realization form a morphological word.}%
\footnote{
    In a lexicalist theory they may be known as X^0 nodes, as in e.g. \citet{bickel2007free}.
    However, recent studies in Minimalist syntax are becoming more suspicious
    of the syntactic status of head movements,
    and here we follow Distributed Morphology and consider them to be formed by 
    post-syntactic operations.
}
Or to be more concise: \form{morphological wordhood is about morphological selection.}
Note that this definition also defines function words,
which contain no roots, but may have behaviors comparable to content words.
We have just seen how the auxiliary \form{has} appears in the English \form{present perfect} in third person:
the ``stem'' \form{have-} is purely a \category{perfect} marker here,
but when it is combined with the third person singular \form{present} marker \form{-s},
it inflects just like an ordinary verb.
This solves the problem in \prettyref{sec:no-function-word-in-syntax}.

We note that the process sketched here generally tends to 
make exponents of grammatical categories closer to the root also closer to the root
in the linear order after phonological realization.
In reality, this is not always the case:
we have both \emph{layered} morphology which represents the hierarchical structure
and \emph{slot-filler} or \emph{template} morphology, which is linear and flat and defined in terms of slots,
which is however still within the formal complexity class of Distributed Morphology,
as local dislocation rules, some of which may have phonological motivations,
can reorder the exponents;
syntactic ordering of grammatical categories is also a possible explanation
\citep{bye2020morpheme}.%
\footnote{
    What seems more problematic to syntax-oriented theories of morphology like Distributed Morphology
    is the existence of extended exponents (e.g. \citealt{bickel2007free}).
    This however can still be captured by assuming fission of a morphological feature \citep{bobaljik2017distributed}.
}
By this logic, almost arbitrary reordering of affixes is also not impossible,
which is indeed what is observed in Chintang;
we nevertheless acknowledge that the affixes are indeed affixes,
because they pass tests for grammatical wordhood (i.e. morphological wordhood in this note),
like obligatoriness, selectivity of hosts, and interaffix dependence,
which clearly indicate that they are collected together by post-syntactic morphological operations \citep{bickel2007free}.

\subsubsection{Accidentally fixed linear orders}\label{sec:morphological-wordhood-linear}

What is sketched above (\prettyref{sec:morphological-wordhood-production})
is the most natural definition of morphological wordhood from the perspective of \emph{production}.
We may also want to define morphological wordhood from the perspective of \emph{parsing}:
whatever seems to have a fixed morphological pattern
is often recognized as a morphological word.
This leads to an interesting corner case of morphological wordhood definition,
although it is not as drastic as \prettyref{sec:derivation-deep-tree} and \prettyref{sec:inflection-flatten-tree}.

Certain ``morphological templates'' identified in a surface-oriented analysis
do not seem to need any post-syntactic relocations.
Consider an imaginary dialect of English,
where there were far less tense-aspect-modality adverbs.
This would result in a lot of \form{have been being asked} sequences
not interrupted by inserted adverbs.
Further let us suppose that this dialect of English had lost the subject-auxiliary inversion rule
(not uncommon in contemporary casual speech: \form{you know what?}).
A linguist analyzing this would face the dilemma of whether to
consider the whole sequence as a morphological word.
It definitely looks like a morphological word,
but no post-syntactic relocation of abstract or concrete morphological pieces is necessary to generate it.%
\footnote{
    To be fair, if a content word ahs layered morphology,
    then the rigid order of formatives in it may also transparently reflect syntax.
    But post-syntactic morphology is doing something here:
    it \emph{selects} grammatical categories it wants to put into the inflectional paradigm,
    and leaves the rest to auxiliaries, particles, etc.
    In \form{have been being asked}, post-syntactic morphology is still doing things,
    but it's mostly constructing the auxiliaries one by one,
    without any non-trivial relation \emph{between} them.
}
This is indeed the case of Modern Japanese discussed in \prettyref{sec:inflection-flatten-tree}.
This is probably more clearly seen in the evolution of Turkish:
certain ``suffixes'' are written as auxiliaries in Ottoman Turkish
and are written as suffixes in modern Turkish,
but the linear forms are exactly the same,
so we see two competing analyses of the same linear surface form.

We may want to comfort ourselves by focusing on the fact that a sequence like this will likely soon be reanalyzed as an authentic morphological template.
We will go back to this issue when discussing clitics -- discussed immediately below.

\subsubsection{Clitics and morphological wordhood}\label{sec:morphological-wordhood-clitics}

Probably a more important phenomenon that may disrupt the definition of morphological wordhood in \prettyref{sec:morphological-wordhood-production} is \concept{clitics}.
A clitic, just like ordinary affixes, needs to be attached to something,
but it is less picky when choosing the host and deciding where to land.
A good example is the Latin conjunction \form{=que},
which can be attached to any inflected head noun,
sometimes an adjective, in a noun phrase coordination.
The fact that they need to be attached to something
seems to suggest that they pass the morphological wordhood test in \prettyref{sec:morphological-wordhood-production},
as they are dislocated by morphological rules as well \citep{harley1999distributed},
and should we consider them to be a part of the morphological word they attach to as well?

If the answer is ``no'', then a clear-cut distinction between clitics and affixes has to be made.
An observation is that the attachment of clitics to other words is ``late'',
while the post-syntactic relocation of typical affixes is ``early'':
typical affixes form their own units first, and then clitics are attached to these units
(e.g. \citealt[\citepage{485}]{jacques2021grammar}).
The ``late incorporation'' definition of clitics seems to be surprisingly stable cross-linguistically.
In Romance languages, personal clitics can only attach to the verb:
but they are nonetheless clitics because they are not compatible with noun phrase arguments,
and therefore they have to be originally pronouns,
only \emph{lately} incorporated into the verb.
Latin \form{=que} is also attached to things when everything else is formed,
so it is attached to nouns or adjectives \emph{lately}.

However, in these cases, the clitics cannot see the internal structure of a morphological word,
and can never cross the boundary of a morphological word:
affixes reordering is possible, but a clitic cannot be incorporated into the units they have already formed
\citep{embick2007linearization}.
Whether this statement is universally true or not is not clear,
and we have evidence against it.
It seems that in Udi, we have personal agreement formatives
that can be attached to both focalized elements and verbs.
So, these formatives are more picky than the Latin \form{=que}, but still much more flexible than typical affixes do,
and it seems they should be analyzed as clitics.
It is then observed that these clitics can invade the verb morphological template
(known as \concept{endocliticization}),
and their positions of clitics in the verb morphological template 
are subject to the control of other formatives \citep{harris2000word}.

Once the boundary of pre-formed morphological words is no longer a problem for cliticization,
distinguishing clitics from affixes becomes less easy.
Chintang, a rather unusual language in which prefixes take arbitrary orders,
also have focus clitics that can be incorporated into the verb and
they occasionally appear between prefixes,
making a focus clitic just like another prefix that can relocate arbitrarily.
Besides verbs, pronouns and adverbs can also be focalized by the same clitic,
but this doesn't say much because there are affixes that apply to both nouns and verbs:
what \emph{can} show that a formative is a clitic is that
it can attach to multiple hosts \emph{in the same construction},
but what we see in Chintang is that we need to shift the focus 
for the clitic to jump to another word.

Tests on the linear order of formatives are usually not completely convincing in these cases.
In theory, if language-specific morphological factors can reorder exponents
to create a flat template with a rigid linear order regardless of syntax,
the same can be done to clitics,
and indeed in Udi, just as is shown above, personal agreement clitics,
when incorporated into verbs, have their pre-specified slots in the morphological template. 
Or we are allowed to reorder both affixes and clitics in a quite wild way,
as is the case in Chintang.
Or no reordering is done: if auxiliaries can receive a rigid linear order without the possibility of intervention of any other materials, then clitics can, too. 

Often, we start by noticing that a sequence of formatives follows a rigid order
which doesn't seem to be completely transparently come from syntax
(having a slot-and-filler structure that doesn't reflect syntactic scopes of the grammatical categories, or having a layered structure not covering all grammatical categories attested in the grammar of the language, showing morphological selection),
while another set of formatives are strictly attached behind them and sometimes after other particles
(e.g. \citealt[\citesec{11.6.2}]{jacques2021grammar}).
But in this case, what actually does the heavy-lifting job
is the morphological coherence of the formatives with the rigid order:
we let the rigid order to guide ourselves to \emph{possible} morphological words,
as many languages do not madly alter the order of formatives,
but we always need further evidence to support morphological wordhood there.
On the other hand, without a rigid order,
morphological wordhood can still be determined,
as in the case of Chintang, by tests like like obligatoriness, selectivity of hosts, and interaffix dependence \citep{bickel2007free}.

In conclusion, the only thing that sets clitics apart is evidence supporting
them being incorporated into a morphological word lately.
What counts as valid evidence includes 
the formative in question being in conflict with some other components in the construction
(thus a formative in conflict with explicit arguments is a pronoun in disguise,
hence likely a clitic),
or the formative being able to attach to multiple hosts
\emph{without any alternation of the construction containing it},
or the formative seeming to interact with a morphological unit already well-formed.
Other criteria, like rigidity of linear order,
are generally not decisive in the most puzzling cases
but can provide candidates for wordhood.

\subsection{Phonological wordhood}\label{sec:phonological}

A final type of wordhood is defined not via morphosyntax, but via phonology:
what forms a unit in phonology is considered a word.
\citet[\citesec{10.3}]{dixon2010basic2} overviews some of common criteria used to define the phonological word.
Cross-linguistically, different phonological processes may happen in different domains \citep[\citepage{62}]{schackow2015grammar},
making a cross-linguistic definition of phonological wordhood impossible.
What is cross-linguistic is the relevant phonological phenomena,
not the exact definition of phonological wordhood.

Moreover, phonological wordhood can be inconsistent with all types of wordhood defined above:
its relation with morphological words can be quite arbitrary,
and it does not always respect syntactic ``wordhood'' defined in \prettyref{sec:derivation-deep-tree} or \prettyref{sec:inflection-flatten-tree}.
Formation of phonological words can ignore a lot of morphosyntactic information.
To illustrate the point, let us start with an easy example (\ref{ex:chinese-poem-compare}).
Prosody plays an important role in modern Mandarin:
an utterance is divided into a series of disyllabic prosodic words from left to right,
with the first syllable being heavy and the second light.
Thus in the reading convention in modern Mandarin Chinese of Classical poetry,
a line like (\ref{ex:chinese-poem}) is read as (\ref{ex:chinese-poem-reading}).
Note that the first prosodic word is inconsistent with the constituency relations in (\ref{ex:chinese-poem}).

\begin{exe}
    \ex\label{ex:chinese-poem-compare} \begin{xlist}
        \ex\label{ex:chinese-poem} 夕贬潮阳路八千 
        \gll [xī]_{\text{temporal}} [-- biǎn [Cháoyáng]_{\text{locative}}]_{\text{verbal predication}} [lù bā qiān]_{\text{non-verbal predication}} \\
        evening {} relegate \category{place} road eight thousand \\
        \glt\translate{In the evening, [I] was relegated to Chaoyang, the road being eight thousand miles.}
        \ex\label{ex:chinese-poem-reading} 夕贬|潮阳|路八千
    \end{xlist}
\end{exe}

In (\ref{ex:chinese-poem-reading}), a phonological word -- 夕贬 -- contains
two morphological words which do not form a syntactic unit
in the sense of \prettyref{sec:derivation-deep-tree} or \prettyref{sec:inflection-flatten-tree},
a phonological word is identical to a morphological word (潮阳, a place name),
and a phonological word is a clause on its own (路八千).
Thus, there is no guarantee that a simple relation exists between morphological and phonological wordhood.

Now, cross-linguistically, a review of possible relations between morphological and phonological wordhood can be found in \citet[\citesec{10.6}]{dixon2010basic2}.
Not many generalizations can be made on what is found.

\subsubsection{A morphological word consisting of more than one phonological word}\label{sec:phonological-multiple-phonological-one-morphological}

First, it is possible that a morphological word consists of more than one phonological word.
Basically this means the morphological word breaks into parts in the middle.

A complex verb (a morphological word) in Mandarin containing a disyllabic root and a directional complement,
like 支楞起来, has to be divided into two phonological words,
one containing the root, another containing the directional complement.
A similar phenomenon is observed in Moloko, where a grammatical word (itself containing a root and affixes) 
attracts several clitics to form a verbal complex,
which then is reorganized into two phonological words \citep[\citepage{202}]{Friesen2017}.

In \citet{bickel2007free}, it is proposed that all prefixes themselves are phonological words and attach to phonological words,
which leads to the nearly free order of formatives before the stem.

What are shown above all have a pattern:
a morphological word breaks into phonological pieces,
with one piece containing the root.
But it is also possible that the two phonological words from one morphological word
both contain roots, when the morphological word is a compound \citep[\citepage{23}]{dixon2010basic2}.

\subsubsection{A phonological word consisting of more than one morphological word}\label{sec:phonological-multiple-morphological-one-phonological}

Second, it is possible for a phonological word to consist of more than one morphological word.
This usually means the morphological words are too lightweight
to appear alone as phonological words.

We may immediately think of cliticization.
If a clitic is not considered to be a part of its host,
then cliticization that involves local dislocation not motivated by pure phonological reasons
can be seen as producing phonological words containing multiple morphological words
(\prettyref{sec:morphological-wordhood-clitics}).
It should be noted that some clitics may not be completely fused into the host they attach to in phonology.

We also have examples of \emph{phonological} cliticization,
which are clearer for demonstrating the idea of a phonological word consisting of multiple morphological words.
\citet[\citepage{49}]{dixon2009basic1} calls \form{to} and \form{the} in \form{to the fat man} clitics,
because they are not stressed, and almost form a phonological word with \form{fat}.
What makes them different from clitics in \prettyref{sec:morphological-wordhood-clitics}
is that the positions of \form{to} and \form{the} are never shifted by any post-syntactic morphological adjustments:
they are attached to \form{fat} only because they are linearly next to it,
and when another word, like \form{famous}, is inserted between them and \form{fat},
they have no reason to be attached to \form{fat}.
On the other hand, clitics in \prettyref{sec:morphological-wordhood-clitics} need to undergo readjustments of their positions.
Thus \form{to} and \form{the} in \form{to the fat man} are purely \emph{phonological} clitics,
and when they are attached to content words following them,
a phonological word containing several morphological words (e.g. \form{to=the=fat}) forms.

The examples given here look like the inverse of what is given in \prettyref{sec:phonological-multiple-phonological-one-morphological}.
There, a morphological word (i.e. the realization of a flattened-tree syntactic ``word'' defined in \prettyref{sec:inflection-flatten-tree}) breaks into pieces,
and one piece contains the root.
Here, morphological pieces -- one of which contains the root -- are put together to form a larger unit in phonology.
But we should note that this does not always have to be the case:
夕贬 in (\ref{ex:chinese-poem-compare}) is also a phonological word containing two morphological words,
and both morphological words are made of purely roots.
Moreover, 夕贬 and \form{to=the=fat} are both \emph{not} flattened-tree syntactic ``words'':
the morphological words in them are there purely out of accident,
without any implications on syntactic structures.

\subsubsection{Irregular correspondences}

Third, a particularly interesting fact is that it is possible for a part of a morphological word
to be attracted to a nearby morphological word to form a phonological word
\citep[\citepage{24}]{dixon2010basic2}.
(\ref{ex:chinese-poem-compare}) already suggests that this is possible,
although 贬 there does not form a morphological word with its locative argument,
and 夕贬 therefore (\ref{ex:chinese-poem-compare}) should still be described as
a phonological word containing two morphological words.

\subsection{Interim summary}\label{sec:all-wordhood}

In synchronic description of morphosyntax,
the concept of wordhood is in principle not a must:
in morphosyntax, we can talk about small constituents, lexicalization (as idiomization),
formatives appearing together because of morphological operations.
We do have word-like units in phonology,
but they are sometimes inconsistent with whatever morphosyntax wordhood we propose.

Anyway, we have done a thorough survey of everything in morphosyntax and in phonology that looks kind of like a word. 
Our findings are summarized in the list below.

\begin{itemize}
    \item Syntactic wordhood based on syntactic constituency (\prettyref{sec:derivation-deep-tree}).
    This definition is too narrow and essentially is a over-narrow definition of the \term{stem}
    (i.e. one or more roots plus derivations),
    which excludes certain processes like valency alternation also known as derivation in some contexts
    (\prettyref{sec:inflection-derivation}).

    \item Syntactic wordhood based on flattened-tree constituency (\prettyref{sec:inflection-flatten-tree}).
    Things considered to be a part of a single word in \prettyref{sec:derivation-deep-tree}
    are also considered to be a part of a single word in \prettyref{sec:inflection-flatten-tree}.
    This definition of wordhood is too broad:
    essentially throws almost every part of grammar into an inflection table,
    as now we have to acknowledge that \form{have}, \form{been} and \form{exercising} in \form{have recently been actively exercising} form a single word.
    Interestingly, criteria (e-f) in \citet[\citepages{15-16}]{dixon2010basic2} are all satisfied in ``words'' defined according to the standard of \prettyref{sec:inflection-flatten-tree}:
    you do not see multiple occurrences of an auxiliary in a verb phrase (flattened-tree version) either!

    \item Criterion (b) in \citet[\citepage{13}]{dixon2010basic2} states that a grammatical word  (i.e. a morphological word, as we have no well-defined syntactic wordhood)
    has a conventionalized coherence and meaning.
    We note that this is true for words defined by intuition,
    but also true for some roots, and certain phrases and even clauses.

    \item Criteria (a, c) in \citet[\citepages{13-16}]{dixon2010basic2} state that 
    a grammatical word is one or more lexical roots to which morphological processes have applied,
    whose formatives always occur together.
    These criteria are essentially syntactic wordhood defined in \prettyref{sec:inflection-flatten-tree}
    narrowed down by purely morphological i.e. realizational considerations:
    formatives that pass the test of \prettyref{sec:inflection-flatten-tree} 
    and are collected into one location by post-syntactic morphological operations
    form a word (\prettyref{sec:morphological-wordhood-production}).
    The main problem of this definition is the existence of clitics,
    which also satisfy the ``togetherness'' condition.
    What sets affixes and clitics apart is that cliticization happens \emph{later} than formation of morphological words consisting of affixes and roots
    because of reduced morphological selectivity.
    Still a clear distinction is not always possible (\prettyref{sec:morphological-wordhood-clitics}).
    Another problem is that in certain languages,
    a list of formative appearing together not allowing intervention of any other materials
    may simply be a transparent reflection of syntax (\prettyref{sec:morphological-wordhood-linear}).
    
    \item Criterion (d) in \citet[\citepage{14}]{dixon2010basic2} states that formatives in a morphological word generally occur in a fixed order.
    This is not always true, as is discussed in \prettyref{sec:morphological-wordhood-linear}:
    a fixed order may be a direct reflection of syntax
    and what we see may just be a sequence of particles and auxiliaries
    (although in this case, reanalysis of the sequence into a morphological word will soon happen),
    and certain uncontroversial morphological words allow reordering of affixes.
    Fixed linear orders often accompany wordhood in many languages,
    but a double check to check if morphological selection is always needed.

    \item Phonological wordhood can be defined according to multiple standards,
    and a cross-linguistic definition is impossible.
    We should note that cliticization is also relevant in forming a phonological word.
    Further, phonological words and morphological words do not necessarily have clear relations
    (\prettyref{sec:phonological}).
    Criterion (g) in \citet[\citepage{18}]{dixon2010basic2} uses pause between words to define  wordhood, which may be seen as one type of phonological wordhood.
\end{itemize}

Given this wild diversity of wordhood criteria,
what is striking is not the absence of clear wordhood in certain languages,
but that these criteria still roughly converge in many languages.
An actual human learner, despite being able to acquire a language with high irregular correspondences between morphological and phonological wordhood,
may still feel the necessity of relatively regular correspondences between the two,
which enables more shortcuts in language processing in the brain (\prettyref{sec:realization}).
We are going to tentatively touch this topic below.

\section{The unit of transmission and evolution}\label{sec:unit-of-transmission}

We expect the unit of transmission and evolution
(that is to say, units that are slightly distorted within their boarders by language change;
cf. genes in biology)
in historical evolution of languages
to be neither too big (clause-like) nor too small (root-like).
It is frequent that roots does not appear in any natural utterance:
Latin is a quite good example.
A big unit like a complicated clause is likely decomposed into pieces when transmitted.%
\footnote{
    Some large units, like stories or legal principles, can be transmitted quite stably,
    as is seen in e.g. Indo-European languages \citep[\citechap{2}]{fortson2011indo}.
    But what are transmitted here are \emph{abstract ideas},
    and quite different sentences can be used to describe these cultural traits.
    They are valuable in providing candidates for cognates,
    but provide little materials for clause-level or phrase-level comparison.
}
So the primary locus of transmission probably will be something -- probably more than one -- in \prettyref{sec:all-wordhood}.
In theory, any historical law proposed on language evolution 
should be based on psycholinguistics of language transmission.
Such a microscopic foundation however is currently lacking,
and the only thing we can do is to go over \prettyref{sec:all-wordhood}
and check whether they are the unit of language transmission.

Moreover, language change involves both phonological and morphosyntactic changes.
Laws governing phonological changes have been listed in \prettyref{sec:intro}.
Laws governing morphosyntactic changes involves reanalysis (alternation of underlying structure with the surface form staying the same), extension (the surface form varies, often after reanalysis introducing new forms, like auxiliaries, while the underlying structure does not undergo major changes), and borrowing.
We will not focus on borrowing in this note.


\subsection{Purely syntactic reanalysis}\label{sec:syntactic-reanalysis}

Reanalysis can work on quite large structures.
Therefore the unit of transmission can be much larger than what we commonly conceive as words, or does it?
Before continuing our discussions on units of transmission in historical linguistics,
we need to first have a look at examples of \emph{syntactic} reanalysis,
and see whether they truly challenge our assumption above that
the basic unit of transmission is something in the list in \prettyref{sec:all-wordhood}.

Here is a good demonstration of the idea of reanalysis running on the level of embedded clauses.
In the history of Finnish, a sound change \form{-m} \textgt \form{-n} makes accusative singular and the genitive singular endings homophonous,
and this eventually leads to reanalysis of something like \form{I saw [him [doing sth.]_{\text{relative clause}}]_{\text{object: noun phrase}}} into 
\form{I saw [his doing sth.]_{\text{object: gerundive clause}}},
creating a new complement clause construction
\citep[\citepages{275-276}]{campbell2013historical}.

It turns out that syntactic reanalysis in this example happens \emph{silently}:
it creates no new forms on the surface. 
This should be \emph{expected}, as we do not expect syntactic reanalysis to create forms 
that are not acceptable at all in the parent language:
it is \emph{re}-analysis, and the two analyses pertaining to the same surface form
must both be valid when the reanalysis happens.
The only surface form change happens because of a regular sound change applied to word endings,
which \emph{triggers} the reanalysis, but is conceptually independent from it.

In this example, we may say that the whole clause structure -- with all the hierarchical syntax -- is the unit being transmitted.
But since the constituency relations are all well kept,
we can actually identify the single spot of syntactic change in this example:
it has to be the function of the genitive ending,
which now has the function of labeling the subject of a gerundive clause.

Auxiliaries are grammaticalized in the same way:
the starting point, i.e. a complement clause construction,
and the destination, i.e. an auxiliary construction, have comparable constituency structure:%
\footnote{
    \citet[\citepage{65}]{cgel} even calls object complement clauses \term{catenative}. 
}
the only difference is that the complement-taking verb loses its status as a content word.
Grammaticalization of auxiliaries differs from the emergence of the gerundive complement clause construction in Finnish
in that in the former, a new grammatical marker appears.
Because all the constituency relations are well kept before and after the syntactic change,
we can summarize the syntactic change into a single spot:
the grammatical status of a morphological word changes
from a verb to a marker of a tense or aspect category.

The emergence of auxiliaries creates no new \emph{forms}, either:
the auxiliaries are still independent morphological words,
the only differences of grammaticalization being 
the rigid hierarchy of auxiliaries, the change of the positions of adverbs, etc. (\prettyref{sec:abstract-syntax-hierarchy}).
But when the adverbs do not appear, the surface form remains the same.

The Latin particle \form{etiam} probably was formed along the line of 
\form{[et [iam]_{\text{focused}} [\dots]_{\text{nucleus clause}}]_{\text{coordinate clause}}} 
\textgt \form{[etiam \dots]_{\text{clause}}}.
Again, the surface form is never lost in the reanalysis.
What is being reanalyzed -- the sequence \form{et iam} -- 
actually form a syntactic ``word'' in the sense of \prettyref{sec:inflection-flatten-tree}:
they both belong to the peripheral system of a clause.
Similarly, when a verb root and a voice marker attached to it are being
reanalyzed into a synchronic root,
we notice that the surface form is not changed,
and what are being reanalyzed form a ``word'' in the sense of \prettyref{sec:inflection-flatten-tree}.

We can summarize our observations above into two.
First, syntactic reanalysis does not alternate the surface form,
although in certain circumstances, like in the formation of \form{etiam},
new formatives are created.
Second, syntactic reanalysis, despiting working on complex structures,
usually has a single locus:
we can always locate the change in one syntactic ``word'' defined in \prettyref{sec:inflection-flatten-tree}.
In the example of Finnish complement clause structure,
the function of the genitive case marker changes,
and we can say that the change is limited within the syntactic ``word'' formed by the 
head noun and the case marker of the noun phrase after the verb.
In the example of auxiliary grammaticalization,
the locus of change is the (inflected forms of) the complement-taking verb.
The same is the case for \form{etiam} and formation of new verb roots.
Everything else in the construction is left unchanged.

Therefore, syntactic reanalysis does not challenge our assumption at the beginning of \prettyref{sec:unit-of-transmission},
that is, the unit of language transmission and evolution is always word-like in \prettyref{sec:all-wordhood}.

\subsection{Morphological reanalysis}\label{sec:reanalysis-form-change}

In \prettyref{sec:syntactic-reanalysis},
we have discussed syntactic reanalysis which keeps the surface form of a construction
while changing its underlying syntax.
In this section, we consider morphological reanalysis,
which does not change the underlying syntactic structure
but alternates the forms of formatives in morphological realization of a construction.

\subsubsection{Introduction: rebracketing of short phrases}

In English we occasionally see \emph{rebracketed} phrases.
Thus we see \form{a nuncle}, which is a result of re-bracketing \form{an=uncle},
the phonological realization of a simple noun phrase with the underlying representation
\form{[a_{\text{\category{singular,indefinite}}} [uncle]_{\text{noun stem}}]_{\text{noun phrase}}},
where the indefinite, singular categories are realized as \form{a},
which, by virtue of appearing before a vowel, becomes its allomorph \form{an},
and then is attached to the noun stem \form{uncle}.
Another example is the emergence of \form{nickname},
which was formed by dropping the initial \form{a} in \form{an eke name}.

The main difference between formation of \form{nuncle} or \form{nickname}
and processes in (\prettyref{sec:syntactic-reanalysis})
is that in the former, the underlying syntax is kept the same:
the only thing that changes is the morphological exponent of the root meaning \translate{uncle}.

Turning to the question of the unit of transmission.
What is being transmitted and undergoes re-bracketing here is the whole sequence \form{an=uncle}.
Note that here boundaries of morphological words do not matter:
\form{an} is not a morphological clitic but a phonological one (\prettyref{sec:phonological-multiple-morphological-one-phonological}),
and therefore it always has a boundary with \form{uncle},
and if this boundary is respected here, the re-bracketing can never appear.
Therefore, morphological words are not the only units of transmission and evolution.

Further, it is highly unlikely that a construction with an arbitrary size can be morphologically reanalyzed:
like the case in \prettyref{sec:syntactic-reanalysis},
there is likely a locus of morphological reanalysis.
The unit of transmission here
is either some sort of phonological wordhood or syntactic ``wordhood''.

\subsubsection{Morphological change without linear order?}

We want to first exclude the feasibility of syntactic ``wordhood'' (as in \prettyref{sec:inflection-flatten-tree})
as a unit of morphological reanalysis \emph{on its own}.

A historical alternation of the \emph{realization} or surface form a flattened-tree syntactic word is something like a particle becoming an affix, 
something in the form of \form{[\textcolor{blue}{A} \textcolor{red}{\dots} \textcolor{blue}{Σ}]} \textgt \form{[\textcolor{red}{\dots} \textcolor{blue}{A-Σ}]},
where \form{A} is a grammatical marker of a construction headed by \form{Σ}.
But this can also be explained from the perspective of linear order:
modifications in the construction headed by \form{Σ}, by definition, do not obligatorily appear,
and it is possible that \form{[A Σ]}, being the most frequent surface form now, possibly forming a phonological word \form{[A-Σ]} (\prettyref{sec:phonological-word-multiple-morphological-word-evolution}), is the unit reanalyzed as a morphological word.

So we see there is a lot of noise here.
If \form{A}'s transformation into a prefix is always accompanied by \form{A} and \form{Σ} being a phonological word,
then what historical linguistic laws are working on is the phonological word,
not the syntactic ``word''.
If we want to demonstrate that the flattened-tree syntactic ``word'' defined in \prettyref{sec:inflection-flatten-tree}
is a unit that historical linguistic laws work on,
we need to demonstrate that \form{A} and \form{Σ}, being one flattened-tree syntactic ``word'', can evolve into a morphological word in one step,
even when the latter is never attested in the parent language.
For instance, a change in the form of \form{[\textcolor{blue}{A} \textcolor{red}{\dots} \textcolor{blue}{Σ}]} \textgt \form{[\textcolor{red}{\dots} \textcolor{blue}{Σ-A}]} (note the reversed order)
is a convincing piece of evidence suggesting that historical linguistic laws work on flattened-tree syntactic ``words'',
because no explanation on surface linear orders is possible.
Yet \form{[\textcolor{blue}{A} \textcolor{red}{\dots} \textcolor{blue}{Σ}]} \textgt \form{[\textcolor{red}{\dots} \textcolor{blue}{Σ-A}]} seems to be rare, if not impossible.
An instance of this is the imaginary evolution of a \emph{preposition} (which may be separated from the head noun by adjectives, etc.) into a case \emph{suffix}.
This trajectory seems to be never attested.

This is probably to be expected, as we do not expect a form unacceptable in the parent language to emerge overnight in language evolution,
even when the syntactic derivation of the form is quite similar to the underlying morphosyntax of the parent language.

\subsection{What kind of phonological word is eligible to reanalysis?}\label{sec:phonological-word-reanalysis-condition}

So now the only choice left to us is to assume that phonological words undergo morphological reanalysis.
Which makes sense, if we consider the formation of \form{etiam} in \prettyref{sec:syntactic-reanalysis}:
although \form{et iam} form a syntactic ``word'' in the sense of \prettyref{sec:inflection-flatten-tree},
they likely do not form a morphological word,
as the linear order \form{et iam} is a transparent reflection of the syntactic structure
(\prettyref{sec:morphological-wordhood-linear}).
The only type of unit \form{et} and \form{iam} can form is a phonological word,
which then is reanalyzed into a morphological word \form{etiam}.

Or we can consider how pronouns become agreement markers.
They are first unstressed, and then attach to the main verb in simplest clauses with only the main verb and the arguments to form a phonological word,
which is later reanalyzed as a morphological word and the personal pronouns are reanalyzed as agreement markers, meaning that the personal markers stop being clitics and start to be affixes,
which can appear even when noun phrase arguments are present.

The next question is, can all phonological words be morphologically reanalyzed?
In theory, any phonological word can be reanalyzed as a particle,
but a reanalysis with internal structures (which gives rise to things like \form{nuncle})
seems to intuitively requires the phonological word to be not too far from a morphological word
or a synchronic ``word'' in the sense of \prettyref{sec:inflection-flatten-tree}.
To test if this is true, we can try to find examples like \form{a nuncle} where we have materials after \form{uncle}:

\subsubsection{Morphological words without phonological wordhood}

In other cases, morphological words seem to be the unit of transmission.
Latin \form{=que} doesn't leave any traces on the evolution of nominal declension of Romance languages.


\subsection{Processes based on the phonological word}\label{sec:phonological-word-multiple-morphological-word-evolution}

Besides the formation of \form{etiam},
we have other examples \form{whodunit} 


\subsection{Extremely big morphological words}

Now we want to know if \emph{all} morphological words are units of transmission.
Small ones definitely do, but what about ultra-complicated ones with multiple slots?
The question ``are all morphological word units of transmission'' therefore is equivalent to 
``is complex morphology stable in language evolution.''
If long prefix chains and suffix chains are not stable at all

Still there is the possibility that what is being transmitted is actually
a phonological segment of the morphological word in question.

\section{Reconstruction based on content words}

Strikingly, all the subtleties mentioned above do not significantly hinder reconstruction works,
if we only focus on content words.

\subsection{Summary: patterns of change observed}

\begin{itemize}
    \item In \emph{syntactic} reanalysis and extension,
    the basic unit of transmission is a small object in a construction,
    possibly an affix or a grammatical word.
    What changes is the grammatical function, not the surface form \prettyref{sec:syntactic-reanalysis}.
    \item 
\end{itemize}

\section{Reconstruction based on grammatical markers}

\section{Possible artifacts in a reconstructed proto-language}

It is unlikely that a reconstruction proto-language of a language family
looks incredibly close to the real common ancestor of that family.
A reconstructed proto-language is a \emph{model} that explains
the similarities and differences we now observe in the daughter languages,
which (a) does not necessarily have the historically correct phonetic values,
and (b) likely conflates phenomena from different dialects and periods into one single system.
Some of the problems can be avoided by more careful reconstructions,
but reviving the historical common ancestor is generally speaking beyond the capacity of historical linguistics.

This means artifacts originating from techniques used in reconstruction,
not real historical information, appear in the reconstructed language.
Here we discuss some possible artifacts relevant to the problem of wordhood.

\subsection{Lack of function words}

Reconstructed languages generally do not have a ton of function words.
This is because function words are almost always results of grammaticalization of content words.
Take a look at function words in contemporary languages and you see grammaticalization.
Take a look at classical texts of dead languages -- and you still see grammaticalization.
Therefore, when a 

\subsection{Complex morphology}

It is possible that two affixes of a reconstructed word were never both highly productive historically.
Thus in a form like \form{Σ-A-B}, it is possible that \form{A} was once a productive derivational suffix,
but had fallen out of use when \form{B}, maybe originally a clitic, was finally grammaticalized.

Another problem is clitics being reconstructed as affixes?

\printbibliography

\end{document}