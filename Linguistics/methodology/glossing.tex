\documentclass{article}

\usepackage{geometry}
\usepackage{titling}
\usepackage{titlesec}
\usepackage{paralist}
\usepackage{float}
\usepackage{footnote}
\usepackage{enumerate}
\usepackage{amsmath, amssymb, amsthm}
\usepackage{gb4e}
\noautomath
\usepackage{bbm}
\usepackage{soul}
\usepackage{graphicx}
\usepackage{siunitx}
\usepackage[table,xcdraw]{xcolor}
\usepackage{tikz}
\usepackage[ruled, vlined, linesnumbered, noend]{algorithm2e}
\usepackage{xr-hyper}
\usepackage[colorlinks]{hyperref} % linkcolor=black, anchorcolor=black, citecolor=black, filecolor=black
\usepackage[most]{tcolorbox}
\usepackage{caption}
\usepackage{subcaption}
\usepackage{booktabs}
\usepackage{multirow}
\usepackage[figuresright]{rotating}
\usepackage{acro}
\usepackage[round]{natbib} 
\usepackage{nameref,zref-xr}
\zxrsetup{toltxlabel}
\zexternaldocument*[cgel-]{../English/cambridge}[cambridge.pdf]
\zexternaldocument*[chinese-]{../Chinese/main}[main.pdf]
\usepackage{prettyref}

\geometry{left=3.18cm,right=3.18cm,top=2.54cm,bottom=2.54cm}
\titlespacing{\paragraph}{0pt}{1pt}{10pt}[20pt]
\setlength{\droptitle}{-5em}

\DeclareMathOperator{\timeorder}{\mathcal{T}}
\DeclareMathOperator{\diag}{diag}
\DeclareMathOperator{\legpoly}{P}
\DeclareMathOperator{\primevalue}{P}
\DeclareMathOperator{\sgn}{sgn}
\newcommand*{\ii}{\mathrm{i}}
\newcommand*{\ee}{\mathrm{e}}
\newcommand*{\const}{\mathrm{const}}
\newcommand*{\suchthat}{\quad \text{s.t.} \quad}
\newcommand*{\argmin}{\arg\min}
\newcommand*{\argmax}{\arg\max}
\newcommand*{\normalorder}[1]{: #1 :}
\newcommand*{\pair}[1]{\langle #1 \rangle}
\newcommand*{\fd}[1]{\mathcal{D} #1}

\newcommand*{\citesec}[1]{\S~{#1}}
\newcommand*{\citechap}[1]{chap.~{#1}}
\newcommand*{\citefig}[1]{Fig.~{#1}}
\newcommand*{\citetable}[1]{Table~{#1}}
\newcommand*{\citefootnote}[1]{footnote~{#1}}

\newrefformat{sec}{\citesec{\ref{#1}}}
\newrefformat{fig}{\citefig{\ref{#1}}}
\newrefformat{tbl}{\citetable{\ref{#1}}}
\newrefformat{chap}{\citechap{\ref{#1}}}

\usetikzlibrary{arrows,shapes,positioning}
\usetikzlibrary{arrows.meta}
\usetikzlibrary{decorations.markings}
\tikzstyle arrowstyle=[scale=1]
\tikzstyle directed=[postaction={decorate,decoration={markings,
    mark=at position .5 with {\arrow[arrowstyle]{stealth}}}}]
\tikzstyle ray=[directed, thick]
\tikzstyle dot=[anchor=base,fill,circle,inner sep=1pt]


\tcbuselibrary{skins, breakable, theorems}

\newtcbtheorem[number within=chapter]{infobox}{Box}%
  {colback=blue!5,colframe=blue!65,fonttitle=\bfseries, breakable}{infobox}

\newcommand*{\concept}[1]{\textbf{#1}}
\newcommand*{\term}[1]{\emph{#1}}
\newcommand*{\corpus}[1]{\emph{#1}}

\newcommand*{\vP}{\textit{v}P}

\DeclareAcronym{blt}{short = BLT, long = Basic Linguistic Theory}
\DeclareAcronym{cgel}{short = CGEL, long = The Cambridge Grammar of the English Language}
\DeclareAcronym{dm}{short = DM, long = Distributed Morphology}
\DeclareAcronym{tag}{long = Tree-adjoining grammar, short = TAG}
\DeclareAcronym{sfp}{long = sentence final particle, short = SFP}
\DeclareAcronym{vp}{long = verb phrase, short = VP}
\DeclareAcronym{np}{long = noun phrase, short = NP}
\DeclareAcronym{adjp}{long = adjective phrase, short = AdjP}
\DeclareAcronym{advp}{long = adverb phrase, short = AdvP}
\DeclareAcronym{pp}{long = preposition phrase, short = PP}
\DeclareAcronym{cls}{long = classifier, short = CLS}
\DeclareAcronym{dist}{long = distal, short = DIST}
\DeclareAcronym{prox}{long = proximate, short = PROX}
\DeclareAcronym{dem}{long = demonstrative, short = DEM}
\DeclareAcronym{dur}{long = durative, short = DUR}
\DeclareAcronym{neg}{long = negative, short = NEG}
\DeclareAcronym{tam}{long = {Tense, Aspect, Mood}, short = TAM}
\DeclareAcronym{pie}{long = Proto-Indo-European, short = PIE}

% Disable unsupported commands in bookmark titles 
\pdfstringdefDisableCommands{%
  \def\\{}%
  \def\texttt#1{<#1>}%
  \def\mathbb#1{#1}%
}
\pdfstringdefDisableCommands{\def\eqref#1{(\ref{#1})}}

\makeatletter
\pdfstringdefDisableCommands{\let\HyPsd@CatcodeWarning\@gobble}
\makeatother

\newcommand{\cgel}{\href{../English/cambridge.pdf}{my notes about CGEL}}
\newcommand{\chinese}{\href{../Chinese/main.pdf}{my notes about Chinese syntax}}

\title{Glossing of descriptive terms, and how to write a grammar}
\author{Jinyuan Wu}

\begin{document}

\maketitle

\section{Theoretical orientation}

Modern descriptive grammars are usually carried out within the framework of \ac{blt} 
\citep{dixon2009basic1,dixon2010basic2,dixon2012basic3},
which, according to Dixon, deviates striking from the bond-to-fail generative approach,
though the two frameworks roughly describe the same grammatical complexity class.
The contemporary generative approach, or the Minimalist one, 
is based on features that are Merged and undergo certain morphophonological processes,
and \ac{blt} may be viewed as the surface-oriented dual theory of it.
(see \citechap{\ref{chinese-chap:theory}} in \chinese)

There are still further divergences within the \ac{blt} approach.
Here are some dimensions of divergence:
\begin{itemize}
    \item Is the theory purely lexicalist, 
    or are there syntactic templates?%
    \footnote{
        Sometimes the term \term{lexicalist} means the syntax works on words 
        and not sub-word units.
        This is not the meaning intended here. 
        The meaning intended here by \term{lexicalist} is 
        ``all grammatical rules can be reduced to how to use certain lexicon entries (lexical or functional)'',
        which may be words or morphemes or features.
        In other words, a lexicalist theory has no or few ``global'' phrase structure rules,
        as opposed to early generative grammars.
        This usage of the term \term{lexicalist} is attested in \citet{matchin2020cortical}. 
    }
    Though in a quick glance, it seems the lexicalist approach agrees with the Minimalist syntax 
    while the templatic approach agrees with the constructionism,
    things are not that simple:
    remember, a Minimalist syntax runs on features which are not directly visible,
    and words and morphemes are just quirky reflections of them.
    The corresponding surface-oriented version of a Minimalist syntax with lots of features 
    that are used to guide the syntactic derivation (e.g. the EPP feature), then,
    inevitably contains syntactic templates that are hard to place under any lexicon entry.
    The Cinque hierarchy of clause structure, for example, contains tons of invisible functional heads,
    and once we ``integrate out'' these functional heads,
    the resulting grammar has a clause template.
    The linguist has to consider whether to introduce 
    a chapter named ``the structure of noun phrases''
    or a chapter named ``the clausal structure''.

    \item How is morphology dealt?
    This parameter has strong association with the previous parameter, 
    since there is no clear distinction between a morpheme and a word.
    In morphology the lexicalist extreme is the Item-and-Arrangement approach,
    while the templatic extreme is the Word-and-Paradigm approach.
    The Item-and-Process approach is somehow in the middle, 
    maybe in a position closer to the former and further from the latter. 
    What brings in more complexity in morphology is 
    there are post-syntactic operations:
    even when the features do spellout into morphemes,
    the Distributed Morphology-style post-syntactic operations 
    blur the correspondence between features and morphemes,
    and hence the idea that words are built up by morphemes 
    does not lead to any constraints on the form of the word,
    raising doubts about whether in a surface-oriented analysis,
    morphemes are of any theoretical significance at all \citep{anderson2017words}.
    The linguist needs to pick up a specific way to show how words are built up.

    \item Top-down (i.e. structuralist partition-based), 
    or bottom-up (i.e. based on the usage of smaller units)? 
    In PSGs there is a clear correspondence between the two, 
    but for actual language documentation things are often complicated.

    \item How are grammatical relations (in other words, dependency relations) introduced? 
    Together with morphemes that bear them, or words, or constituents, or with separate chapters and sections?
    This parameter has certain correlation with the top-down/bottom-up parameter,
    because in a top-down analysis,
    the grammatical functions of constituents in a larger construction 
    are obviously introduced before what fill the constituent slots are discussed.
    On the other hand, 
    a bottom-up grammar tends to introduce grammatical relations when discussing the smallest unit that bear them,
    for example talking about the case marking of various complements in the noun morphology chapter.

    \item What is the relation between a phrase and words contained in it?
    What is the head? What are the complements? What are the modifiers?
    In Minimalist syntax, all functional categories serve as heads,
    but lexical categories are never heads.
    This may appear strange but has underlying consistency 
    (see \citesec{\ref{chinese-sec:headedness}} in \chinese).
    This approach, however, is not acceptable for a surface-oriented grammar,
    and here another concept -- what determines the ``overall'' property of a constituent -- 
    is accepted as the standard to decide what is the head. 
    Thus a \term{n}P and a DP are all headed by the central noun in the surface-oriented analysis,
    because both of them are built surrounding the core noun stem,
    and since the core noun stem is phonetically realized as the central noun -- a lexical word --
    the latter is recognized as the head.
    Disagreements then arise when whether a word is functional or lexical is not that certain.
    Should the preposition be considered as a head? 
    The preposition in a peripheral argument may be seen as the marker of a syntactic case system 
    (so in the generative analysis, we have PP and CaseP),
    and under this analysis, the preposition is not a head.
    But in many languages like English, 
    the preposition category has certain predicative properties,
    making it appear like a lexical category, 
    and then it seems a \ac{np} with a preposition is no longer a \ac{np} -- 
    it is a \ac{pp} headed by the preposition.
    
    \item Are there fine-grained constituency structures, or are there just \acl{np}s and clauses?
    Some grammars, like the \ac{cgel} \citep{cgel}, 
    posits an anatomy of \ac{np}s with the following functional domains:
    head noun -- nominal -- minimal \ac{np} with a determiner -- external modifiers.
    Others just list possible \ac{np} dependents or clausal dependents,
    without discussing which is closer to the head. 
    If the latter approach is taken,
    the linguist has to introduce effects due to the relative position of constituents in another way,
    like ``the O argument in ergative languages is more topic-like''.
    
    \item How is constituent order (often called \term{word order}) introduced?
    Constituent order can be understood as a manifestation of constituent hierarchy,
    while in more functionalist approaches, 
    it is understood as a method parallel to morphological marking that 
    marks the constituent positions in a larger construction.
    Note that the second claim does not go against the first one:
    certain features are indeed reflected by the surface constituent order in generative syntax,
    Certain ideas in the first account that do not involve features (e.g. Antisymmetry)
    cannot be translated transparently back into the second approach, though,
    but they can be framed in the second approach as 
    ``the human language faculty just rejects certain constituent orders anyway''.%
    \footnote{
        One controversy here is the generative feature-driven constituent order often involves movement,
        while functionalists accept constituent order variations ``as they are''.
        This controversy is false, because for many generative linguists, 
        movements can be unmarked, and what movement means is simply 
        dual syntactic function of a constituent 
        or the imperfect relation between constituent order and dependency relations
        (e.g. cross-serial dependencies).
    }
    
    \item Whether a set of canonical constructions is established.
    Viewing non-canonical constructions as transformed from canonical ones (or by adjunction, etc.) 
    is a powerful descriptive tool,
    but it is often the case that certain constructions 
    that are uncontroversially deemed non-canonical do not have a canonical counterpart.
    This is one of the reason transformational rules are finally abandoned in generative syntax.
    Transformational rules (and adjunction, etc.) are still handy when doing description, though:
    no one wants to read a grammar that . % TODO: 用voice做例子
\end{itemize}

In principle, the above parameters are free to choose.
In practice, they have to be fine-tuned or otherwise the grammar will be hard to read.
If a linguist unfortunately decides to write a Latin grammar 
in a top-down, 

\bibliographystyle{plainnat}
\bibliography{typology,famous-grammars}

\end{document}