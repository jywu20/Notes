\documentclass{article}

\usepackage{geometry}
\usepackage{titling}
\usepackage{titlesec}
\usepackage{paralist}
\usepackage{float}
\usepackage{footnote}
\usepackage{marginnote}
\usepackage{enumerate}
\usepackage{amsmath, amssymb, amsthm}
\usepackage{gb4e}
\noautomath
\usepackage{bbm}
\usepackage{soul}
\usepackage{graphicx}
\usepackage{siunitx}
\usepackage[table,xcdraw]{xcolor}
\usepackage{tikz}
\usepackage[ruled, vlined, linesnumbered, noend]{algorithm2e}
\usepackage{xr-hyper}
\usepackage[colorlinks]{hyperref} % linkcolor=black, anchorcolor=black, citecolor=black, filecolor=black
\usepackage[most]{tcolorbox}
\usepackage{caption}
\usepackage{subcaption}
\usepackage{booktabs}
\usepackage{multirow}
\usepackage[figuresright]{rotating}
\usepackage{acro}
\usepackage[round]{natbib} 
\usepackage{nameref,zref-xr}
\zxrsetup{toltxlabel}
\zexternaldocument*[cgel-]{../English/cambridge}[cambridge.pdf]
\zexternaldocument*[chinese-]{../Chinese/main}[main.pdf]
\zexternaldocument*[latin-]{../Latin/latin-notes}[latin-notes.pdf]
\usepackage{prettyref}

\geometry{left=3.18cm,right=3.18cm,top=2.54cm,bottom=2.54cm}
\titlespacing{\paragraph}{0pt}{1pt}{10pt}[20pt]
\setlength{\droptitle}{-5em}

\DeclareMathOperator{\timeorder}{\mathcal{T}}
\DeclareMathOperator{\diag}{diag}
\DeclareMathOperator{\legpoly}{P}
\DeclareMathOperator{\primevalue}{P}
\DeclareMathOperator{\sgn}{sgn}
\newcommand*{\ii}{\mathrm{i}}
\newcommand*{\ee}{\mathrm{e}}
\newcommand*{\const}{\mathrm{const}}
\newcommand*{\suchthat}{\quad \text{s.t.} \quad}
\newcommand*{\argmin}{\arg\min}
\newcommand*{\argmax}{\arg\max}
\newcommand*{\normalorder}[1]{: #1 :}
\newcommand*{\pair}[1]{\langle #1 \rangle}
\newcommand*{\fd}[1]{\mathcal{D} #1}

\newcommand*{\citesec}[1]{\S~{#1}}
\newcommand*{\citechap}[1]{chap.~{#1}}
\newcommand*{\citefig}[1]{Fig.~{#1}}
\newcommand*{\citetable}[1]{Table~{#1}}
\newcommand*{\citefootnote}[1]{footnote~{#1}}

\newrefformat{sec}{\citesec{\ref{#1}}}
\newrefformat{fig}{\citefig{\ref{#1}}}
\newrefformat{tbl}{\citetable{\ref{#1}}}
\newrefformat{chap}{\citechap{\ref{#1}}}
\newrefformat{infobox}{Box~\ref{#1}}

\usetikzlibrary{arrows,shapes,positioning}
\usetikzlibrary{arrows.meta}
\usetikzlibrary{decorations.markings}
\tikzstyle arrowstyle=[scale=1]
\tikzstyle directed=[postaction={decorate,decoration={markings,
    mark=at position .5 with {\arrow[arrowstyle]{stealth}}}}]
\tikzstyle ray=[directed, thick]
\tikzstyle dot=[anchor=base,fill,circle,inner sep=1pt]


\tcbuselibrary{skins, breakable, theorems}

\newtcbtheorem[number within=section]{infobox}{Box}%
  {colback=blue!5,colframe=blue!65,fonttitle=\bfseries, breakable}{infobox}

\newcommand*{\concept}[1]{\textbf{#1}}
\newcommand*{\term}[1]{\emph{#1}}
\newcommand*{\corpus}[1]{\emph{#1}}

\newcommand*{\vP}{\textit{v}P}

\DeclareAcronym{blt}{short = BLT, long = Basic Linguistic Theory}
\DeclareAcronym{cgel}{short = CGEL, long = The Cambridge Grammar of the English Language}
\DeclareAcronym{dm}{short = DM, long = Distributed Morphology}
\DeclareAcronym{tag}{long = Tree-adjoining grammar, short = TAG}
\DeclareAcronym{sfp}{long = sentence final particle, short = SFP}
\DeclareAcronym{vp}{long = verb phrase, short = VP}
\DeclareAcronym{np}{long = noun phrase, short = NP}
\DeclareAcronym{adjp}{long = adjective phrase, short = AdjP}
\DeclareAcronym{advp}{long = adverb phrase, short = AdvP}
\DeclareAcronym{pp}{long = preposition phrase, short = PP}
\DeclareAcronym{cls}{long = classifier, short = CLS}
\DeclareAcronym{dist}{long = distal, short = DIST}
\DeclareAcronym{prox}{long = proximate, short = PROX}
\DeclareAcronym{dem}{long = demonstrative, short = DEM}
\DeclareAcronym{dur}{long = durative, short = DUR}
\DeclareAcronym{neg}{long = negative, short = NEG}
\DeclareAcronym{tam}{long = {Tense, Aspect, Mood}, short = TAM}
\DeclareAcronym{pie}{long = Proto-Indo-European, short = PIE}

% Disable unsupported commands in bookmark titles 
\pdfstringdefDisableCommands{%
  \def\\{}%
  \def\texttt#1{<#1>}%
  \def\mathbb#1{#1}%
}
\pdfstringdefDisableCommands{\def\eqref#1{(\ref{#1})}}

\makeatletter
\pdfstringdefDisableCommands{\let\HyPsd@CatcodeWarning\@gobble}
\makeatother

\newcommand{\cgel}{\href{../English/cambridge.pdf}{my notes about CGEL}}
\newcommand{\chinese}{\href{../Chinese/main.pdf}{my notes about Chinese syntax}}
\newcommand{\latin}{\href{../Latin/latin-notes.pdf}{my notes about Latin}}

\title{Glossing of descriptive terms, and how to read a grammar}
\author{Jinyuan Wu}

\begin{document}

\maketitle

\section{Theoretical orientation}\label{sec:theory}

Modern descriptive grammars are usually carried out within the framework of \ac{blt} 
\citep{dixon2009basic1,dixon2010basic2,dixon2012basic3},
which, according to Dixon, deviates striking from the bond-to-fail generative approach,
though the two frameworks roughly describe the same grammatical complexity class.
The contemporary generative approach, or the Minimalist one, 
is based on features that are Merged and undergo certain morphophonological processes,
and \ac{blt} may be viewed as the surface-oriented dual theory of it.
(see \citechap{\ref{chinese-chap:theory}} in \chinese)

There are still further divergences within the \ac{blt} approach.
Here are some dimensions of divergence:
\begin{itemize}
    \item Is the theory purely lexicalist, 
    or are there syntactic templates?%
    \footnote{
        Sometimes the term \term{lexicalist} means the syntax works on words 
        and not sub-word units.
        This is not the meaning intended here. 
        The meaning intended here by \term{lexicalist} is 
        ``all grammatical rules can be reduced to how to use certain lexicon entries (lexical or functional)'',
        which may be words or morphemes or features.
        In other words, a lexicalist theory has no or few ``global'' phrase structure rules,
        as opposed to early generative grammars.
        This usage of the term \term{lexicalist} is attested in \citet{matchin2020cortical}. 
    }
    Though in a quick glance, it seems the lexicalist approach agrees with the Minimalist syntax 
    while the templatic approach agrees with the constructionism,
    things are not that simple:
    remember, a Minimalist syntax runs on features which are not directly visible,
    and words and morphemes are just quirky reflections of them.
    The corresponding surface-oriented version of a Minimalist syntax with lots of features 
    that are used to guide the syntactic derivation (e.g. the EPP feature), then,
    inevitably contains syntactic templates that are hard to place under any lexicon entry.
    The Cinque hierarchy of clause structure, for example, contains tons of invisible functional heads,
    and once we ``integrate out'' these functional heads,
    the resulting grammar has a clause template.
    The linguist has to consider whether to introduce 
    a chapter named ``the structure of noun phrases''
    or a chapter named ``the clausal structure''.

    \item How is morphology dealt?
    This parameter has strong association with the previous parameter, 
    since there is no clear distinction between a morpheme and a word.
    In morphology the lexicalist extreme is the Item-and-Arrangement approach,
    while the templatic extreme is the Word-and-Paradigm approach.
    The Item-and-Process approach is somehow in the middle, 
    maybe in a position closer to the former and further from the latter. 
    What brings in more complexity in morphology is 
    there are post-syntactic operations:
    even when the features do spellout into morphemes,
    the Distributed Morphology-style post-syntactic operations 
    blur the correspondence between features and morphemes,
    and hence the idea that words are built up by morphemes 
    does not lead to any constraints on the form of the word,
    raising doubts about whether in a surface-oriented analysis,
    morphemes are of any theoretical significance at all \citep{anderson2017words}.
    The linguist needs to pick up a specific way to show how words are built up.

    \item How are grammatical relations (in other words, dependency relations) introduced? 
    Together with morphemes that bear them, or words, or constituents, or with separate chapters and sections?
    This parameter has certain correlation with the top-down/bottom-up parameter,
    because in a top-down analysis,
    the grammatical functions of constituents in a larger construction 
    are obviously introduced before what fill the constituent slots are discussed.
    On the other hand, 
    a bottom-up grammar tends to introduce grammatical relations when discussing the smallest unit that bear them,
    for example talking about the case marking of various complements in the noun morphology chapter.

    \item What is the relation between a phrase and words contained in it?
    What is the head? What are the complements? What are the modifiers?
    In Minimalist syntax, all functional categories serve as heads,
    but lexical categories are never heads.
    This may appear strange but has underlying consistency 
    (see \citesec{\ref{chinese-sec:headedness}} in \chinese).
    This approach, however, is not acceptable for a surface-oriented grammar,
    and here another concept -- what determines the ``overall'' property of a constituent -- 
    is accepted as the standard to decide what is the head. 
    Thus a \term{n}P and a DP are all headed by the central noun in the surface-oriented analysis,
    because both of them are built surrounding the core noun stem,
    and since the core noun stem is phonetically realized as the central noun -- a lexical word --
    the latter is recognized as the head.
    Disagreements then arise when whether a word is functional or lexical is not that certain.
    Should the preposition be considered as a head? 
    The preposition in a peripheral argument may be seen as the marker of a syntactic case system 
    (so in the generative analysis, we have PP and CaseP),
    and under this analysis, the preposition is not a head.
    But in many languages like English, 
    the preposition category has certain predicative properties,
    making it appear like a lexical category, 
    and then it seems a \ac{np} with a preposition is no longer a \ac{np} -- 
    it is a \ac{pp} headed by the preposition.
    
    \item Are there fine-grained constituency structures, or are there just \acl{np}s and clauses?
    Some grammars, like the \ac{cgel} \citep{cgel}, 
    posits an anatomy of \ac{np}s with the following functional domains:
    head noun -- nominal -- minimal \ac{np} with a determiner -- external modifiers.
    Others just list possible \ac{np} dependents or clausal dependents,
    without discussing which is closer to the head. 
    If the latter approach is taken,
    the linguist has to introduce effects due to the relative position of constituents in another way,
    like ``the O argument in ergative languages is more topic-like''.
    The main reason to take the latter approach -- which is the approach advocated in \ac{blt} --
    is only \ac{np}s and clauses have complete semantic significance.
    See \ac{blt} \citesec{1.11}, (33) and (34):
    Dixon does not like the binary-branching (Minimalist) approach (33),
    because it does not illustrate the fact that the functional words are different from lexical ones. 
    But this is more a problem of terminology:
    the term \term{phrase} in \ac{blt} corresponds to a maximal domain like DP or CP in generative syntax,
    while a generative \term{phrase} -- like \term{v}P or AdvMannerP -- 
    corresponds to a grammatical construction in \ac{blt}.
    
    \item How is constituent order (often called \term{word order}) introduced?
    Is there a separate chapter devoted to constituent order?
    Constituent order can be understood as a manifestation of constituent hierarchy,
    while in more functionalist approaches, 
    it is understood as a method parallel to morphological marking that 
    marks the constituent positions in a larger construction.
    Note that the second claim does not go against the first one:
    certain features are indeed reflected by the surface constituent order in generative syntax,
    Certain ideas in the first account that do not involve features (e.g. Antisymmetry)
    cannot be translated transparently back into the second approach, though,
    but they can be framed in the second approach as 
    ``the human language faculty just rejects certain constituent orders anyway''.%
    \footnote{
        One controversy here is the generative feature-driven constituent order often involves movement,
        while functionalists accept constituent order variations ``as they are''.
        This controversy is false, because for many generative linguists, 
        movements can be unmarked, and what movement means is simply 
        dual syntactic function of a constituent 
        or the imperfect relation between constituent order and dependency relations
        (e.g. cross-serial dependencies).
    }

    \item Top-down (i.e. structuralist partition-based), 
    or bottom-up (i.e. based on the usage of smaller units)? 
    In PSGs there is a clear correspondence between the two, 
    but for actual language documentation things are often complicated:
    a top-down grammar is awkward to write 
    because the author has to enumerate all possible configurations in a construction 
    to fully characterize it
    (``a clause is either coordination of clauses or a subject-predicate construction''
    -- oh no, supplementation and pre-nucleus constructions are forgotten),
    while a bottom-up grammar is awkward to read 
    because the reader has to infer all possible configurations in a construction 
    (``the verb is the prototypical content of the predicate slot''
    -- any other possibilities? Nobody knows).
    This parameter is in principle orthogonal to the parameter about how constituent order is introduced,
    but a bottom-up grammar without a chapter (or several chapters) devoted to constituent order 
    will be extremely hard to read:
    the reader may find a sentence like ``the object follows the verb'' in the chapter about verbs.
    Alright, can an adverb intervenes between the verb and the object? No answer.

    \item Whether a set of canonical constructions is established.
    Viewing non-canonical constructions as transformed from canonical ones (or by adjunction, etc.) 
    is a powerful descriptive tool,
    but it is often the case that certain constructions 
    that are uncontroversially deemed non-canonical do not have a canonical counterpart.
    This is one of the reason transformational rules are finally abandoned in generative syntax.
    Transformational rules (and adjunction, etc.) are still handy when doing description, though:
    no one wants to read a grammar that treats positive clauses and negative clause in the same way.
\end{itemize}

In principle, the above parameters are free to choose.
In practice, they have to be fine-tuned or otherwise the grammar will be hard to read.
If a linguist unfortunately decides to write a Chinese grammar 
in a bottom-up manner
in which a grammatical relation is introduced in the chapter about the lexical category about its head,
a reader will soon be stuck in questions like 
what are the possible linear order between object(s), directional complement, and aspectual markers.

\section{Best practices of grammar writing}

% TODO: a table about famous grammars

\subsection{Organization of chapters}

\section{Top-down partition of the clause structure}

\subsection{The constituency tree}

This section gives a purely form-based analysis of clause structure.
Just like when discussing morphology,
we often first show possible morphological devices 
and then discuss what grammatical categories are marked by these devices,
in this section, I first discuss how to give an immediate constituent analysis of a clause,
and then in other sections about how to interpret the constituency tree obtained.

\subsubsection{A clause is built up by one or more nuclei with certain syntactic processes}

The top-level partition of a clause is given as the follows:
\begin{exe}
    \ex\label{ex:clause-def-1} A \concept{clause} is
    \begin{itemize}
        \item the coordination of two clauses (\prettyref{sec:clause-coord}),
        which may involve ellipsis in and/or movement out of the conjuncts, or
        \item a clause with supplementation (\prettyref{sec:clause-supp}), or
        \item a clause without the two.
    \end{itemize}
    \ex\label{ex:clause-def-2} A clause without coordination or supplementation is 
    \begin{itemize}
        \item a clause with pre- or post-nucleus constructions
        (the residue of the nucleus clause undergoing relevant syntactic processes 
        is named the \concept{nucleus}), 
        like the English subject-auxiliary verb inversion or \term{wh}-movement, or 
        \item a nucleus clause (see \eqref{ex:nucleus-def}).
    \end{itemize}
\end{exe}
Note that the distinction between 
coordination and adjunct clause construction (a type of subordination)
may be not so clear for some languages, 
for example Latin (see \latin, \citesec{\ref{latin-sec:subordination-abs}}).
Also, there is no strict application order 
between coordination, supplementation, and pre- and post-nucleus constructions:
in the English question
\corpus{on that particular day -- I mean the day when the unfortunate incident happened -- 
did you pass that site or hear anything usual in that direction},
first a coordination construction is used, 
followed by a subject-auxiliary verb inversion (a pre-nucleus construction)
and then supplementation 
and finally another pre-nucleus construction (topicalization of the time adjunct).
Another remark here is the syntactic processes from nucleus clauses to more complicated ones 
may only work for certain inputs:
in English, for example, the supplementation \corpus{not even \dots} 
is only possible for a clause in negative voice.

\subsubsection{Clausal dependents in the nucleus}

Now it is time to define the nucleus clause:
\begin{exe}
    \ex\label{ex:nucleus-def} A \concept{nucleus clause} is 
    \begin{itemize}
        \item a minimal nucleus clause, or 
        \item a nucleus clause with adjunction.
    \end{itemize}
    \ex\label{ex:minimal-nucleus-def} A \concept{minimal nucleus clause} includes 
    \begin{itemize}
        \item the \concept{predicator}, 
        prototypically a verb but with possible alternatives,
        possibly marked for grammatical categories involved in the clause structure, and 
        \item one or more visible or invisible \concept{complements}.
    \end{itemize}
\end{exe}
Here \concept{adjunction} means adding \concept{adjuncts} into the tree structure, 
in the manner in \ac{tag}.
This is the surface-oriented counterpart of optional projections in Cinque hierarchies.
Adjuncts are contrasted with complements,
the latter being somehow closer to the predicator:

The term \term{adjunct} used in this note means clausal modifier.
\term{Adjunct}, in generative syntax, 
means optional non-head components of any projection,
though nowadays, especially in the Syntactic Cartography program, 
it is often assume that there is no adjoin operation beside the usual Merge,
and so-called adjuncts are specifiers of certain optional functional heads,
and hence the term \term{adjunct} loses its structural significance.
Many descriptive grammars, like \citep{quirk2010comprehensive}, 
use the term \term{adverbial} for the term \term{adjunct}.
A third name used for adjuncts are \term{peripheral argument} in \ac{blt}.

The term \term{complement} may sometimes be used to denote specifically \term{complement clauses}. % TODO: ref
In \ac{blt}, the term \term{complement} is usually replaced by \term{core arguments}.

\ac{cgel} insists on a strict form-function distinction 
and hence the term \term{argument} is reserved for semantics.
\ac{blt}, on the other hand, emphasizes on the semantic basis of syntax, 
and so the term \term{argument} is used.
But here comes a subtle difference between \ac{blt}'s standard of clausal dependents and \ac{cgel}'s:
certain constituents, like the direction complement in Mandarin Chinese 
(see \citesec{\ref{chinese-sec:direction-complement}} in \chinese),
are definitely complements under the standard of \ac{cgel},
but are definitely not arguments, 
and hence they are not recognized as clausal complements in \ac{blt}
-- they are thus recognized as a part of the \ac{blt} predicate (see \prettyref{infobox:term-predicate}).

There 

\begin{infobox}{Terminology: adjunct, adverbial, complement, arguments}{clausal-dependent}
    There are two rough 
\end{infobox}

\subsubsection{Pre- and post-nucleus constructions not well defined for free-order languages}\label{sec:free-order-blt}

It should be noted that for languages with a relatively free constituent order,
it is almost impossible to find a neutral order, 
and hence pre-nucleus and post-nucleus constructions 
cannot be well-defined,
let along the fact that some linguists posit so-called in-VP scrambling
and the pronominal argument construction for radical non-configurational languages
where argument NPs are actually adjuncts,
which are by no means pre- or post-nucleus constructions 
but nonetheless induces changes in the constituent order.
In this case, \eqref{ex:clause-def-2} and \eqref{ex:nucleus-def} should be merged together,
and notions like pre- and post-nucleus constructions are to be replaced by 
discussions on the relation between constituent order and semantic and pragmatic information.

\subsubsection{The inner structure of the nuclues clause for syntactically accusative languages}

\eqref{ex:minimal-nucleus-def} is a flat-tree analysis, 
but there are several evidences suggesting 
a fine-grained hierarchy is useful even for surface-oriented analysis.
For accusative languages, 
the S and A arguments are and hence are identified as the \term{subject}, 
and we have the following facts:
\begin{itemize}
    \item The subject is much easier to be extracted out of the nucleus,
    which can be explained by the theory that 
    it is somehow higher and movement operations are localized.
    \item The quantifiers of the subject and internal complements, 
    explicit or implicit, 
    demonstrate a stable scope hierarchy:
    the scope of the subject quantifier is always larger.
    When talking about a charity organization,
    one may say \corpus{every woman helps three boys}.
    Here, the subject is bounded by $\forall$ and the object is bounded by `there exists three \dots',
    and $\forall$ $>$ \corpus{three} and *\corpus{three} $>$ $\forall$:
    the meaning of the sentence aforementioned is 
    `for each woman, there are three boys that she helps,
    but I do not know who they are,
    and possibly the boys Sarah helps are not the boys Lily helps'.
    After a seemingly trivial passivization, 
    we get \corpus{three boys are helped by every woman},
    which means 
    `there are three boys -- I don't know who, but anyway there are three -- 
    who are helped by every woman in our organization',
    and we have \corpus{three} $>$ $\forall$ and *$\forall$ $>$ \corpus{three}.
    If we assume the semantics is related to the syntactic structure at least partially,
    then this is a piece of evidence that the subject is higher in the syntactic tree,
    no matter what its semantic role is. 
    \item If the subject is indefinite, then it is by default bounded by $\forall$, TODO: really???
    
    Some notes about \ac{blt} \citechap{13}, Appendix 1: 
    TODO: S argument and A argument are by default bounded by $\forall$,
    while O is bounded by $\exists$ -- is this cross-linguistically correct?
    This also explains why verb-object incorporation is frequent:
    \corpus{a cat kills some animals} = \corpus{a cat kills}.
    It seems the only argument -- be it peripheral or core -- 
    that is by default bounded by $\forall$ is S in intransitive clauses and A in transitive clauses
    (which may be seen as a double check ).
    What's the counterpart in syntactic ergative languages?
    \item Verb-argument incorporation, nominalization, etc. 
    (for example compare \corpus{solve problem} and \corpus{problem solving}) 
    usually happens between the verb and the internal complement(s),
    not between the verb and the subject.%
    \footnote{
        Grammaticalization of a span is also in principle possible,
        so incorporation between the verb and the subject 
        may still rarely occur.
        But note that operation on span is usually seen for functional hierarchies,
        in which what are spellout as a single word 
        are highly lightweight functional heads, 
        not more substantial lexical categories.
    }
    \item If there is something looking like reflexive pronouns, 
    then it usually follows the Government and Binding scheme,
    and using this as a test,
    the subject is always predicted to occupy a higher position.
\end{itemize}
The list can go on and on, and hence it is useful to divide the nucleus clause into 
the subject and the predicate:
\begin{exe}
    \ex A nucleus clause is made up by an \concept{external complement}, often named the \concept{subject},
    and a predicate.
    \ex A predicate is made up by 
    \begin{itemize}
        \item a predicator (which is the head of the predicate and the clause) 
        and its \concept{internal complements}, 
        in this case it is a minimal predicate, or
        \item a predicate with adjunction.
    \end{itemize}
\end{exe}
The above rules replace \eqref{ex:nucleus-def} and \eqref{ex:minimal-nucleus-def}.

It should be noted the above concept of \term{predicate} does not always correspond to 
an uncontroversially constituent in the surface structure:
in a VSO language, for example, the predicate is discontinuous.
This urges some to accept a flat-tree approach to describe the nucleus predicate.

TODO: serial verb construction, 

There are some controversies arising from the ``what is the head'' parameter in \prettyref{sec:theory}.
Some verbs are auxiliary verbs.
In the clause \corpus{I should do this},
what is the predicator?
Here we are in the same delimma as the one concerning ``preposition phrase''.
An analysis in which the predicator is \corpus{should} will face the criticism 
that functional words are never heads in a surface-oriented analysis,
or otherwise, in order to be self-consistent,
its bound morpheme counterparts should also be regarded as heads,
which falls back to the generative functional head analysis.
The analysis in which the main verb \corpus{do} is the predicator 
usually occurs with the flat-tree approach,
or otherwise \corpus{should} is analyzed as a dependent similar to the determiner in a \ac{np},
which is acceptable in the structuralist analysis of Chinese non-argument complements 
(see \citechap{\ref{chinese-chap:non-argument-complement}} in \chinese)
but is not prevalent outside the Chinese grammar community.
Also, the analysis that the main verb \corpus{do} is the main verb 
also faces a problem of non-consistency:
the boundary between auxiliary verbs and lexical catenative verbs 
is somehow vague in some languages, 
and in this case, in somewhere in the grammaticalization process the head status 
suddenly flip from verb to another.


The verb complex \citep{hockett1948potawatomi,Friesen2017,Wilbur2014}

\begin{infobox}{Terminology: predicate, predicator, and verb complex}{term-predicate}
    \ac{cgel} does not acknowledge the role of verb complex,
    which is in principle correct, 
    because English is already highly analytic and rigid-order,
    and most information about dependency relations can be reconstructed 
    from the surface-oriented constituency analysis.
    Therefore, the term \term{predicate} is used as in traditional grammar:
    it means the nucleus minus the subject, 
    and it is a constituent.
    The predicator is the head of the predicate, which is always filled by a verb in English.

    In flat-tree approach grammars, i.e. \ac{blt}, 
    the 
\end{infobox}

Parameter: the flat-tree analysis in \ac{blt} means the term \term{predicate} 

TODO: what is included in the verb complex? Auxiliary verb, serial verb construction, of course,
but what about lexical catenative verbs?

\begin{infobox}{Summary of the immediate constituent analysis of clause structure}{summary-clause}
    
\end{infobox}

\subsection{Alignment of core arguments}

\concept{Alignment} means 

\section{Top-down partition of the \acl{np}}

\section{Lexical categories}

\subsection{The noun category}

\subsection{The verb category}

\subsubsection{What is recorded in the dictionary entry of a verb}

Here is a list of what needs to be described in the dictionary entry of a verb
if the dictionary is expected to provide full information instructing 
how to build a sentence from words:
\begin{itemize}
    \item Grammatical categories marked on the verb and 
\end{itemize}

\subsubsection{Semantic classification}

\subsection{Distinguishing nouns and verbs}

\subsection{The adjective category (or categories)}

\section{Morphology}

\section{Coordination}

\subsection{Clausal coordination}\label{sec:clause-coord}

\subsection{Coordination in \ac{np}s}

\section{Supplementation}\label{sec:clause-supp}

\bibliographystyle{plainnat}
\bibliography{typology,famous-grammars}

\end{document}