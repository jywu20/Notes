\PassOptionsToPackage{table}{xcolor}
\documentclass[a4paper, oneside, 12pt]{report}

\usepackage[T1]{fontenc}
\usepackage{libertinus}
\usepackage{geometry}
\usepackage{float}
\usepackage{titling}
\usepackage{titlesec}
\usepackage{paralist}
\usepackage{footnote}
\usepackage[colorinlistoftodos]{todonotes}
\usepackage[inline]{enumitem}
\usepackage{amsmath, amsthm}
\usepackage{gb4e}
\noautomath
\usepackage{bbm}
\usepackage{textcomp}
\usepackage{soul}
\usepackage{graphicx}
\usepackage{siunitx}
\usepackage{tikz}
\usepackage[ruled, vlined, linesnumbered, noend]{algorithm2e}
\usepackage[colorlinks, citecolor = purple, bookmarksnumbered]{hyperref} % linkcolor=black, anchorcolor=black, citecolor=black, filecolor=black
\usepackage[most]{tcolorbox}
\usepackage{caption}
\usepackage{subcaption}
\usepackage{booktabs}
\usepackage{multirow}
\usepackage[figuresright]{rotating}
\usepackage{acro}
\usepackage[round]{natbib} 
\usepackage{prettyref}

\geometry{left=3.18cm,right=3.18cm,top=2.54cm,bottom=2.54cm}
\titlespacing{\paragraph}{0pt}{1pt}{10pt}[20pt]
\setlength{\droptitle}{-5em}

\DeclareMathOperator{\timeorder}{\mathcal{T}}
\DeclareMathOperator{\diag}{diag}
\DeclareMathOperator{\legpoly}{P}
\DeclareMathOperator{\primevalue}{P}
\DeclareMathOperator{\sgn}{sgn}
\newcommand*{\ii}{\mathrm{i}}
\newcommand*{\ee}{\mathrm{e}}
\newcommand*{\const}{\mathrm{const}}
\newcommand*{\suchthat}{\quad \text{s.t.} \quad}
\newcommand*{\argmin}{\arg\min}
\newcommand*{\argmax}{\arg\max}
\newcommand*{\normalorder}[1]{: #1 :}
\newcommand*{\pair}[1]{\langle #1 \rangle}
\newcommand*{\fd}[1]{\mathcal{D} #1}

\newcommand*{\citesec}[1]{\S~{#1}}
\newcommand*{\citechap}[1]{Ch~{#1}}
\newcommand*{\citefig}[1]{Fig.~{#1}}
\newcommand*{\citetable}[1]{Table~{#1}}
\newcommand*{\citepage}[1]{p.~{#1}}
\newcommand*{\citepages}[1]{pp.~{#1}}
\newcommand*{\citefootnote}[1]{fn.~{#1}}
\newcommand*{\citechapsec}[2]{\citechap{#1}.\citesec{#2}}

\newrefformat{sec}{\citesec{\ref{#1}}}
\newrefformat{fig}{\citefig{\ref{#1}}}
\newrefformat{tbl}{\citetable{\ref{#1}}}
\newrefformat{chap}{\citechap{\ref{#1}}}
\newrefformat{fn}{\citefootnote{\ref{#1}}}
\newrefformat{box}{Box~\ref{#1}}
\newrefformat{ex}{\ref{#1}}


% color boxes

\tcbuselibrary{skins, breakable, theorems}

\AtBeginEnvironment{infobox}{\small}
\AtBeginEnvironment{theorybox}{\small}

\newtcbtheorem[number within=chapter]{infobox}{Box}{
    enhanced,
    boxrule=0pt,
    colback=blue!5,
    colframe=blue!5,
    coltitle=blue!50,
    borderline west={4pt}{0pt}{blue!65},
    sharp corners,
    fonttitle=\bfseries, 
    breakable,
    before upper={\parindent15pt\noindent}}{box}
\newtcbtheorem[number within=chapter, use counter from=infobox]{theorybox}{Box}{
    enhanced,
    boxrule=0pt,
    colback=orange!5, 
    colframe=orange!5, 
    coltitle=orange!50,
    borderline west={4pt}{0pt}{orange!65},
    sharp corners,
    fonttitle=\bfseries, 
    breakable,
    before upper={\parindent15pt\noindent}}{box}
\newtcbtheorem[number within=chapter, use counter from=infobox]{learnbox}{Box}{
    enhanced,
    boxrule=0pt,
    colback=green!5,
    colframe=green!5,
    coltitle=green!50,
    borderline west={4pt}{0pt}{green!65},
    sharp corners,
    fonttitle=\bfseries, 
    breakable,
    before upper={\parindent15pt\noindent}}{box}

% Shorthands
\newcommand*{\concept}[1]{\textbf{#1}}
\newcommand*{\term}[1]{\emph{#1}}
\newcommand{\form}[1]{\emph{#1}}

\newcommand{\redp}{\textasciitilde}

\newcommand{\deictictime}{T$_{\text{d}}$}
\newcommand{\referredtime}{T$_{\text{r}}$}
\newcommand{\orientationtime}{T$_{\text{o}}$}

\DeclareAcronym{blt}{short = BLT, long = Basic Linguistic Theory}
\DeclareAcronym{cgel}{short = CGEL, long = The Cambridge Grammar of the English Language}
\DeclareAcronym{dm}{short = DM, long = Distributed Morphology}
\DeclareAcronym{tag}{long = Tree-adjoining grammar, short = TAG}
\DeclareAcronym{sfp}{long = sentence-final particle, short = \textsc{sfp}}
\DeclareAcronym{np}{long = noun phrase, short = NP}
\DeclareAcronym{vp}{long = verb phrase, short = VP}
\DeclareAcronym{pp}{long = preposition phrase, short = PP}
\DeclareAcronym{advp}{long = adverb phrase, short = AdvP}
\DeclareAcronym{cls}{long = classifier, short = CLS}
\DeclareAcronym{dist}{long = distal, short = DIST}
\DeclareAcronym{prox}{long = proximate, short = PROX}
\DeclareAcronym{dem}{long = demonstrative, short = DEM}
\DeclareAcronym{classify}{long = classifier, short = \textsc{cl}}
\DeclareAcronym{dur}{long = durative, short = DUR}
\DeclareAcronym{neg}{long = negative, short = \textsc{neg}}
\DeclareAcronym{cc}{long = copular complement, short = CC}
\DeclareAcronym{cs}{long = copular subject, short = CS}
\DeclareAcronym{tame}{long = {tense, aspect, mood, evidentiality}, short = TAME}
\DeclareAcronym{past}{long = past, short = PST}
\DeclareAcronym{nonpast}{long = non-past, short = NPST}
\DeclareAcronym{present}{long = present, short = PRES}
\DeclareAcronym{progressive}{long = progressive, short = \textsc{poss}}
\DeclareAcronym{perfect}{long = perfect, short = \textsc{perf}}
\DeclareAcronym{passive}{long = passive, short = \textsc{pass}}
\DeclareAcronym{copula}{long = copula, short = COP}
\DeclareAcronym{possessive}{long = possessive, short = \textsc{poss}}
\DeclareAcronym{coca}{long = Corpus of Contemporary American English, short = COCA}

\newcommand{\asis}[1]{\textsc{#1}}
\newcommand{\oneof}[1]{{#1}}
\newcommand*{\homo}[2]{#1$_{\text{#2}}$}
\newcommand{\category}[1]{\textsc{#1}}
\newcommand{\formcat}[1]{\textsc{#1}}
\newcommand{\emptymorpheme}{$\emptyset$}
\newcommand*{\fromto}[2]{\langle {#1}, {#2} \rangle}

\newcommand{\alignment}{\href{../alignment/alignment.pdf}{my notes about alignment}}
\newcommand{\method}{\href{../methodology/glossing.pdf}{this note about my understanding of descriptive grammars}}

\newcommand{\ala}{à la}
\newcommand{\translate}[1]{`#1'}
\newcommand{\vP}{\textit{v}P}

% Make subsubsection labeled
\setcounter{secnumdepth}{4}
\setcounter{tocdepth}{4}
% reset example counter every chapter (but do not include the chapter number to the label)
\counterwithin{exx}{chapter} 

% Reference formats
\renewcommand{\bibname}{References}
\setcitestyle{aysep={}} 

% List format
\setlist[enumerate,1]{label=\alph*\upshape)}

% Source of examples 
\newcommand{\source}[1]{\hspace{\fill}\mbox{}\linebreak[0]\hspace*{\fill}\mbox{(\small #1)}}

\title{Notes about Sanzhi Dargwa}
\author{Jinyuan Wu}

\begin{document}

\maketitle

\chapter{Noun phrase}

\paragraph*{The template of noun phrase} A template of the noun phrase 
can be seen in \citet[\citesec{21.1.3}]{forker2020grammar}.


\chapter{Clausal structure}

There is no clear constraint 
on the constituent order in the main clause;
the constituent order however reflects pragmatic information,
and the relation between the two is not as simple as 
``the most important constituent is fronted'',
but has some structural constraints.
This is similar to the case of Latin or many other so-called 
languages with free constituent orders.

\paragraph*{The verb} The morphological template of the verb
is shown in \citet[\citetable{11.9}]{forker2020grammar}.
Clearly inflectional categories include 
negation, gender agreement, the two TAM slots.
Verb derivation is not very active:
it seems it's impossible to have multiple rounds of compounding.

\paragraph*{Argument structure} Valency changing devices are not abundant in Sanzhi;
we only know an antipassive and a causative,
the former lacking any morphological marking on the verb 
\citep[\citesec{19.2.1}]{forker2020grammar}.

\paragraph*{TAME categories} 
\begin{itemize}
    \item perfective, imperfective \citep[\citepage{206}]{forker2020grammar} \todo{Morphological, or real imperfective-perfective distinction?
    \citet[\citepage{259}]{forker2020grammar} mentioned ``imperfective aspect''}
    \item present, past: the present tense is marked by 
    person enclitics for first and second person, 
    and by the copula for third person \citep[\citepage{250}]{forker2020grammar},
    or maybe the \form{-ne} suffix \citep[\citepage{253}]{forker2020grammar};
    the past tense is marked by past enclitic \form{=de} \citep[\citepage{252}]{forker2020grammar}.
    \item The future meaning seems to be expressed by  
    the modal category: the first TAM slot in the verb is filled by 
    the modal participle marker \form{-an} \citep[\citepage{18.1.2.2}]{forker2020grammar};
    the person enclitics/third-person (tense?) marker \form{-ne} is added after it.
    The marker \form{-ne} indeed seems to be a tense marker: 
    it's in contrastive distribution with the present person marker,
    and it never appears in non-future constructions
    (in obligative constructions, third person is always marked by the copula).
    \todo{Relation between the future and the various obligative categories}
    \item Habitual or not, in the imperfective paradigm: 
    habitual present/past, where personal affixes are attached  
    directly after the stem \citep[\citepages{243, 246}]{forker2020grammar};
    compound present/past, the personal enclitics are attached 
    after the converb; 
    future, obligative, etc., where personal and tense suffixes are attached 
    after the modal participle \form{-an}. 
    (for obligative constructions we have another \form{-te} suffix after \form{-an},
    which retains its historical meaning -- definite denotation -- 
    in some cases \citep[\citepage{256}]{forker2020grammar}).
    \todo{Why the personal enclitic is recognized as a clitic.}
    \item In the perfective paradigm:
    simple: we also have ``perfective simple (not habitual) past'', 
    or, according to its morphological profile, the preterite 
    \citep[\citesec{14.2.2}]{forker2020grammar},
    but we don't have perfective simple present,
    although we do have imperfective preterite;
    imperfective and perfective resultative; 
    perfect, present and past, with \form{=da} etc. after the perfective converb; 
    experiential I and II, present and past. 
    \item For non-finite forms, the imperfective converb 
    is obtained by attaching the imperfective converb suffix \form{-ul(e)} to the imperfective stem
    \citep[\citepage{306}]{forker2020grammar},
    but the perfective converb is obtained by 
    attaching the perfective converb suffix \form{-le} to the preterite participle 
    \citep[\citepage{308}]{forker2020grammar}.
    The term \term{preterite} has morphological meaning only:
    it's usually known as a participle when used in a relative clause, 
    when it's used in finite conjugation, 
    it's simply referred to as the preterite in \citet{forker2020grammar}.
    \item Semi-auxiliary verb constructions (i.e. so-called \term{periphrastic verb forms})
\end{itemize}

\prettyref{tbl:core-tam} shows the core TAM constructions of Sanzhi;
constructions involving auxiliaries that also appear as lexical verbs, 
or what are known as \term{periphrastic verb forms} in \cite{forker2020grammar}
(on the other hand, the term \term{analytic verb forms}
is used for categories in \prettyref{tbl:core-tam} 
that involve non-finite verb forms in their morphological realizations),
are not displayed here.
\begin{table}[H]
    \caption{Core TAM categories of Sanzhi}
    \label{tbl:core-tam}
    \centering
    \small
    \begin{tabular}{llll}
    \toprule
    \multicolumn{2}{l}{}                                      & \multicolumn{2}{c}{tense}                                           \\ 
    \cmidrule(r){3-4}
    \multicolumn{2}{l}{\multirow{-2}{*}{aspect and modality}} & past                               & present                        \\ \cmidrule{2-4}
                                        & simple              & imperfective preterite             & \cellcolor[HTML]{C0C0C0}       \\
                                        & resultative         & imperfective preterite resultative & \cellcolor[HTML]{C0C0C0}       \\
    \cmidrule{2-4}                                        
                                        & habitual            & habitual past                      & habitual present               \\
    \cmidrule{2-4}                                        
                                        & generic             & compound past                      & compound present               \\
    \cmidrule{2-4}                                        
                                        & future              & future in the past                 & 
    \begin{tabular}[c]{@{}l@{}}future, obligative \\ present (no copula \\
    for 1st and 2nd person) \end{tabular}                                    \\
    \multirow{-6}{*}{imperfective}      & obligative          & obligative past                    & 
    \begin{tabular}[c]{@{}l@{}}obligative, \\ obligative present\end{tabular}  \\
    \midrule
                                        & simple              & preterite                          & \cellcolor[HTML]{C0C0C0}       \\
                                        & resultative         & perfective resultative             & \cellcolor[HTML]{C0C0C0}       \\
    \cmidrule{2-4} 
                                        & perfect             & perfect                            & pluperfect                     \\
    \cmidrule{2-4} 
                                        & experiencial I      & experiencial I                     & experiencial past I            \\
    \multirow{-5}{*}{perfective}        & experiencial II     & experiencial II                    & experiential past II           \\ 
    \bottomrule
    \end{tabular}
\end{table}

Now we try to locate the aforementioned TAM features to morphemes, 
being fully aware that this is not always possible. 
The imperfective-perfective distinction is of course marked 
solely by stem alternation.
In the analytic verb forms, 
the simple and the resultative aspects
are marked by the preterite participle suffix, 
and distinction between the two is realized by the 
presence or absence of the copula; 
the generic aspect is marked by the converb suffix; 
the future and the obligative are marked by 
the modal participle suffix.

The distinction between the future and obligative categories is more subtle.
For first and second person, 
we have three morphological parameters:
whether there is a definite denotation \form{-te} clitic after the participle, 
whether what follows \form{-te} (if it's there) is 
the past tense \form{=de},
and if not, whether what follows \form{-te} is a personal clitic 
or the copula.
The appearance of the copula always leads to an obligative meaning:
now if there is no \form{-te}, 
we say we are in the obligative, 
otherwise we say we are in the obligative present. 
The copula seems incompatible with the past marker;
the past obligative meaning is expressed by -\category{dd}=\category{pst}
(see below).
On the other hand, obligative meaning can still be expressed 
without the copula:
the future (\category{ptcp}=\category{1/2}) has modal meaning, 
and so do the future in the past (\category{ptcp}=\category{pst}),
the obligative past (\category{ptcp}-\category{dd}=\category{pst}), 
and the obligative present without copula (\category{ptcp}-\category{dd}=\category{1/2}).
Among them, the obligative past has no ``past future'' meaning, 
while the rest three constructions all have both the future meaning 
and the obligative meaning.
Therefore it seems the \category{dd} suffix is free to appear and go 
in the present tense, 
but in the past tense, it enforces the obligative meaning.

In conclusion, we find that the obligative meaning is sometimes not realized at all 
and sometimes realized as using the copula for the first and second person,
and sometimes realized as \category{-dd}-\category{=pst}.
The \category{-dd} suffix otherwise has no is redundant and is free to come and go.
Now the meanings of all the three morphological parameters are clear.

\citet{forker2020grammar} doesn't mention whether there exists 
temporal adverbs that seem to be a part of the TAM system 
instead of temporal peripheral arguments comparable to location expressions; 
a search in the corpus however reveals that 
\form{hana} \translate{now} has a strong tendency to say either 
at the beginning or at the end of a clause; 
counterexamples exist 
\citep[\citepage{292}, 37, 40]{forker2020grammar}, 
which 

\paragraph*{Alignment} In summary, Sanzhi shows morphological ergativity 
but syntactic accusativity \citep[\citesec{22.3}]{forker2020grammar}.
Ergativity can be observed in agreement besides case marking, 
and this can be explained by assuming that 
the ergative case resembles an inherent case, 
not unlike the \form{by}-phrase in the English passive:
in this case the only argument that is visible to the verb 
is the absolutive argument.
That said, agreement with arguments other than the absolutive argument
is possible \citep[\citesec{20.2.4}]{forker2020grammar}.

\paragraph*{Agreement} In the template of the verb 
we find slots for gender (in the prefix chain), person and number 
(both in the suffix chain and in the prefix chain). 
The gender prefix is actually a portmanteau of 
gender, person and number \citep[\citetable{20.1}]{forker2020grammar};
note that it's possible to have ``semantic agreement''
\citep[\citesec{20.2.2}]{forker2020grammar},
as in, say, fairy tales; 
we may say that there are actually two gender features in the 
relevant noun phrases:
the inner gender feature is the feature that comes with the head noun, 
while the outer gender feature is the feature 
added by the speaker around the noun phrase;
the gender marking of the head noun follows the inner gender feature, 
since the lexicon simply doesn't contain a concrete realization 
of a noun with the wrong gender feature;
but the gender feature visible to the verb 
is likely to be the outer gender feature. 
Person agreement happens in the suffix chain
and has interaction with the TAM category.

\begin{exe}
    \ex asdf \source{ab}
\end{exe}

\paragraph*{Negation} 

\paragraph*{Focus} The SOV and SVO orders are frequent and express neutral information structure; 
the two surface orders however 
are also able to express focus:
the SOV order sometimes expresses object focus, 
and the SVO order sometimes expresses subject focus
\citep[\citetable{27.1}]{forker2020grammar}.
This seems comparable with the English 
\form{only then do we cook} focus construction,
where focus comes together with fronting of the verb 
(although in modern English the focus category is not strong enough 
to attract the whole verb to the front).
Similar constructions -- where a focusing position is added before the nucleus clause 
to which the main verb is fronted and before which the focused constituent is placed-- 
are attested cross-linguistically \citep[\citepage{521}]{forker2020grammar}.

\todo{
    Whether the movement indeed happens 
    can be tested by looking at the position 
    of adverbs.
}

\paragraph*{Topic} The topic sometimes is marked by fronting:
thus the OSV order may be understood as 
object topicalization plus subject focus;

\bibliographystyle{plainnat}
\bibliography{sanzhi.bib}

\end{document}