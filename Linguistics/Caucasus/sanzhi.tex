\PassOptionsToPackage{table}{xcolor}
\documentclass[a4paper, oneside, 12pt]{report}

\usepackage[T1]{fontenc}
\usepackage{libertinus}
\usepackage{geometry}
\usepackage{float}
\usepackage{titling}
\usepackage{titlesec}
\usepackage{paralist}
\usepackage{footnote}
\usepackage[colorinlistoftodos]{todonotes}
\usepackage[inline]{enumitem}
\usepackage{amsmath, amsthm}
\usepackage{gb4e}
\noautomath
\usepackage{bbm}
\usepackage{textcomp}
\usepackage{soul}
\usepackage{graphicx}
\usepackage{siunitx}
\usepackage{tikz}
\usepackage[ruled, vlined, linesnumbered, noend]{algorithm2e}
\usepackage[colorlinks, citecolor = purple, bookmarksnumbered]{hyperref} % linkcolor=black, anchorcolor=black, citecolor=black, filecolor=black
\usepackage[most]{tcolorbox}
\usepackage{caption}
\usepackage{subcaption}
\usepackage{booktabs}
\usepackage{multirow}
\usepackage[figuresright]{rotating}
\usepackage{acro}
\usepackage[round]{natbib} 
\usepackage{prettyref}

\geometry{left=3.18cm,right=3.18cm,top=2.54cm,bottom=2.54cm}
\titlespacing{\paragraph}{0pt}{1pt}{10pt}[20pt]
\setlength{\droptitle}{-5em}

\DeclareMathOperator{\timeorder}{\mathcal{T}}
\DeclareMathOperator{\diag}{diag}
\DeclareMathOperator{\legpoly}{P}
\DeclareMathOperator{\primevalue}{P}
\DeclareMathOperator{\sgn}{sgn}
\newcommand*{\ii}{\mathrm{i}}
\newcommand*{\ee}{\mathrm{e}}
\newcommand*{\const}{\mathrm{const}}
\newcommand*{\suchthat}{\quad \text{s.t.} \quad}
\newcommand*{\argmin}{\arg\min}
\newcommand*{\argmax}{\arg\max}
\newcommand*{\normalorder}[1]{: #1 :}
\newcommand*{\pair}[1]{\langle #1 \rangle}
\newcommand*{\fd}[1]{\mathcal{D} #1}

\newcommand*{\citesec}[1]{\S~{#1}}
\newcommand*{\citechap}[1]{Ch~{#1}}
\newcommand*{\citefig}[1]{Fig.~{#1}}
\newcommand*{\citetable}[1]{Table~{#1}}
\newcommand*{\citepage}[1]{p.~{#1}}
\newcommand*{\citepages}[1]{pp.~{#1}}
\newcommand*{\citefootnote}[1]{fn.~{#1}}
\newcommand*{\citechapsec}[2]{\citechap{#1}.\citesec{#2}}

\newrefformat{sec}{\citesec{\ref{#1}}}
\newrefformat{fig}{\citefig{\ref{#1}}}
\newrefformat{tbl}{\citetable{\ref{#1}}}
\newrefformat{chap}{\citechap{\ref{#1}}}
\newrefformat{fn}{\citefootnote{\ref{#1}}}
\newrefformat{box}{Box~\ref{#1}}
\newrefformat{ex}{\ref{#1}}


% color boxes

\tcbuselibrary{skins, breakable, theorems}

\AtBeginEnvironment{infobox}{\small}
\AtBeginEnvironment{theorybox}{\small}

\newtcbtheorem[number within=chapter]{infobox}{Box}{
    enhanced,
    boxrule=0pt,
    colback=blue!5,
    colframe=blue!5,
    coltitle=blue!50,
    borderline west={4pt}{0pt}{blue!65},
    sharp corners,
    fonttitle=\bfseries, 
    breakable,
    before upper={\parindent15pt\noindent}}{box}
\newtcbtheorem[number within=chapter, use counter from=infobox]{theorybox}{Box}{
    enhanced,
    boxrule=0pt,
    colback=orange!5, 
    colframe=orange!5, 
    coltitle=orange!50,
    borderline west={4pt}{0pt}{orange!65},
    sharp corners,
    fonttitle=\bfseries, 
    breakable,
    before upper={\parindent15pt\noindent}}{box}
\newtcbtheorem[number within=chapter, use counter from=infobox]{learnbox}{Box}{
    enhanced,
    boxrule=0pt,
    colback=green!5,
    colframe=green!5,
    coltitle=green!50,
    borderline west={4pt}{0pt}{green!65},
    sharp corners,
    fonttitle=\bfseries, 
    breakable,
    before upper={\parindent15pt\noindent}}{box}

% Shorthands
\newcommand*{\concept}[1]{\textbf{#1}}
\newcommand*{\term}[1]{\emph{#1}}
\newcommand{\form}[1]{\emph{#1}}

\newcommand{\redp}{\textasciitilde}

\newcommand{\deictictime}{T$_{\text{d}}$}
\newcommand{\referredtime}{T$_{\text{r}}$}
\newcommand{\orientationtime}{T$_{\text{o}}$}

\DeclareAcronym{blt}{short = BLT, long = Basic Linguistic Theory}
\DeclareAcronym{cgel}{short = CGEL, long = The Cambridge Grammar of the English Language}
\DeclareAcronym{dm}{short = DM, long = Distributed Morphology}
\DeclareAcronym{tag}{long = Tree-adjoining grammar, short = TAG}
\DeclareAcronym{sfp}{long = sentence-final particle, short = \textsc{sfp}}
\DeclareAcronym{np}{long = noun phrase, short = NP}
\DeclareAcronym{vp}{long = verb phrase, short = VP}
\DeclareAcronym{pp}{long = preposition phrase, short = PP}
\DeclareAcronym{advp}{long = adverb phrase, short = AdvP}
\DeclareAcronym{cls}{long = classifier, short = CLS}
\DeclareAcronym{dist}{long = distal, short = DIST}
\DeclareAcronym{prox}{long = proximate, short = PROX}
\DeclareAcronym{dem}{long = demonstrative, short = DEM}
\DeclareAcronym{classify}{long = classifier, short = \textsc{cl}}
\DeclareAcronym{dur}{long = durative, short = DUR}
\DeclareAcronym{neg}{long = negative, short = \textsc{neg}}
\DeclareAcronym{cc}{long = copular complement, short = CC}
\DeclareAcronym{cs}{long = copular subject, short = CS}
\DeclareAcronym{tame}{long = {tense, aspect, mood, evidentiality}, short = TAME}
\DeclareAcronym{past}{long = past, short = PST}
\DeclareAcronym{nonpast}{long = non-past, short = NPST}
\DeclareAcronym{present}{long = present, short = PRES}
\DeclareAcronym{progressive}{long = progressive, short = \textsc{poss}}
\DeclareAcronym{perfect}{long = perfect, short = \textsc{perf}}
\DeclareAcronym{passive}{long = passive, short = \textsc{pass}}
\DeclareAcronym{copula}{long = copula, short = COP}
\DeclareAcronym{possessive}{long = possessive, short = \textsc{poss}}
\DeclareAcronym{coca}{long = Corpus of Contemporary American English, short = COCA}

\newcommand{\asis}[1]{\textsc{#1}}
\newcommand{\oneof}[1]{{#1}}
\newcommand*{\homo}[2]{#1$_{\text{#2}}$}
\newcommand{\category}[1]{\textsc{#1}}
\newcommand{\formcat}[1]{\textsc{#1}}
\newcommand{\emptymorpheme}{$\emptyset$}
\newcommand*{\fromto}[2]{\langle {#1}, {#2} \rangle}

\newcommand{\alignment}{\href{../alignment/alignment.pdf}{my notes about alignment}}
\newcommand{\method}{\href{../methodology/glossing.pdf}{this note about my understanding of descriptive grammars}}

\newcommand{\ala}{à la}
\newcommand{\translate}[1]{`#1'}
\newcommand{\vP}{\textit{v}P}

% Make subsubsection labeled
\setcounter{secnumdepth}{4}
\setcounter{tocdepth}{4}
% reset example counter every chapter (but do not include the chapter number to the label)
\counterwithin{exx}{chapter} 

% Reference formats
\renewcommand{\bibname}{References}
\setcitestyle{aysep={}} 

% List format
\setlist[enumerate,1]{label=\alph*\upshape)}

% Source of examples 
\newcommand{\source}[1]{\hspace{\fill}\mbox{}\linebreak[0]\hspace*{\fill}\mbox{(\small #1)}}

\title{Notes about Sanzhi Dargwa}
\author{Jinyuan Wu}

\begin{document}

\maketitle

\chapter{The language and the speakers}

\chapter{Grammatical overview}

\paragraph*{Morphological typology}

\section{Parts of speech division}

\paragraph*{Wordhood}

\paragraph*{Recognizing the verb}
The morphological template of the verb
is shown in \citet[\citetable{11.9}]{forker2020grammar}.
The verb conjugates partly according to
gender (in the prefix chain), person and number 
(both in the suffix chain and in the prefix chain). 
Other clear inflectional categories include 
negation and the two TAM slots.
Verb derivation is not very active:
it seems it's impossible to have multiple rounds of compounding.

\section{Clausal syntax}

\subsection{Clause types or ``moods''} 

\subsection{Constituent orders and information structure}

\paragraph*{The predicator and clausal dependents} 
A clause is a predicator (which may be null) plus constituents around it.
The predicator is often informally known as the \emph{verb}, 
although the term is more frequently used to refer to a lexeme that is prototypically verb-like, 
which may be used as the predicator or not, 
and it's possible to have a clause headed by something else. 
Criteria used to recognize the verb are outlined above in TODO 

There might be auxiliaries in the clause.
The distinction between an auxiliary verb construction 
and a biclausal construction 
can be made by examining e.g. the agreement properties of the alleged auxiliary.
In Sanzhi we do have controversies regarding whether some so-called auxiliary verb constructions 
should always be analyzed in this way
(\prettyref{sec:overview.alignment.gender-agreement}). 

From a surface-based perspective, 
there is no hard constraint on the constituent order in the main clause;
the constituent order however reflects pragmatic information,
and the relation between the two is not as simple as 
``the most important constituent is fronted'',
but has some structural constraints.
First, all constituent order alternations seem to be motivated by 
information structure marking devices that are attested in unrelated languages, 
as is outlined below.
Second, constituent order has some relations with agreement (TODO: link), 
and the latter is usually related to structural factors. 
This is similar to the case of Latin or many other 
languages with so-called free constituent orders (TODO: cite).

\paragraph*{Neutral orders}
The neutral order is SOV or SV; SVO is also attested
\citep[\citetable{27.1}]{forker2020grammar}.
Temporal adverbials appear before the main verb in the SOV case 
\citep[\citepage{520}, (30)]{forker2020grammar}.
Object focusing is possible with the SOV and SVO orders
\citep[\citepages{520-521}]{forker2020grammar}, 
which is not surprising since object focalization agrees with the neutral information structure 
and either does not necessarily trigger explicit movement
or only works within the verb phrase and merely swaps the object with some adverbials, if any.
The alternation between the VO and OV orders: why? TODO

\paragraph*{Verb-second focalization}
The OVS order is induced by focusing both the object and the verb.
This order, when the object and the verb are focused, 
is not uncommon in other languages, particularly in verb-second constructions;
indeed in modern English we have focus constructions like 
\form{only then do we cook} 
where a focusing position is added before the nucleus clause 
to which the main verb is fronted and before which the focused constituent is placed
(although in modern English the focus category is not strong enough 
to attract the whole verb to the front).
Similar constructions are also attested cross-linguistically \citep[\citepage{521}]{forker2020grammar}.
Similarly when the subject is focused, SVO may appear, 
besides its appearance as a neutral order.
Thus the alternation between SVO and SOV is possibly although not necessarily 
the alternation between subject focalization and object focalization (plus subject topicalization).
TODO: the position of adverbs in focusing SVO and normal SVO

\paragraph*{Focalization of the verb only}
Focalization of the verb -- without focalization of anything else -- likely lead to the VSO order.
\citet[\citetable{27.1}]{forker2020grammar} says in this constituent order, 
the subject and the object are all topical; 
this may be a natural consequence of the verb and only the verb being focalized.

\paragraph*{Object topicalization}
The OSV word order is less common.
It seems to be formed by topicalizing the object \citep[\citepage{522}]{forker2020grammar}; 
the subject-plus-verb focalization reading is likely a derivation 
of the object topicalization reading.

\paragraph*{Rightward shift of the subject}\label{sec:overview.information.rightward}
All pragmatics-driven word order alternations introduced so far are fronting operations: 
the verb-second focusing order, 
the fronting of the verb, and the fronting of the object.
When the subject is topicalized, an additional strategy is to place it \emph{after} the rest of the clause: 
hence we have the VS and the OVS orders where the subject is topicalized.
This looks like an ``afterthought'' construction that's 
not uncommon in languages with flexible word orders, 
like Latin \citep[\citepage{39}]{devine2006latin}.


TODO: presentational

\paragraph*{Omission of arguments} 
A final remark is uncontroversial core arguments in Sanzhi clauses can also be dropped 
if they can be recovered from the context.
This is different from valency alternation in that 
argument dropping in valency alternation creates a \emph{new} verb frame 
and hence may lead to slight meaning alternation as well:
the English sentence \form{the dog bites} means 
the dog is mean and has the bad habit to bite people he doesn't like; 
we never say \form{your dog bites} to refer to a \emph{specific} event 
in which the mean dog bites a guest.
On the other hand, in Sanzhi, the latter usage is totally find. \todo{Citation} 

\subsection{Alignment and grammatical relations} 

\paragraph*{Morphological ergativity}
Roughly speaking, Sanzhi shows morphological ergativity 
but syntactic accusativity \citep[\citesec{22.3}]{forker2020grammar}.
Ergativity can be observed in agreement and case marking, 
and this can be explained by assuming that 
the ergative case resembles an inherent case, 
not unlike the \form{by}-phrase in the English passive:
in this case the only argument that is visible to the verb 
is the absolutive argument.

Sanzhi has three agreement systems, 
namely pure number agreement, 
combined gender/number agreement, 
and person agreement \citep[\citepage{373}]{forker2020grammar}.
At the clause level all the three systems are present.
The systems are independent to each other, 
meaning that the controllers of the three agreement systems can be different.


Sanzhi deviates from the ideal morphological ergativity
in the following aspects.
First, it has quirky subject case marking: 
with affective verbs, subject-like argument may take the dative case 
\citep[\citetable{19.1}]{forker2020grammar}.
Second, agreement with arguments other than the absolutive argument
-- i.e. arguments in the ergative or dative cases --
is possible \citep[\citesec{20.2.4}]{forker2020grammar}.

\paragraph*{Person agreement}
Person agreement shows the distinction between speech act participants and third person.
The rules are discussed in \citep[\citesec{20.3.2}]{forker2020grammar}.

Person agreement, on the verb, is realized in the suffix chain
and is related to the realization TAM categories 
\citet[\citesec{11.4}]{forker2020grammar}.

\paragraph*{Gender agreement}\label{sec:overview.alignment.gender-agreement}
Gender agreement, or more precisely mixed gender/number agreement, 
roughly follows the ergative scheme, that's to say, 
the absolutive argument usually controls the agreement \citep[\citepage{377}]{forker2020grammar}.
There are however several notable derivations.

It's possible to have ``semantic agreement''
\citep[\citesec{20.2.2}]{forker2020grammar},
as in, say, fairy tales; 
we may say that there are actually two gender features in the 
noun phrases that trigger the person agreement:
the inner gender feature is the feature that comes with the head noun, 
while the outer gender feature is the feature 
added by the speaker around the noun phrase;
the gender marking of the head noun follows the inner gender feature, 
since the lexicon simply doesn't contain a concrete realization 
of a noun with the wrong gender feature;
but the gender feature visible to the verb 
is likely to be the outer gender feature. 

Agreement with an ergative or dative argument \todo{Also spatial adjuncts} -- 
the so-called deviant agreement --
is possible and only possible when the argument is agentive 
and the target is a copula-auxiliary and clausal adjuncts. \todo{What does ``clausal adjuncts'' mean?}
Therefore, the problem of deviant agreements is still about 
choosing between S and O for the agreement controller;
specifically the T argument of a ditransitive verb never participates at gender agreement 
even when the agreement is deviant.
Note that the nonfinite verb form also has its own gender agreement prefix, 
which always follows the standard, morphologically ergative rules..

Criteria of deviant agreement are not clear yet. 
It's observed that if a non-absolutive argument appears after the verb, 
which is likely due to the rightward shift (\prettyref{sec:overview.information.rightward})
and can be a topicalization device, 
its likelihood to control the gender agreement is increased 
\citep[\citepage{387}, (64)]{forker2020grammar}.
It might be the case that the gender agreement prefix on the auxiliary 
represents an additional agreement with the topic, 
which isn't always consistent with the usual gender agreement with the absolutive argument 
that appears on the main verb.
When the topic is not the absolutive argument, 
the possibility of deviant agreement exists, 
which may or may not be overridden by the agreement with the absolutive argument.
On the other hand, we can stipulate that 
agreement with the absolutive argument, if any, 
always overrides the topical agreement exponent on the main verb, 
presumably because of some localization principles.
The problem with this analysis is that 
there are cases where the deviant gender agreement controller can be by no means topical
\citep[\citepage{388}, (67)]{forker2020grammar}.%
\footnote{
    Her argument based on example (66) on the same page however is not convincing at all: 
    for the speaker to treat an argument as a topic, 
    it's not truly needed that this argument explicitly appears in the context.
    Rather, it's only needed that the speaker thinks the argument is in the context: 
    \form{as for you-know-who, \dots}
}
Therefore, even if the information structure account is right, 
it can't explain all instances of deviant agreement.

Another way to analyze the phenomenon is to stipulate that 
constructions that allow deviant agreement is actually biclausal: 
we can imagine that the analytic TAM constructions have two competing analyses 
and although the monoclausal structures are the most popular, 
the biclausal structures appear from time to time 
and when they appear, the copula is morphologically the main verb in the matrix clause,
which agrees with the only argument of the matrix clause, 
i.e. the non-absolutive argument.
The ``main verb'' -- which, in the biclausal structure, is actually the main verb of the complement clause -- 
however always agrees with the absolutive argument, 
which is a direct constituent of the complement clause and not the matrix clause.
This is at least consistent with the fact that deviant agreement only happens within clauses with analytic verb forms.
The main problem of the biclausal analysis is that the word orders of some examples with deviant agreement 
simply don't support a biclausal analysis.\todo{Discuss the standard of a clause}
For example, consider example (56) in \citep[\citepage{385}]{forker2020grammar}, 
where the verb-auxiliary sequence appears at the beginning of the whole sentence, 
while the relative clause, 
which under the biclausal analysis should be the sole argument (comparable to the absolutive argument) of the main verb, 
appears at the end of the sentence.
It then seems that the complement clause that contains 
both the main verb and the relative clause is broken into two.
Therefore, if we accept the biclausal analysis, 
the complement clause that contains both the absolutive argument and the main verb 
can't be a prototypical, completely sealed clause 
but should be something like the complement clause in the English raising construction.

\paragraph*{Discrepancies between agreement systems} 
The two agreement systems work independently.
For example it's possible for the gender agreement exponent to be from the direct object 
and for the person agreement exponent to be from the subject
\citep[\citepage{400}, (108)]{forker2020grammar}.

\paragraph*{Case marking} 
Case marking depends on the grammatical function of an argument, 
but also its deep semantic role: 
with affective verbs, subject-like argument may take the dative case 
\citep[\citetable{19.1}]{forker2020grammar}. 

\paragraph*{Syntactic pivot} 

\subsection{TAME categories} 

Here we list all TAM categories that can be easily recognized from surface-based evidence, 
and study the combinatorial possibilities of the TAM features.
We end this section by an enumeration of possible TAM categories in Sanzhi.

\citet{forker2020grammar} doesn't mention whether there exists 
temporal adverbs that seem to be a part of the TAM system 
instead of temporal peripheral arguments comparable to location expressions; 
therefore in this note we are going to mainly focus on TAM categories 
that can be read from verb conjugation.

\paragraph*{The imperfective/perfective distinction} 
In Sanzhi we see a clear imperfective/perfective distinction directly in verb morphology 
\citep[\citepage{206}]{forker2020grammar}. 
The imperfective aspect is the merger of the habitual aspect, the progressive aspect, etc.%
\footnote{
    Some may argue that this distinction is merely morphological; 
    at least \citet[\citepage{206}]{forker2020grammar} mentions the imperfective aspect 
    and seems to treat this as a real syntactic feature, 
    not just something like a conjugational class. 
}

\paragraph*{Habitual and generic aspects} 
The habitual and generic aspects are only compatible with the imperfective aspect.


\paragraph*{Tense} Sanzhi shows a distinction between the present and the past.
The present tense is marked by 
person enclitics for first and second person, 
and by the copula for third person \citep[\citepage{250}]{forker2020grammar},
or maybe the \form{-ne} suffix \citep[\citepage{253}]{forker2020grammar};
the past tense is marked by past enclitic \form{=de} \citep[\citepage{252}]{forker2020grammar}.

\paragraph*{Modality} 
The future meaning seems to be expressed by  
the modal category: the first TAM slot in the verb is filled by 
the modal participle marker \form{-an} \citep[\citepage{18.1.2.2}]{forker2020grammar};
the person enclitics/third-person (tense?) marker \form{-ne} is added after it.
The marker \form{-ne} indeed seems to be a tense marker: 
it's in contrastive distribution with the present person marker,
and it never appears in non-future constructions
(in obligative constructions, third person is always marked by the copula).
\todo{Relation between the future and the various obligative categories}

\begin{itemize}
    \item Habitual or not, in the imperfective paradigm: 
    habitual present/past, where personal affixes are attached  
    directly after the stem \citep[\citepages{243, 246}]{forker2020grammar};
    compound present/past, the personal enclitics are attached 
    after the converb; 
    future, obligative, etc., where personal and tense suffixes are attached 
    after the modal participle \form{-an}. 
    (for obligative constructions we have another \form{-te} suffix after \form{-an},
    which retains its historical meaning -- definite denotation -- 
    in some cases \citep[\citepage{256}]{forker2020grammar}).
    \todo{Why the personal enclitic is recognized as a clitic.}

    \item In the perfective paradigm:
    simple: we also have ``perfective simple (not habitual) past'', 
    or, according to its morphological profile, the preterite 
    \citep[\citesec{14.2.2}]{forker2020grammar},
    but we don't have perfective simple present,
    although we do have imperfective preterite;
    imperfective and perfective resultative; 
    perfect, present and past, with \form{=da} etc. after the perfective converb; 
    experiential I and II, present and past. 
    \item For non-finite forms, the imperfective converb 
    is obtained by attaching the imperfective converb suffix \form{-ul(e)} to the imperfective stem
    \citep[\citepage{306}]{forker2020grammar},
    but the perfective converb is obtained by 
    attaching the perfective converb suffix \form{-le} to the preterite participle 
    \citep[\citepage{308}]{forker2020grammar}.
    The term \term{preterite} has morphological meaning only:
    it's usually known as a participle when used in a relative clause, 
    when it's used in finite conjugation, 
    it's simply referred to as the preterite in \citet{forker2020grammar}.
    \item Semi-auxiliary verb constructions (i.e. so-called \term{periphrastic verb forms})
\end{itemize}

\paragraph*{Evidentiality} 
Sanzhi shows at least one evidentiality category: the experiencial \citep[\citesec{14.2.6}]{forker2020grammar}.
There are two experiencial constructions, and both of them have to be in past tense. 
Some other TAM constructions have inferential evidentiality values
\citep[\citepages{262,254-266}]{forker2020grammar}, 
although it's hard to tell whether the meanings are dictated by grammar and are obligatory 
or are from semantic implication (e.g. \citealt[\citepage{262}, (58)]{forker2020grammar}).

\paragraph*{Summary of attested core TAM constructions}
\prettyref{tbl:core-tam} shows the core TAM constructions of Sanzhi;
constructions involving auxiliaries that also appear as lexical verbs, 
or what are known as \term{periphrastic verb forms} in \cite{forker2020grammar}
(on the other hand, the term \term{analytic verb forms}
is used for categories in \prettyref{tbl:core-tam} 
that involve non-finite verb forms in their morphological realizations),
are not displayed here.
\todo{The perfective paradigm is wrong: there seems to be no ``present'' section?}

\begin{table}[H]
    \caption{Core TAM categories of Sanzhi}
    \label{tbl:core-tam}
    \centering
    \small
    \begin{tabular}{llll}
    \toprule
    \multicolumn{2}{l}{}                                      & \multicolumn{2}{c}{tense}                                           \\ 
    \cmidrule(r){3-4}
    \multicolumn{2}{l}{\multirow{-2}{*}{aspect and modality}} & past                               & present                        \\ \midrule
                                        & simple              & imperfective preterite             & \cellcolor[HTML]{C0C0C0}       \\
                                        & resultative         & imperfective preterite resultative & \cellcolor[HTML]{C0C0C0}       \\
    \cmidrule{2-4}                                        
                                        & habitual            & habitual past                      & habitual present               \\
    \cmidrule{2-4}                                        
                                        & generic             & compound past                      & compound present               \\
    \cmidrule{2-4}                                        
                                        & future              & future in the past                 & 
    \begin{tabular}[c]{@{}l@{}}future, obligative \\ present (no copula \\
    for 1st and 2nd person) \end{tabular}                                    \\
    \multirow{-6}{*}{imperfective}      & obligative          & obligative past                    & 
    \begin{tabular}[c]{@{}l@{}}obligative, \\ obligative present\end{tabular}  \\
    \midrule
                                        & simple              & preterite                          & \cellcolor[HTML]{C0C0C0}       \\
                                        & resultative         & perfective resultative             & \cellcolor[HTML]{C0C0C0}       \\
    \cmidrule{2-4} 
                                        & perfect             & pluperfect                         & perfect                        \\
    \cmidrule{2-4} 
                                        & experiencial I      & experiencial past I                & experiencial I                 \\
    \multirow{-5}{*}{perfective}        & experiencial II     & experiencial past II               & experiential II                \\ 
    \bottomrule
    \end{tabular}
\end{table}


\subsection{Verb frames} Valency changing devices are not abundant in Sanzhi;
we only know an antipassive and a causative,
the former lacking any morphological marking on the verb 
\citep[\citesec{19.2.1}]{forker2020grammar}.



\subsection{Negation} It seems Sanzhi doesn't have negative adverbs.
Negation is marked solely by verb conjugation.
\todo{Scope of negation?}
The marking of negation therefore depends on verb conjugation as well, 
either by adding a negative prefix to the verb or by using a negative copula 
\citep[\citesec{11.7}]{forker2020grammar}.

\subsection{Analyses of typical clauses}

As a summary of the above, here several sentences are analyzed in fine details.

\section{Noun and prepositional phrases}

\paragraph*{The template of noun phrase} A template of the noun phrase 
can be seen in \citet[\citesec{21.1.3}]{forker2020grammar}.

\section{What to investigate for lexical items}

\paragraph*{Nouns}

\subsection{Verbs}

\paragraph*{Verbal stems} The verbal stems of Sanzhi verbs have 
lexicalized, unpredictable imperfective/perfective alternation
\citep[\citesec{11.2}]{forker2020grammar}.
Some conjugational classes can still be recognized.
Also there are morphologically exceptional verbs 
\citep[\citesec{11.2.7}]{forker2020grammar}.

\paragraph*{Argument structure} Transitivity; whether the subject should be in dative.

\subsection{Adverbs}

One controversy is whether Sanzhi has temporal adverbs 
comparable to e.g. English \form{usually}, 
which structurally are not close to peripheral arguments of the clause 
but rather seem to be a part of the TAM system.

A search in the corpus reveals that 
\form{hana} \translate{now} has a strong tendency to say either 
at the beginning or at the end of a clause; 
counterexamples exist 
\citep[\citepage{292}, 37, 40]{forker2020grammar}, 
which 

\chapter{Verb morphology}

The gender prefix is actually a portmanteau of 
gender, person and number \citep[\citetable{20.1}]{forker2020grammar}.

\begin{exe}
    \ex asdf \source{ab}
\end{exe}

Now we try to locate the aforementioned TAM features to morphemes, 
being fully aware that this is not always possible. 
The imperfective-perfective distinction is of course marked 
solely by stem alternation.
In the analytic verb forms, 
the simple and the resultative aspects
are marked by the preterite participle suffix, 
and distinction between the two is realized by the 
presence or absence of the copula; 
the generic aspect is marked by the converb suffix; 
the future and the obligative are marked by 
the modal participle suffix.

An interesting observation is the copula seems to be the marker of a resultative meaning: 
the realization of the perfect category 
is a mixture of personal enclitic and copula, 
which may indicate that this category is the merger of ``the perfect'' and ``the perfect resultative''.

The distinction between the future and obligative categories is more subtle.
For first and second person, 
we have three morphological parameters:
whether there is a definite denotation \form{-te} clitic after the participle, 
whether what follows \form{-te} (if it's there) is 
the past tense \form{=de},
and if not, whether what follows \form{-te} is a personal clitic 
or the copula.
The appearance of the copula always leads to an obligative meaning:
now if there is no \form{-te}, 
we say we are in the obligative, 
otherwise we say we are in the obligative present. 
The copula seems incompatible with the past marker;
the past obligative meaning is expressed by -\category{dd}=\category{pst}
(see below).
On the other hand, obligative meaning can still be expressed 
without the copula:
the future (\category{ptcp}=\category{1/2}) has modal meaning, 
and so do the future in the past (\category{ptcp}=\category{pst}),
the obligative past (\category{ptcp}-\category{dd}=\category{pst}), 
and the obligative present without copula (\category{ptcp}-\category{dd}=\category{1/2}).
Among them, the obligative past has no ``past future'' meaning, 
while the rest three constructions all have both the future meaning 
and the obligative meaning.
Therefore it seems the \category{dd} suffix is free to appear and go 
in the present tense, 
but in the past tense, it enforces the obligative meaning.

In conclusion, we find that the obligative meaning is sometimes not realized at all 
and sometimes realized as using the copula for the first and second person,
and sometimes realized as \category{-dd}-\category{=pst}.
The \category{-dd} suffix otherwise has no is redundant and is free to come and go.
Now the meanings of all the three morphological parameters are clear.

\bibliographystyle{plainnat}
\bibliography{sanzhi.bib}

\end{document}