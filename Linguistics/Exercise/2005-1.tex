\documentclass[12pt]{article}

\usepackage{libertinus}
\usepackage{geometry}
\usepackage{titling}
\usepackage{titlesec}
\usepackage{paralist}
\usepackage{multicol} 
\usepackage{footnote}
\usepackage{marginnote}
\usepackage{enumerate}
\usepackage{amsmath, amssymb, amsthm}
\usepackage{gb4e}
\noautomath
\usepackage{bbm}
\usepackage{soul}
\usepackage{graphicx}
\usepackage{siunitx}
\usepackage[table,xcdraw]{xcolor}
\usepackage{tikz}
\usepackage[ruled, vlined, linesnumbered, noend]{algorithm2e}
\usepackage{xr-hyper}
\usepackage[colorlinks]{hyperref} % linkcolor=black, anchorcolor=black, citecolor=black, filecolor=black
\usepackage[most]{tcolorbox}
\usepackage{caption}
\usepackage{subcaption}
\usepackage{booktabs}
\usepackage{multirow}
\usepackage[figuresright]{rotating}
\usepackage{acro}
\usepackage[round]{natbib} 
\usepackage{prettyref}

\geometry{left=3.18cm,right=3.18cm,top=2.54cm,bottom=2.54cm}
\titlespacing{\paragraph}{0pt}{1pt}{10pt}[20pt]
\setlength{\droptitle}{-5em}

\DeclareMathOperator{\timeorder}{\mathcal{T}}
\DeclareMathOperator{\diag}{diag}
\DeclareMathOperator{\legpoly}{P}
\DeclareMathOperator{\primevalue}{P}
\DeclareMathOperator{\sgn}{sgn}
\newcommand*{\ii}{\mathrm{i}}
\newcommand*{\ee}{\mathrm{e}}
\newcommand*{\const}{\mathrm{const}}
\newcommand*{\suchthat}{\quad \text{s.t.} \quad}
\newcommand*{\argmin}{\arg\min}
\newcommand*{\argmax}{\arg\max}
\newcommand*{\normalorder}[1]{: #1 :}
\newcommand*{\pair}[1]{\langle #1 \rangle}
\newcommand*{\fd}[1]{\mathcal{D} #1}

\newcommand*{\citesec}[1]{\S~{#1}}
\newcommand*{\citechap}[1]{chap.~{#1}}
\newcommand*{\citefig}[1]{Fig.~{#1}}
\newcommand*{\citetable}[1]{Table~{#1}}
\newcommand*{\citefootnote}[1]{footnote~{#1}}

\newrefformat{sec}{\citesec{\ref{#1}}}
\newrefformat{fig}{\citefig{\ref{#1}}}
\newrefformat{tbl}{\citetable{\ref{#1}}}
\newrefformat{chap}{\citechap{\ref{#1}}}
\newrefformat{infobox}{Box~\ref{#1}}

\usetikzlibrary{arrows,shapes,positioning}
\usetikzlibrary{arrows.meta}
\usetikzlibrary{decorations.markings}
\tikzstyle arrowstyle=[scale=1]
\tikzstyle directed=[postaction={decorate,decoration={markings,
    mark=at position .5 with {\arrow[arrowstyle]{stealth}}}}]
\tikzstyle ray=[directed, thick]
\tikzstyle dot=[anchor=base,fill,circle,inner sep=1pt]


\tcbuselibrary{skins, breakable, theorems}

\newtcbtheorem{infobox}{Box}{
    enhanced,
    boxrule=0pt,
    colback=blue!5,
    colframe=blue!5,
    coltitle=blue!50,
    borderline west={4pt}{0pt}{blue!65},
    sharp corners,
    fonttitle=\bfseries, 
    breakable,
    before upper={\parindent15pt\noindent}}{box}

\newcommand*{\concept}[1]{\textbf{#1}}
\newcommand*{\term}[1]{\emph{#1}}
\newcommand*{\form}[1]{\emph{#1}}
\newcommand{\translate}[1]{`#1'}

\newcommand*{\vP}{\textit{v}P}

\DeclareAcronym{np}{short = NP, long = noun phrase}
\DeclareAcronym{plural}{short = PL, long = plural}
\DeclareAcronym{blt}{short = BLT, long = Basic Linguistic Theory}
\DeclareAcronym{cgel}{short = CGEL, long = The Cambridge Grammar of the English Language}
\DeclareAcronym{dm}{short = DM, long = Distributed Morphology}
\DeclareAcronym{tag}{long = Tree-adjoining grammar, short = TAG}
\DeclareAcronym{sfp}{long = sentence final particle, short = SFP}
\DeclareAcronym{vp}{long = verb phrase, short = VP}
\DeclareAcronym{cls}{long = classifier, short = CLS}
\DeclareAcronym{dist}{long = distal, short = DIST}
\DeclareAcronym{prox}{long = proximate, short = PROX}
\DeclareAcronym{dem}{long = demonstrative, short = DEM}
\DeclareAcronym{what}{long = \term{wh}-pronoun, short = WH}
\DeclareAcronym{dur}{long = durative, short = DUR}
\DeclareAcronym{neg}{long = negative, short = NEG}
\DeclareAcronym{tame}{long = {Tense, Aspect, Mood, Evidentiality}, short = TAME}
\DeclareAcronym{past}{long = past, short = PST}
\DeclareAcronym{future}{long = future, short = FUT}

\newcommand{\category}[1]{\textsc{#1}}

% Disable unsupported commands in bookmark titles 
\pdfstringdefDisableCommands{%
  \def\\{}%
  \def\texttt#1{<#1>}%
  \def\mathbb#1{#1}%
}
\pdfstringdefDisableCommands{\def\eqref#1{(\ref{#1})}}

\makeatletter
\pdfstringdefDisableCommands{\let\HyPsd@CatcodeWarning\@gobble}
\makeatother

\title{Solving Problem 1 in the Third International Linguistics Olympiad (2005)}
\author{Jinyuan Wu}

\begin{document}

\automath

\maketitle

\section{Introduction}

The Tzotzil language belongs to the Mayan family,
and its grammar is the focus of the first problem 
in the third International Linguistic Olympiad.
15 translated sentences are given.
In this note I give a bottom-up description 
based on information available in the texts given
in the framework of \ac{blt}.

Wordhood is taken for granted in the discussion below;
fortunately in the texts given,
no compound words or similar constructions appears, 
and therefore controversies surrounding wordhood of lexical words, 
if any, are only about whether certain grammatical markers 
are to be recognized as parts of inflectional templates,
and whether we have phonological word boundaries at certain places.
These realizational details will not heavily influence the largely grammatical discussion here. 

\section{Word list}

\subsection{Nouns}

By identifying shared nouns among Tzotzil texts and their translations,
some nouns can be easily identified.
The orthography convention 
requires that proper names are capitalized,
which makes interpreting the sentences much easier.
Here is a list of recognizable nouns:
\begin{itemize}
    \item \form{chan-vun} \translate{school} (\ref{ex:text.5}, \ref{ex:text.9}, \ref{ex:text.13})
    \item \form{k'op} \translate{speech} (\ref{ex:text.14})
    \item \form{`ixim} \translate{corn} (\ref{ex:text.1}, \ref{ex:text.10}) 
\end{itemize}

The following identification is not clear:
\begin{exe}
    \ex \form{na} \translate{house, home} 
    \begin{xlist}
        \ex `Oy `ox `ixim ta [ana] nax. \\
        \translate{You had corn at home today.} (Line 1)
        
        \ex `Oy `ox jlekii [na] po'ot. \\
        \translate{I will soon have a good house.}
    \end{xlist}
\end{exe}
Comparison between (\ref{ex:text.10}) and (\ref{ex:text.3}) shows
an alternation between \form{jp'in} and \form{p'in}.
I think it's likely due to the difference in the number of the \acp{np}. 
The alternation of \form{na} is likely phonological.

\subsection{Adverbs}

Adverbs can be found at the end of a sentence.

\begin{itemize}
    \item \form{junbai} \translate{last year} (\ref{ex:text.9}, \ref{ex:text.15})

    \item \form{lavie} \translate{from now on today} (\ref{ex:text.10}, \ref{ex:text.12})
    
    \item \form{nax} \translate{earlier today} (\ref{ex:text.1}, \ref{ex:text.14})

    \item \form{`ok'ob} \translate{tomorrow} 
\end{itemize}

\subsection{Verbs} 

The form \form{`oy} is likely a verb (\ref{ex:nominal.preposition.1})
meaning \translate{exist}.

It seems the word \form{'ox} appearing frequently after \form{`oy} 
is a tense marker of the non-present tense,
including both past and future.

\subsection{Particles}

\form{ta}

\section{The nominal system}

\subsection{Prepositions}

\begin{exe}
    \ex\label{ex:nominal.preposition.1} \gll `Oy [chan-vun ta batz'i k'op] [ta [Jobel]_{\text{proper name}}]. \\
    exist school in {} talk in San.Cristobal \\
    \glt \translate{There is a Tzotzil school in San Cristobal.} (Line 5)
    \ex 
\end{exe}

\subsection{Determiners}



\section{The verbal system}

\subsection{Clause types}

The marker \form{mi} is placed at the starting point of a sentence to mark the interrogative speech force.

\subsection{\ac{tame}}

The only attested \ac{tame} category in the texts given is the tense,
which seems to have an distinction between \category{present} and \category{non-present}.

\section{Annotated texts}

\begin{exe}
    \ex\label{ex:text.1}
    \gll `Oy    `ox            `ixim ta ana  nax. \\
          exist \category{pst}  corn in home today\\
    \translate{You had corn at home today.} (Line 1)
    
    \ex\label{ex:text.3}
    \gll Ch'abal `ox   chenek' ta jp'in po`ot \\
         ?       exist ?       in ?     ? \\
    \translate{Soon there will be no haricots in my pot.} (Line 3)

    \ex\label{ex:text.5} 
    \gll `Oy chan-vun ta batz'i k'op ta Jobel. \\
          exist school in \\
    \translate{There is a Tzotzil school in San Cristobal.} (Line 5)
    
    \ex\label{ex:text.9} 
    \gll Mi           `oy    `ox            chan-vun ta Jobel          junabi?   \\
        \category{q}  exist  \category{pst} school   in San.Cristobal  last.year \\
    \translate{Was there a school in San Cristobal last year?} (Line 9)
    
    \ex\label{ex:text.10}
    \gll Mi            `oy  `ixim ta p'in lavie? \\
         \category{q}  exist corn in ?    today  \\
    \translate{Is there corn in his pot?} (Line 10)

    \ex\label{ex:text.12} 
    \gll `Oy   `ox                lekil vob ta k'in lavie. \\
         exist \category{nonpres} ?     ?   in ?    today\\
    \translate{There will be good music at the party today.} (Line 12)
    
    \ex\label{ex:text.13} K'usi `oy `ox ta [achan-vun] volje? \\
    \translate{What did you have at school yesterday?} (Line 13)
    
    \ex\label{ex:text.14} 
    \glt Bu `oy    `ox             k'op nax?  \\
             exist  \category{pst} talk today \\
    \translate{Where was the talk today?} (Line 14)
    
    \ex\label{ex:text.15} 
    \gll Ch'abal `ox    schi'il li Romin   e junbai. \\
         ?        exist ?       ?  Domingo   last.year\\
    \translate{Last year Domingo had no friend.} (Line 15)
\end{exe}

\end{document}