\documentclass[12pt]{article}

\usepackage{libertinus}
\usepackage{geometry}
\usepackage{titling}
\usepackage{titlesec}
\usepackage{paralist}
\usepackage{multicol} 
\usepackage{footnote}
\usepackage{marginnote}
\usepackage{enumerate}
\usepackage{amsmath, amssymb, amsthm}
\usepackage{gb4e}
\noautomath
\usepackage{bbm}
\usepackage{soul}
\usepackage{graphicx}
\usepackage{siunitx}
\usepackage[table,xcdraw]{xcolor}
\usepackage{tikz}
\usepackage[ruled, vlined, linesnumbered, noend]{algorithm2e}
\usepackage{xr-hyper}
\usepackage[colorlinks]{hyperref} % linkcolor=black, anchorcolor=black, citecolor=black, filecolor=black
\usepackage[most]{tcolorbox}
\usepackage{caption}
\usepackage{subcaption}
\usepackage{booktabs}
\usepackage{multirow}
\usepackage[figuresright]{rotating}
\usepackage{acro}
\usepackage[round]{natbib} 
\usepackage{prettyref}

\geometry{left=3.18cm,right=3.18cm,top=2.54cm,bottom=2.54cm}
\titlespacing{\paragraph}{0pt}{1pt}{10pt}[20pt]
\setlength{\droptitle}{-5em}

\DeclareMathOperator{\timeorder}{\mathcal{T}}
\DeclareMathOperator{\diag}{diag}
\DeclareMathOperator{\legpoly}{P}
\DeclareMathOperator{\primevalue}{P}
\DeclareMathOperator{\sgn}{sgn}
\newcommand*{\ii}{\mathrm{i}}
\newcommand*{\ee}{\mathrm{e}}
\newcommand*{\const}{\mathrm{const}}
\newcommand*{\suchthat}{\quad \text{s.t.} \quad}
\newcommand*{\argmin}{\arg\min}
\newcommand*{\argmax}{\arg\max}
\newcommand*{\normalorder}[1]{: #1 :}
\newcommand*{\pair}[1]{\langle #1 \rangle}
\newcommand*{\fd}[1]{\mathcal{D} #1}

\newcommand*{\citesec}[1]{\S~{#1}}
\newcommand*{\citechap}[1]{chap.~{#1}}
\newcommand*{\citefig}[1]{Fig.~{#1}}
\newcommand*{\citetable}[1]{Table~{#1}}
\newcommand*{\citefootnote}[1]{footnote~{#1}}

\newrefformat{sec}{\citesec{\ref{#1}}}
\newrefformat{fig}{\citefig{\ref{#1}}}
\newrefformat{tbl}{\citetable{\ref{#1}}}
\newrefformat{chap}{\citechap{\ref{#1}}}
\newrefformat{infobox}{Box~\ref{#1}}

\usetikzlibrary{arrows,shapes,positioning}
\usetikzlibrary{arrows.meta}
\usetikzlibrary{decorations.markings}
\tikzstyle arrowstyle=[scale=1]
\tikzstyle directed=[postaction={decorate,decoration={markings,
    mark=at position .5 with {\arrow[arrowstyle]{stealth}}}}]
\tikzstyle ray=[directed, thick]
\tikzstyle dot=[anchor=base,fill,circle,inner sep=1pt]


\tcbuselibrary{skins, breakable, theorems}

\newtcbtheorem{infobox}{Box}{
    enhanced,
    boxrule=0pt,
    colback=blue!5,
    colframe=blue!5,
    coltitle=blue!50,
    borderline west={4pt}{0pt}{blue!65},
    sharp corners,
    fonttitle=\bfseries, 
    breakable,
    before upper={\parindent15pt\noindent}}{box}

\newcommand*{\concept}[1]{\textbf{#1}}
\newcommand*{\term}[1]{\emph{#1}}
\newcommand*{\form}[1]{\emph{#1}}
\newcommand{\translate}[1]{`#1'}

\newcommand*{\vP}{\textit{v}P}

\DeclareAcronym{plural}{short = PL, long = plural}
\DeclareAcronym{blt}{short = BLT, long = Basic Linguistic Theory}
\DeclareAcronym{cgel}{short = CGEL, long = The Cambridge Grammar of the English Language}
\DeclareAcronym{dm}{short = DM, long = Distributed Morphology}
\DeclareAcronym{tag}{long = Tree-adjoining grammar, short = TAG}
\DeclareAcronym{sfp}{long = sentence final particle, short = SFP}
\DeclareAcronym{vp}{long = verb phrase, short = VP}
\DeclareAcronym{cls}{long = classifier, short = CLS}
\DeclareAcronym{dist}{long = distal, short = DIST}
\DeclareAcronym{prox}{long = proximate, short = PROX}
\DeclareAcronym{dem}{long = demonstrative, short = DEM}
\DeclareAcronym{what}{long = \term{wh}-pronoun, short = WH}
\DeclareAcronym{dur}{long = durative, short = DUR}
\DeclareAcronym{neg}{long = negative, short = NEG}
\DeclareAcronym{tame}{long = {Tense, Aspect, Mood, Evidentiality}, short = TAME}
\DeclareAcronym{past}{long = past, short = PST}
\DeclareAcronym{future}{long = future, short = FUT}

% Disable unsupported commands in bookmark titles 
\pdfstringdefDisableCommands{%
  \def\\{}%
  \def\texttt#1{<#1>}%
  \def\mathbb#1{#1}%
}
\pdfstringdefDisableCommands{\def\eqref#1{(\ref{#1})}}

\makeatletter
\pdfstringdefDisableCommands{\let\HyPsd@CatcodeWarning\@gobble}
\makeatother

\title{Solving Problem 1 in the Third International Linguistics Olympiad (2005)}
\author{Jinyuan Wu}

\begin{document}

\automath

\maketitle

\section{Introduction}

The Tzotzil language belongs to the Mayan family,
and its grammar is the focus of the first problem 
in the third International Linguistic Olympiad.
15 translated sentences are given.
In this note I give a bottom-up description 
based on information available in the texts given
in the framework of \ac{blt}.

Wordhood is taken for granted in the discussion below;
fortunately in the texts given,
no compound words or similar constructions appears, 
and therefore the definition of wordhood 
is a purely realizational issue.



\section{Parts of speech}

\subsection{Nouns}

\begin{exe}
    \ex\label{ex:n-school} \form{chan-vun} \translate{school} 
    \begin{xlist}
        \ex `Oy [chan-vun] ta batz'i k'op ta Jobel. \\
        \translate{There is a Tzotzil school in San Cristobal.} (Line 5)
        
        \ex Mi `oy `ox [chan-vun] ta Jobel junabi? \\
        \translate{Was there a school in San Cristobal last year?} (Line 9)
        
        \ex K'usi `oy `ox ta [achan-vun] volje? \\
        \translate{What did you have at school yesterday?} (Line 13)
    \end{xlist}

    \ex\label{ex:n-speech} \form{k'op} \translate{speech}
    \begin{xlist}
        \ex `Oy chan-vun ta batz'i [k'op] ta Jobel. \\
        \translate{There is a Tzotzil school (lit. school in Tzotzil talk) in San Cristobal.}
        
        \ex Bu `oy `ox [k'op] nax? \\
        \translate{Where was the talk today?}
    \end{xlist}
    
    \ex \form{`ixim} \translate{corn}
    \begin{xlist}
        \ex `Oy `ox [`ixim] ta ana nax. \\
        \translate{You had corn at home today.} (Line 1)
        
        \ex Mi `oy [`ixim] ta p'in lavie? \\
        \translate{Is there corn in his pot?} (Line 10)
    \end{xlist}

    \ex \form{na} \translate{house, home} 
    \begin{xlist}
        \ex `Oy `ox `ixim ta [ana] nax. \\
        \translate{You had corn at home today.} (Line 1)
        
        \ex `Oy `ox jlekii [na] po'ot. \\
        \translate{I will soon have a good house.}
    \end{xlist}
\end{exe}

The orthography convention 
requires that proper names are capitalized,
which makes interpreting the sentences much easier.

\subsection{Adverbs}

Adverbs can be found at the end of a sentence.

\begin{exe}
    \ex \form{junbai} \translate{last year}
    \begin{xlist}
        \ex Mi `oy `ox chan-vun ta Jobel [junabi]? \\
        \translate{Was there a school in San Cristobal last year?} (Line 9)
        \ex Ch'abal `ox schi'il li Romin e [junbai]. \\
        \translate{Last year Domingo had no friend.} (Line 15)
    \end{xlist}

    \ex \form{lavie} \translate{from now on today}
    \begin{xlist}
        \ex `Oy `ox lekil vob ta k'in [lavie]. \\
        \translate{There will be good music at the party today.} (Line 12)
        \ex Mi `oy `ixim ta p'in lavie? \\
        \translate{Is there corn in his pot?} (Line 10)
    \end{xlist}
    
    \ex \form{nax} \translate{earlier today}
    \begin{xlist}
        \ex `Oy `ox `ixim ta ana [nax]. \\
        \translate{You had corn at home today.} (Line 1)
        \ex Bu `oy `ox k'op [nax]? \\
        \translate{Where was the talk today?} (Line 14)
    \end{xlist}

    \ex \form{po`ot} \translate{soon} 

    \ex \form{`ok'ob} \translate{tomorrow}
\end{exe}

\subsection{Verbs} 

It seems the word \form{'ox} appearing frequently after \form{`oy} 
is a tense marker of the non-present tense,
including both past and future.

\subsection{Particles}

\form{ta}

\section{The nominal system}

\subsection{Prepositions}

\begin{exe}
    \ex \gll `Oy [chan-vun ta batz'i k'op] [ta [Jobel]_{\text{proper name}}]. \\
    {} school in {} talk in San.Cristobal \\
    \glt \translate{There is a Tzotzil school in San Cristobal.} (Line 5;
    also see \ref{ex:n-school}, \ref{ex:n-speech})
    \ex 
\end{exe}

\section{The verbal system}



\section{Annotated texts}

\end{document}