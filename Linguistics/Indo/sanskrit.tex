\PassOptionsToPackage{table}{xcolor}
\documentclass[a4paper, oneside, 12pt]{report}

\usepackage[T1]{fontenc}
\usepackage{libertinus}
\usepackage{geometry}
\usepackage{float}
\usepackage{titling}
\usepackage{titlesec}
\usepackage{paralist}
\usepackage{footnote}
\usepackage[colorinlistoftodos]{todonotes}
\usepackage[inline]{enumitem}
\usepackage{amsmath, amsthm}
\usepackage{gb4e}
\noautomath
\usepackage{bbm}
\usepackage{textcomp}
\usepackage{soul}
\usepackage{graphicx}
\usepackage{siunitx}
\usepackage{tikz}
\usepackage[ruled, vlined, linesnumbered, noend]{algorithm2e}
\usepackage[colorlinks, citecolor = purple, bookmarksnumbered]{hyperref} % linkcolor=black, anchorcolor=black, citecolor=black, filecolor=black
\usepackage[most]{tcolorbox}
\usepackage{caption}
\usepackage{subcaption}
\usepackage{booktabs}
\usepackage{multirow}
\usepackage[figuresright]{rotating}
\usepackage{acro}
\usepackage[round]{natbib} 
\usepackage{prettyref}

\geometry{left=3.18cm,right=3.18cm,top=2.54cm,bottom=2.54cm}
\titlespacing{\paragraph}{0pt}{1pt}{10pt}[20pt]
\setlength{\droptitle}{-5em}

\DeclareMathOperator{\timeorder}{\mathcal{T}}
\DeclareMathOperator{\diag}{diag}
\DeclareMathOperator{\legpoly}{P}
\DeclareMathOperator{\primevalue}{P}
\DeclareMathOperator{\sgn}{sgn}
\newcommand*{\ii}{\mathrm{i}}
\newcommand*{\ee}{\mathrm{e}}
\newcommand*{\const}{\mathrm{const}}
\newcommand*{\suchthat}{\quad \text{s.t.} \quad}
\newcommand*{\argmin}{\arg\min}
\newcommand*{\argmax}{\arg\max}
\newcommand*{\normalorder}[1]{: #1 :}
\newcommand*{\pair}[1]{\langle #1 \rangle}
\newcommand*{\fd}[1]{\mathcal{D} #1}

\newcommand*{\citesec}[1]{\S~{#1}}
\newcommand*{\citechap}[1]{Ch~{#1}}
\newcommand*{\citefig}[1]{Fig.~{#1}}
\newcommand*{\citetable}[1]{Table~{#1}}
\newcommand*{\citepage}[1]{p.~{#1}}
\newcommand*{\citepages}[1]{pp.~{#1}}
\newcommand*{\citefootnote}[1]{fn.~{#1}}
\newcommand*{\citechapsec}[2]{\citechap{#1}.\citesec{#2}}

\newrefformat{sec}{\citesec{\ref{#1}}}
\newrefformat{fig}{\citefig{\ref{#1}}}
\newrefformat{tbl}{\citetable{\ref{#1}}}
\newrefformat{chap}{\citechap{\ref{#1}}}
\newrefformat{fn}{\citefootnote{\ref{#1}}}
\newrefformat{box}{Box~\ref{#1}}
\newrefformat{ex}{\ref{#1}}


% color boxes

\tcbuselibrary{skins, breakable, theorems}

\AtBeginEnvironment{infobox}{\small}
\AtBeginEnvironment{theorybox}{\small}

\newtcbtheorem[number within=chapter]{infobox}{Box}{
    enhanced,
    boxrule=0pt,
    colback=blue!5,
    colframe=blue!5,
    coltitle=blue!50,
    borderline west={4pt}{0pt}{blue!65},
    sharp corners,
    fonttitle=\bfseries, 
    breakable,
    before upper={\parindent15pt\noindent}}{box}
\newtcbtheorem[number within=chapter, use counter from=infobox]{theorybox}{Box}{
    enhanced,
    boxrule=0pt,
    colback=orange!5, 
    colframe=orange!5, 
    coltitle=orange!50,
    borderline west={4pt}{0pt}{orange!65},
    sharp corners,
    fonttitle=\bfseries, 
    breakable,
    before upper={\parindent15pt\noindent}}{box}
\newtcbtheorem[number within=chapter, use counter from=infobox]{learnbox}{Box}{
    enhanced,
    boxrule=0pt,
    colback=green!5,
    colframe=green!5,
    coltitle=green!50,
    borderline west={4pt}{0pt}{green!65},
    sharp corners,
    fonttitle=\bfseries, 
    breakable,
    before upper={\parindent15pt\noindent}}{box}

% Shorthands
\newcommand*{\concept}[1]{\textbf{#1}}
\newcommand*{\term}[1]{\emph{#1}}
\newcommand{\form}[1]{\emph{#1}}

\newcommand{\redp}{\textasciitilde}

\newcommand{\deictictime}{T$_{\text{d}}$}
\newcommand{\referredtime}{T$_{\text{r}}$}
\newcommand{\orientationtime}{T$_{\text{o}}$}

\DeclareAcronym{blt}{short = BLT, long = Basic Linguistic Theory}
\DeclareAcronym{cgel}{short = CGEL, long = The Cambridge Grammar of the English Language}
\DeclareAcronym{dm}{short = DM, long = Distributed Morphology}
\DeclareAcronym{tag}{long = Tree-adjoining grammar, short = TAG}
\DeclareAcronym{sfp}{long = sentence-final particle, short = \textsc{sfp}}
\DeclareAcronym{np}{long = noun phrase, short = NP}
\DeclareAcronym{vp}{long = verb phrase, short = VP}
\DeclareAcronym{pp}{long = preposition phrase, short = PP}
\DeclareAcronym{advp}{long = adverb phrase, short = AdvP}
\DeclareAcronym{cls}{long = classifier, short = CLS}
\DeclareAcronym{dist}{long = distal, short = DIST}
\DeclareAcronym{prox}{long = proximate, short = PROX}
\DeclareAcronym{dem}{long = demonstrative, short = DEM}
\DeclareAcronym{classify}{long = classifier, short = \textsc{cl}}
\DeclareAcronym{dur}{long = durative, short = DUR}
\DeclareAcronym{neg}{long = negative, short = \textsc{neg}}
\DeclareAcronym{cc}{long = copular complement, short = CC}
\DeclareAcronym{cs}{long = copular subject, short = CS}
\DeclareAcronym{tam}{long = {tense, aspect, mood}, short = TAM}
\DeclareAcronym{past}{long = past, short = PST}
\DeclareAcronym{nonpast}{long = non-past, short = NPST}
\DeclareAcronym{present}{long = present, short = PRES}
\DeclareAcronym{progressive}{long = progressive, short = \textsc{poss}}
\DeclareAcronym{perfect}{long = perfect, short = \textsc{perf}}
\DeclareAcronym{passive}{long = passive, short = \textsc{pass}}
\DeclareAcronym{copula}{long = copula, short = COP}
\DeclareAcronym{possessive}{long = possessive, short = \textsc{poss}}
\DeclareAcronym{coca}{long = Corpus of Contemporary American English, short = COCA}

\newcommand{\asis}[1]{\textsc{#1}}
\newcommand{\oneof}[1]{{#1}}
\newcommand*{\homo}[2]{#1$_{\text{#2}}$}
\newcommand{\category}[1]{\textsc{#1}}
\newcommand{\formcat}[1]{\textsc{#1}}
\newcommand{\emptymorpheme}{$\emptyset$}
\newcommand*{\fromto}[2]{\langle {#1}, {#2} \rangle}
\newcommand*{\source}[1]{\textit{#1}}

\newcommand{\alignment}{\href{../alignment/alignment.pdf}{my notes about alignment}}
\newcommand{\method}{\href{../methodology/glossing.pdf}{this note about my understanding of descriptive grammars}}

\newcommand{\ala}{à la}
\newcommand{\translate}[1]{`#1'}
\newcommand{\vP}{\textit{v}P}

% Make subsubsection labeled
\setcounter{secnumdepth}{4}
\setcounter{tocdepth}{4}
% reset example counter every chapter (but do not include the chapter number to the label)
\counterwithin{exx}{chapter} 

% Reference formats
\renewcommand{\bibname}{References}
\setcitestyle{aysep={}} 

% List format
\setlist[enumerate,1]{label=\alph*\upshape)}

\title{Notes on Sanskrit grammar}
\author{Jinyuan Wu}

\begin{document}

\maketitle

\chapter{Introduction}

\section{Classification and history}

Sanskrit is one of the oldest attested Indo-European languages.
It is the classical and liturgical language of Hinduism.
It belongs to the Indo-Aryan group,
which also contains Avestan, the classical language of Zoroastrianism
with limited mutual intelligibility with Vedic Sanskrit,
the earliest form of Sanskrit.

From the Vedic texts, it is clear that dialects already existed at the Vedic ages,
although textual evidence for that is elusive \citep{witzel1989tracing}.

Vedic Sanskrit underwent some changes in its evolution into Classical Sanskrit,
or \emph{Sanskrit} in the narrow sense \citep{burrow2001sanskrit}.
Indeed, the English name \form{Sanskrit} comes from its native name \form{saṃskṛtá},
which can be analyzed as the adjectivizer \form{-tá} applied to the compound \form{saṃskṛ-},
which in turn is \form{sam} (\translate{together}) 
plus \form{skṛ} (\translate{do}).
The name makes it clear that Sanskrit had historically gone through intentional standardization.

\section{Speakers}

Sanskrit, being the classical and religious language, 
is one of the scheduled languages in the constitution of India.
Despite its sacred and official status, currently Sanskrit has almost zero native speakers.
Alleged native speakers include those whose reported themselves as Sanskrit native speakers in censuses
and residents of some so-called ``Sanskrit'' villages which are often under Sanskrit revivalization projects.
In these villages, however, exaggeration of Sanskrit capacities of the speaker can be observed,
and a large proportion, if not all, of everyday conservation is carried out in vernaculars.
Reports that there are sizable Sanskrit communities may also be politically motivated
\citep{mccartney2017jhiri}.
The situation is comparable to other classical languages:
continuous education of the language exists widely,
and some may even be fluent enough for oral, spontaneous discourses,
but this diverges considerably from the situation of a true native language. 



\section{Previous studies}

\subsection{The ancient India grammatical tradition}

The grammar of Sanskrit is unique in that it has long been semi-formalized 
by the brilliant ancient grammarian Pāṇini.
His great work \source{Aṣṭādhyāyī} contains around 4,000 \form{sutras} (\translate{lit. string}).
The word \form{sutra} here means a short, brief rule:
this is different from how the word is used in Buddhism,
where it means a complete discourse by Buddha.
A Paninian \form{sutra} is different from a rule in modern linguistics
in that the latter is usually written in a self-contained form
(for example ``in Latin first conjugation, the \form{-\={a}} in the stem disappears 
before the first person singular ending \form{-\={o}}''),
but in Pāṇini's grammar, a \form{sutra} may look like two or three words with no reasonable meaning:
only after previous \form{sutras} are concatenated into the current \form{sutra} do we get a complete rule.
\source{Aṣṭādhyāyī} contains some (meta-linguistic) mechanisms to make sure
every \form{sutra} can be placed in the right context and interpreted.

The Paninian \form{sutras} of course contain lots of grammatical terms.
Some terms are normal Sanskrit nouns,
like \form{kartṛ} \translate{agent} 
(\form{kṛ-} \translate{to do} plus \form{-kṛ} \translate{agent suffix}).
Some terms are not regular Sanskrit nouns;
they are actually expressions referring to a subsequence of the text \form{Śivasūtras},
disguised as Sanskrit nouns with the usual case endings and Sandhi 
(which makes it much harder to beginners to understand,
because you need to first understand a little about Sanskrit to understand the rules).
The \form{Śivasūtras} are a set of fourteen sequences,
here written in one line in (\ref{ex:sivasutras}).
When a \form{sutra} refers to a noun \form{aK},
it meas the sequence \form{a, i, u, ṛ, ḷ},
i.e. the sequence from \form{a} to \form{K}.
Therefore to refer to all vowels, Pāṇini would use the noun \form{aṄ},
and all legitimate syllables would be referred to as \form{aL},
while the two diphthongs \form{ai} and \form{au} can be collectively referred to by \form{aiC}.
Note that the symbols \form{Ṇ}, \form{K}, \form{Ṅ}, etc.
are purely metalinguistic: 
they are just there to enable us to select a subsequence of \form{Śivasūtras}.
Pāṇini himself did not use capital letter, of course;
the capital letters are used by modern Indologists to make distinguishing between actual Sanskrit syllables and metalinguistic symbols easier.

\begin{exe}
    \ex\label{ex:sivasutras} a i u Ṇ
    ṛ ḷ K
    e o Ṅ
    ai au C
    ha ya va ra Ṭ
    la Ṇ
    ña ma ṅa ṇa na M
    jha bha Ñ
    gha ḍha dha Ṣ
    ja ba ga ḍa da Ś
    kha pha cha ṭha tha ca ṭa ta V
    ka pa Y
    śa ṣa sa R
    ha L
\end{exe}


To refer to a single sound, the metalinguistic symbol \form{T} is attached after that sound,
and the result again is disguised as a normal Sanskrit noun.
We can also combine several syllable sequence expressions into one:
this is what happens in the first \form{sutra}.
In this \form{sutra} (\ref{ex:sutra-1-1-1}),
the word \form{ādaic} is actually the compound \form{āt-aic} after sandhi 
in its nominative singular masculine form,
and \form{āt}, or more precisely, \form{āT}, means the sound \form{ā}, 
while \form{aiC} means \form{ai} and \form{au},
and eventually we find \form{ādaic} mean the three sounds: \form{ā}, \form{ai} and \form{au}.
The whole \form{sutra} is a definition:
\translate{\form{ā}, \form{ai} and \form{au} are called \form{vṛddhi}}.

\begin{exe}
    \ex\label{ex:sutra-1-1-1} vṛddhir ādaic
\end{exe}

\paragraph*{Morphophonology} Pāṇini mostly focused on morphophonology.
Paninian morphological rules, similar to their contemporary counterparts,
take the form of term rewriting. 
The most common form is $A \to B / C \underline{\quad} D $,
meaning that $A$ is to be rewritten into $B$ if surrounded by $C$ and $D$.
Pāṇini himself did not use the modern symbols like the arrow or the underscore.
He labeled $A, B, C$ and $D$ using Sanskrit cases:
$A$ is in genitive case, $B$ is nominative, $C$ is ablative, and $D$ is locative
\citep{kiparsky1995painian}.

Among syntactic topics that are touched in \source{Aṣṭādhyāyī},
we have a dependency-based analysis of clausal structure
some discussions about \ac{tam} categories,
and, if we consider derivational morphology to be closely related to syntax,
rules pertaining to compounding.

\paragraph*{Alignment}
The Paninian concept of Sanskrit clausal structure is a \emph{flat} dependency tree:
the notion of the ``tightness'' or ``closeness'' of a dependency arc is absent.
Specifically, Pāṇini did not establish mention subjecthood in his grammar \citep{kiparsky2009architecture}.
The subject-verb agreement is treated as if Sanskrit had focus-trigger alignment.
Pāṇini assumed that 
\begin{enumerate}
    \item each argument has one and only one role (\form{kārakas}), and
    \item the verb ending (\form{la}) either indexes one argument or it indexes the process denoted by the clause, and
    \item a role is to be expressed once and only once by a nominal case ending (accusative, instrumental, etc.) 
    or the verb ending (active = agent, passive = goal, etc.), and
    \item the (morphological) nominative case tells us nothing about the role of the nominative argument.
\end{enumerate}
It can be seen that case determination is triggered by what the verb ending chooses to index:
the ``no double argument role exponents'' rule then pushes the indexed argument to the nominative case,
and all other arguments receive their cases according to their roles.

For example, these constraints are sufficient to derive passivization,
because if the goal/patient is indexed by the verb ending,
this means the verb gets a passive ending,
and since the role goal/patient has already been indexed once,
it can no longer appear in accusative case,
so the goal/patient argument appears as the nominative argument.
The agent, then, cannot be nominative or otherwise its role in the clause is not expressed,
so it takes instrumental case to show that it is the agent.

\paragraph*{Problems of Pāṇini's account}
This description of alignment is adequate for case determination without introducing the concept of the subject.
This description however sounds more or less problematic to a modern reader.
First, the requirement that the roles of the arguments are expressed once and only once is utterly not motivated:
Sanskrit has number and person agreement in both noun phrases and clauses,
and no sensible grammarian would argue that there is any single-exponent condition there.
The second problem is the treatment of passivization: 
in other Indo-European languages we usually do not understand the passive ending
as an expression of ``patient-ness''.
The third problem is Sanskrit is not really a focus-trigger language after all,
and a precise description of Sanskrit needs to mention the preference given to the active voice.

The most serious problem is Pāṇini mentions no ``deep'' valency alternation.
Consider the English sentences \form{he killed someone using that knife} 
and \form{that knife killed someone},
and the fact that passivization can happen for both sentences:
in the second example, grammatically \form{that knife} is no longer an instrument argument,
or otherwise it cannot be passivized.
The same phenomenon happens in Sanskrit.
This, then, means that the statuses of the argument roles are not completely symmetric:
other arguments can be promoted to the syntactic agent position, or in other words the \form{kartṛ},
but the inverse is not possible.
We may say \form{kartṛ} is some sort of internal subject,
i.e. the most prominent or the most external argument before possible passivization.
The commentator Patañjali noticed this phenomenon and
acknowledged that the Sanskrit equivalence of \form{that knife} is indeed the \form{kartṛ},
and the alternation of valency here is due to \form{vivakṣā} \translate{communicative intention}
\citep{keidan2017subjecthood}.
But once we agree that the notion of an internal subject exists in Sanskrit,
we are on a slippery slope that eventually leading to acknowledging 
existence of a clausal subject.

The strangeness of the Paninian \form{kārakas}-based calculus may arise from the fact
that the original target audience of \form{Aṣṭādhyāyī} were not people already familiar with a language that was close enough to Sanskrit,
but people whose native language had verbal coding of arguments and
presumably an alignment system that was not prototypically nominative-accusative
\citep{keidan2017subjecthood}.
Or possibly the calculus presented in \form{Aṣṭādhyāyī} is more an ad-hoc attempt to summarize Sanskrit alignment
in a form as concise as possible.

\chapter{Grammatical overview}

It has been claimed by several authors that subjecthood is not a useful grammatical category in Sanskrit grammar;
the concept is not used by Pāṇini, afterall \citep{keidan2017subjecthood,kiparsky2009architecture}. 


\bibliographystyle{plainnat}
\bibliography{sanskrit.bib}

\end{document}