\documentclass[a4paper, oneside, 12pt]{report}

\usepackage[T1]{fontenc}
\usepackage{libertinus}
\usepackage{underscore}
\usepackage{geometry}
\usepackage{float}
\usepackage{titling}
\usepackage{titlesec}
\usepackage{paralist}
\usepackage{footnote}
\usepackage{enumitem}
\setitemize{noitemsep,topsep=0pt,parsep=0pt,partopsep=0pt}
\usepackage{amsmath, amsthm}
\usepackage{gb4e}
\noautomath
\usepackage{bbm}
\usepackage{textcomp}
\usepackage{soul}
\usepackage{graphicx}
\usepackage{siunitx}
\usepackage[table,xcdraw]{xcolor}
\usepackage{tikz}
\usepackage[ruled, vlined, linesnumbered, noend]{algorithm2e}
\usepackage{xr-hyper}
\usepackage[colorlinks, citecolor = purple, bookmarksnumbered]{hyperref} % linkcolor=black, anchorcolor=black, citecolor=black, filecolor=black
\usepackage[most]{tcolorbox}
\usepackage{caption}
\usepackage{subcaption}
\usepackage{booktabs}
\usepackage{multirow}
\usepackage[figuresright]{rotating}
\usepackage{acro}
\usepackage[round]{natbib} 
\usepackage{prettyref}

\geometry{left=3.18cm,right=3.18cm,top=2.54cm,bottom=2.54cm}
\titlespacing{\paragraph}{0pt}{1pt}{10pt}[20pt]
\setlength{\droptitle}{-5em}

\DeclareMathOperator{\timeorder}{\mathcal{T}}
\DeclareMathOperator{\diag}{diag}
\DeclareMathOperator{\legpoly}{P}
\DeclareMathOperator{\primevalue}{P}
\DeclareMathOperator{\sgn}{sgn}
\newcommand*{\ii}{\mathrm{i}}
\newcommand*{\ee}{\mathrm{e}}
\newcommand*{\const}{\mathrm{const}}
\newcommand*{\suchthat}{\quad \text{s.t.} \quad}
\newcommand*{\argmin}{\arg\min}
\newcommand*{\argmax}{\arg\max}
\newcommand*{\normalorder}[1]{: #1 :}
\newcommand*{\pair}[1]{\langle #1 \rangle}
\newcommand*{\fd}[1]{\mathcal{D} #1}
\newcommand*{\textto}{$\to$}
\newcommand*{\textgt}{$>$ }

\newcommand*{\citesec}[1]{\S~{#1}}
\newcommand*{\citechap}[1]{Ch~{#1}}
\newcommand*{\citefig}[1]{Fig.~{#1}}
\newcommand*{\citetable}[1]{Table~{#1}}
\newcommand*{\citepage}[1]{p.~{#1}}
\newcommand*{\citepages}[1]{pp.~{#1}}
\newcommand*{\citefootnote}[1]{fn.~{#1}}
\newcommand*{\citechapsec}[2]{\citechap{#1}.\citesec{#2}}

\newrefformat{sec}{\citesec{\ref{#1}}}
\newrefformat{fig}{\citefig{\ref{#1}}}
\newrefformat{tbl}{\citetable{\ref{#1}}}
\newrefformat{chap}{\citechap{\ref{#1}}}
\newrefformat{fn}{\citefootnote{\ref{#1}}}
\newrefformat{box}{Box~\ref{#1}}
\newrefformat{ex}{\ref{#1}}


% color boxes

\tcbuselibrary{skins, breakable, theorems}

\AtBeginEnvironment{infobox}{\small}
\AtBeginEnvironment{theorybox}{\small}

\newtcbtheorem[number within=chapter]{infobox}{Box}{
    enhanced,
    boxrule=0pt,
    %colback=blue!5,
    %colframe=blue!5,
    colback=white,
    colframe=white,
    coltitle=blue!60,
    borderline west={4pt}{0pt}{blue!65},
    sharp corners,
    fonttitle=\bfseries, 
    breakable,
    before upper={\parindent15pt\noindent}}{box}
\definecolor{my-orange}{HTML}{F58123}
\newtcbtheorem[number within=chapter, use counter from=infobox]{theorybox}{Box}{
    enhanced,
    boxrule=0pt,
    %colback=orange!5, 
    %colframe=orange!5, 
    colback=white,
    colframe=white,
    coltitle=my-orange!80,
    borderline west={4pt}{0pt}{my-orange!80},
    sharp corners,
    fonttitle=\bfseries, 
    breakable,
    before upper={\parindent15pt\noindent}}{box}
\newtcbtheorem[number within=chapter, use counter from=infobox]{learnbox}{Box}{
    enhanced,
    boxrule=0pt,
    colback=green!5,
    colframe=green!5,
    coltitle=green!50,
    borderline west={4pt}{0pt}{green!65},
    sharp corners,
    fonttitle=\bfseries, 
    breakable,
    before upper={\parindent15pt\noindent}}{box}
\newtcbtheorem[number within=chapter, use counter from=infobox]{todobox}{Box}{
    enhanced,
    boxrule=0pt,
    colback=red!5,
    colframe=red!5,
    coltitle=red!50,
    borderline west={4pt}{0pt}{red!65},
    sharp corners,
    fonttitle=\bfseries, 
    breakable,
    before upper={\parindent15pt\noindent}}{box}

% Shorthands
\newcommand*{\concept}[1]{\textbf{#1}}
\newcommand*{\term}[1]{\emph{#1}}
\newcommand{\form}[1]{\emph{#1}}

\newcommand{\redp}{\textasciitilde}

\newcommand{\deictictime}{T$_{\text{d}}$}
\newcommand{\referredtime}{T$_{\text{r}}$}
\newcommand{\orientationtime}{T$_{\text{o}}$}

\DeclareAcronym{blt}{short = BLT, long = Basic Linguistic Theory}
\DeclareAcronym{cgel}{short = CGEL, long = The Cambridge Grammar of the English Language}
\DeclareAcronym{dm}{short = DM, long = Distributed Morphology}
\DeclareAcronym{tag}{long = Tree-adjoining grammar, short = TAG}
\DeclareAcronym{sfp}{long = sentence-final particle, short = sfp}
\DeclareAcronym{np}{long = noun phrase, short = NP}
\DeclareAcronym{vp}{long = verb phrase, short = VP}
\DeclareAcronym{pp}{long = preposition phrase, short = PP}
\DeclareAcronym{advp}{long = adverb phrase, short = AdvP}
\DeclareAcronym{cls}{long = classifier, short = CLS}
\DeclareAcronym{dist}{long = distal, short = DIST}
\DeclareAcronym{prox}{long = proximate, short = PROX}
\DeclareAcronym{dem}{long = demonstrative, short = DEM}
\DeclareAcronym{classify}{long = classifier, short = \textsc{cl}}
\DeclareAcronym{dur}{long = durative, short = DUR}
\DeclareAcronym{neg}{long = negative, short = \textsc{neg}}
\DeclareAcronym{cc}{long = copular complement, short = CC}
\DeclareAcronym{cs}{long = copular subject, short = CS}
\DeclareAcronym{tame}{long = {tense, aspect, mood, evidentiality}, short = TAME}
\DeclareAcronym{past}{long = past, short = PST}
\DeclareAcronym{nonpast}{long = non-past, short = NPST}
\DeclareAcronym{present}{long = present, short = PRES}
\DeclareAcronym{progressive}{long = progressive, short = \textsc{poss}}
\DeclareAcronym{perfect}{long = perfect, short = \textsc{perf}}
\DeclareAcronym{passive}{long = passive, short = \textsc{pass}}
\DeclareAcronym{copula}{long = copula, short = COP}
\DeclareAcronym{possessive}{long = possessive, short = \textsc{poss}}
\DeclareAcronym{coca}{long = Corpus of Contemporary American English, short = COCA}
\DeclareAcronym{sap}{long = speech act participant, short = SAP}

\newcommand{\asis}[1]{\textsc{#1}}
\newcommand{\oneof}[1]{{#1}}
\newcommand*{\homo}[2]{#1$_{\text{#2}}$}
\newcommand{\category}[1]{\textsc{#1}}
\newcommand{\formcat}[1]{\textsc{#1}}
\newcommand{\emptymorpheme}{$\emptyset$}
\newcommand*{\fromto}[2]{\langle {#1}, {#2} \rangle}

\newcommand{\alignment}{\href{../alignment/alignment.pdf}{my notes about alignment}}
\newcommand{\exerciseone}{\href{../Exercise/2021-3.pdf}{this exercise}}
\newcommand{\method}{\href{../methodology/glossing.pdf}{this note about my understanding of descriptive grammars}}

\newcommand{\ala}{à la}
\newcommand{\translate}[1]{`#1'}
\newcommand{\vP}{\textit{v}P}

% Make subsubsection labeled
\setcounter{secnumdepth}{4}
\setcounter{tocdepth}{4}
% reset example counter every chapter (but do not include the chapter number to the label)
\counterwithin{exx}{chapter} 

% Reference formats
\renewcommand{\bibname}{References}
\setcitestyle{aysep={}} 

% List format
\setlist[enumerate,1]{label=\alph*\upshape)}

\title{Modern Hebrew grammar}
\author{Jinyuan Wu}

\begin{document}

\maketitle

\automath

\chapter{Introduction}

\section{History and classification}

The coverage of the term \term{Hebrew} spans from Iron Age inscriptions to modern Hebrew, i.e. the national language of modern Israel \citep[\citepage{533}]{pat2019semitic}.
It is natural to ask 
(a) whether Hebrew is a monophyletic group,
and especially whether a common ancestor of all varieties of Hebrew exists
in parallel with other Canaanite languages, and 
(b) when considered as a group, what is its genetic relations with other languages.
This section is a brief study on the two questions. 

\subsection{Internal diversity of Hebrew}

Modern Hebrew is a well-documented result of the Hebrew revival movement based on earlier texts, mostly the Hebrew Bible, and the liturgical reading traditions:
the phonology is a hybrid of the Ashkenazi and Sephardi pronunciation traditions \citep[\citepage{574}]{pat2019semitic}.

\begin{todobox}{List of unclassified varieties}{hebrew-varieties}
    \begin{itemize}
        \item Community reading traditions
        \item Mishnaic Hebrew 
        \item Medieval Hebrew 
        \item Biblical Hebrew, 
        \item and inscriptions
    \end{itemize}
\end{todobox}

Mishnaic Hebrew is the language of the Tannaitic literature,
which reflects a living spoken Hebrew 

The problem is therefore reduced to the comparison between Biblical Hebrew and earlier inscriptions.

\subsection{Hebrew in the Semitic language family}

Hebrew clearly belongs to Canaanite languages,
which enjoy more or less mutual intelligibility.
Canaanite demonstrates largely similar morphological paradigms
\citep[\citepages{515-524}]{pat2019semitic}.
What tells Hebrew and Phoenician apart, for instance, mostly consists of minor differences like different consonant inventories (Hebrew has three additional phonemes),
diverging diachronic evolution of the causative and the feminine singular nominal ending,
and different negative and tense markers (\citealt[\citepages{71-76}]{khan2013encyclopedia3}; \citealt[\citepage{513}]{pat2019semitic}).
The difference between (say) Moabite and Hebrew \citep[\citepage{430}]{khan2011semitic} is of the same nature.

It is well established that Canaanite languages form a single branch -- instead of a paraphyletic group --
because of multiple innovative features attested in the Canaanite languages and not elsewhere \citep[\citepages{12,510}]{pat2019semitic}.
These innovations are not attested in Ugaritic or Aramaic \citep[\citepage{460-462,510}]{pat2019semitic}.
The Aramaic group is discussed in \citet{huehnergard1995aramaic}.

Canaanite languages, Aramaic languages, and Ugaritic all belong to the Northwest Semitic language family,
again based on clear shared diagnostic innovations \citep[\citepages{11-13}]{pat2019semitic}.
It has been suggested that Aramaic languages and Canaanite languages seem to have a closer relation, forming a Aramaeo-Canaanite sub-node within the Northwest branch \citep[\citepage{12}]{pat2019semitic}.
A review of the large-scale structure of the Semitic language family can be found in \citet[\citechap{1}]{pat2019semitic}.

\chapter{Typological overview}

\section{Morphological profile}

\subsection{Semitic templatic morphology}

Hebrew, and more generally, Semitic languages or even Afro-Asiatic languages,
are famous for their ``templatic morphology'',
in which a word is built by combining a \concept{template},
i.e. a groups of vowels, sometimes with affixes attached to it and instructions about reduplication,
which contains consonant slots waiting to be filled,
and a \concept{root} i.e. a group of consonants or in other words the \concept{consonant skeleton}.
The template encodes grammatical categories like valency alternation
(the \form{binyan} system) or aspect.

The cross-linguistic status of this rather unusual non-concatenative morphological device has gained wide interest.
That roots without part of speech tags as \emph{abstract} grammatical entities exist cross-linguistically
is relative not controversial:
we have countless examples including Proto-Indo-European roots
and the highly productive roots or ``characters'' (from the perspective of the written language) in modern Sinitic languages
(\prettyref{box:roots-no-pos}).
Semitic languages aren't different in this aspect.
It is also not controversial to say that both in Semitic and non-Semitic languages,
roots frequently undergo regular derivation processes
but still gain fixed semantic interpretation
(consider for instance Arabic \form{kitāb}, which originally was a regularly formed verbal noun),
and thus what is stored in the lexicon is not just the roots and their meanings:
the meanings of derived forms should be stored as well.

\begin{theorybox}{On roots without parts of speech}{roots-no-pos}
    Needless to say, one can argue that certain roots mean actions and are inherently verbal
    while others are inherently nominal as they mean objects.
    But the distinction is more nuanced grammatically.
    Consider the Chinese root 约.
    It can be interpreted as meaning \translate{contract},
    and the verbal meaning \translate{make contract with} derives from the nominal meaning.
    Historically, though, the verbal 约 \translate{confine \textgt make contract with} appeared first.
    None of the two meanings can be reasonably said to be fundamental.
    Moreover, even if one of the meanings is considered to be fundamental,
    we do not have \emph{regular} devices that regularly determine the derived meaning from the original meaning.
    Therefore 约, if acknowledged as a productive entity (it does seem to be one),
    has to be a root whose meaning is determined by the syntactic environment it is in.

    This however does not deny that morphological realization of a root can be directional:
    for instance it is possible that a root's ``original form''
    either looks like a verbal stem or a nominal stem in a language. 
\end{theorybox}

The real controversies surrounding the morphological properties of Semitic languages
are more about the morphological realization side,
where a \term{root} refers to the \emph{consonant skeleton} mentioned above
that allegedly is the phonological content of the root as an abstract entity in question.

\subsubsection{Is there stem-based morphology?}

The first question we should ask is whether \emph{forms} of \emph{stems} formed from the roots (and not just their meanings) are stored in the lexicon as well besides roots
and referred to in morphological processes.
It's theoretically possible that the morphological \emph{forms} are not lexicalized at all,
and if there is anything that is lexicalized,
then it can only be only the \emph{meanings},
like \form{kitāb} example above.
The answer to this question is relatively clear:
in Semitic languages we indeed find stem-based morphology and lexicalization of \form{forms}.

First, it is possible for a stem with affixation to be treated as
a unit without internal structures in skeleton extracting,
meaning that the phonological form of the derived word is lexicalized
at least besides the etymologically simple roots, 
if not in place of them \citep[\citepage{132}]{bolozky2008roots}.

The second piece of evidence of stem-based derivation appears in borrowing.
Modern Hebrew has borrowed considerable words from English and other European languages.
When a borrowed noun undergoes denominal derivation,
two (sometimes competing) factors are relevant:
the argument structure of the denominative verb,
and the tendency to keep the origin of the denominative verb recognizable.
The first decides the vowel pattern of the denominative verb 
(i.e. its \form{binyan}) and also possibly additional affixes,
and the second factor decides where to insert the vowels
\citep[\citepages{133-134}]{bolozky2008roots}.
In this process we observe that there is a strong tendency to 
keep consonant clusters in the noun in situ,
which is observed even for denominative derivation of native bases:
that's to say, a consonant cluster is often regarded as a single consonant
in morphological processes that are often described in the consonant skeleton-based framework,
and when this is not feasible,
a strategy is partial reduplication of the last consonant
\citep[\citepages{134-135}]{bolozky2008roots}.

This, together with the fact vowel deletion happens regularly in Hebrew
and thus forms new consonant clusters regularly,
suggests that a large part of derivational morphology of Hebrew 
previously described using the concept of discontinuous consonant skeletons
can be alternatively described in terms of melodic overwriting of existing vowels:
what is previously perceived as a templatic structure 
may actually be a disyllabic (prosodic?) constraint,
which contains two vowels available for various ablaut processes to manipulate or delete,
while inserting new vowels is not feasible.
This analysis neatly captures the fact that consonant clusters are transferred as a whole
in denominative derivations (as ablaut can't shift the positions of vowels)
and eliminates the need to stipulate a separate cluster preservation rule.

The main obstacle faced by this alternative analysis
is its inability to explain the final consonant reduplication phenomenon
discussed above when there are only two consonant/consonant clusters in a noun
\citep[\citepage{136}]{bolozky2008roots}.
An alternative strategy of ignoring the consonant clusters of triliteral nouns is also attested
\citep[\citepages{139-140}]{bolozky2008roots},


It should however be noted that the observations above are not inconsistent with
the existence of consonant skeletons:
it is possible that in certain consonant skeletons,
a ``consonant slot'' is replaced by a consonant cluster.


\subsubsection{Are there really templates?}

Whether the discontinuous consonant skeletons are correct representations of the mental status of roots
or the correct mental representations of roots (as abstract entities) are closer to what are known as stems
(with some vowels already filled in, and hence certain morphological forms
can be rightfully known as the original forms,
which also leads some to call it \term{word-based} morphology,
although strictly speaking this does not deny the existence of roots as abstract grammatical entities;
vowels dictated by the template are thus assumed to be from ablaut triggered by an abstract affix),


Evidence from frequentative morphology in Ethiopian Semitic languages
seems to suggest that templatic morphology is indeed real:
four-segment outputs of partial and full reduplication as well as the frequentative
have vowel patterns rather similar to that of ordinary quadrilateral roots,
and the vowel quality and gemination patterns can be neatly captured by a templatic analysis.
Nonetheless, the possibility of multiple frequentative formatives renders 
a purely templatic analysis infeasible,%
\footnote{
    One can do the mental gymnastics and argue that the multiple frequentative formatives
    originate from modification of the root first,
    but this requires us to stipulate that there exists a separate process
    to copy vowels in the template as well,
    before things are put together.
    This however is extremely unnatural, if not theoretically infeasible.
}
and the similarity between the frequentative forms of roots with different structures
also strongly suggests an affixal analysis.
We also have both evidence for linguistic reality of roots and derived words:
information from both is needed to gain the correct frequentative form.
In particular, the frequentative of a verb already reduplicated 
sometimes is based on the reduplicated form, and not the root form
\citep{rose2008formation}.
Thus, we conclude that both the templatic morphology
and the inflected words are all real linguistic entities,
at least in certain languages.

\subsubsection{Are there consonant skeletons?}

Still the linguistic reality of templatic morphology
does not mean we need to really stipulate that the phonological forms of roots
are made of discontinuous consonants.
Here we head to the reverse question: if the root is referred to in morphological processes
as frequently as it is perceived.

For instance, the no-overly-similar-consonants rule,
also known as the Obligatory Contour Principle,
dictates that the neighboring consonants in a root shouldn't be too similar,
suggesting a rule targeting the root as a level of morphophonological representations,
and hence the linguist reality of the skeleton consonant root.
But the same phenomenon can be accommodated by
not stipulating two tiers for the vowel template and the consonant skeleton
(as is done in standard templatic morphological analysis),
but by stipulating that we have two tiers of \emph{features} for each sound,
making consonants separated only by vowels able to talk to each other \citep{bat20082}. 

Another piece of evidence invoked to support the linguistic reality of discontinuous consonant skeletons
is that the consonant skeletons of borrowed words seem to be 
extracted and conjugated the same way native words are conjugated.
This however should be considered together with two facts.
First, it is possible for a stem with affixation to be treated as
a unit without internal structures in skeleton extracting,
meaning that the phonological form of the derived word is lexicalized
at least besides the etymologically simple roots, 
if not in place of them \citep[\citepage{132}]{bolozky2008roots}.
Second, when the borrowed terms have consonant clusters,
a consonant cluster tends to be transferred in borrowing as a whole
(i.e. vowels are not inserted between two consonants in a consonant cluster),
possibly to make sure the 

And finally we have to problem of cross-linguistic variations in modern languages.
A rather illustrative fact is that Maltese's Arabic templatic morphology is now more like a relic:
derived verbs frequently retain the vowel pattern of their bases,
without any vowel alternations,
and in cases where the vowels do change,
the change can be described in terms of lowering or fronting
\citep{hoberman2008verbal}.
Thus, even if at a certain stage, templatic morphology was synchronically relevant
and even if it is still synchronically relevant in some languages in a family,
we have no guarantee that it is still valid right now
for other languages in the same family.
Classical Arabic, long considered prototypical for templatic morphology,
has been recently suggested to base a large proportion of its derivational morphology
on ablaut of the \category{imperfective} verb form,
considering the fact that many nominal forms can be predicted based on it;
moreover, consistency of vowel length between singular and plural forms is also invoked to propose
a ``word-based''%
\footnote{
    Note that this is not contradictory to the idea of the root as a synchronically valid unit in grammar:
    the main issue here is about morphological \emph{realization}.
}
morphology \citep{benmamoun20085}.
We should emphasize that the non-necessity of roots with discontinuous phonological forms
\emph{in grammatical descriptions}
should not be naively equated to non-necessity of stipulating such forms
\emph{in the psycholinguistic theory}.
Studies have shown that speakers are much less likely to 

The seemingly contradiction can be resolved by noticing that 

Finally, we note that works cited above are mostly based on modern Hebrew and modern Arabic dialects,
both already strongly influenced by Indo-European languages.
The claim that synchronic descriptions of modern Semitic languages
do not require stipulating consonantal skeletons,
even true, does not say anything about their parent languages.
Indeed given the Ethiopian data in \citet{rose2008formation},
it is likely that some sort of 

\section{The clause}

\subsection{Peripheral systems}

We start by identifying nucleus clauses from possible clause linking devices.


\subsection{Types of nucleus clauses}

The existential clause seems to be incompatible with definite \acp{np}.

\subsection{Subjecthood}


\section{The noun phrase}

\subsection{Syntactic environments}

We start by discussing syntactic environments of noun phrases,
and then their overall internal structures.

\paragraph*{Noun phrases as arguments}
Hebrew in general lacks a case system.
There exists an object marker \form{et}.

\paragraph*{Topicalization and focalization}

\paragraph*{Question formation}
Modern Hebrew \acp{np} do not permit \category{wh}-extractions
\citep{shlonsky2012some}.

\subsection{The construct state, the construct chain, and the linear template}

Just like a clause, a noun phrase can also be divided into several systems.
What prevents us from immediately doing a one-to-one comparison of these systems in Hebrew
with those in English or French 
is a peculiar feature of Hebrew and indeed all Semitic languages:
the construct state.

\paragraph*{Basic properties}

The construct state is a unique genitive construction seen not only in Hebrew
but also in other Semitic languages.
A construct phrase consists of two obligatory components:
first the ``head'' i.e. the possessed,
followed by the possessor \citep[\citepage{26}]{glinert2004grammar}.
It should be noted that the head is not restricted to nouns
\citep[\citepage{25}]{glinert2004grammar}. 
The head is in a special morphological form,
known as the \concept{construct state},
while the possessor is marked for its definiteness.
Here we first focus on the nominal construct phrase.

The head seems to have certain complexity constraints:
it's usually a single word, although coordination also seems possible.
The possessor on the other hand can contain multiple words.
Adjectives modifying the head appears after the possessor
\citep[\citepage{25}]{glinert2004grammar}.

The possessor can also be a construct phrase,
containing one possessor modifying the possessed,
and again definiteness is only marked on the possessor.
And now the possessor within the possessor can again be a construct phrase,
and it goes on and on.
The result is that we have a chain of nouns/coordinated nouns,
with no adjectives separating them \citep{borer1999deconstructing},
with a meaning of \translate{A of B of C of \dots}
and definiteness is marked only on the last noun/coordinated nouns,
which is the only component in the chain not in a construct state.
This is known as a \concept{construct chain}.

All adjectives appear after the construct chain.
They can modify all components in the construct chain,
and the word order is nested,
and the dependency arcs do not cross each other:
thus the word order corresponding to \translate{the colorful cairs of the new class}
is chairs-\category{m.pl} \category{def}-class-\category{f} \category{def}-new-\category{f} \category{def}-colorful-\category{m.pl}
\citep{borer1999deconstructing}.

We note that the fact that the head is \emph{not} marked for definiteness.
The definiteness of the whole phrase seems to be consistent with the definiteness of the possessor,
and therefore, the definiteness of a construct chain 
is determined by the definiteness of the last noun/coordinated noun in the chain.
All adjectives following the chain also agree with the definiteness value 
of the last component in the chain.

\paragraph*{Definiteness spreading}
A natural question is how the Hebrew (and more generally, the Semitic) construct phrase
compares with nominal constructions in other languages.
What is particularly remarkable is the phenomenon of definiteness spreading:
this can of course be construed as a type of agreement.
Typical agreements, like the person and number agreement between the verb and the subject,
however do not \emph{limit} the possibilities of feature configurations:
the verb has no inherent person or number anyway.
Therefore, a feature involved in agreement is typically marked twice but interpreted once.
Definiteness spreading on the other hand \emph{limits} the possibilities of feature configurations,
and eliminates the possibility to have expressions with the meaning of 
\translate{the king of a country} or \translate{a king of the country}.
If analyzed as a type of agreement,
then in definiteness spreading, a feature is marked once,
but interpreted twice \citep{dobrovie2000definiteness}.

Definiteness spreading is not hard to understand in compounding
or the English nominal modification/complementation construction, as in \form{a [field linguist]}:
the modifier in the construction (\form{field} here) is not a full, ``sealed'' and referential \ac{np},
but is structurally reduced%
\footnote{
    Known as \form{nominal} in \citet{cgel},
    which structurally does not include the determiner layer of the full \ac{np}.
}.
In both constructions, the modifier is not referential
and doesn't have independent definiteness.
Note that certain possessive constructions like \form{teacher's desk} or \form{girl's bicycle} also belong to the second type
(and are definitely not a subject-determiner in the determiner system;
\citealt[\citepage{469}]{cgel}, \citealt{alexiadou2005possessors}).
In modern Hebrew, productivity of the construct phrase has decayed,
and a portion of construct phrases has idiomized,
presumably with \emph{syntactic} fossilization as well.
It is quite possible, for instance, that \form{beyt sefer} \translate{school (lit. house of books)}
has already evolved into a compound,
despite still having the morphophonological form of regular constructs
\citep{siloni2003prosodic}.
At least some of them are not too far from these English constructions:
for them, definiteness spreading is purely morphological.

In the rest of the examples, the construct state is still productive,
and the ``modifier'' or possessor is referential.
In these cases, definiteness spreading per se seem comparable to the English \form{the country's king's properties},
which is rendered definite as a whole because of the subject-possessor \form{the country}.
The possessor influencing the definiteness of the whole \ac{np}
is a cross-linguist phenomenon,
which however is also subject to many language-specific constraints
and a unified analysis is not desirable \citep{alexiadou2005possessors}.
In English, definiteness spreading can probably explained by semantic reasons
\citep{dobrovie2000definiteness},
whereas in Semitic languages, it seems \form{king} and \form{properties} each has a \emph{syntactic} definiteness value,
as is shown by the definiteness agreement between these words and adjectives modifying them.
This is not a very local agreement because the construct state chain can be arbitrarily long,
separating the adjectives and the noun they are modifying,
and has to involve at least some grammatical (syntactic or morphological) processes.

Moreover, it seems \emph{indefiniteness} does \emph{not} really spread in Hebrew,
because an indefinite construct chain can take a definite determiner
and become definite and appear in syntactic environments requiring a definite \ac{np}.
For instance, \form{ota tmunat praxim} that-\category{f} picture-\category{f} flowers
is fine and can appear in predicational sentences that require definite subjects
\citep{alexiadou2005possessors}.

A possible way to capture the phenomenon above is to assume that 
the definiteness agreement between the possessed and the possessor 
is due to prosodic reasons \citep{siloni2003prosodic}.
Let's suppose that the genitive relation between the two has to be marked 
by the fact that the two are in one prosodic unit.
Note that this does not involve any previously unknown mechanism,
because it has already been observed that, for example, case marking is dictated
by both syntactic and morphophonological concepts like adjacency.%
\footnote{
    Note that morphological cases 
    and abstract Cases in some theoretical works (used to describe grammatical relations)
    are not exactly the same.
    Postulation of the latter is to capture the fact that grammatical relations 
    are related to adjacency conditions.
}
Turning back to Hebrew constructs,
we note that the head is indeed usually unstressed,
which means it has to form a prosodic unit with the word following it,
i.e. the first word of the possessor.
The impossibility to have \translate{a king of the country} or \translate{the king of a country}
may be simply explained by stipulating that the scope of definiteness is over the prosodic unit.
Note that this definiteness spreading sometimes does not have semantic consequences:
in \citet{winter2005some}, it is reported that \form{toshav maxane ha-plitim} resident camp \category{def}-refugees
entails no meaning of uniqueness:
it may refer to any resident in the refugee camp.
The same is not possible in a \form{shel} construction: \form{ha} cannot be prefixed to \form{toshav},
which means the existence of semantically indefinite but syntactically definite constructs
can't be explained by stipulating that interpretation of syntactic definiteness is different in Hebrew.
The impossibility for the article to appear before the head may be explained by
the inability for a (prosodically reduced) function formative
to be attached to an already unstressed word.


\begin{todobox}{Whether the \form{ha} in this case is syntactic}{ha-syntactic}
    It remains to be seen whether \form{ha} in \form{toshav maxane ha-plitim}
    has any \emph{syntactic} significance:
    for instance we may wonder whether it makes the NP really syntactically definite
    so that it can't be inserted into an existential construction.
\end{todobox}


\paragraph*{Diachronic comments on definiteness spreading}
We note that the strict rule prohibiting \translate{the king of a country}
may also have diachronic reasons.
The [head-\category{cs}_{\category{indef}} possessor-\category{def} adj-\category{indef}]_{\text{\ac{np}}} construction is ambiguous with a nominal predication clause,
and even if it once was possible,
it probably faded from use because of the ambiguity.
As for [head-\category{cs}_{\category{def}} possessor-\category{indef} adj-\category{def}]_{\text{\ac{np}}},
we have at least two reasons against its persistence.
First, if the adjective is a reduced relative clause (TODO: ref; e.g. \citet{cinque2010syntax}, or maybe another section in this note),
then the meaning will have the structure of \translate{\textbf{a} man who is \textbf{the} master of patience}.
This makes sense in English,
but it is possible that ``definiteness'' in Semitic is slightly different from definiteness in English
(for instance the former may involve \emph{specificity}, i.e. whether the thing being referred to has already been pre-established in the discourse;
see \citet{ihsane2001specific})
and a definite relative clause modifying an indefinite noun is considered semantically awkward.

Second, in [head-\category{cs}_{\category{def}} possessor-\category{indef} adj-\category{def}]_{\text{\ac{np}}}, 
a word with a definite marker and a word with an indefinite marker meet.
Possibly there was a morphophonological rule 
prohibiting nouns with different definiteness from appearing together,
and if two nouns with the same definiteness appear together,
the first is realized as a construct state noun.
In Hebrew there is no explicit indefinite marker,
but in e.g. Standard Arabic, indefiniteness is marked by nunation of the case ending 
(that's to say, to add a \form{-n} to the end of a word) ,
and definiteness is marked by \form{al-},
so the existence of such a rule is not unimaginable.

\paragraph*{Relation with the \form{shel} genitive}

It's possible to put a construct phrase into the possessor of a \form{shel} construction
\citep{alexiadou2005possessors}.

\subsection{Systems within the Hebrew noun phrase}

Now we do a tentative comparison between the Hebrew noun phrase and the English noun phrase.

\paragraph*{The determiner system}
At the top is the system of determiner(s):
in English we have \form{\textbf{all these three} topics}.
In Hebrew, there is no 

In English, the determiner system also includes the so-called the Saxon genitive:
\form{\textbf{my sister's} car}, which is in a status quite similar to the subject in the clausal
\citep[\citepages{468,472-478}]{cgel}.
In Hebrew, 

\paragraph*{Modification}
Then we have various modification devices,
which are usually adjectival phrases but sometimes can also be nominals.
These can be divided into reduced relative clauses and direct modifications,
the former taking their scopes over the latter \citep{cinque2010syntax}.
Direct modifications are like \form{the \textbf{ugly big old} bear},
which tend to form a hierarchy (in this case, evaluation \textgt dimension \textgt age) according to their roles in the \ac{np} just like adverbs in clauses.
They refine the meaning of the head noun but not in a intersective or restrictive way.%
\footnote{
    For instance, in \form{the \textbf{young skillful dancers}},
    \form{skillful} refines the concept of \translate{dancer},
    but the resulting concept is not the intersection of the two
    (and hence not \emph{intersective}):
    we're talking about skillfulness in dancing,
    not people who are dancers and are also skillful (in probably something else).
    (It doesn't make sense though to say whether \form{young} is intersective or not here.)
    The adjectives \form{skillful} and \form{young} are also not necessarily restrictive
    (although under certain contexts they can be interpreted so)
    because we are \emph{not} first think about dancers in general
    and then decide to only talk about a subset of them who are young and skillful:
    we are talking about a bunch of people who happens to be good dancers and young.
    See \citet{cinque2010syntax} for more discussions.
}
Modifications that are reduced relative clauses are just the opposite:
they are not in any preferred orders (apart from being somehow far away from the head noun),
they are restrictive and intersective in their meanings.%
\footnote{
    In English we have \form{the \textbf{invisible} visible star}
    or \form{the visible star \textbf{invisible today}},
    in which \form{visible} is a direct modification about the inherent properties of a star,
    and \form{invisible} or \form{invisible today} is a reduced relative clause
    about a statement of the star that happens to be true today.
}
In Hebrew, direct modification is attested:
the English order is reversed in Hebrew \citep{shlonsky2012some}.

\paragraph*{Argument structures}
Noun phrases have their own argument structures.
In many languages, the nominal argument structure is related to possession:
in English for example we have \form{[his] play [of the national anthem]}
(c.f. the clausal version \form{[he] played [the national anthem]}),
where the subject corresponds to the subject-possessor (see above)
and the object corresponds to the \form{of}-phrase.

\begin{todobox}{The syntactic position of the construct phrase}{construct-position}
    It would be interesting to know if the construct phrase possessor
    is closer to the \form{of} argument or the subject-possessor.
    What do we need to declare that a possessor is in the determiner system?
\end{todobox}

\paragraph*{Derivational morphology}
Finally, we go into the structure of the head noun and discuss compounding and affixation.
The differences between compounding (e.g. \form{sickbed}) and constructions like \form{hospital bed}
originate from the fact that the the roots in the latter are \emph{categorized}
while the roots in the former are not.%
\footnote{
    \form{sick}, for example, is not a noun,
    which cannot be a nominal modifier
    (here \term{nominal} means a structurally reduced \ac{np} which does not include the determiner layer; this is the term used in e.g. \citet{cgel} and some other English grammars) or complement.
}

\subsection{Determiners}


\subsection{Modification and complementation}





\chapter{Phonology and the writing system}



\bibliographystyle{plainnat}
\bibliography{modern-hebrew.bib,semitic-history.bib}

\end{document}