\documentclass[a4paper, oneside, 12pt]{report}

\usepackage[T1]{fontenc}
\usepackage{libertinus}
\usepackage{underscore}
\usepackage{geometry}
\usepackage{float}
\usepackage{titling}
\usepackage{titlesec}
\usepackage{paralist}
\usepackage{footnote}
\usepackage{enumitem}
\setitemize{noitemsep,topsep=0pt,parsep=0pt,partopsep=0pt}
\usepackage{amsmath, amsthm}
\usepackage{gb4e}
\noautomath
\usepackage{bbm}
\usepackage{textcomp}
\usepackage{soul}
\usepackage{graphicx}
\usepackage{siunitx}
\usepackage[table,xcdraw]{xcolor}
\usepackage{tikz}
\usepackage[ruled, vlined, linesnumbered, noend]{algorithm2e}
\usepackage{xr-hyper}
\usepackage[colorlinks, citecolor = purple, bookmarksnumbered]{hyperref} % linkcolor=black, anchorcolor=black, citecolor=black, filecolor=black
\usepackage[most]{tcolorbox}
\usepackage{caption}
\usepackage{subcaption}
\usepackage{booktabs}
\usepackage{multirow}
\usepackage[figuresright]{rotating}
\usepackage{acro}
\usepackage[round]{natbib} 
\usepackage{prettyref}

\geometry{left=3.18cm,right=3.18cm,top=2.54cm,bottom=2.54cm}
\titlespacing{\paragraph}{0pt}{1pt}{10pt}[20pt]
\setlength{\droptitle}{-5em}

\DeclareMathOperator{\timeorder}{\mathcal{T}}
\DeclareMathOperator{\diag}{diag}
\DeclareMathOperator{\legpoly}{P}
\DeclareMathOperator{\primevalue}{P}
\DeclareMathOperator{\sgn}{sgn}
\newcommand*{\ii}{\mathrm{i}}
\newcommand*{\ee}{\mathrm{e}}
\newcommand*{\const}{\mathrm{const}}
\newcommand*{\suchthat}{\quad \text{s.t.} \quad}
\newcommand*{\argmin}{\arg\min}
\newcommand*{\argmax}{\arg\max}
\newcommand*{\normalorder}[1]{: #1 :}
\newcommand*{\pair}[1]{\langle #1 \rangle}
\newcommand*{\fd}[1]{\mathcal{D} #1}
\newcommand*{\textto}{$\to$}
\newcommand*{\textgt}{$>$ }

\newcommand*{\citesec}[1]{\S~{#1}}
\newcommand*{\citechap}[1]{Ch~{#1}}
\newcommand*{\citefig}[1]{Fig.~{#1}}
\newcommand*{\citetable}[1]{Table~{#1}}
\newcommand*{\citepage}[1]{p.~{#1}}
\newcommand*{\citepages}[1]{pp.~{#1}}
\newcommand*{\citefootnote}[1]{fn.~{#1}}
\newcommand*{\citechapsec}[2]{\citechap{#1}.\citesec{#2}}

\newrefformat{sec}{\citesec{\ref{#1}}}
\newrefformat{fig}{\citefig{\ref{#1}}}
\newrefformat{tbl}{\citetable{\ref{#1}}}
\newrefformat{chap}{\citechap{\ref{#1}}}
\newrefformat{fn}{\citefootnote{\ref{#1}}}
\newrefformat{box}{Box~\ref{#1}}
\newrefformat{ex}{\ref{#1}}


% color boxes

\tcbuselibrary{skins, breakable, theorems}

\AtBeginEnvironment{infobox}{\small}
\AtBeginEnvironment{theorybox}{\small}

\newtcbtheorem[number within=chapter]{infobox}{Box}{
    enhanced,
    boxrule=0pt,
    colback=blue!5,
    colframe=blue!5,
    coltitle=blue!50,
    borderline west={4pt}{0pt}{blue!65},
    sharp corners,
    fonttitle=\bfseries, 
    breakable,
    before upper={\parindent15pt\noindent}}{box}
\newtcbtheorem[number within=chapter, use counter from=infobox]{theorybox}{Box}{
    enhanced,
    boxrule=0pt,
    colback=orange!5, 
    colframe=orange!5, 
    coltitle=orange!50,
    borderline west={4pt}{0pt}{orange!65},
    sharp corners,
    fonttitle=\bfseries, 
    breakable,
    before upper={\parindent15pt\noindent}}{box}
\newtcbtheorem[number within=chapter, use counter from=infobox]{learnbox}{Box}{
    enhanced,
    boxrule=0pt,
    colback=green!5,
    colframe=green!5,
    coltitle=green!50,
    borderline west={4pt}{0pt}{green!65},
    sharp corners,
    fonttitle=\bfseries, 
    breakable,
    before upper={\parindent15pt\noindent}}{box}
\newtcbtheorem[number within=chapter, use counter from=infobox]{todobox}{Box}{
    enhanced,
    boxrule=0pt,
    colback=red!5,
    colframe=red!5,
    coltitle=red!50,
    borderline west={4pt}{0pt}{red!65},
    sharp corners,
    fonttitle=\bfseries, 
    breakable,
    before upper={\parindent15pt\noindent}}{box}

% Shorthands
\newcommand*{\concept}[1]{\textbf{#1}}
\newcommand*{\term}[1]{\emph{#1}}
\newcommand{\form}[1]{\emph{#1}}

\newcommand{\redp}{\textasciitilde}

\newcommand{\deictictime}{T$_{\text{d}}$}
\newcommand{\referredtime}{T$_{\text{r}}$}
\newcommand{\orientationtime}{T$_{\text{o}}$}

\DeclareAcronym{blt}{short = BLT, long = Basic Linguistic Theory}
\DeclareAcronym{cgel}{short = CGEL, long = The Cambridge Grammar of the English Language}
\DeclareAcronym{dm}{short = DM, long = Distributed Morphology}
\DeclareAcronym{tag}{long = Tree-adjoining grammar, short = TAG}
\DeclareAcronym{sfp}{long = sentence-final particle, short = sfp}
\DeclareAcronym{np}{long = noun phrase, short = NP}
\DeclareAcronym{vp}{long = verb phrase, short = VP}
\DeclareAcronym{pp}{long = preposition phrase, short = PP}
\DeclareAcronym{advp}{long = adverb phrase, short = AdvP}
\DeclareAcronym{cls}{long = classifier, short = CLS}
\DeclareAcronym{dist}{long = distal, short = DIST}
\DeclareAcronym{prox}{long = proximate, short = PROX}
\DeclareAcronym{dem}{long = demonstrative, short = DEM}
\DeclareAcronym{classify}{long = classifier, short = \textsc{cl}}
\DeclareAcronym{dur}{long = durative, short = DUR}
\DeclareAcronym{neg}{long = negative, short = \textsc{neg}}
\DeclareAcronym{cc}{long = copular complement, short = CC}
\DeclareAcronym{cs}{long = copular subject, short = CS}
\DeclareAcronym{tame}{long = {tense, aspect, mood, evidentiality}, short = TAME}
\DeclareAcronym{past}{long = past, short = PST}
\DeclareAcronym{nonpast}{long = non-past, short = NPST}
\DeclareAcronym{present}{long = present, short = PRES}
\DeclareAcronym{progressive}{long = progressive, short = \textsc{poss}}
\DeclareAcronym{perfect}{long = perfect, short = \textsc{perf}}
\DeclareAcronym{passive}{long = passive, short = \textsc{pass}}
\DeclareAcronym{copula}{long = copula, short = COP}
\DeclareAcronym{possessive}{long = possessive, short = \textsc{poss}}
\DeclareAcronym{coca}{long = Corpus of Contemporary American English, short = COCA}
\DeclareAcronym{sap}{long = speech act participant, short = SAP}

\newcommand{\asis}[1]{\textsc{#1}}
\newcommand{\oneof}[1]{{#1}}
\newcommand*{\homo}[2]{#1$_{\text{#2}}$}
\newcommand{\category}[1]{\textsc{#1}}
\newcommand{\formcat}[1]{\textsc{#1}}
\newcommand{\emptymorpheme}{$\emptyset$}
\newcommand*{\fromto}[2]{\langle {#1}, {#2} \rangle}

\newcommand{\alignment}{\href{../alignment/alignment.pdf}{my notes about alignment}}
\newcommand{\exerciseone}{\href{../Exercise/2021-3.pdf}{this exercise}}
\newcommand{\method}{\href{../methodology/glossing.pdf}{this note about my understanding of descriptive grammars}}

\newcommand{\ala}{à la}
\newcommand{\translate}[1]{`#1'}
\newcommand{\vP}{\textit{v}P}

% Make subsubsection labeled
\setcounter{secnumdepth}{4}
\setcounter{tocdepth}{4}
% reset example counter every chapter (but do not include the chapter number to the label)
\counterwithin{exx}{chapter} 

% Reference formats
\renewcommand{\bibname}{References}
\setcitestyle{aysep={}} 

% List format
\setlist[enumerate,1]{label=\alph*\upshape)}

\title{Modern Hebrew grammar}
\author{Jinyuan Wu}

\begin{document}

\maketitle

\automath

\chapter{Typological overview}

\section{The clause}

\subsection{Peripheral systems}

We start by identifying nucleus clauses from possible clause linking devices.


\subsection{Types of nucleus clauses}

The existential clause seems to be incompatible with definite \acp{np}.

\subsection{Subjecthood}


\section{The noun phrase}

\subsection{Syntactic environments}

We start by discussing syntactic environments of noun phrases,
and then their overall internal structures.

\paragraph*{Noun phrases as arguments}
Hebrew in general lacks a case system.
There exists an object marker \form{et}.

\paragraph*{Topicalization and focalization}

\paragraph*{Question formation}
Modern Hebrew \acp{np} do not permit \category{wh}-extractions
\citep{shlonsky2012some}.

\subsection{The construct state, the construct chain, and the linear template}

Just like a clause, a noun phrase can also be divided into several systems.
What prevents us from immediately doing a one-to-one comparison of these systems in Hebrew
with those in English or French 
is a peculiar feature of Hebrew and indeed all Semitic languages:
the construct state.

\paragraph*{Basic properties}

The construct state is a unique genitive construction seen not only in Hebrew
but also in other Semitic languages.
A construct phrase consists of two obligatory components:
first the ``head'' i.e. the possessed,
followed by the possessor \citep[\citepage{26}]{glinert2004grammar}.
It should be noted that the head is not restricted to nouns
\citep[\citepage{25}]{glinert2004grammar}. 
The head is in a special morphological form,
known as the \concept{construct state},
while the possessor is marked for its definiteness.
Here we first focus on the nominal construct phrase.

The head seems to have certain complexity constraints:
it's usually a single word, although coordination also seems possible.
The possessor on the other hand can contain multiple words.
Adjectives modifying the head appears after the possessor
\citep[\citepage{25}]{glinert2004grammar}.

The possessor can also be a construct phrase,
containing one possessor modifying the possessed,
and again definiteness is only marked on the possessor.
And now the possessor within the possessor can again be a construct phrase,
and it goes on and on.
The result is that we have a chain of nouns/coordinated nouns,
with no adjectives separating them \citep{borer1999deconstructing},
with a meaning of \translate{A of B of C of \dots}
and definiteness is marked only on the last noun/coordinated nouns,
which is the only component in the chain not in a construct state.
This is known as a \concept{construct chain}.

All adjectives appear after the construct chain.
They can modify all components in the construct chain,
and the word order is nested,
and the dependency arcs do not cross each other:
thus the word order corresponding to \translate{the colorful cairs of the new class}
is chairs-\category{m.pl} \category{def}-class-\category{f} \category{def}-new-\category{f} \category{def}-colorful-\category{m.pl}
\citep{borer1999deconstructing}.

We note that the fact that the head is \emph{not} marked for definiteness.
The definiteness of the whole phrase seems to be consistent with the definiteness of the possessor,
and therefore, the definiteness of a construct chain 
is determined by the definiteness of the last noun/coordinated noun in the chain.
All adjectives following the chain also agree with the definiteness value 
of the last component in the chain.

\paragraph*{Definiteness spreading}
A natural question is how the Hebrew (and more generally, the Semitic) construct phrase
compares with nominal constructions in other languages.
What is particularly remarkable is the phenomenon of definiteness spreading:
this can of course be construed as a type of agreement.
Typical agreements, like the person and number agreement between the verb and the subject,
however do not \emph{limit} the possibilities of feature configurations:
the verb has no inherent person or number anyway.
Therefore, a feature involved in agreement is typically marked twice but interpreted once.
Definiteness spreading on the other hand \emph{limits} the possibilities of feature configurations,
and eliminates the possibility to have expressions with the meaning of 
\translate{the king of a country} or \translate{a king of the country}.
If analyzed as a type of agreement,
then in definiteness spreading, a feature is marked once,
but interpreted twice \citep{dobrovie2000definiteness}.

Definiteness spreading is not hard to understand in compounding
or the English nominal modification/complementation construction, as in \form{a [field linguist]}:
the modifier in the construction (\form{field} here) is not a full, ``sealed'' and referential \ac{np},
but is structurally reduced%
\footnote{
    Known as \form{nominal} in \citet{cgel},
    which structurally does not include the determiner layer of the full \ac{np}.
}.
In both constructions, the modifier is not referential
and doesn't have independent definiteness.
Note that certain possessive constructions like \form{teacher's desk} or \form{girl's bicycle} also belong to the second type
(and are definitely not a subject-determiner in the determiner system;
\citealt[\citepage{469}]{cgel}, \citealt{alexiadou2005possessors}).
In modern Hebrew, productivity of the construct phrase has decayed,
and a portion of construct phrases has idiomized,
presumably with \emph{syntactic} fossilization as well.
It is quite possible, for instance, that \form{beyt sefer} \translate{school (lit. house of books)}
has already evolved into a compound,
despite still having the morphophonological form of regular constructs
\citep{siloni2003prosodic}.
At least some of them are not too far from these English constructions:
for them, definiteness spreading is purely morphological.

In the rest of the examples, the construct state is still productive,
and the ``modifier'' or possessor is referential.
In these cases, definiteness spreading per se seem comparable to the English \form{the country's king's properties},
which is rendered definite as a whole because of the subject-possessor \form{the country}.
The possessor influencing the definiteness of the whole \ac{np}
is a cross-linguist phenomenon,
which however is also subject to many language-specific constraints
and a unified analysis is not desirable \citep{alexiadou2005possessors}.
In English, definiteness spreading can probably explained by semantic reasons
\citep{dobrovie2000definiteness},
whereas in Semitic languages, it seems \form{king} and \form{properties} each has a \emph{syntactic} definiteness value,
as is shown by the definiteness agreement between these words and adjectives modifying them.
This is not a very local agreement because the construct state chain can be arbitrarily long,
separating the adjectives and the noun they are modifying,
and has to involve at least some grammatical (syntactic or morphological) processes.

Moreover, it seems \emph{indefiniteness} does \emph{not} really spread in Hebrew,
because an indefinite construct chain can take a definite determiner
and become definite and appear in syntactic environments requiring a definite \ac{np}.
For instance, \form{ota tmunat praxim} that-\category{f} picture-\category{f} flowers
is fine and can appear in predicational sentences that require definite subjects
\citep{alexiadou2005possessors}.

A possible way to capture the phenomenon above is to assume that 
the definiteness agreement between the possessed and the possessor 
is due to prosodic reasons \citep{siloni2003prosodic}.
Let's suppose that the genitive relation between the two has to be marked 
by the fact that the two are in one prosodic unit.
Note that this does not involve any previously unknown mechanism,
because it has already been observed that, for example, case marking is dictated
by both syntactic and morphophonological concepts like adjacency.%
\footnote{
    Note that morphological cases 
    and abstract Cases in some theoretical works (used to describe grammatical relations)
    are not exactly the same.
    Postulation of the latter is to capture the fact that grammatical relations 
    are related to adjacency conditions.
}
Turning back to Hebrew constructs,
we note that the head is indeed usually unstressed,
which means it has to form a prosodic unit with the word following it,
i.e. the first word of the possessor.
The impossibility to have \translate{a king of the country} or \translate{the king of a country}
may be simply explained by stipulating that the scope of definiteness is over the prosodic unit.
Note that this definiteness spreading sometimes does not have semantic consequences:
in \citet{winter2005some}, it is reported that \form{toshav maxane ha-plitim} resident camp \category{def}-refugees
entails no meaning of uniqueness:
it may refer to any resident in the refugee camp.
The same is not possible in a \form{shel} construction: \form{ha} cannot be prefixed to \form{toshav}.
The impossibility for the article to appear before the head may be explained by
the inability for a (prosodically reduced) function formative
to be attached to an already unstressed word.


\begin{todobox}{Whether the \form{ha} in this case is syntactic}{ha-syntactic}
    It remains to be seen whether \form{ha} in \form{toshav maxane ha-plitim}
    has any \emph{syntactic} significance:
    for instance we may wonder whether it makes the NP really syntactically definite
    so that it can't be inserted into an existential construction.
\end{todobox}


\paragraph*{Diachronic comments on definiteness spreading}
We note that the strict rule prohibiting \translate{the king of a country}
may also have diachronic reasons.
The [head-\category{cs}_{\category{indef}} possessor-\category{def} adj-\category{indef}]_{\text{\ac{np}}} construction is ambiguous with a nominal predication clause,
and even if it once was possible,
it probably faded from use because of the ambiguity.
As for [head-\category{cs}_{\category{def}} possessor-\category{indef} adj-\category{def}]_{\text{\ac{np}}},
we have at least two reasons against its persistence.
First, if the adjective is a reduced relative clause (TODO: ref; e.g. \citet{cinque2010syntax}, or maybe another section in this note),
then the meaning will have the structure of \translate{\textbf{a} man who is \textbf{the} master of patience}.
This makes sense in English,
but it is possible that ``definiteness'' in Semitic is slightly different from definiteness in English
(for instance the former may involve \emph{specificity}, i.e. whether the thing being referred to has already been pre-established in the discourse;
see \citet{ihsane2001specific})
and a definite relative clause modifying an indefinite noun is considered semantically awkward.

Second, in [head-\category{cs}_{\category{def}} possessor-\category{indef} adj-\category{def}]_{\text{\ac{np}}}, 
a word with a definite marker and a word with an indefinite marker meet.
Possibly there was a morphophonological rule 
prohibiting nouns with different definiteness from appearing together,
and if two nouns with the same definiteness appear together,
the first is realized as a construct state noun.
In Hebrew there is no explicit indefinite marker,
but in e.g. Standard Arabic, indefiniteness is marked by nunation of the case ending 
(that's to say, to add a \form{-n} to the end of a word) ,
and definiteness is marked by \form{al-},
so the existence of such a rule is not unimaginable.

\paragraph*{Relation with the \form{shel} genitive}

It's possible to put a construct phrase into the possessor of a \form{shel} construction
\citep{alexiadou2005possessors}.

\subsection{Systems within the Hebrew noun phrase}

Now we do a tentative comparison between the Hebrew noun phrase and the English noun phrase.

\paragraph*{The determiner system}
At the top is the system of determiner(s):
in English we have \form{\textbf{all these three} topics}.
In Hebrew, there is no 

In English, the determiner system also includes the so-called the Saxon genitive:
\form{\textbf{my sister's} car}, which is in a status quite similar to the subject in the clausal
\citep[\citepages{468,472-478}]{cgel}.
In Hebrew, 

\paragraph*{Modification}
Then we have various modification devices,
which are usually adjectival phrases but sometimes can also be nominals.
These can be divided into reduced relative clauses and direct modifications,
the former taking their scopes over the latter \citep{cinque2010syntax}.
Direct modifications are like \form{the \textbf{ugly big old} bear},
which tend to form a hierarchy (in this case, evaluation \textgt dimension \textgt age) according to their roles in the \ac{np} just like adverbs in clauses.
They refine the meaning of the head noun but not in a intersective or restrictive way.%
\footnote{
    For instance, in \form{the \textbf{young skillful dancers}},
    \form{skillful} refines the concept of \translate{dancer},
    but the resulting concept is not the intersection of the two
    (and hence not \emph{intersective}):
    we're talking about skillfulness in dancing,
    not people who are dancers and are also skillful (in probably something else).
    (It doesn't make sense though to say whether \form{young} is intersective or not here.)
    The adjectives \form{skillful} and \form{young} are also not necessarily restrictive
    (although under certain contexts they can be interpreted so)
    because we are \emph{not} first think about dancers in general
    and then decide to only talk about a subset of them who are young and skillful:
    we are talking about a bunch of people who happens to be good dancers and young.
    See \citet{cinque2010syntax} for more discussions.
}
Modifications that are reduced relative clauses are just the opposite:
they are not in any preferred orders (apart from being somehow far away from the head noun),
they are restrictive and intersective in their meanings.%
\footnote{
    In English we have \form{the \textbf{invisible} visible star}
    or \form{the visible star \textbf{invisible today}},
    in which \form{visible} is a direct modification about the inherent properties of a star,
    and \form{invisible} or \form{invisible today} is a reduced relative clause
    about a statement of the star that happens to be true today.
}
In Hebrew, direct modification is attested:
the English order is reversed in Hebrew \citep{shlonsky2012some}.

\paragraph*{Argument structures}
Noun phrases have their own argument structures.
In many languages, the nominal argument structure is related to possession:
in English for example we have \form{[his] play [of the national anthem]}
(c.f. the clausal version \form{[he] played [the national anthem]}),
where the subject corresponds to the subject-possessor (see above)
and the object corresponds to the \form{of}-phrase.

\begin{todobox}{The syntactic position of the construct phrase}{construct-position}
    It would be interesting to know if the construct phrase possessor
    is closer to the \form{of} argument or the subject-possessor.
    What do we need to declare that a possessor is in the determiner system?
\end{todobox}

\paragraph*{Derivational morphology}
Finally, we go into the structure of the head noun and discuss compounding and affixation.
The differences between compounding (e.g. \form{sickbed}) and constructions like \form{hospital bed}
originate from the fact that the the roots in the latter are \emph{categorized}
while the roots in the former are not.%
\footnote{
    \form{sick}, for example, is not a noun,
    which cannot be a nominal modifier
    (here \term{nominal} means a structurally reduced \ac{np} which does not include the determiner layer; this is the term used in e.g. \citet{cgel} and some other English grammars) or complement.
}

\subsection{Determiners}


\subsection{Modification and complementation}





\chapter{Phonology and the writing system}



\bibliographystyle{plainnat}
\bibliography{modern-hebrew.bib}

\end{document}