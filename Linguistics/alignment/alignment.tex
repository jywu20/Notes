\documentclass{article}

\usepackage{geometry}
\usepackage{titling}
\usepackage{titlesec}
\usepackage{paralist}
\usepackage{footnote}
\usepackage{enumerate}
\usepackage{amsmath, amssymb, amsthm}
\usepackage{gb4e}
\noautomath
\usepackage{bbm}
\usepackage{soul}
\usepackage{graphicx}
\usepackage{siunitx}
\usepackage[table,xcdraw]{xcolor}
\usepackage{tikz}
\usepackage[ruled, vlined, linesnumbered, noend]{algorithm2e}
\usepackage{xr-hyper}
\usepackage[colorlinks]{hyperref} % linkcolor=black, anchorcolor=black, citecolor=black, filecolor=black
\usepackage[most]{tcolorbox}
\usepackage{caption}
\usepackage{subcaption}
\usepackage{booktabs}
\usepackage{multirow}
\usepackage[figuresright]{rotating}
\usepackage{acro}
\usepackage[round]{natbib} 
\usepackage{prettyref}

\geometry{left=3.18cm,right=3.18cm,top=2.54cm,bottom=2.54cm}
\titlespacing{\paragraph}{0pt}{1pt}{10pt}[20pt]
\setlength{\droptitle}{-5em}

\DeclareMathOperator{\timeorder}{\mathcal{T}}
\DeclareMathOperator{\diag}{diag}
\DeclareMathOperator{\legpoly}{P}
\DeclareMathOperator{\primevalue}{P}
\DeclareMathOperator{\sgn}{sgn}
\newcommand*{\ii}{\mathrm{i}}
\newcommand*{\ee}{\mathrm{e}}
\newcommand*{\const}{\mathrm{const}}
\newcommand*{\suchthat}{\quad \text{s.t.} \quad}
\newcommand*{\argmin}{\arg\min}
\newcommand*{\argmax}{\arg\max}
\newcommand*{\normalorder}[1]{: #1 :}
\newcommand*{\pair}[1]{\langle #1 \rangle}
\newcommand*{\fd}[1]{\mathcal{D} #1}

\newcommand*{\citesec}[1]{\S~{#1}}
\newcommand*{\citechap}[1]{chap.~{#1}}
\newcommand*{\citefig}[1]{Fig.~{#1}}
\newcommand*{\citetable}[1]{Table~{#1}}

\newrefformat{sec}{\citesec{\ref{#1}}}
\newrefformat{fig}{\citefig{\ref{#1}}}
\newrefformat{tbl}{\citetable{\ref{#1}}}
\newrefformat{chap}{\citechap{\ref{#1}}}

\usetikzlibrary{arrows,shapes,positioning}
\usetikzlibrary{arrows.meta}
\usetikzlibrary{decorations.markings}
\tikzstyle arrowstyle=[scale=1]
\tikzstyle directed=[postaction={decorate,decoration={markings,
    mark=at position .5 with {\arrow[arrowstyle]{stealth}}}}]
\tikzstyle ray=[directed, thick]
\tikzstyle dot=[anchor=base,fill,circle,inner sep=1pt]


\tcbuselibrary{skins, breakable, theorems}

\newtcbtheorem[number within=chapter]{infobox}{Box}%
  {colback=blue!5,colframe=blue!65,fonttitle=\bfseries, breakable}{infobox}

\newcommand*{\concept}[1]{\textbf{#1}}
\newcommand*{\term}[1]{\emph{#1}}

\newcommand*{\vP}{\textit{v}P}

\DeclareAcronym{blt}{short = BLT, long = Basic Linguistic Theory}
\DeclareAcronym{cgel}{short = CGEL, long = The Cambridge Grammar of the English Language}
\DeclareAcronym{dm}{short = DM, long = Distributed Morphology}
\DeclareAcronym{tag}{long = Tree-adjoining grammar, short = TAG}
\DeclareAcronym{sfp}{long = sentence final particle, short = SFP}
\DeclareAcronym{vp}{long = verb phrase, short = VP}
\DeclareAcronym{cls}{long = classifier, short = CLS}
\DeclareAcronym{dist}{long = distal, short = DIST}
\DeclareAcronym{prox}{long = proximate, short = PROX}
\DeclareAcronym{dem}{long = demonstrative, short = DEM}
\DeclareAcronym{dur}{long = durative, short = DUR}
\DeclareAcronym{neg}{long = negative, short = NEG}

% Disable unsupported commands in bookmark titles 
\pdfstringdefDisableCommands{%
  \def\\{}%
  \def\texttt#1{<#1>}%
  \def\mathbb#1{#1}%
}
\pdfstringdefDisableCommands{\def\eqref#1{(\ref{#1})}}

\makeatletter
\pdfstringdefDisableCommands{\let\HyPsd@CatcodeWarning\@gobble}
\makeatother

\title{Alignment}
\author{Jinyuan Wu}

\begin{document}

\automath

\maketitle

\section{Introduction}

\subsection{Starting point: accusative v.s. ergative}

This note is about alignment in world languages. 
I start from summarizing the ideas reviewed in \citet{aldridge2008generative} in more descriptive terms,
and clarification of the discussion of \concept{ergativity} in the descriptive-typological literature, 
say, Dixon's \ac{blt} \citesec{3.9}.
What I want to do is to reconcile both approaches:
to ``integrate out'' unobservable derivational steps in the generative account and 
present them in terms of \ac{blt}, 
while giving more precise and structure-based definition of so-called ``functional'' terms.

To do so, I need to first coin a definition of \emph{surface} S, A and O arguments.
I emphasize \emph{surface} because languages have \emph{valency changing devices}.
In a clause constructed with a valency changing device, say passivization,
the S arguments semantically corresponds to the O argument 
in the active counterpart of the clause, 
while the E argument semantically corresponds to the A argument in the active counterpart.
In transformational terms, we can say in a passive clause,
the \concept{surface} S argument corresponds to the \concept{deep} A argument,
and the \concept{surface} E argument corresponds to the \concept{deep} O argument
(\ac{blt} \citesec{3.20}).
Though the transformational analysis is no longer supported as a separate stage after PSG in generative syntax,
it is still a good descriptive tool to link two constructions with only slightly differences in their derivations.

\section{The argument structure and the definition of deep S, A, and O arguments}\label{sec:argument-structure}

\subsection{The argument structure: coarse-grained \vP}

\begin{figure}
    \centering
    

\tikzset{every picture/.style={line width=0.75pt}} %set default line width to 0.75pt        

\begin{tikzpicture}[x=0.75pt,y=0.75pt,yscale=-0.85,xscale=0.85]
%uncomment if require: \path (0,408); %set diagram left start at 0, and has height of 408

%Straight Lines [id:da2665478368972569] 
\draw [color={rgb, 255:red, 245; green, 166; blue, 35 }  ,draw opacity=1 ] [dash pattern={on 4.5pt off 4.5pt}]  (264,61) -- (544,61) ;
\draw [shift={(546,61)}, rotate = 180] [fill={rgb, 255:red, 245; green, 166; blue, 35 }  ,fill opacity=1 ][line width=0.08]  [draw opacity=0] (12,-3) -- (0,0) -- (12,3) -- cycle    ;
%Straight Lines [id:da6278442099042232] 
\draw [color={rgb, 255:red, 245; green, 166; blue, 35 }  ,draw opacity=1 ] [dash pattern={on 4.5pt off 4.5pt}]  (295,107) -- (622,107) ;
\draw [shift={(624,107)}, rotate = 180] [fill={rgb, 255:red, 245; green, 166; blue, 35 }  ,fill opacity=1 ][line width=0.08]  [draw opacity=0] (12,-3) -- (0,0) -- (12,3) -- cycle    ;
%Rounded Rect [id:dp9005470977487176] 
\draw  [draw opacity=0][fill={rgb, 255:red, 80; green, 227; blue, 194 }  ,fill opacity=0.2 ] (494,41.22) .. controls (494,32.48) and (501.08,25.4) .. (509.82,25.4) -- (741.18,25.4) .. controls (749.92,25.4) and (757,32.48) .. (757,41.22) -- (757,325.99) .. controls (757,334.73) and (749.92,341.81) .. (741.18,341.81) -- (509.82,341.81) .. controls (501.08,341.81) and (494,334.73) .. (494,325.99) -- cycle ;
%Straight Lines [id:da3555404365615693] 
\draw [color={rgb, 255:red, 155; green, 155; blue, 155 }  ,draw opacity=1 ]   (630.05,206.81) -- (630.05,301) ;
%Straight Lines [id:da5116688308518689] 
\draw [color={rgb, 255:red, 80; green, 227; blue, 194 }  ,draw opacity=1 ]   (703.05,158.83) -- (665.05,140.5) ;
%Straight Lines [id:da7532901881200198] 
\draw [color={rgb, 255:red, 80; green, 227; blue, 194 }  ,draw opacity=1 ]   (665.05,140.5) -- (630.05,158.83) ;
%Straight Lines [id:da3708728686228633] 
\draw [color={rgb, 255:red, 80; green, 227; blue, 194 }  ,draw opacity=1 ]   (658.65,97.63) -- (601.2,71.73) ;
%Straight Lines [id:da18650265857430104] 
\draw [color={rgb, 255:red, 80; green, 227; blue, 194 }  ,draw opacity=1 ]   (601.2,71.73) -- (548.65,97.63) ;
%Rounded Rect [id:dp9951745940140804] 
\draw  [draw opacity=0][fill={rgb, 255:red, 255; green, 255; blue, 255 }  ,fill opacity=1 ] (675,215.16) .. controls (675,210.29) and (678.95,206.33) .. (683.83,206.33) -- (722.17,206.33) .. controls (727.05,206.33) and (731,210.29) .. (731,215.16) -- (731,314.17) .. controls (731,319.05) and (727.05,323) .. (722.17,323) -- (683.83,323) .. controls (678.95,323) and (675,319.05) .. (675,314.17) -- cycle ;
%Rounded Rect [id:dp4970986462373306] 
\draw  [draw opacity=0][fill={rgb, 255:red, 255; green, 255; blue, 255 }  ,fill opacity=1 ] (521.24,153.76) .. controls (521.24,148.82) and (525.25,144.81) .. (530.19,144.81) -- (568.05,144.81) .. controls (572.99,144.81) and (577,148.82) .. (577,153.76) -- (577,314.05) .. controls (577,318.99) and (572.99,323) .. (568.05,323) -- (530.19,323) .. controls (525.25,323) and (521.24,318.99) .. (521.24,314.05) -- cycle ;
%Rounded Rect [id:dp4071663798075751] 
\draw  [draw opacity=0][fill={rgb, 255:red, 80; green, 227; blue, 194 }  ,fill opacity=0.2 ] (106,41.89) .. controls (106,33.39) and (112.89,26.5) .. (121.4,26.5) -- (346.6,26.5) .. controls (355.11,26.5) and (362,33.39) .. (362,41.89) -- (362,324.41) .. controls (362,332.92) and (355.11,339.81) .. (346.6,339.81) -- (121.4,339.81) .. controls (112.89,339.81) and (106,332.92) .. (106,324.41) -- cycle ;
%Straight Lines [id:da5088138181155522] 
\draw [color={rgb, 255:red, 155; green, 155; blue, 155 }  ,draw opacity=1 ]   (217.05,133.93) -- (217.05,299.1) ;
%Straight Lines [id:da998446713157706] 
\draw [color={rgb, 255:red, 80; green, 227; blue, 194 }  ,draw opacity=1 ]   (290.05,133.93) -- (252.05,115.6) ;
%Straight Lines [id:da5752866187220778] 
\draw [color={rgb, 255:red, 80; green, 227; blue, 194 }  ,draw opacity=1 ]   (252.05,115.6) -- (217.05,133.93) ;
%Straight Lines [id:da6021628337779128] 
\draw [color={rgb, 255:red, 80; green, 227; blue, 194 }  ,draw opacity=1 ]   (249.65,94.73) -- (200.2,68.83) ;
%Straight Lines [id:da8361003247119008] 
\draw [color={rgb, 255:red, 80; green, 227; blue, 194 }  ,draw opacity=1 ]   (200.2,68.83) -- (153.65,94.73) ;
%Rounded Rect [id:dp3858870486231194] 
\draw  [draw opacity=0][fill={rgb, 255:red, 255; green, 255; blue, 255 }  ,fill opacity=1 ] (257.24,151.34) .. controls (257.24,145.52) and (261.96,140.81) .. (267.77,140.81) -- (313.48,140.81) .. controls (319.29,140.81) and (324,145.52) .. (324,151.34) -- (324,309.05) .. controls (324,314.86) and (319.29,319.57) .. (313.48,319.57) -- (267.77,319.57) .. controls (261.96,319.57) and (257.24,314.86) .. (257.24,309.05) -- cycle ;
%Rounded Rect [id:dp2951833006567197] 
\draw  [draw opacity=0][fill={rgb, 255:red, 255; green, 255; blue, 255 }  ,fill opacity=1 ] (134.24,118.4) .. controls (134.24,113.11) and (138.54,108.81) .. (143.84,108.81) -- (184.41,108.81) .. controls (189.71,108.81) and (194,113.11) .. (194,118.4) -- (194,305.98) .. controls (194,311.28) and (189.71,315.57) .. (184.41,315.57) -- (143.84,315.57) .. controls (138.54,315.57) and (134.24,311.28) .. (134.24,305.98) -- cycle ;

% Text Node
\draw (630.76,321) node [anchor=south] [inner sep=0.75pt]   [align=left] {hurt};
% Text Node
\draw (665.05,137.5) node [anchor=south] [inner sep=0.75pt]   [align=left] {\begin{minipage}[lt]{89.53pt}\setlength\topsep{0pt}
\begin{center}
predicate:\\smaller verb phrase
\end{center}

\end{minipage}};
% Text Node
\draw (601.2,68.73) node [anchor=south] [inner sep=0.75pt]   [align=left] {verb phrase};
% Text Node
\draw (548.65,100.63) node [anchor=north] [inner sep=0.75pt]   [align=left] {\begin{minipage}[lt]{29.61pt}\setlength\topsep{0pt}
\begin{center}
agent:\\DP
\end{center}

\end{minipage}};
% Text Node
\draw (630.05,161.83) node [anchor=north] [inner sep=0.75pt]   [align=left] {\begin{minipage}[lt]{50.89pt}\setlength\topsep{0pt}
\begin{center}
predicator:\\V
\end{center}

\end{minipage}};
% Text Node
\draw (703.05,161.83) node [anchor=north] [inner sep=0.75pt]   [align=left] {\begin{minipage}[lt]{36.95pt}\setlength\topsep{0pt}
\begin{center}
patient:\\DP
\end{center}

\end{minipage}};
% Text Node
\draw (216.76,321.1) node [anchor=south] [inner sep=0.75pt]   [align=left] {hurt};
% Text Node
\draw (249.65,97.73) node [anchor=north] [inner sep=0.75pt]   [align=left] {Trans - DP};
% Text Node
\draw (179.8,50.81) node [anchor=north west][inner sep=0.75pt]   [align=left] {$\displaystyle v$ - DP};
% Text Node
\draw (211,383) node [anchor=north west][inner sep=0.75pt]   [align=left] {(a)};
% Text Node
\draw (636,383) node [anchor=north west][inner sep=0.75pt]   [align=left] {(b)};


\end{tikzpicture}

    \caption{An example of a routinized \vP, or the so-called argument structure of the verb \term{hurt}}
    \label{fig:example-argument-structure}
\end{figure}

The first question is what morphosyntactic phenomena are related to 
the argument structure and
the definition of deep S, A and O mentioned above.
In generative terms, the \concept{argument structure} is given by the parts of \vP{}
in which specifiers positions are filled by NPs.
Routinization of the structure of \vP{} gives a \ac{tag}-like tree with argument slots,
something like \prettyref{fig:example-argument-structure}.
The information contained in the derivation process of the \vP{} can be roughly summarized as the following:
\begin{itemize}
    \item The relation between the verb and the arguments, or 
    the \concept{semantic (thematic) roles} of the arguments. 
    \begin{itemize}
        \item A Trans-DP dependency relation is to be interpreted 
        as the grammatical relation connecting the verb and the \concept{patient}.
        \item A $v$-DP dependency relation is to be interpreted
        as the grammatical relation connecting the verb and the \concept{agent}.
        \item \prettyref{fig:example-argument-structure} therefore gives the \emph{argument structure} of \term{hurt}:
        it may take an agent argument and a patient argument.
    \end{itemize}
    \item The relation between the arguments. The agent is in a higher position than the patient position.
\end{itemize}

\subsection{Syntactic tendencies related to the argument structure}

Though the argument structure is commonly recognized as mostly semantic 
(\ac{blt} \citetable{3.1}), 
it has syntactic consequences, which are also regulated by pragmatic factors.
Note that it immediately follows from the relative hierarchy position of different arguments
that there is likely to be a stable \emph{binding} relation from the higher argument to the lower argument.
Hence, we may expect that 
\begin{exe}
    \ex \label{ex:argument-binding} If a language has an English-like reflexive pronoun system
    (i.e. \term{myself}, \term{yourself}, \term{himself}, \term{herself}),
    then it is highly likely that a reflexive pronoun can fill the patient position,
    but not the agent position.
\end{exe}
Note that this tendency has \emph{nothing} to do with how the arguments are marked, 
and so the correlation between semantic roles and the distribution of reflexive pronouns holds 
for both accusative and ergative languages.

Similarly, when forming an imperative clause, 
the agent is easily understood from the context, 
and hence it can be expected that among languages there is the following generalization
\citep[\citesec{5.4}]{comrie1989language}:
\begin{exe}
    \ex \label{ex:argument-imperative} Even though \textit{pro}-drop is not allowed in most cases,
    the agent in an imperative clause is likely to be omitted,
    \emph{regardless} of its case marking.
\end{exe}

Thus the largely semantic argument structure has syntactic consequences and is strongly related to pragmatics,
and therefore may be seen as a \emph{construction} with integrated syntactic, semantic and pragmatic information.
But note that it can be broken down into smaller structural units built together 
by the structure-building operation Merge, 
and its semantic and pragmatic functions can be analyzed 
in terms of the structure building or \emph{derivation} process.
This is where my opinion deviates from the strictly constructivist approach.

\subsection{Coarse-graining of semantic roles: introducing the S, A, and O roles}

The problem of the tendencies related to the argument structure discussed above 
is their scopes are too narrow.
\eqref{ex:argument-binding} and \eqref{ex:argument-imperative} tell us nothing about 
a clause headed by a verb with no agent argument.
The next question to ask is for clauses without an agent argument,
whether we have generalizations like \eqref{ex:argument-binding} and \eqref{ex:argument-imperative}.
In other words, the question is whether non-agent arguments have agent-like syntactic properties.
Another equivalent way to frame the question is to ask 
how can semantic roles be coarse-grained (or clustered) further into categories 
with respect to argument structure-related syntactic properties.
The definition of these coarse-grained categories of semantic roles is a prototype-based one
\citep[\citesec{5.2}]{comrie1989language}.

After defining S, A, and O, tendencies like \eqref{ex:argument-binding} and \eqref{ex:argument-imperative}
can be further generalized to all types of verbs.
The generalized version of \eqref{ex:argument-binding} is that 
\begin{exe} 
    \ex A always binds O.
\end{exe}   
Even for Dyirbal, a language with much more ergativity than most syntactically ergative languages, 
this still holds \citep{van2003syntactic}.

\section{Morphological ergativity}\label{sec:morphologcal-ergativity}

\subsection{Marking strategies of S, A, and O}

Once the definition of the S, A and, O notation is obtained, 
the question becomes how they are marked.
But what is meant by ``marking''?
Languages may mark S, A and O via the following ways:
\begin{itemize} %TODO: details
    \item Morphological case.
    \item Constituent order.
    \item Movement, ellipsis, 
\end{itemize}

\begin{table}
    \caption{Morphological and syntactic ergativity}
    \label{tbl:morphological-syntactic-ergativity}
    \centering
    \begin{tabular}{ccc}
    
    \toprule
                               & syntactically accusative     & syntactically ergative       \\ \midrule
    morphologically accusative & typical accusative languages & Dyirbal 1st and 2nd pronouns \\
    morphologically ergative   & Basque, Tzotzil              & Dyirbal                      \\ \bottomrule
    \end{tabular}
\end{table}

Morphological marking occurs at all stages of clause building.
It may occur directly in the 

\section{The split pivot theory of syntactic ergativity}\label{sec:split-pivot}

\subsection{Definition of subjecthood 1: agent property}

\subsection{Definition of subjecthood 2: topic property}

\subsection{Coordination pivot}

\subsection{Control construction}

\subsection{Variation 1: control structures}

\section{Constituent orders and ergativity}

One important question avoided in \prettyref{sec:morphologcal-ergativity} and \prettyref{sec:split-pivot}
is constituent order.
It is in principle possible, for example, 
that a language is typically morphologically and syntactically ergative,
but the constituent order is NP$_{\text{erg}}$-V-NP$_{\text{abs}}$ and NP$_{\text{abs}}$-V.
Should this be the case, 
we would have to come up with a typology of \emph{constituent order} ergativity.
The good news is that ergative languages also tend to avoid this tricky point: 
it seems that \citep{mahajan1994ergativity}
\begin{exe}
    \ex Ergative languages are largely verb-peripheral, or free word order or V2 or \dots
\end{exe}
Hence no reasonable comparison between a transitive clause and an intransitive clause can be made 
with respect to constituent order.

\section{Transitive alignment}


Transitive alignment

\section{Split-S systems}

\section{Austronesian focus system}

\section{Direct-Inverse system}

\bibliographystyle{plainnat}
\bibliography{alignment}

\end{document}