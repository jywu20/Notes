\documentclass[a4paper, oneside, 12pt]{report}

\usepackage{libertinus}
\usepackage{geometry}
\usepackage{float}
\usepackage{titling}
\usepackage{titlesec}
\usepackage{paralist}
\usepackage{footnote}
\usepackage[inline]{enumitem}
\usepackage{amsmath, amsthm}
\usepackage{gb4e}
\noautomath
\usepackage{bbm}
\usepackage{soul}
\usepackage{graphicx}
\usepackage{siunitx}
\usepackage[table,xcdraw]{xcolor}
\usepackage{tikz}
\usepackage[ruled, vlined, linesnumbered, noend]{algorithm2e}
\usepackage{xr-hyper}
\usepackage[colorlinks,citecolor=purple]{hyperref} % linkcolor=black, anchorcolor=black, citecolor=black, filecolor=black
\usepackage[most]{tcolorbox}
\usepackage{caption}
\usepackage{subcaption}
\usepackage{booktabs}
\usepackage{multirow}
\usepackage[figuresright]{rotating}
\usepackage{acro}
\usepackage[round]{natbib} 
\usepackage{prettyref}

\geometry{left=3.18cm,right=3.18cm,top=2.54cm,bottom=2.54cm}
\titlespacing{\paragraph}{0pt}{1pt}{10pt}[20pt]
\setlength{\droptitle}{-5em}

\DeclareMathOperator{\timeorder}{\mathcal{T}}
\DeclareMathOperator{\diag}{diag}
\DeclareMathOperator{\legpoly}{P}
\DeclareMathOperator{\primevalue}{P}
\DeclareMathOperator{\sgn}{sgn}
\newcommand*{\ii}{\mathrm{i}}
\newcommand*{\ee}{\mathrm{e}}
\newcommand*{\const}{\mathrm{const}}
\newcommand*{\suchthat}{\quad \text{s.t.} \quad}
\newcommand*{\argmin}{\arg\min}
\newcommand*{\argmax}{\arg\max}
\newcommand*{\normalorder}[1]{: #1 :}
\newcommand*{\pair}[1]{\langle #1 \rangle}
\newcommand*{\fd}[1]{\mathcal{D} #1}

\newcommand*{\citesec}[1]{\S~{#1}}
\newcommand*{\citechap}[1]{chap.~{#1}}
\newcommand*{\citefig}[1]{Fig.~{#1}}
\newcommand*{\citetable}[1]{Table~{#1}}
\newcommand*{\citepage}[1]{p.~{#1}}
\newcommand*{\citepages}[1]{pp.~{#1}}
\newcommand*{\citefootnote}[1]{fn.~{#1}}

\newrefformat{sec}{\citesec{\ref{#1}}}
\newrefformat{fig}{\citefig{\ref{#1}}}
\newrefformat{tbl}{\citetable{\ref{#1}}}
\newrefformat{chap}{\citechap{\ref{#1}}}
\newrefformat{fn}{\citefootnote{\ref{#1}}}
\newrefformat{box}{Box~\ref{#1}}
\newrefformat{ex}{\ref{#1}}

\tcbuselibrary{skins, breakable, theorems}

\AtBeginEnvironment{infobox}{\small}
\AtBeginEnvironment{theorybox}{\small}

\newtcbtheorem[number within=chapter]{infobox}{Box}{
    enhanced,
    boxrule=0pt,
    colback=blue!5,
    colframe=blue!5,
    coltitle=blue!50,
    borderline west={4pt}{0pt}{blue!65},
    sharp corners,
    fonttitle=\bfseries, 
    breakable,
    before upper={\parindent15pt\noindent}}{box}
\newtcbtheorem[number within=chapter, use counter from=infobox]{theorybox}{Box}{
    enhanced,
    boxrule=0pt,
    colback=orange!5, 
    colframe=orange!5, 
    coltitle=orange!50,
    borderline west={4pt}{0pt}{orange!65},
    sharp corners,
    fonttitle=\bfseries, 
    breakable,
    before upper={\parindent15pt\noindent}}{box}
\newtcbtheorem[number within=chapter, use counter from=infobox]{learnbox}{Box}{
    enhanced,
    boxrule=0pt,
    colback=green!5,
    colframe=green!5,
    coltitle=green!50,
    borderline west={4pt}{0pt}{green!65},
    sharp corners,
    fonttitle=\bfseries, 
    breakable,
    before upper={\parindent15pt\noindent}}{box}
\newtcbtheorem[number within=chapter, use counter from=infobox]{todobox}{Box}{
    enhanced,
    boxrule=0pt,
    colback=red!5,
    colframe=red!5,
    coltitle=red!50,
    borderline west={4pt}{0pt}{red!65},
    sharp corners,
    fonttitle=\bfseries, 
    breakable,
    before upper={\parindent15pt\noindent}}{box}

\newcommand*{\concept}[1]{\textbf{#1}}
\newcommand*{\term}[1]{\emph{#1}}
\newcommand{\form}[1]{\emph{#1}}
\newcommand*{\category}[1]{\textsc{#1}}

%region Acronym

% Theory
\DeclareAcronym{blt}{short = BLT, long = Basic Linguistic Theory}
\DeclareAcronym{cgel}{short = CGEL, long = The Cambridge Grammar of the English Language}
\DeclareAcronym{dm}{short = DM, long = Distributed Morphology}
\DeclareAcronym{tag}{long = Tree-adjoining grammar, short = TAG}

% History

\DeclareAcronym{pie}{long = proto-Indo-European, short = PIE}

% Roles

\DeclareAcronym{sfp}{long = sentence final particle, short = SFP}
\DeclareAcronym{np}{long = noun phrase, short = NP}
\DeclareAcronym{vp}{long = verb phrase, short = VP}
\DeclareAcronym{pp}{long = preposition phrase, short = PP}
\DeclareAcronym{adjp}{long = adjective phrase, short = AdjP}
\DeclareAcronym{advp}{long = adverb phrase, short = AdvP}
\DeclareAcronym{cc}{long = copular complement, short = CC}
\DeclareAcronym{cs}{long = copular subject, short = CS}
\DeclareAcronym{tam}{long = {tense, aspect, mood}, short = TAM}
\DeclareAcronym{tame}{long = {Tense, Aspect, Mood, Evidentiality}, short = TAME}
\DeclareAcronym{copula}{long = copula, short = COP}

% Pronouns 

\DeclareAcronym{dist}{long = distal, short = \textsc{dist}}
\DeclareAcronym{prox}{long = proximate, short = \textsc{prox}}
\DeclareAcronym{dem}{long = demonstrative, short = \textsc{dem}}

% TAME, negative

\DeclareAcronym{neg}{long = negative, short = \textsc{neg}}
\DeclareAcronym{past}{long = \textsc{past}, short = \textsc{pst}}
\DeclareAcronym{imperfect}{long = \textsc{imperfect}, short = \textsc{impf}}
\DeclareAcronym{present}{long = \textsc{present}, short = \textsc{pres}}
\DeclareAcronym{perfect}{long = \textsc{perfect}, short = \textsc{perf}}
\DeclareAcronym{future}{long = \textsc{future}, short = \textsc{fut}}
\DeclareAcronym{pluperfect}{long = \textsc{pluperfect}, short = \textsc{plup}}
\DeclareAcronym{future perfect}{long = \textsc{future perfect}, short = \textsc{fut.perf}}
\DeclareAcronym{passive}{long = passive, short = PASS}
\DeclareAcronym{indicative}{long = \textsc{indicative}, short = \textsc{ind}}
\DeclareAcronym{subjunctive}{long = \textsc{subjunctive}, short = \textsc{sjv}}
\DeclareAcronym{imperative}{long = \textsc{imperative}, short = \textsc{imp}}


%endregion

\newcommand*{\homo}[2]{#1$_{\text{#2}}$}

\newcommand{\ala}{à la}
\newcommand{\translate}[1]{`#1'}
\newcommand{\vP}{\textit{v}P}

\newcommand{\classify}[1]{{\textsc{#1}}}
\newcommand{\literature}[1]{\textit{#1}}

\newcommand*{\focus}[1]{\textbf{#1}}

% Make subsubsection labeled
\setcounter{secnumdepth}{4}
\setcounter{tocdepth}{4}
% reset example counter every chapter (but do not include the chapter number to the label)
\counterwithin{exx}{chapter} 

\renewcommand{\bibname}{References}

\title{Notes on French grammar}
\author{Jinyuan Wu}

\begin{document}

\automath

\maketitle

\chapter{Introduction}

\section{Classification}

French is a Romance language, that's to say, it's a daughter language of Vulgar Latin.

\begin{todobox}{Western Romance language}{western-romance-lang}
    Argue that French is indeed a Western Romance language.
\end{todobox}

\section{Sociolinguistic status}

French is an official language in 27 countries.


\section{Remarkable features}

French syntax resembles English in several aspects:
the SVO basic order,
the existence of auxiliaries, 
the clause-initial topicalization region and 
the clause-final ``miscellaneous'' region, 
and the existence of both ample peripheral argument-like adverbs 
and \ac{tam} adverbs, 
the former usually following the latter, 
reflecting their relevant distances to the main verb in the clausal structure.
The main differences include the follows:
\begin{itemize}
    \item French has negative concord;
    \item French is more morphologically complicated 
    and correspondingly allows at most two auxiliary verbs;
    \item 
\end{itemize}

\chapter{Grammatical overview}

\section{Morphological typology}

\begin{todobox}{Morphological template}{morphological-template}
    French as polysynthetic language.
\end{todobox}

\section{Clausal syntax}

\subsection{The top-level categories}

\paragraph*{Parts of the clause}
We can do a top-down dissection of French clauses.
For sentences, we can first recognize speech force categories like declarative or interrogative,%
\footnote{
    The term \term{sentence} is used here to refer to a somewhat ``complete'' utterance.
    Some, like \citet{cgel}, use the term \term{sentence} to refer to all utterances,
    which include e.g. noun phrase responds to questions.
}
and for embedded clauses we recognize their environments.
Then we investigate coordination and subordination
as well as topicalization and focalization.
Below these categories is the \term{nucleus clause},
which contains a subject and an extended verb phrase,%
\footnote{
    The term \term{verb phrase} may also be used to refer to what we refer to as the \term{verbal complex} here.
}
the latter is made of \ac{tam} markers plus an extended argument structure,
which in turn is made of peripheral arguments and the core argument structure.

\paragraph*{Speech forces}
The structure of a declarative sentence follows \prettyref{fig:finite-clause-template}.
Formation of \category{wh}-questions in French is done by \category{wh}-fronting.
Unlike the case in English, if the argument being interrogated about is introduced by a preposition,
the preposition has to go with the argument \citep[\citepage{60}]{rowlett2007syntax}.
What is unusual is that the most prevalent interrogative construction in French is a biclausal cleft construction
(\ref{ex:grammatical.speech-force.interrogative.wh.1}).
A shorter, structurally simpler form without the cleft structure is also available (\ref{ex:grammatical.speech-force.interrogative.wh.2}),
but in this construction, regular verb fronting (not just the fossilized \form{qu'est}) is involved
and therefore is considered too formal and hence rare in ordinary uses.
There is a similar cleft construction to formulate a yes/no question
(\ref{ex:grammatical.speech-force.interrogative.polar.1}),
and a structurally more direct and yet more formal variety also exists.

\begin{exe}
    \ex\label{ex:grammatical.speech-force.interrogative.wh.1}
    \gll Qu'est_{\text{inverted copula}}-ce [que tu manges]_{\text{relative clause}}? \\
            \category{wh}-be.\category{3sg.pres.ind}-this \category{rel} \category{2sg} eat-\category{2sg.pres.ind} \\
    \glt\translate{What are you eating? (lit. what is this thing that you are eating?)}

    \ex\label{ex:grammatical.speech-force.interrogative.wh.2}
    \gll Que [veux]_{\text{inverted verb}}-tu? \\
    \category{wh} want-\category{2sg.pres.ind}-you \\
    \glt\translate{What do you want?}
    
    \ex\label{ex:grammatical.speech-force.interrogative.polar.1}
    \gll Est_{\text{inverted copula}}-ce [que tu as faim]_{\text{complement clause}}? \\
    be.\category{3sg.pres.ind}-this \category{comp} \category{2sg} have.\category{2sg.pres.ind} hunger \\
    \glt\translate{Are you hungery? (lit. Is it that you have hunger?)}
\end{exe}

\paragraph*{Non-sentential clauses}
Clause combining includes complement clause constructions, relative clause constructions, and clausal coordination/subordination,
the last including conditional constructions, concession constructions, etc.
Non-sentential clauses may be finite, meaning that their structures are similar enough to those of sentences
(specifically, the \ac{tam} marking should be close enough),
or they may be non-finite, with structures deviating from finite structures.

In French, we have \category{indicative}, \category{subjunctive} and \category{conditional} finite clauses.
This distinction is traditionally known as the \category{mood};
it's also generalized to include other clause types (\prettyref{sec:verb-inflection.overview.categories}).
The \category{indicative}/\category{subjunctive} distinction
is about whether the clause is intended to convey an actual situation,
and sometimes the verb dictates the \category{mood} \citep[\citepage{753}]{l1999advanced}.

Non-finite constructions in French include one \category{infinitive} and two \category{participles}. 
Unlike the case in English, in French \category{infinitive} complement clauses
can be introduced by prepositions \form{à} and \form{de}.
This sometimes is due to the complement clause being treated in a way
consistent to noun phrases introduced by \form{à} and \form{de},
and sometimes is due to \form{de} acting as a complementizer
\citep[\citepages{158-159}]{rowlett2007syntax}.

\begin{todobox}{Control and raising}{control-raising}
    Control and raising constructions in French non-finite clause;
    also check backward control in psych-verbs.
\end{todobox}

\paragraph*{Information packaging}
Here we have a rough overview of information structure marking in French.
We have left dislocation and right dislocation topicalization
(\citealt[\citesec{5.3}]{rowlett2007syntax}; \prettyref{sec:information-structure.topic}),
and they may appear together (\ref{ex:grammatical.clause.information-structure.topic.1}).

\begin{exe}
    \ex\label{ex:grammatical.clause.information-structure.topic.1} 
    \gll [Jean]_{\text{left topic},i}, [il_{i}            m’énerve]_{\text{comment}}, [Jean]_{\text{right topic},i} \\
          Jean                          \category{3sg.m}  \category{1sg}-annoy         Jean \\
    \glt\translate{Jean annoys me.} (\citealt[\citepage{174}, (85)]{rowlett2007syntax})
\end{exe}

Focus fronting, in the form of \form{this book I don't know},
is not permitted in Contemporary French, but is good in Modern French
(\ref{ex:grammatical.clause.information-structure.focus.1}).
Apparent focus fronting instances \citep[\citesec{5.4}]{rowlett2007syntax}.

\begin{exe}
    \ex\label{ex:grammatical.clause.information-structure.focus.1}
    \gll  A  Jean j’ai                 donné €20 \\
          to Jean \category{1sg}-have  give  €20 \\
    \glt\translate{I gave Jean €20.}
\end{exe}

\begin{todobox}{Information packaging and complement clauses}{complement-clause-information-marking}
    Is information packaging absent in complement clauses?
\end{todobox}

\paragraph*{Structural template of nucleus finite clauses}
After discussing information marking and clause combining,
we briefly go through the structure of finite clauses,
which is summarized in \prettyref{fig:finite-clause-template}.

\begin{figure}[H]
    \caption{Linear template of finite clauses}
    \label{fig:finite-clause-template}
    \centering
    pre-subject topic or frame - subject - clitics - auxiliary/main verb - TAM and negation adverbials - main verb - object - peripheral arguments - post-nucleus topic or frame
\end{figure}

\paragraph*{The inflected verb and the \ac{tam} and negation adverbials}\label{sec:grammatical.clause.top-level.verb-tam-neg}
French is similar to English in that we have a sequence of \ac{tam} adverbials to the right of the subject,
and the verb bearing the main \ac{tam} inflection (also known as the \term{operator} in some works),
which, when there is no auxiliary verb, is the main verb,
and is the auxiliary verb when there is one,
gets mixed with these adverbials in the linear order.
Probably because of the morphophonologically ``stronger'' \ac{tam} features, 
this inflected verb is somehow promoted to a position ``higher'' than 
that of the English counterpart.
Thus, all adverbials, if not topicalized, 
always follow the verb bearing the main \ac{tam} categories (\ref{ex:grammatical.clause.template.aux.1}), 
and so are the negators. 
This resembles Early Modern English 
(e.g. \form{The wise of the world, and the great ones, said Luther, 
\focus{understand not} God's Word}).
This analysis is consistent with the fact that in non-finite constructions,
the verb appears after the negative adverb \form{pas}
\citep[\citepage{108}]{rowlett2007syntax}.

\begin{exe}
    \ex\label{ex:grammatical.clause.template.aux.1}
    \gll {} [Jean]_{\text{subject}} \focus{a}_{\text{auxiliary}}  [franchement]_{\text{speech-act}}  [besoin de se   laver]_{\text{core argument structure}} \\
            {} J.   has.\category{pres}.\category{3sg}            frankly      need   of self wash \\
        \glt \translate{Jean franky needs a wash.} (\citealt[\citepage{106}, (8a)]{rowlett2007syntax}) 
\end{exe}

\paragraph*{Clitics and the verbal complex}\label{sec:grammatical.clause.top-level.verbal-complex}
French has a personal clitic system corresponding to grammatical relations like 
direct and indirect objects (\prettyref{sec:grammatical.clause.clitic}).
The clitics, when present, seem to be attached before the inflected verb with a certain order
The linear order essentially forms a polysynthetic morphological template
\citep[\citepage{128}]{rowlett2007syntax}.
Especially in spoken French, the personal clitics and the negation marker \form{ne}
have formed a rigid sequence which arguably has polysynthetic features
(\prettyref{sec:grammatical.clause.top-level.verbal-complex}).
A question worth asking is whether in spoken French,
the clitics have already become a part of the post-syntactic morphological template,
whose details, apart from the grammatical categories it realizes,
are invisible to the syntax proper.
This seems to be not the case for neither personal clitics (\prettyref{sec:grammatical.clause.clitic}) nor negation \form{ne}.

\begin{todobox}{Negation}{negation}
    The two negation markers with different scopes.
\end{todobox}



\subsection{Subjecthood}

French is a typical nominative-accusative language:
the subject is defined 

\subsection{TAM categories}\label{sec:grammatical.clause.tam}
\ac{tam} categories in French are marked by verb inflection, auxiliaries,
and \ac{tam} adverbials.
A list of \ac{tam} adverbials are given in \citet[\citepage{103, (3)}]{rowlett2007syntax},
and their relative scopes are relative reflected by the linear order:
the higher an adverbal is, the lefter it appears,
and apparent deviations from this order may be due to 
one adverbial modifying another,
scrambling, or dislocation \citep[\citepages{104-105}]{rowlett2007syntax}.

Here, we investigate \ac{tam} categories marked by verb inflection and auxiliaries.

\paragraph*{Speaker-oriented categories} 
The only speaker-orientated category marked in French verbal inflection is epistemic modality,
marked by the \category{mood} category
(\prettyref{sec:verb-inflection.overview.categories}).

\paragraph*{Primary tense}
The primary tense establishes a \emph{reference time}
and compares its relation with the \emph{speech time}.
The ``speech time'', of course, can also be the time of writing,
or even the time of reading in written texts. 
French has a present/past/future distinction:
this is clearly demonstrated by the \category{simple tenses}
(\prettyref{sec:verb-inflection.overview.categories}). 
The past future tense seen in English (\form{he [would] take part in in if he knew it})
is absent in French: it's actually lacking in Latin as well.

The reference time sets the stage of the situation of affair being described;
the exact time when the situation happens is not necessarily identical to the reference time:
see the category of anteriority. 

\paragraph*{Deontic modality}
Deontic modality, or the category of whether an event is morally desirable,
doesn't seem to have stable verbal marking.
The \category{mood} is mainly about whether the clause is intended to describe a real situation.

\paragraph*{Habituality and similar categories}
In French the category of habitual or frequentative aspectuality is attested
in the distinction between the \category{past historic} and the \category{imperfect}:
the \category{imperfect} is about an event frequently appeared in the past,
while the \category{past historic} refers to a specific past incident.

\paragraph*{Anteriority}
Anteriority, i.e. whether the state of affair being described
happens before or after a reference time 
(whose relation with the speech time is determined by primary tense),
is sometimes recognized as tense and sometimes aspect.

In French we see a simple/perfect distinction illustrated
in the difference between the \category{perfect} and the \category{present},
or between the \category{pluperfect} and the \category{imperfect}
\citep[\citepage{148}]{l1999advanced}.

\paragraph*{Point of view aspectuality}
Events may be perceived as a whole or as something with internal structures;
the latter is what is known as the progressive or imperfective
(note that it's different from the imperfect defined as a type of anteriority).
French doesn't have a very explicit marker of the progressive aspect.
We however note that the \category{double compound past} \citep[\citepage{152}]{l1999advanced}
seems to mark the punctual or perfective point of view aspectuality.

\subsection{Agreement}

Cross-linguistically, agreement can happen at all levels of the clausal structure.
In French clauses, we first observe a clear subject-verb agreement:
the person and number of the subject is also indexed on the verb bearing the main \ac{tam} inflection
(\prettyref{sec:grammatical.clause.top-level.verb-tam-neg}).
One interesting phenomenon is that in auxiliary verb constructions 
or more often in French linguistics, in \category{compound tenses} (\prettyref{sec:verb-inflection.overview.categories}),
when the direct object appears before the main verb, main verb-direct object agreement optionally happens,
but when the direct object appears after the main verb,
this main verb-direct object agreement is not possible.

\begin{todobox}{Agreement typology}{agreement}
    In vP, etc.; check the direct object agreement
\end{todobox}

\subsection{Negation}

French has several negation markers often appearing together.
First we have the \form{ne} particle, whose scope is high
and marks the polarity of the whole clause,

French negation 

\begin{exe}
    \ex\gll [Ma                            grande soeur]_{\text{subject}}  [n’habite]_{\text{verbal complex}}                [pas]_{\text{negation adverb}}     [avec nous]_{\text{location}}. \\
            \category{1sg}.\category{poss} big    sister                   \category{neg}-habitate \category{neg} with \category{1pl} \\
    \glt\translate{My big sister doesn't live with us.}
\end{exe}

\subsection{Clitics}\label{sec:grammatical.clause.clitic}

One interesting aspect of French syntax is the existence of personal clitics.

The relative scopes of the clitics and the \ac{tam} adverbials are not reflected by the linear order because the clitics are all attached to

The personal clitics, despite being grammatical markers,
are likely not agreement morphemes in a fixed realizational morphological template in the verbal complex
that is not visible to the syntax,
but still have some syntactic autonomy.
For example, if we treat the subject clitic as an agreement marker,
we have to explain why the verbal complex has double agreement markers
while one of them can be omitted,
and why the inversion of verbs and subject clitics is possible
\citep{de2005french}.

\subsection{Core argument structure}

\paragraph*{Prepositional complementation} 
The subcategorization frame in some clauses may contain a whole directional construction.
In English, for example, we can have a whole directional construction
with markers giving the ground and the direction,
and also the directional particle describing the direction of the event,
as in \form{the boat drifted up from inside the cave}.
In French prepositional complementation is simpler.
We have arguments coming with \form{à}, \form{de}, and \form{pur}.

\paragraph*{Pronominal verbs}
Some verbs 

\section{The noun phrase}

\subsection{Top-level categories}

\paragraph*{The case system}\label{sec:grammatical.np.peripheral.case}
The case system, once highly regular in Latin, has almost completely eroded in French.
Like modern English, case is only preserved in pro-forms in French.

Noun phrases also appear in directional and locational constructions,
which, cross-linguistically, have a ground-direction hierarchy and multiple markers may appear together
(as in e.g. \form{from under the bed} in English).
In French 

\paragraph*{Number} French has singular and plural numbers.

\paragraph*{Structural template of noun phrases}

\paragraph*{Adjective phrases}
In French, adjective phrases (\prettyref{sec:grammatical.pos.adjective}) are found both pre- and post-head noun.
The positions of the adjective phrases seem to stem from grammatical factors,
but the relevant factors seem to be more complicated than they are in English,
where a single hierarchy is enough to fix the linear order of adjectives,
and all deviations can be explained by the direct modification/reduced relative clause distinction
and/or information structure.

\subsection{Direct modifications}\label{sec:grammatical.np.direct-modify}

\begin{todobox}{Adjective classification}{adjective-classification}
    Direct modification and reduced relative clauses

    Also, does French have things like \form{a London legal adviser}?
    
    Scopes of direct modification: \citep[\citesec{3.7.1}]{rowlett2007syntax}
\end{todobox}


\subsection{Argument structure}

A linear order of arguments in the noun phrase can be observed in French,
which shows a structure parallel to the clausal structure \citep[\citepage{21}]{rowlett2007syntax}.

\begin{todobox}{Relative order of adjectives and}{adjective-pp}
    Direct modification and reduced relative clauses
\end{todobox}

\subsection{Compounding}\label{sec:grammatical.np.compound}

What is informally known as compounding may be 
genuine compounding or nominal modification/complementation with semantic idiomization
\citep[\citechap{5}, \citesec{14.4}]{cgel}.
In order to have genuine compounding or ``syntactic fossilization'',
usually we expect the ``compound'' to have subcategorization frames
not predictable from its constituents
or to have strong complexity constraints in its constituents.
According to these criteria, French does seem to have genuine compounds,
according to criteria (a) and (b) in \citet[\citepage{15}]{rowlett2007syntax}.

\section{Parts of speech}

\subsection{Theoretical caveats}\label{sec:grammatical.pos.theory}

The term \term{part of speech} may refer to several things.
We can define \emph{syntactic} part of speech tags,
and hence the term \term{noun} refers to the head of a noun phrase, etc.
Tags defined in this way are mostly universal,
although we can also define tags like ``noun formed by this-and-this nominalization'',
which involves language-specific constructions.
They however by no means entail exclusivity:
we may find the same form appearing both as a syntactic verb or a syntactic noun.
We can also define \emph{morphological} part of speech tags,
which depend on (realizational) morphological details of the specific language in question,
and therefore in Latin we have nominal- and verbal-like inflections,
but we don't have a completely separate adjective class.
The third type of part of speech tag is defined by classification of lexical items
according their properties that guide their syntactic and morphological parts of speech.
Thus the category \category{noun} in Latin means the bundle of 
the ability to serve (or ``categorize'') as a syntactic noun, its morphological template
(which needs to be supplemented by the exact declension class),
and some language-specific properties like how it's verbalized.
This \emph{lexical} definition of parts of speech is the type of parts of speech we discuss here.

The lexicon contains roots and various idioms (in the generic sense) with fossilized \emph{meanings},
i.e. \term{lexicalized items}.%
\footnote{
    Fossilization of \emph{syntax} includes grammaticalization (i.e. evolution into grammatical marker)
    and collapse of the internal structure,
    like a noun phrase idiom evolving into a compound
    (\prettyref{sec:grammatical.np.compound}).
}
The sizes of these idioms are not all the same.
Morphological words, with given lexical parts of speech,
are often most frequently encountered as lexical terms cross-linguistically,
and in this case dictionaries should usually be organized by words.
But we note that there is no guarantee that lexical parts of speech like \category{noun} or \category{verb}
defined for words directly form by directly putting roots into noun phrases and clauses 
apply to all items in the lexicon:
it's possible that, say, a root that can be categorized as a (syntactic) adjective,
after some sort of derivation, forms another syntactically adjective-like construction.
The heads of the two constructions, when recorded in the lexicon,
should be placed into two lexical parts of speech \citep[\citechap{5}]{paul2014new}. 
Languages can also have very regular derivational morphology and in this case
the dictionary should be organized in term of roots (as in Arabic and Mandarin Chinese),
and they can also have not-so-regular derivational morphology but regular inflectional morphology,
and in this case the dictionary should be organized in terms of stems.%
\footnote{
    This is mostly the case in Japanese,
    although in actual dictionaries sub-phrasal lexical items are still given as words:
    inflections are obtained by ``changing the last syllable'' and adding inflectional suffixes.
}
These subtleties are largely non-existent for the lexicon of French.

Grammatical markers, or ``function words'' or ``function morphemes'',
in theory do not need be included in a dictionary.
But often they are because whether an item is considered a grammatical item may vary
diachronically or even within the population.
Some grammatical markers also have well-defined \emph{syntactic} part of speech tags:
they are usually some sort of pro-forms;
other grammatical markers, like case markers, don't.
Real lexical items, or ``content words'' or ``lexical morphemes'',
may share one feature with grammatical items:
they sometimes may form closed lexical parts of speech.
This is often the case for \category{verb} or \category{adjective} categories identified in some languages.

\subsection{Noun}

The category \category{noun} is distinguished by its morphological template:

\subsection{Verb}

The category \category{verb} can be distinguished by 

\subsection{Adjectives}\label{sec:grammatical.pos.adjective}

What are known as adjectives or adjective phrases include direct modification
(\prettyref{sec:grammatical.np.direct-modify})
and so-called reduced relative clauses.

\subsection{Adverbs}

What are known as adverbials include everything that appears to be a modifier in the clause system.
They include \ac{tam} adverbials (\prettyref{sec:grammatical.clause.tam}),
peripheral arguments,
and occasionally conjunctions.
The term \term{adverb}, as is mentioned above in \prettyref{sec:grammatical.pos.theory},
may syntactically mean the head of an adverbial (if the adverbial is not directly made of a noun phrase or a clause),
or it may mean a lexicalized item that regularly acts as an adverb.
It seems in 

\subsection{Prepositions}

What are known as prepositions, in theory, can be adverbs with complementation
or grammatical markers for the case system.
In French they are largely grammatical:
the class is largely closed and largely integrated into the grammar.
The grammatical complexity of prepositional constructions in French is not high:
preposition stacking, for example, is absent \prettyref{sec:grammatical.np.peripheral.case}.

\chapter{Verb inflection}

\section{Overview}

\subsection{Categories involved in inflection}\label{sec:verb-inflection.overview.categories}

\paragraph*{TAM categories}
The category \category{mood} in French marks the indicative/subjunctive distinction.
Note that imperative and the conditional clause types
are also placed under the concept of \category{mood} in French grammar.
The two non-finite constructions, the \category{infinitive} and the \category{participle},
are also regarded as \category{moods},
known as \category{impersonal moods} because they don't license subjects.
Therefore we have four \category{personal moods} and two \category{impersonal moods} in Latin.

There are 9 \category{tenses} in the \category{indicative} system of French verb conjugation.
According to whether their markings involve auxiliary verbs,
they can be divided into \category{simple tenses}, namely \category{present}, \category{future}, \category{imperfect}, and \category{past historic},
and \category{compound tense}, namely \category{perfect}, \category{future perfect}, \category{pluperfect}, \category{past anterior}, and \category{double compound past}.

\paragraph*{Person and number}
The categories of person and number are also marked on finite verbs,
and their values are determined by subject-verb agreement. 

\subsection{Conjugation classes}

French verbs are classified according to 

\chapter{Verb frames}



\chapter{Information structure}

\section{Topicalization}\label{sec:information-structure.topic}

French has two types of topicalization constructions.
In the left dislocation construction \citep[\citepages{174-175}]{rowlett2007syntax},
the topic is coreferential with a pro-form in the comment
and appears at the left of the comment
(\prettyref{ex:information-structure.topic.simple.1}),
and the coreferential relation can cross clause boundaries
(\ref{ex:information-structure.topic.complement-clause.1}).
The process can happen multiple times
(\prettyref{ex:information-structure.topic.two-topic.1}),
and the pro-form can even appear in the topic of another topic-comment construction
(\ref{ex:information-structure.topic.from-topic.1}).

\begin{exe}
    \ex\label{ex:information-structure.topic.simple.1}
    \gll [Jean]_{\text{topic},i}, [il_{\text{subject},i}  m’aime]_{\text{comment}}. \\
          Jean                     \category{3sg}         \category{1sg}-love \\
    \glt\translate{Jean, he loves me.}

    \ex\label{ex:information-structure.topic.complement-clause.1}
    \gll  [Moi]_{\text{topic},i}, [il             faut          [que            j’_i            aille en ville]_{\text{complement clause}}]_{\text{comment}}. \\
           \category{1sg}          \category{3sg} be.necessary \category{comp}  \category{1sg}  go    in town  \\
    \glt\translate{I, it's necessary that I go to town.}

    \ex\label{ex:information-structure.topic.two-topic.1}
    \gll [Jean]_{\text{topic},i}, [Marie]_{\text{topic},j}, [il_{\text{subject},i}        l’_{\text{object},j}          aime bien]_{\text{comment}}. \\
          Jean      Marie     \category{3sg.m}  \category{3sg.f} like well \\
    \glt\translate{Jean likes Marie.}
    
    \ex\label{ex:information-structure.topic.from-topic.1}
    \gll [Moi]_{\text{topic},i}, [mon_i               frère]_{\text{topic},j}, [sa_j                femme]_{\text{topic},k}, 
    [elle_k           est malade]_{\text{comment}}. \\
          \category{1sg}          \category{1sg.poss} brother                   \category{3sg.poss} wife 
     \category{3sg.f} be.\category{3sg}.\category{pres}.\category{ind}  ill \\
    \glt\translate{My brother’s wife is ill.}
\end{exe}

Right dislocation is also possible
(\prettyref{ex:information-structure.topic.two-topic-right.1}).
In this construction, the right dislocated constituent is still a topic,
but it's a much more restricted one:
it's more presuppositional and is compatible with only known topics \citep[\citepage{181}]{rowlett2007syntax}.


\begin{exe}
    \ex\label{ex:information-structure.topic.two-topic-right.1}
    \gll [Il_i              l’_j             aime bien]_{\text{comment}}, [Jean]_{\text{topic},i}, [Marie]_{\text{topic},j}. \\
         \category{3sg.m}  \category{3sg.f}  like well                     Jean                     Marie \\
    \glt\translate{Jean likes Marie.}
\end{exe}



\bibliographystyle{plainnat}
\bibliography{grammars.bib,typology.bib}

\end{document}