\documentclass[a4paper]{article}

\usepackage{geometry}
\usepackage{caption}
\usepackage{subcaption}
\usepackage{abstract}
\usepackage{paralist}
\usepackage{ulem}
\usepackage{amsmath, amssymb}
\usepackage{qtree}
\usepackage{gb4e}
\usepackage[colorlinks, linkcolor=black, anchorcolor=black, citecolor=black]{hyperref}
\usepackage{prettyref}

\geometry{left=3.18cm,right=3.18cm,top=2.54cm,bottom=2.54cm}

\DeclareMathOperator{\pmergel}{PMergeL}
\DeclareMathOperator{\pmerger}{PMergeR}
\DeclareMathOperator{\hmergel}{HMergeL}
\DeclareMathOperator{\hmerger}{HMergeR}
\DeclareMathOperator{\copyl}{CopyL}
\DeclareMathOperator{\copyr}{CopyR}
\DeclareMathOperator{\agree}{Agree}
\DeclareMathOperator{\move}{Move}
\DeclareMathOperator{\undefined}{undefined}
\setlength{\arraycolsep}{2pt}
\newcommand*{\synbracket}[2][{}]{[_\mathrm{#1} \; \begin{matrix} #2 \end{matrix} \; ]}

\newrefformat{fig}{Figure \ref{#1}}
\newrefformat{exe}{(\ref{#1})}

\newcommand*{\concept}[1]{\underline{\textbf{#1}}}

\title{The Framework of Minimalist Syntax}
\author{wujinq}

\begin{document}

\maketitle

\begin{abstract}
    This article is a summary of Minimalist syntax in modern generative grammar.
    We review the mutually recognized primitives of the Minimalist Program, its main motivation, and its main achievements.
    We also present main disputes - both theoretical and empirical issues around this research program, and list some dominant frameworks trying to settle these issues.  
\end{abstract}

\section{Introduction}

The tradition of the \concept{Principle and Parameter} (henceforward P\&P) approach has achieved huge success since 1980s. 
It, however, is definitely not the end of the generative enterprise, with too many principles and parameters required to cover all empirical observation, sometimes rendering P\&P simply curve-fitting of complicated phenomena instead of an explanatory theory.
If the goal of P\&P was to explore and explain the universal rules of all human languages, then the \concept{Minimalist Program} (henceforward MP) is focused on explore and explain P\&P.
What are obligatory? What can be explained using other concepts? 

That is how MP began - We need to investigate the necessity of every concept used in P\&P, trying to find the real building blocks of human language.
That is why MP is called a \emph{research program} instead of a theory: there are, of course, something basic stipulations in MP that fall in the range of chomsky's derivational syntax.
For example, there is Merge as the fundamental structure building operation, and Move is the origin of non-locality.
Besides these, however, over many issues no mutual consensus is achieved. We present some of these issues in this article and relevant solutions.

We have several goals of MP. The ideal theory about grammar must cover all empirical observations, meaning it should be productive enough, and should never generate something never attested in human language, requiring it to be not too generative.
There should be a series of operations, rendering syntactic derivations to conform our intuition about language and be interpretable, but not too many of them, so it will be easier to enact proper constraints to avoid over-generation and meanwhile maintain empirical coverage.
These (often conflicting) requirements are challenging to meet, 
% TODO

\section{The structure of grammar}

This section deals with syntactic objects, like morphemes, words, phrases, and the operations used to build them.
We are not to give a formalization of Minimalist syntax here so we will not use much technical terms, 
and often discussion about a certain concept will involves concepts yet to be defined.

\subsection{The Y-shaped model}

\subsection{Purely syntactic operations}

\subsubsection{Merge: External and Internal}

\concept{Maximal projection}

\subsubsection{Agree}

% feature assignment

\subsubsection{Anything else?}

Merge, Move, Agree, and possibly Copy - do these form a exhaustive list of purely syntactic operations? 
The answer differs within different frameworks (See \prettyref{sec:framework} for some examples), and any possible additional operations are highly controversial (See \prettyref{sec:issue-op} for some discussion).
Here we only list some possible choices, reserving debate about their existence to \prettyref{sec:issue-op}.

Frequently proposed operations other than Merge, Move, Agree and Copy include:

\begin{itemize}
    \item \concept{Adjoin}, to attach an \concept{adjunct} - something optional and semantically interpreted differently with complements and specifiers - to a maximal projection. 
    \item \concept{Head movement}, to move a head to another head and adjoin these two heads, forming a complex head.
    \item \concept{Match}, to link two constituents' reference to the same object.
\end{itemize}

Some purely syntactic operations can be substituted by post-syntactic operation, and vice versa.

\subsection{Syntax-phonology interface and syntax-morphology interface}

\subsubsection{Transferring and phases}

\subsubsection{Post-syntactic operations}

It should be noted that whether post-syntactic operations really exist is still questionable. (See \prettyref{sec:post-syn-op} for some discussion)
This section only list some possible post-syntactic operations, which may be just phenomenological.

\subsection{Phenomena of interest}

Now we name a few phenomena of particular significance in MP.
Descriptive grammar take them for granted and just dive into detailed explanation of them, while MP should \emph{derive} them, not just simply stating ``there are such phenomenon''.

\subsubsection{Categories}

We have nouns, verbs, adjectives, adverbs, prepositions and a lot of other \concept{categories} in any human languages.
It is almost a miracle that despite the unimaginable possible number of ways to express an event or an idea (for example, formal logic, or embedding vectors used in deep neural networks), everyone in the world tend to use the same toolkit of words as the building block of their utterance.
Nonetheless, categories across languages have slight discrepancy, examples of which includes the fact that in European languages adjectives are more ``nominal'', while in East Asian languages adjectives are more ``verbal'' in contrast, and in some languages such as Japanese, there are actually two systems of adjectives.
MP need to account for the nature of categories, and if there is an underlying universal structure of categories among nature languages, then why categories seem to be mildly flexible.

\subsubsection{Word formation}

\concept{Word formation} means how features in a tree is phonetically realized as \concept{words}, or independent phonological units.
The actual definition of word is a big headache among others.
Some languages, like the Chinese languages, do not even have clear word boundaries.
The process of word formation is also related to how a tree is spelt out (Is it spelt out via spanning, or via phrasal spellout?), what is the structure of the lexicon (Are words prepared there in advance, or is the lexicon purely a collection of abstract features, or maybe it is a museum of interesting structures, storing syntax-semantics-phonology tuples?), and the structure of grammar (Is there a separate module of morphology?).

\subsubsection{Recursion}



\subsubsection{Constraints of derivation}



\section{Frameworks and programs within MP}\label{sec:framework}

\subsection{Cartography}

\subsection{Antisymmetry}

\subsubsection{The failiure of LCA, and a possible alternative}

\subsection{Distributed Morphology}

\concept{Distributed Morhplogy} (henceforward DM) is a late-insertion morphosyntax theory.
Contrary to the lexicalist view that the word is the basic unit manipulated in syntax, DM assume

\subsubsection{Concerns and disputes around DM}

There are, of course, metatheoretical concerns about DM.
One is about the status of post-syntactic operations: they happened \emph{before} Vocabulary Insertion, while all information about morphology or phonology is stored in Vocabulary, and thus at the state of post-syntactic operations, terminal nodes simply do not know how they should be \emph{merged} or \emph{fused}.
There are two ways to ensure correct post-syntactic operations: 
\begin{inparaitem}
    \item[(a)] to introduce some kind of \emph{looking ahead}: we \emph{have to} stipulate that somehow terminal nodes can get access to Vocabulary even before Vocabulary Insertion, or 
    \item[(b)] to assume that there may be some language specific wellformedness rules about the ``correct'' morphology, contrary to the hypothesis that every language specific feature can be attributed to some discrepancy in the lexicon.
\end{inparaitem}

% TODO: this hypothesis has its name as Chomsky-xxx hypothesis

Another concern is about the productivity of post-syntactic operations.
There may already be too many of them: merger, fusion, fission, rearrangement, and a long following list.
Each of them seems to be handy for explanation of some phenomenon, but when every syntactican is adding post-syntactic operations to DM, soon the list is too long, resulting in over-generation and going against the idea of conciseness of Minimalism.

Ironically, though post-syntactic operations are criticized as over-generative, the standard Vocabulary Insertion process is actually too restrictive.
It is quite easy to observe that Fusion is actually a simulation of phrasal spell-out within the framework of terminal node spell-out, implying that restricting spelling-out on terminal nodes actually does not make sense, since it can be easily bypassed via Fusion.
If, on the other hand, we allow phrasal spell-out, then problems immediately occur concerning the principle of Underspecification.

\subsection{Nanosyntax}

\section{Issues regarding operations}\label{sec:issue-op}

\subsection{Adjunction}

\subsection{Constraints on movement}

\subsubsection{Chain and its properties}

% TODO: chain uniformity condition  

\subsubsection{Locality and anti-locality}\label{sec:locality}

\subsubsection{Cyclicity}\label{sec:cyclic}

% successive cyclic
cyclic movement

A constituent should be moved to the edge of the current constituent.

\subsection{Head movement}

% TODO: snowball roll-up

\subsubsection{Definition and issues}

\concept{Head movement} is a kind of movement where a head is raised to another head position, resulting in a complex head, formed by two adjoined heads, as shown in \prettyref{exe:head-movement}.
This operation is a standard component of P\&P, where it is used to explain why there are subject-verb inversion, incorporation of V and v, and many other things that seems to involve complex heads.

% TODO: movement arrows
\begin{exe}
    \ex\label{exe:head-movement} \lb{\alpha}\lb{\alpha} $\alpha$ $\beta$ ]\lb{\beta} \ldots \sout{$\beta$} \ldots]]
\end{exe}

Up to now, head movement seems to be a quite nature operation, but a thorough review of its attributes results in the following puzzles:
\begin{inparaitem}
    \item[(a)] It obviously is not a cyclic movement (\autoref{sec:cyclic}), since the moved material is not moved to the edge of the present constituent.
    Even if we do not accept the strict cyclic condition, head movement still goes against the Extension Condition.
    \item[(b)] It raises problems about c-command relations - if $\beta$ c-commands something in its maximal projection, after movement the relation is broken.
    This is a direct result of (a). 
    \item[(c)] The action of taking out a head that is already merged into the tree and adjoin something onto it is quite strange.
    \item[(d)] Head movement violates the Head Uniformity Condition, because the higher copy of $\alpha$ is a head, but since it does not project any more, it is also a (bare) phrase.
    \item[(e)] Head movement violates the Anti-Locality Hypothesis. 
\end{inparaitem}

\subsubsection{Explanation: sideward movement}

A possible try 

\subsubsection{Alternative: post-syntactic operations}

\subsubsection{Alternative: remnant movement}

\subsubsection{Do we really need a uniform head movement?}

Now we already have a list of alternative to the traditional notion of head movement, each of which can cover a 

\subsection{Post-syntactic operations}\label{sec:post-syn-op}

\subsection{The mechanism of Agree}


\section{The structure of grammar}

\subsection{The status of morphology}

\subsection{What is in the lexicon?}

% TODO: Cleaning up the lexicon, Michal Starke

The lexicon has been the trash bin of the current Minimalist enterprise: whatever exceptional is attributed to ``something weird in the lexicon'', causing great mess.
The result is the lexicon may eventually have its own ``syntax'': its generative power is so strong, that most of the task traditionally distributed to the syntax proper can be handled by the lexicon itself, so the explanation power of the syntax is largely undermined.
Here we list a subset of technology thrown into the lexicon, and critically review whether they are really obligatory.
They are: feature bundles, context sensitive rewriting rules, subcategorization frames, notations for coreference, thermatic roles.
The list is far from exhaustive, but our way to deal with them should provide insights to how to clean up the lexicon.

\subsubsection{Feature bundles, or one feature one head?}

Starting from the notation of feature bundles. A \concept{feature bundle} is a set of features in a pair of brackets, what trivially corresponds to a flat $n$-ary tree.
Including feature bundles in the lexicon thus gives us two tree diagrams: one is binary-branching and movable, the other is flat and unbreakable.
We, therefore, has 

The traditional concept of feature bundle 

\subsubsection{Context sensitive rewritting rules}


\section{Morphosyntax phenomena}

\subsection{Word formation, or what is a word}

\subsubsection{The Lexicalist Hypothesis and why it is wrong}

\subsection{Agreement and concord}

\subsection{Pronouns, reflexives and others}

\section{Phrase and sentence structures}

\subsection{Word order}

\subsection{Modifying and adjunction}

\subsection{Coordination}

\subsection{Relative clauses}

\subsection{Asymmetry}

% more SOV, less VOS

\section{Conclusion: where we are and where to proceed}

% TODO:reference and bibtex

\end{document}