\documentclass[a4paper]{article}

\usepackage{geometry}
\usepackage{caption}
\usepackage{subcaption}
\usepackage{abstract}
\usepackage{paralist}
\usepackage{amsmath, amssymb}
\usepackage{qtree}
\usepackage{gb4e}
\usepackage[colorlinks, linkcolor=black, anchorcolor=black, citecolor=black]{hyperref}
\usepackage{prettyref}

\geometry{left=3.18cm,right=3.18cm,top=2.54cm,bottom=2.54cm}

\DeclareMathOperator{\pmergel}{PMergeL}
\DeclareMathOperator{\pmerger}{PMergeR}
\DeclareMathOperator{\hmergel}{HMergeL}
\DeclareMathOperator{\hmerger}{HMergeR}
\DeclareMathOperator{\copyl}{CopyL}
\DeclareMathOperator{\copyr}{CopyR}
\DeclareMathOperator{\agree}{Agree}
\DeclareMathOperator{\move}{Move}
\DeclareMathOperator{\undefined}{undefined}
\setlength{\arraycolsep}{2pt}
\newcommand*{\synbracket}[2][{}]{[_\mathrm{#1} \; \begin{matrix} #2 \end{matrix} \; ]}

\newrefformat{fig}{Figure \ref{#1}}

\newcommand*{\concept}[1]{\underline{\textbf{#1}}}

\title{The Framework of Minimalist Syntax}
\author{wujinq}

\begin{document}

\maketitle

\begin{abstract}
    This article is a summary of Minimalist syntax in modern generative grammar.
    We review the mutually recognized primitives of the Minimalist Program, its main motivation, and its main achievements.
    We also present main disputes - both theoretical and empirical issues around this research program, and list some dominant frameworks trying to settle these issues.  
\end{abstract}

% TODO:
% successive cyclic
% head movement

\section{Introduction}

The tradition of the \concept{Principle and Parameter} (henceforward P\&P) approach has achieved huge success since 1980s. 
It, however, is definitely not the end of the generative enterprise, with too many principles and parameters required to cover all empirical observation, sometimes rendering P\&P simply curve-fitting of complicated phenomena instead of an explanatory theory.
If the goal of P\&P was to explore and explain the universal rules of all human languages, then the \concept{Minimalist Program} (henceforward MP) is focused on explore and explain P\&P.
What are obligatory? What can be explained using other concepts? 

That is how MP began - We need to investigate the necessity of every concept used in P\&P, trying to find the real building blocks of human language.
That is why MP is called a \emph{research program} instead of a theory: there are, of course, something basic stipulations in MP that fall in the range of chomsky's derivational syntax.
For example, there is Merge as the fundamental structure building operation, and Move is the origin of non-locality.
Besides these, however, over many issues no mutual consensus is achieved. We present some of these issues in this article and relevant solutions.

We have several goals of MP. The ideal theory about grammar must cover all empirical observations, meaning it should be productive enough, and should never generate something never attested in human language, requiring it to be not too generative.
There should be a series of operations, rendering syntactic derivations to conform our intuition about language and be interpretable, but not too many of them, so it will be easier to enact proper constraints to avoid over-generation and meanwhile maintain empirical coverage.
These (often conflicting) requirements are challenging to meet, 
% TODO

\section{The structure of grammar}

This section deals with syntactic objects, like morphemes, words, phrases, and the operations used to build them.
We are not to give a formalization of Minimalist syntax here so we will not use much technical terms, 
and often discussion about a certain concept will involves concepts yet to be defined.

\subsection{The Y-shaped model}

\subsection{Purely syntactic operations}

\subsubsection{Merge: External and Internal}

\concept{Maximal projection}

\subsubsection{Agree}

% feature assignment

\subsubsection{Anything else?}

Merge, Move, Agree, and possibly Copy - do these form a exhaustive list of purely syntactic operations? 
The answer differs within different frameworks (See \prettyref{sec:framework} for some examples), and any possible additional operations are highly controversial (See \prettyref{sec:issue-op} for some discussion).
Here we only list some possible choices, reserving debate about their existence to \prettyref{sec:issue-op}.

Frequently proposed operations other than Merge, Move, Agree and Copy include:

\begin{itemize}
    \item \concept{Adjoin}, to attach an \concept{adjunct} - something optional and semantically interpreted differently with complements and specifiers - to a maximal projection. 
    \item \concept{Head movement}, to move a head to another head and adjoin these two heads, forming a complex head.
    \item \concept{Match}, to link two constituents' reference to the same object.
\end{itemize}

Some purely syntactic operations can be substituted by post-syntactic operation, and vice versa.

\subsection{Syntax-phonology interface and syntax-morphology interface}

\subsubsection{Transferring and phases}

\subsubsection{Post-syntactic operations}

It should be noted that whether post-syntactic operations really exist is still questionable. (See \prettyref{sec:post-syn-op} for some discussion)
This section only list some possible post-syntactic operations, which may be just phenomenological.

\subsection{Categories}

\subsection{Constraints}

\subsubsection{Constituent selection}

\subsubsection{What triggers a movement?}

\subsubsection{Cyclic-successive derivation}

A constituent should be moved to the edge of the current constituent.

\subsubsection{Locality and anti-locality of movements}

\section{Frameworks and programs within MP}\label{sec:framework}

\subsection{Cartography}

\subsection{Antisymmetry}

\subsubsection{The failiure of LCA, and a possible alternative}

\subsection{Distributed Morphology}

\concept{Distributed Morhplogy} (henceforward DM) is a late-insertion morphosyntax theory.
Contrary to the lexicalist view that the word is the basic unit manipulated in syntax, DM assume

\subsubsection{Concerns and disputes around DM}

There are, of course, metatheoretical concerns about DM.
One is about the status of post-syntactic operations: they happened \emph{before} Vocabulary Insertion, while all information about morphology or phonology is stored in Vocabulary, and thus at the state of post-syntactic operations, terminal nodes simply do not know how they should be \emph{merged} or \emph{fused}.
There are two ways to ensure correct post-syntactic operations: 
\begin{inparaitem}
    \item[a.] to introduce some kind of \emph{looking ahead}: we \emph{have to} stipulate that somehow terminal nodes can get access to Vocabulary even before Vocabulary Insertion, or 
    \item[b.] to assume that there may be some language specific wellformedness rules about the ``correct'' morphology, contrary to the hypothesis that every language specific feature can be attributed to some discrepancy in the lexicon.
\end{inparaitem}

% TODO: this hypothesis has its name as Chomsky-xxx hypothesis

Another concern is about the productivity of post-syntactic operations.
There may already be too many of them: merger, fusion, fission, rearrangement, and a long following list.
Each of them seems to be handy for explanation of some phenomenon, but when every syntactican is adding post-syntactic operations to DM, soon the list is too long, resulting in over-generation and going against the idea of conciseness of Minimalism.

Ironically, though post-syntactic operations are criticized as over-generative, the standard Vocabulary Insertion process is actually too restrictive.
It is quite easy to observe that Fusion is actually a simulation of phrasal spell-out within the framework of terminal node spell-out, implying that restricting spelling-out on terminal nodes actually does not make sense, since it can be easily bypassed via Fusion.
If, on the other hand, we allow phrasal spell-out, then problems immediately occur concerning the principle of Underspecification.

\subsection{Nanosyntax}

\section{Issues regarding operations}\label{sec:issue-op}

\subsection{Head movement}

% TODO: snowball roll-up

\subsubsection{Alternative: post-syntactic operations}

\subsubsection{Alternative: remnant movement}

\subsection{Adjunction}

\subsection{Post-syntactic operations}\label{sec:post-syn-op}

\subsection{The mechanism of Agree}

\subsection{Puzzles about the lexicon}

\section{The structure of grammar}

\subsection{The status of morphology}

\section{Morphosyntax phenomena}

\subsection{Word formation, or what is a word}

\subsubsection{The Lexicalist Hypothesis and why it is wrong}

\subsection{Agreement and concord}

\subsection{Pronouns, reflexives and others}

\section{Sentence structures}

\subsection{Word order}

\subsection{Modifying and adjunction}

\subsection{Relative clauses}

\subsection{Asymmetry}

% more SOV, less VOS

\section{Conclusion: where we are and where to proceed}

% TODO:reference and bibtex

\end{document}