\documentclass[UTF8, a4paper, oneside, scheme=plain]{ctexart}

\usepackage{geometry}
\usepackage{titling}
\usepackage{titlesec}
\usepackage{paralist}
\usepackage{footnote}
\usepackage{enumerate}
\usepackage{amsmath, amssymb, amsthm}
\usepackage{gb4e}
\noautomath
\usepackage{bbm}
\usepackage{soul}
\usepackage{graphicx}
\usepackage{siunitx}
\usepackage[table,xcdraw]{xcolor}
\usepackage{tikz}
\usepackage[ruled, vlined, linesnumbered, noend]{algorithm2e}
\usepackage{xr-hyper}
\usepackage[colorlinks]{hyperref} % linkcolor=black, anchorcolor=black, citecolor=black, filecolor=black
\usepackage[most]{tcolorbox}
\usepackage{caption}
\usepackage{subcaption}
\usepackage{booktabs}
\usepackage{multirow}
\usepackage[figuresright]{rotating}
\usepackage{acro}
\usepackage[round]{natbib} 
\usepackage{nameref,zref-xr}
\zxrsetup{toltxlabel}
\zexternaldocument*[draft-]{./main}[main.pdf]
\zexternaldocument*[cgel-]{../English/cambridge}[cambridge.pdf]
\zexternaldocument*[latin-]{../Latin/latin-notes}[latin-notes.pdf]
\zexternaldocument*[alignment-]{../alignment/alignment}[alignment.pdf]
\zexternaldocument*[exercise1-]{../Exercise/2021-3}[2021-3.pdf]
\zexternaldocument*[method-]{../methodology/glossing}[glossing.pdf]
\usepackage{prettyref}

\geometry{left=3.18cm,right=3.18cm,top=2.54cm,bottom=2.54cm}
\titlespacing{\paragraph}{0pt}{1pt}{10pt}[20pt]
\setlength{\droptitle}{-5em}

\DeclareMathOperator{\timeorder}{\mathcal{T}}
\DeclareMathOperator{\diag}{diag}
\DeclareMathOperator{\legpoly}{P}
\DeclareMathOperator{\primevalue}{P}
\DeclareMathOperator{\sgn}{sgn}
\newcommand*{\ii}{\mathrm{i}}
\newcommand*{\ee}{\mathrm{e}}
\newcommand*{\const}{\mathrm{const}}
\newcommand*{\suchthat}{\quad \text{s.t.} \quad}
\newcommand*{\argmin}{\arg\min}
\newcommand*{\argmax}{\arg\max}
\newcommand*{\normalorder}[1]{: #1 :}
\newcommand*{\pair}[1]{\langle #1 \rangle}
\newcommand*{\fd}[1]{\mathcal{D} #1}

\newcommand*{\citesec}[1]{\S~{#1}}
\newcommand*{\citechap}[1]{chap.~{#1}}
\newcommand*{\citefig}[1]{Fig.~{#1}}
\newcommand*{\citetable}[1]{Table~{#1}}
\newcommand*{\citepage}[1]{pp.~{#1}}

\newrefformat{sec}{\citesec{\ref{#1}}}
\newrefformat{fig}{\citefig{\ref{#1}}}
\newrefformat{tbl}{\citetable{\ref{#1}}}
\newrefformat{chap}{\citechap{\ref{#1}}}

\usetikzlibrary{arrows,shapes,positioning}
\usetikzlibrary{arrows.meta}
\usetikzlibrary{decorations.markings}
\tikzstyle arrowstyle=[scale=1]
\tikzstyle directed=[postaction={decorate,decoration={markings,
    mark=at position .5 with {\arrow[arrowstyle]{stealth}}}}]
\tikzstyle ray=[directed, thick]
\tikzstyle dot=[anchor=base,fill,circle,inner sep=1pt]


\tcbuselibrary{skins, breakable, theorems}

\newtcbtheorem[number within=chapter]{infobox}{Box}%
  {colback=blue!5,colframe=blue!65,fonttitle=\bfseries, breakable}{infobox}

\newcommand*{\concept}[1]{\textbf{#1}}
\newcommand*{\term}[1]{\emph{#1}}
\newcommand{\corpus}[1]{\emph{#1}}

\DeclareAcronym{blt}{short = BLT, long = Basic Linguistic Theory}
\DeclareAcronym{cgel}{short = CGEL, long = The Cambridge Grammar of the English Language}
\DeclareAcronym{dm}{short = DM, long = Distributed Morphology}
\DeclareAcronym{tag}{long = Tree-adjoining grammar, short = TAG}
\DeclareAcronym{sfp}{long = sentence final particle, short = SFP}
\DeclareAcronym{np}{long = noun phrase, short = NP}
\DeclareAcronym{vp}{long = verb phrase, short = VP}
\DeclareAcronym{pp}{long = preposition phrase, short = PP}
\DeclareAcronym{cls}{long = classifier, short = CLS}
\DeclareAcronym{dist}{long = distal, short = DIST}
\DeclareAcronym{prox}{long = proximate, short = PROX}
\DeclareAcronym{dem}{long = demonstrative, short = DEM}
\DeclareAcronym{dur}{long = durative, short = DUR}
\DeclareAcronym{neg}{long = negative, short = NEG}

\newcommand*{\homo}[2]{#1$_{\text{#2}}$}

\newcommand{\cgel}{\href{../English/cambridge.pdf}{my notes about CGEL}}
\newcommand{\latin}{\href{../Latin/latin-notes.pdf}{my notes about Latin}}
\newcommand{\alignment}{\href{../alignment/alignment.pdf}{my notes about alignment}}
\newcommand{\exerciseone}{\href{../Exercise/2021-3.pdf}{this exercise}}
\newcommand{\method}{\href{../methodology/glossing.pdf}{this note about how descriptive grammars work}}
\newcommand{\draft}{\href{./main.pdf}{this draft}}

\newcommand{\ala}{à la}
\newcommand{\translate}[1]{`#1'}

\title{Mandarin morphosyntax reading note}
\author{Jinyuan Wu}

\begin{document}

\maketitle

\automath

This note is my reading note of \citet{zhudexigrammar}.
It can be seen as a preparation of \draft,
which is premature and possibly will never be finished,
especially by someone without systematic linguistic training like me.
Still, the theoretical orientation of this note is well introduced in the above link,
as well as in \cgel, \latin, and \method.
\citet{zhudexigrammar} is commonly referred to as a typical structuralist book of Chinese.
I do not say ``structuralist grammar'' because the book is also a textbook about structuralism,
mostly in Bloomfield's brand
and strikingly close to the \ac{cgel} \citep{cgel} approach summarized in the above notes,
with a lot of argumentation, more than what ordinary grammars contain.

\section{About Zhu's book}

\subsection{The object language}

The object language, ``Chinese'', needs some clarification.
It means Standard Modern Chinese or Standard Modern Mandarin,
often abbreviated as Mandarin in the English speaking world.
In mainland China it is called 普通话.
In Taiwan and Singapore it (with small variations) is called 国语.

Standard Mandarin -- like other languages -- is an evolving language.
Certain usages documented in Zhu's book have already been obsoleted.
% TODO: evolving speed

\subsection{Organization of chapters}\label{sec:organization-chapter}

The book can be divided into several parts:
\begin{itemize}
    \item Chapters 1-6 are about morphology and lexical categories.
    Lexical categories discussed in details are either nominal or verbal.
    \item Chapters 7-10 together give a top-down analysis of syntactic constructions without coordination.
    Serial verb constructions are \emph{not} introduced in these chapters.
    \item Chapter 11 is about coordination.
    \item Chapter 12 is about serial verb construction.
    \item Chapter 13-14 are about prepositions and adverbs.
    \item Chapter 15 is about clause types.
    \item Chapter 16 is about \ac{sfp}.
    \item Chapter 17 is about clause linking without canonical coordination.
    \item Chapter 18 is about ellipsis and inversion, 
    which may be roughly said to be about information packaging. 
\end{itemize}

This organization is an example of \citesec{\ref{method-sec:chapter-organization}} in \method.
The relation between the first six chapters and the following four 
is the item and arrangement strategy relation.
Within the chapters 7-10,
we see the top-down partition of clauses and \acs{np}s 
introduced in \citesec{\ref{method-sec:clause-top-down}}
and \citesec{\ref{method-sec:np-top-down}} in \method.
This is typical in structuralist works:
it is a direct reflection of the top-down analysis of syntactic structures 
(see \citesec{\ref{method-sec:item-and-arrangement-grammar}} and \citesec{\ref{method-sec:clause-top-down}}
in \method).

The noun-verb distinction (\citesec{\ref{method-sec:np-clause-chapter}} in \method) 
is only reflected in 
nominal categories being introduced in \citechap{4},
while verbal categories being introduced in \citechap{5}.
The \ac{np} structure is introduced in \citechap{10},
together with their clausal counterparts.

% TODO: chapter 11 and 12

The relation between the first twelve chapters and chapters 13 and 14
is the relation between canonical constructions and their counterparts with adjunction.
The relation between the first fourteen chapters and \citechap{15} is 
the relation between canonical constructions and non-canonical ones
related to the former ones by transformation rules.

Chapters 7, 8, 9, 10, 13, and 14 constitute a system 
quite similar to the chapter 4-8 in \ac{cgel}:
first clausal complements, 
including the external complement -- the subject --
and internal complements,
then \ac{np}s,
then how the two are modified,
by adjectives and adverbs,
or by more complicated \acs{pp}s.

Chapter 16 actually can be placed before \citechap{13}.
This is not the order used in the book,
the reason of which, in generative terms,
seems to be that \ac{sfp}s are merged in higher projections than what is involved 
form \citechap{7} to \citechap{15}.
Zhu, however, regard most of \ac{sfp}s as a part of the predicate.
The contradiction between the arrangement of chapters 
and the explicit analysis of \ac{sfp}s as a part of the predicate in \citesec{16.1.1} 
in \citet{zhudexigrammar}
is actually self-consistent:
the mutual relation between the predicate and the \ac{sfp}s
is parallel to the mutual relation between the verb stem and the aspectual markers:
the aspectual markers are introduced in higher functional projections than the verb stem,
in the same way \ac{sfp}s are introduced in higher functional projections than the predicate.
The verb stem and the aspectual markers being analyzed as two immediate constituents of a ``word''
reflects post-syntactic processes,
not constituency relations and dependency relations created by the syntax proper.
If this is acceptable -- which is the case in most descriptive grammars --
then since phonologically and especially from the perspective of prosody,
\ac{sfp}s are closer to the predicate,
it is of course also acceptable to place the \ac{sfp}s into the predicate.

\subsection{Terminology}\label{sec:terminology}

The terminology used in the book is closer to the \ac{cgel} approach rather than the \ac{blt} approach.
It should be noted that the book is written in Mandarin Chinese,
in which certain linguistic terms 
do not have morpheme-to-morpheme counterparts in English or
already have different meaning than their morpheme-to-morpheme counterparts in English.

To keep the rest of this note fluent,
issues of term translations are summarized in this section.

\subsubsection{Theoretical orientation}

Terminology reflects the theoretical orientation of a grammar.
The term 中心语 \translate{central speech (i.e. head)}
is translated as \term{head} here,
which is the lexical head and not the functional head.
Therefore, we have notions like noun phrase, verb phrase, etc.,
in which the head is defined as the noun, the verb, etc.
and not the determiner or the light verb.

\subsubsection{Word classes}\label{sec:word-class-term}

For lexical categories, 体词 means \translate{referential word} i.e. nominal words.
Its direct translation would be \translate{body-word},
which may be understood by some as \translate{content word} i.e. \translate{lexical word}.
The term 谓词 means \translate{verbal word}.
The direct translation would be \translate{commenting-word},
which may be understood as somehow ``predicative'' in the sense of predicative complements in \ac{cgel}.
This is not correct: 
谓词 means what can head a predicate,
thus verbs and adjectives in Chinese.
The term 实词 \translate{substantial word} means lexical words,
while 虚词 \translate{virtual word} means function words.

\subsubsection{Clause structure}

The term 谓语 means predicate in the \ac{cgel} sense.
The term 述语 means predicator in the \ac{cgel} sense.
Unlike earlier structuralist works which work in the vanilla phrase structure grammar (PSG) framework,
\citet{zhudexigrammar} uses a \ac{cgel}-like PSG,
where a label of a constituent in a larger construction 
contains both its category label and its function label,
for example both ``\ac{np}'' and ``subject''.
This idea is made explicit in \citesec{1.3.10}.
The analysis of 我们班有许多外国留学生 in \citesec{1.3.8} is a good example.
Unlike \ac{cgel},
\citet{zhudexigrammar} uses a more compact format 
in which constituents are illustrated by underlining 
to show the constituency tree.
This is, of course, merely a notational problem,
but somehow it becomes a tradition of the School Grammar analysis of Chinese.

\subsection{About this note}

This note try to rearrange the content of \citet{zhudexigrammar}
in a way that is both acceptable in the approaches in \ac{blt} and \ac{cgel}.
The order of this note is largely bottom-up.
Certain top-down analyses, of course, will be given in the grammar sketch chapter.
When obsolete usages appear, I will point them out.
When the analysis is problematic, I will discuss why it is problematic and how it can be improved.

\section{A grammar sketch}

The first chapter in \citet{zhudexigrammar} may be thought as a grammar sketch chapter,
but it contains much discussion on theoretical issues
(replicating what is discussed in \ac{cgel} \citesec{1.4}).
This section is a more compact grammar sketch,
skipping theoretical commitments which can be found in sources at the beginning of this note.
Chapter 3 is also a short one and may be regarded as a part of the grammar sketch.

I will roughly follow \citet{jacques2021grammar} in the organization of this section.
However, since in Chinese, dependency relations are not mainly coded by morphology,
I will replace the ``nominal morphology'' section by ``noun phrase''
and replace the ``verbal morphology'' section by ``clause structure'',
and do not give constituent order a special section,
since constituent order is covered by the constituency structure.
This is a major difference between \ac{cgel}-like ``structuralist'' grammars 
and \ac{blt}-based ``functionalist'' grammars
(\citesec{\ref{method-sec:theory}} in \method).

\subsection{Parts of speech}

Since Chinese does not rich grammatical relation-bearing morphology,
purely syntactic tests play the major role in determining parts of speech.
Semantics may help but is never decisive 
(\citesec{3.1.1} and \citesec{3.1.2} in \citet{zhudexigrammar}).
The word class division given in the book inevitably meets the problem 
that a word may belong to two categories depending on the context.%
\footnote{
    Formally, we may say the word prototypically belong to one category,
    and its usage as a word in another category
    involves zero derivation or conversion.
    From a Distributed Morphology perspective,
    however, we can also say that the stem of that word can be merged with 
    two categorizers, 
    and here we are faced with the same problem that 
    urged linguists to give up transformational rules.

    The most appropriate term for this process -- 
    zero-derivation, conversion or something else -- is still debated,
    and I will skip this topic in this note.
}
In the analysis adopted here, words belonging to two categories are only the minority,
because otherwise, the two categories can be considered as one 
(\citesec{3.2, 3.3} in \citet{zhudexigrammar}).

\subsubsection{Lexical words}

Lexical words in Chinese can be roughly divided into nominal ones and verbal ones,
or in the Chinese terms, 体词 and 谓词
(for issues on translation between English and Chinese terms, see \prettyref{sec:word-class-term}).
The prototypical role of nominal words 
is to fill argument slots (or to be more precise, to head a phrase that fills an argument slot).
Nominal words rarely appear in the predicator position
(though for stylistic purposes, they sometimes do).
Verbal words prototypically fill argument slots,
but many of them -- and clauses without any morphological marking -- 
can regularly appear in argument slots \citep[\citesec{3.5}]{zhudexigrammar}.

The fact that verbal categories can fill argument slots or in colloquial words ``be used as nouns''
urges some to put the verbal categories under the nominal categories,
so thus there is only one mega lexical category in Chinese:
the nominal category or the Noun.
The analysis adopted here does not aim to organize lexical categories 
in a binary branching classification tree,
so the ordinary nominal-verbal distinction is maintained.

Whether Chinese has a separate adjective category 
has been debated for decades.
Based on a line of reasoning similar to the above verb-as-noun analysis,
some linguists argue that the so-called adjectives should be put under the verb category,
since they can fill the predicator slot without any morphological marking \citep{li1989mandarin}.
Since verbs and most alleged adjectives show different morphological behaviors in duplication, % TODO: ref
the verb-adjective distinction is kept,
and the two are placed under the verbal category.

There still exist a (much smaller) number of alleged adjectives that shows 
different morphosyntactic properties with the adjectives in the verbal category.
They can be marginally used as heads of \ac{np}s,
while they do not have duplication variants.
These ``adjectives'' are thus placed under the nominal category.
Thus we have two types of adjectives
In \citet{zhudexigrammar}, 
nominal adjectives are called 区别词 \translate{distinction word},
while verbal adjectives are called 形容词 \translate{adjective}.

There are more nominal categories than the ordinary noun category and the nominal adjective category.
Numerals, for examples, are in another nominal category.
Chinese has a rich classifier system,
and most classifiers still have strong nominal properties
and thus they constitute yet another nominal category.
\citet{zhudexigrammar} calls them 量词 \translate{measure word},
because many classifiers have the meaning of ``unit''.
There is also a location word class, including 里 in 在房子里,
which is sometimes said to be the postposition class.

\subsubsection{Function words}\label{sec:function-word-introduction}

Unlike the case in English or Latin 
(see \citesec{\ref{latin-sec:particle-relation}} in \latin), 
in Chinese, there is no synchronic or diachronic ways 
to regularly form adverbs from fossilized phrases 
or from adjectives via derivations 
which can be seen as forming a peripheral argument 
with the meaning of ``in the manner of \dots''.
Thus what can be uncontroversially called adverbs in Chinese 
form only a small category,
which is placed as one type of function words in \citet{zhudexigrammar}.
% TODO: 狠狠地这种算状语,但是是副词吗?

So-called Chinese prepositions are all historically verbs.
The distribution of so-called preposition phrases % TODO:把字句中“把”不是介词,是轻动词
is also highly restricted,
rendering people to ask whether they are constituents at all.
Despite \citet{zhudexigrammar} calls them 介词 \translate{adposition},
these words are better regarded as introduced in serial verb constructions
(\prettyref{sec:serial-verb-construction-intro}),
instead of English-like and Latin-like peripheral argument slots.
Thus, in this note, I call these ``prepositions'' \term{coverbs},
following the terminology in \citet{po2015chinese}.
Certain words, like 把 in the disposal construction (\prettyref{sec:disposal}), 
are traditionally analyzed as prepositions,
but a deeper look challenges this view.
Since the term \term{coverb} does not entail 
a preposition-like behavior (taking one and only one complement, projecting into a preposition phrase, etc.),
these words -- 把, 被, 给, etc. -- 
are also named coverbs in this note.

Another group of function words in Chinese is the \ac{sfp}.
They are named 语气词 \translate{specch force word} in \citet{zhudexigrammar},
revealing the fact that they are about in the Force projection(s)
-- though \citet{zhudexigrammar} somehow insists on them being a part of the predicate
(\prettyref{sec:organization-chapter}, TODO: more ref).

Here is a note of mine: 
\citep[\citesec{3.6}]{zhudexigrammar} classifies 
certain categories like location words % TODO: 方位词的正确翻译???
into the nominal category and hence the lexical one,
while the location word category can definitely be enumerated \citep[\citesec{4.4}]{zhudexigrammar}.
On the other hand, 
the author claims that lexical categories are always open 
and function categories are always close \citet[\citesec{3.4}]{zhudexigrammar}.
A conflict thus occurs.
So we need to closely investigate what the author really means by the term \term{lexical word}.
\citesec{3.4} starts with 
``\dots lexical words may fill the subject, object and predicate positions,
while function words cannot \dots''.
Thus the meaning of \term{lexical word} 
is not far from ``being able to appear in utterance independently''.
Then, whether a category is open and whether a category is lexical 
are largely orthogonal 
(though open function categories are rare): % TODO: ref
the Japanese verbal adjective category is lexical but close.

\subsubsection{Overview of all categories}

The comprehensive classification of parts of speech can be found in \citet[\citesec{3.6}]{zhudexigrammar}.
Two categories that are neither lexical nor function 
are the ideophone class and the interjection class.
% TODO: the standard of open and closed


\subsection{Nominal categories, morphology, and the \ac{np}}

\subsubsection{The \ac{np} template}

No morphological case, number, and gender categories are attested in Chinese.
There is a word class system or in other words classifier system, however.
In most cases when a numeral appears in a \ac{np},
a classifier follows immediately after the numeral.
Attributives -- both adjectives and relative clauses -- 
follow the classifier. % TODO: 红色的三个,这样的说法说得通吗?
The demonstrative, if any, appears before the numeral,
and even when there is no numeral,
there is frequently also a classifier.

The template of \ac{np}s, therefore, can be summarized as 
demonstrative--numeral--classifier--attributive(s)--head noun.

\subsection{The verb and the clause}

\subsubsection{The verb}



\subsubsection{The subject in a clause}

Though completely lacking case morphology,
Chinese is a typical syntactically accusative language.
% TODO: subject and topic
The structuralist binary branching works well for Chinese clausal structure 
\citep[\citesec{133-136}]{zhudexigrammar}.
A clause without preposing -- henceforth called a nucleus clause --
can be divided into a subject and a predicate,
plus possible \ac{sfp}s.
The subject is on the left, and the predicate is on the right, followed by \ac{sfp}s.
The predicate may be a 
single verbal word (its function is the predicator) plus possible internal complements, 
possibly modified by adverbs,
and in this case we say the predicate is filled by a verb phrase.
The predicate may also be 

\subsubsection{Verb complementation patterns}

Clausal complements inside the predicate are said to be internal.
Internal complements of the verb include objects and non-argument complements
\citep[1.3.3-1.3.4]{zhudexigrammar},
the latter being called 补语 \translate{complementing speech} in \citet{zhudexigrammar}.
The term 补语 is frequently translated into \term{complement} in English,
but then it conflicts with the wider definition of complements in \ac{cgel},
which includes both arguments and 补语,
and such confusion occurs,
I use the term \term{non-argument complement}.

We are sure that non-argument complements are not arguments,
because they cannot be filled by nominal constituents.
They are indeed complements, if not parts of verb compounding constructions,
for reasons given in \citesec{\ref{draft-sec:complement-name}} in \draft.

Non-argument complements and objects have complicated interplays,
and the boundary of non-argument complements is not always clear.
Many non-argument complement types are mutually exclusive.
The constituent order between some non-argument complements and the object(s) is rigid,
while for other non-argument complements it is more flexible.
Certain non-argument complement constructions are almost examples of verb compounding,
and the so-called complements may be analyzed as a part of the verb complex.
Certain non-argument complements are almost objects.
% TODO: ref, indirect complement

\subsubsection{Serial verb constructions}\label{sec:serial-verb-construction-intro}

Chinese has rich serial verb constructions,
in which the predicate contains more than one main verbs
or possibly a main verb and one or more verbal adjectives and coverbs.
The distinction between serial verb constructions and some non-argument complement constructions 
is highly blurred.

\subsubsection{Sentence-final particles}

\ac{sfp}s are actually clause-final particles,
because they can appear in subordinate clauses,
but since this is the standard term I will not alter it.
They appear strictly at the end of nucleus clauses.
Postposing to the right of \ac{sfp}s is rare, if possible.

\subsubsection{Unattested constructions and categories}

\subsection{Negation}

Chinese does not have a versatile negator.
Negation in 

\subsection{Coordination, clause linking and supplementation}

Coordination occurs in all levels of Chinese syntax:
\ac{np}s, predicates, and clauses.
In these constructions different coordination devices are used.

\subsection{Subordination}

Like all 

\subsection{Typological information and remarkable features}

\subsubsection{Alignment and the topic-comment construction}


\subsubsection{Morphological typology}

Chinese does not have inflection at all,
except the aspectual markers,
which may be argued to be agglutinating suffixes (\prettyref{sec:aspectual}).
The rest of morphology is all derivational.
Morphological devices attested include 
duplication, compounding and affixation.
No internal change, infix or circumfix is attested.

What is the proper definition of \term{words} in Chinese 
is a topic surrounded by lots of debate. % TODO: ref

The conclusion is Chinese is basically an analytic language,
but not among the most analytic ones.

\subsubsection{Prosody and styles}

The grammar of Chinese is especially remarkable in its heavy reliance on prosody and style.
Violation of relevant conditions is not only not recommended,
but sometimes causes grammatical error.

\section{Overview of morphology}

Before starting discussion on more specific topics, 
a brief introduction to the morphology of Chinese is a good idea.
Since the preferred writing system of Chinese,
the Chinese character 汉字 system,
roughly represents the morphemes in Chinese,
it will be introduced first (\prettyref{sec:morpheme-character}).
Then follows long and tedious discussions on what is a word in Chinese.
Grammatical words are discussed (\prettyref{sec:grammatical-word}),
the word-phrase distinction 
and morphological devices on them (\prettyref{sec:morphological-device})
being introduced.
The prosody structure 
-- which, quite unlike many other languages, plays a key role in Chinese grammar and not just phonology --
is introduced as the background of the definition of phonological words (\prettyref{sec:prosody-structure}).
\prettyref{sec:prosodic-word-syntax} 
highlights the significance of prosodic words as building blocks in Chinese.

\subsection{Morphemes and Chinese characters}\label{sec:morpheme-character}

Most Chinese morphemes are monosyllabic. 
There are exceptions, though, most of which are historically or contemporarily borrowed ones
or ideophones.
Examples include 
葡萄 \translate{grape},
巧克力 \translate{chocolate},
摩登 \translate{modern}.
This fact means the preferred writing system -- also the one used in this note --
is Chinese characters,
in which one character corresponds to one syllable
and roughly one morpheme.

Putting some quirky cases aside,
Chinese characters are often good indicators of morphemes.
There are, for example, at least seven morphemes sounding \corpus{xi\={a}n},
and there happens to be seven Chinese characters corresponding to each of them:
仙, 先, 籼, 掀, 锨, 鲜, and 纤.

Like all writing systems, 
Chinese characters do not completely faithfully represent 
the underlying linguistic structure.
Some characters do not mean anything -- 
they are simply the designated characters representing syllables in certain words.
The character 萄 as in 葡萄, for example, 
means nothing more than the syllable \corpus{t\'{a}o},
but it only appears in the morpheme 葡萄 and 葡萄牙 \translate{Portuguese}.
The same is for the character 葡.
Some characters have regular morpheme meanings
but also have merely phonetic meaning in certain words.
The character 登 in 摩登 regularly means \translate{climb},
but in the word 摩登, only its phonetic value \corpus{d\={e}ng} is preserved.
Certain morphemes can denoted by more than one character.
The \ac{sfp} \corpus{ba} can be written as 吧 or 罢,
the latter hinting its etymology but is now rarely used.
Certain characters denote more than one morpheme.
The character 会 may mean \translate{conference} or \translate{be able to do}. 

Thus, Chinese characters provide clues on what is a morpheme,
but they are not decisive \citep[1.1.4]{zhudexigrammar}.

A question causing endless controversy and confusion 
is ``what is a word''. 
\ac{blt} spends a whole chapter (\citechap{10}) on this topic,
and \citesec{\ref{method-sec:what-is-word}} in \method{}
is a brief summary. 
It is often said that Chinese is ``character-based''
or to be precise, ``monosyllabic morpheme-based'',
with no level of grammatical words.
This claim is factually flawed, 
since in Chinese, there \term{are} distinction between 
productive morphemes and words.
What should be noted are the split between phonological words and grammatical words % TODO: ref
and the subtleties concerning word-phrase distinction. % TODO: 类似于幽了他一默这种形态变化导致词变成短语
These are introduced in the following sections.

\subsection{Grammatical words}\label{sec:grammatical-word}

\subsubsection{Chinese has grammatical words}

The first piece of evidence for the existence of grammatical words in Chinese is 
there are disyllabic units in Chinese 
that have conventionalized meanings and its inner structure is invisible 
to any other morphosyntactic rules (except prosody).
The unit 白菜 is made up by two perfectly productive morphemes:
白 \translate{white} and 菜 \translate{vegetable},
but its meaning is not the composition of the two morphemes:
白菜 means \translate{Chinese cabbage}, not \translate{any vegetable with whitish appearance}.
The word has already gained a conventionalized meaning,
and its inner structure is of mostly diachronic interest but not synchronic interest.
Therefore, the disyllabic unit 白菜 is the smallest unit fed into morphosyntax,
and it of course is not a phrase.

Those insisting on the nonexistence of words in Chinese 
may explain the observation made above 
by claiming 白菜 to be an idiom \ac{np}:
it is indeed a lexical entry,
but is regarded as a pre-compiled phrase.
It then should be noted that 
certain grammatical relations seem to be not a part of \ac{np}s and clauses,
highlighting the necessity to introduce a smaller level of constituency.
Consider the following examples:
\begin{exe}
    \ex 
    \begin{xlist}
        \ex\label{ex:nominal-modifier-1} {} [[定义]_{\text{modifier:N}} [[等价]_{\text{complement:adjective}} [性]_{\text{nominalizer}}]_{\text{N}}]_{\text{N}} \translate{equivalence of definitions}
        \ex\label{ex:nominal-modifier-2} {} [[美国]_{\text{modifier:N}} [苹果]_{\text{head:N}}]_{\text{N}}
        \ex\label{ex:meiguo-hongse-pinguo} *[美国]_{\text{modifier:N}} [红色的苹果]_{\text{NP}}
        \ex\label{ex:hongse-meiguo-pinguo} 红色的美国苹果
    \end{xlist}
\end{exe}
From \eqref{ex:nominal-modifier-1} and \eqref{ex:nominal-modifier-2},
it can be seen in certain morphosyntax units,
a bare noun may serve as a (restrictive) modifier.
The constituent of Chinese \ac{np}s is Dem Num A N,
and this bare noun modifier position seems to be more internal than the adjective position,
as is illustrated by \eqref{ex:meiguo-hongse-pinguo} and \eqref{ex:hongse-meiguo-pinguo}.
Furthermore, the bare noun position cannot be filled by a \ac{np}.
The following examples demonstrate this:
\begin{exe}
    \ex \begin{xlist}
        \ex {} [联合国] [秘书长]
        \ex *[[某个组织] [秘书长]]
        \ex 某个组织的秘书长
    \end{xlist}
\end{exe}
The obligatoriness of 的 means the NP 不知道什么组织 can only appear as a modifier via the possessive construction.
It cannot fill the slot of 美国 in 美国苹果.
So the bare noun modifier position is a function label existing in a unit smaller than the \ac{np}
-- and it has to be the word.

Therefore, the term \term{grammatical word} is a useful descriptive concept when it comes to Chinese.

\subsubsection{The blurred line between words and phrases}\label{sec:blur-line-word-phrase}

That is not to say there are no real problems concerning what is a word in Chinese.
In certain scenarios no clear distinction between morphemes and words
-- and hence the distinction between words and phrases --
does seem available.

One peculiar feature of Chinese is 
grammatical relations in morphology often have syntactic counterparts,
with the same constituent order.
Verbs, for example, may have inner predicator-object structures.
The verb 关心 \translate{care for} is certainly analyzable 
as a predicator-object structure,
but it takes objects just like any other verbs:
\begin{exe}
    \ex 他 [[[关]_{\text{predicator:V}} [心]_{\text{object:N}}]_{\text{predicator:V}} [自己的家人]_{\text{object:NP}}]_{\text{predicate:VP}}
\end{exe}
This means 关心 is not a VP but a grammatical word,
or otherwise it is impossible to take another object since there is no valency changing device in use.
The verb 关心 itself poses no threat to the word-phrase distinction,
but certain examples of these ``words with internal syntax'' can be tested to be words,
while they indeed have phrasal counterparts.
Compare the two examples below:
\begin{exe}
    \ex 
    \begin{xlist}
        \ex\label{ex:nianfo-tang} {} [念佛] 堂 
        \ex\label{ex:nianfo-split} 老太太 [念了这么久佛]_{\text{VP1}},却不知道自己在[念哪一尊佛]_{\text{VP2}}
    \end{xlist}
\end{exe}
The verb 念佛 in the first example is similar to 美国 in \eqref{ex:nominal-modifier-2}:
it serves as a bare modifier 
(since in Chinese verbal constituents can fill argument slots directly,
the fact that 念佛 is a verb is not surprising).
The fact 念佛 is able to appear in such a position assures us that 
it is a word.
Then consider \eqref{ex:nianfo-split}.
In VP1, a temporal semi-object is injected between the verb 念 and the object 佛,
while in VP2, an interrogative phrase 哪一尊 is inserted into 佛 
and a \ac{np} object is now taken by the verb 念.
So here is the problem:
what if 念佛 is \emph{always} a VP,
and what \eqref{ex:nianfo-tang} demonstrates is 
the bare noun (or verb) modifier may be sometimes filled by a phrase?

One solution -- the solution taken by \citet[\citesec{1.2.6}]{zhudexigrammar} -- 
is to regard 念佛 in \eqref{ex:nianfo-tang} as a word,
while the two \acs{vp}s in \eqref{ex:nianfo-split} as phrases.
念佛 as in 老太太经常念佛 can be interpreted as a word or as a phrase
without making any difference.
念佛 as a word is something like \corpus{Buddha-praying},
while 念佛 as a verb is something like \corpus{pray to Buddha}.
In the account of \citet[\citepage{82}]{feng2000},
念佛 is a morphosyntactic word,%
\footnote{
    The original term is a 句法词 \translate{syntactic word},
    which of course works well in generative syntax,
    but in a surface-oriented grammar,
    the verb version of 念佛 may be said as created by morphological rules,
    not syntactic rules.
}
which is created by morphosyntactic rules 
and has a inner structure that is (partially) transparent 
for other morphosyntactic rules,
while 关心 is a \translate{lexical word},
which is taken out of the lexicon directly.%
\footnote{
    Those insisting on a universal word-phrase distinction 
    may say ``it is assembled in the lexicon before syntax''.
    The position of mine is if something is assembled synchronically,
    then it has to have something to do with syntax:
    syntax is the only productive engine.
    If a morphological rule is completely invisible to the rest of the grammar,
    it is likely to have lost productivity and becomes historical.
}

Another place where the distinction between words and phrases are subtle is 
the non-argument complement construction.
What is the status of 爬上 in 他笨手笨脚地爬上信号塔?
A word (created by a productive verb compounding rule), 
a phrase (a verb-complement structure),
or just a word sequence without structural significance?
% TODO

\subsubsection{Splitting of a word}\label{sec:splitting}

What makes things more puzzling is even words \emph{without} synchronically morphosyntactic internal structures 
can sometimes be split and extended
with phrasal dependents injected,
though not everyone will accept such usages.
\citet[\citesec{6.5.8}]{chao1965grammar} records the first two non-standard examples of the phenomenon
in the follows,
while the third example is more widely accepted:
\begin{exe}
    \ex\label{ex:word-splitting} \begin{xlist}
        \ex\label{ex:junwanlexun} \% [军完了训] 以后才可以去请护照
        \ex\label{ex:youmo} \% 还 [幽了他一默]
        \ex\label{ex:guanshenmexin} 这件事情你 [关什么心] 啊
    \end{xlist}
\end{exe}
The bracketed constituents in \eqref{ex:youmo} and \eqref{ex:guanshenmexin} 
are both uncontroversially \acs{vp}s:
\eqref{ex:youmo} contains the object 他 and a temporal semi-object 一, % TODO: ref
while in \eqref{ex:guanshenmexin}, 什么心 as a \ac{np} is the object.
The splitting of the verbs clearly origins 
by analogy with \ac{vp}s containing morphosyntactic words 
The bracketed constituent in \eqref{ex:junwanlexun} is equivalent to 军训完了,
the latter being a verb complex, % TODO: ref
but 军完了训 is apparently created by analogy with 
[[吃完了]_{\text{predicator:verb complex}} [饭]_{\text{object:N}}]_{\text{VP}}.
Anyway, whether 军完了训 is directly a \ac{vp} or is first a verb complex 
and then a \ac{vp} is not of much importance:
similar construction never when an object is present.
What can be seen from \eqref{ex:word-splitting}, then, 
is uncontroversial grammatical words can also be split and extended into phrases 
in the same way phrase-like words like 念佛 do in \prettyref{sec:blur-line-word-phrase}.

The motivation of this phenomenon seems to be prosody: 
splitting words into phrases is only observed in \ac{vp}s,
and \ac{vp}s are subject to the prosodic constraint 
that the neither the verb nor the final complement can be too light.
Splitting the verb may help to reduce the ``weight'' of the verb 
so the resulting utterance meets the prosodic constraint better.

\subsection{Morphological devices}\label{sec:morphological-device}

A word created by morphological devices 
may be understood compositionally from its morphemes,
or it may be fossilized with a fixed, conventionalized meaning 
and syntactic function.




\subsubsection{Duplication}

Nouns (\prettyref{sec:duplication-noun}), 
adjectives, and verbs may undergo duplication.

\subsubsection{Compounding}

\subsubsection{Nonconcatenative morphology}

There is little nonconcatenative morphology in modern Chinese.
Certain formulae inherited from Classical Chinese, however, 
may be marginally regarded as nonconcatenative morphology,
since the relevant syntactic rules have largely eroded.
唯命是从, in Old Chinese, is perfectly analyzable:
是 is a marker of object fronting, 
and 唯 roughly means \translate{only}.
The syntax of object fronting marked by 是 is completely gone in modern Chinese,
and  唯 only appears in words like 唯有 and never occurs freely.
The 唯\dots是\dots words, as is listed in \eqref{ex:wei-shi},
may be analyzed as a highly limited template construction,
with 唯\dots是\dots being a template and 
the object being filled into the first slot 
and the verb being filled into the second slot.
\begin{exe}
    \ex\label{ex:wei-shi} \begin{xlist}
        \ex 唯命是从~从命
        \ex 唯你是问~问你
        \ex 唯利是图~图利
        \ex 唯才是举~举才
    \end{xlist}
\end{exe}

\subsection{Prosody words}

\subsubsection{The Chinese prosody structure}\label{sec:prosody-structure}

By paying attention to stops in Chinese utterances,
it can be found that phonological words exist and they are mostly defined by the prosody structure.
In the rest of this note,
the term \term{prosodic word} and \term{phonological word}
will be used interchangeably.
The prosody structure is about how stress is assigned to phonological constituents.
Assigning a prosodic structure is like condensation and clustering:
something is merged with something adjacent,
and the result is merged with something adjacent else.
When two phonological constituents are merged together,
one of them is considered heavier than the other.
If heaviness is to have a simple relation with the length of a phonological constituent,
then usually the more a phonological constituent is,
the heavier it is.
This is consistent with the condensation picture of prosodic segmentation.
Suppose a prosodic constituent attracts a syllable and merges with it.
The latter is not an independent phonological constituent
and cannot be heavy,
so the former is the heavier one and the latter is the lighter one in the larger prosodic constituent.

The smallest unit of prosody structure 
is a prosodic word.
The simplest prosodic word is the disyllabic foot, 
which contains two adjacent syllables in the case of Chinese.
(It can be made by two moras in other languages.)
One is assigned stress and is therefore heavier than the other.
Trisyllabic prosodic words also exist in Chinese,
most of which are borrowed words (e.g. 加拿大 \translate{Canada})
or words formed by coordinating three morphemes (e.g. 数理化 \translate{math, physics, and chemistry}).
They can also be regarded as foots \citet[\citesec{2.2}]{feng2000}.

Prosody is able to see the constituency structure.
Some prosodic rules pertaining to the constituency tree 
guide and limit the assignment of relative heaviness and lightness.
In Chinese, prosodic segmentation is done strictly left-to-right 
in each \ac{np},
and then the \ac{np}s together with verbal constituents 
are used as the input of prosodic segmentation of clauses.
Then there are rules ruling out certain utterances.
The most important rule of this kind in Chinese is
the main verb and one post-verbal constituent
should form the last prosodic constituent in the clause (ignoring \ac{sfp}s),
and when there is no post-verbal constituents,
the main verb receives the natural stress. 
This is actually a rather strong condition.
Certain constituents -- most functional words -- are unable to accept stress at all.
They may be freely merged into the closest prosodic constituents.
Certain constituents -- like \ac{np}s with definite references -- are by default stressed.
When the verb is the last lexical constituent
(as in *他挥动棒子把我打),
it should not be too short or otherwise it is unable to receive stress.
When there are more than one post-verbal constituents,
only one of them can receive stress.
If two post-verbal constituents are both by default stressed,
the sentence is again ruled out. % TODO: ref

\subsubsection{Prosodic words in morphosyntax}\label{sec:prosodic-word-syntax}

Sometimes the significance of prosodic words are primarily phonological.
In 副总经理,
we have two prosodic words,
副总 and 经理,
while the morphological structure of the word is [[副] [总 [经理]]].
It is often the case that 
prosodic partition of a long grammatical word 
does not respect morpheme boundaries.
This is similar to the case in English and Latin poems,
where the prosody arrangement of sentences does not have to respect word boundaries:
\corpus{arma vi\textbar rumque ca\textbar no}.

A large portion of prosodic words, however, enjoy the status of building blocks in Chinese.
The disyllabic verbs 念佛, 军训, 体操 etc. 
in \prettyref{sec:blur-line-word-phrase} and \prettyref{sec:splitting}
are all prosodic words.

\subsection{Summary: the structure of Chinese lexicon}



\section{Parts of speech}

\subsection{Nouns}

There are two defining properties of the noun class:
being able to be modified by a numeral-classifier construction (\prettyref{sec:possession}),
and being unable to be modified by adverbs. % TODO: ref
A word with both of the properties is definitely a noun.
Certain verbal words also appear with numerals and classifiers 
(which may be viewed as zero-derivation into abstract nouns),
but they can always be modified by adverbs,
so they themselves are not nouns \citep[\citesec{4.1.1}]{zhudexigrammar}.

Nouns may be classified according to 
their classes and countability (\prettyref{sec:noun-class-classifier}),
their behaviors in possession (\prettyref{sec:possession}).

\subsubsection{Duplication of nouns}\label{sec:duplication-noun}

Duplication of nouns is mainly restricted to kinship terms,
like 爷爷 \translate{grandpa}, 奶奶 \translate{gradma}, 爸爸, 妈妈.

\subsection{Classifiers}\label{sec:classifiers}

There are roughly seven types of classifiers. % TODO: 翻译

\subsection{Verbs}

\subsection{Verbal adjectives}


\section{Noun phrases}

\subsection{Noun class and the classifier}\label{sec:noun-class-classifier}

Possible classifier in a \ac{np} headed by a noun 
gives the noun class of that noun.
Roughly there are five classes \citep[\citesec{4.1.2}]{zhudexigrammar}:
\begin{itemize} % TODO: ref to sec:classifiers
    \item Countable nouns, whose classifiers themselves denote to discrete objects.
    \item Uncountable nouns, whose classifiers are 
\end{itemize}
Each class has lots of subclasses.

\subsubsection{Numeral}\label{sec:numeral-classifier}

\subsection{The possessive construction}\label{sec:possession}

% TODO: “我爸”这样明显是词法的动词应该放在这里吗?

\subsection{Relative order of noun phrase dependents}

\section{Verbal morphology}

\subsection{The verb complex}

\subsection{Aspectual markers}\label{sec:aspectual}

A separate section has to be devoted to 了, 着, and 过,
because they code the aspectual system in Chinese.
I say \term{aspectual}, not \term{aspect},
partly because there are so many well-accepted usage of the term \term{aspect},
partly because whether the Chinese aspectual system can be safely said to be one of them is still controversial.

\section{Verb types, argument structures and clausal dependents}\label{sec:verb-type-canonical}

% TODO: serial verb construction也和verb type有关,要不要放在这里?

Like the corresponding chapter in \ac{cgel} (\citechap{4}),
this section is mainly about canonical clauses.
Here ``canonical'' means
the clause contains only one main verb 
-- which, as mentioned before, is designated as the predicator -- 
and the clausal complements transparently displays the argument structure:
we only have 王冕经历了父亲的过世, and not 王冕死了父亲.
Non-canonical constructions appear only for making argumentation for a complement type
(\citesec{\ref{method-sec:why-argumentation-function-label}} in \method).

Needless to say, non-canonical usages may be fossilized
and become canonical, 
as in Old Chinese 示,
which is likely to be a fossilized causative construction as in
蔺相如示秦王壁~蔺相如使秦王视璧.
% TODO: ref: 和non-canonical的章节的链接

The complement configuration of a canonical clause headed by a verb 
is the main factor of verb subcategorization.

\subsection{Subject and subjecthood}\label{sec:subject}


\subsubsection{Distinction between subject and topic}\label{sec:topic-subject-distinction}

The constituency tree of a subject-predicate construction 
and a topic-comment construction as in topicalization
is exactly the same,
if function labels are ignored.
This is common among world languages.
In the below two examples, 
\eqref{ex:wo-xihuan-kan-tade-xiaoshuo} is obviously a subject-predicate construction,
while \eqref{ex:xiaoshuo-wo-xihuan-kan-tade} is obviously a topic-comment construction:
\begin{exe}
    \ex\label{ex:wo-xihuan-kan-tade-xiaoshuo} {} [我]_{\text{subject:pronoun}} [喜欢看他写的小说]_{\text{predicate:VP}}
    \ex\label{ex:xiaoshuo-wo-xihuan-kan-tade} {} [小说]_{\text{topic:pronoun}} [我喜欢看他写的]_{\text{comment}}
\end{exe}
The second example obviously is dual to the first subject-predicate construction:
the object of \eqref{ex:wo-xihuan-kan-tade-xiaoshuo} is preposed and hence topicalized.

The subject, if well-defined, is always topic-like,
for it is at a high position compared to other arguments in the clause
and is subject to multiple extractions.
Indeed, typological studies often say the subject is 
something that is both an agentive position and a topic.
Here the term \term{topic} means anything that is relatively ``high''
and is subject to A'-extractions.
The absolutive argument in syntactically ergative languages, for example, 
is also the topic in this sense. % TODO: split pivot analysis
The problem is whether topic in the narrow sense -- as in \eqref{ex:xiaoshuo-wo-xihuan-kan-tade} -- 
and subject are in fact truly one grammatical relation.

In Chinese there is no finiteness category,
or at least there is no strong evidence for a finite-nonfinite distinction \citep{no-finite}.
Therefore, the definition of subject as 
what typically vanishes in nonfinite clauses 
is inviable in Chinese.

Another way to distinguish between subject and topic 
is transformational:
if a clause can be seen as a transformed version of an uncontroversial subject-predicate clause,
and the external topic-like position corresponds to a gap in what follows it,
then the external topic-like position is a topic.
Otherwise it is a subject.
This criterion is exemplified by contrasting 
\eqref{ex:wo-xihuan-kan-tade-xiaoshuo} and \eqref{ex:xiaoshuo-wo-xihuan-kan-tade}:
in \eqref{ex:xiaoshuo-wo-xihuan-kan-tade} 小说 seems to be moved from its base position after 写的,
and hence \eqref{ex:xiaoshuo-wo-xihuan-kan-tade} is a topic-comment clause,
where the initial topic-like 小说 is indeed a topic.
On the other hand, \eqref{ex:wo-xihuan-kan-tade-xiaoshuo} cannot be obtained 
from transformation of another canonical clause,
so the topic-like 小说 is a subject, not a topic.

The problems with this analysis are twofold.
First, there are mechanisms other than topicalization 
that causes fronting of an inner argument,
so if a clause can be seen as a transformed version of a canonical clause,
it is possible that the transformation relevant is a valency changing device and not topicalization.
There is no inflection marking on the verb about valency changing in Chinese.
Therefore, whether preposing means topicalization or valency changing
-- or even the question whether valency changing exists outside the 被-construction or similar constructions --
cannot be settled.

Another problem with the topic-as-moved-argument analysis is cross-linguistically,
topic can be base-generated 
so it is possible for a clause without a canonical correspondence 
to be a topic-comment one.
The lack of inflectional morphology -- this times the case system -- in Chinese
again blocks our research.
(Japanese, on the other hand, has \ac{np}-final case particles,
and thus trivially it can be found that 
Japanese has both base-generated topics and preposed topics,
the latter being identical to scrambling of an internal argument in constituent order,
with the only difference being changing the case particle into the topic particle.) 
What we find here is the topic-subject contrast 
and the movement v.s. base generation contrast,
though having certain correlation,
have no categorical implicational relations.

The famous 王冕死了父亲 problem is a good demonstration of the problem.
It means \translate{Wangmian's father died}
with a seemingly inharmonic constituent order with the meaning.
Some people, in surface-oriented terms, analyze it as a complex topic and focus,
where 王冕 is a base-generated topic:
\begin{exe}
    \ex {} [王冕]_{\text{base-generated topic:proper noun}} [ ---_{i} 死了 [父亲]_{i, \text{final focus:NP}}]_{\text{predicate}}
\end{exe}
Others, however, analyze the structure as an affected construction,
which is psychologically passive but not syntactically so,
with 王冕 being the ``affected'' argument and 
the main verb being fronted to merge with the ``affect'' light verb:
\begin{exe}
    \ex {} [王冕]_{\text{subject (affected):proper noun}} [[死了]_{i, \text{affecting verb}} [父亲 ---_{i}]_{\text{VP}}]
\end{exe}
The two analyses all seem reasonable,
but they are radically different.

What can be concluded here is there is no easy way to tell subjects from topics.
More generally, there is no easy way to distinguish the nucleus clause 
(\citesec{\ref{method-sec:complement-type-necessary}} in \method).

The simplest position is to identify subject with topic.
That is, to assume there is no syntactic divergence 
between the initial NP position in  
\eqref{ex:wo-xihuan-kan-tade-xiaoshuo} and \eqref{ex:xiaoshuo-wo-xihuan-kan-tade}.
This is indeed the position taken in \citep[7.1.3]{zhudexigrammar},
arguably strongly influenced by the constituency-only structuralist stance,
where syntactic functions are labeled by looking at the surface-oriented constituency tree only
and the notion of subject is defined purely in terms of being somehow higher than the rest of the clause.
There are still discrepancies among this approach:
some hold that the relation between the subject and the predicate 
-- regardless of their kinds -- 
is always about argument structures,
and hence the term \term{subject} is appropriate and there is no need to mention \term{topic}.
Others hold an opposite view,
arguing that in Chinese the argument structure is not coded 
and what is coded ins the information structure,
and hence we have \term{topic} only and no need for the term \term{subject}.
The rest of the works, e.g. \citet{huang2016reference}, disagree
with the subject-only or topic-only analysis.
In \citet[\citesec{2.6}]{huang2016reference} [79] and [81a],
for example, 
the clausal initial temporal expressions and the NP 李家 in 李家人最多 
are recognized as information packaging devices and not subjects,
while they are subjects according to \citet[\citesec{7.2, 7.9.1}]{zhudexigrammar}

The position taken in this note is similar to \citet{huang2016reference}.
This is obviously a risky position,
since if there is indeed distinction between subject and topic, 
the approach in \citet{zhudexigrammar} may be inaccurate,
but it is still true if we replace the term \term{subject} with \term{topic in the broad sense}.
If, however, there is no such distinction,
then the approach here is wrong.
Even in the case when the subject-topic distinction is real,
there will still possibly be errors about 
whether a specific construction is
analyzed as a subject or topic.
\citet{zhudexigrammar} is useful in all cases above, though.
The risk is however worth taking, 
because it helps us to dig deeper into the subtle details of the language,
and also because the complete lack of subject-topic distinction 
seems to be not well-justified \citep{sih2000topic}.

The content of \citet[\citechap{7}]{zhudexigrammar},
therefore, will be scattered to 
\prettyref{sec:subject}, \prettyref{sec:valency-changing}, and \prettyref{sec:information-packaging}. 
To keep this note still relevant to Zhu's original book,
here is a list of subtypes of ``subject-predicate constructions'' given by him:
\begin{itemize}
    \item 
\end{itemize}

\subsubsection{Non-agentive subjects}

\subsection{An overview of internal complements}

\begin{table}
    \caption{Semantic (and then syntactic) classification of non-argument complements besides quantity complements}
    \label{tbl:complement-class}
    \begin{tabular}{cccccc}
    \toprule
    \multicolumn{1}{c}{} & directional          & resultive         & possibility               & \begin{tabular}[c]{@{}l@{}}manner and \\ consequence\end{tabular}               & \begin{tabular}[c]{@{}l@{}}time and \\ location\end{tabular}               \\ \midrule
    factual              & \begin{tabular}[c]{@{}l@{}}direction \\ complement\end{tabular} & \begin{tabular}[c]{@{}l@{}}result complement\end{tabular}  & \cellcolor[HTML]{9B9B9B}- & \begin{tabular}[c]{@{}l@{}}manner and \\ consequence \\ complement\end{tabular} & \begin{tabular}[c]{@{}l@{}}time and \\ location \\ complement\end{tabular} \\
    potential            & \multicolumn{3}{c}{potential complement}                             & \cellcolor[HTML]{9B9B9B}-                                                       & \cellcolor[HTML]{9B9B9B}-     \\ \bottomrule                                            
    \end{tabular}
    \end{table}

Classification of internal complements is a topic full of chaos.
There are roughly two classes of internal complements:
those prototypically filled by nominal constituents are given the label \term{object},
while those prototypically filled by verbal constituents are non-argument complements.
The classification is obviously form-oriented and not function-oriented.
Whether the concept of \term{object} has any syntactic significance requires argumentation. % TODO: ref

A purely semantics-oriented analysis of non-argument complements 
can be found in \prettyref{tbl:complement-class}.
This is given in \cite[5.8]{xianhan2004}.
The classification taken in \citet{zhudexigrammar} is a little different.
First, the manner and consequence complement class 
is divided into 状态补语 \translate{state complement}
and 程度补语 \translate{degree complement},
because of the imperfect mapping 
between the semantics and the syntax: 
the class of 程度补语 origins from grammaticalized direction complements and result complements,
and thus its grammatical properties differs from the rest of the manner and consequence complement class
(\citesec{\ref{draft-sec:complement-semantics}} in \draft).
In this note, I accept \term{state complement} and \term{degree complement}
as the translations of 状态补语 and 程度补语, respectively.

Another semantic class of non-argument complements often seen in textbooks is 
数量补语 \translate{quantity complement} \citep[\citesec{7.1}]{zhuqingming2005}.
The status of quantity complements  is kind of controversial.
Since quantity complements look like nominal arguments 
and can occur together with other types of non-argument complements, 
just like objects do, some authors -- including Zhu -- 
kick it from the family of complements and assign various names to it, 
for example semi-object and time expression. 
准宾语 \translate{semi-object} is the name used in \citet{zhudexigrammar}. % TODO: ref

The time and location complement class is also absent in \citet{zhudexigrammar},
because it can be easily reanalyzed as an instance of serial verb construction.

In conclusion, the classification of internal clausal complements in \citet{zhudexigrammar} % TODO: 作图

Due to the highly complicated interaction between all those complement types,
I have to introduce % TODO: 说清楚介绍不同的complement的顺序
\begin{itemize}
    \item Compounding-like non-argument complements:
    There are three types of non-argument complements,
    some (though not all) products of which look like compound words.   
\end{itemize}

\subsection{Direction complements}




\subsection{Degree complements}

The degree complement is a ``miscellaneous'' type,
which includes grammaticalized non-argument complements 
with various analyzable origins \citep[\citesec{9.9}]{zhudexigrammar}.

\subsubsection{Degree complements similar to result complements}

A verbal adjective predicator may be complemented by 
极, 多, and 透.

In modern usages, all the above mentioned degree complements have to be followed by the aspectual 了:
\begin{exe}
    \ex \begin{xlist}
        \ex 这本书[好极了]_{\text{pred: degree comp. const.}}
        \ex *这本书好极
    \end{xlist}
    \ex \begin{xlist}
        \ex 这本书[糟糕透了]_{\text{pred: degree comp. const.}}
        \ex *这本书糟糕透
    \end{xlist}
\end{exe}
More \ac{sfp}s are possible:
\begin{exe}
    \ex 这本书好极了呢!
\end{exe}

\subsubsection{好得很}

% TODO:结构来源

\begin{exe}
    \ex 这次演出好得很
\end{exe}

\subsection{Monotransitive indirect object}

There is conflict between the post-verbal object and the state complement.
Consider:
\begin{exe}
    \ex \begin{xlist}
        \ex *他写文章得读者云里雾里
        \ex *他写得文章读者云里雾里
    \end{xlist}
\end{exe}
The most natural alternative to the two unsuccessful attempts above is the verb-copying construction:
\begin{exe}
    \ex 他写文章写得读者云里雾里
\end{exe}
There is also another alternative,
where the object is preposed before the verb:
\begin{exe}
    \ex 他这话说得我们云里雾里
\end{exe}
In this construction, 
the agentive subject 
This construction, however, is not available for all verbs.
For example, a similar construction with the verb 写 seems to be ungrammatical:
\begin{exe}
    \ex *他这文章写得读者云里雾里
\end{exe}
The distinction between 给 and 写 is not shown 
in object preposing constructions like the follows:

\subsection{The verb 是}

The verb 是 is sometimes referred to as the copula in Chinese.
This captures some of its properties,
though not all. 

\citet[7.8]{zhudexigrammar}

\section{Valency changing devices}\label{sec:valency-changing}

Since Chinese dos not have any morphological marking of verbal valency changing,
one way -- exactly the way \citet{zhudexigrammar} works --


\begin{exe}
    \ex 我泡好茶了
    \ex 茶泡好了
\end{exe}

\subsection{Causative}

The causative is frequent in Old Chinese,
but its usage has been largely limited in modern Mandarin.
An example of the causative is shown below:
\begin{exe}
    \ex Syntactic causative \\
    我们的政策 [繁荣了市场]_{\text{predicate:causative VP}} 
    \ex Periphrastic causative \\
    我们的政策 [使市场繁荣了]_{\text{predicate:VP}}
\end{exe}

\subsubsection{Notional passive}

The difference 

\begin{exe}
    \ex 问题解决了~某人解决了问题
    \ex 茶泡好了~某人泡好了茶
\end{exe}

Since Chinese is a radical \term{pro}-drop language,
the notional passive construction 

\subsection{The affected construction}

The affected construction has already been introduced in \prettyref{sec:topic-subject-distinction},
the most famous example being 
\begin{exe}
    \ex 王冕死了父亲
\end{exe}
Similar constructions do exist, demonstrating this is still a productive construction:
\begin{exe}
    \ex 这座工厂前段时间塌了一堵墙
\end{exe}

\subsection{Instrumental object}

\begin{exe}
    \ex 我们今天准备吃食堂
\end{exe}

\section{Serial verb constructions}

\subsection{Overview}

The term \term{serial verb construction} is used widely in Chinese linguistics,
and one alternative to it is \term{chaining} \citep{po2015chinese}.
There are linguists arguing against the notion of 
\emph{the} serial verb construction,
for the term is just a catch-all term 
for constructions that are hard to analyze otherwise,
with great inhomogeneity inside \citep{paul2008serial}.
Since this note is carried out mainly following \citet{zhudexigrammar},
I will still give so-called serial verb constructions a separate section,
but this is only to keep the above sections not too long.

\subsubsection{The linear constituent order}

All serial verb constructions in Chinese can be analyzed as 
created by recursively applying simple serial verb constructions
\citep[\citesec{12.1.4}]{zhudexigrammar}.
A simple serial verb construction contains 
at least a verbal word and a predicate,
with a possible argument intervening the two
(how this argument is licensed will be discussed shortly).

Here are some examples (SVC is the abbreviation of \term{serial verb construction}):
\begin{exe}
    \ex\label{ex:svc-1} 我们合唱团一般 [[站着]_{\text{V}_1} [唱歌]_{\text{predicate:VP}}]_{\text{SVC}} 
    \ex\label{ex:svc-2} 我 [[没有]_{\text{V_1}} [工夫]_{\text{argument}} [[跟着]_{\text{V_1}} [你]_{\text{argument}} [到处乱跑]_{\text{predicate:VP}}]_{\text{predicate:SVC}}]_{\text{SVC}}
\end{exe}
Here I use the notation adopted in \citet{zhudexigrammar} and name the first verbal word as V_1.
The head of the predicate following V_1 is named V_2,
if it can be well-defined. 
If the predicate following V_1 is not itself a serial verb construction,
V_2 is just its head.
Therefore, the inner most serial verb construction always has a clear and uncontroversial V_2 position.
V_1 can be a verb or a coverb, and V_2 can be a verb, a coverb or a verbal adjective.
Adjectives filling the V_1 position are not attested
\citep[12.1.1, 12.1.2]{zhudexigrammar}.

If, however, the predicate following V_1 is a serial verb construction,
we are faced with the problem to identify a similar position for serial verb constructions.
This is done in \prettyref{sec:svc-embedding},
after simple serial verb constructions are discussed.

There are several other constructions having the same linear order with serial verb constructions,
including predicate coordination, 
predicator-object constructions which take complement clauses as objects,
and non-argument complement constructions
\citep[\citesec{12.1.3}]{zhudexigrammar}.
In the following sections, they are singled out.

\subsubsection{What is the relation between the two verbal words?}\label{sec:svc-no-intervening}

I start the discussion on serial verb constructions 
from the relation between V_1 and V_2 when there is no intervening argument.
Consider the following examples (the first is a copy of \eqref{ex:svc-1}):
\begin{exe}
    \ex\label{ex:zhanzhe-changge} 我们合唱团一般 [[站着]_{\text{V}_1} [唱歌]_{\text{predicate:VP}}]_{\text{SVC}} 
    \ex\label{ex:want-to-eat-1} 我 [[想]_{\text{V}_1} [吃点东西]_{\text{predicate:VP}}]_{\text{VP}}
\end{exe}
The two have exactly the same label-free constituency tree structures. 
Their semantic structures, however, are radically different.
In the first example, V_1 describes the manner of V_2.
In this case, the aspectual marker 着 obligatorily appear.
In \eqref{ex:want-to-eat-1}, however, V_2 -- together with its dependents -- 
is a complement of V_1.
That is why \eqref{ex:want-to-eat-1} is not considered as a serial verb construction:
it is already covered by the scheme of verb phrases.

There are more constructions which 

\subsubsection{The intervening argument}\label{sec:svc-intervening}

Now I discuss the V_1--intervening argument--V_2 constructions.
The intervening argument may simply be a complement of V_1.
In this case, the structure of the serial verb construction 
is largely parallel to how preposition phrases are introduced as adjuncts in English or Latin,
with the only difference being that 
in Chinese, the preposition phrase is replaced by 
a phrase headed by a verb or a coverb.
Thus V_2 -- its strict definition and properties -- is irrelevant to the intervening argument.
The V_1 and the intervening argument therefore forms a constituent,
and we may name it as predicate 1 
and the following predicate is named predicate 2.
Examples of this type include semantic counterparts of the English preposition phrase adjunct constructions:
\begin{exe}
    \ex 合唱团团员 [[在_{\text{V}_1:\text{coverb}} [室外]_{\text{object}}]_{\text{pred1:VP}} [一起唱歌]_{\text{pred2:VP}}]_{\text{SVC}}
\end{exe} 
It is also possible to have VP_1 denoting an action, rather than a state:
\begin{exe}
    \ex 合唱团团员刚才在 [[[拿]_{\text{V}_1} [着]_{\text{aspect}} [谱子]_{\text{object:NP}}]_{\text{pred1:VP}} [唱歌]_{\text{pred2:VP}}]_{\text{SVC}}
\end{exe}
This construction has nothing different from \eqref{ex:zhanzhe-changge}.
Their differences can be all attributed to the subcategorization properties of V_1
introduced in \prettyref{sec:verb-type-canonical}.

In other cases, the intervening argument has dependency relations with V_2.
Here is an example:
\begin{exe}
    \ex\label{ex:qing-ni-zhuangao} 我 [[请]_{\text{V_1}} [你]_{\text{argument}} [转告他一声]_{\text{pred2:VP}}]_{\text{SVC}}
\end{exe}
What is exactly the \emph{structural} relation 
between the intervening argument with V_1 and V_2 is debated.
\citet{zhudexigrammar} says the relation between the intervening argument and V_2
is purely semantic:
there is no constituency relation between it and V_2,
and instead, the intervening argument forms a constituent with V_1.
The argumentation provided for this claim 
is in topolects that have case marking for pronouns,
the intervening argument is always accusative (\citesec{12.2.2}),
and thus it receives the case from V_1 and is in a constituent with V_1.
This is not convincing, 
since the accusative case is a structural case, not an inherent one,
and it can be more strongly influenced by the large scale syntactic environment:
in English, for example,
the subject of an infinitive is accusative,
though it has nothing different semantically or structurally 
from the subject of a finite clause.

The position of this note 
is the syntax of \eqref{ex:qing-ni-zhuangao} is not clearly demonstrated 
by surface-oriented constituency analysis:
to catch all aspects of the serial verb construction with an intervening argument,
complicated movements and span-spellout in the underlying Minimalist derivation 
are necessary for a constituency-based analysis,
and in the surface-oriented version, 
the binary branching constituency relations are largely blurred.
For example, the analysis of \citet[\citesec{9.5}]{deng2010formal} 
of the intervening argument involves multiple head movement (see (82) in \citesec{9.5})
and this is by no means acceptable surface-oriented constituency analysis.
Therefore, for a general-purpose analysis of surface-oriented constituency relations,
a simple flat tree is enough:
I will stop at \eqref{ex:qing-ni-zhuangao} and do not attempt a ``deeper'' analysis.
The rest of the grammar should be described in terms of dependency relations.
The final analysis in a descriptive grammar is therefore more \ac{blt}-style
(see \citefig{\ref{method-fig:serial-verb-construction-1}} in \method),
though the phenomenon embodies a huge success of the light verb theory in generative syntax.
(The approach in this note is still not fully \ac{blt} as Dixon may like,
because V_2 in \eqref{ex:qing-ni-zhuangao} is not bracketed together with 请你,
but in the \ac{blt}-style predicate-as-verbal-span approach, it should be.)

It should be noted that there is not strict restriction on what argument becomes the intervening argument:
it can be the subject of V_2 as in \eqref{ex:qing-ni-zhuangao},
but it is also possible for the intervening argument to be the object, or an object: % TODO: which one? 
\begin{exe}
    \ex 他 [交了 [一篇文章]_{\text{argument}} 给我]_{\text{SVC}}(,让我评论评论)~他给了我 [一篇文章]
\end{exe}

\subsubsection{When the second predicate is also a serial verb construction}\label{sec:svc-embedding}

Now it is time to discuss what is V_2 when the second predicate is also a serial verb construction.
It can be observed from \prettyref{sec:svc-no-intervening} and \prettyref{sec:svc-intervening}
that V_2 is important only when there is an intervening argument 
and the intervening argument has non-trivial relations with V_2.
So consider the following example:
\begin{exe}
    \ex\label{ex:chen-jing-li-1} 他 [%
        [让]_{\text{V}_1}
        [我]_{\text{argument:pronoun}}
        [%
            [在未经他人允许的情况下]_{\text{pred1:SVC}}
            [%
                [交了]_{\text{V}_1} [这份报告]_{\text{argument:NP}} [给陈经理]_{\text{pred2:VP}}%
            ]_{\text{pred2:SVC}}%
        ]_{\text{pred2:SVC}}%
    ]_{\text{predicate:SVC}}
\end{exe}
Here we have four serial verb constructions in total.
The outmost one, 
[[让] [我] [在未经他人允许的情况下交了这份报告给陈经理]],
serves as the predicate of the whole clause.
Now 我 is the intervening argument,
and in a serial verb construction headed with 让 as V_1,
the intervening argument is likely to be the agent of V_2.
So what is V_2?
Semantics hints it is 交:
\eqref{ex:chen-jing-li-1} entails 
\begin{exe}
    \ex 我交了这份报告给陈经理
\end{exe}
So we find in the serial verb construction [[在未经他人允许的情况下] [交了这份报告给陈经理]],
which is the last constituent of the outmost serial verb construction
and is without an intervening argument with non-trivial relation with a V_2,
the first constituent -- 在未经他人允许的情况下 --
is irrelevant to the search for V_2.
Thus, it can be concluded that in two-constituent serial verb constructions,
if the first constituent is for modifying purposes,
then it is also syntactically so:
the second constituent is the head.

On the other hand, it can also be demonstrated that the search for V_2 stops 
when something -- labeled as $C$ -- like \eqref{ex:want-to-eat-1} or \eqref{ex:qing-ni-zhuangao} occurs,
and V_2 of the larger construction is designated as the V_1 of $C$.
Consider the following examples:
\begin{exe}
    \ex 他 [%
        [让]_{\text{V}_1} 
        [我]_{\text{argument:NP}} 
        [[想]_{\text{V}_1} [换个工作]_{\text{pred2:VP}}%
    ]_{\text{pred2:SVC}}]_{\text{predicate:SVC}} 了
\end{exe}
Here 我想换个工作
% TODO: 他让我买了一份报纸看

\subsubsection{Summary: the structure of serial verb constructions}

After the above discussions,
we find the Chinese serial verb constructions can be constructed by any of the following ways:
\begin{itemize}
    \item 
\end{itemize}

% TODO: 这东西花钱买回来搁着不用;似乎这个结构的成立也和他动词变自动词有关

It should be noted the scheme shown in this section does cover all possible serial verb constructions,
it does not exclude ungrammatical ones.
The following example has a constituency structure identical to what we see above
but is nonetheless not grammatical:
\begin{exe}
    \ex *我 [[把]_{\text{V}_1} [他]_{\text{argument}} [打了张三一顿]_{\text{predicate:VP}}]_{\text{SVC}}
\end{exe}
However, by just changing a word, the example becomes grammatical:
\begin{exe}
    \ex 我 [[让]_{\text{V}_1} [他]_{\text{argument}} [打了张三一顿]_{\text{predicate:VP}}]_{\text{SVC}}
\end{exe}
The difference can only be explained by the idiosyncratic properties of V_1.
This raises a question:
for lexical verbs, do their subcategorization information revealed by \prettyref{sec:verb-type-canonical}
completely decide their behaviors in serial verb constructions?
% TODO:似乎不是的,例如有些词只在把字句里面能够用:他把我的包放到了抽屉里

\subsection{Distinguishing serial verb constructions from others}

Unfortunately, there are occasionally grammar books 
saying certain utterances that can be explained perfectly well 
with other constructions
are made of serial verb constructions.
The authors of these books do not really know 
what we need a new term \term{serial verb construction}.
Below are some constructions that are frequently confused with serial verb constructions.

\subsubsection{Verb-complement constructions}



The distinction between the complement clause construction % TODO: ref
and serial verb construction is somehow subtle yet robust.
Compare \eqref{ex:want-to-eat-1} with the following clause:
\begin{exe}
    \ex\label{ex:want-to-eat-2} 我想有食欲不是坏事
\end{exe}
The two have the same main verb 想 and similar constituency trees.
This raises the question whether \eqref{ex:want-to-eat-1} contains a serial verb construction at all,
or even whether serial verb constructions exist at all.
It is, of course possible, to have verbs with complementation patterns like the

\subsection{The disposal construction: 把, 让, 使}\label{sec:disposal}

Potential complement constructions and 把 are mutually exclusive:
\begin{exe}
    \ex *他能把事情做得完
\end{exe}

\subsection{The passive construction: 被}

\section{Sentence final particles}

\section{Negation}

\section{The clause structure}

Now it is time to assemble components introduced in the above sections into clauses.
A clause is prototypically made up by a subject-predicate construction,
but certain information packaging processes are applicable (\prettyref{sec:information-packaging}). 
As for the subject-predicate construction i.e. the nucleus clause,
the most frequent filler of the predicate slot is either 
a verb phrase or a ``verb phrase'' headed by a verbal adjective (\prettyref{sec:verbal-predicate}),
but nominal predicate is also possible under certain circumstances (\prettyref{sec:nominal-predicate}). 


\subsection{Verbal predicate}\label{sec:verbal-predicate}

\subsection{Nominal predicate}\label{sec:nominal-predicate}

\citet[\citesec{7.6}]{zhudexigrammar}


\subsection{Information packaging}\label{sec:information-packaging}

\subsubsection{The default information structure}

% TODO: topics

\section{Subordination}

\section{Translation guides}

\bibliographystyle{plainnat}
\bibliography{references/grammars.bib,references/controversy.bib,references/generative.bib,references/aspects.bib,references/general-typology.bib}

\end{document}