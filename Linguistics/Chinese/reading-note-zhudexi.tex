\documentclass[UTF8, a4paper, oneside, scheme=plain]{ctexart}

\usepackage{geometry}
\usepackage{titling}
\usepackage{titlesec}
\usepackage{paralist}
\usepackage{footnote}
\usepackage{enumerate}
\usepackage{amsmath, amssymb, amsthm}
\usepackage{gb4e}
\noautomath
\usepackage{bbm}
\usepackage{soul}
\usepackage{graphicx}
\usepackage{siunitx}
\usepackage[table,xcdraw]{xcolor}
\usepackage{tikz}
\usepackage[ruled, vlined, linesnumbered, noend]{algorithm2e}
\usepackage{xr-hyper}
\usepackage[colorlinks]{hyperref} % linkcolor=black, anchorcolor=black, citecolor=black, filecolor=black
\usepackage[most]{tcolorbox}
\usepackage{caption}
\usepackage{subcaption}
\usepackage{booktabs}
\usepackage{multirow}
\usepackage[figuresright]{rotating}
\usepackage{acro}
\usepackage[round]{natbib} 
\usepackage{nameref,zref-xr}
\zxrsetup{toltxlabel}
\zexternaldocument*[draft-]{./main}[main.pdf]
\zexternaldocument*[cgel-]{../English/cambridge}[cambridge.pdf]
\zexternaldocument*[latin-]{../Latin/latin-notes}[latin-notes.pdf]
\zexternaldocument*[alignment-]{../alignment/alignment}[alignment.pdf]
\zexternaldocument*[exercise1-]{../Exercise/2021-3}[2021-3.pdf]
\zexternaldocument*[method-]{../methodology/glossing}[glossing.pdf]
\usepackage{prettyref}

\geometry{left=3.18cm,right=3.18cm,top=2.54cm,bottom=2.54cm}
\titlespacing{\paragraph}{0pt}{1pt}{10pt}[20pt]
\setlength{\droptitle}{-5em}

\DeclareMathOperator{\timeorder}{\mathcal{T}}
\DeclareMathOperator{\diag}{diag}
\DeclareMathOperator{\legpoly}{P}
\DeclareMathOperator{\primevalue}{P}
\DeclareMathOperator{\sgn}{sgn}
\newcommand*{\ii}{\mathrm{i}}
\newcommand*{\ee}{\mathrm{e}}
\newcommand*{\const}{\mathrm{const}}
\newcommand*{\suchthat}{\quad \text{s.t.} \quad}
\newcommand*{\argmin}{\arg\min}
\newcommand*{\argmax}{\arg\max}
\newcommand*{\normalorder}[1]{: #1 :}
\newcommand*{\pair}[1]{\langle #1 \rangle}
\newcommand*{\fd}[1]{\mathcal{D} #1}

\newcommand*{\citesec}[1]{\S~{#1}}
\newcommand*{\citechap}[1]{chap.~{#1}}
\newcommand*{\citefig}[1]{Fig.~{#1}}
\newcommand*{\citetable}[1]{Table~{#1}}

\newrefformat{sec}{\citesec{\ref{#1}}}
\newrefformat{fig}{\citefig{\ref{#1}}}
\newrefformat{tbl}{\citetable{\ref{#1}}}
\newrefformat{chap}{\citechap{\ref{#1}}}

\usetikzlibrary{arrows,shapes,positioning}
\usetikzlibrary{arrows.meta}
\usetikzlibrary{decorations.markings}
\tikzstyle arrowstyle=[scale=1]
\tikzstyle directed=[postaction={decorate,decoration={markings,
    mark=at position .5 with {\arrow[arrowstyle]{stealth}}}}]
\tikzstyle ray=[directed, thick]
\tikzstyle dot=[anchor=base,fill,circle,inner sep=1pt]


\tcbuselibrary{skins, breakable, theorems}

\newtcbtheorem[number within=chapter]{infobox}{Box}%
  {colback=blue!5,colframe=blue!65,fonttitle=\bfseries, breakable}{infobox}

\newcommand*{\concept}[1]{\textbf{#1}}
\newcommand*{\term}[1]{\emph{#1}}
\newcommand{\corpus}[1]{\emph{#1}}

\DeclareAcronym{blt}{short = BLT, long = Basic Linguistic Theory}
\DeclareAcronym{cgel}{short = CGEL, long = The Cambridge Grammar of the English Language}
\DeclareAcronym{dm}{short = DM, long = Distributed Morphology}
\DeclareAcronym{tag}{long = Tree-adjoining grammar, short = TAG}
\DeclareAcronym{sfp}{long = sentence final particle, short = SFP}
\DeclareAcronym{np}{long = noun phrase, short = NP}
\DeclareAcronym{vp}{long = verb phrase, short = VP}
\DeclareAcronym{pp}{long = preposition phrase, short = PP}
\DeclareAcronym{cls}{long = classifier, short = CLS}
\DeclareAcronym{dist}{long = distal, short = DIST}
\DeclareAcronym{prox}{long = proximate, short = PROX}
\DeclareAcronym{dem}{long = demonstrative, short = DEM}
\DeclareAcronym{dur}{long = durative, short = DUR}
\DeclareAcronym{neg}{long = negative, short = NEG}

\newcommand*{\homo}[2]{#1$_{\text{#2}}$}

\newcommand{\cgel}{\href{../English/cambridge.pdf}{my notes about CGEL}}
\newcommand{\latin}{\href{../Latin/latin-notes.pdf}{my notes about Latin}}
\newcommand{\alignment}{\href{../alignment/alignment.pdf}{my notes about alignment}}
\newcommand{\exerciseone}{\href{../Exercise/2021-3.pdf}{this exercise}}
\newcommand{\method}{\href{../methodology/glossing.pdf}{this note about how descriptive grammars work}}
\newcommand{\draft}{\href{./main.pdf}{this draft}}

\newcommand{\ala}{à la}
\newcommand{\translate}[1]{`#1'}

\title{Mandarin morphosyntax reading note}
\author{Jinyuan Wu}

\begin{document}

\maketitle

\automath

This note is my reading note of \citet{zhudexigrammar}.
It can be seen as a preparation of \draft,
which is premature and possibly will never be finished,
especially by someone without systematic linguistic training like me.
Still, the theoretical orientation of this note is well introduced in the above link,
as well as in \cgel, \latin, and \method.
\citet{zhudexigrammar} is commonly referred to as a typical structuralist book of Chinese.
I do not say ``structuralist grammar'' because the book is also a textbook about structuralism,
mostly in Bloomfield's brand
and strikingly close to the \ac{cgel} \citep{cgel} approach summarized in the above notes,
with a lot of argumentation, more than what ordinary grammars contain.

\section{About Zhu's book}

\subsection{The object language}

The object language, ``Chinese'', needs some clarification.
It means Standard Modern Chinese or Standard Modern Mandarin,
often abbreviated as Mandarin in the English speaking world.
In mainland China it is called 普通话.
In Taiwan and Singapore it (with small variations) is called 国语.

Standard Mandarin -- like other languages -- is an evolving language.
Certain usages documented in Zhu's book have already been obsoleted.
% TODO: evolving speed

\subsection{Organization of chapters}\label{sec:organization-chapter}

The book can be divided into several parts:
\begin{itemize}
    \item Chapters 1-6 are about morphology and lexical categories.
    Lexical categories discussed in details are either nominal or verbal.
    \item Chapters 7-10 together give a top-down analysis of syntactic constructions without coordination.
    Serial verb constructions are \emph{not} introduced in these chapters.
    \item Chapter 11 is about coordination.
    \item Chapter 12 is about serial verb construction.
    \item Chapter 13-14 are about prepositions and adverbs.
    \item Chapter 15 is about clause types.
    \item Chapter 16 is about \ac{sfp}.
    \item Chapter 17 is about clause linking without canonical coordination.
    \item Chapter 18 is about ellipsis and inversion, 
    which may be roughly said to be about information packaging. 
\end{itemize}

This organization is an example of \citesec{\ref{method-sec:chapter-organization}} in \method.
The relation between the first six chapters and the following four 
is the item and arrangement strategy relation.
Within the chapters 7-10,
we see the top-down partition of clauses and \acs{np}s 
introduced in \citesec{\ref{method-sec:clause-top-down}}
and \citesec{\ref{method-sec:np-top-down}} in \method.
This is typical in structuralist works:
it is a direct reflection of the top-down analysis of syntactic structures 
(see \citesec{\ref{method-sec:item-and-arrangement-grammar}} and \citesec{\ref{method-sec:clause-top-down}}
in \method).

The noun-verb distinction (\citesec{\ref{method-sec:np-clause-chapter}} in \method) 
is only reflected in 
nominal categories being introduced in \citechap{4},
while verbal categories being introduced in \citechap{5}.
The \ac{np} structure is introduced in \citechap{10},
together with their clausal counterparts.

% TODO: chapter 11 and 12

The relation between the first twelve chapters and chapters 13 and 14
is the relation between canonical constructions and their counterparts with adjunction.
The relation between the first fourteen chapters and \citechap{15} is 
the relation between canonical constructions and non-canonical ones
related to the former ones by transformation rules.

Chapters 7, 8, 9, 10, 13, and 14 constitute a system 
quite similar to the chapter 4-8 in \ac{cgel}:
first clausal complements, 
including the external complement -- the subject --
and internal complements,
then \ac{np}s,
then how the two are modified,
by adjectives and adverbs,
or by more complicated \acs{pp}s.

Chapter 16 actually can be placed before \citechap{13}.
This is not the order used in the book,
the reason of which, in generative terms,
seems to be that \ac{sfp}s are merged in higher projections than what is involved 
form \citechap{7} to \citechap{15}.
Zhu, however, regard most of \ac{sfp}s as a part of the predicate.
The contradiction between the arrangement of chapters 
and the explicit analysis of \ac{sfp}s as a part of the predicate in \citesec{16.1.1} 
in \citet{zhudexigrammar}
is actually self-consistent:
the mutual relation between the predicate and the \ac{sfp}s
is parallel to the mutual relation between the verb stem and the aspectual markers:
the aspectual markers are introduced in higher functional projections than the verb stem,
in the same way \ac{sfp}s are introduced in higher functional projections than the predicate.
The verb stem and the aspectual markers being analyzed as two immediate constituents of a ``word''
reflects post-syntactic processes,
not constituency relations and dependency relations created by the syntax proper.
If this is acceptable -- which is the case in most descriptive grammars --
then since phonologically and especially from the perspective of prosody,
\ac{sfp}s are closer to the predicate,
it is of course also acceptable to place the \ac{sfp}s into the predicate.

\subsection{Terminology}\label{sec:terminology}

The terminology used in the book is closer to the \ac{cgel} approach rather than the \ac{blt} approach.
It should be noted that the book is written in Mandarin Chinese,
in which certain linguistic terms 
do not have morpheme-to-morpheme counterparts in English or
already have different meaning than their morpheme-to-morpheme counterparts in English.

To keep the rest of this note fluent,
issues of term translations are summarized in this section.

\subsubsection{Theoretical orientation}

Terminology reflects the theoretical orientation of a grammar.
The term \term{head} is the lexical head and not the functional head.
Therefore, we have notions like noun phrase, verb phrase, etc.,
in which the head is defined as the noun, the verb, etc.
and not the determiner or the light verb.

\subsubsection{Word classes}\label{sec:word-class-term}

For lexical categories, 体词 means \translate{referential word} i.e. nominal words.
Its direct translation would be \translate{body-word},
which may be understood by some as \translate{content word} i.e. \translate{lexical word}.
The term 谓词 means \translate{verbal word}.
The direct translation would be \translate{commenting-word},
which may be understood as somehow ``predicative'' in the sense of predicative complements in \ac{cgel}.
This is not correct: 
谓词 means what can head a predicate,
thus verbs and adjectives in Chinese.
The term 实词 \translate{substantial word} means lexical words,
while 虚词 \translate{virtual word} means function words.

\subsubsection{Clause structure}

The term 谓语 means predicate in the \ac{cgel} sense.
The term 述语 means predicator in the \ac{cgel} sense.
Unlike earlier structuralist works which work in the vanilla phrase structure grammar (PSG) framework,
\citet{zhudexigrammar} uses a \ac{cgel}-like PSG,
where a label of a constituent in a larger construction 
contains both its category label and its function label,
for example both ``\ac{np}'' and ``subject''.
This idea is made explicit in \citesec{1.3.10}.
The analysis of 我们班有许多外国留学生 in \citesec{1.3.8} is a good example.
Unlike \ac{cgel},
\citet{zhudexigrammar} uses a more compact format 
in which constituents are illustrated by underlining 
to show the constituency tree.
This is, of course, merely a notational problem,
but somehow it becomes a tradition of the School Grammar analysis of Chinese.

\subsection{About this note}

This note try to rearrange the content of \citet{zhudexigrammar}
in a way that is both acceptable in the approaches in \ac{blt} and \ac{cgel}.
The order of this note is largely bottom-up.
Certain top-down analyses, of course, will be given in the grammar sketch chapter.
When obsolete usages appear, I will point them out.
When the analysis is problematic, I will discuss why it is problematic and how it can be improved.

\section{A grammar sketch}

The first chapter in \citet{zhudexigrammar} may be thought as a grammar sketch chapter,
but it contains much discussion on theoretical issues
(replicating what is discussed in \ac{cgel} \citesec{1.4}).
This section is a more compact grammar sketch,
skipping theoretical commitments which can be found in sources at the beginning of this note.
Chapter 3 is also a short one and may be regarded as a part of the grammar sketch.

I will roughly follow \citet{jacques2021grammar} in the organization of this section.
However, since in Chinese, dependency relations are not mainly coded by morphology,
I will replace the ``nominal morphology'' section by ``noun phrase''
and replace the ``verbal morphology'' section by ``clause structure'',
and do not give constituent order a special section,
since constituent order is covered by the constituency structure.
This is a major difference between \ac{cgel}-like ``structuralist'' grammars 
and \ac{blt}-based ``functionalist'' grammars
(\citesec{\ref{method-sec:theory}} in \method).

\subsection{Parts of speech}

Since Chinese does not rich grammatical relation-bearing morphology,
purely syntactic tests play the major role in determining parts of speech.
Semantics may help but is never decisive 
(\citesec{3.1.1} and \citesec{3.1.2} in \citet{zhudexigrammar}).
The word class division given in the book inevitably meets the problem 
that a word may belong to two categories depending on the context.%
\footnote{
    Formally, we may say the word prototypically belong to one category,
    and its usage as a word in another category
    involves zero derivation or conversion.
    From a Distributed Morphology perspective,
    however, we can also say that the stem of that word can be merged with 
    two categorizers, 
    and here we are faced with the same problem that 
    urged linguists to give up transformational rules.

    The most appropriate term for this process -- 
    zero-derivation, conversion or something else -- is still debated,
    and I will skip this topic in this note.
}
In the analysis adopted here, words belonging to two categories are only the minority,
because otherwise, the two categories can be considered as one 
(\citesec{3.2, 3.3} in \citet{zhudexigrammar}).

\subsubsection{Lexical words}

Lexical words in Chinese can be roughly divided into nominal ones and verbal ones,
or in the Chinese terms, 体词 and 谓词
(for issues on translation between English and Chinese terms, see \prettyref{sec:word-class-term}).
The prototypical role of nominal words 
is to fill argument slots (or to be more precise, to head a phrase that fills an argument slot).
Nominal words rarely appear in the predicator position
(though for stylistic purposes, they sometimes do).
Verbal words prototypically fill argument slots,
but many of them -- and clauses without any morphological marking -- 
can regularly appear in argument slots \citep[\citesec{3.5}]{zhudexigrammar}.

The fact that verbal categories can fill argument slots or in colloquial words ``be used as nouns''
urges some to put the verbal categories under the nominal categories,
so thus there is only one mega lexical category in Chinese:
the nominal category or the Noun.
The analysis adopted here does not aim to organize lexical categories 
in a binary branching classification tree,
so the ordinary nominal-verbal distinction is maintained.

Whether Chinese has a separate adjective category 
has been debated for decades.
Based on a line of reasoning similar to the above verb-as-noun analysis,
some linguists argue that the so-called adjectives should be put under the verb category,
since they can fill the predicator slot without any morphological marking \citep{li1989mandarin}.
Since verbs and most alleged adjectives show different morphological behaviors in duplication, % TODO: ref
the verb-adjective distinction is kept,
and the two are placed under the verbal category.

There still exist a (much smaller) number of alleged adjectives that shows 
different morphosyntactic properties with the adjectives in the verbal category.
They can be marginally used as heads of \ac{np}s,
while they do not have duplication variants.
These ``adjectives'' are thus placed under the nominal category.
Thus we have two types of adjectives
In \citet{zhudexigrammar}, 
nominal adjectives are called 区别词 \translate{distinction word},
while verbal adjectives are called 形容词 \translate{adjective}.

There are more nominal categories than the ordinary noun category and the nominal adjective category.
Numerals, for examples, are in another nominal category.
Chinese has a rich classifier system,
and most classifiers still have strong nominal properties
and thus they constitute yet another nominal category.
\citet{zhudexigrammar} calls them 量词 \translate{measure word},
because many classifiers have the meaning of ``unit''.
There is also a location word class, including 里 in 在房子里,
which is sometimes said to be the postposition class.

\subsubsection{Function words}

Unlike the case in English or Latin 
(see \citesec{\ref{latin-sec:particle-relation}} in \latin), 
in Chinese, there is no synchronic or diachronic ways 
to regularly form adverbs from fossilized phrases 
or from adjectives via derivations 
which can be seen as forming a peripheral argument 
with the meaning of ``in the manner of \dots''.
Thus what can be uncontroversially called adverbs in Chinese 
form only a small category,
which is placed as one type of function words in \citet{zhudexigrammar}.
% TODO: 狠狠地这种算状语,但是是副词吗?

So-called Chinese prepositions are all historically verbs.
The distribution of so-called preposition phrases % TODO:把字句中“把”不是介词,是轻动词
is also highly restricted,
rendering people to ask whether they are constituents at all.
Despite \citet{zhudexigrammar} calls them 介词 \translate{adposition},
these words are better regarded as introduced in serial verb constructions
(\prettyref{sec:serial-verb-construction-intro}),
instead of English-like and Latin-like peripheral argument slots.
Thus, in this note, I call these ``prepositions'' \term{coverbs},
following the terminology in \citet{po2015chinese}.

Another group of function words in Chinese is the \ac{sfp}.
They are named 语气词 \translate{specch force word} in \citet{zhudexigrammar},
revealing the fact that they are about in the Force projection(s)
-- though \citet{zhudexigrammar} somehow insists on them being a part of the predicate
(\prettyref{sec:organization-chapter}, TODO: more ref).

\subsubsection{Overview of all categories}

The comprehensive classification of parts of speech can be found in \citet[\citesec{3.6}]{zhudexigrammar}.
Two categories that are neither lexical nor function 
are the ideophone class and the interjection class.
% TODO: the standard of open and closed

\subsection{Morphology}

Chinese does not have inflection at all,
except the aspectual markers,
which may be argued to be agglutinating suffixes (\prettyref{sec:aspectual}).
The rest of morphology is all derivational.
Morphological devices attested include 
duplication, compounding and affixation.
No internal change, infix or circumfix is attested.

A question causing endless controversy and confusion 
is ``what is a word''. % TODO: criteria

\subsection{Nominal categories, morphology, and the \ac{np}}

\subsubsection{The \ac{np} template}

No morphological case, number, and gender categories are attested in Chinese.
There is a word class system or in other words classifier system, however.
In most cases when a numeral appears in a \ac{np},
a classifier follows immediately after the numeral.
Attributives -- both adjectives and relative clauses -- 
follow the classifier. % TODO: 红色的三个,这样的说法说得通吗?
The demonstrative, if any, appears before the numeral,
and even when there is no numeral,
there is frequently also a classifier.

The template of \ac{np}s, therefore, can be summarized as 
demonstrative--numeral--classifier--attributive(s)--head noun.

\subsection{The verb and the clause}

\subsubsection{The verb}



\subsubsection{The subject in a clause}

Though completely lacking case morphology,
Chinese is a typical syntactically accusative language.
% TODO: subject and topic
The structuralist binary branching works well for Chinese clausal structure 
\citep[\citesec{133-136}]{zhudexigrammar}.
A clause without preposing -- henceforth called a nucleus clause --
can be divided into a subject and a predicate,
plus possible \ac{sfp}s.
The subject is on the left, and the predicate is on the right, followed by \ac{sfp}s.
The predicate may be a 
single verbal word (its function is the predicator) plus possible internal complements, 
possibly modified by adverbs,
and in this case we say the predicate is filled by a verb phrase.
The predicate may also be 

\subsubsection{Verb complementation patterns}

Clausal complements inside the predicate are said to be internal.
Internal complements of the verb include objects and non-argument complements
\citep[1.3.3-1.3.4]{zhudexigrammar},
the latter being called 补语 \translate{complementing speech} in \citet{zhudexigrammar}.
The term 补语 is frequently translated into \term{complement} in English,
but then it conflicts with the wider definition of complements in \ac{cgel},
which includes both arguments and 补语,
and such confusion occurs,
I use the term \term{non-argument complement}.

We are sure that non-argument complements are not arguments,
because they cannot be filled by nominal constituents.
They are indeed complements, if not parts of verb compounding constructions,
for reasons given in \citesec{\ref{draft-sec:complement-name}} in \draft.

Non-argument complements and objects have complicated interplays,
and the boundary of non-argument complements is not always clear.
Many non-argument complement types are mutually exclusive.
The constituent order between some non-argument complements and the object(s) is rigid,
while for other non-argument complements it is more flexible.
Certain non-argument complement constructions are almost examples of verb compounding,
and the so-called complements may be analyzed as a part of the verb complex.
Certain non-argument complements are almost objects.
% TODO: ref, indirect complement

\subsubsection{Serial verb constructions}\label{sec:serial-verb-construction-intro}

Chinese has rich serial verb constructions,
in which the predicate contains more than one main verbs
or possibly a main verb and one or more verbal adjectives and coverbs.
The distinction between serial verb constructions and some non-argument complement constructions 
is highly blurred.

\subsubsection{Sentence-final particles}

\ac{sfp}s are actually clause-final particles,
because they can appear in subordinate clauses,
but since this is the standard term I will not alter it.
They appear strictly at the end of nucleus clauses.
Postposing to the right of \ac{sfp}s is rare, if possible.

\subsubsection{Unattested constructions and categories}

\subsection{Negation}

Chinese does not have a versatile negator.
Negation in 

\subsection{Coordination, clause linking and supplementation}

Coordination occurs in all levels of Chinese syntax:
\ac{np}s, predicates, and clauses.
In these constructions different coordination devices are used.

\subsection{Subordination}

Like all 

\subsection{Prosody and styles}

\subsection{Remarkable features}

\section{Nominal categories}

\subsection{Nouns}

There are two defining properties of the noun class:
being able to be modified by a numeral-classifier construction (\prettyref{sec:possession}),
and being unable to be modified by adverbs. % TODO: ref
A word with both of the properties is definitely a noun.
Certain verbal words also appear with numerals and classifiers 
(which may be viewed as zero-derivation into abstract nouns),
but they can always be modified by adverbs,
so they themselves are not nouns \citep[\citesec{4.1.1}]{zhudexigrammar}.

Nouns may be classified according to 
their classes and countability (\prettyref{sec:noun-class-classifier}),
their behaviors in possession (\prettyref{sec:possession}).

\subsubsection{Duplication of nouns}

Duplication of nouns is mainly restricted to kinship terms,
like 爷爷 \translate{grandpa}, 奶奶 \translate{gradma}, 爸爸, 妈妈.

\subsection{Classifiers}\label{sec:classifiers}

There are roughly seven types of classifiers. % TODO: 翻译

\section{Verbal categories}

\subsection{Verbs}

\subsection{Verbal adjectives}


\section{Noun phrases}

\subsection{Noun class and the classifier}\label{sec:noun-class-classifier}

Possible classifier in a \ac{np} headed by a noun 
gives the noun class of that noun.
Roughly there are five classes \citep[\citesec{4.1.2}]{zhudexigrammar}:
\begin{itemize} % TODO: ref to sec:classifiers
    \item Countable nouns, whose classifiers themselves denote to discrete objects.
    \item Uncountable nouns, whose classifiers are 
\end{itemize}
Each class has lots of subclasses.

\subsubsection{Numeral}\label{sec:numeral-classifier}

\subsection{The possessive construction}\label{sec:possession}

% TODO: “我爸”这样明显是词法的动词应该放在这里吗?

\subsection{Relative order of noun phrase dependents}

\section{Verbal morphology}

\subsection{Aspectual markers}\label{sec:aspectual}

A separate section has to be devoted to 了, 着, and 过,
because they code the aspectual system in Chinese.
I say \term{aspectual}, not \term{aspect},
partly because there are so many well-accepted usage of the term \term{aspect},
partly because whether the Chinese aspectual system can be safely said to be one of them is still controversial.

\section{Verb types, argument structures and clausal dependents}\label{sec:verb-type-canonical}

% TODO: serial verb construction也和verb type有关,要不要放在这里?

Like the corresponding chapter in \ac{cgel} (\citechap{4}),
this section is mainly about canonical clauses.
Here ``canonical'' means the clausal complements transparently displays the argument structure:
we only have 王冕经历了父亲的过世, and not 王冕死了父亲.
Non-canonical constructions appear only for making argumentation for a complement type
(\citesec{\ref{method-sec:why-argumentation-function-label}} in \method).

Needless to say, non-canonical usages may be fossilized
and become canonical, 
as in Old Chinese 示,
which is likely to be a fossilized causative construction as in
蔺相如示秦王壁~蔺相如使秦王视璧.
% TODO: ref: 和non-canonical的章节的链接

The complement configuration of a canonical clause headed by a verb 
is the main factor of verb subcategorization.

\subsection{Subject and subjecthood}



\subsection{An overview of internal complements}

\begin{table}
    \caption{Semantic (and then syntactic) classification of non-argument complements besides quantity complements}
    \label{tbl:complement-class}
    \begin{tabular}{cccccc}
    \toprule
    \multicolumn{1}{c}{} & directional          & resultive         & possibility               & \begin{tabular}[c]{@{}l@{}}manner and \\ consequence\end{tabular}               & \begin{tabular}[c]{@{}l@{}}time and \\ location\end{tabular}               \\ \midrule
    factual              & \begin{tabular}[c]{@{}l@{}}direction \\ complement\end{tabular} & \begin{tabular}[c]{@{}l@{}}result complement\end{tabular}  & \cellcolor[HTML]{9B9B9B}- & \begin{tabular}[c]{@{}l@{}}manner and \\ consequence \\ complement\end{tabular} & \begin{tabular}[c]{@{}l@{}}time and \\ location \\ complement\end{tabular} \\
    potential            & \multicolumn{3}{c}{potential complement}                             & \cellcolor[HTML]{9B9B9B}-                                                       & \cellcolor[HTML]{9B9B9B}-     \\ \bottomrule                                            
    \end{tabular}
    \end{table}

Classification of non-argument complements is a topic full of chaos.
A purely semantics-oriented analysis of non-argument complements 
can be found in \prettyref{tbl:complement-class}.
This is given in \cite[5.8]{xianhan2004}.
The classification taken in \citet{zhudexigrammar} is a little different.
First, the manner and consequence complement class 
is divided into 状态补语 \translate{state complement}
and 程度补语 \translate{degree complement},
because of the imperfect mapping 
between the semantics and the syntax: 
the class of 程度补语 origins from grammaticalized direction complements and result complements,
and thus its grammatical properties differs from the rest of the manner and consequence complement class
(\citesec{\ref{draft-sec:complement-semantics}} in \draft).
In this note, I accept \term{state complement} and \term{degree complement}
as the translations of 状态补语 and 程度补语, respectively.

Another semantic class of non-argument complements often seen in textbooks is 
数量补语 \translate{quantity complement} \citep[\citesec{7.1}]{zhuqingming2005}.
The status of quantity complements  is kind of controversial.
Since quantity complements look like nominal arguments 
and can occur together with other types of non-argument complements, 
just like objects do, some authors -- including Zhu -- 
kick it from the family of complements and assign various names to it, 
for example semi-object and time expression. 
准宾语 \translate{semi-object} is the name used in \citet{zhudexigrammar}. % TODO: ref

The time and location complement class is also absent in \citet{zhudexigrammar},
because it can be easily reanalyzed as an instance of serial verb construction.

In conclusion, the classification of internal clausal complements in \citet{zhudexigrammar} % TODO: 作图

\subsection{Monotransitive indirect object}

\section{Serial verb constructions and related constructions}

\subsection{Overview}

\subsubsection{The linear constituent order}

All serial verb constructions in Chinese can be analyzed as 
created by recursively applying simple serial verb constructions
\citep[\citesec{12.1.4}]{zhudexigrammar}.
A simple serial verb construction contains 
at least a verbal word and a predicate,
with a possible argument intervening the two
(how this argument is licensed will be discussed shortly).

Here are some examples (SVC is the abbreviation of \term{serial verb construction}):
\begin{exe}
    \ex\label{ex:svc-1} 我们合唱团一般 [[站着]_{\text{V}_1} [唱歌]_{\text{predicate:VP}}]_{\text{SVC}} 
    \ex\label{ex:svc-2} 我 [[没有]_{\text{V_1}} [工夫]_{\text{argument}} [[跟着]_{\text{V_1}} [你]_{\text{argument}} [到处乱跑]_{\text{predicate:VP}}]_{\text{predicate:SVC}}]_{\text{SVC}}
\end{exe}
Here I use the notation adopted in \citet{zhudexigrammar} and name the first verbal word as V_1.
The head of the predicate following V_1 is named V_2,
if it can be well-defined. 
If the predicate following V_1 is not itself a serial verb construction,
V_2 is just its head.
Therefore, the inner most serial verb construction always has a clear and uncontroversial V_2 position.
If, however, the predicate following V_1 is a serial verb construction,
we are faced with the problem to identify a similar position for serial verb constructions.

There are several reasons for us to define 

V_1 can be a verb or a coverb, and V_2 can be a verb, a coverb or a verbal adjective.
Adjectives filling the V_1 position are not attested
\citep[12.1.1, 12.1.2]{zhudexigrammar}.

\subsubsection{The intervening argument}

The intervening argument may simply be a complement of V_1.
In this case, the structure of the serial verb construction 
is largely parallel to how preposition phrases are introduced as adjuncts in English or Latin,
with the only difference being that 
in Chinese, the preposition phrase is replaced by 
a phrase headed by a verb or a coverb.
Thus V_2 -- its strict definition and properties -- is irrelevant to the intervening argument.
The V_1 and the intervening argument therefore forms a constituent,
and we may name it as predicate 1 
and the following predicate is named predicate 2.
Examples of this type include most semantic counterparts of the English preposition phrase adjunct constructions:
\begin{exe}
    \ex 合唱团团员 [[在_{\text{V}_1:\text{coverb}} [室外]_{\text{object}}]_{\text{pred1:VP}} [一起唱歌]_{\text{pred2:VP}}]_{\text{SVC}}
\end{exe} 

Another case may be that the intervening argument has dependency relations with V_2.


\subsubsection{Summary: the structure of serial verb constructions}

It should be noted the scheme shown in this section does cover all possible serial verb constructions,
it does not exclude ungrammatical ones.
The following example is fine in constituency structure
but is nonetheless not grammatical:
\begin{exe}
    \ex *我 [[把]_{\text{V}_1} [他]_{\text{argument}} [打了张三一顿]_{\text{predicate:VP}}]_{\text{SVC}}
\end{exe}
However, by just changing a word, the example becomes grammatical:
\begin{exe}
    \ex 我 [[让]_{\text{V}_1} [他]_{\text{argument}} [打了张三一顿]_{\text{predicate:VP}}]_{\text{SVC}}
\end{exe}
The difference can only be explained by the idiosyncratic properties of V_1.
This raises a question:
for lexical verbs, do their subcategorization information revealed by \prettyref{sec:verb-type-canonical}
completely decide their behaviors in serial verb constructions?
% TODO:似乎不是的,例如有些词只在把字句里面能够用:他把我的包放到了抽屉里

\subsubsection{Distinguishing serial verb constructions from others}

There are several other constructions having the same linear order with serial verb constructions,
including predicate coordination, 
predicator-object constructions 
where the object position is filled by another predicator-object construction,
and non-argument complement constructions with similar objects
\citep[\citesec{12.1.3}]{zhudexigrammar}.

\subsection{The disposal construction: 把, 让, 使}

\subsection{The passive construction: 被}

\section{Sentence final particles}

\section{Clauses}

Now it is time to assemble components introduced in the above sections into clauses.


\bibliographystyle{plainnat}
\bibliography{references/grammars.bib,references/controversy.bib,references/generative.bib,references/aspects.bib,references/general-typology.bib}

\end{document}