\documentclass[UTF8, a4paper, oneside, scheme=plain]{ctexart}

\usepackage{geometry}
\usepackage{titling}
\usepackage{titlesec}
\usepackage{paralist}
\usepackage{footnote}
\usepackage{enumerate}
\usepackage{amsmath, amssymb, amsthm}
\usepackage{gb4e}
\noautomath
\usepackage{bbm}
\usepackage{soul}
\usepackage{graphicx}
\usepackage{siunitx}
\usepackage[table,xcdraw]{xcolor}
\usepackage{tikz}
\usepackage[ruled, vlined, linesnumbered, noend]{algorithm2e}
\usepackage{xr-hyper}
\usepackage[colorlinks]{hyperref} % linkcolor=black, anchorcolor=black, citecolor=black, filecolor=black
\usepackage[most]{tcolorbox}
\usepackage{caption}
\usepackage{subcaption}
\usepackage{booktabs}
\usepackage{multirow}
\usepackage[figuresright]{rotating}
\usepackage{acro}
\usepackage[round]{natbib} 
\usepackage{nameref,zref-xr}
\zxrsetup{toltxlabel}
\zexternaldocument*[cgel-]{../English/cambridge}[cambridge.pdf]
\zexternaldocument*[latin-]{../Latin/latin-notes}[latin-notes.pdf]
\zexternaldocument*[alignment-]{../alignment/alignment}[alignment.pdf]
\zexternaldocument*[exercise1-]{../Exercise/2021-3}[2021-3.pdf]
\zexternaldocument*[method-]{../methodology/glossing}[glossing.pdf]
\usepackage{prettyref}

\geometry{left=3.18cm,right=3.18cm,top=2.54cm,bottom=2.54cm}
\titlespacing{\paragraph}{0pt}{1pt}{10pt}[20pt]
\setlength{\droptitle}{-5em}

\DeclareMathOperator{\timeorder}{\mathcal{T}}
\DeclareMathOperator{\diag}{diag}
\DeclareMathOperator{\legpoly}{P}
\DeclareMathOperator{\primevalue}{P}
\DeclareMathOperator{\sgn}{sgn}
\newcommand*{\ii}{\mathrm{i}}
\newcommand*{\ee}{\mathrm{e}}
\newcommand*{\const}{\mathrm{const}}
\newcommand*{\suchthat}{\quad \text{s.t.} \quad}
\newcommand*{\argmin}{\arg\min}
\newcommand*{\argmax}{\arg\max}
\newcommand*{\normalorder}[1]{: #1 :}
\newcommand*{\pair}[1]{\langle #1 \rangle}
\newcommand*{\fd}[1]{\mathcal{D} #1}

\newcommand*{\citesec}[1]{\S~{#1}}
\newcommand*{\citechap}[1]{chap.~{#1}}
\newcommand*{\citefig}[1]{Fig.~{#1}}
\newcommand*{\citetable}[1]{Table~{#1}}

\newrefformat{sec}{\citesec{\ref{#1}}}
\newrefformat{fig}{\citefig{\ref{#1}}}
\newrefformat{tbl}{\citetable{\ref{#1}}}
\newrefformat{chap}{\citechap{\ref{#1}}}

\usetikzlibrary{arrows,shapes,positioning}
\usetikzlibrary{arrows.meta}
\usetikzlibrary{decorations.markings}
\tikzstyle arrowstyle=[scale=1]
\tikzstyle directed=[postaction={decorate,decoration={markings,
    mark=at position .5 with {\arrow[arrowstyle]{stealth}}}}]
\tikzstyle ray=[directed, thick]
\tikzstyle dot=[anchor=base,fill,circle,inner sep=1pt]


\tcbuselibrary{skins, breakable, theorems}

\newtcbtheorem[number within=chapter]{infobox}{Box}%
  {colback=blue!5,colframe=blue!65,fonttitle=\bfseries, breakable}{infobox}

\newcommand*{\concept}[1]{\textbf{#1}}
\newcommand*{\term}[1]{\emph{#1}}
\newcommand{\corpus}[1]{\emph{#1}}

\DeclareAcronym{blt}{short = BLT, long = Basic Linguistic Theory}
\DeclareAcronym{cgel}{short = CGEL, long = The Cambridge Grammar of the English Language}
\DeclareAcronym{dm}{short = DM, long = Distributed Morphology}
\DeclareAcronym{tag}{long = Tree-adjoining grammar, short = TAG}
\DeclareAcronym{sfp}{long = sentence final particle, short = SFP}
\DeclareAcronym{np}{long = noun phrase, short = NP}
\DeclareAcronym{vp}{long = verb phrase, short = VP}
\DeclareAcronym{pp}{long = preposition phrase, short = PP}
\DeclareAcronym{cls}{long = classifier, short = CLS}
\DeclareAcronym{dist}{long = distal, short = DIST}
\DeclareAcronym{prox}{long = proximate, short = PROX}
\DeclareAcronym{dem}{long = demonstrative, short = DEM}
\DeclareAcronym{dur}{long = durative, short = DUR}
\DeclareAcronym{neg}{long = negative, short = NEG}

\newcommand*{\homo}[2]{#1$_{\text{#2}}$}

\newcommand{\cgel}{\href{../English/cambridge.pdf}{my notes about CGEL}}
\newcommand{\latin}{\href{../Latin/latin-notes.pdf}{my notes about Latin}}
\newcommand{\alignment}{\href{../alignment/alignment.pdf}{my notes about alignment}}
\newcommand{\exerciseone}{\href{../Exercise/2021-3.pdf}{this exercise}}
\newcommand{\method}{\href{../methodology/glossing.pdf}{this note about how descriptive grammars work}}

\newcommand{\ala}{à la}
\newcommand{\translate}[1]{`#1'}

\title{Mandarin morphosyntax reading note}
\author{Jinyuan Wu}

\begin{document}

\maketitle

This note is my reading note of \citet{zhudexigrammar}.
It can be seen as a preparation of \href{./main.pdf}{this book},
which is premature and possibly will never be finished,
especially by someone without systematic linguistic training like me.
Still, the theoretical orientation of this note is well introduced in the above link,
as well as in \cgel, \latin, and \method.
\citet{zhudexigrammar} is commonly referred to as a typical structuralist book of Chinese.
I do not say ``structuralist grammar'' because the book is also a textbook about structuralism,
mostly in Bloomfield's brand
and strikingly close to the \ac{cgel} \citep{cgel} approach summarized in the above notes,
with a lot of argumentation, more than what ordinary grammars contain.

\section{About Zhu's book}

\subsection{The object language}

The object language, ``Chinese'', needs some clarification.
It means Standard Modern Chinese or Standard Modern Mandarin,
often abbreviated as Mandarin in the English speaking world.
In mainland China it is called 普通话.
In Taiwan and Singapore it (with small variations) is called 国语.

Standard Mandarin -- like other languages -- is an evolving language.
Certain usages documented in Zhu's book have already been obsoleted.
% TODO: evolving speed

\subsection{Organization of chapters}

The book can be divided into several parts:
\begin{itemize}
    \item Chapters 1-6 are about morphology and lexical categories.
    Lexical categories discussed in details are either nominal or verbal.
    \item Chapters 7-10 together give a top-down analysis of syntactic constructions without coordination.
    Serial verb constructions are \emph{not} introduced in these chapters.
    \item Chapter 11 is about coordination.
    \item Chapter 12 is about serial verb construction.
    \item Chapter 13-14 are about prepositions and adverbs.
    \item Chapter 15 is about clause types.
    \item Chapter 16 is about \ac{sfp}.
    \item Chapter 17 is about clause linking without canonical coordination.
    \item Chapter 18 is about ellipsis and inversion, 
    which may be roughly said to be about information packaging. 
\end{itemize}

This organization is an example of \citesec{\ref{method-sec:chapter-organization}} in \method.
The relation between the first six chapters and the following four 
is the item and arrangement strategy relation.
Within the chapters 7-10,
we see the top-down partition of clauses and \acs{np}s 
introduced in \citesec{\ref{method-sec:clause-top-down}}
and \citesec{\ref{method-sec:np-top-down}} in \method.
This is typical in structuralist works:
it is a direct reflection of the top-down analysis of syntactic structures 
(see \citesec{\ref{method-sec:item-and-arrangement-grammar}} and \citesec{\ref{method-sec:clause-top-down}}
in \method).

The noun-verb distinction (\citesec{\ref{method-sec:np-clause-chapter}} in \method) 
is only reflected in 
nominal categories being introduced in \citechap{4},
while verbal categories being introduced in \citechap{5}.
The \ac{np} structure is introduced in \citechap{10},
together with their clausal counterparts.

% TODO: chapter 11 and 12

The relation between the first twelve chapters and chapters 13 and 14
is the relation between canonical constructions and their counterparts with adjunction.
The relation between the first fourteen chapters and \citechap{15} is 
the relation between canonical constructions and non-canonical ones
related to the former ones by transformation rules.

Chapters 7, 8, 9, 10, 13, and 14 constitute a system 
quite similar to the chapter 4-8 in \ac{cgel}:
first clausal complements, 
including the external complement -- the subject --
and internal complements,
then \ac{np}s,
then how the two are modified,
by adjectives and adverbs,
or by more complicated \acs{pp}s.

Chapter 16 actually can be placed before \citechap{13}.
This is not the order used in the book,
the reason of which, in generative terms,
seems to be that \ac{sfp}s are merged in higher projections than what is involved 
form \citechap{7} to \citechap{15}.
Zhu, however, regard most of \ac{sfp}s as a part of the predicate.
The contradiction between the arrangement of chapters 
and the explicit analysis of \ac{sfp}s as a part of the predicate in \citesec{16.1.1} 
in \citet{zhudexigrammar} 
will be discussed in % TODO

\subsection{Terminology}

The terminology used in the book is closer to the \ac{cgel} approach rather than the \ac{blt} approach.
It should be noted that the book is written in Mandarin Chinese,
in which certain linguistic terms 
do not have morpheme-to-morpheme counterparts in English or
already have different meaning than their morpheme-to-morpheme counterparts in English.

For lexical categories, 体词 means \translate{referential word} i.e. nominal words.
Its direct translation would be \translate{body-word},
which may be understood by some as \translate{content word} i.e. \translate{lexical word}.
The term 谓词 means \translate{verbal word}.
The direct translation would be \translate{commenting-word},
which may be understood as somehow ``predicative'' in the sense of predicative complements in \ac{cgel}.
This is not correct: 
谓词 means what can head a predicate,
thus verbs and adjectives in Chinese.
The term 实词 \translate{substantial word} means lexical words,
while 虚词 \translate{virtual word} means function words.

The term 谓语 means predicate in the \ac{cgel} sense.
The term 述语 means predicator in the \ac{cgel} sense.
Unlike earlier structuralist works which work in the vanilla phrase structure grammar (PSG) framework,
\citet{zhudexigrammar} uses a \ac{cgel}-like PSG,
where a label of a constituent in a larger construction 
contains both its category label and its function label,
for example both ``\ac{np}'' and ``subject''.
This idea is made explicit in \citesec{1.3.10}.
The analysis of 我们班有许多外国留学生 in \citesec{1.3.8} is a good example.
Unlike \ac{cgel},
\citet{zhudexigrammar} uses a more compact format 
in which constituents are illustrated by underlining 
to show the constituency tree.
This is, of course, merely a notational problem,
but somehow it becomes a tradition of the School Grammar analysis of Chinese.

\subsection{About this note}

This note try to rearrange the content of \citet{zhudexigrammar}
in a way that is both acceptable in the approaches in \ac{blt} and \ac{cgel}.
The order of this note is largely bottom-up.
Certain top-down analyses, of course, will be given in the grammar sketch chapter.

\section{Grammar sketch}

The first chapter in \citet{zhudexigrammar} may be thought as a grammar sketch chapter,
but it contains much discussion on theoretical issues
(replicating what is discussed in \ac{cgel} \citesec{1.4}).
This section is a more compact grammar sketch,
skipping theoretical commitments which can be found in sources at the beginning of this note.
Chapter 3 is also a short one and may be regarded as a part of the grammar sketch.

I will roughly follow \citet{jacques2021grammar} in the organization of this section.
However, since in Chinese, dependency relations are not mainly coded by morphology,
I will replace the ``nominal morphology'' section by ``noun phrase''
and replace the ``verbal morphology'' section by ``clause structure'',
and do not give constituent order a special section,
since constituent order is covered by the constituency structure.
This is a major difference between \ac{cgel}-like ``structuralist'' grammars 
and \ac{blt}-based ``functionalist'' grammars
(\citesec{\ref{method-sec:theory}} in \method).

\subsection{Parts of speech}

Since Chinese does not rich grammatical relation-bearing morphology,
purely syntactic tests play the major role in determining parts of speech.
Semantics may help but is never decisive 
(\citesec{3.1.1} and \citesec{3.1.2} in \citet{zhudexigrammar}).
The word class division given in the book inevitably meets the problem 
that a word may belong to two categories depending on the context.%
\footnote{
    Formally, we may say the word prototypically belong to one category,
    and its usage as a word in another category
    involves zero derivation or conversion.
    But here we are faced with the same problem that 
    urged linguists to give up transformational rules:
    
    The most appropriate term for this process:
    zero-derivation, conversion or something else is still debated,
    and I will skip this topic in this note.
}
The analysis adopted here mini

\section{Morphology}

\subsection{Morphemes}

\subsection{Duplication}

\subsection{Compounding}

\subsection{What is a word?}

\section{Sentence final particles}

\bibliographystyle{plainnat}
\bibliography{references/grammars.bib,references/controversy.bib,references/generative.bib,references/aspects.bib,references/general-typology.bib}

\end{document}