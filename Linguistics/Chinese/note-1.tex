\documentclass[UTF8, a4paper, oneside, scheme=plain]{ctexrep}

\usepackage{geometry}
\usepackage{float}
\usepackage{titling}
\usepackage{titlesec}
\usepackage{paralist}
\usepackage{footnote}
\usepackage{enumerate}
\usepackage{amsmath, amssymb, amsthm}
\usepackage{gb4e}
\noautomath
\usepackage{bbm}
\usepackage{soul}
\usepackage{graphicx}
\usepackage{siunitx}
\usepackage[table,xcdraw]{xcolor}
\usepackage{tikz}
\usepackage[ruled, vlined, linesnumbered, noend]{algorithm2e}
\usepackage{xr-hyper}
\usepackage[colorlinks, citecolor = purple]{hyperref} % linkcolor=black, anchorcolor=black, citecolor=black, filecolor=black
\usepackage[most]{tcolorbox}
\usepackage{caption}
\usepackage{subcaption}
\usepackage{booktabs}
\usepackage{multirow}
\usepackage[figuresright]{rotating}
\usepackage{acro}
\usepackage[round]{natbib} 
\usepackage{nameref,zref-xr}
\zxrsetup{toltxlabel}
\zexternaldocument*[cgel-]{../English/cambridge}[cambridge.pdf]
\zexternaldocument*[alignment-]{../alignment/alignment}[alignment.pdf]
\zexternaldocument*[exercise1-]{../Exercise/2021-3}[2021-3.pdf]
\zexternaldocument*[method-]{../methodology/glossing}[glossing.pdf]
\usepackage{prettyref}

\geometry{left=3.18cm,right=3.18cm,top=2.54cm,bottom=2.54cm}
\titlespacing{\paragraph}{0pt}{1pt}{10pt}[20pt]
\setlength{\droptitle}{-5em}

\DeclareMathOperator{\timeorder}{\mathcal{T}}
\DeclareMathOperator{\diag}{diag}
\DeclareMathOperator{\legpoly}{P}
\DeclareMathOperator{\primevalue}{P}
\DeclareMathOperator{\sgn}{sgn}
\newcommand*{\ii}{\mathrm{i}}
\newcommand*{\ee}{\mathrm{e}}
\newcommand*{\const}{\mathrm{const}}
\newcommand*{\suchthat}{\quad \text{s.t.} \quad}
\newcommand*{\argmin}{\arg\min}
\newcommand*{\argmax}{\arg\max}
\newcommand*{\normalorder}[1]{: #1 :}
\newcommand*{\pair}[1]{\langle #1 \rangle}
\newcommand*{\fd}[1]{\mathcal{D} #1}

\newcommand*{\citesec}[1]{\S~{#1}}
\newcommand*{\citechap}[1]{chap.~{#1}}
\newcommand*{\citefig}[1]{Fig.~{#1}}
\newcommand*{\citetable}[1]{Table~{#1}}
\newcommand*{\citepage}[1]{pp.~{#1}}
\newcommand*{\citefootnote}[1]{fn.~{#1}}

\newrefformat{sec}{\citesec{\ref{#1}}}
\newrefformat{fig}{\citefig{\ref{#1}}}
\newrefformat{tbl}{\citetable{\ref{#1}}}
\newrefformat{chap}{\citechap{\ref{#1}}}
\newrefformat{fn}{\citefootnote{\ref{#1}}}
\newrefformat{box}{Box~\ref{#1}}

\usetikzlibrary{arrows,shapes,positioning}
\usetikzlibrary{arrows.meta}
\usetikzlibrary{decorations.markings}
\tikzstyle arrowstyle=[scale=1]
\tikzstyle directed=[postaction={decorate,decoration={markings,
    mark=at position .5 with {\arrow[arrowstyle]{stealth}}}}]
\tikzstyle ray=[directed, thick]
\tikzstyle dot=[anchor=base,fill,circle,inner sep=1pt]


\tcbuselibrary{skins, breakable, theorems}

\newtcbtheorem[number within=chapter]{infobox}{Box}{
    enhanced,
    boxrule=0pt,
    colback=blue!5,
    colframe=blue!5,
    coltitle=blue!50,
    borderline west={4pt}{0pt}{blue!65},
    sharp corners,
    fonttitle=\bfseries, 
    breakable,
    before upper={\parindent15pt\noindent}}{box}
\newtcbtheorem[number within=chapter, use counter from=infobox]{theorybox}{Box}{
    enhanced,
    boxrule=0pt,
    colback=orange!5, 
    colframe=orange!5, 
    coltitle=orange!50,
    borderline west={4pt}{0pt}{orange!65},
    sharp corners,
    fonttitle=\bfseries, 
    breakable,
    before upper={\parindent15pt\noindent}}{box}
\newtcbtheorem[number within=chapter, use counter from=infobox]{learnbox}{Box}{
    enhanced,
    boxrule=0pt,
    colback=green!5,
    colframe=green!5,
    coltitle=green!50,
    borderline west={4pt}{0pt}{green!65},
    sharp corners,
    fonttitle=\bfseries, 
    breakable,
    before upper={\parindent15pt\noindent}}{box}

\newcommand*{\concept}[1]{\textbf{#1}}
\newcommand*{\term}[1]{\emph{#1}}
\newcommand{\corpus}[1]{\emph{#1}}

\DeclareAcronym{blt}{short = BLT, long = Basic Linguistic Theory}
\DeclareAcronym{cgel}{short = CGEL, long = The Cambridge Grammar of the English Language}
\DeclareAcronym{dm}{short = DM, long = Distributed Morphology}
\DeclareAcronym{tag}{long = Tree-adjoining grammar, short = TAG}
\DeclareAcronym{sfp}{long = sentence final particle, short = SFP}
\DeclareAcronym{np}{long = noun phrase, short = NP}
\DeclareAcronym{vp}{long = verb phrase, short = VP}
\DeclareAcronym{pp}{long = preposition phrase, short = PP}
\DeclareAcronym{cls}{long = classifier, short = CLS}
\DeclareAcronym{dist}{long = distal, short = DIST}
\DeclareAcronym{prox}{long = proximate, short = PROX}
\DeclareAcronym{dem}{long = demonstrative, short = DEM}
\DeclareAcronym{quant}{long = quantifier, short = QNT}
\DeclareAcronym{dur}{long = durative, short = DUR}
\DeclareAcronym{neg}{long = negative, short = NEG}
\DeclareAcronym{cc}{long = copular complement, short = CC}
\DeclareAcronym{cs}{long = copular subject, short = CS}
\DeclareAcronym{tame}{long = {tense, aspect, mood, evidentiality}, short = TAME}
\DeclareAcronym{past}{long = past, short = PST}
\DeclareAcronym{nonpast}{long = non-past, short = NPST}
\DeclareAcronym{present}{long = present, short = PRES}
\DeclareAcronym{progressive}{long = progressive, short = PROG}
\DeclareAcronym{perfect}{long = perfect, short = PERF}
\DeclareAcronym{passive}{long = passive, short = PASS}
\DeclareAcronym{copula}{long = copula, short = COP}
\DeclareAcronym{polite}{long = polite, short = POL}
\DeclareAcronym{possessive}{long = possessive, short = POSS}

\newcommand*{\homo}[2]{#1$_{\text{#2}}$}

\newcommand{\cgel}{\href{../English/cambridge.pdf}{my notes about CGEL}}
\newcommand{\latin}{\href{../Latin/latin-notes.pdf}{my notes about Latin}}
\newcommand{\alignment}{\href{../alignment/alignment.pdf}{my notes about alignment}}
\newcommand{\exerciseone}{\href{../Exercise/2021-3.pdf}{this exercise}}
\newcommand{\method}{\href{../methodology/glossing.pdf}{this note about my understanding of descriptive grammars}}

\newcommand{\ala}{à la}
\newcommand{\translate}[1]{`#1'}
\newcommand{\vP}{\textit{v}P}

% Make subsubsection labeled
\setcounter{secnumdepth}{4}
\setcounter{tocdepth}{4}
% reset example counter every chapter (but do not include the chapter number to the label)
\counterwithin{exx}{chapter} 

\renewcommand{\bibname}{References}

\title{Mandarin Chinese notes}
\author{Jinyuan Wu}

\begin{document}

\maketitle

\automath

\chapter{Overview}

\section{Historical notes}

\section{Prosody}

One distinct feature of Mandarin is its morphosyntax relies strongly on \emph{prosody} \citep{feng2000}. 

\section{Parts of speech}

\section{Clause structure}

\subsection{Alignment}

Mandarin is an typical accusative language.
Mandarin clauses have a rather rigid constituent order:
It's usually classified as having a SVO clausal constituent order,
and the subject and the object(s) can be told from the positions in the clause 
(\ref{ex:get-sick}, \ref{ex:svo-example}).
Certain ``SOV'' orders can be obtained by invoking the disposal construction
(\prettyref{sec:disposal-construction}), as in \eqref{ex:ba-example}.

\begin{exe}
    \ex \gll 我 生病 了 \\
    1 get.sick \acs{sfp} \\
    \glt \translate{I got sick.}
    \label{ex:get-sick}

    \ex \gll [我]_{\text{subject}} 今天 去 看 [电影]_{\text{object}} 了 \\
    1 today to watch movie \acs{sfp} \\
    \glt \translate{I went to watch a movie today.} 
    \label{ex:svo-example}

    \ex \gll [我]_{\text{subject}} 今天 把 [ 一 个 碗 ]_{\text{object}} 摔 碎 了 \\
    1 today BA {} one \acs{quant} bowl {} break crack \acs{sfp} \\
    \glt \translate{I broke one bowl today.}
    \label{ex:ba-example}
\end{exe}

The usual tests of syntactic accusative alignment can be run on Mandarin
(\ref{ex:inter-sentence}).

\begin{exe}
    \ex \gll 陈 经理 昨天 没有 和 他的 客户 聊 过 。 他 生病 了 。 \\
    {Chen (surname)} manager yesterday \acs{neg} with 3sg-\acs{possessive} client talk \acs{sfp}
    {} 3sg get.sick \acs{sfp} \\
    \glt \translate{Manager Chen didn't talk with his client yesterday. He (Chen, not his client) got sick.}
    \label{ex:inter-sentence}
\end{exe}

\subsection{\acs{tame} categories}

Mandarin lacks the category of tense -- 
all tense information is expressed by time adverbs.
Modality is marked similarly be adverbs or complement clause constructions.
Yet there is a system marking the aspect (\prettyref{sec:aspectual}). 
\eqref{ex:quguo-qule} is an example.

\begin{exe}
    \ex \begin{xlist}
        \ex \gll 我 去 过 上海 了 \\
        1 go GUO Shanghai \acs{sfp} \\
        \glt \translate{I have been in Shanghai.}
        \ex \gll 我 去 了 上海 了 \\
        1 go LE Shanghai \acs{sfp} \\
        \glt \translate{I have gone to Shanghai.}
    \end{xlist}
    \label{ex:quguo-qule}
\end{exe}

\subsection{Negation}

Like the case in standard English, 
there is no negative concord in Mandarin Chinese.
There is, however, no uniform negation operator like the English \emph{not}. 
Several negation operators and strategies are used frequently (\prettyref{sec:negation}).
Verbs can be negated by 不 while nouns generally cannot, 
and this is a criterion to tell verbs from nouns. 
There is another negation operator 没, 
which has subtle differences in its meaning and syntactic properties compared with 不
(\ref{ex:chiqincai}, \ref{ex:buchi-meichi}).
On the other hand, the negative potential complement construction,
i.e. the V不了 construction,
isn't obtained by inserting a negator in the clause \eqref{ex:zuobuliao-example}.

\begin{exe}
    \ex \begin{xlist}
        \ex \gll 我 不 喜欢 吃 芹菜 \\
        1 \acs{neg} like eat celery \\
        \glt \translate{I don't like eating celery.} \\
        \ex * 我 没 喜欢 吃 芹菜
    \end{xlist}
    \label{ex:chiqincai}
\end{exe}

\begin{exe}
    \ex \begin{xlist}
        \ex \gll 我 不 吃 早饭 \\
        1 \acs{neg} eat breakfast \\
        \glt \translate{I don't eat breakfast. (I usually don't, I don't want any today, etc.)}
        \ex \gll 我 没 吃 早饭 \\
        1 \acs{neg} eat breakfast \\
        \glt \translate{I didn't eat breakfast. (I may usually do, but somehow I didn't today.)}
    \end{xlist}
    \label{ex:buchi-meichi}
\end{exe}

\begin{exe}
    \ex \begin{xlist}
        \ex 我做 [不了]_{\text{potential complement, negative}} 这件事。
        \ex[*]{我\{没有/并非/不\}_{\text{negative operator}} 做 [得了]_{\text{potential complement, positive}} 这件事。}
    \end{xlist}    
    \label{ex:zuobuliao-example}
\end{exe}

\subsection{The topic-comment structure}



\section{Clause combining}

\section{Remarkable features}

\subsection{Serializing}

It's often said Mandarin is a serializing language:
it contains 
A closer look, however, 

\chapter{Prosody and the writing system}

\chapter{Parts of speech}

\chapter{Nominal derivation}

\chapter{The structure of noun phrase}

\chapter{The verbal complex}

Mandarin is generally regarded as a prototypical analytic language,
without traditionally acknowledged verb inflections.
Indeed it will be weird to posit something like a paradigm in Mandarin,
but it doesn't mean there is no such thing as verbal affixation that are active
in the morphosyntax 
(instead of not fully productive and arguably historical derivations).
Some items involved here however may have partial mobility.
Consider \eqref{ex:movable-suffix}:
In the first sentence, 
了 is an aspectual suffix (\prettyref{sec:aspectual}),
while 走 is a verb which never appear without an argument in uncontroversial phrasal grammar.
So we conclude 了 and 走 are suffixes,
and by structural comparison, 
we conclude 过来 in \eqref{ex:sanpinqishuiugolai-1} 
is also a suffix, with the same status as 走.
But there comes \eqref{ex:sanpinqishuiugolai-2},
in which 过来 moves to the end of the sentence.

\begin{exe}
    \ex \begin{xlist}
        \ex \gll 他 带 走 了 他的 文件  \\ 
        3sg carry go.away \acs{perfect} 3sg-\acs{possessive} file \\
        \glt \translate{He carried his files away.}
        \ex \gll 他 带 [过来] 了 三 瓶 汽水 \\
        3sg carry come \acs{perfect} three bottle.\acs{quant} soda \\
        \glt \translate{He carried here three bottles of soda.} 
        \label{ex:sanpinqishuiugolai-1}
        \ex 他带了三瓶汽水[过来]
        \label{ex:sanpinqishuiugolai-2}
    \end{xlist}
    \label{ex:movable-suffix}
\end{exe}

To avoid the useless quarrelling about what is a word and whether a grammar point is morphology
(which isn't that important in non-lexicalist generative theories, anyway),
I use the term \term{verbal complex} to cover 
the main verb and the  ``suffixes'' in \eqref{ex:movable-suffix}.
There are roughly three systems in the verbal complex.
The first is the uncontroversial derivation system,
like 化 \translate{-ize}.
The second is the verbal complement system,
which includes three subsystems:
the resultative complements, the directional complements, 
and the potential complements (\prettyref{sec:verbal-complement}).
The third is the aspectual system (\prettyref{sec:aspectual}).

\begin{infobox}{On the notion of \term{complements}}{complement-name}
    The Chinese term 补语 corresponding to my \term{verbal complement}
    is frequently translated into the English term \term{complement}.
    This creates some confusion,
    because the term \term{complement} can also denote 
    clausal dependents that are arguments of the main verb, as in \citet{cgel}.
    The term \term{non-argument complement} may be used to avoid this confusion.
    There are, however, further confusions:
    Should we regard a clausal dependent that records the quantity or amount of an action 
    as a non-argument complement?
    This construction can also be seen in Latin, 
    like the Latin accusative expression of time \citep[\citesec{423}]{greenough2013allen}.
    Thus, I use the term \term{verbal complement} to refer to 
    things like 完 as in 做完了.
\end{infobox}

\eqref{ex:hua-wan-le-1} is an example in which 
all the three systems appear.
In real world speeches, such combinations have relatively lower distributions,
possibly because of the prosodic constraint 
that verb shouldn't be too heavy unless it appears at the end of a clause
(TDOO: ref).

\begin{exe}
    \ex \dots 并且企业 [数字 [化]_{\text{derivation}} [完]_{\text{complement}} [了]_{\text{aspectual}}]_{\text{V}} 之后还不一定赚钱 \dots
    \label{ex:hua-wan-le-1}
\end{exe}

You may note the so-called serial verb constructions aren't mentioned here.
\citet{paul2008serial} and \citet[\citesec{9.4}]{deng2010formal} 
summarizes several constructions that are
frequently referred to as serial verb constructions,
and points out after deeper investigation,
they can all be described in terms of the usual complement clause constructions,
purpose clause constructions, etc. 
that are well attested cross-linguistically.

\section{Verbal derivations}

\section{Verbal complements}\label{sec:verbal-complement}

\section{The aspectual system}\label{sec:aspectual}

\section{Separable verbs}

It's sometimes possible to split a verb and inject some clausal dependents into it.
The interaction between this separation operation and the structure of the verbal complex is of some interest.

\chapter{Verb and arguments}

\chapter{Valency changing}

There are two ways of valency changing in Mandarin.
The first is via a coverb construction, 
as in the disposal constructions (\prettyref{sec:disposal-construction}),
TODO 
The second is \emph{doing nothing} to the verb 
and relying on the unusual semantic roles of clausal complements 
to inform the listener about the valency changing,
as in TODO: ref.
Since there is no morphological marking,
constructions of this type are often recognized as topic-comment structures,
in which the ``topic'' -- which is the subject under closer investigation -- 
is said to be freely occupied by any semantic (and not necessarily syntactic) argument in the clause,
though this claim can be falsified by detailed syntactic tests (\prettyref{box:topic-subject}).

\section{The disposal constructions}\label{sec:disposal-construction}

\chapter{Simple clauses}

\section{Overview}

A sentence can be divided into several clauses linked by clause linking constructions 
(\prettyref{chap:clause-linking}).
A clause can be divided into
one or more topics (if any) and a comment,
the latter being the nucleus clause,
and possibly \ac{sfp}s.
The comment may further be divided into a subject (if any),
a series of adverbials, 
the verbal complex, and post-verbal constituents,
the most important types including object(s), 
the second part of a separable verb,
certain directional complements,
and purpose clauses.
This chapter is denoted to everything on the clause level,
postponing details in subordination and clause linking to the next several chapters.

\begin{infobox}{The term \term{clause}}{clause-def}
    Some people, like \citet[\citepage{140}]{deng2010formal}
    as well as \citet{dixon2009basic},
    use the term \term{clause} for subject-predict constructions 
    that don't receive complete marking of speech forces.
    (In generative terms, \term{clause} is for lower level CPs or even TPs.)
    So in this way, \acs{sfp}s shouldn't be discussed in this chapter because 
    they are of course dependents in the sentence level.
    They may be discussed together with other sentence-level constructions like \prettyref{chap:clause-linking}.
    But this notion of clause certainly goes against the tradition in descriptive grammars.
    So the approach of this note is to acknowledge everything larger than TP as a clause,
    which may or may not be a sentence,
    and discuss its structure in this chapter,
    while ``adjunctions'' -- or in other words, optional dependents -- 
    are discussed in, say, \prettyref{chap:clause-linking},
    for the sake of convenience.
    The narrative order of this note is not the ideal ``small unit -- large unit'' scheme,
    but the ``simple large unit -- complicated large unit'' scheme.
    Needless to say,
    when it comes to clause combining, 
    the problem of what the clause really is -- with or without \ac{sfp}s, for example --
    is still relevant,
    but it is not answered by saying ``the construction takes a clause, not a sentence''.
\end{infobox}

\section{The topic-comment structure}

I follow \citet{sih2000topic}'s approach and define a topic as an unmarked \acs{np} 
that has certain relations with a position in the clause after it
and is indeed the topic in the information structure
(i.e. some (probably already known) object to which new information is added).
Constructions like 连\dots都\dots are not discussed in this section -- 
they are to be found in TODO: ref.

\begin{infobox}{Rejection of the notion of dangling topics}{topic-subject}
    Some people, like \citet[\citesec{7.1}]{zhudexigrammar},
    equate \term{subject} with \term{topic} in Mandarin grammar.
    Some (especially those from the functional-typological tradition) go further 
    and assert that ``the notion of the subject (as the position of the most agentive argument) 
    isn't grammaticalized in Mandarin Chinese'',
    and therefore the topic is just an \acs{np} which the comment is ``about''.
    This view is rejected in this note,
    because such accounts usually end up in severe overgeneration. 

    \paragraph*{Type 1} In the first type of ``dangling topic'',
    it's impossible for any \acs{np} in the comment to be syntactically related to the topic.
    Such cases are however rather unproductive. 
    In \eqref{ex:dayuchixiaoyu} and \eqref{ex:nikankanwo},
    the orders of the constituents can never be changed.
    Nor is it possible to change a word or two in the bracketed ``comments''.
    A reasonable assumption is these bracketed ``comments''
    are actually idioms, 
    which are to be regarded as a single verbal element that can't be further analyzed.

    \begin{exe}
        \ex\label{ex:dayuchixiaoyu} 他们[大鱼吃小鱼](,厮杀成一片)
        \ex\label{ex:nikankanwo} 他们[你看看我我看看你]
    \end{exe}

    \paragraph*{Type 2} 
\end{infobox}

\subsection{Preposition objects}

\begin{exe}
    \ex 这件事你不能就麻烦他一个人
    \ex 你不能[在这件事上]_{\text{adverbial:\acs{pp}}} 就麻烦他一个人
\end{exe}
This is also a demonstration of the preposition status of 在 in this sentence,
because if it's a verb or an auxiliary verb,
it will be hard to have its object topicalized and have it deleted at the same time,
but deletion of the preposition in topicalization is well-attested cross-linguistically.

\section{Negation}\label{sec:negation}

\section{Sentence final particles}


\chapter{Subordination}

\section{Overview}



\begin{infobox}{Non-existence of finite-nonfinite distinction in Mandarin}{finiteness}
    Cross-linguistically, we find a finite-nonfinite distinction in subordination.
    This distinction is arguably absent in Mandarin,
    even after detailed syntactic tests \citep{no-finite}.
\end{infobox}

\chapter{Clause linking}\label{chap:clause-linking}

\bibliographystyle{plainnat}
\bibliography{references/grammars,references/aspects,references/general-typology}

\end{document}