\documentclass[UTF8, a4paper, oneside, scheme=plain]{ctexrep}

\usepackage{libertinus}
\usepackage{geometry}
\usepackage{float}
\usepackage{titling}
\usepackage{titlesec}
\usepackage{paralist}
\usepackage{footnote}
\usepackage{enumerate}
\usepackage{amsmath, amssymb, amsthm}
\usepackage{gb4e}
\noautomath
\usepackage{bbm}
\usepackage{textcomp}
\usepackage{soul}
\usepackage{graphicx}
\usepackage{siunitx}
\usepackage[table,xcdraw]{xcolor}
\usepackage{tikz}
\usepackage[ruled, vlined, linesnumbered, noend]{algorithm2e}
\usepackage{xr-hyper}
\usepackage[colorlinks, citecolor = purple]{hyperref} % linkcolor=black, anchorcolor=black, citecolor=black, filecolor=black
\usepackage[most]{tcolorbox}
\usepackage{caption}
\usepackage{subcaption}
\usepackage{booktabs}
\usepackage{multirow}
\usepackage[figuresright]{rotating}
\usepackage{acro}
\usepackage[round]{natbib} 
\usepackage{nameref,zref-xr}
\zxrsetup{toltxlabel}
\zexternaldocument*[cgel-]{../English/cambridge}[cambridge.pdf]
\zexternaldocument*[alignment-]{../alignment/alignment}[alignment.pdf]
\zexternaldocument*[exercise1-]{../Exercise/2021-3}[2021-3.pdf]
\zexternaldocument*[method-]{../methodology/glossing}[glossing.pdf]
\usepackage{prettyref}

\geometry{left=3.18cm,right=3.18cm,top=2.54cm,bottom=2.54cm}
\titlespacing{\paragraph}{0pt}{1pt}{10pt}[20pt]
\setlength{\droptitle}{-5em}

\DeclareMathOperator{\timeorder}{\mathcal{T}}
\DeclareMathOperator{\diag}{diag}
\DeclareMathOperator{\legpoly}{P}
\DeclareMathOperator{\primevalue}{P}
\DeclareMathOperator{\sgn}{sgn}
\newcommand*{\ii}{\mathrm{i}}
\newcommand*{\ee}{\mathrm{e}}
\newcommand*{\const}{\mathrm{const}}
\newcommand*{\suchthat}{\quad \text{s.t.} \quad}
\newcommand*{\argmin}{\arg\min}
\newcommand*{\argmax}{\arg\max}
\newcommand*{\normalorder}[1]{: #1 :}
\newcommand*{\pair}[1]{\langle #1 \rangle}
\newcommand*{\fd}[1]{\mathcal{D} #1}

\newcommand*{\citesec}[1]{\S~{#1}}
\newcommand*{\citechap}[1]{chap.~{#1}}
\newcommand*{\citefig}[1]{Fig.~{#1}}
\newcommand*{\citetable}[1]{Table~{#1}}
\newcommand*{\citepage}[1]{pp.~{#1}}
\newcommand*{\citefootnote}[1]{fn.~{#1}}

\newrefformat{sec}{\citesec{\ref{#1}}}
\newrefformat{fig}{\citefig{\ref{#1}}}
\newrefformat{tbl}{\citetable{\ref{#1}}}
\newrefformat{chap}{\citechap{\ref{#1}}}
\newrefformat{fn}{\citefootnote{\ref{#1}}}
\newrefformat{box}{Box~\ref{#1}}
\newrefformat{ex}{\ref{#1}}

% color boxes

\tcbuselibrary{skins, breakable, theorems}

\newtcbtheorem[number within=chapter]{infobox}{Box}{
    enhanced,
    boxrule=0pt,
    colback=blue!5,
    colframe=blue!5,
    coltitle=blue!50,
    borderline west={4pt}{0pt}{blue!65},
    sharp corners,
    fonttitle=\bfseries, 
    breakable,
    before upper={\parindent15pt\noindent}}{box}
\newtcbtheorem[number within=chapter, use counter from=infobox]{theorybox}{Box}{
    enhanced,
    boxrule=0pt,
    colback=orange!5, 
    colframe=orange!5, 
    coltitle=orange!50,
    borderline west={4pt}{0pt}{orange!65},
    sharp corners,
    fonttitle=\bfseries, 
    breakable,
    before upper={\parindent15pt\noindent}}{box}
\newtcbtheorem[number within=chapter, use counter from=infobox]{learnbox}{Box}{
    enhanced,
    boxrule=0pt,
    colback=green!5,
    colframe=green!5,
    coltitle=green!50,
    borderline west={4pt}{0pt}{green!65},
    sharp corners,
    fonttitle=\bfseries, 
    breakable,
    before upper={\parindent15pt\noindent}}{box}

\newcommand*{\concept}[1]{\textbf{#1}}
\newcommand*{\term}[1]{\emph{#1}}
\newcommand{\corpus}[1]{\emph{#1}}

\newcommand{\redp}{\textasciitilde}

\DeclareAcronym{blt}{short = BLT, long = Basic Linguistic Theory}
\DeclareAcronym{cgel}{short = CGEL, long = The Cambridge Grammar of the English Language}
\DeclareAcronym{dm}{short = DM, long = Distributed Morphology}
\DeclareAcronym{tag}{long = Tree-adjoining grammar, short = TAG}
\DeclareAcronym{sfp}{long = sentence-final particle, short = \textsc{sfp}}
\DeclareAcronym{np}{long = noun phrase, short = NP}
\DeclareAcronym{vp}{long = verb phrase, short = VP}
\DeclareAcronym{pp}{long = preposition phrase, short = PP}
\DeclareAcronym{cls}{long = classifier, short = CLS}
\DeclareAcronym{dist}{long = distal, short = DIST}
\DeclareAcronym{prox}{long = proximate, short = PROX}
\DeclareAcronym{dem}{long = demonstrative, short = DEM}
\DeclareAcronym{classify}{long = classifier, short = \textsc{cl}}
\DeclareAcronym{dur}{long = durative, short = DUR}
\DeclareAcronym{neg}{long = negative, short = \textsc{neg}}
\DeclareAcronym{cc}{long = copular complement, short = CC}
\DeclareAcronym{cs}{long = copular subject, short = CS}
\DeclareAcronym{tame}{long = {tense, aspect, mood, evidentiality}, short = TAME}
\DeclareAcronym{past}{long = past, short = PST}
\DeclareAcronym{nonpast}{long = non-past, short = NPST}
\DeclareAcronym{present}{long = present, short = PRES}
\DeclareAcronym{progressive}{long = progressive, short = \textsc{poss}}
\DeclareAcronym{perfect}{long = perfect, short = \textsc{perf}}
\DeclareAcronym{passive}{long = passive, short = \textsc{pass}}
\DeclareAcronym{copula}{long = copula, short = COP}
\DeclareAcronym{possessive}{long = possessive, short = \textsc{poss}}

\newcommand{\asis}[1]{\textsc{#1}}
\newcommand{\oneof}[1]{{#1}}
\newcommand*{\homo}[2]{#1$_{\text{#2}}$}

\newcommand{\cgel}{\href{../English/cambridge.pdf}{my notes about CGEL}}
\newcommand{\latin}{\href{../Latin/latin-notes.pdf}{my notes about Latin}}
\newcommand{\alignment}{\href{../alignment/alignment.pdf}{my notes about alignment}}
\newcommand{\exerciseone}{\href{../Exercise/2021-3.pdf}{this exercise}}
\newcommand{\method}{\href{../methodology/glossing.pdf}{this note about my understanding of descriptive grammars}}

\newcommand{\ala}{à la}
\newcommand{\translate}[1]{`#1'}
\newcommand{\vP}{\textit{v}P}

% Make subsubsection labeled
\setcounter{secnumdepth}{4}
\setcounter{tocdepth}{4}
% reset example counter every chapter (but do not include the chapter number to the label)
\counterwithin{exx}{chapter} 

% Reference formats
\renewcommand{\bibname}{References}
\setcitestyle{aysep={}} 

\title{Mandarin Chinese notes}
\author{Jinyuan Wu}

\begin{document}

\maketitle

\automath

\chapter{Introduction}

\section{The language and the speakers}

Mandarin Chinese is a predominant language in the world,
belonging to the Sinitic family.
Indeed,  

\section{Previous studies and theoretical orientation of this work}

\subsection{Structuralist generative grammars}\label{sec:previous.structuralist}

There are already sufficient works concerning the grammar of Mandarin Chinese.
The tradition used in colleges (and occasionally high schools) 
is largely structuralist \ala{} Bloomfield:
A clause is divided into a topic and a comment,
and the comment is divided into a subject and a predicate,
and the predicate is divided into a predicator and an object, etc.
Examples of works in this tradition usually have names like 现代汉语 \translate{Modern Chinese}
\citep{xianhan2004}.
This tradition is still seen in many contemporary grammars,
like \citet{cgel}.
This tradition is largely coherent with the generative tradition,
and indeed there are books which are generative in essence 
but organized in the traditional and structuralist framework \citep{deng2010formal}.

\subsection{Teaching materials}

Teaching materials of Mandarin Chinese are also largely influenced by the structuralist tradition.
TODO

\subsection{Mandarin in the functional-typological tradition}

Mandarin also gains much attention in the functional-typological tradition.
\citet{li1989mandarin} is a ``functional'' reference grammar of Mandarin, TODO
The reason for Mandarin's popularity seems to be the fact that 
it breaks many previous typological generalizations about isolating languages and constituent orders
\citep[\citechap{8}]{paul2014new}.

\subsection{Theoretical commitment of this work}\label{sec:theory}

This notes attempts to reconcile the several approaches in previous researches.
It's my belief that the differences between the generative tradition, 
the traditional structuralist school 
and a large part of the so-called functionalist 
(or ``Basic Linguistic Theory'' \citep{dixon2009basic}) 
are mainly notational from a practical perspective.
Below, by the capitalized Generativism,
I mean a mixture of Minimalism, Distributed Morphology and Cartography
(but with less flavor of Antisymmetry),
which I mean are instructive on grammar description.
This leaves out Lexicalist traditions, 
due to the limit of space.
Indeed, what I want to do here 
is to incorporate the new perspectives in generative syntax, 
such as \cite{paul2014new} and \cite{paul2008serial},
into the structuralist tradition
in an accessible and typology-informed way.

Since modern generative syntax contains lots of hidden functional heads,
the notion of, say, a DP which is the specifier of T,
is to be replaced by an \acs{np} filling a subject position 
in a surface-oriented structuralist constituency analysis. 
Both ``\acs{np}'' and ``subject'' should be labeled on a sub-syntactic tree
in the structuralist tradition,
while in generative syntax,
the label ``subject'' is a secondary concept:
it's an abbreviation of SpecTP (or in an even more fine-grained way, SpecSubjP or SpecNomP).
So functional heads can be replaced by syntactic function labels.
On the other hand,
in Distributed Morphology we have roots,
which reside at the center of an extended verbal or nominal projection
(NP-NumP-DP, or vP-TP-CP),
and they are recognized as \term{heads} 
(in this note referred to as \concept{lexical heads})
in traditional structuralism.
This unifies the notation of
\citet{cgel}, \citet{chao1965grammar}, \citet{zhudexigrammar}
in the traditional structuralist perspective
and Generativism.

The \ac{blt} approach \citep{dixon2009basic}, 
or the grammar writing approach accepted in 
modern descriptive linguistics and typology,
is based on \emph{dependency relations} on the other hand.
It's also possible to formulate generative grammar in terms of dependency relations,
so it can be expected that the \acs{blt} approach is still largely equivalent 
with the grammatical complexity of generative grammar.
\acs{blt} only recognizes two major types of constituents:
\acs{np}s and clauses,
which are essentially \term{domains} or \term{fields} in generative syntax:
The former is the DP domain while the latter is the \vP{}-TP-CP domain.
The complicated binary constituency tree is replaced by a flat tree 
with lots of dependency relations labeled inside,
containing the same amount of information.
(We may also say the \acl{blt}'s standard of constituency 
is ``being a relatively independent construction''. 
Then the equivalence between this version of mild constructivism and minimalism 
is clear \citep{construction-minimalism}.)
Note that the \acl{blt} approach still (although quite implicitly) admits 
that there is a rank of ``closeness'' or ``height'' among dependency relations: 
the A argument seems to be somehow higher than the O argument, etc.,
and this information is conveyed by the fine-grained constituency relations 
in Generativism.

The focus on dependency relation in \acs{blt} also leads to 
a notational difference on what is a constituent.
A sequence like \corpus{has been exploring} is not a constituent in the sense of 
the structuralist constituency test in grammars like \citet{cgel} and generativism,
but are still recognized as a unit in \acs{blt} 
because its parts always appear together in the surface-oriented analysis.
This what is \citet[\citepage{109}]{dixon2009basic} defines 
as a \term{verb phrase},
which excludes the object. 
From the constituency-based point, 
such a ``phrase'' is usually a larger functional domain minus a smaller functional domain;
here it's the functional projection below the subject -- the verb phrase in \citet{cgel} --
minus the object.
This definition is meaningful 
because syntax is cyclic 
and the object is a DP and therefore is a phase, 
and above it is another phase, 
and although the introduction of the subject 
(by, say, a Nom head which may contain the nominative case) 
doesn't seal a phase, 
it still changes the properties of the syntactic tree, 
so what happen above the object and below the subject 
-- the \acs{tame} functional heads, the verb root, etc. -- 
are tightly linked to each other
(note that here we are just translating between 
the constituency-based analysis and the dependency-based analysis), 
so from a dependency relation-oriented analysis, 
the sequence \corpus{has been exploring} does qualify as a phrase; 
in a constituency grammar 
we recognize this sequence has a status 
but we don't call it a phrase or a constituent.
This difference is notational but may be confusing.

A further difference between the \acl{blt} and Generativism (as well as the structuralist tradition)
is the former is claimed to be semantic-based.
But first, of course it's possible to have a meaning-first version of generative syntax,
and second,
there seems to be a ``gluing'' layer between pure semantics and 
the phonetic realization 
and can't be equated with either of the two: 
the semantics of complement-taking verbs 
may be coded as a complement clause construction, 
a relative clause construction 
(compare \corpus{I see a man running} and \corpus{I see a running man}),
and superficially similar utterances may have different ``structures''.
This is recognized by Dixon, 
who distinguishes the ``prototypical'' coding strategy of a semantic concept 
and other strategies.
So there is also no substantial disagreement here.

The Generativism I take here doesn't emphasize on wordhood as a universal concept,
and indeed this is what I want here:
A word is simply a mini phrase (in the \acs{blt} sense),
either the realization of a mini constituency tree,
or the realization of a span of functional heads 
and possibly the lexical head or in other words the root.
The controversy about what is a word in Mandarin 
has been around for decades,
which I believe is due to the desire to 
find \emph{the} word as a universal unit,
without thinking of the tenet of Generativism 
that phonetic realization doesn't always 
transparently reflect in the syntax proper,
and that what is universal is likely to be 
prototypes of functional heads and how they are arranged together;
this means we may have categorizer phrases like nP or vP,
which can be recognized as minimal words,
but whether things like compound words are recognized as words or phrases 
depends more on grammatical traditions
and external factors like prosody.

\section{Plan of the book}

\subsection{Origin of data and how to represent them}

I will use less structuralist constituency trees compared with structuralist grammars.
That's to say, for example, 
the ``serial verb construction'' is not divided into two verb phrases 
following rigidly the structuralist tenets,
but is analyzed with a flat-tree structure instead,
following \acs{blt}.
But I will also do lots of in-depth morphosyntactic tests,
instead of just staring at the surface realization,
as many typologists may do.

\section{Remarkable features of Mandarin}

\subsection{Lack of morphology?}

Despite of lack of inflection
and lack of contextual alternation of morphemes,
Chinese does have some local and syntactically unmotivated operations
which are just like morphophonological rules,
although they don't necessarily operate on phrases.

An example of this is the verb copying phenomenon,
as in 看了一会书 (compare 看书了一会);
看书 \translate{to read} (intransitive; lit. \translate{to read books}) 
is a fossilized verb-object structure 
and this verb-object structure may still have synchronic effects.
More radical examples however also exist,
like \%体了一堂操, which is likely to be linked to 
[[体操]_{\text{noun-as-verb}}了[一堂]_{\text{time object}}]_{\text{\acs{vp}}}.
In casual speech,
verbs borrowed from other languages may also be split 
and the two fragments of the verb then surround the semi-object
(\prettyref{ex:remarkable.debug}).

\begin{exe}
    \ex\label{ex:remarkable.debug} 我 [debug]_{\text{topic:{\acs{vp}}}} [de不出来]_{\text{predicate:\acs{vp}}} [啊]_{\text{\acs{sfp}}}
\end{exe}

This phenomenon -- the verb being split and the time semi-object getting embedded into the verb -- 
looks just like infixing,
although here this infixing operation targets a \acs{vp} instead of a smaller unit.
This justifies the assumption taken at the end of \prettyref{sec:theory}
that there is no clear boundary between words and phrases 
and therefore syntax and morphology:
It's possible for a phrase to undergo 
rearrangement without clear syntax motivation
that usually happens within a word.

\subsection{The overwhelming influence of prosody}

One distinct feature of Mandarin is its morphosyntax relies strongly on \emph{prosody} \citep{feng2000}. 
Other components in phonology, strikingly, 
doesn't have much influence on Chinese morphosyntax,
and it will be largely skipped in this note.



\chapter{Grammatical overview}


\section{Parts of speech}

\section{The noun phrase}


\section{Clause structure}

\subsection{Alignment}



\subsection{\acs{tame} categories}

Mandarin lacks the category of tense -- 
all tense information is expressed by time adverbs.
Modality is marked similarly be adverbs or complement clause constructions.
Yet there is a system marking the aspect (\prettyref{sec:aspectual}). 
\eqref{ex:quguo-qule} is an example.

\begin{exe}
    \ex \begin{xlist}
        \ex \gll 我 去 过 上海 了 \\
        1 go \asis{guo} Shanghai \acs{sfp} \\
        \glt \translate{I have been in Shanghai.}
        \ex \gll 我 去 了 上海 了 \\
        1 go \asis{le} Shanghai \acs{sfp} \\
        \glt \translate{I have gone to Shanghai.}
    \end{xlist}
    \label{ex:quguo-qule}
\end{exe}

\subsection{Negation}



\subsection{The topic-comment structure}



\section{Clause combining}

Mandarin Chinese has usual clause linker devices (\prettyref{chap:clause-linking}),
as well as complement clauses (\prettyref{sec:complement-clause})
and relative clauses (TODO: ref). TODO: what else?

It's often said Mandarin is a serializing (i.e. with serial verb constructions) language.
A closer look, however, reveals this is not the case:
These constructions are either adverbial clause constructions 
or complement clause constructions, 
or maybe certain kind of light verb constructions (\prettyref{sec:no-serial-verb}).
The internal heterogeneity renders the term \term{serial verb construction} useless.

\chapter{Phonology and the writing system}

\chapter{Parts of speech}

\section{From morphemes to clauses: levels of units in Mandarin morphosyntax}

\subsection{The existence of words as mini-constituents}

\subsubsection{``Mini-constituent'' words}\label{sec:pos.architecture.word.mini-constituent}

A question causing endless controversy and confusion 
is ``what is a word''. 
\ac{blt} spends a whole chapter (\citechap{10}) on this topic.
It is often said that Chinese is ``character-based''
or to be precise, ``monosyllabic morpheme-based'',
with no level of grammatical words.
This claim is factually flawed, 
since in Chinese, there \term{are} distinction between 
productive morphemes and words.
What should be noted are the split between phonological words and grammatical words % TODO: ref
and the subtleties concerning word-phrase distinction. % TODO: 类似于幽了他一默这种形态变化导致词变成短语
These are introduced in the following sections.

The first piece of evidence for the existence of grammatical words in Chinese is 
there are disyllabic units in Chinese 
that have conventionalized meanings and its inner structure is invisible 
to any other morphosyntactic rules (except prosody).
The unit 白菜 is made up by two perfectly productive morphemes:
白 \translate{white} and 菜 \translate{vegetable},
but its meaning is not the composition of the two morphemes:
白菜 means \translate{Chinese cabbage}, not \translate{any vegetable with whitish appearance}.
The word has already gained a conventionalized meaning,
and its inner structure is of mostly diachronic interest but not synchronic interest.
Therefore, the disyllabic unit 白菜 is the smallest unit fed into morphosyntax,
and it of course is not a phrase.

Those insisting on the nonexistence of words in Chinese 
may explain the observation made above 
by claiming 白菜 to be an idiom \ac{np}:
it is indeed a lexical entry,
but is regarded as a pre-compiled phrase.
It then should be noted that 
certain grammatical relations seem to be not a part of \ac{np}s and clauses,
highlighting the necessity to introduce a smaller level of constituency.
Consider the following examples:
\begin{exe}
    \ex 
    \begin{xlist}
        \ex\label{ex:nominal-modifier-1} {} [[定义]_{\text{modifier:N}} [[等价]_{\text{complement:adjective}} [性]_{\text{nominalizer}}]_{\text{N}}]_{\text{N}} \translate{equivalence of definitions}
        \ex\label{ex:nominal-modifier-2} {} [[美国]_{\text{modifier:N}} [苹果]_{\text{head:N}}]_{\text{N}}
        \ex\label{ex:meiguo-hongse-pinguo} *[美国]_{\text{modifier:N}} [红色的苹果]_{\text{NP}}
        \ex\label{ex:hongse-meiguo-pinguo} 红色的美国苹果
    \end{xlist}
\end{exe}
From \eqref{ex:nominal-modifier-1} and \eqref{ex:nominal-modifier-2},
it can be seen in certain morphosyntax units,
a bare noun may serve as a (restrictive) modifier.
The constituent of Chinese \ac{np}s is Dem Num A N,
and this bare noun modifier position seems to be more internal than the adjective position,
as is illustrated by \eqref{ex:meiguo-hongse-pinguo} and \eqref{ex:hongse-meiguo-pinguo}.
Furthermore, the bare noun position cannot be filled by an \ac{np}.
The following examples demonstrate this:
\begin{exe}
    \ex \begin{xlist}
        \ex {} [联合国] [秘书长]
        \ex *[[某个组织] [秘书长]]
        \ex 某个组织的秘书长
    \end{xlist}
\end{exe}
The obligatoriness of 的 means the NP 某个组织 can only appear as a modifier via the possessive construction.
It cannot fill the slot of 美国 in 美国苹果.
So the bare noun modifier position is a function label existing in a unit smaller than the \ac{np}
-- and it has to be the word.

Therefore, the term \term{grammatical word} is a useful descriptive concept when it comes to Chinese
at least for some parts of the grammar.

\subsubsection{Fossilized phrasal structures}\label{sec:pos.morpheme-to-clause.fossilized}

One peculiar feature of Chinese is 
grammatical relations in morphology often have syntactic counterparts,
with the same constituent order.
Verbs, for example, may have inner predicator-object structures.
The verb 关心 \translate{care for} is certainly analyzable 
as a predicator-object structure,
but it takes objects just like any other verbs:
\begin{exe}
    \ex 他 [[[关]_{\text{predicator:V}} [心]_{\text{object:N}}]_{\text{predicator:V}} [自己的家人]_{\text{object:NP}}]_{\text{predicate:VP}}
\end{exe}
This means 关心 is not a VP but a grammatical word,
or otherwise it is impossible to take another object since there is no valency changing device in use.
The verb 关心 itself poses no threat to the word-phrase distinction,
but certain examples of these ``words with internal syntax'' can be tested to be words,
while they indeed have phrasal counterparts.
Compare the two examples below:
\begin{exe}
    \ex 
    \begin{xlist}
        \ex\label{ex:nianfo-tang} {} [念佛] 堂 
        \ex\label{ex:nianfo-split} 老太太 [念了这么久佛]_{\text{VP1}},却不知道自己在[念哪一尊佛]_{\text{VP2}}
    \end{xlist}
\end{exe}
The verb 念佛 in the first example is similar to 美国 in \eqref{ex:nominal-modifier-2}:
it serves as a bare modifier 
(since in Chinese verbal constituents can fill argument slots directly,
the fact that 念佛 is a verb is not surprising).
The fact 念佛 is able to appear in such a position assures us that 
it is a word.
Then consider \eqref{ex:nianfo-split}.
In VP1, a temporal semi-object is injected between the verb 念 and the object 佛,
while in VP2, an interrogative phrase 哪一尊 is inserted into 佛 
and an \ac{np} object is now taken by the verb 念.
So here is the problem:
what if 念佛 is \emph{always} a VP,
and what \eqref{ex:nianfo-tang} demonstrates is 
the bare noun (or verb) modifier may be sometimes filled by a phrase?

One solution -- the solution taken by \citet[\citesec{1.2.6}]{zhudexigrammar} -- 
is to regard 念佛 in \eqref{ex:nianfo-tang} as a word,
while the two \acs{vp}s in \eqref{ex:nianfo-split} as phrases.
念佛 as in 老太太经常念佛 can be interpreted as a word or as a phrase
without making any difference.
念佛 as a word is something like \corpus{Buddha-praying},
while 念佛 as a verb is something like \corpus{pray to Buddha}.
In the account of \citet[\citepage{82}]{feng2000},
念佛 is a morphosyntactic word
(the original term being a 句法词 \translate{syntactic word}),
which is created by morphosyntactic rules 
and has a inner structure that is (partially) transparent 
for other morphosyntactic rules,
while 关心 is a \translate{lexical word} 
(not the same with \term{lexical word} in the rest of this note),
which is taken out of the lexicon directly.

\begin{theorybox}{No generative rule in the lexicon}{no-generative-rule-lexicon}
    Those insisting on a universal word-phrase distinction 
    may say ``it is assembled in the lexicon before syntax''.
    The position of mine is if something is assembled synchronically,
    then it has to have something to do with syntax:
    syntax is the only productive engine.
    If a morphological device is completely invisible to the rest of the grammar,
    it is likely to have lost productivity and becomes historical.
\end{theorybox}

Another place where the distinction between words and phrases are subtle is 
the non-argument complement construction.
What is the status of 爬上 in 他笨手笨脚地爬上信号塔?
A word (created by a productive verb compounding rule), 
a phrase (a verb-complement structure),
or just a word sequence without structural significance?
Here I follow the opinion in \citep[\citepage{86}]{feng2000} and \citep{tham2015resultative}
and assume these are grammatical words,
because the word sequences in question are never extended,
or are extended highly limitedly,
while phrases, in principle, can be extended infinitely.
This goes against the analysis in \citep[\citesec{1.2.7}]{zhudexigrammar}.

\subsection{The blurred boundary between words and phrases}\label{sec:pos.morpheme-to-clause.blur-line-word-phrase}

That is not to say there are no real problems concerning what is a word in Chinese.
In certain scenarios no clear distinction between morphemes and words
-- and hence the distinction between words and phrases --
seems available.
\prettyref{sec:pos.morpheme-to-clause.fossilized} just shows this.
In this section I give two other cases in which 
we find blurred boundary between words and phrases.

\subsubsection{Splitting of word}

One thing that makes things more puzzling 
is even words \emph{without} synchronically morphosyntactic internal structures 
can sometimes be split and extended
with phrasal dependents injected,
though not everyone will accept such usages.
\citet[\citesec{6.5.8}]{chao1965grammar} records the first two non-standard examples of the phenomenon
in the follows,
while the third example is more widely accepted:
\begin{exe}
    \ex\label{ex:word-splitting} \begin{xlist}
        \ex\label{ex:junwanlexun} \% [军完了训] 以后才可以去请护照
        \ex\label{ex:youmo} \% 还 [幽了他一默]
        \ex\label{ex:guanshenmexin} 这件事情你 [关什么心] 啊
    \end{xlist}
\end{exe}
The bracketed constituents in \eqref{ex:youmo} and \eqref{ex:guanshenmexin} 
are both uncontroversially \acs{vp}s:
\eqref{ex:youmo} contains the object 他 and a temporal semi-object 一, % TODO: ref
while in \eqref{ex:guanshenmexin}, 什么心 as an \ac{np} is the object.
The splitting of the verbs clearly origins 
by analogy with \ac{vp}s containing morphosyntactic words 
The bracketed constituent in \eqref{ex:junwanlexun} is equivalent to 军训完了,
the latter being a verb complex, % TODO: ref
but 军完了训 is apparently created by analogy with 
[[吃完了]_{\text{predicator:verb complex}} [饭]_{\text{object:N}}]_{\text{VP}}.
Anyway, whether 军完了训 is directly a \ac{vp} or is first a verb complex 
and then a \ac{vp} is not of much importance:
similar construction never when an object is present.
What can be seen from \eqref{ex:word-splitting}, then, 
is uncontroversial grammatical words can also be split and extended into phrases 
in the same way phrase-like words like 念佛 do in 
\prettyref{sec:pos.morpheme-to-clause.blur-line-word-phrase}.

Splitting a word is of course morphological,
and is of a nontrivial type,
but here in Chinese, what is injected in after splitting the verb are phrasal dependents 
and the resulting morphosyntactic unit is a phrase,
so what we find here is syntax seems to rely on morphological devices.
Without prejudice, this is not quite shocking,
since what are said to be head movements 
also involve fusion in Distributed Morphology 
and therefore may be described as morphological.

The motivation of this phenomenon seems to be prosody: 
splitting words into phrases is only observed in \ac{vp}s,
and \ac{vp}s are subject to the prosodic constraint 
that the neither the verb nor the final complement can be too light.
Splitting the verb may help to reduce the ``weight'' of the verb 
so the resulting utterance meets the prosodic constraint better.

\subsubsection{What is a mini-constituent?}\label{sec:pos.architecture.word-confusing.constituent}

Recall that the standard for a constituent varies.
The English form \corpus{explore-ed} 
still needs to take an object, 
and some syntactic and semantic analysis 
actually implies that the tense/aspect categories are somehow higher than the object.
Thus it can be recognized as a unit -- a constituent -- because 
its inner components have strong interlink in dependency relations
(\prettyref{sec:theory}).
In principle, we can also analyze it as 
\corpus{[-ed [explore \emph{object}]_{\text{argument structure}}]_{\text{\acs{tame} marking}}};
in practice we refrain from doing so
(except in fully theoretical works like Distributed Morphology), 
because \corpus{explored} is a \emph{phonological} word 
(\prettyref{sec:pos.word.phonological}),
and this urges us to recognize it also as a morphosyntactic unit.
The longer sequence \corpus{have explored} 
has a parallel structure, 
but is not a phonological unit, 
and therefore some grammars say it's a verb phrase,
while others say it's not a phrase. (TODO: ref)
Both analyses, of course, are well-justified -- 
which one is to be used 
depends on the focus of the grammar
(\prettyref{sec:theory}).

In Mandarin, although prototypical inflection is absent, 

TODO: examples: 拿出了一个闹钟

\subsection{Phonological words}\label{sec:pos.word.phonological}

A phonological word may be a mini-constituent 
(\prettyref{sec:pos.architecture.word.mini-constituent}), 
or it may be a sequence that according to 
the constituency-based analysis is not a grammatical unit 
but does package highly related items.
In these cases it is also a morphosyntactic word,
although the definition of the latter is admittedly subjective
(although the hierarchical structure of the grammar reflected by the definition(s) is not).
Indeed, in the second case, 
sometimes it's the phonological wordhood 
that settles down 
whether a sequence is recognized as a word in everyday use or not.
But of course it's possible that the definition of the phonological word 
differs from the morphosyntactic word. 
This is also the case in Mandarin.
Before talking about how the two concepts match, 
I first explore what is a phonological word in Mandarin.

\subsubsection{The Chinese prosody structure}\label{sec:prosody-structure}

By paying attention to stops in Chinese utterances,
it can be found that phonological words exist and they are mostly defined by the prosody structure.
In the rest of this note,
the term \term{prosodic word} and \term{phonological word}
will be used interchangeably.
The prosody structure is about how stress is assigned to phonological constituents.
Assigning a prosodic structure is like condensation and clustering:
something is merged with something adjacent,
and the result is merged with something adjacent else.
When two phonological constituents are merged together,
one of them is considered heavier than the other.
If heaviness is to have a simple relation with the length of a phonological constituent,
then usually the more a phonological constituent is,
the heavier it is.
This is consistent with the condensation picture of prosodic segmentation.
Suppose a prosodic constituent attracts a syllable and merges with it.
The latter is not an independent phonological constituent
and cannot be heavy,
so the former is the heavier one and the latter is the lighter one in the larger prosodic constituent.

The smallest unit of prosody structure 
is a prosodic word.
The simplest prosodic word is the disyllabic foot, 
which contains two adjacent syllables in the case of Chinese.
(It can be made by two moras in other languages.)
One is assigned stress and is therefore heavier than the other.
Trisyllabic prosodic words also exist in Chinese,
though they are highly limited.
Most of which are borrowed words (e.g. 加拿大 \translate{Canada})
or words formed by coordinating three morphemes (e.g. 数理化 \translate{math, physics, and chemistry}).
They can also be regarded as foots \citet[\citesec{2.2}]{feng2000}.

Longer morphosyntactic units are inevitably broke into smaller disyllabic or trisyllabic prosodic words
in their prosodic structures:
加利福尼亚 may be segmented into 加利\textbar 福尼亚.

Prosody is able to see the constituency structure.
Some prosodic rules pertaining to the constituency tree 
guide and limit the assignment of relative heaviness and lightness.
In Chinese, prosodic segmentation is done strictly left-to-right 
in each \ac{np},
and then the \ac{np}s together with verbal constituents 
are used as the input of prosodic segmentation of clauses.
Then there are rules ruling out certain utterances.
The most important rule of this kind in Chinese is
the main verb and one post-verbal constituent
should form the last prosodic constituent in the clause (ignoring \ac{sfp}s),
and when there is no post-verbal constituents,
the main verb receives the natural stress. 
This is actually a rather strong condition.
Certain constituents -- most functional words -- are unable to accept stress at all.
They may be freely merged into the closest prosodic constituents.
Certain constituents -- like \ac{np}s with definite references -- are by default stressed.
When the verb is the last lexical constituent
(as in *他挥动棒子把我打),
it should not be too short or otherwise it is unable to receive stress.
When there are more than one post-verbal constituents,
only one of them can receive stress.
If two post-verbal constituents are both by default stressed,
the sentence is again ruled out. % TODO: ref

\subsubsection{Prosodic words in morphosyntax}\label{sec:prosodic-word-syntax}

Sometimes the significance of prosodic words are primarily phonological.
In 副总经理,
we have two prosodic words,
副总 and 经理,
while the morphological structure of the word is [[副] [总 [经理]]].
It is often the case that 
prosodic partition of a long grammatical word 
does not respect morpheme boundaries.
This is similar to the case in English and Latin poems,
where the prosody arrangement of sentences does not have to respect word boundaries:
\corpus{arma vi\textbar rumque ca\textbar no}.
By the way, this is also a criterion for testing wordhood:
for a grammatical word 副总经理,
mismatch between the morphosyntactic tree and the prosodic tree is acceptable,
while for a phrase,
such a mismatch immediately leads to ill-formedness.

It is possible
-- and quite likely, since prosody has access to syntax --
that a prosodic word has morphosyntactic significance, 
i.e. it is a morphosyntactic constituent.
Here is a subtle question:
is it always a grammatical word,
and never a phrase?
Here I make the assertion that a prosodic word that is a morphosyntactic constituent
is always a grammatical word.
For example, a prosodic word with a predicator-object inner structure 
like 念佛 is of course a morphosyntactic constituent,
and I recognize it as a word instead of a phrase 
(\prettyref{sec:pos.morpheme-to-clause.blur-line-word-phrase}).
In this case, I call it a \emph{morphosyntactic} prosodic word.
Whether this works is to be seen in the following sections.

A large portion of morphosyntactic prosodic words,
like the disyllabic verbs 念佛, 军训, 体操 etc. 
in \prettyref{sec:pos.morpheme-to-clause.blur-line-word-phrase} 
are all prosodic words,
and as is said in \prettyref{sec:pos.morpheme-to-clause.blur-line-word-phrase},
if a prosodic word is made up by two morphemes with a synchronic device,
the two morphemes can be interpreted as grammatical words and morphemes in different syntactic contexts,
and constructions with similar output 
exist even for those without analyzable internal structures 
(\prettyref{sec:pos.morpheme-to-clause.blur-line-word-phrase}).

\subsection{Types of clauses}

\subsection{The structure of Mandarin grammar}

After establishing the status of \term{words} in Mandarin grammar, 
we can now discuss the overall architecture of Mandarin grammar 
in a more disciplined way.
\prettyref{fig:morpheme-to-phrase} summarizes the organization of Chinese lexicon 
as well as how larger units are built from lexical items.
Overlapping of blobs means ``having the same form''.
Thus, the blob representing monosyllabic words is completely in the blob of monosyllabic morphemes.
The same is for the relation between non-prosodic simple words 
(which are neither monosyllabic nor disyllabic) and polysyllabic morphemes.
Red arrows mean synchronic morphosyntactic devices,
while orange arrows means historical evolution,
like grammaticalization and/or fossilization.

\begin{figure}[H]
    \centering
    

\tikzset{every picture/.style={line width=0.3pt}} %set default line width to 0.75pt        

\begin{tikzpicture}[x=0.75pt,y=0.75pt,yscale=-0.8,xscale=0.8]
%uncomment if require: \path (0,552); %set diagram left start at 0, and has height of 552

%Shape: Ellipse [id:dp25211786396044866] 
\draw  [color={rgb, 255:red, 155; green, 155; blue, 155 }  ,draw opacity=1 ][fill={rgb, 255:red, 155; green, 155; blue, 155 }  ,fill opacity=0.2 ] (105.33,392.3) .. controls (105.33,341.28) and (154.13,299.93) .. (214.33,299.93) .. controls (274.53,299.93) and (323.33,341.28) .. (323.33,392.3) .. controls (323.33,443.31) and (274.53,484.67) .. (214.33,484.67) .. controls (154.13,484.67) and (105.33,443.31) .. (105.33,392.3) -- cycle ;
%Shape: Ellipse [id:dp011093449258029242] 
\draw  [color={rgb, 255:red, 155; green, 155; blue, 155 }  ,draw opacity=1 ][fill={rgb, 255:red, 155; green, 155; blue, 155 }  ,fill opacity=0.2 ] (164.06,334.33) .. controls (167.18,318.1) and (187.28,308.32) .. (208.94,312.5) .. controls (230.6,316.67) and (245.62,333.22) .. (242.49,349.45) .. controls (239.37,365.69) and (219.27,375.47) .. (197.61,371.29) .. controls (175.95,367.12) and (160.93,350.57) .. (164.06,334.33) -- cycle ;

%Shape: Ellipse [id:dp8023514623566042] 
\draw  [color={rgb, 255:red, 155; green, 155; blue, 155 }  ,draw opacity=1 ][fill={rgb, 255:red, 155; green, 155; blue, 155 }  ,fill opacity=0.2 ] (336,398) .. controls (336,362.51) and (373.68,333.74) .. (420.17,333.74) .. controls (466.65,333.74) and (504.33,362.51) .. (504.33,398) .. controls (504.33,433.49) and (466.65,462.26) .. (420.17,462.26) .. controls (373.68,462.26) and (336,433.49) .. (336,398) -- cycle ;
%Shape: Ellipse [id:dp321544508802865] 
\draw  [color={rgb, 255:red, 155; green, 155; blue, 155 }  ,draw opacity=1 ][fill={rgb, 255:red, 155; green, 155; blue, 155 }  ,fill opacity=0.2 ] (404.81,386.06) .. controls (402.26,369.72) and (417.86,353.72) .. (439.65,350.31) .. controls (461.45,346.91) and (481.19,357.39) .. (483.74,373.73) .. controls (486.29,390.06) and (470.69,406.07) .. (448.9,409.47) .. controls (427.1,412.88) and (407.36,402.4) .. (404.81,386.06) -- cycle ;

%Shape: Ellipse [id:dp9397609165336205] 
\draw  [color={rgb, 255:red, 155; green, 155; blue, 155 }  ,draw opacity=1 ][fill={rgb, 255:red, 155; green, 155; blue, 155 }  ,fill opacity=0.2 ] (382.01,362.67) .. controls (324.14,364.88) and (275.22,314.07) .. (272.74,249.18) .. controls (270.25,184.3) and (315.16,129.9) .. (373.03,127.69) .. controls (430.9,125.47) and (479.82,176.28) .. (482.3,241.17) .. controls (484.78,306.06) and (439.88,360.45) .. (382.01,362.67) -- cycle ;

%Curve Lines [id:da3228764928640613] 
\draw [color={rgb, 255:red, 208; green, 2; blue, 27 }  ,draw opacity=1 ]   (230.86,300.6) .. controls (223.77,257.91) and (236.35,243.61) .. (271.13,248.93) ;
\draw [shift={(272.74,249.18)}, rotate = 189.46] [fill={rgb, 255:red, 208; green, 2; blue, 27 }  ,fill opacity=1 ][line width=0.08]  [draw opacity=0] (12,-3) -- (0,0) -- (12,3) -- cycle    ;
%Curve Lines [id:da796668618520048] 
\draw [color={rgb, 255:red, 245; green, 166; blue, 35 }  ,draw opacity=1 ] [dash pattern={on 4.5pt off 4.5pt}]  (646.67,256.26) .. controls (666.57,354.69) and (582.51,395.85) .. (503.85,408.08) ;
\draw [shift={(502.67,408.26)}, rotate = 351.36] [fill={rgb, 255:red, 245; green, 166; blue, 35 }  ,fill opacity=1 ][line width=0.08]  [draw opacity=0] (12,-3) -- (0,0) -- (12,3) -- cycle    ;
%Curve Lines [id:da025052452235164946] 
\draw [color={rgb, 255:red, 208; green, 2; blue, 27 }  ,draw opacity=1 ]   (551.67,168.26) .. controls (514.24,147.58) and (486.19,158.91) .. (471.97,189.83) ;
\draw [shift={(471.33,191.26)}, rotate = 293.63] [fill={rgb, 255:red, 208; green, 2; blue, 27 }  ,fill opacity=1 ][line width=0.08]  [draw opacity=0] (12,-3) -- (0,0) -- (12,3) -- cycle    ;
%Shape: Ellipse [id:dp6184914092498568] 
\draw  [color={rgb, 255:red, 155; green, 155; blue, 155 }  ,draw opacity=1 ][fill={rgb, 255:red, 155; green, 155; blue, 155 }  ,fill opacity=0.2 ] (536,203.67) .. controls (536,172.74) and (571.89,147.67) .. (616.17,147.67) .. controls (660.44,147.67) and (696.33,172.74) .. (696.33,203.67) .. controls (696.33,234.59) and (660.44,259.67) .. (616.17,259.67) .. controls (571.89,259.67) and (536,234.59) .. (536,203.67) -- cycle ;

%Curve Lines [id:da27466382204201545] 
\draw [color={rgb, 255:red, 208; green, 2; blue, 27 }  ,draw opacity=1 ]   (514,363.26) .. controls (519.31,299.58) and (676.42,360.64) .. (637.6,261.76) ;
\draw [shift={(637,260.26)}, rotate = 67.91] [fill={rgb, 255:red, 208; green, 2; blue, 27 }  ,fill opacity=1 ][line width=0.08]  [draw opacity=0] (12,-3) -- (0,0) -- (12,3) -- cycle    ;
%Curve Lines [id:da11077382821910953] 
\draw [color={rgb, 255:red, 245; green, 166; blue, 35 }  ,draw opacity=1 ] [dash pattern={on 4.5pt off 4.5pt}]  (473.67,293.26) .. controls (484.56,310.09) and (489.57,323.98) .. (432.42,337.84) ;
\draw [shift={(430.67,338.26)}, rotate = 346.65] [fill={rgb, 255:red, 245; green, 166; blue, 35 }  ,fill opacity=1 ][line width=0.08]  [draw opacity=0] (12,-3) -- (0,0) -- (12,3) -- cycle    ;
%Curve Lines [id:da3266231201686014] 
\draw [color={rgb, 255:red, 155; green, 155; blue, 155 }  ,draw opacity=1 ][fill={rgb, 255:red, 155; green, 155; blue, 155 }  ,fill opacity=0.2 ]   (89,90.26) .. controls (85.67,243.26) and (163.67,345.26) .. (210.67,340.26) .. controls (257.67,335.26) and (204.67,217.26) .. (283.67,140.26) .. controls (362.67,63.26) and (423,124.59) .. (414,207.59) .. controls (405,290.59) and (320.67,294.26) .. (350.67,376.26) .. controls (380.67,458.26) and (558.67,427.26) .. (634.67,364.26) .. controls (710.67,301.26) and (728.67,127.26) .. (728.67,90.26) ;
%Shape: Polygon Curved [id:ds7705333747253118] 
\draw  [color={rgb, 255:red, 155; green, 155; blue, 155 }  ,draw opacity=1 ][fill={rgb, 255:red, 155; green, 155; blue, 155 }  ,fill opacity=0.2 ] (300,283.93) .. controls (412,321.93) and (522,319.93) .. (522,395.93) .. controls (522,471.93) and (377,501.93) .. (274,517.93) .. controls (171,533.93) and (84,477.93) .. (79,395.93) .. controls (74,313.93) and (188,245.93) .. (300,283.93) -- cycle ;

% Text Node
\draw (126,154) node [anchor=north west][inner sep=0.75pt]  [color={rgb, 255:red, 0; green, 0; blue, 0 }  ,opacity=1 ] [align=left] {phrases};
% Text Node
\draw (377.52,245.18) node   [align=left] {\begin{minipage}[lt]{76.4pt}\setlength\topsep{0pt}
\begin{center}
morphosyntactic\\prosodic\\words
\end{center}

\end{minipage}};
% Text Node
\draw (245.5,420.7) node   [align=left] {\begin{minipage}[lt]{60.24pt}\setlength\topsep{0pt}
\begin{center}
monosyllabic\\morphemes
\end{center}

\end{minipage}};
% Text Node
\draw (203.28,341.89) node  [font=\scriptsize] [align=left] {\begin{minipage}[lt]{48.74pt}\setlength\topsep{0pt}
\begin{center}
monosyllabic\\words
\end{center}

\end{minipage}};
% Text Node
\draw (618.17,202.67) node   [align=left] {\begin{minipage}[lt]{59.71pt}\setlength\topsep{0pt}
\begin{center}
non-prosodic\\complex\\words
\end{center}

\end{minipage}};
% Text Node
\draw (427.5,430.7) node  [font=\small] [align=left] {\begin{minipage}[lt]{49.89pt}\setlength\topsep{0pt}
\begin{center}
polysyllabic\\morphemes
\end{center}

\end{minipage}};
% Text Node
\draw (444.28,379.89) node  [font=\tiny] [align=left] {\begin{minipage}[lt]{48.8pt}\setlength\topsep{0pt}
\begin{center}
non-prosodic\\simple words
\end{center}

\end{minipage}};
% Text Node
\draw (304,460.93) node [anchor=north west][inner sep=0.75pt]   [align=left] {morphemes};


\end{tikzpicture}

    \caption{From morphemes to phrases.}
    \label{fig:morpheme-to-phrase}
\end{figure}

\subsubsection{Morphemes}

There are two types of morphemes:
one is the monosyllabic type, examples of which include 红, 大, 你, etc.,
the other is the polysyllabic type,
examples of which include 葡萄, 巧克力, etc.
A subset of the first type is able to serve as grammatical words,
like personal pronouns 你, 我 and 他.
Some monosyllabic morphemes only appear in in-word slots 
(like the modifier slot in \eqref{ex:nominal-modifier-2}),
and they are unable to serve as words.
Function affixes, of course, also belong to this type.
Certain archaic words are also no longer free in modern Chinese,
like 观 \translate{observe}:
we have 观鸟 \translate{bird watching},
but 观 never appears as a single verb, 
nor does it undergo delimitative reduplication of verbs:
\begin{exe}
    \ex \begin{xlist}
        \ex *我要去观观那些鸟
        \ex 我要去看看那些鸟
    \end{xlist}
\end{exe}
An overwhelmingly portion of morphemes of the second type serve as grammatical words.
Certain borrowed affixes may be unable to serve as words in certain periods.
As times goes by, however, they gradually become free morphemes.
If clause linking markers like 之所以 and 是因为 are recognized as single morpheme words,
then they may be included into the non-prosodic simple word blob 
and however be unable to serve as phrases.
These markers, however, never appear in other places,
and their exact status is of no descriptive and comparative interest. 

\subsubsection{Prosodic words with morphosyntactic significance}

What singles Chinese out is certain prosodic words have morphosyntactic significance 
(\prettyref{sec:prosodic-word-syntax}).
A large number of them can be both grammatical words and phrases,
the latter being the extended versions of the former.
These prosodic words may be two monosyllabic morphemes glued together by synchronic morphosyntactic rules,
like 种树 and 交给,
or they may be fossilized two-morpheme ones
(or three-morpheme ones in rare cases,
as in 数理化 mentioned in \prettyref{sec:prosody-structure}),
where the relations between the morphemes are no longer available to the grammar,
so in this case,
they have largely identical behaviors with the single-morpheme case below.
Prosodic words with morphosyntactic significance may also be single-morpheme words,
as is the case of 幽默 (and rare trisyllabic cases like 加拿大).
A final origin of morphosyntactic words 
is well-accepted and hence fossilized abbreviation of complex words.
The noun 空调 is a quite frequent word in modern Chinese,
but few will stop and realize it is, at least historically, 
the abbreviation of 空气调节器 (which is the literal translation of \corpus{air conditioner}).

When a morphosyntactic prosodic word is made up by two synchronically analyzable morphemes,
and both of them are able to be grammatical words themselves,
we are left in the situation described in \prettyref{sec:pos.morpheme-to-clause.blur-line-word-phrase},
and the morphosyntactic prosodic word can be extended into a phrase 
by inserting more phrasal dependents 
or by adjoining modifiers to one of the two morphemes.
Even when this is not the case,
word splitting is sometimes also possible 
(\prettyref{sec:pos.morpheme-to-clause.blur-line-word-phrase}).

\subsubsection{Non-prosodic complex words}

Non-prosodic complex grammatical words also exist in Chinese.
They are not monosyllabic and are not prosodic words,
so they definitely contain more than two syllables. 
Just like the case of morphosyntactic prosodic words,
non-prosodic complex grammatical words 
may be created by synchronic morphosyntactic rules,
or they may be fossilized or have no internal structure.

Compared to prosodic words,
non-prosodic complex words are less ``active'' in syntax:
splitting them is possible in certain cases
but is much less frequent.
This may be a result of pragmatics:
complex words are created to cover a meaning that needs some explanation,
and once a complex word is well-accepted,
its form and meaning soon gets fixed 
(because people will not burden themselves),
and fossilization occurs rapidly.
The term 美利坚合众国 has an analyzable internal structure,
but it has already gained a fixed meaning 
and its parts are never taken out,
despite both 美利坚 and 合众国 can serve as grammatical words.

\subsubsection{Between morphemes and phrases}

After introducing all blobs in \prettyref{fig:morpheme-to-phrase},
I now discuss their relations.
In English there is a group of clearly defined and largely homogenous morphosyntactic units 
lying between morphemes and phrases
(which means they can be neither morphemes nor phrases,
and can be constructed by the former and can be used to build the latter),
which is just the grammatical word.
In Chinese, however, despite the existence of in-word grammatical relations,
this is not the case.

Let us start with the relation between grammatical words and phrases.
Most of grammatical words are themselves phrases,
major exceptions including 
compounds of a verb and its non-argument complement, like 看出 and 走上,
monosyllabic location words like 上 and 前,
and coverbs.
Transitive verbs can regularly fill argument slots 
and thus are able to be used as one-word phrases,
though they themselves are not sufficient to build one-word predicate \ac{vp}s,
but certain grammatical words,
like those discussed at the end of \prettyref{sec:pos.morpheme-to-clause.blur-line-word-phrase},
are not phrases on their on, regardless of their types.
On the other hand, the compound of verbs and non-argument complements 
are never able to fill argument slots on their own:
\begin{exe}
    \ex\label{ex:transitive-verb-phrase-disyllabic} Transitive disyllabic verbs as one-word phrases
    \begin{xlist}
        \ex {} [看书]_{\text{subject:verb}} 是一件有趣的事情
        \ex *[走进]_{\text{subject:verb}} 意味着您已经同意了我们的服务条款 
        \ex {} [走进这个建筑]_{\text{subject:VP}} 意味着您已经同意了我们的服务条款
    \end{xlist}
\end{exe}
Monosyllabic location words like 前 are definitely words 
because they can be attached to an arbitrary \ac{np} 
to denote a place near the place denoted by that \ac{np},
as in [[那座老旧的房子 [前]_{\text{location word}}]_{\text{NP}} 有一口井]_{\text{clause}},
and what only appears as immediate constituents of phrases are of course words,%
\footnote{
    Here I have to emphasize again that I'm talking about morphosyntactic words here.
    It is possible that something is a morphosyntactic word is incorporated into a word nearby phonologically,
    though this is not the case in Chinese.
}%
but they never appear independently as \ac{np}s.
Monosyllabic location words and coverbs, however,
are clearly pre-defined in the grammar and 
for some authors (like Dixon) 
they are to be excluded from the discussion on the lexicon,
because they are not and are not made from lexemes.
Most polysyllabic grammatical words are able to constitute one-word phrases.
This is true for non-prosodic simple words,
and similarly,
a overwhelming number of non-prosodic complex words are able to be one-word phrases,
and there is almost no attested counterexample.

Now I discuss the relation between grammatical words and morphemes.
A monosyllabic word is of course made by only one morpheme.
There are also polysyllabic morphemes that can be words on their owns,
but the number of them is much less than monosyllabic ones.
Trisyllabic or larger complex words can be abbreviated into morphosyntactic prosodic words,
and they can also be constructed with

It can be seen in Chinese, 
though wordhood is still distinguishable morphosyntactically,
grammatical words have much larger overlapping with morphemes and phrases: 
monosyllabic words are also morphemes, 
while non-prosodic complex words and non-prosodic simple words have part-time jobs as phrases.
This may be the deriving force for some linguists 
(who are too eager to ``find diversity'')
to reject the existence of grammatical words,
though as I argued before,
via in-depth morphosyntactic tests,
the notion of grammatical words can still be established.
Is there a subgroup of morphosyntactic words 
with uniform behaviors and 
a considerable number of members that are neither morphemes nor phrases, then?
It is the group of prosodic words.
As is illustrated by \prettyref{sec:pos.morpheme-to-clause.blur-line-word-phrase},
prosodic words are subject to richer syntactic devices 
compared with other grammatical words.
Polysyllabic grammatical words are also subject to these processes,
but they are more restricted in that aspect.
(If syntactic processes in \prettyref{sec:pos.morpheme-to-clause.blur-line-word-phrase} 
are productive for a polysyllabic constituent with a fixed meaning,
it is highly likely this constituent is not a word but an idiom.)
Therefore, the category of morphosyntactic prosodic words does have uniform behaviors.

What can be concluded is 
morphosyntactic prosodic words in Chinese 
fill the role of grammatical words in English:
they are intermediate units between (monosyllabic) words and phrases,
and they are also intermediate units between morphemes and (complex) words,
occasionally.
The fuss around wordhood in Chinese 
arises from the split of several definitions of the term \term{word}:
wordhood defined by morphosyntactic test include is wider 
than the coverage of morphosyntactic prosodic words,
and it is the former that has too much overlapping with morphemes and phrases.
These prosodic words are therefore the nature response 
when native speakers without much linguistic training 
come up with intuitively when talking about ``words''
i.e. intermediate building blocks.
They are also building blocks of the language
despite being synchronically analyzable,
beside the commonly recognized grammatical words. 
This is a particularity of Chinese.

\section{Overview of classification criteria}

\subsection{Word class labels: noun, verb, adjectives, etc.}

\subsubsection{The nominal-verbal division}

Lexical words in Chinese can be roughly divided into nominal ones and verbal ones,
or in the Chinese terms, 体词 and 谓词.
The prototypical role of nominal words 
is to fill predicate slots (or to be more precise, to head a phrase that fills an argument slot).
Nominal words rarely appear in the verbal complex,
though for stylistic purposes, they sometimes do.
Verbal words prototypically appear in the verbal complex
(\prettyref{chap:verbal-complex}),
but many of them -- and clauses without any morphological marking -- 
can regularly appear in argument slots \citep[\citesec{3.5}]{zhudexigrammar}.

The fact that verbal categories can fill argument slots or in colloquial words ``be used as nouns''
urges some to put the verbal categories under the nominal categories,
so thus there is only one mega lexical category in Chinese:
the nominal category or the Noun.
The analysis adopted here does not aim to organize lexical categories 
in a binary branching classification tree,
so the ordinary nominal-verbal distinction is maintained:
verbs being able to fill argument slots is not typologically rare, actually,
and this shared feature itself does not bring nouns and verbs close enough 
for them to be merged together.

\subsubsection{Two adjectival classes}

Whether Chinese has a separate adjective category 
has been debated for decades.
Based on a line of reasoning similar to the above verb-as-noun analysis,
some linguists argue that the so-called adjectives should be put under the verb category,
since they can fill the predicator slot without any morphological marking \citep{li1989mandarin}.
Since verbs and most alleged adjectives show different morphological behaviors in reduplication, % TODO: ref
the verb-adjective distinction is kept,
and the two are placed under the verbal category.

There still exist a (much smaller) number of alleged adjectives that shows 
different morphosyntactic properties with the adjectives in the verbal category 
\citep[\citechap{5}]{paul2014new}.
They can be marginally used as heads of \ac{np}s,
while they do not have reduplication variants.
These ``adjectives'' are thus placed under the nominal category.
Thus we have two types of adjectives.
In \citet{zhudexigrammar}, 
nominal adjectives are called 区别词 \translate{distinction word},
while verbal adjectives are called 形容词 \translate{adjective}.

\subsubsection{Other nominal categories}

There are more nominal categories than the ordinary noun category and the nominal adjective category.
Numerals, for examples, are in another nominal category.
Chinese has a rich classifier system,
and most classifiers still have strong nominal properties
and thus they constitute yet another nominal category.
\citet{zhudexigrammar} calls them 量词 \translate{measure word},
because many classifiers have the meaning of ``unit''.
There is also a locative particle class, including 里 in 在房子里,
which is sometimes said to be the postposition class
because they sometimes have adposition-like properties (TODO: ref: topicalization, and what else?).

\subsection{Open and close classes; the lexical-functional distinction}

\subsubsection{The hierarchy of openness}

The distinction between lexical and functional classes is sometimes subtle.
\citep[\citesec{3.6}]{zhudexigrammar} classifies 
certain categories like locative particles % TODO: 方位词的正确翻译???
into the nominal class and hence the lexical one,
while the locative particle class can definitely be enumerated \citep[\citesec{4.4}]{zhudexigrammar}.
On the other hand, 
the author claims that lexical classes are always open 
and function classes are always closed \citet[\citesec{3.4}]{zhudexigrammar}.
A conflict thus occurs.

The problem here is we have a gradient hierarchy 
from the prototypical lexical classes 
to the prototypical function classes.
The most lexical class is open to new members, 
not a part of the grammar,
and its members are able to be lexical heads
(and thus has a ``real part-of-speech label'' 
like ``noun'' or ``verb'') 
of, say, an \acs{np} or a verbal complex.
A less open class is not so open to new members 
(just like Japanese verbs and adjectives),
but is still not a part of the grammar 
and its members are able to be lexical heads. 
A even more closed class is not open to new members,
and is a part of the grammar,
but its members are still able to be lexical heads.
Pronouns are in this type.
A prototypical function class, then, 
is not open to new members, hardwired in the grammar, 
and its members are never lexical heads.
Derivational suffixes are in this type.
This last type of forms 
bring no real part-of-speech label to its realization.
We still classify these functional items in the grammar into classes,
but these classes are somehow less ``real''.

It's of course not easy to tell a newly discovered part of speech 
(or \term{form class}, which may be a word class or an affix class)
What's the status of an orientation preverb, 
which may be found in Japhug \citep{jacques2021grammar}?
It's a part of the grammar,
but does it carry a real part-of-speech label (like ``directional adverb'')?
And speaking of adverbs, what's the status of the English \corpus{allegedly}?
An adverb filling a peripheral argument position,
or an evidentiality marker?
We really need to know a lot about language to fix the position of a form class.
A common practice is just to shun the details and just say whether a class is lexical or functional,
drawing a hard line between the two.
So \citet{zhudexigrammar} mainly uses the criterion of whether there is a real part-of-speech label,
and then directive particles are classified into the nominal class 
and they are in turn considered lexical.
But he mistakenly confuses the notion of lexical classes with the notion of open classes,
and then we get the self-conflicting asserts in \citet[\citesec{3.4}]{zhudexigrammar}.

\subsubsection{Openness of Mandarin form classes}

\subsection{Summary: a tentative part of speech analysis}

\section{Prepositions}\label{sec:preposition-pos}

Though all Mandarin prepositions have verb origins 
and therefore may be classified as a subclass of verbs by some,
it's necessary to distinguish a separate preposition class.
Criteria of prepositions include TODO: ref

\begin{infobox}{The term \term{coverb}}{coverb}
    In 
\end{infobox}


\chapter{The structure of noun phrase}

No morphological case, number, and gender categories are attested in Mandarin.
There is a word class system or in other words classifier system, however.
In most cases when a numeral appears in an \ac{np},
a classifier follows immediately after the numeral.
Attributives -- both adjectives and relative clauses -- 
follow the classifier. % TODO: 红色的三个,这样的说法说得通吗?
The demonstrative, if any, appears before the numeral,
and even when there is no numeral,
there is frequently also a classifier.

The template of \ac{np}s, therefore, belongs to the 
Dem-Num-A-N type,
with the classifier residing between Num and A. 


\chapter{The verbal complex}\label{chap:verbal-complex}

\section{Introduction}

Mandarin is generally regarded as a prototypical analytic language,
without traditionally acknowledged verb inflections.
Indeed it will be weird to posit something like a paradigm in Mandarin,
but it doesn't mean there is no such thing as verbal affixation that are active
in the morphosyntax 
(instead of not fully productive and arguably historical derivations).
Some items involved here however may have partial mobility.
Consider \eqref{ex:movable-suffix}:
In the first sentence, 
了 is an aspectual suffix (\prettyref{sec:aspectual}),
while 走 is a verb which never appear without an argument in uncontroversial phrasal grammar.
So we conclude 了 and 走 are suffixes,
and by structural comparison, 
we conclude 过来 in \eqref{ex:sanpinqishuiugolai-1} 
is also a suffix, with the same status as 走.
But there comes \eqref{ex:sanpinqishuiugolai-2},
in which 过来 moves to the end of the sentence.

\begin{exe}
    \ex \begin{xlist}
        \ex \gll 他 带 走 了 他的 文件  \\ 
        3sg carry go.away \acs{perfect} 3sg-\acs{possessive} file \\
        \glt \translate{He carried his files away.}
        \ex \gll 他 带 [过来] 了 三 瓶 汽水 \\
        3sg carry come \acs{perfect} three bottle.\acs{classify} soda \\
        \glt \translate{He carried here three bottles of soda.} 
        \label{ex:sanpinqishuiugolai-1}
        \ex 他带了三瓶汽水[过来]
        \label{ex:sanpinqishuiugolai-2}
    \end{xlist}
    \label{ex:movable-suffix}
\end{exe}

To avoid the useless quarrelling about what is a word and whether a grammar point is morphology
(which isn't that important in non-lexicalist generative theories, anyway),
I use the term \term{verbal complex} to cover 
the main verb and the  ``suffixes'' in \eqref{ex:movable-suffix}.
There are roughly three systems in the verbal complex.
The first is the uncontroversial derivation system,
like 化 \translate{-ize}.
The second is the verbal complement system,
which includes three subsystems:
the resultative complements, the directional complements, 
and the potential complements (\prettyref{sec:verbal-complement}).
The third is the aspectual system (\prettyref{sec:aspectual}).

\begin{infobox}{On the notion of \term{complements}}{complement-name}
    The Chinese term 补语 corresponding to my \term{verbal complement}
    is frequently translated into the English term \term{complement}.
    This creates some confusion,
    because the term \term{complement} can also denote 
    clausal dependents that are arguments of the main verb, as in \citet{cgel}.
    The term \term{non-argument complement} may be used to avoid this confusion.
    There are, however, further confusions:
    Should we regard a clausal dependent that records the quantity or amount of an action 
    as a non-argument complement?
    This construction can also be seen in Latin, 
    like the Latin accusative expression of time \citep[\citesec{423}]{greenough2013allen}.
    Thus, I use the term \term{verbal complement} to refer to 
    things like 完 as in 做完了.
\end{infobox}

\eqref{ex:hua-wan-le-1} is an example in which 
all the three systems appear.
In real world speeches, such combinations have relatively lower distributions,
possibly because of the prosodic constraint 
that verb shouldn't be too heavy unless it appears at the end of a clause
(TDOO: ref).

\begin{exe}
    \ex \dots 并且企业 [数字 [化]_{\text{derivation}} [完]_{\text{complement}} [了]_{\text{aspectual}}]_{\text{V}} 之后还不一定赚钱 \dots
    \label{ex:hua-wan-le-1}
\end{exe}

Besides the systems shown in \eqref{ex:hua-wan-le-1},
the separation of a verb further complicates the behavior of the verbal complex 
(\prettyref{sec:separable-verbs}).

You may note the so-called serial verb constructions aren't mentioned here.
\citet{paul2008serial} and \citet[\citesec{9.4}]{deng2010formal} 
summarizes several constructions that are
frequently referred to as serial verb constructions,
and points out after deeper investigation,
they can all be described in terms of the usual complement clause constructions,
purpose clause constructions, etc. 
that are well attested cross-linguistically (\prettyref{sec:no-serial-verb}).

\section{Verbal derivations}

\section{Verbal complements}\label{sec:verbal-complement}

\section{The aspectual system}\label{sec:aspectual}

\section{Separable verbs}\label{sec:separable-verbs}

It's sometimes possible to split a verb and inject some clausal dependents into it.
The interaction between this separation operation and the structure of the verbal complex is of some interest.

\chapter{Verb and arguments}

\chapter{Valency changing}

There are two ways of valency changing in Mandarin.
The first is via a coverb construction, 
as in the disposal constructions (\prettyref{sec:disposal-construction}),
TODO 
The second is \emph{doing nothing} to the verb 
and relying on the unusual semantic roles of clausal complements 
to inform the listener about the valency changing,
as in TODO: ref.
Since there is no morphological marking,
constructions of this type are often recognized as topic-comment structures,
in which the ``topic'' -- which is the subject under closer investigation -- 
is said to be freely occupied by any semantic (and not necessarily syntactic) argument in the clause,
though this claim can be falsified by detailed syntactic tests (\prettyref{sec:topic-subject}).

\section{The disposal constructions}\label{sec:disposal-construction}

\section{The passive constructions}\label{sec:affected-construction}

\begin{exe}
    \ex 我被他打了一拳
\end{exe}

\section{The causative}

\section{The affected construction}



\section{Instrumental object}

\begin{exe}
    \ex 我们今天准备吃食堂
\end{exe}

\chapter{Simple clauses}

\section{Overall remarks about the clause structure}

A sentence can be divided into several clauses linked by clause linking constructions 
(\prettyref{chap:clause-linking}).
This chapter is denoted to the simple clause,
postponing details in subordination and clause linking to the next several chapters.
Mandarin has rich topicalization phenomena,
and thus a clause can be divided into
one or more topics (if any) and a comment,
the latter being the nucleus clause
plus possible \acl{sfp}s.
The comment -- the nucleus clause -- may further be divided into a subject (if any),
a series of adverbials, 
the verbal complex, and post-verbal constituents,
the most important types including object(s), 
the second part of a separable verb,
certain directional complements,
and purpose clauses.

\begin{infobox}{The term \term{clause}}{clause-def}
    Some people, like \citet[\citepage{140}]{deng2010formal}
    as well as \citet{dixon2009basic},
    use the term \term{clause} for subject-predict constructions 
    that don't receive complete marking of speech forces.
    (In generative terms, \term{clause} is for lower level CPs or even TPs.)
    So in this way, \acl{sfp}s shouldn't be discussed in this chapter because 
    they are of course dependents in the sentence level.
    They may be discussed together with other sentence-level constructions like \prettyref{chap:clause-linking}.
    But this notion of clause certainly goes against the tradition in descriptive grammars.
    So the approach of this note is to acknowledge everything larger than TP as a clause,
    which may or may not be a sentence,
    and discuss its structure in this chapter,
    while ``adjunctions'' -- or in other words, optional dependents -- 
    are discussed in, say, \prettyref{chap:clause-linking},
    for the sake of convenience.
    The narrative order of this note is not the ideal ``small unit -- large unit'' scheme,
    but the ``simple large unit -- complicated large unit'' scheme.
    Needless to say,
    when it comes to clause combining, 
    the problem of what the clause really is -- with or without \ac{sfp}s, for example --
    is still relevant,
    but it is not answered by saying ``the construction takes a clause, not a sentence''.
\end{infobox}

As is implied by my using the term \term{subject},
Mandarin is an typical accusative language.
Clausal dependents are recognizable from the rather rigid constituent order:
Mandarin is usually classified as having a SVO clausal constituent order,
and the subject and the object(s) can be told from the positions in the clause 
(\ref{ex:get-sick}, \ref{ex:svo-example}).
Certain ``SOV'' orders can be obtained by invoking the disposal construction
(\prettyref{sec:disposal-construction}), as in \eqref{ex:ba-example}.

\begin{exe}
    \ex \gll 我 生病 了 \\
    1 get.sick \acs{sfp} \\
    \glt \translate{I got sick.}
    \label{ex:get-sick}

    \ex \gll [我]_{\text{subject}} 今天 去 看 [电影]_{\text{object}} 了 \\
    1 today to watch movie \acs{sfp} \\
    \glt \translate{I went to watch a movie today.} 
    \label{ex:svo-example}

    \ex \gll [我]_{\text{subject}} 今天 把 [ 一 个 碗 ]_{\text{object}} 摔 碎 了 \\
    1 today BA {} one \acs{classify} bowl {} break crack \acs{sfp} \\
    \glt \translate{I broke one bowl today.}
    \label{ex:ba-example}
\end{exe}

The normal tests of syntactic accusative alignment can be run on Mandarin
(\ref{ex:inter-sentence}).

\begin{exe}
    \ex \gll 陈 经理 昨天 没有 和 他的 客户 聊 过 。 他 生病 了 。 \\
    {Chen (surname)} manager yesterday \acs{neg} with 3sg-\acs{possessive} client talk \acs{sfp}
    {} 3sg get.sick \acs{sfp} \\
    \glt \translate{Manager Chen didn't talk with his client yesterday. He (Chen, not his client) got sick.}
    \label{ex:inter-sentence}
\end{exe}

\section{Types of nucleus clauses}

\begin{infobox}{About the subject-predicate binary division}{subject-predicate}
    \citet{dixon2009basic} argues against the definition of \term{predicate} 
    as the main verb (or adjective) plus somehow ``internal'' arguments.
    He uses the term \term{predicate} to refer to the verbal complex instead.
    However, since I will need to compare the topic-comment construction 
    with the inner structure of the nucleus clause,
    the term \term{predicate} will still be used in the way \citet{dixon2009basic} dislikes,
    because it's the counterpart of the comment in the topic-comment construction.
\end{infobox}

\subsection{There is no serial-verb construction or complex predicate}\label{sec:no-serial-verb}

\section{Negation}\label{sec:negation}

Like the case in standard English, 
there is no negative concord in Mandarin Chinese.
There is, however, no uniform negation operator like the English \emph{not}. 
Several negation operators and strategies are used frequently (\prettyref{sec:negation}).
Verbs can be negated by 不 while nouns generally cannot, 
and this is a criterion to tell verbs from nouns. 
There is another negation operator 没, 
which has subtle differences in its meaning and syntactic properties compared with 不
(\ref{ex:chiqincai}, \ref{ex:buchi-meichi}).
On the other hand, the negative potential complement construction,
i.e. the V不了 construction,
isn't obtained by inserting a negator in the clause \eqref{ex:zuobuliao-example}.

\begin{exe}
    \ex \begin{xlist}
        \ex \gll 我 不 喜欢 吃 芹菜 \\
        1 \acs{neg} like eat celery \\
        \glt \translate{I don't like eating celery.} \\
        \ex * 我 没 喜欢 吃 芹菜
    \end{xlist}
    \label{ex:chiqincai}
\end{exe}

\begin{exe}
    \ex \begin{xlist}
        \ex \gll 我 不 吃 早饭 \\
        1 \acs{neg} eat breakfast \\
        \glt \translate{I don't eat breakfast. (I usually don't, I don't want any today, etc.)}
        \ex \gll 我 没 吃 早饭 \\
        1 \acs{neg} eat breakfast \\
        \glt \translate{I didn't eat breakfast. (I may usually do, but somehow I didn't today.)}
    \end{xlist}
    \label{ex:buchi-meichi}
\end{exe}

\begin{exe}
    \ex \begin{xlist}
        \ex \gll 我 做 [ 不 了 ]_{\text{potential complement, negative}} 这 件 事 \\
        1 do {} \acs{neg} finish {} this \acs{classify} affair \\
        \glt \translate{I'm not able to do this.}
        \ex[*]{\gll 我 \oneof{没有/并非/不} 做 [ 得 了 ]_{\text{potential complement, positive}} 这 件 事 \\
        1 \acs{neg} do {} \asis{de} finish {} this \acs{classify} affair \\}
    \end{xlist}    
    \label{ex:zuobuliao-example}
\end{exe}

\section{Sentence final particles}

\section{The topic-comment structure}

I follow \citet{sih2000topic}'s approach and define a topic as an unmarked \acs{np} 
that has certain relations with a position in the clause after it
and is indeed the topic in the information structure
(i.e. some (probably already known) object to which new information is added).
Constructions like 连\dots都\dots are not discussed in this section -- 
they are to be found in TODO: ref.

\subsection{Topicalization of possessor}

\eqref{ex:tagezigaogaode} and \eqref{ex:tagezigaogaode} are a pair of sentences 
with and without topicalization of the possessor in the subject.

\begin{exe}
    \ex \begin{xlist}
        \ex\label{ex:tagezigaogaode}  
        \gll [他]_{\text{topic}} [ [个子]_{\text{subject}} 高高 的 ]_{\text{comment}} \\
        3sg {} stature tall\redp{}\asis{todo} \asis{de} \\
        \glt \translate{As for him, the stature is tall.}
        \ex\label{ex:tadegezigaogaode} \gll [ 他 的 个子 ]_{\text{subject}} 高高 的 \\
        {} 3sg \acs{possessive} stature {} tall\redp{}\asis{todo} \asis{de} \\
        \glt \translate{His stature is tall.}
    \end{xlist}
\end{exe}

\subsection{Topicalization of preposition objects}\label{sec:topicalization-of-preposition-objects}

\begin{exe}
    \ex\label{ex:zhejianshinibunengjiumafantayigeren} 这件事你不能就麻烦他一个人
    \ex 你不能[为了这件事]_{\text{adverbial:\acs{pp}}} 就麻烦他一个人
\end{exe}
This is also a demonstration of the preposition status of 在 in this sentence (\prettyref{sec:preposition-pos}),
because if it's a verb or an auxiliary verb,
it will be hard to have its object topicalized and have it deleted at the same time,
but deletion of the preposition in topicalization is well-attested cross-linguistically.

\subsection{Rejecting the notion of dangling topics}\label{sec:topic-subject}

Some people, like \citet[\citesec{7.1}]{zhudexigrammar},
equate \term{subject} with \term{topic} in Mandarin grammar.
Some (especially those from the functional-typological tradition) go further 
and assert that ``the notion of the subject (as the position of the most agentive argument) 
isn't grammaticalized in Mandarin Chinese'',
and therefore the topic is just an \acs{np} which the comment is ``about'',
and this base-generated and syntactically unconstrained topic 
is called a ``dangling topic''.
This view is rejected in this note,
because such accounts usually end up in severe overgeneration. 
Here I briefly summarize \citet{sih2000topic}'s argumentation.

\subsubsection{Type 1: Idiomatic phrasal predicate looking like a comment} 

In the first type of ``dangling topic'',
it's impossible for any \acs{np} in the comment to be syntactically related to the topic.
Such cases are however rather unproductive. 
In \eqref{ex:dayuchixiaoyu} and \eqref{ex:nikankanwo},
the orders of the constituents can never be changed.
Nor is it possible to change a word or two in the bracketed ``comments''.
A reasonable assumption is these bracketed ``comments''
are actually idioms, 
which are to be regarded as a single verbal element that can't be further analyzed.
Thus, in \eqref{ex:dayuchixiaoyu} and \eqref{ex:nikankanwo},
the so-called topic is an ordinary subject,
and the so-called comment is a predicate.

\begin{exe}
    \ex\label{ex:dayuchixiaoyu} 他们[大鱼吃小鱼](,厮杀成一片)
    \ex\label{ex:nikankanwo} 他们[你看看我我看看你]
\end{exe}

\subsubsection{Type 2: Quantificational adverbial looking like the inner subject}

The second type of ``dangling topic'' is like \eqref{ex:shui-dou-bu-pa}.
A topic-comment analysis of \eqref{ex:shui-dou-bu-pa} 

\begin{exe}
    \ex\label{ex:shui-dou-bu-pa} \gll 他们 谁 都 不 怕 \\
    3pl who even \acs{neg} fear \\
    \glt \translate{They don't fear anyone.}
\end{exe}

\subsubsection{Type 3: Ellipsis leaving a subject and one predicate}

Some people accept \eqref{ex:nasuofangzixingkuimeixiaxue}.
Here the \acs{np} 那所房子 definitely doesn't come from the words following it,
and is therefore recognized as a topic by some (TODO: ref). 
Note, however, that 幸亏 serves as a clause linker outside \eqref{ex:nasuofangzixingkuimeixiaxue}:
\eqref{ex:xingkui-buran-ex} is a demonstration of the 幸亏……不然…… linking construction,
and we also have its topicalized version \eqref{ex:xingkui-buran-fronted}. (TODO: whether this is parenthesis)
We also know in a clause linking construction,
often one clause can be omitted in the utterance because it's content can be easily inferred (TODO: ref).
So now the origin of \eqref{ex:nasuofangzixingkuimeixiaxue} is clear:
We can get it by omitting the second clause in the comment part of \eqref{ex:xingkui-buran-fronted}.
Indeed, if we replace 幸亏 by anything that is adverbial but not a clause linker,
the resulting sentence -- which now contains a real dangling topic -- is not grammatical.

\begin{exe}
    \ex \label{ex:nasuofangzixingkuimeixiaxue} \gll \% 那 所 房子 幸亏 没 下雪 \\
    {} that \acs{classify} house fortunate \acs{neg} snow \\
    \glt \translate{For that house, fortunately it didn't snow (or otherwise something bad would happen).}

    \ex\label{ex:xingkui-buran-ex} \gll [幸亏] 去年 没 下雪 , [不然] 那 所 房子 早就 塌 了 \\
    fortunate last.year \acs{neg} snow {} otherwise that \acs{classify} house already collapse \acs{sfp} \\
    \glt \translate{Fortunately it didn't snow last year, or otherwise that house has already collapsed.}

    \ex\label{ex:xingkui-buran-fronted} 
    \gll [ 那 所 房子 ]_{\text{topic}} [ 幸亏 去年 没 下雪 , 不然 早就 塌 了 ]_{\text{comment}} \\
    {} that \acs{classify} house {} {} fortunate last.year \acs{neg} snow {}  otherwise already collapse \acs{sfp} \\
\end{exe}

\subsubsection{Type 4: Extraction from prepositional adverbials}

\eqref{ex:zhejianshinibunengjiumafantayigeren} in \prettyref{sec:topicalization-of-preposition-objects} 
is sometimes regarded as an instance of the dangling topic construction.
However, as is shown in \prettyref{sec:topicalization-of-preposition-objects},
it may just be from topicalization of an \acs{np} in an adverbial,
with the preposition (and/or the locative particle) removed.

\subsubsection{Type 5: Nominal predicate}

\begin{exe}
    \ex 这种青菜一斤三十块钱
\end{exe}

\subsubsection{Type 6: Locational adverbial mistaken for the subject}

\begin{exe}
    \ex \gll \% 物价 纽约 最 贵  \\
    {} price New.York most expensive \\
    \glt \translate{The price in New York is the most expensive.}
\end{exe}

\subsubsection{Tentative conclusion}

The conclusion is all topics in Chinese are closely linked to a position in the comment,
be it a core argument position or a peripheral one.
So the notion of dangling topics is to be rejected in Mandarin grammar,
and we can always recover the ``canonical'' i.e. non-topic-comment clause
from a topic-comment structure.
After this, if the canonical clause can be divided into an \acs{np}
or a complement clause and a verbal constituent following it,
we can uncontroversially say the first is the subject while the second is the predicate. (TODO: predicate def)
So equating the subject with the topic is also wrong.

It's possible to find the semantic role of the subject isn't agentive;
in this case I assert there is a valency changing mechanism here.

\begin{infobox}{What to expect when people talk about the subject or the topic}{subject-topic}
    Unfortunately, despite the syntactic tests presented above,
    there are still many people -- even many native speakers -- 
    promoting the idea that the Mandarin topic has nothing different with the subject.
    Here is a list of TODO: ref
\end{infobox}

\chapter{Relative clause constructions}

Due to 

\chapter{Complement clause constructions}\label{sec:complement-clause}


\begin{infobox}{Non-existence of finite-nonfinite distinction in Mandarin}{finiteness}
    Cross-linguistically, we find a finite-nonfinite distinction in subordination.
    This distinction is arguably absent in Mandarin,
    even after detailed syntactic tests \citep{no-finite}.
\end{infobox}

\chapter{Clause linking}\label{chap:clause-linking}

\bibliographystyle{plainnat}
\bibliography{references/grammars,references/aspects,references/general-typology,references/controversy}

\end{document}