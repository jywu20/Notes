\documentclass[UTF8, a4paper, oneside, scheme=plain]{ctexrep}

\usepackage{libertinus}
\usepackage{geometry}
\usepackage{float}
\usepackage{titling}
\usepackage{titlesec}
\usepackage{paralist}
\usepackage{footnote}
\usepackage{enumerate}
\usepackage{amsmath, amssymb, amsthm}
\usepackage{gb4e}
\noautomath
\usepackage{bbm}
\usepackage{textcomp}
\usepackage{soul}
\usepackage{graphicx}
\usepackage{siunitx}
\usepackage[table,xcdraw]{xcolor}
\usepackage{tikz}
\usepackage[ruled, vlined, linesnumbered, noend]{algorithm2e}
\usepackage{xr-hyper}
\usepackage[colorlinks, citecolor = purple]{hyperref} % linkcolor=black, anchorcolor=black, citecolor=black, filecolor=black
\usepackage[most]{tcolorbox}
\usepackage{caption}
\usepackage{subcaption}
\usepackage{booktabs}
\usepackage{multirow}
\usepackage[figuresright]{rotating}
\usepackage{acro}
\usepackage[citestyle=authoryear,backend=bibtex,natbib=true,doi=false,isbn=false,url=false]{biblatex}
\addbibresource{references/grammars.bib}
\addbibresource{references/aspects.bib}
\addbibresource{references/general-typology.bib}
\addbibresource{references/controversy.bib}
\usepackage{prettyref}

\geometry{left=3.18cm,right=3.18cm,top=2.54cm,bottom=2.54cm}
\titlespacing{\paragraph}{0pt}{1pt}{10pt}[20pt]
\setlength{\droptitle}{-5em}

\DeclareMathOperator{\timeorder}{\mathcal{T}}
\DeclareMathOperator{\diag}{diag}
\DeclareMathOperator{\legpoly}{P}
\DeclareMathOperator{\primevalue}{P}
\DeclareMathOperator{\sgn}{sgn}
\newcommand*{\ii}{\mathrm{i}}
\newcommand*{\ee}{\mathrm{e}}
\newcommand*{\const}{\mathrm{const}}
\newcommand*{\suchthat}{\quad \text{s.t.} \quad}
\newcommand*{\argmin}{\arg\min}
\newcommand*{\argmax}{\arg\max}
\newcommand*{\normalorder}[1]{: #1 :}
\newcommand*{\pair}[1]{\langle #1 \rangle}
\newcommand*{\fd}[1]{\mathcal{D} #1}

\newcommand*{\citesec}[1]{\S~{#1}}
\newcommand*{\citechap}[1]{chap.~{#1}}
\newcommand*{\citefig}[1]{Fig.~{#1}}
\newcommand*{\citetable}[1]{Table~{#1}}
\newcommand*{\citepage}[1]{p.~{#1}}
\newcommand*{\citepages}[1]{pp.~{#1}}
\newcommand*{\citefootnote}[1]{fn.~{#1}}

\newrefformat{sec}{\citesec{\ref{#1}}}
\newrefformat{fig}{\citefig{\ref{#1}}}
\newrefformat{tbl}{\citetable{\ref{#1}}}
\newrefformat{chap}{\citechap{\ref{#1}}}
\newrefformat{fn}{\citefootnote{\ref{#1}}}
\newrefformat{box}{Box~\ref{#1}}
\newrefformat{ex}{\ref{#1}}

% color boxes

\tcbuselibrary{skins, breakable, theorems}

\newtcbtheorem[number within=chapter]{infobox}{Box}{
    enhanced,
    boxrule=0pt,
    colback=blue!5,
    colframe=blue!5,
    coltitle=blue!50,
    borderline west={4pt}{0pt}{blue!65},
    sharp corners,
    fonttitle=\bfseries, 
    breakable,
    before upper={\parindent15pt\noindent}}{box}
\newtcbtheorem[number within=chapter, use counter from=infobox]{theorybox}{Box}{
    enhanced,
    boxrule=0pt,
    colback=orange!5, 
    colframe=orange!5, 
    coltitle=orange!50,
    borderline west={4pt}{0pt}{orange!65},
    sharp corners,
    fonttitle=\bfseries, 
    breakable,
    before upper={\parindent15pt\noindent}}{box}
\newtcbtheorem[number within=chapter, use counter from=infobox]{learnbox}{Box}{
    enhanced,
    boxrule=0pt,
    colback=green!5,
    colframe=green!5,
    coltitle=green!50,
    borderline west={4pt}{0pt}{green!65},
    sharp corners,
    fonttitle=\bfseries, 
    breakable,
    before upper={\parindent15pt\noindent}}{box}

\newcommand*{\concept}[1]{\textbf{#1}}
\newcommand*{\term}[1]{\emph{#1}}
\newcommand{\corpus}[1]{\emph{#1}}

\newcommand{\redp}{\textasciitilde}

\DeclareAcronym{blt}{short = BLT, long = Basic Linguistic Theory}
\DeclareAcronym{cgel}{short = CGEL, long = The Cambridge Grammar of the English Language}
\DeclareAcronym{dm}{short = DM, long = Distributed Morphology}
\DeclareAcronym{tag}{long = Tree-adjoining grammar, short = TAG}
\DeclareAcronym{sfp}{long = sentence-final particle, short = \textsc{sfp}}
\DeclareAcronym{np}{long = noun phrase, short = NP}
\DeclareAcronym{vp}{long = verb phrase, short = VP}
\DeclareAcronym{pp}{long = preposition phrase, short = PP}
\DeclareAcronym{cls}{long = classifier, short = CLS}
\DeclareAcronym{dist}{long = distal, short = DIST}
\DeclareAcronym{prox}{long = proximate, short = PROX}
\DeclareAcronym{dem}{long = demonstrative, short = DEM}
\DeclareAcronym{classify}{long = classifier, short = \textsc{cl}}
\DeclareAcronym{dur}{long = durative, short = DUR}
\DeclareAcronym{neg}{long = negative, short = \textsc{neg}}
\DeclareAcronym{cc}{long = copular complement, short = CC}
\DeclareAcronym{cs}{long = copular subject, short = CS}
\DeclareAcronym{tame}{long = {tense, aspect, mood, evidentiality}, short = TAME}
\DeclareAcronym{past}{long = past, short = PST}
\DeclareAcronym{nonpast}{long = non-past, short = NPST}
\DeclareAcronym{present}{long = present, short = PRES}
\DeclareAcronym{progressive}{long = progressive, short = \textsc{poss}}
\DeclareAcronym{perfect}{long = perfect, short = \textsc{perf}}
\DeclareAcronym{passive}{long = passive, short = \textsc{pass}}
\DeclareAcronym{copula}{long = copula, short = COP}
\DeclareAcronym{possessive}{long = possessive, short = \textsc{poss}}

\newcommand{\asis}[1]{\textsc{#1}}
\newcommand{\oneof}[1]{{#1}}
\newcommand*{\homo}[2]{#1$_{\text{#2}}$}

\newcommand{\cgel}{\href{../English/cambridge.pdf}{my notes about CGEL}}
\newcommand{\latin}{\href{../Latin/latin-notes.pdf}{my notes about Latin}}
\newcommand{\alignment}{\href{../alignment/alignment.pdf}{my notes about alignment}}
\newcommand{\exerciseone}{\href{../Exercise/2021-3.pdf}{this exercise}}
\newcommand{\method}{\href{../methodology/glossing.pdf}{this note about my understanding of descriptive grammars}}

\newcommand{\ala}{à la}
\newcommand{\translate}[1]{`#1'}
\newcommand{\vP}{\textit{v}P}
\newcommand*{\category}[1]{\textsc{#1}}

% Make subsubsection labeled
\setcounter{secnumdepth}{4}
\setcounter{tocdepth}{4}
% reset example counter every chapter (but do not include the chapter number to the label)
\counterwithin{exx}{chapter} 

% Reference formats
\renewcommand*{\nameyeardelim}{\space} % No comma between year and name
\DeclareNameAlias{sortname}{family-given} % Putting the family name before the given name
\DeclareNameAlias{default}{family-given} 
\DeclareFieldFormat{labelnumberwidth}{} % No number label like [12] in the reference list
\setlength{\biblabelsep}{0pt} % No space for these labels

\title{Mandarin Chinese notes}
\author{Jinyuan Wu}

\begin{document}

\maketitle

\automath

\chapter{Introduction}

\section{The language and the speakers}

Mandarin Chinese is a predominant language in the world,
belonging to the Sinitic family.
Indeed,  

\section{Previous studies and theoretical orientation of this work}

\subsection{Structuralist generative grammars}\label{sec:previous.structuralist}

There are already sufficient works concerning the grammar of Mandarin Chinese.
The tradition used in colleges (and occasionally high schools) 
is largely structuralist \ala{} Bloomfield:
A clause is divided into a topic and a comment,
and the comment is divided into a subject and a predicate,
and the predicate is divided into a predicator and an object, etc.
Examples of works in this tradition usually have names like 现代汉语 \translate{Modern Chinese}
\citep{xianhan2004}.
This tradition is still seen in many contemporary grammars,
like \citet{cgel}.
This tradition is largely coherent with the generative tradition,
and indeed there are books which are generative in essence 
but organized in the traditional and structuralist framework \citep{deng2010formal}.

\subsection{Teaching materials}

Teaching materials of Mandarin Chinese are also largely influenced by the structuralist tradition.
TODO

\subsection{Mandarin in the functional-typological tradition}

Mandarin also gains much attention in the functional-typological tradition.
\citet{li1989mandarin} is a ``functional'' reference grammar of Mandarin, TODO: more references
The reason for Mandarin's popularity seems to be the fact that 
it breaks many previous typological generalizations about isolating languages and constituent orders
\citep[\citechap{8}]{paul2014new}.

\subsection{Theoretical commitment of this work}\label{sec:theory}

This notes attempts to reconcile the several approaches in previous researches.
It's my belief that the differences between the generative tradition, 
the traditional structuralist school 
and a large part of the so-called functionalist 
(or ``Basic Linguistic Theory'' \citep{dixon2009basic}) 
are mainly notational from a practical perspective.
Below, by the capitalized Generativism,
I mean a mixture of Minimalism, Distributed Morphology and Cartography
(but with less flavor of Antisymmetry),
which I mean are instructive on grammar description.
This leaves out Lexicalist traditions, 
like the framework used in \citet{deng2010formal}.
Due to the limit of space I will not talk about 
why I believe a qualitative distinction between morphology and syntax 
doesn't work for natural languages; 
I only want to point out that at least in Mandarin, 
the inner structures of words and phrases 
have striking resemblance, 
and the boundary between words and phrases 
is hard to draw, 
and this note has to rely on prosodic factors to draw the line
(\prettyref{sec:pos.word.phonological}).

What I want to do here, then, 
is to incorporate the new perspectives in generative syntax, 
such as \citet{paul2014new} and \citet{paul2008serial},
into the structuralist tradition
in an accessible and typology-informed way.

Since modern generative syntax contains lots of hidden functional heads,
the notion of, say, a DP which is the specifier of T,
is to be replaced by an \acs{np} filling a subject position 
in a surface-oriented structuralist constituency analysis. 
Both ``\acs{np}'' and ``subject'' should be labeled on a sub-syntactic tree
in the structuralist tradition,
while in generative syntax,
the label ``subject'' is a secondary concept:
it's an abbreviation of SpecTP (or in an even more fine-grained way, SpecSubjP or SpecNomP).
So functional heads can be replaced by syntactic function labels:
after eliminating the syntactic functional heads, 
the label SpecNomP should be replaced by ``subject'',
and the label NomP should be replaced by ``subject-predicate structure''.
On the other hand,
in Distributed Morphology we have roots,
which reside at the center of an extended verbal or nominal projection
(NP-NumP-DP, or vP-TP-CP),
and they are recognized as \term{heads} 
(in this note referred to as \concept{lexical heads})
in traditional structuralism.
This unifies the notation of
\citet{cgel}, \citet{chao1965grammar}, \citet{zhudexigrammar}
in the traditional structuralist perspective
and Generativism.

The \ac{blt} approach \citep{dixon2009basic}, 
or the grammar writing approach accepted in 
modern descriptive linguistics and typology,
is based on \emph{dependency relations} on the other hand.
It's also possible to formulate generative grammar in terms of dependency relations,
so it can be expected that the \acs{blt} approach is still largely equivalent 
with the grammatical complexity of generative grammar.
\acs{blt} only recognizes two major types of constituents:
\acs{np}s and clauses,
which are essentially \term{domains} or \term{fields} in generative syntax:
The former is the DP domain while the latter is the \vP{}-TP-CP domain.
The complicated binary constituency tree is replaced by a flat tree 
with lots of dependency relations labeled inside,
containing the same amount of information.
(We may also say the \acl{blt}'s standard of constituency 
is ``being a relatively independent construction''. 
Then the equivalence between this version of mild constructivism and minimalism 
is clear \citep{construction-minimalism}.)
Note that the \acl{blt} approach still (although quite implicitly) admits 
that there is a rank of ``closeness'' or ``height'' among dependency relations: 
the A argument seems to be somehow higher than the O argument, etc.,
and this information is conveyed by the fine-grained constituency relations 
in Generativism.

The focus on dependency relation in \acs{blt} also leads to 
a notational difference on what is a constituent.
A sequence like \corpus{has been exploring} is not a constituent in the sense of 
the structuralist constituency test in grammars like \citet{cgel} and generativism,
but are still recognized as a unit in \acs{blt} 
because its parts always appear together in the surface-oriented analysis.
This what is \citet[\citepage{109}]{dixon2009basic} defines 
as a \term{verb phrase},
which excludes the object. 
From the constituency-based point, 
such a ``phrase'' is usually a larger functional domain minus a smaller functional domain;
here it's the functional projection below the subject -- the verb phrase in \citet{cgel} --
minus the object.
This definition is meaningful 
because syntax is cyclic 
and the object is a DP and therefore is a phase, 
and above it is another phase, 
and although the introduction of the subject 
(by, say, a Nom head which may contain the nominative case) 
doesn't seal a phase, 
it still changes the properties of the syntactic tree, 
so what happen above the object and below the subject 
-- the \acs{tame} functional heads, the verb root, etc. -- 
are tightly linked to each other
(note that here we are just translating between 
the constituency-based analysis and the dependency-based analysis), 
so from a dependency relation-oriented analysis, 
the sequence \corpus{has been exploring} does qualify as a phrase; 
in a constituency grammar 
we recognize this sequence has a status 
but we don't call it a phrase or a constituent.
This difference is notational but may be confusing.

A further difference between the \acl{blt} and Generativism (as well as the structuralist tradition)
is the former is claimed to be semantic-based.
But first, of course it's possible to have a meaning-first version of generative syntax,
and second,
there seems to be a ``gluing'' layer between pure semantics and 
the phonetic realization 
and can't be equated with either of the two: 
the semantics of complement-taking verbs 
may be coded as a complement clause construction, 
a relative clause construction 
(compare \corpus{I see a man running} and \corpus{I see a running man}),
and superficially similar utterances may have different ``structures''.
This is recognized by Dixon, 
who distinguishes the ``prototypical'' coding strategy of a semantic concept 
and other strategies.
So there is also no substantial disagreement here.

The Generativism I take here doesn't emphasize on wordhood as a universal concept,
and indeed this is what I want here:
A word is simply a mini phrase (in the \acs{blt} sense),
either the realization of a mini constituency tree,
or the realization of a span of functional heads 
and possibly the lexical head or in other words the root.
The controversy about what is a word in Mandarin 
has been around for decades,
which I believe is due to the desire to 
find \emph{the} word as a universal unit,
without thinking of the tenet of Generativism 
that phonetic realization doesn't always 
transparently reflect in the syntax proper,
and that what is universal is likely to be 
prototypes of functional heads and how they are arranged together;
this means we may have categorizer phrases like nP or vP,
which can be recognized as minimal words,
but whether things like compound words are recognized as words or phrases 
depends more on grammatical traditions
and external factors like prosody.

I need to note here that 
many generative works, like \citet{deng2010formal},
are \emph{lexicalist}:%
\footnote{
    The term \term{lexicalist} sometimes means that  
    it's lexical words like verbs or nouns that carry grammatical structures with them, 
    and there is no separate phrase-structure rules 
    seen in early Chomskyan generative works.
    Since in Distributed Morphology, 
    we can use whether a root can be well spelt out 
    with or without certain functional heads surrounding it 
    to control its subcategorization,
    Minimalism is also lexicalist in this sense:
    to say that the root of an intransitive verb 
    only gets spelt out together with a Trans head 
    is equivalent to say that the verb carries a verb-object structure with it.
} 
they made a clear distinction between what happens in lexicon 
and what happens in syntax. 
Due to space limit I will not discuss in detail
the problems with their assumptions;
I only want to point out that advocates of the lexicalist hypothesis 
often can't be consistent:
on \citepages{242}, \citet{deng2010formal} says the syntactic structures 
can also be found in morphology,
while on \citepage{262} he says 
the syntactic structures listed above 
can be reduced to the X-bar theory.
This means it's highly likely that the inner structure of words 
also fits in the X-bar framework,
which in turn means his classification of syntactic structures 
based on specifier-complement distinction is not accurate enough:
a better approach is to divide syntactic structures into domains, 
and thus grammatical relations in a lower domain look like 
head-complement relations in the traditional X-bar theory, 
and grammatical relations in a higher domain look like 
specifier-head relations in the traditional X-bar theory.
TODO: adjunct 

\section{Origin of data and how to represent them}

\subsection{Intuition}

One problem is scholars working on Mandarin grammar 
have their own judgement on what is acceptable and what is not, 
which is sometimes not shared by the majority of Mandarin speakers.
Some works, like \citet{huang2013}, 
are heavily criticized for 
representing not empirical observations of Mandarin
but distorted and man-made examples 
which are used to ``support'' a pre-accepted theory.
My personal opinion is this reflects individual varieties, 
which is probably influenced by non-Mandarin varieties of Chinese,
instead of academic misconduct.
We can find sentences that go against the intuition of most Mandarin speakers 
in the main text -- as opposed to examples -- of their works as well, 
which implies these authors are probably 
sincere about the acceptability of their ``weird'' examples.
Speaking of Huang, (\prettyref{ex:weird-1}) appears in \citet{huang2007},
which doesn't sound acceptable for me but seems to be completely fine for him.

\begin{exe}
    \ex\label{ex:weird-1} 我们建议汉语动词具有下列特征而有别于英语动词 \citep{huang2007}
\end{exe}

I will use less structuralist constituency trees compared with structuralist grammars.
That's to say, for example, 
the ``serial verb construction'' is not divided into two verb phrases 
following rigidly the structuralist tenets,
but is analyzed with a flat-tree structure instead,
following \acs{blt}.
But I will also do lots of in-depth morphosyntactic tests,
instead of just staring at the surface realization,
as many typologists may do.


\section{Plan of the book}

\section{Remarkable features of Mandarin}

\subsection{The overwhelming influence of prosody}

One distinct feature of Mandarin is its morphosyntax relies strongly on \emph{prosody} \citep{feng2000}. 
Other components in phonology, strikingly, 
doesn't have much influence on Chinese morphosyntax,
and it will be largely skipped in this note.

\subsection{Lack of word (and therefore morphology)?}

Despite of lack of inflection
and lack of contextual alternation of morphemes,
Chinese does have some local and syntactically unmotivated operations
which are just like morphophonological rules,
although they don't necessarily operate on phrases.

An example of this is the verb copying phenomenon,
as in 看了一会书 (compare 看书了一会);
看书 \translate{to read} (intransitive; lit. \translate{to read books}) 
is a fossilized verb-object structure 
and this verb-object structure may still have synchronic effects.
More radical examples however also exist,
like \%体了一堂操, which is likely to be linked to 
[[体操]_{\text{noun-as-verb}}了[一堂]_{\text{time object}}]_{\text{\acs{vp}}}.
In casual speech,
verbs borrowed from other languages may also be split 
and the two fragments of the verb then surround the semi-object
(\prettyref{ex:remarkable.debug}).

\begin{exe}
    \ex\label{ex:remarkable.debug} 我 [debug]_{\text{topic:{\acs{vp}}}} [de不出来]_{\text{predicate:\acs{vp}}} [啊]_{\text{\acs{sfp}}}
\end{exe}

This phenomenon -- the verb being split and the time semi-object getting embedded into the verb -- 
looks just like infixing,
although here this infixing operation targets a \acs{vp} instead of a smaller unit.
This justifies the assumption taken at the end of \prettyref{sec:theory}
that there is no clear boundary between words and phrases 
and therefore syntax and morphology:
It's possible for a phrase to undergo 
rearrangement without clear syntax motivation
that usually happens within a word.

\subsection{The so-called serial verb constructions}

It's often said Mandarin is a serializing (i.e. with serial verb constructions) language.
A closer look, however, reveals this is not the case:
These constructions are either adverbial clause constructions 
or complement clause constructions, 
or maybe certain kind of light verb constructions (\prettyref{sec:no-serial-verb}).
The internal heterogeneity renders the term \term{serial verb construction} useless.

\subsection{Ionized verbs}

Some verbal units -- commonly recognized as verbs, i.e. grammatical words -- in Mandarin
may be ionized into two parts, 
with a constituent residing between the two. 
This involves two mechanisms:
the first is TODO: 念佛-like, in which a verb phrase is fossilized, 
similar to English \corpus{in case} or \corpus{by definition},  
but its object can be extended into a complex \acs{np},
and the second seems to be a morphophonological device, 
by which even a verb without synchronically analyzable inner structure 
is teared into two parts and a clausal dependent is inserted between the two (TODO: ref).
Despite its striking properties for English speakers, 
ingredients of this phenomenon are all well attested cross-linguistically.

\subsection{Voices}

TODO: several passive constructions

\chapter{Phonology and the writing system}

\section{Prosody}\label{sec:prosody-structure}

By paying attention to stops in Chinese utterances,
it can be found that phonological words exist and they are mostly defined by the prosody structure.
In the rest of this note,
the term \term{prosodic word} and \term{phonological word}
will be used interchangeably.
The prosody structure is about how stress is assigned to phonological constituents.
Assigning a prosodic structure is like condensation and clustering:
something is merged with something adjacent,
and the result is merged with something adjacent else.
When two phonological constituents are merged together,
one of them is considered heavier than the other.
If heaviness is to have a simple relation with the length of a phonological constituent,
then usually the more a phonological constituent is,
the heavier it is.
This is consistent with the condensation picture of prosodic segmentation.
Suppose a prosodic constituent attracts a syllable and merges with it.
The latter is not an independent phonological constituent
and cannot be heavy,
so the former is the heavier one and the latter is the lighter one in the larger prosodic constituent.

The smallest unit of prosody structure 
is a prosodic word.
The simplest prosodic word is the disyllabic foot, 
which contains two adjacent syllables in the case of Chinese.
(It can be made by two moras in other languages.)
One is assigned stress and is therefore heavier than the other.
Trisyllabic prosodic words also exist in Chinese,
though they are highly limited.
Most of which are borrowed words (e.g. 加拿大 \translate{Canada})
or words formed by coordinating three morphemes (e.g. 数理化 \translate{math, physics, and chemistry}).
They can also be regarded as foots \citep[\citesec{2.2}]{feng2000}.

Longer morphosyntactic units are 
inevitably broke into smaller disyllabic or trisyllabic prosodic words
in their prosodic structures,
often regardless of their morphosyntactic structure:
加利福尼亚 may be segmented into 加利\textbar 福尼亚, 
although the word contains only one morpheme.
In 副总经理,
we have two prosodic words,
副总 and 经理,
while the morphosyntactic structure of the word is [[副] [总 [经理]]]
(\prettyref{sec:pos.noun.adj-modify}).
This is similar to the case in English and Latin poems,
where the prosody arrangement of sentences does not have to respect word boundaries:
\corpus{arma vi\textbar rumque ca\textbar no}.
It's however also possible that 
a prosodic word has morphosyntactic significance
(\prettyref{sec:pos.word.phonological}).

Prosody is able to see the constituency structure
and prosodic constraints are important in Mandarin grammar.
Some prosodic rules pertaining to the constituency tree 
guide and limit the assignment of relative heaviness and lightness.
In Chinese, prosodic segmentation is done strictly left-to-right 
in each \ac{np},
and then the \ac{np}s together with verbal constituents 
are used as the input of prosodic segmentation of clauses.
Certain forms are therefore ruled out
(\prettyref{sec:clause.prosodic-constraint}), 
not by morphosyntactic reasons but for prosodic reasons.

\section{Chinese characters}\label{sec:chinese-character}

The preferred writing system of Mandarin is the Chinese character system.
Except some characters made in early modern ages,
like 兛 \translate{kilogram} or 砼 \translate{concrete (lit. human-labor stone)},
a Chinese character corresponds to a syllable.
However, Chinese characters don't just represent the sound.
Putting some quirky cases aside,
Chinese characters are often good indicators of morphemes
(\prettyref{sec:pos.morpheme.primitive}).
There are, for example, at least seven morphemes sounding \corpus{xi\={a}n},
and there happens to be seven Chinese characters corresponding to each of them:
仙, 先, 籼, 掀, 锨, 鲜, and 纤.

Like all writing systems, 
Chinese characters do not completely faithfully represent 
the underlying linguistic structure.
Some characters do not mean anything -- 
they are simply the designated characters representing syllables 
in certain polysyllabic morphemes.
The character 萄 as in 葡萄, for example, 
means nothing more than the syllable \corpus{t\'{a}o},
but it only appears in the morpheme 葡萄 and 葡萄牙 \translate{Portuguese}.
The same is for the character 葡.
Some characters have regular morpheme meanings
but also have merely phonetic meaning in certain words.
The character 登 in 摩登 regularly means \translate{climb},
but in the word 摩登, only its phonetic value \corpus{d\={e}ng} is preserved.
Certain morphemes can be denoted by more than one character.
The \ac{sfp} \corpus{ba} can be written as 吧 or 罢,
the latter hinting its etymology but is now rarely used.
Certain characters denote more than one morpheme.
The character 会 may mean \translate{conference} or \translate{be able to do}. 

Thus, Chinese characters provide clues on what is a morpheme,
but they are not decisive \citep[1.1.4]{zhudexigrammar}.

\chapter{Parts of speech}

\section{From morphemes to clauses: levels of units in Mandarin morphosyntax}

\subsection{Morphemes}\label{sec:pos.morpheme}

A morpheme is a minimal unit in grammatical analysis.
The meaning of the term has some ambiguity:
sometimes, a grammatical item has a clearly analyzable inner structure 
but the structure no longer has any morphosyntactic significance
(\prettyref{sec:pos.morpheme.fossilization}, \prettyref{sec:pos.morpheme.abbreviation}).
In this case, we say the item is \emph{synchronically} a morpheme.

I start the introduction of Mandarin morphosyntax 
with morphemes for good reasons. 
Although in most language documentation projects, 
words -- whatever this term mean (\prettyref{sec:pos.word}) -- 
are the smallest unit appearing in the dictionary, 
this is not the case with Mandarin:
the standard practice of lexicography 
is to record Chinese characters and their ``meanings''; 
in linguistic terms, 
an entry of a Chinese character includes the follows:
its pronunciation(s); 
historical or synchronic monosyllabic morphemes that may be represented by that character
and polysyllabic morphemes containing that character,
which may have different pronunciations;
(\prettyref{sec:pos.morpheme.primitive},
\prettyref{sec:pos.morpheme.abbreviation},
\prettyref{sec:pos.morpheme.fossilization}); 
grammatical words that are made up regularly using one of the morphemes listed above
and have already gained a stable meaning.
Thus, for someone who wants to do a thorough grammatical analysis 
of an utterance, 
finding the morphemes -- instead of grammatical words recognized in \prettyref{sec:pos.word} -- 
is the first step.

\subsubsection{Primitive content morphemes}\label{sec:pos.morpheme.primitive}

Most native primitive content morphemes 
are monosyllabic, examples of which include 红, 大, 你, etc.
(\prettyref{sec:pos.word.monosyllabic}).
Polysyllabic primitive content morphemes,
like 葡萄, 巧克力, 哥斯达黎加 etc., 
are mostly borrowed words
in different historical stages.
There do exist seemingly native polysyllabic primitive content morphemes,
like TODO: historical analysis of 轱辘, etc.

Only a subset of the monosyllabic morphemes are able to serve as grammatical words.
Some monosyllabic morphemes only appear in in-word slots 
(like the modifier slot in \eqref{ex:nominal-modifier-2}),
and they are unable to serve as words.
Many of them are historically free morphemes
but have become obsolete in contemporary usage.
The morpheme 观 \translate{observe}, for example,
still exists in 观鸟 \translate{bird watching}
but never appears as a single verb, 
nor does it undergo delimitative reduplication of verbs
(\prettyref{ex:pos.obsolete-1}).

\begin{exe}
    \ex\label{ex:pos.obsolete-1} \begin{xlist}
        \ex *我要去观观那些鸟
        \ex 我要去看看那些鸟
    \end{xlist}
\end{exe}

It should be noted that bound morphemes are not a homogenous class.
Specifically, that a unit contains a bound morpheme as its immediate constituent
doesn't mean that unit is a grammatical word.
\citet[\citesec{8.3.2}]{zhudexigrammar} notes that 
despite 亏 is a bound morpheme and 饭 is a free morpheme,
verb-object constructions 吃饭 and 吃亏 has largely similar morphosyntactic behaviors,
in which the object 亏 may be modified just like any other \acs{np}, 
as in 吃了个大亏.
We can even move 亏 out of the verb-object structure
and topicalize it (\prettyref{ex:pos.morpheme.chikui}; 
\citet{zhudexigrammar} seems to be unaware of this fact).
Thus, we reject the analysis given by \citet[\citesec{8.3.2}]{zhudexigrammar} himself
that 吃亏 is a verb because it contains a bound morpheme, 
and conclude that 吃亏 is to be regarded 
as a usual \acs{vp},
enjoying the same status of 吃饭.%
\footnote{
    Still, this doesn't answer the question whether disyllabic verb-object structures 
    have something significantly different 
    from longer verb-object structures. TODO
}
Cross-linguistically, 
it's quite common for some words to appear mainly in idioms, 
but this doesn't mean syntactically
these idioms are words.

\begin{exe}
    \ex\label{ex:pos.morpheme.chikui} \begin{xlist}
        \ex 我吃了这个亏
        \ex 这个亏我今天吃了
    \end{xlist}
\end{exe}

The boundary between free and bound primitive content morphemes 
seems to vary among registers and conversational context. 
(\prettyref{ex:pos.morpheme.kui-2}) is usually not acceptable,
but if the idiom \acs{vp} 吃亏 has appeared frequently enough, 
it gradually becomes acceptable.
(And in this case, the fact that 亏 usually doesn't appear as a noun 
even has a focusing effect.)
(\prettyref{ex:pos.morpheme.quan}) is not acceptable in daily conversation,
because the morpheme 犬 is usually a bound morpheme 
and should be displaced by the more colloquial 狗.
The sentence however is perfectly fine in 
a police officer's recollection of a detective story involving K-9 dogs.
Thus the professional background licenses 犬 as a free morpheme.

\begin{exe}
    \ex\label{ex:pos.morpheme.kui-2} 
    ? 这个亏让我记了一辈子
    \ex\label{ex:pos.morpheme.quan} 
    ? 当时我的犬发现现场的气味有点不对劲
\end{exe}

\begin{infobox}{The notion of \term{free morphemes}}{free-morpheme}
    Free morphemes are morphemes that can be words themselves.
    Of course, this involves the question what is a word. 
    We will see 

    Another definition seen in \citet[\citesec{1.1.2}]{zhudexigrammar}
    is that a free morpheme is a morpheme that may appear as an utterance.
    This definition contradicts with other discussions in \citet{zhudexigrammar},
    because most function words never appear as a single utterance.
    So this definition shouldn't be taken seriously.
\end{infobox}

An overwhelmingly portion of polysyllabic morphemes 
are able to constitute one-morpheme word and also one-word phrase.
There exist however a small number of polysyllabic morphemes 
that seem to be unable to be words themselves. 
The root 日耳曼 (from \corpus{German}) 
appears in compound nouns like 日耳曼人 \translate{Germanic people} 
or 日耳曼血统 \translate{Germanic descent}, 
but almost never appears as a noun itself; 
similarly, 达达 in 达达主义 also never appears as a word itself.

\subsubsection{Fossilization}\label{sec:pos.morpheme.fossilization}

A lot of words have internal structures parallel to 
those observed in syntax \citep[\citesec{2.6}]{zhudexigrammar},
but the structures have already completely fossilized, 
so they may be synchronically regarded as containing only one morpheme.

A completely fossilized structure 
may be historically created with an obsolete syntactic device, 
or be not in the expected part of speech inferred from its etymological
(i.e. it appears in syntactic environments that are not expected 
for its inner structure).
The verb 关心 has an internal verb-object structure, 
but is able to take an object.
Since well-attested double object constructions in Mandarin
are all unable to cover this usage,
we conclude 关心 has already been fossilized into one single synchronic morpheme.%
\footnote{
    It's still possible for 关心 to be split, 
    but this happens without considering the internal structure of the verb
    (\prettyref{sec:verb-splitting}).
}
Some words with fossilized phrasal origin 
contain gap inside, 
indicating that at some early stages 
they were parts of formulaic speeches 
and were later reanalyzed as words.
The word 例如 \translate{for example}
is a connective adverb, 
but it has a subject-predicate with a gap,
and likely arose from a reanalysis of the formula 
[[例]_{\text{subject}} 如 [\dots]_{\text{object}}]_{\text{clause}}
\translate{an example is like \dots}.
This also explains why it appears predominantly at the start of a clause.

One thing is worth mentioning concerning fossilization: 
although fossilization in syntactic structure 
is often connected with 
a conventionalized meaning and being small in size,
the three parameters are not that interdependent, 
although we can observe some weaker-than-expected correlation.
Regarding the relation between syntactic fossilization 
and the size of the unit in question, 
it should be noted that many languages have 
idioms that have archaic syntactic structures, 
like \corpus{till death do us part} in English 
(involving archaic verb-final clausal structures)
and 放心不下 (involving an early stage of a type of verbal complement structure) 
in Mandarin.
Regarding the relation between syntactic and semantic fossilization, 
note that most idioms -- what the word refers to in everyday speech -- 
are formed by regular syntactic devices and yet have gained conventionalized meanings, 
while it's also possible that some structures have already been semi-productive
and yet the meanings of their products can still be 
regularly inferred compositionally.
These facts pose an awkward problem to use: 
since the syntactic device giving rise to \corpus{till death do us part}
has already died in contemporary speech, 
should we claim that the whole expression is a morpheme?
And if not, how confident are we 
when we say some units with fossilized syntactic structures 
are all monomorphemic?
TODO: answer to the question

\begin{theorybox}{No generative rule in the lexicon}{no-generative-rule-lexicon}
    Those insisting on a universal word-phrase distinction 
    may say ``fossilized structures are assembled in the lexicon before syntax''.
    The position of this note, however, is if something is assembled synchronically,
    then it has to have something to do with syntax:
    syntax is the only productive engine.
    If a morphological device is completely invisible to the rest of the grammar,
    it is likely to have lost productivity and becomes historical; 
    its products are therefore synchronically morphemes, 
    instead of words with inner structures.
\end{theorybox}

\subsubsection{Abbreviation}\label{sec:pos.morpheme.abbreviation}

There is a strong tendency to make the abbreviation a prosodic word.
The abbreviation of a binary-branching structure
usually consists of the first syllables of its two immediate constituents,
regardless of the inner structures of the two immediate constituents.
Thus [副 [总经理]] \translate{vice general manager} is usually abbreviated as 副总, 
and 总工程师 is usually abbreviated as 总工.
It's possible that an abbreviation replaces the original word completely.
空调 is historically the abbreviation of 空气调节器, 
the word-by-word translation of \corpus{air conditioner},
but the latter is no longer in active use.

Trisyllabic prosodic words do exist in Mandarin, 
and trisyllabic abbreviations also exist.
The majority of them are three-morpheme coordination structures like 数理化, 
which is the abbreviation of 
数学物理化学 \translate{math, physics, and chemistry}.

\subsubsection{Function items and semi-function items} 

Another type of morphemes is the type of function items,%
\footnote{
    Here the term \term{grammatical word} 
    means a word defined in the grammar, not phonology;
    a \term{function item} means an item that is not lexical
    and therefore is a part of the grammar. 
    The adjective \term{grammatical} means being related to grammar
    and it has two explanations:
    being defined in grammar and being contained by the grammar; 
    In this note, I use the term \term{grammatical} to refer to the first meaning, 
    and \term{function} to refer to the second meaning.
} 
like function words, inflection suffixes (which are lacking in Mandarin).
Distinguishing between words and phrases 
means to classify morphosyntactic units 
according to their possible internal
grammatical relations and categories.
Thus, we distinguish between \acs{np}s and compound nouns, 
for they possess different internal structures 
(\prettyref{sec:pos.noun.compound}).
A function item, on the other hand, is a \emph{label} of a grammatical category
and has no inner structure.
Thus, defining whether something is a function word or a function morpheme
is purely based on \emph{external} factors:
if something appears in a morphosyntactic unit commonly referred to as a phrase, 
than it's a function word, 
and if it appears in a morphosyntactic unit commonly referred to as a grammatical word, 
than it's a bound function morpheme.
But there seems to be no particularly strong correlation 
between whether a function item is grammatically a function word
and whether it is phonologically a word: 
a suffix in a long grammatical word can be a phonological word, 
while a function item, like the Latin \corpus{-que}, that is an immediate constituent of a phrase 
may be glued to another phonological word. 
In general, the meaning of \term{function word} is not well-defined, 
and an accurate description of a function item 
inevitably involves both of its phonological status and its grammatical status.
That's why I refrain from discussing wordhood of function items in the following sections.

Monosyllabic location words, 
for example, appears in \acs{np}s 
and therefore should be analyzed as words;
they however never constituent one-word \acs{np}s themselves
(\prettyref{sec:pos.locational}).
Whether they are words therefore becomes a mystery,
and this mystery is merely a wrongly asked question.

Certain borrowed affixes may be unable to serve as words in certain periods.
As times goes by, however, they gradually become free morphemes.
If clause linking markers like 之所以 and 是因为 are recognized as single morpheme words,
then they may be included into the non-prosodic simple word blob 
and however be unable to serve as phrases.
These markers, however, never appear in other places,
and their exact status is of no descriptive and comparative interest.

\begin{infobox}{On some confusing notions of morpheme classification}{confusing-function}
    Many publications on Mandarin based on structuralism \ala Bloomfield 
    have some outdated, confusing, and not really necessary 
    notions on classifying morphemes.
    \citet[\citepage{16}]{zhudexigrammar} claims that 
    free morphemes don't have fixed positions, 
    while bound morphemes sometimes have fixed positions.
    Cross-linguistically, this is simply wrong: 
    in SOV languages like Japanese, 
    the position of the verb is predominantly 
    after all clausal dependents and before inflectional endings, 
    but we would all agree that verb roots are prototypical free morphemes.
    On the other hand, 
    bound morphemes like derivational suffixes 
    do have fixed positions, 
    but that simply comes from the fact that 
    they have fixed positions in the syntactic structure -- 
    and indeed so-called free morphemes can only occupy a limited number 
    of positions in the syntactic structure 
    (lexical head position, compounding attributive position, etc.)
    and this may lead to a constituent order effect, 
    as is shown in the above example of Japanese.
\end{infobox}

\subsection{The existence of words as mini-constituents}\label{sec:pos.word}

\subsubsection{Comparing grammatical wordhood and phonological wordhood}\label{sec:pos.word.phonological}

A question causing endless controversy and confusion 
is ``what is a word''. 
\ac{blt} spends a whole chapter (\citechap{10}) on this topic.
It is often said that Chinese is ``character-based''
or to be precise, ``monosyllabic morpheme-based'',
with no level of grammatical words.
This claim is factually flawed, 
since in Chinese, there \term{are} distinction between 
productive morphemes and words.
What should be noted are
that demarcation of phonological words does not always follow morphosyntactic structures
(\prettyref{sec:prosody-structure}),
and that there are subtleties concerning word-phrase distinction. 
These are introduced in the following sections.

We expect a grammatical word to be a mini-constituent, 
which is therefore easily subject to conventionalization, 
and has limited interaction with the syntactic environment,
although definitely not absolutely no interaction -- 
even in English, we have things like \corpus{pre- and post-processing}, 
in which a phrasal structure -- coordination -- interacts 
with the inner structures of two bare nouns, 
and therefore a clear boundary between word or phrase 
or between morphology and syntax is in principle impossible.
Thus, comparing morphosyntactic levels 
and \emph{phonological} wordhood 
may be a good idea for us to draw a boundary 
between words and phrases.
This helps us find native speakers' intuition 
about the smallest unit in natural (i.e. non-linguistic, not in language games, etc.) conversation,
which strongly influences how new words are created 
or how borrowing happens, etc. (\prettyref{sec:pos.word.perception}).

A phonological word,
which in Mandarin is defined by prosody (\prettyref{sec:prosody-structure}),
may be a single-morpheme with a well-defined part of speech tag (\prettyref{sec:pos.morpheme}),
like 幽默 \translate{humor (homophonic translation)} 
and relatively rare trisyllabic cases 
like 加拿大 \translate{Canada (homophonic translation)},
or a mini-constituent 
(\prettyref{sec:pos.architecture.word.mini-constituent}), 
like 白菜 \translate{Chinese cabbage (lit. white vegetable)} 
(\prettyref{sec:pos.noun.fossilized-structure})
or 种树 \translate{plant tree} (\prettyref{sec:pos.verb.idiomatic-verb-object}), 
or a sequence that according to 
the constituency-based analysis is not a grammatical unit 
but does package highly related items, 
like 交给 \translate{transfer-give (the latter being a verbal complement)} 
(\prettyref{sec:pos.architecture.word-confusing.constituent}).
It's of course possible that a phonological word 
has no morphosyntactic significance at all.

The next question is 
whether all prosodic words that are also morphosyntactic units 
are small enough, 
or some of them are actually phrases.
This question is hard to answer for 
disyllabic verb-object constructions,
but the tentative conclusion of this note 
is they can be regarded as grammatical words
(\prettyref{sec:pos.verb.idiomatic-verb-object}).
In other cases, prosodic words with morphosyntactic significance
are all small units and either have conventionalized meanings 
or are unable to be extended infinitely.
Thus, I recognize them as \term{grammatical words},
fixing the boundary between words and phrases in a way 
that is somehow subjective but consistent with 
the standard employed in many other world languages
(\prettyref{sec:pos.architecture.word-confusing.constituent}).

There are also non-prosodic grammatical words in Mandarin,
including monosyllabic words and polysyllabic words like 哥斯达黎加 
(\prettyref{sec:pos.morpheme.primitive})
and complex non-prosodic words like 
总工程师 (\prettyref{sec:pos.word.complex}).
Thus I call prosodic words with morphosyntactic significance \concept{morphosyntactic prosodic words}.

\subsubsection{Monosyllabic words: grammatical words, but not prosodic words}\label{sec:pos.word.monosyllabic}

As is said in \prettyref{sec:pos.morpheme.primitive}, 
some -- although by no means all -- monosyllabic morphemes 
are free morphemes and

\subsubsection{Disyllabic mini-constituent words}\label{sec:pos.architecture.word.mini-constituent}

There are disyllabic units in Chinese 
that have conventionalized meanings and its inner structure is invisible 
to any other morphosyntactic rules 
(\prettyref{sec:pos.morpheme.fossilization},
\prettyref{sec:pos.noun.fossilized-structure},
\prettyref{sec:pos.verb.fossilized-structure}).
Other disyllabic word are made up by two morphemes with a synchronically available device.

A controversy is whether some disyllabic prosodic words that have morphosyntactic significance
are actually phrases,
in which the two syllables may be analyzed 
as two grammatical words
instead of two morphemes or two meaningless syllables.
This is usually not the case for nouns, 
but is indeed true for verbs
(\prettyref{sec:pos.verb.idiomatic-verb-object}).
Examples of them include 念佛, 军训, 体操 etc. 

Then it's possible that the morphosyntactic prosodic word can be extended into a phrase 
by inserting more phrasal dependents 
or by adjoining modifiers to one of the two morphemes.
But note that similar processes are possible even for 
disyllabic verbs without analyzable inner structures,
which are surely grammatical words (\prettyref{sec:verb-splitting}).

\subsubsection{Psosodic words with semi-inflection}\label{sec:pos.architecture.word-confusing.constituent}

The standard for a constituent varies,
and so does the definition of a mini-constituent.
The English form \corpus{explore-ed} 
still needs to take an object, 
and some syntactic and semantic analysis 
actually implies that the tense/aspect categories are somehow higher than the object.
Thus it can be recognized as a unit -- a constituent -- because 
its inner components have strong interlink in dependency relations
(\prettyref{sec:theory}).
In principle, we can also analyze it as 
\corpus{[-ed [explore \emph{object}]_{\text{argument structure}}]_{\text{\acs{tame} marking}}};
in practice we refrain from doing so
(except in fully theoretical works like Distributed Morphology), 
because \corpus{explored} is a \emph{phonological} word 
(\prettyref{sec:pos.word.phonological}),
and this urges us to recognize it also as a morphosyntactic unit
(essentially, by \emph{redefining} what is a unit).
The longer sequence \corpus{have explored} 
has a parallel structure, 
but is not a phonological unit, 
and therefore some grammars say it's a verb phrase,
while others say it's not a phrase. (TODO: ref)
Both analyses, of course, are well-justified -- 
which one is to be used 
depends on the focus of the grammar,
and by alternating the terminology 
the two analyses are equivalent
(\prettyref{sec:theory}).
The definition of the term \term{word} 
therefore in principle carries no assertion regarding 
the nature of the language in question; 
we do need to choose the definition carefully enough, though, 
to avoid possible confusion.

In Mandarin, although prototypical inflection is absent, 
in verbal complement constructions 
and the aspect system, 
we indeed can see something similar to \corpus{explored} mentioned above, 
the inner parts of which have strong dependency
but do not form a constituent in the most strict sense.
What is the status of 爬上 in 他笨手笨脚地爬上信号塔?
A word (created by a productive verb compounding rule), 
a phrase (a verb-complement structure),
or just a word sequence without structural significance?
Here I follow the opinion in \citep[\citepage{86}]{feng2000} and \citet{tham2015resultative}
and call it a word,
because a sequence like 爬上 is extended in a highly limited way 
(the only possibilities being 爬得上 and 爬不上),
while phrases, in principle, can be extended infinitely, 
and it is also a prosodic word.
This goes against the analysis in \citep[\citesec{1.2.7}]{zhudexigrammar}.


\subsubsection{Non-prosodic complex words}\label{sec:pos.word.complex}

Certain grammatical relations seem to be not a part of \ac{np}s and clauses,
again highlighting the necessity to introduce a smaller level of constituency
(\prettyref{sec:pos.noun.compound}, TODO: verb),
commonly known as grammatical words.
The structures listed above in principle 
can be extended without an upper bound, 
and therefore they are not monosyllabic and are not prosodic words.

Just like the case of morphosyntactic prosodic words,
these \concept{non-prosodic complex grammatical words} 
may be created by synchronic morphosyntactic rules,
or they may be fossilized or have no internal structure.

Compared to prosodic words,
non-prosodic complex words are less ``active'' in syntax:
splitting them is possible in certain cases
but is much less frequent.
This may be a result of pragmatics:
complex words are created to cover a meaning that needs some explanation,
and once a complex word is well-accepted,
its form and meaning soon gets fixed 
(because people will not burden themselves),
and fossilization occurs rapidly.
The term 美利坚合众国 has an analyzable internal structure,
but it has already gained a fixed meaning 
and its parts are never taken out,
despite both 美利坚 and 合众国 can serve as grammatical words.


\subsection{Phrases}

\subsubsection{Noun phrase}

Nouns are able to be lexical heads of \acs{np}s,
including non-prosodic simple words like 哥斯达黎加,
as well as non-prosodic complex words.
One-word phrases are always possible,
and there is almost no attested counterexample.

\subsubsection{The verb phrase}

Transitive verbs can regularly fill argument slots 
and thus are able to be used as one-word phrases,
though they themselves are not sufficient to build one-word predicate \ac{vp}s. TODO: really???
Certain grammatical words,
like some verbal complement structures 

\begin{exe}
    \ex\label{ex:transitive-verb-phrase-disyllabic} 
    \begin{xlist}
        \ex {} [看书]_{\text{subject:verb}} 是一件有趣的事情
        \ex *[走进]_{\text{subject:verb}} 意味着您已经同意了我们的服务条款 
        \ex {} [走进这个建筑]_{\text{subject:VP}} 意味着您已经同意了我们的服务条款
    \end{xlist}
\end{exe}

\subsection{Types of clauses}


\subsection{The structure of Mandarin grammar}

After establishing the status of \term{words} in Mandarin grammar, 
we can now discuss the overall architecture of Mandarin grammar 
in a more disciplined way.
\prettyref{fig:morpheme-to-phrase} summarizes the organization of Chinese lexicon 
as well as how larger units are built from lexical items.
Overlapping of blobs means ``having the same form''.
Thus, the blob representing monosyllabic words is completely in the blob of monosyllabic morphemes.
The same is for the relation between non-prosodic simple words 
(which are neither monosyllabic nor disyllabic) and polysyllabic morphemes.
Red arrows mean synchronic morphosyntactic devices,
while orange arrows means historical evolution,
like grammaticalization and/or fossilization,.

\begin{figure}[H]
    \centering
    

\tikzset{every picture/.style={line width=0.3pt}} %set default line width to 0.75pt        

\begin{tikzpicture}[x=0.75pt,y=0.75pt,yscale=-0.8,xscale=0.8]
%uncomment if require: \path (0,552); %set diagram left start at 0, and has height of 552

%Shape: Ellipse [id:dp25211786396044866] 
\draw  [color={rgb, 255:red, 155; green, 155; blue, 155 }  ,draw opacity=1 ][fill={rgb, 255:red, 155; green, 155; blue, 155 }  ,fill opacity=0.2 ] (105.33,392.3) .. controls (105.33,341.28) and (154.13,299.93) .. (214.33,299.93) .. controls (274.53,299.93) and (323.33,341.28) .. (323.33,392.3) .. controls (323.33,443.31) and (274.53,484.67) .. (214.33,484.67) .. controls (154.13,484.67) and (105.33,443.31) .. (105.33,392.3) -- cycle ;
%Shape: Ellipse [id:dp011093449258029242] 
\draw  [color={rgb, 255:red, 155; green, 155; blue, 155 }  ,draw opacity=1 ][fill={rgb, 255:red, 155; green, 155; blue, 155 }  ,fill opacity=0.2 ] (164.06,334.33) .. controls (167.18,318.1) and (187.28,308.32) .. (208.94,312.5) .. controls (230.6,316.67) and (245.62,333.22) .. (242.49,349.45) .. controls (239.37,365.69) and (219.27,375.47) .. (197.61,371.29) .. controls (175.95,367.12) and (160.93,350.57) .. (164.06,334.33) -- cycle ;

%Shape: Ellipse [id:dp8023514623566042] 
\draw  [color={rgb, 255:red, 155; green, 155; blue, 155 }  ,draw opacity=1 ][fill={rgb, 255:red, 155; green, 155; blue, 155 }  ,fill opacity=0.2 ] (336,398) .. controls (336,362.51) and (373.68,333.74) .. (420.17,333.74) .. controls (466.65,333.74) and (504.33,362.51) .. (504.33,398) .. controls (504.33,433.49) and (466.65,462.26) .. (420.17,462.26) .. controls (373.68,462.26) and (336,433.49) .. (336,398) -- cycle ;
%Shape: Ellipse [id:dp321544508802865] 
\draw  [color={rgb, 255:red, 155; green, 155; blue, 155 }  ,draw opacity=1 ][fill={rgb, 255:red, 155; green, 155; blue, 155 }  ,fill opacity=0.2 ] (404.81,386.06) .. controls (402.26,369.72) and (417.86,353.72) .. (439.65,350.31) .. controls (461.45,346.91) and (481.19,357.39) .. (483.74,373.73) .. controls (486.29,390.06) and (470.69,406.07) .. (448.9,409.47) .. controls (427.1,412.88) and (407.36,402.4) .. (404.81,386.06) -- cycle ;

%Shape: Ellipse [id:dp9397609165336205] 
\draw  [color={rgb, 255:red, 155; green, 155; blue, 155 }  ,draw opacity=1 ][fill={rgb, 255:red, 155; green, 155; blue, 155 }  ,fill opacity=0.2 ] (382.01,362.67) .. controls (324.14,364.88) and (275.22,314.07) .. (272.74,249.18) .. controls (270.25,184.3) and (315.16,129.9) .. (373.03,127.69) .. controls (430.9,125.47) and (479.82,176.28) .. (482.3,241.17) .. controls (484.78,306.06) and (439.88,360.45) .. (382.01,362.67) -- cycle ;

%Curve Lines [id:da3228764928640613] 
\draw [color={rgb, 255:red, 208; green, 2; blue, 27 }  ,draw opacity=1 ]   (230.86,300.6) .. controls (223.77,257.91) and (236.35,243.61) .. (271.13,248.93) ;
\draw [shift={(272.74,249.18)}, rotate = 189.46] [fill={rgb, 255:red, 208; green, 2; blue, 27 }  ,fill opacity=1 ][line width=0.08]  [draw opacity=0] (12,-3) -- (0,0) -- (12,3) -- cycle    ;
%Curve Lines [id:da796668618520048] 
\draw [color={rgb, 255:red, 245; green, 166; blue, 35 }  ,draw opacity=1 ] [dash pattern={on 4.5pt off 4.5pt}]  (646.67,256.26) .. controls (666.57,354.69) and (582.51,395.85) .. (503.85,408.08) ;
\draw [shift={(502.67,408.26)}, rotate = 351.36] [fill={rgb, 255:red, 245; green, 166; blue, 35 }  ,fill opacity=1 ][line width=0.08]  [draw opacity=0] (12,-3) -- (0,0) -- (12,3) -- cycle    ;
%Curve Lines [id:da025052452235164946] 
\draw [color={rgb, 255:red, 208; green, 2; blue, 27 }  ,draw opacity=1 ]   (551.67,168.26) .. controls (514.24,147.58) and (486.19,158.91) .. (471.97,189.83) ;
\draw [shift={(471.33,191.26)}, rotate = 293.63] [fill={rgb, 255:red, 208; green, 2; blue, 27 }  ,fill opacity=1 ][line width=0.08]  [draw opacity=0] (12,-3) -- (0,0) -- (12,3) -- cycle    ;
%Shape: Ellipse [id:dp6184914092498568] 
\draw  [color={rgb, 255:red, 155; green, 155; blue, 155 }  ,draw opacity=1 ][fill={rgb, 255:red, 155; green, 155; blue, 155 }  ,fill opacity=0.2 ] (536,203.67) .. controls (536,172.74) and (571.89,147.67) .. (616.17,147.67) .. controls (660.44,147.67) and (696.33,172.74) .. (696.33,203.67) .. controls (696.33,234.59) and (660.44,259.67) .. (616.17,259.67) .. controls (571.89,259.67) and (536,234.59) .. (536,203.67) -- cycle ;

%Curve Lines [id:da27466382204201545] 
\draw [color={rgb, 255:red, 208; green, 2; blue, 27 }  ,draw opacity=1 ]   (514,363.26) .. controls (519.31,299.58) and (676.42,360.64) .. (637.6,261.76) ;
\draw [shift={(637,260.26)}, rotate = 67.91] [fill={rgb, 255:red, 208; green, 2; blue, 27 }  ,fill opacity=1 ][line width=0.08]  [draw opacity=0] (12,-3) -- (0,0) -- (12,3) -- cycle    ;
%Curve Lines [id:da11077382821910953] 
\draw [color={rgb, 255:red, 245; green, 166; blue, 35 }  ,draw opacity=1 ] [dash pattern={on 4.5pt off 4.5pt}]  (473.67,293.26) .. controls (484.56,310.09) and (489.57,323.98) .. (432.42,337.84) ;
\draw [shift={(430.67,338.26)}, rotate = 346.65] [fill={rgb, 255:red, 245; green, 166; blue, 35 }  ,fill opacity=1 ][line width=0.08]  [draw opacity=0] (12,-3) -- (0,0) -- (12,3) -- cycle    ;
%Curve Lines [id:da3266231201686014] 
\draw [color={rgb, 255:red, 155; green, 155; blue, 155 }  ,draw opacity=1 ][fill={rgb, 255:red, 155; green, 155; blue, 155 }  ,fill opacity=0.2 ]   (89,90.26) .. controls (85.67,243.26) and (163.67,345.26) .. (210.67,340.26) .. controls (257.67,335.26) and (204.67,217.26) .. (283.67,140.26) .. controls (362.67,63.26) and (423,124.59) .. (414,207.59) .. controls (405,290.59) and (320.67,294.26) .. (350.67,376.26) .. controls (380.67,458.26) and (558.67,427.26) .. (634.67,364.26) .. controls (710.67,301.26) and (728.67,127.26) .. (728.67,90.26) ;
%Shape: Polygon Curved [id:ds7705333747253118] 
\draw  [color={rgb, 255:red, 155; green, 155; blue, 155 }  ,draw opacity=1 ][fill={rgb, 255:red, 155; green, 155; blue, 155 }  ,fill opacity=0.2 ] (300,283.93) .. controls (412,321.93) and (522,319.93) .. (522,395.93) .. controls (522,471.93) and (377,501.93) .. (274,517.93) .. controls (171,533.93) and (84,477.93) .. (79,395.93) .. controls (74,313.93) and (188,245.93) .. (300,283.93) -- cycle ;

% Text Node
\draw (126,154) node [anchor=north west][inner sep=0.75pt]  [color={rgb, 255:red, 0; green, 0; blue, 0 }  ,opacity=1 ] [align=left] {phrases};
% Text Node
\draw (377.52,245.18) node   [align=left] {\begin{minipage}[lt]{76.4pt}\setlength\topsep{0pt}
\begin{center}
morphosyntactic\\prosodic\\words
\end{center}

\end{minipage}};
% Text Node
\draw (245.5,420.7) node   [align=left] {\begin{minipage}[lt]{60.24pt}\setlength\topsep{0pt}
\begin{center}
monosyllabic\\morphemes
\end{center}

\end{minipage}};
% Text Node
\draw (203.28,341.89) node  [font=\scriptsize] [align=left] {\begin{minipage}[lt]{48.74pt}\setlength\topsep{0pt}
\begin{center}
monosyllabic\\words
\end{center}

\end{minipage}};
% Text Node
\draw (618.17,202.67) node   [align=left] {\begin{minipage}[lt]{59.71pt}\setlength\topsep{0pt}
\begin{center}
non-prosodic\\complex\\words
\end{center}

\end{minipage}};
% Text Node
\draw (427.5,430.7) node  [font=\small] [align=left] {\begin{minipage}[lt]{49.89pt}\setlength\topsep{0pt}
\begin{center}
polysyllabic\\morphemes
\end{center}

\end{minipage}};
% Text Node
\draw (444.28,379.89) node  [font=\tiny] [align=left] {\begin{minipage}[lt]{48.8pt}\setlength\topsep{0pt}
\begin{center}
non-prosodic\\simple words
\end{center}

\end{minipage}};
% Text Node
\draw (304,460.93) node [anchor=north west][inner sep=0.75pt]   [align=left] {morphemes};


\end{tikzpicture}

    \caption{From morphemes to phrases.}
    \label{fig:morpheme-to-phrase}
\end{figure}

The sub-phrasal units in \prettyref{fig:morpheme-to-phrase}
are represented in \prettyref{tbl:sub-phrasal-units}.
The +? symbols in the ``morpheme'' column 
mean possible fossilization, 
corresponding to the lower two orange arrows in \prettyref{fig:morpheme-to-phrase}.
The upper orange arrow in \prettyref{fig:morpheme-to-phrase}
means word creation by abbreviation. 
The +? symbol appearing in the ``phrase'' column 
means some authors may agree that 走进 is a constituent 
and therefore they may say it's a phrase
(\prettyref{sec:pos.architecture.word-confusing.constituent}).

\begin{sidewaystable}
    \centering
    \caption{Sub-phrasal units}
    \label{tbl:sub-phrasal-units}
    \begin{tabular}{@{}lllllll@{}}
    \toprule
    index & length                                                                                                       & morpheme & word & phrase & example           & comment                                                                                                                                    \\ \midrule
    1     & \multirow{3}{*}{monosyllable}                                                                                & +        & -    & -      & 竞, 语              & usually from Classical Chinese                                                                                                             \\ \cmidrule{3-7}
    2     &                                                                                                              & +        & -    & -      & 性 in 等价性, 前 in 窗前 & \begin{tabular}[c]{@{}l@{}}suffixes, location words, \\ and other semi-grammatical items\end{tabular}                                      \\ \cmidrule{3-7}
    3     &                                                                                                              & +        & +    & +      & 蛇, 猫              & usually from Classical Chinese                                                                                                             \\ \midrule
    4     & \multirow{4}{*}{disyllable}                                                                                  & +        & -    & -      & 达达 in 达达主义        & usually borrowed terms, rare                                                                                                               \\ \cmidrule{3-7}
    5     &                                                                                                              & +        & +    & -      & 觊觎, 葡萄, 空调        & \begin{tabular}[c]{@{}l@{}}from Classical Chinese, borrowed items, \\ abbreviations, and deep fossilization\\ of type 6 and 7\end{tabular} \\ \cmidrule{3-7}
    6     &                                                                                                              & -/(+?)   & +    & -/(+?) & 走进, 来到            & \begin{tabular}[c]{@{}l@{}}constituents in \ac{blt} only; \\ made of manosyllables\end{tabular}                                            \\ \cmidrule{3-7}
    7     &                                                                                                              & -/(+?)   & +    & +      & 念佛, 看书, 吃饭        & \begin{tabular}[c]{@{}l@{}}mini-constituents; \\ made of monosyllables\end{tabular}                                                        \\ \midrule
    8     & \multirow{3}{*}{\begin{tabular}[c]{@{}l@{}}trisyllable \\ or larger \end{tabular}} & +        & -    & -      & 日耳曼               & usually borrowed terms, rare                                                                                                               \\ \cmidrule{3-7}
    9     &                                                                                                              & +        & +    & +      & 哥斯达黎加, 滑铁卢,数理化    & usually borrowed terms or abbreviations                                                                                                    \\ \cmidrule{3-7}
    10    &                                                                                                              & -/(+?)   & +    & +      & 副总经理, 总干事         & complex word, made of monosyllables and disyllables                                                                                        \\ \bottomrule
\end{tabular}
\end{sidewaystable}

\subsection{Perception of basic units in Mandarin}\label{sec:pos.word.perception}

In English there is a group of clearly defined and largely homogenous morphosyntactic units 
lying between morphemes and phrases
(which means they can be neither morphemes nor phrases,
and can be constructed by the former and can be used to build the latter),
which is just the grammatical word.
They are recognized as basic units by society 
(i.e. outside of the linguistic community):
the length of an article is measured by counting words, not letters, for example.

In Chinese, however, despite the fact that 
we can still identify grammatical words,
the much larger overlap between grammatical words and morphemes and phrases
means Chinese characters -- representation of syllables in polysyllabic morphemes 
and rough representation of monosyllabic morphemes -- 
are recognized as basic units of the language.
This may be the deriving force for some linguists 
(who are too eager to ``find diversity'')
to reject the existence of grammatical words in a hurry,
although I've already shown this is not the case.

Still, native speakers generally 
confirm that there is something between the morpheme and the phrase, 
which people call 词.
In everyday speech, 
non-prosodic complex words have limited use, 
and monosyllabic words are rarely used 
because their formality (\prettyref{sec:pos.morpheme.primitive}).  
Thus, from \prettyref{fig:morpheme-to-phrase}, 
we find it's the class of prosodic morphosyntactic words 
that plays the role of ``words''
as a level between morphemes and phrases,
as well as intermediate units between morphemes and complex words.
Prosodic morphosyntactic words 
are also usually the final destine 
of abbreviation (\prettyref{sec:pos.morpheme.abbreviation}).
These morphosyntactic prosodic words are therefore the nature response 
when native speakers without much linguistic training 
come up with intuitively when talking about ``words'' or 词,
i.e. intermediate building blocks.

The fuss around wordhood in Chinese 
arises from the split of several definitions of the term \term{word}:
wordhood defined by morphosyntactic test include is wider 
than the coverage of morphosyntactic prosodic words,
and the range of morphosyntactic words indeed 
has too much overlapping with morphemes and phrases.

\section{Overview of classification criteria}

\subsection{Word class labels: noun, verb, adjectives, etc.}

\subsubsection{The nominal-verbal division}

Lexical words in Chinese can be roughly divided into nominal ones and verbal ones,
or in the Chinese terms, 体词 and 谓词.
The prototypical role of nominal words 
is to fill predicate slots (or to be more precise, to head a phrase that fills an argument slot).
Nominal words rarely appear in the verbal complex,
though for stylistic purposes, they sometimes do.
Verbal words prototypically appear in the verbal complex
(\prettyref{chap:verbal-complex}),
but many of them -- and clauses without any morphological marking -- 
can regularly appear in argument slots \citep[\citesec{3.5}]{zhudexigrammar}.

The fact that verbal categories can fill argument slots or in colloquial words ``be used as nouns''
urges some to put the verbal categories under the nominal categories,
so thus there is only one mega lexical category in Chinese:
the nominal category or the Noun.
The analysis adopted here does not aim to organize lexical categories 
in a binary branching classification tree,
so the ordinary nominal-verbal distinction is maintained:
verbs being able to fill argument slots is not typologically rare, actually,
and this shared feature itself does not bring nouns and verbs close enough 
for them to be merged together.

\subsubsection{Two adjectival classes}

Whether Chinese has a separate adjective category 
has been debated for decades.
Based on a line of reasoning similar to the above verb-as-noun analysis,
some linguists argue that the so-called adjectives should be put under the verb category,
since they can fill the predicator slot without any morphological marking \citep{li1989mandarin}.
Since verbs and most alleged adjectives show different morphological behaviors in reduplication, % TODO: ref
the verb-adjective distinction is kept,
and the two are placed under the verbal category.

There still exist a (much smaller) number of alleged adjectives that shows 
different morphosyntactic properties with the adjectives in the verbal category 
\citep[\citechap{5}]{paul2014new}.
They can be marginally used as heads of \ac{np}s,
while they do not have reduplication variants.
These ``adjectives'' are thus placed under the nominal category.
Thus we have two types of adjectives.
In \citet{zhudexigrammar}, 
nominal adjectives are called 区别词 \translate{distinction word},
while verbal adjectives are called 形容词 \translate{adjective}.

\subsubsection{Other nominal categories}

There are more nominal categories than the ordinary noun category and the nominal adjective category.
Numerals, for examples, are in another nominal category.
Chinese has a rich classifier system,
and most classifiers still have strong nominal properties
and thus they constitute yet another nominal category.
\citet{zhudexigrammar} calls them 量词 \translate{measure word},
because many classifiers have the meaning of ``unit''.
There is also a locative particle class, including 里 in 在房子里,
which is sometimes said to be the postposition class
because they sometimes have adposition-like properties (TODO: ref: topicalization, and what else?).

\subsection{Open and close classes; the lexical-functional distinction}

\subsubsection{The hierarchy of openness}

The distinction between lexical and functional classes is sometimes subtle.
\citep[\citesec{3.6}]{zhudexigrammar} classifies 
certain categories like locative particles % TODO: 方位词的正确翻译???
into the nominal class and hence the lexical one,
while the locative particle class can definitely be enumerated \citep[\citesec{4.4}]{zhudexigrammar}.
On the other hand, 
the author claims that lexical classes are always open 
and function classes are always closed \citet[\citesec{3.4}]{zhudexigrammar}.
A conflict thus occurs.

The problem here is we have a gradient hierarchy 
from the prototypical lexical classes 
to the prototypical function classes.
The most lexical class is open to new members, 
not a part of the grammar,
and its members are able to be lexical heads
(and thus has a ``real part-of-speech label'' 
like ``noun'' or ``verb'') 
of, say, an \acs{np} or a verbal complex.
A less open class is not so open to new members 
(just like Japanese verbs and adjectives),
but is still not a part of the grammar 
and its members are able to be lexical heads. 
A even more closed class is not open to new members,
and is a part of the grammar,
but its members are still able to be lexical heads.
Pronouns are in this type.
A prototypical function class, then, 
is not open to new members, hardwired in the grammar, 
and its members are never lexical heads.
Derivational suffixes are in this type.
This last type of forms 
bring no real part-of-speech label to its realization.
We still classify these function items in the grammar into classes,
but these classes are somehow less ``real''.

It's of course not easy to tell a newly discovered part of speech 
(or \term{form class}, which may be a word class or an affix class)
What's the status of an orientation preverb, 
which may be found in Japhug \citep{jacques2021grammar}?
It's a part of the grammar,
but does it carry a real part-of-speech label (like ``directional adverb'')?
And speaking of adverbs, what's the status of the English \corpus{allegedly}?
An adverb filling a peripheral argument position,
or an evidentiality marker?
We really need to know a lot about language to fix the position of a form class.
A common practice is just to shun the details and just say whether a class is lexical or functional,
drawing a hard line between the two.
So \citet{zhudexigrammar} mainly uses the criterion of whether there is a real part-of-speech label,
and then directive particles are classified into the nominal class 
and they are in turn considered lexical.
But he mistakenly confuses the notion of lexical classes with the notion of open classes,
and then we get the self-conflicting asserts in \citet[\citesec{3.4}]{zhudexigrammar}.

\subsubsection{Openness of Mandarin form classes}

\subsection{Summary: a tentative part of speech analysis}

\section{Nouns}

\subsection{Monosyllabic nouns}

\subsection{Nouns with historically analyzable inner structure}\label{sec:pos.noun.fossilized-structure}

The unit 白菜 is made up by two perfectly productive morphemes:
白 \translate{white} and 菜 \translate{vegetable},
but its meaning is not the composition of the two morphemes:
白菜 means \translate{Chinese cabbage}, not \translate{any vegetable with whitish appearance}.
The word has already gained a conventionalized meaning,
and its inner structure is of mostly diachronic interest but not synchronic interest.
Therefore, the disyllabic unit 白菜 is the smallest unit fed into morphosyntax,
and it of course is not a phrase.

Those insisting on the nonexistence of words in Chinese 
may explain the observation made above 
by claiming 白菜 to be an idiom \ac{np}:
it is indeed a lexical entry,
but is regarded as a pre-compiled phrase. TODO: 


\subsection{Nominal bound morphemes}

\subsection{Compound nouns}\label{sec:pos.noun.compound}

From \eqref{ex:nominal-modifier-1} and \eqref{ex:nominal-modifier-2},
it can be seen in certain morphosyntax units,
a bare noun may serve as a (restrictive) modifier.
The constituent of Chinese \ac{np}s is Dem Num A N,
and this bare noun modifier position seems to be more internal than the adjective position,
as is illustrated by \eqref{ex:meiguo-hongse-pinguo} and \eqref{ex:hongse-meiguo-pinguo}.

\begin{exe}
    \ex 
    \begin{xlist}
        \ex\label{ex:nominal-modifier-1} {} [[定义]_{\text{modifier:N}} [[等价]_{\text{complement:adjective}} [性]_{\text{nominalizer}}]_{\text{N}}]_{\text{N}} \translate{equivalence of definitions}
        \ex\label{ex:nominal-modifier-2} {} [[美国]_{\text{modifier:N}} [苹果]_{\text{head:N}}]_{\text{N}}
        \ex\label{ex:meiguo-hongse-pinguo} *[美国]_{\text{modifier:N}} [红色的苹果]_{\text{NP}}
        \ex\label{ex:hongse-meiguo-pinguo} 红色的美国苹果
    \end{xlist}
\end{exe}

Furthermore, the bare noun position cannot be filled by an \ac{np}.
The following examples demonstrate this:
\begin{exe}
    \ex \begin{xlist}
        \ex {} [联合国] [秘书长]
        \ex *[[某个组织] [秘书长]]
        \ex 某个组织的秘书长
    \end{xlist}
\end{exe}
The obligatoriness of 的 means the NP 某个组织 can only appear as a modifier via the possessive construction.
It cannot fill the slot of 美国 in 美国苹果.
So the bare noun modifier position is a function label existing in a unit smaller than the \ac{np}
-- and it has to be the word.

The modifier position can also be filled by 
a disyllabic 

\begin{exe}
    \ex 染发行业
    \ex *染头发行业
\end{exe}

\subsection{Adjectival modification in complex noun}\label{sec:pos.noun.adj-modify}

\section{Verbs}

\subsection{Verbs with fossilized internal structures}\label{sec:pos.verb.fossilized-structure}

The verb 关心 \translate{care for} is certainly analyzable 
as a predicator-object structure,
but it takes objects just like any other verbs:
\begin{exe}
    \ex 他 [[[关]_{\text{predicator:V}} [心]_{\text{object:N}}]_{\text{predicator:V}} [自己的家人]_{\text{object:NP}}]_{\text{predicate:VP}}
\end{exe}
This means 关心 is not a VP but a grammatical word,
or otherwise it is impossible to take another object since there is no valency changing device in use.

\subsection{Idiomatic disyllabic verb-object structures}\label{sec:pos.verb.idiomatic-verb-object}

Unlike the case of \prettyref{sec:pos.verb.fossilized-structure}, 
there exist some disyllabic prosodic words 
which are verbal constituents 
and are often recognized as words. 
Examples of them include 看书, 做饭, etc.
These disyllabic verb-object structures 
behave just like ordinary \acs{vp}s 
when the object is modified by, say, adjectival or nominal attributives;
when a semi-object is inserted between the verb and the object,
they also follow the same pattern observed in 
ordinary \acs{vp}s (\prettyref{sec:verb-phrase.ionization.similar}).
However, disyllabic verb-object structures 
do seem to have one thing different with longer \acs{vp}s:
they can appear in compound nouns,
while the latter can't (\prettyref{sec:pos.noun.compound}).
Thus, the morphosyntactic tests outlined in this note decide 
that they are both words and phrases.

(\prettyref{ex:pos.verb.verb-object.nianfo-1}) gives an example of this duality.
The verb 念佛 in the first example is similar to 美国 in \eqref{ex:nominal-modifier-2}:
it serves as a bare modifier 
(since in Chinese verbal constituents can fill argument slots directly,
the fact that 念佛 is a verb is not surprising).
The fact 念佛 is able to appear in such a position assures us that 
it is a word.
Then consider \eqref{ex:nianfo-split}.
In VP1, a temporal semi-object is injected between the verb 念 and the object 佛,
while in VP2, an interrogative phrase 哪一尊 is inserted into 佛 
and an \ac{np} object is now taken by the verb 念.

\begin{exe}
    \ex\label{ex:pos.verb.verb-object.nianfo-1}
    \begin{xlist}
        \ex\label{ex:nianfo-tang} {} [念佛] 堂 
        \ex\label{ex:nianfo-split} 老太太 [念了这么久佛]_{\text{VP1}},却不知道自己在[念哪一尊佛]_{\text{VP2}}
    \end{xlist}
\end{exe}

The solution used in this note 
is to regard 念佛 in \eqref{ex:nianfo-tang} as a word,
while the two \acs{vp}s in \eqref{ex:nianfo-split} as phrases.
念佛 as in 老太太经常念佛 can be interpreted as a word or as a phrase
without making any difference.
念佛 as a word is something like \corpus{Buddha-praying},
while 念佛 as a verb is something like \corpus{pray to Buddha}.
This agrees with the account of \citet[\citepage{82}]{feng2000},
in which 念佛 is a morphosyntactic word
(the original term being a 句法词 \translate{syntactic word}),
which is created by morphosyntactic rules 
and has a inner structure that is (partially) transparent 
for other morphosyntactic rules,
while 关心 is a \translate{lexical word} 
(not the same with \term{lexical word} in the rest of this note 
which means words that are not a part of the grammar),
which is taken out of the lexicon directly
and has no synchronically analyzable inner structure.
This also agrees with the claim in \prettyref{sec:pos.word.phonological}
that all prosodic words with morphosyntactic significance 
are indeed grammatical words 
and not phrases.

\begin{infobox}{Notes on one previous analysis}{zhu-verb-object-idiom-analysis}
    The solution here is a generalization of 
    the solution taken by \citet[\citesec{1.2.6}]{zhudexigrammar}.
    \citet{zhudexigrammar} recognizes three classes of grammatical structures, 
    组合式 \translate{composition-style}, 粘合式 \translate{gluing-style}, 
    and 复合词式 \translate{compound word-style}.
    (TODO: ref) 
    A gluing-up-style structure, in his definition, 
    seems to be a regularly formed structure 
    containing elements smaller than usual phrases 
    -- bare grammatical words? --
    and yet is a phrase itself,
    while a compound structure is within a grammatical word.
    Thus, in his analysis, 
    吃饭 and 念佛 are gluing-up structures 
    and are phrases. 
    This analysis is based on the idea that 
    if a structure contains a bound morpheme as one of its immediate constituent, 
    it has to be a grammatical word, 
    while if a structure contains free morphemes 
    and doesn't have a highly established meaning, 
    then it has to be a phrase; 
    the first idea has been refuted in \prettyref{sec:pos.morpheme.primitive}, 
    while the second idea confuses 
    the parameter of syntactic size 
    and the parameter of established meaning, 
    which are almost orthogonal to each other. 
    So his analysis is not valid, 
    and his gluing-up structure should be merged with 
    the compound word structure
    (but we need to note that in the context in this box, 
    the term \term{compound word}
    refers to words with synchronically analyzable internal structures, 
    \emph{not} words in \prettyref{sec:pos.verb.fossilized-structure}).
    Indeed, he recognizes that the syntactic behaviors of the two are highly close to each other 
    (\citepages{128-129}).
\end{infobox}

\subsection{Idiomatic clauses as a predicate}

Some clauses (usually idioms), like 大鱼吃小鱼, or 你看看我我看看你,
despite having perfectly analyzable internal structures, 
are used collectively as a \emph{verb} (\prettyref{sec:clause.dangling-topic.1}).

\section{Locational words}\label{sec:pos.locational}

Monosyllabic location words like 前 may be analyzed as words 
because they can be attached to an arbitrary \ac{np} 
to denote a place near the place denoted by that \ac{np},
as in [[那座老旧的房子 [前]_{\text{location word}}]_{\text{NP}} 有一口井]_{\text{clause}},
and what only appears as immediate constituents of phrases are of course words,%
\footnote{
    I'm talking about morphosyntactic words here.
    It is possible that something is a morphosyntactic word 
    appearing at the level of phrase
    is incorporated into a word nearby phonologically,
    though this is not the case in Chinese.
}%
but they never appear independently as \ac{np}s.


\section{Prepositions}\label{sec:preposition-pos}

Though all Mandarin prepositions have verb origins 
and therefore may be classified as a subclass of verbs by some,
it's necessary to distinguish a separate preposition class.
Criteria of prepositions include TODO: ref

\begin{infobox}{The term \term{coverb}}{coverb}
    In 
\end{infobox}

\section{Other grammatical ``coverbs''}

TODO: 把, 被, etc.; put the surface constituent orders here 
and link them to the chapter about the verb phrase

\chapter{The structure of noun phrase}

No morphological case, number, and gender categories are attested in Mandarin.
There is a word class system or in other words classifier system, however.
In most cases when a numeral appears in an \ac{np},
a classifier follows immediately after the numeral.
Attributives -- both adjectives and relative clauses -- 
follow the classifier. % TODO: 红色的三个,这样的说法说得通吗?
The demonstrative, if any, appears before the numeral,
and even when there is no numeral,
there is frequently also a classifier.

The template of \ac{np}s, therefore, belongs to the 
Dem-Num-A-N type,
with the classifier residing between Num and A. 


\chapter{The verbal complex and the verb phrase}\label{chap:verbal-complex}

\section{Introduction}

Mandarin is generally regarded as a prototypical analytic language,
without traditionally acknowledged verb inflections,
but several highly productive (as opposed to arguable historical derivation) 
verbal affixation devices have been attested.
I use the term \term{verbal complex} to cover 
the main verb and the suffixes.
The structure of the verbal complex is highly intertwined 
with verb valence, 
and therefore describing the verbal complex on its own 
is not feasible.
Instead, this chapter describes the \acs{vp} 
-- the verbal complex plus internal clausal complements, 
with the subject being an external complement --
as a whole.

There are several notable systems in the verb phrase: 
the verbal complement system, 
the objects (with various numbers, positions and semantic roles), 
the lexical aspect (without explicit marking but with grammatical consequences), 
negation (if any), 
and the aspectual marker. 

Some types of verbal complements -- the resultative complements, the directional complements, 
and the potential complements (\prettyref{sec:verbal-complement}) --
and the aspectual system (\prettyref{sec:aspectual})
are materialized as suffixes in the verbal complex.
Inside the verb stem, 
we still have a derivation system, 
like 化 \translate{-ize}.
Therefore, there are roughly three subsystems 
in the suffixes in verbal complex.

\eqref{ex:hua-wan-le-1} is an example in which 
all the three systems appear.
In real world speeches, such combinations have relatively lower distributions,
possibly because of the prosodic constraint 
that verb shouldn't be too heavy unless it appears at the end of a clause
(sec:vp.prosody).

\begin{exe}
    \ex \dots 并且企业 [数字 [化]_{\text{derivation}} [完]_{\text{complement}} [了]_{\text{aspectual}}]_{\text{V}} 之后还不一定赚钱 \dots
    \label{ex:hua-wan-le-1}
\end{exe}

It should be noted that 
verbal complements have limited mobility and therefore are not strictly affixes.
The linear order between the aspect marker and the verbal complement, if any, 
is not completely fixed \eqref{ex:movable-suffix}.
In the first sentence of \eqref{ex:movable-suffix}, 
了 is an aspectual suffix (\prettyref{sec:aspectual}),
while 走 is a verb which never appear without an argument in uncontroversial phrasal grammar.
So we conclude 了 and 走 are suffixes,
and by structural comparison, 
we conclude 过来 in \eqref{ex:sanpinqishuiugolai-1} 
is also a suffix, with the same status as 走.
But there comes \eqref{ex:sanpinqishuiugolai-2},
in which 过来 moves to the end of the sentence.

\begin{exe}
    \ex \begin{xlist}
        \ex \gll 他 带 走 了 他的 文件  \\ 
        3sg carry go.away \acs{perfect} 3sg-\acs{possessive} file \\
        \glt \translate{He carried his files away.}
        \ex \gll 他 带 [过来] 了 三 瓶 汽水 \\
        3sg carry come \acs{perfect} three bottle.\acs{classify} soda \\
        \glt \translate{He carried here three bottles of soda.} 
        \label{ex:sanpinqishuiugolai-1}
        \ex 他带了三瓶汽水[过来]
        \label{ex:sanpinqishuiugolai-2}
    \end{xlist}
    \label{ex:movable-suffix}
\end{exe}

Besides the phenomenon shown in \eqref{ex:movable-suffix},
it's also possible that
a part of the verb that not a morphosyntactic constituent at all 
is scattered to the end of the \acs{vp},
also the acceptability of this structure 
varies among people 
(\prettyref{sec:verb-splitting}).
The formation of the verbal complex therefore involves 
multiple levels of operations 
motivated by syntactic (i.e. more compositional) 
and morphological (in the sense of \prettyref{box:morphology-meaning}) operations.

The verbal complement system is another remarkable feature of Mandarin, 

\begin{infobox}{On the notion of \term{complements}}{complement-name}
    The Chinese term 补语 corresponding to my \term{verbal complement}
    is frequently translated into the English term \term{complement}.
    This creates some confusion,
    because the term \term{complement} can also denote 
    clausal dependents that are arguments of the main verb, as in \citet{cgel}.
    The term \term{non-argument complement} may be used to avoid this confusion.
    There are, however, further confusions:
    Should we regard a clausal dependent that records the quantity or amount of an action 
    as a non-argument complement?
    This construction can also be seen in Latin, 
    like the Latin accusative expression of time \citep[\citesec{423}]{greenough2013allen}.
    Thus, I use the term \term{verbal complement} to refer to 
    things like 完 as in 做完了.
\end{infobox}

The so-called serial verb constructions aren't mentioned here.
\citet{paul2008serial} and \citet[\citesec{9.4}]{deng2010formal} 
summarizes several constructions that are
frequently referred to as serial verb constructions,
and points out after deeper investigation,
they can all be described in terms of the usual complement clause constructions,
purpose clause constructions, etc. 
that are well attested cross-linguistically (\prettyref{sec:no-serial-verb}).


\section{Verbal complements and internal objects}\label{sec:verbal-complement}

Depending on their sizes, 
verbal complements can be divided into 
phrasal verbal complements 
and suffixal verbal complements. 
Suffixal verbal complements 
include \concept{resultative complements}, 
\concept{directional complements}, 
and \concept{potential complements}. 
Phrasal verbal complements include \concept{state complements} and 
\concept{prepositional complements}.
In each \acs{vp} there is at most one verbal complement: 
thus, the existence of a directional suffix
automatically excludes all phrasal verbal complements.

\begin{infobox}{Phrasal directional complement?}{phrasal-directional}
    The \acs{vp}-final disyllabic directional complement seen in 
    (\prettyref{ex:sanpinqishuiugolai-2})
    may also be seen as a phrasal verbal complement 
    \citep[\citepage{120}]{deng2010formal}, 
    but this is due to his strictly lexicalist analysis; 
    in my approach outlined in \prettyref{sec:theory}, 
    even though the directional complement in  
    (\prettyref{ex:sanpinqishuiugolai-2})
    doesn't appear together with the rest of the verbal complex, 
    the fact that it's small in size and  
    it's constrained in productivity
    means it should be put together with other directional complements. 
\end{infobox}

There are also a class of clausal dependents 
that are traditionally analyzed as objects,
but are extremely inactive in structure building 
after the argument structure is finished.
They seem to be in contrast distribution with verbal complements.
They therefore are to be classified together with verbal complements
\citep[\citepages{188-190}]{deng2010formal}.

\subsection{The state complement}



\begin{infobox}{Structural ambiguity after 得}{structural-ambiguity-de}
    There are two structures corresponding to the 得-\acs{np}-\acs{vp} sequence.
    In one case, like 这文章写得谁也看不懂, the \acs{np}-\acs{vp} sequence after 得 is a clause, 
    which, together with 得, constitutes 
    a state complement construction.
    In another case, like 这条山路走得我累死了, 
    the \acs{np} immediately after 得 
    is the \category{theme} argument associated with the \category{become} argument structure, 
    while the \acs{vp} is a clause with an empty subject 
    that constituents the state complement construction.
    This difference can be tested by trying to remove the current subject 
    and see whether we can still find a related grammatical sentence. 
    In the second case, we have 我走得累死了, 
    which has a structure parallel to 这文章写得谁也看不懂, 
    while in the first case it's impossible. 
\end{infobox}

\subsection{Verbal objects}

Some are complement clause constructions

Purpose clause

\begin{exe}
    \ex 你当我傻吗
    \ex 我准备明天去骑马
\end{exe}

\begin{exe}
    \ex 他跪下来求我
\end{exe}

TODO: 为动,死国可乎 etc.

What about 笑天下可笑之人

It should be noted that the position of the complement clause
is lower than the direct object.

\begin{exe}
    \ex 他把这个消息告诉了我
    \ex 他告诉了我这个消息
    \ex 他告诉我张三脑袋被驴踢了
    \ex *他把张三脑袋被驴踢了告诉了我
\end{exe}

\section{Subject and direct object in the argument structure}

\subsection{Overview}

Cross-linguistically, 
the argument structure concerning the subject and the object of \acs{vp}s 
(not verbs: a verb may project into several different \acs{vp}s) 
involving zero or one objects 
(not including so-called semi-objects in \prettyref{sec:verb-splitting})
may be prototypically divided into the following contrasting classes 
\citep[\citechap{6}]{deng2010formal}:
\begin{itemize}
    \item The \category{be} type, describing a static state, with one argument.
    \item The \category{do} type, describing a dynamic event, with one agentive argument 
        and one possible patientive argument.
        The agentive argument always goes to the subject position 
        (and therefore goes out of the \acs{vp}).
    \item The \category{become} type, describing a dynamic event, 
        with one argument being the participant of this event 
        and the ``state-transition'' caused by the event. 
        The argument may be agentive or patientive semantically.
        The argument is raised to the subject position.
    \item The \category{cause-become} type, a dynamic event, 
        with one argument being the causer, 
        and the other argument being the participant of this event.
        The causer is raised to the subject position.
\end{itemize}
An intransitive verb from the third category
is often called an \concept{unaccusative verb}, 
and in contrast, 
intransitive verbs belonging to the second type 
are \concept{unergative verbs}.
The terminology is unfortunately confusing
because this has nothing to do with alignment:
an ergative language can still have unergative verbs.
A verb allowing alternation of the third and fourth verb frames therefore 
is a S=O ambitransitive verb,
and a verb with an optional patientive argument 
is therefore a S=A ambitransitive verb.
It should be noted that although a \category{cause-become} \acs{vp}
is just a \category{cause} \acs{vp} plus a causer, 
this doesn't mean all verbs appearing in one of these structures 
are S=O ambitransitive verbs:
死, for example, only appears in the \category{become} structure 
and not the \category{cause}-\category{become} structure.

Certain gradience exists in the distinction between the two transitive classes
\citep{lin2021unaccusativity}. TODO: elaborate

There is a strong correlation between the argument structure 
and the lexical aspect of \acs{vp}s
\citep{laws2010core,toratani1997typology,aljovic2000unaccusativity}.
Specifically, a \category{become} or \category{cause}-\category{become} \acs{vp}
tends to be telic, 
because it describes a transition of states 
and therefore the event denoted is bounded by definition.

The above types can all be observed in Mandarin



我喜欢她

???我把她喜欢

??她被我喜欢

被不喜欢的人喜欢 -- but this is definitely a ``suffer'' or ``affectee'' construction

Thus, we conclude 喜欢 is a \category{do} verb, 
and this can be expected: 
to like someone (or in general, to have some thoughts about someone) 
doesn't change the inside state of the target.
Thus, the \category{become} argument structure is not available. 


There exist several variants of the \acs{vp} valencies listed above.
The following 

TODO: 台上坐着主席团

\begin{exe}
    \ex 他们表扬了我
    \ex 我被表扬了
    \ex ??他们把我表扬了
\end{exe}

One possibility might be that the \category{become} structure 
can take another light verb structure inside, 
and thus 王冕 is introduced by a \category{become} light verb.

我丢了手机、我把手机丢了 is 我 introduced by \category{cause}?

Double-object \acs{vp}s may be divided into 
two subclasses
\citep[\citesec{7.2}]{deng2010formal}:
\begin{enumerate}
    \item Describing a telic event which roughly means to give something, 
        with one agentive argument, 
        one receiver and one theme (TODO: elaborate).
    \item Describing a telic event with a meaning of obtaining something,
        with one agentive argument corresponding to the receiver, 
        one argument similar to the TODO: experiencer? What's the role? that corresponds to the source, 
        and one argument similar to the transitive object in the \category{do} type 
        that corresponds the the object being transferred.
\end{enumerate}

The patientive argument and the TODO: numeral object like 他抢了我[一块钱] 
are not active for further processing; 
thus the \acs{vp} can be further divided into two layers.
TODO: 被 construction: what happens to the so-called passivized argument?
Evidences suggest that the intransitive object is very internal, 
appearing in almost the same position with complement clause, etc.
but why is it subject to 被 construction? 
TODO: for short 被-construction, 
maybe it's because short 被-construction is generated by directly attaching 被 to, say, [抢了一块钱]: 
我被抢了一块钱 \translate{lit. I suffer from one-dollar robbing}
李四被捕了 doesn't have a counterpart without 被: 
捕 here may be regarded as a deponent verb. 
This also implies that 被 has already been grammaticalized 
and is no longer a lexical verb.

\begin{theorybox}{Light verb projections and verb frame}{light-verb}
    What we are seeing here is actually light verb structures.
    That a verb can only appear in a \category{cause-become} structure 
    can be explained by stipulating that 
    the verb root can only get appropriately spelt out
    together with the \category{cause} and \category{become} light verbs. 
\end{theorybox}

The telicity category described above interacts non-trivially with 
the aspectual system (\prettyref{sec:aspectual}).

\subsection{Adjective predicator}

One intriguing trait of Mandarin is a \category{be} structure 
without a degree adverbial is considered problematic
in a matrix clause,
and yet is perfectly fine in a subordinated clause.

\begin{exe}
    \ex ?他个子高
    \ex 他个子比较高
    \ex 他个子还算高
    \ex 他个子高高的
    \ex 他睡不下这张床,因为他个子高啊
\end{exe}

\subsection{The unmarked \acs{vp} with \category{cause}-\category{become} argument structure}

\subsubsection{Prototypical transitive and intransitive \category{become} structures}
\label{sec:verb-phrase.cause-become.ordinary}

The \category{cause}-\category{become} structure TODO

This structure -- the intransitive verb frame of a S=O ambitransitive verb -- 
is sometimes known as the \concept{notional passive}, 
in which no explicit passive markers like 被 
(\prettyref{sec:verb-phrase.object.short-bei})
or 给 (\prettyref{sec:ver-phrase.gei}) appear, 
but the meaning is passive
(\prettyref{ex:verb-phrase.notional-pass.1},
\prettyref{ex:verb-phrase.notional-pass.trans-1}; 
compare the existence of the passive voice in the English translations
and the absence of any passive marker in the Mandarin examples).
If a verb can appear in the notional passive,
then it has no problem appearing in the \corpus{bǎ}-construction
(\prettyref{ex:verb-phrase.notional-pass.ba-1}, \prettyref{sec:verb-phrase.object.ba.cause-become}).
The long \corpus{bèi}-construction, however, 
may seem somehow strange
(\prettyref{ex:verb-phrase.notional-pass.bei-1}, \prettyref{sec:verb-phrase.bei.passive-alternation}).

\begin{exe}
    \ex\label{ex:verb-phrase.notional-pass.1} 
    \gll 茶 泡 好 了 \\
    tea soak well \category{asp} \\
    \translate{Tea has been prepared.
    (lit. Tea has soaked well)}
    \ex\label{ex:verb-phrase.notional-pass.trans-1} 
    \gll 我 已经 泡 好 茶 了 \\
    1 already soak well tea \category{asp} \\
    \glt \translate{I have already prepared tea. (lit. I already have soaked tea well.)}
    \ex\label{ex:verb-phrase.notional-pass.ba-1}
    \gll 我 把 茶 泡 好 了 \\
    1 \category{ba} tea soak well \category{asp} \\
    \glt \translate{I have already prepared tea.}
    \ex\label{ex:verb-phrase.notional-pass.bei-1} 
    \gll ? 茶 被 我 泡 好 了 \\
    {} tea \category{bei} 1 soak well \category{asp} \\
    \glt \translate{Tea is prepared by me.}
\end{exe}

\subsubsection{The \category{be}-\category{become} alternation}

A verb or adjective (TODO: what category?) that usually appear in a \category{be} structure 
(\prettyref{ex:verb-phrase.be-become.source-1})
can be semi-regularly (see below) inserted into a \category{become} structure 
(\prettyref{ex:verb-phrase.be-become.1})
or further, in a \category{cause}-\category{become} structure 
(\prettyref{ex:verb-phrase.be-become.cause-1}).

\begin{exe}
    \ex\label{ex:verb-phrase.be-become.source-1} 他对物理学的知识一向丰富
    \ex\label{ex:verb-phrase.be-become.1} 经过这次实地考察,我们对这片山区的了解更加清楚了
    \ex\label{ex:verb-phrase.be-become.cause-1} 这次考察丰富了我们对地质学的认识
\end{exe}

Note that if the predicator of 
a clause with the structure of (\prettyref{ex:verb-phrase.be-become.cause-1})
is usually found in a \category{be} structure, 
then it is clearly to be interpreted as a \category{cause}-\category{become} structure, 
but this doesn't mean the \category{cause}-\category{become} structure
is always available: 
(\prettyref{ex:verb-phrase.be-become.cause-2}) seems problematic,
although its predicator, 红火, clearly is an adjective (TODO: ref)
and is able to appear in a \category{be} structure 
(\prettyref{ex:verb-phrase.be-become.2}).

\begin{exe}
    \ex\label{ex:verb-phrase.be-become.2} 
    \gll 我们 家 的 日子 真是 越来越 红火 了 \\
    our home \category{poss} live truly more.and.more booming \category{asp} \\
    \glt \translate{Our lives are increasingly improving.}
    \ex\label{ex:verb-phrase.be-become.cause-2} ???新政策红火了我们的日子
\end{exe}

\category{become}-structures or \category{cause}-\category{become} structures 
derived from \category{be}-structures 
are usually unable to appear in \category{ba}- or \category{bei}-constructions
(\prettyref{ex:verb-phrase.be.impossible-1},
\prettyref{ex:verb-phrase.be.impossible-2},
\prettyref{ex:verb-phrase.be.impossible-3}).
The reason for unavailability of \corpus{bèi}-constructions
seems to be the same as the reason in \prettyref{sec:verb-phrase.bei.passive-alternation}.
TODO: why ba is not good?

\begin{exe}
    \ex\label{ex:verb-phrase.be.impossible-1} *我们的知识被丰富了
    \ex\label{ex:verb-phrase.be.impossible-2} *这场展览把我们的知识丰富了
    \ex\label{ex:verb-phrase.be.impossible-3} *我们的知识被这场展览丰富了
\end{exe}

\subsubsection{The \category{do}-\category{become} alternation}\label{sec:ver-phrase.gei}

Not all verbs can appear in the notional passive construction.
There exists another construction -- the \corpus{gěi}-passive construction -- 
that has similar meaning with the notional passive
(i.e. the internal state of the subject somehow changes 
without the external cause -- if any -- being specified) and
can be observed on its own 
in very colloquial and non-standard speech
(\prettyref{ex:verb-phrase.gei.1}),
sometimes for stylist and humor purposes.

\begin{exe}
    \ex\label{ex:verb-phrase.gei.1} \% 李四给杀了
\end{exe}

The alternation between the notional passive and the \corpus{gěi}-construction 
is quite intriguing.
It means the two are not free variants of the marker of the \category{become} argument structure,
but are in contrast distribution:
给 usually appears with verbs that prototypically appear in the \category{do} structure
(\prettyref{ex:verb-phrase.gei.alternation-1},
\prettyref{ex:verb-phrase.gei.alternation-2},
\prettyref{ex:verb-phrase.gei.alternation-3}).
给 also renders the whole predicate strongly telic.
It seems to never appear without the aspectual marker 了.
It's likely that 给 has already developed a distinct usage 
as a valency changing marker, 
which turns a \category{do} verb into a \category{become} one,
while the notional passive construction directly applies to \category{become} verb.

\begin{exe}
    \ex\label{ex:verb-phrase.gei.alternation-1} \begin{xlist}
        \ex ???我给喝醉了
        \ex 我喝醉了
    \end{xlist}
    \ex\label{ex:verb-phrase.gei.alternation-2} \begin{xlist}
        \ex ??? 茶给泡好了
        \ex 茶泡好了
    \end{xlist}
    \ex\label{ex:verb-phrase.gei.alternation-3} \begin{xlist}
        \ex \% 我给灌醉了
        \ex *我灌醉了
    \end{xlist}
\end{exe}

给 also appears as a part of the \corpus{bǎ}-construction
(\prettyref{ex:verb-phrase.gei.ba-1})
and the long \corpus{bèi}-construction
(\prettyref{ex:verb-phrase.gei.bei-1}),
although after removing 给,
the instances of the \corpus{bǎ}- and \corpus{bèi}-constructions above 
are still grammatical.
In \corpus{bǎ}- and \corpus{bèi}-constructions, 
the appearance of 给 seems to have no link 
with the \category{do}-\category{become} distinction
(\prettyref{ex:verb-phrase.gei.ba-2}; c.f. \prettyref{ex:verb-phrase.gei.alternation-1}).
The \corpus{gěi}-construction however is not compatible 
with the short \corpus{bèi}-construction
(\prettyref{ex:verb-phrase.gei.bei-2}).
A reasonable guess, then, 
is that in the \corpus{bǎ}-construction
and the long \corpus{bèi}-construction,
给 marks an embedded \category{become} structure,
regardless of whether a \category{do} structure is embedded inside,
while the short \corpus{bèi} construction is completely isolated 
from this pipeline.

\begin{exe}
    \ex\label{ex:verb-phrase.gei.ba-1} 他们把李四给杀了
    \ex\label{ex:verb-phrase.gei.bei-1} 李四被他们给杀了
    \ex\label{ex:verb-phrase.gei.bei-2} *李四被给杀了
    \ex\label{ex:verb-phrase.gei.ba-2} 这瓶酒把我给喝醉了
\end{exe}

Sometimes, 把 in the \corpus{bǎ}-construction can be replaced by 给.
It seems if the verb is \category{become} (我笑麻了),
then 给 has the same function as 把's, 
while if the verb is \category{do},
给 has the same function as \category{become}.

\begin{exe}
    \ex \% 有人给我发了一封邮件,属实给我笑麻了
    \ex \% 这个情况给我整不会了
\end{exe}

\begin{exe}
    \ex\label{ex:verb-phrase.gei.2} \% 李四给他们杀了
\end{exe}

\subsection{The short \corpus{bèi}-passive construction}\label{sec:verb-phrase.object.short-bei}

Although 给 and 被 seem similar at the same glance, 
the former is able to appear in a ba-construction, 
while the latter is never able to do so. 

\begin{exe}
    \ex 李四被他们杀了
    \ex 李四给他们杀了
    \ex *他们把李四被杀了
\end{exe}


\subsection{The \corpus{bǎ}-construction}\label{sec:verb-phrase.object.ba}

\subsubsection{\corpus{Bǎ}-construction and the \category{cause}-\category{become} structure}
\label{sec:verb-phrase.object.ba.cause-become}

The \corpus{bǎ}-construction
is an alternative way to realize 
the \category{cause}-\category{become} structure
(\prettyref{sec:verb-phrase.cause-become.ordinary}),
regardless of whether this \category{cause}-\category{become} structure 
comes from a \category{do} structure 
(\citealt[\citepages{98-99}]{deng2010formal});
this gives rise to some mismatch between the \corpus{bǎ}-construction
and the long \corpus{bèi}-construction 
(\prettyref{sec:verb-phrase.bei.no-ba}).

The S=O ambitransitive valency alternation 
in examples \prettyref{ex:verb-phrase.ba.cause.1}, \prettyref{ex:verb-phrase.ba.cause.2},
as well as the non-availability of the long \corpus{bèi}-construction
(\prettyref{ex:verb-phrase.ba.bei-1}, \prettyref{ex:verb-phrase.ba.bei-2})
and the state-changing semantics,
means these examples contain a typical \category{cause}-\category{become} structure 
described in \prettyref{sec:verb-phrase.cause-become.ordinary}.
In both examples, 
\corpus{bǎ}-constructions are available 
(\prettyref{ex:verb-phrase.ba.ba-1}, \prettyref{ex:verb-phrase.ba.ba-2}).

\begin{exe}
    \ex\label{ex:verb-phrase.ba.cause.1} \begin{xlist}
        \ex 我喝醉了
        \ex ?这瓶酒喝醉了我
    \end{xlist}
    \ex\label{ex:verb-phrase.ba.bei-1} ???我被这瓶酒喝醉了
    \ex\label{ex:verb-phrase.ba.ba-1} 这瓶酒把我喝醉了
\end{exe}

\begin{exe}
    \ex\label{ex:verb-phrase.ba.cause.2} \begin{xlist}
        \ex 我走得累死了
        \ex 这条路走得我累死了
    \end{xlist}
    \ex\label{ex:verb-phrase.ba.bei-2} ???我被这条路走得累死了
    \ex\label{ex:verb-phrase.ba.ba-2} 这条路把我走得累死了
\end{exe}

It's also possible that the 

\begin{exe}
    \ex *我害惨了
    \ex 你这下把我害惨了
    \ex 你这下害惨我了
    \ex 我被你害惨了
\end{exe}

\subsubsection{Comparison with lexical causatives}

The \corpus{bǎ}-construction is also referred to as 
the disposal construction or the causative construction,
which involves the auxiliary 把, followed by a patientive argument 
and then a residue \acs{vp},
with the direct object moved out;
the subject is agentive (\prettyref{ex:verb-phrase.ba.ex-1}).
It should be noted that the \corpus{bǎ}-construction is not
a causative construction that \emph{blindly} takes any existing argument structure as the input.
It seems what appears after 把 is never quite agentive:
this is only possible with a lexical causative verb like 让
(\prettyref{ex:verb-phrase.ba.2}, \prettyref{ex:verb-phrase.ba.correct-2}).
Thus, the \corpus{bǎ}-construction is unable to attach a \category{causer} argument 
to any existing argument structure. 

\begin{exe}
    \ex \label{ex:verb-phrase.ba.ex-1}
    \gll [他们]_{\text{subject}} [把 [李四]_{\text{object}} 杀 了]_{\text{predicate:\corpus{bǎ}-\acs{vp}}} \\
    3pl \category{ba} (name) kill \category{asp} \\
    \glt \translate{They have killed Li Si.}
\end{exe}

\begin{exe}
    \ex\label{ex:verb-phrase.ba.2} \gll * 他 把 我 去 爬山 \\
    {} 3 \category{ba} 1 go climb.mountain \\
    \glt \translate{He lets me climb mountains.}
    \ex\label{ex:verb-phrase.ba.correct-2} \gll 他 让 我 去 爬山 \\
    3 let 1 go climb.mountain \\
    \glt \translate{He lets me climb mountains.}
\end{exe}

\subsection{The long \corpus{bèi}-construction}

\subsubsection{Long \corpus{bèi}-construction without \corpus{bǎ} counterpart}
\label{sec:verb-phrase.bei.no-ba}

However, the \category{cause}-\category{become} argument structure
doesn't seem to be the only source of the long \corpus{bèi}-construction.
There exist long \corpus{bèi}-constructions whose \corpus{bǎ}-counterparts 
are at best awkward and usually not acceptable.

\begin{exe}
    \ex 李四被张三抢了一顶帽子
    \ex 被不喜欢的人喜欢是很让人为难的一件事
    \ex 我被他批评了一番,感觉非常不爽
\end{exe}

\begin{exe}
    \ex ??张三把李四抢了一顶帽子
    \ex *他真是把自己的老婆喜欢啊
    \ex ?他把我批评了一番
\end{exe}

\subsubsection{Alternation between the notional passive and the long \corpus{bèi}-passive}
\label{sec:verb-phrase.bei.passive-alternation}

One interesting observation is the long \corpus{bèi}-construction
is less acceptable for 
a verb that is able to appear in the notional passive construction
(\prettyref{ex:verb-phrase.notional-pass.bei-1},
\prettyref{ex:verb-phrase.bei.notional-pass-conflict-1},
\prettyref{ex:verb-phrase.bei.notional-pass-conflict-2}).
That's to say, 
if a verb can appear in the \category{become} structure \term{on its own},
it's less likely to appear in the long \corpus{bèi}-passive.
This might be motivated by semantic reasons:
since both the long and the short \corpus{bèi}-constructions 
have the meaning of ``being influenced by some external factors'',
a verb indicating an ``automatic'' internal state change 
has semantic incompatibility with these constructions. 
Indeed, when both situations seem plausible, 
both the notional passive and the long \corpus{bèi}-passive 
are available (\prettyref{ex:verb-phrase.bei.notional-pass-conflict-3}).

\begin{exe}
    \ex\label{ex:verb-phrase.bei.notional-pass-conflict-1} \begin{xlist}
        \ex 茶泡好了
        \ex ?茶被我泡好了
    \end{xlist}
    \ex\label{ex:verb-phrase.bei.notional-pass-conflict-2} \begin{xlist}
        \ex *他杀了
        \ex 他被强盗给杀了
    \end{xlist}
    \ex\label{ex:verb-phrase.bei.notional-pass-conflict-3} \begin{xlist}
        \ex 你的提案已经交到程序委员会了
        \ex 你的提案已经被交到程序委员会了
    \end{xlist}
\end{exe}


\subsection{The affective construction}

\begin{exe}
    \ex \gll 王冕 死 了 父亲 \\
    (name) die \category{asp} father \\
    \glt \translate{Wang Mian's father died.}
    \ex 才几个月,这家工厂就已经坏了三台机床了
\end{exe}

\section{Semi-objects and verb ionization}\label{sec:verb-splitting}

There may be a numeral expression in the \acs{vp}
that gives the ``quantity'' of the event, 
which is called the \concept{semi-object} 
\citep[\citesec{8.6}]{zhudexigrammar}.
A semi-object may be a counting expression, 
a time expression, 
or a pure numerical expression (TODO: examples, and more concise terms).
Their syntactic and semantic functions 
are closer to numeral attributives in \acs{np}s.
They are called ``objects '' purely because they are within the \acs{vp}
and are \acs{np}s themselves;
this note doesn't recognize them as objects;
the term \term{semi-object} is only used to TODO: so is it really necessary to use the term?

\begin{infobox}{The coverage of the term \term{semi-object}}{semi-object-coverage}
    Apart from the numeral attributives in \acs{vp} discussed above, 
    numerals appearing at the end of \acs{vp}s 
    are also sometimes called semi-objects \citep[\citepage{117}]{deng2010formal}.
    The syntactic function numerals in this latter case 
    is closer to verbal complements (TODO: ref).
    And actually the term \term{semi-object} works better in the latter case!
    TODO: why are we sure that the two types of semi-objects have different syntactic positions?
\end{infobox}

\subsection{Types of verb ionization structures}

It's sometimes possible to split a verb 
and inject some clausal dependents into it
(\prettyref{ex:junwanlexun}, \prettyref{ex:youmo}, \prettyref{ex:guanshenmexin}).
This phenomenon is known as 
\concept{verb separation} or \concept{verb ionization} \citet[\citesec{6.5.8}]{chao1965grammar}.
The injected clausal dependent is usually 
a resultative complement (\prettyref{ex:junwanlexun}) 
or a semi-object
(\prettyref{ex:guanshenmexin}, \prettyref{ex:verb-phrase.separation.xuexi}), 
but in marginal cases, 
an object personal pronoun is also possible
beside a short semi-object (\prettyref{ex:guanshenmexin}). 

\begin{exe}
    \ex\label{ex:junwanlexun} 
    \gll \% [军 完 了 训]_{\text{\acs{vp}}} 以后 才 可以 去 请 护照 \\
    {} military finish \category{asp} training after only.after(TODO) can go ask.for passport \\
    \glt \translate{(We) can only apply for a passport after finishing military training.} 
    \citet[\citesec{6.5.8}]{chao1965grammar}
    \ex\label{ex:youmo} \% 还 [幽了他一默]_{\text{\acs{vp}}}
    \ex\label{ex:guanshenmexin} 这件事情你 [关什么心]_{\text{\acs{vp}}} 啊
    \ex\label{ex:verb-phrase.separation.xuexi} 我[学了两个小时的习]_{\text{\acs{vp}}}
\end{exe}

Whenever a semi-object is inserted, 
the resulting structure is uncontroversially an \acs{vp}
(\prettyref{ex:youmo}, \prettyref{ex:guanshenmexin}, \prettyref{ex:verb-phrase.separation.xuexi}).
Insertion of a resultative complement 
seems to be only available when the verb is intransitive,
so the resulting structure is also an \acs{vp} on its own.%
\footnote{
    And therefore whether 军完了训 is directly a \ac{vp} or is first a verb complex 
    and then a \ac{vp} is not of much importance.
}
It's usually possible to move the inserted constituent out
(\prettyref{ex:verb-phrase.separation.junxun-2},
\prettyref{ex:verb-phrase.separation.guanxin-2},
\prettyref{ex:verb-phrase.separation.xuexi-2}), 
except in (\prettyref{ex:youmo}, TODO: general condition).
When this is possible, 
the resulting structure has exactly the same meaning 
with the verb separation structure.

\begin{exe}
    \ex\label{ex:verb-phrase.separation.junxun-2} 军训完了以后才可以去请护照
    \ex\label{ex:verb-phrase.separation.guanxin-2} 这件事你关心什么啊
    \ex\label{ex:verb-phrase.separation.xuexi-2} 我学习了两个小时
\end{exe}

\subsubsection{Motivation of verb ionization}

The splitting of the verbs clearly origins 
by analogy with \ac{vp}s containing morphosyntactic words.
军完了训 is apparently created by analogy with 
[[吃完了]_{\text{predicator:verb complex}} [饭]_{\text{object:N}}]_{\text{VP}}.
The motivation of this analogy seems to be prosody: 
splitting words into phrases is only observed in \ac{vp}s,
and \ac{vp}s are subject to the prosodic constraint 
that the neither the verb nor the final complement can be too light.
Splitting the verb may help to reduce the ``weight'' of the verb 
so the resulting utterance meets the prosodic constraint better.

The phenomenon of verb ionization
looks just like infixing,
in which a word is split even when it has no analyzable morphosyntactic inner structure.
Here, however, this infixing operation creates an \acs{vp} instead of a grammatical word.
This justifies the assumption taken at the end of \prettyref{sec:theory}
that there is no clear boundary between words and phrases 
and therefore syntax and morphology:
It's possible for a phrase to undergo 
rearrangement without clear syntax motivation
that usually happens within a word.

\begin{theorybox}{The two faces of the term \term{morphology}}{morphology-meaning}
    When the term \term{morphology} is used, 
    people use it to refer to two things:
    the first is what happens within a grammatical word, 
    the second is phenomena that involve high localized
    operations that don't look like syntax 
    but are also not prototypically phonological on the other hand. 
    
    Verb ionization involves 
    splitting a word,
    and is therefore of course morphological (in a highly non-trivial way)
    in the second sense.
    However, what is injected in after splitting the verb are phrasal dependents 
    and the resulting morphosyntactic unit is a phrase,
    so verb ionization is syntactic and not morphological in the first sense.
    What I want to emphasize here is surface realization of a unit 
    and its abstract constituency/dependency structure
    may be not always the same, 
    although the latter strongly influences the former.
\end{theorybox}

\subsubsection{Comparison with similar constructions}\label{sec:verb-phrase.ionization.similar}

It should be noted that some so-called verb ionization examples 
seem to be analyzable as 
trivially extending the object: 
compare 看书 and 看了三本书.
Some verbs with verb-object inner structures 
contain an internal object that usually doesn't appear as a full \acs{np}, 
but that is not absolute: 
洗了一个舒服的澡 and 一个舒服的澡对睡眠有好处.
These constructions are excluded from the category of verb ionization,
although as is shown above, 
they may historically motivate the emergence of verb ionization.

Other verbal items that are prosodic words
seem to be extendable by 
both modification within the object 
and verb ionization.
The prosodic word 念佛 may be extended into 念阿弥陀佛 (extending the object)
as well as 念三声佛 (inserting a semi-object, verb ionization).
This happens for 染发 as well: 
we have both 染了一头蓝发 (extending the object) 
and 染了一次头发.
Since the semi-object can also intervene
between the verb and the direct object in uncontroversial \acs{vp}s 
(\prettyref{ex:verb-phrase.separation.nianfo-full-vp}), 
it seems disyllabic (and therefore prosodic) verb-object structures 
have the same behaviors with 
longer verb-object structures.

\begin{exe}
    \ex\label{ex:verb-phrase.separation.nianfo-full-vp} 老太太念了十多年的阿弥陀佛,却说不清阿弥陀佛是谁
\end{exe}

\section{Negation}

Mandarin has two attested negators: 不 and 没.
不 is always used together with the habitual aspect:
他是回民,不吃猪肉.
没 is used together with a non-habitual aspect:
我那顿饭没吃猪肉.

When 不 is used together with a potential complement, 
we need to remove 得 and insert 不 in its position

我不是算不清楚账,但是那天不知怎么的就是没算清楚帐

\section{The aspectual system}\label{sec:aspectual}

\subsection{\acs{tame} categories}

Mandarin lacks the category of tense -- 
all tense information is expressed by time adverbs.
Modality is marked similarly be adverbs or complement clause constructions.
Yet there is a system marking the aspect (\prettyref{sec:aspectual}). 
\eqref{ex:quguo-qule} is an example.

\begin{exe}
    \ex \begin{xlist}
        \ex \gll 我 去 过 上海 了 \\
        1 go \asis{guo} Shanghai \acs{sfp} \\
        \glt \translate{I have been in Shanghai.}
        \ex \gll 我 去 了 上海 了 \\
        1 go \asis{le} Shanghai \acs{sfp} \\
        \glt \translate{I have gone to Shanghai.}
    \end{xlist}
    \label{ex:quguo-qule}
\end{exe}

标语贴在墙上 标语已经在墙上贴着了: 
this means the preposition 在 actually is morphologically merged with the verb 贴, 
or otherwise we are unable to explain why 
in the first example, 着 can never appear, 
while in the second example, 着 can appear.

Although 着 can appear in a matrix clause, 
its distribution is wider in temporal adverbials. 

*他笑着。
他[笑着]走了进来

\section{Prosodic constraint}\label{sec:vp.prosody}

All Mandarin \acs{vp}s follow the following prosodic constraint:
if the verbal complex is transitive,
then the constituent after it should receive prosodic focus; 
otherwise the verbal complex should be able to receive prosodic focus. 
This means in a transitive clause, 
the verbal complex can't be too heavy, 
while in an intransitive clause (TODO: 把 construction), 
the verbal complex can't be too light. 

It should be noted that this constraint doesn't apply 
to other type of syntactic constructions, 
even though a verb root appears. 
Thus, *[种植树]_{\text{\acs{vp}}} \translate{plant trees} 
is not grammatical
because in this \acs{vp} the verbal complex is heavier than the object, 
while it's the object that is supposed to receive prosodic focus, 
but [种植牙]_{\text{compound noun}} is perfectly fine. 

\section{The short bei-construction}

Although the only difference between the short bei-construction 
and the long bei-construction 
seems to be that the former lacks the semantic agent, 
the two constructions have important grammatical differences 
that seem to be not motivated by semantics. 

\section{The notional passive, the ba-construction, and the long bei-construction}

把, 给, 被 appearing together: 给 is the \category{become} verb; 
李四给人杀了 = 李四 \category{become} [人杀了 e], 
where 李四 moves out of the DoP.

\begin{exe}
    \ex 李四给杀了
    \ex 李四给他们杀了
    \ex 他们把李四杀了
    \ex 他们把李四给杀了
    \ex *他们把李四被杀了
    \ex 李四被他们杀了
    \ex 李四被他们给杀了
\end{exe}

On the other hand, long bei-constructions are probably 
from the \category{cause}-\category{become} structure,
which can be demonstrated by the long-distance dependency relations 
observed in both the ba-construction  
and the long bei-construction
\citep[\citesec{4.2.1.5}]{huang2013}.
那块手表被李四用一个锤子砸烂了 
Here the \category{become}-structure would be 
?那块手表用一个锤子给砸烂了. 
李四 appears as a \category{causer},
and if the derivation stops here, 
把 is inserted, and we get 李四用一个锤子把那块手表给砸烂了.
If, however, on top of the \category{cause}-\category{become} structure, 
we insert 被 and move 那块手表 out, 
把 is no longer spelt out 
and we just get 那块手表被李四用一个锤子砸烂了.
Note that in the above procedure, 
the most internal clausal dependents are completely inactive: 
we can replace 砸烂 by 砸成了一堆破铜烂铁, 
and everything is still completely grammatical.

\begin{infobox}{Valency increasing, or just different subcategorization frames?}{status-of-ba}
    One difference between my analysis here and the analysis in \citet[\citepage{202}]{deng2010formal}
    is that the latter assumes that the input to the gei-construction 
    is a verb in the \category{become} frame.
    This however is unable to explain why we get 他给人家杀了,
    in which the agent 人家 appears after 给.
    There are however definitely vagueness in whether 
    the gei-construction can be applied to an existing \category{do} argument structure.
\end{infobox}

\section{The structure and position of the verbal complex}

When ba-, gei- and bei-constructions are not available,
the verbal complex always appears at the initial position 
of the core \acs{vp}, 
that is, it appears immediately after the adverbials 
and after the negator if there is any. 
Thus, Mandarin is usually classified as an SVO language.
Although the existence of the ba--construction
casts doubt on this classification, 
TODO

\section{There is no serial-verb construction or complex predicate}\label{sec:no-serial-verb}

The term \term{serial-verb construction} refers to several different things in the literature.
Sometimes it refers to the verbal complement system
\citep{chen2016mandarin}, 
although in the topological literature 
there is no longer considered as a good usage 
\citep[\citesec{10.1}; note that %
the V2s in Yakkha complex predication highly resembles Mandarin directional verbal complements 
in their formal aspects]{schackow2015grammar}. 
In this sense, we of course have serial-verb constructions in Mandarin.

\chapter{Verb valency}

After introducing all possible forms of the \acs{vp}
and its relation with the subject,
I now give a thorough classification of verbs 
according to their semantics and valency.

There are two ways of valency changing in Mandarin.
The first is via a coverb construction, 
as in the disposal constructions (\prettyref{sec:disposal-construction}),
TODO 
The second is \emph{doing nothing} to the verb 
and relying on the unusual semantic roles of clausal complements 
to inform the listener about the valency changing,
as in TODO: ref.
Since there is no morphological marking,
constructions of this type are often recognized as topic-comment structures,
in which the ``topic'' -- which is the subject under closer investigation -- 
is said to be freely occupied by any semantic (and not necessarily syntactic) argument in the clause,
though this claim can be falsified by detailed syntactic tests (\prettyref{sec:topic-subject}).

\section{The disposal constructions}\label{sec:disposal-construction}

\section{The passive constructions}\label{sec:affected-construction}

\begin{exe}
    \ex 我被他打了一拳
\end{exe}

\section{The causative}

\section{The affected construction}



\section{Instrumental object}

\begin{exe}
    \ex 我们今天准备吃食堂
\end{exe}

\chapter{Simple clauses}

\section{Overall remarks about the clause structure}

A sentence can be divided into several clauses linked by clause linking constructions 
(\prettyref{chap:clause-linking}).
This chapter is denoted to the simple clause,
postponing details in subordination and clause linking to the next several chapters.
Mandarin has rich topicalization phenomena,
and thus a clause can be divided into
one or more topics (if any) and a comment,
the latter being the nucleus clause
plus possible \acl{sfp}s.
The comment -- the nucleus clause -- may further be divided into a subject (if any),
a series of adverbials, 
the verbal complex, and post-verbal constituents,
the most important types including object(s), 
the second part of a separable verb,
certain directional complements,
and purpose clauses.

\begin{infobox}{The term \term{clause}}{clause-def}
    Some people, like \citet[\citepage{140}]{deng2010formal}
    as well as \citet{dixon2009basic},
    use the term \term{clause} for subject-predict constructions 
    that don't receive complete marking of speech forces.
    (In generative terms, \term{clause} is for lower level CPs or even TPs.)
    So in this way, \acl{sfp}s shouldn't be discussed in this chapter because 
    they are of course dependents in the sentence level.
    They may be discussed together with other sentence-level constructions like \prettyref{chap:clause-linking}.
    But this notion of clause certainly goes against the tradition in descriptive grammars.
    So the approach of this note is to acknowledge everything larger than TP as a clause,
    which may or may not be a sentence,
    and discuss its structure in this chapter,
    while ``adjunctions'' -- or in other words, optional dependents -- 
    are discussed in, say, \prettyref{chap:clause-linking},
    for the sake of convenience.
    The narrative order of this note is not the ideal ``small unit -- large unit'' scheme,
    but the ``simple large unit -- complicated large unit'' scheme.
    Needless to say,
    when it comes to clause combining, 
    the problem of what the clause really is -- with or without \ac{sfp}s, for example --
    is still relevant,
    but it is not answered by saying ``the construction takes a clause, not a sentence''.
\end{infobox}

As is implied by my using the term \term{subject},
Mandarin is an typical accusative language.
Clausal dependents are recognizable from the rather rigid constituent order:
Mandarin is usually classified as having a SVO clausal constituent order,
and the subject and the object(s) can be told from the positions in the clause 
(\ref{ex:get-sick}, \ref{ex:svo-example}).
Certain ``SOV'' orders can be obtained by invoking the disposal construction
(\prettyref{sec:disposal-construction}), as in \eqref{ex:ba-example}.

\begin{exe}
    \ex \gll 我 生病 了 \\
    1 get.sick \acs{sfp} \\
    \glt \translate{I got sick.}
    \label{ex:get-sick}

    \ex \gll [我]_{\text{subject}} 今天 去 看 [电影]_{\text{object}} 了 \\
    1 today to watch movie \acs{sfp} \\
    \glt \translate{I went to watch a movie today.} 
    \label{ex:svo-example}

    \ex \gll [我]_{\text{subject}} 今天 把 [ 一 个 碗 ]_{\text{object}} 摔 碎 了 \\
    1 today \category{ba} {} one \acs{classify} bowl {} break crack \acs{sfp} \\
    \glt \translate{I broke one bowl today.}
    \label{ex:ba-example}
\end{exe}

The normal tests of syntactic accusative alignment can be run on Mandarin
(\ref{ex:inter-sentence}).

\begin{exe}
    \ex \gll 陈 经理 昨天 没有 和 他的 客户 聊 过 。 他 生病 了 。 \\
    {Chen (surname)} manager yesterday \acs{neg} with 3sg-\acs{possessive} client talk \acs{sfp}
    {} 3sg get.sick \acs{sfp} \\
    \glt \translate{Manager Chen didn't talk with his client yesterday. He (Chen, not his client) got sick.}
    \label{ex:inter-sentence}
\end{exe}

\section{Prosodic constraint}\label{sec:clause.prosodic-constraint}

The main verb and one post-verbal constituent
should form the last prosodic constituent in the clause (ignoring \ac{sfp}s),
and when there is no post-verbal constituents,
the main verb receives the natural stress. 
This is actually a rather strong condition.
Certain constituents -- most functional words -- are unable to accept stress at all.
They may be freely merged into the closest prosodic constituents.
Certain constituents -- like \ac{np}s with definite references -- are by default stressed.
When the verb is the last lexical constituent
(as in *他挥动棒子把我打),
it should not be too short or otherwise it is unable to receive stress.
When there are more than one post-verbal constituents,
only one of them can receive stress.
If two post-verbal constituents are both by default stressed,
the sentence is again ruled out. % TODO: ref

\section{Types of nucleus clauses}

\begin{infobox}{About the subject-predicate binary division}{subject-predicate}
    \citet{dixon2009basic} argues against the definition of \term{predicate} 
    as the main verb (or adjective) plus somehow ``internal'' arguments.
    He uses the term \term{predicate} to refer to the verbal complex instead.
    However, since I will need to compare the topic-comment construction 
    with the inner structure of the nucleus clause,
    the term \term{predicate} will still be used in the way \citet{dixon2009basic} dislikes,
    because it's the counterpart of the comment in the topic-comment construction.
\end{infobox}



\section{Negation}\label{sec:negation}

Like the case in standard English, 
there is no negative concord in Mandarin Chinese.
There is, however, no uniform negation operator like the English \emph{not}. 
Several negation operators and strategies are used frequently (\prettyref{sec:negation}).
Verbs can be negated by 不 while nouns generally cannot, 
and this is a criterion to tell verbs from nouns. 
There is another negation operator 没, 
which has subtle differences in its meaning and syntactic properties compared with 不
(\ref{ex:chiqincai}, \ref{ex:buchi-meichi}).
On the other hand, the negative potential complement construction,
i.e. the V不了 construction,
isn't obtained by inserting a negator in the clause \eqref{ex:zuobuliao-example}.

\begin{exe}
    \ex \begin{xlist}
        \ex \gll 我 不 喜欢 吃 芹菜 \\
        1 \acs{neg} like eat celery \\
        \glt \translate{I don't like eating celery.} \\
        \ex * 我 没 喜欢 吃 芹菜
    \end{xlist}
    \label{ex:chiqincai}
\end{exe}

\begin{exe}
    \ex \begin{xlist}
        \ex \gll 我 不 吃 早饭 \\
        1 \acs{neg} eat breakfast \\
        \glt \translate{I don't eat breakfast. (I usually don't, I don't want any today, etc.)}
        \ex \gll 我 没 吃 早饭 \\
        1 \acs{neg} eat breakfast \\
        \glt \translate{I didn't eat breakfast. (I may usually do, but somehow I didn't today.)}
    \end{xlist}
    \label{ex:buchi-meichi}
\end{exe}

\begin{exe}
    \ex \begin{xlist}
        \ex \gll 我 做 [ 不 了 ]_{\text{potential complement, negative}} 这 件 事 \\
        1 do {} \acs{neg} finish {} this \acs{classify} affair \\
        \glt \translate{I'm not able to do this.}
        \ex[*]{\gll 我 \oneof{没有/并非/不} 做 [ 得 了 ]_{\text{potential complement, positive}} 这 件 事 \\
        1 \acs{neg} do {} \asis{de} finish {} this \acs{classify} affair \\}
    \end{xlist}    
    \label{ex:zuobuliao-example}
\end{exe}

\section{Sentence final particles}

\section{Cleft construction}

他是昨天才知道这个消息的
他昨天才知道这个消息

他是王教授招进来的
王教授把他招进来

Note that in the second example, 
the verbal complex 招进来 is too heavy, 
and the ba-construction is used to meet the prosodic constraint.

\section{The topic-comment structure}

I follow \citet{sih2000topic}'s approach and define a topic as an unmarked \acs{np} 
that has certain relations with a position in the clause after it
and is indeed the topic in the information structure
(i.e. some (probably already known) object to which new information is added).
Constructions like 连\dots都\dots are not discussed in this section -- 
they are to be found in TODO: ref.

\subsection{Topicalization of possessor}

\eqref{ex:tagezigaogaode} and \eqref{ex:tagezigaogaode} are a pair of sentences 
with and without topicalization of the possessor in the subject.

\begin{exe}
    \ex \begin{xlist}
        \ex\label{ex:tagezigaogaode}  
        \gll [他]_{\text{topic}} [ [个子]_{\text{subject}} 高高 的 ]_{\text{comment}} \\
        3sg {} stature tall\redp{}\asis{todo} \asis{de} \\
        \glt \translate{As for him, the stature is tall.}
        \ex\label{ex:tadegezigaogaode} \gll [ 他 的 个子 ]_{\text{subject}} 高高 的 \\
        {} 3sg \acs{possessive} stature {} tall\redp{}\asis{todo} \asis{de} \\
        \glt \translate{His stature is tall.}
    \end{xlist}
\end{exe}

\subsection{Topicalization of preposition objects}\label{sec:topicalization-of-preposition-objects}

\begin{exe}
    \ex\label{ex:zhejianshinibunengjiumafantayigeren} 这件事你不能就麻烦他一个人
    \ex 你不能[为了这件事]_{\text{adverbial:\acs{pp}}} 就麻烦他一个人
\end{exe}
This is also a demonstration of the preposition status of 在 in this sentence (\prettyref{sec:preposition-pos}),
because if it's a verb or an auxiliary verb,
it will be hard to have its object topicalized and have it deleted at the same time,
but deletion of the preposition in topicalization is well-attested cross-linguistically.

\subsection{Origins of so-called ``dangling topics''}\label{sec:topic-subject}

Some people, like \citet[\citesec{7.1}]{zhudexigrammar},
equate \term{subject} with \term{topic} in Mandarin grammar.
Some (especially those from the functional-typological tradition) go further 
and assert that ``the notion of the subject (as the position of the most agentive argument) 
isn't grammaticalized in Mandarin Chinese'',
and therefore the topic is just an \acs{np} which the comment is ``about'',
and this base-generated and syntactically unconstrained topic 
is called a ``dangling topic''.
This view is rejected in this note,
because such accounts usually end up in severe overgeneration. 
Here I briefly summarize \citet{sih2000topic}'s argumentation.

\subsubsection{Type 1: Idiomatic phrasal predicate looking like a comment}\label{sec:clause.dangling-topic.1}

In the first type of ``dangling topic'',
it's impossible for any \acs{np} in the comment to be syntactically related to the topic.
Such cases are however rather unproductive. 
In \eqref{ex:dayuchixiaoyu} and \eqref{ex:nikankanwo},
the orders of the constituents can never be changed.
Nor is it possible to change a word or two in the bracketed ``comments''.
A reasonable assumption is these bracketed ``comments''
are actually idioms, 
which are to be regarded as a single verbal element that can't be further analyzed.
Thus, in \eqref{ex:dayuchixiaoyu} and \eqref{ex:nikankanwo},
the so-called topic is an ordinary subject,
and the so-called comment is a predicate.

\begin{exe}
    \ex\label{ex:dayuchixiaoyu} 他们[大鱼吃小鱼](,厮杀成一片)
    \ex\label{ex:nikankanwo} 他们[你看看我我看看你]
\end{exe}

\subsubsection{Type 2: Quantificational adverbial looking like the inner subject}

The second type of ``dangling topic'' is like \eqref{ex:shui-dou-bu-pa}.
A topic-comment analysis of \eqref{ex:shui-dou-bu-pa} 

\begin{exe}
    \ex\label{ex:shui-dou-bu-pa} \gll 他们 谁 都 不 怕 \\
    3pl who even \acs{neg} fear \\
    \glt \translate{They don't fear anyone.}
\end{exe}

\subsubsection{Type 3: Ellipsis leaving a subject and one predicate}

Some people accept \eqref{ex:nasuofangzixingkuimeixiaxue}.
Here the \acs{np} 那所房子 definitely doesn't come from the words following it,
and is therefore recognized as a topic by some (TODO: ref). 
Note, however, that 幸亏 serves as a clause linker outside \eqref{ex:nasuofangzixingkuimeixiaxue}:
\eqref{ex:xingkui-buran-ex} is a demonstration of the 幸亏……不然…… linking construction,
and we also have its topicalized version \eqref{ex:xingkui-buran-fronted}. (TODO: whether this is parenthesis)
We also know in a clause linking construction,
often one clause can be omitted in the utterance because it's content can be easily inferred (TODO: ref).
So now the origin of \eqref{ex:nasuofangzixingkuimeixiaxue} is clear:
We can get it by omitting the second clause in the comment part of \eqref{ex:xingkui-buran-fronted}.
Indeed, if we replace 幸亏 by anything that is adverbial but not a clause linker,
the resulting sentence -- which now contains a real dangling topic -- is not grammatical.

\begin{exe}
    \ex \label{ex:nasuofangzixingkuimeixiaxue} \gll \% 那 所 房子 幸亏 没 下雪 \\
    {} that \acs{classify} house fortunate \acs{neg} snow \\
    \glt \translate{For that house, fortunately it didn't snow (or otherwise something bad would happen).}

    \ex\label{ex:xingkui-buran-ex} \gll [幸亏] 去年 没 下雪 , [不然] 那 所 房子 早就 塌 了 \\
    fortunate last.year \acs{neg} snow {} otherwise that \acs{classify} house already collapse \acs{sfp} \\
    \glt \translate{Fortunately it didn't snow last year, or otherwise that house has already collapsed.}

    \ex\label{ex:xingkui-buran-fronted} 
    \gll [ 那 所 房子 ]_{\text{topic}} [ 幸亏 去年 没 下雪 , 不然 早就 塌 了 ]_{\text{comment}} \\
    {} that \acs{classify} house {} {} fortunate last.year \acs{neg} snow {}  otherwise already collapse \acs{sfp} \\
\end{exe}

\subsubsection{Type 4: Extraction from prepositional adverbials}

\eqref{ex:zhejianshinibunengjiumafantayigeren} in \prettyref{sec:topicalization-of-preposition-objects} 
is sometimes regarded as an instance of the dangling topic construction.
However, as is shown in \prettyref{sec:topicalization-of-preposition-objects},
it may just be from topicalization of an \acs{np} in an adverbial,
with the preposition (and/or the locative particle) removed.

\subsubsection{Type 5: Nominal predicate}

\begin{exe}
    \ex 这种青菜一斤三十块钱
\end{exe}

\subsubsection{Type 6: Locational adverbial mistaken for the subject}

\begin{exe}
    \ex \gll \% 物价 纽约 最 贵  \\
    {} price New.York most expensive \\
    \glt \translate{The price in New York is the most expensive.}
\end{exe}

\subsubsection{Tentative conclusion}

The conclusion is all topics in Chinese are closely linked to a position in the comment,
be it a core argument position or a peripheral one.
So the notion of dangling topics is to be rejected in Mandarin grammar,
and we can always recover the ``canonical'' i.e. non-topic-comment clause
from a topic-comment structure.
After this, if the canonical clause can be divided into an \acs{np}
or a complement clause and a verbal constituent following it,
we can uncontroversially say the first is the subject while the second is the predicate. (TODO: predicate def)
So equating the subject with the topic is also wrong.

It's possible to find the semantic role of the subject isn't agentive;
in this case I assert there is a valency changing mechanism here.

\begin{infobox}{What to expect when people talk about the subject or the topic}{subject-topic}
    Unfortunately, despite the syntactic tests presented above,
    there are still many people -- even many native speakers -- 
    promoting the idea that the Mandarin topic has nothing different with the subject.
    Here is a list of TODO: ref
\end{infobox}

\chapter{Relative clause constructions}





\chapter{Complement clause constructions}\label{sec:complement-clause}


\begin{infobox}{Non-existence of finite-nonfinite distinction in Mandarin}{finiteness}
    Cross-linguistically, we find a finite-nonfinite distinction in subordination.
    This distinction is arguably absent in Mandarin,
    even after detailed syntactic tests \citep{no-finite}.
\end{infobox}

\chapter{Clause linking}\label{chap:clause-linking}

Mandarin Chinese has usual clause linker devices (\prettyref{chap:clause-linking}),
as well as complement clauses (\prettyref{sec:complement-clause})
and relative clauses (TODO: ref). TODO: what else?

\chapter{Summary and discussion}

\section{A typological summary}

\section{About the theoretical framework}

Many linguists call for a framework-less and completely open-minded approach towards syntactic analysis.
It's true (almost tautologically) that a grammar of a language 
should be organized according to the object language's 
own features.
Still, there exists the problem about \emph{how much} variation 
a linguist should expect when working with a totally unfamiliar language.

I'm not in the position to discuss whether the generative community 
is on the right track 
or whether the tendency to work on complex clause structures 
hinders the race against time to capture 
endangered languages.
What I do know -- which is illustrated in the discussion above -- 
is that kind of generativism I adopted in \prettyref{sec:theory}
does seem to work for Mandarin Chinese, 
despite the latter didn't play a strong role 
in the historical development of this framework.
We see the lexical-decomposition analysis 
and the \acs{vp}-shell theory 
neatly capture the structure of Mandarin \acs{vp}.
We see the category of clause can and should be 
further divided into subcategories with various internal complexities,
and the sizes of these subcategories can be placed on 
a monotonically increasing hierarchy,
which agrees well with the \vP-TP-CP hierarchy.
We see that on one hand, 
we can recognize grammatical words in Mandarin, 
and on the other hand, 
grammatical words are just mini-phrases.
And, most importantly, we have shown that most -- if not all -- mysterious traits 
of Mandarin have ingredients already well-known in other languages.
This is by no means a denial of linguistic diversity: 
on the contrary, 
that languages have choices over how to recombine these ingredients
helps us understand \emph{why} there is linguistic diversity at all.

\subsection{Necessity of large-volume grammars}

The next question is, 
since all natural languages have comparable complexity, 
whether the same thing should be done for less known languages.

\subsection{About how to teach Mandarin}

\printbibliography[title=References]

\end{document}