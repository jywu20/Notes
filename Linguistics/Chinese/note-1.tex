\documentclass[UTF8, a4paper, oneside, scheme=plain]{ctexrep}

\usepackage{libertinus}
\usepackage{geometry}
\usepackage{float}
\usepackage{titling}
\usepackage{titlesec}
\usepackage{paralist}
\usepackage{footnote}
\usepackage{enumerate}
\usepackage{amsmath, amssymb, amsthm}
\usepackage{gb4e}
\noautomath
\usepackage{bbm}
\usepackage{textcomp}
\usepackage{soul}
\usepackage{graphicx}
\usepackage{siunitx}
\usepackage[table,xcdraw]{xcolor}
\usepackage{tikz}
\usepackage[ruled, vlined, linesnumbered, noend]{algorithm2e}
\usepackage{xr-hyper}
\usepackage[colorlinks, citecolor = purple]{hyperref} % linkcolor=black, anchorcolor=black, citecolor=black, filecolor=black
\usepackage[most]{tcolorbox}
\usepackage{caption}
\usepackage{subcaption}
\usepackage{booktabs}
\usepackage{multirow}
\usepackage[figuresright]{rotating}
\usepackage{acro}
\usepackage[round]{natbib} 
\usepackage{nameref,zref-xr}
\zxrsetup{toltxlabel}
\zexternaldocument*[cgel-]{../English/cambridge}[cambridge.pdf]
\zexternaldocument*[alignment-]{../alignment/alignment}[alignment.pdf]
\zexternaldocument*[exercise1-]{../Exercise/2021-3}[2021-3.pdf]
\zexternaldocument*[method-]{../methodology/glossing}[glossing.pdf]
\usepackage{prettyref}

\geometry{left=3.18cm,right=3.18cm,top=2.54cm,bottom=2.54cm}
\titlespacing{\paragraph}{0pt}{1pt}{10pt}[20pt]
\setlength{\droptitle}{-5em}

\DeclareMathOperator{\timeorder}{\mathcal{T}}
\DeclareMathOperator{\diag}{diag}
\DeclareMathOperator{\legpoly}{P}
\DeclareMathOperator{\primevalue}{P}
\DeclareMathOperator{\sgn}{sgn}
\newcommand*{\ii}{\mathrm{i}}
\newcommand*{\ee}{\mathrm{e}}
\newcommand*{\const}{\mathrm{const}}
\newcommand*{\suchthat}{\quad \text{s.t.} \quad}
\newcommand*{\argmin}{\arg\min}
\newcommand*{\argmax}{\arg\max}
\newcommand*{\normalorder}[1]{: #1 :}
\newcommand*{\pair}[1]{\langle #1 \rangle}
\newcommand*{\fd}[1]{\mathcal{D} #1}

\newcommand*{\citesec}[1]{\S~{#1}}
\newcommand*{\citechap}[1]{chap.~{#1}}
\newcommand*{\citefig}[1]{Fig.~{#1}}
\newcommand*{\citetable}[1]{Table~{#1}}
\newcommand*{\citepage}[1]{pp.~{#1}}
\newcommand*{\citefootnote}[1]{fn.~{#1}}

\newrefformat{sec}{\citesec{\ref{#1}}}
\newrefformat{fig}{\citefig{\ref{#1}}}
\newrefformat{tbl}{\citetable{\ref{#1}}}
\newrefformat{chap}{\citechap{\ref{#1}}}
\newrefformat{fn}{\citefootnote{\ref{#1}}}
\newrefformat{box}{Box~\ref{#1}}

% color boxes

\tcbuselibrary{skins, breakable, theorems}

\newtcbtheorem[number within=chapter]{infobox}{Box}{
    enhanced,
    boxrule=0pt,
    colback=blue!5,
    colframe=blue!5,
    coltitle=blue!50,
    borderline west={4pt}{0pt}{blue!65},
    sharp corners,
    fonttitle=\bfseries, 
    breakable,
    before upper={\parindent15pt\noindent}}{box}
\newtcbtheorem[number within=chapter, use counter from=infobox]{theorybox}{Box}{
    enhanced,
    boxrule=0pt,
    colback=orange!5, 
    colframe=orange!5, 
    coltitle=orange!50,
    borderline west={4pt}{0pt}{orange!65},
    sharp corners,
    fonttitle=\bfseries, 
    breakable,
    before upper={\parindent15pt\noindent}}{box}
\newtcbtheorem[number within=chapter, use counter from=infobox]{learnbox}{Box}{
    enhanced,
    boxrule=0pt,
    colback=green!5,
    colframe=green!5,
    coltitle=green!50,
    borderline west={4pt}{0pt}{green!65},
    sharp corners,
    fonttitle=\bfseries, 
    breakable,
    before upper={\parindent15pt\noindent}}{box}

\newcommand*{\concept}[1]{\textbf{#1}}
\newcommand*{\term}[1]{\emph{#1}}
\newcommand{\corpus}[1]{\emph{#1}}

\newcommand{\redp}{\textasciitilde}

\DeclareAcronym{blt}{short = BLT, long = Basic Linguistic Theory}
\DeclareAcronym{cgel}{short = CGEL, long = The Cambridge Grammar of the English Language}
\DeclareAcronym{dm}{short = DM, long = Distributed Morphology}
\DeclareAcronym{tag}{long = Tree-adjoining grammar, short = TAG}
\DeclareAcronym{sfp}{long = sentence-final particle, short = \textsc{sfp}}
\DeclareAcronym{np}{long = noun phrase, short = NP}
\DeclareAcronym{vp}{long = verb phrase, short = VP}
\DeclareAcronym{pp}{long = preposition phrase, short = PP}
\DeclareAcronym{cls}{long = classifier, short = CLS}
\DeclareAcronym{dist}{long = distal, short = DIST}
\DeclareAcronym{prox}{long = proximate, short = PROX}
\DeclareAcronym{dem}{long = demonstrative, short = DEM}
\DeclareAcronym{classify}{long = classifier, short = \textsc{cl}}
\DeclareAcronym{dur}{long = durative, short = DUR}
\DeclareAcronym{neg}{long = negative, short = \textsc{neg}}
\DeclareAcronym{cc}{long = copular complement, short = CC}
\DeclareAcronym{cs}{long = copular subject, short = CS}
\DeclareAcronym{tame}{long = {tense, aspect, mood, evidentiality}, short = TAME}
\DeclareAcronym{past}{long = past, short = PST}
\DeclareAcronym{nonpast}{long = non-past, short = NPST}
\DeclareAcronym{present}{long = present, short = PRES}
\DeclareAcronym{progressive}{long = progressive, short = \textsc{poss}}
\DeclareAcronym{perfect}{long = perfect, short = \textsc{perf}}
\DeclareAcronym{passive}{long = passive, short = \textsc{pass}}
\DeclareAcronym{copula}{long = copula, short = COP}
\DeclareAcronym{possessive}{long = possessive, short = \textsc{poss}}

\newcommand{\asis}[1]{\textsc{#1}}
\newcommand{\oneof}[1]{{#1}}
\newcommand*{\homo}[2]{#1$_{\text{#2}}$}

\newcommand{\cgel}{\href{../English/cambridge.pdf}{my notes about CGEL}}
\newcommand{\latin}{\href{../Latin/latin-notes.pdf}{my notes about Latin}}
\newcommand{\alignment}{\href{../alignment/alignment.pdf}{my notes about alignment}}
\newcommand{\exerciseone}{\href{../Exercise/2021-3.pdf}{this exercise}}
\newcommand{\method}{\href{../methodology/glossing.pdf}{this note about my understanding of descriptive grammars}}

\newcommand{\ala}{à la}
\newcommand{\translate}[1]{`#1'}
\newcommand{\vP}{\textit{v}P}

% Make subsubsection labeled
\setcounter{secnumdepth}{4}
\setcounter{tocdepth}{4}
% reset example counter every chapter (but do not include the chapter number to the label)
\counterwithin{exx}{chapter} 

% Reference formats
\renewcommand{\bibname}{References}
\setcitestyle{aysep={}} 

\title{Mandarin Chinese notes}
\author{Jinyuan Wu}

\begin{document}

\maketitle

\automath

\chapter{Introduction}

\section{The language and the speakers}

Mandarin Chinese is a predominant language in the world,
belonging to the Sinitic branch 

\section{Previous studies and theoretical orientation of this work}

\subsection{Structuralist grammars}

There are already sufficient works concerning the grammar of Mandarin Chinese.
The tradition used in colleges (and occasionally high schools) 
is largely structuralist \ala{} Bloomfield:
A clause is divided into a topic and a comment,
and the comment is divided into a subject and a predicate,
and the predicate is divided into a predicator and an object, etc.
Examples of works in this tradition usually have names like 现代汉语 \translate{Modern Chinese}
\citep{xianhan2004}.
This tradition is still seen in many contemporary grammars,
like \citet{cgel}.
This tradition is largely coherent with the generative tradition,
and indeed there are books which are generative in essence 
but organized in the traditional and structuralist framework \citep{deng2010formal}.

\subsection{Teaching materials}

Teaching materials of Mandarin Chinese are also largely influenced by the structuralist tradition.
TODO

\subsection{Mandarin in the functional-typological tradition}

Mandarin also gains much attention in the functional-typological tradition.
\citet{li1989mandarin} is a ``functional'' reference grammar of Mandarin, TODO
The reason for Mandarin's popularity seems to be the fact that 
it breaks many previous typological generalizations about isolating languages and constituent orders
\citep[\citechap{8}]{paul2014new}.

\subsection{Theoretical commitment of this work}

The differences between the generative tradition, 
the traditional structuralist school 
and a large part of the so-called functionalist 
(or ``Basic Linguistic Theory'' \citep{dixon2009basic}) 
are mainly notational from a practical perspective.

Since modern generative syntax contains lots of hidden functional heads,
the notion of, say, a DP which is the specifier of T,
is to be replaced by an \acs{np} filling a subject position 
in a surface-oriented structuralist constituency analysis. 
Both ``\acs{np}'' and ``subject'' should be labeled on a sub-syntactic tree
in the structuralist tradition,
while in generative syntax,
the label ``subject'' is a secondary concept:
it's an abbreviation of SpecTP (or in an even more fine-grained way, SpecSubjP or SpecNomP).

The Basic Linguistic Theory approach, 
or the functional-typological grammar writing approach,
is based on \emph{dependency relations} on the other hand.
It's also possible to formulate generative grammar in terms of dependency relations,
so it can be expected that the \acs{blt} approach is still largely equivalent 
with the grammatical complexity of generative grammar.
\acs{blt} only recognizes two types of constituents:
\acs{np}s and clauses,
which are essentially \term{domains} in generative syntax:
The former is the DP domain while the latter is the \vP{}-TP-CP domain.
The complicated binary constituency tree is replaced by a flat tree 
with lots of dependency relations labeled inside,
containing the same amount of information.
(We may also say the \acl{blt}'s standard of constituency 
is ``being a relatively independent construction''. 
Then the equivalence between this version of mild constructivism and minimalism 
is clear \citep{construction-minimalism}.)
The \acl{blt} approach still (although quite implicitly) admits 
that there is a rank of ``closeness'' or ``height'' among dependency relations: 
the A argument seems to be somehow higher than the O argument, etc.,
and this information is conveyed by the fine-grained constituency relations 
in generativism.

A further difference between the \acl{blt} and generativism (as well as the structuralist tradition)
is the former is claimed to be semantic-based.
But first, of course it's possible to have a meaning-first version of generative syntax,
and second,
there seems to be a ``gluing'' layer between pure semantics and 
the phonetic realization 
and can't be equated with either of the two: 
the semantics of complement-taking verbs 
may be coded as a complement clause construction, 
a relative clause construction 
(compare \corpus{I see a man running} and \corpus{I see a running man}),
and superficially similar utterances may have different ``structures''.
This is recognized by Dixon, 
who distinguishes the ``prototypical'' coding strategy of a semantic concept 
and other strategies.
So there is also no substantial disagreement here.

This note is an attempt to reconcile the several approaches in previous researches.
I will use less structuralist constituency trees compared with structuralist grammars.
That's to say, for example, 
the ``serial verb construction'' is not divided into two verb phrases 
following rigidly the structuralist tenets,
but is analyzed with a flat-tree structure instead,
following \acs{blt}.
But I will also do lots of in-depth morphosyntactic tests,
instead of just staring at the surface realization,
as many typologists may do.

\chapter{Grammatical overview}

\section{Prosody}

One distinct feature of Mandarin is its morphosyntax relies strongly on \emph{prosody} \citep{feng2000}. 

The phonology, strikingly, doesn't have much influence on Chinese morphosyntax,
and it will be largely skipped in this note.

\section{Parts of speech}

\section{The noun phrase}

No morphological case, number, and gender categories are attested in Mandarin.
There is a word class system or in other words classifier system, however.
In most cases when a numeral appears in an \ac{np},
a classifier follows immediately after the numeral.
Attributives -- both adjectives and relative clauses -- 
follow the classifier. % TODO: 红色的三个,这样的说法说得通吗?
The demonstrative, if any, appears before the numeral,
and even when there is no numeral,
there is frequently also a classifier.

The template of \ac{np}s, therefore, belongs to the 
Dem-Num-A-N type,
with the classifier residing between Num and A. 

\section{Clause structure}

\subsection{Alignment}

Mandarin is an typical accusative language.
Mandarin clauses have a rather rigid constituent order:
It's usually classified as having a SVO clausal constituent order,
and the subject and the object(s) can be told from the positions in the clause 
(\ref{ex:get-sick}, \ref{ex:svo-example}).
Certain ``SOV'' orders can be obtained by invoking the disposal construction
(\prettyref{sec:disposal-construction}), as in \eqref{ex:ba-example}.

\begin{exe}
    \ex \gll 我 生病 了 \\
    1 get.sick \acs{sfp} \\
    \glt \translate{I got sick.}
    \label{ex:get-sick}

    \ex \gll [我]_{\text{subject}} 今天 去 看 [电影]_{\text{object}} 了 \\
    1 today to watch movie \acs{sfp} \\
    \glt \translate{I went to watch a movie today.} 
    \label{ex:svo-example}

    \ex \gll [我]_{\text{subject}} 今天 把 [ 一 个 碗 ]_{\text{object}} 摔 碎 了 \\
    1 today BA {} one \acs{classify} bowl {} break crack \acs{sfp} \\
    \glt \translate{I broke one bowl today.}
    \label{ex:ba-example}
\end{exe}

The normal tests of syntactic accusative alignment can be run on Mandarin
(\ref{ex:inter-sentence}).

\begin{exe}
    \ex \gll 陈 经理 昨天 没有 和 他的 客户 聊 过 。 他 生病 了 。 \\
    {Chen (surname)} manager yesterday \acs{neg} with 3sg-\acs{possessive} client talk \acs{sfp}
    {} 3sg get.sick \acs{sfp} \\
    \glt \translate{Manager Chen didn't talk with his client yesterday. He (Chen, not his client) got sick.}
    \label{ex:inter-sentence}
\end{exe}

\subsection{\acs{tame} categories}

Mandarin lacks the category of tense -- 
all tense information is expressed by time adverbs.
Modality is marked similarly be adverbs or complement clause constructions.
Yet there is a system marking the aspect (\prettyref{sec:aspectual}). 
\eqref{ex:quguo-qule} is an example.

\begin{exe}
    \ex \begin{xlist}
        \ex \gll 我 去 过 上海 了 \\
        1 go \asis{guo} Shanghai \acs{sfp} \\
        \glt \translate{I have been in Shanghai.}
        \ex \gll 我 去 了 上海 了 \\
        1 go \asis{le} Shanghai \acs{sfp} \\
        \glt \translate{I have gone to Shanghai.}
    \end{xlist}
    \label{ex:quguo-qule}
\end{exe}

\subsection{Negation}

Like the case in standard English, 
there is no negative concord in Mandarin Chinese.
There is, however, no uniform negation operator like the English \emph{not}. 
Several negation operators and strategies are used frequently (\prettyref{sec:negation}).
Verbs can be negated by 不 while nouns generally cannot, 
and this is a criterion to tell verbs from nouns. 
There is another negation operator 没, 
which has subtle differences in its meaning and syntactic properties compared with 不
(\ref{ex:chiqincai}, \ref{ex:buchi-meichi}).
On the other hand, the negative potential complement construction,
i.e. the V不了 construction,
isn't obtained by inserting a negator in the clause \eqref{ex:zuobuliao-example}.

\begin{exe}
    \ex \begin{xlist}
        \ex \gll 我 不 喜欢 吃 芹菜 \\
        1 \acs{neg} like eat celery \\
        \glt \translate{I don't like eating celery.} \\
        \ex * 我 没 喜欢 吃 芹菜
    \end{xlist}
    \label{ex:chiqincai}
\end{exe}

\begin{exe}
    \ex \begin{xlist}
        \ex \gll 我 不 吃 早饭 \\
        1 \acs{neg} eat breakfast \\
        \glt \translate{I don't eat breakfast. (I usually don't, I don't want any today, etc.)}
        \ex \gll 我 没 吃 早饭 \\
        1 \acs{neg} eat breakfast \\
        \glt \translate{I didn't eat breakfast. (I may usually do, but somehow I didn't today.)}
    \end{xlist}
    \label{ex:buchi-meichi}
\end{exe}

\begin{exe}
    \ex \begin{xlist}
        \ex \gll 我 做 [ 不 了 ]_{\text{potential complement, negative}} 这 件 事 \\
        1 do {} \acs{neg} finish {} this \acs{classify} affair \\
        \glt \translate{I'm not able to do this.}
        \ex[*]{\gll 我 \oneof{没有/并非/不} 做 [ 得 了 ]_{\text{potential complement, positive}} 这 件 事 \\
        1 \acs{neg} do {} \asis{de} finish {} this \acs{classify} affair \\}
    \end{xlist}    
    \label{ex:zuobuliao-example}
\end{exe}

\subsection{The topic-comment structure}



\section{Clause combining}

Mandarin Chinese has usual clause linker devices (\prettyref{chap:clause-linking}),
as well as complement clauses (\prettyref{sec:complement-clause})
and relative clauses (TODO: ref). TODO: what else?

It's often said Mandarin is a serializing (i.e. with serial verb constructions) language.
A closer look, however, reveals this is not the case:
These constructions are either adverbial clause constructions 
or complement clause constructions, 
or maybe certain kind of light verb constructions (\prettyref{sec:no-serial-verb}).
The internal heterogeneity renders the term \term{serial verb construction} useless.

\chapter{Prosody and the writing system}

\chapter{Parts of speech}

\section{Introduction}

Lexical words in Chinese can be roughly divided into nominal ones and verbal ones,
or in the Chinese terms, 体词 and 谓词.
The prototypical role of nominal words 
is to fill predicate slots (or to be more precise, to head a phrase that fills an argument slot).
Nominal words rarely appear in the verbal complex,
though for stylistic purposes, they sometimes do.
Verbal words prototypically appear in the verbal complex,
but many of them -- and clauses without any morphological marking -- 
can regularly appear in argument slots \citep[\citesec{3.5}]{zhudexigrammar}.

The fact that verbal categories can fill argument slots or in colloquial words ``be used as nouns''
urges some to put the verbal categories under the nominal categories,
so thus there is only one mega lexical category in Chinese:
the nominal category or the Noun.
The analysis adopted here does not aim to organize lexical categories 
in a binary branching classification tree,
so the ordinary nominal-verbal distinction is maintained:
verbs being able to fill argument slots is not typologically rare, actually,
and this shared feature itself does not bring nouns and verbs close enough 
for them to be merged together.

Whether Chinese has a separate adjective category 
has been debated for decades.
Based on a line of reasoning similar to the above verb-as-noun analysis,
some linguists argue that the so-called adjectives should be put under the verb category,
since they can fill the predicator slot without any morphological marking \citep{li1989mandarin}.
Since verbs and most alleged adjectives show different morphological behaviors in reduplication, % TODO: ref
the verb-adjective distinction is kept,
and the two are placed under the verbal category.

There still exist a (much smaller) number of alleged adjectives that shows 
different morphosyntactic properties with the adjectives in the verbal category 
\citep[\citechap{5}]{paul2014new}.
They can be marginally used as heads of \ac{np}s,
while they do not have reduplication variants.
These ``adjectives'' are thus placed under the nominal category.
Thus we have two types of adjectives.
In \citet{zhudexigrammar}, 
nominal adjectives are called 区别词 \translate{distinction word},
while verbal adjectives are called 形容词 \translate{adjective}.

There are more nominal categories than the ordinary noun category and the nominal adjective category.
Numerals, for examples, are in another nominal category.
Chinese has a rich classifier system,
and most classifiers still have strong nominal properties
and thus they constitute yet another nominal category.
\citet{zhudexigrammar} calls them 量词 \translate{measure word},
because many classifiers have the meaning of ``unit''.
There is also a locative particle class, including 里 in 在房子里,
which is sometimes said to be the postposition class
because they sometimes have adposition-like properties (TODO: ref: topicalization, and what else?).

\begin{infobox}{The notion of lexical and function words}{lexical-function}
    The lexical-functional distinction is sometimes subtle.
    \citep[\citesec{3.6}]{zhudexigrammar} classifies 
    certain categories like locative particles % TODO: 方位词的正确翻译???
    into the nominal class and hence the lexical one,
    while the locative particle class can definitely be enumerated \citep[\citesec{4.4}]{zhudexigrammar}.
    On the other hand, 
    the author claims that lexical classes are always open 
    and function classes are always closed \citet[\citesec{3.4}]{zhudexigrammar}.
    A conflict thus occurs.

    The problem here is we have a gradient hierarchy 
    from the prototypical lexical classes 
    to the prototypical function classes.
    The most lexical class is open to new members, 
    not a part of the grammar,
    and its members are able to be lexical heads%
    \footnote{
        Not functional heads \ala{} modern generative syntax:
        realizations of functional heads are function words or suffixes, 
        not lexical items.
        We may also say a lexical head has a ``real part-of-speech label'' 
        like ``noun'' or ``verb'',
        which, in the language of Distributed Morphology, means that 
        such a lexical head appears at the core of a functional project as the root,
        and at somewhere a categorizer has to be merged into the derivation.
        A functional head in Distributed Morphology, on the other hand, 
        doesn't bring any real part-of-speech label to its realization.
        We still classify functional items in the grammar into classes,
        but these classes are somehow less ``real''.
    }   
    of, say, an \acs{np} or a verbal complex.
    A less open class is not so open to new members 
    (just like Japanese verbs and adjectives),
    but is still not a part of the grammar 
    and its members are able to be lexical heads. 
    A even more closed class is not open to new members,
    and is a part of the grammar,
    but its members are still able to be lexical heads.
    Pronouns are in this type.
    A prototypical function class, then, 
    is not open to new members, hardwired in the grammar, 
    and its members are never lexical heads.
    Derivational suffixes are in this type.

    It's of course not easy to tell a newly discovered part of speech 
    (or \term{form class}, which may be a word class or an affix class)
    What's the status of an orientation preverb, 
    which may be found in Japhug \citep{jacques2021grammar}?
    It's a part of the grammar,
    but does it carry a real part-of-speech label (like ``directional adverb'')?
    And speaking of adverbs, what's the status of the English \corpus{allegedly}?
    An adverb filling a peripheral argument position,
    or an evidentiality marker?
    We really need to know a lot about language to fix the position of a form class.
    A common practice is just to shun the details and just say whether a class is lexical or functional,
    drawing a hard line between the two.
    So \citet{zhudexigrammar} mainly uses the criterion of whether there is a real part-of-speech label,
    and then directive particles are classified into the nominal class 
    and they are in turn considered lexical.
    But he mistakenly confuses the notion of lexical classes with the notion of open classes,
    and then we get the asserts in \citet[\citesec{3.4}]{zhudexigrammar}.
\end{infobox}

\section{Prepositions}\label{sec:preposition-pos}

Though all Mandarin prepositions have verb origins 
and therefore may be classified as a subclass of verbs by some,
it's necessary to distinguish a separate preposition class.
Criteria of prepositions include TODO: ref

\begin{infobox}{The term \term{coverb}}
    In 
\end{infobox}

\chapter{Nominal derivation}

\chapter{The structure of noun phrase}

\chapter{The verbal complex}

\section{Introduction}

Mandarin is generally regarded as a prototypical analytic language,
without traditionally acknowledged verb inflections.
Indeed it will be weird to posit something like a paradigm in Mandarin,
but it doesn't mean there is no such thing as verbal affixation that are active
in the morphosyntax 
(instead of not fully productive and arguably historical derivations).
Some items involved here however may have partial mobility.
Consider \eqref{ex:movable-suffix}:
In the first sentence, 
了 is an aspectual suffix (\prettyref{sec:aspectual}),
while 走 is a verb which never appear without an argument in uncontroversial phrasal grammar.
So we conclude 了 and 走 are suffixes,
and by structural comparison, 
we conclude 过来 in \eqref{ex:sanpinqishuiugolai-1} 
is also a suffix, with the same status as 走.
But there comes \eqref{ex:sanpinqishuiugolai-2},
in which 过来 moves to the end of the sentence.

\begin{exe}
    \ex \begin{xlist}
        \ex \gll 他 带 走 了 他的 文件  \\ 
        3sg carry go.away \acs{perfect} 3sg-\acs{possessive} file \\
        \glt \translate{He carried his files away.}
        \ex \gll 他 带 [过来] 了 三 瓶 汽水 \\
        3sg carry come \acs{perfect} three bottle.\acs{classify} soda \\
        \glt \translate{He carried here three bottles of soda.} 
        \label{ex:sanpinqishuiugolai-1}
        \ex 他带了三瓶汽水[过来]
        \label{ex:sanpinqishuiugolai-2}
    \end{xlist}
    \label{ex:movable-suffix}
\end{exe}

To avoid the useless quarrelling about what is a word and whether a grammar point is morphology
(which isn't that important in non-lexicalist generative theories, anyway),
I use the term \term{verbal complex} to cover 
the main verb and the  ``suffixes'' in \eqref{ex:movable-suffix}.
There are roughly three systems in the verbal complex.
The first is the uncontroversial derivation system,
like 化 \translate{-ize}.
The second is the verbal complement system,
which includes three subsystems:
the resultative complements, the directional complements, 
and the potential complements (\prettyref{sec:verbal-complement}).
The third is the aspectual system (\prettyref{sec:aspectual}).

\begin{infobox}{On the notion of \term{complements}}{complement-name}
    The Chinese term 补语 corresponding to my \term{verbal complement}
    is frequently translated into the English term \term{complement}.
    This creates some confusion,
    because the term \term{complement} can also denote 
    clausal dependents that are arguments of the main verb, as in \citet{cgel}.
    The term \term{non-argument complement} may be used to avoid this confusion.
    There are, however, further confusions:
    Should we regard a clausal dependent that records the quantity or amount of an action 
    as a non-argument complement?
    This construction can also be seen in Latin, 
    like the Latin accusative expression of time \citep[\citesec{423}]{greenough2013allen}.
    Thus, I use the term \term{verbal complement} to refer to 
    things like 完 as in 做完了.
\end{infobox}

\eqref{ex:hua-wan-le-1} is an example in which 
all the three systems appear.
In real world speeches, such combinations have relatively lower distributions,
possibly because of the prosodic constraint 
that verb shouldn't be too heavy unless it appears at the end of a clause
(TDOO: ref).

\begin{exe}
    \ex \dots 并且企业 [数字 [化]_{\text{derivation}} [完]_{\text{complement}} [了]_{\text{aspectual}}]_{\text{V}} 之后还不一定赚钱 \dots
    \label{ex:hua-wan-le-1}
\end{exe}

Besides the systems shown in \eqref{ex:hua-wan-le-1},
the separation of a verb further complicates the behavior of the verbal complex 
(\prettyref{sec:separable-verbs}).

You may note the so-called serial verb constructions aren't mentioned here.
\citet{paul2008serial} and \citet[\citesec{9.4}]{deng2010formal} 
summarizes several constructions that are
frequently referred to as serial verb constructions,
and points out after deeper investigation,
they can all be described in terms of the usual complement clause constructions,
purpose clause constructions, etc. 
that are well attested cross-linguistically (\prettyref{sec:no-serial-verb}).

\section{Verbal derivations}

\section{Verbal complements}\label{sec:verbal-complement}

\section{The aspectual system}\label{sec:aspectual}

\section{Separable verbs}\label{sec:separable-verbs}

It's sometimes possible to split a verb and inject some clausal dependents into it.
The interaction between this separation operation and the structure of the verbal complex is of some interest.

\chapter{Verb and arguments}

\chapter{Valency changing}

There are two ways of valency changing in Mandarin.
The first is via a coverb construction, 
as in the disposal constructions (\prettyref{sec:disposal-construction}),
TODO 
The second is \emph{doing nothing} to the verb 
and relying on the unusual semantic roles of clausal complements 
to inform the listener about the valency changing,
as in TODO: ref.
Since there is no morphological marking,
constructions of this type are often recognized as topic-comment structures,
in which the ``topic'' -- which is the subject under closer investigation -- 
is said to be freely occupied by any semantic (and not necessarily syntactic) argument in the clause,
though this claim can be falsified by detailed syntactic tests (\prettyref{sec:topic-subject}).

\section{The disposal constructions}\label{sec:disposal-construction}

\section{The passive constructions}\label{sec:affected-construction}

\begin{exe}
    \ex 我被他打了一拳
\end{exe}

\section{The causative}

\section{The affected construction}



\section{Instrumental object}

\begin{exe}
    \ex 我们今天准备吃食堂
\end{exe}

\chapter{Simple clauses}

\section{Introduction}

A sentence can be divided into several clauses linked by clause linking constructions 
(\prettyref{chap:clause-linking}).
A clause can be divided into
one or more topics (if any) and a comment,
the latter being the nucleus clause,
and possibly \acl{sfp}s.
The comment -- the nucleus clause -- may further be divided into a subject (if any),
a series of adverbials, 
the verbal complex, and post-verbal constituents,
the most important types including object(s), 
the second part of a separable verb,
certain directional complements,
and purpose clauses.
This chapter is denoted to everything on the clause level,
postponing details in subordination and clause linking to the next several chapters.

\begin{infobox}{The term \term{clause}}{clause-def}
    Some people, like \citet[\citepage{140}]{deng2010formal}
    as well as \citet{dixon2009basic},
    use the term \term{clause} for subject-predict constructions 
    that don't receive complete marking of speech forces.
    (In generative terms, \term{clause} is for lower level CPs or even TPs.)
    So in this way, \acl{sfp}s shouldn't be discussed in this chapter because 
    they are of course dependents in the sentence level.
    They may be discussed together with other sentence-level constructions like \prettyref{chap:clause-linking}.
    But this notion of clause certainly goes against the tradition in descriptive grammars.
    So the approach of this note is to acknowledge everything larger than TP as a clause,
    which may or may not be a sentence,
    and discuss its structure in this chapter,
    while ``adjunctions'' -- or in other words, optional dependents -- 
    are discussed in, say, \prettyref{chap:clause-linking},
    for the sake of convenience.
    The narrative order of this note is not the ideal ``small unit -- large unit'' scheme,
    but the ``simple large unit -- complicated large unit'' scheme.
    Needless to say,
    when it comes to clause combining, 
    the problem of what the clause really is -- with or without \ac{sfp}s, for example --
    is still relevant,
    but it is not answered by saying ``the construction takes a clause, not a sentence''.
\end{infobox}

\section{Types of nucleus clauses}

\begin{infobox}{About the subject-predicate binary division}{subject-predicate}
    \citet{dixon2009basic} argues against the definition of \term{predicate} 
    as the main verb (or adjective) plus somehow ``internal'' arguments.
    He uses the term \term{predicate} to refer to the verbal complex instead.
    However, since I will need to compare the topic-comment construction 
    with the inner structure of the nucleus clause,
    the term \term{predicate} will still be used in the way \citet{dixon2009basic} dislikes,
    because it's the counterpart of the comment in the topic-comment construction.
\end{infobox}

\subsection{There is no serial-verb construction or complex predicate}\label{sec:no-serial-verb}

\section{Negation}\label{sec:negation}

\section{Sentence final particles}

\section{The topic-comment structure}

I follow \citet{sih2000topic}'s approach and define a topic as an unmarked \acs{np} 
that has certain relations with a position in the clause after it
and is indeed the topic in the information structure
(i.e. some (probably already known) object to which new information is added).
Constructions like 连\dots都\dots are not discussed in this section -- 
they are to be found in TODO: ref.

\subsection{Topicalization of possessor}

\eqref{ex:tagezigaogaode} and \eqref{ex:tagezigaogaode} are a pair of sentences 
with and without topicalization of the possessor in the subject.

\begin{exe}
    \ex \begin{xlist}
        \ex\label{ex:tagezigaogaode}  
        \gll [他]_{\text{topic}} [ [个子]_{\text{subject}} 高高 的 ]_{\text{comment}} \\
        3sg {} stature tall\redp{}\asis{todo} \asis{de} \\
        \glt \translate{As for him, the stature is tall.}
        \ex\label{ex:tadegezigaogaode} \gll [ 他 的 个子 ]_{\text{subject}} 高高 的 \\
        {} 3sg \acs{possessive} stature {} tall\redp{}\asis{todo} \asis{de} \\
        \glt \translate{His stature is tall.}
    \end{xlist}
\end{exe}

\subsection{Topicalization of preposition objects}\label{sec:topicalization-of-preposition-objects}

\begin{exe}
    \ex\label{ex:zhejianshinibunengjiumafantayigeren} 这件事你不能就麻烦他一个人
    \ex 你不能[为了这件事]_{\text{adverbial:\acs{pp}}} 就麻烦他一个人
\end{exe}
This is also a demonstration of the preposition status of 在 in this sentence (\prettyref{sec:preposition-pos}),
because if it's a verb or an auxiliary verb,
it will be hard to have its object topicalized and have it deleted at the same time,
but deletion of the preposition in topicalization is well-attested cross-linguistically.

\subsection{Rejecting the notion of dangling topics}\label{sec:topic-subject}

Some people, like \citet[\citesec{7.1}]{zhudexigrammar},
equate \term{subject} with \term{topic} in Mandarin grammar.
Some (especially those from the functional-typological tradition) go further 
and assert that ``the notion of the subject (as the position of the most agentive argument) 
isn't grammaticalized in Mandarin Chinese'',
and therefore the topic is just an \acs{np} which the comment is ``about'',
and this base-generated and syntactically unconstrained topic 
is called a ``dangling topic''.
This view is rejected in this note,
because such accounts usually end up in severe overgeneration. 
Here I briefly summarize \citet{sih2000topic}'s argumentation.

\subsubsection{Type 1: Idiomatic phrasal predicate looking like a comment} 

In the first type of ``dangling topic'',
it's impossible for any \acs{np} in the comment to be syntactically related to the topic.
Such cases are however rather unproductive. 
In \eqref{ex:dayuchixiaoyu} and \eqref{ex:nikankanwo},
the orders of the constituents can never be changed.
Nor is it possible to change a word or two in the bracketed ``comments''.
A reasonable assumption is these bracketed ``comments''
are actually idioms, 
which are to be regarded as a single verbal element that can't be further analyzed.
Thus, in \eqref{ex:dayuchixiaoyu} and \eqref{ex:nikankanwo},
the so-called topic is an ordinary subject,
and the so-called comment is a predicate.

\begin{exe}
    \ex\label{ex:dayuchixiaoyu} 他们[大鱼吃小鱼](,厮杀成一片)
    \ex\label{ex:nikankanwo} 他们[你看看我我看看你]
\end{exe}

\subsubsection{Type 2: Quantificational adverbial looking like the inner subject}

The second type of ``dangling topic'' is like \eqref{ex:shui-dou-bu-pa}.
A topic-comment analysis of \eqref{ex:shui-dou-bu-pa} 

\begin{exe}
    \ex\label{ex:shui-dou-bu-pa} \gll 他们 谁 都 不 怕 \\
    3pl who even \acs{neg} fear \\
    \glt \translate{They don't fear anyone.}
\end{exe}

\subsubsection{Type 3: Ellipsis leaving a subject and one predicate}

Some people accept \eqref{ex:nasuofangzixingkuimeixiaxue}.
Here the \acs{np} 那所房子 definitely doesn't come from the words following it,
and is therefore recognized as a topic by some (TODO: ref). 
Note, however, that 幸亏 serves as a clause linker outside \eqref{ex:nasuofangzixingkuimeixiaxue}:
\eqref{ex:xingkui-buran-ex} is a demonstration of the 幸亏……不然…… linking construction,
and we also have its topicalized version \eqref{ex:xingkui-buran-fronted}. (TODO: whether this is parenthesis)
We also know in a clause linking construction,
often one clause can be omitted in the utterance because it's content can be easily inferred (TODO: ref).
So now the origin of \eqref{ex:nasuofangzixingkuimeixiaxue} is clear:
We can get it by omitting the second clause in the comment part of \eqref{ex:xingkui-buran-fronted}.
Indeed, if we replace 幸亏 by anything that is adverbial but not a clause linker,
the resulting sentence -- which now contains a real dangling topic -- is not grammatical.

\begin{exe}
    \ex \label{ex:nasuofangzixingkuimeixiaxue} \gll \% 那 所 房子 幸亏 没 下雪 \\
    {} that \acs{classify} house fortunate \acs{neg} snow \\
    \glt \translate{For that house, fortunately it didn't snow (or otherwise something bad would happen).}

    \ex\label{ex:xingkui-buran-ex} \gll [幸亏] 去年 没 下雪 , [不然] 那 所 房子 早就 塌 了 \\
    fortunate last.year \acs{neg} snow {} otherwise that \acs{classify} house already collapse \acs{sfp} \\
    \glt \translate{Fortunately it didn't snow last year, or otherwise that house has already collapsed.}

    \ex\label{ex:xingkui-buran-fronted} 
    \gll [ 那 所 房子 ]_{\text{topic}} [ 幸亏 去年 没 下雪 , 不然 早就 塌 了 ]_{\text{comment}} \\
    {} that \acs{classify} house {} {} fortunate last.year \acs{neg} snow {}  otherwise already collapse \acs{sfp} \\
\end{exe}

\subsubsection{Type 4: Extraction from prepositional adverbials}

\eqref{ex:zhejianshinibunengjiumafantayigeren} in \prettyref{sec:topicalization-of-preposition-objects} 
is sometimes regarded as an instance of the dangling topic construction.
However, as is shown in \prettyref{sec:topicalization-of-preposition-objects},
it may just be from topicalization of an \acs{np} in an adverbial,
with the preposition (and/or the locative particle) removed.

\subsubsection{Type 5: Nominal predicate}

\begin{exe}
    \ex 这种青菜一斤三十块钱
\end{exe}

\subsubsection{Type 6: Locational adverbial mistaken for the subject}

\begin{exe}
    \ex \gll \% 物价 纽约 最 贵  \\
    {} price New.York most expensive \\
    \glt \translate{The price in New York is the most expensive.}
\end{exe}

\subsubsection{Tentative conclusion}

The conclusion is all topics in Chinese are closely linked to a position in the comment,
be it a core argument position or a peripheral one.
So the notion of dangling topics is to be rejected in Mandarin grammar,
and we can always recover the ``canonical'' i.e. non-topic-comment clause
from a topic-comment structure.
After this, if the canonical clause can be divided into an \acs{np}
or a complement clause and a verbal constituent following it,
we can uncontroversially say the first is the subject while the second is the predicate. (TODO: predicate def)
So equating the subject with the topic is also wrong.

It's possible to find the semantic role of the subject isn't agentive;
in this case I assert there is a valency changing mechanism here.

\begin{infobox}{What to expect when people talk about the subject or the topic}{subject-topic}
    Unfortunately, despite the syntactic tests presented above,
    there are still many people -- even many native speakers -- 
    promoting the idea that the Mandarin topic has nothing different with the subject.
    Here is a list of TODO: ref
\end{infobox}

\chapter{Relative clause constructions}

Due to 

\chapter{Complement clause constructions}\label{sec:complement-clause}


\begin{infobox}{Non-existence of finite-nonfinite distinction in Mandarin}{finiteness}
    Cross-linguistically, we find a finite-nonfinite distinction in subordination.
    This distinction is arguably absent in Mandarin,
    even after detailed syntactic tests \citep{no-finite}.
\end{infobox}

\chapter{Clause linking}\label{chap:clause-linking}

\bibliographystyle{plainnat}
\bibliography{references/grammars,references/aspects,references/general-typology,references/controversy}

\end{document}