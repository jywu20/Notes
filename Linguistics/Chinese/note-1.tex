\documentclass[UTF8, a4paper, oneside, scheme=plain, 12pt]{ctexrep}

\usepackage{libertinus}
\usepackage{geometry}
\usepackage{float}
\usepackage{titling}
\usepackage{titlesec}
\usepackage{paralist}
\usepackage{footnote}
\usepackage{enumerate}
\usepackage{amsmath, amssymb, amsthm}
\usepackage{gb4e}
\noautomath
\usepackage{bbm}
\usepackage{textcomp}
\usepackage{soul}
\usepackage{graphicx}
\usepackage{siunitx}
\usepackage[table,xcdraw]{xcolor}
\usepackage{tikz}
\usepackage[ruled, vlined, linesnumbered, noend]{algorithm2e}
\usepackage{xr-hyper}
\usepackage[colorlinks, citecolor = purple]{hyperref} % linkcolor=black, anchorcolor=black, citecolor=black, filecolor=black
\usepackage[most]{tcolorbox}
\usepackage{caption}
\usepackage{subcaption}
\usepackage{booktabs}
\usepackage{multirow}
\usepackage[figuresright]{rotating}
\usepackage{acro}
\usepackage[citestyle=authoryear,backend=bibtex,natbib=true,doi=false,isbn=false,url=false]{biblatex}
\addbibresource{references/grammars.bib}
\addbibresource{references/aspects.bib}
\addbibresource{references/general-typology.bib}
\addbibresource{references/controversy.bib}
\usepackage{prettyref}

\geometry{left=3.18cm,right=3.18cm,top=2.54cm,bottom=2.54cm}
\titlespacing{\paragraph}{0pt}{1pt}{10pt}[20pt]
\setlength{\droptitle}{-5em}

\DeclareMathOperator{\timeorder}{\mathcal{T}}
\DeclareMathOperator{\diag}{diag}
\DeclareMathOperator{\legpoly}{P}
\DeclareMathOperator{\primevalue}{P}
\DeclareMathOperator{\sgn}{sgn}
\newcommand*{\ii}{\mathrm{i}}
\newcommand*{\ee}{\mathrm{e}}
\newcommand*{\const}{\mathrm{const}}
\newcommand*{\suchthat}{\quad \text{s.t.} \quad}
\newcommand*{\argmin}{\arg\min}
\newcommand*{\argmax}{\arg\max}
\newcommand*{\normalorder}[1]{: #1 :}
\newcommand*{\pair}[1]{\langle #1 \rangle}
\newcommand*{\fd}[1]{\mathcal{D} #1}

\newcommand*{\citesec}[1]{\S~{#1}}
\newcommand*{\citechap}[1]{chap.~{#1}}
\newcommand*{\citefig}[1]{Fig.~{#1}}
\newcommand*{\citetable}[1]{Table~{#1}}
\newcommand*{\citepage}[1]{p.~{#1}}
\newcommand*{\citepages}[1]{pp.~{#1}}
\newcommand*{\citefootnote}[1]{fn.~{#1}}

\newrefformat{sec}{\citesec{\ref{#1}}}
\newrefformat{fig}{\citefig{\ref{#1}}}
\newrefformat{tbl}{\citetable{\ref{#1}}}
\newrefformat{chap}{\citechap{\ref{#1}}}
\newrefformat{fn}{\citefootnote{\ref{#1}}}
\newrefformat{box}{Box~\ref{#1}}
\newrefformat{ex}{\ref{#1}}

% color boxes

\tcbuselibrary{skins, breakable, theorems}

\newtcbtheorem[number within=chapter]{infobox}{Box}{
    enhanced,
    boxrule=0pt,
    colback=blue!5,
    colframe=blue!5,
    coltitle=blue!50,
    borderline west={4pt}{0pt}{blue!65},
    sharp corners,
    fonttitle=\bfseries, 
    breakable,
    before upper={\parindent15pt\noindent}}{box}
\newtcbtheorem[number within=chapter, use counter from=infobox]{theorybox}{Box}{
    enhanced,
    boxrule=0pt,
    colback=orange!5, 
    colframe=orange!5, 
    coltitle=orange!50,
    borderline west={4pt}{0pt}{orange!65},
    sharp corners,
    fonttitle=\bfseries, 
    breakable,
    before upper={\parindent15pt\noindent}}{box}
\newtcbtheorem[number within=chapter, use counter from=infobox]{learnbox}{Box}{
    enhanced,
    boxrule=0pt,
    colback=green!5,
    colframe=green!5,
    coltitle=green!50,
    borderline west={4pt}{0pt}{green!65},
    sharp corners,
    fonttitle=\bfseries, 
    breakable,
    before upper={\parindent15pt\noindent}}{box}

\newcommand*{\concept}[1]{\textbf{#1}}
\newcommand*{\term}[1]{\emph{#1}}
\newcommand{\form}[1]{\emph{#1}}
\newcommand{\work}[1]{\textit{#1}}

\newcommand{\redp}{\textasciitilde}

\DeclareAcronym{blt}{short = BLT, long = Basic Linguistic Theory}
\DeclareAcronym{cgel}{short = CGEL, long = The Cambridge Grammar of the English Language}
\DeclareAcronym{dm}{short = DM, long = Distributed Morphology}
\DeclareAcronym{tag}{long = Tree-adjoining grammar, short = TAG}
\DeclareAcronym{sfp}{long = sentence-final particle, short = \textsc{sfp}}
\DeclareAcronym{np}{long = noun phrase, short = NP}
\DeclareAcronym{vp}{long = verb phrase, short = VP}
\DeclareAcronym{pp}{long = preposition phrase, short = PP}
\DeclareAcronym{cls}{long = classifier, short = CLS}
\DeclareAcronym{dist}{long = distal, short = DIST}
\DeclareAcronym{prox}{long = proximate, short = PROX}
\DeclareAcronym{dem}{long = demonstrative, short = DEM}
\DeclareAcronym{classify}{long = classifier, short = \textsc{cl}}
\DeclareAcronym{dur}{long = durative, short = DUR}
\DeclareAcronym{neg}{long = negative, short = \textsc{neg}}
\DeclareAcronym{cc}{long = copular complement, short = CC}
\DeclareAcronym{cs}{long = copular subject, short = CS}
\DeclareAcronym{tame}{long = {tense, aspect, mood, evidentiality}, short = TAME}
\DeclareAcronym{past}{long = past, short = PST}
\DeclareAcronym{nonpast}{long = non-past, short = NPST}
\DeclareAcronym{present}{long = present, short = PRES}
\DeclareAcronym{progressive}{long = progressive, short = \textsc{poss}}
\DeclareAcronym{perfect}{long = perfect, short = \textsc{perf}}
\DeclareAcronym{passive}{long = passive, short = \textsc{pass}}
\DeclareAcronym{copula}{long = copula, short = COP}
\DeclareAcronym{possessive}{long = possessive, short = \textsc{poss}}

\newcommand{\asis}[1]{\textsc{#1}}
\newcommand{\oneof}[1]{{#1}}
\newcommand*{\homo}[2]{#1$_{\text{#2}}$}

\newcommand{\cgel}{\href{../English/cambridge.pdf}{my notes about CGEL}}
\newcommand{\latin}{\href{../Latin/latin-notes.pdf}{my notes about Latin}}
\newcommand{\alignment}{\href{../alignment/alignment.pdf}{my notes about alignment}}
\newcommand{\exerciseone}{\href{../Exercise/2021-3.pdf}{this exercise}}
\newcommand{\method}{\href{../methodology/glossing.pdf}{this note about my understanding of descriptive grammars}}

\newcommand{\ala}{à la}
\newcommand{\translate}[1]{`#1'}
\newcommand{\vP}{\textit{v}P}
\newcommand*{\category}[1]{\textsc{#1}}
\newcommand*{\specialunit}[1]{$<$\textit{#1}$>$}

% Make subsubsection labeled
\setcounter{secnumdepth}{4}
\setcounter{tocdepth}{4}
% reset example counter every chapter (but do not include the chapter number to the label)
\counterwithin{exx}{chapter} 

% Reference formats
\renewcommand*{\nameyeardelim}{\space} % No comma between year and name
\DeclareNameAlias{sortname}{family-given} % Putting the family name before the given name
\DeclareNameAlias{default}{family-given} 
\DeclareFieldFormat{labelnumberwidth}{} % No number label like [12] in the reference list
\setlength{\biblabelsep}{0pt} % No space for these labels

\title{Mandarin Chinese notes}
\author{Jinyuan Wu}

\begin{document}

\maketitle

\automath

\tableofcontents



\chapter{Introduction}

\section{The language and the speakers}

\subsection{Speaker population}

Standard Mandarin Chinese,
known as 普通话 \translate{generic speech}, 
sometimes 现代汉语 \translate{contemporary Chinese} 
(when the word is used in the narrow sense) 
in mainland China, 
国语 \translate{national language} in Taiwan, Hongkong and Macau,
and 华语 \translate{Chinese language} in Malaysia and Singapore,
and colloquially simply 汉语 \translate{Han language} or 中文 \translate{Chinese}
(again when the words are used in the narrow sense),%
\footnote{
    Strictly speaking, 中文 refers to the written form of Mandarin Chinese; 
    but the word has been appropriated to cover the whole language; 
    hence we have the seeming strange expression 
    说中文 \translate{speak Mandarin (lit. speak Chinese writing)},
    which does not make sense if taken literally.
}
is a predominant language in the world,
belonging to the Sinitic family.
It's currently spoken by a majority of the population in mainland China,
as well as Taiwan, 
and increasingly in Hong Kong and Macau.
It's the official form of Chinese 
(i.e. the Sinitic language family perceived as a macrolanguage)
in mainland China and Taiwan, 
as well as in several other countries or regions 
with traditional link to Han culture, 
such as Singapore and Chinese communities in Malaysia. 

Standard Mandarin is almost always in the superior status 
in the regions mentioned above,
except in Hongkong and Macau, 
where Cantonese enjoys the superior position,
although under the pressure from mainland China, 
Standard Mandarin may gain more importance.
Standard Mandarin is the language for school, business, politics, 
and academics and technology; 
English has picked up significance importance in the last two fields,
appearing in course materials, research articles and a small portion of theses 
in prestigious institutions of mainland China and Taiwan,
but unlike the case in India,
English has never become the language
for scientific and technological oral communication.
Even in Hongkong and Macau, 
written texts are usually in Mandarin.%
\footnote{
    Written Cantonese does exist but is never as influential as written Mandarin.
}
Indeed, the superior status of Standard Mandarin is directly reflected in its various names: 
it's explicitly called the generic speech intended for 
communication between all groups in China, 
the national language, 
and \emph{the} Chinese language.

\subsection{Status of documentation}

The popularity of Standard Mandarin means it's a fairly well documented languages.
Grammars, dictionaries and teaching materials can be easily found 
(\prettyref{sec:data}).
Despite its popularity, Mandarin has several features that 
are still not fully TODO

A comprehensive study of Mandarin informed by typological information 
is therefore still urgently needed. 

\subsection{Variance and standardization}

The term \term{Mandarin}, in its most generalized sense as the English translation 
of the Chinese term 官话, 
include Southwestern Mandarin 西南官话, 
Northern Mandarin 北方官话,
Jiao-Liao Mandarin 胶辽官话, 
and more.
The Standard Mandarin -- the variety discussed in this note -- 
is based on Beijing Mandarin,
minus some traits regarded too vernacular and less respected. 

Variances can still be attested within Standard Mandarin.
Officially, Standard Mandarin 
``has the Beijing accent as its standard accent, 
the Beijing topolect%
\footnote{
    The original term is 方言, 
    which is often translated as \translate{dialect};
    however, varieties like Cantonese, 
    which are rather distant from Mandarin
    with almost zero intelligibility, 
    are also recognized as 方言, 
    and the word 方言 is even used to refer to 
    non-Sinitic languages in colloquial uses.
    Therefore the term \term{topolect} is used 
    to refer to a variety of Chinese that is attested in a region,
    without any implication on how close it is to Standard Mandarin.
} as its foundation,
and canonical modern (Mandarin) vernacular literatures 
as its grammatical specification''
\citep[\citepage{18}]{deng2010formal};
the definition however involves great uncertainty.
There is no single Beijing accent, both historically and contemporarily: 
accent variance can be attested 
between people with Eight Banner heritage and 
Han Chinese outside the Eight Banner system;
all the accents have stable differences with Standard Mandarin.
Considerable grammatical variances exist among 
canonical Mandarin vernacular literature works;
sentences written by some authors, like Lu Xun,
are no longer acceptable today when taken out of 
the literature context. 

Institutional standardization of Mandarin is mainly about the lexicon.
\work{Contemporary Chinese Dictionary} (现代汉语词典)
is a comprehensive dictionary used for the officially organized 
Putonghua Proficiency Test.
No grammatical standardization happened,
and native speakers' acceptability judgement vary 
on some fine details.
This note intends to illustrate the shared core 
of all varieties within Standard Mandarin.

\section{Theoretical commitment}\label{sec:theory}

{\small

\subsection{Previous approaches}

The tradition of Mandarin grammar documentation
frequently seen in college-level introductory courses (and occasionally in high schools) 
is largely structuralist \ala{} Bloomfield;
such kind of introductory works 
usually have names like 现代汉语 \translate{Modern Chinese}
\citep{xianhan2004};
systematic teaching materials for non-native speakers  
are also largely influenced by the structuralist tradition.
This tradition is still seen in contemporary large-scale grammars 
for languages other than Mandarin,
like \citet{cgel}.
This tradition is largely coherent with the generative tradition;
it's possible to 
organize a generative work in the traditional and structuralist framework \citep{deng2010formal}.

Mandarin also gains much attention in the functional-typological tradition.
\citet{li1989mandarin} is a ``functional'' reference grammar of Mandarin, TODO: more references
The reason for Mandarin's popularity seems to be the fact that 
it breaks many previous typological generalizations about isolating languages and constituent orders
\citep[\citechap{8}]{paul2014new}.


\subsection{Constituency and dependency}

This notes attempts to reconcile 
the traditional and generative structuralist schools
and the functional-typological approach
i.e. ``Basic Linguistic Theory'' (BLT; \citealt{dixon2009basic})
used in modern grammar writing.
Below the capitalized Generativism
refers to a mixture of Minimalism, Distributed Morphology and Cartography
(but with less flavor of Antisymmetry),
which I believe are instructive on grammar description.
This leaves out Lexicalist traditions, 
like the framework used in \citet{deng2010formal};
the reasons are discussed in \prettyref{sec:theory.word}.

What I want to do here, then, 
is to incorporate the new perspectives in generative syntax, 
such as \citet{paul2014new} and \citet{paul2008serial},
into the structuralist tradition
in an accessible and typology-informed way.

Since modern generative syntax contains lots of hidden functional heads,
the notion of, say, a DP which is the specifier of T,
is to be replaced by an \acs{np} filling a subject position 
in a surface-oriented structuralist constituency analysis. 
Both ``\acs{np}'' and ``subject'' should be labeled on a sub-syntactic tree
in the structuralist tradition,
while in generative syntax,
the label ``subject'' is a secondary concept:
it's an abbreviation of SpecTP (or in an even more fine-grained way, SpecSubjP or SpecNomP).
So functional heads can be replaced by syntactic function labels:
after eliminating the syntactic functional heads, 
the label SpecNomP should be replaced by ``subject'',
and the label NomP should be replaced by ``subject-predicate structure''.
On the other hand,
in Distributed Morphology we have roots,
which reside at the center of an extended verbal or nominal projection
(NP-NumP-DP, or vP-TP-CP),
and they are recognized as \term{heads} 
(in this note referred to as \concept{lexical heads})
in traditional structuralism.
This unifies the notation of
\citet{cgel}, \citet{chao1965grammar}, \citet{zhudexigrammar}
in the traditional structuralist perspective
and Generativism.

Often, a grammatical concept is a bundle of smaller concepts; 
in most grammars, the term \term{subject} is not just SpecTP, 
but the bundle of SpecTP \emph{and} the deep Spec\vP;
or, if we want to follow the convention in the last paragraph, 
we may say ``the concept of subject 
is the concept of the innermost syntactic pivot 
plus the concept of the outmost agentive argument''; 
of course, it's possible that in some languages 
one or more bundle concept should be broken into more elementary concept,
as in the split pivot analysis of ergativity.

The \ac{blt} approach \citep{dixon2009basic}, 
or the grammar writing approach accepted in 
modern descriptive linguistics and typology,
is based on \emph{dependency relations} on the other hand.
It's also possible to formulate generative grammar in terms of dependency relations,
so it can be expected that the \acs{blt} approach is still largely equivalent 
with the grammatical complexity of generative grammar.
\acs{blt} only recognizes two major types of constituents:
\acs{np}s and clauses,
which are essentially \term{domains} or \term{fields} in generative syntax:
The former is the DP domain while the latter is the \vP{}-TP-CP domain.
The complicated binary constituency tree is replaced by a flat tree 
with lots of dependency relations labeled inside,
containing the same amount of information.%
\footnote{
    We may also say the \acl{blt}'s standard of constituency 
    is ``being a relatively independent construction''. 
    Then the equivalence between this version of mild constructivism and minimalism 
    is clear \citep{construction-minimalism}.
}
Note that the \acl{blt} approach still (although quite implicitly) admits 
that there is a rank of ``closeness'' or ``height'' among dependency relations: 
the A argument seems to be somehow higher than the O argument, etc.,
and this information is conveyed by the fine-grained constituency relations 
in Generativism.

The focus on dependency relation in \acs{blt} also leads to 
a notational difference on what is a constituent.
A sequence like \form{has been exploring} is not a constituent in the sense of 
the structuralist constituency test in grammars like \citet{cgel} and generativism,
but are still recognized as a unit in \acs{blt} 
because its parts always appear together in the surface-oriented analysis.
This what is \citet[\citepage{109}]{dixon2009basic} defines 
as a \term{verb phrase},
which excludes the object. 
From the constituency-based point, 
such a ``phrase'' is usually a larger functional domain minus a smaller functional domain;
here it's the functional projection below the subject -- the verb phrase in \citet{cgel} --
minus the object.
This definition is meaningful 
because syntax is cyclic 
and the object is a DP and therefore is a phase, 
and above it is another phase, 
and although the introduction of the subject 
(by, say, a Nom head which may contain the nominative case) 
doesn't seal a phase, 
it still changes the properties of the syntactic tree, 
so what happen above the object and below the subject 
-- the \acs{tame} functional heads, the verb root, etc. -- 
are tightly linked to each other
(note that here we are just translating between 
the constituency-based analysis and the dependency-based analysis), 
so from a dependency relation-oriented analysis, 
the sequence \form{has been exploring} does qualify as a phrase; 
in a constituency grammar 
we recognize this sequence has a status 
but we don't call it a phrase or a constituent.
This difference is notational but may be confusing.

A further difference between the \acl{blt} and Generativism (as well as the structuralist tradition)
is the former is claimed to be semantic-based.
But first, of course it's possible to have a meaning-first version of generative syntax,
and second,
there seems to be a ``gluing'' layer between pure semantics and 
the phonetic realization 
and can't be equated with either of the two: 
the semantics of complement-taking verbs 
may be coded as a complement clause construction, 
a relative clause construction 
(compare \form{I see a man running} and \form{I see a running man}),
and superficially similar utterances may have different ``structures''.
This is recognized by Dixon, 
who distinguishes the ``prototypical'' coding strategy of a semantic concept 
and other strategies.
So there is also no substantial disagreement here.

Another concept often seen in generative syntax 
-- as well as many \acs{blt}-like works -- is adjunct.
This notion is replaced in Cartography 
by optional specifiers of functional heads, 
and this is also the viewpoint of this note:
we don't hold a clear-cut argument-adjunct distinction, 
or in the terms of \citet{cgel},
a complement-modifier distinction:
the distinction is to be understood as a bundle of more primitive concepts,
like whether an constituent has a position closer to the lexical head, 
or whether the functional head licensing the constituent is obligatory,%
\footnote{
    Or otherwise the lexical head -- the stem in the functional projections -- 
    fails to spell out \citep[\citechap{9}]{siddiqi2009syntax}.
}
etc. 

\subsection{Wordhood}\label{sec:theory.word}

The controversy about what is a word in Mandarin 
has been around for decades,
which I believe is due to the desire to 
find \emph{the} word as a universal unit,
without thinking of the tenet of Generativism 
that phonetic realization doesn't always 
transparently reflect in the syntax proper,
and that what is universal is likely to be 
prototypes of functional heads and how they are arranged together.

Wordhood may be defined by the following standards:
\begin{itemize}
    \item Phonology, for example prosody. This tells us how an utterance is segmented.
    \item The realizational part of morphology, 
    i.e. sequences that are created ``as a whole'', 
    possibly having undergone some sort of highly local rules 
    that may or may not have phonological motivation, 
    like vowel change in Latin inflection.
    This part usually tells us native speakers' intuition 
    about the smallest unit in natural (i.e. non-linguistic, not in language games, etc.) conversation,
    which strongly influences how new words are created 
    or how borrowing happens, etc. (\prettyref{sec:pos.word.perception}).
    \item Since I reject lexicalism and contend that there is only one generative engine in syntax, 
    no universal standard of wordhood can be defined in syntax proper.
    We may define wordhood as the status of a mini-phrase in the \acs{blt} sense,
    which is usually very small and therefore has limited interaction with the syntactic environment.%
    \footnote{
        Although definitely not absolutely no interaction -- 
        even in English, we have things like \form{pre- and post-processing}, 
        in which a phrasal structure -- coordination -- interacts 
        with the inner structures of two bare nouns, 
        and therefore a clear boundary between word or phrase 
        or between morphology and syntax is in principle impossible.
    }
    It is either a mini constituency tree (in the traditional structuralist sense),
    or the realization of a span of functional heads 
    for example, \acs{blt}'s definition of verb phrase.%
    \footnote{
        In Generativism this kind of words are not constituents.
        But in some versions of Distributed Morphology, 
        in the post-syntactic step, 
        the functional heads realized by inflection are rearranged into a mini-tree, 
        so we can also say inflection creates mini-constituents.
    } 
    We may say the first may be described as derivational,
    the most prototypical case being compounding,
    while the second is inflectional,
    i.e. how functional heads are glued to the root (the lexical head) 
    with high localized, phonological (but yet not phonologically motivated) rules,
    like ``the tense affix is always attached to the end of the verb stem, 
    with this, this and this vowel alternation''.
\end{itemize}

The third definition still has
inherent obliqueness, demonstrated by the following English example. 
In the English form \form{explore-ed} 
we have the verb stem and the tense marker;  
generative syntax tells us that the tense/aspect categories are 
introduced in the same phase as that of the verb stem, 
and therefore according to the \ac{blt}'s definition of 
\term{verb phrase}, 
\form{explore} and \form{-ed} are to be recognized as 
one constituent, 
and indeed in \citet{prins2011web}, 
a similar construction is recognized as the verb phrase.
But in principle, we can also analyze it as 
\form{[-ed [explore \emph{(something)}]_{\text{\vP{} (argument structure)}}]_{\text{TP (\acs{tame} marking)}}},
and in this analysis 
\form{explore} and \form{-ed} are no longer recognized as 
two branches of a constituent. 
Both analyses, of course, are well-justified -- 
by alternating the terminology in the aforementioned way 
the two analyses are equivalent.

But now we are in a delimma. 
If we reject \form{-ed explore}'s status as a constituent,
then it's not a word in the third definition,
which seems ridiculous.
In practice, most descriptive grammarians will choose to recognize 
\form{-ed} and \form{explore-} as forming one unit, 
because \form{explored} is a \emph{phonological} word.
So this means we need to choose \acs{blt}'s definition of constituency.
But then the longer sequence \form{have explored} 
has a parallel structure, 
but is not a phonological unit. 
No one recognizes it as a word, 
although it has the same status of \form{explored},
and if the latter is a word, 
it seems the former should also be recognized as a word.

Thus, we find that in the third definition, 
factors outside morphosyntax 
have to be used to favor one certain way to write the grammar. 
The third definition does work for derivational morphology,
but the morphemes involved are usually very close to each other 
in the syntax tree, 
so in the surface form they will appear together anyway.
Thus this case where the third definition really works
is already covered by the second definition.
But the third definition is not useless: 
if two syntactic units are completely irrelevant to each other, 
i.e. they can never form a word in the third definition, 
it's very unlikely that they can form a word in the second definition.
This means the second definition deserves the title of \term{grammatical word},
and we should expect this, 
since in Distributed Morphology, 
post-syntactic processes that give us the second definition 
work closely with syntax.

The definition of the term \term{word} 
therefore in principle asserts nothing about 
the nature of the language in question.
Whether the definition is handy does tell us 
something about the language. 
We still need to choose the definition wisely. 

Many generative works, like \citet{deng2010formal},
are \emph{lexicalist}:%
\footnote{
    The term \term{lexicalist} sometimes means that  
    it's lexical words like verbs or nouns that carry grammatical structures with them, 
    and there is no separate phrase-structure rules 
    seen in early Chomskyan generative works.
    Since in Distributed Morphology, 
    we can use whether a root can be well spelt out 
    with or without certain functional heads surrounding it 
    to control its subcategorization,
    Minimalism is also lexicalist in this sense:
    to say that the root of an intransitive verb 
    only gets spelt out together with a Trans head 
    is equivalent to say that the verb carries a verb-object construction with it.
} 
they made a clear distinction between what happens in lexicon 
and what happens in phrasal syntax. 
Due to space limit, 
I only want to point out two facts.

First, an empirical observation is that 
as in \form{pre- and post-revolutionary France},
word structure does engage with phrasal syntax (coordination in this case);
a morphological device completely invisible to the rest of the grammar
likely has rather limited productivity and becomes \emph{historical},
its products being synchronically morphemes
with no synchronically inner structures.
There seems to be only one productive engine in language, not two. 

Second, inconsistency can be observed in lexicalist works:
\citet[\citepages{242}]{deng2010formal} says the syntactic structures 
can also be found in morphology,
while on \citepage{262} he says 
the syntactic structures listed in the book 
can be reduced to the X-bar theory;
so the inner structure of words likely
also fits in the X-bar framework.
This can be easily explained by assuming that his classification of syntactic structures 
based on specifier-complement distinction is not accurate enough:
we assume several domains or fields in a DP or CP,
and grammatical relations in a lower domain look like 
traditional head-complement relations (like the object position, 
which in a Cartographic theory should be something like SpecTransP, 
not CompVP),
where the root (here the verb root) is considered as the head,
and grammatical relations in a higher domain look like  
traditional specifier-head relations.
Many phenomena previously attributed to the specifier-complement distinction 
now can be captured by something like the phase theory;%
\footnote{
    This doesn't mean the current version of the phase theory 
    is adequate; the point is what was once attribute to the specifier-complement distinction
    can be derived from more general locality conditions. 
}
units smaller than phrases 
but larger than prototypical grammatical words 
are sometimes recognized as structures like [X^0 Y^0],
but they seem to be better recognized as 
maximal projections in lower positions in the functional hierarchy.
A hard boundary between words and phrases 
is not possible even for English \citep{bruening2018lexicalist},
and phonological (often prosodic) factors are usually needed to draw the line 
(\prettyref{sec:pos.word.phonological}).

This note will not pay too much attention 
to the word-phrase distinction or the morpheme-word distinction
for function items.%
\footnote{
    Here the term \term{grammatical word} 
    means a word defined in the grammar, not phonology;
    a \term{function item} means an item that is not lexical
    and therefore is a part of the grammar. 
    The adjective \term{grammatical} means being related to grammar
    and it has two explanations:
    being defined in grammar and being contained by the grammar; 
    In this note, I use the term \term{grammatical} to refer to the first meaning, 
    and \term{function} to refer to the second meaning.
} 
Distinguishing between words and phrases 
means to classify morphosyntactic units 
according to their possible internal
grammatical relations and categories.
Thus, we distinguish between \acs{np}s and compound nouns, 
for they possess different internal structures 
(\prettyref{sec:pos.noun.compound}).
A function item, on the other hand, is a \emph{label} of a grammatical category
and has no inner structure.
Thus, defining whether something is a function word or a function morpheme
is purely based on \emph{external} factors:
if something appears in a morphosyntactic unit commonly referred to as a phrase, 
than it's a function word, 
and if it appears in a morphosyntactic unit commonly referred to as a grammatical word, 
than it's a bound function morpheme.
But there seems to be no particularly strong correlation 
between whether a function item is grammatically a function word
and whether it is phonologically a word: 
a suffix in a long grammatical word can be a phonological word, 
while a function item, like the Latin \form{-que}, that is an immediate constituent of a phrase 
may be glued to another phonological word. 
In general, the meaning of \term{function word} is not well-defined, 
and an accurate description of a function item 
inevitably involves both of its phonological status and its grammatical status.
That's why I refrain from discussing wordhood of function items in the following sections.

\subsection{Grammatical categories}

Inspired by Cartography, 
we divide the verbal system into the \vP{} 
(core and peripheral arguments, lexical aspect, etc.;
for example, a verb can only appear in a \category{cause-become} structure 
can be explained by stipulating that 
the verb root can only get appropriately spelt out
together with the \category{cause} and \category{become} light verbs), 
TP (\ac{tame} marking), 
and CP (the clause type, speech force, topicalization, etc.; 
in Mandarin it's responsible for the sentence final particles); 
here $v$ means light verb, 
and at the core of \vP, 
we still have projections  
which contains structures that are traditionally analyzed 
as derivational morphology, 
like compounding.

Similarly the nominal system can be divided into 
the nominalizer phrase, 
the argument structure of the noun, 
the numerical modification (NumP), 
and the most external grammatical concepts like 
determiner, in-\acs{np} topicalization, etc. 

Below I will discuss on topics like 
``whether something is more like a concept in the verb phrase 
or in the core clause 
or in the sentence'':
what I'm doing is to decide whether 
the syntactic device in question is 
in the \vP{} or TP or CP.

It can be seen that structural resemblance between the two system 
exists in several domains; 
besides the correspondence between 
the nominal and verbal argument structure, 
TP and NumP, 
and DP and CP, 
a more straightforward structural parallel 
is that coordination constructions in the nominal system 
and the verbal system are often close. 
This might be an evidence for more abstract concepts 
like ``point of view'' (and of course, ``coordination'')
that are ``generic'', 
which, after being filled by the label ``nominal'' or ``verbal'', 
becomes more concrete grammatical categories
\citep{wiltschko2014universal}. 

These alleged universal functional hierarchies
of course are in conflict with the fact that 
grammatical categories actually observed in languages 
show great variances. 
These variances may be explained by substantial alternation 
of the functional hierarchies 
(and thus it might be the case that 
it's not the the whole hierarchy that is fixed cross-linguistically,
but its subsystems, which may be motivated by the aforementioned abstract concepts),
or they may be explained by fusion of universal categories 
(because, say, one value of one category and another value of another category 
can't appear together or the spellout process fails).
Since this note aims to compare Mandarin with other languages 
not only semantically but also structurally, 
efforts will be made to ``decompose'' seemingly unique 
categories of Mandarin into well-attested categories in other languages -- 
and the uniqueness of Mandarin can also be understood more concretely. 
(A successful example: ``Latin \category{tense} 
collectively correspond to the future-non future, 
past-non past, perfect-non perfect, and progressive-non progressive 
systems in English \dots'')



\subsection{The plan of investigation}

The methodology of this note is now briefly summarized here.
I start from recognizing lexical and functional morphemes,
and then investigate the morphosyntactic status 
of phonological words,
including derivational morphology 
and inflection (Mandarin has no prototypical inflection 
but there do exist what I call semi-inflection; 
\prettyref{sec:pos.architecture.word-confusing.constituent})
mentioned above;
thus a reasonable definition of grammatical wordhood can be obtained.
Then it's time for top-level concepts like noun phrases and clauses
(in the rest of this note, the term \term{phrase} is reserved 
for constructions larger than the grammatical word;
constituency may be defined in the \acs{blt} sense), 
and also their internal modification and complementation schemes. 
Constructions above the layer of grammatical words
are divided into different layers, as outlined in the last section. 
This note is intended as a bottom-up one, 
but in practice, the anatomy of noun phrases and clauses 
is always the starting point of the investigation. 

Parts of speech divisions are then made according to the 
syntactically environment -- thus information regarding top-level concepts is needed -- and not semantics: 
what appears in the nominal system is considered a noun, 
and vice versa. 
As is said above, 
I assume ``nominal properties'' or ``verbal properties'' 
to be universal in all languages; 
detailed parts of speech classes, however, 
always contain some language-specific information,
since they are defined by the interplay 
between the root and the surrounding functional system. 

Another problem is the division of labor 
between the grammar and the dictionary, 
or in other words, the lexical-functional distinction,
which is needed when investigating into morphemes. 
There are three interlinked parameters involved in the lexical-functional distinction:
openness to new members, 
integration with grammar (whether it's already coded in the grammar),
and the necessity of a part-of-speech label (see below).
\begin{itemize}
    \item The most lexical class is open to new members, 
    not a part of the grammar,
    and its members are able to be lexical heads of phrases
    (and thus has a ``real part-of-speech label'' 
    like ``noun'' or ``verb'') 
    of, say, an \acs{np} or a verbal complex.
    \item A less open class is not so open to new members 
    (just like Japanese verbs and adjectives),
    but is still not a part of the grammar 
    and its members are able to be lexical heads. 
    \item A even more closed class is not open to new members,
    and is a part of the grammar,
    but its members are still able to be lexical heads.
    Pronouns are in this type.
    \item A prototypical function class, then, 
    is not open to new members, hardwired in the grammar, 
    and its members are never lexical heads.
    Derivational suffixes are in this type.
    This last type of forms 
    bring no real part-of-speech label to its realization.
    We still classify these function items in the grammar into classes,
    but these classes are somehow less ``real''.
    When doing part of speech classification,
    deciding these parameters for each class 
    is also important.
\end{itemize}

The definition of morpheme also needs clarification.
If we study synchronic morphemes, 
i.e. minimal forms carrying some meanings, 
then entries like \form{till death do us part} 
are to be recognized as morphemes; 
it seems then the term \term{morpheme} is better defined 
as a historical linguistic term; 
when doing synchronic studies, 
it's better to stick to concepts like root and affixes, 
which are put together in a shared manner with syntax.



It's also important to recognize fossilized constructions
from syntactically regularly formed ones. 
Fossilization may be syntactic or semantic; 
in the first case, 
the productivity of the construction involved is reduced, 
although the meaning may still be compositionally analyzable; 
in the second case a form gets a conventionalized meaning
but not necessarily gets syntactically fossilized.
It may be the case that
syntactically fossilized constructions 
are documented in List B (containing information about compatibility between parts in spellout) 
and semantically fossilized constructions are documented in List C 
(containing information about accepted meanings).
As fossilization goes on, 
syntactically fossilized constructions 
has gradually reduced structures 
(from a verb phrase to a grammatical word, for example), 
and the original compositional interpretation of a semantically fossilized construction
gradually fades away. 
The final stage of fossilization is collapsing into a single root
(documented in List A).
It's however hard to tell the stage of fossilization of a construction; 
the form \form{till death do us part} 
at first was a formula used in wedding 
and may be seen as documented in List C, 
but contemporary English grammar 
is no longer able to assemble other elements from List A into it;
a somehow radical analysis would then be placing it in List A. 

}

\section{Data}\label{sec:data}

\subsection{Intuition}

One problem is scholars working on Mandarin grammar 
have their own judgement on what is acceptable and what is not, 
which is sometimes not shared by the majority of Mandarin speakers.
Some works, like \citet{huang2013}, 
are heavily criticized for 
representing not empirical observations of Mandarin
but distorted and man-made examples 
which are used to ``support'' a pre-accepted theory.
My personal opinion is this reflects individual varieties, 
which is probably influenced by non-Mandarin varieties of Chinese,
instead of academic misconduct.
We can find sentences that go against the intuition of most Mandarin speakers 
in the main text -- as opposed to examples -- of their works as well, 
which implies these authors are probably 
sincere about the acceptability of their ``weird'' examples.
Speaking of Huang, (\prettyref{ex:weird-1}) appears in \citet{huang2007},
which doesn't sound acceptable for me but seems to be completely fine for him.

\begin{exe}
    \ex\label{ex:weird-1} 我们建议汉语动词具有下列特征而有别于英语动词 \citep{huang2007}
\end{exe}

\subsection{Existing literature}

The main problem is these examples are however not always reliable: 
some authors are not native speakers of Mandarin 
(even though their first languages are Sinitic), 

This note will cite some examples in the literature, 


\subsection{Representation of data}

Examples of Mandarin in this note are represented 
in Chinese characters,
the preferred writing system of Mandarin (\prettyref{sec:chinese-character});
this deviates from common practice in linguistics,
but the vast amount of written sources means that 
reading Chinese character seems unavoidable for 
a serious student of the language,
and since Mandarin does not have complicated phonological rules 
that obscure the morphosyntactic structure of utterances,
this practice is not in risk of being confusing.

\section{Remarkable features of Mandarin}

\subsection{The overwhelming influence of prosody}

One distinct feature of Mandarin is its morphosyntax relies strongly on \emph{prosody} \citep{feng2000}. 
Other components in phonology, strikingly, 
doesn't have much influence on Chinese morphosyntax,
and it will be largely skipped in this note.

\subsection{Lack of word (and therefore morphology)?}

Despite of lack of inflection
and lack of contextual alternation of morphemes,
Chinese does have some local and syntactically unmotivated operations
which are just like morphophonological rules,
although they don't necessarily operate on phrases.

An example of this is the verb copying phenomenon,
as in 看了一会书 (compare 看书了一会);
看书 \translate{to read} (intransitive; lit. \translate{to read books}) 
has a fossilized internal verb-object structure 
and this structure may still have synchronic effects.
More radical examples however also exist,
like \%体了一堂操, which is likely to be linked to 
[[体操]_{\text{noun-as-verb}}了[一堂]_{\text{time object}}]_{\text{\acs{vp}}}.
In casual speech,
verbs borrowed from other languages may also be split 
and the two fragments of the verb then surround the semi-object
(\prettyref{ex:remarkable.debug}).

\begin{exe}
    \ex\label{ex:remarkable.debug} 我 [debug]_{\text{topic:{\acs{vp}}}} [de不出来]_{\text{predicate:\acs{vp}}} [啊]_{\text{\acs{sfp}}}
    \ex \gll 怎么 可能 政变 政 一半 呢  \\
    how possibly coup {} half \category{sfp} \\
    \glt 
\end{exe}

This phenomenon -- the verb being split and the time semi-object getting embedded into the verb -- 
looks just like infixing,
although here this infixing operation targets a \acs{vp} instead of a smaller unit.
This justifies the assumption taken at the end of \prettyref{sec:theory}
that there is no clear boundary between words and phrases 
and therefore syntax and morphology:
It's possible for a phrase to undergo 
rearrangement without clear syntax motivation
that usually happens within a word.

\subsection{The so-called serial verb constructions}

It's often said Mandarin is a serializing (i.e. with serial verb constructions) language.
A closer look, however, reveals this is not the case:
These constructions are either adverbial clause constructions 
or complement clause constructions, 
or maybe certain kind of light verb constructions (\prettyref{sec:no-serial-verb}).
The internal heterogeneity renders the term \term{serial verb construction} useless.

\subsection{Ionized verbs}

Some verbal units -- commonly recognized as verbs, i.e. grammatical words -- in Mandarin
may be ionized into two parts, 
with a constituent residing between the two. 
This involves two mechanisms:
the first is discussed in \prettyref{sec:pos.verb.idiomatic-verb-object}, 
in which a verb phrase is fossilized, 
similar to English \form{in case} or \form{by definition},  
but its object can be extended into a complex \acs{np},
and the second seems to be a morphophonological device, 
by which even a verb without synchronically analyzable inner structure 
is teared into two parts and a clausal dependent is inserted between the two 
(\prettyref{sec:verb-splitting}).
Despite its striking properties for English speakers, 
ingredients of this phenomenon are all well attested cross-linguistically.

\subsection{Voices}

Mandarin has several valency alternation devices, 
including at least two structures 
that may be called passives,
commonly known as \form{bèi}-constructions
(\prettyref{sec:verb-phrase.object.short-bei},
\prettyref{sec:verb-phrase.bei}).
They however possess some features that tell them apart 
from passives in West European languages:
the most salient feature is 
they all seem to assume certain level of perceptiveness.
There also exists a \form{bǎ}-construction 
(\prettyref{sec:verb-phrase.object.ba}),
which is sometimes known as the causative construction,
although the coverage of this construction 
is narrow compared with lexical causatives (\prettyref{sec:verb-phrase.object.ba.causative});
the structure TODO: do to cause-become 



\subsection{Verbal complements}

The verbal complement system is another remarkable feature of Mandarin, 
which is sometimes classified as a type of serial verb constructions.


\chapter{Phonology and the writing system}

\section{Prosody}\label{sec:prosody-structure}

By paying attention to stops in Chinese utterances,
it can be found that phonological words exist and they are mostly defined by the prosody structure.
In the rest of this note,
the term \term{prosodic word} and \term{phonological word}
will be used interchangeably, 
since other phenomena like sandhi that may be used to 
determine the phonological word boundary 
are rare in Mandarin 
(but not complete absent: TODO: ref).
The prosody structure is about how stress is assigned to phonological constituents.
Assigning a prosodic structure is like condensation and clustering:
something is merged with something adjacent,
and the result is merged with something adjacent else.
When two phonological constituents are merged together,
one of them is considered heavier than the other.
If heaviness is to have a simple relation with the length of a phonological constituent,
then usually the more a phonological constituent is,
the heavier it is.
This is consistent with the condensation picture of prosodic segmentation.
Suppose a prosodic constituent attracts a syllable and merges with it.
The latter is not an independent phonological constituent
and cannot be heavy,
so the former is the heavier one and the latter is the lighter one in the larger prosodic constituent.

The smallest unit of prosody structure 
is a prosodic word.
The simplest prosodic word is the disyllabic foot, 
which contains two adjacent syllables in the case of Chinese.
(It can be made by two moras in other languages.)
One is assigned stress and is therefore heavier than the other.
Trisyllabic prosodic words also exist in Chinese,
though they are highly limited.
Most of which are borrowed words (e.g. 加拿大 \translate{Canada})
or words formed by coordinating three morphemes (e.g. 数理化 \translate{math, physics, and chemistry}).
They can also be regarded as foots \citep[\citesec{2.2}]{feng2000}.

Longer morphosyntactic units are 
inevitably broke into smaller disyllabic or trisyllabic prosodic words
in their prosodic structures,
often regardless of their morphosyntactic structure:
加利福尼亚 may be segmented into 加利\textbar 福尼亚, 
although the word contains only one morpheme.
In 副总经理,
we have two prosodic words,
副总 and 经理,
while the morphosyntactic structure of the word is [[副] [总 [经理]]]
(\prettyref{sec:pos.noun.adj-modify}).
This is similar to the case in English and Latin poems,
where the prosody arrangement of sentences does not have to respect word boundaries:
\form{arma vi\textbar rumque ca\textbar no}.
It's however also possible that 
a prosodic word has morphosyntactic significance
(\prettyref{sec:pos.word.phonological}).

Prosody is able to see the constituency structure
and prosodic constraints are important in Mandarin grammar.
Some prosodic rules pertaining to the constituency tree 
guide and limit the assignment of relative heaviness and lightness.
In Chinese, prosodic segmentation is done strictly left-to-right 
in each \ac{np},
and then the \ac{np}s together with verbal constituents 
are used as the input of prosodic segmentation of clauses.
Certain forms are therefore ruled out
(\prettyref{sec:vp.prosody}), 
not by morphosyntactic reasons but for prosodic reasons.

\section{Chinese characters}\label{sec:chinese-character}

The preferred writing system of Mandarin is the Chinese character system.
Except some characters made in early modern ages,
like 兛 \translate{kilogram} or 砼 \translate{concrete (lit. human-labor stone)},
a Chinese character corresponds to a syllable.
However, Chinese characters don't just represent the sound.
Putting some quirky cases aside,
Chinese characters are often good indicators of morphemes
(\prettyref{sec:pos.morpheme.primitive}).
There are, for example, at least seven morphemes sounding \form{xi\={a}n},
and there happens to be seven Chinese characters corresponding to each of them:
仙, 先, 籼, 掀, 锨, 鲜, and 纤.

Like all writing systems, 
Chinese characters do not completely faithfully represent 
the underlying linguistic structure.
Some characters do not mean anything -- 
they are simply the designated characters representing syllables 
in certain polysyllabic morphemes.
The character 萄 as in 葡萄, for example, 
means nothing more than the syllable \form{t\'{a}o},
but it only appears in the morpheme 葡萄 and 葡萄牙 \translate{Portuguese}.
The same is for the character 葡.
Some characters have regular morpheme meanings
but also have merely phonetic meaning in certain words.
The character 登 in 摩登 regularly means \translate{climb},
but in the word 摩登, only its phonetic value \form{d\={e}ng} is preserved.
Certain morphemes can be denoted by more than one character.
The \ac{sfp} \form{ba} can be written as 吧 or 罢,
the latter hinting its etymology but is now rarely used.
Certain characters denote more than one morpheme.
The character 会 may mean \translate{conference} or \translate{be able to do}. 

Thus, Chinese characters provide clues on what is a morpheme,
but they are not decisive \citep[1.1.4]{zhudexigrammar}.

\chapter{Parts of speech}

About the structure of the note: 
since now the part about verb frames is too long, 
it's worth it to break the chapter about verb phrase 
into several parts, 
just like the grammar of Latin or Sanzhi; 
then there should be a grammatical sketch part, 
which I'm unwilling to write because 
I will then be faced with the dilemma to decide where to start 
introducing a concept; 
the current plan is to put the sketch of NP or clauses 
into ``distribution and grammatical relations'' sections about nouns or verbs.

\section{From morphemes to clauses}

\subsection{Morphemes}\label{sec:pos.morpheme}

I start the introduction of Mandarin morphosyntax 
with morphemes. 
Although in most language documentation projects, 
words -- whatever this term mean (\prettyref{sec:pos.word}) -- 
are the smallest unit appearing in the dictionary, 
this is not the case with Mandarin:
the standard practice of lexicography 
is to record Chinese characters and their ``meanings''; 
in linguistic terms, 
an entry of a Chinese character includes the follows:
\begin{itemize}
    \item its pronunciation(s), almost always a syllable in Mandarin Chinese
    (\prettyref{sec:chinese-character}); 
    \item historical or synchronic monosyllabic morphemes that may be represented by that character;
    \item polysyllabic morphemes containing that character,
    which may have different pronunciations 
    (\prettyref{sec:pos.morpheme.primitive}-\prettyref{sec:pos.morpheme.abbreviation});
    \item grammatical words that are made up regularly using one of the morphemes listed above
    and have already gained a stable meaning
    (for example, under the entry of the character 打, 
    which is a light verb,
    we may find 打坐, which means \translate{do sitting (and meditate)}).
\end{itemize}
The morpheme is therefore the closest concept to Chinese characters 
-- how most educated Mandarin speakers perceive the basic unit of Mandarin
-- that bears some linguistic significance.
Thus, to use existing dictionaries to understand an utterance,
finding the morphemes -- instead of grammatical words recognized in \prettyref{sec:pos.word} -- 
is the first step.
The meaning of the term however has some ambiguity,
since analyzable inner structures of words 
does not always have synchronic morphosyntactic significance
(\prettyref{sec:pos.morpheme.fossilization}, \prettyref{sec:pos.morpheme.abbreviation}).
In this case, we say the item is \emph{synchronically} a morpheme.%
\footnote{
    In the end, discussions using the term \term{morpheme} can -- and should -- 
    be reduced into discussions about features, roots, fossilization, etc.
    The concept of \term{morpheme} enjoys no position 
    in primitive concepts in linguistics, 
    and has inherent ambiguity. 
    The English expression \form{till death do us part} 
    has a conventionalized usage 
    and an archaic internal structure, 
    but people rarely call it a morpheme. 
    This intuition that morphemes shouldn't be too long 
    essentially comes from the desire to define the morpheme
    as an etymological concept:
    the smallest unit that can be distinguished historically
    is a morpheme 
    (whether the relevant syntactic mechanisms are productive 
    is out of question).
    This definition of morpheme
    has its own weakness: 
    morphemes can be fused as time goes by, 
    and how forms like Latin conjugation endings should be divided 
    depends on how long you are willing to go backwards 
    in the history of the language.
}

\subsubsection{Primitive content morphemes}\label{sec:pos.morpheme.primitive}

Most native primitive content morphemes 
are monosyllabic, examples of which include 红, 大, 你, etc.
(\prettyref{sec:pos.word.monosyllabic}).
Polysyllabic primitive content morphemes,
like 葡萄, 巧克力, 哥斯达黎加 etc., 
are mostly borrowed words
in different historical stages.
There do exist seemingly native polysyllabic primitive content morphemes,
like 轱辘 \translate{wheel}; 
but it has been proposed that this word is related to PIE. TODO: ref 

Only a subset of the monosyllabic morphemes are free,
i.e. able to form grammatical words
on their own.%
\footnote{
    This involves the question what is a word. 
    This question is dealt in \prettyref{sec:pos.word}.

    Another definition seen in \citet[\citesec{1.1.2}]{zhudexigrammar}
    is that a free morpheme is a morpheme that may appear as an utterance.
    This definition contradicts with other parts in \citet{zhudexigrammar},
    because most function words never appear as a single utterance
    and yet are still recognized as words
    and therefore free morphemes by himself.
    This definition of free morpheme is also of little value syntactically,
    since it's much more consistent to check the ability of larger constructions to appear 
    as a single utterance, 
    and then check whether some of them are monomorphemic. 

    Yet another definition seen in \citet[\citepage{16}]{zhudexigrammar} claims that 
    free morphemes don't have fixed positions, 
    while bound morphemes sometimes have fixed positions.
    Cross-linguistically, this is simply wrong: 
    free morphemes can only occupy a limited number 
    of positions in the syntactic structure 
    (lexical head position, compounding attributive position, etc.)
    and this may lead to a constituent order effect.
    In Japanese,  
    the position of the verb is predominantly 
    after all clausal dependents and before inflectional endings, 
    but we would all agree that verb roots are prototypical free morphemes.
    On the other hand, reordering of derivational affixes, although rare, 
    is not completely unattested. 
}
Bound morphemes may be found as 
derivational affixes (\ref{ex:nominal-modifier-1}) 
or as TODO: list .
Many of them are historically free morphemes
but have become obsolete in contemporary usage.
The morpheme 观 \translate{observe}, for example,
still exists in 观鸟 \translate{bird watching}
but never appears as a single verb, 
nor does it undergo delimitative reduplication of verbs
(\prettyref{ex:pos.obsolete-1}).

\begin{exe}
    \ex\label{ex:pos.obsolete-1} \begin{xlist}
        \ex *我要去观观那些鸟
        \ex 我要去看看那些鸟
    \end{xlist}
\end{exe}

It should however be noted that bound morphemes
are not always immediate constituents of grammatical words; 
that's to say, they may have limited word-forming abilities 
in some lexicalized constructions.
\citet[\citesec{8.3.2}]{zhudexigrammar} notes that 
despite 亏 is a bound morpheme and 饭 is a free morpheme,
verb-object constructions 吃饭 and 吃亏 has largely similar morphosyntactic behaviors,
in which the object 亏 may be modified just like any other \acs{np}, 
as in (\prettyref{ex:pos.morpheme.chikui}b).
We can even move 亏 out of the verb-object construction 
and topicalize it (\prettyref{ex:pos.morpheme.chikui}c; 
\citet{zhudexigrammar} seems to be unaware of this fact).
Thus, we conclude that 吃亏 can be regarded 
as a usual \acs{vp},
enjoying the same status of 吃饭.%
\footnote{
    But it can still be regarded as a grammatical word
    at least in some circumstances
    (\prettyref{sec:pos.verb.idiomatic-verb-object}).
}
Cross-linguistically, 
it's quite common for some words to appear mainly in idioms, 
but this doesn't mean syntactically
these idioms are words.

\begin{exe}
    \ex\label{ex:pos.morpheme.chikui} \begin{xlist}
        \ex 我吃了这个亏
        \ex 我吃了一个大亏
        \ex 这个亏我今天吃了,但是你们之后别再想从我这里拿到一分钱好处!
    \end{xlist}
\end{exe}

The boundary between free and bound primitive content morphemes 
seems to vary among registers and conversational context. 
(\prettyref{ex:pos.morpheme.kui-2}) is somehow strange without a context, 
but if the idiom \acs{vp} 吃亏 has appeared frequently enough
in previous speech, 
it gradually becomes acceptable;
the fact that 亏 usually doesn't appear as a noun 
but does appear as a noun here 
has a stylist focusing effect.
(\prettyref{ex:pos.morpheme.quan}) is not acceptable in daily conversation,
because the morpheme 犬 is usually a bound morpheme 
and should be displaced by the more colloquial 狗.
The sentence however is perfectly fine in 
a police officer's recollection of a detective story involving K-9 dogs.
Thus the professional background licenses 犬 as a free morpheme.

\begin{exe}
    \ex\label{ex:pos.morpheme.kui-2} 
    ? 这个亏让我记了一辈子
    \ex\label{ex:pos.morpheme.quan} 
    \gll \% 当时 我 的 犬 发现 现场 的 气味 有点 不对劲 \\
    {} at.that.time 1 \category{poss} canine find scenario \category{poss} smell kind.of abnormal \\
    \glt \translate{At that time, my canine found the smell of the scenario was kind of abnormal.}
    (acceptable in professional context)
\end{exe}

Polysyllabic morphemes are mostly free. 
There exist however a small number of polysyllabic morphemes 
that seem to be unable to be words themselves. 
The root 日耳曼 (from \form{German}) 
appears in compound nouns like 日耳曼人 \translate{Germanic people} 
or 日耳曼血统 \translate{Germanic descent}, 
but almost never appears as a noun itself; 
similarly, 达达 in 达达主义 also never appears as a word itself.

\subsubsection{Fossilization}\label{sec:pos.morpheme.fossilization}

A lot of words have internal structures parallel to 
those observed in syntax \citep[\citesec{2.6}]{zhudexigrammar},
but the structures have already completely fossilized, 
so they may be synchronically regarded as containing only one morpheme.
Thus, synchronically they are not different from polysyllabic morphemes.

A completely fossilized structure 
may be historically created with an obsolete syntactic device, 
or be not in the expected part of speech inferred from its etymological
(i.e. it appears in syntactic environments that are not expected 
for its inner structure).
The verb 关心 has an internal verb-object structure, 
but is able to take an object.
Since well-attested double object constructions in Mandarin
are all unable to cover this usage,
we conclude 关心 has already been fossilized into one single synchronic morpheme.%
\footnote{
    It's still possible for 关心 to be split, 
    but this happens without considering the internal structure of the verb
    (\prettyref{sec:verb-splitting}).
}
Some words with fossilized phrasal origin 
contain gap inside, 
indicating that at some early stages 
they were parts of formulaic speeches 
and were later reanalyzed as words.
The word 例如 \translate{for example}
is a connective adverb, 
but it has a subject-predicate with a gap,
and likely arose from a reanalysis of the formula (\ref{ex:pos.fossilized.1}).
This also explains why it appears predominantly at the start of a clause.

\begin{exe}
    \ex\label{ex:pos.fossilized.1} {} [[例]_{\text{subject}} 如 [\dots]_{\text{object}}]_{\text{clause}} \\
    \translate{an example is like \dots}. 
\end{exe}

Although fossilization in syntactic structure 
is often connected with 
a conventionalized meaning and being small in size,
the three parameters are not completely interdependent, 
although we can observe some weaker-than-expected correlation.
Regarding the relation between syntactic fossilization 
and the size of the unit in question, 
it should be noted that many languages have 
idioms that have archaic syntactic structures, 
like \form{till death do us part} in English 
(involving archaic verb-final clausal structures)
and 放心不下 (involving an early stage of a type of verbal complement structure) 
in Mandarin.
Regarding the relation between syntactic and semantic fossilization, 
note that most idioms -- what the word refers to in everyday speech -- 
are formed by regular syntactic devices and yet have gained conventionalized meanings, 
while it's also possible that some structures have already been semi-productive
and yet the meanings of their products can still be 
regularly inferred compositionally
(\prettyref{sec:pos.noun.adj-modify}).


\subsubsection{Abbreviation}\label{sec:pos.morpheme.abbreviation}

Mandarin has a strong tendency to abbreviate an existing construction into a prosodic word.
The abbreviation of a binary-branching structure
usually consists of the first syllables of its two immediate constituents,
regardless of the inner structures of the two immediate constituents.
Thus [副 [总经理]] \translate{vice general manager} is usually abbreviated as 副总, 
and 总工程师 is usually abbreviated as 总工.
It's possible that an abbreviation replaces the original word completely.
空调 is historically the abbreviation of 空气调节器, 
the word-by-word translation of \form{air conditioner},
but the latter is no longer in active use.

Trisyllabic prosodic words do exist in Mandarin, 
and trisyllabic abbreviations also exist.
The majority of them are three-morpheme coordination structures like 数理化, 
which is the abbreviation of 
数学物理化学 \translate{math, physics, and chemistry}.

\subsubsection{Function items and semi-function items} 

Another type of morphemes is the type of 

Monosyllabic location words, 
for example, appears in \acs{np}s 
and therefore should be analyzed as words;
they however never constituent one-word \acs{np}s themselves
(\prettyref{sec:pos.locational}).
Whether they are words therefore becomes a mystery,
and this mystery is merely a wrongly asked question.

Certain borrowed affixes may be unable to serve as words in certain periods.
As times goes by, however, they gradually become free morphemes.
If clause linking markers like 之所以 and 是因为 are recognized as single morpheme words,
then they may be included into the non-prosodic simple word blob 
and however be unable to serve as phrases.
These markers, however, never appear in other places,
and their exact status is of no descriptive and comparative interest.

\subsection{Wordhood}\label{sec:pos.word}

A question causing endless controversy and confusion 
is ``what is a word''. 
\citet{dixon2010basic2} spends a whole chapter (\citechap{10}) on this topic.
It is often said that Chinese is ``character-based''
or to be precise, ``monosyllabic morpheme-based'',
with no level of grammatical words.
This claim, despite having its value 
to remind learners about the uniqueness of Mandarin grammar,
is misleading: 
it's indeed possible to recognize units that 

What do make Mandarin wordhood somehow opaque 
are the fact that a clear word-phrase distinction is sometimes hard to draw, 
and that TODO 

\subsubsection{Phonological words in morphosyntax}\label{sec:pos.word.phonological}

Comparing morphosyntactic levels 
and \emph{phonological} wordhood 
may be a good idea for us to draw a boundary 
between words and phrases.


A phonological word,
which in Mandarin is defined by prosody (\prettyref{sec:prosody-structure}),
may be syntactically one of the follows: 
\begin{itemize}
    \item a single-morpheme with a well-defined part of speech tag (\prettyref{sec:pos.morpheme}),
    like 幽默 \translate{humor (homophonic translation)} 
    and relatively rare trisyllabic cases 
    like 加拿大 \translate{Canada (homophonic translation)}, or 
    \item a mini-constituent with \acs{np} or \acs{vp}-like internal structure
    (\prettyref{sec:pos.architecture.word.mini-constituent}), 
    like 白菜 \translate{Chinese cabbage (lit. white vegetable)} 
    (\prettyref{sec:pos.noun.fossilized-structure})
    or 种树 \translate{plant tree} (\prettyref{sec:pos.verb.idiomatic-verb-object}), or 
    \item a sequence with arguable affixation 
    like 交给 \translate{transfer-give (the latter being a verbal complement)} 
    (\prettyref{sec:pos.architecture.word-confusing.constituent}).
    \item It's of course possible that a phonological word 
    has no morphosyntactic significance at all.
\end{itemize}

The next question is 
whether all prosodic words that are also morphosyntactic units 
are small enough to be grammatical words, 
or some of them are actually phrases.
The main issues are presented as follows: 
\begin{itemize}
    \item Disyllabic verb-object constructions
    are highly similar to ordinary verb phrases; 
    they however participate in compounding 
    and therefore are words (probably after dephrasalization) 
    at least in some occasions
    (\prettyref{sec:pos.verb.idiomatic-verb-object}).

    \item Disyllabic attributive constructions like 小狗 
        
    \item Verbal complement constructions 
\end{itemize}
Thus, I use the term \concept{morphosyntactic prosodic words}
to refer to prosodic words with morphosyntactic significance, 
indicating that they are all words in some circumstances, 
but may also be phrases in other cases.

fixing the boundary between words and phrases in a way 
that is somehow subjective but consistent with 
the standard employed in many other world languages
(TODO: \prettyref{sec:pos.architecture.word-confusing.constituent}).


There are also non-prosodic grammatical words in Mandarin,
including monosyllabic words and polysyllabic words like 哥斯达黎加 
(\prettyref{sec:pos.morpheme.primitive})
and complex non-prosodic words like 
总工程师 (\prettyref{sec:pos.word.complex}).


\subsubsection{Monosyllabic words: grammatical words, but not prosodic words}\label{sec:pos.word.monosyllabic}

As is said in \prettyref{sec:pos.morpheme.primitive}, 
some -- although by no means all -- monosyllabic morphemes 
are free morphemes and

\subsubsection{Disyllabic mini-constituents: prosodic words, and also grammatical words}\label{sec:pos.architecture.word.mini-constituent}

There are disyllabic units in Chinese 
that have conventionalized meanings and its inner structure is invisible 
to any other morphosyntactic rules 
(\prettyref{sec:pos.morpheme.fossilization},
\prettyref{sec:pos.noun.fossilized-structure},
\prettyref{sec:pos.verb.fossilized-structure}).
Other disyllabic word are made up by two morphemes with a synchronically available device.

A controversy is whether some disyllabic prosodic words that have morphosyntactic significance
are actually phrases,
in which the two syllables may be analyzed 
as two grammatical words
instead of two morphemes or two meaningless syllables.
This is usually not the case for nouns, 
but is indeed true for verbs
(\prettyref{sec:pos.verb.idiomatic-verb-object}).
Examples of them include 念佛, 军训, 体操 etc. 

Then it's possible that the morphosyntactic prosodic word can be extended into a phrase 
by inserting more phrasal dependents 
or by adjoining modifiers to one of the two morphemes.
But note that similar processes are possible even for 
disyllabic verbs without analyzable inner structures,
which are surely grammatical words (\prettyref{sec:verb-splitting}).

\subsubsection{Psosodic words created by semi-inflection}\label{sec:pos.architecture.word-confusing.constituent}


In Mandarin, although prototypical inflection is absent, 
in verbal complement constructions 
and the aspect system, 
we indeed can see something similar to \form{explored} mentioned above, 
the inner parts of which have strong dependency
but do not form a constituent in the most strict sense.
What is the status of 爬上 in 他笨手笨脚地爬上信号塔?
A word (created by a productive verb compounding rule), 
a phrase (a verb-complement structure),
or just a word sequence without structural significance?
Here I follow the opinion in \citep[\citepage{86}]{feng2000} and \citet{tham2015resultative}
and call it a word,
because a sequence like 爬上 is extended in a highly limited way 
(the only possibilities being 爬得上 and 爬不上),
while phrases, in principle, can be extended infinitely, 
and it is also a prosodic word.
This goes against the analysis in \citep[\citesec{1.2.7}]{zhudexigrammar}.


\subsubsection{Non-prosodic complex words}\label{sec:pos.word.complex}

Certain grammatical relations seem to be not a part of \ac{np}s and clauses,
again highlighting the necessity to introduce a smaller level of constituency
(\prettyref{sec:pos.noun.compound}, TODO: verb),
commonly known as grammatical words.
The structures listed above in principle 
can be extended without an upper bound, 
and therefore they are not monosyllabic and are not prosodic words.

Just like the case of morphosyntactic prosodic words,
these \concept{non-prosodic complex grammatical words} 
may be created by synchronic morphosyntactic rules,
or they may be fossilized or have no internal structure.

Compared to prosodic words,
non-prosodic complex words are less ``active'' in syntax:
splitting them is possible in certain cases
but is much less frequent.
This may be a result of pragmatics:
complex words are created to cover a meaning that needs some explanation,
and once a complex word is well-accepted,
its form and meaning soon gets fixed 
(because people will not burden themselves),
and fossilization occurs rapidly.
The term 美利坚合众国 has an analyzable internal structure,
but it has already gained a fixed meaning 
and its parts are never taken out,
despite both 美利坚 and 合众国 can serve as grammatical words.


\subsection{Larger units}

\subsubsection{Noun phrase}

Nouns are able to be lexical heads of \acs{np}s,
including non-prosodic simple words like 哥斯达黎加,
as well as non-prosodic complex words.
One-word phrases are always possible,
and there is almost no attested counterexample.

\subsubsection{The verb phrase}

Transitive verbs can regularly fill argument slots 
and thus are able to be used as one-word phrases,
though they themselves are not sufficient to build one-word predicate \ac{vp}s. TODO: really???
Certain grammatical words,
like some verbal complement structures 

\begin{exe}
    \ex\label{ex:transitive-verb-phrase-disyllabic} 
    \begin{xlist}
        \ex {} [看书]_{\text{subject:verb}} 是一件有趣的事情
        \ex *[走进]_{\text{subject:verb}} 意味着您已经同意了我们的服务条款 
        \ex {} [走进这个建筑]_{\text{subject:VP}} 意味着您已经同意了我们的服务条款
    \end{xlist}
\end{exe}

\subsection{The structure of Mandarin grammar}

After establishing the status of \term{words} in Mandarin grammar, 
we can now discuss the overall architecture of Mandarin grammar 
in a more disciplined way.
\prettyref{fig:morpheme-to-phrase} summarizes the organization of Chinese lexicon 
as well as how larger units are built from lexical items.
Overlapping of blobs means ``having the same form''.
Thus, the blob representing monosyllabic words is completely in the blob of monosyllabic morphemes.
The same is for the relation between non-prosodic simple words 
(which are neither monosyllabic nor disyllabic) and polysyllabic morphemes.
Red arrows mean synchronic morphosyntactic devices,
while orange arrows means historical evolution,
like grammaticalization and/or fossilization,.

\begin{figure}[H]
    \centering
    

\tikzset{every picture/.style={line width=0.3pt}} %set default line width to 0.75pt        

\begin{tikzpicture}[x=0.75pt,y=0.75pt,yscale=-0.8,xscale=0.8]
%uncomment if require: \path (0,552); %set diagram left start at 0, and has height of 552

%Shape: Ellipse [id:dp25211786396044866] 
\draw  [color={rgb, 255:red, 155; green, 155; blue, 155 }  ,draw opacity=1 ][fill={rgb, 255:red, 155; green, 155; blue, 155 }  ,fill opacity=0.2 ] (105.33,392.3) .. controls (105.33,341.28) and (154.13,299.93) .. (214.33,299.93) .. controls (274.53,299.93) and (323.33,341.28) .. (323.33,392.3) .. controls (323.33,443.31) and (274.53,484.67) .. (214.33,484.67) .. controls (154.13,484.67) and (105.33,443.31) .. (105.33,392.3) -- cycle ;
%Shape: Ellipse [id:dp011093449258029242] 
\draw  [color={rgb, 255:red, 155; green, 155; blue, 155 }  ,draw opacity=1 ][fill={rgb, 255:red, 155; green, 155; blue, 155 }  ,fill opacity=0.2 ] (164.06,334.33) .. controls (167.18,318.1) and (187.28,308.32) .. (208.94,312.5) .. controls (230.6,316.67) and (245.62,333.22) .. (242.49,349.45) .. controls (239.37,365.69) and (219.27,375.47) .. (197.61,371.29) .. controls (175.95,367.12) and (160.93,350.57) .. (164.06,334.33) -- cycle ;

%Shape: Ellipse [id:dp8023514623566042] 
\draw  [color={rgb, 255:red, 155; green, 155; blue, 155 }  ,draw opacity=1 ][fill={rgb, 255:red, 155; green, 155; blue, 155 }  ,fill opacity=0.2 ] (336,398) .. controls (336,362.51) and (373.68,333.74) .. (420.17,333.74) .. controls (466.65,333.74) and (504.33,362.51) .. (504.33,398) .. controls (504.33,433.49) and (466.65,462.26) .. (420.17,462.26) .. controls (373.68,462.26) and (336,433.49) .. (336,398) -- cycle ;
%Shape: Ellipse [id:dp321544508802865] 
\draw  [color={rgb, 255:red, 155; green, 155; blue, 155 }  ,draw opacity=1 ][fill={rgb, 255:red, 155; green, 155; blue, 155 }  ,fill opacity=0.2 ] (404.81,386.06) .. controls (402.26,369.72) and (417.86,353.72) .. (439.65,350.31) .. controls (461.45,346.91) and (481.19,357.39) .. (483.74,373.73) .. controls (486.29,390.06) and (470.69,406.07) .. (448.9,409.47) .. controls (427.1,412.88) and (407.36,402.4) .. (404.81,386.06) -- cycle ;

%Shape: Ellipse [id:dp9397609165336205] 
\draw  [color={rgb, 255:red, 155; green, 155; blue, 155 }  ,draw opacity=1 ][fill={rgb, 255:red, 155; green, 155; blue, 155 }  ,fill opacity=0.2 ] (382.01,362.67) .. controls (324.14,364.88) and (275.22,314.07) .. (272.74,249.18) .. controls (270.25,184.3) and (315.16,129.9) .. (373.03,127.69) .. controls (430.9,125.47) and (479.82,176.28) .. (482.3,241.17) .. controls (484.78,306.06) and (439.88,360.45) .. (382.01,362.67) -- cycle ;

%Curve Lines [id:da3228764928640613] 
\draw [color={rgb, 255:red, 208; green, 2; blue, 27 }  ,draw opacity=1 ]   (230.86,300.6) .. controls (223.77,257.91) and (236.35,243.61) .. (271.13,248.93) ;
\draw [shift={(272.74,249.18)}, rotate = 189.46] [fill={rgb, 255:red, 208; green, 2; blue, 27 }  ,fill opacity=1 ][line width=0.08]  [draw opacity=0] (12,-3) -- (0,0) -- (12,3) -- cycle    ;
%Curve Lines [id:da796668618520048] 
\draw [color={rgb, 255:red, 245; green, 166; blue, 35 }  ,draw opacity=1 ] [dash pattern={on 4.5pt off 4.5pt}]  (646.67,256.26) .. controls (666.57,354.69) and (582.51,395.85) .. (503.85,408.08) ;
\draw [shift={(502.67,408.26)}, rotate = 351.36] [fill={rgb, 255:red, 245; green, 166; blue, 35 }  ,fill opacity=1 ][line width=0.08]  [draw opacity=0] (12,-3) -- (0,0) -- (12,3) -- cycle    ;
%Curve Lines [id:da025052452235164946] 
\draw [color={rgb, 255:red, 208; green, 2; blue, 27 }  ,draw opacity=1 ]   (551.67,168.26) .. controls (514.24,147.58) and (486.19,158.91) .. (471.97,189.83) ;
\draw [shift={(471.33,191.26)}, rotate = 293.63] [fill={rgb, 255:red, 208; green, 2; blue, 27 }  ,fill opacity=1 ][line width=0.08]  [draw opacity=0] (12,-3) -- (0,0) -- (12,3) -- cycle    ;
%Shape: Ellipse [id:dp6184914092498568] 
\draw  [color={rgb, 255:red, 155; green, 155; blue, 155 }  ,draw opacity=1 ][fill={rgb, 255:red, 155; green, 155; blue, 155 }  ,fill opacity=0.2 ] (536,203.67) .. controls (536,172.74) and (571.89,147.67) .. (616.17,147.67) .. controls (660.44,147.67) and (696.33,172.74) .. (696.33,203.67) .. controls (696.33,234.59) and (660.44,259.67) .. (616.17,259.67) .. controls (571.89,259.67) and (536,234.59) .. (536,203.67) -- cycle ;

%Curve Lines [id:da27466382204201545] 
\draw [color={rgb, 255:red, 208; green, 2; blue, 27 }  ,draw opacity=1 ]   (514,363.26) .. controls (519.31,299.58) and (676.42,360.64) .. (637.6,261.76) ;
\draw [shift={(637,260.26)}, rotate = 67.91] [fill={rgb, 255:red, 208; green, 2; blue, 27 }  ,fill opacity=1 ][line width=0.08]  [draw opacity=0] (12,-3) -- (0,0) -- (12,3) -- cycle    ;
%Curve Lines [id:da11077382821910953] 
\draw [color={rgb, 255:red, 245; green, 166; blue, 35 }  ,draw opacity=1 ] [dash pattern={on 4.5pt off 4.5pt}]  (473.67,293.26) .. controls (484.56,310.09) and (489.57,323.98) .. (432.42,337.84) ;
\draw [shift={(430.67,338.26)}, rotate = 346.65] [fill={rgb, 255:red, 245; green, 166; blue, 35 }  ,fill opacity=1 ][line width=0.08]  [draw opacity=0] (12,-3) -- (0,0) -- (12,3) -- cycle    ;
%Curve Lines [id:da3266231201686014] 
\draw [color={rgb, 255:red, 155; green, 155; blue, 155 }  ,draw opacity=1 ][fill={rgb, 255:red, 155; green, 155; blue, 155 }  ,fill opacity=0.2 ]   (89,90.26) .. controls (85.67,243.26) and (163.67,345.26) .. (210.67,340.26) .. controls (257.67,335.26) and (204.67,217.26) .. (283.67,140.26) .. controls (362.67,63.26) and (423,124.59) .. (414,207.59) .. controls (405,290.59) and (320.67,294.26) .. (350.67,376.26) .. controls (380.67,458.26) and (558.67,427.26) .. (634.67,364.26) .. controls (710.67,301.26) and (728.67,127.26) .. (728.67,90.26) ;
%Shape: Polygon Curved [id:ds7705333747253118] 
\draw  [color={rgb, 255:red, 155; green, 155; blue, 155 }  ,draw opacity=1 ][fill={rgb, 255:red, 155; green, 155; blue, 155 }  ,fill opacity=0.2 ] (300,283.93) .. controls (412,321.93) and (522,319.93) .. (522,395.93) .. controls (522,471.93) and (377,501.93) .. (274,517.93) .. controls (171,533.93) and (84,477.93) .. (79,395.93) .. controls (74,313.93) and (188,245.93) .. (300,283.93) -- cycle ;

% Text Node
\draw (126,154) node [anchor=north west][inner sep=0.75pt]  [color={rgb, 255:red, 0; green, 0; blue, 0 }  ,opacity=1 ] [align=left] {phrases};
% Text Node
\draw (377.52,245.18) node   [align=left] {\begin{minipage}[lt]{76.4pt}\setlength\topsep{0pt}
\begin{center}
morphosyntactic\\prosodic\\words
\end{center}

\end{minipage}};
% Text Node
\draw (245.5,420.7) node   [align=left] {\begin{minipage}[lt]{60.24pt}\setlength\topsep{0pt}
\begin{center}
monosyllabic\\morphemes
\end{center}

\end{minipage}};
% Text Node
\draw (203.28,341.89) node  [font=\scriptsize] [align=left] {\begin{minipage}[lt]{48.74pt}\setlength\topsep{0pt}
\begin{center}
monosyllabic\\words
\end{center}

\end{minipage}};
% Text Node
\draw (618.17,202.67) node   [align=left] {\begin{minipage}[lt]{59.71pt}\setlength\topsep{0pt}
\begin{center}
non-prosodic\\complex\\words
\end{center}

\end{minipage}};
% Text Node
\draw (427.5,430.7) node  [font=\small] [align=left] {\begin{minipage}[lt]{49.89pt}\setlength\topsep{0pt}
\begin{center}
polysyllabic\\morphemes
\end{center}

\end{minipage}};
% Text Node
\draw (444.28,379.89) node  [font=\tiny] [align=left] {\begin{minipage}[lt]{48.8pt}\setlength\topsep{0pt}
\begin{center}
non-prosodic\\simple words
\end{center}

\end{minipage}};
% Text Node
\draw (304,460.93) node [anchor=north west][inner sep=0.75pt]   [align=left] {morphemes};


\end{tikzpicture}

    \caption{From morphemes to phrases.}
    \label{fig:morpheme-to-phrase}
\end{figure}

The sub-phrasal units mentioned in \prettyref{fig:morpheme-to-phrase}
are represented in \prettyref{tbl:sub-phrasal-units}.
The +? symbols in the ``morpheme'' column 
mean possible fossilization, 
corresponding to the lower two orange arrows in \prettyref{fig:morpheme-to-phrase}.
The upper orange arrow in \prettyref{fig:morpheme-to-phrase}
means word creation by abbreviation. 
The +? symbol appearing in the ``phrase'' column 
means some authors may agree that 走进 is a constituent 
and therefore they may say it's a phrase
(\prettyref{sec:pos.architecture.word-confusing.constituent}).

\begin{sidewaystable}
    \centering
    \caption{Sub-phrasal units}
    \label{tbl:sub-phrasal-units}
    \begin{tabular}{@{}lllllll@{}}
    \toprule
    index & length                                                                                                       & morpheme & word & phrase & example           & comment                                                                                                                                    \\ \midrule
    1     & \multirow{3}{*}{monosyllable}                                                                                & +        & -    & -      & 竞, 语              & usually from Classical Chinese                                                                                                             \\ \cmidrule{3-7}
    2     &                                                                                                              & +        & -    & -      & 性 in 等价性, 前 in 窗前 & \begin{tabular}[c]{@{}l@{}}suffixes, location words, \\ and other semi-grammatical items\end{tabular}                                      \\ \cmidrule{3-7}
    3     &                                                                                                              & +        & +    & +      & 蛇, 猫              & usually from Classical Chinese                                                                                                             \\ \midrule
    4     & \multirow{4}{*}{disyllable}                                                                                  & +        & -    & -      & 达达 in 达达主义        & usually borrowed terms, rare                                                                                                               \\ \cmidrule{3-7}
    5     &                                                                                                              & +        & +    & -      & 觊觎, 葡萄, 空调        & \begin{tabular}[c]{@{}l@{}}from Classical Chinese, borrowed items, \\ abbreviations, and deep fossilization\\ of type 6 and 7\end{tabular} \\ \cmidrule{3-7}
    6     &                                                                                                              & -/(+?)   & +    & -/(+?) & 走进, 来到            & \begin{tabular}[c]{@{}l@{}}constituents in \ac{blt} only; \\ made of manosyllables\end{tabular}                                            \\ \cmidrule{3-7}
    7     &                                                                                                              & -/(+?)   & +    & +      & 念佛, 看书, 吃饭        & \begin{tabular}[c]{@{}l@{}}mini-constituents; \\ made of monosyllables\end{tabular}                                                        \\ \midrule
    8     & \multirow{3}{*}{\begin{tabular}[c]{@{}l@{}}trisyllable \\ or larger \end{tabular}} & +        & -    & -      & 日耳曼               & usually borrowed terms, rare                                                                                                               \\ \cmidrule{3-7}
    9     &                                                                                                              & +        & +    & +      & 哥斯达黎加, 滑铁卢,数理化    & usually borrowed terms or abbreviations                                                                                                    \\ \cmidrule{3-7}
    10    &                                                                                                              & -/(+?)   & +    & +      & 副总经理, 总干事         & complex word, made of monosyllables and disyllables                                                                                        \\ \bottomrule
\end{tabular}
\end{sidewaystable}

\subsection{Perception of basic units in Mandarin}\label{sec:pos.word.perception}

In English there is a group of clearly defined and largely homogenous morphosyntactic units 
lying between morphemes and phrases
(which means they can be neither morphemes nor phrases,
and can be constructed by the former and can be used to build the latter),
which is just the grammatical word.
They are recognized as basic units by society 
(i.e. outside of the linguistic community):
the length of an article is measured by counting words, not letters, for example.

In Chinese, however, despite the fact that 
we can still identify grammatical words,
the much larger overlap between grammatical words and morphemes and phrases
means Chinese characters -- representation of syllables in polysyllabic morphemes 
and rough representation of monosyllabic morphemes -- 
are recognized as basic units of the language.
This may be the deriving force for some linguists 
(who are too eager to ``find diversity'')
to reject the existence of grammatical words in a hurry,
although I've already shown this is not the case.

Still, native speakers generally 
confirm that there is something between the morpheme and the phrase, 
which people call 词.
In everyday speech, 
non-prosodic complex words have limited use, 
and monosyllabic words are rarely used 
because their formality (\prettyref{sec:pos.morpheme.primitive}).  
Thus, from \prettyref{fig:morpheme-to-phrase}, 
we find it's the class of prosodic morphosyntactic words 
that plays the role of ``words''
as a level between morphemes and phrases,
as well as intermediate units between morphemes and complex words.
Prosodic morphosyntactic words 
are also usually the final product 
of abbreviation (\prettyref{sec:pos.morpheme.abbreviation}),
and they observe frequent fossilization
(\prettyref{sec:pos.noun.fossilized-structure},
\prettyref{sec:pos.verb.fossilized-structure}).
These morphosyntactic prosodic words are therefore the nature response 
when native speakers without much linguistic training 
come up with intuitively when talking about ``words'' or 词,
i.e. intermediate building blocks.

The fuss around wordhood in Chinese 
arises from the split of several definitions of the term \term{word}:
wordhood defined by morphosyntactic test include is wider 
than the coverage of morphosyntactic prosodic words,
and the range of morphosyntactic words indeed 
has too much overlapping with morphemes and phrases.

\subsection{The functional-lexical distinction}

\begin{infobox}{Confusion related to the lexical-functional distinction}{lexical-functional-distinction}
    \citep[\citesec{3.6}]{zhudexigrammar} classifies 
    certain categories like locative particles % TODO: 方位词的正确翻译???
    into the nominal class and hence the lexical one,
    while the locative particle class can definitely be enumerated \citep[\citesec{4.4}]{zhudexigrammar}.
    On the other hand, 
    the author claims that lexical classes are always open 
    and function classes are always closed \citet[\citesec{3.4}]{zhudexigrammar}.
    A conflict thus occurs.
    We now see that's because he failed to see there are three parameters 
    related to the lexical-functional distinction.  
    Still, it's of course not easy to tell a newly discovered part of speech 
    (or \term{form class}, which may be a word class or an affix class)
    What's the status of an orientation preverb, 
    which may be found in Japhug \citep{jacques2021grammar}?
    It's a part of the grammar,
    but does it carry a real part-of-speech label (like ``directional adverb'')?
    And speaking of adverbs, what's the status of the English \form{allegedly}?
    An adverb filling a peripheral argument position,
    or an evidentiality marker?
    We really need to know a lot about language to fix the position of a form class.
    A common practice is just to shun the details and just say 
    whether a large word class is lexical or functional,
    drawing a somehow arbitrary hard line between the two.
    So \citet{zhudexigrammar} mainly uses the criterion of whether there is a real part-of-speech label,
    and then directive particles are classified into the nominal class 
    and they are in turn considered lexical.
    But he mistakenly confuses the notion of lexical classes with the notion of open classes,
    and then we get the self-conflicting asserts in \citet[\citesec{3.4}]{zhudexigrammar}.
\end{infobox}


\section{Part of speech labels}

\subsection{Word class labels: noun, verb, adjectives, etc.}

\subsubsection{The nominal-verbal division}

Lexical words in Chinese can be roughly divided into nominal ones and verbal ones,
or in the Chinese terms, 体词 and 谓词.
The prototypical role of nominal words 
is to fill predicate slots (or to be more precise, to head a phrase that fills an argument slot).
Nominal words rarely appear in the verbal complex,
though for stylistic purposes, they sometimes do.
Verbal words prototypically appear in the verbal complex
(\prettyref{chap:verbal-complex}),
but many of them -- and clauses without any morphological marking -- 
can regularly appear in argument slots \citep[\citesec{3.5}]{zhudexigrammar}.

Verbs can be negated by 不 while nouns generally cannot, 
and this is a criterion to tell verbs from nouns. TODO: others

The fact that verbal categories can fill argument slots or in colloquial words ``be used as nouns''
urges some to put the verbal categories under the nominal categories,
so thus there is only one mega lexical category in Chinese:
the nominal category or the Noun.
The analysis adopted here does not aim to organize lexical categories 
in a binary branching classification tree,
so the ordinary nominal-verbal distinction is maintained:
verbs being able to fill argument slots is not typologically rare, actually,
and this shared feature itself does not bring nouns and verbs close enough 
for them to be merged together.

\subsubsection{Two adjectival classes}

Whether Chinese has a separate adjective category 
has been debated for decades.
Based on a line of reasoning similar to the above verb-as-noun analysis,
some linguists argue that the so-called adjectives should be put under the verb category,
since they can fill the predicator slot without any morphological marking \citep{li1989mandarin}.
Since verbs and most alleged adjectives show different morphological behaviors in reduplication, % TODO: ref
the verb-adjective distinction is kept,
and the two are placed under the verbal category.

There still exist a (much smaller) number of alleged adjectives that shows 
different morphosyntactic properties with the adjectives in the verbal category 
\citep[\citechap{5}]{paul2014new}.
They can be marginally used as heads of \ac{np}s,
while they do not have reduplication variants.
These ``adjectives'' are thus placed under the nominal category.
Thus we have two types of adjectives.
In \citet{zhudexigrammar}, 
nominal adjectives are called 区别词 \translate{distinction word},
while verbal adjectives are called 形容词 \translate{adjective}.

\subsubsection{Other nominal categories}

There are more nominal categories than the ordinary noun category and the nominal adjective category.
Numerals, for examples, are in another nominal category.
Chinese has a rich classifier system,
and most classifiers still have strong nominal properties
and thus they constitute yet another nominal category.
\citet{zhudexigrammar} calls them 量词 \translate{measure word},
because many classifiers have the meaning of ``unit''.
There is also a locative particle class, including 里 in 在房子里,
which is sometimes said to be the postposition class
because they sometimes have adposition-like properties (TODO: ref: topicalization, and what else?).


\subsection{Summary: a tentative part of speech analysis}

\section{Nouns}

\subsection{Monosyllabic nouns}

\subsection{Nouns with historically analyzable inner structure}\label{sec:pos.noun.fossilized-structure}

The unit 白菜 is made up by two perfectly productive morphemes:
白 \translate{white} and 菜 \translate{vegetable},
but its meaning is not the composition of the two morphemes:
白菜 means \translate{Chinese cabbage}, not \translate{any vegetable with whitish appearance}.
The word has already gained a conventionalized meaning,
and its inner structure is of mostly diachronic interest but not synchronic interest.
Therefore, the disyllabic unit 白菜 is the smallest unit fed into morphosyntax,
and it of course is not a phrase.

Those insisting on the nonexistence of words in Chinese 
may explain the observation made above 
by claiming 白菜 to be an idiom \ac{np}:
it is indeed a lexical entry,
but is regarded as a pre-compiled phrase. TODO: 

Also: 大车, 大师傅, etc.

\subsection{Nominal bound morphemes}

\subsection{Compound nouns}\label{sec:pos.noun.compound}

From \eqref{ex:nominal-modifier-1} and \eqref{ex:nominal-modifier-2},
it can be seen in certain morphosyntax units,
a bare noun may serve as a (restrictive) modifier.
The constituent of Chinese \ac{np}s is Dem Num A N,
and this bare noun modifier position seems to be more internal than the adjective position,
as is illustrated by \eqref{ex:meiguo-hongse-pinguo} and \eqref{ex:hongse-meiguo-pinguo}.

\begin{exe}
    \ex 
    \begin{xlist}
        \ex\label{ex:nominal-modifier-1} {} [[定义]_{\text{modifier:N}} [[等价]_{\text{complement:adjective}} [性]_{\text{nominalizer}}]_{\text{N}}]_{\text{N}} \translate{equivalence of definitions}
        \ex\label{ex:nominal-modifier-2} {} [[美国]_{\text{modifier:N}} [苹果]_{\text{head:N}}]_{\text{N}}
        \ex\label{ex:meiguo-hongse-pinguo} *[美国]_{\text{modifier:N}} [红色的苹果]_{\text{NP}}
        \ex\label{ex:hongse-meiguo-pinguo} 红色的美国苹果
    \end{xlist}
\end{exe}

Furthermore, the bare noun position cannot be filled by an \ac{np}.
The following examples demonstrate this:
\begin{exe}
    \ex \begin{xlist}
        \ex {} [联合国] [秘书长]
        \ex *[[某个组织] [秘书长]]
        \ex 某个组织的秘书长
    \end{xlist}
\end{exe}
The obligatoriness of 的 means the NP 某个组织 can only appear as a modifier via the possessive construction.
It cannot fill the slot of 美国 in 美国苹果.
So the bare noun modifier position is a function label existing in a unit smaller than the \ac{np}
-- and it has to be the word.

The modifier position can also be filled by 
a disyllabic 

\begin{exe}
    \ex 染发行业
    \ex *染头发行业
\end{exe}

\subsection{Adjectival modification in complex noun}\label{sec:pos.noun.adj-modify}

小狗, 大盘子, as opposed to 大车

In \citet[\citepages{84-85}]{feng2000}, he emphasizes that 
a prosodic word doesn't have to be a grammatical word,
and specifically, attributive modification structures like 小狗 
(\prettyref{sec:pos.noun.adj-modify})
don't have to be words.
But he then precedes to say 
that these constructions are also not prototypical phrases
\citep[\citepages{85-86}]{feng2000}.
His conclusion is that these constructions 
are ``syntactic words'',
which, in the X-bar scheme, 
is smaller than a maximal projection but larger than terminal nodes.
Due to the lexical-decomposition stance of this note,
this analysis is discarded in this note;
I simply label them as grammatical words

\section{Verbs}

\subsection{Verbs with fossilized internal structures}\label{sec:pos.verb.fossilized-structure}

The verb 关心 \translate{care for} is certainly analyzable 
as a predicator-object structure,
but it takes objects just like any other verbs:
\begin{exe}
    \ex 他 [[[关]_{\text{predicator:V}} [心]_{\text{object:N}}]_{\text{predicator:V}} [自己的家人]_{\text{object:NP}}]_{\text{predicate:VP}}
\end{exe}
This means 关心 is not a VP but a grammatical word,
or otherwise it is impossible to take another object since there is no valency changing device in use.

\subsection{Disyllabic verb-object constructions}\label{sec:pos.verb.idiomatic-verb-object}

Unlike the case of \prettyref{sec:pos.verb.fossilized-structure}, 
there exist some disyllabic prosodic words 
which are verbal constituents 
and are often recognized as words. 
Examples of them include 看书, 做饭, 吃饭, etc.
These disyllabic verb-object constructions 
behave just like ordinary \acs{vp}s 
when the object is modified by, say, adjectival or nominal attributives;
when a semi-object is inserted between the verb and the object,
they also follow the same pattern observed in 
ordinary \acs{vp}s (\prettyref{sec:verb-phrase.ionization.similar}).
However, disyllabic verb-object constructions 
do seem to have one thing different with longer \acs{vp}s:
they can appear in compound nouns,
while the latter can't (\prettyref{sec:pos.noun.compound}).
Thus, the morphosyntactic tests outlined in this note decide 
that they are both words and phrases.

(\prettyref{ex:pos.verb.verb-object.nianfo-1}) gives an example of this duality.
The verb 念佛 in the first example is similar to 美国 in \eqref{ex:nominal-modifier-2}:
it serves as a bare modifier 
(since in Chinese verbal constituents can fill argument slots directly,
the fact that 念佛 is a verb is not surprising).
The fact 念佛 is able to appear in such a position assures us that 
it is a word.
Then consider \eqref{ex:nianfo-split}.
In VP1, a temporal semi-object is injected between the verb 念 and the object 佛,
while in VP2, an interrogative phrase 哪一尊 is inserted into 佛 
and an \ac{np} object is now taken by the verb 念.

\begin{exe}
    \ex\label{ex:pos.verb.verb-object.nianfo-1}
    \begin{xlist}
        \ex\label{ex:nianfo-tang} {} [念佛] 堂 
        \ex\label{ex:nianfo-split} 老太太 [念了这么久佛]_{\text{VP1}},却不知道自己在[念哪一尊佛]_{\text{VP2}}
    \end{xlist}
\end{exe}

The solution used in this note 
is to regard 念佛 in \eqref{ex:nianfo-tang} as a word,
while the two \acs{vp}s in \eqref{ex:nianfo-split} as phrases.
念佛 as in 老太太经常念佛 can be interpreted as a word or as a phrase
without making any difference.
念佛 as a word is something like \form{Buddha-chanting},
while 念佛 as a verb is something like \form{chant (the name of) Buddha}.
This agrees with the account of \citet[\citepage{82}]{feng2000},
in which 念佛 is a morphosyntactic word
(the original term being a 句法词 \translate{syntactic word}),
which is created by morphosyntactic rules 
and has a inner structure that is (partially) transparent 
for other morphosyntactic rules,
while 关心 is a \translate{lexical word} 
(not the same as \term{lexical word} in the rest of this note 
which means words that are not stored in the grammar),
which is taken out of the lexicon directly
and has no synchronically analyzable inner structure.

Note that we don't really need to set an intermediate layer 
between ordinary grammatical words and phrases:
the fact that disyllabic \acs{vp}s like 念佛 
are also grammatical words 
can be easily accounted for 
by dephrasalization.%
\footnote{
    As in \form{his [holier-than-thee] attitude} in English. 
}
The limit that only disyllabic \acs{vp}s 
can be grammatical words as well 
may have a purely prosodic explanation:
to make prosodic processing of a grammatical word easier,
Mandarin has a strong tendency to dephrase 
only units that are prosodic words. 
This also agrees with the claim in \prettyref{sec:pos.word.phonological}
that all prosodic words with morphosyntactic significance 
are indeed grammatical words 
and are never \emph{only} phrases.

Various degrees of fossilization exist for verb-object constructions. 
In some of them, the object is no longer frequent 
in ordinary speech, 
but can still receive modification within the verb-object construction, 
and sometimes can be moved out of the verb-object construction
(\ref{ex:pos.morpheme.chikui}, \ref{ex:pos.verb.verb-object.fossilize-1}); 
some verb-object constructions have almost completely fossilized 
and act as a whole in synchronic syntax 
(\ref{ex:pos.verb.verb-object.fossilize-2},
\ref{ex:pos.verb.verb-object.fossilize-3}). 
These historical verb-object constructions usually have restricted abilities 
(\ref{ex:pos.verb.verb-object.fossilize-2}b)
to have the object modified; 
insertion of a semi-object is usually possible 
(\ref{ex:pos.verb.verb-object.fossilize-2}c),
which is not surprising since this involves verb ionization, 
which does not consider the internal structure of the verb 
(\prettyref{ex:pos.verb.verb-object.fossilize-2}).
Some constructions can be analyzed both 
as semi-object insertion 
and object modification 
(\ref{ex:pos.verb.verb-object.fossilize-3}b). 

\begin{exe}
    \ex\label{ex:pos.verb.verb-object.fossilize-1} \begin{xlist}
        \ex 我今天吃了这个亏
        \ex 这个亏我今天吃了,但是你们之后别再想从我这里拿到一分钱好处!
    \end{xlist}

    \ex\label{ex:pos.verb.verb-object.fossilize-2} \begin{xlist}
        \ex 我们今年招了一个很有才干的学生
        \ex *我们今年招了一个很有才干的生
        \ex 我们今年计划招两次生
    \end{xlist}

    \ex\label{ex:pos.verb.verb-object.fossilize-3} \begin{xlist}
        \ex 她正在起草一份报告
        \ex 先起个草再说,别管是不是写得下去
    \end{xlist}
\end{exe}

In conclusion, 
disyllabic verb-object constructions, no matter historical or synchronic, 
are classified as \prettyref{fig:verb-object-classification},
which their full \acs{vp} versions, if any, illustrated. 
It can be seen that all disyllabic verb-object constructions, 
historical or synchronic, 
can be seen as grammatical words in some circumstances; 
without fossilization, 
the object of a disyllabic verb-object construction 
can be arbitrarily modified and we get a \acs{vp}; 
with fossilization, 
semi-object insertion is always possible, 
while for forms like 起草, 
whether the extended version 起个草
comes from object modification or semi-object insertion 
is not clear. 
The terminology in the figure
is introduced in \citet[\citepages{128-129}]{zhudexigrammar}.
\citet{zhudexigrammar} recognizes three classes of grammatical structures:  
组合式 \translate{composition-style}, 粘合式 \translate{gluing-style}, 
and 复合词式 \translate{compound word-style}:%
\footnote{
    However, it's necessary to realize Zhu's own analysis 
    of wordhood of verb-object constructions 
    is flawed. 
    It seems in Zhu's analysis, 
    it has to be a grammatical word in this case (and hence a ``compound word-style construction''); 
    this however is refuted by the example of 吃亏 
    (\ref{ex:pos.morpheme.chikui}, \ref{ex:pos.verb.verb-object.fossilize-1}).

    He also seems to think that if a structure contains free morphemes 
    and doesn't have a highly established meaning, 
    then it has to be a phrase,
    and hence we have the distinction 
    between ``glued-up construction'' and ``compound word-style construction'',
    the first being made of free morphemes. 
    This is a confusion between wordhood and fossilization.
    I define wordhood purely from the perspective of structural analysis 
    (ability to appear as a part of a compound word, etc.), 
    and from this perspective, 
    when used as a grammatical word, 
    a glued-up construction doesn't appear to be quite different 
    from a compound word-style construction. 
    Indeed, Zhu recognizes that the syntactic behaviors of 
    the glued-up construction and the compound word-style construction 
    are highly close to each other.
}
\begin{itemize}
    \item A composition-style verb-object construction is just a \acs{vp}.
    \item A glued-up structure 
    is a regularly formed structure 
    containing elements smaller than usual phrases 
    -- possibly bare grammatical words --
    and can serve as a verb phrase itself
    (\ref{ex:pos.verb.verb-object.nianfo-1}b),
    or as a grammatical word, 
    probably after dephrasalization 
    (\ref{ex:pos.verb.verb-object.nianfo-1}a); 
    this includes the case of 念佛 or 吃饭.
    \item A compound word-style structure contains 
    at least one bound morpheme as one of its immediate constituent, 
    which includes the case of both 吃亏, 起草, and 招生,
    but the latter two are even more fossilized
    and belows to \prettyref{sec:pos.verb.fossilized-structure}.
\end{itemize}


\begin{figure}[H]
    \centering
    \caption{Classification of disyllabic verb-object constructions, 
    \label{fig:verb-object-classification}
    fossilized and synchronic}
    {\small \begin{tikzpicture}[x=0.75pt,y=0.75pt,yscale=-1,xscale=1]
    %uncomment if require: \path (0,478); %set diagram left start at 0, and has height of 478
    
    %Shape: Rectangle [id:dp8523951309460049] 
    \draw  [draw opacity=0][fill={rgb, 255:red, 74; green, 144; blue, 226 }  ,fill opacity=0.2 ] (69,36.6) -- (91.17,36.6) -- (91.17,256.27) -- (69,256.27) -- cycle ;
    %Shape: Rectangle [id:dp5406916790863352] 
    \draw  [draw opacity=0][fill={rgb, 255:red, 80; green, 227; blue, 194 }  ,fill opacity=0.2 ] (69,184) -- (91.17,184) -- (91.17,336.27) -- (69,336.27) -- cycle ;
    %Straight Lines [id:da159005947041438] 
    \draw [color={rgb, 255:red, 74; green, 181; blue, 226 }  ,draw opacity=1 ][line width=1.5]  [dash pattern={on 1.69pt off 2.76pt}]  (91.17,184) -- (321.17,184) ;
    %Straight Lines [id:da5149018356050215] 
    \draw [color={rgb, 255:red, 80; green, 227; blue, 214 }  ,draw opacity=1 ][line width=1.5]  [dash pattern={on 1.69pt off 2.76pt}]  (91.17,256.27) -- (321.17,256.27) ;
    %Straight Lines [id:da016885432514876397] 
    \draw [color={rgb, 255:red, 80; green, 199; blue, 227 }  ,draw opacity=1 ][line width=1.5]  [dash pattern={on 1.69pt off 2.76pt}]  (468.17,380.32) -- (468.17,164.6) ;
    \draw [shift={(468.17,160.6)}, rotate = 90] [fill={rgb, 255:red, 80; green, 199; blue, 227 }  ,fill opacity=1 ][line width=0.08]  [draw opacity=0] (9.91,-4.76) -- (0,0) -- (9.91,4.76) -- (6.58,0) -- cycle    ;
    %Shape: Rectangle [id:dp574744310659201] 
    \draw  [draw opacity=0][fill={rgb, 255:red, 184; green, 233; blue, 134 }  ,fill opacity=0.1 ] (69,336.27) -- (91.17,336.27) -- (91.17,464.6) -- (69,464.6) -- cycle ;
    %Straight Lines [id:da10122532545557017] 
    \draw [color={rgb, 255:red, 112; green, 230; blue, 138 }  ,draw opacity=1 ][line width=1.5]  [dash pattern={on 1.69pt off 2.76pt}]  (91.17,336.27) -- (321.17,336.27) ;
    %Shape: Rectangle [id:dp1531409107055255] 
    \draw  [draw opacity=0][fill={rgb, 255:red, 189; green, 16; blue, 224 }  ,fill opacity=0.1 ] (124,88.14) -- (342.4,88.14) -- (342.4,107) -- (124,107) -- cycle ;
    %Shape: Rectangle [id:dp759439411401527] 
    \draw  [draw opacity=0][fill={rgb, 255:red, 126; green, 211; blue, 33 }  ,fill opacity=0.1 ] (412,88.14) -- (523,88.14) -- (523,107) -- (412,107) -- cycle ;
    %Shape: Rectangle [id:dp7746367637292535] 
    \draw  [draw opacity=0][fill={rgb, 255:red, 33; green, 211; blue, 193 }  ,fill opacity=0.1 ] (342.4,88.14) -- (412,88.14) -- (412,107) -- (342.4,107) -- cycle ;
    
    % Text Node
    \draw (66,38.6) node [anchor=north west][inner sep=0.75pt]  [color={rgb, 255:red, 74; green, 144; blue, 226 }  ,opacity=1 ,rotate=-90] [align=left] {verb phrase};
    % Text Node
    \draw (66,334.27) node [anchor=north east] [inner sep=0.75pt]  [color={rgb, 255:red, 80; green, 227; blue, 194 }  ,opacity=1 ,rotate=-90] [align=left] {\begin{minipage}[lt]{46.89pt}\setlength\topsep{0pt}
    \begin{center}
    ``compound word-like''
    \end{center}
    
    \end{minipage}};
    % Text Node
    \draw (119,131) node [anchor=north west][inner sep=0.75pt]   [align=left] {吃一顿饭};
    % Text Node
    \draw (132,212) node [anchor=north west][inner sep=0.75pt]   [align=left] {吃饭};
    % Text Node
    \draw (240.5,290) node [anchor=north west][inner sep=0.75pt]   [align=left] {吃亏};
    % Text Node
    \draw (225,131) node [anchor=north west][inner sep=0.75pt]   [align=left] {吃一个大亏};
    % Text Node
    \draw  [draw opacity=0][fill={rgb, 255:red, 255; green, 255; blue, 255 }  ,fill opacity=1 ]  (162.67,172) -- (249.67,172) -- (249.67,196) -- (162.67,196) -- cycle  ;
    \draw (206.17,184) node  [color={rgb, 255:red, 74; green, 181; blue, 226 }  ,opacity=1 ] [align=left] {dephrasalize};
    % Text Node
    \draw  [draw opacity=0][fill={rgb, 255:red, 255; green, 255; blue, 255 }  ,fill opacity=1 ]  (140.17,244.27) -- (272.17,244.27) -- (272.17,268.27) -- (140.17,268.27) -- cycle  ;
    \draw (206.17,256.27) node  [color={rgb, 255:red, 80; green, 227; blue, 214 }  ,opacity=1 ] [align=left] {(syntactic) fossilize};
    % Text Node
    \draw (451.5,389.6) node [anchor=north west][inner sep=0.75pt]   [align=left] {招生};
    % Text Node
    \draw (233.2,85.57) node [anchor=south] [inner sep=0.75pt]   [align=left] {\begin{minipage}[lt]{56.24pt}\setlength\topsep{0pt}
    \begin{center}
    verb-object \\structure
    \end{center}
    
    \end{minipage}};
    % Text Node
    \draw (408.4,48.8) node [anchor=north west][inner sep=0.75pt]   [align=left] {\begin{minipage}[lt]{100pt}\setlength\topsep{0pt}
    \begin{center}
    verb-semi object \\structure
    \end{center}
    
    \end{minipage}};
    % Text Node
    \draw (432,131) node [anchor=north west][inner sep=0.75pt]   [align=left] {招了一次生};
    % Text Node
    \draw (66,186) node [anchor=north west][inner sep=0.75pt]  [color={rgb, 255:red, 74; green, 144; blue, 226 }  ,opacity=1 ,rotate=-90] [align=left] {\begin{minipage}[lt]{54.82pt}\setlength\topsep{0pt}
    \begin{center}
    ``glued-up'' \\structure
    \end{center}
    
    \end{minipage}};
    % Text Node
    \draw (471.17,270.46) node [anchor=south] [inner sep=0.75pt]  [color={rgb, 255:red, 80; green, 199; blue, 227 }  ,opacity=1 ,rotate=-90] [align=left] {\begin{minipage}[lt]{56.84pt}\setlength\topsep{0pt}
    \begin{center}
    semi-object \\insertion
    \end{center}
    
    \end{minipage}};
    % Text Node
    \draw (67.75,403.51) node [anchor=north] [inner sep=0.75pt]  [color={rgb, 255:red, 184; green, 233; blue, 134 }  ,opacity=1 ,rotate=-90] [align=left] {\begin{minipage}[lt]{76.94pt}\setlength\topsep{0pt}
    \begin{center}
    synchronically \\mono-morpheme\\verb
    \end{center}
    
    \end{minipage}};
    % Text Node
    \draw (351.7,389.6) node [anchor=north west][inner sep=0.75pt]   [align=left] {起草};
    % Text Node
    \draw (349.6,131) node [anchor=north west][inner sep=0.75pt]   [align=left] {起个草};
    % Text Node
    \draw  [draw opacity=0][fill={rgb, 255:red, 255; green, 255; blue, 255 }  ,fill opacity=1 ]  (140.17,324.27) -- (272.17,324.27) -- (272.17,348.27) -- (140.17,348.27) -- cycle  ;
    \draw (206.17,336.27) node  [color={rgb, 255:red, 112; green, 230; blue, 138 }  ,opacity=1 ] [align=left] {(syntactic) fossilize};
    % Text Node
    \draw (368.4,57.4) node [anchor=north west][inner sep=0.75pt]   [align=left] {?};
    
    
    \end{tikzpicture}
    }
\end{figure}

The above discussion is all about syntactic aspects. 
Semantic fossilization is an orthogonal parameter; 
吃饭, despite being a ``glued-up'' structure, 
has an established meaning of \translate{eating something}, 
and not just eating rice; 
吃亏, on the other hand, is semantically compositional, 
although 亏 is infrequent as a noun in ordinary speech. 

\subsection{Idiomatic clauses as a predicate}

Some clauses (usually idioms), like 大鱼吃小鱼, or 你看看我我看看你,
despite having perfectly analyzable internal structures, 
are used collectively as a \emph{verb} (\prettyref{sec:clause.dangling-topic.1}).

\section{Locational words}\label{sec:pos.locational}

Monosyllabic location words like 前 may be analyzed as words 
because they can be attached to an arbitrary \ac{np} 
to denote a place near the place denoted by that \ac{np},
as in [[那座老旧的房子 [前]_{\text{location word}}]_{\text{NP}} 有一口井]_{\text{clause}},
and what only appears as immediate constituents of phrases are of course words,%
\footnote{
    I'm talking about morphosyntactic words here.
    It is possible that something is a morphosyntactic word 
    appearing at the level of phrase
    is incorporated into a word nearby phonologically,
    though this is not the case in Chinese.
}%
but they never appear independently as \ac{np}s.


\section{Prepositions}\label{sec:preposition-pos}

Though all Mandarin prepositions have verb origins 
and therefore may be classified as a subclass of verbs by some,
it's necessary to distinguish a separate preposition class.
Criteria of prepositions include 
being able to be deleted in topicalization
(\prettyref{sec:topicalization-of-preposition-objects}), 

\begin{infobox}{The term \term{coverb}}{coverb}
    In 
\end{infobox}

\section{Other grammatical ``coverbs''}

TODO: 把, 被, etc.; put the surface constituent orders here 
and link them to the chapter about the verb phrase

\chapter{Noun phrase}

No morphological case, number, and gender categories are attested in Mandarin.
There is a word class system or in other words classifier system, however.
In most cases when a numeral appears in an \ac{np},
a classifier follows immediately after the numeral.
Attributives -- both adjectives and relative clauses -- 
follow the classifier. % TODO: 红色的三个,这样的说法说得通吗?
The demonstrative, if any, appears before the numeral,
and even when there is no numeral,
there is frequently also a classifier.

The template of \ac{np}s, therefore, belongs to the 
Dem-Num-A-N type,
with the classifier residing between Num and A. 

\chapter{The verbal complex}\label{chap:vp.verbal-complex}

Although Mandarin is generally regarded as a prototypical analytic language,
no traditionally acknowledged verb inflections,
several highly productive (as opposed to arguable historical derivation) 
verbal affixation devices have been attested.
Several other morphological devices,
including reduplication and verb copying,
are also attested.  

The Mandarin verb sees more active morphosyntactic operations  
compared with prototypical verb morphology.
the verb can split (\prettyref{sec:verb-splitting})
the ``verbal complement'' component of it 
can be released from the verb morphological template
in certain occasions 
(\prettyref{sec:vp.direction}),
and the order of the aspectual marker and the disyllabic directional complement 
may change to meet prosodic requirements 
(\prettyref{sec:vp.direction}, \ref{ex:sanpinqishuiugolai-3}).

The structure of the verbal complex is highly intertwined 
with verb valence, 
and therefore describing the verbal complex on its own 
(which is identified as the verb phrase by some)
is not feasible.
The \acs{vp} 
has to be analyzed as a whole.

\section{Affixational morphology}\label{sec:vp.verb.affix}

Affixational morphology in the verbal complex, from left to right in the template, 
are listed as follows:
\begin{enumerate}
    \item Inside the verb stem, 
    we have a derivation system, 
    like 化 \translate{-ize} (TODO: ref). 

    \item The ``verbal complement'' slot,
    which may contain a verbal complement (\prettyref{chap:verbal-complement}),
    or the preposition part of the prepositional complement 
    (\prettyref{sec:verb-phrase.internal.preposition}),
    or nothing, but never more than one. 

    It should be noted that 
    verbal complements have limited mobility 
    and therefore are not always realized as suffixes 
    (\prettyref{sec:vp.direction}, \ref{ex:movable-suffix}). 
    We may analyze the slot as incorporation. 

    \item The aspectual system (\prettyref{sec:aspectual})
    are also usually realized as suffixes in the verbal complex.
    Note that not all aspectual markers are realized as suffixes:
    the progressive 正在 is not (TODO: ref).
    Also, some verbal complements seem to have already grammaticalized 
    into aspectual markers, 
    which means we may have at most two -- instead of one -- aspectual markers 
    in a verbal complex. (TODO: 吃上了饭了)
\end{enumerate}

\eqref{ex:hua-wan-le-1} is an example in which 
all the three systems appear.
In real world speeches, such combinations have relatively lower distributions,
possibly because of the prosodic constraint 
that verb shouldn't be too heavy unless it appears at the end of a clause
(\prettyref{sec:vp.prosody}).
In this example, the verbal complex 数字化完了 
splits into two prosodic words.

\begin{exe}
    \ex \dots 并且企业 [数字 [化]_{\text{derivation}} [完]_{\text{complement}} [了]_{\text{aspectual}}]_{\text{V}} 之后还不一定赚钱 \dots
    \label{ex:hua-wan-le-1}
\end{exe}




In some occasions, a personal pronoun can also be 
phonologically incorporated into the verbal complex,
between the verbal complex position 
and the aspectual marker
(\ref{ex:vp.verbal-complex.incorporation-1},
\ref{ex:vp.verbal-complex.incorporation-2}); 
this is less acceptable when the pronoun is replaced by a longer \acs{np}
(\ref{ex:vp.verbal-complex.incorporation-3}).

\begin{exe}
    \ex\label{ex:vp.verbal-complex.incorporation-1} \begin{xlist}
        \ex Ordinary order 
        \gll 他 [给 了] 我 三 本 书 \\ 
        3 give \category{asp} 1 three \category{cl} book \\
        \glt \translate{He gave me three books.}
        
        \ex Personal pronoun incorporation
        \gll ?他 [给 我 了] 三 本 书 \\
        3 give 1 \category{asp} three \category{cl} book \\
        \glt \translate{He gave me three books.}
    \end{xlist}
    
    \ex\label{ex:vp.verbal-complex.incorporation-2} \begin{xlist}
        \ex 他送给了我三本书
        \ex 他送了给我三本书
    \end{xlist}

    \ex\label{ex:vp.verbal-complex.incorporation-3} \begin{xlist}
        \ex 他给了学生三本书 
        \ex *他给学生了三本书
    \end{xlist}
\end{exe}



\section{Verb ionization}\label{sec:verb-splitting}

\subsection{Classification}

It's sometimes possible to split a verb 
and inject some clausal dependents into it,
while the two parts are not morphosyntactic constituents at all.
This phenomenon is known as 
\concept{verb separation} or \concept{verb ionization} 
(\citealt[\citesec{6.5.8}]{chao1965grammar};
\prettyref{sec:verb-splitting}).
The output of verb ionization is an \acs{vp}
with greatly reduced dephrasalization ability. 
Possible injected clausal dependents include 
\begin{itemize}
    \item the verbal complement (\prettyref{ex:junwanlexun}),
    \item the semi-object
    (\prettyref{ex:guanshenmexin}; TODO: ref; note the relation with pseudo-attributive),
    \item the pseudo-attributive (\prettyref{ex:verb-phrase.separation.xuexi}; TODO: ref);
    but in marginal cases, and 
    \item a personal pronoun which is the object (\prettyref{ex:guanshenmexin}). 
\end{itemize}


\begin{exe}
    \ex\label{ex:junwanlexun} 
    \gll \% [军 完 了 训]_{\text{\acs{vp}}} 以后 才 可以 去 请 护照 \\
    {} military finish \category{asp} training after only.after(TODO) can go ask.for passport \\
    \glt \translate{(We) can only apply for a passport after finishing military training.} 
    \citet[\citesec{6.5.8}]{chao1965grammar}
    \ex\label{ex:youmo} \% 还 [幽了他一默]_{\text{\acs{vp}}}
    \ex\label{ex:guanshenmexin} 这件事情你 [关什么心]_{\text{\acs{vp}}} 啊
    \ex\label{ex:verb-phrase.separation.xuexi} 我[学了两个小时的习]_{\text{\acs{vp}}}
\end{exe}

It's usually possible to move the inserted constituent out
(\prettyref{ex:verb-phrase.separation.junxun-2},
\prettyref{ex:verb-phrase.separation.guanxin-2},
\prettyref{ex:verb-phrase.separation.xuexi-2}), 
except in (\prettyref{ex:youmo}, TODO: general condition).
When this is possible, 
the resulting structure has exactly the same meaning 
with the verb separation structure.

\begin{exe}
    \ex\label{ex:verb-phrase.separation.junxun-2} 军训完了以后才可以去请护照
    \ex\label{ex:verb-phrase.separation.guanxin-2} 这件事你关心什么啊
    \ex\label{ex:verb-phrase.separation.xuexi-2} 我学习了两个小时
\end{exe}



Verb ionization is compatible with the ordinary affixation processes: 
when verb ionization happens, 
the first half of the verb plays the role of the verb stem in \prettyref{sec:vp.verb.affix}:
in (\ref{ex:junwanlexun}), for example, 
the first half of the verb 军训 
is glued together with the inserted resultative complement 完,
followed by the aspectual marker 了;
we may say \prettyref{sec:vp.verb.affix} happens after verb ionization
in the pipeline of Mandarin verbal morphology.



TODO: review the following; 
possible mechanism of object postponing??

Insertion of a resultative complement 
seems to be only available when the verb is intransitive,
so the resulting structure is also an \acs{vp} on its own.%
\footnote{
    And therefore whether 军完了训 is directly a \ac{vp} or is first a verb complex 
    and then a \ac{vp} is not of much importance.
}




The acceptability of this structure 
varies among people 
(\prettyref{sec:verb-splitting}).


\subsection{Motivation of verb ionization}

The splitting of the verbs clearly origins 
by analogy with \ac{vp}s containing morphosyntactic words.
军完了训 is apparently created by analogy with 
[[吃完了]_{\text{predicator:verb complex}} [饭]_{\text{object:N}}]_{\text{VP}}.
The motivation of this analogy seems to be prosody: 
splitting words into phrases is only observed in \ac{vp}s,
and \ac{vp}s are subject to the prosodic constraint 
that the neither the verb nor the final complement can be too light.
Splitting the verb may help to reduce the ``weight'' of the verb 
so the resulting utterance meets the prosodic constraint better.

The phenomenon of verb ionization
looks just like infixing,
in which a word is split even when it has no analyzable morphosyntactic inner structure.
Here, however, this infixing operation creates an \acs{vp} instead of a grammatical word.
This justifies the assumption taken at the end of \prettyref{sec:theory}
that there is no clear boundary between words and phrases 
and therefore syntax and morphology:
It's possible for a phrase to undergo 
rearrangement without clear syntax motivation
that usually happens within a word.

Verb ionization sits at the boundary of morphology and syntax.
It involves alternation of the shape of the verb 
and is therefore morphological; 
but what is injected into the verb is a clause complement  
and the resulting morphosyntactic unit is a phrase,
so verb ionization is also syntactic. 
This is an illustration of 

\subsection{Comparison with similar constructions}\label{sec:verb-phrase.ionization.similar}

\subsubsection{Modification of the object in verb-object construction}

Some constructions classified as verb ionization 
seem to be analyzable as 
ordinary object modification
(\ref{ex:verb-complex.ionization.similar.object-modify-1}).
Some verb phrase idioms 
contain an internal object that usually doesn't appear as a full \acs{np}, 
but that is not absolute
(\ref{ex:verb-complex.ionization.similar.object-modify-2}).
These constructions are excluded from the category of verb ionization,
although as is shown above, 
they may historically motivate the emergence of verb ionization.

\begin{exe}
    \ex\label{ex:verb-complex.ionization.similar.object-modify-1} \begin{xlist}
        \ex 我喜欢看书 
        \ex 我看了三本书
    \end{xlist}
    
    \ex\label{ex:verb-complex.ionization.similar.object-modify-2} \begin{xlist}
        \ex 我洗了一个舒服的澡 
        \ex ???一个澡对睡眠有好处 
        \ex 一个舒服的澡对睡眠有好处.
    \end{xlist}
\end{exe}

\subsubsection{Pseudo-attributive construction}

Other idiomatic verb phrases 
seem to be extendable by 
both modification within the object 
and verb ionization.
The prosodic word 念佛 may be extended into 念阿弥陀佛 (extending the object)
as well as 念三声佛 (inserting a semi-object, verb ionization).
This happens for 染发 as well: 
we have both 染了一头蓝发 (extending the object) 
and 染了一次头发.
Since the semi-object can also intervene
between the verb and the direct object in uncontroversial \acs{vp}s 
(\prettyref{ex:verb-phrase.separation.nianfo-full-vp}), 
it seems disyllabic (and therefore prosodic) verb-object constructions 
have the same behaviors with 
longer verb-object constructions.

\begin{exe}
    \ex\label{ex:verb-phrase.separation.nianfo-full-vp} 老太太念了十多年的阿弥陀佛,却说不清阿弥陀佛是谁
\end{exe}

\section{Reduplication}

Reduplication of verb marks the delimitative aspect, 
i.e. `do something but not with an excessive amount'.

\begin{exe}
    \ex 拜了一拜
    \ex 看了他一看
\end{exe}

\section{Verb copying}

All types of verbal complement
may obligatorily or optionally trigger double occurrences of the main verb,
knowing as verb copying. 

This construction is understood as a 

\begin{exe}
    \ex 你找这本书找到了吗?(~你找到这本书了吗?)
    \ex 他爬山永远爬不上去。(~\%他永远爬不上去山。)
    \ex 我找这本书就是找不到。(~我就是找不到这本书。)
    \ex 我写文章写不出来啊。(~我写不出来文章啊。)
    \ex 我打球打得胳膊酸痛。(*我打球得胳膊酸痛。)
    \ex 我打球打得篮筐坏了。(*我打球得篮筐坏了。)
    \ex 熊打他连续打了三巴掌。(~熊连续打了他三巴掌。)
    \ex 我敲桌子敲了两下。(~我敲了两下桌子。)
\end{exe}

\subsection{Comparison with adverbials}

\begin{exe}
    \ex \begin{xlist}
        \ex 我吃西瓜喜欢吃红瓤的
        \ex 我吃西瓜专门挑红瓤的
    \end{xlist}
\end{exe}

\section{Negation}\label{sec:vp.neg}
 
There is no negative concord in Mandarin Chinese.
There is, however, no uniform negation operator like the English \emph{not}. 
Mandarin has two attested negators: 不 and 没.
不 is always used together with the habitual aspect
(\ref{ex:vp.negation.bu-1}).
没 is used together with a non-habitual aspect
(\ref{ex:vp.negation.mei-1}).

\begin{exe}
    \ex\label{ex:vp.negation.bu-1} \gll 他 平时 不 吃 猪肉 \\ 
    3 ordinary.day \category{neg} eat pork \\ 
    \glt \translate{He usually doesn't eat pork (for religious reasons, or for personal taste, etc.)}

    \ex\label{ex:vp.negation.mei-1} \gll 我 那 顿 饭 没 吃 猪肉 \\
    1 that \category{cl} meal \category{neg} eat pork \\ 
    \glt \translate{I didn't eat pork in that meal.}
\end{exe}

When 不 is used together with a potential complement, 
we need to remove 得 and insert 不 in its position

\begin{exe}
    \ex 我 不 是 算 不 清楚 账, 但是 那 天 不 知 怎么 的 就是 没 算 清楚 帐
\end{exe}


\chapter{Verb valency}\label{chap:vp.argument}


我丢了手机、我把手机丢了 is 我 introduced by \category{cause}?

\section{Overview}


A well-known cross-linguistic generalization of argument structure 
is that ambitransitive verbs can be divided into S=O ones 
and S=A ones;
the former are usually related to (change of) state, 
the latter are related to a ``spontaneous'' action.
Intransitive verbs can be classified according to their 
resemblance to the S=A case or the S=O case, 
and transitive verbs can be classified similarly. 
Thus, concerning the subject and the object%
\footnote{
    Not including so-called semi-objects 
    -- their behaviors seem to be largely orthogonal to the content of this section.
}
of \acs{vp}s, 
intransitive and monotransitive verbs
may be prototypically divided into the following contrasting classes 
\citep[\citechap{6}]{deng2010formal}:%
\footnote{
    An intransitive verb from the fourth category
    is often called an \concept{unaccusative verb}, 
    and in contrast, 
    intransitive verbs belonging to the second type 
    are \concept{unergative verbs}.
    The terminology is unfortunately confusing
    because this has nothing to do with alignment:
    an ergative language can still have unergative verbs.
    Since this note pretends to be a descriptive one, 
    below I will try to use terms like ``a verb frequently appears in the \category{become} structure''
    in place of ``an unaccusative verb'',
    which also agrees with the lexical-decomposition flavor of 
    my theoretical commitment better. 
}
\begin{itemize}
    \item The \category{be} type, describing a static state, with one argument
    (\prettyref{sec:verb-phrase.be}).

    \item The intransitive \category{do} type, describing a dynamic event, with one agentive argument.

    \item The monotransitive \category{do} type, describing a dynamic event,  
        with one agentive argument and 
        and one patientive argument.
        In both \category{do} constructions, 
        the agentive argument always goes to the subject position 
        and therefore goes out of the \acs{vp}
        (\prettyref{sec:verb-phrase.do.standard}).
        
        A verb allowing alternation between the two \category{do} frames 
        is a S=A ambitransitive verb. 
        
    \item The \category{become} type, describing a dynamic event, 
        with one argument being the participant of this event 
        and the ``state-transition'' caused by the event (\prettyref{sec:vp.become}). 
        The argument may be agentive or patientive semantically.
        The argument is raised to the subject position.
        
    \item The \category{cause-become} type, a dynamic event, 
        with one argument being the causer, 
        and the other argument being the participant of this event.
        The causer is raised to the subject position
        (\prettyref{sec:verb-phrase.cause-become.ordinary}).
        
        A verb allowing alternation between the two \category{become} frames 
        is a S=O ambitransitive verb.
\end{itemize}
Certain gradience exists in the distinction between the aforementioned 
verb valency classes 
\citep{lin2021unaccusativity,huang2007}.
The five types listed above are to be understood as verb frames, 
into which verb stems can be inserted, 
instead of inherent properties of verbs. 

Double-object \acs{vp}s may be divided into 
two subclasses
\citep[\citesec{7.2}]{deng2010formal}:
\begin{itemize}
    \item A \category{become} construction about giving something,
        with one agentive argument, 
        one receiver and one theme (TODO: elaborate).
        The receiver is not syntactically active 
        in valency changing. 
    \item A \category{do} construction event with a meaning of obtaining something,
        with one agentive argument corresponding to the receiver, 
        one argument similar to the TODO: experiencer? What's the role? that corresponds to the source, 
        and one argument about the object being transferred.
\end{itemize}

Mandarin is known for its flexible valency alternation
\citep[\citepage{76}]{huang2018handbook}, 
which is described in detail in the following sections.
Mandarin valency changing has no morphological marking on the verb;
valency changing is inferred from the unusual semantic roles of clausal complements.
Thus, distinguishing between 
valency changing and information structure operations like topicalization 
is often challenging.
The case is made even worse by the prevalent idea that the topic 
is freely occupied by any semantic (and not necessarily syntactic) argument in the clause,
an analysis that essentially rejects the idea 
of grammaticalized argument structure, 
though this claim can be falsified by detailed syntactic tests 
(\prettyref{sec:topic-subject}).
Some constructions that have triggered endless debates 
which include the \category{experience} construction 
(\prettyref{sec:verb-phrase.experience}, commonly known as the 王冕死了父亲 structure),
and several types of existential constructions TODO: 台上坐着主席团
However, under in-depth scrutiny, 
all these constructions can be reduced into argument structure alternation 
and alternative ways to realize the verbal complex.
Valency alternation depends on the properties of the verb:
for example, some verbs are always transitive 
and can never be intransitivized
(\prettyref{sec:verb-phrase.cause-become.ordinary}). 

Verb valency or in other words verb frame
is of great importance in Mandarin grammar. 
Apart from the S=O/A distinction 
and the action/state/change of state semantic distinction, 
the follows can also be observed:
\begin{itemize}
    \item Realizational properties:     
        an example is that for a stative verb selecting a prepositional complement,
        the prepositional complement follows the verb 
        and the preposition is incorporated into the verb 
        (\prettyref{sec:verb-phrase.internal.preposition}),
        while appearance of the preposition after the verb 
        is not possible for an action verb
        (\prettyref{sec:verb-phrase.do.standard}). 

    \item Constraint on valency changing: 
    a \category{be} verb or adjective can only be converted to 
    a \category{become} one (\prettyref{sec:vp.be-become}); 
    the instrumental or locational object construction 
    is restricted to \category{do} verbs (\prettyref{sec:vp.oblique-incorporation}); 
    the ``experiencing'' construction is restricted to 
    \category{become} verbs (\prettyref{sec:verb-phrase.experience}).

    \item Correlation between the argument structure 
    and the lexical aspect of \acs{vp}s
    \citep{laws2010core,toratani1997typology,aljovic2000unaccusativity}:
    change of state verbs tend to be telic, 
    because it describes a transition of states 
    and therefore the event denoted is bounded by definition.
    The lexical aspect then interacts non-trivially with 
    the aspect system (\prettyref{sec:aspectual}).
    
    \item Appearance in seemingly non-SVO constructions (\prettyref{sec:verb-phrase.object.ba},
    \prettyref{sec:verb-phrase.object.short-bei}, 
    \prettyref{sec:verb-phrase.bei}).
\end{itemize}
Therefore the verb frame largely decides the structure 
of the nucleus clause (\prettyref{chap:simple-clause}).

\begin{exe}
    \ex 他们表扬了我
    \ex 我被表扬了
    \ex ??他们把我表扬了
\end{exe}




The patientive argument and the TODO: numeral object like 他抢了我[一块钱] 
are not active for further processing; 
thus the \acs{vp} can be further divided into two layers.
TODO: 被 construction: what happens to the so-called passivized argument?
Evidences suggest that the intransitive object is very internal, 
appearing in almost the same position with complement clause, etc.
but why is it subject to 被 construction? 
TODO: for short 被-construction, 
maybe it's because short 被-construction is generated by directly attaching 被 to, say, [抢了一块钱]: 
我被抢了一块钱 \translate{lit. I suffer from one-dollar robbing}
李四被捕了 doesn't have a counterpart without 被: 
捕 here may be regarded as a deponent verb. 
This also implies that 被 has already been grammaticalized 
and is no longer a lexical verb.




\section{Stative constructions}\label{sec:verb-phrase.be}

\subsection{Adjective predicator}

One intriguing trait of Mandarin is that 
when the predicator is an adjective but is  
without a degree adverbial,
the clause is considered problematic
as a matrix clause,
and yet is perfectly fine as a subordinated clause.

\begin{exe}
    \ex ?他个子高
    \ex 他个子比较高
    \ex 他个子还算高
    \ex 他个子高高的
    \ex 他睡不下这张床,因为他个子高啊
\end{exe}

\subsection{Prepositional complement construction}\label{sec:verb-phrase.internal.preposition}


Verbs that take prepositional phrases as complements also exist in Mandarin;
unlike the case in English, 
when a prepositional phrase appears after the verb,
we can be sure that the verb is stative 
and no further internal complements are allowed for them.%
\footnote{
    Action verbs can also select prepositional complements 
    but then they have to appear before the verb
    (\prettyref{sec:vp.oblique-incorporation}, 
    \prettyref{ex:vp.oblique-incorporation.2.omission}).
}
(\ref{ex:vp.preposition-complement.1}, \ref{ex:vp.preposition-complement.2})
are examples in which the only internal complement is a prepositional phrase. 
No valency alternation constructions involving these prepositional phrases are possible,
unlike, say, English, 
where pseudo-passive constructions like 
\form{I don't want to be stared at} exist.

\begin{exe}
    \ex\label{ex:vp.preposition-complement.1} \gll 他 生 [于 1990 年] \\
    3 be.born at {} year \\
    \glt \translate{He was born in the year of 1990.}
    \ex\label{ex:vp.preposition-complement.2} \gll 我 住 [在 上海] \\ 
    1 live at Shanghai \\
    \glt \translate{I live in Shanghai.}
\end{exe}

There is evidence suggesting that the preposition in the prepositional complement
is incorporated into the verbal complex. 
In (\ref{ex:vp.preposition-complement.1}, \ref{ex:vp.preposition-complement.2}),
生于 and 住在 are prosodic words.
Morphologically, the phonologically light aspectual marker 了 
sometimes is attested \emph{after} the preposition
(\ref{ex:vp.preposition-complement.3.1});
although this construction seems somehow infelicitous to me, 
and the alternative form without the aspectual marker 
seems better (\ref{ex:vp.preposition-complement.3.2}),
the alternative \category{asp}-\category{prep} morpheme order 
is clearly not acceptable (\ref{ex:vp.preposition-complement.3.3}).

\begin{exe}
    \ex\label{ex:vp.preposition-complement.3} \begin{xlist}
        \ex\label{ex:vp.preposition-complement.3.1} 
        \gll 为什么 他们 都 住 在 了 LIC? \\
        why \category{3sg} all live at \category{asp} \specialunit{place} \\ 
        \glt \translate{Why do they all live in Long Island City?}
        (\href{https://zhuanlan.zhihu.com/p/457465036}{online post})

        \ex\label{ex:vp.preposition-complement.3.2} 
        \gll 为什么 他们 都 住 在 LIC? \\
        why \category{3sg} all live at \specialunit{place} \\ 
        \glt \translate{Why do they all live in Long Island City?}
        
        \ex\label{ex:vp.preposition-complement.3.3} 
        *为什么他们都住了在LIC
    \end{xlist}
\end{exe}


\section{Monotransitive and intransitive action verbs}


\subsection{The prototypical \category{do} structure}\label{sec:verb-phrase.do.standard}

\begin{exe}
    \ex 我喜欢她
    \ex 
    \begin{xlist}
        \ex\label{ex:vp.do.standard.ba-1} ???我把她喜欢
        \ex ??她被我喜欢
        \ex 被不喜欢的人喜欢是非常糟糕的事情
    \end{xlist}
\end{exe}

 -- but this is definitely a ``suffer'' or ``affectee'' construction

Thus, we conclude 喜欢 is a \category{do} verb, 
and this can be expected: 
to like someone (or in general, to have some thoughts about someone) 
doesn't change the inside state of the target.
Thus, the \category{become} argument structure is not available. 

\subsection{The \category{do}-\category{instrument} or \category{do}-\category{place} structure}
\label{sec:vp.oblique-incorporation}

The main verb sometimes can move before an oblique or peripheral argument, 
and the \acs{np} then appears to be the object
(\ref{ex:verb-valency.do-instrument.1}; \citealt[\citechap{4}, \citesec{5}]{feng2000}).


\begin{exe}
    \ex\label{ex:verb-valency.do-instrument.1} \begin{xlist}
        \ex Prototypical place and object \\
        我们今天准备 [在食堂] 吃 [饭]

        \ex\label{ex:vp.oblique-incorporation.2.omission} 
        Omission of object \\ 
        \%我们今天准备在食堂吃
        
        \ex Place-as-object construction \\
        我们今天准备吃食堂
    \end{xlist}
    
    \ex 你们吃过这里的食堂了吗?
\end{exe}

The structure seems to have prosodic motivation:
when the standard patientive argument (饭 in the example)
is deleted, either because of information structure
(e.g. it carries a piece of old information) 
or because of valency alternation,
the stress will be assigned to the main verb, 
which is too light to receive it;
one way to redeem the structure 
is to move the verb before 
and get a V-\category{instrument/place} linear order, 
which doesn't break any prosodic constraint.

Since the post-verbal complement is not truly object-like  
in the underlying structure, 
it can be expected that 
it's unable to participate in \form{bǎ}-construction 
or \form{bèi}-constructions.

\section{Change of state constructions}

\subsection{The intransitive \category{become} construction}\label{sec:vp.become}

A \category{become} verb may or may not license a \category{causer} argument.
Some never do. 
The verb 死, for example, doesn't allow the causer argument
and is therefore always intransitive
(\prettyref{ex:verb-phrase.cause-become.ordinary.1},
\prettyref{ex:verb-phrase.cause-become.ordinary.1-no-transitive}).
When the \category{cause}-\category{become} structure is not available,
the \form{bǎ}-construction 
and the two \form{bèi}-constructions,
which require the \category{causer} argument, 
are also not available
(\ref{ex:verb-phrase.cause-become.ordinary.1-no-ba},
\prettyref{sec:verb-phrase.object.ba.cause-become}).

\begin{exe}
    \ex\label{ex:verb-phrase.cause-become.ordinary.1} \gll 楼 下 一 只 猫 死 了  \\
    building down one \category{cl} cat die \category{asp} \\ 
    \glt {Outside the building, a cat died.}
    \ex\label{ex:verb-phrase.cause-become.ordinary.1-no-transitive} \gll * 这 个 混蛋 死 了 一 只 猫 \\
    {} \category{dem} \category{cl} thug die one \category{cl} cat \\ 
    \translate{This thug killed a cat.} 
    \ex\label{ex:verb-phrase.cause-become.ordinary.1-no-ba} *这个混蛋把一只猫死了
\end{exe}

\subsection{The intransitive \category{become}-\category{theme} construction}

\begin{exe}
    \ex \begin{xlist}
        \ex *我剥皮了这个橘子
        \ex 我把这个橘子剥了皮
    \end{xlist}
\end{exe}

The \category{theme} argument is unable to participate in any further valency alternations.

This is essentially a piece of evidence to postulate a OV deep structure: 
the surface SVO order is obtained by moving the verb before the object, 
and hence since 剥皮 can't be moved forward, 
we find the reason why the first sentence is not acceptable. 

\begin{exe}
    \ex ?我剥了这个橘子皮
\end{exe}

The construction may also be considered as an external possession construction?

\subsection{The transitive \category{cause}-\category{become} construction}
\label{sec:verb-phrase.cause-become.ordinary}

When the \category{causer} argument is licensed,
we observe S=O ambitransitive valence alternation
(\prettyref{ex:verb-phrase.notional-pass.1}, 
\prettyref{ex:verb-phrase.notional-pass.trans-1}).
The intransitive verb frame of a S=O ambitransitive verb 
is sometimes known as the \concept{notional passive}, 
in which no explicit passive markers like 被 
(\prettyref{sec:verb-phrase.object.short-bei})
or 给 (\prettyref{sec:ver-phrase.gei}) appear, 
but the semantics is somehow passive,
because the subject lacks the ability to carry out the action itself
(\prettyref{ex:verb-phrase.notional-pass.1},
\prettyref{ex:verb-phrase.notional-pass.trans-1}; 
compare the existence of the passive voice in the English translations
and the absence of any passive marker in the Mandarin examples).
A \category{cause}-\category{become} verb
has no problem appearing in the \form{bǎ}-construction
(\prettyref{ex:verb-phrase.notional-pass.ba-1}, \prettyref{sec:verb-phrase.object.ba.cause-become}).
The long \form{bèi}-construction, however, 
may seem somehow strange
(\prettyref{ex:verb-phrase.notional-pass.bei-1}, \prettyref{sec:verb-phrase.bei.passive-alternation}).

\begin{exe}
    \ex\label{ex:verb-phrase.notional-pass.1} 
    \gll 茶 泡 好 了 \\
    tea soak well \category{asp} \\
    \translate{Tea has been prepared.
    (lit. Tea has soaked well)}
    \ex\label{ex:verb-phrase.notional-pass.trans-1} 
    \gll 我 已经 泡 好 茶 了 \\
    1 already soak well tea \category{asp} \\
    \glt \translate{I have already prepared tea. (lit. I already have soaked tea well.)}
    \ex\label{ex:verb-phrase.notional-pass.ba-1}
    \gll 我 把 茶 泡 好 了 \\
    1 \category{ba} tea soak well \category{asp} \\
    \glt \translate{I have already prepared tea.}
    \ex\label{ex:verb-phrase.notional-pass.bei-1} 
    \gll ? 茶 被 我 泡 好 了 \\
    {} tea \category{bei} 1 soak well \category{asp} \\
    \glt \translate{Tea is prepared by me.}
\end{exe}

Also, if we assume that the \form{bǎ}-construction
is limited to the \category{cause}-\category{become} structure,
then the \category{cause}-\category{become} structure 
that comes from a transitive \category{do} structure 
doesn't have a corresponding \category{become} structure.

\begin{exe}
    \ex 他把一条流浪狗捡回了家
    \ex *这条流浪狗捡回了家
\end{exe}

The fact that some \category{become} structures 
don't have corresponding \category{cause}-\category{become} structures,
and some \category{cause}-\category{become} structures 
don't have corresponding \category{become} structures
means there should be further distinctions 
within the \category{become} verb class.

\subsection{The \category{experience}-\category{become} structure}\label{sec:verb-phrase.experience}

When a \category{become} structure
doesn't allow a \category{causer} argument being attached to it,
sometimes the \category{experience}-\category{become} structure is available,
in which the \category{experiencer} is promoted to the subject position
(\prettyref{ex:verb-phrase.experience.1},
\prettyref{ex:verb-phrase.experience.2}).
This structure usually has a negative meaning:
the subject (the \category{experiencer}) suffers from the action happening on the object.

\begin{exe}
    \ex\label{ex:verb-phrase.experience.1} \gll 王冕 死 了 父亲 \\
    (name) die \category{asp} father \\
    \glt \translate{Wang Mian's father died.
    (lit. Wang Mian experienced his father's death.)}
    \ex\label{ex:verb-phrase.experience.2} 才几个月,这家工厂就已经坏了三台机床了
\end{exe}

One phenomenon worth remarking 
is in the \category{experience}-\category{become} structure,
the \category{experiencer} argument 
always seems to have a possession relation 
with the patientive argument.
The experience analysis here semantically
imposes this tendency,
but it seems when there is no possession relation between the two arguments, 
the sentence is completely ungrammatical,
and not just implausible:
the meaning of (\prettyref{ex:verb-phrase.experience.3})
is quite plausible,
but the sentence is not fine.
On the other hand, 
adding a possessive into the object \acs{np}
that refers to the subject 
has the correct semantics,
but the result is still not acceptable
(\prettyref{ex:verb-phrase.experience-4}).
The possessor is never licensed for the object,
which means the \category{experience}-\category{become} structure 
is not simply adding a \category{experiencer} argument 
on top of a \category{become} structure:
it's an external possession construction as well (TODO).
External possession is also seen in 
long and short \form{bèi}-constructions
(\ref{ex:verb-phrase.long-bei.external-possession-1}).

\begin{exe}
    \ex\label{ex:verb-phrase.experience.3} *原本李四就靠张三的妈妈接济,现在李四死了张三的妈妈了,村里人自然要看不起他
    \ex\label{ex:verb-phrase.experience-4} *这家工厂坏了他们的三台机床
\end{exe}

\begin{infobox}{Alternative analysis of the \category{experience} construction}{alternative-analysis-experience}
    \citet[\citesec{212}]{deng2010formal} analyzes the \category{experience} structure 
    as a \category{become} structure with 
    the subject 王冕 being the argument introduced by \category{become},
    and the object 父亲 being an internal object attached to the verb 死.
    The main problem of this analysis is 
    this goes against his previous analysis -- 
    which I accept in this note -- 
    that in, say, 某人死了, it's the \acs{np} 某人 denoting the person who dies that 
    appears as the argument in the \category{become} structure.
    Comparing this prototypical verb frame of 死 
    with (\prettyref{ex:verb-phrase.experience.1}),
    we find if Deng's analysis were true,
    then the same semantic argument would be linked to two different syntactic positions,
    which, unless motivated by rather strong structural evidences, 
    should not be accepted.
\end{infobox}

The \category{experience}-\category{become} structure is ``finalized'':
the \form{bǎ}-construction and the long and short \form{bèi}-constructions
are never permitted to appear.
This is not surprising:
the \category{experience}-\category{become} structure 
is semantically distant from the \category{cause}-\category{become} structure,
and therefore the \form{bǎ}-construction 
is not applicable.
On the other hand,
both the short and the long \form{bèi}-constructions 
serve to promote one argument in the \acs{vp} 
to the experiencer position,
and imposing the \form{bèi}-constructions 
on top of the \category{experience}-\category{become} structure
violates the principle of economy.

\subsection{The \category{cause}-\category{benefit}-\category{become} construction}
\label{sec:verb-phrase.dative}


\subsection{The \category{be}-\category{become} alternation}\label{sec:vp.be-become}

A verb or adjective (TODO: what category?) that usually appear in a stative construction 
(\prettyref{ex:verb-phrase.be-become.source-1})
can be semi-regularly (see below) inserted into a \category{become} structure 
(\prettyref{ex:verb-phrase.be-become.1})
or further, in a \category{cause}-\category{become} structure 
(\prettyref{ex:verb-phrase.be-become.cause-1}).

\begin{exe}
    \ex\label{ex:verb-phrase.be-become.source-1} 
    \gll 他 对 物理学 的 知识 一向 丰富 \\
    3 towards physics \category{poss} knowledge always abundant \\
    \glt \translate{His knowledge in physics is always abundant.} 
    \ex\label{ex:verb-phrase.be-become.1} 
    \gll 经过 这 次 实地考察, 我们 对 这 片 山区 的 知识 更加 丰富 了 \\
    going.through \category{dem} \category{cl} real-place-investigation 
    1pl towards \category{dem} \category{cl} mountain-area \category{de} knowledge 
    more abundant \category{asp} \\
    \glt \translate{After this field work, our knowledge of this mountain 
    becomes more abundant.}
    \ex\label{ex:verb-phrase.be-become.cause-1}
    \gll 这 次 考察 丰富 了 我们 对 地质学 的 认识 \\
    \category{dem} \category{cl} investigation abundant \category{asp} 1pl towards geology \category{poss} \category{knowing} \\ 
    \glt \translate{This explanation makes our knowledge of geology abundant.}
\end{exe}

Note that if the predicator of 
a clause with the structure of (\prettyref{ex:verb-phrase.be-become.cause-1})
is usually found in a stative construction, 
then it is clearly to be interpreted as a \category{cause}-\category{become} structure, 
but this doesn't mean the \category{cause}-\category{become} structure
is always available: 
(\prettyref{ex:verb-phrase.be-become.cause-2}) seems problematic,
although its predicator, 红火, clearly is an adjective (TODO: ref)
and is able to appear in a stative construction 
(\prettyref{ex:verb-phrase.be-become.2}).

\begin{exe}
    \ex\label{ex:verb-phrase.be-become.2} 
    \gll 我们 家 的 日子 真是 越来越 红火 了 \\
    our home \category{poss} live truly more.and.more booming \category{asp} \\
    \glt \translate{Our lives are increasingly improving.}
    \ex\label{ex:verb-phrase.be-become.cause-2} ???新政策红火了我们的日子
\end{exe}

\form{Bǎ}- or \form{bèi}-constructions are less productive 
for \category{become}-structures or \category{cause}-\category{become} structures 
derived from \category{be}-structures 
(\prettyref{ex:verb-phrase.be.impossible-1},
\prettyref{ex:verb-phrase.be.impossible-2},
\prettyref{ex:verb-phrase.be.impossible-3}),
though not completely unattested.
The problem with the \form{bèi}-constructions
seems to be that the \category{become} construction 
headed by an adjective tends to pick up the 
``spontaneous state change'' meaning 
and therefore is semantically incompatible with 
the \form{bèi}-constructions
(\prettyref{sec:verb-phrase.bei.passive-alternation},
\prettyref{sec:verb-phrase.experience}).
The problem with the \form{bǎ}-construction 
also seems to be semantic:
the \form{bǎ}-construction is not compatible with 
a main predicator that is too stative. 
By adding suffixes that are usually applied to verbs like 起来, 
the acceptability is slightly improved 
(\ref{ex:verb-phrase.be.impossible-4}).

\begin{exe}
    \ex\label{ex:verb-phrase.be.impossible-1} *我们的知识被丰富了
    \ex\label{ex:verb-phrase.be.impossible-2} ?这场展览把我们的知识丰富了
    \ex\label{ex:verb-phrase.be.impossible-3} ?我们的知识被这场展览丰富了
    \ex\label{ex:verb-phrase.be.impossible-4}
    \gll  iPad 版 需要 把 功能 丰富 起来!\\ 
    iPad version need \category{ba} function rich \category{todo} \\ 
    \glt \translate{The iPad version (of this software) needs to enrich its functionalities!} 
    (an online user feedback)
\end{exe}

\subsection{The \category{do}-\category{become} alternation}\label{sec:ver-phrase.gei}

The verb 跑, for example, 
can appear in the \category{experience}-\category{become} verb frame  
(\prettyref{ex:verb-phrase.overview.1}), 
but insertion of the aspect marker 正在 
rules out the \category{experience}-\category{become} verb frame 
(\prettyref{ex:verb-phrase.overview.2})
but permits the \category{do} construction 
(\ref{ex:verb-phrase.overview.3}).

\begin{exe}
    \ex\label{ex:verb-phrase.overview.1} \gll 昨天 城北 那 座 监狱 跑 了 一 个 犯人 \\
    yesterday north.city \category{dem} \category{cl} prison run \category{asp} one \category{cl} inmate \\
    \glt \translate{Yesterday, an inmate escaped (lit. `ran') from the prison to the north of the city.} 
    \ex\label{ex:verb-phrase.overview.2} *城北那座监狱正在跑一个犯人
    \ex\label{ex:verb-phrase.overview.3} 有一个犯人现在正在逃跑
\end{exe}

Not all verbs can appear in the notional passive construction.
There exists another construction -- the \form{gěi}-passive construction -- 
that has similar meaning with the notional passive
(i.e. the internal state of the subject somehow changes 
without the external cause -- if any -- being specified) and
can be observed on its own 
in very colloquial and non-standard speech
(\prettyref{ex:verb-phrase.gei.1}),
sometimes for stylist and humor purposes.

\begin{exe}
    \ex\label{ex:verb-phrase.gei.1} \% 李四给杀了
\end{exe}

The alternation between the notional passive and the \form{gěi}-construction 
is quite intriguing.
It means the two are not free variants of the marker of the \category{become} argument structure,
but are in contrast distribution:
给 usually appears with verbs that prototypically appear in the \category{do} structure
(\prettyref{ex:verb-phrase.gei.alternation-1},
\prettyref{ex:verb-phrase.gei.alternation-2},
\prettyref{ex:verb-phrase.gei.alternation-3}).
给 also renders the whole predicate strongly telic.
It seems to never appear without the aspectual marker 了.
It's likely that 给 has already developed a distinct usage 
as a valency changing marker, 
which turns a \category{do} verb into a \category{become} one,
while the notional passive construction directly applies to \category{become} verb.

\begin{exe}
    \ex\label{ex:verb-phrase.gei.alternation-1} \begin{xlist}
        \ex ???我给喝醉了
        \ex 我喝醉了
    \end{xlist}
    \ex\label{ex:verb-phrase.gei.alternation-2} \begin{xlist}
        \ex ??? 茶给泡好了
        \ex 茶泡好了
    \end{xlist}
    \ex\label{ex:verb-phrase.gei.alternation-3} \begin{xlist}
        \ex \% 我给灌醉了
        \ex *我灌醉了
    \end{xlist}
\end{exe}

给 also appears as a part of the \form{bǎ}-construction
(\prettyref{ex:verb-phrase.gei.ba-1})
and the long \form{bèi}-construction
(\prettyref{ex:verb-phrase.gei.bei-1}),
although after removing 给,
the instances of the \form{bǎ}- and \form{bèi}-constructions above 
are still grammatical.
In \form{bǎ}- and \form{bèi}-constructions, 
the appearance of 给 seems to have no link 
with the \category{do}-\category{become} distinction
(\prettyref{ex:verb-phrase.gei.ba-2}; c.f. \prettyref{ex:verb-phrase.gei.alternation-1}).
The \form{gěi}-construction however is not compatible 
with the short \form{bèi}-construction
(\prettyref{ex:verb-phrase.gei.bei-2}).
A reasonable guess, then, 
is that in the \form{bǎ}-construction
and the long \form{bèi}-construction,
给 marks an embedded \category{become} structure,
regardless of whether a \category{do} structure is embedded inside,
while the short \form{bèi} construction is completely isolated 
from this pipeline.

\begin{exe}
    \ex\label{ex:verb-phrase.gei.ba-1} 他们把李四给杀了
    \ex\label{ex:verb-phrase.gei.bei-1} 李四被他们给杀了
    \ex\label{ex:verb-phrase.gei.bei-2} *李四被给杀了
    \ex\label{ex:verb-phrase.gei.ba-2} 这瓶酒把我给喝醉了
\end{exe}

\subsection{The existential construction}

\begin{exe}
    \ex 高速公路上开着一辆宝马
    \ex 墙上打了三个洞
\end{exe}

The construction may be recognized as a focus construction by some; 
the fact that it's impossible to recast the construction 
into a more ``prototypical'' one 
without changing the aspectual marking of the verb, however, 
implies that the formation of the construction 
is rooted in the core of the \acs{vp}, 
and is unlikely to be merely an information packaging phenomenon. 

Note the behavior in coordination.
It can be seen that here we have the possessor raising construction again. 
\begin{exe}
    \ex 这条高速公路是新修建的,非常气派,上面开着很多车
\end{exe}

\subsection{Semi-object}\label{sec:vp.internal.semi-object}

There may be a numeral expression in the \acs{vp}
that gives the ``quantity'' of the event, 
which is called the \concept{semi-object} 
\citep[\citesec{8.6}]{zhudexigrammar}.
A semi-object may be a counting expression, 
a time expression, 
or a pure numerical expression (TODO: examples, and more concise terms).
Their semantic functions 
are closer to numeral attributives in \acs{np}s.
They are called ``objects '' purely because they are within the \acs{vp}
and are \acs{np}s themselves;
this note doesn't recognize them as objects;
the term \term{semi-object} is only used to TODO: so is it really necessary to use the term?

\begin{infobox}{The coverage of the term \term{semi-object}}{semi-object-coverage}
    Apart from the numeral attributives in \acs{vp} discussed above, 
    numerals appearing at the end of \acs{vp}s 
    are also sometimes called semi-objects \citep[\citepage{117}]{deng2010formal}.
    The syntactic function numerals in this latter case 
    is closer to verbal complements (TODO: ref).
    And actually the term \term{semi-object} works better in the latter case!
    TODO: why are we sure that the two types of semi-objects have different syntactic positions?
\end{infobox}

\section{Ditransitive constructions}

\subsection{The simple \category{cause}-\category{benefit}-\category{become} construction}\label{sec:verb-valency.giving}

The \category{cause}-\category{benefit}-\category{become} construction
is predominantly seen with verbs about giving
can be divided into two subtypes: 
in one type the \category{beneficiary} argument 
is related to a resultative complement (\prettyref{sec:verb-valency.giving.complement}),
and another one, described here, 
is a double-object construction 
besides the \category{affect} construction
(\prettyref{sec:verb-phrase.do.affect}).
A \category{cause}-\category{benefit}-\category{become} construction
is just like a \category{cause}-\category{benefict} structure,
but it has an additional \category{beneficiary} argument 
lying between the \category{causer} and the \category{theme};
the S=O ambitransitive transitivity alternation 
(\prettyref{sec:verb-phrase.cause-become.ordinary})
can be observed, 
where the subject and the \category{them} argument 
play the role of S and O, respectively 
(\prettyref{ex:verb-phrase.cause.experience.alternation}).

\begin{exe}
    \ex\label{ex:verb-phrase.cause.experience.alternation}  \begin{xlist}
        \ex \gll {} [他]_{\text{subject, \category{causer}}} [给 了]_{\text{verbal complex}} [我]_{\text{\category{beneficiary}}} [一 本 书]_{\text{\category{theme}}} \\
        {} 3 give \category{asp} 1 one \category{cl} book \\
        \glt \translate{He gave me a book.}
        \ex\label{ex:verb-phrase.cause.experience.1-become} 
        \gll [这 本 书]_{\text{subject,\category{theme}}} [给 了]_{\text{verbal complex}} [我]_{\text{beneficiary}} \\
        \category{dem} \category{cl} book give.toward \category{asp} 1 \\ 
        \glt \translate{This book is given to me as a gift.}

        \ex\label{ex:verb-phrase.cause.experience.1-ba-correct} 他把一本书给了我

        \ex\label{ex:verb-phrase.cause.experience.1-bei-correct} 这本书被他给了我
    \end{xlist}
\end{exe}

The beneficiary argument 
is not syntactically active in valency alternation.
It can't be promoted to the subject position 
in intransitivization, 
the expected result being a \category{become} construction
(\ref{ex:verb-phrase.cause.experience.1-notional-passive}; 
c.f. \ref{ex:verb-phrase.cause.experience.1-become}).
Similarly, a legit \form{bǎ}-construction can't be formed by  
identifying the \category{causer} argument and  
the \category{beneficiary} argument 
as the arguments in the \category{cause}-\category{become} construction 
(\prettyref{ex:verb-phrase.cause.experience.1-ba}; c.f. \ref{ex:verb-phrase.cause.experience.1-ba-correct}).
This inactive characteristic of the \category{beneficiary} argument
may be explained by assuming that it receives an inherent case
and behaves just like a prepositional complement
\citep{huang2007}.%
\footnote{
    This fact seems to be unnoticed by \citet[\citepage{112}]{deng2010formal},
    who analyzes the dative construction 
    as 我 \category{cause} 张三 \category{become} 一本书 送.
    Then the first two arguments are in a \category{cause}-\category{become} structure 
    and therefore should be able to be alternatively 
    realized as a \form{bǎ}-construction,
    with 张三 appearing directly after 把.
    This, however, doesn't seem to be the case: 
    the correct \form{bǎ}-version is 我把一本书送给了张三,
    instead of 我把张三送了一本书.
} 

\begin{exe}
    \ex\label{ex:verb-phrase.cause.experience.1-notional-passive}  *我给了一本书
    \ex\label{ex:verb-phrase.cause.experience.1-ba} *他把我给了一本书
    \ex *我被他给了一本书
\end{exe}

The \category{beneficiary} argument lying between the highest \category{causer} argument 
and the lowest \category{theme} argument 
can be dated back to Old Chinese
(\prettyref{ex:verb-phrase.experience.oc-1}).
A difference between Old Chinese and modern Mandarin shown in the example 
is the \category{beneficiary} argument 
has become less active, 
for it's no longer able to be the subject; 
in Old Chinese, however, the \category{cause}-\category{benefict}-\category{become}
construction has a internal makeup comparable to the 
\category{do}-\category{affect} construction,
where the argument representing the object being transferred in the event 
is a rather internal and syntactically inactive one 
(\prettyref{sec:verb-phrase.do.affect}).
Few Old Chinese \category{cause}-\category{benefit}-\category{become} verbs 
are directly inherited by Mandarin;
for those Mandarin verbs with etymological relation 
with Old Chinese \category{cause}-\category{benefit}-\category{become} verbs,
the verb frame usually has already gone huge reanalysis,
with the alternation shown in 
(\prettyref{ex:verb-phrase.cause.experience.alternation}, 
\prettyref{ex:verb-phrase.experience.oc-1}) being unavailable
(\prettyref{ex:verb-phrase.experience.oc-to-m-1}).


\begin{exe}
    \ex \label{ex:verb-phrase.experience.oc-1} Old Chinese 
    \begin{xlist}
        \ex 王授我牛羊三千
        \ex\label{ex:verb-phrase.experience.oc-1b} 我受牛羊三千
    \end{xlist}
    \ex \label{ex:verb-phrase.experience.oc-to-m-1} 我接受了三千头牛羊
\end{exe}

The number of prototypical 
\category{cause}-\category{benefit}-\category{become} verbs 
is limited.
Most \category{giving} verbs in modern Mandarin 
fall under the category introduced in \prettyref{sec:verb-valency.giving.complement},
which employs a strategy 
that uses a verbal complement structure 
with close tie with \category{cause}-\category{benefit}-\category{become} construction.

\subsection{\category{cause}-\category{benefict}-\category{become} construction by resultative complement}\label{sec:verb-valency.giving.complement}

which is a standard ambitransitive \category{cause}-\category{become} structure,
with the receiver argument 我 being an internal object or prepositional complement,
depending on the nomenclature,
and the preposition 给 is attached to the verbal complex
(\prettyref{sec:verb-phrase.internal.preposition}).
There are several pieces of evidences 
that 给 is a preposition: 
it \emph{has to} be deleted when the receiver argument governed by it 
is topicalized 

\citep{paul2010applicative}

\begin{exe}
    \ex\label{ex:verb-phrase.cause.experience.1} 
    \gll {} [他] 送 给 了 [我] 一 本 书 \\
    {} 3 give.gift give.towards \category{asp} 1 one \category{cl} book \\
    \glt \translate{He gave me a book as a gift.}
    \ex\label{ex:verb-phrase.cause.experience.1-cause-become}
    \gll {} [他]_{\text{subject, \category{cause}}} 送 了 [这 本 书] [给 我] \\
    {} 3 give.gift \category{asp} \category{dem} \category{cl} book give.towards 1 \\
    \glt \translate{He gave the book to me as a gift.}
\end{exe}

\begin{exe}
    \ex \% 李四这人真的看人下菜碟。 
    [我]_{\text{topic}, i} [他]_{\text{subject}} 送了一本书 ---_i ,王大爷他就啥也没送
    \glt \translate{(colloquial).}
    \ex *我他送给了一本书
\end{exe}

\begin{exe}
    \ex 他送给了我一本书
    \ex 他送了给我一本书
    \ex 他给了一本书给我
    \ex 
\end{exe}

\subsection{The \category{do}-\category{affect} construction}\label{sec:verb-phrase.do.affect}

The \category{affect} construction may appear 
below the agent argument 
but above the patient argument or the internal object/verbal complement 
in a \category{do} structure.
When there is indeed a patient argument 
or an internal object, 
we get a double-object construction 
(\prettyref{ex:verb-phrase.affect.1}).
This double-object construction however 
is structurally different from the more frequent 
dative construction 
(\prettyref{sec:verb-phrase.dative},
\prettyref{sec:verb-valency.giving}).

The \category{affectee} argument is strongly patientive:
unlike the \category{experiencer} argument that bears a similar meaning 
but appears with the \category{become} structure 
(\prettyref{sec:verb-phrase.experience}),
a \category{affectee} argument can never appear in the subject,
which may be partially due to parsing effects,
for structural ambiguity usually arises 
if the \category{affectee} argument moves to the subject position
(\prettyref{ex:verb-phrase.affectee.1-ambiguity}).
The two \form{bèi}-constructions are both available 
for the \category{do}-\category{affect} construction,
for obvious semantic reason
(\prettyref{ex:verb-phrase.affectee.1-long-bei}, 
\prettyref{ex:verb-phrase.affectee.1-short-bei}).
On the other hand, the \form{bǎ}-construction 
is problematic (\prettyref{ex:verb-phrase.affectee.1-ba}).

\begin{exe}
    \ex\label{ex:verb-phrase.affect.1} 
    \gll [他]_{\text{subject, agent}} 抢 了 [我]_{\text{\category{affectee}}} [十 块 钱]_{\text{internal object}} \\ 
    3 rob \category{asp} 1 ten \category{cl} money \\
    \glt{He robbed ten dollars/\form{yuan}.}
    \ex\label{ex:verb-phrase.affectee.1-ambiguity} \gll 我 抢 了 十 块 钱 \\
    1 rob \category{asp} ten \category{cl} money \\
    \glt{I robbed ten dollars/\form{yuan}. (*I was robbed of ten dollars/\form{yuan}.)} 
    \ex\label{ex:verb-phrase.affectee.1-long-bei} \gll 我 被 他 抢 了 十 块 钱 \\
    1 \category{bei} 3 rob \category{asp} ten \category{cl} money \\ 
    \glt{I was robbed of ten dollars/\form{yuan} by him.}
    \ex\label{ex:verb-phrase.affectee.1-short-bei} \gll 我 被 抢 了 十 块 钱 \\ 
    1 \category{bei} rob ten \category{cl} money \\ 
    \glt \translate{I was robbed of ten dollars.}
    \ex\label{ex:verb-phrase.affectee.1-ba} \gll ? 他 把 我 抢 了 十 块 钱 \\
    {} 3 \category{ba} 1 rob \category{asp} ten \category{cl} money \\
    \glt{He robbed ten dollars/\form{yuan} from me.}
\end{exe}

The \acs{vp}-final theme argument is syntactically inactive in valency alternation;
it seems comparable to the semi-object 
(\prettyref{sec:vp.internal.semi-object}; \citealt{huang2007})

TODO: 打了我一下,幽了他一默

One important behavior of the \category{do}-\category{affect} construction 
is it can't be used with the durative aspect marker 着.

\section{Complement clause constructions}



\chapter{Verbal complements}\label{chap:verbal-complement}

\concept{Verbal complements}% 
\footnote{
    The term \term{verbal} here is used to highlight that 
    the non-argument complement itself is 
    verb-like, either synchronically or historically.
    Complement clauses are also ``verbal'', 
    but they are larger than a single word 
    and are \emph{not} included in the category of verbal complement. 
} 
including \concept{resultative complements}, 
\concept{directional complements} (\prettyref{sec:vp.direction}), 
and \concept{potential complements}; 
they are usually suffixal, but not always
(\prettyref{sec:vp.direction}).

Sometimes phrasal complements like \concept{state complements} (TODO: 得, ref), 
\concept{semi-objects} and \concept{prepositional complements} 
are also recognized as non-argument complements, 
although in this note I recognize them as arguments (TODO: discussion).
Note that the state complement construction 
and the prepositional complement construction
also bring suffixation to the verb (TODO: ref).
This may explain why 
if we add the state complement and the prepositional complement 
into the category of non-argument complements, 
in each \acs{vp} there is at most one non-argument complement: 
since there is at most one verbal complement slot in the verbal complex
(\prettyref{sec:vp.verb.affix}),
the existence of one excludes the existence of others.

There are also a class of clausal dependents 
that are traditionally analyzed as objects,
but are extremely inactive in structure building 
after the argument structure is finished.
They seem to be in contrast distribution with verbal complements.
They therefore are to be classified together with verbal complements
\citep[\citepages{188-190}]{deng2010formal}.

\begin{infobox}{Verbal complements and ordinary arguments: which is higher?}{verbal-complement-position}
    It's hard to tell whether verbal complements are closer to the main verb 
    or they are actually further from the main verb 
    (ahd ordinary arguments like the object are closely linked to the verb).
    Frequently used structural tests, 
    like the binding test (observing the distribution of reflexives),
    the coordination test, etc. all don't work. 
    From the perspective of semantic interpretation, 
    both possibilities are highly plausible -- 
    if the verbal complement is closer to the main verb,
    we can say it modifies the action in question,
    while if the verbal complement is attached to the main verb plus its arguments,
    we can say it modifies the whole event including the participants.
    Both interpretations seem to be largely equivalent.
\end{infobox}

\section{The directional complement}\label{sec:vp.direction}

\subsection{Directional complement in and out of verbal complex}

The linear order between the aspect marker and the directional complement, if any, 
is not completely fixed \eqref{ex:movable-suffix}.
In the first sentence of \eqref{ex:movable-suffix}, 
了 is an aspectual suffix (\prettyref{sec:aspectual}),
while 走 is a verb which never appear without an argument in uncontroversial phrasal grammar.
So we conclude 了 and 走 are suffixes,
and by structural comparison, 
we conclude 过来 in \eqref{ex:sanpinqishuiugolai-1} 
is also a suffix, with the same status as 走.
But there comes \eqref{ex:sanpinqishuiugolai-2},
in which 过来 moves to the end of the clause.
Finally, the order of 了 and 过来 can be swapped, 
possibly to make the verbal complex prosodically harmonic, 
since now it contains two disyllabic prosodic words 
(\ref{ex:sanpinqishuiugolai-3}).  

\begin{exe}
    \ex \begin{xlist}
        \ex \gll 他 [带 走 了] 他的 文件  \\ 
        3sg carry go.away \acs{perfect} 3sg-\acs{possessive} file \\
        \glt \translate{He carried his files away.}
        \ex \gll 他 [带 [过来]_{\text{directional complement}} 了] 三 瓶 汽水 \\
        3sg carry come \acs{perfect} three bottle.\acs{classify} soda \\
        \glt \translate{He carried here three bottles of soda.} 
        \label{ex:sanpinqishuiugolai-1}

        \ex \label{ex:sanpinqishuiugolai-2}
        \gll 他 带 了 三 瓶 汽水 [过来]_{\text{directional complement}} \\
        3 carry \category{asp} three \category{cls}.bottle soda come \\
        \glt \translate{He carried here three bottles of soda.}
        
        \ex\label{ex:sanpinqishuiugolai-3}
        \gll 他 带 了 [过来]_{\text{directional complement}} 三 瓶 汽水  \\
        3 carry \category{asp} come three \category{cls}.bottle soda  \\
        \glt \translate{He carried here three bottles of soda.}
    \end{xlist}
    \label{ex:movable-suffix}
\end{exe}

\subsection{Monosyllabic directional complement}



\subsection{Disyllabic directional complements}

\begin{infobox}{Phrasal directional complement?}{phrasal-directional}
    The \acs{vp}-final disyllabic directional complement seen in 
    (\prettyref{ex:sanpinqishuiugolai-2})
    may also be seen as a phrasal verbal complement 
    \citep[\citepage{120}]{deng2010formal}, 
    but this is due to his strictly lexicalist analysis; 
    in my approach outlined in \prettyref{sec:theory}, 
    even though the directional complement in  
    (\prettyref{ex:sanpinqishuiugolai-2})
    doesn't appear together with the rest of the verbal complex, 
    the fact that it's small in size and  
    it's constrained in productivity
    means it should be put together with other directional complements. 
\end{infobox}

\section{The state complement}



\begin{infobox}{Structural ambiguity after 得}{structural-ambiguity-de}
    There are two structures corresponding to the 得-\acs{np}-\acs{vp} sequence.
    In one case, like 这文章写得谁也看不懂, the \acs{np}-\acs{vp} sequence after 得 is a clause, 
    which, together with 得, constitutes 
    a state complement construction.
    In another case, like 这条山路走得我累死了, 
    the \acs{np} immediately after 得 
    is the \category{theme} argument associated with the \category{become} argument structure, 
    while the \acs{vp} is a clause with an empty subject 
    that constituents the state complement construction.
    This difference can be tested by trying to remove the current subject 
    and see whether we can still find a related grammatical sentence. 
    In the second case, we have 我走得累死了, 
    which has a structure parallel to 这文章写得谁也看不懂, 
    while in the first case it's impossible. 
\end{infobox}

\section{Complement clause constructions}

Some are complement clause constructions

Purpose clause

\begin{exe}
    \ex 你当我傻吗
    \ex 我准备明天去骑马
\end{exe}

\begin{exe}
    \ex 他跪下来求我
\end{exe}

TODO: 为动,死国可乎 etc.

What about 笑天下可笑之人

It should be noted that the position of the complement clause
is lower than the direct object.

\begin{exe}
    \ex 他把这个消息告诉了我
    \ex 他告诉了我这个消息
    \ex 他告诉我张三脑袋被驴踢了
    \ex *他把张三脑袋被驴踢了告诉了我
\end{exe}

Here, the \acs{np} 一顶帽子 seems to be the internal object, 
which specifies ``the amount'' of the action,
while the 

\begin{exe}
    \ex 他抢了我一顶帽子
    \ex 我被他抢了一顶帽子
\end{exe}




\chapter{Peripheral arguments}\label{chap:vp.peripheral}

It should be noted that there exist core arguments with 
meanings or even forms (\prettyref{sec:verb-phrase.internal.preposition}) 
similar to real peripheral arguments;
there are however structural reasons to distinguish
between the two.
A main criterion is whether the argument can appear 
in a post-verbal position in the \acs{vp}:
if the argument is a core argument, 
then it appears behind the verb 
(\prettyref{sec:verb-phrase.internal.preposition}, \prettyref{sec:vp.oblique-incorporation})
to satisfy the prosodic constraint (\prettyref{sec:vp.prosody}),
although not necessarily 
(\prettyref{sec:vp.oblique-incorporation}, \ref{ex:vp.oblique-incorporation.2.omission}).
On the other hand, 
peripheral arguments never appear after the verb, 
even though this satisfies the prosodic constraint. 
This means prosodic checking in modern Mandarin happens
before peripheral arguments are added into the \acs{vp},
although historically this was not the case
\citep[\citepages{155-157}]{feng2000}.

\chapter{Tense, aspect, and modality}\label{chap:tam}

\section{Lexical aspect}\label{sec:lexical-aspect}

TODO: relation with argument structure 

\section{The aspectual system}\label{sec:aspectual}

Yet there is a system marking the aspect (\prettyref{sec:aspectual}). 
\eqref{ex:quguo-qule} is an example.

\begin{exe}
    \ex \begin{xlist}
        \ex \gll 我 去 过 上海 了 \\
        1 go \asis{guo} Shanghai \acs{sfp} \\
        \glt \translate{I have been in Shanghai.}
        \ex \gll 我 去 了 上海 了 \\
        1 go \asis{le} Shanghai \acs{sfp} \\
        \glt \translate{I have gone to Shanghai.}
    \end{xlist}
    \label{ex:quguo-qule}
\end{exe}

TODO: If the aspectual 了 marker is below the external light verb head, 
    then it can be only recognized as something more similar to the lexical aspect; 
    but it has rich interaction with negation (没吃完, *没吃完了, etc.),
    the latter seeming to be a clause-level phenomenon?
    But if we consider it to be higher than the external light verb head, 
    then the 把书交给了他 structure is hard to capture,
    since here we will agree that the verbal complex 交给了 
    is lower than 把, which roughly has the same position of the external light verb head,
    and 了 appears in the verbal complex; 
    this can be explained by affix lowering though; 
    but then arguably 把 should bar the affix lowering process; 
    but in the Cartographic viewpoint, 
    affix lowering is not prevented by the empty functional heads; 
    the minimal movement constraints seem to be more about a tendency.

\section{Is there a tense system in Mandarin?}

From a surface form-oriented perspective, 
Mandarin lacks the category of tense -- 
all semantic tense information is 
expressed by time adverbs 
and the default values determined by the aspect.
The position of the time adverb 
however seems to be slightly different from that 
of uncontroversial peripheral arguments. 
The default position of at least some time adverbs 
is higher than modality auxiliaries; 
alternation of this order results in 
clauses that are either slightly infelicitous 
or pragmatically marked. 

\begin{exe}
    \ex \begin{xlist}
        \ex 我[明天]可能能和你讨论一下
        \ex 我可能[明天]能和你讨论一下
        \ex ??我可能能[明天]和你讨论一下
    \end{xlist}
    
    \ex \begin{xlist}
        \ex ?我在周四可能能和你讨论一下
        \ex 我可能能在周四和你讨论一下
    \end{xlist}

    \ex 我之后可能每天会来 
\end{exe}

\begin{itemize}
    \item Supporting tense node: 
        \begin{itemize}
            \item Whether We Tense-Agree Overtly or No
        \end{itemize}
    \item No tense node:
    \begin{itemize}
        \item 
    \end{itemize}
\end{itemize}

\begin{exe}
    \ex 标语贴在墙上 
    \ex 标语已经在墙上贴着了
\end{exe}

\begin{exe}
    \ex 我每天都在床上哭
    \ex ?我每天在床上都哭
\end{exe}
 
this means the preposition 在 actually is morphologically merged with the verb 贴, 
or otherwise we are unable to explain why 
in the first example, 着 can never appear, 
while in the second example, 着 can appear.

Although 着 can appear in a matrix clause, 
its distribution is wider in temporal adverbials. 

*他笑着。
他[笑着]走了进来



\section{Modal auxiliaries}\label{sec:vp.aux}

Modality is marked similarly be adverbs or complement clause constructions.

\section{There is no serial-verb construction or complex predicate}\label{sec:no-serial-verb}

The term \term{serial-verb construction} refers to several different things in the literature.
Sometimes it refers to the verbal complement system
\citep{chen2016mandarin}, 
although in the topological literature 
there is no longer considered as a good usage 
\citep[\citesec{10.1}; note that %
the V2s in Yakkha complex predication highly resembles Mandarin directional verbal complements 
in their formal aspects]{schackow2015grammar}. 
In this sense, we of course have serial-verb constructions in Mandarin.




\chapter{Simple clauses}\label{chap:simple-clause}

A sentence can be divided into several clauses 
linked by clause linking constructions 
(\prettyref{chap:clause-linking}).%
\footnote{
    \citet{cgel} uses the term \term{sentence} 
    to refer to a natural unit in dialogue,
    which I refer to as a \term{utterance}.
    The term \term{sentence} here refers to 
    a clause that qualifies as an utterance. 

    Some people, like \citet[\citepage{140}]{deng2010formal}
    as well as \citet{dixon2009basic},
    use the term \term{clause} for subject-predict constructions 
    with no speech force marking.
    (\citet{deng2010formal} uses 句子 as the Mandarin counterpart of \term{sentence}
    and 小句 as the counterpart of \term{clause}.)
    In this way, \acl{sfp}s shouldn't be
    regarded as a part of the clause, 
    and they may be discussed together with 
    other higher level constructions like 
    clause linking \prettyref{chap:clause-linking}.
    But this notion of clause certainly goes against the tradition in descriptive grammars.

    This note refers to all units that are larger than the 
    subject-predicate construction as clauses, 
    which may or may not be sentence.
    The internal complexity of a clause 
    is still relevant for example in clause combining.
}
Mandarin has ample information marking phenomena,
and thus a clause can be divided into
one or more topics, if any, and a comment,
the latter being the nucleus clause
plus possible \acl{sfp}s. 
Note that topicalization and coordination can happen successively
(TODO: ref),
and coordination can also happen inside the nucleus clause
(TODO: ref).
The nucleus clause contains a subject (if any) and a predicate.%
\footnote{
    \citet{dixon2009basic} argues against the definition of \term{predicate} 
    as the main verb (or adjective) plus somehow ``internal'' arguments.
    He uses the term \term{predicate} to refer to the verbal complex instead.
    However, since I will need to compare the topic-comment construction 
    with the inner structure of the nucleus clause,
    the term \term{predicate} will still be used in the way \citet{dixon2009basic} dislikes,
    because it's the counterpart of the comment role in the topic-comment construction.
}

This chapter is a summary of the more or less obligatory parts 
in the clause structure,
namely the nucleus clause plus \acl{sfp}s,
introduced in the chapters before;
following the order from simpler constructions to more complicated constructions, 
information marking, coordination and subordination
are introduced in chapters following this chapter.
The latter chapters contain more high-level constructions 
outside constructions in this chapter  
as well as refinements to the \emph{internal structure} 
of constructions introduced in this chapter.

TODO: relative height of SFPs and information marking 

\section{The verb phrase}\label{sec:simple-clause.vp}

\subsection{The extended region}

The full \acs{vp} (\concept{extended \acs{vp}}) can always be first divided into two regions.
The left region,
which I name the \concept{extension region}, 
contains the follows
(\ref{ex:vp.ex.1}, \prettyref{fig:vp.ex.1}):
\begin{itemize}
    \item \acs{tame} auxiliaries and adverbs not realized in the verbal complex 
    (\prettyref{chap:tam});
    \item peripheral arguments like 
    temporal and spatial locations (\prettyref{chap:vp.peripheral}); 
    \item fronted the object TODO 
    \item it's also possible that a prepositional complement is fronted to this region 
\end{itemize}

\begin{exe}
    \ex \label{ex:vp.ex.1}
    \gll 我 [明天 可能 能 在 我 的 办公室 跟 你 [讨论 一下]_{\text{core\acs{vp}}}]_{\text{extended \acs{vp}}} \\
    1 tomorrow \category{aux}:possible \category{aux}:ability at my \category{poss} 
    office with 2 discuss a.little.bit \\ 
    \glt \translate{Tomorrow possiblity I can have a brief discussion with you in my office.}
    
    \ex\label{ex:vp.ex.2}
    \gll 我 明天 可能 能 在 我 的 办公室 跟 你 把 这 个 问题 讨论 一下 \\
    1 tomorrow \category{aux}:possible \category{aux}:ability 
    at my \category{poss} office 
    with 2 
    \category{ba} this \category{cls} problem 
    discuss a.little.bit \\
    \glt \translate{Tomorrow possiblity I can have a brief discussion of this problem with you in my office.}
\end{exe}

\begin{figure}[H]
    {
        \small 
        \begin{tikzpicture}[x=0.75pt,y=0.75pt,yscale=-0.8,xscale=0.8]
    %uncomment if require: \path (0,740); %set diagram left start at 0, and has height of 740
    
%Straight Lines [id:da5419452521165726] 
\draw    (690.67,583.33) -- (742,583.33) ;
%Straight Lines [id:da08544086505959503] 
\draw    (713,511.59) -- (706.23,533.35) -- (690.67,583.33) ;
%Straight Lines [id:da5166120577578295] 
\draw    (713,511.59) -- (742,583.33) ;
%Straight Lines [id:da7597929142889452] 
\draw    (410,368.26) -- (410,584.33) ;
%Straight Lines [id:da7951507355099614] 
\draw    (312.67,294.26) -- (312.67,584.33) ;
%Straight Lines [id:da27035558491953804] 
\draw    (622.67,583.33) -- (642,583.33) ;
%Straight Lines [id:da24028174129699176] 
\draw    (630,511.59) -- (622.67,583.33) ;
%Straight Lines [id:da917015263091598] 
\draw    (630,511.59) -- (642,583.33) ;
%Straight Lines [id:da14883621053934304] 
\draw    (470,583.33) -- (547,583.33) ;
%Straight Lines [id:da9342077761978527] 
\draw    (500,441.59) -- (470,583.33) ;
%Straight Lines [id:da3488354163696341] 
\draw    (500,441.59) -- (547,583.33) ;
%Straight Lines [id:da5030694457277152] 
\draw    (673,441.59) -- (628.67,462.33) ;
%Straight Lines [id:da7531987353356717] 
\draw    (673,441.59) -- (710.67,462.33) ;
%Straight Lines [id:da40179355911176984] 
\draw    (589,370.26) -- (502,398.33) ;
%Straight Lines [id:da5639473207238246] 
\draw    (589,370.26) -- (672,398.33) ;
%Straight Lines [id:da5704299439338378] 
\draw    (497,294.26) -- (410,322.33) ;
%Straight Lines [id:da8053817579360183] 
\draw    (497,294.26) -- (587,322.33) ;
%Straight Lines [id:da8066766119523336] 
\draw    (402,220.26) -- (315,248.33) ;
%Straight Lines [id:da8072938752013339] 
\draw    (402,220.26) -- (492,248.33) ;
%Straight Lines [id:da35978458309470907] 
\draw    (199.67,219.76) -- (199.67,475.49) -- (199.67,583.33) ;
%Straight Lines [id:da927464810721838] 
\draw    (294,146.26) -- (199.33,176.93) ;
%Straight Lines [id:da9186562853518254] 
\draw    (294,146.26) -- (400.33,176.93) ;
%Straight Lines [id:da7086875966035955] 
\draw    (190,71.59) -- (95.33,102.26) ;
%Straight Lines [id:da7175081469679203] 
\draw    (190,71.59) -- (296.33,102.26) ;
%Straight Lines [id:da26209295176984004] 
\draw    (87.67,146.56) -- (87.67,583.33) ;
%Straight Lines [id:da43013746492722116] 
\draw [color={rgb, 255:red, 80; green, 227; blue, 194 }  ,draw opacity=1 ][line width=2.25]    (191.67,623.26) -- (643.78,623.26) ;
%Straight Lines [id:da47071956794766767] 
\draw [color={rgb, 255:red, 74; green, 144; blue, 226 }  ,draw opacity=1 ][line width=2.25]    (686.78,623.26) -- (755,623.26) ;

% Text Node
\draw (78.67,590.17) node [anchor=north west][inner sep=0.75pt]   [align=left] {我 \ \ \ \ \ };
% Text Node
\draw (183.67,590.17) node [anchor=north west][inner sep=0.75pt]   [align=left] {明天};
% Text Node
\draw (293.67,590.17) node [anchor=north west][inner sep=0.75pt]   [align=left] {可能};
% Text Node
\draw (400.67,590.17) node [anchor=north west][inner sep=0.75pt]   [align=left] {能};
% Text Node
\draw (614.67,590.17) node [anchor=north west][inner sep=0.75pt]   [align=left] {跟你};
% Text Node
\draw (463.67,590.17) node [anchor=north west][inner sep=0.75pt]   [align=left] {在我的办公室};
% Text Node
\draw (687.67,590.17) node [anchor=north west][inner sep=0.75pt]   [align=left] {讨论一下};
% Text Node
\draw (713,508.59) node [anchor=south] [inner sep=0.75pt]   [align=left] {\begin{minipage}[lt]{38.85pt}\setlength\topsep{0pt}
\begin{center}
head:\\core VP
\end{center}

\end{minipage}};
% Text Node
\draw (630,508.59) node [anchor=south] [inner sep=0.75pt]   [align=left] {\begin{minipage}[lt]{51.96pt}\setlength\topsep{0pt}
\begin{center}
comitative:\\PP
\end{center}

\end{minipage}};
% Text Node
\draw (673,438.59) node [anchor=south] [inner sep=0.75pt]   [align=left] {\begin{minipage}[lt]{80.61pt}\setlength\topsep{0pt}
\begin{center}
head:\\extended VP
\end{center}

\end{minipage}};
% Text Node
\draw (500,438.59) node [anchor=south] [inner sep=0.75pt]   [align=left] {\begin{minipage}[lt]{40.93pt}\setlength\topsep{0pt}
\begin{center}
location:\\PP
\end{center}

\end{minipage}};
% Text Node
\draw (589,367.26) node [anchor=south] [inner sep=0.75pt]   [align=left] {\begin{minipage}[lt]{60.61pt}\setlength\topsep{0pt}
\begin{center}
head:\\extended VP
\end{center}

\end{minipage}};
% Text Node
\draw (410,365.26) node [anchor=south] [inner sep=0.75pt]   [align=left] {\begin{minipage}[lt]{64.42pt}\setlength\topsep{0pt}
\begin{center}
modality aux\\ {[\category{ability}]}
\end{center}

\end{minipage}};
% Text Node
\draw (497,291.26) node [anchor=south] [inner sep=0.75pt]   [align=left] {\begin{minipage}[lt]{60.61pt}\setlength\topsep{0pt}
\begin{center}
head:\\extended VP
\end{center}

\end{minipage}};
% Text Node
\draw (312.67,291.26) node [anchor=south] [inner sep=0.75pt]   [align=left] {\begin{minipage}[lt]{64.42pt}\setlength\topsep{0pt}
\begin{center}
modality aux\\ {[\category{possibility}]}
\end{center}

\end{minipage}};
% Text Node
\draw (402,217.26) node [anchor=south] [inner sep=0.75pt]   [align=left] {\begin{minipage}[lt]{60.61pt}\setlength\topsep{0pt}
\begin{center}
head:\\extended VP
\end{center}

\end{minipage}};
% Text Node
\draw (199.67,216.76) node [anchor=south] [inner sep=0.75pt]   [align=left] {\begin{minipage}[lt]{64.14pt}\setlength\topsep{0pt}
\begin{center}
time location:\\adverb
\end{center}

\end{minipage}};
% Text Node
\draw (294,143.26) node [anchor=south] [inner sep=0.75pt]   [align=left] {\begin{minipage}[lt]{60.61pt}\setlength\topsep{0pt}
\begin{center}
predicate:\\extended VP
\end{center}

\end{minipage}};
% Text Node
\draw (190,68.59) node [anchor=south] [inner sep=0.75pt]   [align=left] {\begin{minipage}[lt]{65.38pt}\setlength\topsep{0pt}
\begin{center}
nucleus clause
\end{center}

\end{minipage}};
% Text Node
\draw (87.67,143.56) node [anchor=south] [inner sep=0.75pt]   [align=left] {\begin{minipage}[lt]{39.55pt}\setlength\topsep{0pt}
\begin{center}
subject:\\pronoun
\end{center}

\end{minipage}};
% Text Node
\draw (417.72,626.26) node [anchor=north] [inner sep=0.75pt]  [color={rgb, 255:red, 80; green, 227; blue, 194 }  ,opacity=1 ] [align=left] {extension};
% Text Node
\draw (720.89,626.26) node [anchor=north] [inner sep=0.75pt]  [color={rgb, 255:red, 74; green, 144; blue, 226 }  ,opacity=1 ] [align=left] {core VP};


    
    
    \end{tikzpicture}
    
    }
    \caption{Tree diagram of (\ref{ex:vp.ex.1})}
    \label{fig:vp.ex.1}
\end{figure}

\subsection{The core region: the prototypical VO form}

Below the extension region, we find the core \acs{vp}.
The surface structure shows great variance, 
which contains the follows:
\begin{itemize}
    \item The \concept{verbal complex} (\prettyref{chap:vp.verbal-complex})%
    \footnote{
        The term \term{verbal complex} is used to highlight
        its derivation from prototypical verb derivation and inflection.
        See the beginning of \prettyref{chap:vp.verbal-complex}.
    }
    is at the beginning 
    -- sometimes with fragments residing at the end (\prettyref{sec:verb-splitting}) -- 
    of the core \acs{vp}.
    
    Grammatical systems contained in the verbal complex include 
    aspectual marking, verbal complement and verb derivation
    (\prettyref{sec:vp.verb.affix}). 
    Lexical aspect and verb valency has no explicit marking 
    but are connected to the aspectual marker 
    and the verbal complement (TODO:ref).

    \item The argument structure (\prettyref{chap:vp.argument}), 
    including both the roles of the arguments 
    in the situation described by the verb
    (i.e. ``deep argument slots''),%
    \footnote{
        Note that these positions are still syntactic concepts,
        like \category{agent} or \category{causer} in \prettyref{chap:vp.argument} 
        since they are determined at least partly by syntactic criteria;
        classification of truly semantic argument roles 
        is much complicated. 
    }
    and the roles of the arguments in the clause 
    (i.e. ``surface argument slots'').
    The subject is also a part of the argument structure;
    its properties as a clausal pivot
    however involve grammatical concepts beyond the scope of the \acs{vp}%
    \footnote{
        For example we need to compare it with the topic 
        (\prettyref{sec:topic-subject}).
    }
    and are dealt with not in \prettyref{chap:vp.argument} 
    but in this chapter (\prettyref{sec:simple-clause.nucleus.subject}).

    \item Non-argument complements, known as 补语 in Chinese linguistic community,%
    \footnote{
        The term 补语 literally means \translate{complementation speech}, 
        and is therefore often translated as \term{complement}.
        In this note I use the term \term{complement}
        to refer to grammatical constituents that are somehow more closely 
        related to the lexical head, 
        and I choose the (somehow tedious but explicit) term 
        \term{non-argument complement}.
    }
    i.e. complements besides the subject and the object. 
    This is a rather heterogeneous category,
    its boundary (expectedly) being somewhat unclear
    (\prettyref{chap:verbal-complement}).
\end{itemize}

There exist several seemingly non-SVO constructions.
The most salient cases are the \form{bǎ}-construction (\prettyref{sec:verb-phrase.object.ba}),
the \form{bèi}-constructions 
(\prettyref{sec:verb-phrase.object.short-bei}, 
\prettyref{sec:verb-phrase.bei}),
and similar constructions that interrupt the standard SVO constituent order (TODO: ref, 让, etc.).

\subsubsection{The position of the verb}

Complements in the core \acs{vp} are strictly 
after the starting point of the verbal complex;
when \form{bǎ}-, \form{gěi}- and \form{bèi}-constructions are not used,
they are always after the main verb.
Thus, Mandarin is usually classified as an SVO language.

The case of the \form{bǎ}-construction
casts doubt on this classification in typologists, 
but the auxiliary 把 may be seen as the starting point of the verbal complex
(and hence we assume the verbal complex is discontinuous here; TODO: ref).

\subsubsection{Argument structure v.s. information structure}

Some, like LaPolla, claim that Mandarin lacks grammaticalized argument structure altogether.
It will be too lengthy to review and refute the relevant ideas here.
Here I reanalyze some examples in \citet{lapolla20091} 
and point out why LaPolla's account is 
not restrictive enough to rule out unacceptable sentences
or unacceptable readings of acceptable sentences
and sometimes too restrictive that it rules out 
acceptable or marginal sentences.



\begin{exe}
    \ex 正说着,门外突然响起了砰砰砰的急急的敲门声
\end{exe}

\begin{itemize}
    \item 他现在知道我了 v.s. 我现在知道他了: semantic role and information structure can be orthogonal to each other. 
    \item 
\end{itemize}

The methodological issue with the ``Mandarin has no argument structure'' analysis 
is that an analysis that explains the attested utterances is not enough:
we also need to think about why something is \emph{not} acceptable.

\subsection{The short \form{bèi}-passive construction}\label{sec:verb-phrase.object.short-bei}

Although the only difference between the short bei-construction 
and the long bei-construction 
seems to be that the former lacks the semantic agent, 
the two constructions have important grammatical differences 
that seem to be not motivated by semantics. 

Although 给 and 被 seem similar at the same glance, 
the former is able to appear in a ba-construction, 
while the latter is never able to do so. 

\begin{exe}
    \ex 李四被他们杀了
    \ex 李四给他们杀了
    \ex *他们把李四被杀了
\end{exe}

把, 给, 被 appearing together: 给 is the \category{become} verb; 
李四给人杀了 = 李四 \category{become} [人杀了 e], 
where 李四 moves out of the DoP.

\begin{exe}
    \ex 李四给杀了
    \ex 李四给他们杀了
    \ex 他们把李四杀了
    \ex 他们把李四给杀了
    \ex *他们把李四被杀了
    \ex 李四被他们杀了
    \ex 李四被他们给杀了
\end{exe}

On the other hand, long bei-constructions are probably 
from the \category{cause}-\category{become} structure,
which can be demonstrated by the long-distance dependency relations 
observed in both the ba-construction  
and the long bei-construction
\citep[\citesec{4.2.1.5}]{huang2013}.
那块手表被李四用一个锤子砸烂了 
Here the \category{become}-structure would be 
?那块手表用一个锤子给砸烂了. 
李四 appears as a \category{causer},
and if the derivation stops here, 
把 is inserted, and we get 李四用一个锤子把那块手表给砸烂了.
If, however, on top of the \category{cause}-\category{become} structure, 
we insert 被 and move 那块手表 out, 
把 is no longer spelt out 
and we just get 那块手表被李四用一个锤子砸烂了.
Note that in the above procedure, 
the most internal clausal dependents are completely inactive: 
we can replace 砸烂 by 砸成了一堆破铜烂铁, 
and everything is still completely grammatical.

\begin{infobox}{Valency increasing, or just different subcategorization frames?}{status-of-ba}
    One difference between my analysis here and the analysis in \citet[\citepage{202}]{deng2010formal}
    is that the latter assumes that the input to the gei-construction 
    is a verb in the \category{become} frame.
    This however is unable to explain why we get 他给人家杀了,
    in which the agent 人家 appears after 给.
    There are however definitely vagueness in whether 
    the gei-construction can be applied to an existing \category{do} argument structure.
\end{infobox}


\subsection{The \form{bǎ}-construction}\label{sec:verb-phrase.object.ba}

TODO: 只不过把这个转变的过程,不愿意用精确的语言去解释 -- it seems 把 plus the \acs{np} after it indeed can be shifted out of the core clause

Also: 我还不能把门开着啊

\subsubsection{The overall structure}

The \form{bǎ}-construction is also referred to as 
the disposal construction or the causative construction.
The formation of the \form{bǎ}-construction is relatively clear: 
the ``direct object'' (see below) is first moved out
before the core \acs{vp} 
(and then appear before the main verb),
and the auxiliary 把 is then inserted before the direct object,
and then the subject is promoted before the auxiliary 把 (\prettyref{ex:verb-phrase.ba.ex-1}).

\begin{exe}
    \ex \label{ex:verb-phrase.ba.ex-1}
    \begin{xlist}
        \ex \gll [他们]_{\text{subject}} [杀 了 李四]_{\text{\acs{vp}}} \\
        3pl kill \category{asp} (name) \\
        \glt \translate{They killed Li Si.}

        \ex \gll [他们]_{\text{subject}} [把 [李四]_{\text{object}} 杀 了]_{\text{predicate:\form{bǎ}-\acs{vp}}} \\
        3pl \category{ba} (name) kill \category{asp} \\
        \glt \translate{They killed Li Si.}
    \end{xlist}
\end{exe}

The auxiliary 把 is frequently analyzed as a preposition in the literature; 
the \form{bǎ}-construction therefore may be considered 
as an exceptional object marking construction,
where the object is transformed into a prepositional phrase. 
But the 把-\acs{np} sequence is never observed 
outside of the \form{bǎ}-construction; 
also, the omission of 把 is accepted by some 
(\ref{ex:vp.ba.coordination-1}),
which is rare -- if not impossible -- 
for other prepositions in coordination constructions in Mandarin. 
Thus, in this note 把 is considered an auxiliary verb.

\begin{exe}
    \ex\label{ex:vp.ba.coordination-1} \% 他把饭做完了,衣服也洗好了
\end{exe}

The ``direct object'' here depends on the original argument structure. 
For monotransitive constructions 
the direct object is just the object;
for the \category{cause}-\category{benefit}-\category{become} construction 
the direct object is the \category{theme} argument 
(\prettyref{sec:verb-phrase.dative}, \ref{ex:verb-phrase.cause.experience.alternation}).
The \category{do}-\category{affect} construction 
however usually doesn't participate in the \form{bǎ}-construction.
The conditions under which the \form{bǎ}-construction is licensed 
are discussed in \prettyref{sec:verb-phrase.object.ba.cause-become}.



\subsubsection{The input argument structure}
\label{sec:verb-phrase.object.ba.cause-become}

The \form{bǎ}-construction
prototypically take a verb frame already with the \category{causer} role,  
i.e. the \category{cause}-\category{become} structure
and the \category{cause}-\category{benefit}-\category{become} construction
(\prettyref{sec:verb-phrase.cause-become.ordinary}, 
\ref{ex:verb-phrase.notional-pass.ba-1};
\prettyref{sec:verb-phrase.dative};
\citealt[\citepages{98-99}]{deng2010formal}).
A \category{become} verb without a \category{causer} argument 
can't receive one by appearing in the \form{bǎ}-construction
(\ref{ex:verb-phrase.cause-become.ordinary.1-no-ba}).

Some \category{do} verbs are also able 
to appear in the \form{bǎ}-construction, 
but \category{do} verbs deviating severely from the structure and semantics of  
the \category{cause}-\category{become} 
are prohibited from appearing
(\ref{ex:vp.do.standard.ba-1}, 
\ref{ex:verb-phrase.affectee.1-ba},
\prettyref{sec:verb-phrase.cause-become.ordinary}).

The durative aspect marker 着 almost never appears 
with the \form{bǎ}-construction, TODO: why?? 
which shouldn't be the case 
if the \form{bǎ}-construction can contain a \category{do}-structure.

It should however be emphasized that the \category{cause}-\category{become} structure 
appearing in the \form{bǎ}-construction 
doesn't necessarily have a \category{become} counterpart,
when the main verb is originally a \category{do} verb.

\begin{exe}
    \ex *我害惨了
    \ex 你这下把我害惨了
    \ex 你这下害惨我了
    \ex 我被你害惨了
\end{exe}

Due to the prosodic constraint (\prettyref{sec:vp.prosody}), 
in the \form{bǎ}-construction, 
it's not acceptable if the verb is monosyllabic and is not followed by other complements,
since now the \acs{vp}-final stress 
falls on the verb 
which is too light to receive the stress. 
Thus, in the \form{bǎ}-construction, 
monosyllabic verbs usually have the aspectual suffix 了.
This doesn't seem to be due to inherent properties 
of the \form{bǎ}-construction: 
when the prosodic problem is not present, 
the aspectual marker 正在 is also acceptable.

\begin{exe}
    \ex 他已经把这项工作做完了
    \ex 他正在把三种溶液放进一个瓶子里
\end{exe}

\subsubsection{Argument structures only appearing in \form{bǎ}-construction}

\begin{exe}
    \ex 我把卡车装满了稻草 \citep[\citepage{153}]{huang2013}
\end{exe}

\subsubsection{Comparison with \form{bèi}-constructions}

It is observed that the \form{bǎ}-construction
and the long \form{bèi}-construction
have close relation to each other, 
and the latter is usually analyzed 
as a variant of the former.
However, the \form{bǎ}-construction
is required to be somehow similar to a \category{cause}-\category{become} construction,
while this is not necessarily true for the long \form{bèi}-construction
gives rise to some mismatch between the \form{bǎ}-construction
and the long \form{bèi}-construction.

In some cases where the \form{bǎ}-construction is not available 
(\prettyref{sec:verb-phrase.object.ba.cause-become}),
the long \form{bèi}-construction is available
(\prettyref{sec:verb-phrase.bei.no-ba}).

On the other hand,
the S=O ambitransitive valency alternation 
in examples \prettyref{ex:verb-phrase.ba.cause.1}, \prettyref{ex:verb-phrase.ba.cause.2},
as well as the non-availability of the long \form{bèi}-construction
(\prettyref{ex:verb-phrase.ba.bei-1}, \prettyref{ex:verb-phrase.ba.bei-2})
and the state-changing semantics,
means these examples contain a typical \category{cause}-\category{become} structure 
described in \prettyref{sec:verb-phrase.cause-become.ordinary}.
In both examples, 
\form{bǎ}-constructions are available 
(\prettyref{ex:verb-phrase.ba.ba-1}, \prettyref{ex:verb-phrase.ba.ba-2}),
while the long \form{bèi}-construction 
seems problematic 
(\ref{ex:verb-phrase.ba.bei-1}, \ref{ex:verb-phrase.ba.bei-2}).

\begin{exe}
    \ex\label{ex:verb-phrase.ba.cause.1} \begin{xlist}
        \ex 我喝醉了
        \ex ?这瓶酒喝醉了我
    \end{xlist}
    \ex\label{ex:verb-phrase.ba.ba-1} 这瓶酒把我喝醉了
    \ex\label{ex:verb-phrase.ba.bei-1} ???我被这瓶酒喝醉了
\end{exe}

\begin{exe}
    \ex\label{ex:verb-phrase.ba.cause.2} \begin{xlist}
        \ex 我走得累死了
        \ex 这条路走得我累死了
    \end{xlist}
    \ex\label{ex:verb-phrase.ba.ba-2} 这条路把我走得累死了
    \ex\label{ex:verb-phrase.ba.bei-2} ???我被这条路走得累死了
\end{exe}




\subsubsection{Comparison with lexical causatives}\label{sec:verb-phrase.object.ba.causative}


It should be noted that the \form{bǎ}-construction is not
a causative construction that \emph{blindly} takes any existing argument structure as the input.
It seems what appears after 把 is never quite agentive:
this is only possible with a lexical causative verb like 让
(\prettyref{ex:verb-phrase.ba.2}, \prettyref{ex:verb-phrase.ba.correct-2}).
Thus, the \form{bǎ}-construction is unable to attach a \category{causer} argument 
to any existing argument structure. 

\begin{exe}
    \ex\label{ex:verb-phrase.ba.2} \gll * 他 把 我 去 爬山 \\
    {} 3 \category{ba} 1 go climb.mountain \\
    \glt \translate{He lets me climb mountains.}
    \ex\label{ex:verb-phrase.ba.correct-2} \gll 他 让 我 去 爬山 \\
    3 let 1 go climb.mountain \\
    \glt \translate{He lets me climb mountains.}
\end{exe}

\subsection{The long \form{bèi}-construction}\label{sec:verb-phrase.bei}

Summarizing the two subtypes of the long \form{bèi}-construction
and one case in which it's not available
(\prettyref{sec:verb-phrase.bei.ba},
\prettyref{sec:verb-phrase.bei.no-ba},
\prettyref{sec:verb-phrase.bei.passive-alternation}),
I conclude that the mechanism underlying the long \form{bèi}-construction
can be summarized as 
extracting an internal argument that is affected by the event 
and move it before 被.
Whether an argument structure can be an input of the long \form{bèi}-construction
depends on whether this affectee argument is available;
the \category{do}-\category{become} distinction seems to be not important here.

\subsubsection{Long \form{bèi}-construction with \form{bǎ} counterpart}
\label{sec:verb-phrase.bei.ba}

The close relation between the long \form{bèi}-construction 
and the \form{bǎ}-construction
has long been noticed.

\subsubsection{Long \form{bèi}-construction without \form{bǎ} counterpart}
\label{sec:verb-phrase.bei.no-ba}

However, the \category{cause}-\category{become} argument structure
doesn't seem to be the only source of the long \form{bèi}-construction.
There exist long \form{bèi}-constructions whose \form{bǎ}-counterparts 
are at best awkward and usually not acceptable.
The argument structures appearing in these examples 
include the \category{do}-\category{affect} construction 
(\prettyref{ex:verb-phrase.long-bei.no-ba.1}, 
\prettyref{sec:verb-phrase.do.affect}),
the \category{do} structure 
(\prettyref{ex:verb-phrase.long-bei.no-ba.2})

\begin{exe}
    \ex\label{ex:verb-phrase.long-bei.no-ba.1} 李四被张三抢了一顶帽子
    \ex\label{ex:verb-phrase.long-bei.no-ba.2} 被不喜欢的人喜欢是很让人为难的一件事
    \ex 我被他批评了一番,感觉非常不爽
\end{exe}

\begin{exe}
    \ex ??张三把李四抢了一顶帽子
    \ex *他真是把自己的老婆喜欢啊
    \ex 他把我批评了一番
\end{exe}

(\prettyref{ex:verb-phrase.long-bei.no-ba.1}) is a good example 
demonstrating the long \form{bèi}-construction
allows more freedom in choosing 
than the \form{bǎ}-construction.

TODO: compare 他被强盗杀了父亲 with 王冕死了父亲; also 他被杀了父亲

\begin{exe}
    \ex\label{ex:verb-phrase.long-bei.external-possession-1} \gll 他 被 强盗 杀 了 父亲 \\
    3 \category{bei} robber kill \category{asp} father \\
    \glt{He suffers from robbers' killing of his (*someone else's) father.}
\end{exe}

This illustrates the difference between \category{do} and \category{become} as well: 
王冕死了父亲 is a \category{become} argument structure 
with an experiencer attached,
while 强盗杀了他父亲 is 

他被抢了三十块钱 

强盗抢了他三十块钱

The alternation of the presence of 被 between the two input argument structures 
may reflect the same mechanism outlined in the alternation 
between the notional passive and the bei passive:

\subsubsection{Alternation between the notional passive and the long \form{bèi}-passive}
\label{sec:verb-phrase.bei.passive-alternation}

One interesting observation is \emph{not} every \category{cause}-\category{become} structure 
leads to a good long \form{bèi}-construction: 
the long \form{bèi}-construction
is less acceptable for 
a verb that is able to appear in the notional passive construction,
i.e. a verb can appear in the \category{become} structure \emph{on its own}
(\prettyref{ex:verb-phrase.notional-pass.bei-1},
\prettyref{ex:verb-phrase.bei.notional-pass-conflict-1},
\prettyref{ex:verb-phrase.bei.notional-pass-conflict-2}).
This might be motivated by semantic reasons:
since both the long and the short \form{bèi}-constructions 
have the meaning of ``being influenced by some external factors'',
a verb indicating an ``automatic'' internal state change 
has semantic incompatibility with these constructions. 
Indeed, when both situations seem plausible, 
both the notional passive and the long \form{bèi}-passive 
are available (\prettyref{ex:verb-phrase.bei.notional-pass-conflict-3}).

\begin{exe}
    \ex\label{ex:verb-phrase.bei.notional-pass-conflict-1} \begin{xlist}
        \ex 茶泡好了
        \ex ?茶被我泡好了
    \end{xlist}
    \ex\label{ex:verb-phrase.bei.notional-pass-conflict-2} \begin{xlist}
        \ex *他杀了
        \ex 他被强盗给杀了
    \end{xlist}
    \ex\label{ex:verb-phrase.bei.notional-pass-conflict-3} \begin{xlist}
        \ex 你的提案已经交到程序委员会了
        \ex 你的提案已经被交到程序委员会了
    \end{xlist}
\end{exe}

\subsection{Attested variances}

The above discussion is based on my own intuition.
This by no means represents all Mandarin speakers. 
This section lists some notable variances of the patterns summarized above. 

\subsubsection{把, 被 replaced by 给}

Sometimes, 把 in the \form{bǎ}-construction can be replaced by 给.
It seems if the verb is \category{become} (我笑麻了),
then 给 has the same function as 把's, 
while if the verb is \category{do},
给 has the same function as \category{become}.

\begin{exe}
    \ex \% 有人给我发了一封邮件,属实给我笑麻了
    \ex \% 这个情况给我整不会了
\end{exe}

\begin{exe}
    \ex\label{ex:verb-phrase.gei.2} \% 李四给他们杀了
\end{exe}

\subsection{Prosodic constraint on core \acs{vp}}\label{sec:vp.prosody}

All Mandarin \acs{vp}s follow the following prosodic constraint:
if the verbal complex is transitive,
then the constituent after it should receive prosodic focus; 
otherwise the verbal complex should be able to receive prosodic focus. 
This means in a transitive clause, 
the verbal complex can't be too heavy, 
while in an intransitive clause (TODO: 把 construction), 
the verbal complex can't be too light. 

It should be noted that this constraint doesn't apply 
to other type of syntactic constructions, 
even though a verb root appears. 
Thus, *[种植树]_{\text{\acs{vp}}} \translate{plant trees} 
is not grammatical
because in this \acs{vp} the verbal complex is heavier than the object, 
while it's the object that is supposed to receive prosodic focus, 
but [种植牙]_{\text{compound noun}} is perfectly fine. 

The prosodic constraint also doesn't apply to peripheral arguments.


\subsection{Non-existence of the serial verb construction}

The so-called serial verb constructions aren't mentioned here.
\citet{paul2008serial} and \citet[\citesec{9.4}]{deng2010formal} 
summarizes several constructions that are
frequently referred to as serial verb constructions,
and points out after deeper investigation,
they can all be described in terms of the usual complement clause constructions,
purpose clause constructions, etc. 
that are well attested cross-linguistically (\prettyref{sec:no-serial-verb}).



\section{The nucleus clause}



\subsection{General properties of the subject}\label{sec:simple-clause.nucleus.subject}

As is implied by my using the term \term{subject},
Mandarin is an typical accusative language.
The subject is recognizable from several properties listed below.

\subsubsection{Constituent order: subject before verb phrase}

The subject is always before the verb phrase (\ref{ex:get-sick}, \ref{ex:svo-example});
even when ``SOV'' order is obtained by the disposal construction 
(\prettyref{sec:verb-phrase.object.ba})
this is still the case \eqref{ex:ba-example}.
The subject therefore seems to be largely unaffected 
by the internal affairs within the verb phrase.

\begin{exe}
    \ex \gll 我 生病 了 \\
    1 get.sick \acs{sfp} \\
    \glt \translate{I got sick.}
    \label{ex:get-sick}

    \ex \gll [我]_{\text{subject}} 今天 去 看 [电影]_{\text{object}} 了 \\
    1 today to watch movie \acs{sfp} \\
    \glt \translate{I went to watch a movie today.} 
    \label{ex:svo-example}

    \ex \gll [我]_{\text{subject}} 今天 把 [ 一 个 碗 ]_{\text{object}} 摔 碎 了 \\
    1 today \category{ba} {} one \acs{classify} bowl {} break crack \acs{sfp} \\
    \glt \translate{I broke one bowl today.}
    \label{ex:ba-example}
\end{exe}

\subsubsection{Accusative alignment: scope}

The normal tests of syntactic accusative alignment can be run on Mandarin
(\ref{ex:inter-sentence}).

\begin{exe}
    \ex \gll 陈 经理 昨天 没有 和 他的 客户 聊 过 。 他 生病 了 。 \\
    {Chen (surname)} manager yesterday \acs{neg} with 3sg-\acs{possessive} client talk \acs{sfp}
    {} 3sg get.sick \acs{sfp} \\
    \glt \translate{Manager Chen didn't talk with his client yesterday. He (Chen, not his client) got sick.}
    \label{ex:inter-sentence}
\end{exe}

\subsubsection{Subject compared to topic}

\subsection{Verbal predicates}

The predicate may be a verbal one or a nominal one; 
the internal structure is discussed in \prettyref{sec:simple-clause.vp}.


\subsection{Nominal predicates}

Unmarked nominal predication is rare in modern Mandarin,
but does exist in certain cases.


\begin{exe}
    \ex 你这个笨蛋!
\end{exe}

\section{Negation}\label{sec:negation}

In Mandarin negation may modify the core \acs{vp} 
or a modal construction; 
it may also influence the structure of the verbal complex  
(\prettyref{sec:vp.neg}).
Several negation operators and strategies are used frequently.

There is another negation operator 没, 
which has subtle differences in its meaning and syntactic properties compared with 不
(\ref{ex:chiqincai}, \ref{ex:buchi-meichi}).
On the other hand, the negative potential complement construction,
i.e. the V不了 construction,
isn't obtained by inserting a negator in the clause \eqref{ex:zuobuliao-example}.

\begin{exe}
    \ex \begin{xlist}
        \ex \gll 我 不 喜欢 吃 芹菜 \\
        1 \acs{neg} like eat celery \\
        \glt \translate{I don't like eating celery.} \\
        \ex * 我 没 喜欢 吃 芹菜
    \end{xlist}
    \label{ex:chiqincai}
\end{exe}

\begin{exe}
    \ex \begin{xlist}
        \ex \gll 我 不 吃 早饭 \\
        1 \acs{neg} eat breakfast \\
        \glt \translate{I don't eat breakfast. (I usually don't, I don't want any today, etc.)}
        \ex \gll 我 没 吃 早饭 \\
        1 \acs{neg} eat breakfast \\
        \glt \translate{I didn't eat breakfast. (I may usually do, but somehow I didn't today.)}
    \end{xlist}
    \label{ex:buchi-meichi}
\end{exe}

\begin{exe}
    \ex \begin{xlist}
        \ex \gll 我 做 [ 不 了 ]_{\text{potential complement, negative}} 这 件 事 \\
        1 do {} \acs{neg} finish {} this \acs{classify} affair \\
        \glt \translate{I'm not able to do this.}
        \ex[*]{\gll 我 \oneof{没有/并非/不} 做 [ 得 了 ]_{\text{potential complement, positive}} 这 件 事 \\
        1 \acs{neg} do {} \asis{de} finish {} this \acs{classify} affair \\}
    \end{xlist}    
    \label{ex:zuobuliao-example}
\end{exe}

\section{Sentence final particles}\label{sec:sfp}

\section{The clause-final ``afterthought'' region}

In informal speech, 
some auxiliaries or adverbials may be postponed 
to the end of the clause 
looking like an afterthought (\ref{ex:simple-clause.afterthought.1}).
This afterthought region clearly follows the \acs{sfp}
(\ref{ex:simple-clause.afterthought.2}).

\begin{exe}
    \ex\label{ex:simple-clause.afterthought.1} \% 他吃完饭了可能
    \ex\label{ex:simple-clause.afterthought.2} *他吃完饭可能了
\end{exe}

\section{Minor constructions}

\subsection{Pseudo-attributive constructions}

One peculiar trait of Mandarin is that in some \acs{vp}s, 
we find attributive structures that are usually found in \acs{np}s.
These constructions can usually be seen as 

\subsection{\acs{np}-V-\acs{np}-\form{de}-X}

\begin{exe}
    \ex 你看你的书去
    \ex 你去当你的数学老师,不要来掺和这些琐事
\end{exe}

\subsection{The verb copying construction}

\subsection{The pseudo-numeral construction}

\begin{exe}
    \ex 我当了一年的程序员
    \ex 他们运了三个月的粮食,才算完成任务
\end{exe}

The pseudo-numeral construction often has an alternative semi-object form? TODO

\begin{exe}
    \ex 他们运粮食运了三个月
\end{exe}

\subsection{In subject (the so-called pseudo-attributive construction)}

\begin{exe}
    \ex 他的老师当得好
\end{exe}

There exist some examples in which no verb that is syntactically coreferential with the 
blank verb exists in each of them
(\prettyref{ex:verb-phrase.vp-argument.subject-semantic.1},
\prettyref{ex:verb-phrase.vp-argument.subject-semantic.2}; 
\citealt[\citepage{239}]{deng2010formal}).
Also, a general tendency is that abstract nouns are less acceptable 
in a pseudo-attributive construction 
(\ref{ex:verb-phrase.vp-argument.subject-semantic.3}).
These evidences supporting the analysis that the blank verb in the subject gerund 
is be recovered by semantic and pragmatic factors, 
not by structural considerations.

\begin{exe}
    \ex\label{ex:verb-phrase.vp-argument.subject-semantic.1} 
    \begin{xlist}
        \ex 他的周瑜还算是比较压场的
        \ex 他演周瑜还算是比较压场的
    \end{xlist}
    \ex\label{ex:verb-phrase.vp-argument.subject-semantic.2} 
    \begin{xlist}
        \ex \% 今天下午是谁的原告、谁的被告?
        \ex 今天下午是谁做原告、谁做被告?
    \end{xlist}
    \ex\label{ex:verb-phrase.vp-argument.subject-semantic.3} \begin{xlist}
        \ex 他这甲方当得真是恶劣 
        \ex *他这对方当得真是恶劣
    \end{xlist} 
\end{exe}

Despite the disputation on the details of the analysis 
of the so-called pseudo-attributive construction,
it can be clearly seen that 
the ``pseudo-attributive'' enjoys a similar role 
as the status of \form{his} in \form{his playing the national anthem won him applause}.
If we accept this status is a kind of attributives, 
then there is nothing ``pseudo''
with the attributive status of the pseudo-attributive
\citep[\citepage{242}]{deng2010formal}.






\chapter{Information structure}\label{chap:information-structure}

\section{The topic-comment construction}

I follow \citet{sih2000topic}'s approach and define a topic as an \acs{np} 
without extra syntactic marking 
that has certain relations with a position in the clause after it
and is indeed the topic in the information structure
(i.e. some (probably already known) object to which new information is added).
Constructions like 连\dots都\dots are not discussed in this section -- 
they are to be found in TODO: ref.

TODO: analysis of 这个能吃吗 -- without the presence of the modal verb (?) 能, the clause is no longer acceptable. Why? The clause can also be extended by an agentive \acs{np}, 
which apparently is the subject 这种植物一般人能吃吗
The initial \acs{np} then seems to be a topic.
Indeed, 这个吃吗 is acceptable in casual speech; 
it can be attested when two people are discussing 
on what to eat tonight, 
and one points at an item on the menu and 
propose it to another.

Topicalization can happen multiple times in informal speech;
the resulting utterance 
is often hard to parse and comprehend without context 
and is easily rejected as ungrammatical at the first glance, 
although the same native speaker may produce it hours ago.

\begin{exe}
    \ex \begin{exe}
        \ex 有的课老师作业答案自己也不知道的
        \ex 有的课老师自己也不知道作业答案的
        \ex 在有的课中,老师自己也不知大作业答案的
    \end{exe}
\end{exe}

\subsection{Topicalization of possessor}

\eqref{ex:tagezigaogaode} and \eqref{ex:tagezigaogaode} are a pair of sentences 
with and without topicalization of the possessor in the subject.

\begin{exe}
    \ex \begin{xlist}
        \ex\label{ex:tagezigaogaode}  
        \gll [他]_{\text{topic}} [[个子]_{\text{subject}} 高高 的 ]_{\text{comment}} \\
        3sg  stature tall\redp{}\asis{todo} \asis{de} \\
        \glt \translate{As for him, the stature is tall.}
        \ex\label{ex:tadegezigaogaode} \gll [ 他 的 个子 ]_{\text{subject}} 高高 的 \\
        {} 3sg \acs{possessive} stature {} tall\redp{}\asis{todo} \asis{de} \\
        \glt \translate{His stature is tall.}
    \end{xlist}
\end{exe}

\subsection{Topicalization of preposition objects}\label{sec:topicalization-of-preposition-objects}

\begin{exe}
    \ex\label{ex:zhejianshinibunengjiumafantayigeren} 这件事你不能就麻烦他一个人
    \ex 你不能[为了这件事]_{\text{adverbial:\acs{pp}}} 就麻烦他一个人
\end{exe}
This is also a demonstration of the preposition status of 在 in this sentence (\prettyref{sec:preposition-pos}),
because if it's a verb or an auxiliary verb,
it will be hard to have its object topicalized and have it deleted at the same time,
but deletion of the preposition in topicalization is well-attested cross-linguistically.

\subsection{Origins of so-called ``dangling topics''}\label{sec:topic-subject}

Some people, like \citet[\citesec{7.1}]{zhudexigrammar},
equate \term{subject} with \term{topic} in Mandarin grammar.
Some (especially those from the functional-typological tradition) go further 
and assert that ``the notion of the subject (as the position of the most agentive argument) 
isn't grammaticalized in Mandarin Chinese'',
and therefore the topic-comment construction 
is construed as simply the syntactic coding of aboutness,
and this base-generated and syntactically unconstrained topic 
is called a ``dangling topic''.
This view is rejected in this note,
because such accounts usually end up in severe overgeneration. 
Here I briefly summarize \citet{sih2000topic}'s argumentation.

\subsubsection{Type 1: Idiomatic phrasal predicate looking like a comment}\label{sec:clause.dangling-topic.1}

In the first type of ``dangling topic'',
it's impossible for any \acs{np} in the comment to be syntactically related to the topic
(\ref{ex:dayuchixiaoyu}, \ref{ex:nikankanwo}).
Such cases however should be analyzed as instances of the
subject-predicate construction,
where the predicate is a dephrasalized clause.

We notice that in such examples, 
the ``comment'' often has already undergone various degrees of fossilization.
Changing the comment usually makes the sentences much less felicitous 
(\ref{ex:dayuchixiaoyu-2}),
at best highly marked.
This is strange if the attested examples are topic-comment constructions,
but makes sense if dephrasalization is needed 
to put the clause 大鱼吃小鱼 etc. to the ``comment'' position.

Thus, in \eqref{ex:dayuchixiaoyu} and \eqref{ex:nikankanwo},
the so-called topic is an ordinary subject,
and the so-called comment is a predicate.
The meaning of the result of dephrasalization 
may be compared with the English colloquial 
\form{I was like, \dots} construction.

\begin{exe}
    \ex\label{ex:dayuchixiaoyu} 他们[大鱼吃小鱼](,厮杀成一片)
    \ex\label{ex:nikankanwo} 他们[你看看我我看看你]
    
    \ex\label{ex:dayuchixiaoyu-2} \begin{xlist}
        \ex *他们小鱼咬大鱼 
        \ex *他们虾米啃泥底
    \end{xlist}
\end{exe}

\subsubsection{Type 2: Quantificational adverbial looking like the inner subject}

The second type of ``dangling topic'' is like \eqref{ex:shui-dou-bu-pa}.
A topic-comment analysis of \eqref{ex:shui-dou-bu-pa} 

\begin{exe}
    \ex\label{ex:shui-dou-bu-pa} \gll 他们 谁 都 不 怕 \\
    3pl who even \acs{neg} fear \\
    \glt \translate{They don't fear anyone.}
\end{exe}

\subsubsection{Type 3: Ellipsis leaving a subject and one predicate}

Some people accept \eqref{ex:nasuofangzixingkuimeixiaxue}.
Here the \acs{np} 那所房子 definitely doesn't come from the words following it,
and is therefore recognized as a topic by some (TODO: ref). 
Note, however, that 幸亏 serves as a clause linker outside \eqref{ex:nasuofangzixingkuimeixiaxue}:
\eqref{ex:xingkui-buran-ex} is a demonstration of the 幸亏……不然…… linking construction,
and we also have its topicalized version \eqref{ex:xingkui-buran-fronted}. (TODO: whether this is parenthesis)
We also know in a clause linking construction,
often one clause can be omitted in the utterance because it's content can be easily inferred (TODO: ref).
So now the origin of \eqref{ex:nasuofangzixingkuimeixiaxue} is clear:
We can get it by omitting the second clause in the comment part of \eqref{ex:xingkui-buran-fronted}.
Indeed, if we replace 幸亏 by anything that is adverbial but not a clause linker,
the resulting sentence -- which now contains a real dangling topic -- is not grammatical.

\begin{exe}
    \ex \label{ex:nasuofangzixingkuimeixiaxue} \gll \% 那 所 房子 幸亏 没 下雪 \\
    {} that \acs{classify} house fortunate \acs{neg} snow \\
    \glt \translate{For that house, fortunately it didn't snow (or otherwise something bad would happen).}

    \ex\label{ex:xingkui-buran-ex} \gll [幸亏] 去年 没 下雪 , [不然] 那 所 房子 早就 塌 了 \\
    fortunate last.year \acs{neg} snow {} otherwise that \acs{classify} house already collapse \acs{sfp} \\
    \glt \translate{Fortunately it didn't snow last year, or otherwise that house has already collapsed.}

    \ex\label{ex:xingkui-buran-fronted} 
    \gll [ 那 所 房子 ]_{\text{topic}} [ 幸亏 去年 没 下雪 , 不然 早就 塌 了 ]_{\text{comment}} \\
    {} that \acs{classify} house {} {} fortunate last.year \acs{neg} snow {}  otherwise already collapse \acs{sfp} \\
\end{exe}

\subsubsection{Type 4: Extraction from prepositional adverbials}

\eqref{ex:zhejianshinibunengjiumafantayigeren} in \prettyref{sec:topicalization-of-preposition-objects} 
is sometimes regarded as an instance of the dangling topic construction.
However, as is shown in \prettyref{sec:topicalization-of-preposition-objects},
it may just be from topicalization of an \acs{np} in an adverbial,
with the preposition (and/or the locative particle) removed.

\subsubsection{Type 5: Nominal predicate}

\begin{exe}
    \ex 这种青菜一斤三十块钱
\end{exe}

\subsubsection{Type 6: Locational adverbial mistaken for the subject}

\begin{exe}
    \ex \gll \% 物价 纽约 最 贵  \\
    {} price New.York most expensive \\
    \glt \translate{The price in New York is the most expensive.}
\end{exe}

\subsubsection{Tentative conclusion}

The conclusion is all topics in Chinese are closely linked to a position in the comment,
be it a core argument position or a peripheral one.
So the notion of dangling topics is to be rejected in Mandarin grammar,
and we can always recover the ``canonical'' i.e. non-topic-comment clause
from a topic-comment construction.
After this, if the canonical clause can be divided into an \acs{np}
or a complement clause and a verbal constituent following it,
we can uncontroversially say the first is the subject while the second is the predicate. (TODO: predicate def)
So equating the subject with the topic is also wrong.

It's possible to find the semantic role of the subject isn't agentive;
in this case I assert there is a valency changing mechanism here.

\begin{infobox}{What to expect when people talk about the subject or the topic}{subject-topic}
    Unfortunately, despite the syntactic tests presented above,
    there are still many people -- even many native speakers -- 
    promoting the idea that the Mandarin topic has nothing different with the subject.
    Here is a list of TODO: ref
\end{infobox}

\section{Focusing}

The focused argument, when being a interrogative pronoun, 
should always be stressed. 

\begin{exe}
    \ex 我们 这里 什么 都 有
\end{exe}

\section{Cleft construction}

The cleft construction is formed by inserting 是 and 的 
before a clause and extracting one constituent in the clause before 是
(\ref{ex:clause.cleft.subject-1}, \ref{ex:clause.cleft.object-1}). 
The construction seems to be a biclausal construction, 
since it has the same structure 
as the ordinary predicative construction
(\ref{ex:clause.cleft.ordinary-1}). 
The preposed constituent is usually the subject
(\ref{ex:clause.cleft.subject-1}) 
or the direct object (\ref{ex:clause.cleft.object-1}; TODO: direct object def). 
Peripheral arguments are unable to be moved out. 

An interesting property of Mandarin cleft construction 
is what is focused -- semantically and phonologically -- is 
\emph{not} the preposed constituent, 
but the clause from which it's moved out
\citep[\citepage{108}]{huang2018handbook}. 

\begin{exe}
    \ex\label{ex:clause.cleft.subject-1} \begin{xlist}
        \ex 他昨天才知道这个消息
        \ex {} [他]_{\text{preposed}, i} 是 [---_i 昨天才知道这个消息] 的 
    \end{xlist}

    \ex\label{ex:clause.cleft.object-1} \begin{xlist}
        \ex 王教授把他招进来了
        \ex 他是王教授招进来的%
        \footnote{
            Here the verbal complex 招进来 is too heavy, 
            and the ba-construction is used to meet the prosodic constraint.
        }
    \end{xlist}
    \ex\label{ex:clause.cleft.ordinary-1} 天空是蓝色的
    
    \ex 是他昨天发现了这个故障
\end{exe}

\chapter{Relative clause constructions}\label{chap:relative}





\chapter{Complement clause constructions}\label{sec:complement-clause}


Finite-nonfinite distinction can often be found 
in complement clause constructions. 
This distinction is arguably absent in Mandarin,
even after detailed syntactic tests \citep{no-finite}.
Complement-taking verbs I have checked all allow 
appearance of modal and aspect marking in the complement clause; 
the so-called unacceptable appearance of modality or aspectuality 
in complement clauses that is sometimes invoked 
in the literature as evidence for nonfiniteness
seems to be purely semantic and pragmatic.

\begin{exe}
    \ex 我劝张三早点回家
    \ex 我劝张三吃了这碗饭
    \ex 那些不明白情况就劝你一定要大度的人,你要离他远一点
    \ex 这道题把我难得都要哭了
    \ex 这种重体力劳动弄得我一个胖子一个月瘦了十斤
    \ex 失败的人生让我每天都在床上哭
\end{exe}

\chapter{Coordination}\label{chap:clause-linking}

Mandarin Chinese has usual clause linker devices (\prettyref{chap:clause-linking}),
as well as complement clauses (\prettyref{sec:complement-clause})
and relative clauses (\prettyref{chap:relative}).
Fused-head relative clauses seem to be absent in Mandarin:
compare (\ref{ex:clause-combining.1}) with 
its English translation.

\begin{exe}
    \ex\label{ex:clause-combining.1} 
    \gll 我们 [在 [他们 昨天 扎营] 的 地方]_{\text{location}} 发现 了 熄灭的 篝火 \\
    \category{1pl} at \category{3pl} yesterday camp \category{rel} place 
    find \category{asp} extinguished bonfire \\ 
    \glt \translate{We found extinguished bonfires
    where they camped yesterday.}
\end{exe}

In the cause-consequence pair construction, for example,
the cause branch may be a clause or an \acs{np},
and the construction is therefore classified as subordination
for the obvious asymmetry between the two branches.

\begin{exe}
    \ex\label{ex:coordination.1} 
    \gll 因为 我 轻蔑 权威, 所以 命运 惩罚 我, 使 我 自己 竟 也 成 了 权威 \\
    because 1 contemn authority 
    therefore fortune punish 1 
    let 1 \category{refl} unexpectedly also become \category{asp} authority \\
    \translate{To punish me for my contempt for authority, fate made me an authority myself.
    (lit. Because I contemn the authority, 
    fortune punishes me and 
    unexpectedly makes me also an authority.)} (Quote from Einstein)
    \ex\label{ex:coordination.2} 因为我对权威的轻蔑,所以命运惩罚我,使我自己竟也成了权威
\end{exe}

\begin{figure}[H]
    \centering
    \begin{tikzpicture}[x=0.75pt,y=0.75pt,yscale=-0.8,xscale=0.8]
    %uncomment if require: \path (0,527); %set diagram left start at 0, and has height of 527
    
    %Straight Lines [id:da3715805745148597] 
    \draw    (69.83,331.94) -- (69.83,463.94) ;
    %Straight Lines [id:da2135419254533084] 
    \draw [color={rgb, 255:red, 189; green, 16; blue, 224 }  ,draw opacity=1 ][fill={rgb, 255:red, 189; green, 16; blue, 224 }  ,fill opacity=1 ]   (68.83,274.94) -- (135.83,244.94) ;
    %Shape: Triangle [id:dp7635220301199279] 
    \draw   (209.83,331.94) -- (256.83,462.94) -- (162.83,462.94) -- cycle ;
    %Straight Lines [id:da13568399614586713] 
    \draw    (205.83,274.94) -- (135.83,244.94) ;
    %Shape: Triangle [id:dp8714537495481391] 
    \draw   (561.35,394.94) -- (635.83,462.94) -- (484.83,462.94) -- cycle ;
    %Shape: Triangle [id:dp21340536727050363] 
    \draw   (419.05,394.94) -- (434.83,462.94) -- (402.83,462.94) -- cycle ;
    %Straight Lines [id:da08687604033684582] 
    \draw    (420.83,349.94) -- (485.83,321.94) ;
    %Straight Lines [id:da20577198722094936] 
    \draw    (561.83,349.94) -- (485.83,321.94) ;
    %Shape: Triangle [id:dp5453796206573645] 
    \draw   (329.79,325.6) -- (337.54,463.94) -- (321.83,463.94) -- cycle ;
    %Straight Lines [id:da7993985506417396] 
    \draw    (329.79,276.6) -- (404.83,245.94) ;
    %Straight Lines [id:da08378221000220742] 
    \draw    (485.79,276.6) -- (404.83,245.94) ;
    %Straight Lines [id:da309202803716613] 
    \draw [color={rgb, 255:red, 189; green, 16; blue, 224 }  ,draw opacity=1 ][fill={rgb, 255:red, 189; green, 16; blue, 224 }  ,fill opacity=1 ]   (135.83,195.94) -- (267.83,152.94) ;
    %Straight Lines [id:da66791443927485] 
    \draw [color={rgb, 255:red, 189; green, 16; blue, 224 }  ,draw opacity=1 ][fill={rgb, 255:red, 189; green, 16; blue, 224 }  ,fill opacity=1 ]   (407.83,195.94) -- (267.83,152.94) ;
    
    % Text Node
    \draw (267.83,149.94) node [anchor=south] [inner sep=0.75pt]   [align=left] {\begin{minipage}[lt]{100pt}\setlength\topsep{0pt}
    \begin{center}
    sentential clause:\\coordinated clause
    \end{center}
    
    \end{minipage}};
    % Text Node
    \draw (209.83,328.94) node [anchor=south] [inner sep=0.75pt]   [align=left] {\begin{minipage}[lt]{100pt}\setlength\topsep{0pt}
    \begin{center}
    core:\\nucleus clause
    \end{center}
    
    \end{minipage}};
    % Text Node
    \draw (53,464.23) node [anchor=north west][inner sep=0.75pt]   [align=left] {因为};
    % Text Node
    \draw (153,464.23) node [anchor=north west][inner sep=0.75pt]   [align=left] {我轻蔑权威};
    % Text Node
    \draw (69.83,328.94) node [anchor=south] [inner sep=0.75pt]  [color={rgb, 255:red, 189; green, 16; blue, 224 }  ,opacity=1 ] [align=left] {\begin{minipage}[lt]{100pt}\setlength\topsep{0pt}
    \begin{center}
    coordinator:\\因为
    \end{center}
    
    \end{minipage}};
    % Text Node
    \draw (135.83,241.94) node [anchor=south] [inner sep=0.75pt]  [color={rgb, 255:red, 189; green, 16; blue, 224 }  ,opacity=1 ] [align=left] {\begin{minipage}[lt]{100pt}\setlength\topsep{0pt}
    \begin{center}
    cause branch:\\clause
    \end{center}
    
    \end{minipage}};
    % Text Node
    \draw (390,464.23) node [anchor=north west][inner sep=0.75pt]   [align=left] {惩罚我};
    % Text Node
    \draw (465,464.23) node [anchor=north west][inner sep=0.75pt]   [align=left] {使我自己竟也成为了权威};
    % Text Node
    \draw (485.83,318.94) node [anchor=south] [inner sep=0.75pt]   [align=left] {\begin{minipage}[lt]{100pt}\setlength\topsep{0pt}
    \begin{center}
    predicate:\\coordinated VP
    \end{center}
    
    \end{minipage}};
    % Text Node
    \draw (561.35,391.94) node [anchor=south] [inner sep=0.75pt]   [align=left] {\begin{minipage}[lt]{100pt}\setlength\topsep{0pt}
    \begin{center}
    core:\\verb phrase
    \end{center}
    
    \end{minipage}};
    % Text Node
    \draw (419.05,391.94) node [anchor=south] [inner sep=0.75pt]   [align=left] {\begin{minipage}[lt]{100pt}\setlength\topsep{0pt}
    \begin{center}
    core:\\verb phrase
    \end{center}
    
    \end{minipage}};
    % Text Node
    \draw (310,464.23) node [anchor=north west][inner sep=0.75pt]   [align=left] {命运};
    % Text Node
    \draw (329.79,322.6) node [anchor=south] [inner sep=0.75pt]   [align=left] {\begin{minipage}[lt]{37.58pt}\setlength\topsep{0pt}
    \begin{center}
    subject:\\NP
    \end{center}
    
    \end{minipage}};
    % Text Node
    \draw (404.83,242.94) node [anchor=south] [inner sep=0.75pt]  [color={rgb, 255:red, 189; green, 16; blue, 224 }  ,opacity=1 ] [align=left] {\begin{minipage}[lt]{200pt}\setlength\topsep{0pt}
    \begin{center}
    consequence branch:\\clause
    \end{center}
    
    \end{minipage}};
    
    
    \end{tikzpicture}
    
    \caption{Analysis of (\ref{ex:coordination.1})}
\end{figure}

However, the 

TODO: 而 for modification 

\begin{exe}
    \ex 我因为怕冷,一直不喜欢游泳
    \ex 我平常喜欢游泳,不过昨天因为感冒了所以没有去游泳
    \ex 我因为怕冷而不喜欢游泳
\end{exe}

\chapter{Notable variances}

Sooner or later, a student of the Chinese language(s) will encounter 
linguistic phenomena deviating from the Standard Mandarin.
This note will not touch non-Mandarin topolects,
nor will it go deep into 

\section{Christian Mandarin}

\subsection{The language of Chinese Union Version bible}

The Chinese Union Version of bible, 
i.e. 和合本,
uses a variety of Mandarin that is quite distinct 
from Standard Mandarin in both lexicon and syntax.
(\ref{ex:variance.union-bible.1}) is a vivid illustration 
of how the word choices of Chinese Union Version differ 
from Standard Mandarin:
\begin{itemize}
    \item The verb 得 \translate{obtain} 
    is almost no longer in use in contemporary Mandarin;
    disyllabic verbs like 获得 or 得到 
    have filled its position.
    \item The intransitive valency of the verb 经营 \translate{manage}
    is rather unusual in modern Standard Mandarin:
    经营 in contemporary usage almost always selects an object, 
    as in 经营产业 \translate{manage businesses}.
    \item The noun (phrase? TODO) 恶人 \translate{bad person} 
    is still in use but its frequency 
    has reduced significantly;
    most frequently, it appears in idioms 
    that retain archaic styles, 
    as in 恶人先告状 \translate{The bad guy files a case (against others) first.} 
    \item The noun 果效 \translate{consequence, result}
    is a unique trait of Chinese Union Version: 
    it seems to be only attested in Chinese Union Version 
    and other bible translations affected by the former 
    as well as contemporary writings on Christianity.
    The noun 效果 is much more frequent in all other uses. 
\end{itemize}
A modern rephrase of (\ref{ex:variance.union-bible.1})
may look like (\ref{ex:variance.union-bible.1-modern});
however, most Christians would rather stay with 
the more archaic and weird-looking (\ref{ex:variance.union-bible.1-modern}),
since the modern version usually becomes less brief 
and, most importantly, shows no distinction
between mundane and spiritual affairs in its wording. 

\begin{exe}
    \ex\label{ex:variance.union-bible.1} 恶人经营, 得虚浮的工价; 撒义种的, 得实在的果效 (Proverbs 11:18)
    \ex\label{ex:variance.union-bible.1-modern} 罪恶的人经营产业,得到虚浮的报酬;播撒正义的种子的人,得到实在的效果
\end{exe}

The 

\subsection{Features of contemporary Chinese Christian writers}


\subsection{The Studium Biblicum Version}

\section{Early modern literature}

\section{Academic writing}

Scientific writing can be signified by the following traits:
\begin{itemize}
    \item Reduced usage of sentence final particles.
    \item In some cases, pro-drop clauses are used  
        to translate passive constructions. TODO: like 研究了一种……
\end{itemize}

\section{Contemporary internet languages}

\subsection{Cuteness, Japanese anime, and influence by translation}



\chapter{Summary and discussion}

\section{A typological summary}

\section{About the theoretical framework}

Many linguists call for a framework-less and completely open-minded approach towards syntactic analysis.
It's true (almost tautologically) that a grammar of a language 
should be organized according to the object language's 
own features.
Still, there exists the problem about \emph{how much} variation 
a linguist should expect when working with a totally unfamiliar language.

I'm not in the position to discuss whether the generative community 
is on the right track 
or whether the tendency to work on complex clause structures 
hinders the race against time to capture 
endangered languages.
What I do know -- which is illustrated in the discussion above -- 
is that kind of generativism I adopted in \prettyref{sec:theory}
does seem to work for Mandarin Chinese, 
despite the latter didn't play a strong role 
in the historical development of this framework.
We see the lexical-decomposition analysis 
and the \acs{vp}-shell theory 
neatly capture the structure of Mandarin \acs{vp}.
We see the category of clause can and should be 
further divided into subcategories with various internal complexities,
and the sizes of these subcategories can be placed on 
a monotonically increasing hierarchy,
which agrees well with the \vP-TP-CP hierarchy.
We see that on one hand, 
we can recognize grammatical words in Mandarin, 
and on the other hand, 
grammatical words are just mini-phrases.
And, most importantly, we have shown that most -- if not all -- mysterious traits 
of Mandarin have ingredients already well-known in other languages.
This is by no means a denial of linguistic diversity: 
on the contrary, 
that languages have choices over how to recombine these ingredients
helps us understand \emph{why} there is linguistic diversity at all.

\subsection{Necessity of large-volume grammars}

The next question is, 
since all natural languages have comparable complexity, 
whether the same thing should be done for less known languages.

\subsection{About how to teach Mandarin}

\printbibliography[title=References]

\end{document}