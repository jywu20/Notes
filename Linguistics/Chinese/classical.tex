\documentclass[UTF8, a4paper, oneside, scheme=plain, 12pt]{ctexrep}

\usepackage{libertinus}
\usepackage{geometry}
\usepackage{float}
\usepackage{titling}
\usepackage{titlesec}
\usepackage{paralist}
\usepackage{footnote}
\usepackage{enumerate}
\usepackage{amsmath, amssymb, amsthm}
\usepackage{gb4e}
\noautomath
\usepackage{bbm}
\usepackage{textcomp}
\usepackage{soul}
\usepackage{graphicx}
\usepackage{siunitx}
\usepackage[table,xcdraw]{xcolor}
\usepackage{tikz}
\usepackage[ruled, vlined, linesnumbered, noend]{algorithm2e}
\usepackage{xr-hyper}
\usepackage[colorlinks, citecolor = purple]{hyperref} % linkcolor=black, anchorcolor=black, citecolor=black, filecolor=black
\usepackage[most]{tcolorbox}
\usepackage{caption}
\usepackage{subcaption}
\usepackage{booktabs}
\usepackage{multirow}
\usepackage[figuresright]{rotating}
\usepackage{acro}
\usepackage[citestyle=authoryear,backend=bibtex,natbib=true,doi=false,isbn=false,url=false]{biblatex}
\addbibresource{references/classical-grammars.bib}
\addbibresource{references/general-typology.bib}
\usepackage{prettyref}

\geometry{left=3.18cm,right=3.18cm,top=2.54cm,bottom=2.54cm}
\titlespacing{\paragraph}{0pt}{1pt}{10pt}[20pt]
\setlength{\droptitle}{-5em}

\DeclareMathOperator{\timeorder}{\mathcal{T}}
\DeclareMathOperator{\diag}{diag}
\DeclareMathOperator{\legpoly}{P}
\DeclareMathOperator{\primevalue}{P}
\DeclareMathOperator{\sgn}{sgn}
\newcommand*{\ii}{\mathrm{i}}
\newcommand*{\ee}{\mathrm{e}}
\newcommand*{\const}{\mathrm{const}}
\newcommand*{\suchthat}{\quad \text{s.t.} \quad}
\newcommand*{\argmin}{\arg\min}
\newcommand*{\argmax}{\arg\max}
\newcommand*{\normalorder}[1]{: #1 :}
\newcommand*{\pair}[1]{\langle #1 \rangle}
\newcommand*{\fd}[1]{\mathcal{D} #1}

\newcommand*{\citesec}[1]{\S~{#1}}
\newcommand*{\citechap}[1]{Ch.~{#1}}
\newcommand*{\citechaps}[1]{Chs.~{#1}}
\newcommand*{\citefig}[1]{Fig.~{#1}}
\newcommand*{\citetable}[1]{Table~{#1}}
\newcommand*{\citepage}[1]{p.~{#1}}
\newcommand*{\citepages}[1]{pp.~{#1}}
\newcommand*{\citefootnote}[1]{fn.~{#1}}

\newrefformat{sec}{\citesec{\ref{#1}}}
\newrefformat{fig}{\citefig{\ref{#1}}}
\newrefformat{tbl}{\citetable{\ref{#1}}}
\newrefformat{chap}{\citechap{\ref{#1}}}
\newrefformat{fn}{\citefootnote{\ref{#1}}}
\newrefformat{box}{Box~\ref{#1}}
\newrefformat{ex}{\ref{#1}}

% color boxes

\tcbuselibrary{skins, breakable, theorems}

\newtcbtheorem[number within=chapter]{infobox}{Box}{
    enhanced,
    boxrule=0pt,
    colback=blue!5,
    colframe=blue!5,
    coltitle=blue!50,
    borderline west={4pt}{0pt}{blue!65},
    sharp corners,
    fonttitle=\bfseries, 
    breakable,
    before upper={\parindent15pt\noindent}}{box}
\newtcbtheorem[number within=chapter, use counter from=infobox]{theorybox}{Box}{
    enhanced,
    boxrule=0pt,
    colback=orange!5, 
    colframe=orange!5, 
    coltitle=orange!50,
    borderline west={4pt}{0pt}{orange!65},
    sharp corners,
    fonttitle=\bfseries, 
    breakable,
    before upper={\parindent15pt\noindent}}{box}
\newtcbtheorem[number within=chapter, use counter from=infobox]{learnbox}{Box}{
    enhanced,
    boxrule=0pt,
    colback=green!5,
    colframe=green!5,
    coltitle=green!50,
    borderline west={4pt}{0pt}{green!65},
    sharp corners,
    fonttitle=\bfseries, 
    breakable,
    before upper={\parindent15pt\noindent}}{box}
\newtcbtheorem[number within=chapter, use counter from=infobox]{todobox}{Box}{
    enhanced,
    boxrule=0pt,
    colback=red!5,
    colframe=red!5,
    coltitle=red!50,
    borderline west={4pt}{0pt}{red!65},
    sharp corners,
    fonttitle=\bfseries, 
    breakable,
    before upper={\parindent15pt\noindent}}{box}

\newcommand*{\concept}[1]{\textbf{#1}}
\newcommand*{\term}[1]{\emph{#1}}
\newcommand{\form}[1]{\emph{#1}}
\newcommand{\work}[1]{\textit{#1}}

\newcommand{\redp}{\textasciitilde}

\DeclareAcronym{blt}{short = BLT, long = Basic Linguistic Theory}
\DeclareAcronym{cgel}{short = CGEL, long = The Cambridge Grammar of the English Language}
\DeclareAcronym{dm}{short = DM, long = Distributed Morphology}
\DeclareAcronym{tag}{long = Tree-adjoining grammar, short = TAG}
\DeclareAcronym{sfp}{long = sentence-final particle, short = \textsc{sfp}}
\DeclareAcronym{np}{long = noun phrase, short = NP}
\DeclareAcronym{vp}{long = verb phrase, short = VP}
\DeclareAcronym{pp}{long = preposition phrase, short = PP}
\DeclareAcronym{cls}{long = classifier, short = CLS}
\DeclareAcronym{dist}{long = distal, short = DIST}
\DeclareAcronym{prox}{long = proximate, short = PROX}
\DeclareAcronym{dem}{long = demonstrative, short = DEM}
\DeclareAcronym{classify}{long = classifier, short = \textsc{cl}}
\DeclareAcronym{dur}{long = durative, short = DUR}
\DeclareAcronym{neg}{long = negative, short = \textsc{neg}}
\DeclareAcronym{cc}{long = copular complement, short = CC}
\DeclareAcronym{cs}{long = copular subject, short = CS}
\DeclareAcronym{tam}{long = {tense, aspect, mood}, short = TAM}
\DeclareAcronym{past}{long = past, short = PST}
\DeclareAcronym{nonpast}{long = non-past, short = NPST}
\DeclareAcronym{present}{long = present, short = PRES}
\DeclareAcronym{progressive}{long = progressive, short = \textsc{poss}}
\DeclareAcronym{perfect}{long = perfect, short = \textsc{perf}}
\DeclareAcronym{passive}{long = passive, short = \textsc{pass}}
\DeclareAcronym{copula}{long = copula, short = COP}
\DeclareAcronym{possessive}{long = possessive, short = \textsc{poss}}

\newcommand{\asis}[1]{\textsc{#1}}
\newcommand{\oneof}[1]{{#1}}
\newcommand*{\homo}[2]{#1$_{\text{#2}}$}

\newcommand{\cgel}{\href{../English/cambridge.pdf}{my notes about CGEL}}
\newcommand{\latin}{\href{../Latin/latin-notes.pdf}{my notes about Latin}}
\newcommand{\alignment}{\href{../alignment/alignment.pdf}{my notes about alignment}}
\newcommand{\exerciseone}{\href{../Exercise/2021-3.pdf}{this exercise}}
\newcommand{\method}{\href{../methodology/glossing.pdf}{this note about my understanding of descriptive grammars}}

\newcommand{\ala}{à la}
\newcommand{\translate}[1]{`#1'}
\newcommand{\vP}{\textit{v}P}
\newcommand*{\category}[1]{\textsc{#1}}
\newcommand*{\specialunit}[1]{$<$\textit{#1}$>$}
\newcommand{\before}{$>$}

% Make subsubsection labeled
\setcounter{secnumdepth}{4}
\setcounter{tocdepth}{4}
% reset example counter every chapter (but do not include the chapter number to the label)
\counterwithin{exx}{chapter} 

% Reference formats
\renewcommand*{\nameyeardelim}{\space} % No comma between year and name
\DeclareNameAlias{sortname}{family-given} % Putting the family name before the given name
\DeclareNameAlias{default}{family-given} 
\DeclareFieldFormat{labelnumberwidth}{} % No number label like [12] in the reference list
\setlength{\biblabelsep}{0pt} % No space for these labels

\makeindex

\title{Notes about Classical Chinese}
\author{Jinyuan Wu}

\begin{document}

\automath

\maketitle

\chapter{Introduction}

This note is about Classical Chinese,
the high variety of more than two millennia of diglossia in China.

\section{The name of the language}

The language is known natively as 文言 (\translate{lit. cultured speech})
or sometimes 古文 (\translate{lit. ancient articles})
or 古汉语 (\translate{lit. ancient Chinese}).
Note that there were several stages of the development of Chinese
and Classical Chinese is mostly (but not completely) based on Old Chinese
(\prettyref{sec:introduction.history}).

The language is sometimes known as \form{Wen-li} by Western missionaries,
especially in Bible translation.
This seems to be a misunderstanding of the word 文理,
which is a nominal compound and means rhetorics (i.e. 文) and meanings (i.e. 理) of literature works.
An educated person therefore would be described as ``通文理'' (\translate{fluent in rhetorics and meanings}).
Such a person of course would have decent understanding of Classical Chinese,
and hence 文理 was probably mistranslated as ``Classical Chinese'',
although the word was not natively used to refer to the latter.

\section{Historical background}\label{sec:introduction.history}

Since there was no attempt at explicit and systematic grammatical standardization
(\prettyref{sec:introduction.previous.tradition}),
prescriptive authority of Classical Chinese was a collection of canonical literature works
consensually regarded as classical (\prettyref{sec:introduction.text}).
The whole canon was finished before the collapse of Han
and therefore falls under the term Old Chinese.
Both temporal and regional variances can be observed in Old Chinese texts, though,
and not all varieties contribute to Classical Chinese equally.
In this section, we briefly overview the history of Sinitic language(s)
and analyze how they shape Classical Chinese.

\subsection{Pre-classical period}

The earliest attested Sinitic texts were oracle bone inscriptions,
a 20th century archeological re-discovery not known to Classical Chinese authors.
For them, the earliest available texts are 
documents preserved in 《尚书》 (lit. \translate{venerated documents}),
often known as \work{Book of Documents} in English.
Since these texts are from ancient kings 
whose deeds were romanticized by Confucian scholars,
these texts were highly venerated and yet deemed as 诘屈聱牙 (\translate{twisted, hard to pronounce}).
They were something that had to be read with commentaries,
the latter written in easier Classical Chinese.
These documents therefore should be regarded as pre-Classical,
although they did contribute sporadic phrases
and grammatical words (e.g. the copula 惟 or the pronoun 厥)
that were occasionally used in Classical Chinese works as a way to polish an article.

One thing worth mentioning is that 
the language of \work{Book of Documents} and the language of oracle bone inscriptions are not identical:
the aforementioned pronoun 厥 appears frequently in \work{Book of Documents},
but it appears neither in oracle bone inscriptions nor in Spring and Autumn works. 
Possibly, \work{Book of Documents} contains predominantly early Zhou dynasty texts,
while oracle bones dates back to Shang,
and the differences we are observing reflect dialectal differences between the ruling classes of the two dynasties.

Another fairly early source is 《诗经》 (lit. \translate{poem classics}),
also known as \work{Odes},
which contains poems dates back to as early as early Zhou.

\subsection{Spring and Autumn and Warring States}

The majority of treatise texts that shaped Classical Chinese prose
were written when Zhou was already weakened
and the Spring and Autumn period, a time filled with chaotic wars between dukedoms, already started.
The language of this period diverges tremendously from the pre-Classical period.
For example, the copula 惟 had died out of use 
and the copula construction had been largely replaced by the nominal predication construction
(\prettyref{sec:grammatical.clause.nominal}).
The lexicon also underwent huge changes.

There are clues suggesting regional variances.
Students of Confucius noticed that when he recited Classical texts and presided rituals,
he used 雅言 or \translate{elegant speech} (\work{Analects} 7:18).
This suggests at a possible diglossia at his age,
with the ``elegant speech'' conceivably being the language of intellectuals of Zhou Dynasty,
although the differences between the high and low varieties 
definitely would be much smaller than, say, the differences 
between Standard Mandarin and Classical Chinese,
as is shown by comparison between \work{Analects} and \work{Odes}.

More solid evidence for regional variance is found by comparing
楚辞 \work{Verses of Chu} and the linguistic mainstream.
The former shows a Kra–Dai substrate.

\begin{todobox}{Chu dialect}{chu-dialect}
    Find references.
\end{todobox}

\subsection{Han dynasty}

The last batch of uncontroversially classical works were composed during Han dynasty,
among them the most important being \work{Records of the Grand Historian}.
The language of \form{Records of the Grand Historian} shows notable but largely qualitative differences
compared with earlier historical works,
the most important one being 《左传》.
Notable changes include more pre-verbal adverbials,
reduction of prepositional verbs,
regularization of constituent orders,
and also proliferation of disyllable words
It is therefore suggested that Han dynasty texts and pre-Qin texts 
reflect two stages of post-Zhou developments of Chinese,
although the change was not as radical as the change from \form{Book of Documents} to Spring and Autumn texts 
\citep[\citepages{260-264}]{he2005shiji}.

\subsection{Post-Classical influences}

Expectedly, despite purification attempts,
vernacular elements made their ways into not only administrative documents
but also pure literature and scholar works.
Classical Chinese or 文言, in the broadest sense,
is a term that covers all genres whose grammars are roughly based on the Old Chinese canon
but may have a varieties of innovations.

\section{Texts}\label{sec:introduction.text}

The great historical work 《史记》 (\translate{lit. historical records}),
often known as \form{Records of the Grand Historian} in English
(a translation of 太史公记, the earliest known title of the work),
laid the paradigm of official historiography of all Chinese dynasties after Han.
The author 司马迁 \form{Sima Qian} is known as the \form{Lord Grand Historian} or 太史公.
太史 \translate{grand historian} was the title of



\section{Previous studies}

\paragraph*{Grammatical traditions}\label{sec:introduction.previous.tradition}
Classical Chinese authors had conversations about grammaticality 
and uses of grammatical particles.
These discussions are reminiscent of how English native speakers
with some exposure to the study of English grammar but no formal training:
``delete the \form{the} here and your sentence looks more concise''.
No attempts were made to extract grammatical relations or structural templates from the surface forms
and to organize the grammar as a machine producing acceptable utterances:
discussions on grammatical topics were either for education or for rhetorics.
The situation is not unlike what a Roman grammarian or \form{grammaticus} did

\begin{todobox}{Ancient Chinese grammatical tradition and Roman tradition}{china-rome-compare}
    Is the situation somehow close to what a Roman grammarian (\form{grammaticus}) would do?
    It seems that Roman grammarians also didn't care about abstract structures.
    See:
    \begin{itemize}
        \item Use and Function of Grammatical Examples in Roman Grammarians
        \item Quintilian’s ‘Grammar’ (Inst.1.4-8) and its Importance for the History of Roman Grammar
        \item What Graeco-Roman Grammar was about
    \end{itemize}
\end{todobox}

On the other hand, phonology was an active topic in ancient China.
This was probably due to the influence of 

\paragraph*{Perspectives of European missionaries}

\paragraph*{Modern descriptions}

\chapter{Grammatical overview}

\section{The overall clausal structure}

Like all other languages, a Classical Chinese clause can be a simple clause
or a complex one constructed from subordination and coordination \citep[\citechaps{3-5}]{meiguang2018}.
A simple Classical Chinese clause is a nucleus,
which may be either a verbal predication construction
or a nominal predication construction,
plus possible sentence final particles and/or topicalization.
Topicalization can also happen for a complex clause \citep[\citechap{4} \citesec{3.3}]{meiguang2018}.
It appears that all embedded clauses in Classical Chinese cannot have discourse-related devices 
like topicalization and sentence final particles.

\subsection{Nominal predication}\label{sec:grammatical.clause.nominal}

The top-level structure of a Classical Chinese clause may contain only two noun phrases/fused relative clauses
(\ref{ex:grammatical.clause.nominal.isa.1},
\ref{ex:grammatical.clause.nominal.havea.1}).
A nominal predication may express an ``is-a'' relation between the subject and the subject complement,
which is the case of (\ref{ex:grammatical.clause.nominal.isa.1}).
Some nominal predication constructions however express a possessive relation between the two 
(\ref{ex:grammatical.clause.nominal.havea.1}).

\begin{exe}
    \ex\label{ex:grammatical.clause.nominal.isa.1} [良人者]_{\text{subject}} [所仰望而终身]_{\text{predicate}} 也
    \ex\label{ex:grammatical.clause.nominal.havea.1} 蟹六跪而二螯
\end{exe}

All the constructions mentioned above are without a copula.
In the pre-Classical copula age there is a copula 惟,
which however had largely died out of use in Classical texts.

\subsection{The nucleus with verbal predication}

\paragraph*{Core grammatical relations} 
In the case of verbal predication
a nucleus is a subject plus either a nominal or verbal predication,
with possible \ac{tam} and peripheral argument modifications.
The constituent order is almost always SVO
(\ref{ex:grammatical.clause.svo.declarative.1}, \ref{ex:grammatical.clause.svo.interrogative.1});
SOV is however attested in negative (\ref{ex:grammatical.clause.sov.neg.1})
or interrogative situations (\ref{ex:grammatical.clause.sov.interrogative.1}).

\begin{exe}
    \ex\label{ex:grammatical.clause.svo.declarative.1} 
    [子张]_{\text{subject}} [学]_{\text{verb}} [干禄]_{\text{object}}
    \ex\label{ex:grammatical.clause.svo.interrogative.1} 
    [子]_{\text{subject}} [奚]_{\text{reason}} 不 [为]_{\text{verb}} [政]_{\text{object}}
    \ex\label{ex:grammatical.clause.sov.neg.1} 
    恐 [年岁 之 [不吾与]_{\text{verbal predication: Neg-OV}}]_{\text{complement clause}}
    \ex\label{ex:grammatical.clause.sov.interrogative.1} 
    以五十步笑百步,则 [何如]_{\text{SOV clause}}
\end{exe}

\paragraph*{Complex predicates} 

\begin{todobox}{Classical Chinese complex predicate}{cp}
    Directional complement and resultative complement
\end{todobox}

\paragraph*{Peripheral arguments and \ac{tam} marking} 
Adverbial constituents in the nucleus can be divided into \ac{tam} ones 
and so-called peripheral arguments, including location, manner, instrument, etc.
The peripheral arguments are usually post-verbal
(\ref{ex:grammatical.clause.peripheral.postverbal.1},
\ref{ex:grammatical.clause.peripheral.postverbal.2},
\ref{ex:grammatical.clause.peripheral.postverbal.3}).
Pre-verbal peripheral arguments are however still possible
(\ref{ex:grammatical.clause.peripheral.preverbal.1},
\ref{ex:grammatical.clause.peripheral.preverbal.2}).

\begin{exe}
    \ex\label{ex:grammatical.clause.peripheral.postverbal.1} 侍饮于长者
    \ex\label{ex:grammatical.clause.peripheral.postverbal.2} 孟孙问孝于我
    \ex\label{ex:grammatical.clause.peripheral.postverbal.3} 祷尔于上下神祇
    \ex\label{ex:grammatical.clause.peripheral.preverbal.1} 韩生南向坐
    \ex\label{ex:grammatical.clause.peripheral.preverbal.2} 於人之罪无所忘
\end{exe}

The \ac{tam} adverbials are almost always preverbal.

\begin{exe}
    \ex 文王既没,文不在兹乎
    \ex 孔子既得合葬于防
    \ex 我未之能易也
\end{exe}

When \ac{tam} adverbs and peripheral arguments both appear before the verb,
the order is always \ac{tam} \before peripheral argument.
The reverse order is never attested.

\begin{exe}
    \ex 三王既以定法度
\end{exe}

\begin{todobox}{Adverbials combination}{adverbial-combine}
    Is it possible to use multiple pre-verbal adverbials?
    What's the relevant order constraint?
\end{todobox}

\subsection{Sentence final particles}

Classical sentence final particles have a variety of functions.
It may mark the interrogative force (\ref{ex:grammatical.clause.sfp.interrogative.1}), 
a judgemental meaning (\ref{ex:grammatical.clause.sfp.judgement.1}),
and aspectual values (\ref{ex:grammatical.clause.sfp.aspectual.1}).

\begin{exe}
    \ex\label{ex:grammatical.clause.sfp.interrogative.1} 大车无輗,小车无軏,其何以行之哉
    \ex\label{ex:grammatical.clause.sfp.judgement.1} 人而无信,不知其可也
    \ex\label{ex:grammatical.clause.sfp.aspectual.1} 温故而知新,可以为师矣
\end{exe}

\subsection{Topicalization}

\begin{todobox}{Topic and subject}{topic-subject}
    What's the relation between the topic and the subject?
\end{todobox}

\section{The noun phrase}

The Classical Chinese \ac{np} has a 

\chapter{Look-up tables for particles}

\paragraph*{者} The particle 者 most frequently appears as a relativizer, a complementizer,
or in the \form{zhe}-\form{ye} construction.
The three functions can be uniformly analyzed as the function of a low-level determiner \citep{aldridge2009old}. 

\printbibliography[title=References]

\end{document}