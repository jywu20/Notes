\documentclass[UTF8, a4paper, oneside, scheme=plain, 12pt]{ctexrep}

\usepackage{libertinus}
\usepackage{geometry}
\usepackage{float}
\usepackage{titling}
\usepackage{titlesec}
\usepackage{paralist}
\usepackage{footnote}
\usepackage{enumerate}
\usepackage{amsmath, amssymb, amsthm}
\usepackage{gb4e}
\noautomath
\usepackage{bbm}
\usepackage{textcomp}
\usepackage{soul}
\usepackage{graphicx}
\usepackage{siunitx}
\usepackage[table,xcdraw]{xcolor}
\usepackage{tikz}
\usepackage[ruled, vlined, linesnumbered, noend]{algorithm2e}
\usepackage{xr-hyper}
\usepackage[colorlinks, citecolor = purple]{hyperref} % linkcolor=black, anchorcolor=black, citecolor=black, filecolor=black
\usepackage[most]{tcolorbox}
\usepackage{caption}
\usepackage{subcaption}
\usepackage{booktabs}
\usepackage{multirow}
\usepackage[figuresright]{rotating}
\usepackage{acro}
\usepackage[citestyle=authoryear,backend=bibtex,natbib=true,doi=false,isbn=false,url=false]{biblatex}
\addbibresource{references/grammars.bib}
\addbibresource{references/aspects.bib}
\addbibresource{references/general-typology.bib}
\addbibresource{references/controversy.bib}
\usepackage{prettyref}

\geometry{left=3.18cm,right=3.18cm,top=2.54cm,bottom=2.54cm}
\titlespacing{\paragraph}{0pt}{1pt}{10pt}[20pt]
\setlength{\droptitle}{-5em}

\DeclareMathOperator{\timeorder}{\mathcal{T}}
\DeclareMathOperator{\diag}{diag}
\DeclareMathOperator{\legpoly}{P}
\DeclareMathOperator{\primevalue}{P}
\DeclareMathOperator{\sgn}{sgn}
\newcommand*{\ii}{\mathrm{i}}
\newcommand*{\ee}{\mathrm{e}}
\newcommand*{\const}{\mathrm{const}}
\newcommand*{\suchthat}{\quad \text{s.t.} \quad}
\newcommand*{\argmin}{\arg\min}
\newcommand*{\argmax}{\arg\max}
\newcommand*{\normalorder}[1]{: #1 :}
\newcommand*{\pair}[1]{\langle #1 \rangle}
\newcommand*{\fd}[1]{\mathcal{D} #1}

\newcommand*{\citesec}[1]{\S~{#1}}
\newcommand*{\citechap}[1]{chap.~{#1}}
\newcommand*{\citefig}[1]{Fig.~{#1}}
\newcommand*{\citetable}[1]{Table~{#1}}
\newcommand*{\citepage}[1]{p.~{#1}}
\newcommand*{\citepages}[1]{pp.~{#1}}
\newcommand*{\citefootnote}[1]{fn.~{#1}}

\newrefformat{sec}{\citesec{\ref{#1}}}
\newrefformat{fig}{\citefig{\ref{#1}}}
\newrefformat{tbl}{\citetable{\ref{#1}}}
\newrefformat{chap}{\citechap{\ref{#1}}}
\newrefformat{fn}{\citefootnote{\ref{#1}}}
\newrefformat{box}{Box~\ref{#1}}
\newrefformat{ex}{\ref{#1}}

\newcommand*{\textgt}{$>$ }
\newcommand*{\textlt}{$<$ }
\newcommand*{\textto}{$\to$ }

% color boxes

\tcbuselibrary{skins, breakable, theorems}

\newtcbtheorem[number within=chapter]{infobox}{Box}{
    enhanced,
    boxrule=0pt,
    %colback=blue!5,
    %colframe=blue!5,
    colback=white,
    colframe=white,
    coltitle=blue!60,
    borderline west={4pt}{0pt}{blue!65},
    sharp corners,
    fonttitle=\bfseries, 
    breakable,
    before upper={\parindent15pt\noindent}}{box}
\definecolor{my-orange}{HTML}{F58123}
\newtcbtheorem[number within=chapter, use counter from=infobox]{theorybox}{Box}{
    enhanced,
    boxrule=0pt,
    %colback=orange!5, 
    %colframe=orange!5, 
    colback=white,
    colframe=white,
    coltitle=my-orange!65,
    borderline west={4pt}{0pt}{my-orange!65},
    sharp corners,
    fonttitle=\bfseries, 
    breakable,
    before upper={\parindent15pt\noindent}}{box}
\newtcbtheorem[number within=chapter, use counter from=infobox]{todobox}{Box}{
    enhanced,
    boxrule=0pt,
    colback=red!5,
    colframe=red!5,
    coltitle=red!50,
    borderline west={4pt}{0pt}{red!65},
    sharp corners,
    fonttitle=\bfseries, 
    breakable,
    before upper={\parindent15pt\noindent}}{box}
\newtcbtheorem[number within=chapter, use counter from=infobox]{perspectivebox}{Box}{
    enhanced,
    boxrule=0pt,
    %colback=red!5,
    %colframe=red!5,
    colback=white,
    colframe=white,
    coltitle=red!50,
    borderline west={4pt}{0pt}{red!65},
    sharp corners,
    fonttitle=\bfseries, 
    breakable,
    before upper={\parindent15pt\noindent}}{box}

\AtBeginEnvironment{infobox}{\small}
\AtBeginEnvironment{todobox}{\small}
\AtBeginEnvironment{theorybox}{\small}

\newcommand*{\concept}[1]{\textbf{#1}}
\newcommand*{\term}[1]{\emph{#1}}
\newcommand{\form}[1]{\emph{#1}}
\newcommand{\work}[1]{\textit{#1}}

\newcommand{\redp}{\textasciitilde}

\DeclareAcronym{blt}{short = BLT, long = Basic Linguistic Theory}
\DeclareAcronym{cgel}{short = CGEL, long = The Cambridge Grammar of the English Language}
\DeclareAcronym{dm}{short = DM, long = Distributed Morphology}
\DeclareAcronym{tag}{long = Tree-adjoining grammar, short = TAG}
\DeclareAcronym{sfp}{long = sentence-final particle, short = \textsc{sfp}}
\DeclareAcronym{np}{long = noun phrase, short = NP}
\DeclareAcronym{vp}{long = verb phrase, short = VP}
\DeclareAcronym{pp}{long = preposition phrase, short = PP}
\DeclareAcronym{cls}{long = classifier, short = CLS}
\DeclareAcronym{dist}{long = distal, short = DIST}
\DeclareAcronym{prox}{long = proximate, short = PROX}
\DeclareAcronym{dem}{long = demonstrative, short = DEM}
\DeclareAcronym{classify}{long = classifier, short = \textsc{cl}}
\DeclareAcronym{dur}{long = durative, short = DUR}
\DeclareAcronym{neg}{long = negative, short = \textsc{neg}}
\DeclareAcronym{cc}{long = copular complement, short = CC}
\DeclareAcronym{cs}{long = copular subject, short = CS}
\DeclareAcronym{tame}{long = {tense, aspect, mood, evidentiality}, short = TAME}
\DeclareAcronym{past}{long = past, short = PST}
\DeclareAcronym{nonpast}{long = non-past, short = NPST}
\DeclareAcronym{present}{long = present, short = PRES}
\DeclareAcronym{progressive}{long = progressive, short = \textsc{poss}}
\DeclareAcronym{perfect}{long = perfect, short = \textsc{perf}}
\DeclareAcronym{passive}{long = passive, short = \textsc{pass}}
\DeclareAcronym{copula}{long = copula, short = COP}
\DeclareAcronym{possessive}{long = possessive, short = \textsc{poss}}

\newcommand{\asis}[1]{\textsc{#1}}
\newcommand{\oneof}[1]{{#1}}
\newcommand*{\homo}[2]{#1$_{\text{#2}}$}

\newcommand{\ala}{à la}
\newcommand{\translate}[1]{`#1'}
\newcommand{\vP}{\textit{v}P}
\newcommand*{\category}[1]{\textsc{#1}}
\newcommand*{\wordroot}[1]{$\sqrt{\text{#1}}$}
\newcommand*{\specialunit}[1]{$<$\textit{#1}$>$}

% Make subsubsection labeled
\setcounter{secnumdepth}{4}
\setcounter{tocdepth}{4}
% reset example counter every chapter (but do not include the chapter number to the label)
\counterwithin{exx}{chapter} 

% Reference formats
\renewcommand*{\nameyeardelim}{\space} % No comma between year and name
\DeclareNameAlias{sortname}{family-given} % Putting the family name before the given name
\DeclareNameAlias{default}{family-given} 
\DeclareFieldFormat{labelnumberwidth}{} % No number label like [12] in the reference list
\setlength{\biblabelsep}{0pt} % No space for these labels

\title{Mandarin Chinese notes}
\author{Jinyuan Wu}

\begin{document}

\automath

\maketitle

\chapter{Introduction}

\section{History and classification}

Being historically accurate or not, we can at least say that Middle Chinese, as described in the rhyme books and rhyme tables,
\emph{is} a well-defined proto-language in terms of Neogrammarian historical linguistics,
because the phonology modern Sinitic languages can be derived by regularly applying sound change laws to it.

\section{Methodology}

\subsection{Theoretical framework}

{
\small
The underlying theoretical framework of this grammar is Distributed Morphology \citep{siddiqi2009syntax} and Cartography \citep{cinque1999adverbs}.
This is to say, we assume that
\begin{itemize}
    \item[(a)] a grammatical construction is made of a \emph{root} surrounded by a hierarchy of \emph{functional heads} (corresponding to grammatical features and categories in traditional grammar) and their specifiers (in the generative syntactic sense), and 
    \item[(b)] the functional heads often follow a relatively cross-linguistically stable hierarchy,
    which strongly influences the linear order of auxiliaries, adverbs (which are in specifier positions), etc. and their scopes
    (e.g. \prettyref{fig:vp.ex.1}), and 
    \item[(c)] post-syntactic morphological and phonological operations can bring roots and functional heads together,
    which is subject to \emph{phase theory} (or similar cyclic constraints of syntactic derivation),
    which then are subject to phonological realization, and 
    \item[(d)] the whole process is guided by the lexicon,
    in which the list of roots, idiomized (i.e. \term{lexicalized}) meanings of constructions,
    and details of phonological realization
    (which, by the way, gives us subcategorization:
    a root that can only be phonologically realized with the \category{transitive} functional head
    which introduces the direct object is a transitive verb; see \citealt{siddiqi2009syntax}),
    and the last two lists in theory can be independent to each other.
\end{itemize}

Grammar, in this framework, is about how roots are ``dressed up'' by grammatical constructions,
and grammatical constructions can be fully described
in terms of hierarchical organization of grammatical categories based on constituency relations, and their phonological realization.
The framework adopted here therefore has a clear lexical/functional distinction,
which can be tested by checking whether a formative is within a fixed hierarchy 
(e.g. \form{have been being observed} and the similar ordering of tense, aspect and modality adverbs in English: \category{tense} \textgt{}\category{perfect} \textgt{}\category{progressive} \textgt{}\category{passive}; 
in Mandarin see e.g. \prettyref{sec:grammatical.clause.tam}).
Certain gradience however is allowed because multiple analyses may appear at the same time
in the mental grammar of a speaker of the language.

\subsection{Relation with descriptive grammars}

The framework above may seem strange for descriptive linguists.
Here we ``explain'' concepts in descriptive grammars in terms of concepts in the framework
of Distributed Morphology plus Cartography.

\begin{itemize}
    \item[(a)] Definition of head. \term{Heads} in our framework are all \emph{functional} heads,
    i.e. (markers of) grammatical categories.
    Under the more traditional definition of \term{head},
    we have \emph{noun} phrases and \emph{verb} phrases,
    and we have constructions without heads, like coordination constructions.
    In our framework, this traditional definition of \term{head} also makes sense:
    it corresponds to the core i.e. the center root of a construction,
    which, of course, cannot always be non-ambiguously defined in constructions like coordination.

    This traditional usage of the term \term{head} appears in the generative literature as well:
    \citet[\citepage{120}]{paul2014new}, for instance, talks about the \term{head noun} of something he calls a DP.
    The two definitions of \term{head} overlap in grammaticalization:
    a so-called prepositional phrase can be a complement-taking adverb phrase
    in which the \term{head preposition} is a head in the traditional sense,
    or an analytic case phrase, where the \term{head preposition} is actually a case particle and a functional head.
    The former can be reanalyzed as the latter.
    On the other hand, some functional heads, like sentence final particles
    (see e.g. \prettyref{fig:grammatical.clause.high-level.topic.1}),
    are never recognized as heads in traditional grammars:
    they are instead called \term{markers}.

    \item[(b)] Constituency and dependency relations.
    In mainstream generative syntax, constituency (or more abstractly, c-command relations) is the only primitive for structure building.
    Moreover, because of phase theory (or alternative theories aiming to explain the relevant phenomena),
    constituency is \term{cyclic} or ``layered'':
    thus in a clause, the vP is finished first, followed by TP and CP.
    This provides an alternative way to define constituents,
    in which the main verb (on the notion of what is a verb or more generally what is a word, see below),
    and the tense, aspect and modality markers, but \emph{not} arguments, are put into one unit,
    often called the verb phrase \citep[\citepage{39}]{quirk1985}.
    Still the hierarchical relations between the functional heads 
    (e.g. \category{tense} \textgt{}\category{perfect} \textgt{}\category{progressive})
    exist and need to be accounted for, so we see discussions on them in \citet[\citepages{79,121}]{quirk1985}.
    The flat-tree analysis in \citep[\citepage{39}]{quirk1985} makes sense in our framework too.
    We can see this in \prettyref{fig:vp.ex.1} as well.
    
    Another issue is the relation between constituency and dependency.
    The two are basically two notations for the same thing \citep{boston2009dependency}.
    Here hierarchical relations can be represented by assigning ``closeness'' values to dependency arcs.
    Dependency analyses are particularly wieldy when movements are frequent.
    \citet[\citepage{55}]{cgel}, for instance, mentions \term{indirect complements},
    which are originally a part of the core argument structure of an adjective
    in a noun phrase and has to appear post-nominally.
    Its relation with the adjective that licenses it therefore is ideally reflected 
    by a dependency arc in a descriptive grammar,
    although a movement-based constituency analysis is of course possible.
    
    \item[(c)] Pre-defined grammatical constructions.
    In lexicalist schools of generative grammar,
    we have the X-bar theory, in which some heads are heads in the sense of (a).
    In lexicalist X-bar theory, the distinction between adjuncts, specifiers and complements
    are used to explain their differences in their syntactic behaviors.
    This distinction is absent in Cartography,
    as almost all things are specifiers, and the distinction is to be reinterpreted
    as the distinction between different types of specifiers.
    So the X-bar scheme can be seen as a pre-defined schema of grammatical constructions.
    
    We can further derive more grammatical structures 
    (like subject-predicate structures, verb-object structures, coordination structures, and so on)
    by incorporating the vP-TP-CP and the nP-NumP-DP hierarchies into the X-bar scheme,
    which is exactly what is done in \citet{deng2010formal},
    which results in a descriptive formalism quite similar to that in \citet{cgel}:
    we replace TopP by a \term{form} label \term{topic-comment construction},
    and we replace SpecTopP by a \term{function} label \term{topic},
    and nodes in the tree diagram are labeled like ``sentence: topic-comment constructions''
    or ``topic: noun phrase'';
    on the other hand, phonological realization of functional heads
    do not have the form:function labels:
    we only label them according to the grammatical categories they mark,
    like ``evaluative particle [\category{diminutive}]'' (\prettyref{fig:grammatical.clause.high-level.topic.1}).
    We should be cautious that in hierarchies described by \citet{cinque1999adverbs},
    there can be too many form labels,
    and sometimes we have to conflate them into things like \form{extended verb phrase}
    (\prettyref{fig:vp.ex.1}).
\end{itemize}

It can be verified that our framework does not strongly deviate from the so-called Basic Linguistic Theory
\citep{dixon2009basic},
i.e. the grammatical framework used in most descriptive grammars.
Our framework is also largely consistent with the framework in \citet{cgel},
except the fact that \citet{cgel} do not make an explicit lexical/functional distinction:
for instance, they first analyze all prepositions as if they are a content word class
in \citechap{7},
and then go on to discuss grammaticalized prepositions from \citepage{647}.
On the other hand, \citet{dixon2009basic}, despite its fierce attack on generative syntax
(\prettyref{sec:grammatical.previous}),
advocates for analyses that are perfectly consistent with the descriptive framework in this section,
where functional and content items are strictly separated
\citep[\citepage{49}]{dixon2009basic}.

\subsection{Wordhood}\label{sec:intro.theory.word}

\term{Words} do not have a primitive status in our descriptive framework.
We may want to define wordhood syntactically,
but if it's defined as a small constituency in the sense of generative syntax,
then inflectional endings, and even voice markers,
are \emph{not} in the same word with the root
(or otherwise the whole verb-object phrase is to be recognized as a word),
and if it's defined as a flat-tree constituent,
then how shouldn't \form{has been working} is a word?
We can only define \emph{morphological words} and \emph{phonological words}.
Formatives brought together by post-syntactic operations form one morphological word:
hence a verb word in English is a complex containing
the verb root plus possible derivation affixes and also the tense suffix.
It is also possible that first a bunch of formatives are gathered together,
and later some other formatives join them:
this is known as cliticization.
Phonological words are to be defined according to prosody
or domains of phonological rules.

It is less wieldy to use constituency trees to represent 
the scopes of grammatical categories represented by formatives
in a morphological word.
In (\prettyref{sec:grammatical.clause.core-vp}, \ref{ex:grammatical.clause.core-vp.verbal-complex.1}),
for instance, the verbal complement 完 has its scope over 做作业,
and the aspectual suffix 了 has its scope over 做完作业.
We can represent this fact in the way of \prettyref{fig:grammatical.clause.core-vp.verbal-complex.1.1},
but this introduces invisible nodes in the syntactic tree. 

\subsection{Lexicalization}\label{sec:intro.theory.lexicon}

We understand lexicalization in two aspects:
realizational and semantic.
(Note that subcategorization can be seen as a consequence of realizational lexicalization:
saying a verb is transitive is equivalent to say it can only be phonologically realized
in the presence of a \category{transitivity} feature; \citealt{siddiqi2009syntax}.)
An arbitrarily large grammatical unit,
which is phonologically realized in a combinatorial way, can have an idiomized meaning:
consider \form{kick the bucket}.
A grammatical word \emph{without} an idiomized meaning
can be phonologically realized as a whole:
thus \form{feet} is just \wordroot{foot}-\category{nominal}-\category{plural},
but it has an irregular form.
The two are theoretically independent to each other,
although idiomization often leads to \emph{syntactic} reanalysis,
often towards a simpler direction (and thus can be called syntactic fossilization). 
Note that being semantically idiomized has syntactic consequences,
like reduced acceptability of certain movement operations.

\subsection{Derivation, inflection, parts of speech}

All other concepts, like the derivation/inflection distinction
or the argument/adjunct distinction, are in theory secondary,
and in this grammar we do not attempt to do demarcations of this point,
and merely focus on the relevant grammatical phenomena related to these distinctions.

The two types of lexicalization often converges on morphological words,
or to be more precise, their \emph{lexemes}.
Consider the structure [\dots \category{do} [\dots \category{transitive} \wordroot{hit}]_{\text{TransP}}]_{\text{vP}}.
This treelet has an established meaning \translate{to hit sth.},
and post-syntactic morphological operations gather the three formatives,
\category{do}, \category{transitive} and \wordroot{hit}, together into one morphological word,
and it gets realized as \form{hit} because of a corresponding lexical entry.
The tense and aspect markers are phonological implemented in a regular way,
So we say that there is a transitive verb \term{lexeme} \form{hit} stored in the lexicon,
and markers tense, person, etc. are given as a \term{paradigm} of it.%
\footnote{
    Note that here we assume a layered morphology of inflection,
    while template morphology which does not transparently show the hierarchy of functional heads
    is also possible, which however is not beyond the descriptive capacity of our framework
    \citep{bye2020morpheme}.
}

The concept of lexeme however implies that we have a clear derivation/inflection distinction.
Our opinion is that such a distinction is generally not possible to make cross-linguistically.
For instance, we may want to draw a line between derivation and inflection structurally
according to the two definitions of syntactic wordhood above (\prettyref{sec:intro.theory.word}),
but this definition excludes any valency alternation from derivation.

A \term{part of speech} refers to a class of lexicalized items in the lexicon with shared features.
Thus \wordroot{hit}-\category{transitive}-\category{do}
and \wordroot{rise}-\category{become}
are both assigned the part of speech tag \term{verb},
because they prototypically appear at the center of clauses
and can both receive endings \form{-s} and \form{-ed} and \form{-ing}.
Whether we have a \category{transitive} feature or not in the phonologically lexicalized entry
dictates whether the clause is transitive,
which gives rise to \term{subcategorization}.
Functional formatives, in principle, do not need part of speech tags:
in practice we often given them a tag so writing a dictionary becomes easier.
The lexicon of different languages are organized in different ways,
and therefore part of speech division expected has strong cross-linguistic variances.

Some languages, like Chinese, do not have rich morphology,
and it leaves beginners an impression that these languages have no parts of speech in their lexicons.
This, despite being theoretically possible, is highly implausible,
as this implies that the meanings of root-affix tuples are also not deterministic:
\wordroot{eat}-\category{nominal}, for instance,
would be understood as the action of eating or something to eat if this were true,
depending on the context.
Mandarin definitely \emph{has} part of speech divisions
as this is not the case in Mandarin.

\subsubsection{Arguments and adjuncts}

The standard of being an argument can be defined according to 
criteria listed in \prettyref{sec:grammatical.clause.peripheral}.
This distinction is also impossible to define in pure structural terms,
and shows strong cross-linguistic variances.

\subsection{Relation with previous works}\label{sec:grammatical.previous}

The purpose of theoretical linguistics is to see the complexity class of human languages,
while the purpose of descriptive linguistics is easier descriptions
-- at the expanse of having more ``primitive'' concepts
which actually do not allow more possible languages being described.
This grammar aims to strike a balance between readability and theoretical cross-linguistic comparisons.

We choose this framework for several reasons.
First, Mandarin has a relatively mature structuralist description tradition,
which is quite similar to that in \citet{cgel},
and that it can be seamlessly incorporated into modern generative syntax 
has long been noticed (e.g. \citealt{deng2010formal}).
Second, many descriptive linguists align themselves with the functionalist approaches to grammar,
but this is more because of problematic \emph{practices} in generative schools,
like relying on often unstable acceptability judgments
or overly focusing on complicated clauses.
The actual \emph{theory} they use however often depart from contemporary functionalist theories.
The term \term{construction}, for instance,
appears frequently in the descriptive and typological literature
(it will frequently appear in this grammar, too),
but it is much less frequently used in the sense of various schools of Construction Grammar:
in typical language description works, a construction is often still analyzed 
in a decompositional and combinatorial way,
Therefore, it makes sense to see how far structuralist analyses can go
being informed by modern generative syntax.


}





\chapter{Grammatical overview}

\section{Morphological typology}

Mandarin lacks prototypical inflectional morphology but has rich derivational morphology.
Compounding is the most frequent morphological device,
and partly due to influences of European languages, partly due to grammaticalization,
affixation is also frequently seen.
Plus, reduplication plays an important role in Mandarin verbal and adjectival derivation.

One caveat about talking about derivational morphology is that it assumes
the existence of a well-defined wordhood.
It's often claimed that Mandarin lacks the word/phrase distinction.
In this note, we will show that wordhood can indeed be clearly defined 
by syntactic, morphological and phonological standards.
What makes Mandarin unique is that the three definitions of wordhood do not always overlap.

\begin{todobox}{Different standards of woodhood}{wordhood}
    Define
\end{todobox}

Because of this, Chinese lexicography is usually based on \emph{morphemes},

\section{Clauses}

\subsection{Clauses are made of nucleus clauses and high-level categories}

A clause (\prettyref{box:clause-sentence-def}) can be divided into several clauses 
linked by \concept{clause linking} constructions,
including \concept{coordination} and \concept{subordination}.
(Note that coordination can also happen inside the nucleus clause;
\prettyref{sec:grammatical.clause.subject.clause}.)
Mandarin has ample information marking phenomena,
and thus a clause can be divided into
one or more \concept{topics}, if any, and a \concept{comment},
the latter being the \concept{nucleus clause}
plus possible \concept{sentence final particles}.

\begin{theorybox}{Terminology: \term{clause}, \term{sentence}, and the like}{clause-sentence-def}
    \citet{cgel} uses the term \term{sentence} 
    to refer to a natural unit in dialogue,
    which I refer to as a \term{utterance}.
    The term \term{sentence} here refers to 
    a clause that qualifies as an utterance. 

    Some people, like \citet[\citepage{140}]{deng2010formal}
    as well as \citet{dixon2009basic},
    use the term \term{clause} for subject-predict constructions 
    with no speech force marking.
    (\citet{deng2010formal} uses 句子 as the Mandarin counterpart of \term{sentence}
    and 小句 as the counterpart of \term{clause}.)
    In this way, \acl{sfp}s strictly shouldn't be
    regarded as a part of the clause, 
    and they may be discussed together with 
    other higher level constructions like clause linking. 
    This notion of clause correctly highlights the hierarchy in clausal structures.
    The problem with this terminology however is that in traditional grammars,
    the term \term{clause} does refer to units that have \ac{sfp}s.
    
    This note therefore refers to all units larger than the 
    subject-predicate construction as clauses, 
    which may or may not be sentence.
    The subject-predicate construction is instead named the \emph{nucleus} clause.
    The internal complexity of a clause 
    is still relevant for example in clause combining.
\end{theorybox}

These devices can coexist: in (\ref{ex:grammatical.clause.high-level.topic.1}),
diagrammed in \prettyref{fig:grammatical.clause.high-level.topic.1}
topicalization and a sentence final particle appear together.
Note that here we assume that the scope of topic is over the scope of the evaluative particle.
The relative scopes have subtle semantic effects and \citet{pan2015mandarin} notes that in Mandarin,
no preference is made among these subtle semantic differences,
meaning that it is also possible that the scope of the sentence final particle
being larger than that of the topic.

\begin{exe}
    \ex\label{ex:grammatical.clause.high-level.topic.1}
    \gll {} [张三]_{\text{topic},i}, [[他]_{\text{subject},i} 就 是 [个 王八蛋]_{\text{copular complement}}]_{\text{nucleus clause}} 罢了! \\
    {} \category{name} 3 just be \category{cls} turtle-egg \category{sfp} \\
    \translate{Zhang San is a son of a bitch!}
\end{exe}

\begin{figure}[H]
    {
        \centering
        \small
        \begin{tikzpicture}[x=0.75pt,y=0.75pt,yscale=-1,xscale=1]
    %uncomment if require: \path (0,300); %set diagram left start at 0, and has height of 300
    
    %Straight Lines [id:da2012956362095869] 
    \draw    (257,260) -- (448,260) ;
    %Straight Lines [id:da24303093387997055] 
    \draw    (355,227.31) -- (448,260) ;
    %Straight Lines [id:da14587666508930852] 
    \draw    (355,227.31) -- (257,260) ;
    %Straight Lines [id:da25151294497023513] 
    \draw    (532,227.31) -- (532,260.31) ;
    %Straight Lines [id:da5251263367750888] 
    \draw    (435,158.31) -- (532,183.31) ;
    %Straight Lines [id:da14069077735379665] 
    \draw    (359,183.31) -- (435,158.31) ;
    %Straight Lines [id:da20288315197393314] 
    \draw    (177,158.31) -- (192,260.31) ;
    %Straight Lines [id:da3486544796360199] 
    \draw    (154,260.31) -- (192,260.31) ;
    %Straight Lines [id:da02126425119297326] 
    \draw    (177,158.31) -- (154,260.31) ;
    %Straight Lines [id:da996798510579373] 
    \draw    (177,117.31) -- (311,93.31) ;
    %Straight Lines [id:da519603489229032] 
    \draw    (438,117.31) -- (311,93.31) ;
    
    % Text Node
    \draw (352.5,263) node [anchor=north] [inner sep=0.75pt]   [align=left] {他就是个王八蛋};
    % Text Node
    \draw (355,223.31) node [anchor=south] [inner sep=0.75pt]   [align=left] {\begin{minipage}[lt]{69.54pt}\setlength\topsep{0pt}
    \begin{center}
    factual branch:\\nucleus clause
    \end{center}
    
    \end{minipage}};
    % Text Node
    \draw (532,263.31) node [anchor=north] [inner sep=0.75pt]   [align=left] {罢了};
    % Text Node
    \draw (532,223.31) node [anchor=south] [inner sep=0.75pt]   [align=left] {\begin{minipage}[lt]{87.07pt}\setlength\topsep{0pt}
    \begin{center}
    evaluative particle\\{[\category{diminutive}]}
    \end{center}
    
    \end{minipage}};
    % Text Node
    \draw (435,154.31) node [anchor=south] [inner sep=0.75pt]   [align=left] {\begin{minipage}[lt]{75.28pt}\setlength\topsep{0pt}
    \begin{center}
    comment:\\sentential clause
    \end{center}
    
    \end{minipage}};
    % Text Node
    \draw (176,156.31) node [anchor=south] [inner sep=0.75pt]   [align=left] {\begin{minipage}[lt]{27.63pt}\setlength\topsep{0pt}
    \begin{center}
    topic:\\NP
    \end{center}
    
    \end{minipage}};
    % Text Node
    \draw (173,263.31) node [anchor=north] [inner sep=0.75pt]   [align=left] {张三};
    % Text Node
    \draw (311,90.31) node [anchor=south] [inner sep=0.75pt]   [align=left] {\begin{minipage}[lt]{127.06pt}\setlength\topsep{0pt}
    \begin{center}
    sentence:\\topic-comment construction
    \end{center}
    
    \end{minipage}};
    
    
    \end{tikzpicture}
    
    }
    \caption{Tree diagram of (\ref{ex:grammatical.clause.high-level.topic.1})}
    \label{fig:grammatical.clause.high-level.topic.1}
\end{figure}

Similarly, topicalization and clause linking can happen successively as well
(\ref{ex:grammatical.clause.high-level.topic-and-coordination.1}),
diagrammed in \prettyref{fig:grammatical.clause.high-level.topic-and-coordination.1},
shows an example of topicalization after subordination.
It is also possible to link two topic-comment clauses.

\begin{exe}
    \ex\label{ex:grammatical.clause.high-level.topic-and-coordination.1}
    \gll{} [我]_{\text{topic},i} [幸亏 ---_i 昨天 没 来]_{\text{nucleus clause}}, [否则 ---_i 就 被 困 住 了]_{\text{nucleus clause}} \\
    {} 2 fortunately {} yesterday \category{neg} come or.otherwise {} then \category{bei} trap \category{v2} \category{asp} \\
    \translate{Fortunately, I didn't come yesterday, or otherwise I would have been trapped.}
\end{exe}

\begin{figure}[H]
    {
        \small
        \begin{tikzpicture}[x=0.75pt,y=0.75pt,yscale=-0.8,xscale=0.8]
    %uncomment if require: \path (0,517); %set diagram left start at 0, and has height of 517
    
    %Straight Lines [id:da2627393797771076] 
    \draw    (628,456.31) -- (799,456.31) ;
    %Straight Lines [id:da5606219139716366] 
    \draw    (712,418.31) -- (799,456.31) ;
    %Straight Lines [id:da424518697731333] 
    \draw    (712,418.31) -- (628,456.31) ;
    %Straight Lines [id:da12245703561211851] 
    \draw    (528,418.31) -- (528,457.31) ;
    %Straight Lines [id:da8265933785037657] 
    \draw    (712,359.31) -- (616,313.31) ;
    %Straight Lines [id:da2646279688497606] 
    \draw    (528,359.31) -- (616,313.31) ;
    %Straight Lines [id:da8784074024110043] 
    \draw    (237,456.31) -- (428,456.31) ;
    %Straight Lines [id:da13759720547882903] 
    \draw    (334,313.31) -- (428,456.31) ;
    %Straight Lines [id:da3565761561879872] 
    \draw    (334,313.31) -- (237,456.31) ;
    %Straight Lines [id:da46724975619438913] 
    \draw    (468,224.31) -- (334,268.31) ;
    %Straight Lines [id:da606218866385489] 
    \draw    (468,224.31) -- (618,268.31) ;
    %Straight Lines [id:da17514038586144876] 
    \draw    (135,225.31) -- (135,457.31) ;
    %Straight Lines [id:da9654714437988359] 
    \draw    (309,136.31) -- (138,180.31) ;
    %Straight Lines [id:da6202905359610531] 
    \draw    (309,136.31) -- (471,180.31) ;
    
    % Text Node
    \draw (135,460.31) node [anchor=north] [inner sep=0.75pt]   [align=left] {我_i};
    % Text Node
    \draw (332.5,459.31) node [anchor=north] [inner sep=0.75pt]   [align=left] {---_i 幸亏昨天没来};
    % Text Node
    \draw (713.5,459.31) node [anchor=north] [inner sep=0.75pt]   [align=left] {---_i 就被困住了};
    % Text Node
    \draw (528,460.31) node [anchor=north] [inner sep=0.75pt]   [align=left] {否则};
    % Text Node
    \draw (528,407.5) node [anchor=south] [inner sep=0.75pt]   [align=left] {\begin{minipage}[lt]{100pt}\setlength\topsep{0pt}
    \begin{center}
    conjunction \\ {[\category{counterfactual}]}
    \end{center}
    
    \end{minipage}};
    % Text Node
    \draw (707,407.5) node [anchor=south] [inner sep=0.75pt]   [align=left] {\begin{minipage}[lt]{99.04pt}\setlength\topsep{0pt}
    \begin{center}
    counterfactual clause:\\nucleus clause
    \end{center}
    
    \end{minipage}};
    % Text Node
    \draw (616,310.81) node [anchor=south] [inner sep=0.75pt]   [align=left] {\begin{minipage}[lt]{102.68pt}\setlength\topsep{0pt}
    \begin{center}
    counterfactual branch:\\nucleus clause
    \end{center}
    
    \end{minipage}};
    % Text Node
    \draw (341,310.81) node [anchor=south] [inner sep=0.75pt]   [align=left] {\begin{minipage}[lt]{69.54pt}\setlength\topsep{0pt}
    \begin{center}
    factual branch:\\nucleus clause
    \end{center}
    
    \end{minipage}};
    % Text Node
    \draw (469,221.81) node [anchor=south] [inner sep=0.75pt]   [align=left] {\begin{minipage}[lt]{131.05pt}\setlength\topsep{0pt}
    \begin{center}
    comment:\\counterfactual subordination
    \end{center}
    
    \end{minipage}};
    % Text Node
    \draw (135,221.81) node [anchor=south] [inner sep=0.75pt]   [align=left] {\begin{minipage}[lt]{39.56pt}\setlength\topsep{0pt}
    \begin{center}
    topic:\\pronoun
    \end{center}
    
    \end{minipage}};
    % Text Node
    \draw (309,133.31) node [anchor=south] [inner sep=0.75pt]   [align=left] {\begin{minipage}[lt]{127.06pt}\setlength\topsep{0pt}
    \begin{center}
    sentence:\\topic-comment construction
    \end{center}
    
    \end{minipage}};
    
    
    \end{tikzpicture}
    
    }
    \caption{Tree diagram of (\ref{ex:grammatical.clause.high-level.topic-and-coordination.1})}
    \label{fig:grammatical.clause.high-level.topic-and-coordination.1}
\end{figure}

\subsection{Subject and predicate}\label{sec:grammatical.clause.subject}

The nucleus clause contains a subject (if any) and what is often known as a predicate,%
\footnote{
    \citet{dixon2009basic} argues against the definition of \term{predicate} 
    as the main verb (or adjective) plus somehow ``internal'' arguments.
    He uses the term \term{predicate} to refer to the verbal complex instead.
    However, since I will need to compare the topic-comment construction 
    with the inner structure of the nucleus clause,
    the term \term{predicate} will still be used in the way \citet{dixon2009basic} dislikes,
    because it's the counterpart of the comment role in the topic-comment construction.
}
which usually is a (extended) verb phrase but may also be a nominal. 
In the Mandarin simple nucleus clause,
the definition of the subject, as opposed to the topic,
is not trivially clear.
Here we note that the nucleus clause has a neutral structure (\prettyref{sec:grammatical.clause.subject.topic}),
in which a subject appearing at the initial is both the argument structure pivot
(\prettyref{sec:grammatical.clause.subject.argument})
and the clause-level pivot.

\subsubsection{The neutral order}\label{sec:grammatical.clause.subject.topic}

The notion that in Mandarin, subject is the same as topic is prevalent.
Taking one step further, one may argue that Mandarin has no argument structure at all
and the word order in a clause is shaped by only information structure \citep{lapolla20091}.
This grammar rejects this analysis.

First, we note that a information structure neutral order can be defined for most, if not all, clauses.
An example is provided in (\ref{ex:grammatical.clause.subject.no-topic}).
The two arguments, 饭 and 吃, can be reordered in a seemingly free way depending on their topicality,
violating the common generalization that Mandarin has a SVO order.
We however note that (\ref{ex:grammatical.clause.subject.no-topic.4})
is completely unacceptable with the intended meaning.
Playing with more possible orders, and we will find that the arguments seem to be
only permitted to move \emph{leftwards} (and thus \ref{ex:grammatical.clause.subject.no-topic.4} is not possible),
consistent with the assumption that a neutral ordered nucleus clause is formed first,
followed by topicalization.
By analyzing subtle pragmatics differences, we find (\ref{ex:grammatical.clause.subject.no-topic.1}) seems to be the ``neutral'' order 
(although it imposes weak topicality to 你 \translate{you},
and 吃饭 \translate{eat (lit. eat meal)} is focalized).

\begin{exe}
    \ex\label{ex:grammatical.clause.subject.no-topic} 
    \begin{xlist}
        \ex\label{ex:grammatical.clause.subject.no-topic.1}
        \gll 你 吃 饭 了 吗 \\
        2 eat meal \category{sfp} \category{sfp} \\
        \glt\translate{Have you eaten?} 
        \ex\label{ex:grammatical.clause.subject.no-topic.2}
        \gll 饭 你 吃 了 吗 \\
        meal 2 eat \category{sfp} \category{sfp} \\
        \glt\translate{Have you eaten?}
        \ex\label{ex:grammatical.clause.subject.no-topic.3}
        \gll 你 饭 吃 了 吗 \\
        2 meal eat \category{sfp} \category{sfp} \\
        \glt\translate{Have you eaten?}
        \ex\label{ex:grammatical.clause.subject.no-topic.4}
        \gll *饭 吃 你 了 吗  \\
        meal eat 2 \category{sfp} \category{sfp} \\
        \glt\translate{Intended meaning: have you eaten? (Actual meaning: has meal eaten you?)}
    \end{xlist}
\end{exe}

We also note that there is no dangling topic in Mandarin (\prettyref{sec:topic-subject}).
This means \emph{all} topics originate from somewhere within the nucleus clause.
On the other hand, the subject, if well-defined by the usual pivot tests, is a part of the nucleus clause,
and therefore in Mandarin, topic and subject are different.

\subsubsection{Subject as pivot of argument structure}\label{sec:grammatical.clause.subject.argument}

Being the initial constituent%
\footnote{
    Note that it is possible that certain constituents, like temporal constituents,
    naturally appear before the subject.
}
in clauses like (\ref{ex:grammatical.clause.subject.no-topic.1}) 
has a clear relation to being the most prominent or the most \emph{external} argument
-- the agent or causer or the patient in passive constructions.
Examples like (\ref{ex:grammatical.clause.subject.valency.1}) can be explained by valency alternation.

\begin{exe}
    \ex\label{ex:grammatical.clause.subject.valency.1} 茶泡好了
\end{exe}

\subsubsection{Subject as pivot of clause}\label{sec:grammatical.clause.subject.clause}

Certain ``clause linking'' constructions are actually verb phrase linking constructions.
At the first glance, (\ref{ex:grammatical.clause.subject.clause.1}) looks just like (\ref{ex:grammatical.clause.high-level.topic-and-coordination.1}),
but further grammatical tests show that the two are structurally different.
It is not possible for the conjunction 既 to appear before the subject;
further, it is not possible for the two branches to have different subjects
(\ref{ex:grammatical.clause.subject.clause.1-no-good}).
Therefore, the 既…右… coordination construction (and many more) is for connecting two verb phrases,
and we note that that the element shared by the two branches is
always the subject defined in \prettyref{sec:grammatical.clause.subject.argument}:
hence we find that in Mandarin,
we have both well-defined argument structure and clausal pivots,
which are identical.
This justifies using the term \term{subject} in describing Mandarin,
and confirms that Mandarin is a nominative-accusative language.

\begin{exe}
    \ex\label{ex:grammatical.clause.subject.clause.1} \gll {} [我]_{\text{subject}} 既 [不 想 用 这 个 方案]_{\text{VP}}, 又 [不 想 用 那 个 方案]_{\text{VP}} \\
    {} 1 \category{conj} \category{neg} want use this \category{cls} plan \category{conj} \category{neg} want use that \category{cls} plan \\
    \glt\translate{I don't want to use this plan, and nor do I want to use that plan.}
    \ex\label{ex:grammatical.clause.subject.clause.1-no-good} \begin{xlist}
        \ex\label{ex:grammatical.clause.subject.clause.1-no-good.1} *既我不想用这个方案,又不想用那个方案
        \ex *我既不想用这个方案,他又不想用那个方案
        \ex *我既不想用这个方案,他又不想用那个方案
    \end{xlist}
\end{exe}

\subsection{The predicate}

(\ref{ex:vp.ex.1}) is an illustration of a complicated nucleus clause.
Its constituent structure is shown in \prettyref{fig:vp.ex.1},
following the notation in \citet{cgel}.
We need to warn that the main information contained in \prettyref{fig:vp.ex.1} 
is the \emph{scopes} of constituents surrounding the core verb phrase,
while the function labels (e.g. \term{head}) and the form labels (e.g. \form{extended VP})
in \prettyref{fig:vp.ex.1} may be misleading,
as 能在我的办公室跟你讨论一下 and 可能能在我的办公室跟你讨论一下
are both labeled as extended VPs, but clearly they have slightly different syntactic statuses:
the auxiliary 可能 can be attached to the former
but it can never appear twice and hence cannot be attached to the latter.


\begin{exe}
    \ex \label{ex:vp.ex.1}
    \gll 我 [明天 可能 能 在 我 的 办公室 跟 你 [讨论 一下]_{\text{core\acs{vp}}}]_{\text{extended \acs{vp}}} \\
    1 tomorrow \category{aux}:possible \category{aux}:ability at my \category{poss} 
    office with 2 discuss a.little.bit \\ 
    \glt \translate{Tomorrow possiblity I can have a discussion with you in my office.}
\end{exe}

\begin{figure}[H]
    {
        \small 
        \begin{tikzpicture}[x=0.75pt,y=0.75pt,yscale=-0.8,xscale=0.8]
    %uncomment if require: \path (0,740); %set diagram left start at 0, and has height of 740
    
%Straight Lines [id:da5419452521165726] 
\draw    (690.67,583.33) -- (742,583.33) ;
%Straight Lines [id:da08544086505959503] 
\draw    (713,511.59) -- (706.23,533.35) -- (690.67,583.33) ;
%Straight Lines [id:da5166120577578295] 
\draw    (713,511.59) -- (742,583.33) ;
%Straight Lines [id:da7597929142889452] 
\draw    (410,368.26) -- (410,584.33) ;
%Straight Lines [id:da7951507355099614] 
\draw    (312.67,294.26) -- (312.67,584.33) ;
%Straight Lines [id:da27035558491953804] 
\draw    (622.67,583.33) -- (642,583.33) ;
%Straight Lines [id:da24028174129699176] 
\draw    (630,511.59) -- (622.67,583.33) ;
%Straight Lines [id:da917015263091598] 
\draw    (630,511.59) -- (642,583.33) ;
%Straight Lines [id:da14883621053934304] 
\draw    (470,583.33) -- (547,583.33) ;
%Straight Lines [id:da9342077761978527] 
\draw    (500,441.59) -- (470,583.33) ;
%Straight Lines [id:da3488354163696341] 
\draw    (500,441.59) -- (547,583.33) ;
%Straight Lines [id:da5030694457277152] 
\draw    (673,441.59) -- (628.67,462.33) ;
%Straight Lines [id:da7531987353356717] 
\draw    (673,441.59) -- (710.67,462.33) ;
%Straight Lines [id:da40179355911176984] 
\draw    (589,370.26) -- (502,398.33) ;
%Straight Lines [id:da5639473207238246] 
\draw    (589,370.26) -- (672,398.33) ;
%Straight Lines [id:da5704299439338378] 
\draw    (497,294.26) -- (410,322.33) ;
%Straight Lines [id:da8053817579360183] 
\draw    (497,294.26) -- (587,322.33) ;
%Straight Lines [id:da8066766119523336] 
\draw    (402,220.26) -- (315,248.33) ;
%Straight Lines [id:da8072938752013339] 
\draw    (402,220.26) -- (492,248.33) ;
%Straight Lines [id:da35978458309470907] 
\draw    (199.67,219.76) -- (199.67,475.49) -- (199.67,583.33) ;
%Straight Lines [id:da927464810721838] 
\draw    (294,146.26) -- (199.33,176.93) ;
%Straight Lines [id:da9186562853518254] 
\draw    (294,146.26) -- (400.33,176.93) ;
%Straight Lines [id:da7086875966035955] 
\draw    (190,71.59) -- (95.33,102.26) ;
%Straight Lines [id:da7175081469679203] 
\draw    (190,71.59) -- (296.33,102.26) ;
%Straight Lines [id:da26209295176984004] 
\draw    (87.67,146.56) -- (87.67,583.33) ;
%Straight Lines [id:da43013746492722116] 
\draw [color={rgb, 255:red, 80; green, 227; blue, 194 }  ,draw opacity=1 ][line width=2.25]    (191.67,623.26) -- (643.78,623.26) ;
%Straight Lines [id:da47071956794766767] 
\draw [color={rgb, 255:red, 74; green, 144; blue, 226 }  ,draw opacity=1 ][line width=2.25]    (686.78,623.26) -- (755,623.26) ;

% Text Node
\draw (78.67,590.17) node [anchor=north west][inner sep=0.75pt]   [align=left] {我 \ \ \ \ \ };
% Text Node
\draw (183.67,590.17) node [anchor=north west][inner sep=0.75pt]   [align=left] {明天};
% Text Node
\draw (293.67,590.17) node [anchor=north west][inner sep=0.75pt]   [align=left] {可能};
% Text Node
\draw (400.67,590.17) node [anchor=north west][inner sep=0.75pt]   [align=left] {能};
% Text Node
\draw (614.67,590.17) node [anchor=north west][inner sep=0.75pt]   [align=left] {跟你};
% Text Node
\draw (463.67,590.17) node [anchor=north west][inner sep=0.75pt]   [align=left] {在我的办公室};
% Text Node
\draw (687.67,590.17) node [anchor=north west][inner sep=0.75pt]   [align=left] {讨论一下};
% Text Node
\draw (713,508.59) node [anchor=south] [inner sep=0.75pt]   [align=left] {\begin{minipage}[lt]{38.85pt}\setlength\topsep{0pt}
\begin{center}
head:\\core VP
\end{center}

\end{minipage}};
% Text Node
\draw (630,508.59) node [anchor=south] [inner sep=0.75pt]   [align=left] {\begin{minipage}[lt]{51.96pt}\setlength\topsep{0pt}
\begin{center}
comitative:\\PP
\end{center}

\end{minipage}};
% Text Node
\draw (673,438.59) node [anchor=south] [inner sep=0.75pt]   [align=left] {\begin{minipage}[lt]{80.61pt}\setlength\topsep{0pt}
\begin{center}
head:\\extended VP
\end{center}

\end{minipage}};
% Text Node
\draw (500,438.59) node [anchor=south] [inner sep=0.75pt]   [align=left] {\begin{minipage}[lt]{40.93pt}\setlength\topsep{0pt}
\begin{center}
location:\\PP
\end{center}

\end{minipage}};
% Text Node
\draw (589,367.26) node [anchor=south] [inner sep=0.75pt]   [align=left] {\begin{minipage}[lt]{60.61pt}\setlength\topsep{0pt}
\begin{center}
head:\\extended VP
\end{center}

\end{minipage}};
% Text Node
\draw (410,365.26) node [anchor=south] [inner sep=0.75pt]   [align=left] {\begin{minipage}[lt]{64.42pt}\setlength\topsep{0pt}
\begin{center}
modality aux\\ {[\category{ability}]}
\end{center}

\end{minipage}};
% Text Node
\draw (497,291.26) node [anchor=south] [inner sep=0.75pt]   [align=left] {\begin{minipage}[lt]{60.61pt}\setlength\topsep{0pt}
\begin{center}
head:\\extended VP
\end{center}

\end{minipage}};
% Text Node
\draw (312.67,291.26) node [anchor=south] [inner sep=0.75pt]   [align=left] {\begin{minipage}[lt]{64.42pt}\setlength\topsep{0pt}
\begin{center}
modality aux\\ {[\category{possibility}]}
\end{center}

\end{minipage}};
% Text Node
\draw (402,217.26) node [anchor=south] [inner sep=0.75pt]   [align=left] {\begin{minipage}[lt]{60.61pt}\setlength\topsep{0pt}
\begin{center}
head:\\extended VP
\end{center}

\end{minipage}};
% Text Node
\draw (199.67,216.76) node [anchor=south] [inner sep=0.75pt]   [align=left] {\begin{minipage}[lt]{64.14pt}\setlength\topsep{0pt}
\begin{center}
time location:\\adverb
\end{center}

\end{minipage}};
% Text Node
\draw (294,143.26) node [anchor=south] [inner sep=0.75pt]   [align=left] {\begin{minipage}[lt]{60.61pt}\setlength\topsep{0pt}
\begin{center}
predicate:\\extended VP
\end{center}

\end{minipage}};
% Text Node
\draw (190,68.59) node [anchor=south] [inner sep=0.75pt]   [align=left] {\begin{minipage}[lt]{65.38pt}\setlength\topsep{0pt}
\begin{center}
nucleus clause
\end{center}

\end{minipage}};
% Text Node
\draw (87.67,143.56) node [anchor=south] [inner sep=0.75pt]   [align=left] {\begin{minipage}[lt]{39.55pt}\setlength\topsep{0pt}
\begin{center}
subject:\\pronoun
\end{center}

\end{minipage}};
% Text Node
\draw (417.72,626.26) node [anchor=north] [inner sep=0.75pt]  [color={rgb, 255:red, 80; green, 227; blue, 194 }  ,opacity=1 ] [align=left] {extension};
% Text Node
\draw (720.89,626.26) node [anchor=north] [inner sep=0.75pt]  [color={rgb, 255:red, 74; green, 144; blue, 226 }  ,opacity=1 ] [align=left] {core VP};


    
    
    \end{tikzpicture}
    
    }
    \caption{Tree diagram of (\ref{ex:vp.ex.1})}
    \label{fig:vp.ex.1}
\end{figure}

It can be seen that when we have verbal prediction,
the full, extended \acs{vp} following the subject 
can be divided into an extension region and the core \acs{vp}
(\prettyref{sec:grammatical.clause.core-vp}).
The extension region contains 
\acs{tame} auxiliaries and adverbs not realized in the verbal complex (\prettyref{sec:grammatical.clause.tam}),
and peripheral arguments like temporal and spatial locations (\prettyref{sec:grammatical.clause.peripheral}).
Sometimes the object may be fronted and 
it's also possible that a prepositional complement is fronted to this region.

It should be noted that in the disposal and passive constructions,
the manner phrase may appear \emph{after} the auxiliary (\ref{ex:vp.ex.2}),
and in this case the boundary of the core \acs{vp}
can't be clearly defined at the surface level,
which shouldn't be surprising as we do not expect to
always see a clear-cut argument/adjunct distinction.

\begin{exe}
    \ex\label{ex:vp.ex.2}
    \gll 我 明天 可能 能 在 我 的 办公室 跟 你 [把]_{\text{auxiliary}}  这 个 问题 [好好]_{\text{manner}} 讨论 一下 \\
    1 tomorrow \category{aux}:possible \category{aux}:ability 
    at my \category{poss} office 
    with 2 
    \category{ba} this \category{cls} problem good
    discuss a.little.bit \\
    \glt \translate{Tomorrow possiblity I can have a good discussion of this problem with you in my office.}
\end{exe}

\subsubsection{Tense, aspect and modality marking}\label{sec:grammatical.clause.tam}

In (\ref{ex:vp.ex.1}), it can be clearly seen that Mandarin has modal auxiliaries:
the order (and also the scope) of 可能 and 能 is strictly 可能 \textgt{}能 and never the inverse,
suggesting that these modality markers are grammaticalized items.
More analytic markers of \ac{tame} categories can be found:
it seems 据说 is a peripheristic marker of evidentiality, for instance:
in (\ref{ex:grammatical.clause.tam.1}),
the order of the \acs{tame} markers is always 据说 \textgt{}可能, and not the inverse,
suggesting that 据说 is a part of the \acs{tame} grammatical hierarchy.%
\footnote{
    Note that English adverbs like \form{allegedly} follow the same generalization:
    we have e.g. \form{he allegedly possibly did this} but never \form{he possibly allegedly did this}. See \citet{cinque1999adverbs}.
}

\begin{exe}
    \ex\label{ex:grammatical.clause.tam.1} \gll 
    [这 辆 车]_{\text{subject:NP}} [据说]_{\text{evidentiality}} [可能]_{\text{modality}} 不 太 靠谱 \\
    this \category{cls} car is.said \category{aux} \category{neg} very reliable \\
    \glt\translate{It is said that this car may not be very reliable.}
\end{exe}

Whether Mandarin has something comparable to tense in more prototypical tensed languages is not clear.
An observation is that Mandarin speakers often do not fully subconsciously acquire the tense category when learning tensed languages like English.
This, however, does not fully exclude the possibility of an impoverished tense system.
TODO: ref

Mandarin has ample devices to mark point-of-view aspect.
This is primarily done by the verbal complex (e.g. \ref{ex:grammatical.clause.core-vp.verbal-complex.1}),
via the (semi-)inflectional marking of aspect by 了, 着 and 过 in the verbal complex
(\prettyref{sec:grammatical.clause.core-vp}).
but analytic devices exist.
In (\ref{ex:grammatical.clause.tam.2}),
we find that the aspect marker 正 is separated from the core \ac{vp} by a manner phrase,
proving that 正 is not a part of the verbal complex.
The sentence is diagrammed in 

\begin{exe}
    \ex\label{ex:grammatical.clause.tam.2} 他 在很认真地写作业
\end{exe}

\begin{figure}[H]
    \centering
    {
        \small
        \begin{tikzpicture}[x=0.75pt,y=0.75pt,yscale=-0.8,xscale=0.8]
    %uncomment if require: \path (0,443); %set diagram left start at 0, and has height of 443
    
    %Straight Lines [id:da528878482445523] 
    \draw    (521,366.69) -- (625.33,366.69) ;
    %Straight Lines [id:da40108809842151916] 
    \draw    (567.33,338.27) -- (625.33,366.69) ;
    %Straight Lines [id:da27474423107368695] 
    \draw    (567.33,338.27) -- (521,366.69) ;
    %Straight Lines [id:da9052525494601512] 
    \draw    (421.33,266.27) -- (421.33,367.27) ;
    %Straight Lines [id:da7905703091701273] 
    \draw    (530.33,172.27) -- (646,219) ;
    %Straight Lines [id:da6687808665035286] 
    \draw    (418,219) -- (530.33,172.27) ;
    %Straight Lines [id:da6091113777043599] 
    \draw    (315,127) -- (417.33,93.27) ;
    %Straight Lines [id:da5783470379038482] 
    \draw    (527,127) -- (417.33,93.27) ;
    %Straight Lines [id:da011193879040605426] 
    \draw    (312.33,172.27) -- (312.33,367.27) ;
    %Straight Lines [id:da2541456739251504] 
    \draw    (647.33,266.27) -- (731.33,294.27) ;
    %Straight Lines [id:da7805223808386317] 
    \draw    (691,366.69) -- (783.33,366.69) ;
    %Straight Lines [id:da905598438052631] 
    \draw    (732.33,338.27) -- (783.33,366.69) ;
    %Straight Lines [id:da992603763061734] 
    \draw    (732.33,338.27) -- (691,366.69) ;
    %Straight Lines [id:da11089280689158798] 
    \draw    (563.33,295.27) -- (647.33,266.27) ;
    
    % Text Node
    \draw (573.17,369.69) node [anchor=north] [inner sep=0.75pt]   [align=left] {很认真地};
    % Text Node
    \draw (418,222) node [anchor=north] [inner sep=0.75pt]   [align=left] {\begin{minipage}[lt]{100.39pt}\setlength\topsep{0pt}
    \begin{center}
    aspect marker\\{[\category{imperfective}]}
    \end{center}
    
    \end{minipage}};
    % Text Node
    \draw (421.33,370.27) node [anchor=north] [inner sep=0.75pt]   [align=left] {在};
    % Text Node
    \draw (647.33,225.27) node [anchor=north] [inner sep=0.75pt]   [align=left] {\begin{minipage}[lt]{60.61pt}\setlength\topsep{0pt}
    \begin{center}
    head:\\extended VP
    \end{center}
    
    \end{minipage}};
    % Text Node
    \draw (530.33,168.27) node [anchor=south] [inner sep=0.75pt]   [align=left] {\begin{minipage}[lt]{60.61pt}\setlength\topsep{0pt}
    \begin{center}
    predicate:\\extended VP
    \end{center}
    
    \end{minipage}};
    % Text Node
    \draw (315,130) node [anchor=north] [inner sep=0.75pt]   [align=left] {\begin{minipage}[lt]{39.55pt}\setlength\topsep{0pt}
    \begin{center}
    subject:\\pronoun
    \end{center}
    
    \end{minipage}};
    % Text Node
    \draw (312.33,370.27) node [anchor=north] [inner sep=0.75pt]   [align=left] {他};
    % Text Node
    \draw (417.33,90.27) node [anchor=south] [inner sep=0.75pt]   [align=left] {\begin{minipage}[lt]{65.38pt}\setlength\topsep{0pt}
    \begin{center}
    nucleus clause
    \end{center}
    
    \end{minipage}};
    % Text Node
    \draw (737.17,369.69) node [anchor=north] [inner sep=0.75pt]   [align=left] {写作业};
    % Text Node
    \draw (731.33,297.27) node [anchor=north] [inner sep=0.75pt]   [align=left] {\begin{minipage}[lt]{38.85pt}\setlength\topsep{0pt}
    \begin{center}
    head:\\core VP
    \end{center}
    
    \end{minipage}};
    % Text Node
    \draw (563.33,298.27) node [anchor=north] [inner sep=0.75pt]   [align=left] {\begin{minipage}[lt]{65.11pt}\setlength\topsep{0pt}
    \begin{center}
    manner:\\adverb phrase
    \end{center}
    
    \end{minipage}};
    
    
    \end{tikzpicture}
    
    }
    \caption{Tree diagram of (\ref{ex:grammatical.clause.tam.2})}
    \label{fig:grammatical.clause.tam.2}
\end{figure}

\subsubsection{Negation}\label{sec:grammatical.clause.negation}

The negator can appear at any position in the auxiliary chain described in \prettyref{sec:grammatical.clause.tam}.
Its linear order is consistent with its scope,
which in turn introduces subtle semantic differences
(\ref{ex:grammatical.clause.negation.1}).

\begin{exe}
    \ex\label{ex:grammatical.clause.negation.1} \begin{xlist}
        \ex\label{ex:grammatical.clause.negation.1.1} 
        \gll [他]_{\text{subject}} 不 [可能 能 出 国]_{\text{negated}} \\
        3 \category{neg} \category{aux} \category{aux} go.outside.of country \\
        \glt\translate{It's not possible that he has the ability to go abroad.}
    
        \ex\label{ex:grammatical.clause.negation.1.2} 
        \gll [他]_{\text{subject}} 可能 不 [能 出 国]_{\text{negated}} \\
        3 \category{aux} \category{neg} \category{aux} go.outside.of country \\
        \glt\translate{It's possible that he doesn't have the ability to go abroad.}
    
        \ex\label{ex:grammatical.clause.negation.1.3} 
        \gll [他]_{\text{subject}} 可能 能 不 [出 国]_{\text{negated}} \\
        3 \category{aux} \category{aux} \category{neg} go.outside.of country \\
        \glt\translate{It's possible that he has the ability to not go abroad.}
    \end{xlist}
\end{exe}

(\ref{ex:grammatical.clause.negation.1}) shows a negation device that is more flexible in its scope
that the English negation.
(\ref{ex:grammatical.clause.negation.1.2}) can be word-to-word translated to English as
\form{he possibly cannot go abroad},
but (\ref{ex:grammatical.clause.negation.1.1}) and (\ref{ex:grammatical.clause.negation.1.3})
can only be faithfully translated using complement clause constructions.

\subsubsection{Peripheral arguments}\label{sec:grammatical.clause.peripheral}

The term \term{peripheral argument} is from \citet{dixon2009basic}.
We intentionally use the term here instead of the more frequent \term{adjunct},
because there are both \acs{tame} adjuncts and circumstantial adjuncts,
the latter known as peripheral arguments in \citet{dixon2009basic}.

A clear distinction between core and peripheral arguments,
more often known as the argument/adjunct distinction, is not always possible.
Some criteria used for the distinction are about structural closeness of the argument to the main verb, or in other words scope:
the manner expression usually has a scope wider than the core verb phrase,
and thus the former is classified as a peripheral argument.
Other criteria are based on licensing: intransitive use of a transitive verb is prohibited by the lexicon,
or, in more technical terms, the verb root appearing in a verbal environment but without transitivity is not allowed by the lexicon \citep{siddiqi2009syntax}.
Thus \form{well} in \form{he treats us well} seems to be an argument, although it's a manner expression.
Yet other criteria are based on argument indexation and flagging:
an argument with oblique case marking does not leave agreement markers on the main verb,
while an argument with structural case (nominative, accusative) does
if the language has agreement marking,
and the latter is recognized as a core argument.
Following this standard, many so-called oblique arguments,
like \form{this} in \form{I think [of this]}, would be classified as peripheral,
although they are clearly licensed by the lexical entry of the verb.
These criteria correlate with each other but in a non-deterministic way.

Because Mandarin has no verb agreement, only the first two criteria can be used,
and the problems listed above all occur.
The status of comitative 跟你 in (\ref{ex:vp.ex.1}) is not so clear, for instance:
it is fairly low in \prettyref{fig:vp.ex.1}, but it is not obligatory.
We also note that reordering of peripheral arguments is possible,
but mixing them with \acs{tame} markers sounds problematic to say the least
(\ref{ex:grammatical.clause.peripheral.order.1}).
Note that fronting of the comitative to a higher position is possible (\prettyref{sec:topic.subject}).

\begin{exe}
    \ex\label{ex:grammatical.clause.peripheral.order.1} \begin{xlist}
        \ex 我明天可能能在我的办公室跟你讨论一下 (=\ref{ex:vp.ex.1})
        \ex 我明天可能能跟你在我的办公室讨论一下
        \ex ??我明天可能跟你能在我的办公室讨论一下
    \end{xlist}
\end{exe}

\subsection{The structure of the core verb phrase}\label{sec:grammatical.clause.core-vp}

In the surface form, the core \acs{vp} contains the core arguments and the \concept{verbal complex}
(\prettyref{sec:grammatical.clause.core-vp.verbal-complex}).
The bracketed constituent in (\ref{ex:grammatical.clause.core-vp.verbal-complex.1})
is a typical core verb phrase.

\begin{exe}
    \ex\label{ex:grammatical.clause.core-vp.verbal-complex.1}
    \gll 我 [[做 完 了]_{\text{verbal complex}} 作业]_{\text{core VP}} \\
    1 do finish \category{asp} homework \\
    \glt\translate{I have finished the homework.}
\end{exe}

Mandarin has two types of constructions in which the main verb appears at the final:
the \category{disposal} construction, and the so-called \category{passive} construction,
also known as the \form{ba}-construction and the \form{bei}-construction.

\begin{exe}
    \ex 我把作业做完了
\end{exe}

\subsubsection{The verbal complex}\label{sec:grammatical.clause.core-vp.verbal-complex}

The verbal complex contains three slots:
the verb stem,
the so-called \concept{verbal complement},
known as 补语 in Chinese linguistic community,%
\footnote{
    The term 补语 literally means \translate{complementation speech}, 
    and is therefore often translated as \term{complement}.
    In this note I use the term \term{complement}
    to refer to grammatical constituents that are somehow more closely 
    related to the lexical head, 
    and I choose the (somehow tedious but explicit) term 
    \term{non-argument complement}.
},
and the aspectual marker.
In (\ref{ex:grammatical.clause.core-vp.verbal-complex.1}),
完 is the verbal complement, and 了 is the aspectual marker.
There the verbal complex contains all the possible components it can have.
It's possible for only one of the two or even none of it to appear.

We note that the aspect marker 了 actually has a scope \emph{over} the core verb phrase.
Following \prettyref{fig:grammatical.clause.tam.2},
we can represent the structure of (\ref{ex:grammatical.clause.core-vp.verbal-complex.1})
in \prettyref{fig:grammatical.clause.core-vp.verbal-complex.1.1}.
But further, we note that it is likely that 完 here is a lexical aspect marker,
and likely has scope over the whole 做完作业 argument structure,
so we need to add one more node between the aspect marker node
and the predicator node in \prettyref{fig:grammatical.clause.core-vp.verbal-complex.1.1}.

\begin{figure}[H]
    \centering
    {
        \small
        \begin{tikzpicture}[x=0.75pt,y=0.75pt,yscale=-0.8,xscale=0.8]
    %uncomment if require: \path (0,470); %set diagram left start at 0, and has height of 470
    
    %Straight Lines [id:da7905703091701273] 
    \draw    (530.33,172.27) -- (646,219) ;
    %Straight Lines [id:da6687808665035286] 
    \draw    (418,219) -- (530.33,172.27) ;
    %Straight Lines [id:da6091113777043599] 
    \draw    (315,127) -- (417.33,93.27) ;
    %Straight Lines [id:da5783470379038482] 
    \draw    (527,127) -- (417.33,93.27) ;
    %Straight Lines [id:da011193879040605426] 
    \draw    (312.33,172.27) -- (312.33,367.27) ;
    %Straight Lines [id:da2541456739251504] 
    \draw    (647.33,266.27) -- (731.33,294.27) ;
    %Straight Lines [id:da7805223808386317] 
    \draw    (691,366.69) -- (783.33,366.69) ;
    %Straight Lines [id:da905598438052631] 
    \draw    (732.33,338.27) -- (783.33,366.69) ;
    %Straight Lines [id:da992603763061734] 
    \draw    (732.33,338.27) -- (691,366.69) ;
    %Straight Lines [id:da11089280689158798] 
    \draw    (563.33,295.27) -- (647.33,266.27) ;
    %Straight Lines [id:da3706258778100655] 
    \draw    (564.33,338.27) -- (564.33,367.27) ;
    %Curve Lines [id:da663135552126335] 
    \draw  [dash pattern={on 0.84pt off 2.51pt}]  (420,393.27) .. controls (497.75,444.23) and (533.56,424.12) .. (572.6,395.06) ;
    \draw [shift={(575,393.27)}, rotate = 143.13] [fill={rgb, 255:red, 0; green, 0; blue, 0 }  ][line width=0.08]  [draw opacity=0] (10.72,-5.15) -- (0,0) -- (10.72,5.15) -- (7.12,0) -- cycle    ;
    %Straight Lines [id:da5230430384886475] 
    \draw    (418.33,273.27) -- (418.33,367.27) ;
    
    % Text Node
    \draw (564.33,370.27) node [anchor=north] [inner sep=0.75pt]   [align=left] {做完\textcolor[rgb]{0.29,0.56,0.89}{了}};
    % Text Node
    \draw (418,222) node [anchor=north] [inner sep=0.75pt]   [align=left] {\begin{minipage}[lt]{65.39pt}\setlength\topsep{0pt}
    \begin{center}
    \textcolor[rgb]{0.29,0.56,0.89}{aspect marker}\\\textcolor[rgb]{0.29,0.56,0.89}{[\category{perfective}]}
    \end{center}
    
    \end{minipage}};
    % Text Node
    \draw (647.33,225.27) node [anchor=north] [inner sep=0.75pt]   [align=left] {\begin{minipage}[lt]{38.85pt}\setlength\topsep{0pt}
    \begin{center}
    head:\\core VP
    \end{center}
    
    \end{minipage}};
    % Text Node
    \draw (530.33,168.27) node [anchor=south] [inner sep=0.75pt]   [align=left] {\begin{minipage}[lt]{60.61pt}\setlength\topsep{0pt}
    \begin{center}
    predicate:\\extended VP
    \end{center}
    
    \end{minipage}};
    % Text Node
    \draw (315,130) node [anchor=north] [inner sep=0.75pt]   [align=left] {\begin{minipage}[lt]{39.55pt}\setlength\topsep{0pt}
    \begin{center}
    subject:\\pronoun
    \end{center}
    
    \end{minipage}};
    % Text Node
    \draw (312.33,370.27) node [anchor=north] [inner sep=0.75pt]   [align=left] {我};
    % Text Node
    \draw (417.33,90.27) node [anchor=south] [inner sep=0.75pt]   [align=left] {\begin{minipage}[lt]{65.38pt}\setlength\topsep{0pt}
    \begin{center}
    nucleus clause
    \end{center}
    
    \end{minipage}};
    % Text Node
    \draw (737.17,369.69) node [anchor=north] [inner sep=0.75pt]   [align=left] {作业};
    % Text Node
    \draw (731.33,297.27) node [anchor=north] [inner sep=0.75pt]   [align=left] {\begin{minipage}[lt]{38.85pt}\setlength\topsep{0pt}
    \begin{center}
    object:\\ NP
    \end{center}
    
    \end{minipage}};
    % Text Node
    \draw (563.33,298.27) node [anchor=north] [inner sep=0.75pt]   [align=left] {\begin{minipage}[lt]{69.83pt}\setlength\topsep{0pt}
    \begin{center}
    predicator:\\verbal complex
    \end{center}
    
    \end{minipage}};
    
    
    \end{tikzpicture}
    
    }
    \caption{One tree diagram of (\ref{ex:grammatical.clause.core-vp.verbal-complex.1})}
    \label{fig:grammatical.clause.core-vp.verbal-complex.1.1}
\end{figure}

The category of verbal complements is rather heterogeneous,
its boundary (expectedly) being somewhat unclear;
it includes verbal complements or in other words complex predicates, 
complement clauses, 
and oblique arguments. 

The verbal complex is sometimes \emph{separable},
meaning that other constituents within the verb phrase can be incorporated into it.
In (\ref{ex:grammatical.clause.core-vp.verbal-complex.separable.1}),
for instance, the object is incorporated into the verbal complex.
The incorporated constituent is not limited to the direct object.

\begin{exe}
    \ex\label{ex:grammatical.clause.core-vp.verbal-complex.separable.1} \begin{xlist}
        \ex 这件事你关什么心啊
        \ex 这件事你关心什么啊
    \end{xlist}
\end{exe}
 

\subsubsection{Transitivity}

Verb frames in Mandarin can be divided into the \category{do} class (about actions),
the \category{become} class (about changes of states),
the \category{cause}-\category{become} class (about something causing a state to change),
and the purely stative \category{be} class.

Just like the case in English, \category{do} verb frames often allow S/A-ambivalence,
where the P argument of a transitive verb frame (i.e. the more internal, patient-like argument) can be removed, leaving only the subject (\ref{ex:grammatical.clause.core-vp.valency.do}).

\begin{exe}
    \ex\label{ex:grammatical.clause.core-vp.valency.do} \begin{xlist}
        \ex 他喜欢玩
        \ex 他喜欢玩玩具
    \end{xlist}
\end{exe}

On the other hand, we observe regular alternations between \category{become} and \category{cause}-\category{become} verbs,
and hence S/P-ambivalence (\ref{ex:grammatical.clause.core-vp.valency.become}).
It should be noted that not all \category{become} verbs can receive a \category{cause}-\category{become} verb frame (\ref{ex:grammatical.clause.core-vp.valency.become-only}).
A rather interesting phenomenon in Mandarin is the \category{experience}-\category{become} construction,
in which the subject \emph{experiences} the effect of a change-of-state situation
(\ref{ex:grammatical.clause.core-vp.valency.experience}).

\begin{exe}
    \ex\label{ex:grammatical.clause.core-vp.valency.become} \begin{xlist}
        \ex 茶泡好了
        \ex 我泡好茶了
    \end{xlist}

    \ex\label{ex:grammatical.clause.core-vp.valency.become-only}
    \begin{xlist}
        \ex 这只猫死了
        \ex *坏人死了这只猫
    \end{xlist}
    
    \ex\label{ex:grammatical.clause.core-vp.valency.experience}
    \begin{xlist}
        \ex 王冕死了父亲
    \end{xlist}
\end{exe}

The verb frames naturally have correlations with the lexical aspect of the clause:
change-of-state clauses are naturally telic and often cannot be in the progressive aspect
(\ref{ex:grammatical.clause.core-vp.valency.progressive.1}).
But counterexamples exist
(\ref{ex:grammatical.clause.core-vp.valency.progressive.2}).

\begin{exe}
    \ex\label{ex:grammatical.clause.core-vp.valency.progressive.1} *这只猫正在死
    \ex\label{ex:grammatical.clause.core-vp.valency.progressive.2} 我正在泡茶
\end{exe}

The A and P arguments discussed above are not the only types of arguments in Mandarin clauses.

First, the duration of a Mandarin \category{do} clause
can be measured by a so-called \concept{semi-object}
(\ref{ex:grammatical.clause.core-vp-valency.semi-object}).
This is related to the so-called pseudo-attributive construction
(\ref{ex:grammatical.clause.core-vp-valency.pseudo-attributive};
\prettyref{sec:grammatical.clause.core-vp.pseudo-attributive}).

\begin{exe}
    \ex\label{ex:grammatical.clause.core-vp-valency.semi-object} 我工作了一年
    \ex\label{ex:grammatical.clause.core-vp-valency.pseudo-attributive} 干了一个月的活
\end{exe}

Second, some verb frames have what we may call \concept{internal objects}, usually licensed by a verbal complement.
These arguments are immune to any further syntactic operations.
The simplest case is the preposition complement construction (\ref{ex:grammatical.clause.core-vp-valency.internal-object-1}).
The \category{cause}-\category{become}-internal object structure is also possible,
although due to various constraints, sometimes it can only be realized as a \category{disposal} construction in the surface form (\ref{ex:grammatical.clause.core-vp-valency.internal-object-2}).


\begin{exe}
    \ex\label{ex:grammatical.clause.core-vp-valency.internal-object-1}
    我住在上海
    \ex\label{ex:grammatical.clause.core-vp-valency.internal-object-2} 
    \begin{xlist}
        \ex 卡车装满了稻草
        \ex 他把卡车装满了稻草
    \end{xlist}
\end{exe}

Finally, there seems to be a \category{do}-\category{affect}-\category{patient} construction
(\ref{ex:grammatical.clause.core-vp.valency.do-affect-patient.1}).
This construction is tentatively classified as a subtype of \category{do} verb frames,
mainly because we have no semantic evidence for a \category{cause}-to-\category{lose} analysis,
especially in (\ref{ex:grammatical.clause.core-vp.valency.do-affect-patient.2}),
where it's hard to argue that 他 and 耳光 form a mini verb frame meaning that the person in question loses something
\citep{huang2007}.
This however raises the question whether some semi-objects are to be analyzed in the same way,
and not as a wide-scope quantity or frequency phrase: cf. 打了他一下.

\begin{exe}
    \ex\label{ex:grammatical.clause.core-vp.valency.do-affect-patient.1} 阿飞抢了我一顶帽子
    \ex\label{ex:grammatical.clause.core-vp.valency.do-affect-patient.2} 打了他一个耳光
\end{exe}

\subsubsection{Verb-particle constructions}

Just like English, Mandarin has directional and resultative particles in the argument structure.

\subsubsection{Pseudo-attributive constructions}\label{sec:grammatical.clause.core-vp.pseudo-attributive}

In the pseudo-attributive construction,
a constituent usually appearing as a determiner of a noun phrase, 
like a numeral (\ref{ex:grammatical.clause.core-vp.pseudo-attributive.numeral})
or a possessive (\ref{ex:grammatical.clause.core-vp.pseudo-attributive.possessive}),
is inserted into a verb phrase.
This constituent is often known as a pseudo-attributive.
We notice that a pseudo-attributive can be incorporated into the verbal complex
even when the verbal complex has no analyzable internal structure
(\ref{ex:grammatical.clause.core-vp.pseudo-attributive.numeral.incorporation}),
or when its structure clearly is not subject to nominal attributive modification
(\ref{ex:grammatical.clause.core-vp.pseudo-attributive.possessive.incorporation}).
The explanation 

\begin{exe}
    \ex\label{ex:grammatical.clause.core-vp.pseudo-attributive.numeral} 我们干了一年的活
    \ex\label{ex:grammatical.clause.core-vp.pseudo-attributive.possessive} 你当你的老师
    \ex\label{ex:grammatical.clause.core-vp.pseudo-attributive.numeral.incorporation} 幽了他一默
    \ex\label{ex:grammatical.clause.core-vp.pseudo-attributive.possessive.incorporation} 你保你的守
\end{exe}

\subsubsection{Verb copying construction}

\begin{exe}
    \ex 做工做了一星期
\end{exe}

\section{Noun phrase}

\subsection{The determiner region}

\begin{exe}
    \ex 这 二十 张 桌子
\end{exe}

\subsection{Adjectives}

\begin{exe}
    \ex 大白褂子
\end{exe}

Subcategorization exists within these so-called compounds.
Thus 

\begin{exe}
    \ex \begin{xlist}
        \ex 研究生招生工作
        \ex 中小学校招生工作
        \ex 本市中小学校招生工作
        \ex *中小学校本市招生工作
    \end{xlist}
\end{exe}

\section{Structure of the lexicon}

After a survey of the grammatical system of Mandarin Chinese,
we examine what the lexicon has to feed into the grammar.

\begin{todobox}{Word or phrases}{word-or-phrase}
    粘合式, 组合式

    念佛堂 etc.: compounds, or nominal attributives, or whatever?
\end{todobox}

\subsection{Roots}



\chapter{Information structure}

\section{Subject being topicalized}\label{sec:topic.subject}

\begin{exe}
    \ex 我明天跟你可能能在我的办公室讨论一下
\end{exe}

\section{(Absence of) dangling topics}\label{sec:topic-subject}

Some people, like \citet[\citesec{7.1}]{zhudexigrammar},
equate \term{subject} with \term{topic} in Mandarin grammar.
Some (especially those from the functional-typological tradition) go further 
and assert that ``the notion of the subject (as the position of the most agentive argument) 
isn't grammaticalized in Mandarin Chinese'',
and therefore the topic-comment construction 
is construed as simply the syntactic coding of aboutness,
and this base-generated and syntactically unconstrained topic 
is called a ``dangling topic''.
This view is rejected in this note,
because such accounts usually end up in severe overgeneration. 
Here I briefly summarize \citet{sih2000topic}'s argumentation.

\subsection{Type 1: Idiomatic phrasal predicate looking like a comment}\label{sec:clause.dangling-topic.1}

In the first type of ``dangling topic'',
it's impossible for any \acs{np} in the comment to be syntactically related to the topic
(\ref{ex:dayuchixiaoyu}, \ref{ex:nikankanwo}).
Such cases however should be analyzed as instances of the
subject-predicate construction,
where the predicate is a dephrasalized clause.

We notice that in such examples, 
the ``comment'' often has already undergone various degrees of fossilization.
Changing the comment usually makes the sentences much less felicitous 
(\ref{ex:dayuchixiaoyu-2}),
at best highly marked.
This is strange if the attested examples are topic-comment constructions,
but makes sense if dephrasalization is needed 
to put the clause 大鱼吃小鱼 etc. to the ``comment'' position.

Thus, in \eqref{ex:dayuchixiaoyu} and \eqref{ex:nikankanwo},
the so-called topic is an ordinary subject,
and the so-called comment is a predicate.
The meaning of the result of dephrasalization 
may be compared with the English colloquial 
\form{I was like, \dots} construction.

\begin{exe}
    \ex\label{ex:dayuchixiaoyu} 他们[大鱼吃小鱼](,厮杀成一片)
    \ex\label{ex:nikankanwo} 他们[你看看我我看看你]
    
    \ex\label{ex:dayuchixiaoyu-2} \begin{xlist}
        \ex *他们小鱼咬大鱼 
        \ex *他们虾米啃泥底
    \end{xlist}
\end{exe}

\subsection{Type 2: Quantificational adverbial looking like the inner subject}

The second type of ``dangling topic'' is like \eqref{ex:shui-dou-bu-pa}.
A topic-comment analysis of \eqref{ex:shui-dou-bu-pa} 

\begin{exe}
    \ex\label{ex:shui-dou-bu-pa} \gll 他们 谁 都 不 怕 \\
    3pl who even \acs{neg} fear \\
    \glt \translate{They don't fear anyone.}
\end{exe}

\subsection{Type 3: Ellipsis leaving a subject and one predicate}

Some people accept \eqref{ex:nasuofangzixingkuimeixiaxue}.
Here the \acs{np} 那所房子 definitely doesn't come from the words following it,
and is therefore recognized as a topic by some (TODO: ref). 
Note, however, that 幸亏 serves as a clause linker outside \eqref{ex:nasuofangzixingkuimeixiaxue}:
\eqref{ex:xingkui-buran-ex} is a demonstration of the 幸亏……不然…… linking construction,
and we also have its topicalized version \eqref{ex:xingkui-buran-fronted}. (TODO: whether this is parenthesis)
We also know in a clause linking construction,
often one clause can be omitted in the utterance because it's content can be easily inferred (TODO: ref).
So now the origin of \eqref{ex:nasuofangzixingkuimeixiaxue} is clear:
We can get it by omitting the second clause in the comment part of \eqref{ex:xingkui-buran-fronted}.
Indeed, if we replace 幸亏 by anything that is adverbial but not a clause linker,
the resulting sentence -- which now contains a real dangling topic -- is not grammatical.

\begin{exe}
    \ex \label{ex:nasuofangzixingkuimeixiaxue} \gll \% 那 所 房子 幸亏 没 下雪 \\
    {} that \acs{classify} house fortunate \acs{neg} snow \\
    \glt \translate{For that house, fortunately it didn't snow (or otherwise something bad would happen).}

    \ex\label{ex:xingkui-buran-ex} \gll [幸亏] 去年 没 下雪 , [不然] 那 所 房子 早就 塌 了 \\
    fortunate last.year \acs{neg} snow {} otherwise that \acs{classify} house already collapse \acs{sfp} \\
    \glt \translate{Fortunately it didn't snow last year, or otherwise that house has already collapsed.}

    \ex\label{ex:xingkui-buran-fronted} 
    \gll [ 那 所 房子 ]_{\text{topic}} [ 幸亏 去年 没 下雪 , 不然 早就 塌 了 ]_{\text{comment}} \\
    {} that \acs{classify} house {} {} fortunate last.year \acs{neg} snow {}  otherwise already collapse \acs{sfp} \\
\end{exe}

\subsection{Type 4: Extraction from prepositional adverbials}

\eqref{ex:zhejianshinibunengjiumafantayigeren} in \prettyref{sec:topicalization-of-preposition-objects} 
is sometimes regarded as an instance of the dangling topic construction.
However, as is shown in \prettyref{sec:topicalization-of-preposition-objects},
it may just be from topicalization of an \acs{np} in an adverbial,
with the preposition (and/or the locative particle) removed.

\subsection{Type 5: Nominal predicate}

\begin{exe}
    \ex 这种青菜一斤三十块钱
\end{exe}

\subsection{Type 6: Locational adverbial mistaken for the subject}

\begin{exe}
    \ex \gll \% 物价 纽约 最 贵  \\
    {} price New.York most expensive \\
    \glt \translate{The price in New York is the most expensive.}
\end{exe}

\subsection{Tentative conclusion}

The conclusion is all topics in Chinese are closely linked to a position in the comment,
be it a core argument position or a peripheral one.
So the notion of dangling topics is to be rejected in Mandarin grammar,
and we can always recover the ``canonical'' i.e. non-topic-comment clause
from a topic-comment construction.
After this, if the canonical clause can be divided into an \acs{np}
or a complement clause and a verbal constituent following it,
we can uncontroversially say the first is the subject while the second is the predicate. (TODO: predicate def)
So equating the subject with the topic is also wrong.

It's possible to find the semantic role of the subject isn't agentive;
in this case I assert there is a valency changing mechanism here.

\begin{infobox}{What to expect when people talk about the subject or the topic}{subject-topic}
    Unfortunately, despite the syntactic tests presented above,
    there are still many people -- even many native speakers -- 
    promoting the idea that the Mandarin topic has nothing different with the subject.
    Here is a list of TODO: ref
\end{infobox}


\end{document}