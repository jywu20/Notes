\documentclass[UTF8, a4paper, oneside, scheme=plain, 12pt]{ctexrep}

\usepackage{libertinus}
\usepackage{geometry}
\usepackage{float}
\usepackage{titling}
\usepackage{titlesec}
\usepackage{paralist}
\usepackage{footnote}
\usepackage{enumerate}
\usepackage{amsmath, amssymb, amsthm}
\usepackage{gb4e}
\noautomath
\usepackage{bbm}
\usepackage{textcomp}
\usepackage{soul}
\usepackage{graphicx}
\usepackage{siunitx}
\usepackage[table,xcdraw]{xcolor}
\usepackage{tikz}
\usepackage[ruled, vlined, linesnumbered, noend]{algorithm2e}
\usepackage{xr-hyper}
\usepackage[colorlinks, citecolor = purple]{hyperref} % linkcolor=black, anchorcolor=black, citecolor=black, filecolor=black
\usepackage[most]{tcolorbox}
\usepackage{caption}
\usepackage{subcaption}
\usepackage{booktabs}
\usepackage{multirow}
\usepackage[figuresright]{rotating}
\usepackage{acro}
\usepackage[citestyle=authoryear,backend=bibtex,natbib=true,doi=false,isbn=false,url=false]{biblatex}
\addbibresource{references/grammars.bib}
\addbibresource{references/aspects.bib}
\addbibresource{references/general-typology.bib}
\addbibresource{references/controversy.bib}
\usepackage{prettyref}

\geometry{left=3.18cm,right=3.18cm,top=2.54cm,bottom=2.54cm}
\titlespacing{\paragraph}{0pt}{1pt}{10pt}[20pt]
\setlength{\droptitle}{-5em}

\DeclareMathOperator{\timeorder}{\mathcal{T}}
\DeclareMathOperator{\diag}{diag}
\DeclareMathOperator{\legpoly}{P}
\DeclareMathOperator{\primevalue}{P}
\DeclareMathOperator{\sgn}{sgn}
\newcommand*{\ii}{\mathrm{i}}
\newcommand*{\ee}{\mathrm{e}}
\newcommand*{\const}{\mathrm{const}}
\newcommand*{\suchthat}{\quad \text{s.t.} \quad}
\newcommand*{\argmin}{\arg\min}
\newcommand*{\argmax}{\arg\max}
\newcommand*{\normalorder}[1]{: #1 :}
\newcommand*{\pair}[1]{\langle #1 \rangle}
\newcommand*{\fd}[1]{\mathcal{D} #1}

\newcommand*{\citesec}[1]{\S~{#1}}
\newcommand*{\citechap}[1]{chap.~{#1}}
\newcommand*{\citefig}[1]{Fig.~{#1}}
\newcommand*{\citetable}[1]{Table~{#1}}
\newcommand*{\citepage}[1]{p.~{#1}}
\newcommand*{\citepages}[1]{pp.~{#1}}
\newcommand*{\citefootnote}[1]{fn.~{#1}}

\newrefformat{sec}{\citesec{\ref{#1}}}
\newrefformat{fig}{\citefig{\ref{#1}}}
\newrefformat{tbl}{\citetable{\ref{#1}}}
\newrefformat{chap}{\citechap{\ref{#1}}}
\newrefformat{fn}{\citefootnote{\ref{#1}}}
\newrefformat{box}{Box~\ref{#1}}
\newrefformat{ex}{\ref{#1}}

% color boxes

\tcbuselibrary{skins, breakable, theorems}

\newtcbtheorem[number within=chapter]{infobox}{Box}{
    enhanced,
    boxrule=0pt,
    colback=blue!5,
    colframe=blue!5,
    coltitle=blue!50,
    borderline west={4pt}{0pt}{blue!65},
    sharp corners,
    fonttitle=\bfseries, 
    breakable,
    before upper={\parindent15pt\noindent}}{box}
\newtcbtheorem[number within=chapter, use counter from=infobox]{theorybox}{Box}{
    enhanced,
    boxrule=0pt,
    colback=orange!5, 
    colframe=orange!5, 
    coltitle=orange!50,
    borderline west={4pt}{0pt}{orange!65},
    sharp corners,
    fonttitle=\bfseries, 
    breakable,
    before upper={\parindent15pt\noindent}}{box}
\newtcbtheorem[number within=chapter, use counter from=infobox]{learnbox}{Box}{
    enhanced,
    boxrule=0pt,
    colback=green!5,
    colframe=green!5,
    coltitle=green!50,
    borderline west={4pt}{0pt}{green!65},
    sharp corners,
    fonttitle=\bfseries, 
    breakable,
    before upper={\parindent15pt\noindent}}{box}
\newtcbtheorem[number within=chapter, use counter from=infobox]{todobox}{Box}{
    enhanced,
    boxrule=0pt,
    colback=red!5,
    colframe=red!5,
    coltitle=red!50,
    borderline west={4pt}{0pt}{red!65},
    sharp corners,
    fonttitle=\bfseries, 
    breakable,
    before upper={\parindent15pt\noindent}}{box}

\newcommand*{\concept}[1]{\textbf{#1}}
\newcommand*{\term}[1]{\emph{#1}}
\newcommand{\form}[1]{\emph{#1}}
\newcommand{\work}[1]{\textit{#1}}

\newcommand{\redp}{\textasciitilde}

\DeclareAcronym{blt}{short = BLT, long = Basic Linguistic Theory}
\DeclareAcronym{cgel}{short = CGEL, long = The Cambridge Grammar of the English Language}
\DeclareAcronym{dm}{short = DM, long = Distributed Morphology}
\DeclareAcronym{tag}{long = Tree-adjoining grammar, short = TAG}
\DeclareAcronym{sfp}{long = sentence-final particle, short = \textsc{sfp}}
\DeclareAcronym{np}{long = noun phrase, short = NP}
\DeclareAcronym{vp}{long = verb phrase, short = VP}
\DeclareAcronym{pp}{long = preposition phrase, short = PP}
\DeclareAcronym{cls}{long = classifier, short = CLS}
\DeclareAcronym{dist}{long = distal, short = DIST}
\DeclareAcronym{prox}{long = proximate, short = PROX}
\DeclareAcronym{dem}{long = demonstrative, short = DEM}
\DeclareAcronym{classify}{long = classifier, short = \textsc{cl}}
\DeclareAcronym{dur}{long = durative, short = DUR}
\DeclareAcronym{neg}{long = negative, short = \textsc{neg}}
\DeclareAcronym{cc}{long = copular complement, short = CC}
\DeclareAcronym{cs}{long = copular subject, short = CS}
\DeclareAcronym{tame}{long = {tense, aspect, mood, evidentiality}, short = TAME}
\DeclareAcronym{past}{long = past, short = PST}
\DeclareAcronym{nonpast}{long = non-past, short = NPST}
\DeclareAcronym{present}{long = present, short = PRES}
\DeclareAcronym{progressive}{long = progressive, short = \textsc{poss}}
\DeclareAcronym{perfect}{long = perfect, short = \textsc{perf}}
\DeclareAcronym{passive}{long = passive, short = \textsc{pass}}
\DeclareAcronym{copula}{long = copula, short = COP}
\DeclareAcronym{possessive}{long = possessive, short = \textsc{poss}}

\newcommand{\asis}[1]{\textsc{#1}}
\newcommand{\oneof}[1]{{#1}}
\newcommand*{\homo}[2]{#1$_{\text{#2}}$}

\newcommand{\ala}{à la}
\newcommand{\translate}[1]{`#1'}
\newcommand{\vP}{\textit{v}P}
\newcommand*{\category}[1]{\textsc{#1}}
\newcommand*{\specialunit}[1]{$<$\textit{#1}$>$}

% Make subsubsection labeled
\setcounter{secnumdepth}{4}
\setcounter{tocdepth}{4}
% reset example counter every chapter (but do not include the chapter number to the label)
\counterwithin{exx}{chapter} 

% Reference formats
\renewcommand*{\nameyeardelim}{\space} % No comma between year and name
\DeclareNameAlias{sortname}{family-given} % Putting the family name before the given name
\DeclareNameAlias{default}{family-given} 
\DeclareFieldFormat{labelnumberwidth}{} % No number label like [12] in the reference list
\setlength{\biblabelsep}{0pt} % No space for these labels

\title{Mandarin Chinese notes}
\author{Jinyuan Wu}

\begin{document}

\automath

\maketitle

\chapter{Grammatical overview}

\section{Morphological typology}

Mandarin lacks prototypical inflectional morphology but has rich derivational morphology.
Compounding is the most frequent morphological device,
and partly due to influences of European languages, partly due to grammaticalization,
affixation is also frequently seen.
Plus, reduplication plays an important role in Mandarin verbal and adjectival derivation.

One caveat about talking about derivational morphology is that it assumes
the existence of a well-defined wordhood.
It's often claimed that Mandarin lacks the word/phrase distinction.
In this note, we will show that wordhood can indeed be clearly defined 
by syntactic, morphological and phonological standards.
What makes Mandarin unique is that the three definitions of wordhood do not always overlap.

\begin{todobox}{Different standards of woodhood}{wordhood}
    Define
\end{todobox}

Because of this, Chinese lexicography is usually based on \emph{morphemes},

\section{Clauses}

\subsection{Top-level concepts}

\paragraph*{Clauses are made of nucleus clauses and high-level categories}
A clause can be divided into several clauses 
linked by clause linking constructions.%
\footnote{
    \citet{cgel} uses the term \term{sentence} 
    to refer to a natural unit in dialogue,
    which I refer to as a \term{utterance}.
    The term \term{sentence} here refers to 
    a clause that qualifies as an utterance. 

    Some people, like \citet[\citepage{140}]{deng2010formal}
    as well as \citet{dixon2009basic},
    use the term \term{clause} for subject-predict constructions 
    with no speech force marking.
    (\citet{deng2010formal} uses 句子 as the Mandarin counterpart of \term{sentence}
    and 小句 as the counterpart of \term{clause}.)
    In this way, \acl{sfp}s strictly shouldn't be
    regarded as a part of the clause, 
    and they may be discussed together with 
    other higher level constructions like clause linking. 
    This notion of clause correctly highlights the hierarchy in clausal structures.
    The problem with this terminology however is that in traditional grammars,
    the term \term{clause} does refer to units that have \ac{sfp}s.
    
    This note therefore refers to all units larger than the 
    subject-predicate construction as clauses, 
    which may or may not be sentence.
    The subject-predicate construction is instead named the \emph{nucleus} clause.
    The internal complexity of a clause 
    is still relevant for example in clause combining.
}
Mandarin has ample information marking phenomena,
and thus a clause can be divided into
one or more topics, if any, and a comment,
the latter being the nucleus clause
plus possible \ac{sfp}s (\ref{ex:grammatical.clause.high-level.topic.1}). 
Note that topicalization and coordination can happen successively
(TODO: ref),
and coordination can also happen inside the nucleus clause
(TODO: ref).

\begin{exe}
    \ex\label{ex:grammatical.clause.high-level.topic.1}
    \gll {} [张三]_{\text{topic},i}, [[他]_{\text{subject},i} 就 是 [个 王八蛋]_{\text{copular complement}}]_{\text{nucleus clause}} 罢了! \\
    {} \category{name} 3 just be \category{cls} turtle-egg \category{sfp} \\
    \translate{Zhang San is a son of a bitch!}
\end{exe}

\paragraph*{The nucleus clause is a subject plus a predicate}
The nucleus clause contains a subject (if any) and what is often known as a predicate,%
\footnote{
    \citet{dixon2009basic} argues against the definition of \term{predicate} 
    as the main verb (or adjective) plus somehow ``internal'' arguments.
    He uses the term \term{predicate} to refer to the verbal complex instead.
    However, since I will need to compare the topic-comment construction 
    with the inner structure of the nucleus clause,
    the term \term{predicate} will still be used in the way \citet{dixon2009basic} dislikes,
    because it's the counterpart of the comment role in the topic-comment construction.
}
which usually is a (extended) verb phrase but may also be a nominal. 
In the Mandarin simple nucleus clause,
the definition of the subject, as opposed to the topic,
is not trivially clear.
The concept of the subject, and also the nominative-accusative typological classification of Mandarin, is justified in \prettyref{sec:grammatical.clause.subject}.

When we have verbal prediction, the full, extended \acs{vp} following the subject 
can be divided into an extension region and the core \acs{vp}.
The extension region contains 
\acs{tame} auxiliaries and adverbs not realized in the verbal complex (\prettyref{sec:grammatical.clause.tam}),
and peripheral arguments like temporal and spatial locations (\prettyref{sec:grammatical.clause.peripheral}).
Sometimes the object may be fronted and 
it's also possible that a prepositional complement is fronted to this region.

\paragraph*{Details in the extended verb phrase}
In the surface form, core \acs{vp} contains the core arguments and the verbal complex%
\footnote{
    The term \term{verbal complex} is used to highlight
    its derivation from prototypical verb derivation and inflection.
    See the beginning of \prettyref{chap:vp.verbal-complex}.
}
(\prettyref{chap:vp.verbal-complex}).
The verbal complex involves aspectual marking (\prettyref{sec:grammatical.clause.tam}), 
verb derivation and verbal complements (see below);
lexical aspect is not grammatically marked but is important.

The argument structure (\prettyref{chap:vp.argument}), 
including both the roles of the arguments 
in the situation described by the verb
(i.e. ``deep argument slots'', like agent, patient, etc.),%
\footnote{
    Note that these positions are still syntactic concepts,
    like \category{agent} or \category{causer} in \prettyref{chap:vp.argument} 
    since they are determined at least partly by syntactic criteria;
    classification of truly semantic argument roles 
    is much complicated. 
}
and the roles of the arguments in the clause 
(i.e. ``surface argument slots'', like subject, object, etc.).
The deep and surface positions of internal arguments 
are already treated in \prettyref{chap:vp.argument}. 
The subject is also a part of the argument structure;
its properties as a clausal pivot
however involve grammatical concepts beyond the scope of the \acs{vp}
(\prettyref{sec:grammatical.clause.subject}).
The claim that there is no grammaticalized argument structure in Mandarin 
is examined in \prettyref{sec:clause.vp.argument-vs-information}.

\paragraph*{Verbal complements, or complex predicates}
Mandarin has lots of clausal complements that are not prototypically arguments, 
known as 补语 in Chinese linguistic community.%
\footnote{
    The term 补语 literally means \translate{complementation speech}, 
    and is therefore often translated as \term{complement}.
    In this note I use the term \term{complement}
    to refer to grammatical constituents that are somehow more closely 
    related to the lexical head, 
    and I choose the (somehow tedious but explicit) term 
    \term{non-argument complement}.
}
This is a rather heterogeneous category,
its boundary (expectedly) being somewhat unclear;
it includes verbal complements or in other words complex predicates, 
complement clauses, 
and oblique arguments. 

\paragraph*{Examples of the nucleus clause}
(\ref{ex:vp.ex.1}) is an illustration of a complicated nucleus clause,
whose constituent structure is shown in \prettyref{fig:vp.ex.1}.

\begin{exe}
    \ex \label{ex:vp.ex.1}
    \gll 我 [明天 可能 能 在 我 的 办公室 跟 你 [讨论 一下]_{\text{core\acs{vp}}}]_{\text{extended \acs{vp}}} \\
    1 tomorrow \category{aux}:possible \category{aux}:ability at my \category{poss} 
    office with 2 discuss a.little.bit \\ 
    \glt \translate{Tomorrow possiblity I can have a discussion with you in my office.}
\end{exe}

\begin{figure}[H]
    {
        \small 
        \begin{tikzpicture}[x=0.75pt,y=0.75pt,yscale=-0.8,xscale=0.8]
    %uncomment if require: \path (0,740); %set diagram left start at 0, and has height of 740
    
%Straight Lines [id:da5419452521165726] 
\draw    (690.67,583.33) -- (742,583.33) ;
%Straight Lines [id:da08544086505959503] 
\draw    (713,511.59) -- (706.23,533.35) -- (690.67,583.33) ;
%Straight Lines [id:da5166120577578295] 
\draw    (713,511.59) -- (742,583.33) ;
%Straight Lines [id:da7597929142889452] 
\draw    (410,368.26) -- (410,584.33) ;
%Straight Lines [id:da7951507355099614] 
\draw    (312.67,294.26) -- (312.67,584.33) ;
%Straight Lines [id:da27035558491953804] 
\draw    (622.67,583.33) -- (642,583.33) ;
%Straight Lines [id:da24028174129699176] 
\draw    (630,511.59) -- (622.67,583.33) ;
%Straight Lines [id:da917015263091598] 
\draw    (630,511.59) -- (642,583.33) ;
%Straight Lines [id:da14883621053934304] 
\draw    (470,583.33) -- (547,583.33) ;
%Straight Lines [id:da9342077761978527] 
\draw    (500,441.59) -- (470,583.33) ;
%Straight Lines [id:da3488354163696341] 
\draw    (500,441.59) -- (547,583.33) ;
%Straight Lines [id:da5030694457277152] 
\draw    (673,441.59) -- (628.67,462.33) ;
%Straight Lines [id:da7531987353356717] 
\draw    (673,441.59) -- (710.67,462.33) ;
%Straight Lines [id:da40179355911176984] 
\draw    (589,370.26) -- (502,398.33) ;
%Straight Lines [id:da5639473207238246] 
\draw    (589,370.26) -- (672,398.33) ;
%Straight Lines [id:da5704299439338378] 
\draw    (497,294.26) -- (410,322.33) ;
%Straight Lines [id:da8053817579360183] 
\draw    (497,294.26) -- (587,322.33) ;
%Straight Lines [id:da8066766119523336] 
\draw    (402,220.26) -- (315,248.33) ;
%Straight Lines [id:da8072938752013339] 
\draw    (402,220.26) -- (492,248.33) ;
%Straight Lines [id:da35978458309470907] 
\draw    (199.67,219.76) -- (199.67,475.49) -- (199.67,583.33) ;
%Straight Lines [id:da927464810721838] 
\draw    (294,146.26) -- (199.33,176.93) ;
%Straight Lines [id:da9186562853518254] 
\draw    (294,146.26) -- (400.33,176.93) ;
%Straight Lines [id:da7086875966035955] 
\draw    (190,71.59) -- (95.33,102.26) ;
%Straight Lines [id:da7175081469679203] 
\draw    (190,71.59) -- (296.33,102.26) ;
%Straight Lines [id:da26209295176984004] 
\draw    (87.67,146.56) -- (87.67,583.33) ;
%Straight Lines [id:da43013746492722116] 
\draw [color={rgb, 255:red, 80; green, 227; blue, 194 }  ,draw opacity=1 ][line width=2.25]    (191.67,623.26) -- (643.78,623.26) ;
%Straight Lines [id:da47071956794766767] 
\draw [color={rgb, 255:red, 74; green, 144; blue, 226 }  ,draw opacity=1 ][line width=2.25]    (686.78,623.26) -- (755,623.26) ;

% Text Node
\draw (78.67,590.17) node [anchor=north west][inner sep=0.75pt]   [align=left] {我 \ \ \ \ \ };
% Text Node
\draw (183.67,590.17) node [anchor=north west][inner sep=0.75pt]   [align=left] {明天};
% Text Node
\draw (293.67,590.17) node [anchor=north west][inner sep=0.75pt]   [align=left] {可能};
% Text Node
\draw (400.67,590.17) node [anchor=north west][inner sep=0.75pt]   [align=left] {能};
% Text Node
\draw (614.67,590.17) node [anchor=north west][inner sep=0.75pt]   [align=left] {跟你};
% Text Node
\draw (463.67,590.17) node [anchor=north west][inner sep=0.75pt]   [align=left] {在我的办公室};
% Text Node
\draw (687.67,590.17) node [anchor=north west][inner sep=0.75pt]   [align=left] {讨论一下};
% Text Node
\draw (713,508.59) node [anchor=south] [inner sep=0.75pt]   [align=left] {\begin{minipage}[lt]{38.85pt}\setlength\topsep{0pt}
\begin{center}
head:\\core VP
\end{center}

\end{minipage}};
% Text Node
\draw (630,508.59) node [anchor=south] [inner sep=0.75pt]   [align=left] {\begin{minipage}[lt]{51.96pt}\setlength\topsep{0pt}
\begin{center}
comitative:\\PP
\end{center}

\end{minipage}};
% Text Node
\draw (673,438.59) node [anchor=south] [inner sep=0.75pt]   [align=left] {\begin{minipage}[lt]{80.61pt}\setlength\topsep{0pt}
\begin{center}
head:\\extended VP
\end{center}

\end{minipage}};
% Text Node
\draw (500,438.59) node [anchor=south] [inner sep=0.75pt]   [align=left] {\begin{minipage}[lt]{40.93pt}\setlength\topsep{0pt}
\begin{center}
location:\\PP
\end{center}

\end{minipage}};
% Text Node
\draw (589,367.26) node [anchor=south] [inner sep=0.75pt]   [align=left] {\begin{minipage}[lt]{60.61pt}\setlength\topsep{0pt}
\begin{center}
head:\\extended VP
\end{center}

\end{minipage}};
% Text Node
\draw (410,365.26) node [anchor=south] [inner sep=0.75pt]   [align=left] {\begin{minipage}[lt]{64.42pt}\setlength\topsep{0pt}
\begin{center}
modality aux\\ {[\category{ability}]}
\end{center}

\end{minipage}};
% Text Node
\draw (497,291.26) node [anchor=south] [inner sep=0.75pt]   [align=left] {\begin{minipage}[lt]{60.61pt}\setlength\topsep{0pt}
\begin{center}
head:\\extended VP
\end{center}

\end{minipage}};
% Text Node
\draw (312.67,291.26) node [anchor=south] [inner sep=0.75pt]   [align=left] {\begin{minipage}[lt]{64.42pt}\setlength\topsep{0pt}
\begin{center}
modality aux\\ {[\category{possibility}]}
\end{center}

\end{minipage}};
% Text Node
\draw (402,217.26) node [anchor=south] [inner sep=0.75pt]   [align=left] {\begin{minipage}[lt]{60.61pt}\setlength\topsep{0pt}
\begin{center}
head:\\extended VP
\end{center}

\end{minipage}};
% Text Node
\draw (199.67,216.76) node [anchor=south] [inner sep=0.75pt]   [align=left] {\begin{minipage}[lt]{64.14pt}\setlength\topsep{0pt}
\begin{center}
time location:\\adverb
\end{center}

\end{minipage}};
% Text Node
\draw (294,143.26) node [anchor=south] [inner sep=0.75pt]   [align=left] {\begin{minipage}[lt]{60.61pt}\setlength\topsep{0pt}
\begin{center}
predicate:\\extended VP
\end{center}

\end{minipage}};
% Text Node
\draw (190,68.59) node [anchor=south] [inner sep=0.75pt]   [align=left] {\begin{minipage}[lt]{65.38pt}\setlength\topsep{0pt}
\begin{center}
nucleus clause
\end{center}

\end{minipage}};
% Text Node
\draw (87.67,143.56) node [anchor=south] [inner sep=0.75pt]   [align=left] {\begin{minipage}[lt]{39.55pt}\setlength\topsep{0pt}
\begin{center}
subject:\\pronoun
\end{center}

\end{minipage}};
% Text Node
\draw (417.72,626.26) node [anchor=north] [inner sep=0.75pt]  [color={rgb, 255:red, 80; green, 227; blue, 194 }  ,opacity=1 ] [align=left] {extension};
% Text Node
\draw (720.89,626.26) node [anchor=north] [inner sep=0.75pt]  [color={rgb, 255:red, 74; green, 144; blue, 226 }  ,opacity=1 ] [align=left] {core VP};


    
    
    \end{tikzpicture}
    
    }
    \caption{Tree diagram of (\ref{ex:vp.ex.1})}
    \label{fig:vp.ex.1}
\end{figure}

It should be noted that in the disposal and passive constructions,
the manner phrase may appear \emph{after} the auxiliary (\ref{ex:vp.ex.2}),
and in this case the boundary of the core \acs{vp}
can't be defined at the surface level.
This hints at the existence of in-\acs{vp} information structure marking in Mandarin
(TODO: comparison with Latin).

\begin{exe}
    \ex\label{ex:vp.ex.2}
    \gll 我 明天 可能 能 在 我 的 办公室 跟 你 [把]_{\text{auxiliary}}  这 个 问题 [好好]_{\text{manner}} 讨论 一下 \\
    1 tomorrow \category{aux}:possible \category{aux}:ability 
    at my \category{poss} office 
    with 2 
    \category{ba} this \category{cls} problem good
    discuss a.little.bit \\
    \glt \translate{Tomorrow possiblity I can have a good discussion of this problem with you in my office.}
\end{exe}

\subsection{Clause types, clause combining, information structure, and sentence final particles}

\subsection{The subject}\label{sec:grammatical.clause.subject}

\subsection{Tense, aspect and modality marking}\label{sec:grammatical.clause.tam}

\subsection{Peripheral arguments}\label{sec:grammatical.clause.peripheral}

\section{Parts of speech}

After a survey of the grammatical system of Mandarin Chinese,
we examine what the lexicon has to feed into the grammar.

\end{document}