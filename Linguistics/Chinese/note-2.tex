\documentclass[UTF8, a4paper, oneside, scheme=plain, 12pt]{ctexrep}

\usepackage{libertinus}
\usepackage{geometry}
\usepackage{float}
\usepackage{titling}
\usepackage{titlesec}
\usepackage{paralist}
\usepackage{footnote}
\usepackage{enumerate}
\usepackage{amsmath, amsthm}
\usepackage[pysep={}]{xpinyin}
\usepackage{gb4e}
\noautomath
\usepackage{bbm}
\usepackage{textcomp}
\usepackage{soul}
\usepackage{graphicx}
\usepackage{siunitx}
\usepackage[table,xcdraw]{xcolor}
\usepackage{tikz}
\usepackage[ruled, vlined, linesnumbered, noend]{algorithm2e}
\usepackage{xr-hyper}
\usepackage[colorlinks, citecolor = purple]{hyperref} % linkcolor=black, anchorcolor=black, citecolor=black, filecolor=black
\usepackage[most]{tcolorbox}
\usepackage{caption}
\usepackage{subcaption}
\usepackage{booktabs}
\usepackage{multirow}
\usepackage[figuresright]{rotating}
\usepackage{acro}
\usepackage[citestyle=authoryear,backend=bibtex,natbib=true,doi=false,isbn=false,url=false]{biblatex}
\addbibresource{references/grammars.bib}
\addbibresource{references/aspects.bib}
\addbibresource{references/general-typology.bib}
\addbibresource{references/controversy.bib}
\addbibresource{references/historical-phonology.bib}
\usepackage{prettyref}

\geometry{left=3.18cm,right=3.18cm,top=2.54cm,bottom=2.54cm}
\titlespacing{\paragraph}{0pt}{1pt}{10pt}[20pt]
\setlength{\droptitle}{-5em}

\DeclareMathOperator{\timeorder}{\mathcal{T}}
\DeclareMathOperator{\diag}{diag}
\DeclareMathOperator{\legpoly}{P}
\DeclareMathOperator{\primevalue}{P}
\DeclareMathOperator{\sgn}{sgn}
\newcommand*{\ii}{\mathrm{i}}
\newcommand*{\ee}{\mathrm{e}}
\newcommand*{\const}{\mathrm{const}}
\newcommand*{\suchthat}{\quad \text{s.t.} \quad}
\newcommand*{\argmin}{\arg\min}
\newcommand*{\argmax}{\arg\max}
\newcommand*{\normalorder}[1]{: #1 :}
\newcommand*{\pair}[1]{\langle #1 \rangle}
\newcommand*{\fd}[1]{\mathcal{D} #1}

\newcommand*{\citesec}[1]{\S~{#1}}
\newcommand*{\citechap}[1]{chap.~{#1}}
\newcommand*{\citefig}[1]{Fig.~{#1}}
\newcommand*{\citetable}[1]{Table~{#1}}
\newcommand*{\citepage}[1]{p.~{#1}}
\newcommand*{\citepages}[1]{pp.~{#1}}
\newcommand*{\citefootnote}[1]{fn.~{#1}}

\newrefformat{sec}{\citesec{\ref{#1}}}
\newrefformat{fig}{\citefig{\ref{#1}}}
\newrefformat{tbl}{\citetable{\ref{#1}}}
\newrefformat{chap}{\citechap{\ref{#1}}}
\newrefformat{fn}{\citefootnote{\ref{#1}}}
\newrefformat{box}{Box~\ref{#1}}
\newrefformat{ex}{\ref{#1}}

\newcommand*{\textgt}{$>$ }
\newcommand*{\textlt}{$<$ }
\newcommand*{\textto}{$\to$ }

% color boxes

\tcbuselibrary{skins, breakable, theorems}

\newtcbtheorem[number within=chapter]{infobox}{Box}{
    enhanced,
    boxrule=0pt,
    %colback=blue!5,
    %colframe=blue!5,
    colback=white,
    colframe=white,
    coltitle=blue!60,
    borderline west={4pt}{0pt}{blue!65},
    sharp corners,
    fonttitle=\bfseries, 
    breakable,
    before upper={\parindent15pt\noindent}}{box}
\definecolor{my-orange}{HTML}{F58123}
\newtcbtheorem[number within=chapter, use counter from=infobox]{theorybox}{Box}{
    enhanced,
    boxrule=0pt,
    %colback=orange!5, 
    %colframe=orange!5, 
    colback=white,
    colframe=white,
    coltitle=my-orange!80,
    borderline west={4pt}{0pt}{my-orange!80},
    sharp corners,
    fonttitle=\bfseries, 
    breakable,
    before upper={\parindent15pt\noindent}}{box}
\newtcbtheorem[number within=chapter, use counter from=infobox]{todobox}{Box}{
    enhanced,
    boxrule=0pt,
    colback=red!5,
    colframe=red!5,
    coltitle=red!50,
    borderline west={4pt}{0pt}{red!65},
    sharp corners,
    fonttitle=\bfseries, 
    breakable,
    before upper={\parindent15pt\noindent}}{box}
\newtcbtheorem[number within=chapter, use counter from=infobox]{perspectivebox}{Box}{
    enhanced,
    boxrule=0pt,
    %colback=red!5,
    %colframe=red!5,
    colback=white,
    colframe=white,
    coltitle=red!50,
    borderline west={4pt}{0pt}{red!65},
    sharp corners,
    fonttitle=\bfseries, 
    breakable,
    before upper={\parindent15pt\noindent}}{box}

\AtBeginEnvironment{infobox}{\small}
\AtBeginEnvironment{todobox}{\small}
\AtBeginEnvironment{theorybox}{\small}

\newcommand*{\concept}[1]{\textbf{#1}}
\newcommand*{\term}[1]{\emph{#1}}
\newcommand{\form}[1]{\emph{#1}}
\newcommand{\work}[1]{\textit{#1}}

\newcommand{\redp}{\textasciitilde}

\DeclareAcronym{blt}{short = BLT, long = Basic Linguistic Theory}
\DeclareAcronym{cgel}{short = CGEL, long = The Cambridge Grammar of the English Language}
\DeclareAcronym{dm}{short = DM, long = Distributed Morphology}
\DeclareAcronym{tag}{long = Tree-adjoining grammar, short = TAG}
\DeclareAcronym{sfp}{long = sentence-final particle, short = \textsc{sfp}}
\DeclareAcronym{np}{long = noun phrase, short = NP}
\DeclareAcronym{vp}{long = verb phrase, short = VP}
\DeclareAcronym{pp}{long = preposition phrase, short = PP}
\DeclareAcronym{cls}{long = classifier, short = CLS}
\DeclareAcronym{dist}{long = distal, short = DIST}
\DeclareAcronym{prox}{long = proximate, short = PROX}
\DeclareAcronym{dem}{long = demonstrative, short = DEM}
\DeclareAcronym{classify}{long = classifier, short = \textsc{cl}}
\DeclareAcronym{dur}{long = durative, short = DUR}
\DeclareAcronym{neg}{long = negative, short = \textsc{neg}}
\DeclareAcronym{cc}{long = copular complement, short = CC}
\DeclareAcronym{cs}{long = copular subject, short = CS}
\DeclareAcronym{tame}{long = {tense, aspect, mood, evidentiality}, short = TAME}
\DeclareAcronym{past}{long = past, short = PST}
\DeclareAcronym{nonpast}{long = non-past, short = NPST}
\DeclareAcronym{present}{long = present, short = PRES}
\DeclareAcronym{progressive}{long = progressive, short = \textsc{poss}}
\DeclareAcronym{perfect}{long = perfect, short = \textsc{perf}}
\DeclareAcronym{passive}{long = passive, short = \textsc{pass}}
\DeclareAcronym{copula}{long = copula, short = COP}
\DeclareAcronym{possessive}{long = possessive, short = \textsc{poss}}

\newcommand{\asis}[1]{\textsc{#1}}
\newcommand{\oneof}[1]{{#1}}
\newcommand*{\homo}[2]{#1$_{\text{#2}}$}

\newcommand{\ala}{à la}
\newcommand{\translate}[1]{`#1'}
\newcommand{\vP}{\textit{v}P}
\newcommand*{\category}[1]{\textsc{#1}}
\newcommand*{\wordroot}[1]{$\sqrt{\text{\textsc{#1}}}$}
\newcommand*{\specialunit}[1]{$<$\textit{#1}$>$}
\newcommand{\personname}[1]{#1-\category{name}}

% Make subsubsection labeled
\setcounter{secnumdepth}{4}
\setcounter{tocdepth}{4}
% reset example counter every chapter (but do not include the chapter number to the label)
\counterwithin{exx}{chapter} 

% Reference formats
\renewcommand*{\nameyeardelim}{\space} % No comma between year and name
\DeclareNameAlias{sortname}{family-given} % Putting the family name before the given name
\DeclareNameAlias{default}{family-given} 
\DeclareFieldFormat{labelnumberwidth}{} % No number label like [12] in the reference list
\setlength{\biblabelsep}{0pt} % No space for these labels

\title{A grammar of Standard Mandarin Chinese}
\author{Jinyuan Wu}

\begin{document}

\automath

\maketitle

\chapter{Introduction}

\begin{todobox}{Topics to be covered}{topics}
    \begin{itemize}
        \item Relative clauses 
        \item Lexical aspect and its relation with le zhe guo
        \item Relation with Middle Chinese
        \item More discussions on wordhood (and avoid using terms hinting at wordhood in the description)
        \item Def. of verb stem
    \end{itemize}
\end{todobox}

\section{History and classification}

\subsection{Early history}

A pattern observable in the history of the Sinitic family is that divergent development and drastic loss of linguistic diversity happened alternately.
Unlike the case of Tibetan, which has a East Bodish sister group,
the modern Sinitic family has no sister group with a shared Pre-Proto-Sinitic ancestor.
This suggests that even before the Shang dynasty (which was the earliest archaeologically demonstrable dynasty in the history of China),
language uniformity to a certain extent was already in place.

The earliest evidence of what we can call Chinese is from 
oracle bone inscriptions and the transmitted text of 尚书 \translate{\work{Book of Documents}},
corresponding to the languages of Shang and Zhou dynasties, respectively.
These, together with the phonetic components of Chinese characters
and the rhyme patterns in 诗经 \translate{\work{Book of Odes}},
can be used to reconstruct Pre-Classical Old Chinese,
which had typological features radically different from stereotypical ``Sinitic'' features
\citep{baxter2014old}.
It should be noted that the languages of oracle bone inscriptions,
the \work{Book of Documents}, and the \work{Odes} are clearly different,
meaning that linguistic divergence had already appeared in the pre-Classical period,
and reconstruction of Pre-Classical Chinese should consider the dialectal differences
\citep[\citepages{478-9}]{harbsmeier2016irrefutable}.

The classical period started from the Spring and Autumn period and to late Han dynasty.
Dialectal divergence was clearly still a thing during this period,
but uniformity of at least the literary language already appeared:
all transmitted texts, even 楚辞, are written in what looks like Classical Chinese
\citep[\citepage{447}]{harbsmeier2016irrefutable},
although regional phonological differences of this standard language are to be expected \citep[\citepage{488}]{harbsmeier2016irrefutable}.

Whether 

\subsection{Middle Chinese}

Being historically accurate or not, we can at least say that Middle Chinese, as described in the rhyme books and rhyme tables,
\emph{is} a well-defined proto-language in terms of Neogrammarian historical linguistics,
because the phonology modern Sinitic languages can be derived by regularly applying sound change laws to it.


\section{Methodology}

{
\small

This part, printed in smaller letters, is for linguistic nerds.
\citet{culicover2004cambridge} criticizes \citet{cgel}
for its lack of explicit connections to contemporary syntactic theories,
and we decide to specifically add a section on how our methodology compares to contemporary
theoretical and descriptive linguistic frameworks.

\subsection{Theoretical framework}\label{sec:theory.framework}

The underlying theoretical framework of this grammar is inspired by Distributed Morphology \citep{siddiqi2009syntax} and Cartography \citep{cinque1999adverbs}.
This is to say, we assume that
\begin{itemize}
    \item[(a)] a grammatical construction is made of a \emph{root} surrounded by a hierarchy of \emph{functional heads} (corresponding to grammatical features and categories in traditional grammar) and their specifiers (in the generative syntactic sense), and 
    \item[(b)] the functional heads often follow a relatively cross-linguistically stable hierarchy,
    which strongly influences the linear order of auxiliaries, adverbs (which are in specifier positions), etc. and their scopes
    (e.g. \prettyref{fig:vp.ex.1}), and 
    \item[(c)] post-syntactic morphological and phonological operations can bring roots and functional heads together,
    which is subject to \emph{phase theory} (or similar cyclic constraints of syntactic derivation),
    which then are subject to phonological realization, and 
    \item[(d)] the whole process is guided by the lexicon,
    in which the list of roots, idiomized (i.e. \term{lexicalized}) meanings of constructions,
    and details of phonological realization
    (which, by the way, gives us subcategorization:
    a root that can only be phonologically realized with the \category{transitive} functional head
    which introduces the direct object is a transitive verb; see \citealt{siddiqi2009syntax}),
    and the last two lists in theory can be independent to each other
    (\prettyref{sec:intro.theory.lexicon}).
\end{itemize}

Grammar, in this framework, is about how roots are ``dressed up'' by grammatical constructions,
and grammatical constructions can be fully described
in terms of hierarchical organization of grammatical categories based on constituency relations, and their phonological realization.
The framework adopted here therefore has a clear lexical/functional distinction,
which can be tested by checking whether a formative is within a fixed hierarchy 
(e.g. \form{have been being observed} and the similar ordering of tense, aspect and modality adverbs in English: \category{tense} \textgt{}\category{perfect} \textgt{}\category{progressive} \textgt{}\category{passive}; 
in Mandarin see e.g. \prettyref{sec:grammatical.clause.tam}).
Certain gradience however is allowed because multiple analyses may appear at the same time
in the mental grammar of a speaker of the language.

Whether the current version of Distributed Morphology of Cartography is literally correct or not 
is far beyond the scope of this grammar.
Instead, we rely on the general idea of them to build an \emph{informal} descriptive framework
which enables cross-linguistic comparison.

\subsection{Relation with descriptive grammars}\label{sec:theory.descriptive}

The framework above may seem strange for descriptive linguists.
Here we ``explain'' concepts in descriptive grammars in terms of concepts in the framework
of Distributed Morphology plus Cartography.

\begin{itemize}
    \item[(a)] Definition of head. \term{Heads} in \prettyref{sec:theory.framework} are all \emph{functional} heads,
    i.e. (markers of) grammatical categories.
    Under the more traditional definition of \concept{head},
    we have \emph{noun} phrases and \emph{verb} phrases,
    and we have constructions without heads, like coordination constructions.
    In our framework, this traditional definition of \term{head} also makes sense:
    it corresponds to the core i.e. the center root of a construction,
    which, of course, cannot always be non-ambiguously defined in constructions like coordination.
    We also use the term \term{head} to refer to constituents that contain the core of a construction:
    thus the central nP or the central NumP in a DP is known as the \term{head} in the traditional grammar sense.
    This is the terminology in \citet{cgel}.

    This traditional usage of the term \term{head} appears in the generative literature as well:
    \citet[\citepage{120}]{paul2014new}, for instance, talks about the \term{head noun} of something he calls a DP.
    The two definitions of \term{head} overlap when there is grammaticalization:
    a so-called prepositional phrase can be a complement-taking adverb phrase
    in which the \term{head preposition} is a head in the traditional sense,
    or an analytic case phrase, where the \term{head preposition} is actually a case particle and a functional head.
    The former can be reanalyzed as the latter.
    On the other hand, some functional heads, like sentence final particles
    (see e.g. \prettyref{fig:grammatical.clause.high-level.topic.1}),
    are never recognized as heads in traditional grammars:
    they are instead called \term{markers} in grammars like \citet{cgel}.

    \item[(b)] Constituency and dependency relations.
    In mainstream generative syntax, constituency (or more abstractly, c-command relations) is the only primitive for structure building.
    Besides that, because of phase theory (or alternative theories aiming to explain the relevant phenomena),
    constituency is \term{cyclic} or ``layered'':
    thus in a clause, the vP is finished first, followed by TP and CP.
    This provides an alternative way to define constituents,
    in which the main verb (on the notion of what is a verb or more generally what is a word, see below),
    and the tense, aspect and modality markers, but \emph{not} arguments, are put into one unit,
    often called the verb phrase \citep[\citepage{39}]{quirk1985}.
    Still the hierarchical relations between the functional heads 
    (e.g. \category{tense} \textgt{}\category{perfect} \textgt{}\category{progressive})
    exist and need to be accounted for, so we see discussions on them in \citet[\citepages{79,121}]{quirk1985}.
    The flat-tree analysis in \citet[\citepage{39}]{quirk1985} can be adopted in our framework too,
    %The sentence below is not based on verb phrase in the BLT sense:
    %We can see this by comparing \prettyref{fig:vp.ex.1} and \prettyref{fig:vp.ex.1-alternative}.
    although this grammar will not heavily make use of the flat-tree analysis.
    
    Another issue is the relation between constituency and dependency.
    The two are basically two notations for the same thing \citep{boston2009dependency}.
    Here hierarchical relations can be represented by assigning ``closeness'' values to dependency arcs.
    Dependency analyses are particularly wieldy when movements are frequent.
    \citet[\citepage{55}]{cgel}, for instance, mentions \term{indirect complements},
    which are originally a part of the core argument structure of an adjective
    in a noun phrase and has to appear post-nominally.
    Its relation with the adjective that licenses it therefore is ideally reflected 
    by a dependency arc in a descriptive grammar,
    although a movement-based constituency analysis is of course possible.
    
    \item[(c)] Pre-defined grammatical constructions.
    In lexicalist schools of generative grammar,
    we have the X-bar theory, in which some heads are heads in the sense of (a).
    In lexicalist X-bar theory, the distinction between adjuncts, specifiers and complements
    are used to explain their differences in their syntactic behaviors.
    This distinction is absent in Cartography,
    as almost all things are specifiers, and the distinction is to be reinterpreted
    as the distinction between different types of specifiers.
    So the X-bar scheme can be seen as a pre-defined schema of grammatical constructions.
    
    We can further derive more grammatical structures 
    (like subject-predicate structures, predicator-object structures, coordination structures, and so on)
    by incorporating the vP-TP-CP and the nP-NumP-DP hierarchies into the X-bar scheme,
    which is exactly what is done in \citet{deng2010formal},
    which results in a descriptive formalism quite similar to that in \citet{cgel}:
    we replace the label TopP by a \term{form} label \term{topic-comment construction},
    and we replace SpecTopP by a \term{function} label \term{topic},
    and nodes in the tree diagram are labeled like ``sentence: topic-comment construction''
    or ``topic: noun phrase'';
    on the other hand, phonological realization of functional heads
    do not have the form-plus-function labels:
    we only label them according to the grammatical categories they mark,
    like ``evaluative particle [\category{delimitative}]'' (\prettyref{fig:grammatical.clause.high-level.topic.1}).%
    \footnote{
        \citet{culicover2004cambridge} notes that this eliminates the necessity of functional heads,
        and that when describing a single language (by extending \citet{cgel} into what he calls GODZILLA-CGEL),
        the parameters apply to concrete, content items, not abstract functional heads.
        The reason the function head analysis is preferred (or, alternatively, the function-form analysis is preferred) is 
        ``it is a description of the way in which the Language Faculty behaves\dots [i]t is also a description of the possible relationships among these expressions.''
        That's to say, if grammatical variances can be most easily captured by
        stipulating the existence or non-existence of certain functional heads
        and the morphological or phonological status of them
        (like whether C is ``strong'' enough to attract the main verb to it,
        or whether T is ``weak'' enough to lower to the main verb),
        then functional heads are preferred for theoretical linguistics.
        This however says nothing about what formalism to use 
        when describing a specific language (\prettyref{sec:grammatical.previous}).
    }
    We should be cautious that in hierarchies described by \citet{cinque1999adverbs},
    there can be too many form labels,
    and sometimes we have to conflate them into things like \form{extended verb phrase}
    (\prettyref{fig:vp.ex.1}).
    CompFP in [SpecFP [F CompFP]] may have a conventional \emph{function} label as well
    (like \form{comment} in topic-comment constructions);
    otherwise we just call it the \term{head} (see (a); see e.g. \prettyref{fig:grammatical.clause.high-level.topic.1} and \prettyref{fig:vp.ex.1}).

    In a surface-oriented tree diagram of an utterance,
    which roughly corresponds to the derivation tree after post-syntactic morphological operations,
    non-terminal nodes have gotten their labels like \term{sentence: topic-comment construction},
    and terminal nodes corresponding to functional heads have gotten labels like \category{delimitative}.
    Special attention needs to be paid to how to labeling content words,
    i.e. clusters of functional heads and roots gathered at one node of the derivation tree
    (this is related to morphological wordhood; \prettyref{sec:intro.theory.word}).
    In this grammar, we use terms \term{noun} or \term{verb}
    to refer to \emph{morphological words} that have the n or v categorizers,
    and use these terms to label content word terminal nodes (which always appear around either one functional head or a root).
    Of course, these terms conflate many different possible word structures
    (which sometimes have grammatical consequences; \citealt[\citepages{59-60}]{siddiqi2009syntax}),
    just like terms like \term{extended verb phrase} can mean a lot of things
    (\prettyref{sec:theory.gradience}).
    Constituency tree diagram based on the surface form is
    less wieldy when formatives in the content word
    originate both from syntax within the word and post-syntactic operations moving 
    other functional heads to the word,
    although it's definitely doable (\prettyref{fig:grammatical.clause.core-vp.verbal-complex.2}).
\end{itemize}

It can be verified that our framework does not strongly deviate from the so-called Basic Linguistic Theory
\citep{dixon2009basic},
i.e. the grammatical framework used in most descriptive grammars.
Our framework is also largely consistent with the framework in \citet{cgel},
except the fact that \citet{cgel} do not make an explicit lexical/functional distinction:
for instance, they first analyze all prepositions as if they are a content word class
in \citechap{7},
and then go on to discuss grammaticalized prepositions from \citepage{647}.
On the other hand, \citet{dixon2009basic}, despite its fierce attack on generative syntax
(\prettyref{sec:grammatical.previous}),
advocates for analyses that are perfectly consistent with the descriptive framework in this section,
where functional and content items are strictly separated
\citep[\citepage{49}]{dixon2009basic}.

\subsection{Wordhood}\label{sec:intro.theory.word}

Because our framework is strongly inspired by Distributed Morphology,
the notion of \term{words} do not have a primitive status
in describing syntax in our descriptive framework.
That's to say, all grammatical constructions and processes
can in principles be described in terms of roots, constituency/dependency,
without using any concept that implies an absolute word/phrase distinction.

We may still want to define wordhood syntactically,
based on language-specific criteria.
We should note difficulties of certain degrees are bond to appear.
If wordhood is defined as a small constituency, where \term{constituency} means what the word means in mainstream generative syntax,
then inflectional endings, and even voice markers,
are \emph{not} in the same word with the root
(or otherwise the whole predicator-object phrase is to be recognized as a word),
and if it's defined as a flat-tree constituent (see (b) in \prettyref{sec:theory.descriptive}),
then how shouldn't \form{has been working} is a word?

\concept{Morphological words} and \concept{phonological words},
on the other hand, are well-defined, as the fact is cross-linguistically,
roots and grammatical formatives in the abstract syntax are bundled into groups for phonological realization.
Formatives brought together by \concept{post-syntactic} operations form one morphological word:
hence a verb word in English is a complex containing
the verb root plus possible derivation affixes and also the tense suffix.
It is also possible that first a bunch of formatives are gathered together,
and later some other formatives join them:
this is known as cliticization.
Phonological words are to be defined according to prosody
or domains of phonological rules.

It is less wieldy to use constituency trees based on the surface form to represent 
the scopes of grammatical categories represented by formatives
in a morphological word.
In \prettyref{sec:grammatical.clause.core-vp}, (\ref{ex:grammatical.clause.core-vp.verbal-complex.1}),
for instance, the verbal complement 完 has its scope over 做作业,
and the aspectual suffix 了 has its scope over 做完作业.
We can represent this fact in the way of \prettyref{fig:grammatical.clause.core-vp.verbal-complex.1.1},
which, despite being largely surface-oriented, still involves invisible nodes in the syntactic tree. 

Grammar is naturally divided into abstract syntax i.e. syntax proper and post-syntactic operations.
We may want to call this distinction the syntax/morphology distinction,
although syntax/morphology distinction in traditional grammar means something slightly different and
this may cause confusion: some derivational morphology like compounding can be described using the same framework we use for prototypical phrasal syntax,
as they are both mostly about the syntax proper,
while what is commonly placed under the category of syntax,
like verb fronting in question formation in English,
can be explained by a V-to-C head movement, which can be captured as a post-syntactic operation 
\citep[\citepage{24}]{siddiqi2009syntax}.
Unfortunately the term \term{realizational morphology},
which looks like a good name to emphasize the post-syntactic part of morphology, 
has been used to refer to the Word-and-Paradigm theory morphology,
which is beyond the scope of this section.

One practical problem is whether we should talk about roots explicitly in 
constituency trees for derivational processes shown in this grammar.
This is also related to theoretical issues,
like whether it is necessary to allow root-root Merge without any functional heads \citep{zhang2007root},
and whether in \form{grammaticality judgment},
given that the nominalizer \form{-ment} has scope over the whole form
(\translate{the action or result of judging grammaticality}),
the form can be represented
as something like \form{[[judge [[[grammatic] -al]_{\text{A}} -ity]_{\text{N}}]-ment]_{\text{N}}},
in which \form{judge} at first appears as a root,
which merges with \form{grammaticality}.
The second option, even if true, does not necessitate appearance of the label \term{root},
because the categorizer n eventually is merged with \form{judge-}
and in a surface-oriented constituency tree, which roughly corresponds to the derivational tree after post-syntactic operations, we can say that \form{grammaticality judgment} is a noun-noun compound
(\prettyref{box:dephrasal-compound}),
although the second noun is formed \emph{after} the constituent structure of the whole form is formed.
A clearer example demonstrating this is \form{hard worker},
which, from semantic clues, likely has a structure of \form{[work hard]-er},
as \form{worker} in this form does not have its usual meaning.
On the other hand, even if root-root merging is not truly permitted,
some sort of functional projections that take two roots -- none of which is categorized,
unlike the case of \form{grammaticality judgment} in which the first branch is a nP --
likely exist cross-linguistically, which gives rise to English \form{sick-bed}%
\footnote{
    There is no other construction in which the adjective \form{sick} appears to mean \form{sickness},
    which means it's highly likely that \form{sick} in this form is licensed as a root,
    not a categorized adjective.
}
or \form{spectr-o-scopy} \citep{di2005decomposing,scher2014unifying}.%
\footnote{
    \citet{di2005decomposing} is based on a lexicalist formalism,
    but transferring the analysis to our root-and-functional head framework is easy
    \citep{scher2014unifying}.
}
In constituency trees for these forms, at least \form{spectr-} or \form{sick-} should be labeled as a root.

\subsection{Lexicalization}\label{sec:intro.theory.lexicon}

We understand \concept{lexicalization} in two aspects:
the morphophonological and semantic.
Note that subcategorization can be seen as a consequence of morphophonological lexicalization:
saying a verb is transitive is equivalent to say it can only be phonologically realized
in the presence of a \category{transitivity} feature \citep{siddiqi2009syntax}.
An arbitrarily large grammatical unit,
which is phonologically realized in a combinatorial way, can have an idiomized meaning:
consider \form{kick the bucket}.
A grammatical word \emph{without} an idiomized meaning
can be phonologically realized as a whole:
thus \form{feet} is just \wordroot{foot}-\category{nominal}-\category{plural},
but it has an irregular form.
The two are theoretically independent to each other,
although idiomization often leads to \emph{syntactic} reanalysis,
often towards a simpler direction (and thus can be called syntactic fossilization). 
Note that being semantically idiomized has syntactic consequences,
like reduced acceptability of certain movement operations.

\subsection{Derivation, inflection, parts of speech}\label{sec:intro.theory.pos}

All other concepts, like the derivation/inflection distinction
or the argument/adjunct distinction, are in theory secondary,
and in this grammar we do not attempt to do demarcations of this point,
and merely focus on the relevant grammatical phenomena related to these distinctions.

The two types of lexicalization often converges into the concept of \emph{lexemes}.
Consider the structure [\dots \category{do} [\dots \category{transitive} \wordroot{hit}]_{\text{TransP}}]_{\text{vP}}.
This treelet has an established meaning \translate{to hit sth.},
and post-syntactic morphological operations gather the three formatives,
\category{do}, \category{transitive} and \wordroot{hit}, together into one morphological word,
and it gets realized as \form{hit} because of a corresponding lexical entry.
The tense and aspect markers are phonological implemented in a regular way,
So we say that there is a transitive verb \concept{lexeme} \form{hit} stored in the lexicon,
and markers tense, person, etc. are given as a \concept{paradigm} of it.%
\footnote{
    Note that here we assume a layered morphology of inflection,
    while template morphology which does not transparently show the hierarchy of functional heads
    is also possible, which however is not beyond the descriptive capacity of our framework
    \citep{bye2020morpheme}.
}

The concept of lexeme however implies that we have a clear derivation/inflection distinction, which is problematic.
For instance, we may want to draw a line between derivation and inflection structurally
according to the two definitions of syntactic wordhood above (\prettyref{sec:intro.theory.word}),
but this definition excludes any valency alternation from derivation.
Or we may want to say that derivation is generally more lexicalized,
but lexicalization is not a single parameter (\prettyref{sec:intro.theory.lexicon}),
and many languages have productive derivational devices.
Or we may want to say that derivation is correlated with layered morphology
while inflection is correlated with template morphology,
but layered inflection (consider Japanese)
and template derivation all exist.
Eventually, we find that the \concept{derivation/inflection distinction}
just reflects how close a grammatical category is to the root in one construction
(template morphology is more likely to develop for
relatively external categories in the TP projections),
but how internal a grammatical category should be to be considered derivational is not clear.
Our opinion is that such a distinction is generally not possible to make cross-linguistically.
Hence the term \term{lexeme} does not always mean the same thing cross-linguistically.

A \concept{part of speech} refers to a class of lexicalized items in the lexicon with shared features.
When a derivation/inflection distinction is defined (per language-specific criteria),
the lexicalized items in question are lexemes.
Thus \wordroot{hit}-\category{transitive}-\category{do}
and \wordroot{rise}-\category{become}
are both assigned the part of speech tag \term{verb},
because they prototypically appear at the center of clauses
and can both receive endings \form{-s} and \form{-ed} and \form{-ing}.
Whether we have a \category{transitive} feature or not in the phonologically lexicalized entry
dictates whether the clause is transitive,
which gives rise to \term{subcategorization}, or in other words different \term{verb frames}
(and also valency in other constructions).
Functional formatives, in principle, do not need part of speech tags:
in practice we often given them a tag so writing a dictionary becomes easier.
The lexicon of different languages are organized in different ways,
and therefore part of speech division expected has strong cross-linguistic variances.
We also note that it's perfectly fine for parts of speech to have blurred boundaries,
as they are not primitives in the structure of the lexicon.
This is true for both lexical part of speech categories and
so-called categories of functional formatives.%
\footnote{
    This is one criticism \citet{culicover2004cambridge} raises against \citet{cgel}.
    He notices that \citet{cgel} sometimes uses some evidence to 
    justify categorization of one (sometimes functional) item while ignoring other evidence,
    and insightful questions, ``what follows from the categorization?''
}

Some languages, like Chinese, do not have rich morphology,
and it leaves beginners an impression that these languages have no parts of speech in their lexicons.
This, despite being theoretically possible, is highly implausible,
as this implies that the meanings of root-affix tuples are also not deterministic:
\wordroot{eat}-\category{nominal}, for instance,
would be understood as the action of eating or something to eat if this were true,
depending on the context.
Mandarin definitely \emph{has} part of speech divisions
as this is not the case in Mandarin.

\subsection{Arguments and adjuncts}

The standard of being an \concept{argument} can be defined according to 
criteria listed in \prettyref{sec:grammatical.clause.peripheral}.
This distinction is also impossible to define in pure structural terms,
and shows strong cross-linguistic variances.
We use the term \concept{complementation} to refer to something taking an argument,
and \concept{modification} to refer to something taking an adjunct.
As the boundary between arguments and adjuncts is blurred,
the distinction between complementation and modification is also blurred.

\subsection{Relation with previous works}\label{sec:grammatical.previous}

The purpose of theoretical linguistics is to see the complexity class of human languages,
while the purpose of descriptive linguistics is easier descriptions
-- at the expanse of having more ``primitive'' concepts
which actually do not allow more possible languages being described.
This grammar aims to strike a balance between readability and theoretical cross-linguistic comparisons.

We choose this framework for several reasons.
First, Mandarin has a relatively mature structuralist description tradition,
which is quite similar to that in \citet{cgel},
and that it can be seamlessly incorporated into modern generative syntax 
has long been noticed (e.g. \citealt{deng2010formal}).
Second, many descriptive linguists align themselves with the functionalist approaches to grammar,
but this is more because of problematic \emph{practices} in generative schools,
like relying on often unstable acceptability judgments
or overly focusing on complicated clauses.
The actual \emph{theory} they use however often depart from contemporary functionalist theories.
The term \term{construction}, for instance,
appears frequently in the descriptive and typological literature
(it will frequently appear in this grammar, too),
but it is much less frequently used in the sense of various schools of Construction Grammar:
in typical language description works, a construction is often still analyzed 
in a decompositional and combinatorial way,
Therefore, it makes sense to see how far structuralist analyses can go
being informed by modern generative syntax.

\subsection{Gradience}\label{sec:theory.gradience}

\citet{quirk1985} frequently mentions gradience in grammar.
We take grammatical gradience to be a heterogeneous concept.

Sometimes gradience is purely due to defective analysis:
if two constructions are labeled in the same way,
and one can participate in a certain syntactic process and the other can't,
one may argue that the acceptability of that process is gradient.
This however is purely an artifact of the grammarian being not careful enough,
and means nothing besides the necessity of having more fine grained labels
(see e.g. the discussion on the meaning of the term \term{extended VP} around \prettyref{fig:vp.ex.1}).

A specific case of this type of ``gradience'' is what counts as a clause.
A TP is more exposed to its syntactic environment than a CP.
This doesn't mean that the concept of \term{clause} is gradient
in the same sense that the real number axis is gradient:
the ``scale'' of clause-ness may actually be discrete.

Another type of gradience, like frequency effects in fossilization, is indeed continuous.
This however can be analyzed as existence of two or more competing grammars in the mind of a native speaker,
each of which are assigned a probability.
The continuity observed in this kind of gradient phenomena
does not invalidate discreteness of syntactic structures.%
\footnote{
    Readers with knowledge on condensed matter physics may compare Ising model with Heisenberg model
    to see what's going on here:
    continuous output can arise simply because of continuous weights assigned to each possible configuration of a system,
    and not continuously deformable configurations of a system.
}

Yet another type of gradience comes from dialectal differences:
it's possible that the internal knowledge of grammar in each speaker's mind is discrete,
but speakers differ with each other in subtle ways,
so a frequency statistics in the population shows gradient effects.

We therefore avoid using the term \term{gradience} without further specification.
We say things like ``multiple analyses are possible'', or ``speakers differ in their acceptance of the example'' instead when necessary.

}

\section{Previous studies}

\chapter{Phonology and the writing system}

\section{Vowels}

\section{Tones}

\subsection{Tone sandhi}\label{sec:phonology.tone-sandhi}

\subsubsection{Successive third tone}\label{sec:phonology.tone-sandhi.third}

The most salient sandhi rule is that 
when two syllables with the third tone appears together, 
the first's tone is reassigned to the second tone in 
fluent utterances. 

保管好: first and second syllables undergo tone sandhi together 

\section{Prosody}\label{sec:prosody-structure}

Phonological words exist in Mandarin 
and they are mostly defined by the prosody structure.
In the rest of this note,
the term \term{prosodic word} and \term{phonological word}
will be used interchangeably, 
although tone sandhi can penetrate 
the boundary between prosodic words
(\prettyref{sec:phonology.tone-sandhi.third}).
The prosody structure is about how stress is assigned to phonological constituents.
Assigning a prosodic structure is like condensation and clustering:
something is merged with something adjacent,
and the result is merged with something adjacent else.
When two phonological constituents are merged together,
one of them is considered heavier than the other.
If heaviness is to have a simple relation with the length of a phonological constituent,
then usually the more a phonological constituent is,
the heavier it is.
This is consistent with the condensation picture of prosodic segmentation.
Suppose a prosodic constituent attracts a syllable and merges with it.
The latter is not an independent phonological constituent
and cannot be heavy,
so the former is the heavier one and the latter is the lighter one in the larger prosodic constituent.

\subsection{Prosodic word}

The smallest unit of prosody structure 
is a prosodic word.
The simplest prosodic word is the disyllabic foot, 
which contains two adjacent syllables in the case of Chinese.
(It can be made by two moras in other languages.)
One is assigned stress and is therefore heavier than the other.
Trisyllabic prosodic words also exist in Chinese,
though they are highly limited.
Most of which are borrowed words (e.g. 加拿大 \translate{Canada})
or words formed by coordinating three morphemes (e.g. 数理化 \translate{math, physics, and chemistry}).
They can also be regarded as foots \citep[\citesec{2.2}]{feng2000}.

\subsubsection{Prosodic segmentation ignoring morphosyntax}\label{sec:prosody.word.ignorance}

Longer morphosyntactic units are 
inevitably broke into smaller disyllabic or trisyllabic prosodic words
in their prosodic structures,
often regardless of their morphosyntactic structure:
加利福尼亚 may be segmented into 加利\textbar 福尼亚, 
although the word contains only one morpheme.
In 副总经理,
we have two prosodic words,
副总 and 经理,
while the morphosyntactic structure of the word is [[副] [总 [经理]]]
(\prettyref{sec:pos.noun.adj-modify}).
This is similar to the case in English and Latin poems,
where the prosody arrangement of sentences does not have to respect word boundaries:
\form{arma vi\textbar rumque ca\textbar no}.
It's however also possible that 
a prosodic word has morphosyntactic significance
(\prettyref{sec:pos.word.phonological}).

Prosody is able to see the constituency structure
and prosodic constraints are important in Mandarin grammar.
Some prosodic rules pertaining to the constituency tree 
guide and limit the assignment of relative heaviness and lightness.
In Chinese, prosodic segmentation is done strictly left-to-right 
in each \ac{np},
and then the \ac{np}s together with verbal constituents 
are used as the input of prosodic segmentation of clauses.
Certain forms are therefore ruled out
(\prettyref{sec:vp.prosody}), 
not by morphosyntactic reasons but for prosodic reasons.

\section{Chinese characters}\label{sec:chinese-character}

The preferred writing system of Mandarin is the Chinese character system.
Except some characters made in early modern ages,
like 兛 \translate{kilogram} or 砼 \translate{concrete (lit. human-labor stone)},
a Chinese character corresponds to a syllable.
However, Chinese characters don't just represent the sound.
Putting some quirky cases aside,
Chinese characters are often good indicators of morphemes
(\prettyref{sec:pos.morpheme.primitive}).
There are, for example, at least seven morphemes sounding \form{xi\={a}n},
and there happens to be seven Chinese characters corresponding to each of them:
仙, 先, 籼, 掀, 锨, 鲜, and 纤.

Like all writing systems, 
Chinese characters do not completely faithfully represent 
the underlying linguistic structure.
Some characters do not mean anything -- 
they are simply the designated characters representing syllables 
in certain polysyllabic morphemes.
The character 萄 as in 葡萄, for example, 
means nothing more than the syllable \form{t\'{a}o},
but it only appears in the morpheme 葡萄 and 葡萄牙 \translate{Portuguese}.
The same is for the character 葡.
Some characters have regular morpheme meanings
but also have merely phonetic meaning in certain words.
The character 登 in 摩登 regularly means \translate{climb},
but in the word 摩登, only its phonetic value \form{d\={e}ng} is preserved.
Certain morphemes can be denoted by more than one character.
The \ac{sfp} \form{ba} can be written as 吧 or 罢,
the latter hinting its etymology but is now rarely used.
Certain characters denote more than one morpheme.
The character 会 may mean \translate{conference} or \translate{be able to do}. 

Thus, Chinese characters provide clues on what is a morpheme,
but they are not decisive \citep[1.1.4]{zhudexigrammar}.


\chapter{Grammatical overview}

\section{Morphological typology}

A grammar of a language can be summarized into the abstract syntax of clauses (\prettyref{sec:grammatical.clause}) and \acp{np} (\prettyref{sec:grammatical.np}),
how syntactic objects are morphophonologically realized,
and the interaction between roots and syntactic environments
or in other words parts of speech categorization
(\prettyref{sec:grammatical.lexicon}).
We start our introduction by talking about the morphological realizational part of Mandarin grammar,
i.e. morphological typology.

\subsection{Wordhood}\label{sec:grammatical.wordhood}

A frequent claim is that Mandarin lacks the word/phrase distinction.
Syntactically, among world languages,
``lexical grammar'' and prototypical syntax aren't truly radically different,
and this can be seen even in English (\prettyref{box:english-word-phrase}),
so this claim is trivially true, without providing any new insights.
This grammar does not assume existence of a unit called ``word'' a priori in constituency relations in syntax, either.%
\footnote{
    For theoretical discussions, see \prettyref{sec:intro.theory.word} and related sections.
}

TODO: discuss how subcategorization frames are always applied to small objects.
Below: wordhood by lexicalization

If we try to define wordhood according to a clearly defined standard, like verb valency --
what selects arguments in an unpredictable way is considered a word,
and anything larger than it is a phrase --
then a word/phrase distinction \emph{can} be done in Mandarin:
we can demonstrate that constructions in \prettyref{sec:grammatical.clause.core-vp.derivation},
\prettyref{sec:grammatical.np.derivation} are words,
while constructions in \prettyref{sec:grammatical.np.adjectives}, \prettyref{sec:grammatical.np.complementation}
are phrases, even though they look kind of like compounding intuitively.

\begin{theorybox}{The blurry boundary between words and phrases in English}{english-word-phrase}
    Regular syntactic processes happen to both phrases and what are common considered words:
    a quick google search reveals that \form{tooth- and back-ache}
    or \form{pre- and post-revolutionary France} are both considered acceptable by many,
    On the other hand, irregularity and fossilization of historical terms
    can be seen in idiomized phrases as well, as in \form{till death do us part}.
\end{theorybox}

The usual tests for morphological and phonological wordhood can be run on Mandarin as well,
and we again get unambiguous results.
What is described in \prettyref{sec:grammatical.clause.core-vp.verbal-complex},
for instance, is clearly a morphological word,
and phonological wordhood can be defined in terms of prosody (\prettyref{sec:prosody.word.ignorance}).

What is truly non-trivial is whether and how these three definitions of wordhood overlap.
The main inconsistency between various morphosyntactic wordhood criteria appears for verbs
(\prettyref{sec:verbal-complex.wordhood}).
On the other hand, prosodic wordhood can often differ from grammatical wordhood 
(\prettyref{sec:verbal-complex.v-c-a}, \ref{ex:verbal-complex.linear-order.reorder.1.1}).

We emphasize that our description of syntax does \emph{not} depend on any wordhood definition:
discussions in this section are mainly for cross-linguistic comparison.

\subsection{Inflection and derivation}

The common wisdom that Mandarin lacks morphology is automatically refuted
by \prettyref{sec:grammatical.wordhood}
as a grammatical containing multiple pieces has to have morphology.
Having established that in Mandarin, roots and grammatical formatives are organized 
in ways comparable to other languages,
with well-defined grammatical wordhood,
we can continue to classify the observed multi-piece morphological words.

We find that Mandarin lacks prototypical inflectional morphology 
but still has something that may be analyzed as verbal inflection (\prettyref{sec:grammatical.clause.core-vp.verbal-complex}),
and it has a decent amount of derivational morphology
(\prettyref{sec:grammatical.clause.core-vp.derivation},
\prettyref{sec:grammatical.np.derivation}).

\subsection{Morphological devices}

Compounding is the most frequent morphological device,
and partly due to influences of European languages, partly due to grammaticalization,
affixation is also frequently seen.
Plus, reduplication plays an important role in Mandarin verbal and adjectival derivation.



\section{Clauses}\label{sec:grammatical.clause}

\subsection{Sentential categories and the nucleus clause}\label{sec:grammatical.clause.high-level}

A clause (\prettyref{box:clause-sentence-def}) can be divided into several clauses 
linked by \concept{clause linking} constructions,
including \concept{coordination} and \concept{subordination}.
(Note that coordination can also happen inside the nucleus clause;
\prettyref{sec:grammatical.clause.subject.clause}.)
Mandarin has ample information marking phenomena,
and thus a clause can be divided into
one or more \concept{topics}, if any, and a \concept{comment},
the latter being the \concept{nucleus clause} (\prettyref{sec:grammatical.clause.subject})
plus possible \concept{sentence final particles}.

\begin{theorybox}{Terminology: \term{clause}, \term{sentence}, and the like}{clause-sentence-def}
    \citet{cgel} uses the term \term{sentence} 
    to refer to a natural unit in dialogue,
    which I refer to as a \term{utterance}.
    The term \concept{sentence} here in this grammar refers to 
    a clause that qualifies as an utterance. 

    Some people, like \citet[\citepage{140}]{deng2010formal}
    as well as \citet{dixon2009basic},
    use the term \term{clause} for subject-predict constructions 
    with no speech force marking.
    (\citet{deng2010formal} uses 句子 as the Mandarin counterpart of \term{sentence}
    and 小句 as the counterpart of \term{clause}.)
    In this way, \acl{sfp}s strictly shouldn't be
    regarded as a part of the clause, 
    and they may be discussed together with 
    other higher level constructions like clause linking. 
    This notion of clause correctly highlights the hierarchy in clausal structures.
    The problem with this terminology however is that in traditional grammars,
    the term \term{clause} does refer to units that have \ac{sfp}s.
    
    This note therefore refers to all units larger than the 
    subject-predicate construction as clauses, 
    which may or may not be sentence.
    The subject-predicate construction is instead named the \emph{nucleus} clause.
    The internal complexity of a clause 
    is still relevant for example in clause combining.
\end{theorybox}

These high-level devices -- topic-comment construction, sentence final particles, and clause linking
-- can coexist: in (\ref{ex:grammatical.clause.high-level.topic.1}),
diagrammed in \prettyref{fig:grammatical.clause.high-level.topic.1}
topicalization and a sentence final particle appear together.
Note that here we assume that the scope of topic is over the scope of the evaluative particle.
The relative scopes have subtle semantic effects and \citet{pan2015mandarin} notes that in Mandarin,
no preference is made among these subtle semantic differences,
meaning that it is also possible that the scope of the sentence final particle
being larger than that of the topic.
A more telling example with clearly attestable semantic differences is (\ref{ex:grammatical.clause.high-level.question.1}).

\begin{exe}
    \ex\label{ex:grammatical.clause.high-level.topic.1}
    张三,他就是个王八蛋而已!
    \gll {} [\pinyin{Zhang1} \pinyin{San1}]_{\text{topic},i}, [\pinyin{ta1}_i \pinyin{jiu4} \pinyin{shi4} \pinyin{ge} \pinyin{wang2ba1dan4}]_{\text{nucleus clause (\ref{ex:grammatical.clause.subject.copular.1})}} \pinyin{er2yi3}! \\
    {} \category{name} {} 3 just be \category{cls} turtle-egg \category{sfp} \\
    \translate{Zhang San is just a son of a bitch, and that's it!}
\end{exe}

\begin{figure}[H]
    {
        \centering
        \small
        \begin{tikzpicture}[x=0.75pt,y=0.75pt,yscale=-1,xscale=1]
    %uncomment if require: \path (0,300); %set diagram left start at 0, and has height of 300
    
    %Straight Lines [id:da2012956362095869] 
    \draw    (257,260) -- (448,260) ;
    %Straight Lines [id:da24303093387997055] 
    \draw    (355,227.31) -- (448,260) ;
    %Straight Lines [id:da14587666508930852] 
    \draw    (355,227.31) -- (257,260) ;
    %Straight Lines [id:da25151294497023513] 
    \draw    (532,227.31) -- (532,260.31) ;
    %Straight Lines [id:da5251263367750888] 
    \draw    (435,158.31) -- (532,183.31) ;
    %Straight Lines [id:da14069077735379665] 
    \draw    (359,183.31) -- (435,158.31) ;
    %Straight Lines [id:da20288315197393314] 
    \draw    (177,158.31) -- (192,260.31) ;
    %Straight Lines [id:da3486544796360199] 
    \draw    (154,260.31) -- (192,260.31) ;
    %Straight Lines [id:da02126425119297326] 
    \draw    (177,158.31) -- (154,260.31) ;
    %Straight Lines [id:da996798510579373] 
    \draw    (177,117.31) -- (311,93.31) ;
    %Straight Lines [id:da519603489229032] 
    \draw    (438,117.31) -- (311,93.31) ;
    
    % Text Node
    \draw (352.5,263) node [anchor=north] [inner sep=0.75pt]   [align=left] {他就是个王八蛋};
    % Text Node
    \draw (355,223.31) node [anchor=south] [inner sep=0.75pt]   [align=left] {\begin{minipage}[lt]{69.54pt}\setlength\topsep{0pt}
    \begin{center}
    factual branch:\\nucleus clause
    \end{center}
    
    \end{minipage}};
    % Text Node
    \draw (532,263.31) node [anchor=north] [inner sep=0.75pt]   [align=left] {罢了};
    % Text Node
    \draw (532,223.31) node [anchor=south] [inner sep=0.75pt]   [align=left] {\begin{minipage}[lt]{87.07pt}\setlength\topsep{0pt}
    \begin{center}
    evaluative particle\\{[\category{diminutive}]}
    \end{center}
    
    \end{minipage}};
    % Text Node
    \draw (435,154.31) node [anchor=south] [inner sep=0.75pt]   [align=left] {\begin{minipage}[lt]{75.28pt}\setlength\topsep{0pt}
    \begin{center}
    comment:\\sentential clause
    \end{center}
    
    \end{minipage}};
    % Text Node
    \draw (176,156.31) node [anchor=south] [inner sep=0.75pt]   [align=left] {\begin{minipage}[lt]{27.63pt}\setlength\topsep{0pt}
    \begin{center}
    topic:\\NP
    \end{center}
    
    \end{minipage}};
    % Text Node
    \draw (173,263.31) node [anchor=north] [inner sep=0.75pt]   [align=left] {张三};
    % Text Node
    \draw (311,90.31) node [anchor=south] [inner sep=0.75pt]   [align=left] {\begin{minipage}[lt]{127.06pt}\setlength\topsep{0pt}
    \begin{center}
    sentence:\\topic-comment construction
    \end{center}
    
    \end{minipage}};
    
    
    \end{tikzpicture}
    
    }
    \caption{Tree diagram of one possible structure of (\ref{ex:grammatical.clause.high-level.topic.1})}
    \label{fig:grammatical.clause.high-level.topic.1}
\end{figure}

Similarly, topicalization and clause linking can happen successively as well
(\ref{ex:grammatical.clause.high-level.topic-and-coordination.1}),
diagrammed in \prettyref{fig:grammatical.clause.high-level.topic-and-coordination.1},
shows an example of topicalization after subordination.
It is also possible to link two topic-comment clauses.

\begin{exe}
    \ex\label{ex:grammatical.clause.high-level.topic-and-coordination.1}
    我幸亏昨天没来,否则就被困住了。
    \gll{} [\pinyin{Wo3}]_{\text{topic},i} [\pinyin{xing4kui1} ---_i \pinyin{zuo2tian1} \pinyin{mei2} \pinyin{lai2}]_{\text{nucleus clause}}, [\pinyin{fou3ze2} ---_i \pinyin{jiu3} \pinyin{bei4} \pinyin{kun4} \pinyin{zhu4} \pinyin{le}]_{\text{nucleus clause}} \\
    {} 2 fortunately {} yesterday \category{neg} come or.otherwise {} then \category{bei} trap \category{v2} \category{asp} \\
    \translate{Fortunately, I didn't come yesterday, or otherwise I would have been trapped.}
\end{exe}

\begin{figure}[H]
    {
        \centering
        \small
        \begin{tikzpicture}[x=0.75pt,y=0.75pt,yscale=-0.8,xscale=0.8]
    %uncomment if require: \path (0,517); %set diagram left start at 0, and has height of 517
    
    %Straight Lines [id:da2627393797771076] 
    \draw    (628,456.31) -- (799,456.31) ;
    %Straight Lines [id:da5606219139716366] 
    \draw    (712,418.31) -- (799,456.31) ;
    %Straight Lines [id:da424518697731333] 
    \draw    (712,418.31) -- (628,456.31) ;
    %Straight Lines [id:da12245703561211851] 
    \draw    (528,418.31) -- (528,457.31) ;
    %Straight Lines [id:da8265933785037657] 
    \draw    (712,359.31) -- (616,313.31) ;
    %Straight Lines [id:da2646279688497606] 
    \draw    (528,359.31) -- (616,313.31) ;
    %Straight Lines [id:da8784074024110043] 
    \draw    (237,456.31) -- (428,456.31) ;
    %Straight Lines [id:da13759720547882903] 
    \draw    (334,313.31) -- (428,456.31) ;
    %Straight Lines [id:da3565761561879872] 
    \draw    (334,313.31) -- (237,456.31) ;
    %Straight Lines [id:da46724975619438913] 
    \draw    (468,224.31) -- (334,268.31) ;
    %Straight Lines [id:da606218866385489] 
    \draw    (468,224.31) -- (618,268.31) ;
    %Straight Lines [id:da17514038586144876] 
    \draw    (135,225.31) -- (135,457.31) ;
    %Straight Lines [id:da9654714437988359] 
    \draw    (309,136.31) -- (138,180.31) ;
    %Straight Lines [id:da6202905359610531] 
    \draw    (309,136.31) -- (471,180.31) ;
    
    % Text Node
    \draw (135,460.31) node [anchor=north] [inner sep=0.75pt]   [align=left] {我_i};
    % Text Node
    \draw (332.5,459.31) node [anchor=north] [inner sep=0.75pt]   [align=left] {---_i 幸亏昨天没来};
    % Text Node
    \draw (713.5,459.31) node [anchor=north] [inner sep=0.75pt]   [align=left] {---_i 就被困住了};
    % Text Node
    \draw (528,460.31) node [anchor=north] [inner sep=0.75pt]   [align=left] {否则};
    % Text Node
    \draw (528,407.5) node [anchor=south] [inner sep=0.75pt]   [align=left] {\begin{minipage}[lt]{100pt}\setlength\topsep{0pt}
    \begin{center}
    conjunction \\ {[\category{counterfactual}]}
    \end{center}
    
    \end{minipage}};
    % Text Node
    \draw (707,407.5) node [anchor=south] [inner sep=0.75pt]   [align=left] {\begin{minipage}[lt]{99.04pt}\setlength\topsep{0pt}
    \begin{center}
    counterfactual clause:\\nucleus clause
    \end{center}
    
    \end{minipage}};
    % Text Node
    \draw (616,310.81) node [anchor=south] [inner sep=0.75pt]   [align=left] {\begin{minipage}[lt]{102.68pt}\setlength\topsep{0pt}
    \begin{center}
    counterfactual branch:\\nucleus clause
    \end{center}
    
    \end{minipage}};
    % Text Node
    \draw (341,310.81) node [anchor=south] [inner sep=0.75pt]   [align=left] {\begin{minipage}[lt]{69.54pt}\setlength\topsep{0pt}
    \begin{center}
    factual branch:\\nucleus clause
    \end{center}
    
    \end{minipage}};
    % Text Node
    \draw (469,221.81) node [anchor=south] [inner sep=0.75pt]   [align=left] {\begin{minipage}[lt]{131.05pt}\setlength\topsep{0pt}
    \begin{center}
    comment:\\counterfactual subordination
    \end{center}
    
    \end{minipage}};
    % Text Node
    \draw (135,221.81) node [anchor=south] [inner sep=0.75pt]   [align=left] {\begin{minipage}[lt]{39.56pt}\setlength\topsep{0pt}
    \begin{center}
    topic:\\pronoun
    \end{center}
    
    \end{minipage}};
    % Text Node
    \draw (309,133.31) node [anchor=south] [inner sep=0.75pt]   [align=left] {\begin{minipage}[lt]{127.06pt}\setlength\topsep{0pt}
    \begin{center}
    sentence:\\topic-comment construction
    \end{center}
    
    \end{minipage}};
    
    
    \end{tikzpicture}
    
    }
    \caption{Tree diagram of (\ref{ex:grammatical.clause.high-level.topic-and-coordination.1})}
    \label{fig:grammatical.clause.high-level.topic-and-coordination.1}
\end{figure}

It is also possible to have multiple sentence final particles.
Their relative orders reflect their scopes (\prettyref{sec:sfp.all}).
(\ref{ex:grammatical.clause.high-level.question.1}) is a modification of (\ref{ex:grammatical.clause.high-level.topic.1}),
in which the particle 吗 marks the interrogative \concept{speech act} (known as speech \term{force} in some works),
which appears to the right of the delimitative particle 而已 and has scope over it.
The relative scope of topicalization and the two particles is not specified,
and this sentence can be translated as \translate{Is it the case that as for Zhangsan, he stops at being a son of a bitch?}
or \translate{As for Zhangsan, does he stop at being a son of bitch?}
The first translation and the corresponding structure (\prettyref{fig:grammatical.clause.high-level.question.1.1}) is more likely to be a genuine question.
The second translation and the corresponding structure (\prettyref{fig:grammatical.clause.high-level.question.1.2}) is more likely to be a rhetoric question
(implied meaning: he's possibly even worse than that).

\begin{exe}
    \ex\label{ex:grammatical.clause.high-level.question.1} 张三,他就是个王八蛋而已吗
    \gll {} [\pinyin{Zhang1san1}]_{\text{topic},i}, [\pinyin{ta1}_i \pinyin{jiu4} \pinyin{shi4} \pinyin{ge} \pinyin{wang2ba1dan4}]_{\text{nucleus clause (\ref{ex:grammatical.clause.subject.copular.1})}} \pinyin{er2yi3} \pinyin{ma1}? \\
    {} \category{name} 3 just be \category{cls} turtle-egg \category{sfp} \category{sfp} \\
\end{exe}

\begin{figure}[H]
    {
        \centering
        \small
        \begin{tikzpicture}[x=0.75pt,y=0.75pt,yscale=-0.8,xscale=0.8]
    %uncomment if require: \path (0,348); %set diagram left start at 0, and has height of 348
    
    %Straight Lines [id:da9428646144627109] 
    \draw    (198,314.69) -- (389,314.69) ;
    %Straight Lines [id:da2762997173362961] 
    \draw    (296,282) -- (389,314.69) ;
    %Straight Lines [id:da40545842803887666] 
    \draw    (296,282) -- (198,314.69) ;
    %Straight Lines [id:da10223987482458685] 
    \draw    (473,282) -- (473,315) ;
    %Straight Lines [id:da6137808415823605] 
    \draw    (376,213) -- (473,238) ;
    %Straight Lines [id:da3439645613590703] 
    \draw    (300,238) -- (376,213) ;
    %Straight Lines [id:da22121071701908346] 
    \draw    (153,213) -- (168,315) ;
    %Straight Lines [id:da5866979316900394] 
    \draw    (130,315) -- (168,315) ;
    %Straight Lines [id:da30595968699275544] 
    \draw    (153,213) -- (130,315) ;
    %Straight Lines [id:da30126912402396855] 
    \draw    (149,172) -- (266.33,141.6) ;
    %Straight Lines [id:da2364930211450167] 
    \draw    (379,172) -- (266.33,141.6) ;
    %Straight Lines [id:da8459157641594554] 
    \draw    (571,141.6) -- (571,315) ;
    %Straight Lines [id:da6536616587089727] 
    \draw    (267,94) -- (410.33,62.6) ;
    %Straight Lines [id:da3716776414047742] 
    \draw    (571,94) -- (410.33,62.6) ;
    
    % Text Node
    \draw (293.5,317.69) node [anchor=north] [inner sep=0.75pt]   [align=left] {他就是个王八蛋};
    % Text Node
    \draw (296,278) node [anchor=south] [inner sep=0.75pt]   [align=left] {\begin{minipage}[lt]{65.38pt}\setlength\topsep{0pt}
    \begin{center}
    head:\\nucleus clause
    \end{center}
    
    \end{minipage}};
    % Text Node
    \draw (473,318) node [anchor=north] [inner sep=0.75pt]   [align=left] {而已};
    % Text Node
    \draw (473,278) node [anchor=south] [inner sep=0.75pt]   [align=left] {\begin{minipage}[lt]{87.07pt}\setlength\topsep{0pt}
    \begin{center}
    evaluative particle \\ {[\category{delimitative}]}
    \end{center}
    
    \end{minipage}};
    % Text Node
    \draw (376,209) node [anchor=south] [inner sep=0.75pt]   [align=left] {\begin{minipage}[lt]{75.27pt}\setlength\topsep{0pt}
    \begin{center}
    comment:\\ \category{sfp} modification
    \end{center}
    
    \end{minipage}};
    % Text Node
    \draw (153,210) node [anchor=south] [inner sep=0.75pt]   [align=left] {\begin{minipage}[lt]{27.63pt}\setlength\topsep{0pt}
    \begin{center}
    topic:\\NP
    \end{center}
    
    \end{minipage}};
    % Text Node
    \draw (149,318) node [anchor=north] [inner sep=0.75pt]   [align=left] {张三};
    % Text Node
    \draw (266.33,138.6) node [anchor=south] [inner sep=0.75pt]   [align=left] {\begin{minipage}[lt]{200.27pt}\setlength\topsep{0pt}
    \begin{center}
    head:\\ topic-comment construction
    \end{center}
    
    \end{minipage}};
    % Text Node
    \draw (571,138.6) node [anchor=south] [inner sep=0.75pt]   [align=left] {\begin{minipage}[lt]{85.45pt}\setlength\topsep{0pt}
    \begin{center}
    speech act particle\\ {[\category{interrogative}]}
    \end{center}
    
    \end{minipage}};
    % Text Node
    \draw (571,318) node [anchor=north] [inner sep=0.75pt]   [align=left] {吗};
    % Text Node
    \draw (410.33,59.6) node [anchor=south] [inner sep=0.75pt]   [align=left] {\begin{minipage}[lt]{74.11pt}\setlength\topsep{0pt}
    \begin{center}
    sentence:\\ \category{sfp} modification
    \end{center}
    
    \end{minipage}};
    
    
    \end{tikzpicture}
    
    }
    \caption{Tree diagram of one possible structure of (\ref{ex:grammatical.clause.high-level.question.1})}
    \label{fig:grammatical.clause.high-level.question.1.1}
\end{figure}

\begin{figure}[H]
    {
        \centering
        \small
        \begin{tikzpicture}[x=0.75pt,y=0.75pt,yscale=-0.82,xscale=0.82]
    %uncomment if require: \path (0,350); %set diagram left start at 0, and has height of 350
    
    %Straight Lines [id:da039208741852972184] 
    \draw    (217,318.69) -- (408,318.69) ;
    %Straight Lines [id:da19942581874701737] 
    \draw    (315,286) -- (408,318.69) ;
    %Straight Lines [id:da25213640627062217] 
    \draw    (315,286) -- (217,318.69) ;
    %Straight Lines [id:da4257768539212] 
    \draw    (492,286) -- (492,319) ;
    %Straight Lines [id:da6711679136534192] 
    \draw    (395,217) -- (492,242) ;
    %Straight Lines [id:da8972586554532624] 
    \draw    (319,242) -- (395,217) ;
    %Straight Lines [id:da30788028272591994] 
    \draw    (166.33,149.6) -- (187,319) ;
    %Straight Lines [id:da23931546049438723] 
    \draw    (149,319) -- (187,319) ;
    %Straight Lines [id:da32269467462696366] 
    \draw    (166.33,149.6) -- (149,319) ;
    %Straight Lines [id:da31798198759619045] 
    \draw    (398,176) -- (485.33,149.6) ;
    %Straight Lines [id:da36231534147357114] 
    \draw    (591,176) -- (485.33,149.6) ;
    %Straight Lines [id:da110277422523411] 
    \draw    (591,217) -- (590,319) ;
    %Straight Lines [id:da14236204537587704] 
    \draw    (166,105) -- (317.33,73.6) ;
    %Straight Lines [id:da18846344122013636] 
    \draw    (481.33,106.6) -- (317.33,73.6) ;
    
    % Text Node
    \draw (312.5,321.69) node [anchor=north] [inner sep=0.75pt]   [align=left] {他就是个王八蛋};
    % Text Node
    \draw (315,282) node [anchor=south] [inner sep=0.75pt]   [align=left] {\begin{minipage}[lt]{65.38pt}\setlength\topsep{0pt}
    \begin{center}
    head:\\nucleus clause
    \end{center}
    
    \end{minipage}};
    % Text Node
    \draw (492,322) node [anchor=north] [inner sep=0.75pt]   [align=left] {而已};
    % Text Node
    \draw (492,282) node [anchor=south] [inner sep=0.75pt]   [align=left] {\begin{minipage}[lt]{87.07pt}\setlength\topsep{0pt}
    \begin{center}
    evaluative particle \\ {\category{delimitative}]}
    \end{center}
    
    \end{minipage}};
    % Text Node
    \draw (395,213) node [anchor=south] [inner sep=0.75pt]   [align=left] {\begin{minipage}[lt]{75.27pt}\setlength\topsep{0pt}
    \begin{center}
    comment:\\sentential clause
    \end{center}
    
    \end{minipage}};
    % Text Node
    \draw (166.33,146.6) node [anchor=south] [inner sep=0.75pt]   [align=left] {\begin{minipage}[lt]{27.63pt}\setlength\topsep{0pt}
    \begin{center}
    topic:\\NP
    \end{center}
    
    \end{minipage}};
    % Text Node
    \draw (168,322) node [anchor=north] [inner sep=0.75pt]   [align=left] {张三};
    % Text Node
    \draw (485.33,146.6) node [anchor=south] [inner sep=0.75pt]   [align=left] {\begin{minipage}[lt]{74.11pt}\setlength\topsep{0pt}
    \begin{center}
    head:\\ \category{sfp} modification
    \end{center}
    
    \end{minipage}};
    % Text Node
    \draw (591,214) node [anchor=south] [inner sep=0.75pt]   [align=left] {\begin{minipage}[lt]{85.45pt}\setlength\topsep{0pt}
    \begin{center}
    speech act particle\\ {[\category{interrogative}]}
    \end{center}
    
    \end{minipage}};
    % Text Node
    \draw (590,322) node [anchor=north] [inner sep=0.75pt]   [align=left] {吗};
    % Text Node
    \draw (317.33,70.6) node [anchor=south] [inner sep=0.75pt]   [align=left] {\begin{minipage}[lt]{74.11pt}\setlength\topsep{0pt}
    \begin{center}
    sentence:\\ \category{sfp} modification
    \end{center}
    
    \end{minipage}};
    
    
    \end{tikzpicture}
    
    }
    \caption{Tree diagram of another possible structure of (\ref{ex:grammatical.clause.high-level.question.1})}
    \label{fig:grammatical.clause.high-level.question.1.2}
\end{figure}

When the relative scopes of the topic and the sentence final particles are not important,
we can also choose the represent the sentence as \prettyref{fig:grammatical.clause.high-level.question.1.3},
which is just the diagrammatic version of bracketing in (\ref{ex:grammatical.clause.high-level.question.1}).

\begin{figure}[H]
    {
        \centering
        \small
        \begin{tikzpicture}[x=0.75pt,y=0.75pt,yscale=-0.85,xscale=0.85]
    %uncomment if require: \path (0,300); %set diagram left start at 0, and has height of 300
    
    %Straight Lines [id:da9618238865968819] 
    \draw    (225,252) -- (416,252) ;
    %Straight Lines [id:da027426105873617934] 
    \draw    (323,219.31) -- (416,252) ;
    %Straight Lines [id:da46184558003103304] 
    \draw    (323,219.31) -- (225,252) ;
    %Straight Lines [id:da6440331536377604] 
    \draw    (331.33,154.94) -- (525,192) ;
    %Straight Lines [id:da20332587806363012] 
    \draw    (528,219.31) -- (488,252) ;
    %Straight Lines [id:da7119387006121559] 
    \draw    (528,219.31) -- (546,252) ;
    %Straight Lines [id:da8655608578359899] 
    \draw    (178,219.31) -- (178,252) ;
    %Straight Lines [id:da42574803474951617] 
    \draw    (331.33,154.94) -- (324,192) ;
    %Straight Lines [id:da13941809614357192] 
    \draw    (331.33,154.94) -- (179,192) ;
    
    % Text Node
    \draw (320.5,255) node [anchor=north] [inner sep=0.75pt]   [align=left] {他就是个王八蛋};
    % Text Node
    \draw (488,255) node [anchor=north] [inner sep=0.75pt]   [align=left] {而已};
    % Text Node
    \draw (178,255) node [anchor=north] [inner sep=0.75pt]   [align=left] {张三};
    % Text Node
    \draw (546,255) node [anchor=north] [inner sep=0.75pt]   [align=left] {吗};
    % Text Node
    \draw (323,216.31) node [anchor=south] [inner sep=0.75pt]   [align=left] {\begin{minipage}[lt]{65.38pt}\setlength\topsep{0pt}
    \begin{center}
    nucleus clause
    \end{center}
    
    \end{minipage}};
    % Text Node
    \draw (528,216.31) node [anchor=south] [inner sep=0.75pt]   [align=left] {\begin{minipage}[lt]{103.32pt}\setlength\topsep{0pt}
    \begin{center}
    sentence final particles
    \end{center}
    
    \end{minipage}};
    % Text Node
    \draw (323,216.31) node [anchor=south] [inner sep=0.75pt]   [align=left] {\begin{minipage}[lt]{65.38pt}\setlength\topsep{0pt}
    \begin{center}
    nucleus clause
    \end{center}
    
    \end{minipage}};
    % Text Node
    \draw (178,216.31) node [anchor=south] [inner sep=0.75pt]   [align=left] {\begin{minipage}[lt]{24.8pt}\setlength\topsep{0pt}
    \begin{center}
    topic
    \end{center}
    
    \end{minipage}};
    % Text Node
    \draw (331.33,151.94) node [anchor=south] [inner sep=0.75pt]   [align=left] {\begin{minipage}[lt]{75.27pt}\setlength\topsep{0pt}
    \begin{center}
    sentential clause
    \end{center}
    
    \end{minipage}};
    
    
    \end{tikzpicture}
    
    }
    \caption{A flat-tree representation of (\ref{ex:grammatical.clause.high-level.question.1}).
    Additional information: the topic and sentence final particles have scope over the nucleus clause;
    their relative scope not specified.}
    \label{fig:grammatical.clause.high-level.question.1.3}
\end{figure}

\subsection{Subject and predicate}\label{sec:grammatical.clause.subject}

The \concept{nucleus clause} contains a \concept{subject} (if any) and what is often known as a \concept{predicate} (\prettyref{sec:grammatical.clause.predicate}, \prettyref{box:predicate}),
which usually is a (extended) \concept{verb phrase}.

\begin{exe}
    \ex\label{ex:grammatical.clause.subject.copular.1} 他就是个王八蛋!
    [\pinyin{ta1}]_{\text{subject},i} \pinyin{jiu4} \pinyin{shi4} [\pinyin{ge} \pinyin{wang2ba1dan4}]_{\text{copular complement}}
\end{exe}

\begin{theorybox}{Terminology: \term{predicate} and \term{predicator}}{predicate}
    \citet{dixon2009basic} argues against the definition of \term{predicate} 
    as the main verb (or adjective) plus somehow ``internal'' arguments.
    He uses the term \term{predicate} to refer to the verbal complex instead.
    However, since I will need to compare the topic-comment construction 
    with the inner structure of the nucleus clause,
    the term \term{predicate} will still be used in the way \citet{dixon2009basic} dislikes,
    because it's the counterpart of the comment role in the topic-comment construction.
    
    On the other hand, in this grammar, the term \term{predicator}
    refers to what selects arguments in a clause,
    which typically is just the main verb,
    although strictly speaking, the main verb, as a morphological word,
    contains grammatical markers that have scope over the whole argument structure
    (\prettyref{fig:grammatical.clause.core-vp.verbal-complex.2}).
    The term \term{predicator} is also used in \citet{cgel}.
\end{theorybox}

The verb phase contains all the \concept{arguments} in the clause besides the subject
(\prettyref{sec:grammatical.clause.core-vp.transitivity}, \prettyref{box:complement}),
and sometimes particles (\prettyref{sec:grammatical.clause.core-vp.particles}).
In the Mandarin simple nucleus clause,
the definition of the subject, as opposed to the topic,
is not trivially clear.
Here we note that the nucleus clause has a neutral structure (\prettyref{sec:grammatical.clause.subject.topic}),
in which a subject appearing at the initial is both the argument structure pivot
(\prettyref{sec:grammatical.clause.subject.argument})
and the clause-level pivot.

\begin{theorybox}{Terminology: complement, argument}{complement}
    A construction contains a head and a bunch of other constituents selected by the head.
    In a clause, the head is the main verb,
    which selects a subject, one or more objects and probably prepositional phrases,
    which are collectively called \concept{arguments}.
    \citet{cgel} also call them \concept{complements}.
    Note that the term \term{complement}, in the context of Chinese linguistics,
    often refers to a slot in the verbal complex described in \prettyref{sec:verbal-complex.v-c-a},
    known as 补语 in Mandarin.
    To avoid confusion, I call 补语 \concept{verbal complements}.
\end{theorybox}

\subsubsection{Existence of a neutral order}\label{sec:grammatical.clause.subject.topic}

The notion that in Mandarin, subject is the same as topic is prevalent.
Taking one step further, one may argue that Mandarin has no argument structure at all
and the word order in a clause is shaped by only information structure \citep{lapolla20091}.
This grammar rejects this analysis.

First, we note that a information structure neutral order can be defined for most, if not all, clauses.
An example is provided in (\ref{ex:grammatical.clause.subject.no-topic}).
The two arguments, 饭 and 吃, can be reordered in a seemingly free way depending on their topicality,
violating the common generalization that Mandarin has a SVO order.
We however note that (\ref{ex:grammatical.clause.subject.no-topic.4})
is completely unacceptable with the intended meaning.
Playing with more possible orders, and we will find that the arguments seem to be
only permitted to move \emph{leftwards} (and thus \ref{ex:grammatical.clause.subject.no-topic.4} is not possible),
consistent with the assumption that a neutral ordered nucleus clause is formed first,
followed by topicalization.
By analyzing subtle pragmatics differences, we find (\ref{ex:grammatical.clause.subject.no-topic.1}) seems to be the ``neutral'' order 
(although it imposes weak topicality to 你 \translate{you},
and 吃饭 \translate{eat (lit. eat meal)} is focalized).

\begin{exe}
    \ex\label{ex:grammatical.clause.subject.no-topic} 
    \begin{xlist}
        \ex\label{ex:grammatical.clause.subject.no-topic.1}
        \gll 你 吃 饭 了 吗 \\
        2 eat meal \category{sfp} \category{sfp} \\
        \glt\translate{Have you eaten?} 
        \ex\label{ex:grammatical.clause.subject.no-topic.2}
        \gll 饭 你 吃 了 吗 \\
        meal 2 eat \category{sfp} \category{sfp} \\
        \glt\translate{Have you eaten?}
        \ex\label{ex:grammatical.clause.subject.no-topic.3}
        \gll 你 饭 吃 了 吗 \\
        2 meal eat \category{sfp} \category{sfp} \\
        \glt\translate{Have you eaten?}
        \ex\label{ex:grammatical.clause.subject.no-topic.4}
        \gll *饭 吃 你 了 吗  \\
        meal eat 2 \category{sfp} \category{sfp} \\
        \glt\translate{Intended meaning: have you eaten? (Actual meaning: has meal eaten you?)}
    \end{xlist}
\end{exe}

We also note that there is no dangling topic in Mandarin (\prettyref{sec:topic-subject}).
This means \emph{all} topics originate from somewhere within the nucleus clause.
On the other hand, the subject, if well-defined by the usual pivot tests, is a part of the nucleus clause,
and therefore in Mandarin, topic and subject are different.

\subsubsection{Subject as pivot of argument structure}\label{sec:grammatical.clause.subject.argument}

Being the initial constituent%
\footnote{
    Note that it is possible that certain constituents, like temporal constituents,
    naturally appear before the subject.
}
in clauses like (\ref{ex:grammatical.clause.subject.no-topic.1}) 
has a clear relation to being the most prominent or the most \emph{external} argument
-- the agent or causer or the patient in passive constructions.

In (\ref{ex:grammatical.clause.subject.no-topic.1}),
the initial 你 \emph{has to} be the agent in the clause,
who intentionally initiates the action of eating.
(\ref{ex:grammatical.clause.subject.valency.1}),
on the other hand, is the intransitive use of a \category{cause}-\category{become} verb
(and \emph{not} pro-drop and topicalization; \prettyref{sec:valency.become.subject-or-topic}),
and by virtue of appearing at the initial of
the information structure-neutral clause,
茶 is not the agent: instead, it involuntarily \emph{undergoes} the situation,
as the clause has a \category{become} structure.

\begin{exe}
    \ex\label{ex:grammatical.clause.subject.valency.1} 茶泡好了
\end{exe}

\subsubsection{Subject as pivot of clause}\label{sec:grammatical.clause.subject.clause}

Certain ``clause linking'' constructions are actually verb phrase linking constructions.
At the first glance, (\ref{ex:grammatical.clause.subject.clause.1}) looks just like (\ref{ex:grammatical.clause.high-level.topic-and-coordination.1}),
but further grammatical tests show that the two are structurally different.
It is not possible for the conjunction 既 to appear before the subject;
further, it is not possible for the two branches to have different subjects
(\ref{ex:grammatical.clause.subject.clause.1-no-good}).
Therefore, the 既…又… coordination construction (and many more) is for connecting two verb phrases,
and we note that that the element shared by the two branches is
always the subject defined in \prettyref{sec:grammatical.clause.subject.argument}:

\begin{exe}
    \ex\label{ex:grammatical.clause.subject.clause.1} \gll {} [我]_{\text{subject}} 既 [不 想 用 这 个 方案]_{\text{VP}}, 又 [不 想 用 那 个 方案]_{\text{VP}} \\
    {} 1 \category{conj} \category{neg} want use this \category{cls} plan \category{conj} \category{neg} want use that \category{cls} plan \\
    \glt\translate{I don't want to use this plan, and nor do I want to use that plan.}
    \ex\label{ex:grammatical.clause.subject.clause.1-no-good} \begin{xlist}
        \ex\label{ex:grammatical.clause.subject.clause.1-no-good.1} *既我不想用这个方案,又不想用那个方案
        \ex *我既不想用这个方案,他又不想用那个方案
        \ex *我既不想用这个方案,他又不想用那个方案
    \end{xlist}
\end{exe}

In particular, we note that in these verb phrase coordination structures,
the shared subject corresponds to the sole argument of an intransitive construction
and the external argument of a transitive construction
(\ref{ex:grammatical.clause.subject.clause.coordination.pivot.2}). 

\begin{exe}
    \ex\label{ex:grammatical.clause.subject.clause.coordination.pivot.2} 张三既不打游戏,也不睡
\end{exe}

Thus, we find that in Mandarin,
we have both well-defined argument structure and clausal pivots,
which are identical.
This justifies using the term \term{subject} in describing Mandarin,
and confirms that Mandarin is a nominative-accusative language.

\subsubsection{Omision of subject}\label{sec:grammatical.clause.subject.pro-drop}

When the reference of the subject can be resolved,
it can be left blank.
In (\ref{ex:grammatical.clause.subject.pro-drop.1}),
the subject is null,
but from the conversational context,
it likely referred to the recipient of the question:
\translate{have you eaten?}

\begin{exe}
    \ex\label{ex:grammatical.clause.subject.pro-drop.1} 吃了吗
\end{exe}

\subsection{The predicate}\label{sec:grammatical.clause.predicate}

(\ref{ex:vp.ex.1}) is an illustration of a nucleus clause with a complicated predicate.
Its constituent structure is shown in \prettyref{fig:vp.ex.1},
following the notation in \citet{cgel}.
We need to warn that the main information contained in \prettyref{fig:vp.ex.1} 
is the \emph{scopes} of constituents surrounding the core verb phrase,
while the function labels (e.g. \term{head}) and the form labels (e.g. \form{extended VP})
in \prettyref{fig:vp.ex.1} conflate a series of constructions with subtle differences,
as 能在我的办公室跟你讨论一下 and 可能能在我的办公室跟你讨论一下
are both labeled as extended VPs, but clearly they have slightly different syntactic statuses:
the auxiliary 可能 can be attached to the former
but it can never appear twice and hence cannot be attached to the latter.
Giving the two different labels however makes the tree diagram tedious to read,
so \prettyref{fig:vp.ex.1} is a compromise.

\begin{exe}
    \ex \label{ex:vp.ex.1}
    \gll 我 [明天 可能 能 在 我 的 办公室 跟 你 [讨论 一下]_{\text{core\acs{vp}}}]_{\text{extended \acs{vp}}} \\
    1 tomorrow \category{aux}:possible \category{aux}:ability at my \category{poss} 
    office with 2 discuss a.little.bit \\ 
    \glt \translate{Tomorrow possiblity I can have a discussion with you in my office.}
\end{exe}

\begin{figure}[H]
    {
        \small 
        \begin{tikzpicture}[x=0.75pt,y=0.75pt,yscale=-0.8,xscale=0.8]
    %uncomment if require: \path (0,740); %set diagram left start at 0, and has height of 740
    
%Straight Lines [id:da5419452521165726] 
\draw    (690.67,583.33) -- (742,583.33) ;
%Straight Lines [id:da08544086505959503] 
\draw    (713,511.59) -- (706.23,533.35) -- (690.67,583.33) ;
%Straight Lines [id:da5166120577578295] 
\draw    (713,511.59) -- (742,583.33) ;
%Straight Lines [id:da7597929142889452] 
\draw    (410,368.26) -- (410,584.33) ;
%Straight Lines [id:da7951507355099614] 
\draw    (312.67,294.26) -- (312.67,584.33) ;
%Straight Lines [id:da27035558491953804] 
\draw    (622.67,583.33) -- (642,583.33) ;
%Straight Lines [id:da24028174129699176] 
\draw    (630,511.59) -- (622.67,583.33) ;
%Straight Lines [id:da917015263091598] 
\draw    (630,511.59) -- (642,583.33) ;
%Straight Lines [id:da14883621053934304] 
\draw    (470,583.33) -- (547,583.33) ;
%Straight Lines [id:da9342077761978527] 
\draw    (500,441.59) -- (470,583.33) ;
%Straight Lines [id:da3488354163696341] 
\draw    (500,441.59) -- (547,583.33) ;
%Straight Lines [id:da5030694457277152] 
\draw    (673,441.59) -- (628.67,462.33) ;
%Straight Lines [id:da7531987353356717] 
\draw    (673,441.59) -- (710.67,462.33) ;
%Straight Lines [id:da40179355911176984] 
\draw    (589,370.26) -- (502,398.33) ;
%Straight Lines [id:da5639473207238246] 
\draw    (589,370.26) -- (672,398.33) ;
%Straight Lines [id:da5704299439338378] 
\draw    (497,294.26) -- (410,322.33) ;
%Straight Lines [id:da8053817579360183] 
\draw    (497,294.26) -- (587,322.33) ;
%Straight Lines [id:da8066766119523336] 
\draw    (402,220.26) -- (315,248.33) ;
%Straight Lines [id:da8072938752013339] 
\draw    (402,220.26) -- (492,248.33) ;
%Straight Lines [id:da35978458309470907] 
\draw    (199.67,219.76) -- (199.67,475.49) -- (199.67,583.33) ;
%Straight Lines [id:da927464810721838] 
\draw    (294,146.26) -- (199.33,176.93) ;
%Straight Lines [id:da9186562853518254] 
\draw    (294,146.26) -- (400.33,176.93) ;
%Straight Lines [id:da7086875966035955] 
\draw    (190,71.59) -- (95.33,102.26) ;
%Straight Lines [id:da7175081469679203] 
\draw    (190,71.59) -- (296.33,102.26) ;
%Straight Lines [id:da26209295176984004] 
\draw    (87.67,146.56) -- (87.67,583.33) ;
%Straight Lines [id:da43013746492722116] 
\draw [color={rgb, 255:red, 80; green, 227; blue, 194 }  ,draw opacity=1 ][line width=2.25]    (191.67,623.26) -- (643.78,623.26) ;
%Straight Lines [id:da47071956794766767] 
\draw [color={rgb, 255:red, 74; green, 144; blue, 226 }  ,draw opacity=1 ][line width=2.25]    (686.78,623.26) -- (755,623.26) ;

% Text Node
\draw (78.67,590.17) node [anchor=north west][inner sep=0.75pt]   [align=left] {我 \ \ \ \ \ };
% Text Node
\draw (183.67,590.17) node [anchor=north west][inner sep=0.75pt]   [align=left] {明天};
% Text Node
\draw (293.67,590.17) node [anchor=north west][inner sep=0.75pt]   [align=left] {可能};
% Text Node
\draw (400.67,590.17) node [anchor=north west][inner sep=0.75pt]   [align=left] {能};
% Text Node
\draw (614.67,590.17) node [anchor=north west][inner sep=0.75pt]   [align=left] {跟你};
% Text Node
\draw (463.67,590.17) node [anchor=north west][inner sep=0.75pt]   [align=left] {在我的办公室};
% Text Node
\draw (687.67,590.17) node [anchor=north west][inner sep=0.75pt]   [align=left] {讨论一下};
% Text Node
\draw (713,508.59) node [anchor=south] [inner sep=0.75pt]   [align=left] {\begin{minipage}[lt]{38.85pt}\setlength\topsep{0pt}
\begin{center}
head:\\core VP
\end{center}

\end{minipage}};
% Text Node
\draw (630,508.59) node [anchor=south] [inner sep=0.75pt]   [align=left] {\begin{minipage}[lt]{51.96pt}\setlength\topsep{0pt}
\begin{center}
comitative:\\PP
\end{center}

\end{minipage}};
% Text Node
\draw (673,438.59) node [anchor=south] [inner sep=0.75pt]   [align=left] {\begin{minipage}[lt]{80.61pt}\setlength\topsep{0pt}
\begin{center}
head:\\extended VP
\end{center}

\end{minipage}};
% Text Node
\draw (500,438.59) node [anchor=south] [inner sep=0.75pt]   [align=left] {\begin{minipage}[lt]{40.93pt}\setlength\topsep{0pt}
\begin{center}
location:\\PP
\end{center}

\end{minipage}};
% Text Node
\draw (589,367.26) node [anchor=south] [inner sep=0.75pt]   [align=left] {\begin{minipage}[lt]{60.61pt}\setlength\topsep{0pt}
\begin{center}
head:\\extended VP
\end{center}

\end{minipage}};
% Text Node
\draw (410,365.26) node [anchor=south] [inner sep=0.75pt]   [align=left] {\begin{minipage}[lt]{64.42pt}\setlength\topsep{0pt}
\begin{center}
modality aux\\ {[\category{ability}]}
\end{center}

\end{minipage}};
% Text Node
\draw (497,291.26) node [anchor=south] [inner sep=0.75pt]   [align=left] {\begin{minipage}[lt]{60.61pt}\setlength\topsep{0pt}
\begin{center}
head:\\extended VP
\end{center}

\end{minipage}};
% Text Node
\draw (312.67,291.26) node [anchor=south] [inner sep=0.75pt]   [align=left] {\begin{minipage}[lt]{64.42pt}\setlength\topsep{0pt}
\begin{center}
modality aux\\ {[\category{possibility}]}
\end{center}

\end{minipage}};
% Text Node
\draw (402,217.26) node [anchor=south] [inner sep=0.75pt]   [align=left] {\begin{minipage}[lt]{60.61pt}\setlength\topsep{0pt}
\begin{center}
head:\\extended VP
\end{center}

\end{minipage}};
% Text Node
\draw (199.67,216.76) node [anchor=south] [inner sep=0.75pt]   [align=left] {\begin{minipage}[lt]{64.14pt}\setlength\topsep{0pt}
\begin{center}
time location:\\adverb
\end{center}

\end{minipage}};
% Text Node
\draw (294,143.26) node [anchor=south] [inner sep=0.75pt]   [align=left] {\begin{minipage}[lt]{60.61pt}\setlength\topsep{0pt}
\begin{center}
predicate:\\extended VP
\end{center}

\end{minipage}};
% Text Node
\draw (190,68.59) node [anchor=south] [inner sep=0.75pt]   [align=left] {\begin{minipage}[lt]{65.38pt}\setlength\topsep{0pt}
\begin{center}
nucleus clause
\end{center}

\end{minipage}};
% Text Node
\draw (87.67,143.56) node [anchor=south] [inner sep=0.75pt]   [align=left] {\begin{minipage}[lt]{39.55pt}\setlength\topsep{0pt}
\begin{center}
subject:\\pronoun
\end{center}

\end{minipage}};
% Text Node
\draw (417.72,626.26) node [anchor=north] [inner sep=0.75pt]  [color={rgb, 255:red, 80; green, 227; blue, 194 }  ,opacity=1 ] [align=left] {extension};
% Text Node
\draw (720.89,626.26) node [anchor=north] [inner sep=0.75pt]  [color={rgb, 255:red, 74; green, 144; blue, 226 }  ,opacity=1 ] [align=left] {core VP};


    
    
    \end{tikzpicture}
    
    }
    \caption{Tree diagram of (\ref{ex:vp.ex.1})}
    \label{fig:vp.ex.1}
\end{figure}

It can be seen that when we have verbal prediction,
the full, extended \acs{vp} following the subject 
can be divided into an extension region and the core \acs{vp}
(\prettyref{sec:grammatical.clause.core-vp}).
The extension region contains 
\acs{tame} auxiliaries and adverbs not realized in the verbal complex (\prettyref{sec:grammatical.clause.tam}),
and peripheral arguments like temporal and spatial locations (\prettyref{sec:grammatical.clause.peripheral}).
Sometimes the object may be fronted and 
it's also possible that a prepositional complement is fronted to this region.

%Not necessary to mention this
% The existence of the two domains in \prettyref{fig:vp.ex.1}
% enables us to redraw the tree in a flat way, shown in \prettyref{fig:vp.ex.1-alternative}.
% This flat-tree analysis has to be accompanied by the scope information of
% constituents in the extension region:
% that's to say, 可能 has scope over that of 能, etc.
% Once these annotations are added, \prettyref{fig:vp.ex.1-alternative}
% contains the same information as in \prettyref{fig:vp.ex.1}.
% Both analyses can be found in various literatures
% (cf. \citealt[\citepages{39,79,121}]{quirk1985}, and the tree diagrams in \citealt{cgel}),
% and we need to keep in mind that their differences are much smaller than what appear to be.
% 
% \begin{figure}[H]
%     {
%         \small
%         \begin{tikzpicture}[x=0.75pt,y=0.75pt,yscale=-0.8,xscale=0.8]
    %uncomment if require: \path (0,300); %set diagram left start at 0, and has height of 300
    
    %Straight Lines [id:da8955636807072126] 
    \draw    (101.57,123.98) -- (101.57,167.96) ;
    %Straight Lines [id:da3512435842306001] 
    \draw    (395.57,123.98) -- (201,169.98) ;
    %Straight Lines [id:da41529527687388235] 
    \draw    (395.57,123.98) -- (311,169.98) ;
    %Straight Lines [id:da416796280147414] 
    \draw    (395.57,123.98) -- (408,169.98) ;
    %Straight Lines [id:da19421854474989353] 
    \draw    (395.57,123.98) -- (519,169.98) ;
    %Straight Lines [id:da34431461000230335] 
    \draw    (395.57,123.98) -- (630,169.98) ;
    %Straight Lines [id:da7169790249724577] 
    \draw    (378.57,46.98) -- (105,95.98) ;
    %Straight Lines [id:da410995253580696] 
    \draw    (378.57,46.98) -- (389,95.98) ;
    %Straight Lines [id:da6624518557331187] 
    \draw    (378.57,46.98) -- (707,95.98) ;
    %Straight Lines [id:da37421433836653706] 
    \draw    (708.57,123.98) -- (708.57,167.96) ;
    
    % Text Node
    \draw (86.67,223.91) node [anchor=north west][inner sep=0.75pt]   [align=left] {我 \ \ \ \ \ };
    % Text Node
    \draw (184.67,223.91) node [anchor=north west][inner sep=0.75pt]   [align=left] {明天};
    % Text Node
    \draw (294.67,223.91) node [anchor=north west][inner sep=0.75pt]   [align=left] {可能};
    % Text Node
    \draw (401.67,223.91) node [anchor=north west][inner sep=0.75pt]   [align=left] {能};
    % Text Node
    \draw (615.67,223.91) node [anchor=north west][inner sep=0.75pt]   [align=left] {跟你};
    % Text Node
    \draw (464.67,223.91) node [anchor=north west][inner sep=0.75pt]   [align=left] {在我的办公室};
    % Text Node
    \draw (688.67,223.91) node [anchor=north west][inner sep=0.75pt]   [align=left] {讨论一下};
    % Text Node
    \draw (389,98.98) node [anchor=north] [inner sep=0.75pt]  [color={rgb, 255:red, 0; green, 0; blue, 0 }  ,opacity=1 ] [align=left] {extension region};
    % Text Node
    \draw (707,98.98) node [anchor=north] [inner sep=0.75pt]  [color={rgb, 255:red, 0; green, 0; blue, 0 }  ,opacity=1 ] [align=left] {head};
    % Text Node
    \draw (105,98.98) node [anchor=north] [inner sep=0.75pt]  [color={rgb, 255:red, 0; green, 0; blue, 0 }  ,opacity=1 ] [align=left] {subject};
    % Text Node
    \draw (101.57,172.98) node [anchor=north] [inner sep=0.75pt]  [color={rgb, 255:red, 0; green, 0; blue, 0 }  ,opacity=1 ] [align=left] {pronoun};
    % Text Node
    \draw (201,172.98) node [anchor=north] [inner sep=0.75pt]  [color={rgb, 255:red, 0; green, 0; blue, 0 }  ,opacity=1 ] [align=left] {\begin{minipage}[lt]{64.14pt}\setlength\topsep{0pt}
    \begin{center}
    time location:\\adverb
    \end{center}
    
    \end{minipage}};
    % Text Node
    \draw (311,172.98) node [anchor=north] [inner sep=0.75pt]  [color={rgb, 255:red, 0; green, 0; blue, 0 }  ,opacity=1 ] [align=left] {\begin{minipage}[lt]{53.49pt}\setlength\topsep{0pt}
    \begin{center}
    aux\\ {[\category{possibility}]}
    \end{center}
    
    \end{minipage}};
    % Text Node
    \draw (408,172.98) node [anchor=north] [inner sep=0.75pt]  [color={rgb, 255:red, 0; green, 0; blue, 0 }  ,opacity=1 ] [align=left] {\begin{minipage}[lt]{36.68pt}\setlength\topsep{0pt}
    \begin{center}
    aux\\ {[\category{ability}]}
    \end{center}
    
    \end{minipage}};
    % Text Node
    \draw (519,172.98) node [anchor=north] [inner sep=0.75pt]  [color={rgb, 255:red, 0; green, 0; blue, 0 }  ,opacity=1 ] [align=left] {\begin{minipage}[lt]{40.93pt}\setlength\topsep{0pt}
    \begin{center}
    location:\\PP
    \end{center}
    
    \end{minipage}};
    % Text Node
    \draw (630,172.98) node [anchor=north] [inner sep=0.75pt]  [color={rgb, 255:red, 0; green, 0; blue, 0 }  ,opacity=1 ] [align=left] {\begin{minipage}[lt]{51.96pt}\setlength\topsep{0pt}
    \begin{center}
    comitative:\\PP
    \end{center}
    
    \end{minipage}};
    % Text Node
    \draw (723.67,172.98) node [anchor=north] [inner sep=0.75pt]  [color={rgb, 255:red, 0; green, 0; blue, 0 }  ,opacity=1 ] [align=left] {core VP};
    % Text Node
    \draw (378.57,43.98) node [anchor=south] [inner sep=0.75pt]  [color={rgb, 255:red, 0; green, 0; blue, 0 }  ,opacity=1 ] [align=left] {nucleus clause};
    
    
    \end{tikzpicture}
    
%     }
%     \caption{Alternative flat-tree representation of (\ref{ex:vp.ex.1})}
%     \label{fig:vp.ex.1-alternative}
% \end{figure}

It should be noted that in the disposal and passive constructions,
the manner phrase may appear \emph{after} the auxiliary (\ref{ex:vp.ex.2}),
and in this case the boundary of the core \acs{vp}
can't be clearly defined at the surface level,
which shouldn't be surprising as we do not expect to
always see a clear-cut argument/adjunct distinction.

\begin{exe}
    \ex\label{ex:vp.ex.2}
    \gll 我 明天 可能 能 在 我 的 办公室 跟 你 [把]_{\text{auxiliary}}  这 个 问题 [好好]_{\text{manner}} 讨论 一下 \\
    1 tomorrow \category{aux}:possible \category{aux}:ability 
    at my \category{poss} office 
    with 2 
    \category{ba} this \category{cls} problem good
    discuss a.little.bit \\
    \glt \translate{Tomorrow possiblity I can have a good discussion of this problem with you in my office.}
\end{exe}

\subsubsection{Tense, aspect and modality marking}\label{sec:grammatical.clause.tam}

In (\ref{ex:vp.ex.1}), it can be clearly seen that Mandarin has modal auxiliaries:
the order (and also the scope) of 可能 and 能 is strictly 可能 \textgt{}能 and never the inverse,
suggesting that these modality markers are grammaticalized items.
More analytic markers of \ac{tame} categories can be found:
it seems 据说 is a peripheristic marker of evidentiality, for instance:
in (\ref{ex:grammatical.clause.tam.1}),
the order of the \acs{tame} markers is always 据说 \textgt{}可能, and not the inverse,
suggesting that 据说 is a part of the \acs{tame} grammatical hierarchy.%
\footnote{
    Note that English adverbs like \form{allegedly} follow the same generalization:
    we have e.g. \form{he allegedly possibly did this} but never \form{he possibly allegedly did this}. See \citet{cinque1999adverbs}.
}

\begin{exe}
    \ex\label{ex:grammatical.clause.tam.1} 
    这辆车据说可能不太靠谱
    \gll 
    [\pinyin{Zhe4} \pinyin{liang4} \pinyin{che1}]_{\text{subject:NP}} [\pinyin{ju4shuo1}]_{\text{evidentiality}} [\pinyin{ke3neng2}]_{\text{modality}} \pinyin{bu2} \pinyin{tai4} \pinyin{kao4pu3.} \\
    this \category{cls} car is.said \category{aux} \category{neg} very reliable \\
    \glt\translate{It is said that this car may not be very reliable.}
\end{exe}

Whether Mandarin has something comparable to tense in more prototypical tensed languages is not clear.
An observation is that Mandarin speakers often do not fully subconsciously acquire the tense category when learning tensed languages like English.
This, however, does not fully exclude the possibility of an impoverished tense system.
On the other hand, based on the positional distribution of certain time adverbs and interpretive evidence,
we can actually argue for the existence of a tense category,
which gets its value by agreement with the tense-like time adverb
(\prettyref{sec:tam.tense}).

Mandarin has ample devices to mark point-of-view aspect.
This is primarily done by the verbal complex (e.g. \ref{ex:grammatical.clause.core-vp.verbal-complex.1}),
via the (semi-)inflectional marking of aspect by 了, 着 and 过 in the verbal complex
(\prettyref{sec:grammatical.clause.core-vp}),
but analytic devices exist.
In (\ref{ex:grammatical.clause.tam.2}),
we find that the aspect marker 在 is separated from the core \ac{vp} by a manner phrase,
proving that 在 is not morphologically merged to the verbal complex.
The sentence is diagrammed in \prettyref{fig:grammatical.clause.tam.2};
cf. \prettyref{fig:grammatical.clause.core-vp.verbal-complex.1.1}.

\begin{exe}
    \ex\label{ex:grammatical.clause.tam.2} 他 在很认真地写作业
\end{exe}

\begin{figure}[H]
    \centering
    {
        \small
        \begin{tikzpicture}[x=0.75pt,y=0.75pt,yscale=-0.8,xscale=0.8]
    %uncomment if require: \path (0,443); %set diagram left start at 0, and has height of 443
    
    %Straight Lines [id:da528878482445523] 
    \draw    (521,366.69) -- (625.33,366.69) ;
    %Straight Lines [id:da40108809842151916] 
    \draw    (567.33,338.27) -- (625.33,366.69) ;
    %Straight Lines [id:da27474423107368695] 
    \draw    (567.33,338.27) -- (521,366.69) ;
    %Straight Lines [id:da9052525494601512] 
    \draw    (421.33,266.27) -- (421.33,367.27) ;
    %Straight Lines [id:da7905703091701273] 
    \draw    (530.33,172.27) -- (646,219) ;
    %Straight Lines [id:da6687808665035286] 
    \draw    (418,219) -- (530.33,172.27) ;
    %Straight Lines [id:da6091113777043599] 
    \draw    (315,127) -- (417.33,93.27) ;
    %Straight Lines [id:da5783470379038482] 
    \draw    (527,127) -- (417.33,93.27) ;
    %Straight Lines [id:da011193879040605426] 
    \draw    (312.33,172.27) -- (312.33,367.27) ;
    %Straight Lines [id:da2541456739251504] 
    \draw    (647.33,266.27) -- (731.33,294.27) ;
    %Straight Lines [id:da7805223808386317] 
    \draw    (691,366.69) -- (783.33,366.69) ;
    %Straight Lines [id:da905598438052631] 
    \draw    (732.33,338.27) -- (783.33,366.69) ;
    %Straight Lines [id:da992603763061734] 
    \draw    (732.33,338.27) -- (691,366.69) ;
    %Straight Lines [id:da11089280689158798] 
    \draw    (563.33,295.27) -- (647.33,266.27) ;
    
    % Text Node
    \draw (573.17,369.69) node [anchor=north] [inner sep=0.75pt]   [align=left] {很认真地};
    % Text Node
    \draw (418,222) node [anchor=north] [inner sep=0.75pt]   [align=left] {\begin{minipage}[lt]{100.39pt}\setlength\topsep{0pt}
    \begin{center}
    aspect marker\\{[\category{imperfective}]}
    \end{center}
    
    \end{minipage}};
    % Text Node
    \draw (421.33,370.27) node [anchor=north] [inner sep=0.75pt]   [align=left] {在};
    % Text Node
    \draw (647.33,225.27) node [anchor=north] [inner sep=0.75pt]   [align=left] {\begin{minipage}[lt]{60.61pt}\setlength\topsep{0pt}
    \begin{center}
    head:\\extended VP
    \end{center}
    
    \end{minipage}};
    % Text Node
    \draw (530.33,168.27) node [anchor=south] [inner sep=0.75pt]   [align=left] {\begin{minipage}[lt]{60.61pt}\setlength\topsep{0pt}
    \begin{center}
    predicate:\\extended VP
    \end{center}
    
    \end{minipage}};
    % Text Node
    \draw (315,130) node [anchor=north] [inner sep=0.75pt]   [align=left] {\begin{minipage}[lt]{39.55pt}\setlength\topsep{0pt}
    \begin{center}
    subject:\\pronoun
    \end{center}
    
    \end{minipage}};
    % Text Node
    \draw (312.33,370.27) node [anchor=north] [inner sep=0.75pt]   [align=left] {他};
    % Text Node
    \draw (417.33,90.27) node [anchor=south] [inner sep=0.75pt]   [align=left] {\begin{minipage}[lt]{65.38pt}\setlength\topsep{0pt}
    \begin{center}
    nucleus clause
    \end{center}
    
    \end{minipage}};
    % Text Node
    \draw (737.17,369.69) node [anchor=north] [inner sep=0.75pt]   [align=left] {写作业};
    % Text Node
    \draw (731.33,297.27) node [anchor=north] [inner sep=0.75pt]   [align=left] {\begin{minipage}[lt]{38.85pt}\setlength\topsep{0pt}
    \begin{center}
    head:\\core VP
    \end{center}
    
    \end{minipage}};
    % Text Node
    \draw (563.33,298.27) node [anchor=north] [inner sep=0.75pt]   [align=left] {\begin{minipage}[lt]{65.11pt}\setlength\topsep{0pt}
    \begin{center}
    manner:\\adverb phrase
    \end{center}
    
    \end{minipage}};
    
    
    \end{tikzpicture}
    
    }
    \caption{Tree diagram of (\ref{ex:grammatical.clause.tam.2})}
    \label{fig:grammatical.clause.tam.2}
\end{figure}

\subsubsection{Negation}\label{sec:grammatical.clause.negation}

The negator can appear at any position in the auxiliary chain described in \prettyref{sec:grammatical.clause.tam}.
Its linear order is consistent with its scope,
which in turn introduces subtle semantic differences
(\ref{ex:grammatical.clause.negation.1}).

\begin{exe}
    \ex\label{ex:grammatical.clause.negation.1} \begin{xlist}
        \ex\label{ex:grammatical.clause.negation.1.1} 
        \gll [他]_{\text{subject}} 不 [可能 能 出 国]_{\text{negated}} \\
        3 \category{neg} \category{aux} \category{aux} go.outside.of country \\
        \glt\translate{It's not possible that he has the ability to go abroad.}
    
        \ex\label{ex:grammatical.clause.negation.1.2} 
        \gll [他]_{\text{subject}} 可能 不 [能 出 国]_{\text{negated}} \\
        3 \category{aux} \category{neg} \category{aux} go.outside.of country \\
        \glt\translate{It's possible that he doesn't have the ability to go abroad.}
    
        \ex\label{ex:grammatical.clause.negation.1.3} 
        \gll [他]_{\text{subject}} 可能 能 不 [出 国]_{\text{negated}} \\
        3 \category{aux} \category{aux} \category{neg} go.outside.of country \\
        \glt\translate{It's possible that he has the ability to not go abroad.}
    \end{xlist}
\end{exe}

(\ref{ex:grammatical.clause.negation.1}) shows a negation device that is more flexible in its scope
that the English negation.
(\ref{ex:grammatical.clause.negation.1.2}) can be word-to-word translated to English as
\form{he possibly cannot go abroad},
but (\ref{ex:grammatical.clause.negation.1.1}) and (\ref{ex:grammatical.clause.negation.1.3})
can only be faithfully translated using complement clause constructions.

\subsubsection{Peripheral arguments}\label{sec:grammatical.clause.peripheral}

The term \term{peripheral argument} is from \citet{dixon2009basic}.
We intentionally use the term here instead of the more frequent \term{adjunct},
because there are both \acs{tame} adjuncts and circumstantial adjuncts,
the latter known as peripheral arguments in \citet{dixon2009basic}.

A clear distinction between core and peripheral arguments,
more often known as the argument/adjunct distinction, is not always possible.
Some criteria used for the distinction are about structural closeness of the argument to the main verb, or in other words scope:
the manner expression usually has a scope wider than the core verb phrase,
and thus the former is classified as a peripheral argument.
Other criteria are based on licensing: intransitive use of a transitive verb is prohibited by the lexicon,
or, in more technical terms, the verb root appearing in a verbal environment but without transitivity is not allowed by the lexicon \citep{siddiqi2009syntax}.
Thus \form{well} in \form{he treats us well} seems to be an argument, although it's a manner expression.
Yet other criteria are based on argument indexation and flagging:
an argument with oblique case marking does not leave agreement markers on the main verb,
while an argument with structural case (nominative, accusative) does
if the language has agreement marking,
and the latter is recognized as a core argument.
Following this standard, many so-called oblique arguments,
like \form{this} in \form{I think [of this]}, would be classified as peripheral,
although they are clearly licensed by the lexical entry of the verb.
These criteria correlate with each other but in a non-deterministic way.

Because Mandarin has no verb agreement, only the first two criteria can be used,
and the problems listed above all occur.
The status of comitative 跟你 in (\ref{ex:vp.ex.1}) is not so clear, for instance:
it is fairly low in \prettyref{fig:vp.ex.1}, but it is not obligatory.
We also note that reordering of peripheral arguments is possible,
but mixing them with \acs{tame} markers sounds problematic to say the least
(\ref{ex:grammatical.clause.peripheral.order.1}).
Note that fronting of the comitative to a higher position is possible (\prettyref{sec:topic.subject}).

\begin{exe}
    \ex\label{ex:grammatical.clause.peripheral.order.1} \begin{xlist}
        \ex 我明天可能能在我的办公室跟你讨论一下 (=\ref{ex:vp.ex.1})
        \ex 我明天可能能跟你在我的办公室讨论一下
        \ex ??我明天可能跟你能在我的办公室讨论一下
    \end{xlist}
\end{exe}

\subsection{The structure of the core verb phrase}\label{sec:grammatical.clause.core-vp}

In the surface form, the core \acs{vp} contains the core arguments and the \concept{verb}
(\prettyref{sec:grammatical.clause.core-vp.verbal-complex}).
The bracketed constituent in (\ref{ex:grammatical.clause.core-vp.verbal-complex.1})
is a typical core verb phrase.

\begin{exe}
    \ex\label{ex:grammatical.clause.core-vp.verbal-complex.1}
    \gll 我 [[做 完 了]_{\text{verb}} 作业]_{\text{core VP}} \\
    1 do finish \category{asp} homework \\
    \glt\translate{I have finished the homework.}
\end{exe}

Mandarin has two types of constructions in which the main verb appears at the final:
the \category{disposal} construction, and the so-called \category{passive} construction,
also known as the \form{ba}-construction and the \form{bei}-construction.

\begin{exe}
    \ex 我把作业做完了
\end{exe}

\subsubsection{The verbal complex}\label{sec:grammatical.clause.core-vp.verbal-complex}

Due to a lack of better terms, the term \concept{verb} has to have multiple meanings
(\prettyref{box:verb-definition}).
Here it refers to what uncontroversially is a morphological word,
which may receive a clearer name \concept{verbal complex}
especially when having a complicated internal structure,
as is discussed in this section.

\begin{theorybox}{The term \term{verb}}{verb-definition}
    We have just defined a \term{verb} as (a) the central morphological word of a clause.
    On the other hand, the term \term{verb} may refer to (b) a form with an argument structure
    that needs to be specified by lexical information.
    This is to say its internal structure, if any,
    can only be formed by \term{compounding} (\prettyref{box:dephrasal-compound}).
    This is a purely syntactic concept (without information from morphological realization).
    
    What is commonly referred to as a \term{verb} 
    can also be (c) any form that appears at the center of a clause and is lexicalized (\prettyref{sec:grammatical.lexicon}).
    It is a part of speech (\prettyref{sec:grammatical.lexicon.words}) that differs from roots (\prettyref{sec:grammatical.lexicon.roots}).
    Of course, all verbs defined according to (b) are verbs defined according to (c),
    but certain forms sometimes classified as verbs in terms of lexicalization are actually idioms,
    and are structurally ``larger'' than verbs defined in terms of the argument structure
    (\prettyref{sec:grammatical.lexicon.idiom}).

    Parts of an idiom are typically not considered to form one verbal complex.
    So some parts of a verb in the sense (c) are not included in a verb in the sense (a).
    On the other hand, formatives for aspect marking in the verbal complex (\prettyref{fig:grammatical.clause.core-vp.verbal-complex.2})
    and incorporated materials (\prettyref{ex:grammatical.clause.core-vp.verbal-complex.separable.1}) need not be stored in the lexicon.
    So some parts of a verb in the sense (a) are not included in a verb in the sense (c).
    We use terms \term{verbal complex} for (a), \term{compound} for (b) (\prettyref{box:dephrasal-compound}),
    and \term{verbal lexical entry} i.e. \term{lexeme} for (c)
    to specify what we mean more clearly whenever such confusion arises.

    Note that all the issues have counterparts in English grammar,
    but due to the less productive internal structures of English verbs,
    and the more regular lexeme-plus-conjugation, main verb-plus-auxiliary structure of the verbal complex,
    these issues can often be overlooked.
\end{theorybox}

The boundary of the verbal complex, i.e. the morphological word that contains the root that heads the whole clause,
is sometimes hard to say,
because some suffixes actually look like clitics in certain circumstances
(e.g. \prettyref{sec:verbal-complex.v-c-a.clitic}),
and in other cases we have incorporation (e.g. \prettyref{sec:verbal-complex.v-c-a.incorporation}).
We note that disagreement on wordhood does not influence the description of at least syntax
(\prettyref{sec:grammatical.clause.core-vp.derivation}).

Several types of verbal complexes exist in Mandarin
(\prettyref{sec:verbal-complex.linear-order}).
In (\ref{ex:grammatical.clause.core-vp.verbal-complex.1}),
完 is the verbal complement, and 了 is the aspect marker
(\prettyref{sec:verbal-complex.v-c-a}).
We note that the scope of the aspect marker 了 is \emph{over} the core verb phrase
as it governs the whole nucleus clause.
Following \prettyref{fig:grammatical.clause.tam.2},
we can represent the structure of (\ref{ex:grammatical.clause.core-vp.verbal-complex.1})
in \prettyref{fig:grammatical.clause.core-vp.verbal-complex.1.1}.

\begin{figure}[H]
    \centering
    {
        \small
        \begin{tikzpicture}[x=0.75pt,y=0.75pt,yscale=-0.8,xscale=0.8]
    %uncomment if require: \path (0,470); %set diagram left start at 0, and has height of 470
    
    %Straight Lines [id:da7905703091701273] 
    \draw    (530.33,172.27) -- (646,219) ;
    %Straight Lines [id:da6687808665035286] 
    \draw    (418,219) -- (530.33,172.27) ;
    %Straight Lines [id:da6091113777043599] 
    \draw    (315,127) -- (417.33,93.27) ;
    %Straight Lines [id:da5783470379038482] 
    \draw    (527,127) -- (417.33,93.27) ;
    %Straight Lines [id:da011193879040605426] 
    \draw    (312.33,172.27) -- (312.33,367.27) ;
    %Straight Lines [id:da2541456739251504] 
    \draw    (647.33,266.27) -- (731.33,294.27) ;
    %Straight Lines [id:da7805223808386317] 
    \draw    (691,366.69) -- (783.33,366.69) ;
    %Straight Lines [id:da905598438052631] 
    \draw    (732.33,338.27) -- (783.33,366.69) ;
    %Straight Lines [id:da992603763061734] 
    \draw    (732.33,338.27) -- (691,366.69) ;
    %Straight Lines [id:da11089280689158798] 
    \draw    (563.33,295.27) -- (647.33,266.27) ;
    %Straight Lines [id:da3706258778100655] 
    \draw    (564.33,338.27) -- (564.33,367.27) ;
    %Curve Lines [id:da663135552126335] 
    \draw  [dash pattern={on 0.84pt off 2.51pt}]  (420,393.27) .. controls (497.75,444.23) and (533.56,424.12) .. (572.6,395.06) ;
    \draw [shift={(575,393.27)}, rotate = 143.13] [fill={rgb, 255:red, 0; green, 0; blue, 0 }  ][line width=0.08]  [draw opacity=0] (10.72,-5.15) -- (0,0) -- (10.72,5.15) -- (7.12,0) -- cycle    ;
    %Straight Lines [id:da5230430384886475] 
    \draw    (418.33,273.27) -- (418.33,367.27) ;
    
    % Text Node
    \draw (564.33,370.27) node [anchor=north] [inner sep=0.75pt]   [align=left] {做完\textcolor[rgb]{0.29,0.56,0.89}{了}};
    % Text Node
    \draw (418,222) node [anchor=north] [inner sep=0.75pt]   [align=left] {\begin{minipage}[lt]{65.39pt}\setlength\topsep{0pt}
    \begin{center}
    \textcolor[rgb]{0.29,0.56,0.89}{aspect marker}\\\textcolor[rgb]{0.29,0.56,0.89}{[\category{perfective}]}
    \end{center}
    
    \end{minipage}};
    % Text Node
    \draw (647.33,225.27) node [anchor=north] [inner sep=0.75pt]   [align=left] {\begin{minipage}[lt]{38.85pt}\setlength\topsep{0pt}
    \begin{center}
    head:\\core VP
    \end{center}
    
    \end{minipage}};
    % Text Node
    \draw (530.33,168.27) node [anchor=south] [inner sep=0.75pt]   [align=left] {\begin{minipage}[lt]{60.61pt}\setlength\topsep{0pt}
    \begin{center}
    predicate:\\extended VP
    \end{center}
    
    \end{minipage}};
    % Text Node
    \draw (315,130) node [anchor=north] [inner sep=0.75pt]   [align=left] {\begin{minipage}[lt]{39.55pt}\setlength\topsep{0pt}
    \begin{center}
    subject:\\pronoun
    \end{center}
    
    \end{minipage}};
    % Text Node
    \draw (312.33,370.27) node [anchor=north] [inner sep=0.75pt]   [align=left] {我};
    % Text Node
    \draw (417.33,90.27) node [anchor=south] [inner sep=0.75pt]   [align=left] {\begin{minipage}[lt]{65.38pt}\setlength\topsep{0pt}
    \begin{center}
    nucleus clause
    \end{center}
    
    \end{minipage}};
    % Text Node
    \draw (737.17,369.69) node [anchor=north] [inner sep=0.75pt]   [align=left] {作业};
    % Text Node
    \draw (731.33,297.27) node [anchor=north] [inner sep=0.75pt]   [align=left] {\begin{minipage}[lt]{38.85pt}\setlength\topsep{0pt}
    \begin{center}
    object:\\ NP
    \end{center}
    
    \end{minipage}};
    % Text Node
    \draw (563.33,298.27) node [anchor=north] [inner sep=0.75pt]   [align=left] {\begin{minipage}[lt]{69.83pt}\setlength\topsep{0pt}
    \begin{center}
    predicator:\\verbal complex
    \end{center}
    
    \end{minipage}};
    
    
    \end{tikzpicture}
    
    }
    \caption{One tree diagram representation of (\ref{ex:grammatical.clause.core-vp.verbal-complex.1})}
    \label{fig:grammatical.clause.core-vp.verbal-complex.1.1}
\end{figure}

Actually, it is likely that 完 here is a lexical aspect marker,
and likely has scope over the whole 做完作业 argument structure,
so we need to add one more node between the aspect marker node
and the predicator node in \prettyref{fig:grammatical.clause.core-vp.verbal-complex.1.1}.
The possible types of formatives in the verbal complex
and their structural origins are described in \prettyref{sec:verbal-complex.linear-order}.

Aspect marking and compounding can coexist.
In \ref{ex:grammatical.clause.core-vp.verbal-complex.2},
the main verb consists of a dephrasal compound (\prettyref{fig:grammatical.clause.core-vp.compounding.1}),
and the aspect affixation in \prettyref{fig:grammatical.clause.core-vp.verbal-complex.1.1} also exists.
\prettyref{fig:grammatical.clause.core-vp.verbal-complex.2} 
is a diagrammatic representation of the verb structure in (\ref{ex:grammatical.clause.core-vp.verbal-complex.2}).


\begin{exe}
    \ex\label{ex:grammatical.clause.core-vp.verbal-complex.2} 上任第一天关心了这件事
\end{exe}

\begin{figure}[H]
    {
        \centering
        \small
        \begin{tikzpicture}[x=0.75pt,y=0.75pt,yscale=-0.85,xscale=0.85]
    %uncomment if require: \path (0,540); %set diagram left start at 0, and has height of 540
    
    %Straight Lines [id:da7062644688685291] 
    \draw [color={rgb, 255:red, 0; green, 0; blue, 0 }  ,draw opacity=1 ]   (407.33,321.2) -- (407.33,370) ;
    %Straight Lines [id:da6887191294274365] 
    \draw [color={rgb, 255:red, 0; green, 0; blue, 0 }  ,draw opacity=1 ]   (484.33,321.2) -- (484.33,370) ;
    %Straight Lines [id:da6238914258810929] 
    \draw [color={rgb, 255:red, 0; green, 0; blue, 0 }  ,draw opacity=1 ]   (444.57,255.04) -- (407.33,272.2) ;
    %Straight Lines [id:da5196839174944481] 
    \draw [color={rgb, 255:red, 0; green, 0; blue, 0 }  ,draw opacity=1 ]   (444.57,255.04) -- (485.33,272.2) ;
    %Straight Lines [id:da7285800726266157] 
    \draw    (586.57,255.04) -- (555.33,370) ;
    %Straight Lines [id:da9361549393106426] 
    \draw [color={rgb, 255:red, 0; green, 0; blue, 0 }  ,draw opacity=1 ]   (516.57,184.04) -- (444.57,210.04) ;
    %Straight Lines [id:da6865149711315705] 
    \draw [color={rgb, 255:red, 0; green, 0; blue, 0 }  ,draw opacity=1 ]   (516.57,184.04) -- (585.57,210.04) ;
    %Straight Lines [id:da8113936854933245] 
    \draw    (408.57,118.04) -- (309.57,142.04) ;
    %Straight Lines [id:da14440790051563346] 
    \draw    (408.57,118.04) -- (518.57,142.04) ;
    %Straight Lines [id:da23985586602033182] 
    \draw    (326.57,52.47) -- (242.57,74.04) ;
    %Straight Lines [id:da256699364704867] 
    \draw    (326.57,52.47) -- (411.57,74.04) ;
    %Straight Lines [id:da12377656510197987] 
    \draw    (310.33,187.47) -- (310.33,370) ;
    %Curve Lines [id:da2209865311467033] 
    \draw  [dash pattern={on 0.84pt off 2.51pt}]  (311.57,394.33) .. controls (354.14,415.12) and (450.62,427.09) .. (510.76,392.4) ;
    \draw [shift={(512.57,391.33)}, rotate = 149.04] [fill={rgb, 255:red, 0; green, 0; blue, 0 }  ][line width=0.08]  [draw opacity=0] (10.72,-5.15) -- (0,0) -- (10.72,5.15) -- (7.12,0) -- cycle    ;
    %Straight Lines [id:da08319541152867371] 
    \draw    (586.57,255.04) -- (614.33,370) ;
    %Straight Lines [id:da04438888328459056] 
    \draw    (614.33,370) -- (555.33,370) ;
    
    % Text Node
    \draw (407.33,373) node [anchor=north] [inner sep=0.75pt]  [color={rgb, 255:red, 0; green, 0; blue, 0 }  ,opacity=1 ] [align=left] {关};
    % Text Node
    \draw (310.33,373) node [anchor=north] [inner sep=0.75pt]   [align=left] {};
    % Text Node
    \draw (584.83,373) node [anchor=north] [inner sep=0.75pt]   [align=left] {这件事};
    % Text Node
    \draw (484.33,373) node [anchor=north] [inner sep=0.75pt]  [color={rgb, 255:red, 0; green, 0; blue, 0 }  ,opacity=1 ] [align=left] {心};
    % Text Node
    \draw (407.33,279.2) node [anchor=north] [inner sep=0.75pt]  [color={rgb, 255:red, 0; green, 0; blue, 0 }  ,opacity=1 ] [align=left] {\begin{minipage}[lt]{50.88pt}\setlength\topsep{0pt}
    \begin{center}
    predicator:\\verb
    \end{center}
    
    \end{minipage}};
    % Text Node
    \draw (484.33,279.2) node [anchor=north] [inner sep=0.75pt]  [color={rgb, 255:red, 0; green, 0; blue, 0 }  ,opacity=1 ] [align=left] {\begin{minipage}[lt]{33pt}\setlength\topsep{0pt}
    \begin{center}
    object:\\noun
    \end{center}
    
    \end{minipage}};
    % Text Node
    \draw (444.57,213.04) node [anchor=north] [inner sep=0.75pt]  [color={rgb, 255:red, 0; green, 0; blue, 0 }  ,opacity=1 ] [align=left] {\begin{minipage}[lt]{132.9pt}\setlength\topsep{0pt}
    \begin{center}
    predicator:\\dephasal compound from VP
    \end{center}
    
    \end{minipage}};
    % Text Node
    \draw (585.57,213.04) node [anchor=north] [inner sep=0.75pt]   [align=left] {\begin{minipage}[lt]{39.56pt}\setlength\topsep{0pt}
    \begin{center}
    object:\\pronoun
    \end{center}
    
    \end{minipage}};
    % Text Node
    \draw (518.57,145.04) node [anchor=north] [inner sep=0.75pt]   [align=left] {\begin{minipage}[lt]{46.33pt}\setlength\topsep{0pt}
    \begin{center}
    predicate:\\VP
    \end{center}
    
    \end{minipage}};
    % Text Node
    \draw (309.57,145.04) node [anchor=north] [inner sep=0.75pt]   [align=left] {\begin{minipage}[lt]{65.39pt}\setlength\topsep{0pt}
    \begin{center}
    aspect marker\\{[\category{perfective}]}
    \end{center}
    
    \end{minipage}};
    % Text Node
    \draw (411.57,77.04) node [anchor=north] [inner sep=0.75pt]   [align=left] {\begin{minipage}[lt]{60.61pt}\setlength\topsep{0pt}
    \begin{center}
    predicate:\\extended VP
    \end{center}
    
    \end{minipage}};
    % Text Node
    \draw (242.57,77.04) node [anchor=north] [inner sep=0.75pt]   [align=left] {\begin{minipage}[lt]{11.23pt}\setlength\topsep{0pt}
    \begin{center}
    \dots
    \end{center}
    
    \end{minipage}};
    % Text Node
    \draw (457.6,423) node [anchor=north west][inner sep=0.75pt]  [rotate=-90]  {$\Rightarrow $};
    % Text Node
    \draw (452,458) node [anchor=north] [inner sep=0.75pt]   [align=left] {关心了};
    % Text Node
    \draw (516.33,373) node [anchor=north] [inner sep=0.75pt]   [align=left] {\mbox{-}了};
    
    
    \end{tikzpicture}
    
    }
    \caption{Tree diagram representation of (\ref{ex:grammatical.clause.core-vp.verbal-complex.2})}
    \label{fig:grammatical.clause.core-vp.verbal-complex.2}
\end{figure}

A verb is sometimes \emph{separable},
meaning that other constituents within the verb phrase can be incorporated into it.
In (\ref{ex:grammatical.clause.core-vp.verbal-complex.separable.1}),
for instance, the object is incorporated into the verbal complex.
The incorporated constituent is not limited to the direct object.
The linear order of the clause after verb splitting
does not always transparently reflect the constituency relations in the clause
(e.g. \ref{ex:grammatical.clause.core-vp.pseudo-attributive.numeral.incorporation}).

\begin{exe}
    \ex\label{ex:grammatical.clause.core-vp.verbal-complex.separable.1} \begin{xlist}
        \ex 这件事你关什么心啊
        \ex 这件事你关心什么啊
    \end{xlist}
\end{exe}
 
Below, when we explicitly refer to \term{the verb},
often it is related to the morphological word i.e. the verbal complex
and the relevant morphological realizational properties.

\subsubsection{Intransitive and monotransitive constructions}\label{sec:grammatical.clause.core-vp.transitivity}

Verb frames in Mandarin can be divided into the \category{do} class (about actions),
the \category{become} class (about changes of states),
the \category{cause}-\category{become} class (about something causing a state to change),
and the purely stative \category{be} class.

Just like the case in English, \category{do} verb frames often allow S/A-ambivalence,
where the P argument of a transitive verb frame (i.e. the more internal, patient-like argument) can be removed, leaving only the subject (\ref{ex:grammatical.clause.core-vp.valency.do}).

\begin{exe}
    \ex\label{ex:grammatical.clause.core-vp.valency.do} \begin{xlist}
        \ex 他喜欢玩
        \ex 他喜欢玩玩具
    \end{xlist}
\end{exe}

On the other hand, we observe regular alternations between \category{become} and \category{cause}-\category{become} verbs,
and hence S/P-ambivalence (\ref{ex:grammatical.clause.core-vp.valency.become}).
It should be noted that not all \category{become} verbs can receive a \category{cause}-\category{become} verb frame (\ref{ex:grammatical.clause.core-vp.valency.become-only}).
We also note that the alternation in (\ref{ex:grammatical.clause.core-vp.valency.become})
cannot be explained by topicalization (\prettyref{sec:valency.become.subject-or-topic}).

\begin{exe}
    \ex\label{ex:grammatical.clause.core-vp.valency.become} \begin{xlist}
        \ex\label{ex:grammatical.clause.core-vp.valency.become.1} 茶泡好了
        \ex 我泡好茶了
    \end{xlist}

    \ex\label{ex:grammatical.clause.core-vp.valency.become-only}
    \begin{xlist}
        \ex\label{ex:grammatical.clause.core-vp.valency.become-only.1} 这只猫死了
        \ex *坏人死了这只猫
    \end{xlist}
\end{exe}

A rather interesting phenomenon in Mandarin is the \category{experience}-\category{become} construction,
in which the subject \emph{experiences} the effect of a change-of-state situation
(\ref{ex:grammatical.clause.core-vp.valency.experience}).

\begin{exe}
    \ex\label{ex:grammatical.clause.core-vp.valency.experience}
    \begin{xlist}
        \ex 王冕死了父亲
    \end{xlist}
\end{exe}

The verb frames naturally have correlations with the lexical aspect of the clause:
change-of-state clauses are naturally telic and often cannot be in the progressive aspect
(\ref{ex:grammatical.clause.core-vp.valency.progressive.1}).
But counterexamples exist
(\ref{ex:grammatical.clause.core-vp.valency.progressive.2}).

\begin{exe}
    \ex\label{ex:grammatical.clause.core-vp.valency.progressive.1} *这只猫正在死
    \ex\label{ex:grammatical.clause.core-vp.valency.progressive.2} 我正在泡茶
\end{exe}

\subsubsection{So-called semi-objects}

The A and P arguments discussed above are not the only types of arguments in Mandarin clauses.
The semi-objects (准宾语 in Chinese grammatical tradition; \citealt[\citepage{132}]{zhudexigrammar}) is introduced in this section;
the others are introduced in the following sections.

The duration of a Mandarin \category{do} clause
can be measured by a so-called \concept{semi-object}
(\ref{ex:grammatical.clause.core-vp-valency.semi-object}).
This is related to the so-called pseudo-attributive construction,
where the semi-object becomes the determiner of the nominalized core \ac{vp}
(\ref{ex:grammatical.clause.core-vp-valency.pseudo-attributive};
\prettyref{sec:grammatical.clause.core-vp.pseudo-attributive}).
Note that the semi-object construction is more permissive 
than the pseudo-attributive construction,
the latter only allowing the object to be a bare noun
(\ref{ex:grammatical.clause.derivation.glued-up.pseudo-attributive.1}).

\begin{exe}
    \ex\label{ex:grammatical.clause.core-vp-valency.semi-object} 我工作了一年
    \ex\label{ex:grammatical.clause.core-vp-valency.pseudo-attributive} 干了一个月的活
\end{exe}

\subsubsection{Internal arguments}\label{sec:grammatical.clause.core-vp.internal}

Some verb frames have what we may call \concept{internal objects}, usually licensed by a verbal complement.
These arguments are immune to any further syntactic operations.
The simplest case is the preposition complement construction (\ref{ex:grammatical.clause.core-vp-valency.internal-object-1}).
The \category{cause}-\category{become}-internal object structure is also possible,
although due to various constraints, sometimes it can only be realized as a \category{disposal} construction in the surface form (\ref{ex:grammatical.clause.core-vp-valency.internal-object-2}).


\begin{exe}
    \ex\label{ex:grammatical.clause.core-vp-valency.internal-object-1}
    我住在上海
    \ex\label{ex:grammatical.clause.core-vp-valency.internal-object-2} 
    \begin{xlist}
        \ex 卡车装满了稻草
        \ex 他把卡车装满了稻草
    \end{xlist}
\end{exe}

\subsubsection{External possession}

External possession is not commonly mentioned in Chinese grammar tradition,
but nevertheless is attested in the argument structure of Mandarin.

\begin{exe}
    \ex 这些橘子已经剥了皮了
    \ex\begin{xlist}
        \ex *这些橘子已经剥了它们的皮了
        \ex *这些橘子已经剥了很难剥的皮了
    \end{xlist}
\end{exe}

Finally, there seems to be a \category{do}-\category{affect}-\category{patient} construction
(\ref{ex:grammatical.clause.core-vp.valency.do-affect-patient.1}).
This construction is tentatively classified as a subtype of \category{do} verb frames,
mainly because we have no semantic evidence for a \category{cause}-to-\category{lose} analysis,
especially in (\ref{ex:grammatical.clause.core-vp.valency.do-affect-patient.2}),
where it's hard to argue that 他 and 耳光 form a mini verb frame meaning that the person in question loses something
\citep{huang2007}.
This however raises the question whether some semi-objects are to be analyzed in the same way,
and not as a wide-scope quantity or frequency phrase: cf. 打了他一下.

\begin{exe}
    \ex\label{ex:grammatical.clause.core-vp.valency.do-affect-patient.1} 阿飞抢了我一顶帽子
    \ex\label{ex:grammatical.clause.core-vp.valency.do-affect-patient.2} 打了他一个耳光
\end{exe}

\subsubsection{Verb-particle constructions and secondary predications}\label{sec:grammatical.clause.core-vp.particles}

Just like English, Mandarin has directional and resultative particles in the argument structure.

The presence of certain verbal complements alternates the argument structure.
买 is transitive, but (\ref{ex:grammatical.clause.core-vp.verbal-complement.valency.1})
demonstrates that the presence of the complement 贵 makes 
a transitive clause structure unacceptable.
This suggests that the underlying structure of (\ref{ex:grammatical.clause.core-vp.verbal-complement.valency.1.1})
differs from typical verb-particle constructions:
it is possible that 买 here is actually a manner expression,
and 贵 is the real main predicator.

\begin{exe}
    \ex\label{ex:grammatical.clause.core-vp.verbal-complement.valency.1} 
    \begin{xlist}
        \ex\label{ex:grammatical.clause.core-vp.verbal-complement.valency.1.1} 这辆车买贵了
        \ex *我买贵了这辆车
    \end{xlist}
\end{exe}

\subsubsection{Pseudo-attributive constructions}\label{sec:grammatical.clause.core-vp.pseudo-attributive}

In the pseudo-attributive construction,
a constituent usually appearing as a determiner of a noun phrase, 
like a numeral (\ref{ex:grammatical.clause.core-vp.pseudo-attributive.numeral})
or a possessive (\ref{ex:grammatical.clause.core-vp.pseudo-attributive.possessive}),
is inserted into a verb phrase.
This constituent is often known as a pseudo-attributive.

\begin{exe}
    \ex\label{ex:grammatical.clause.core-vp.pseudo-attributive.numeral} 我们干了一年的活
    \ex\label{ex:grammatical.clause.core-vp.pseudo-attributive.possessive} 你当你的老师
\end{exe}

\citet{huang2008} suggests that the pseudo-attributive is indeed an attributive:
what likely happens here is that the core \ac{vp} undergoes some sort of nominalization,
and the whole clause thus has a structure similar to 
我们 \category{do} [[一年]_{\text{quantity}}的干活]_{\text{nominalized core \ac{vp}}} 
(cf. English \form{we've being doing this work for one year}).
Then, because of the morpho-phonological requirement in Mandarin,
the verb 干 is fronted, forming (\ref{ex:grammatical.clause.core-vp.pseudo-attributive.numeral}).
When the main verb is disyllabic, it is sometimes possible to only front the first syllable
(\ref{ex:grammatical.clause.core-vp.pseudo-attributive.possessive.incorporation}).

\begin{exe}
    \ex\label{ex:grammatical.clause.core-vp.pseudo-attributive.possessive.incorporation} 你保你的守
\end{exe}

An empirical observation is that the pseudo-attributive construction
is generally incompatible with full-fledged complex \ac{np} objects
(\ref{ex:grammatical.clause.derivation.glued-up.pseudo-attributive.1}).
A parallel observation is that 
so-called glued-up predicator-object forms -- which, again, contain only bare noun objects --
can appear in nominal complementation without any modification
(\ref{ex:grammatical.clause.core-vp.compounding.2}).
In both constructions, we find verb-bare noun object combinations undergoing
what seems to be nominalization,
which is not possible when the object is a full fledged \ac{np}.
The nominalization analysis of the pseudo-attributive construction in \citet{huang2008}
is hence corroborated by this observation.

We also have the ditransitive pseudo-attributive construction,
in which the nominalized core \ac{vp} has a recipient.
(\ref{ex:grammatical.clause.core-vp.pseudo-attributive.numeral.incorporation})
is an example.
Note that splitting of the verb (as in \ref{ex:grammatical.clause.core-vp.pseudo-attributive.possessive.incorporation}) also appears in this example.

\begin{exe}
    \ex\label{ex:grammatical.clause.core-vp.pseudo-attributive.numeral.incorporation} 幽了他一默 \translate{give him humor}
\end{exe}

\subsubsection{Verb copying construction}

\begin{exe}
    \ex 做工做了一星期
\end{exe}

\subsection{Derivation}\label{sec:grammatical.clause.core-vp.derivation}

We have now reached the center of the clause structure,
and start to deal with concepts that typically fall under the category of derivation.
The exact meaning of the term \term{derivation} of course involves the question of wordhood.
In this grammar, we maintain that there is no need to treat the grammar above and below the line of wordhood
in drastically different ways a priori: we explore ``syntax within the word'' in this section,
and then move on to discuss purely morphological operations that gather formatives together to form morphological words in \prettyref{chap:verbal-complex}.
\prettyref{fig:grammatical.clause.core-vp.verbal-complex.1.1} is a good illustration of our approach.
Whether a definition of wordhood naturally emerge is discussed later in
\prettyref{sec:grammatical.wordhood}, \prettyref{sec:verbal-complex.wordhood}.

\subsubsection{Compounding}\label{sec:grammatical.clause.core-vp.derivation.compounding}

Frequently, the verb has an analyzable internal structure.
Sometimes that structure has no connection to any regular verbal constructions at all.
In (\ref{ex:grammatical.clause.derivation.standard-compounding.1}),
矛盾, if taking literally as a noun-level coordination, means \translate{spear and shield},
but is used as a verb.
That it is used as a verb cannot be explained by any regular syntactic construction:
what happens here is probably that 矛 and 盾 form a symmetric coordination,
which, due to idiomization (in this case, the irresistible force paradox),
appears in a verbal syntactic context.

\begin{exe}
    \ex\label{ex:grammatical.clause.derivation.standard-compounding.1} 他感到很矛盾
\end{exe}

In other cases, the internal structure of the verb \emph{is} analyzable within modern Mandarin syntax,
but the argument structure of the main verb still can't
be regularly predicted from the internal structure.
An example is given in (\ref{ex:grammatical.clause.core-vp.compounding.1}).

\begin{exe}
    \ex\label{ex:grammatical.clause.core-vp.compounding.1} 张三的女朋友很关心他
\end{exe}

In this example, we note that 关心 has an internal predicator-object structure.
The sequence 关心他 can't be explained by any attested verb frames in Mandarin
(\prettyref{chap:verb-frames}),
and therefore 关心 can only be seen as a \concept{dephrasal compound}
(on the meaning of the term \term{compound} see \prettyref{box:dephrasal-compound})
originating from a verb phrase
(possibly a glued-up verb phrase; \prettyref{sec:grammatical.clause.core-vp.derivation.glue}),
which then ``wears'' a transitive verb frame around it and functions like a verb
(the blue part in \prettyref{fig:grammatical.clause.core-vp.compounding.1}).
On the other hand, forms in \prettyref{sec:grammatical.clause.core-vp.derivation.glue}
do \emph{not} wear new valency frames around them,
and are not compounds in the proper sense (\prettyref{box:dephrasal-compound}). 

\begin{figure}[H]
    {
        \centering
        \small
        \begin{tikzpicture}[x=0.75pt,y=0.75pt,yscale=-0.85,xscale=0.85]
    %uncomment if require: \path (0,423); %set diagram left start at 0, and has height of 423
    
    %Straight Lines [id:da6830156861271671] 
    \draw    (192.57,137.04) -- (256.33,389) ;
    %Straight Lines [id:da9563574064337684] 
    \draw    (142.33,389) -- (256.33,389) ;
    %Straight Lines [id:da8413125935069653] 
    \draw    (192.57,137.04) -- (142.33,389) ;
    %Straight Lines [id:da9672768868143817] 
    \draw    (321.57,203.04) -- (321.33,389) ;
    %Straight Lines [id:da6508741786753789] 
    \draw [color={rgb, 255:red, 74; green, 144; blue, 226 }  ,draw opacity=1 ]   (419.33,340.2) -- (419.33,389) ;
    %Straight Lines [id:da9498407641716957] 
    \draw [color={rgb, 255:red, 74; green, 144; blue, 226 }  ,draw opacity=1 ]   (496.33,340.2) -- (496.33,389) ;
    %Straight Lines [id:da09739709587323908] 
    \draw [color={rgb, 255:red, 74; green, 144; blue, 226 }  ,draw opacity=1 ]   (456.57,274.04) -- (419.33,291.2) ;
    %Straight Lines [id:da6569066964886561] 
    \draw [color={rgb, 255:red, 74; green, 144; blue, 226 }  ,draw opacity=1 ]   (456.57,274.04) -- (497.33,291.2) ;
    %Straight Lines [id:da38350549257900046] 
    \draw    (598.57,274.04) -- (598.33,389) ;
    %Straight Lines [id:da6417623797559647] 
    \draw [color={rgb, 255:red, 74; green, 144; blue, 226 }  ,draw opacity=1 ]   (528.57,203.04) -- (456.57,229.04) ;
    %Straight Lines [id:da16338163169645048] 
    \draw [color={rgb, 255:red, 74; green, 144; blue, 226 }  ,draw opacity=1 ]   (528.57,203.04) -- (597.57,229.04) ;
    %Straight Lines [id:da8422162278245698] 
    \draw    (420.57,137.04) -- (321.57,161.04) ;
    %Straight Lines [id:da8624797524727791] 
    \draw    (420.57,137.04) -- (530.57,161.04) ;
    %Straight Lines [id:da8123281099854852] 
    \draw    (302.57,64.04) -- (192.57,89.04) ;
    %Straight Lines [id:da7960684783335494] 
    \draw    (302.57,64.04) -- (420.57,89.04) ;
    
    % Text Node
    \draw (321.33,392) node [anchor=north] [inner sep=0.75pt]   [align=left] {很};
    % Text Node
    \draw (419.33,392) node [anchor=north] [inner sep=0.75pt]  [color={rgb, 255:red, 74; green, 144; blue, 226 }  ,opacity=1 ] [align=left] {关};
    % Text Node
    \draw (199.33,392) node [anchor=north] [inner sep=0.75pt]   [align=left] {张三的女朋友};
    % Text Node
    \draw (598.33,392) node [anchor=north] [inner sep=0.75pt]   [align=left] {他};
    % Text Node
    \draw (496.33,392) node [anchor=north] [inner sep=0.75pt]  [color={rgb, 255:red, 74; green, 144; blue, 226 }  ,opacity=1 ] [align=left] {心};
    % Text Node
    \draw (419.33,298.2) node [anchor=north] [inner sep=0.75pt]  [color={rgb, 255:red, 74; green, 144; blue, 226 }  ,opacity=1 ] [align=left] {\begin{minipage}[lt]{50.88pt}\setlength\topsep{0pt}
    \begin{center}
    predicator:\\verb root
    \end{center}
    
    \end{minipage}};
    % Text Node
    \draw (496.33,298.2) node [anchor=north] [inner sep=0.75pt]  [color={rgb, 255:red, 74; green, 144; blue, 226 }  ,opacity=1 ] [align=left] {\begin{minipage}[lt]{33pt}\setlength\topsep{0pt}
    \begin{center}
    object:\\noun
    \end{center}
    
    \end{minipage}};
    % Text Node
    \draw (456.57,232.04) node [anchor=north] [inner sep=0.75pt]  [color={rgb, 255:red, 74; green, 144; blue, 226 }  ,opacity=1 ] [align=left] {\begin{minipage}[lt]{132.9pt}\setlength\topsep{0pt}
    \begin{center}
    predicator:\\dephasal compound from VP
    \end{center}
    
    \end{minipage}};
    % Text Node
    \draw (597.57,232.04) node [anchor=north] [inner sep=0.75pt]   [align=left] {\begin{minipage}[lt]{39.56pt}\setlength\topsep{0pt}
    \begin{center}
    \textcolor[rgb]{0.29,0.56,0.89}{object}:\\pronoun
    \end{center}
    
    \end{minipage}};
    % Text Node
    \draw (530.57,164.04) node [anchor=north] [inner sep=0.75pt]   [align=left] {\begin{minipage}[lt]{26.5pt}\setlength\topsep{0pt}
    \begin{center}
    head:\\VP
    \end{center}
    
    \end{minipage}};
    % Text Node
    \draw (321.57,164.04) node [anchor=north] [inner sep=0.75pt]   [align=left] {\begin{minipage}[lt]{33.88pt}\setlength\topsep{0pt}
    \begin{center}
    degree:\\adverb
    \end{center}
    
    \end{minipage}};
    % Text Node
    \draw (420.57,92.04) node [anchor=north] [inner sep=0.75pt]   [align=left] {\begin{minipage}[lt]{46.33pt}\setlength\topsep{0pt}
    \begin{center}
    predicate:\\VP
    \end{center}
    
    \end{minipage}};
    % Text Node
    \draw (192.57,92.04) node [anchor=north] [inner sep=0.75pt]   [align=left] {\begin{minipage}[lt]{37.59pt}\setlength\topsep{0pt}
    \begin{center}
    subject:\\NP
    \end{center}
    
    \end{minipage}};
    % Text Node
    \draw (302.57,61.04) node [anchor=south] [inner sep=0.75pt]   [align=left] {\begin{minipage}[lt]{65.38pt}\setlength\topsep{0pt}
    \begin{center}
    nucleus clause
    \end{center}
    
    \end{minipage}};
    
    
    \end{tikzpicture}
    
    }
    \caption{Tree diagram analysis of \eqref{ex:grammatical.clause.core-vp.compounding.1}.
    The blue part refers to the verb frame of 关心.}
    \label{fig:grammatical.clause.core-vp.compounding.1}
\end{figure}

\begin{theorybox}{Terminology: \term{dephrasal} and \term{compound}}{dephrasal-compound}
    The terms \term{dephrasal} and \term{compound} here are used just out of convenience,
    without implying that syntax stops at the boundary between prototypical phrases and compounds:
    what is traditionally recognized as a compound may show quite ``phrasal'' behaviors 
    (\prettyref{sec:grammatical.np.complementation}).
    Some syntactic tests are given in \citet[\citepages{448-451}]{cgel},
    but even at the level of compounds defined by \citeauthor{cgel},
    some regular syntactic processes happen (\prettyref{box:english-word-phrase}).
    In this section and in the rest of this grammar, we just \emph{define} what happens \emph{before} the verb frame being introduced \concept{compounding}.
    This does not deny that syntactic processes apply to this definition of compounds too,
    and indeed (\ref{ex:grammatical.clause.core-vp.compounding.coordination.1})
    can be seen as an instance.

    A compound with an internal structure that resembles a phrase
    is known as a \concept{dephrasal compound}.
    Whether actually the internal syntax of a dephrasal compound is the same as prototypical phrasal syntax is another question.
    It could just be recategorization, i.e. a verb phrase as a whole wears a verb frame outside of it,
    or it could involve syntactic erosion and has an internal structure that is unlike structures commonly referred to as phrasal.
    We can use English as an example:
    \form{sickbed} can be regularly interpreted as \translate{bed related to sickness}, 
    i.e. the bed on which one lies sick.
    The form \form{sick-}, meaning \form{sickness} here,
    clearly is not be used as an adjectival phrase,
    and is likely just an uncategorized root,
    being glued to another root \form{bed} in some sort of ``aboutness'' construction.

    The distinction between the two types of compounding is observed both in the nominal and the verbal systems.
    What is described in \prettyref{sec:grammatical.np.derivation.compounding} is clearly the second type, or ``real'' compounds,
    and so is the verb in \prettyref{ex:grammatical.clause.core-vp.compounding.coordination.2}.
    We may want to use this distinction to demarcate the boundary between words and phrases,
    but note that cross-linguistically, it is also possible to embed a form that is commonly considered a phrase
    into a not-so-prototypical construction.
    This is common in English informal speech,
    like \form{[killing 682]-ology} from \href{https://scp-wiki.wikidot.com/forum/t-410418/scp-1066}{this forum discussion}.
    As usual, syntactic criteria hardly provide clear demarcation of wordhood
    (\prettyref{box:english-word-phrase}),
    and ``real'' compounds (i.e. the second type) can be analyzed using concepts from prototypical phrasal syntax as well \citep{scher2014unifying}.
    
    The distinction between the two types of compounding,
    which determines whether certain marginal forms are acceptable,
    is also related to whether we should label 关 in \prettyref{fig:grammatical.clause.core-vp.compounding.1} as a root or as a verb.
    Since \prettyref{fig:grammatical.clause.core-vp.compounding.1} is an analysis based on the surface form,
    we represent 关 as a verb for now.
    This labeling strategy becomes problematic 
    when we are dealing with cases like \form{hard worker},
    which is better represented as \form{[work hard]-er},
    and \form{work} and the nominalizer \form{-er} form a morphological word
    which does \emph{not} have the lexicalized meaning of the word \form{worker}
    \translate{employees, especially factory workers etc.}:
    in this case saying ``\form{worker} is a noun''
    is misleading in that \form{worker} is a morphological word
    and its internal parts do not form a grammatical constituency.
    This is even more clearly revealed by the form \form{do-gooder}.
    No such forms exist in Mandarin,
    so the labeling in \prettyref{fig:grammatical.clause.core-vp.compounding.1}
    should at least be descriptively accurate,
    although it is not the only possible labeling.
    
    Finally, constructions in \prettyref{sec:grammatical.np.complementation} and \prettyref{sec:grammatical.clause.core-vp.derivation.glue} are often referred to as compounding in some teaching materials,
    but they are structurally different from both types of compounds defined above,
    and are better analyzed as 粘合式结构 or \term{glued-up structures},
    in terms of \citet{zhudexigrammar} (cf. \prettyref{sec:grammatical.clause.core-vp.derivation.glue}).
\end{theorybox}

More types of compounding constructions exists besides dephrasal compounding.
The structure of some so-called compound verbs reminds us of verb phrase-level coordination (\prettyref{sec:grammatical.clause.subject.clause}).
The two branches in (\ref{ex:grammatical.clause.core-vp.compounding.coordination.1}), 吃 and 喝,
for instance, can both take 东西 as their objects.
Whether (\ref{ex:grammatical.clause.core-vp.compounding.coordination.1})
is to be analyzed synchronically as the coordination of 吃 and 喝 sharing the same object is not clear.

\begin{exe}
    \ex\label{ex:grammatical.clause.core-vp.compounding.coordination.1} 哪怕没有吃喝东西
\end{exe}

Some verbs however are clearly not formed by synchronic verb phrase-level coordination:
in (\ref{ex:grammatical.clause.core-vp.compounding.coordination.2}),
the two roots 讥 and 讽 are both Classical and never 
appears as heads of clauses on their own.
This however does not mean the roots are not productive:
from 讽 we have 反讽, 讽诵, 讽经, and more.
(\ref{ex:grammatical.clause.core-vp.compounding.coordination.2}) therefore uncontroversially involves 
``real'' compounding,
different from the synchronically still analyzable (\ref{ex:grammatical.clause.core-vp.compounding.coordination.1}).
Accompanied by this distinction is a constraint in complexity:
coordination at the \ac{vp}-level can be quite complicated,
while real compounds involving roots like 讽 typically cannot be very long.

\begin{exe}
    \ex\label{ex:grammatical.clause.core-vp.compounding.coordination.2} 不要随便讥讽人
\end{exe}

\subsubsection{Affixation}

Affixation is not very productive in Mandarin's verbal system.
Verbalization suffixation exists
(\prettyref{sec:verbal-complex.v-c-a}),
and certain 

\subsubsection{Glued-up predicator-object structures}\label{sec:grammatical.clause.core-vp.derivation.glue}

A type of predicator-object forms that at the first glance resembles \prettyref{fig:grammatical.clause.core-vp.compounding.1} 
is what \citet[\citepages{128-9}]{zhudexigrammar}
calls \concept{glued-up predicator-object structures} (粘合式述宾结构).
In a glued-up predicator-object form,
the object being a bare noun, without any modification or complementation:
see e.g. 营救人质 in (\ref{ex:grammatical.clause.core-vp.compounding.2.3}).

\begin{exe}
    \ex\label{ex:grammatical.clause.core-vp.compounding.2.3} 这个工作组要去[营救人质]_{\text{glued-up}},去[营救困在石油园区里的倒霉鬼]_{\text{VP}}
\end{exe}

At the first glance, there is nothing special with the object being a bare noun,
because if 营救 takes an object,
then 营救人质, in which the \category{wh}-pronoun, the numeral and the classifier
are all removed from the object is of course also well-formed.
There are however properties of glued-up predicator-object forms that are not shared by
prototypical verb phrases which can take arbitrarily complicated objects.
The contrast between (\ref{ex:grammatical.clause.core-vp.compounding.2.1})
and (\ref{ex:grammatical.clause.core-vp.compounding.2.2}) means that 
营救人质 (a glued-up predicator-object structure) can be used as a nominal in \prettyref{sec:grammatical.np.complementation}
while 营救他们, which is a prototypical verb phrase, or in the terminology of \citet[\citepages{128-9}]{zhudexigrammar},
a \concept{compositional predicator-object structure} (组合式述宾结构), can't 
(cf. \ref{ex:grammatical.np.complementation.2}, a similar contrast).

\begin{exe}
    \ex\label{ex:grammatical.clause.core-vp.compounding.2}
    \begin{xlist}
        \ex\label{ex:grammatical.clause.core-vp.compounding.2.1} {} [营救人质]小队
        \ex\label{ex:grammatical.clause.core-vp.compounding.2.2} {} *[营救他们]小队
    \end{xlist}
\end{exe}

We tentatively analyze these glued-up predicator-object forms as 
verbs taking bare nouns which do not have clear references (instead of full fledged ones) as objects
(\prettyref{box:glued-up-double-structure}).
First, we note that a glued-up predicator-object form regularly takes no other object.

A second piece of evidence comes from the pseudo-attributive construction
(\prettyref{sec:grammatical.clause.core-vp.pseudo-attributive}).
We note that the pseudo-attributive can only appear
when the object is a bare noun 
(\ref{ex:grammatical.clause.derivation.glued-up.pseudo-attributive.1.1}):
it cannot appear when the object is a full fledge \ac{np}
(\ref{ex:grammatical.clause.derivation.glued-up.pseudo-attributive.1.2}).
This contrast demonstrates existence of structural difference
between a bare noun object and a full fledge \ac{np} object.
(\ref{ex:grammatical.clause.derivation.glued-up.pseudo-attributive.1.3}),
on the other hand, is a semi-object construction,
and does not impose any requirements to the complexity of the object.
As the distinction between the two types of objects has been motivated independently by the pseudo-attributive construction,
it seems optimal to also attribute the peculiarity of glued-up predicator-object forms to this distinction
(see also \prettyref{sec:grammatical.clause.core-vp.pseudo-attributive}).

\begin{exe}
    \ex\label{ex:grammatical.clause.derivation.glued-up.pseudo-attributive.1}
    \begin{xlist}
        \ex\label{ex:grammatical.clause.derivation.glued-up.pseudo-attributive.1.1} 他看了一天的书
        \ex\label{ex:grammatical.clause.derivation.glued-up.pseudo-attributive.1.2} *他看了一天的那本书
        \ex\label{ex:grammatical.clause.derivation.glued-up.pseudo-attributive.1.3} 他看了一天书/那本书
    \end{xlist}
\end{exe}

\begin{infobox}{Alternative analysis}{glued-up-double-structure}
    We are tempted to adopt a dual-structure analysis:
    one has undergone syntactic fossilization and become a dephrasal compound
    (as in e.g. \ref{ex:grammatical.clause.core-vp.compounding.2.1}), 
    and the other is regularly built by phrasal structural rules of the verb phrase.
    This, however, inevitably involves the problem of definition of ``compounding'':
    in \prettyref{sec:grammatical.np.complementation}, for instance,
    the nominals, despite looking like compounds, can be analyzed as
    formed by prototypical argument structures.
\end{infobox}

A glued-up predicator-object form often -- but not necessarily -- is disyllabic,
in which the verb has one syllable and the object has one syllable,
without any other constituents.
This however is not absolute, as is shown by the intentionally chosen example (\ref{ex:grammatical.clause.core-vp.compounding.2.3}).

\subsubsection{Bound objects}\label{sec:grammatical.clause.core-vp.derivation.object-bound}

Another type of unusual grammatical units that look like \acp{vp} 
license objects that do not typically appear as heads of \acp{np}
(\ref{ex:grammatical.clause.derivation.idiom.1.1} vs. \ref{ex:grammatical.clause.derivation.idiom.1.2}).
\citet[\citepage{129}]{zhudexigrammar} argues that based on the fact that 
as the objects are ``bound morphemes'', these units can only be compounds.
This assumes homogeneity of ``bound morphemes'' (\prettyref{sec:grammatical.lexicon.roots})
and a strict word/phrase distinction based on the morpheme-word-phrase hierarchy,
which we do not stipulate a priori.
In the case of (\ref{ex:grammatical.clause.derivation.idiom.1}),
we note that the object 亏 can be modified by 这个,
and can be topicalized (\ref{ex:grammatical.clause.derivation.idiom.1.3}).
This is clearly not the behavior of a typical compound verb.
Instead of invoking the word/phrase distinction,
we tentatively assume that the appearance of 亏,
for some reasons, can only be licensed by the presence of a handful of other grammatical objects,
吃 included, and once it is licensed in a clause,
its behavior is the same as a typical noun.
This explains unacceptability of (\ref{ex:grammatical.clause.derivation.idiom.1.2})
and acceptability of (\ref{ex:grammatical.clause.derivation.idiom.1.3}).

\begin{exe}
    \ex\label{ex:grammatical.clause.derivation.idiom.1} \begin{xlist}
        \ex\label{ex:grammatical.clause.derivation.idiom.1.1} 我昨天[吃亏]了
        \ex\label{ex:grammatical.clause.derivation.idiom.1.2} *一个亏
        \ex\label{ex:grammatical.clause.derivation.idiom.1.3} 这个亏我不能吃
    \end{xlist}
\end{exe}

Under our analysis, 吃亏 and the like are all verb phrases,
and they do not necessarily have verb stem counterparts.
We therefore reject the analysis of \citet[\citepage{129}]{zhudexigrammar}.

\section{Noun phrase}\label{sec:grammatical.np}

\subsection{The determiner region}\label{sec:grammatical.np.determiner}

Following the procedure in \prettyref{sec:grammatical.clause.high-level},
we start analyzing the noun phrase by recognizing the high-level categories first,
which mean what are commonly called \concept{determiners}.
Following the convention in \citet{cgel},
we name the part of the \ac{np} below the determiner region the \concept{nominal} (\prettyref{box:nominal}).
A nominal can form a \ac{np} on its own (e.g. \ref{ex:grammatical.np.adjective.1}).

\begin{theorybox}{Terminology: \term{nominal}}{nominal}
    The term \term{nominal} refers to different things in different places.
    In Indo-European comparative linguistics, a \term{nominal} is a word that has a noun-like pattern.
    This is not hwo we use the term \term{nominal} here.

    Besides, sometimes we have to use \term{nominal} as an adjective in this grammar
    to refer to anything that is related to the noun phrase.
    This causes confusion, but we have no better terms.
\end{theorybox}

(\ref{ex:grammatical.np.determiner.1}) is a demonstration of 
the determiner region of the \ac{np} in Mandarin.
A hierarchy, possessor \textgt{}demonstrative \textgt{}numeral \textgt{}classifier,
can be identified.
Alternating the linear order in this hierarchy renders the whole \ac{np}
completely unacceptable, except maybe (\ref{ex:grammatical.np.determiner.1-alternate}).

\begin{exe}
    \ex\label{ex:grammatical.np.determiner.1}
    \gll 我的 这 二十 件 [大 白 褂子]_{\text{nominal}} \\
    1-\category{poss} this twenty \category{cls} big white robe-\category{deminutive} \\
    \glt\translate{these twenty big white robes of mine}

    \ex\label{ex:grammatical.np.determiner.1-alternate}
    \gll ?这 二十 件 我的 [大 白 褂子]_{\text{nominal}} \\
    this twenty \category{cls} 1-\category{poss} big white robe-\category{deminutive} \\
    \glt\translate{these twenty big white robes of mine}
\end{exe}


\subsection{Adjectives}\label{sec:grammatical.np.adjectives}

Although the noun phrase in (\ref{ex:grammatical.np.adjective.1}) looks quite compact,
we note that it has structurally nothing different from the English \form{a big white robe}.
In particular, the order size \textgt{}color and the relevant scope effects is shared
by (\ref{ex:grammatical.np.adjective.1}) and \form{a big white robe}.

\begin{exe}
    \ex\label{ex:grammatical.np.adjective.1} 大白褂子
\end{exe}

The structure of (\ref{ex:grammatical.np.adjective.1}) therefore is represented in 
\prettyref{fig:grammatical.np.adjective.1}.
We note that here the label \term{nominal} is used in a slightly sloppy way,
just like how the label \term{extended VP} is used in \prettyref{fig:vp.ex.1}:
if we have the hierarchy size \textgt{}color,
then by definition the syntactic properties of a color modification construction
is not the same as the syntactic properties of a size modification construction.
Calling 白褂子 a ``NP with color modification'' however makes the tree diagram tedious to read
and \prettyref{fig:grammatical.np.adjective.1} is a compromise.

\begin{figure}[H]
    \centering
    {
        \small
        \begin{tikzpicture}[x=0.75pt,y=0.75pt,yscale=-0.8,xscale=0.8]
    %uncomment if require: \path (0,323); %set diagram left start at 0, and has height of 323
    
    %Straight Lines [id:da2510017208796249] 
    \draw    (306.33,235.8) -- (306.33,299) ;
    %Straight Lines [id:da9406911787357282] 
    \draw    (374.33,256.11) -- (374.33,299) ;
    %Straight Lines [id:da7402573305476123] 
    \draw    (336.57,184.61) -- (373,207) ;
    %Straight Lines [id:da8392725146835093] 
    \draw    (303,207) -- (336.57,184.61) ;
    %Straight Lines [id:da7987182606039094] 
    \draw    (223.33,163.8) -- (223.33,299) ;
    %Straight Lines [id:da3274548515528497] 
    \draw    (282.57,118.61) -- (335,139) ;
    %Straight Lines [id:da10356491942990143] 
    \draw    (227,139) -- (282.57,118.61) ;
    %Straight Lines [id:da0910995248012354] 
    \draw    (151.33,93.8) -- (151.33,299) ;
    %Straight Lines [id:da18618563141514666] 
    \draw    (215.57,47.61) -- (281,71) ;
    %Straight Lines [id:da11820335341740118] 
    \draw    (151,72) -- (215.57,47.61) ;
    
    % Text Node
    \draw (223.33,302) node [anchor=north] [inner sep=0.75pt]   [align=left] {白};
    % Text Node
    \draw (306.33,302) node [anchor=north] [inner sep=0.75pt]   [align=left] {褂};
    % Text Node
    \draw (151.33,302) node [anchor=north] [inner sep=0.75pt]   [align=left] {大};
    % Text Node
    \draw (374.33,302) node [anchor=north] [inner sep=0.75pt]   [align=left] {子};
    % Text Node
    \draw (308,212) node [anchor=north] [inner sep=0.75pt]   [align=left] {\begin{minipage}[lt]{23.67pt}\setlength\topsep{0pt}
    \begin{center}
    head
    \end{center}
    
    \end{minipage}};
    % Text Node
    \draw (377,212) node [anchor=north] [inner sep=0.75pt]   [align=left] {\begin{minipage}[lt]{55.64pt}\setlength\topsep{0pt}
    \begin{center}
    suffix\\ {[\category{diminutive}]}
    \end{center}
    
    \end{minipage}};
    % Text Node
    \draw (335,142) node [anchor=north] [inner sep=0.75pt]   [align=left] {\begin{minipage}[lt]{51.67pt}\setlength\topsep{0pt}
    \begin{center}
    head:\\deminutive
    \end{center}
    
    \end{minipage}};
    % Text Node
    \draw (227,142) node [anchor=north] [inner sep=0.75pt]   [align=left] {\begin{minipage}[lt]{24.27pt}\setlength\topsep{0pt}
    \begin{center}
    color
    \end{center}
    
    \end{minipage}};
    % Text Node
    \draw (281,74) node [anchor=north] [inner sep=0.75pt]   [align=left] {\begin{minipage}[lt]{26.5pt}\setlength\topsep{0pt}
    \begin{center}
    head:\\NP
    \end{center}
    
    \end{minipage}};
    % Text Node
    \draw (151,75) node [anchor=north] [inner sep=0.75pt]   [align=left] {\begin{minipage}[lt]{18.62pt}\setlength\topsep{0pt}
    \begin{center}
    size
    \end{center}
    
    \end{minipage}};
    % Text Node
    \draw (215.57,44.61) node [anchor=south] [inner sep=0.75pt]   [align=left] {\begin{minipage}[lt]{17.31pt}\setlength\topsep{0pt}
    \begin{center}
    NP
    \end{center}
    
    \end{minipage}};
    
    
\end{tikzpicture}
    
    
    }
    \caption{Tree diagram of (\ref{ex:grammatical.np.adjective.1})}
    \label{fig:grammatical.np.adjective.1}
\end{figure}

\begin{theorybox}{The term \term{noun}}{noun-definition}
    Similar to the case of the verb (\prettyref{box:verb-definition}),
    the exact meaning of \term{a noun} is also ambiguous.
    Unlike the case of the verb, however,
    in Mandarin, the morphological noun word contains no inflectional formative,
    so we have two definitions of the term \term{noun},
    one comparable to definition (b) in \prettyref{box:verb-definition},
    one comparable to definition (c).
    The main inconsistency between (b) and (c), as is discussed in \prettyref{box:verb-definition},
    is that certain idioms fall under (c) but not (b).
    See \prettyref{sec:grammatical.lexicon.idiom}.
\end{theorybox}

In this particular case, 
if we swap 大 and 白, the result is acceptable (\ref{ex:grammatical.np.adjective.2};
whether the diminutive suffix 子 can be kept has regional variances,
as the suffix tends to be dropped by speakers outside northern China),
but means something completely different:
白大褂 has a fixed, non-compositional meaning \translate{lab coat}.
The irregularity of the structure of (\ref{ex:grammatical.np.adjective.2})
and its fixed meaning suggests that (\ref{ex:grammatical.np.adjective.2}) has undergone syntactic fossilization
and is now a dephrasal compound (\prettyref{box:dephrasal-compound}),
while (\ref{ex:grammatical.np.adjective.1}) is a \ac{np}.

\begin{exe}
    \ex\label{ex:grammatical.np.adjective.2} 白大褂
\end{exe}

The historical origin of 白大褂 likely is syntactic fossilization
of the regular adjective modification phrase [[白]_{\text{color adj}} [大褂]_{\text{NP}}]_{\text{NP}},
in which 大褂 in turn is a dephrasal compound.
The structure of (\ref{ex:grammatical.np.adjective.2})
therefore can be represented as \prettyref{fig:grammatical.np.adjective.2}.

\begin{figure}[H]
    \centering
    {
        \centering
        \begin{tikzpicture}[x=0.75pt,y=0.75pt,yscale=-0.8,xscale=0.8]
    %uncomment if require: \path (0,341); %set diagram left start at 0, and has height of 341
    
    %Straight Lines [id:da6713211557880338] 
    \draw    (241.33,180.61) -- (241.33,215.61) ;
    %Straight Lines [id:da23215636633605796] 
    \draw    (300.57,113.61) -- (353,134) ;
    %Straight Lines [id:da5661834479615483] 
    \draw    (245,134) -- (300.57,113.61) ;
    %Straight Lines [id:da4956781666176777] 
    \draw    (169.33,113.61) -- (168.33,215.61) ;
    %Straight Lines [id:da37838072273073653] 
    \draw    (233.57,42.61) -- (299,66) ;
    %Straight Lines [id:da47624080247562606] 
    \draw    (169,67) -- (233.57,42.61) ;
    %Straight Lines [id:da7080823814281227] 
    \draw    (352.33,180.61) -- (352.33,215.61) ;
    
    % Text Node
    \draw (241.33,218.61) node [anchor=north] [inner sep=0.75pt]   [align=left] {大};
    % Text Node
    \draw (352.33,218.61) node [anchor=north] [inner sep=0.75pt]   [align=left] {褂};
    % Text Node
    \draw (168.33,218.61) node [anchor=north] [inner sep=0.75pt]   [align=left] {白};
    % Text Node
    \draw (353,137) node [anchor=north] [inner sep=0.75pt]   [align=left] {\begin{minipage}[lt]{26.5pt}\setlength\topsep{0pt}
    \begin{center}
    head:\\noun
    \end{center}
    
    \end{minipage}};
    % Text Node
    \draw (245,137) node [anchor=north] [inner sep=0.75pt]   [align=left] {\begin{minipage}[lt]{21.45pt}\setlength\topsep{0pt}
    \begin{center}
    size:\\Adj
    \end{center}
    
    \end{minipage}};
    % Text Node
    \draw (299,69) node [anchor=north] [inner sep=0.75pt]   [align=left] {\begin{minipage}[lt]{200.82pt}\setlength\topsep{0pt}
    \begin{center}
    head:\\dephrasal noun from NP
    \end{center}
    
    \end{minipage}};
    % Text Node
    \draw (169,70) node [anchor=north] [inner sep=0.75pt]   [align=left] {\begin{minipage}[lt]{100.09pt}\setlength\topsep{0pt}
    \begin{center}
    color:\\Adj
    \end{center}
    
    \end{minipage}};
    % Text Node
    \draw (233.57,39.61) node [anchor=south] [inner sep=0.75pt]   [align=left] {\begin{minipage}[lt]{200.82pt}\setlength\topsep{0pt}
    \begin{center}
    dephrasal noun from NP
    \end{center}
    
    \end{minipage}};
    
    
    \end{tikzpicture}
    
    }
    \caption{Tree diagram of (\ref{ex:grammatical.np.adjective.2})}
    \label{fig:grammatical.np.adjective.2}
\end{figure}

\subsection{Argument structure}\label{sec:grammatical.np.complementation}

In many works, forms like (\ref{ex:grammatical.np.complementation.1}) are often called compounds:
the three branches, 研究生, 招生 and 工作 are simply put together,
and the meaning is transparent.
There is no structure-based universally accepted definition of the term \term{compound},
so it is impossible to know if the term is being used correctly in these works.
What we do know is that (\ref{ex:grammatical.np.complementation.1})
is most comparable to so-called glued-up \acp{vp} (\prettyref{sec:grammatical.clause.core-vp.derivation.glue}).

First, we have evidence for argument structures. 
We first note that the pre-head position can be filled by a large number of forms
(\ref{ex:grammatical.np.complementation.1.2}).
Further, we note that (\ref{ex:grammatical.np.complementation.1-alternation.1}) sounds fine,
while (\ref{ex:grammatical.np.complementation.1-alternation.2}) is not acceptable.
This means 招生 \translate{(student) recruitment} first selects the patient
(i.e. students who are recruited)
and then selects the agent (i.e. who recruits students),
and this order is reflected by the linear order.

\begin{exe}
    \ex\label{ex:grammatical.np.complementation.1}
    \gll 研究生 招生 工作 \\
    research-student recruit-student work \\
    \glt\translate{Graduate student recruitment} 

    \ex\label{ex:grammatical.np.complementation.1.2}\begin{xlist}
        \ex 学生招生工作
        \ex 学员招生工作
        \ex 本科生招生工作
        \ex 女飞行学员招生工作
    \end{xlist}
    
    \ex\label{ex:grammatical.np.complementation.1-alternation} \begin{xlist}
        \ex \label{ex:grammatical.np.complementation.1-alternation.1}
        \gll 高校 研究生 招生 工作 \\
        high-school research-student recruit-student work \\
        \glt\translate{Recruitment of graduate students by higher education institutions}%
        \footnote{
            In Standard Mandarin, 高校 means \translate{higher school} 
            or \translate{higher education institutions}, i.e. colleges.
            High schools are translated as 高级中学 \translate{lit. high-grade middle schools},
            abbreviated as 高中.
            Not to be confused with 高校 in Japanese,
            which does mean \translate{highschools}.
        }

        \ex \label{ex:grammatical.np.complementation.1-alternation.2}
        \gll *研究生 高校 招生 工作 \\
        research-student high-school recruit-student work \\
        \glt Intended word-by-word translation: \translate{Recruitment by higher education institutions of graduate students}
    \end{xlist}
\end{exe}

We in turn note that 工作 in (\ref{ex:grammatical.np.complementation.1}) takes 研究生招生 as its argument.
We know this first because 工作 can be replaced by many other forms 
(\ref{ex:grammatical.np.complementation.3}),
and the test in (\ref{ex:grammatical.np.complementation.1-alternation})
applies to all examples in (\ref{ex:grammatical.np.complementation.3}) equally.
This means licensing of the argument 研究生 is done by 招生 alone,
and not by 工作.
This analysis also makes sense semantically:
研究生招生 is an action, and 研究生招生工作 further nominalizes it.
We therefore has \prettyref{fig:grammatical.np.complementation.1-alternation.1}.

\begin{exe}
    \ex\label{ex:grammatical.np.complementation.3} \begin{xlist}
        \ex\label{ex:grammatical.np.complementation.3.1} 研究生招生工作
        \ex 研究生招生任务
        \ex 研究所招生计划
    \end{xlist}
\end{exe}

\begin{figure}[H]
    \centering
    {
        \small
        \begin{tikzpicture}[x=0.75pt,y=0.75pt,yscale=-0.8,xscale=0.8]
    %uncomment if require: \path (0,407); %set diagram left start at 0, and has height of 407
    
    %Straight Lines [id:da16375746371124433] 
    \draw    (335.33,334) -- (335.33,376.89) ;
    %Straight Lines [id:da08764546411311858] 
    \draw    (233.33,333) -- (215.33,375.89) ;
    %Straight Lines [id:da8880589776804259] 
    \draw    (284.57,259.89) -- (335,287) ;
    %Straight Lines [id:da2918181436609215] 
    \draw    (233,287) -- (284.57,259.89) ;
    %Straight Lines [id:da43085298722325416] 
    \draw    (206.57,187.89) -- (283,213) ;
    %Straight Lines [id:da1049713657424457] 
    \draw    (130,213) -- (206.57,187.89) ;
    %Straight Lines [id:da8061026652202706] 
    \draw    (128.33,259.89) -- (115.33,375.89) ;
    %Straight Lines [id:da3958745690581922] 
    \draw    (128.33,259.89) -- (145.33,375.89) ;
    %Straight Lines [id:da3222281404892736] 
    \draw    (115.33,375.89) -- (145.33,375.89) ;
    %Straight Lines [id:da7614353029737813] 
    \draw    (233.33,333) -- (246.33,375.89) ;
    %Straight Lines [id:da30862512747001924] 
    \draw    (215.33,375.89) -- (246.33,375.89) ;
    %Straight Lines [id:da5895026659250475] 
    \draw    (416.57,186.89) -- (416.33,376.89) ;
    %Straight Lines [id:da02286446825514532] 
    \draw    (318.57,112.89) -- (413.57,141.89) ;
    %Straight Lines [id:da5275163387032914] 
    \draw    (210,143) -- (318.57,112.89) ;
    
    % Text Node
    \draw (230.83,378.89) node [anchor=north] [inner sep=0.75pt]   [align=left] {研究生};
    % Text Node
    \draw (335.33,379.89) node [anchor=north] [inner sep=0.75pt]   [align=left] {招生};
    % Text Node
    \draw (416.33,379.89) node [anchor=north] [inner sep=0.75pt]   [align=left] {工作};
    % Text Node
    \draw (335,290) node [anchor=north] [inner sep=0.75pt]   [align=left] {\begin{minipage}[lt]{26.5pt}\setlength\topsep{0pt}
    \begin{center}
    head:\\noun
    \end{center}
    
    \end{minipage}};
    % Text Node
    \draw (233,290) node [anchor=north] [inner sep=0.75pt]   [align=left] {\begin{minipage}[lt]{38.39pt}\setlength\topsep{0pt}
    \begin{center}
    patient:\\nominal
    \end{center}
    
    \end{minipage}};
    % Text Node
    \draw (283,216) node [anchor=north] [inner sep=0.75pt]   [align=left] {\begin{minipage}[lt]{38.39pt}\setlength\topsep{0pt}
    \begin{center}
    head:\\nominal
    \end{center}
    
    \end{minipage}};
    % Text Node
    \draw (130,216) node [anchor=north] [inner sep=0.75pt]   [align=left] {\begin{minipage}[lt]{38.39pt}\setlength\topsep{0pt}
    \begin{center}
    agent:\\nominal
    \end{center}
    
    \end{minipage}};
    % Text Node
    \draw (130.33,378.89) node [anchor=north] [inner sep=0.75pt]   [align=left] {高校};
    % Text Node
    \draw (210,146) node [anchor=north] [inner sep=0.75pt]   [align=left] {\begin{minipage}[lt]{38.39pt}\setlength\topsep{0pt}
    \begin{center}
    action:\\nominal
    \end{center}
    
    \end{minipage}};
    % Text Node
    \draw (413.57,144.89) node [anchor=north] [inner sep=0.75pt]   [align=left] {\begin{minipage}[lt]{26.5pt}\setlength\topsep{0pt}
    \begin{center}
    head:\\noun
    \end{center}
    
    \end{minipage}};
    % Text Node
    \draw (318.57,109.89) node [anchor=south] [inner sep=0.75pt]   [align=left] {\begin{minipage}[lt]{38.39pt}\setlength\topsep{0pt}
    \begin{center}
    nominal
    \end{center}
    
    \end{minipage}};
    
    
    \end{tikzpicture}
    
    }
    \caption{Tree diagram of (\ref{ex:grammatical.np.complementation.1-alternation.1})}
    \label{fig:grammatical.np.complementation.1-alternation.1}
\end{figure}

What makes (\ref{ex:grammatical.np.complementation.1}) intuitively ``compound-like'' (whatever the term means) is that
the argument(s) it selects is expected to be about \emph{classes} of objects and can't be referential.
That's to say, (\ref{ex:grammatical.np.complementation.2})
is \emph{not} acceptable, because the \ac{np} 我那个同学 \translate{that classmate of mine}
refers to a certain person, while arguments licensed in (\ref{ex:grammatical.np.complementation.1})
can only refer to a certain culturally meaningful set of things or people.

\begin{exe}
    \ex\label{ex:grammatical.np.complementation.2} *我那个同学招生工作
\end{exe}

Finally, it is possible for arguments to be enriched into full fledged noun phrases.
Mandarin does not have counterparts of English post-nominal \form{his playing [of the national anthem]},
and the full fledged argument can only be promoted to the possessor position,
as is illustrated in (\ref{ex:grammatical.np.complementation.4}).

\begin{exe}
    \ex\label{ex:grammatical.np.complementation.4} 这个学校的研究生招生工作混乱得很
\end{exe}

\subsection{Derivation}\label{sec:grammatical.np.derivation}

We finally arrives at the core of the noun phrase.
Similarly to what we do in \prettyref{sec:grammatical.clause.core-vp.derivation},
we explore syntactic processes before any argument or attributive is licensed.

\subsubsection{Dephrasal forms}

We have already demonstrated that forms 大褂 and 白大褂 in \prettyref{fig:grammatical.np.adjective.2}
have undergone syntactic fossilization and are now dephrasal compounds.
Similar syntax-in-word forms are not non-existent in English,
as we have \form{do-gooder} or \form{third worldist}.

招生 in (\ref{ex:grammatical.np.complementation.1}) is another form with sub-argument structure internal parts.
Although it has a predicator-object internal structure (\translate{recruit students}),
this structure says nothing about its valency.
There are verbs with a predicator-object internal structure
that take objects (\ref{ex:grammatical.clause.core-vp.compounding.1}),
and there are nouns with predicator-object internal structure 
that takes one preceding argument (\ref{ex:grammatical.np.complementation.1}):
the predicator-object structure says nothing about how these forms
interact with constituents outside them.
So 招生 in (\ref{ex:grammatical.np.complementation.1})
certainly has undergone dephrasalization and
can be treated as a root most of the time.

\subsubsection{Other types of compounds}\label{sec:grammatical.np.derivation.compounding}

Not all compounds have internal phrasal syntax.
Some forms contain roots that never head a noun phrase independently.
机 is a prototypical example of this:
in \work{Xiàndài Hànyǔ Cídiǎn}, none usage of this root has a part of speech tag
(except when it is used as a surname).
This prompts many to analyze all forms containing this root as complex words.
This analysis assumes homogeneity of \term{complex words},
and ignores possible diverse structural origin of the so-called bound roots.
It is for instance possible that these ``bound roots'' 
obligatory take arguments like transitive verbs do,
in the way described in \prettyref{sec:grammatical.np.complementation}.
For 机, we have (\ref{ex:grammatical.np.complementation.machine}).
Similarly for 计 we have 血压计, 光度计, 强度计.
A certain kind of argument structures still seems to exist:
机 always takes a glued-up verb phrase as its left branch,
while 计 always takes a nominal representing a quantity.

\begin{exe}
    \ex\label{ex:grammatical.np.complementation.machine}  计算机, 缝纫机, 打字机, 拖拉机
\end{exe}

What make them deviate from \prettyref{sec:grammatical.np.complementation} is that forms like (\ref{ex:grammatical.np.complementation.4}) are not possible
(\ref{ex:grammatical.np.complementation.numerical-calculation-machine}).
More generally, we typically observe that complexity of the ``argument'' seems to be severely constrained.

\begin{exe}
    \ex\label{ex:grammatical.np.complementation.numerical-calculation-machine} *大规模科学计算的机
\end{exe}

Another instance of complexity constraint 
is displayed in (\ref{ex:grammatical.np.compounding.chant-buddha}),
where both 念佛 and 念阿弥陀佛 can be interpreted as glued-up predicator-object forms (\prettyref{sec:grammatical.clause.core-vp.derivation.glue})
and should be able to be taken as nominal arguments
(\prettyref{sec:grammatical.clause.core-vp.derivation.glue}, \ref{ex:grammatical.clause.core-vp.compounding.2});
in reality, though, only the shorter form (\ref{ex:grammatical.np.compounding.chant-buddha.1}) is acceptable,
while the longer form (\ref{ex:grammatical.np.compounding.chant-buddha.2}) is not.
This fact cannot be explained by fossilization,
as these roots can regularly form new forms.
Therefore, the structure building mechanism responsible for (\ref{ex:grammatical.np.compounding.chant-buddha.1})
is different from constructions in \prettyref{sec:grammatical.np.complementation}.

\begin{exe}
    \ex\label{ex:grammatical.np.compounding.chant-buddha}
    \begin{xlist}
        \ex\label{ex:grammatical.np.compounding.chant-buddha.1} {} [念佛] 堂 
        \ex\label{ex:grammatical.np.compounding.chant-buddha.2} {} *[念阿弥陀佛] 堂 
    \end{xlist}
\end{exe}

Based on the evidence above, we maintain that these forms are instances
of constructions ``smaller'' than those in \prettyref{sec:grammatical.np.complementation}.
Forms in (\ref{ex:grammatical.np.complementation.machine})
therefore may be known as (real, as opposed to those in \prettyref{sec:grammatical.np.complementation}) \term{compounds}
(cf. discussions in \prettyref{box:dephrasal-compound}).
``Real compounding'' is quite active in Mandarin:
all the three head nouns in (\ref{ex:grammatical.np.complementation.3}),
namely 工作, 任务 and 计划, fall under this type.
We note that the internal structures of the three forms do not predict their valency,%
\footnote{
    工作, for instance, looks verbal according to its internal structure,
    and it does have a verbal usage, which however is intransitive
    and takes no argument, unlike its nominal usage in (\ref{ex:grammatical.np.complementation.3}).
}
which justifies calling them compounds (\prettyref{box:dephrasal-compound}).

\subsubsection{Affixation}

Affixation is much more active in the nominal system in Mandarin
than it is in the verbal system.
(\ref{ex:grammatical.np.affixation.1}, \ref{ex:grammatical.np.affixation.2})
illustrate how a nominalization suffix turns a verb phrase into a noun.

\begin{exe}
    \ex\label{ex:grammatical.np.affixation.1} 驯兽师
    \ex\label{ex:grammatical.np.affixation.2} 定义等价性
\end{exe}

We note that the valency of 等价 in (\ref{ex:grammatical.np.affixation.2})
seems to be preserved after nominalization.
The structural parallelism is shown in (\ref{ex:grammatical.np.affixation.2-var});
in particular, the contrast between (\ref{ex:grammatical.np.affixation.2-var.1})
and (\ref{ex:grammatical.np.affixation.2-var.2})
is consistent with what we see in (\ref{ex:grammatical.np.complementation.2}) in 
\prettyref{sec:grammatical.np.complementation}.
We may say that the argument structure (but not a full nucleus clause) is formed and nominalized as a whole
\citep{kornfilt2011afterword}.
The \ac{np} 这两个定义 has to be promoted to the possessor position
and hence 的 cannot be omitted, leading to unacceptability of (\ref{ex:grammatical.np.affixation.2-var.2})
(cf. discussions on \ref{ex:grammatical.np.complementation.4}).

\begin{exe}
    \ex\label{ex:grammatical.np.affixation.2-var} \begin{xlist}
        \ex {} [这两个定义等价]_{\text{clause}}
        \ex\label{ex:grammatical.np.affixation.2-var.1} {} [这两个定义的等价性]_{\text{\ac{np}}}
        \ex\label{ex:grammatical.np.affixation.2-var.2} {} *这两个定义等价性
    \end{xlist}
\end{exe}

Mandarin also has prefixes in the nominal system, like 老 in 老鼠.
Reduplication also applies to nouns, as in 团团伙伙.

The distinction between compounding and affixation 
is essentially the distinction between content and grammatical morphemes.
性 in (\ref{ex:grammatical.np.affixation.2-var}) has to be an affix,
as it is incorporated into the grammar of Mandarin (more specifically, nominalization).
A clear distinction is not always possible or necessary,
as the progress of grammaticalization from the former to the latter varies
among different speakers.

\subsection{Complexity constraints}\label{sec:grammatical.np.complexity}

Despite the rich grammatical categories and functions revealed by the sketch above,
\acp{np} in actual Mandarin texts seems to be constrained in complexity.
It is rare, if not impossible, for \form{de}-less adjectives and complementation to appear at the same time.
This means word-to-word translation of long noun phrases in English is often \emph{impossible},
if not awkward (\prettyref{sec:translation.long-noun-phrases}).

\section{Structure of the lexicon}\label{sec:grammatical.lexicon}

After a survey of the grammatical system of Mandarin Chinese,
we examine what the lexicon has to feed into the grammar.
\concept{Lexicalization} means a form being stored in the lexicon,
with a (semi-)fixed surface form and/or meaning.
Note that the two aspects of lexicalization can be limitedly independent.
In the discussions above, the most important instance of lexicalization of forms
is probably fixed valency
(\prettyref{sec:grammatical.clause.core-vp.transitivity}, \prettyref{sec:grammatical.np.complementation}),
which does not necessarily correlate with unpredictable meanings:
the meaning of (\ref{ex:grammatical.clause.core-vp.compounding.1}) is largely predictable
once a reader knows the meaning of the roots.
Further, entries stored in the lexicon are not necessarily what are intuitively considered words.
In the above sections, we examine the structure of the lexicon of Mandarin.

\subsection{Roots}\label{sec:grammatical.lexicon.roots}

Roots that are productively used in (real) compounding 
(\prettyref{sec:grammatical.clause.core-vp.derivation.compounding},
\prettyref{sec:grammatical.np.derivation.compounding})
that themselves do not act as heads of noun phrases or clauses
by definition cannot be classified as nouns or verbs.

\begin{infobox}{Cross-linguistic significance}{root-cross-linguistic}
    Another field where the role of \term{roots} is being debated is Semitic linguistics.
    It should be noted, however, that in Semitic languages,
    the debate is more about morphological \term{realization}:
    the distinction between so-called \term{stem-} and \term{root-}based analyses
    is more about the underlying mechanism of non-concatenative morphology
    and whether in the mental lexicon of Arabic or Hebrew,
    the consonants in a root are indeed stored in a line.
\end{infobox}

Traditionally, roots are classified into bound or free ones.
This distinction assumes a clear, unambiguous definition of wordhood.

\subsection{Lexemes}\label{sec:grammatical.lexicon.words}

It is not likely that roots alone are the only things stored in the lexicon.
Consider 吃. It has the ability to head a clause,
but never the ability to head a noun phrase.
This fact -- that 吃 \emph{cannot} be understood as referring the action of eating in a nominal environment -- has to be dictated by the lexicon,
together with the fact that 吃 typically takes an object.
Therefore, 吃-in-verbal-environment -- that's to say, 吃 as a \term{verb} or more clearly as a \term{verb lexeme} (\prettyref{box:verb-definition}) -- 
is a lexical entry in Mandarin.

Similarly we have \term{nouns} stored in the lexicon.


\subsection{Idiomization}\label{sec:grammatical.lexicon.idiom}

Certain forms are formed by prototypical phrasal syntax
but have established meanings.
吃饭, for instance, 

\subsection{Lexicography}

The complicated structure of the Mandarin lexicon has consequences in dictionary editing.
现代汉语词典, which is the dictionary that defines standard usages of Mandarin,
is organized according to Chinese characters first.
Each character's ``meanings'',
i.e. the morphemes it may represent,
are first listed under the entry of the character.
If a ``meaning'' is categorized and can be used as a single morphological word,
a part of speech label like \category{noun} or \category{verb} is given.
Otherwise, if it is a bound root that only appears in compounds,
no part of speech tag is given.
A list of words and idioms containing the character are then given.
Most of them are disyllabic, and they are often with part of speech tags.
Longer idioms, usually phrasal or even clausal,
do not have part of speech tags.

\chapter{Nouns}

\section{Bound roots}

Some morphemes are clearly content morphemes and are productive,
but they never appear on their own as (minimal) noun phrases.
A good example is 工 \translate{worker}.
Its meaning cannot be summarized as something like \translate{agent of an action},
so it cannot be a grammatical formative.
It is also productive:
we have 劳工 \translate{laborer}, 操作工 \translate{operator}, 纺织工 \translate{textile worker}, 车工 \translate{(lathe) turner}, 船工 \translate{boat worker}, 电工 \translate{electrician}, and much more.

\begin{exe}
    \ex 水温计
    \ex 油温计
\end{exe}

It is generally not possible to let a bound root directly merge with a form with complicated internal structures.
This seems to suggest that Mandarin has ``real'' compounding in contrast to what is commonly referred to as compounding but is actually nominal complementation (\prettyref{sec:grammatical.np.derivation.compounding}).

It should be noted that what are considered bound roots very considerably depending on the register.
In scientific writing, for instance, 鲸 is \emph{not} a bound root.
In casual speech though 鲸鱼, a form that is misleading if read compositionally, is more common,
indicating that the ability of 鲸 to be an independent \ac{np} has been lost.



\chapter{The verbal complex}\label{chap:verbal-complex}

\section{Types of verbal complexes}\label{sec:verbal-complex.linear-order}

Cross-linguistically, a \concept{verbal complex} is
any sequence consisting of the verb(s) in a clause,
relevant grammatical markers (tense, aspect, personal pronouns, etc.)
and sometimes incorporated constituents,
which has a fixed internal structure and cannot be interrupted by other constituents.
Different types of verbal complexes may arise from
heterogeneous morphosyntactic mechanisms.

Several types of verbal complexes exist in Mandarin.
Mandarin is generally recognized as a language lacking inflectional morphology,
but it is clear that certain types of constructions
can only be explained in terms of something like agglutinative inflectional morphology.
We refrain from calling these uninterrupted sequences simply as \term{inflected verbs},
because the ``affixes'' in them can sometimes reorder or even split from the rest of the sequences,
and some sequences are longer than a prosodic word
(\ref{ex:verbal-complex.linear-order.reorder.1}).
The term \term{verbal complex} is therefore used in this grammar.

\subsection{The verb stem-complement-aspect chain}\label{sec:verbal-complex.v-c-a}

The first type contains three uncontroversial slots:
the verb stem,
the so-called \concept{verbal complement},
known as 补语 in Chinese linguistic community (\prettyref{sec:verbal-complex.complement}),%
\footnote{
    The term 补语 literally means \translate{complementation speech}, 
    and is therefore often translated as \term{complement}.
    In this note I use the term \term{complement}
    to refer to grammatical constituents that are somehow more closely 
    related to the lexical head, 
    and I choose the (somehow tedious but explicit) term 
    \term{non-argument complement}.
},
and the aspectual marker.
The verb stem itself may contain suffixes.
Usually there is at most one suffix, often a verbalizer like 化.

Whether or not a suffix appear is due to syntactic and sometimes semantic and pragmatic factors
(see the relevant sections on each formative).
It's possible for only one of the two or even none of it to appear.

(\ref{ex:verbal-complex.linear-order.1}) is an example in which 
the aspect marker, the verbal complement, and the verbalization marker in the verb stem all appear.

\begin{exe}
    \ex\label{ex:verbal-complex.linear-order.1}
    \gll \dots 并且 企业 [[数字 [化]_{\text{derivation}}]_{\text{stem}} [完]_{\text{complement}} [了]_{\text{aspectual}}]_{\text{verbal complex}} 之后 还 不一定 赚 钱 \dots \\
    and enterprise digit \category{verbalize} \category{asp} \category{asp} after 
   \category{todo} not.necessarily earn money \\ 
   \glt \translate{\dots and enterprises, after digitalizing, do not necessarily earn a lot\dots}
   \label{ex:hua-wan-le-1}
\end{exe}

\subsubsection{Directional complement as clitics?}\label{sec:verbal-complex.v-c-a.clitic}

When the verbal complement is disyllabic, which actually contains two formatives instead of one
(\prettyref{sec:verbal-complex.directional}),
the order of the aspect marker and the directional complement can be swapped
(\ref{ex:verbal-complex.linear-order.reorder.1}).
This seems to be a consequence of prosody,
as the verbal complex in (\ref{ex:verbal-complex.linear-order.reorder.1.1})
can be neatly divided into two disyllabic prosodic words (站了/起来),
and therefore a reordering is desirable.
This suggests that the so-called suffixes, i.e. 起来 and 了,
are actually enclitics at the current stage of Mandarin:
the verb stem-verbal complement-aspect marker order is not purely determined by morphological considerations,
but partly by phonology and prosody.

\begin{exe}
    \ex\label{ex:verbal-complex.linear-order.reorder.1} \begin{xlist}
        \ex\label{ex:verbal-complex.linear-order.reorder.1.1} 张三站了起来
        \ex\label{ex:verbal-complex.linear-order.reorder.1.2} 张三站起来了
    \end{xlist}
\end{exe}

We note that (\ref{ex:verbal-complex.linear-order.reorder.1.2}) allows a figurative reading:
\translate{Zhang San finally rises up and stands proudly and independently},
while (\ref{ex:verbal-complex.linear-order.reorder.1.1})
does not allow such an interpretation.
This however can be attributed to 站起来 being first semantically lexicalized as an idiom
and then syntactically fossilized,
and now forming a synchronic verb root with a given meaning.

\subsection{Incorporation of personal pronouns}\label{sec:verbal-complex.v-c-a.incorporation}

An interesting phenomenon, related to the possible clitic status of the verbal complement and the aspect marker,
is that personal pronouns sometimes can be incorporated into the verbal complex.

\begin{exe}
    \ex 我的一个朋友告诉我了这个消息
\end{exe}

\subsection{The V-Neg-V construction}\label{sec:verbal-complex.v-neg-v}

A type of verbal complexes, often known as the V-不-V construction,
is used to form interrogative sentences
(\ref{ex:sec:verbal-complex.v-neg-v.1}).
When the verb is disyllabic, often only the first syllable is kept in the first copy of the verb in the verbal complex
(\ref{ex:sec:verbal-complex.v-neg-v.2}).

\begin{exe}
    \ex\label{ex:sec:verbal-complex.v-neg-v.1} 你到底吃不吃
    \ex\label{ex:sec:verbal-complex.v-neg-v.2} 你打不打算去黄山?
\end{exe}

The ``verb'' appearing in the V-Neg-V construction
can also be an auxiliary (\ref{ex:sec:verbal-complex.v-neg-v.3};
for why recognizing an auxiliary class see \prettyref{sec:grammatical.clause.tam}).

\begin{exe}
    \ex\label{ex:sec:verbal-complex.v-neg-v.3} 你能不能过一阵子帮我一个忙
\end{exe}

From acceptability of (\ref{ex:sec:verbal-complex.v-neg-v.3}),
it follows that coexistence of a V-Neg-V verbal complex
and a verb stem-complement-aspect verbal complex (\prettyref{sec:verbal-complex.v-c-a})
is possible (\ref{ex:sec:verbal-complex.v-neg-v.4}).

\begin{exe}
    \ex\label{ex:sec:verbal-complex.v-neg-v.4} 你能不能过一阵子送给我点土特产
\end{exe}

\subsection{Verbal complexes with manner and consequence complements}
\label{sec:verbal-complex.de}

Some verbal complexes merely contain the suffix 得.


\begin{exe}
    \ex\label{ex:verbal-complex.de.1} 他骑马骑得气喘吁吁
    \ex 他骑得马气喘吁吁
\end{exe}

\subsection{Verb copying}

We already observe verb copying in (\ref{ex:verbal-complex.de.1}),
which is obligatory, or otherwise the 

\begin{exe}
    \ex 这帮流氓打架打赢了,可是还是进了牢房
\end{exe}

\section{Wordhood}\label{sec:verbal-complex.wordhood}

All constructions shown in \prettyref{sec:verbal-complex.linear-order}
can be reasonably considered as morphological words,
although we can argue that the suffixes are enclitics
(\prettyref{sec:verbal-complex.v-c-a}).
It is instructive to see how various criteria for wordhood apply to these constructions,
and whether a clear derivation/inflection distinction can be established.

\subsection{Lexicalization}

\begin{exe}
    \ex 约翰考上了哈佛
\end{exe}

We know nowadays it's possible to enter Harvard without standardized tests,
so 考上 can be completely factually wrong,
and even for most students who do have standardized test scores,
the admission process is holistic, and 考上, if taken literally,
can be rather misleading.
College admission in China, on the other hand, is mostly based on the standardized College Entrance Exam,
and 考上 has naturally gained an idiomized meaning of \translate{get admitted to},
which explains its usage here.

It is not clear whether 考上 should be considered to form a synchronic root here,
as it can be replaced by 考取 or 考进 freely.
What happens is more likely that 考 plus an accomplishment verbal complement is considered to form an idiom.
Still, we note that 考取 sounds more archaic.
There might be several layers of idiomization.

\subsection{Is valency alternation marked by verbal complement derivation?}\label{sec:verbal-complex.derivation.valency}

We note that the preposition of a prepositional argument  
(\prettyref{sec:grammatical.clause.core-vp.internal}) in a clause
can be incorporated into the verbal complex of that clause
as the verbal complement (\prettyref{sec:verbal-complex.prepositional}),
and if the verbal complex is to be recognized as a morphological word,
the verbal complement indicates the existence of an internal argument
(\ref{ex:verbal-complex.derivation.valency.1.1}).
Now, if another verbal complex does not license an internal argument 
(\ref{ex:verbal-complex.derivation.valency.1.2}),
we find that the verbal complex is marking the valency of the clause.

\begin{exe}
    \ex\label{ex:verbal-complex.derivation.valency.1} \begin{xlist}
        \ex\label{ex:verbal-complex.derivation.valency.1.1} 他住在了广州
        \ex\label{ex:verbal-complex.derivation.valency.1.2} 他去过广州,感觉不错,就住下了
    \end{xlist}
\end{exe}

The situation here is quite comparable to that of Latin preverbs.
Whether this is to be called derivation is more terminological,
as it depends on the definition of the verb stem.
We do note that the verbal complement slot is sometimes filled by 


\section{Verb derivation}

\subsection{Historical compounding}\label{sec:verbal-complex.derivation.compound}

驱逐、计算


\section{Verbal complements}\label{sec:verbal-complex.complement}

The category of verbal complements is rather heterogeneous,
its boundary (expectedly) being somewhat unclear;
it includes verbal complements or in other words complex predicates, 
complement clauses, 
and oblique arguments. 

\subsection{Directional complements}\label{sec:verbal-complex.directional}

The directional complement is either monosyllabic or disyllabic
(\ref{ex:verbal-complex.linear-order.reorder.1}).
In the latter case, 

\subsection{Resultative complements}\label{sec:verbal-complex.resultative}

\subsection{Time and location complements}\label{sec:verbal-complex.prepositional}

In some clauses, the verbal complement slot is filled by a preposition
from an argument marked by that preposition.

\begin{exe}
    \ex 摄影师卖掉伦敦市中心大房子,竟然住在了这里
\end{exe}

\subsection{Other things commonly known as complements}

\section{Aspect markers}

\begin{exe}
    \ex 标语贴在墙上 
    \ex 标语已经在墙上贴着了
\end{exe}
 
this means the preposition 在 actually is morphologically merged with the verb 贴, 
or otherwise we are unable to explain why 
in the first example, 着 can never appear, 
while in the second example, 着 can appear.

Although 着 can appear in a matrix clause, 
its distribution is wider in temporal adverbials. 

*他笑着。
他[笑着]走了进来

\chapter{Verb frames}\label{chap:verb-frames}

\section{\category{become} and \category{cause}-\category{become} verb frames}

\begin{exe}
    \ex\label{ex:valency.become} \begin{xlist}
        \ex\label{ex:valency.become.1} 茶泡好了
        \ex 我泡好茶了
    \end{xlist}

    \ex\label{ex:valency.become-only}
    \begin{xlist}
        \ex\label{ex:valency.become-only.1} 这只猫死了
        \ex *坏人死了这只猫
    \end{xlist}
\end{exe}

\subsection{Subject or topic?}\label{sec:valency.become.subject-or-topic}

At the first glance, the alternation in (\ref{ex:valency.become})
can be explained by assuming that the 
subject argument who prepared tea is omitted (\prettyref{ex:grammatical.clause.subject.no-topic.1}),
and the object 茶 is \emph{topicalized}.
This analysis eliminates the necessity of postulating certain valency alternation devices,
and is consistent with \citet{lapolla20091}.
This analysis however has to be rejected,
because subject omission, i.e. pro-drop,
is otherwise only used when the subject is known,
while in (\ref{ex:valency.become.1}),
the subject is indefinite and \emph{unknown}.

This is particularly clear when we do not front the object 茶.
(\ref{ex:valency.become.1.not-information-structure.1}) sounds awkward,
because since 茶 stays in the core \ac{vp},
the subject position is empty, and yet its reference cannot be resolved
without pragmatic information.
On the contrary, when (\ref{ex:valency.become.1.not-information-structure.1})
is placed in a conversational context,
e.g. (\ref{ex:valency.become.1.not-information-structure.2}),
it is perfectly acceptable, because in (\ref{ex:valency.become.1.not-information-structure.2}),
the reference of the null subject can be resolved as the conversational participant,
i.e. the target of the question:
\translate{have \emph{you} prepared the tea?}

\begin{exe}
    \ex \begin{xlist}
        \ex\label{ex:valency.become.1.not-information-structure.1}
        \gll \#[泡 好]_{\text{verbal complex}} 茶 了 \\
        soak good tea \category{sfp} \\
        \glt Intended reading: \translate{Someone has prepared the tea.}
    
        \ex\label{ex:valency.become.1.not-information-structure.2} 泡好茶了吗?
    \end{xlist}
\end{exe}

Contrasting (\ref{ex:valency.become.1})
and (\ref{ex:valency.become.1.not-information-structure.1}),
we find that although the former involves an object fronting operation,
it is \emph{less} pragmatically loaded than the latter.
This can be explained by the assumption that 
the object fronting operation in (\ref{ex:valency.become})
is an argument structure alternation,
which involves minimal information structure operation,
while (\ref{ex:valency.become.1.not-information-structure.1})
assumes an identifiable null subject.
Therefore, we consider the alternation in (\ref{ex:valency.become})
a valency alternation, and not pro-drop plus topicalization.
This also justifies the existence of a subject position in Mandarin clauses.



\section{Obligatory \form{bǎ}}

\begin{exe}
    \ex 把参考手册当小说看
\end{exe}

\chapter{Tense, aspect, modality}

\section{A possible tense system?}\label{sec:tam.tense}

From a surface form-oriented perspective, 
Mandarin lacks the category of tense -- 
all semantic tense information is 
expressed by time adverbs 
and the default values determined by the aspect.

\subsection{The location of some time adverbs}

There exists a position for time adverbs that precedes modality auxiliaries.
This makes it slightly different from that 
of uncontroversial peripheral arguments
(\ref{ex:tame.tense.adverb-position.1}).

\begin{exe}
    \ex\label{ex:tame.tense.adverb-position.1}
    \gll 我 [明天] 可能 能 和 你 讨论 一下 \\
    1 tomorrow \category{aux} \category{aux} with 2 discuss for.a.while \\
    \glt\translate{I can have a discussion with you tomorrow.}
\end{exe}

This position seems to be the position that some adverbs most frequently appear in.
Alternation of this order results in 
clauses that are either slightly infelicitous 
or pragmatically marked. 
In (\ref{ex:tame.tense.adverb-position.1.var}),
we move 明天 \translate{tomorrow} rightwards,
and we find that the more rightwards it goes,
the less felicitous the sentence becomes.

We also note that there can be at most one 明天-like ``time point'' adverb.
之后 \translate{later} has the same distribution with 明天.
It and 明天 both appearing is categorically rejected 
(\ref{ex:tame.tense.adverb-position.1.3}).

\begin{exe}
    \ex\label{ex:tame.tense.adverb-position.1.var} \begin{xlist}
        \ex 我[明天]可能能和你讨论一下
        \ex 我可能[明天]能和你讨论一下
        \ex ?我可能能[明天]和你讨论一下
        \ex ??我可能能和你明天讨论一下
    \end{xlist}

    \ex\label{ex:tame.tense.adverb-position.1.3} *我之后可能能和你明天讨论一下
\end{exe}

The distribution of 明天 

Interestingly, not all temporal adverbs are able to move to the tense-like position: 

\begin{exe}
    \ex \begin{xlist}
        \ex ?我在周四可能能和你讨论一下
        \ex 我可能能在周四和你讨论一下
    \end{xlist}
\end{exe}




A further piece of evidence hinting at a higher position for time adverbs 
is that they are easier to topicalize.
It seems they are closer to the subject, 
instead of ordinary peripheral arguments.

\begin{exe}
    \ex 明天我可能能和你在办公室讨论一下
    \ex 我明天可能能和你在办公室讨论一下
    \ex \#在办公室,我明天可能能和你讨论一下
\end{exe}


\subsection{Interpretation of the time}

If a clause has no time adverb at all,
it is assumed that the situation described by the clause 
is the case \emph{now}.
Thus (\ref{ex:tame.tense.interpretation.1}) is obligatorily interpreted as something happen in the present.
We note that making the subject a deceased person does \emph{not}
enable a past interpretation:
(\ref{ex:tame.tense.interpretation.2.1}) is semantically infelicitous,
because Ji Xianlin, an important philologist, has passed away,
and (\ref{ex:tame.tense.interpretation.2.1}) implies that he still lives here now.
The intended past interpretation has to be enforced by introducing a time adverb
(\ref{ex:tame.tense.interpretation.2.2}).

\begin{exe}
    \ex\label{ex:tame.tense.interpretation.1} 
    \gll 我 住 在 这里 \\
    1 live at here \\
    \glt\translate{I live here.}

    \ex\label{ex:tame.tense.interpretation.2} 
    \begin{xlist}
        \ex\label{ex:tame.tense.interpretation.2.1} 
        \gll \#季羡林 住 在 这里  \\
        \category{name} live at here \\
        \glt\translate{Ji Xianlin lives here. (Intended meaning: Ji Xianlin lived here.)}

        \ex\label{ex:tame.tense.interpretation.2.2} 
        \gll 季羡林 曾经 住 在 这里  \\
        \category{name} previously live at here \\
        \glt\translate{Ji Xianlin previously lived here.}
    \end{xlist}
\end{exe}

\citet{sybesma2007whether} argues that the behaviors of the alleged tense system
are comparable to those of the Dutch tense system,
namely that the tense value is determined by agreement with the tense-like adverb.

\section{Frequency}

\begin{exe}
    \ex 我之后可能每天会来 

    \ex *我每天可能能和你讨论一下
    \ex 我可能每天能和你讨论一下 
    \ex 我可能能每天和你讨论一下 
    \ex ?我可能能和你每天讨论一下
    \ex 我之后可能能和你每天讨论一下 -- two time adverbs, one about frequency, one about time point 
\end{exe}

\begin{exe}
    \ex 我每天都在床上哭
    \ex ?我每天在床上都哭
\end{exe}

\chapter{Information structure}

\section{Marker of topicalization}


As is mentioned in \prettyref{sec:grammatical.clause.high-level},
topicalization is prevalent in Mandarin grammar,
and involves fronting of the topic before the rest of the clause, i.e. the comment.
\prettyref{fig:grammatical.clause.high-level.topic.1} is an example of topicalization.
Besides that example, the topic can be marked by a particle
(\prettyref{ex:information.topic.marker.1}).

\begin{exe}
    \ex\label{ex:information.topic.marker.1} 我啊,最讨厌言行不一的人
\end{exe}

\section{Subject being topicalized}\label{sec:topic.subject}

\begin{exe}
    \ex 我明天跟你可能能在我的办公室讨论一下
\end{exe}

\section{(Absence of) dangling topics}\label{sec:topic-subject}

Some people, like \citet[\citesec{7.1}]{zhudexigrammar},
equate \term{subject} with \term{topic} in Mandarin grammar.
Some (especially those from the functional-typological tradition) go further 
and assert that ``the notion of the subject (as the position of the most agentive argument) 
isn't grammaticalized in Mandarin Chinese'',
and therefore the topic-comment construction 
is construed as simply the syntactic coding of aboutness,
and this base-generated and syntactically unconstrained topic 
is called a ``dangling topic''.
This view is rejected in this note,
because such accounts usually end up in severe overgeneration. 
Here I briefly summarize \citet{sih2000topic}'s argumentation.

\subsection{Type 1: Idiomatic phrasal predicate looking like a comment}\label{sec:clause.dangling-topic.1}

In the first type of ``dangling topic'',
it's impossible for any \acs{np} in the comment to be syntactically related to the topic
(\ref{ex:dayuchixiaoyu}, \ref{ex:nikankanwo}).
Such cases however should be analyzed as instances of the
subject-predicate construction,
where the predicate is a dephrasalized clause.

We notice that in such examples, 
the ``comment'' often has already undergone fossilization of various degrees.
Changing the comment usually makes the sentences much less felicitous 
(\ref{ex:dayuchixiaoyu-2}),
at best highly marked.
This is strange if the attested examples are topic-comment constructions,
but makes sense if dephrasalization is needed 
to put the clause 大鱼吃小鱼 etc. to the ``comment'' position.

Thus, in \eqref{ex:dayuchixiaoyu} and \eqref{ex:nikankanwo},
the so-called topic is an ordinary subject,
and the so-called comment is a predicate.
The meaning of the result of dephrasalization 
may be compared with the English colloquial 
\form{I was like, \dots} construction.

\begin{exe}
    \ex\label{ex:dayuchixiaoyu} 他们[大鱼吃小鱼](,厮杀成一片)
    \ex\label{ex:nikankanwo} 他们[你看看我我看看你]
    
    \ex\label{ex:dayuchixiaoyu-2} \begin{xlist}
        \ex *他们小鱼咬大鱼 
        \ex *他们虾米啃泥底
    \end{xlist}
\end{exe}

\subsection{Type 2: Quantificational adverbial looking like the inner subject}

The second type of ``dangling topic'' is like \eqref{ex:shui-dou-bu-pa}.
A topic-comment analysis of \eqref{ex:shui-dou-bu-pa} 

\begin{exe}
    \ex\label{ex:shui-dou-bu-pa} \gll 他们 谁 都 不 怕 \\
    3pl who even \acs{neg} fear \\
    \glt \translate{They don't fear anyone.}
\end{exe}

\subsection{Type 3: Ellipsis leaving a subject and one predicate}

Some people accept \eqref{ex:nasuofangzixingkuimeixiaxue}.
Here the \acs{np} 那所房子 definitely doesn't come from the words following it,
and is therefore recognized as a topic by some (TODO: ref). 
Note, however, that 幸亏 serves as a clause linker outside \eqref{ex:nasuofangzixingkuimeixiaxue}:
\eqref{ex:xingkui-buran-ex} is a demonstration of the 幸亏……不然…… linking construction,
and we also have its topicalized version \eqref{ex:xingkui-buran-fronted}. (TODO: whether this is parenthesis)
We also know in a clause linking construction,
often one clause can be omitted in the utterance because it's content can be easily inferred (TODO: ref).
So now the origin of \eqref{ex:nasuofangzixingkuimeixiaxue} is clear:
We can get it by omitting the second clause in the comment part of \eqref{ex:xingkui-buran-fronted}.
Indeed, if we replace 幸亏 by anything that is adverbial but not a clause linker,
the resulting sentence -- which now contains a real dangling topic -- is not grammatical.

\begin{exe}
    \ex \label{ex:nasuofangzixingkuimeixiaxue} \gll \% 那 所 房子 幸亏 没 下雪 \\
    {} that \acs{classify} house fortunate \acs{neg} snow \\
    \glt \translate{For that house, fortunately it didn't snow (or otherwise something bad would happen).}

    \ex\label{ex:xingkui-buran-ex} \gll [幸亏] 去年 没 下雪 , [不然] 那 所 房子 早就 塌 了 \\
    fortunate last.year \acs{neg} snow {} otherwise that \acs{classify} house already collapse \acs{sfp} \\
    \glt \translate{Fortunately it didn't snow last year, or otherwise that house has already collapsed.}

    \ex\label{ex:xingkui-buran-fronted} 
    \gll [ 那 所 房子 ]_{\text{topic}} [ 幸亏 去年 没 下雪 , 不然 早就 塌 了 ]_{\text{comment}} \\
    {} that \acs{classify} house {} {} fortunate last.year \acs{neg} snow {}  otherwise already collapse \acs{sfp} \\
\end{exe}

\subsection{Type 4: Extraction from prepositional adverbials}

\eqref{ex:zhejianshinibunengjiumafantayigeren} in \prettyref{sec:topicalization-of-preposition-objects} 
is sometimes regarded as an instance of the dangling topic construction.
However, as is shown in \prettyref{sec:topicalization-of-preposition-objects},
it may just be from topicalization of an \acs{np} in an adverbial,
with the preposition (and/or the locative particle) removed.

\subsection{Type 5: Nominal predicate}

\begin{exe}
    \ex 这种青菜一斤三十块钱
\end{exe}

\subsection{Type 6: Locational adverbial mistaken for the subject}

\begin{exe}
    \ex \gll \% 物价 纽约 最 贵  \\
    {} price New.York most expensive \\
    \glt \translate{The price in New York is the most expensive.}
\end{exe}

\subsection{Tentative conclusion}

The conclusion is all topics in Chinese are closely linked to a position in the comment,
be it a core argument position or a peripheral one.
So the notion of dangling topics is to be rejected in Mandarin grammar,
and we can always recover the ``canonical'' i.e. non-topic-comment clause
from a topic-comment construction.
After this, if the canonical clause can be divided into an \acs{np}
or a complement clause and a verbal constituent following it,
we can uncontroversially say the first is the subject while the second is the predicate. (TODO: predicate def)
So equating the subject with the topic is also wrong.

It's possible to find the semantic role of the subject isn't agentive;
in this case I assert there is a valency changing mechanism here.

\begin{infobox}{What to expect when people talk about the subject or the topic}{subject-topic}
    Unfortunately, despite the syntactic tests presented above,
    there are still many people -- even many native speakers -- 
    promoting the idea that the Mandarin topic has nothing different with the subject.
    Here is a list of TODO: ref
\end{infobox}

\section{Pseudo-cleft construction}

\begin{exe}
    \ex 疾病是每个人都不想碰上的
\end{exe}

\chapter{Sentence final particles}

\section{Enumeration of possible particles}\label{sec:sfp.all}

\chapter{Subordination and coordination}

\section{Conditional constructions}



\chapter{Advices for translation}

\section{Long noun phrase}\label{sec:translation.long-noun-phrases}

Mandarin generally does not permit overly complicated \acp{np} (\prettyref{sec:grammatical.np.complexity}),
and when it does, these \acp{np} are usually considered suboptimal.
This means when translating long \acp{np} to Mandarin,
one needs to break a clause containing complex \acp{np} into several ones.

\begin{exe}
    \ex\begin{xlist}
        \ex \form{the current extremely difficult situtaion caused by ignorance of the previous administration obliges us to take a radically different approach}
        \ex 由于上一任政府的疏忽大意导致的极为困难的目前的情况要求我们采取一种全新的方法
        \ex 上一任政府疏忽大意,导致了如今的情况极为困难。因此,我们需要采取一种新的方法
    \end{xlist}
\end{exe}

\chapter{Conclusion and discussion}

\section{Traditional controversies}

\subsection{Separable verbs}

It has long been noticed that in Mandarin,
a verb can be split into two, with syntactic constituents appearing between the two.
Our analyses reveal that so-called \concept{separable verbs} or \concept{verb ionization} are heterogeneous.
Some instances of verb ionization are because the ``verb'' in question
is either a verb phrase (\prettyref{sec:grammatical.clause.core-vp.derivation.object-bound}) or has a verb phrase counterpart (\prettyref{sec:grammatical.clause.core-vp.derivation.glue}).
Others are due to incorporation of 

\section{Describing Chinese language(s)}

The significance of having a unified framework of description of morphosyntax of Standard Mandarin
goes far beyond documentation of a single language.
The current situation is that most, if not all, non-Standard Mandarin Sinitic languages
are rapidly dying out.
Having a solid point of reference in describing them is important.
Just as a grammar of French can inspire 

\printbibliography

\end{document}