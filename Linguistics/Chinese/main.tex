\documentclass[UTF8, a4paper, oneside, scheme=plain]{ctexbook}

\usepackage{geometry}
\usepackage{titling}
\usepackage{titlesec}
\usepackage{paralist}
\usepackage{footnote}
\usepackage{enumerate}
\usepackage{amsmath, amssymb, amsthm}
\usepackage{gb4e}
\noautomath
\usepackage{bbm}
\usepackage{soul}
\usepackage{graphicx}
\usepackage{siunitx}
\usepackage[table,xcdraw]{xcolor}
\usepackage{tikz}
\usepackage[ruled, vlined, linesnumbered, noend]{algorithm2e}
\usepackage{xr-hyper}
\usepackage[colorlinks]{hyperref} % linkcolor=black, anchorcolor=black, citecolor=black, filecolor=black
\usepackage[most]{tcolorbox}
\usepackage{caption}
\usepackage{subcaption}
\usepackage{booktabs}
\usepackage[figuresright]{rotating}
\usepackage{acro}
\usepackage[round]{natbib} 
\usepackage{prettyref}
\usepackage{imakeidx}
\usepackage{subfiles}

\geometry{left=3.18cm,right=3.18cm,top=2.54cm,bottom=2.54cm}
\titlespacing{\paragraph}{0pt}{1pt}{10pt}[20pt]
\setlength{\droptitle}{-5em}

\DeclareMathOperator{\timeorder}{\mathcal{T}}
\DeclareMathOperator{\diag}{diag}
\DeclareMathOperator{\legpoly}{P}
\DeclareMathOperator{\primevalue}{P}
\DeclareMathOperator{\sgn}{sgn}
\newcommand*{\ii}{\mathrm{i}}
\newcommand*{\ee}{\mathrm{e}}
\newcommand*{\const}{\mathrm{const}}
\newcommand*{\suchthat}{\quad \text{s.t.} \quad}
\newcommand*{\argmin}{\arg\min}
\newcommand*{\argmax}{\arg\max}
\newcommand*{\normalorder}[1]{: #1 :}
\newcommand*{\pair}[1]{\langle #1 \rangle}
\newcommand*{\fd}[1]{\mathcal{D} #1}

\newcommand*{\citesec}[1]{\S~{#1}}
\newcommand*{\citechap}[1]{chap.~{#1}}
\newcommand*{\citefig}[1]{Fig.~{#1}}
\newcommand*{\citetable}[1]{Table~{#1}}

\newrefformat{sec}{\citesec{\ref{#1}}}
\newrefformat{fig}{\citefig{\ref{#1}}}
\newrefformat{tbl}{\citetable{\ref{#1}}}
\newrefformat{chap}{\citechap{\ref{#1}}}

\usetikzlibrary{arrows,shapes,positioning}
\usetikzlibrary{arrows.meta}
\usetikzlibrary{decorations.markings}
\tikzstyle arrowstyle=[scale=1]
\tikzstyle directed=[postaction={decorate,decoration={markings,
    mark=at position .5 with {\arrow[arrowstyle]{stealth}}}}]
\tikzstyle ray=[directed, thick]
\tikzstyle dot=[anchor=base,fill,circle,inner sep=1pt]


\tcbuselibrary{skins, breakable, theorems}

\newtcbtheorem[number within=chapter]{infobox}{Box}%
  {colback=blue!5,colframe=blue!65,fonttitle=\bfseries, breakable}{back}

\newcommand*{\concept}[1]{\textbf{#1}}

\numberwithin{equation}{chapter}

\numberwithin{equation}{chapter}

\DeclareAcronym{blt}{short = BLT, long = Basic Linguistic Theory}
\DeclareAcronym{cgel}{short = CGEL, long = The Cambridge Grammar of the English Language}
\DeclareAcronym{dm}{short = DM, long = Distributed Morphology}
\DeclareAcronym{tag}{long = Tree-adjoining grammar, short = TAG}
\DeclareAcronym{sfp}{long = sentence final particle, short = SFP}
\DeclareAcronym{vp}{long = verb phrase, short = VP}

\setcounter{secnumdepth}{3}
\counterwithin{xnumi}{chapter} % the chngcntr way (preferred)
\counterwithin{exx}{chapter} % reset example counter every chapter
\exewidth{(1.234)} % leave enough room for the example number

\makeindex

\title{Aspects of Mandarin Morphosyntax}
\author{Jinyuan Wu}

\begin{document}

\maketitle

\automath

\part{Introduction}

\chapter{Introduction to Modern Standard Mandarin Chinese}

\subfile{chapters/introduction.tex}

\chapter{Preliminaries and the theoretical framework}

\subfile{chapters/theory-framework.tex}

\chapter{Syntactic overview and previous works}

\subfile{chapters/grammar-sketch.tex}

\part{Nouns and noun phrases}

\chapter{Noun phrases: an overview}

\subfile{chapters/noun-phrase.tex}

\chapter{The adjective category and adjective phrases}

\subfile{chapters/adjectives.tex}

\part{Simple clauses}

\chapter{Structure of simple clauses: an overview}\label{chap:basic-clause-structure}

\subfile{chapters/clause-structure.tex}

\chapter{The verb category}

\subfile{chapters/verbs.tex}

\chapter{Aspectual system(s) and other possible TAME categories}\label{chap:aspect}

\subfile{chapters/aspectual.tex}

\chapter{Subjects and objects}% TODO: what can be subject, what can be object

\subfile{chapters/subject-object.tex}

\chapter{Non-argument complements}

\subfile{chapters/non-argument-complements.tex}

\chapter{Serial verb constructions} %TODO: relation with coverbs

\chapter{Prepositions and coverbs in clauses}\label{chap:coverbs}

\subfile{chapters/coverbs.tex}

\chapter{Disposal constructions}\label{chap:disposal}

\chapter{Passive constructions and the 被-construction}\label{chap:passive}

\subfile{chapters/passive.tex}

\chapter{Sentence final particles}

% TODO: 包括了着过,不包括“吗”等

\subfile{chapters/sentence-final-particles.tex}

\part{Sentence types, pragmatics and the information structure}% TODO: the term "pragmatics" - it actually means pragmatics-oriented syntactic structures

\part{Multiple clause constructions}

\chapter{Relative constructions}\label{chap:relative}

\chapter{Complement clauses}\label{chap:comp-clause}

\chapter{Coordination}

\part{Prosody}

\chapter{Overview of the Chinese prosody structure}\label{chap:prosody-overview}

\part{Word formation and etymology}

% TODO: A不A结构
% TODO: 和合体、动漫翻译腔(语气词大量出现)、学术翻译腔、文白夹杂、早期现代白话作品

\bibliographystyle{plainnat}
\bibliography{references/grammars.bib,references/controversy.bib,references/generative.bib,references/aspects.bib,references/general-typology.bib}

\printindex

\end{document}