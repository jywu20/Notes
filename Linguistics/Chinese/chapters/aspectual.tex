This chapter is devoted to the aspectual system in Chinese, 
and the controversies about whether there is only one aspectual system in Chinese, 
whether there is a simple tense system, and whether Chinese shows certain degree of evidentiality.

\section{The post-verbal aspectual suffixes}\label{sec:post-verbal-aspect}

Verb suffixes that are usually regarded as marking aspect include 了, 着 and 过 . 
It should be noted that the aspectual 了 is not the same as the sentence final particle 了 \prettyref{sec:sfp-le}, 
both for distributional and semantic reasons.
The aspectual 了 is in a post-verbal position and is about aspectual information, 
while the sentence final particle 了 is typically analyzed as indicating something new has occurred 
\citep[\citesec{16.2.1}]{zhudexigrammar}.
The former is usually denoted as \homo{了}{1}, and the latter \homo{了}{2} \citep{peng2005le12}. 
What I discuss in this chapter is \homo{了}{1} and not \homo{了}{2}. 

A sketchy description about 了, 着 and 过 is that 
了 means \concept{perfective}, 着 means \concept{durative}, and 过 means \concept{experiential} 
\cite[\citechap{6}]{li1989mandarin}.

There are also verb prefixes that are considered aspectual.

The final device that appears aspectual is reduplication of the verb. 
It roughly means delimitative, i.e. `do something but not with an excessive amount'.
