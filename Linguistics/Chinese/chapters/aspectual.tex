This chapter is devoted to the aspectual system in Chinese, 
and the controversies about whether there is only one aspectual system in Chinese, 
whether there is a simple tense system, and whether Chinese shows certain degree of evidentiality.

\section{Aspect marking devices}\label{sec:aspect-marking-device}

\subsection{The post-verbal aspectual suffixes}\label{sec:post-verbal-aspect}

Verb suffixes that are usually regarded as marking aspect include 了, 着 and 过 . 
It should be noted that the aspectual 了 is not the same as the sentence final particle 了 \prettyref{sec:sfp-le}, 
both for distributional and semantic reasons.
The aspectual 了 is in a post-verbal position and is about aspectual information, 
while the sentence final particle 了 is typically analyzed as indicating something new has occurred 
\citep[\citesec{16.2.1}]{zhudexigrammar}.
The former is usually denoted as \homo{了}{1}, and the latter \homo{了}{2} \citep{peng2005le12}. 
What I discuss in this chapter is \homo{了}{1} and not \homo{了}{2}. 

A sketchy description about 了, 着 and 过 is that 
了 means \concept{perfective}, 着 means \concept{durative} or \concept{imperfective}, 
and 过 means \concept{experiential} 
\cite[\citechap{6}]{li1989mandarin}.

\subsection{The verb prefixes}\label{sec:pre-verbal-aspect}

There are also verb prefixes that are considered aspectual. 
The prefix 在 as in the following example obviously is aspectual:
\begin{exe}
    \ex \gll 我 [在]_{\text{aspectual prefix}} 穿 衣服 。\\
    1sg ZAI wear clothes \\
    \glt `I am wearing clothes (= in the process of getting dressed).'
\end{exe} 
Note that though the above example is durative, it has a slightly different meaning with 着, 
as is illustrated by the following example: 
\begin{exe}
    \ex \gll 我 穿 着 衣服 。 \\
    1sg wear ZHE clothes \\
    \glt `I am wearing clothes (so I am not naked, but I have already finish the dressing-up process).'
\end{exe}

在 frequently appears together with 正, forming 正在. 
Actually, the \emph{present progressive tense} in English is translated as 正在进行时 in Chinese.

\subsection{Delimitative duplication}\label{sec:delimitative-duplication}

The final device that appears aspectual is reduplication of the verb. 
It roughly means delimitative, i.e. `do something but not with an excessive amount'.

So here is a list of aspectual markers in Chinese: 
\begin{itemize}
    \item The perfective \homo{了}{1}
    \item The imperfective 在
    \item The durative 着
    \item The experiential 过
    \item The delimitative duplication
\end{itemize}

\subsection{Compositional usage of aspectual markers}

Markers in \prettyref{sec:aspect-marking-device} are not in complementary distribution.
Some markers are indeed in complementary distribution. 
If a nucleus predicate contains at least one of 在 and 着, 
then it never contains 过 or \homo{了}{1}.
Also, no more than two aspectual markers appear.
These rules are categorical. Other combinations are more or less acceptable,
provided the ordering of the aspectual markers is correct.

过 and 了 can be used together in the order of 过了: 
\begin{exe}
    \ex \gll 他 吃 过 了 饭 。\\
    3sg eat GUO LE meal \\
    \glt `He has eaten.'
\end{exe}
在 and 着 may also be used together:
\begin{exe}
    \ex \gll 我们 还 在 走 着 。 \\
    1pl still ZAI walk ZHE \\
    \glt `We are still walking.'
\end{exe}
Note, however, the [在V着] construction is less productive than 过了. It does not work for certain verbs:
\begin{exe}
    \ex *他在穿着衣服。
\end{exe} 

This fact may be a motivation of \citet[\citesec{6.9}]{li1989mandarin}'s 
rejection of the aspectual status of 着 and 过 --
they claim they are ``manner and experience indicators''.
着 and 过 \emph{are} manner and experience indicators -- 
but since there are more than one systems named ``aspect'' (\ac{blt} 3.15), 
it may be helpful to think of them as being a part of a aspectual system, 
which is good for cross-linguistic comparison.

Finally, 正, despite not an aspectual marker, works together with any non-delimitative aspects,
though the frequencies may differ.
Here is a list of examples where 正 appears:
\begin{exe}
    \ex \begin{xlist}
        \ex 他正在吃饭。
        \ex 他正吃着饭(,忽然听到窗外传来一声巨响)。
        \ex 他正在吃着饭?(呢)。
        \ex ??他正吃饭。
        \ex 他正吃过饭(,邻居就找上门了)。
        \ex 他正吃过了饭(,邻居就找上门了)。
        \ex 他吃过了饭。
    \end{xlist}
\end{exe}

The alignment of aspectual markers is therefore shown in \prettyref{tbl:aspectual-affix}.

\section{The perfective \homo{了}{1}}



This chapter is devoted to the aspectual system in Chinese, 
and the controversies about whether there is only one aspectual system in Chinese, 
whether there is a simple tense system, and whether Chinese shows certain degree of evidentiality.

\section{Aspect marking devices}\label{sec:aspect-marking-device}

\subsection{Post-verbal aspectual suffixes}\label{sec:post-verbal-aspect}

Verb suffixes that are usually regarded as marking aspect include 了, 着 and 过 . 
It should be noted that the aspectual 了 is not the same as the sentence final particle 了 \prettyref{sec:sfp-le}, 
both for distributional and semantic reasons.
The aspectual 了 is in a post-verbal position and is about aspectual information, 
while the sentence final particle 了 is typically analyzed as indicating something new has occurred 
\citep[\citesec{16.2.1}]{zhudexigrammar}.
The former is usually denoted as \homo{了}{1}, and the latter \homo{了}{2} \citep{peng2005le12}. 
What I discuss in this chapter is \homo{了}{1} and not \homo{了}{2}. 

A sketchy description about 了, 着 and 过 is that 
了 means \concept{perfective}, 着 means \concept{durative} or \concept{imperfective}, 
and 过 means \concept{experiential} 
\cite[\citechap{6}]{li1989mandarin}.

\subsection{Verb prefixes}\label{sec:pre-verbal-aspect}

There are also verb prefixes that are considered aspectual. 
The prefix 在 as in the following example obviously is aspectual:
\begin{exe}
    \ex \gll 我 [在]_{\text{aspectual prefix}} 穿 衣服 。\\
    1sg ZAI wear clothes \\
    \glt `I am wearing clothes. (I am in the process of getting dressed, but I have not finished yet.)'
\end{exe} 
Note that though the above example is durative, it has a slightly different meaning with 着, 
as is illustrated by the following example: 
\begin{exe}
    \ex \gll 我 穿 着 衣服 。 \\
    1sg wear ZHE clothes \\
    \glt `I am wearing clothes (so I am not naked).'
\end{exe}

在 frequently appears together with 正, forming 正在. 
Actually, the \emph{present progressive tense} in English is translated as 正在进行时 in Chinese.
正 itself is not an aspectual marker, at least not yet in this stage of Standard Mandarin Chinese.
Its role is more adverbial-like.
This fact can be seen from the following minimal pair:
\begin{exe}
    \ex \begin{xlist}
        \ex 我在穿衣服。
        \ex *我正穿衣服。% TODO: 我正穿衣服也许是正确的,但是是否表示进行时? 或者也许是%?
    \end{xlist}
\end{exe}

\subsection{Delimitative reduplication}\label{sec:delimitative-reduplication}

The final device that appears aspectual is reduplication of the verb. 
It roughly means delimitative, i.e. `do something but not with an excessive amount'.

So here is a list of aspectual markers in Chinese: 
\begin{itemize}
    \item The perfective \homo{了}{1}
    \item The imperfective 在
    \item The durative 着
    \item The experiential 过
    \item The delimitative duplication
\end{itemize}

\subsection{Compositional usage of aspectual markers}

Markers in \prettyref{sec:aspect-marking-device} are not in contrastive distribution.
Some markers are indeed in contrastive distribution. 
If a nucleus predicate contains at least one of 在 and 着, 
then it never contains 过 or \homo{了}{1}.
Also, no more than two aspectual markers appear.
These rules are categorical. Other combinations are more or less acceptable,
provided the ordering of the aspectual markers is correct.

过 and 了 can be used together in the order of 过了: 
\begin{exe}
    \ex \gll 他 吃 过 了 饭 。\\
    3sg eat GUO LE meal \\
    \glt `He has eaten.'
\end{exe}
在 and 着 may also be used together by someone, though this construction appears much less frequently:
\begin{exe}
    \ex \gll 我们 还 在 走 着 路。 \\
    1pl still ZAI walk ZHE road \\
    \glt `We are still walking.'
\end{exe}
Note, however, the [在V着] construction is less productive than 过了. It does not work for certain verbs:
\begin{exe}
    \ex *他在穿着衣服。
\end{exe} 

This fact may be a motivation of \citet[\citesec{6.9}]{po2015chinese}'s 
rejection of the aspectual status of 着 and 过 --
they claim they are ``manner and experience indicators''.
着 and 过 \emph{are} manner and experience indicators -- 
but since there are more than one systems named ``aspect'' (\ac{blt} \citesec{3.15}), 
it may be helpful to think of them as being a part of a aspectual system, 
which is good for cross-linguistic comparison.

Finally, 正, despite not an aspectual marker, works together with any non-delimitative aspects,
though the frequencies may differ.
Here is a list of examples where 正 appears:
\begin{exe}
    \ex \begin{xlist}
        \ex 他正在吃饭。
        \ex 他正吃着饭(,忽然听到窗外传来一声巨响)。
        \ex 他正在吃着饭?(呢)。
        \ex ??他正吃饭。
        \ex 他正吃过饭(,邻居就找上门了)。
        \ex 他正吃过了饭(,邻居就找上门了)。
        \ex 他吃过了饭。
    \end{xlist}
\end{exe}

The alignment of aspectual markers is therefore shown in \prettyref{tbl:aspectual-affix}.

\section{The perfective \homo{了}{1}}



This chapter is devoted to the aspectual system in Chinese, 
and the controversies about whether there is only one aspectual system in Chinese, 
whether there is a simple tense system, and whether Chinese shows certain degree of evidentiality.

\section{Aspect marking devices}\label{sec:aspect-marking-device}

\subsection{Post-verbal aspectual suffixes}\label{sec:post-verbal-aspect}

Verb suffixes that are usually regarded as marking aspect include 了, 着 and 过 . 
It should be noted that the aspectual 了 is not the same as the sentence final particle 了 \prettyref{sec:sfp-le}, 
both for distributional and semantic reasons.
The aspectual 了 is in a post-verbal position and is about aspectual information, 
while the sentence final particle 了 is typically analyzed as indicating something new has occurred 
\citep[\citesec{16.2.1}]{zhudexigrammar}.
The former is usually denoted as \homo{了}{1}, and the latter \homo{了}{2} \citep{peng2005le12}. 
What I discuss in this chapter is \homo{了}{1} and not \homo{了}{2}. 

A sketchy description about 了, 着 and 过 is that 
了 means \concept{perfective}, 着 means \concept{durative} or \concept{imperfective}, 
and 过 means \concept{experiential} 
\cite[\citechap{6}]{li1989mandarin}.

\subsection{Verb prefixes}\label{sec:pre-verbal-aspect}

There are also verb prefixes that are considered aspectual. 
The prefix 在 as in the following example obviously is aspectual:
\begin{exe}
    \ex \gll 我 [在]_{\text{aspectual prefix}} 穿 衣服 。\\
    1sg ZAI wear clothes \\
    \glt `I am wearing clothes. (I am in the process of getting dressed, but I have not finished yet.)'
\end{exe} 
Note that though the above example is durative, it has a slightly different meaning with 着, 
as is illustrated by the following example: 
\begin{exe}
    \ex \gll 我 穿 着 衣服 。 \\
    1sg wear ZHE clothes \\
    \glt `I am wearing clothes (so I am not naked).'
\end{exe}

在 frequently appears together with 正, forming 正在. 
Actually, the \emph{present progressive tense} in English is translated as 正在进行时 in Chinese.
正 itself is not an aspectual marker, at least not yet in this stage of Standard Mandarin Chinese.
Its role is more adverbial-like.
This fact can be seen from the following minimal pair:
\begin{exe}
    \ex \begin{xlist}
        \ex 我在穿衣服。
        \ex *我正穿衣服。% TODO: 我正穿衣服也许是正确的,但是是否表示进行时? 或者也许是%?
    \end{xlist}
\end{exe}

\subsection{Delimitative reduplication}\label{sec:delimitative-reduplication}

The final device that appears aspectual is reduplication of the verb. 
It roughly means delimitative, i.e. `do something but not with an excessive amount'.

So here is a list of aspectual markers in Chinese: 
\begin{itemize}
    \item The perfective \homo{了}{1}
    \item The imperfective 在
    \item The durative 着
    \item The experiential 过
    \item The delimitative duplication
\end{itemize}

\subsection{Compositional usage of aspectual markers}

Markers in \prettyref{sec:aspect-marking-device} are not in contrastive distribution.
Some markers are indeed in contrastive distribution. 
If a nucleus predicate contains at least one of 在 and 着, 
then it never contains 过 or \homo{了}{1}.
Also, no more than two aspectual markers appear.
These rules are categorical. Other combinations are more or less acceptable,
provided the ordering of the aspectual markers is correct.

过 and 了 can be used together in the order of 过了: 
\begin{exe}
    \ex \gll 他 吃 过 了 饭 。\\
    3sg eat GUO LE meal \\
    \glt `He has eaten.'
\end{exe}
在 and 着 may also be used together by someone, though this construction appears much less frequently:
\begin{exe}
    \ex \gll 我们 还 在 走 着 路。 \\
    1pl still ZAI walk ZHE road \\
    \glt `We are still walking.'
\end{exe}
Note, however, the [在V着] construction is less productive than 过了. It does not work for certain verbs:
\begin{exe}
    \ex *他在穿着衣服。
\end{exe} 

This fact may be a motivation of \citet[\citesec{6.9}]{po2015chinese}'s 
rejection of the aspectual status of 着 and 过 --
they claim they are ``manner and experience indicators''.
着 and 过 \emph{are} manner and experience indicators -- 
but since there are more than one systems named ``aspect'' (\ac{blt} \citesec{3.15}), 
it may be helpful to think of them as being a part of a aspectual system, 
which is good for cross-linguistic comparison.

Finally, 正, despite not an aspectual marker, works together with any non-delimitative aspects,
though the frequencies may differ.
Here is a list of examples where 正 appears:
\begin{exe}
    \ex \begin{xlist}
        \ex 他正在吃饭。
        \ex 他正吃着饭(,忽然听到窗外传来一声巨响)。
        \ex 他正在吃着饭?(呢)。
        \ex ??他正吃饭。
        \ex 他正吃过饭(,邻居就找上门了)。
        \ex 他正吃过了饭(,邻居就找上门了)。
        \ex 他吃过了饭。
    \end{xlist}
\end{exe}

The alignment of aspectual markers is therefore shown in \prettyref{tbl:aspectual-affix}.

\section{The perfective \homo{了}{1}}



This chapter is devoted to the aspectual system in Chinese, 
and the controversies about whether there is only one aspectual system in Chinese, 
whether there is a simple tense system, and whether Chinese shows certain degree of evidentiality.

\section{Aspect marking devices}\label{sec:aspect-marking-device}

\subsection{Post-verbal aspectual suffixes}\label{sec:post-verbal-aspect}

Verb suffixes that are usually regarded as marking aspect include 了, 着 and 过 . 
It should be noted that the aspectual 了 is not the same as the sentence final particle 了 \prettyref{sec:sfp-le}, 
both for distributional and semantic reasons.
The aspectual 了 is in a post-verbal position and is about aspectual information, 
while the sentence final particle 了 is typically analyzed as indicating something new has occurred 
\citep[\citesec{16.2.1}]{zhudexigrammar}.
The former is usually denoted as \homo{了}{1}, and the latter \homo{了}{2} \citep{peng2005le12}. 
What I discuss in this chapter is \homo{了}{1} and not \homo{了}{2}. 

A sketchy description about 了, 着 and 过 is that 
了 means \concept{perfective}, 着 means \concept{durative} or \concept{imperfective}, 
and 过 means \concept{experiential} 
\cite[\citechap{6}]{li1989mandarin}.

\subsection{Verb prefixes}\label{sec:pre-verbal-aspect}

There are also verb prefixes that are considered aspectual. 
The prefix 在 as in the following example obviously is aspectual:
\begin{exe}
    \ex \gll 我 [在]_{\text{aspectual prefix}} 穿 衣服 。\\
    1sg ZAI wear clothes \\
    \glt `I am wearing clothes. (I am in the process of getting dressed, but I have not finished yet.)'
\end{exe} 
Note that though the above example is durative, it has a slightly different meaning with 着, 
as is illustrated by the following example: 
\begin{exe}
    \ex \gll 我 穿 着 衣服 。 \\
    1sg wear ZHE clothes \\
    \glt `I am wearing clothes (so I am not naked).'
\end{exe}

在 frequently appears together with 正, forming 正在. 
Actually, the \emph{present progressive tense} in English is translated as 正在进行时 in Chinese.
正 itself is not an aspectual marker, at least not yet in this stage of Standard Mandarin Chinese.
Its role is more adverbial-like.
This fact can be seen from the following minimal pair:
\begin{exe}
    \ex \begin{xlist}
        \ex 我在穿衣服。
        \ex *我正穿衣服。% TODO: 我正穿衣服也许是正确的,但是是否表示进行时? 或者也许是%?
    \end{xlist}
\end{exe}

\subsection{Delimitative reduplication}\label{sec:delimitative-reduplication}

The final device that appears aspectual is reduplication of the verb. 
It roughly means delimitative, i.e. `do something but not with an excessive amount'.

So here is a list of aspectual markers in Chinese: 
\begin{itemize}
    \item The perfective \homo{了}{1}
    \item The imperfective 在
    \item The durative 着
    \item The experiential 过
    \item The delimitative duplication
\end{itemize}

\subsection{Compositional usage of aspectual markers}

Markers in \prettyref{sec:aspect-marking-device} are not in contrastive distribution.
Some markers are indeed in contrastive distribution. 
If a nucleus predicate contains at least one of 在 and 着, 
then it never contains 过 or \homo{了}{1}.
Also, no more than two aspectual markers appear.
These rules are categorical. Other combinations are more or less acceptable,
provided the ordering of the aspectual markers is correct.

过 and 了 can be used together in the order of 过了: 
\begin{exe}
    \ex \gll 他 吃 过 了 饭 。\\
    3sg eat GUO LE meal \\
    \glt `He has eaten.'
\end{exe}
在 and 着 may also be used together by someone, though this construction appears much less frequently:
\begin{exe}
    \ex \gll 我们 还 在 走 着 路。 \\
    1pl still ZAI walk ZHE road \\
    \glt `We are still walking.'
\end{exe}
Note, however, the [在V着] construction is less productive than 过了. It does not work for certain verbs:
\begin{exe}
    \ex *他在穿着衣服。
\end{exe} 

This fact may be a motivation of \citet[\citesec{6.9}]{po2015chinese}'s 
rejection of the aspectual status of 着 and 过 --
they claim they are ``manner and experience indicators''.
着 and 过 \emph{are} manner and experience indicators -- 
but since there are more than one systems named ``aspect'' (\ac{blt} \citesec{3.15}), 
it may be helpful to think of them as being a part of a aspectual system, 
which is good for cross-linguistic comparison.

Finally, 正, despite not an aspectual marker, works together with any non-delimitative aspects,
though the frequencies may differ.
Here is a list of examples where 正 appears:
\begin{exe}
    \ex \begin{xlist}
        \ex 他正在吃饭。
        \ex 他正吃着饭(,忽然听到窗外传来一声巨响)。
        \ex 他正在吃着饭?(呢)。
        \ex ??他正吃饭。
        \ex 他正吃过饭(,邻居就找上门了)。
        \ex 他正吃过了饭(,邻居就找上门了)。
        \ex 他吃过了饭。
    \end{xlist}
\end{exe}

The alignment of aspectual markers is therefore shown in \prettyref{tbl:aspectual-affix}.

\section{The perfective \homo{了}{1}}



\input{tables/aspectual.tex}