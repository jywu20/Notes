\documentclass[../main.tex]{subfiles}

\begin{document}

% 有一定的可能,此处的complement不是CGEL中的complement-modifier区分中的一部分,而是external modifier一类的东西?

\section{Overview}

\subsection{The name of non-argument complements}\label{sec:complement-name}

In the School Grammar, 补语 or (non-argument) complement in a clause means anything 
that is licensed and selected by the verb 
yet does not fill a typical argument slot. 
From the perspective of argument structure, a non-argument complement does not denote to 
any \emph{object} that is involved into the action in question.
From the perspective of inner structure, a non-argument complement is always predicative 
\citep[\citesec{9.1}]{zhudexigrammar}. % TODO: link
In \ac{cgel} about English, the term \emph{argument} never appears,
because in English, all clausal complements are arguments, 
so I emphasize \emph{non-argument} in the title of this chapter. 

The non-argument complements are indeed complements in the complement-modifier dichotomy. 
Consider, for example, the following examples 
of quantity complements (\prettyref{sec:quantity-complement}):
\begin{exe}
    \ex \begin{xlist}
        \ex 我打了他[一下]。
        \ex 我打了他[一巴掌]。
        \ex 我打了他[一拳]。
        \ex *我打了他[一口]。
        \ex 我咬了他[一下]。
        \ex 我咬了他[一口]。
        \ex *我咬了他[一巴掌]。
        \ex *我咬了他[一拳]。
    \end{xlist}
\end{exe}
From the facts above we conclude that the verb licenses the quantity complement. 
Should a quantity complement appears, its measure word cannot vary arbitrarily. 
The verb 打 allows 下, 巴掌 or 拳 as the measure word of its quantity complement, but it rejects 口. 
On the other hand, the verb 咬 only accepts 下 and 口 and rejects 巴掌 or 拳.
It is ungrammatical to use a complement that is not licensed by the verb, 
though the complement may not be obligatory. 
This is just like complements in NPs in English.

Sometimes a complement is not only \emph{licensed}, but also \emph{required}, just like an argument,
but it is by no means nominal, nor is it a clause, 
and hence should not be seen as an argument (\prettyref{sec:complement-argument-clause}). 
Consider the following examples about location complements (\prettyref{sec:location-complement}):
\begin{exe}
    \ex \begin{xlist}
        \ex 我住在北京。
        \ex *我住。
    \end{xlist}
\end{exe}

A remaining question is whether it is better 
to categorize these so-called non-argument complements as \emph{functional words} like determiners or classifiers,
since their variations are much more limited than arguments.
Direction complements, for example, are limited to particles like 
上, 去, 上去, 下来, etc. (\prettyref{sec:direction-complement}),
which is just the case of classifiers % TODO: reference
and the case of orientation preverbs in Japhug \citep[\citechap{15}]{jacques2021grammar}. 
But on the other hand, a manner or consequence complement may take an inside clause, % TODO: ref
and a time or location complement takes an inside NP, % TODO: ref
so these types of non-argument complements are by all means complements.
Now the problem is non-argument complements except the quantity complement 
have largely complementary distributions,
so it is not reasonable to reject the complement status for direction complements 
while keep the complement status for manner and consequence complements and time and location complements.
Therefore, I follow the tradition in the literature and call all of them complements.

That being said, it should be kept in mind that sometimes the complement functions just like a functional word.
It largely determines the structure of the nucleus predicate % TODO: ref

\subsection{Grammatical properties}

\subsubsection{Constituent order}

Non-argument complements involve highly complicated grammatical phenomena. Sometimes a non-argument 
complement precedes the object, as in the case of result complements (\prettyref{sec:result-complement}):
\begin{exe}
    \ex 熊_{\text{subject}} 拍_{\text{verb stem}} 晕_{{\text{result complement}}} 了_{\text{perfect aspect}} 他_{\text{object}}。
    \label{eq:order-result-complement}
\end{exe}
But for quantity complements (\prettyref{sec:quantity-complement}), the inverse order is attested sometimes:
\begin{exe}
    \ex 熊_{\text{subject}} 拍_{\text{verb stem}}  了_{\text{perfect aspect}} 他_{\text{object}} [三巴掌]_{\text{quantity complement}}。
\end{exe}
For a quantity complement with 巴掌 as the classifier, the verb-complement-object order is somehow strange:
\begin{exe}
    \ex ???熊拍了三巴掌我。
\end{exe}
While for a quantity complement with 个 as the classifier, the verb-complement-object order in \eqref{eq:order-result-complement} is more acceptable:
\begin{exe}
    \ex 小朋友拍了几下球。
    \ex ??小朋友拍了球几下。
\end{exe}

One particular interesting property of non-argument complements is 
that they often trigger double occurrences of the main verb. 
All types of non-argument complements show certain degree of this phenomenon:
\begin{exe}
    \ex 你找这本书找到了吗?(~你找到这本书了吗?)
    \ex 他爬山永远爬不上去。(~\%他永远爬不上去山。)
    \ex 我找这本书就是找不到。(~我就是找不到这本书。)
    \ex 我写文章写不出来啊。(~我写不出来文章啊。)
    \ex 我打球打得胳膊酸痛。(*我打球得胳膊酸痛。)
    \ex 我打球打得篮筐坏了。(*我打球得篮筐坏了。)
    \ex 熊打他连续打了三巴掌。(~熊连续打了他三巴掌。)
    \ex 我敲桌子敲了两下。(~我敲了两下桌子。)
\end{exe}
Sometimes there exists a grammatical counterpart of the above examples with the main verb occurring only once, 
sometimes there does not.

\subsubsection{Interplay between non-argument complements and adverbials}

In the following sections, the interplay between non-argument complements and objects is discussed.
It seems at the first glance that pre-verbal adverbials have no non-trivial interaction with complements:
\begin{exe}
    \ex 我_{\text{subject}} 昨天_{\text{time adverbial}} [在那条街上]_{\text{place adverbial}} [狠狠地]_{\text{manner adverbial}} [打了]_{\text{verb, perfect}} 他_{\text{object}} [一顿]_{\text{complement}} % TODO: Aspect
\end{exe}
There are, however, some non-trivial interplay between adverbials and complements. 
\eqref{ex:no-both-adverbial-and-promoted-complement} is a case in which
a complement seems to be moved to an adverbial position. % TODO: more examples

\subsubsection{Limited compositional usage of non-argument complements}\label{sec:complement-no-composition}

Compositional using of non-argument complements is much more limited. Consider, for example, 
the following sentences:
\begin{exe}
    \ex \begin{xlist}
        \ex 那头熊拍了他[一巴掌]_{\text{quantity complement}}。
        \ex 那头熊拍[晕]_{\text{result complement}}了他。
        \ex[*]{那头熊拍晕了他一巴掌。}\label{ex:paiyun-yibazhang-illegal}
    \end{xlist}
    \label{ex:paiyunle-yibazhang}
\end{exe}
This shows quantity complements (\prettyref{sec:quantity-complement}) and result complements 
(\prettyref{sec:result-complement}) do not always allow compositional appearing. 

Note that it \emph{is} possible sometimes for a quantity complement and a result complement occur together, 
but some subtleties occur in such constructions -- see, for example, \prettyref{sec:complement-non-verb-dependency}.

For the case of \eqref{ex:paiyunle-yibazhang}, one possible way to make them appear together is via the following construction:
\begin{exe}
    \ex 那头熊[一巴掌]_{\text{quantity complement?}}拍[晕]_{\text{result complement}}了他。
\end{exe}
where the quantity complement is promoted to a pre-verbal position. In this construction no more manner adverbial can appear:
\begin{exe}
    \ex[*]{那头熊[用力地]_{\text{manner adverbial}}[一巴掌]_{\text{promoted quantity complement}}拍晕了他。}
    \label{ex:no-both-adverbial-and-promoted-complement}
\end{exe}
while other adverbials can still appear:
\begin{exe}
    \ex 那头熊[昨天]_{\text{time adverbial}} [在那座山下面的小树林里]_{\text{place adverbial}} 一巴掌拍晕了他。
\end{exe}
So it seems the fronted quantity complement is in complementary distribution with the ordinary manner adverbial.
Note that the quantity complement cannot be promoted to the pre-verbal position if it is the only non-argument 
complement:
\begin{exe}
    \ex[*]{那头熊一巴掌拍了他。}
\end{exe}

Some compositional using is indeed possible. Below is the example of a direction complement 
(\prettyref{sec:direction-complement}) and a quantity complement (\prettyref{sec:quantity-complement})
appearing together:
\begin{exe}
    \ex 那头熊打[中]_{\text{direction complement}} 了 他 [两次]_{\text{quantity complement}}。
\end{exe}

\subsubsection{Indirect dependency between complements}\label{sec:complement-non-verb-dependency}

A slightly adjusted version of \eqref{ex:paiyunle-yibazhang} is shown here:
\begin{exe}
    \ex 那头熊拍晕了他三次。
    \label{ex:paiyun-yixia}
\end{exe}
Now compositional usage of a quantity complement and a result complement is possible\dots Or is it?
If investigated more closely, the meaning of \eqref{ex:paiyun-yixia} is \emph{not} the composition 
of 那头熊拍晕了他 and 那头熊拍了他三次. \eqref{ex:paiyun-yixia} means ``the bear squatted him into
coma, and he fell into coma for three times'', while in the clause 那头熊拍了他三次, what happens 
three times is the bear's squat, not any other events. So we find when a quantity complement 
and a result complement occur together, there is dependency relation between the two complements, 
\emph{not} the quantity complement and the main verb. In other words, in \eqref{ex:paiyun-yixia}
the quantity complement is an \emph{indirect} dependent (in the sense of \ac{cgel}, \citesec{5.14.1}). 
It is licensed by the result complement, not the main verb. This explains why \eqref{ex:paiyun-yibazhang-illegal}
is ungrammatical: because 一巴掌 is a grammatical complement of 拍, but not a grammatical complement of 晕.

\subsubsection{Non-canonical subject}

Complements are even related to the role of subject. Consider the following examples:
\begin{exe}
    \ex Result complement
    \begin{xlist}
        \ex 这个电影真是笑死我了。
        \ex 这个电影真是把我笑死了。
        \ex \%我被这个电影笑死了。 % TODO
    \end{xlist}
    \ex Manner and consequence complement 
    \begin{xlist}
        \ex 这条山路走得我累得不行。
        \ex \%这条山路把我走得累得不行。 % TODO
        \ex ??我被这条山路走得累得不行。
    \end{xlist}
\end{exe}

In all these examples, the subject is not filled by the agent -- or is there an agent at all in the main clause?
But the clauses are all active ones. 

\subsubsection{Complement and negation}

Potential complement TODO

\subsection{Classification of non-argument complements}\label{sec:complement-classification}

Here I summarize basic types of non-argument complements. The prototypical semantic functions of 
non-argument complements other than quantity complements are shown in \prettyref{tbl:complement-class}.
Sections from \prettyref{sec:direction-complement} to \prettyref{sec:location-complement} discuss in 
detail the inner structure, external syntactic functions and mutual interactions of these 
complements. Since complements in \prettyref{tbl:complement-class} have complementary distribution,
one way to classify nucleus predicates according to the complements they contain. 

The status of quantity complements (\prettyref{sec:quantity-complement}) is kind of controversial.
Since quantity complements look like nominal arguments and can occur together with other types of 
complements, just like objects do, some authors kick it from the family of complements and 
assign various names to it, mostly quasi-object and time expression. I will discuss these aspects 
at the beginning of \prettyref{sec:quantity-complement}.

\begin{table}
    \caption{Semantic (and then syntactic) classification of non-argument complements besides quantity complements}
    \label{tbl:complement-class}
    \begin{tabular}{cccccc}
    \toprule
    \multicolumn{1}{c}{} & directional          & resultive         & possibility               & \begin{tabular}[c]{@{}l@{}}manner and \\ consequence\end{tabular}               & \begin{tabular}[c]{@{}l@{}}time and \\ location\end{tabular}               \\ \midrule
    factual              & \begin{tabular}[c]{@{}l@{}}direction \\ complement\end{tabular} & \begin{tabular}[c]{@{}l@{}}result complement\end{tabular}  & \cellcolor[HTML]{9B9B9B}- & \begin{tabular}[c]{@{}l@{}}manner and \\ consequence \\ complement\end{tabular} & \begin{tabular}[c]{@{}l@{}}time and \\ location \\ complement\end{tabular} \\
    potential            & \multicolumn{3}{c}{potential complement}                             & \cellcolor[HTML]{9B9B9B}-                                                       & \cellcolor[HTML]{9B9B9B}-     \\ \bottomrule                                            
    \end{tabular}
    \end{table}

\section{Direction complements}\label{sec:direction-complement}

Direction complements are post-verbal particles (\prettyref{sec:clause-constituent-order-overview})
and describe the direction of the action described by the main verb.
Possible direction complements form a smallish class of particles.
A member of this class may be from a small set of monosyllabic movement verbs, % TODO: verb classification
or formally a monosyllabic movement verb but with a different meaning % TODO: 选上课了,这个算什么?
from its ordinary meaning when used as a verb, 
or a disyllabic movement verb made of two monosyllabic movement verbs,
or a construction with a quite similar form of the aforementioned disyllabic movement verb 
but does not stand as a movement verb itself. 

The limited variation range of direction complements makes 
some authors remove them from the family of complements \citep[\citesec{8.5}]{po2015chinese}.
The name \citet{po2015chinese} gives for direction complements is \emph{direction indicator}.
The name used in this book is still \emph{direction complement} because 
direction complements and other types of complements except quantifier complements 
are in complementary distribution and are better classified as complements 
since they fill the position of noncontroversial complements, 
direction complements are sometimes similar to result complements, % TODO: link
and direction complements are closely linked to potential complements 
(\prettyref{sec:direction-potential-complement}).
However, the limited and heterogeneous nature of the range of direction complements 
means it is better to enumerate each possible direction complement, 
just in the same manner people list grammatical words.

\subsection{The monosyllabic 来 and 去}

The two monosyllabic movement verbs, 来 and 去, can fill the direction complement position of movement verbs. %TODO and other types of verbs?
Their meanings as direction complements are close to their meaning as movement verbs:
来 means `moving toward to the given point' and 去 means `moving away from the given point',
where the ``given point'' is frequently the speaker, 
but may also be the listener or some point relevant in the dialogue.
The distributions of 来 and 去 are almost identical. 
The differences include the phenomena that some verbs only occur together one of 来 and 去,
and that some verb-complement complex has an idiomatic meaning.

\subsubsection{Constituent order in clauses with 来 and 去}

Here are some examples of clauses with complement 来:
\begin{exe}
    \ex No object \gll 他 回_{\text{main verb}} [来]_{\text{direction complement}} 了_{\text{\ac{sfp}}} 。 \\
        3sg go.back  come \\ % TODO: glossing of le
        \glt `He has comes back (to where the speaker or the listener or someone with close relation with them lives).'
    \ex With a goal object (\prettyref{sec:object-goal}) \label{ex:hui-shanghai-lai-le}  \begin{xlist}
        \ex 
        \gll 我 回_{\text{main verb}} 上海_{\text{object: goal}} [来]_{\text{direction complement}} 了_{\text{\ac{sfp}}}。\\ % TODO: glossing of 上海 and 了,goal object动词还有别的constituent没有
        1sg go.back Shanghai come \\
        \glt `I have come back to Shanghai (where the listener lives or where the speaker's family lives).'
        \ex *我回来上海了。
    \end{xlist}
    \ex 我回上海来了一趟。% TODO:???和数量补语的搭配需要注意
    \ex \begin{xlist}
        \ex With a  他带来了两只鸡。 % TODO: what object?
        \ex 他带了两只鸡来。
    \end{xlist}
\end{exe}

\eqref{ex:hui-shanghai-lai-le} means the goal object of the main verb, if any,
is to be placed between 

\subsubsection{When only one of 来 and 去 may appear}


\begin{exe}
    \ex \begin{xlist}
        \ex 我买来了一只鸡。
        \ex *我买去了一只鸡。
    \end{xlist}
\end{exe}

\section{Result complements}\label{sec:result-complement}

A result complement 

做完

\section{Potential complements}\label{sec:potential-complement}

\subsection{Result complements about direction}\label{sec:direction-potential-complement}

看得懂

\section{Manner and consequence complements}\label{sec:manner-consequence-complement}

Manner complements are also called degree complements 程度补语 for obvious reasons.
Consequence complements are also called state complements 状态补语, since they describe the state of 
objects undergoing the action in question \citep[\citesec{5.8.4}]{xianhan2004}.

\subsubsection{Manner and consequence complements and pseudo-attributive constructions}

\begin{exe}
    \ex Pseudo-attributive constructions 
    \begin{xlist}
        \ex 他的老师当得好。
        \ex 他的工作做得细致。
    \end{xlist}
\end{exe}
\citep{huang2008}

\section{Time and location complements}\label{sec:location-complement}

Time and location complements are often called as \emph{prepositional complements}, because they are 
realized as prepositional phrases (\prettyref{chap:coverbs}). 
These complements illustrate where and when the action they modify happens.
Here are some examples of time and location complements:
\begin{exe}
    \ex Location of action \begin{xlist}
        \ex \gll 他   住   [ 在_{\text{preposition}} 北京 ]_{\text{location complement}}。 \\
                 3sg  live {} at                           Beijing \\
            \glt `He lives in Beijing.' 
        \ex \gll 我  蹲     [ 在_{\text{preposition}}  [ 厕所 里 ]_{\text{location phrase}} ]_{\text{location complement}}。 \\
            1sg squat  {} at {} toilet inside \\
            \glt `I squat in the toilet.'
    \end{xlist}

    \ex Time of action \begin{xlist} % TODO: time phrase??
        \ex \gll 鲁迅 生 [ 于_{\text{preposition}} 1881年_{\text{time phrase}} ]_{\text{time complement}}。\\
               Lu.Xun be.born {} at 1881 \\
            \glt `Lu Xun was born in 1881.'
        \ex \gll 我 等 他 一直 等 [ 到_{\text{preposition}} 昨天_{\text{time phrase}} ]_{\text{time complement}}。 \\
                 1sg wait 3sg continuously wait {} towards yesterday \\
            \glt `I waited for him until yesterday.'
    \end{xlist}
\end{exe}

Since the distinction between Chinese prepositions and verbs is somehow vague, with lots of prepositions
being better named as coverbs (\prettyref{chap:coverbs}), location complements are sometimes hard to be 
separated from serial verb constructions. Consider the following example:
\begin{exe}
    \ex Destination of object 
    \gll 我 交 了 一 本 书 [ 给_{\text{coverb}} 他 ]_{\text{???}}。\\
                 1sg hand.over PERF one CLF.book.like book {} give 3sg \\
            \glt `I give him a book.'
\end{exe}
The bracketed words give information about towards which place the object 书 moves. Should we analyze it 
as a marginal location complement? % TODO: 和serial verb construction建立联系,并且讨论两者的区别
% TODO:这个构造又和“我把笔芯装进笔杆里了”不同,其中“我装笔芯进笔杆里”读不通

\section{Quantity complements}\label{sec:quantity-complement}

The \concept{quantity complement}, also considered as a subset of time expressions (and not listed in complements)
by some authors \citep[Sections 7.2-7.4]{po2015chinese} or as quasi-objects by others \citep[\citesec{5.7}]{xianhan2004}, is usually filled by a % TODO: “一个”这样的短语应该叫做什么;这里有加括号悖论,就是“一个理论”可以分解成“一个”+“理论”

Quantity complement 

% 他看了三天的书

\cite{huang2008}

\end{document}