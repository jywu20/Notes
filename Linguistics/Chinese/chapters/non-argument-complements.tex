\documentclass[../main.tex]{subfiles}

\begin{document}

% 有一定的可能,此处的complement不是CGEL中的complement-modifier区分中的一部分,而是external modifier一类的东西?

\section{Overview}

\subsection{The name and status of non-argument complements}

In the School Grammar, 补语 or (non-argument) complement in a clause means anything that is licensed and 
selected by the verb yet does not fill a typical argument slot. In \ac{cgel}, arguments are also called 
as complements, so we emphasize \emph{non-argument} in the title of this chapter. 

The non-argument complements are indeed complements. Consider, for example, the following examples
of quantity complements (see \prettyref{sec:quantity-complement} for more details):
\begin{exe}
    \ex \begin{xlist}
        \ex 我打了他[一下]。
        \ex 我打了他[一巴掌]。
        \ex 我打了他[一拳]。
        \ex *我打了他[一口]。
        \ex 我咬了他[一下]。
        \ex 我咬了他[一口]。
        \ex *我咬了他[一巴掌]。
        \ex *我咬了他[一拳]。
    \end{xlist}
\end{exe}
From the facts above we conclude that the verb licenses the quantity complement. Should a quantity
complement appears, its measure word cannot vary arbitrarily. The verb 打 allows 下, 巴掌 or 拳 as 
the measure word of its quantity complement, but it rejects 口. 
On the other hand, the verb 咬 only accepts 下 and 口 and rejects 巴掌 or 拳.
It is ungrammatical to use a complement that is not licensed by the verb, though they may not be 
obligatory. This is just like complements in NPs in English.

Sometimes a complement is not only \emph{licensed}, but also \emph{required}, just like an argument. 
Consider the following examples about location complements (see \prettyref{sec:location-complement}):
\begin{exe}
    \ex \begin{xlist}
        \ex 我住在北京。
        \ex *我住。
    \end{xlist}
\end{exe}

\subsection{Interplay between non-argument complements and other constituents}

Non-argument complements involve highly complicated grammatical phenomena. Sometimes a non-argument 
complement precedes the object, as is in the case of result complements (see \prettyref{sec:result-complement}):
\begin{exe}
    \ex 熊_{\text{subject}} [拍]_{\text{verb stem}} [晕]_{{\text{result complement}}} [了]_{\text{perfect aspect}} [他]_{\text{object}}。
\end{exe}
But for quantity complements (see \prettyref{sec:quantity-complement}), the inverse order is the case:
\begin{exe}
    \ex 熊_{\text{subject}} [拍]_{\text{verb stem}}  [了]_{\text{perfect aspect}} [他]_{\text{object}} [三巴掌]_{\text{quantity complement}}。
\end{exe}
In the following sections, the interplay between non-argument complements and objects is included.

It can be easily seen that pre-verbal adverbials have no non-trivial interaction with complements:
\begin{exe}
    \ex 我_{\text{subject}} [昨天]_{\text{time adverbial}} [在那条街上]_{\text{place adverbial}} [狠狠地]_{\text{manner adverbial}} [打了]_{\text{verb, perfect}} [他]_{\text{object}} 一顿_{\text{complement}} % TODO: Aspect
\end{exe}
On the other hand, compositional using of non-argument complements is much more limited:
\begin{exe}
    \ex \begin{xlist}
        \ex 那头熊拍了他[一巴掌]_{\text{quantity complement}}。
        \ex 那头熊拍[晕]_{\text{result}}了他。
        \ex[*]{那头熊拍晕了他一巴掌。}
    \end{xlist}
\end{exe}
Some compositional using is indeed possible:
\begin{exe}
    \ex 那头熊打[中]_{\text{TODO}} 了 他 [两次]_{\text{quantity complement}}。
\end{exe}

\subsection{Classification of non-argument complements}

% TODO:山路走得我累死了是哪一类?

\section{Direction complements}\label{sec:direction-complement}

\section{Result complements}\label{sec:result-complement}

A result complement 

做完

\section{Potential complements}\label{sec:potential-complement}

看得懂

\section{Location complements}\label{sec:location-complement}

\begin{exe}
    \ex \begin{xlist}
        \ex 他住在北京。
        \ex 我蹲在厕所里。
    \end{xlist}
\end{exe}

\section{Quantity complements}\label{sec:quantity-complement}

The \concept{quantity complement}, also considered as a subset of time expressions (and not listed in complements)
by some \citet[sec. 7.2-7.4]{po2015chinese}, is usually filled by a % TODO: “一个”这样的短语应该叫做什么;这里有加括号悖论,就是“一个理论”可以分解成“一个”+“理论”

Quantity complement 

\section{Interaction between different types of complements}

% TODO: more tests

One possible way to make two non-argument complements appear together is via the following construction:
\begin{exe}
    \ex 那头熊[一巴掌]拍晕了他。
\end{exe}
where the quantity complement is promoted to a pre-verbal position. In this construction no more manner adverbial can appear:
\begin{exe}
    \ex[*]{那头熊[用力地]_{\text{manner adverbial}}[一巴掌]_{\text{promoted quantity complement}}拍晕了他。}
\end{exe}
while other adverbials can still appear:
\begin{exe}
    \ex 那头熊[昨天]_{\text{time adverbial}} [在那座山下面的小树林里]_{\text{place adverbial}} 一巴掌拍晕了他。
\end{exe}
So it seems the fronted quantity complement is in complementary distribution with the ordinary manner adverbial.
Note that the quantity complement cannot be promoted to the pre-verbal position if it is the only non-argument 
complement:
\begin{exe}
    \ex[*]{那头熊一巴掌拍了他。}
\end{exe}

\end{document}