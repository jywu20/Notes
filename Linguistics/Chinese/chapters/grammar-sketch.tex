\section{Word classes}\label{sec:word-class-intro}

How to recognize word classes in languages formerly not well understood is always a huge problem.
People can debate for decades on questions about ``whether Chinese has adjectives'' (what do they mean 
by the term ``adjective'', anyway?) or ``whether Chinese verbs can be regarded as a subclass of nouns''
(what do they mean by the term ``noun''?). In the debate name-calling is quite common. 
A most prevalent activity is to accuse each other of 
advocating Eurocentrism, often for exactly opposite reasons. People distinguishing an adjective 
category in Chinese may be claimed as Eurocentrists, because they ``use the terms (in this case 
the notion of adjectives) in European languages as a universal standard in non-European languages''.
People rejecting an adjective category are also charged with the crime of racism, because 
``they use properties of European adjectives to find an adjective category and fail to identify 
the diverse behavior of adjectival categories in world languages''. In \ac{blt} you can find 
(quite humorous, I have to admit) depiction of ``formalists'' trying to fruitlessly find uniform adjectival 
behaviors in all human languages, while another bunch of formalists are trying to eliminate the 
notion of adjectives as a primitive feature of human language, as opposed to the notion of nouns and verbs \citep{Mitrovi2020}.

\section{Remarkable features}

\subsection{Controversial grammatical words}

\subsection{Prosody}\label{sec:prosody-intro}

One distinct feature of Chinese is its morphosyntax relies strongly on \emph{prosody} \citep{feng2013}. 


\section{Grammars and research papers}

There are already multiple comprehensive grammars of the Chinese language, and a terminology system 
suitable to describe Chinese has emerged, with both Chinese and English as possible metalanguages. 
An unfortunate fact (which is not restricted to the study of Chinese language) is the title of a grammar
has nothing to say about its coverage and organization. 

\subsection{Chinese School Grammar}\label{sec:school-grammar}

Like the school grammar 学校文法 in Japanese grammar, there has been a grammatical research tradition 
in Chinese grammar which is the dominant school of Chinese grammar in schools, introductory textbooks of 
which are often just named as 现代汉语 \citep[e.g.][]{xianhan2004}. Following the example of Japanese 
学校文法, we may call this grammar system the \concept{Chinese School Grammar}.
This school of grammar research has no interest in languages outside the Sinitic family -- or even outside 
Standard Mandarin. Its concepts are tailored for describing the modern Standard Mandarin language. 

The School Grammar approach is, maybe quite surprisingly for many, quite ``structuralist'' in the broad 
sense.%
\footnote{
    Early generative grammarians often describe non-generative structuralists using the term \emph{structuralist}
    without any modification, so the term \emph{structuralist} gains its meaning of ``pre-(or even anti-)
    generativism''. Nonetheless, since putting the disputation of what is in the mind aside, 
    generative syntax and structuralist syntax share their phrase structure grammar and transformation 
    metalanguages, both of them are \emph{structuralist} in the broad sense \citep{newmeyer1986has}.
}
Modern descriptive grammars (or ``BLT grammars'') tend to have more coarse-grained analysis \citep{dryer2006descriptive},
while the School Grammar insists a binary branching trees \emph{a la} Bloomberg or \emph{a la} Minimalism
or \emph{a la} \ac{cgel}. We can, for example, see the fact in \prettyref{chap:basic-clause-structure}.

This grammar system deeply influenced how Chinese is taught to foreigners. % TODO: 和中文语法维基对比

\subsection{Latin-like grammars}

There are less structure-oriented approaches to Chinese grammar, which are organized like traditional 
Latin grammars (and English grammars), in which one starts from word categories and discuss what can be 
used after or before words in a category. The famous Latin grammar \citet{greenough2013allen} is 
a model of this type of grammars. The way it treats syntax is via local semantics-syntax mapping 
and constraints defined on dependency relations: an ablative noun may express means, manner, 
accompaniment, degree of difference, quality, price or specification; a dative noun can be used 
together with an adjective; the complement of a preposition always follows the preposition.

It is a wise idea to write a grammar of Latin in this scheme. It is less a wise idea to do so for Chinese,
the latter being a language with relatively rigid consistent order, by which dependency relations 
are manifested. Unfortunately, some contemporary grammars, like \citet{po2015chinese}, are still 
written in this way. Readers will not find a chapter about the overall structure of NPs or clauses 
in \citet{po2015chinese}. They will only find separate chapters about how small components -- classifier
constructions, adjectives and adjective phrases, coverbs -- are made up and used.

Grammars of this type are handy for learners who already have a grasp of the basics, who do not really need 
discussion on segmentation of sentences and phrases and just want a grammar as a dictionary when they 
parsing seemingly weird constructions. For others, this approach of Chinese grammar is of limited use,
but certain it is not worthless. It at least serves as a system of consistent terminology, which, in 
the context of languages without a native grammar tradition, is quite valuable. It is probably also 
the most uncontroversial way to write a grammar, since as we see in \prettyref{sec:descriptive-framework},
there are huge discrepancies among various grammar schools. It is also the most robust approach to 
morphosyntax in a diachronic context. Even when the morphosyntax evolves with certain constructions broken 
and new constructions emerging, the dictionary-like grammars are still of great use. 

\subsection{The GB-style school of Huang et al.}\label{sec:gb-grammar}

Probably the most controversial school of Chinese morphosyntax is the mostly GB-based generative school led by James Huang 黄正德, 
Sze-Wing Tang 邓思颖 and others. Several milestone works are completed by this school, but the grammaticality 
data in most of these works are fiercely questioned by outsiders. Obviously bizarrely ungrammatical examples 
like \citep[sec. 5.4.2, (65)]{huang2013} 
\begin{exe}
    \ex\label{ex:huang-weird-1} 把他们,我打得手都肿了
\end{exe}
appear in these works, making how seriously people should take them a question mark.

If critics take a closer look to the ``deeply flawed'' works, they will soon realize another bizarre fact that 
these questionable grammaticality judgments are often irrelevant to the big picture, despite the fact that 
the authors take these judgments quite seriously and pay long discussion on their implications.
Back to our example of \eqref{ex:huang-weird-1}, in the related discussion of disposal constructions in 
Chapter 5 of \citet{huang2013}, what the authors are trying to argue for is that 把 in the disposal constructions 
is like a light verb (but not a typical light verb) and \emph{not} a preposition as many claim 
(see the discussion in \prettyref{chap:disposal}). The fact that \eqref{ex:huang-weird-1} is ungrammatical 
for most Chinese speakers actually \emph{lighten} the burden of the authors to defend this position.
Scholars in this school are sometimes just like Don Quixote, screwing their brains trying to explain
phenomena that do not exist at the first place.

As the reader can expect, the discussion on whether the GB-based school has done anything valuable 
is also mingled with meaningless conversations about Eurocentrism. ``Well, they are just following 
Chomsky's teaching that all languages are European languages! The best contribution they can make for 
linguistics is to go die as soon as possible!'' One may furiously shout out. ``You know nothing about 
the Chinese language, about how it resembles European languages or not. You are creating linguistic 
human zoos by blocking discussion about cross-linguistic similarities!'' Another may reply.
What the chaos reveals % TODO

Another disadvantage of the generative approach is it almost never produces anything that interests an outsider.
In practice, a generative grammar may resemble a Latin-based and dictionary-like grammar, since the former 
segments the language into $v$P layers, TP layers, CP layers, DP layers, etc. and analyze each layer 
about what can and cannot be used together. For users uninterested in theoretical topics,
generative grammars are just more structured Latin-based and dictionary-like grammars with obscure jargons.

That being said, this proto-book relies heavily on the GB-based school, since setting the good old structuralist
School Grammar aside, this is the most structure-oriented approach.

\subsection{Theoretical functional approaches}

Unlike the generative schools, functional approaches to Chinese grammar establish their 
