\documentclass[../main.tex]{subfiles}

\begin{document}

This chapter is about the verb category, about what qualifies as a verb and 
how verbs can be distinguished from words with similar clausal distribution, e.g. adjectives,
and the subcategorization of verbs.
Grammatical relations involved in subcategorization here includes but is not limited to argument structure.

Some people call certain aspects of verb subcategorization \term{agreement} 
\citep[\citesec{6.6, 6.7}]{li1989mandarin}.
This is not the term taken in this book, 
because the more generally accepted meaning of \emph{agreement} between a word and its complement 
is that the former can take different forms when the latter varies, each of which is grammatical,
while what is covered in \citet[\citesec{6.6, 6.7}]{li1989mandarin} is simply about what the object cannot be.
The term \emph{selection} is therefore better than \emph{agreement}.

\section{Separable verbs}

Verbs in the Chinese language can be classified according to multiple criteria defined by possible clause 
structures headed by them. % TODO: difference between state verb, action verb, etc.
% 离合词的历史成因和韵律成因
% TODO: 了着过和action verb/stative verb的区分

\end{document}