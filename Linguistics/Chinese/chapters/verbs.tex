\documentclass[../main.tex]{subfiles}

\begin{document}

Some people call verb subcategorization \emph{agreement} \citep[\citesec{6.6, 6.7}]{li1989mandarin}.
This is not the term taken in this book, 
because the more generally accepted meaning of \emph{agreement} between a word and its complement 
is that the former can take different forms when the latter varies, each of which is grammatical,
while what is covered in \citet[\citesec{6.6, 6.7}]{li1989mandarin} is simply about what the object cannot be.
The term \emph{selection} is therefore better than \emph{agreement}.

\section{Separable verbs}

Verbs in the Chinese language can be classified according to multiple criteria defined by possible clause 
structures headed by them. % TODO: difference between state verb, action verb, etc.
% 离合词的历史成因和韵律成因
% TODO: 了着过和action verb/stative verb的区分

\end{document}