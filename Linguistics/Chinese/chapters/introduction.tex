\documentclass[../main.tex]{subfiles}

\begin{document}

\section{Genetic affiliation}

% TODO: 茶堡话是否有形态学类似性?

\section{Demography, sociolinguistic situation and varieties}

\section{What language? What grammar?}

This book is about aspects of the grammar of the contemporary Chinese language (现代汉语). 
Each word in this phrase can trigger controversy. Before starting substantial discussion, it is a wise idea 
to clarify what I am actually talking about. 

\subsection{Standard Mandarin and its variation}

% 新文化运动时期的作品读着不顺口

In the rest of the book, we use the term \emph{Chinese} 中文,(现代)汉语,华语 interchangeably with 
the more precise term \emph{Standard Mandarin Chinese}.

\subsection{The possibility to have a structure-based grammar}\label{sec:structure-based}

People familiar with Chinese often say its grammar depends more on the context, and some goes as far as 
claiming that a structure-based approach -- or even a truth-value semantics-based approach -- is infeasible 
when studying Chinese. And indeed, \citet{li1989mandarin}, arguably the most recognized grammar of the Chinese 
language, is a functionalist one. Our opinion is that though of course the context can influence strongly the 
grammar, this is not without limit. Pragmatic information may trigger \emph{pro}-drop or forbid it, but it 
rarely triggers omission of the object. Sentences in dialogues may have more sentence final particles than 
written ones, but spoken sentences never have sentence \emph{initial} particles. Though the context means 
a lot in Chinese, it is safe to assume we still have an underlying rigid structure beside purely semantic 
or pragmatic information.

The structure-based approach is by no means a rejection of functionalist studies. Rather, the former explores 
what features can be employed by the latter, so it can be expected the two approaches are complementary.

The next question is how to catch the structure. It is impossible to sit there and just ``observe the world without bias''. People always do observation within a 
framework. Some may argue that typologists must be ready to invent completely new concepts when documenting 
a language (see, for example, \citet{haspelmath2008framework}), which is, of course, in principle true, 
but practically it is common for people to implicitly take some concepts for granted and carry out 
valuable works. 
R.M.W Dixon, a famous opponent of generative syntax, mocks ``formalists'' 
who fruitlessly try to find concepts exact corresponding to Indo-European ones in underdocumented languages 
in his \ac{blt} \citep{dixon2009basic} and advocates ``describing a language in its own terms'',
but he immediately goes on to discuss how to write a grammar \emph{in terms of basic linguistic theory},
where we have predefined terms like \emph{clause}, \emph{sentence}, \emph{argument}, a \emph{deep structure}
(This is indeed the term used by him!) which is made up by constituent hierarchy and so on.%
\footnote{
    It is often justified that it is acceptable to do so 
    because the predefined terms are just for \emph{inspiring} people 
    and not meant to be used in describing any language, 
    and thus \ac{blt} is not a framework in the way generative approaches are. 
    This justification seems also to be used by Dixon, since in \ac{blt} he writes 
    that \ac{blt} is a toolkit of description devices 
    and not all of them should be used 
    and actual language description always feeds back to the Basic Linguistic Theory toolbox, 
    so the toolbox is not a rigid framework.
    
    This justification is not valid.
    The first reason is it contains a factual error: 
    generative linguistics do not expect to find every concept 
    that has been discovered in English in newly documented languages.
    For example, there have always been some generative linguists arguing 
    that Chinese does not have adjectives (\prettyref{sec:word-class-intro})
    or that the passive construction is radically from the one in English 
    (\prettyref{sec:passive-cross-linguistic}).
    
    In ``hard sciences'' like physics, it is quite common 
    that a theory breaks the framework in a rigid sense 
    but everyone agrees the theory is just \emph{enriching} the framework, 
    and it is also quite common that 
    a so-called framework is just a toolkit of possible mechanisms and ways to deal them
    (see, for example, the case in condensed matter many-body theory), 
    but still everyone agrees there \emph{is} and \emph{needs to be} a framework after all. 
}
Indeed, the strange fact that structuralist (and ``arbitrary'' and ``purely empirical'') analyses of 
languages always fall into the same metalanguage -- the one with headed (I talk about the term in 
\prettyref{sec:headedness}) phrase structures (IC analysis) and a set of shared concepts like predicate, 
arguments, etc. -- is one motivation of the birth of generative syntax, which is formalized in Chomsky's 
famous Syntactic Structures \citep{chomsky2009syntactic}. The same fallacy can be seen in construction grammar,
where people talk about stored routinized constructions -- but routinization of \emph{what}? 
It seems if we are to discuss purely structural aspects of a language, assuming a grammatical framework 
about possible structure building mechanisms is inevitable. This is actually not a bad thing. I will 
talk about the framework in following sections, and we will find Minimalism, tree-adjoining
grammar, the implicit framework employed in many language documentation works, etc. can be reconciled.

\subsection{``Not limited in Indo-European grammar perspectives''}

Another frequently mentioned motto in the study of Chinese language is ``Don't be limited to Indo-European
perspectives''. Again this is a correct statement but does not give much concrete methodological suggestions.
In self-identified ``non-(or even anti-)generative'' communities (the Language Hat, some Twitter circles, 
among others), this motto is also invoked to argue against formalist approaches. This accusation is very alarming and often contains many serious and insightful criticisms, but the claim itself may not factually 
hold, especially in recent years, since many generative linguistics are now highly interested in
underdocumented languages, and many theoretical proposals \citep{preminger2014agreement} are based on %TODO: more references 
these languages rather than so-called Indo-European perspectives. (Another related accusation is 
generative works do not view a language in a holistic way -- how to solve the problem is also 
discussed in \prettyref{sec:generative-no-good}.) We should keep in mind 
that what works in English does not necessarily work in unfamiliar languages in question, but if 
a formal universal (for example, ``the phonetic realization of pronouns is dependent to c-command relations'')
seems truly reasonable in the new language, we should not hesitate to keep it.
What terms in Indo-European language studies should be avoided? Accusing each other as Indo-European-oriented 
often leads to unproductive results and unnecessary chaos. This book includes some examples: 
see \prettyref{sec:word-class-intro}, \prettyref{sec:gb-grammar}. % TODO: full list

\end{document}