\documentclass[../main.tex]{subfiles}

\begin{document}

\section{Tense-like sentence final particles}

\subsection{The particle 了2}\label{sec:sfp-le}

\citep[\citesec{11.2.2}]{meiguang2018} considers the \ac{sfp} 了 as a marker of a rather weak tense system,
which hints the listener to pay attention to the speech time. 
This analysis entails the changing situation analysis, 
since emphasis on the speech time means emphasis on what has already happened, i.e. 
how the situation has changed.
This is illustrated by the following example: 
\begin{exe}
    \ex \begin{xlist}
        \ex \gll (- 你 在 干 什么 ?) - 我 在 穿 衣服 。 \\
        {} 2sg DUR do what {} {} 1sg DUR wear clothes \\
        \glt `(What are you doing?) I am wearing clothes.' 
        \ex \gll 我 在 穿 衣服 了 。 \\
        1sg DUR wear clothes LE \\
        \glt `I am wearing clothes \emph{now/already} (so please don't urge me over and over again!)'
    \end{xlist}
\end{exe}

\end{document}