\documentclass[../main.tex]{subfiles}

\begin{document}

\section{Tense-like sentence final particles}

\subsection{The particle \homo{了}{2}}\label{sec:sfp-le}

\citep[\citesec{11.2.2}]{meiguang2018} considers the \ac{sfp} 了 as a marker of a rather weak tense system,
which hints the listener to pay attention to the speech time. 
This analysis entails the changing situation analysis when the dialogue is about a current event, 
since emphasis on the speech time means emphasis on what has already happened, i.e. 
how the situation has changed.
This is illustrated by the following example: 
\begin{exe}
    \ex \begin{xlist}
        \ex \gll (- 你 在 干 什么 ?) - 我 在 穿 衣服 。 \\
        {} 2sg DUR do what {} {} 1sg DUR wear clothes \\
        \glt `(What are you doing?) I am wearing clothes.' 
        \ex \gll 我 在 穿 衣服 了 。 \\
        1sg DUR wear clothes LE \\
        \glt `I am wearing clothes \emph{now/already} (so please don't urge me over and over again!)'
    \end{xlist}
\end{exe}
However, consider the following example:
\begin{exe}
    \ex \gll 我 去年 就 来 这里 了 。\\
    1sg last.year JIU come here LE \\ % TODO: 就是什么意思?
    \glt `I have been living at this place since last year.'
\end{exe}
In the above example, the focus is definitely not the \emph{speech time},
but the . %TODO: 这个算是什么时间?参考时间还是事件时间?
My position is therefore that \homo{了}{2} is not about speech time and hence not a tense marker, 
but a \emph{sentential aspect} marker \citep{pan2022deriving} indicating a change of the situation. 

\subsection{The particle 呢}



\end{document}