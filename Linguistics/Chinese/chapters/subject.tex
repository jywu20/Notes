\section{The notion of subject}

% TODO: 一些方言有格标记,能否用它们来判断主语和话题有无区别?

\subsection{Subject and topic}

The relation between the subject and the predicate is highly diverse.
It is so diverse that some people in the School Grammar camp 
equate the notion of \term{subject} in Chinese with \term{topic} 
\citep[\citesec{5.6}]{xianhan2004}. 
Under this definition of \emph{subject}, then, in a topic-comment construction, % TODO: topic
the fronted topic is analyzed as the subject: 
\begin{exe}
    \ex 但是 [你最对不起的人]_{\text{subject}, i},[你]_{\text{subject before topicalization}} 反而轻轻地忘了 $e$_i。
    \label{ex:ni-zui-dui-bu-qi-1}
\end{exe}
So topicalization changes the subject.
This does not break the consistency in their analysis, 
because in their analysis a predicate may have an inside subject, 
so the structure of \eqref{ex:ni-zui-dui-bu-qi-1} is 
\begin{exe}
    \ex 但是_{\text{initial time adverbial}} [ [你最对不起的人]_{\text{subject}} [ [你]_{\text{subject}} [反而轻轻地忘了]_{\text{predicate: VP}} ]_{\text{predicate: subject-predicate construction}} ]_{\text{subject-predicate construction}}
\end{exe}

\section{Pseudo-attributive constructions in the subject}

% 他的老师当得好