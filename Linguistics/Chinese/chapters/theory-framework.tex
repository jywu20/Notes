\documentclass[../main.tex]{subfiles}

\begin{document}

\section{What language? What grammar?}

This proto-book is about aspects of the grammar of the contemporary Chinese language (现代汉语). 
Each word in this phrase can trigger controversy. Before we start substantial discussion, it is a wise idea 
to clarify what we are actually talking about. 

\subsection{Standard Mandarin and its variation}

% 新文化运动时期的作品读着不顺口

In the rest of the proto-book, we use the term \emph{Chinese} 中文,(现代)汉语,华语 interchangeably with 
the more precise term \emph{Standard Mandarin Chinese}.

\subsection{The possibility to have a structure-based grammar}

The structure-based approach is by no means a rejection of functionalist studies. Rather, the former explores 
what features can be employed by the latter.

\subsection{``Not limited in Indo-European grammar studies''}

% 印欧语研究能够告诉我们什么

\section{The descriptive framework}

It is impossible to sit there and just ``observe the world without bias''. People always do observation within a 
framework. Some may argue that typologists must be ready to invent completely new concepts when documenting 
a language (see, for example, \citet{haspelmath2008framework}), which is, of course, in principle true, 
but practically it is common for people to implicitly take some concepts for granted and carry out 
valuable works. R.M.W Dixon, a famous opponent of generative syntax, mocks ``formalists'' who fruitlessly try 
to find concepts exact corresponding to Indo-European ones in underdocumented languages in his 
\ac{blt} \citep{dixon2009basic} and advocates ``describing a language in its own terms'',
but he immediately goes on to discuss how to write a grammar \emph{in terms of basic linguistic theory},
where we have predefined terms like \emph{clause}, \emph{sentence}, \emph{argument} and so on. %
\footnote{
    It is often justified that it is acceptable to do so because the predefined terms are just for inspiring 
    people and not meant to be used in describing any language, and thus \ac{blt} is not a framework. This justification is not valid, because in 
    ``hard sciences'' like physics, it is quite common that a theory ``breaks'' the framework in a rigid sense 
    but everyone agrees the theory is just \emph{enriching} the framework, and there is a framework after all.
}%
Indeed, the strange fact that structuralist (and ``arbitrary'' and ``purely empirical'') analyses of 
languages always fall into the same metalanguage -- the one with headed (we will talk about the term in 
\prettyref{sec:headedness}) phrase structures (IC analysis) and a set of shared concepts like predicate, 
arguments, etc. -- is one motivation of the birth of generative syntax, which is formalized in Chomsky's 
famous Syntactic Structures \citep{chomsky2009syntactic}.

The grammatical framework we use in this proto-book is very similar to the framework in \ac{cgel} \citep{cgel,pullum2008expressive}. In the following sections, we discuss why choosing such a framework and the relation 
between the framework and contemporary generative syntax.

\subsubsection{Infeasibility of using derivational syntax as a descriptive tool}

Contemporary generative syntax tends to work with features , which proves not suitable for language describing from sketch \citep{dryer2006descriptive}.

\subsubsection{Dependency relations and phrase structures}

\subsubsection{The notion of \emph{head}}\label{sec:headedness}

\subsubsection{Empty categories and fusion-head constructions}

\subsubsection{Summary}

After length discussions, it is time to go back and summarize the framework we adopt in this proto-book.

\subsection{The surface-oriented yet generativism-informed grammar framework} 

\subsection{Descriptive concepts}

% Predicator,NP等概念

\subsubsection{Categories and the notion of ``word''}

A \concept{category}\index{category} is defined as a type of constructions with similar distributions.
We will first discuss basic syntactic constructions and identify positions that can be filled in them, 
and then search possible constructions in these positions. This is how categories can be recognized.

A construction that is small enough is said to be a \concept{word} or more precisely, 
a \concept{grammatical word}\index{word!grammatical}.
We emphasize \emph{grammatical} because it is quite common to use the term \emph{word} 词 to denote 
a \concept{prosody word}\index{word!prosody}. Prosody is important in Chinese, % TODO to a chapter about prosody

In traditional grammars concerning Latin, a common practice is to roughly define word classes (nouns, verbs, etc.) 
according to their meaning and then discuss where they can be used. 
In this proto-book we do not take this approach. Though we will review a lot of work based on the meaning-first 
approach, the way we distinguish word classes is mainly distributional. If two words can appear in similar
positions, they are classified into one \concept{word class}\index{word class} or \concept{part of speech}.
A word class is just a category about words.  

In other words, we define concepts like \emph{noun-like} and \emph{verb-like} \emph{before} listing criteria of 
what is a noun and what is a verb. Criteria for word classes are always language-specific, but we have more 
confidence that at least some \emph{features} -- like the nominal feature \textit{n} or the verbal feature 
\textit{v} -- are cross-linguistic and may be attributed to the language faculty in the broad sense. 

\section{Methodology}

% 关于语料来源和分析方法

\end{document}