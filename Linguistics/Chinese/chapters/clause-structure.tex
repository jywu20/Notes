\documentclass[../main.tex]{subfiles}

\begin{document}

This chapter gives a sketch of the structure of \concept{simple clauses}\index{clause!simple}, 
i.e. clauses in which there are no further subordinated or coordinated clauses. 
This being said, matrix clauses\index{clause!matrix} (the surrounding environment of subordinated clauses)
will also appear briefly in this part, so what we are talking about in this chapter -- and this part -- is 
actually the $v$P layer and the TP layer, with an unmarked CP background (so topicalization etc. are not 
discussed in this part). It is of course possible 
to further break the $v$P layer -- or in more descriptive terms, the argument structure\index{argument structure} 
-- from the TP layer, but this is not how things are done in more descriptive works.

The word order of a clause is basically the follows:

subject - adverbials - coverbs - adverbials - main verb - suffixes - direct object - indirect object - sentence final particles

The School Grammar segments these constituents in a way quite similar to the \ac{cgel} approach. 
Take the analysis in \citet[chap. 5]{xianhan2004} as an example.
A clause is first divided into 主语 (subject) and 谓语, and 谓语 is then divided into 述语, 宾语 (object) and 补语,
and 谓语 has modifiers named 状语, while modifiers in NPs are named as 定语. We can almost identify 
谓语 as \emph{predicate} in \ac{cgel}, and 述语 as \emph{predicator}. 

\end{document}