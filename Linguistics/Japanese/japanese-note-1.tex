\documentclass[UTF8, a4paper, oneside, scheme=plain]{ctexart}

\usepackage{geometry}
\usepackage{float}
\usepackage{titling}
\usepackage{titlesec}
\usepackage{paralist}
\usepackage{footnote}
\usepackage{enumerate}
\usepackage{amsmath, amssymb, amsthm}
\usepackage{gb4e}
\noautomath
\usepackage{bbm}
\usepackage{soul}
\usepackage{graphicx}
\usepackage{siunitx}
\usepackage[table,xcdraw]{xcolor}
\usepackage{tikz}
\usepackage[ruled, vlined, linesnumbered, noend]{algorithm2e}
\usepackage{xr-hyper}
\usepackage[colorlinks]{hyperref} % linkcolor=black, anchorcolor=black, citecolor=black, filecolor=black
\usepackage[most]{tcolorbox}
\usepackage{caption}
\usepackage{subcaption}
\usepackage{booktabs}
\usepackage{multirow}
\usepackage[figuresright]{rotating}
\usepackage{acro}
\usepackage[round]{natbib} 
\usepackage{nameref,zref-xr}
\zxrsetup{toltxlabel}
\zexternaldocument*[cgel-]{../English/cambridge}[cambridge.pdf]
\zexternaldocument*[latin-]{../Latin/latin-notes}[latin-notes.pdf]
\zexternaldocument*[alignment-]{../alignment/alignment}[alignment.pdf]
\zexternaldocument*[exercise1-]{../Exercise/2021-3}[2021-3.pdf]
\zexternaldocument*[method-]{../methodology/glossing}[glossing.pdf]
\usepackage{prettyref}

\geometry{left=3.18cm,right=3.18cm,top=2.54cm,bottom=2.54cm}
\titlespacing{\paragraph}{0pt}{1pt}{10pt}[20pt]
\setlength{\droptitle}{-5em}

\DeclareMathOperator{\timeorder}{\mathcal{T}}
\DeclareMathOperator{\diag}{diag}
\DeclareMathOperator{\legpoly}{P}
\DeclareMathOperator{\primevalue}{P}
\DeclareMathOperator{\sgn}{sgn}
\newcommand*{\ii}{\mathrm{i}}
\newcommand*{\ee}{\mathrm{e}}
\newcommand*{\const}{\mathrm{const}}
\newcommand*{\suchthat}{\quad \text{s.t.} \quad}
\newcommand*{\argmin}{\arg\min}
\newcommand*{\argmax}{\arg\max}
\newcommand*{\normalorder}[1]{: #1 :}
\newcommand*{\pair}[1]{\langle #1 \rangle}
\newcommand*{\fd}[1]{\mathcal{D} #1}

\newcommand*{\citesec}[1]{\S~{#1}}
\newcommand*{\citechap}[1]{chap.~{#1}}
\newcommand*{\citefig}[1]{Fig.~{#1}}
\newcommand*{\citetable}[1]{Table~{#1}}
\newcommand*{\citepage}[1]{pp.~{#1}}
\newcommand*{\citefootnote}[1]{fn.~{#1}}

\newrefformat{sec}{\citesec{\ref{#1}}}
\newrefformat{fig}{\citefig{\ref{#1}}}
\newrefformat{tbl}{\citetable{\ref{#1}}}
\newrefformat{chap}{\citechap{\ref{#1}}}
\newrefformat{fn}{\citefootnote{\ref{#1}}}

\usetikzlibrary{arrows,shapes,positioning}
\usetikzlibrary{arrows.meta}
\usetikzlibrary{decorations.markings}
\tikzstyle arrowstyle=[scale=1]
\tikzstyle directed=[postaction={decorate,decoration={markings,
    mark=at position .5 with {\arrow[arrowstyle]{stealth}}}}]
\tikzstyle ray=[directed, thick]
\tikzstyle dot=[anchor=base,fill,circle,inner sep=1pt]


\tcbuselibrary{skins, breakable, theorems}

\newtcbtheorem[number within=chapter]{infobox}{Box}%
  {colback=blue!5,colframe=blue!65,fonttitle=\bfseries, breakable}{infobox}

\newcommand*{\concept}[1]{\textbf{#1}}
\newcommand*{\term}[1]{\emph{#1}}
\newcommand{\corpus}[1]{\emph{#1}}

\DeclareAcronym{blt}{short = BLT, long = Basic Linguistic Theory}
\DeclareAcronym{cgel}{short = CGEL, long = The Cambridge Grammar of the English Language}
\DeclareAcronym{dm}{short = DM, long = Distributed Morphology}
\DeclareAcronym{tag}{long = Tree-adjoining grammar, short = TAG}
\DeclareAcronym{sfp}{long = sentence final particle, short = SFP}
\DeclareAcronym{np}{long = noun phrase, short = NP}
\DeclareAcronym{vp}{long = verb phrase, short = VP}
\DeclareAcronym{pp}{long = preposition phrase, short = PP}
\DeclareAcronym{cls}{long = classifier, short = CLS}
\DeclareAcronym{dist}{long = distal, short = DIST}
\DeclareAcronym{prox}{long = proximate, short = PROX}
\DeclareAcronym{dem}{long = demonstrative, short = DEM}
\DeclareAcronym{dur}{long = durative, short = DUR}
\DeclareAcronym{neg}{long = negative, short = NEG}
\DeclareAcronym{cc}{long = copular complement, short = CC}
\DeclareAcronym{cs}{long = copular subject, short = CS}
\DeclareAcronym{tam}{long = {tense, aspect, mood}, short = TAM}

\newcommand*{\homo}[2]{#1$_{\text{#2}}$}

\newcommand{\cgel}{\href{../English/cambridge.pdf}{my notes about CGEL}}
\newcommand{\latin}{\href{../Latin/latin-notes.pdf}{my notes about Latin}}
\newcommand{\alignment}{\href{../alignment/alignment.pdf}{my notes about alignment}}
\newcommand{\exerciseone}{\href{../Exercise/2021-3.pdf}{this exercise}}
\newcommand{\method}{\href{../methodology/glossing.pdf}{this note about how descriptive grammars work}}
\newcommand{\draft}{\href{./main.pdf}{this draft}}

\newcommand{\ala}{à la}
\newcommand{\translate}[1]{`#1'}

\title{Japanese grammar notes}
\author{Jinyuan Wu}

\begin{document}

\maketitle

\automath

This is a reading note of \citet{akiyama2012japanese}, \citet{tsutsui1989dictionary},
as well as lots of books and articles listed in the reference.
The methodology followed is in \method,
i.e. ``largely generatively informed but surface-oriented and flat-tree in the appearance''.

\section{Overview}

\subsection{Historical notes}

Japanese has a strict modifier-first constituent order,
and here the term \term{modifier} includes 
arguments in a clause (``modifiers of the verb or the verbal adjective''),
and even \acs{np}s with respect to case particles.
This is probably related to a strong head-final tendency in the linearization:
if a so-called modifier is introduced as a specifier 
in a functional projection with a root as the core,
then obviously the root and the functional heads are realized into one unit (for example a verb complex) 
and the ``modifier'' precedes the unit to ensure the (functional) head-final rule,
and therefore in the surface-oriented analysis,
we also get a modifier-head constituent order (where \term{head} means lexical heads).
If there is no core root,
then trivially the ``modifier'' is realized in a position before the spellout of functional heads,
and the latter is regarded as somehow a head in the \acs{cgel} sense,
and again we get a modifier-head constituent order,
if we understand things like case particles as heads,
which is the \ac{cgel} approach but not the \acs{blt} approach.

\subsection{Phonology and orthography}

\subsection{Parts of speech}

\subsubsection{Lexical words: nouns and verbs, and everything else}

Japanese has a clear noun-verb distinction:
nouns are subject to case marking,
which is basically adding a particle to the \acs{np} 
which can be dropped especially in casual speech,
while verbs always appear as one of the stem forms plus agglutinative endings.

There are two adjective classes:
the verbal adjectives (or \corpus{i}-adjectives)
and the nominal adjectives (or \corpus{na}-adjectives),
with different syntactic distribution 
(verbal adjectives may fill the predicate slot on their own; nominal adjectives never do so)
and morphological appearances
(verbal adjectives are more like verbs).

One rare property of modern Japanese is the verb class and the verbal adjective class 
are already closed classes:
they rarely accept new members (though not entirely impossible).
What makes Japanese rarer is despite being closed,
the verbal adjective class is large.

\subsubsection{Function items}

I specifically use the term \term{fucntion items} instead of \term{function words}
in the title of this section,
because the word-or-morpheme-or-phrase problem is especially serious in Japanese morphosyntax
(TODO: ref: school grammar, education grammar).
Function items in Japanese include particles, auxiliary verbs, TODO %
\footnote{
    Though particles are in the grammar and do not really carry category labels,
    the traditional practice to list all particles and classify them 
    is practically desirable, 
    as it provides a quick way to navigate across grammatical systems.
}

Particles can be roughly divided according to the grammatical systems they express:
case markers (格助詞, kaku-joshi),
parallel markers (並立助詞, heiritsu-joshi),
sentence final particles (終助詞, shū-joshi),
interjectory particles (間投助詞, kantō-joshi),
adverbial particles (副助詞, fuku-joshi),
binding particles (係助詞, kakari-joshi),
conjunctive particles (接続助詞, setsuzoku-joshi),
and phrasal particles (準体助詞, juntai-joshi).
Here the term \term{particle} is used,
instead of, say, \term{clitic},
though we can only be sure that the case ending is a grammatical word 
(since it appears in the level of phrases),
partly because this is the tradition way to call it,
partly because TODO: whether particles can receive stress, etc.

TODO: hyperlink

Japanese lacks the prototypically pronoun class:
so-called pronouns are customized referential nouns like \translate{that girl},
and thus the pronoun class is not closed and strictly speaking is not a part of the grammar.
The article class is also not attested.

\subsection{Noun phrases}

\subsubsection{Unattested categories}

Gender and number are not grammatical categories in Japanese.



As is usually the case,
structural cases -- cases assigned directly by the vP-TP-CP functional hierarchy 
-- never appear together with the topic marker,
but peripheral cases are allowed to appear with the topic marker.

\subsection{Clauses}

\subsubsection{The template of clause structure}

Japanese is basically SOV in clause constituent order,
and it's also famous for allowing scrambling,
i.e. not-so-radical and often pragmatically marked constituent order deviation from the prototype.%
\footnote{
    Many scholars, especially those skeptical about generativism,
    often insist on ``analyzing non-canonical constituent order as they are''
    and reject the notion of scrambling.
    But there are indeed evidences -- from purely syntactic ones to psychological ones -- 
    for the existence of scrambling mechanism \citep{imamura2015effects,imamura2016processing,yatsushiro2003vp}.
    Since this note is mostly descriptive and is about surface-oriented analysis,
    scrambling is to be discussed in a mostly flat-tree way in the rest of this note,
    and yet I still keep an eye on the hierarchical effects.
}
In a scrambled clause, 
the predicate%
\footnote{
    In this note the term \term{predicate} takes the \ac{blt} meaning.
}
is still strictly at the final position 
(possibly with \acs{sfp}s following it).
The topic and the subject (or maybe we'd better call it focus -- see \prettyref{sec:alignment}), 
if any, are usually in the initial position.
They are dislocated only in marked cases.
On the other hand, 
internal arguments are scrambled in a freer manner:
the indirect object and the direct object can be switched without making a big difference.

\subsubsection{The predicate}\label{sec:predicate-abs}

The predicate position -- the so-called ``V'' in SOV -- 
can be filled either by a verb or by a verbal adjective,
plus a chain of function words.
The mapping between slots in the predicate and the generative TP-CP structure is almost one-to-one,
probably again as a result of the agglutinative nature of Japanese,
where functional morphemes are just speltout as they are with no further processing.
The template of the predicate is basically
the main verb plus a series of so-called ``auxiliary verbs'' 
plus possible negation marker plus tense marker,
and plus a possible \ac{sfp}. 
Verbal adjectives are incompatible with auxiliary verbs.

Still, the case of the Japanese predicate is not identical to the ideal agglutinative case,
because there is still interaction between function morphemes.
Previous works analyze this interplay in quite different ways.
The School Grammar assumes both the main verb and the ``auxiliary verbs'' may have different endings 
decided by morphemes in higher positions (that's to say, by morphemes following them),
as in the English periphrastic conjugation \corpus{have been doing sth.}
(\prettyref{sec:verb-complex-overview}).
It is, however, possible to use morphophonological rules 
in place of a certain portion of these selection-and-affix-lowering mechanisms,
mainly the irrealis form,
though probably not all,
because there are authentic auxiliary verbs whose verb status is well established 
appearing in the predicate,
and the continuative stem (for both lexical verbs and ``auxiliary verbs''),
if explained in purely morphophonological terms,
poses the somehow very unnatural constraint on the deep form of ``auxiliary verbs''
that they all begin with \corpus{i}.
There are also uncontroversial phonological rules acting between morphemes in the predicate.

In copular clauses,
\citet{tsutsui1989dictionary} analyzes the copula and the \acs{cc} together as the predicate,
which makes sense if we understood the \acs{cc} as the predicate,
and in this perspective, the copula is merely a particle indicating the predicate status of the \acs{cc},
but in another perspective, the copula is obviously a verb,
which takes the \acs{cc} as its internal argument.

\subsubsection{Arguments and the nominative-accusative alignment}\label{sec:alignment}

It doesn't take much effort to find that Japanese is nominative-accusative:
for intransitive verbs,
the S argument is marked in the same way as the A argument of transitive verbs,
usually by the \corpus{ga} particle or by the \corpus{wa} particle.
There are actually some subtlety regarding the marking of the S/A argument (\prettyref{sec:topic-subject}),
about the exact meaning of \corpus{ga} and \corpus{wa}.
I call \corpus{ga} the nominative particle instead of the focus particle 
and follow the tradition, to place it together with uncontroversial case particles,
while placing \corpus{wa} together with information structure marking particles TODO: exact name of the two types of particles,
and the justification is done in TODO

\subsubsection{Valency changing}

The so-called Japanese passive is actually affective:
in the affective construction, the A argument undergoes an action caused by others. 
Thus in the affective construction we can still see things like direct objects.

\subsubsection{\Acl{tam} markers}

Japanese has the following \ac{tam} categories:
a past/non-past tense distinction,
TODO: mood

Due to the good correspondence between the hierarchy of grammatical categories in the functional projection 
and the components in the verb complex,
we can deduce that in Japanese, the Mood hea -- I mean Dixon's \term{modality} -- 
is lower than the T head, as is the case in English, 
from the fact that the modality marker in Japanese is on the left side of the tense marker.

Many authors are restricted in the Latin paradigm of grammar writing 
and mix the two kinds of ``moods'' together.
In Japanese, the syntactic marking of modality and the syntactic marking of speech act 
should be split,
and speaking of the syntactic marking of speech act,
clause type like imperative, etc. should also be separated from more detailed speech act marking.
The latter is also the case in Chinese.

\subsubsection{Sentence final particles}

\acs{sfp}s are important in Japanese grammar:
TODO: sfp for relative clauses? (Or in other words, clause or sentence?)

\subsection{Remarkable features}

\subsubsection{Honorifics}

\subsection{Existing studies}

TODO: school grammar and education grammar

\section{Phonology and the kana system}

\subsection{Phonological rules}

This note is mainly about morphosyntax.
However, phonological rules seem to be significant in at least some parts of Japanese grammar.


\section{Noun phrases}

\subsection{Morphology}

For Sino-Japanese nouns, the prefix \corpus{go-} is used to add politeness.
For native nouns, the prefix \corpus{o-} is used instead.
TODO: how to write in kanji?

\subsection{Compound nouns}

Compound nouns may be formed by the particle \corpus{no}:
\corpus{rekishi no kurasu} \translate{history class}.

Some compound nouns can be formed without \corpus{no}.
Barron p. 27

\subsection{Suffixes}

\corpus{-ya}: the place things are sold; the person organizing it

\corpus{-ka}: expert

\subsection{Final particles}

There are basically two systems of particles after \ac{np}s.
The first is the case system (\concept{case markers}, 格助詞), including
TODO: distribution (and the following ones)
\begin{itemize}
    \item Nominative: \corpus{ga}, 
    appearing in certain circumstances as the focus marker (\prettyref{sec:topic-subject}).
    \item Accusative: \corpus{o}
    \item Dative: \corpus{ni}: time and location 
    \item Genitive: \corpus{no} 
    \item Lative: \corpus{e}, used for destination direction (like in "to some place")
    \item Ablative: \corpus{kara}, used for source direction (like in "from some place")
    \item Instrumental/Locative: \corpus{de}
\end{itemize}

The second system is the 

The systems are not completely compatible.
A well known generalization is structural case markers are erased when \ac{np}s are topicalized,
while inherent case markers may be kept.


\section{The verb complex}

Due to the chaos created by previous approaches,
and the puzzling nature of Japanese verbal morphology,
this section is full of argumentation.

\subsection{Problems and previous studies}\label{sec:verb-complex-overview}

The verb complex -- the content of the predicate position -- is discussed in this section.
Since the verb complex is the collective realization of the vP-TP-CP functional projection,
a change in the verb complex may also be a change in 
the arguments (\prettyref{sec:arguments}),
TODO: list all of them.
I follow the practice in \citet{jacques2021grammar}
and use the verb complex as the table of content of what happens in the clause.

One astonishing aspect of Japanese is 
each verbal element -- the main verb or the main verbal adjective, so-called ``auxiliary verbs'' (see below) -- 
in the verb complex is in one of a handful of so-called conjugation forms (the name is problematic -- see below),
if we believe the analysis of the School Grammar.
This shared morphological appearance may be seen as justification of the term \term{auxiliary verb},
since they have similar appearances as lexical verbs,
though they can also be perceived as inflectional endings 
which apply to verbs already with certain degrees of inflection,
which is even more likely 
because elements in the predicate are not subject to further syntactic operation
(see below discussion on Education Grammar).
From then on, I will call all morphosyntactic objects 
appearing in the predicate and have verb-like inflectional behaviors \concept{verbal} elements,
regardless of whether they are lexical or functional.

These conjugation forms have names like ``irrealis'' or ``hypothetical'',
but the names of these forms doesn't say much about their distributions,
except the imperative form (used to convey a rather rude and direct command).
Thus, if we believe in the analysis of the School Grammar,
the so-called conjugation forms are better understood as the 
so-called English \corpus{ing}-participle and \corpus{ed}-participle forms 
in \corpus{have been doing},
which would be better glossed as 
\corpus{have-en be-ing do} plus phonetic readjustment rules%
\footnote{
    Here we assume a trivial spellout rule set,
    and let morphophonological rules -- local dislocation of the position of suffixes --
    do the heavy-lifting jobs.
    A more popular approach recently is to move the burden to the spellout rules.
    For English \corpus{have been being done},
    we may posit the following rules.
    The auxiliary verb \corpus{being} is the default spellout of the progressive aspect:
    if \corpus{-ing} is close to a verb stem, then it goes after that verb stem,
    but in \corpus{being done}, the verb stem has already been incorporated into \corpus{done},
    so \corpus{-ing} is spelt out as \corpus{being}.
    Similarly we have \corpus{been}, which is the default spellout of the so-called 
    Asp$_{\text{\corpus{en}}}$ head \citep{ramchand2014152}.
    Thus, the ``stems'' of auxiliary verbs in the English verb phrase 
    are actually inserted later as a last resort,
    and this agrees with the ``semantic bleaching'' picture:
    the stems of auxiliary verbs, like \corpus{be}, 
    have already lost all semantic content,
    and hence should not appear in the syntax proper 
    (or otherwise they still receive semantic interpretation at LF).
    What remain are just phonetic forms.

    To apply this approach to Japanese,
    we need to notice that now it's the stems of ``auxiliary verbs'' that contains grammatical information,
    and now the continuative ending \corpus{-i} is inserted as a last resort.
    In other morphosyntactic environments,
    like when being close to T and Neg,
    the \corpus{-i} is removed and replaced by an appropriate tense-polarity-honorifics complex.
    This analysis actually has some surface-oriented significance (TODO: ref).
}\label{fn:spellout-based}
if we want an one-to-one correspondence between morphemes and grammatical categories.
What carry grammatical categories are \emph{not} the so-called conjugation ending of 
the irrealis form, the continuative form, etc.,
but the verbal elements themselves -- 
in the case of English,
\corpus{have} (present perfect) and \corpus{be} (passive).

On the other hand, the so-called ``conjugation ending'' of each element in the verb complex 
is \emph{not} determined by agreement with the arguments 
or by clausal grammatical categories:
the former never exists in Japanese and the latter is coded by the sequence of functional verbal items,
where for inner verbal elements,
the conjugation form is purely decided 
by the higher verbal element (in Japanese, the closest verbal element in the right), 
arguably by some sorts of ``affix lowering''.
This periphrastic inflectional strategy in English \corpus{have been doing} 
is used in Japanese in a much larger scale,
because there are lots of auxiliary verbs.

What makes Japanese and English different 
is the verb complex in Japanese doesn't allow intervening of things like adverbs,
while in English it's pretty acceptable.
Thus, the morphosyntactic standard of wordhood hints that 
the whole verb complex may be understood as a huge grammatical word,
since once finished, it never interacts with the outer world.
This indeed seems to be the convention used in many romanization solutions.

If the above is the whole story,
then the verb complex of Japanese is a mixture of two conjugation strategies:
its inner makeup has certain periphrastic flavor as in \corpus{has been doing},
and yet adjunction operations typically seen in periphrastic conjugation are absent, 
and the so-called ``auxiliary verbs'' are close to suffixes instead of words.
This of course isn't completely impossible in principle:
we may assume there is an PF rule requiring all verbal elements to spellout together,
and any argument in the functional hierarchy is squeezed out.
Indeed, it's said in certain varieties of spoken French, 
the tendency of squeezing large arguments (the pronouns are still there) 
out of the sequence of verbal elements 
has been there for a while,
rendering some linguists asserting we are witnessing the emergence of a new polysynthetic language.

What's unusual for Japanese, if we believe the whole story told by the School Grammar, is 
first the astonishing lack of semantic significance of the so-called stem forms,
and second, despite this,
even after the most internal grammatical items (like the passive or causative markers),
the verb stem is in the continuative form, 
and for so-called ``auxiliary verbs'' appearing after other ``auxiliary verbs'',
the former are almost always in the continuative form.
So actually, we may assume that 
the continuative form is actually the ``natural'' form,
and the terminal form is obtained by an Item-and-Process (or may be Word-and-Paradigm) way:
by replacing \corpus{-(i)} by \corpus{-(r)u},
and there is actually no position for the ``verb stem'' in the mental grammar.%
\footnote{
    Indeed, it's common to find that the strictly morpheme-based analysis
    is often further from the generative analysis,
    though the latter is often said to be morpheme-based,
    and the former is usually poorly theoretically informed (\prettyref{fn:morpheme-ill-informed}).
    In the current case, the generative analysis is in \prettyref{fn:spellout-based},
    and a quick examination of \prettyref{fn:spellout-based} tells us 
    its equivalence is just ``T and Neg projections induce changing \corpus{-i} to \corpus{-u}''.
}

The problem of the ambiguity between words and suffixes with the School Grammar
inspired a group of scholars to find an alternative with a morphological approach to the verb complex,
and the result is what's known as Education Grammar today.
The School Grammar analyzes the verb complex in a highly detailed and semi-analytical way,
segmenting it into the verb stem, a list of auxiliary verbs, the negative marker, and the tense marker,
each of them is in one of the ``conjugation forms'',
while the Education Grammar analyzes the verb complex 
simply as the stem plus \emph{one} long, long conjugation ending.
The status of the so-called \corpus{te}-form of verbs as one of the basic conjugation forms, for example,
is rejected by the School Grammar:
it's analyzed as the conjunctive form plus \corpus{-te},
with phonological readjustment (TODO: ref).
This means we may find more conjugation forms in Education Grammar books.

The main problem with Education Grammar is it comes to another extreme,
completely ignoring the inner structure of the verb complex.
(And out of some short-sighted considerations like 
``making second-language learners be able to at least pick up \emph{some} Japanese in one lesson'',
the \corpus{masu}-form is introduced too early,
ending up in endless confusions.)
The approach taken in this note is a third way.
I recognize that the true conjugation endings of a lexeme 
are not just a handful of ``conjugation forms'' mentioned in School Grammar,
and the term \corpus{conjugation ending} is used to denote 
any possible combination of function items in the verb complex,
and hence I recognize the so-called \corpus{te}-form is a conjugation form,
so if a chain of function morphemes has developed a conventionalized meaning,
it's easy to just describe it as a newly emerging conjugation form.
On the other hand,
the possible forms of the verb complex are decomposed as detailed as possible,
and I'll still study auxiliary verbs,
though they are parts of a complex word, not a phrase.
Contrary to the School Grammar, however,
I'll go into the syllables and allow the deep form of morphemes -- lexical or functional --
to begin and end in consonants.
As we will see later,
this actually reduces the necessity of posting 
``conjugation forms for each verbal element in the verb complex'',
as so-called endings of verbal elements 
are frequently about \emph{morphophonological} instead of morphosyntactic interaction between morphemes,
though there is also morphosyntactic interaction.%
\footnote{
    This is actually a methodological lesson.
    In principle, the ideal workflow is to first recognize phonological rules
    (in this step morphosyntactic analysis is of course needed),
    and after morphemes in the deep structure are recognized,
    bracketing them into words and phrases should be done
    without depending much on phonology.
    In practice, though, linguists often implicitly 
    consider large phonological units as morphosyntactic units as well 
    without questioning if there is mismatch between phonological boundary and morphosyntactic boundary,
    and this often ends up in the failure of recognizing the exact meaning of alleged ``morphemes''.
    In the case of Japanese,
    criterions like ``being made of one or more complete mora''
    are for phonological words, not morphosyntactic words.
    School Grammarians, however, never go into the structure of syllables,
    and they simply take the claim that a morphosyntactic unit contains integer syllables for granted.
    The result is the ``conjugation forms'' of verbal elements are completely incomprehensible.
    And if we \emph{then} try to investigate the inner structure of School Grammar ``verbal elements'',
    we find nothing that can deepen our understanding about why things are so.
    As Chomsky pointed out in his early age,
    American structuralist partition of speech purely with intuition 
    -- a mixture of phonology and morphosyntax --
    is theoretically bad-informed.
}\label{fn:morpheme-ill-informed}
There are actually some evidences supporting this
if we have a closer look to the comparison between the two conjugation classes 
(\prettyref{sec:consonant-class}, \prettyref{sec:vowel-class}).
Thus, the superficial resemblance between ``auxiliary verbs'' and lexical verbal words 
is merely a reflection of the same phonological rules applied to the verb complex
(\prettyref{sec:conjugation-compare}),
and hence ``auxiliary verbs'' lose their morphosyntactic status as verbal elements 
and are rightfully renamed as conjugation suffixes.

Here is a list of School Grammar ``conjugation forms'' and relevant distributional information,
and tentative explanations of their underlying origins:
\begin{itemize}
    \item Irrealis form (未然形). 
    This form appears in some negative forms, TODO: ref
    as well as in the causative and passive constructions 
    and the combination of the two. TODO: ref
    This narrow distribution means 
    \item Volitional form (意志形), which historically originates from the irrealis form. TODO
    \item Continuative form or adverbial form (連用形), 
    sometimes also called the infinitive form
    for its frequent appearance after other verbal elements.
    \item Euphonic form (音便形), 
    historically originates from the continuative form plus suffixes starting with \corpus{t}
    (TODO: ref).
    This form is posited by some grammarians 
    who insist on a largely concatenative analysis of verbal morphology 
    (because otherwise the morphological rules are hard to express within the kana system).
    \item Terminal form or dictionary form (終止形, 辞书形)
    \item Attributive form (連体形), which, for regular verbs and verbal adjectives, 
    is the same as the dictionary form in modern Japanese.
    TODO: Nominal adjectives and copula 
    \item Hypothetical form (仮定形). When used as condition
    \item Imperative form (命令形)
\end{itemize}
The imperative form, the terminal form and the attributive form are somehow ``finalized'' forms: 
there can be no pure auxiliary verbs following them,
though semi-auxiliary verbs like \corpus{dar\={o}}, certain particles like the \ac{sfp}s are still possible.
The former case, however, may be argue to be a complement clause construction.

In the perspective of the School Grammar, the irrealis form, the continuative form, etc. 
are actually conjugation stems.
But this term goes in conflict with my notion of \term{stem} in \prettyref{sec:conjugation-class},
and may also create confusion because auxiliary verbs also have these forms 
but it's rather strange to use the term \term{stem} for function items.
So in the following,
I will simply call the five to seven forms listed above (depending on how you count them)
as \concept{School Grammra conjugation forms}.

\subsection{Conjugation classes and verb stems}\label{sec:conjugation-class}

\subsubsection{The consonant and vowel conjugations}

According to whether the final sound of the stem is a vowel or consonant,
Regular Japanese verbs can be divided into \concept{c-stem verbs} and \concept{v-stem verbs}.
C-stem verbs are also called 五段動詞 or \concept{group-1 verbs},
because changing the conjugation ending means the final mora may appear in every row of the kana chart, 
and v-stem verbs are also called 一段動詞 or \concept{group-2 verbs},
because the last or the second but last mora is always in the same row with the last mora in the dictionary form.
一段動詞 can be further divided into 上一段動詞 or \corpus{iru}-verbs or \concept{group-2a verbs} 
and 下一段動詞 or \corpus{eru}-verbs or \concept{group-2b verbs}:
the final vowel of \corpus{iru}-verbs is \corpus{i},
and \corpus{-ru} is actually the conjugation ending of the terminal form,
and similarly the final vowel of \corpus{eru}-verbs is \corpus{e}.

There are two important irregular verbs: \corpus{suru} \translate{to do} and \corpus{kuru} \translate{to come}.
TODO: where to discuss them, 
The two verbs, together with their compounding with TODO,
are collectively called \concept{group-3 verbs}.

\subsubsection{The consonant conjugation}\label{sec:consonant-class}

A verb whose dictionary form is not in group-3 and doesn't end in \corpus{-eru} or \corpus{-iru} 
is definitely a c-stem verb, i.e. a group-1 verb.
Basically, by removing the final \corpus{-u}, we get the stem;
if a verb ends in \corpus{-tsu},
then the stem ends in \corpus{-t-}.
Verbs ending in \corpus{-au}, \corpus{-iu}, and \corpus{-ou} are also c-stem verbs,
because we can assume there is a hidden \corpus{w} before \corpus{u},
which appears in some of the negative forms. TODO
A handful of verbs ending in \corpus{-eru} and \corpus{-iru} are also c-stem verbs. TODO

When doing conjugation, Phonological rules deserve attention.
If the stem ends in \corpus{-s-}, 
then it becomes \corpus{-sh-} before a suffix beginning with \corpus{i}.
If the stem ends in \corpus{-t-},
it becomes \corpus{-ch-} before a suffix beginning with \corpus{i}.%
\footnote{
    \citet{akiyama2012japanese} defines the stem as the result of directly removing the final \corpus{u},
    and thus we would have a \corpus{-ts-} to \corpus{-t-} rule 
    for anything else than the dictionary form or a form with a suffix beginning with \corpus{i}.
    I find this analysis uneconomical,
    since the \corpus{t}-to-\corpus{ts} change is easily explained by palatalization,
    while the inverse is mysterious.
}

Here is the relation between the stem and the School Grammar conjugation forms:
\begin{itemize}
    \item The continuative form of a c-stem verb is obtained by adding \corpus{-i} to the stem.
    \item The terminal form and the attributive form is obtained by adding \corpus{-u}.
    Note that for v-stem verbs the ending is \corpus{-ru}.
    Since Japanese doesn't allow two successive consonants,
    we may assume there is a phonological rule deleting the \corpus{r} sound for c-stem verbs.
    Thus, the terminal form is just the stem plus \corpus{-ru}.
    \item The hypothetical form is obtained by adding \corpus{-e}.
    \item The irrealis form is obtained by adding \corpus{-a}.
    \item The volitional form is obtained by adding \corpus{-o}.
\end{itemize}
I highlight again here that the names hint little about their distributions.
The name \term{irrealis form} makes sense in negation,
but it means little in a causative construction.
These final vowels are likely to reflect 
the initial vowel of the following inflectional suffix,
or in the case that the suffix starts with a consonant,
the vowel of the initial syllable
(\prettyref{sec:vowel-class}, \prettyref{sec:conjugation-compare}).
The case of consonant-initial suffix sometimes gives rise to more complicated morphophonological phenomena,
which are not purely concatenative in the surface form (TODO: ref). 

\subsubsection{The vowel conjugation}\label{sec:vowel-class}

What aren't group-3 verbs and group-1 verbs are v-stem verbs, i.e. group-2 verbs.
The stem is obtained by dropping the final \corpus{-ru}.
The rest of the School Grammar conjugation forms of a v-stem verb are the same as the stem,
except the imperative form,
which is formed by adding \corpus{-ro} to the stem 
(\corpus{-yo} is used as an alternative in writing).

This pattern, compared with the School Grammar conjugation of c-stem verbs, 
hints the nature of School Grammar conjugation:
the vowels appearing after the stem of c-stem verbs 
may just be initial vowels of the following inflectional suffixes 
i.e. auxiliary verbs in the School Grammar,
or simply inflectional suffixes themselves.
Since Japanese has five vowels (TODO: more precise description),
this results in exactly five so-called conjugation forms
if we insist that the verb must contain complete syllables,
ending in \corpus{a}, \corpus{e}, \corpus{i}, \corpus{o}, and \corpus{u}, respectively,
and hence the name 五段動詞.
Historically, Japanese had a tendency to avoid successive vowels.
This restriction has already been dropped in modern Japanese,
but we can expect its residue still works in the verb complex.
Thus, the initial vowel of any inflectional suffix following a v-stem verb is dropped,
and compared with c-stem verbs,
the School Grammar conjugation forms of v-stem verbs 
don't really reflect the underlying vowel of a suffix following them.

As is said in \prettyref{sec:consonant-class},
the dictionary form seems to be simply the verb stem plus \corpus{-ru},
with possible phonological transformations.
Adding a suffix starting with a consonant to a v-stem is trivial.
It's not so trivial for c-stems,
because since Japanese strictly forbids successive consonants,
there must be phonological rules resolving the conflict.
Different readjustment rules give rise to different final vowels 
of the alleged verb conjugation form in the School Grammar
(and if a suffix induces a certain School Grammar conjugation form for a c-stem verb,
say, the conjunctive form,
then the School Grammar says the same suffix is compatible only with the conjunctive form of v-stem verbs,
though the suffix actually does nothing),
or sometimes more complicated cases.
I'll tak about them later in \prettyref{sec:conjugation-compare}.

\subsubsection{Comparison between and unification of the two conjugations}\label{sec:conjugation-compare}

\prettyref{fig:vowel-start-suffix-verb} summarizes the morphophonological behaviors 
of the v-stem and c-stem after an inflectional suffix starting with a vowel in the deep structure.
The syllable segmenting and vowel dropping rules mean the realization of the second morpheme
(colored as green, without initial vowel and therefore contains complete syllables)
happens to be the same after the first morpheme,
regardless of the ending of the first morpheme,
while the initial vowel of the second morpheme
decides the School Grammar ``conjugation form'' of the first morpheme.
Note that this procedure doesn't consider the nature of the first morpheme:
as long as we are in the verb complex,
the result of the mechanism in this figure remains the same.
This means in the School Grammar,
if morpheme 1 is itself an inflectional suffix,
its morphological behaviors resemble those of lexical verbs.

\begin{figure}[H]
    \centering
    

\tikzset{every picture/.style={line width=0.3pt}} %set default line width to 0.75pt        

\begin{tikzpicture}[x=0.75pt,y=0.75pt,yscale=-0.8,xscale=0.8]
%uncomment if require: \path (0,419); %set diagram left start at 0, and has height of 419

%Shape: Rectangle [id:dp10799671156840329] 
\draw   (127,59) -- (242,59) -- (242,87.26) -- (127,87.26) -- cycle ;
%Shape: Rectangle [id:dp7035738070234729] 
\draw  [draw opacity=0][fill={rgb, 255:red, 255; green, 255; blue, 255 }  ,fill opacity=1 ] (75,51) -- (145,51) -- (145,101.26) -- (75,101.26) -- cycle ;
%Shape: Rectangle [id:dp9892285153067764] 
\draw   (367,59) -- (252,59) -- (252,87.26) -- (367,87.26) -- cycle ;
%Shape: Rectangle [id:dp10620314732910852] 
\draw  [draw opacity=0][fill={rgb, 255:red, 255; green, 255; blue, 255 }  ,fill opacity=1 ] (419,51) -- (349,51) -- (349,101.26) -- (419,101.26) -- cycle ;
%Straight Lines [id:da8787945793208012] 
\draw [color={rgb, 255:red, 74; green, 144; blue, 226 }  ,draw opacity=1 ][line width=1.5]    (141,328) -- (269,328) ;
%Straight Lines [id:da8747858480961948] 
\draw [color={rgb, 255:red, 80; green, 227; blue, 194 }  ,draw opacity=1 ][line width=1.5]    (275,328) -- (332,328) ;
%Shape: Rectangle [id:dp6953628572273307] 
\draw   (395,59) -- (510,59) -- (510,87.26) -- (395,87.26) -- cycle ;
%Shape: Rectangle [id:dp7297753746089688] 
\draw  [draw opacity=0][fill={rgb, 255:red, 255; green, 255; blue, 255 }  ,fill opacity=1 ] (343,51) -- (413,51) -- (413,101.26) -- (343,101.26) -- cycle ;
%Shape: Rectangle [id:dp6894546413420739] 
\draw   (635,59) -- (520,59) -- (520,87.26) -- (635,87.26) -- cycle ;
%Shape: Rectangle [id:dp45213728860443014] 
\draw  [draw opacity=0][fill={rgb, 255:red, 255; green, 255; blue, 255 }  ,fill opacity=1 ] (687,51) -- (617,51) -- (617,101.26) -- (687,101.26) -- cycle ;
%Shape: Rectangle [id:dp3701432068489623] 
\draw  [draw opacity=0][fill={rgb, 255:red, 255; green, 255; blue, 255 }  ,fill opacity=1 ] (419,167) -- (349,167) -- (349,217.26) -- (419,217.26) -- cycle ;
%Shape: Rectangle [id:dp817155790614746] 
\draw   (395,175) -- (510,175) -- (510,203.26) -- (395,203.26) -- cycle ;
%Shape: Rectangle [id:dp37068543633689766] 
\draw  [draw opacity=0][fill={rgb, 255:red, 255; green, 255; blue, 255 }  ,fill opacity=1 ] (343,167) -- (413,167) -- (413,217.26) -- (343,217.26) -- cycle ;
%Shape: Rectangle [id:dp6500545934497894] 
\draw   (635,175) -- (520,175) -- (520,203.26) -- (635,203.26) -- cycle ;
%Shape: Rectangle [id:dp2082546715911775] 
\draw  [draw opacity=0][fill={rgb, 255:red, 255; green, 255; blue, 255 }  ,fill opacity=1 ] (687,167) -- (617,167) -- (617,217.26) -- (687,217.26) -- cycle ;
%Straight Lines [id:da5254065506928987] 
\draw [color={rgb, 255:red, 74; green, 144; blue, 226 }  ,draw opacity=1 ][line width=1.5]    (436,328) -- (506,328) ;
%Straight Lines [id:da8747110539789369] 
\draw [color={rgb, 255:red, 80; green, 227; blue, 194 }  ,draw opacity=1 ][line width=1.5]    (545,328) -- (602,328) ;

% Text Node
\draw (205,72.5) node [anchor=west] [inner sep=0.75pt]   [align=left] {V};
% Text Node
\draw (224.17,72.5) node [anchor=west] [inner sep=0.75pt]   [align=left] {C};
% Text Node
\draw (173,72.33) node [anchor=west] [inner sep=0.75pt]   [align=left] {$\displaystyle \cdots $};
% Text Node
\draw (257,72.5) node [anchor=west] [inner sep=0.75pt]   [align=left] {V};
% Text Node
\draw (276.17,72.5) node [anchor=west] [inner sep=0.75pt]   [align=left] {C};
% Text Node
\draw (296,72.33) node [anchor=west] [inner sep=0.75pt]   [align=left] {$\displaystyle \cdots $};
% Text Node
\draw (143,117) node [anchor=north west][inner sep=0.75pt]   [align=left] {morpheme 1};
% Text Node
\draw (263,117) node [anchor=north west][inner sep=0.75pt]   [align=left] {morpheme 2};
% Text Node
\draw (205,306.5) node [anchor=west] [inner sep=0.75pt]  [color={rgb, 255:red, 74; green, 144; blue, 226 }  ,opacity=1 ] [align=left] {V};
% Text Node
\draw (224.17,306.5) node [anchor=west] [inner sep=0.75pt]  [color={rgb, 255:red, 74; green, 144; blue, 226 }  ,opacity=1 ] [align=left] {C};
% Text Node
\draw (173,306.33) node [anchor=west] [inner sep=0.75pt]  [color={rgb, 255:red, 74; green, 144; blue, 226 }  ,opacity=1 ] [align=left] {$\displaystyle \cdots $};
% Text Node
\draw (257,306.5) node [anchor=west] [inner sep=0.75pt]  [color={rgb, 255:red, 74; green, 144; blue, 226 }  ,opacity=1 ] [align=left] {V};
% Text Node
\draw (276.17,306.5) node [anchor=west] [inner sep=0.75pt]  [color={rgb, 255:red, 80; green, 227; blue, 194 }  ,opacity=1 ] [align=left] {C};
% Text Node
\draw (296,306.33) node [anchor=west] [inner sep=0.75pt]  [color={rgb, 255:red, 80; green, 227; blue, 194 }  ,opacity=1 ] [align=left] {$\displaystyle \cdots $};
% Text Node
\draw (16,60) node [anchor=north west][inner sep=0.75pt]   [align=left] {deep\\form};
% Text Node
\draw (16,292) node [anchor=north west][inner sep=0.75pt]   [align=left] {School\\Grammar};
% Text Node
\draw (491.75,72.5) node [anchor=west] [inner sep=0.75pt]   [align=left] {V};
% Text Node
\draw (441,72.33) node [anchor=west] [inner sep=0.75pt]   [align=left] {$\displaystyle \cdots $};
% Text Node
\draw (525,72.5) node [anchor=west] [inner sep=0.75pt]   [align=left] {V};
% Text Node
\draw (544.17,72.5) node [anchor=west] [inner sep=0.75pt]   [align=left] {C};
% Text Node
\draw (564,72.33) node [anchor=west] [inner sep=0.75pt]   [align=left] {$\displaystyle \cdots $};
% Text Node
\draw (411,117) node [anchor=north west][inner sep=0.75pt]   [align=left] {morpheme 1};
% Text Node
\draw (531,117) node [anchor=north west][inner sep=0.75pt]   [align=left] {morpheme 2};
% Text Node
\draw (16,192) node [anchor=north west][inner sep=0.75pt]   [align=left] {vowel \\deletion};
% Text Node
\draw (491.75,188.5) node [anchor=west] [inner sep=0.75pt]   [align=left] {V};
% Text Node
\draw (441,188.33) node [anchor=west] [inner sep=0.75pt]   [align=left] {$\displaystyle \cdots $};
% Text Node
\draw (544.17,188.5) node [anchor=west] [inner sep=0.75pt]   [align=left] {C};
% Text Node
\draw (564,188.33) node [anchor=west] [inner sep=0.75pt]   [align=left] {$\displaystyle \cdots $};
% Text Node
\draw (411,236) node [anchor=north west][inner sep=0.75pt]   [align=left] {morpheme 1};
% Text Node
\draw (531,236) node [anchor=north west][inner sep=0.75pt]   [align=left] {morpheme 2};
% Text Node
\draw (491.75,306.5) node [anchor=west] [inner sep=0.75pt]  [color={rgb, 255:red, 74; green, 144; blue, 226 }  ,opacity=1 ] [align=left] {V};
% Text Node
\draw (441,306.33) node [anchor=west] [inner sep=0.75pt]  [color={rgb, 255:red, 74; green, 144; blue, 226 }  ,opacity=1 ] [align=left] {$\displaystyle \cdots $};
% Text Node
\draw (544.17,306.5) node [anchor=west] [inner sep=0.75pt]  [color={rgb, 255:red, 80; green, 227; blue, 194 }  ,opacity=1 ] [align=left] {C};
% Text Node
\draw (564,306.33) node [anchor=west] [inner sep=0.75pt]  [color={rgb, 255:red, 80; green, 227; blue, 194 }  ,opacity=1 ] [align=left] {$\displaystyle \cdots $};
% Text Node
\draw (146,341) node [anchor=north west][inner sep=0.75pt]  [color={rgb, 255:red, 74; green, 144; blue, 226 }  ,opacity=1 ] [align=left] {\begin{minipage}[lt]{80.99pt}\setlength\topsep{0pt}
\begin{center}
School Grammar:\\five forms
\end{center}

\end{minipage}};
% Text Node
\draw (392,341) node [anchor=north west][inner sep=0.75pt]  [color={rgb, 255:red, 74; green, 144; blue, 226 }  ,opacity=1 ] [align=left] {\begin{minipage}[lt]{80.99pt}\setlength\topsep{0pt}
\begin{center}
School Grammar:\\one forms
\end{center}

\end{minipage}};


\end{tikzpicture}

    \caption{Morphophonological processing of two neighboring morphemes in the verb complex
    when the second morpheme starts with a vowel, 
    and the School Grammar analysis.}
    \label{fig:vowel-start-suffix-verb}
\end{figure}

The case of suffixes with initial consonants is more complicated.
Japanese has a set of rules eliminating CC-type of sound sequences.
The simplest type is to remove the second consonant:
this happens when the second consonant is \corpus{r} or \corpus{s} (TODO: more),
and this results in,
say, the irrealis form in valency changing constructions (\prettyref{sec:valency-changing-form}).
Since the consonant deletion doesn't happen for v-stem verbs,
a single suffix starting with a consonant
splits into two surface forms:
the one for v-stem verbs is close to its deep form,
while the one for c-stem verbs loses its first syllable
(the consonant is deleted, and the vowel is merged into the preceding verb).
Things are not that simple, however, for the so-called \corpus{te}-form and \corpus{ta}-form:
these cases are still analyzed by the School Grammar as 
conjunctive form plus auxiliary verb,
though with a largely regular phonological readjustment.
This process is an instance of 音便 in conjugation,
which essentially creates a new form in the School Grammar:
the euphonic form (\prettyref{sec:verb-complex-overview}).
TODO: rule on \citet[\citepage{99}]{akiyama2012japanese}

Thus, the School Grammar conjugation forms are deciphered 
into the following morphophonological environments:
\begin{itemize}
    \item Irrealis form: concatenating with \corpus{-ana-} (TODO), 
    \corpus{-sase-}, and \corpus{-rare-} (\prettyref{sec:valency-changing-form}). 
    \item Dictionary form: concatenating with \corpus{-ru},
    with deletion of \corpus{r} after a c-stem (\prettyref{sec:consonant-class}, \prettyref{sec:vowel-class}).
    \item Imperative form:
    \item Conjunctive form: TODO!!! Where does \corpus{i} come from?
    \citet{volpe2005japanese} says nothing about its origin -- 
    it just discuses true nominalization.
    \citet{fukuda2012aspectual} discusses the structure of the verbal functional hierarchy,
    but details in the spellout is skipped.
\end{itemize}
To say an School Grammar auxiliary verb selects a ``conjugation form''
is equivalent to say the ``auxiliary verb'', in the deep structure,
creates the corresponding morphophonological environment:
thus to say an ``auxiliary verb'' selects the irrealis form 
is equivalent to say the deep form of the ``auxiliary verb'' 
starts either with \corpus{a} or with \corpus{Ca}.

On the other hand, the prevalence of the continuative form seems to be an evidence 
for its real presence in morphosyntax TODO: really??

\subsubsection{Group-3 verbs}

dictionary form continuative form 

kuru ki 

suru shi

\subsection{Inner structure: valency changing}\label{sec:valency-changing}

\subsubsection{Morphology}\label{sec:valency-changing-form}

The passive suffix is \corpus{-rare-},
and the causative suffix is \corpus{-sase-}.
When applied upon c-stem verbs,
the \corpus{r} and \corpus{s} sounds are dropped,
and the School Grammar therefore analyzes 
the passive of c-stem verbs as the irrealis form plus \corpus{-re-},
and similarly,
the causative of c-stem verbs is the irrealis form plus \corpus{-se-}.

It's also possible to apply the passive to the causative:
we get \translate{be made to do sth.}
In this case, the suffix is the expected 
\corpus{-saserare-},
called the passive causative.
Again, the \corpus{s} sound is dropped when used together with a c-stem verb.
The School Grammar says in this case,
the \corpus{-sase-} is in the irrealis form,
because \corpus{-rare-} applies to the irrealis form of c-stem verbs 
and hence any v-stem verbs appearing before \corpus{-rare-} is also in the irrealis form.

\subsection{The tense, honorifics, polarity complex}

In the Education Grammar,
the categories of tense, honorifics and polarity 
are deemed as realized by a single, fused ending without analyzable inner structure.
This is adequate for descriptive usage,
since these categories never appear more than once in a verb complex.

\subsubsection{The all-in-one paradigm}



\corpus{nai}: adjective conjugation

\subsubsection{Honorifics}

TODO: -masu = -mas-ru, -mashita = -mas-ta (or mas-ita)?
-mashi- is the conjunctive form (for comparison, see \citet[\citepage{99}]{akiyama2012japanese}), but why here is the conjunctive form?
What's the origin of the \corpus{i}?

\subsubsection{Negation and morphology-syntax mismatch}

\citet{spencer2008negation}: This paper seems to assume the auxiliaries in the verb complex 
are ``verbal'' in morphology,
because the author is just doubtful about whether the negator looks like an adjective:
he would not be confused if its conjugation were verb-like.

\subsubsection{Tense}

In the affirmative polarity (TODO: any condition else?),
the non-past tense is marked by \corpus{-ru},
while the past tense is marked by \corpus{-ta} (TODO: The School Grammar says the past is marked by the conjunctive plus \corpus{ta}. What does the conjunctive mean? Morphophonological, or morphosyntactic? And what's the underlying form of \corpus{-ta}?).
The two suffixes are subject to multiple phonological rules.


\subsection{periphrastic conjugations}

\subsubsection{The progressive}

\section{Argument structures and alignment}\label{sec:arguments}

This section is about subcategorization frames of verbs and the alignment.

\subsection{Copular clauses}

The \ac{cc} of the copula \corpus{desu} doesn't have a case particle:
this persuades some people to analyze it as a part of the predicate.
TODO: what about ``the kind of de-education makes him a monster''?

\subsection{Canonical transitive and intransitive verbs}

\section{Simple clauses}

\subsection{Topic and subject}\label{sec:topic-subject}

The subject is often said to be both agentive and topic-like in the typological literature.
What makes Japanese different is despite its accusative nature,
there is still a problem concerning what exact is the subject in Japanese.
The \corpus{wa} particle and the \corpus{ga} particle are commonly called 
the topic particle and the subject particle, respectively.
This correspondence fails the predicate has a stative or habitual meaning:
in that case, \corpus{ga} has the meaning of \translate{it's \dots that},
usually with an exhaustive meaning -- 
only the \ac{np} before \corpus{ga} satisfies the predicate,
and nothing else.
This is probably due to the fact \corpus{ga} assigns focus to the \ac{np} before it:
in English, \corpus{YOU do the job} means it's you -- and only you -- who does the job.
Thus, \corpus{ga} may be better glossed as a focus marker,
and this means in the surface-oriented constituent structure of main clauses, 
there is no such thing as the subject: 
anything moves to SpecTP immediately moves to a higher position in CP, 
making the notion \term{subject} a latent concept without its own particle.
Indeed, we have so-called adverbial nominative in Japanese \citet[\citesec{6.1}]{endo2007locality}.
Note, however, that the alignment in relative clauses is quite simple:
in relative clauses the topic marker \corpus{wa} never appears 
and \corpus{ga} is simply the marker of nominative case.

Also, this perception of the \corpus{wa}-\corpus{ga} distinction, 
doesn't cover some important aspects even in the main clause.
It's possible to have 
\begin{exe}
    \ex Ame ga hutte imasu. \translate{The rain is falling.}
\end{exe}
But this sentence doesn't focus on the rain -- 
it's just a casual description of the weather.
On the other hand, the clause 
\begin{exe}
    \ex Ame wa hutte imasu, \dots
\end{exe}
has a contrastive meaning: 
it means \translate{the rain is falling, but \dots},
probably used together with \translate{but there is no snow today}.
So, although we just conclude that \corpus{wa} is a topic marker 
(and is about old information),
in certain uses it definitely brings the attention of the listener to the \ac{np} it marks.
Thus, if we start from the above two examples, 
we may conclude that \corpus{ga} is indeed a neutral subject marker,
while \corpus{wa} involves more pragmatic marking,
which is further supported by the fact that 
\corpus{wa}-\ac{np}s are harder to question:
\begin{exe}
    \ex \begin{xlist}
        \ex Dare ga kimasita ka?
        \ex *Dare wa kimasita ka?
    \end{xlist}
\end{exe}
A further piece of evidence is in subordinated clauses
(of course, for quoted speech etc., the function of \corpus{wa} and \corpus{ga} 
is the same as the case in the matrix clause),
\corpus{ga} has a quite clear function -- the subject marker --
and \corpus{wa} only appears when there is contrast between two clauses.
Thus, the mysterious uses of the two particles appear to be a problem at the sentence level.

If we just focus on the possibilities of the meaning of \corpus{wa} and \corpus{ga},
we find there are two different uses of \corpus{wa} and three of \corpus{ga}
\citep[\citechap{2}]{kuno1973thestructure}:
\corpus{wa} may be used as a \emph{theme} 
(in the sense of introducing old, already known information which is about the ``background'' of the sentence)
marker, and as a contrastive marker 
(\translate{as for $X$, \dots, but also \dots}),
and \corpus{ga} may be used for neutral descriptions (without any specific information structure)
for actions or temporary states,
or for exhaustive listing,
and sometimes, to mark an O argument in a clause without explicit valency changing.

And now the problem is to test the distributions of all these usages.

For the theme marker function of \corpus{wa},
we need to note that cross-linguistically,
the theme is either \emph{anaphoric} or \emph{generic}:
the former term means the topic denotes a specific%
\footnote{
    Prototypically, a generic \ac{np} is a plural noun without determiner, 
    like \corpus{people},
    while a generic \ac{np} is a proper name, like \corpus{John}.
    But what about an \ac{np} like \corpus{interesting guys}?
    In English, this can be decided by looking at the article: 
    \corpus{the interesting guys} is likely to be specific
    (though \corpus{the beavers in Canada} is still generic),
    while \corpus{interesting guys} is likely to be generic:
    the first \ac{np} superficially contains only one restrictive attributive \corpus{interesting},
    but it can be also understood as \corpus{the interesting guys who I mentioned in my previous speech},
    with the final relative clause deleted.
    But in Japanese there is no article,
    and therefore things are more subtle. (TODO: ref)

    Another issue regarding the specific-generic distinction is 
    it seems both of them can be translated into $\forall x \in A, \ldots$.
    But here there is a difference in how $A$ is constructed:
    for a specific $A$, it is constructed by ``enumerating'' all objects inside,
    while for a generic $A$, it is constructed by restricting an existing set.
}
range of items
and has already appeared in the discourse,
like \corpus{speaking of the boys I met yesterday, they are energetic},
while the latter means the theme denotes to a broad range of items,
like \corpus{speaking of boys, they are often energetic}.
In English, generic (hence not specific) topics and subjects 
usually don't contain \corpus{the} i.e. are not definite,
though the implicational relation can't be reversed, 
since the object can be indefinite while still receiving a specific reading.

Some \ac{np}s are neither anaphoric nor generic:
an \ac{np} can be specific but hasn't appear in the discourse.
And this is exactly when \corpus{wa} fails:
consider the following examples:
\begin{exe}
    \ex *Omosiroi hito wa party ni kimasita
\end{exe}
Here \translate{interesting people} can't receive a generic reading,
or otherwise the meaning of the clause is 
\translate{all people in the world that are interesting came to the party}.
So if \corpus{wa} is there, 
it has to receive an anaphoric reading,
but now since \corpus{omosiroi hito} never appears in previous speech,
this is impossible,
and thus the clause is not legit.
Now the puzzle in 
\begin{exe}
    \ex *Ame wa hutte imasu.
\end{exe}
can be solved. 
Of course the sentence can't be read as \translate{as for all rains, they are falling},
but then \corpus{wa} is not legit:
the rain falling today is yet to be talked about.
We are, however, permitted to use \corpus{wa} in a following clause with the meaning of 
\translate{\dots until the night it was still falling},
because now the same rain has already been talked about.

\ac{np}s that are anaphoric or generic have a deterministic range:
sentences containing them can always start with \corpus{for all \dots},
while \ac{np}s that are neither anaphoric nor generic don't:
they start with \corpus{there exist some \dots} instead.
This seems to be a more generalized criterion,
because \ac{np}s equivalent to \corpus{many people} or \corpus{somebody} 
never appear before \corpus{wa},
while \corpus{today's rain} -- 
with a pretty unique reference, even when it never appears in the previous speech -- 
appears before \corpus{wa}.
TODO: Japanese \translate{most of \dots} can also appear before \corpus{wa}??

One way to solve the problem is to assume each main clause must have a topic and a focus
in the information structure,%
\footnote{
    Here the term \term{information structure} is 
}
which, semantically, doesn't sound like a weird requirement,
since dialogues have backgrounds.
The topic and the focus can be \ac{np}s or the predicate 
(I think here \citet{heycock2008} means verb + object and not the verb complex).
Now here is the language specific rules for Japanese:
whenever an \ac{np} is the topic, it has to be marked by \corpus{wa} 
(because it moves to the \emph{syntactic} topic position, etc.).
So in a main clause about events,
there is an event argument (\translate{in this event, \dots}),
which can fill the topic position, 
and the visible part of the clause -- or maybe simply the subject -- becomes the focus.
The two cases correspond to 
the neutral reading and the ``only the subject does \dots'' reading.
For a main clause about states, however,
this is no longer the case:
now either the subject becomes the topic and the predicate becomes the focus,
which makes the subject moves to the topic position and gets marked by \corpus{wa} instead of \corpus{ga}
(and this often creates an implication that the clause is not exhaustive -- 
see \citet[(2)]{heycock2008}),
or the subject becomes the focus and somehow the predicate becomes the topic 
(\translate{-- Who is late? -- [JACK]_{\text{new information}} [is late]_{\text{old information}}}),
and hence we get the exhaustive \corpus{ga}.
When the is already a topic, of course we can regard the whole comment as the focus,
and thus the \corpus{ga} following the topic doesn't receive an exhaustive reading:
it's just the subject marker.

Thus, if we don't dive into the contrastive/theme distinction
(or to avoid confusing pragmatic and syntactic terms,
we may simply put \term{contrastive/non-contrastive distinction}),
then the \corpus{wa}-\corpus{ga} distinction can be indeed summarized the topic-subject distinction.
The fact that in some clauses \corpus{ga}-\ac{np}s receive an exhaustive reading 
can be attributed to the assigning of topic and focus.

Now here is the problem of multiple nominative construction.
Some people think in this construction, 
some of \corpus{ga} particles appearing are actually focus markers 
\citep{vermeulen2002ga,vermeulen2005two}.
This may be a natural grammaticalization:
since the nominative marker appears so-frequently with focus attached to it,
it may be interpreted as a part-time focus marker.

As for the contrastive/non-contrastive distinction,
note that in Japanese the contrastive construction is not only about contrast of the topic 
(\translate{as for \dots, we have \dots, but also \dots}):
it's possible to contrast two clauses with different topics,
each of which is marked by \corpus{wa}.
Thus, there may be just one \corpus{wa}:
the ``single-clause contrastive'',
which means \translate{as for \dots, I know \dots, but I know nothing about anything else}.
When there is no contrastive construction,
the contrastive \corpus{wa} receives the typical theme reading 
\translate{as for \dots, \dots},
and the contrastive meaning is easily derived when in a contrastive construction
\citep[\citesec{3.4.2}]{heycock2008}.

\section{Clause combining}

\section{Information packaging}


\bibliographystyle{plainnat}
\bibliography{grammars,details,../methodology/famous-grammars,theory}

\end{document}