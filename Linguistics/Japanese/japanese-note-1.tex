\documentclass[UTF8, a4paper, oneside, scheme=plain]{ctexart}

\usepackage{geometry}
\usepackage{titling}
\usepackage{titlesec}
\usepackage{paralist}
\usepackage{footnote}
\usepackage{enumerate}
\usepackage{amsmath, amssymb, amsthm}
\usepackage{gb4e}
\noautomath
\usepackage{bbm}
\usepackage{soul}
\usepackage{graphicx}
\usepackage{siunitx}
\usepackage[table,xcdraw]{xcolor}
\usepackage{tikz}
\usepackage[ruled, vlined, linesnumbered, noend]{algorithm2e}
\usepackage{xr-hyper}
\usepackage[colorlinks]{hyperref} % linkcolor=black, anchorcolor=black, citecolor=black, filecolor=black
\usepackage[most]{tcolorbox}
\usepackage{caption}
\usepackage{subcaption}
\usepackage{booktabs}
\usepackage{multirow}
\usepackage[figuresright]{rotating}
\usepackage{acro}
\usepackage[round]{natbib} 
\usepackage{nameref,zref-xr}
\zxrsetup{toltxlabel}
\zexternaldocument*[draft-]{./main}[main.pdf]
\zexternaldocument*[cgel-]{../English/cambridge}[cambridge.pdf]
\zexternaldocument*[latin-]{../Latin/latin-notes}[latin-notes.pdf]
\zexternaldocument*[alignment-]{../alignment/alignment}[alignment.pdf]
\zexternaldocument*[exercise1-]{../Exercise/2021-3}[2021-3.pdf]
\zexternaldocument*[method-]{../methodology/glossing}[glossing.pdf]
\usepackage{prettyref}

\geometry{left=3.18cm,right=3.18cm,top=2.54cm,bottom=2.54cm}
\titlespacing{\paragraph}{0pt}{1pt}{10pt}[20pt]
\setlength{\droptitle}{-5em}

\DeclareMathOperator{\timeorder}{\mathcal{T}}
\DeclareMathOperator{\diag}{diag}
\DeclareMathOperator{\legpoly}{P}
\DeclareMathOperator{\primevalue}{P}
\DeclareMathOperator{\sgn}{sgn}
\newcommand*{\ii}{\mathrm{i}}
\newcommand*{\ee}{\mathrm{e}}
\newcommand*{\const}{\mathrm{const}}
\newcommand*{\suchthat}{\quad \text{s.t.} \quad}
\newcommand*{\argmin}{\arg\min}
\newcommand*{\argmax}{\arg\max}
\newcommand*{\normalorder}[1]{: #1 :}
\newcommand*{\pair}[1]{\langle #1 \rangle}
\newcommand*{\fd}[1]{\mathcal{D} #1}

\newcommand*{\citesec}[1]{\S~{#1}}
\newcommand*{\citechap}[1]{chap.~{#1}}
\newcommand*{\citefig}[1]{Fig.~{#1}}
\newcommand*{\citetable}[1]{Table~{#1}}
\newcommand*{\citepage}[1]{pp.~{#1}}

\newrefformat{sec}{\citesec{\ref{#1}}}
\newrefformat{fig}{\citefig{\ref{#1}}}
\newrefformat{tbl}{\citetable{\ref{#1}}}
\newrefformat{chap}{\citechap{\ref{#1}}}

\usetikzlibrary{arrows,shapes,positioning}
\usetikzlibrary{arrows.meta}
\usetikzlibrary{decorations.markings}
\tikzstyle arrowstyle=[scale=1]
\tikzstyle directed=[postaction={decorate,decoration={markings,
    mark=at position .5 with {\arrow[arrowstyle]{stealth}}}}]
\tikzstyle ray=[directed, thick]
\tikzstyle dot=[anchor=base,fill,circle,inner sep=1pt]


\tcbuselibrary{skins, breakable, theorems}

\newtcbtheorem[number within=chapter]{infobox}{Box}%
  {colback=blue!5,colframe=blue!65,fonttitle=\bfseries, breakable}{infobox}

\newcommand*{\concept}[1]{\textbf{#1}}
\newcommand*{\term}[1]{\emph{#1}}
\newcommand{\corpus}[1]{\emph{#1}}

\DeclareAcronym{blt}{short = BLT, long = Basic Linguistic Theory}
\DeclareAcronym{cgel}{short = CGEL, long = The Cambridge Grammar of the English Language}
\DeclareAcronym{dm}{short = DM, long = Distributed Morphology}
\DeclareAcronym{tag}{long = Tree-adjoining grammar, short = TAG}
\DeclareAcronym{sfp}{long = sentence final particle, short = SFP}
\DeclareAcronym{np}{long = noun phrase, short = NP}
\DeclareAcronym{vp}{long = verb phrase, short = VP}
\DeclareAcronym{pp}{long = preposition phrase, short = PP}
\DeclareAcronym{cls}{long = classifier, short = CLS}
\DeclareAcronym{dist}{long = distal, short = DIST}
\DeclareAcronym{prox}{long = proximate, short = PROX}
\DeclareAcronym{dem}{long = demonstrative, short = DEM}
\DeclareAcronym{dur}{long = durative, short = DUR}
\DeclareAcronym{neg}{long = negative, short = NEG}
\DeclareAcronym{cc}{long = copular complement, short = CC}
\DeclareAcronym{cs}{long = copular subject, short = CS}
\DeclareAcronym{tam}{long = {tense, aspect, mood}, short = TAM}

\newcommand*{\homo}[2]{#1$_{\text{#2}}$}

\newcommand{\cgel}{\href{../English/cambridge.pdf}{my notes about CGEL}}
\newcommand{\latin}{\href{../Latin/latin-notes.pdf}{my notes about Latin}}
\newcommand{\alignment}{\href{../alignment/alignment.pdf}{my notes about alignment}}
\newcommand{\exerciseone}{\href{../Exercise/2021-3.pdf}{this exercise}}
\newcommand{\method}{\href{../methodology/glossing.pdf}{this note about how descriptive grammars work}}
\newcommand{\draft}{\href{./main.pdf}{this draft}}

\newcommand{\ala}{à la}
\newcommand{\translate}[1]{`#1'}

\title{Japanese grammar notes}
\author{Jinyuan Wu}

\begin{document}

\maketitle

This is a reading note of \citet{tsutsui1989dictionary} 
as well as lots of books and articles listed in the reference.
The methodology followed is in \method,
i.e. ``largely generatively informed but surface-oriented and flat-tree in the appearance''.

\section{Overview}

\subsection{Historical notes}

Japanese has a strict modifier-first constituent order,
and here the term \term{modifier} includes 
arguments in a clause (``modifiers of the verb or the verbal adjective''),
and even \acs{np}s with respect to case particles.
This is probably related to a strong head-final tendency in the linearization:
if a so-called modifier is introduced as a specifier 
in a functional projection with a root as the core,
then obviously the root and the functional heads are realized into one unit (for example a verb complex) 
and the ``modifier'' precedes the unit to ensure the (functional) head-final rule,
and therefore in the surface-oriented analysis,
we also get a modifier-head constituent order (where \term{head} means lexical heads).
If there is no core root,
then trivially the ``modifier'' is realized in a position before the spellout of functional heads,
and the latter is regarded as somehow a head in the \acs{cgel} sense,
and again we get a modifier-head constituent order,
if we understand things like case particles as heads,
which is the \ac{cgel} approach but not the \acs{blt} approach.

\subsection{Phonology and orthography}

\subsection{Parts of speech}

Japanese has a clear noun-verb distinction:
nouns are subject to case marking,
which is basically adding a particle to the \acs{np} 
which can be dropped especially in casual speech,
while verbs always appear as one of the stem forms plus agglutinative endings.

There are two adjective classes:
the verbal adjectives (or \corpus{i}-adjectives)
and the nominal adjectives (or \corpus{na}-adjectives),
with different syntactic distribution 
(verbal adjectives may fill the predicate slot on their own; nominal adjectives never do so)
and morphological appearances
(verbal adjectives are more like verbs).

One rare property of modern Japanese is the verb class and the verbal adjective class 
are already closed classes:
they rarely accept new members (though not entirely impossible).
What makes Japanese rarer is despite being closed,
the verbal adjective class is large.

Function words in Japanese can be roughly divided according to the grammatical systems they express.%
\footnote{
    Though particles are in the grammar and do not really carry category labels,
    the traditional practice to list all particles and classify them 
    is practically desirable, 
    as it provides a quick way to navigate across grammatical systems.
}
The particle class includes
case markers (格助詞, kaku-joshi),
parallel markers (並立助詞, heiritsu-joshi),
sentence final particles (終助詞, shū-joshi),
interjectory particles (間投助詞, kantō-joshi),
adverbial particles (副助詞, fuku-joshi),
binding particles (係助詞, kakari-joshi),
conjunctive particles (接続助詞, setsuzoku-joshi),
and phrasal particles (準体助詞, juntai-joshi).
Here the term \term{particle} is used,
instead of, say, \term{clitic},
though we can only be sure that the case ending is a grammatical word 
(since it appears in the level of phrases),
partly because this is the tradition way to call it,
partly because TODO: whether particles can receive stress, etc.

TODO: hyperlink

Japanese lacks the prototypically pronoun class:
so-called pronouns are customized referential nouns like \translate{that girl},
and thus the pronoun class is not closed and strictly speaking is not a part of the grammar.
The article class is also not attested.

\subsection{Noun phrases}

\subsubsection{Unattested categories}

Gender and number are not grammatical categories in Japanese.



As is usually the case,
structural cases -- cases assigned directly by the vP-TP-CP functional hierarchy 
-- never appear together with the topic marker,
but peripheral cases are allowed to appear with the topic marker.

\subsection{Clauses}

\subsubsection{The template of clause structure}

Japanese is basically SOV in clause constituent order,
and it's also famous for allowing scrambling,
i.e. not-so-radical and often pragmatically marked constituent order deviation from the prototype.%
\footnote{
    Many scholars, especially those skeptical about generativism,
    often insist on ``analyzing non-canonical constituent order as they are''
    and reject the notion of scrambling.
    But there are indeed evidences -- from purely syntactic ones to psychological ones -- 
    for the existence of scrambling mechanism \citep{imamura2015effects,imamura2016processing,yatsushiro2003vp}.
    Since this note is mostly descriptive and is about surface-oriented analysis,
    scrambling is to be discussed in a mostly flat-tree way in the rest of this note.
}
In a scrambled clause, 
the predicate%
\footnote{
    In this note the term \term{predicate} takes the \ac{blt} meaning.
}
is still strictly at the final position 
(possibly with \acs{sfp}s following it).
The topic and the subject (or maybe we'd better call it focus -- see \prettyref{sec:alignment}), 
if any, are usually in the initial position.
They are dislocated only in marked cases.
On the other hand, 
internal arguments are scrambled in a freer manner:
the indirect object and the direct object can be switched without making a big difference.

\subsubsection{The predicate}\label{sec:predicate-abs}

The predicate position -- the so-called ``V'' in SOV -- 
can be filled either by a verb or by a verbal adjective,
plus a chain of function words.
The mapping between slots in the predicate and the generative TP-CP structure is almost one-to-one,
probably again as a result of the agglutinative nature of Japanese,
where functional morphemes are just speltout as they are with no further processing.
The template of the predicate is basically
the main verb plus a series of so-called ``auxiliary verbs'' 
plus possible negation marker plus tense marker,
and plus a possible \ac{sfp}. 
Verbal adjectives are incompatible with auxiliary verbs.

Still, the case of the Japanese predicate is not identical to the ideal agglutinative case,
because there is still interaction between function morphemes.
Previous works analyze this interplay in quite different ways.
The School Grammar assumes both the main verb and the auxiliary verbs may have different endings 
decided by morphemes in higher positions (that's to say, by morphemes following them),
as in the English periphrastic conjugation \corpus{have been doing sth.}
(\prettyref{sec:verb-complex-overview}).
The position of this note, however, is to use morphophonological rules 
in place of this selection-and-affix-lowering approach.
There are also uncontroversial phonological rules acting between morphemes in the predicate.

In copular clauses,
\citet{tsutsui1989dictionary} analyzes the copula and the \acs{cc} together as the predicate,
which makes sense if we understood the \acs{cc} as the predicate,
and in this perspective, the copula is merely a particle indicating the predicate status of the \acs{cc},
but in another perspective, the copula is obviously a verb,
which takes the \acs{cc} as its internal argument.

\subsubsection{Arguments and the nominative-accusative alignment}\label{sec:alignment}

It doesn't take much effort to find that Japanese is nominative-accusative:
for intransitive verbs,
the S argument is marked in the same way as the A argument of transitive verbs,
usually by the \corpus{ga} particle or by the \corpus{wa} particle.
There are actually some subtlety regarding the marking of the S/A argument (\prettyref{sec:topic-subject}),
about the exact meaning of \corpus{ga} and \corpus{wa}.
I call \corpus{ga} the nominative particle instead of the focus particle 
and follow the tradition, to place it together with uncontroversial case particles,
while placing \corpus{wa} together with information structure marking particles TODO: exact name of the two types of particles,
and the justification is done in TODO

\subsubsection{Valency changing}

The so-called Japanese passive is actually affective:
in the affective construction, the A argument undergoes an action caused by others. 
Thus in the affective construction we can still see things like direct objects.

\subsubsection{\ac{tam} markers}

Japanese has the following \ac{tam} categories:
a past/non-past tense distinction,
TODO: mood

\subsubsection{Sentence final particles}

\acs{sfp}s are important in Japanese grammar:
TODO: sfp for relative clauses? (Or in other words, clause or sentence?)

\subsection{Remarkable features}

\subsubsection{Honorifics}

\subsection{Existing studies}

TODO: school grammar and education grammar

\section{Phonological rules}

This note is mainly about morphosyntax.
However, phonological rules seem to be significant in at least some parts of Japanese grammar.


\section{Noun phrases}

\subsection{Final particles}

There are basically two systems of particles after \ac{np}s.
The first is the case system (\concept{case markers}, 格助詞), including
TODO: distribution (and the following ones)
\begin{itemize}
    \item Nominative: \corpus{ga}, 
    appearing in certain circumstances as the focus marker (\prettyref{sec:topic-subject}).
    \item Accusative: \corpus{o}
    \item Dative: \corpus{ni}: time and location 
    \item Genitive: \corpus{no} 
    \item Lative: \corpus{e}, used for destination direction (like in "to some place")
    \item Ablative: \corpus{kara}, used for source direction (like in "from some place")
    \item Instrumental/Locative: \corpus{de}
\end{itemize}

The second system is the 

The systems are not completely compatible.
A well known generalization is structural case markers are erased when \ac{np}s are topicalized,
while inherent case markers may be kept.


\section{The verb complex}

\subsection{Overview and previous studies}\label{sec:verb-complex-overview}

The verb complex -- the content of the predicate position -- is discussed in this section.
Since the verb complex is the collective realization of the vP-TP-CP functional projection,
a change in the verb complex may also be a change in 
the arguments (\prettyref{sec:arguments}),
TODO: list all of them.
I follow the practice in \citet{jacques2021grammar}
and use the verb complex as the table of content of what happens in the clause.

One astonishing aspect of Japanese is 
each verbal element -- the main verb or the main verbal adjective, so-called ``auxiliary verbs'' (see below) -- 
in the verb complex is in one of a handful of so-called conjugation forms (the name is problematic -- see below),
if we believe the analysis of the School Grammar.
This shared morphological appearance may be seen as justification of the term \term{auxiliary verb},
since they have similar appearances as lexical verbs,
though they can also be perceived as inflectional endings 
which apply to verbs already with certain degrees of inflection,
which is even more likely 
because elements in the predicate are not subject to further syntactic operation
(see below discussion on Education Grammar).
From then on, I will call all morphosyntactic objects 
appearing in the predicate and have verb-like inflectional behaviors \concept{verbal} elements,
regardless of whether they are lexical or functional.

These conjugation forms have names like ``irrealis'' or ``hypothetical'',
but the names of these forms doesn't say much about their distributions,
except the imperative form (used to convey a rather rude and direct command).
Thus, if we believe in the analysis of the School Grammar,
the so-called conjugation forms are better understood as the 
so-called English \corpus{ing}-participle and \corpus{ed}-participle forms 
in \corpus{have been doing},
which would be better glossed as 
\corpus{have-en be-ing do} plus phonetic readjustment rules
if we want an one-to-one correspondence between morphemes and grammatical categories.
What carry grammatical categories are \emph{not} the so-called conjugation ending of 
the irrealis form, the continuative form, etc.,
but the verbal elements themselves -- 
in the case of English,
\corpus{have} (present perfect) and \corpus{be} (passive).

On the other hand, the so-called ``conjugation ending'' of each element in the verb complex 
is \emph{not} determined by agreement with the arguments 
or by clausal grammatical categories:
the former never exists in Japanese and the latter is coded by the sequence of functional verbal items,
where for inner verbal elements,
the conjugation form is purely decided 
by the higher verbal element (in Japanese, the closest verbal element in the right), 
arguably by some sorts of ``affix lowering''.
This periphrastic inflectional strategy in English \corpus{have been doing} 
is used in Japanese in a much larger scale,
because there are lots of auxiliary verbs.

What makes Japanese and English different 
is the verb complex in Japanese doesn't allow intervening of things like adverbs,
while in English it's pretty acceptable.
Thus, the morphosyntactic standard of wordhood hints that 
the whole verb complex may be understood as a huge grammatical word,
since once finished, it never interacts with the outer world.
This indeed seems to be the convention used in many romanization solutions.

The problem, now, is that if all the above are correct,
then the verb complex of Japanese is a rather weird construction:
its inner makeup has certain periphrastic flavor as in \corpus{has been doing},
and yet it's closed for typical adjunction operations seen in periphrastic conjugation.
This isn't completely impossible in principle:
we may assume there is an PF rule requiring all verbal elements to spellout together,
and any argument in the functional hierarchy is squeezed out.
But of course it's bound to be typologically rare,
and there may also be some crosslinguistic constraints forbidding such a heavy spellout strategy.

This problem with the School Grammar -- 
ambiguity between affixes and grammatical words within the verb complex -- 
inspired a group of scholars to find an alternative,
and the result is what's known as Education Grammar today.
The School Grammar analyzes the verb complex in a highly detailed and semi-analytical way,
segmenting it into the verb stem, a list of auxiliary verbs, the negative marker, and the tense marker,
each of them is in one of the ``conjugation forms'',
while the Education Grammar analyzes the verb complex 
simply as the stem plus \emph{one} long, long conjugation ending.
The status of the so-called \corpus{te}-form of verbs as one of the basic conjugation forms, for example,
is rejected by the School Grammar:
it's analyzed as TODO.
This means we may find more conjugation forms in Education Grammar books.

The main problem with Education Grammar is it comes to another extreme,
completely ignoring the inner structure of the verb complex.
(And out of some short-sighted considerations like 
``making second-language learners be able to at least pick up \emph{some} Japanese in one lesson'',
the \corpus{masu}-form is introduced too early,
ending up in endless confusions.)
The approach taken in this note is a third way.
I recognize that the true conjugation endings of a lexeme 
are not just a handful of ``conjugation forms'' mentioned in School Grammar,
and the term \corpus{conjugation ending} is used to denote 
any possible combination of function items in the verb complex,
and hence I recognize the so-called \corpus{te}-form is a conjugation form,
so if a chain of function morphemes has developed a conventionalized meaning,
it's easy to just describe it as a newly emerging conjugation form.
On the other hand,
the possible forms of the verb complex are decomposed as detailed as possible,
and I'll still study auxiliary verbs,
though they are parts of a complex word, not a phrase.
Contrary to the School Grammar, however,
I'll go into the syllables and allow the deep form of morphemes -- lexical or functional --
to begin and end in consonants.
As we will see later,
this actually reduces the necessity of posting 
``conjugation forms for each verbal element in the verb complex'',
as so-called endings of verbal elements 
are more about \emph{morphophonological} instead of morphosyntactic interaction 
between morphemes.%
\footnote{
    This is actually a methodological lesson.
    In principle, the ideal workflow is to first recognize phonological rules
    (in this step morphosyntactic analysis is of course needed),
    and after morphemes in the deep structure are recognized,
    bracketing them into words and phrases should be done
    without depending much on phonology.
    In practice, though, linguists often implicitly 
    consider large phonological units as morphosyntactic units as well 
    without questioning if there is mismatch between phonological boundary and morphosyntactic boundary,
    and this often ends up in the failure of recognizing the exact meaning of alleged ``morphemes''.
    In the case of Japanese,
    criterions like ``being made of one or more complete mora''
    are for phonological words, not morphosyntactic words.
    School Grammarians, however, never go into the structure of syllables,
    and they simply take the claim that a morphosyntactic unit contains integer syllables for granted.
    The result is the ``conjugation forms'' of verbal elements are completely incomprehensible.
    And if we \emph{then} try to investigate the inner structure of School Grammar ``verbal elements'',
    we find nothing that can deepen our understanding about why things are so.
    As Chomsky pointed out in his early age,
    American structuralist partition of speech purely with intuition 
    -- a mixture of phonology and morphosyntax --
    is theoretically bad-informed.
}
There are actually some evidences supporting this -- 
for relevant discussions see \prettyref{sec:consonant-class}, \prettyref{sec:vowel-class}.

Here is a list of School Grammar ``conjugation forms'' and relevant distributional information,
and tentative explanations of their underlying origins:
\begin{itemize}
    \item Irrealis form (未然形). 
    This form appears in 
    \item Volitional form (意志形), which Historically originates from the irrealis form. TODO
    \item Continuative form or adverbial form (連用形), 
    sometimes also called the infinitive form
    for its frequent appearance after other verbal elements.
    \item Terminal form or dictionary form (終止形, 辞书形)
    \item Attributive form (連体形), which, for regular verbs and verbal adjectives, 
    is the same as the dictionary form in modern Japanese.
    TODO: Nominal adjectives and copula 
    \item Hypothetical form (仮定形)
    \item Imperative form (命令形)
\end{itemize}
The imperative form, the terminal form and the attributive form are ``finalized'' forms: 
there can be no pure auxiliary verbs following them,
though semi-auxiliary verbs like \corpus{dar\={o}} and \ac{sfp}s are still possible.

In the perspective of the School Grammar, the irrealis form, the continuative form, etc. 
are actually conjugation stems.
But this term goes in conflict with my notion of \term{stem} in \prettyref{sec:conjugation-class},
and may also create confusion because auxiliary verbs also have these forms 
but it's rather strange to use the term \term{stem} for function items.
So in the following,
I will simply call the five or six or seven forms listed above (depending on how you count them)
as \concept{School Grammra conjugation forms}.

\subsection{Conjugation classes and verb stems}\label{sec:conjugation-class}

\subsubsection{The consonant and vowel conjugations}

According to whether the final sound of the stem is a vowel or consonant,
Regular Japanese verbs can be divided into \concept{c-stem verbs} and \concept{v-stem verbs}.
C-stem verbs are also called 五段動詞 or \concept{group-1 verbs},
because changing the conjugation ending means the final mora may appear in every row of the kana chart, 
and v-stem verbs are also called 一段動詞 or \concept{group-2 verbs},
because the last or the second but last mora is always in the same row with the last mora in the dictionary form.
一段動詞 can be further divided into 上一段動詞 or \corpus{iru}-verbs or \concept{group-2a verbs} 
and 下一段動詞 or \corpus{eru}-verbs or \concept{group-2b verbs}:
the final vowel of \corpus{iru}-verbs is \corpus{i},
and \corpus{-ru} is actually the conjugation ending of the terminal form,
and similarly the final vowel of \corpus{eru}-verbs is \corpus{e}.

There are two important irregular verbs: \corpus{suru} \translate{to do} and \corpus{kuru} \translate{to come}.
TODO: where to discuss them, 
The two verbs, together with their compounding with TODO,
are collectively called \concept{group-3 verbs}.

\subsubsection{The consonant conjugation}\label{sec:consonant-class}

A verb whose dictionary form is not in group-3 and doesn't end in \corpus{-eru} or \corpus{-iru} 
is definitely a c-stem verb, i.e. a group-1 verb.
Basically, by removing the final \corpus{-u}, we get the stem;
if a verb ends in \corpus{-tsu},
then the stem ends in \corpus{-t-}.
Verbs ending in \corpus{-au}, \corpus{-iu}, and \corpus{-ou} are also c-stem verbs,
because we can assume there is a hidden \corpus{w} before \corpus{u},
which appears in some of the negative forms. TODO
A handful of verbs ending in \corpus{-eru} and \corpus{-iru} are also c-stem verbs. TODO

When doing conjugation, Phonological rules deserve attention.
If the stem ends in \corpus{-s-}, 
then it becomes \corpus{-sh-} before a suffix beginning with \corpus{i}.
If the stem ends in \corpus{-t-},
it becomes \corpus{-ch-} before a suffix beginning with \corpus{i}.%
\footnote{
    \citet{akiyama2012japanese} defines the stem as the result of directly removing the final \corpus{u},
    and thus we would have a \corpus{-ts-} to \corpus{-t-} rule 
    for anything else than the dictionary form or a form with a suffix beginning with \corpus{i}.
    I find this analysis uneconomical,
    since the \corpus{t}-to-\corpus{ts} change is easily explained by palatalization,
    while the inverse is mysterious.
}

Here is the relation between the stem and the School Grammar conjugation forms:
\begin{itemize}
    \item The continuative form of a c-stem verb is obtained by adding \corpus{-i} to the stem.
    \item The terminal form and the attributive form is obtained by adding \corpus{-u}.
    Note that for v-stem verbs the ending is \corpus{-ru}.
    Since Japanese doesn't allow two successive consonants,
    we may assume there is a phonological rule deleting the \corpus{r} sound for c-stem verbs.
    Thus, the terminal form is just the stem plus \corpus{-ru}.
    \item The hypothetical form is obtained by adding \corpus{-e}.
    \item The irrealis form is obtained by adding \corpus{-a}.
    \item The volitional form is obtained by adding \corpus{-o}.
\end{itemize}

\subsubsection{The vowel conjugation}\label{sec:vowel-class}

What aren't group-3 verbs and group-1 verbs are v-stem verbs, i.e. group-2 verbs.
The stem is obtained by dropping the final \corpus{-ru}.

The rest of the School Grammar conjugation forms of a v-stem verb are the same as the stem,
except the imperative form,
which is formed by adding \corpus{-ro} to the stem 
(\corpus{-yo} is used as an alternative in writing).
This pattern hints there may 

\subsubsection{Group-3 verbs}

dictionary form continuative form 

kuru ki 

suru shi

\subsection{Inner structure: valency changing}



\subsection{Honorifics}

TODO: 
-masu = -mas-u, -mashita = -mas-ita?

\subsection{Tense}

The present tense is marked by nothing. TODO: then what about the ``the last word ending in dictionary form''? this is the ending of all verb complexes?

\subsection{periphrastic conjugations}

\subsubsection{The progressive}

\section{Argument structures and alignment}\label{sec:arguments}

This section is about subcategorization frames of verbs and the alignment.

\subsection{Canonical transitive and intransitive verbs}

\section{Simple clauses}

\subsection{Topic and subject}\label{sec:topic-subject}

The subject is often said to be both agentive and topic-like in the typological literature.
What makes Japanese different is despite its accusative nature,
there is still a problem concerning what exact is the subject in Japanese.
The \corpus{wa} particle and the \corpus{ga} particle are commonly called 
the topic particle and the subject particle, respectively.
This correspondence fails the predicate has a stative or habitual meaning:
in that case, \corpus{ga} has the meaning of \translate{it's \dots that},
usually with an exhaustive meaning -- 
only the \ac{np} before \corpus{ga} satisfies the predicate,
and nothing else.
This is probably due to the fact \corpus{ga} assigns focus to the \ac{np} before it:
in English, \corpus{YOU do the job} means it's you -- and only you -- who does the job.
Thus, \corpus{ga} may be better glossed as a focus marker,
and this means in the surface-oriented constituent structure of main clauses, 
there is no such thing as the subject: 
anything moves to SpecTP immediately moves to a higher position in CP, 
making the notion \term{subject} a latent concept without its own particle.
Indeed, we have so-called adverbial nominative in Japanese \citet[\citesec{6.1}]{endo2007locality}.
Note, however, that the alignment in relative clauses is quite simple:
in relative clauses the topic marker \corpus{wa} never appears 
and \corpus{ga} is simply the marker of nominative case.

\section{Clause combining}

\section{Information packaging}


\bibliographystyle{plainnat}
\bibliography{grammars,details,../methodology/famous-grammars}

\end{document}