\documentclass[UTF8, a4paper, oneside, scheme=plain]{ctexart}

\usepackage{geometry}
\usepackage{titling}
\usepackage{titlesec}
\usepackage{paralist}
\usepackage{footnote}
\usepackage{enumerate}
\usepackage{amsmath, amssymb, amsthm}
\usepackage{gb4e}
\noautomath
\usepackage{bbm}
\usepackage{soul}
\usepackage{graphicx}
\usepackage{siunitx}
\usepackage[table,xcdraw]{xcolor}
\usepackage{tikz}
\usepackage[ruled, vlined, linesnumbered, noend]{algorithm2e}
\usepackage{xr-hyper}
\usepackage[colorlinks]{hyperref} % linkcolor=black, anchorcolor=black, citecolor=black, filecolor=black
\usepackage[most]{tcolorbox}
\usepackage{caption}
\usepackage{subcaption}
\usepackage{booktabs}
\usepackage{multirow}
\usepackage[figuresright]{rotating}
\usepackage{acro}
\usepackage[round]{natbib} 
\usepackage{nameref,zref-xr}
\zxrsetup{toltxlabel}
\zexternaldocument*[draft-]{./main}[main.pdf]
\zexternaldocument*[cgel-]{../English/cambridge}[cambridge.pdf]
\zexternaldocument*[latin-]{../Latin/latin-notes}[latin-notes.pdf]
\zexternaldocument*[alignment-]{../alignment/alignment}[alignment.pdf]
\zexternaldocument*[exercise1-]{../Exercise/2021-3}[2021-3.pdf]
\zexternaldocument*[method-]{../methodology/glossing}[glossing.pdf]
\usepackage{prettyref}

\geometry{left=3.18cm,right=3.18cm,top=2.54cm,bottom=2.54cm}
\titlespacing{\paragraph}{0pt}{1pt}{10pt}[20pt]
\setlength{\droptitle}{-5em}

\DeclareMathOperator{\timeorder}{\mathcal{T}}
\DeclareMathOperator{\diag}{diag}
\DeclareMathOperator{\legpoly}{P}
\DeclareMathOperator{\primevalue}{P}
\DeclareMathOperator{\sgn}{sgn}
\newcommand*{\ii}{\mathrm{i}}
\newcommand*{\ee}{\mathrm{e}}
\newcommand*{\const}{\mathrm{const}}
\newcommand*{\suchthat}{\quad \text{s.t.} \quad}
\newcommand*{\argmin}{\arg\min}
\newcommand*{\argmax}{\arg\max}
\newcommand*{\normalorder}[1]{: #1 :}
\newcommand*{\pair}[1]{\langle #1 \rangle}
\newcommand*{\fd}[1]{\mathcal{D} #1}

\newcommand*{\citesec}[1]{\S~{#1}}
\newcommand*{\citechap}[1]{chap.~{#1}}
\newcommand*{\citefig}[1]{Fig.~{#1}}
\newcommand*{\citetable}[1]{Table~{#1}}
\newcommand*{\citepage}[1]{pp.~{#1}}

\newrefformat{sec}{\citesec{\ref{#1}}}
\newrefformat{fig}{\citefig{\ref{#1}}}
\newrefformat{tbl}{\citetable{\ref{#1}}}
\newrefformat{chap}{\citechap{\ref{#1}}}

\usetikzlibrary{arrows,shapes,positioning}
\usetikzlibrary{arrows.meta}
\usetikzlibrary{decorations.markings}
\tikzstyle arrowstyle=[scale=1]
\tikzstyle directed=[postaction={decorate,decoration={markings,
    mark=at position .5 with {\arrow[arrowstyle]{stealth}}}}]
\tikzstyle ray=[directed, thick]
\tikzstyle dot=[anchor=base,fill,circle,inner sep=1pt]


\tcbuselibrary{skins, breakable, theorems}

\newtcbtheorem[number within=chapter]{infobox}{Box}%
  {colback=blue!5,colframe=blue!65,fonttitle=\bfseries, breakable}{infobox}

\newcommand*{\concept}[1]{\textbf{#1}}
\newcommand*{\term}[1]{\emph{#1}}
\newcommand{\corpus}[1]{\emph{#1}}

\DeclareAcronym{blt}{short = BLT, long = Basic Linguistic Theory}
\DeclareAcronym{cgel}{short = CGEL, long = The Cambridge Grammar of the English Language}
\DeclareAcronym{dm}{short = DM, long = Distributed Morphology}
\DeclareAcronym{tag}{long = Tree-adjoining grammar, short = TAG}
\DeclareAcronym{sfp}{long = sentence final particle, short = SFP}
\DeclareAcronym{np}{long = noun phrase, short = NP}
\DeclareAcronym{vp}{long = verb phrase, short = VP}
\DeclareAcronym{pp}{long = preposition phrase, short = PP}
\DeclareAcronym{cls}{long = classifier, short = CLS}
\DeclareAcronym{dist}{long = distal, short = DIST}
\DeclareAcronym{prox}{long = proximate, short = PROX}
\DeclareAcronym{dem}{long = demonstrative, short = DEM}
\DeclareAcronym{dur}{long = durative, short = DUR}
\DeclareAcronym{neg}{long = negative, short = NEG}
\DeclareAcronym{cc}{long = copular complement, short = CC}
\DeclareAcronym{cs}{long = copular subject, short = CS}

\newcommand*{\homo}[2]{#1$_{\text{#2}}$}

\newcommand{\cgel}{\href{../English/cambridge.pdf}{my notes about CGEL}}
\newcommand{\latin}{\href{../Latin/latin-notes.pdf}{my notes about Latin}}
\newcommand{\alignment}{\href{../alignment/alignment.pdf}{my notes about alignment}}
\newcommand{\exerciseone}{\href{../Exercise/2021-3.pdf}{this exercise}}
\newcommand{\method}{\href{../methodology/glossing.pdf}{this note about how descriptive grammars work}}
\newcommand{\draft}{\href{./main.pdf}{this draft}}

\newcommand{\ala}{à la}
\newcommand{\translate}[1]{`#1'}

\title{Japanese grammar notes}
\author{Jinyuan Wu}

\begin{document}

\maketitle

This is a reading note of \citet{tsutsui1989dictionary} 
as well as lots of books and articles listed in the reference.
The methodology followed is in \method,
i.e. ``largely generatively informed but surface-oriented and flat-tree in the appearance''.

\section{Overview}

\subsection{Historical notes}

Japanese has a strict modifier-first constituent order,
and here the term \term{modifier} includes 
arguments in a clause (``modifiers of the verb or the verbal adjective''),
and even \acs{np}s with respect to case particles.
This is probably related to a strong head-final tendency in the linearization:
if a so-called modifier is introduced as a specifier 
in a functional projection with a root as the core,
then obviously the root and the functional heads are realized into one unit (for example a verb complex) 
and the ``modifier'' precedes the unit to ensure the (functional) head-final rule,
and therefore in the surface-oriented analysis,
we also get a modifier-head constituent order (where \term{head} means lexical heads).
If there is no core root,
then trivially the ``modifier'' is realized in a position before the spellout of functional heads,
and the latter is regarded as somehow a head in the \acs{cgel} sense,
and again we get a modifier-head constituent order,
if we understand things like case particles as heads,
which is the \ac{cgel} approach but not the \acs{blt} approach.

\subsection{Phonology and orthography}

\subsection{Parts of speech}

Japanese has a clear noun-verb distinction:
nouns are subject to case marking,
which is basically adding a particle to the \acs{np} 
which can be dropped especially in casual speech,
while verbs always appear as one of the stem forms plus agglutinative endings.

There are two adjective classes:
the verbal adjectives (or \corpus{i}-adjectives)
and the nominal adjectives (or \corpus{na}-adjectives),
with different syntactic distribution 
(verbal adjectives may fill the predicate slot on their own; nominal adjectives never do so)
and morphological appearances
(verbal adjectives are more like verbs).

One rare property of modern Japanese is the verb class and the verbal adjective class 
are already closed classes:
they rarely accept new members (though not entirely impossible).
What makes Japanese rarer is despite being closed,
the verbal adjective class is large.

Function words in Japanese can be roughly divided according to the grammatical systems they express.%
\footnote{
    Though particles are in the grammar and do not really carry category labels,
    the traditional practice to list all particles and classify them 
    is practically desirable, 
    as it provides a quick way to navigate across grammatical systems.
}
The particle class includes
case markers (格助詞, kaku-joshi),
parallel markers (並立助詞, heiritsu-joshi),
sentence final particles (終助詞, shū-joshi),
interjectory particles (間投助詞, kantō-joshi),
adverbial particles (副助詞, fuku-joshi),
binding particles (係助詞, kakari-joshi),
conjunctive particles (接続助詞, setsuzoku-joshi),
and phrasal particles (準体助詞, juntai-joshi).
Here the term \term{particle} is used,
instead of, say, \term{clitic},
though we can only be sure that the case ending is a grammatical word 
(since it appears in the level of phrases),
partly because this is the tradition way to call it,
partly because TODO: whether particles can receive stress, etc.

TODO: hyperlink

Japanese lacks the prototypically pronoun class:
so-called pronouns are customized referential nouns like \translate{that girl},
and thus the pronoun class is not closed and strictly speaking is not a part of the grammar.
The article class is also not attested.

\subsection{Noun phrases}

\subsubsection{Unattested categories}

Gender and number are not grammatical categories in Japanese.



As is usually the case,
structural cases -- cases assigned directly by the vP-TP-CP functional hierarchy 
-- never appear together with the topic marker,
but peripheral cases are allowed to appear with the topic marker.

\subsection{Clauses}

\subsubsection{The template of clause structure}

Japanese is basically SOV in clause constituent order,
and it's also famous for allowing scrambling,
i.e. not-so-radical and often pragmatically marked constituent order deviation from the prototype.%
\footnote{
    Many scholars, especially those skeptical about generativism,
    often insist on ``analyzing non-canonical constituent order as they are''
    and reject the notion of scrambling.
    But there are indeed evidences -- from purely syntactic ones to psychological ones -- 
    for the existence of scrambling mechanism \citep{imamura2015effects,imamura2016processing,yatsushiro2003vp}.
    Since this note is mostly descriptive and is about surface-oriented analysis,
    scrambling is to be discussed in a mostly flat-tree way in the rest of this note.
}
In a scrambled clause, 
the predicate%
\footnote{
    In this note the term \term{predicate} takes the \ac{blt} meaning.
}
is still strictly at the final position 
(possibly with \acs{sfp}s following it).
The topic and the subject (or maybe we'd better call it focus -- see \prettyref{sec:alignment}), 
if any, are usually in the initial position.
They are dislocated only in marked cases.
On the other hand, 
internal arguments are scrambled in a freer manner:
the indirect object and the direct object can be switched without making a big difference.

\subsubsection{The predicate}

The predicate position -- the so-called ``V'' in SOV -- 
can be filled either by a verb or by a verbal adjective,
plus a chain of function words.
The mapping between slots in the predicate and the generative TP-CP structure is almost one-to-one,
probably again as a result of the agglutinative nature of Japanese,
where functional morphemes are just speltout as they are with no further processing.
The template of the predicate is basically
the main verb plus a series of so-called ``auxiliary verbs'' 
plus possible negation marker plus tense marker,
and plus a possible \ac{sfp}. 
Verbal adjectives are incompatible with auxiliary verbs.

Still, this is not identical to the ideal agglutinative case,
and in Japanese there is still interaction between function morphemes.
The morphosyntactic interaction is about verb forms.
In a purely analytic version of English,
we would have \corpus{have-en be-ing do sth.},
while in reality, what we get is \corpus{have been doing sth.}
And the Japanese predicate is like an enhanced \corpus{have been doing sth.}.
Both the main verb and the auxiliary verbs may have different endings 
decided by morphemes in higher positions (that's to say, by morphemes following them),
and this may be seen as justification of the term \term{auxiliary verb},
since they have similar appearances as lexical verbs,
though they can also be perceived as inflectional endings 
which apply to verbs already with certain degrees of inflection,
which is even more likely because elements in the predicate are not subject to further syntactic operation
(\prettyref{sec:verb-complex-overview}).
Phonological rules are another way of interaction between morphemes in the predicate.
Indeed, this has given rise to some new conjugation forms of verbs.

In copular clauses,
\citet{tsutsui1989dictionary} analyzes the copula and the \acs{cc} together as the predicate,
which makes sense if we understood the \acs{cc} as the predicate,
and in this perspective, the copula is merely a particle indicating the predicate status of the \acs{cc},
but in another perspective, the copula is obviously a verb,
which takes the \acs{cc} as its internal argument.

\subsubsection{Arguments and the nominative-accusative alignment}\label{sec:alignment}

It doesn't take much effort to find that Japanese is nominative-accusative:
for intransitive verbs,
the S argument is marked in the same way as the A argument of transitive verbs,
usually by the \corpus{ga} particle or by the \corpus{wa} particle.
There are actually some subtlety regarding the marking of the S/A argument (\prettyref{sec:topic-subject}),
about the exact meaning of \corpus{ga} and \corpus{wa}.
I call \corpus{ga} the nominative particle instead of the focus particle 
and follow the tradition, to place it together with uncontroversial case particles,
while placing \corpus{wa} together with information structure marking particles TODO: exact name of the two types of particles,
and the justification is done in TODO

\subsubsection{Valency changing}

The so-called Japanese passive is actually affective:
in the affective construction, the A argument undergoes an action caused by others. 
Thus in the affective construction we can still see things like direct objects.

\subsubsection{Sentence final particles}

\acs{sfp}s are important in Japanese grammar:
TODO: sfp for relative clauses? (Or in other words, clause or sentence?)

\subsection{Remarkable features}

\subsubsection{Honorifics}

\section{Noun phrases}

\subsection{Final particles}

There are basically two systems of particles after \ac{np}s.
The first is the case system (\concept{case markers}, 格助詞), including
TODO: distribution (and the following ones)
\begin{itemize}
    \item Nominative: \corpus{ga}, 
    appearing in certain circumstances as the focus marker (\prettyref{sec:topic-subject}).
    \item Accusative: \corpus{o}
    \item Dative: \corpus{ni}: time and location 
    \item Genitive: \corpus{no} 
    \item Lative: \corpus{e}, used for destination direction (like in "to some place")
    \item Ablative: \corpus{kara}, used for source direction (like in "from some place")
    \item Instrumental/Locative: \corpus{de}
\end{itemize}

The second system is the 

The systems are not completely compatible.
A well known generalization is structural case markers are erased when \ac{np}s are topicalized,
while inherent case markers may be kept.


\section{The verb complex}

\subsection{Overview}\label{sec:verb-complex-overview}

The verb complex -- the content of the predicate position -- is discussed in this section.
Since the verb complex is the collective realization of the vP-TP-CP functional projection,
a change in the verb complex may also be a change in 
the arguments (\prettyref{sec:arguments}),
TODO: list all of them.
I follow the practice in \citet{jacques2021grammar}
and use the verb complex as the table of content of what happens in the clause.

Each verbal element -- the main verb, auxiliary verbs, the main verbal adjective -- 
in the verb complex is in one of the following forms.
Here is the list and relevant distributional information:
\begin{itemize}
    \item Irrealis form (未然形). 
    \item Continuative form or adverbial form (連用形), sometimes also called the infinite form.
    \item Terminal form or dictionary form (終止形, 辞书形)
    \item Attributive form (連体形)
    \item Hypothetical form (仮定形)
    \item Imperative form (命令形)
\end{itemize}
The names of the conjugation forms hint something -- but not all -- about their distributions.

Note that which form to choose is \emph{not} determined by agreement with the arguments 
or by clausal categories:
the former never exists in Japanese and the latter is achieved agglutinatively.
Rather, the inflectional strategy in English \corpus{have been doing} 
-- for the outmost (in Japanese, the rightmost) verbal element,
the conjugation form reflects clause-level categories,
while for inner verbal elements the conjugation form is purely decided 
by the higher verbal element (in Japanese, the closest verbal element in the right) -- 
is used in Japanese in a much larger scale.
What makes Japanese and English different 
is the verb complex in Japanese doesn't allow intervening of things like adverbs,
while in English it's pretty acceptable.
Thus, the morphosyntactic standard of wordhood hints that 
the whole verb complex may be understood as a huge grammatical word,
since once finished, it never interacts with the outer world.
This indeed seems to be the convention used in many romanization solutions.

This ambiguity between affixes and grammatical words within complex words sometimes creates
disputation between existing grammar systems.
The status of so-called \corpus{te}-form of verbs as one of the basic conjugation forms, for example,
is rejected by the School Grammar.

\subsection{Conjugation classes and verb stem}

According to whether the final sound of the stem is a vowel or consonant,
Japanese verbs can be divided into \concept{c-stem verbs} and \concept{v-stem verbs}.
C-stem verbs are also called 五段動詞 or type-1 verbs,
because changing the conjugation ending means the final mora may appear in every row of the kana chart, 
and v-stem verbs are also called 一段動詞 or type-2 verbs,
because the last or the second but last mora is always in the same row with the last mora in the dictionary form.
一段動詞 can be further divided into 上一段動詞 or \corpus{iru}-verbs or type-2a verbs
and 下一段動詞 or \corpus{eru}-verbs or type-2b verbs:
the final vowel of \corpus{iru}-verbs is \corpus{i},
and \corpus{-ru} is actually the conjugation ending of the terminal form,
and similarly the final vowel of \corpus{eru}-verbs is \corpus{e}.

\section{Argument structures and alignment}\label{sec:arguments}

This section is about subcategorization frames of verbs and the alignment.

\subsection{Canonical transitive and intransitive verbs}

\section{Simple clauses}

\subsection{Topic and subject}\label{sec:topic-subject}

The subject is often said to be both agentive and topic-like in the typological literature.
What makes Japanese different is despite its accusative nature,
there is still a problem concerning what exact is the subject in Japanese.
The \corpus{wa} particle and the \corpus{ga} particle are commonly called 
the topic particle and the subject particle, respectively.
This correspondence fails the predicate has a stative or habitual meaning:
in that case, \corpus{ga} has the meaning of \translate{it's \dots that},
usually with an exhaustive meaning -- 
only the \ac{np} before \corpus{ga} satisfies the predicate,
and nothing else.
This is probably due to the fact \corpus{ga} assigns focus to the \ac{np} before it:
in English, \corpus{YOU do the job} means it's you -- and only you -- who does the job.
Thus, \corpus{ga} may be better glossed as a focus marker,
and this means in the surface-oriented constituent structure of main clauses, 
there is no such thing as the subject: 
anything moves to SpecTP immediately moves to a higher position in CP, 
making the notion \term{subject} a latent concept without its own particle.
Indeed, we have so-called adverbial nominative in Japanese \citet[\citesec{6.1}]{endo2007locality}.
Note, however, that the alignment in relative clauses is quite simple:
in relative clauses the topic marker \corpus{wa} never appears 
and \corpus{ga} is simply the marker of nominative case.

\section{Clause combining}

\section{Information packaging}


\bibliographystyle{plainnat}
\bibliography{grammars,details,../methodology/famous-grammars}

\end{document}