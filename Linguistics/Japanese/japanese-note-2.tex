\documentclass[UTF8, a4paper, oneside, scheme=plain]{ctexrep}

\usepackage{geometry}
\usepackage{float}
\usepackage{titling}
\usepackage{titlesec}
\usepackage{paralist}
\usepackage{footnote}
\usepackage{enumerate}
\usepackage{amsmath, amssymb, amsthm}
\usepackage{gb4e}
\noautomath
\usepackage{bbm}
\usepackage{soul}
\usepackage{graphicx}
\usepackage{siunitx}
\usepackage[table,xcdraw]{xcolor}
\usepackage{tikz}
\usepackage[ruled, vlined, linesnumbered, noend]{algorithm2e}
\usepackage{xr-hyper}
\usepackage[colorlinks]{hyperref} % linkcolor=black, anchorcolor=black, citecolor=black, filecolor=black
\usepackage[most]{tcolorbox}
\usepackage{caption}
\usepackage{subcaption}
\usepackage{booktabs}
\usepackage{multirow}
\usepackage[figuresright]{rotating}
\usepackage{acro}
\usepackage[round]{natbib} 
\usepackage{nameref,zref-xr}
\zxrsetup{toltxlabel}
\zexternaldocument*[cgel-]{../English/cambridge}[cambridge.pdf]
\zexternaldocument*[alignment-]{../alignment/alignment}[alignment.pdf]
\zexternaldocument*[exercise1-]{../Exercise/2021-3}[2021-3.pdf]
\zexternaldocument*[method-]{../methodology/glossing}[glossing.pdf]
\usepackage{prettyref}

\geometry{left=3.18cm,right=3.18cm,top=2.54cm,bottom=2.54cm}
\titlespacing{\paragraph}{0pt}{1pt}{10pt}[20pt]
\setlength{\droptitle}{-5em}

\DeclareMathOperator{\timeorder}{\mathcal{T}}
\DeclareMathOperator{\diag}{diag}
\DeclareMathOperator{\legpoly}{P}
\DeclareMathOperator{\primevalue}{P}
\DeclareMathOperator{\sgn}{sgn}
\newcommand*{\ii}{\mathrm{i}}
\newcommand*{\ee}{\mathrm{e}}
\newcommand*{\const}{\mathrm{const}}
\newcommand*{\suchthat}{\quad \text{s.t.} \quad}
\newcommand*{\argmin}{\arg\min}
\newcommand*{\argmax}{\arg\max}
\newcommand*{\normalorder}[1]{: #1 :}
\newcommand*{\pair}[1]{\langle #1 \rangle}
\newcommand*{\fd}[1]{\mathcal{D} #1}

\newcommand*{\citesec}[1]{\S~{#1}}
\newcommand*{\citechap}[1]{chap.~{#1}}
\newcommand*{\citefig}[1]{Fig.~{#1}}
\newcommand*{\citetable}[1]{Table~{#1}}
\newcommand*{\citepage}[1]{pp.~{#1}}
\newcommand*{\citefootnote}[1]{fn.~{#1}}

\newrefformat{sec}{\citesec{\ref{#1}}}
\newrefformat{fig}{\citefig{\ref{#1}}}
\newrefformat{tbl}{\citetable{\ref{#1}}}
\newrefformat{chap}{\citechap{\ref{#1}}}
\newrefformat{fn}{\citefootnote{\ref{#1}}}
\newrefformat{box}{Box~\ref{#1}}

\usetikzlibrary{arrows,shapes,positioning}
\usetikzlibrary{arrows.meta}
\usetikzlibrary{decorations.markings}
\tikzstyle arrowstyle=[scale=1]
\tikzstyle directed=[postaction={decorate,decoration={markings,
    mark=at position .5 with {\arrow[arrowstyle]{stealth}}}}]
\tikzstyle ray=[directed, thick]
\tikzstyle dot=[anchor=base,fill,circle,inner sep=1pt]


\tcbuselibrary{skins, breakable, theorems}

\newtcbtheorem[number within=chapter]{infobox}{Box}%
{colback=blue!5,colframe=blue!65,fonttitle=\bfseries, breakable}{box}
\newtcbtheorem[number within=chapter, use counter from=infobox]{theorybox}{Theoretical aspect}%
{colback=orange!5, colframe=orange!65, fonttitle=\bfseries, breakable}{box}
\newtcbtheorem[number within=chapter, use counter from=infobox]{learnbox}{Learning note}%
{colback=green!5,colframe=green!65,fonttitle=\bfseries, breakable}{box}

\newcommand*{\concept}[1]{\textbf{#1}}
\newcommand*{\term}[1]{\emph{#1}}
\newcommand{\corpus}[1]{\emph{#1}}

\DeclareAcronym{blt}{short = BLT, long = Basic Linguistic Theory}
\DeclareAcronym{cgel}{short = CGEL, long = The Cambridge Grammar of the English Language}
\DeclareAcronym{dm}{short = DM, long = Distributed Morphology}
\DeclareAcronym{tag}{long = Tree-adjoining grammar, short = TAG}
\DeclareAcronym{sfp}{long = sentence final particle, short = SFP}
\DeclareAcronym{np}{long = noun phrase, short = NP}
\DeclareAcronym{vp}{long = verb phrase, short = VP}
\DeclareAcronym{pp}{long = preposition phrase, short = PP}
\DeclareAcronym{cls}{long = classifier, short = CLS}
\DeclareAcronym{dist}{long = distal, short = DIST}
\DeclareAcronym{prox}{long = proximate, short = PROX}
\DeclareAcronym{dem}{long = demonstrative, short = DEM}
\DeclareAcronym{dur}{long = durative, short = DUR}
\DeclareAcronym{neg}{long = negative, short = NEG}
\DeclareAcronym{cc}{long = copular complement, short = CC}
\DeclareAcronym{cs}{long = copular subject, short = CS}
\DeclareAcronym{tam}{long = {tense, aspect, mood}, short = TAM}
\DeclareAcronym{past}{long = past, short = PST}
\DeclareAcronym{nonpast}{long = non-past, short = NPST}
\DeclareAcronym{present}{long = present, short = PRES}
\DeclareAcronym{progressive}{long = progressive, short = PROG}
\DeclareAcronym{perfect}{long = perfect, short = PERF}
\DeclareAcronym{passive}{long = passive, short = PASS}
\DeclareAcronym{copula}{long = copula, short = COP}
\DeclareAcronym{polite}{long = polite, short = POL}

\newcommand*{\homo}[2]{#1$_{\text{#2}}$}

\newcommand{\cgel}{\href{../English/cambridge.pdf}{my notes about CGEL}}
\newcommand{\latin}{\href{../Latin/latin-notes.pdf}{my notes about Latin}}
\newcommand{\alignment}{\href{../alignment/alignment.pdf}{my notes about alignment}}
\newcommand{\exerciseone}{\href{../Exercise/2021-3.pdf}{this exercise}}
\newcommand{\method}{\href{../methodology/glossing.pdf}{this note about my understanding of descriptive grammars}}

\newcommand{\ala}{à la}
\newcommand{\translate}[1]{`#1'}
\newcommand{\vP}{\textit{v}P}

% Make subsubsection labeled
\setcounter{secnumdepth}{4}
\setcounter{tocdepth}{4}
% reset example counter every chapter (but do not include the chapter number to the label)
\counterwithin{exx}{chapter} 

\title{Japanese grammar notes}
\author{Jinyuan Wu}

\begin{document}

\maketitle

\automath

This note is a more well-organized version of \href{./japanese-note-1.pdf}{this note}.
It's a reading note of \citet{akiyama2012japanese}, \citet{tsutsui1989dictionary},
as well as lots of books and articles listed in the reference.
The methodology followed is in \method,
i.e. ``largely generatively informed but surface-oriented and flat-tree in the appearance''.

There is basically nothing new in the note. 
Sometimes you will find word-by-word copying of the books and papers in the reference.
Tables are generally from Wikipedia pages.

Boxes in this note should \emph{not} be considered as paragraphs:
they are just there to show my opinion towards certain theoretical aspects 
or language-specific argumentation
or learning advices.
The paragraph after a box is to be considered as next to the paragraph before the box.

TODO list:
\begin{itemize}
    \item The structure of \acs{np}, especially the mutual correlation of particles
    \item Anything between the valency changing suffixes and the tense, polarity, politeness marker
    \item When can appear before the \corpus{te}-form
    \item The end of clauses: what particles appear?
\end{itemize}

\chapter{Introduction}

\section{The Japanese language and its history}

Though this note is purely about contemporary Japanese and contains nothing about old texts,
knowing a little bit of the history helps a lot in understanding, say,
why there are two honorifics systems for things (\prettyref{sec:object-honorifics-prefix}),
why the adjective class is large yet closed (\prettyref{sec:lexical-word-overview}), etc.

\section{Previous studies}

Japanese is a relatively well-documented language,
with a native grammar study tradition.

\section{Language and culture}

\chapter{Overview of Japanese grammar}


\section{Phonology and the writing system}

Japanese is roughly a mora-timing language,
though with some deviation from the prototypical ones.
Japanese has a rather trivial inventory of 5 vowel phonemes (\prettyref{sec:vowel}) 
and 15 consonant phonemes (\prettyref{sec:consonant}),
which can be combined into around 50 moras,
and there are three types of special moras -- 
the syllable-final nasal, vowel lengthening, and consonant germination 
(\prettyref{sec:mora-scheme}).

The above features results in a native writing system,
which includes two sets of mora-based syllabaries called kanas (\prettyref{sec:kana})
and Japanese Chinese characters, the kanji (\prettyref{sec:kanji}).
The spelling system used in this note is a romanization system (\prettyref{sec:romanization}).

\section{Parts of speech}

\subsection{Lexical words: nouns and verbs, and everything else}\label{sec:lexical-word-overview}

Japanese has a clear noun-verb distinction.
This can be found by looking at the morphology:
nouns are subject to case marking,
which is basically adding a particle to the \acs{np} 
which can be dropped especially in casual speech,
while verbs always appear as one of the stem forms plus agglutinative endings.

The following should be recorded concerning a noun:
its compatibility with honorifics affixation (\prettyref{sec:nominal-polite})

The following things should be recorded about a verb:
its conjugation class (\prettyref{sec:conjugation-class}),
the argument structure (\prettyref{chap:arguments}),
which may also affect how the arguments are marked,
TODO: unaccusative and unergative,
and its lexical aspect (\prettyref{sec:lexical-aspect}).

There are two adjective classes:
the verbal class (or \corpus{i}-adjectives)
and the nominal class (or \corpus{na}-adjectives),
with different syntactic distribution 
(verbal adjectives may fill the predicate slot on their own; nominal adjectives never do so)
and morphological appearances
(verbal adjectives are more like verbs).
Since a so-called nominal adjective 
can actually be decomposed into a nominal and a non-standard copula (TODO: ref),
in this note, I'll consider nominal adjectives as \emph{adjectival nouns},
and hence the class of verbal adjectives are simply called \emph{adjectives}.

One rare property of modern Japanese is the verb class and the verbal adjective class 
are already closed classes:
they rarely accept new members (though not entirely impossible).
What makes Japanese rarer is despite being closed,
the verbal adjective class is large.

The vocabulary of Japanese can be divided into three parts according to the etymology:
the native words,
Sino-Japanese words,
and recently borrowed words.
This distinction is sometimes of grammatical significance (TODO: politeness).

\subsection{Function items}

I specifically use the term \term{fucntion items} instead of \term{function words}
in the title of this section,
because the word-or-morpheme-or-phrase problem is especially serious in Japanese morphosyntax
(TODO: ref: school grammar, education grammar).
In functional items,
we have particles in \acs{np}s (\prettyref{chap:particle-in-np}),
\acs{sfp}s in matrix clauses (\prettyref{sec:sfp}),
particles used for clause linking (\prettyref{sec:particle-linking-clause}),
and TODO: 助动词.

\begin{theorybox}{So-called category of functional items}{function-category}
    Though particles are in the grammar and do not really carry category labels like ``noun'' or ``verb''
    and it actually makes no sense to discuss the categories of them,
    the traditional practice to list all particles and classify them 
    is practically desirable, 
    as it provides a quick way to navigate across grammatical systems.
\end{theorybox}

Japanese lacks the prototypically pronoun class:
so-called pronouns are customized referential nouns like \translate{that girl},
and thus the pronoun class is not closed and strictly speaking is not a part of the grammar.
The article class is also not attested.

\section{Noun phrases}

In Japanese \acs{np}s, gender and number are not marked.
The case is marked by an \ac{np}-final particle (\prettyref{sec:case-particle}).
The information structure also receive explicit marking by particles,
including the topic marker \corpus{wa}, 
and in some cases the nominative marker \corpus{ga} has the reading of focus marker
(\prettyref{sec:topic-subject}).

\section{The verb and the clause}

\section{Constituent order}\label{sec:constituent-order}

Japanese has a strict modifier-first constituent order,
and here the term \term{modifier} includes 
arguments in a clause (``modifiers of the verb or the verbal adjective''),
and even \acs{np}s with respect to case particles.

\begin{theorybox}{the notion of head and modifier}{head-modifier}
    This notion of head and modifier is \acs{cgel}-like, 
    and is probably related to a strong head-final tendency in the linearization:
    if a so-called modifier is introduced as a specifier 
    in a functional projection with a root as the core,
    then obviously the root and the functional heads are realized into one unit (for example a verbal complex) 
    and the ``modifier'' precedes the unit to ensure the (functional) head-final rule,
    and therefore in the surface-oriented analysis,
    we also get a modifier-head constituent order (where \term{head} means lexical heads).
    If there is no core root,
    then trivially the ``modifier'' is realized in a position before the spellout of functional heads,
    and the latter is regarded as somehow a head in the \acs{cgel} sense,
    and again we get a modifier-head constituent order,
    if we understand things like case particles as heads,
    which is the \ac{cgel} approach but not the \acs{blt} approach.
\end{theorybox}

Despite the strict modifier-head constituent order,
in the clause, the order of core and peripheral arguments and adverbials is relatively flexible,
which usually reflects the information structure.
Relevant mechanisms include 
topicalization and the ordering between \corpus{wa}-\ac{np}s and \corpus{ga}-\ac{np}s 
(\prettyref{sec:wa-ga-template}, TODO: ref), scrambling (TODO: ref), 
TODO: others

\section{Clause combining}

\section{Remarkable features}

\subsection{Politeness}

Some languages, like Chinese, have a hierarchy of politeness coded in the lexicon.
In Japanese this kind of lexical politeness also exists (\prettyref{sec:personal-pronoun}),
but some components of the grammar are also about politeness
(\prettyref{sec:nominal-polite}, \prettyref{sec:tense-polarity-polite}, TODO).
Some parts of the grammar do not involve any category about politeness,
but using them is shunned if the speaker wants to be polite
(\prettyref{sec:imperative-form}).
Even spelling is related to politeness (\prettyref{sec:spelling-conventions}).

\subsection{The role of gender}

Japanese doesn't give any place to grammatical gender.
However, the gender of the \emph{speaker} is important:
there is one dialect for men,
and another for women.
Grammar points involving gender of the speaker include TODO

\section{The structure of this note}

\begin{theorybox}{The organization in reference grammars}{grammar-org}
    The structure of this note and the contents of chapters follow 
    the examples set by \citet{Friesen2017}, \citet{jacques2021grammar}, \citet{Grimm2021},
    the famous \acs{cgel} \citep{cgel}, and of course Dixon's three volumes of \acs{blt}.
    The nominal chapters (TODO: ref), 
    
    The notion of verbal complex (\prettyref{chap:verb-complex}) is also found in 
    \citet{Friesen2017}.
\end{theorybox}

\chapter{Phonology and the writing system}

\section{Vowels}\label{sec:vowel}

\section{Consonants}\label{sec:consonant}

\section{Phonotactics}

\subsection{The scheme of moras}\label{sec:mora-scheme}

Japanese is usually analyzed as a mora language:
each mora occupies one rhythmic unit and is of the same length.
There are still some deviations from the perfect mora-timing model:
moras with devoiced vowels may be shorter (TODO: ref),
and so is geminated consonants (see below),
and syllable-based language games and sound changes are observed 
\citep[\citesec{3.3}]{tsujimura2013introduction}.

The allowed types of moras include V, CV, jV, CjV, R, N, and Q.
In the following I will talk about the meaning of the symbols j, R, N, and Q. 

\subsubsection{Semivowel-related moras}

Here the symbol j is the glide /j/,
which may appear after a non-glide consonant and before a vowel,
and the sequence of the three phonemes is still one mora.
The appearance of j is called 拗音 \corpus{yōon} in Japanese.
The glide is only compatible with /a/, /u/ and /o/.
The symbol N is a moraic nasal:
it constitutes a single mora, called 撥音 \corpus{hatsuon} 
and never appears at the initial of a word.
Thus, except for CjV, Japanese doesn't allow multiple consonants,
and except for (C)VN, Japanese syllables are always open.

\subsubsection{Prolonged vowels and consonants}

The symbol R means a chroneme,
which prolongs the last vowel,
called 長音 \corpus{chōon}.
It's compatible with any vowels.
Q means geminating the following consonant.
It's called 促音 \corpus{sokuon},
and may be realized as ``pause for a mora'' before the consonant. 
The two abstract phonemes represent adjustment of vowel and consonant lengths,
which are all distinctive in Japanese phonology.

The \corpus{ch\={o}on} and \corpus{sokuon} are important to understand 
certain morphological phenomenon in Japanese,
in which the underlying form of a prolonged vowel or consonant 
is actually a vowel or consonant plus a suffix or prefix.
For example, when \corpus{o} is followed by \corpus{u}, 
the two vowels are realized as a prolonged \corpus{o}: \corpus{\={o}}
(TODO: ref).

I should focus here (especially to readers familiar with syllable languages)
that N and Q are moras themselves,
though the contents of them depend on the phonological environment.

\section{Accent}

\section{Historical sound change}

\subsection{In verbal conjugation}

\subsubsection{Removing the second \corpus{r} or \corpus{s}}\label{sec:removing-rs}

When two consonants meet each other (TODO: in the verbal complex only?),
and the second consonant is \corpus{r} or \corpus{s},
the second consonant is removed.
This is exemplified in marking of the passive and the causative (\prettyref{sec:valency-irrealis}),
the alternation of the ending in the terminal form (\prettyref{sec:terminal-form}),
and TODO.

\subsubsection{Consonant plus \corpus{t}}\label{sec:t-euphonic}

The contraction between a consonant and the sound \corpus{t} is relevant in 
the past tense (\prettyref{sec:indicative-paradigm}),
the \corpus{te}-form (\prettyref{sec:te-form}),
and the condition form (TODO:ref),
and results in a new form in the internal morphology of verbal elements (\prettyref{sec:euphonic-form}).

\section{The writing system}

\subsection{Overview}

The mora-based phonology of Japanese results in two syllabaries used to write Japanese,
called \term{kanas} 仮名.
There are two kinds of kana in contemporary use:
one is hiragana 平仮名, the other is katakana 片仮名.
Hiragana is used to write grammatical items (like inflectional endings and particles)
and a subset of words with native etymology,
while katakana is used for newly borrowed words.
Ideophones are traditionally written in katakana,
though sometimes they are written in hiragana for a softened, adorable appearance.

There exists several romanization systems for Japanese,
which are called \corpus{rōmaji} ローマ字 \translate{Roman letters}.
The Hepburn romanization is designed for non-native speakers,
which roughly reflects the actually contemporary pronunciation.
There are other systems of romanization, which are discussed in \prettyref{sec:romanization}.

\subsection{Differences between romanizations}\label{sec:romanization}

Since I (and intended readers) of this note are all non-native speakers,
it's a good idea to first introduce the romanization systems,
and for the same reason,
the system used in this note is the revised Hepburn romanization,
which tells us more about the phonological structure of Japanese 
in the eyes of an outsider.

All systems of romanization use the same set of letters to represent the five vowels:
\corpus{a}, \corpus{i}, \corpus{u}, \corpus{e}, and \corpus{o}.
The letters for the consonants are also largely the same.
Consonants without any flavor of the glide j are represented by 
\corpus{k}, \corpus{g}, \corpus{s}, \corpus{z}, \corpus{t}, \corpus{d}, \corpus{n}, 
\corpus{h}, \corpus{b}, \corpus{p}, \corpus{m}, \corpus{r}, and \corpus{w}.



\subsection{Table of kanas}\label{sec:kana}

Below is a table enumerating all hiraganas and their revised-Hepburn romaji correspondence, 
called \corpus{gojūon} 五十音 \translate{fifty sounds}:
\begin{table}[H]
    \caption{The \corpus{gojūon} 五十音}
    \label{tbl:hiragana-chart}
    \centering
    \begin{tabular}{cccccc}
        \toprule
         & a & i & u & e & o \\ 
        \midrule
        $\emptyset$ & あ \corpus{a} & い \corpus{i} & う \corpus{u} & え \corpus{e} & お \corpus{o} \\ 
        k & か \corpus{ka} & き \corpus{ki} & く \corpus{ku} & け \corpus{ke} & こ \corpus{ko} \\ 
        g & が \corpus{ga} & ぎ \corpus{gi} & ぐ \corpus{gu} & げ \corpus{ge} & ご \corpus{go} \\ 
        s & さ \corpus{sa} & し \corpus{si} & す \corpus{su} & せ \corpus{se} & そ \corpus{so} \\ 
        z & ざ & じ & ず & ぜ & ぞ \\ 
        t & た & ち & つ & て & と \\ 
        d & だ & ぢ & づ & で & ど \\ 
        n & な & に & ぬ & ね & の \\ 
        h & は & ひ & ふ & へ & ほ \\ 
        b & ば & び & ぶ & べ & ぼ \\ 
        p & ぱ & ぴ & ぷ & ぺ & ぽ \\ 
        m & ま & み & む & め & も \\ 
        y & や & ~ & ゆ & ~ & よ \\ 
        r & ら & り & る & れ & ろ \\ 
        w & わ & ゐ & ~ & ゑ & を \\ 
        \bottomrule
    \end{tabular}
\end{table}

The kana table contains some information about the underlying forms of certain moras.
(That's to say, it doesn't exactly faithfully represent how modern Japaneses speak.)
Certain phonological rules can already be observed in the kana-romaji correspondence in the above table.

\begin{learnbox}{Remembering kanas}{kana-remember}
    Hiraganas can be remembered by recalling the spelling of grammatical items.
    The nominative case particle is か.
    The genitive case particle is の.
    From the passive marker られ \corpus{rare} and the causative marker させ \corpus{sase} 
    (\prettyref{sec:valency-irrealis}),
    we recognize four kanas.
    The dictionary form ending of v-stem verbs is る (\prettyref{sec:terminal-form}).
    The dictionary form ending of verbal adjectives is い (\prettyref{sec:adjective-terminal-form}).
    The \corpus{te}-form of verbs ends in て (\prettyref{sec:te-form}).
    The verb politeness markers ます \corpus{masu} and ました \corpus{mashita} appear frequently 
    (\prettyref{sec:tense-polarity-polite}).
    TODO
\end{learnbox}

The \corpus{hatsuon} is written as ん. 
The \corpus{y\={o}on} is represented as a \corpus{Ci} kana plus a small-sized \corpus{jV} kana.
Thus \corpus{kya} is written as きゃ \corpus{kiya}.

There is no symbol representing \corpus{ch\={o}ons} explicitly:
they are instead represented as two kanas representing their underlying forms.
Thus \corpus{n\={o}} is written as のう \corpus{nou}.
The \corpus{sokuon} Q is written as っ, a small-sized つ.

\subsection{The Kanji}\label{sec:kanji}


\subsection{Spelling conventions}\label{sec:spelling-conventions}

Verbs and adjectives are written as their kanji spellings (if there are any)
plus the kana representing the final mora:
来る, 青い.
When conjugated, the final or sometimes the last two or even the last three
kana indicates the word's internal morphology (\prettyref{sec:conjugation-class}). 
These kanas are called \term{okurigana} 送り仮名.

Certain functional items receive spelling that do not faithfully represent their pronunciations.
The topic marker \corpus{wa} (\prettyref{sec:wa-topic}) is written as は,
though the kana ordinarily means \corpus{ha}.
It's impossible to collect all these spelling conventions for functional items at the same place:
I'll talk about these conventions together with the morphosyntactic distribution of the functional items.

When several spelling forms of an item are possible,
usually there are subtle pragmatic differences between them,
including politeness (the more kanjis, the more polite), TODO

\subsection{Symbols and punctuation}\label{sec:writing-symbols}



TODO: punctuations, like 、 and 。

\chapter{Nominal categories}

\begin{theorybox}{How to enumerate nominal classes}{how-to-nominal-class}
    The organization of nominal categories is the same as \citet[\citechap{3}, \citechap{4}]{Friesen2017}.
\end{theorybox}

\section{Pronouns}

\subsection{Personal pronouns}\label{sec:personal-pronoun}

Japanese lacks personal pronouns in the canonical meaning.

Though the so-called personal pronouns are often related to gender, 
it's not absolute: 
the correlation between pronouns and gender is more likely 
to be \emph{external} to the grammar:
though the first person pronoun \corpus{boku} is usually used by males,
it's possible for a woman to use the pronoun 
because of her dialect or
as a suggestion about her unique characteristics.

\subsection{Demonstratives and interrogative pronouns}

Japanese demonstratives and interrogative pronouns are rather regular in their forms:
almost each of them can be split into two parts,
one indicating the range of the objects denoted,
the other indicating the syntactic role of the demonstrative.
They are shown in \prettyref{tbl:demonstrative}.

\begin{table}[H]
    \centering
    \caption{Japanese demonstratives and interrogative pronouns}
    \label{tbl:demonstrative}
    \begin{tabular}{cllll}
    \toprule
                    & \multicolumn{1}{c}{\corpus{ko-}}                                                  & \multicolumn{1}{c}{\corpus{so-}}                                                  & \multicolumn{1}{c}{\corpus{a-}}                                                           & \multicolumn{1}{c}{\corpus{do-}}                                                       \\ \toprule
    \corpus{-re}    & \begin{tabular}[c]{@{}l@{}}\corpus{kore}\\ \translate{this one}\end{tabular}      & \begin{tabular}[c]{@{}l@{}}\corpus{sore}\\ \translate{that one}\end{tabular}      & \begin{tabular}[c]{@{}l@{}}\corpus{are}\\ \translate{that one over there}\end{tabular}    & \begin{tabular}[c]{@{}l@{}}\corpus{dore}\\ \translate{which one?}\end{tabular}         \\ \midrule
    \corpus{-no}    & \begin{tabular}[c]{@{}l@{}}\corpus{kono}\\ \translate{(of) this}\end{tabular}     & \begin{tabular}[c]{@{}l@{}}\corpus{sono}\\ \translate{(of) that}\end{tabular}     & \begin{tabular}[c]{@{}l@{}}\corpus{ano}\\ \translate{(of) that over there}\end{tabular}   & \begin{tabular}[c]{@{}l@{}}\corpus{dono}\\ \translate{(of) what?}\end{tabular}         \\ \midrule
    \corpus{-nna}   & \begin{tabular}[c]{@{}l@{}}\corpus{konna}\\ \translate{like this}\end{tabular}    & \begin{tabular}[c]{@{}l@{}}\corpus{sonna}\\ \translate{like that}\end{tabular}    & \begin{tabular}[c]{@{}l@{}}\corpus{anna}\\ \translate{like that over there}\end{tabular}  & \begin{tabular}[c]{@{}l@{}}\corpus{donna}\\ \translate{what sort of?}\end{tabular}     \\ \midrule
    \corpus{-ko}    & \begin{tabular}[c]{@{}l@{}}\corpus{koko}\\ \translate{here}\end{tabular}          & \begin{tabular}[c]{@{}l@{}}\corpus{soko}\\ \translate{there}\end{tabular}         & \begin{tabular}[c]{@{}l@{}}\corpus{asoko}\\ \translate{over there}\end{tabular}           & \begin{tabular}[c]{@{}l@{}}\corpus{doko}\\ \translate{where?}\end{tabular}             \\ \midrule
    \corpus{-chira} & \begin{tabular}[c]{@{}l@{}}corpus{kochira}\\ \translate{this way}\end{tabular}    & \begin{tabular}[c]{@{}l@{}}\corpus{sochira}\\ \translate{that way}\end{tabular}   & \begin{tabular}[c]{@{}l@{}}\corpus{achira}\\ \translate{that way over there}\end{tabular} & \begin{tabular}[c]{@{}l@{}}\corpus{dochira}\\ \translate{which way?}\end{tabular}      \\ \midrule
    \corpus{-u}     & \begin{tabular}[c]{@{}l@{}}\corpus{kō}\\ \translate{in this manner}\end{tabular}  & \begin{tabular}[c]{@{}l@{}}\corpus{sō}\\ \translate{in that manner}\end{tabular}  & \begin{tabular}[c]{@{}l@{}}\corpus{ā }\\ \translate{in that (other) manner}\end{tabular}  & \begin{tabular}[c]{@{}l@{}}\corpus{dō}\\ \translate{how? in what manner?}\end{tabular} \\ \midrule
    \corpus{-itsu}  & \begin{tabular}[c]{@{}l@{}}\corpus{koitsu}\\ \translate{this person}\end{tabular} & \begin{tabular}[c]{@{}l@{}}\corpus{soitsu}\\ \translate{that person}\end{tabular} & \begin{tabular}[c]{@{}l@{}}\corpus{aitsu}\\ \translate{that (other) person}\end{tabular}  & \begin{tabular}[c]{@{}l@{}}\corpus{doitsu}\\ \translate{who?}\end{tabular}             \\ \bottomrule
    \end{tabular} 
\end{table}

It can be seen Japanese demonstratives are divided into three types according to what they refer to: 
proximal demonstratives, medial demonstrative, and distal demonstratives.

TODO: case particle compatibility

\section{Numerals}

\chapter{Nominal morphology}

\section{Nominal derivations}

\subsection{Suffixation}

\subsubsection{The agentive \corpus{-te}}\label{sec:agentive-te}

Only event verbs are able to take the \corpus{-te} suffix:
verbs about states are unable to undergo such a process.
This is an instance of the Japanese lexical aspect of verbs (\prettyref{sec:lexical-aspect}).

\section{Honorifics}\label{sec:nominal-polite}

\subsection{Prefixes for things}\label{sec:object-honorifics-prefix}

For Sino-Japanese nouns, the prefix \corpus{go-} is used to add politeness.
For native nouns, the prefix \corpus{o-} is used instead.
Some everyday Sino-Japanese nouns receive the \corpus{o-} prefix as well.
TODO: how to write in kanji?

\subsection{Suffixes for people}

It should be noted the honorifics suffixes can be applied to so-called personal pronouns: (TODO: ref)

\subsubsection{\corpus{San}}

This is the most neutral title.
It can be used in most situations:
to address men or women, to address 

\subsubsection{\corpus{Sama}}

\corpus{Sama} (様, さま) is used to address someone of higher status:
customers, judges, religious authorities, even deity.
It also appears frequently in formal writing, 
but may be too polite for an ordinary conversation.

\subsubsection{\corpus{Kun}}

The title \corpus{kun} (君, くん) has two meanings when used to address a male or a female.
It can be used for boys and men that are younger than yourself or at the same age as you,
especially when you two have a close relation.
For example, 
in workplace it's usually used by a senior member to address a relatively junior member 
who he or she has kind of personal relation (otherwise it may be considered rude);
it's also used by males or females to address a male they have already known for quite a while,
and/or have emotional attachment;
in the Legislature it's used to address members.

When used for females, \corpus{kun} is a more respectful (and much less frequent) version of \corpus{chan},
with the meaning of childlike cuteness kept:
the female addressed is, for example, sweet and kind,
so people use \corpus{kun} to imply respectful endearment.
It's still not a formal title.

\subsubsection{\corpus{Chan}}

\corpus{Chan} is an informal title used for an endearing sense.
It can be used for young children, close friends or lovers, family members, and adorable animals as well.
It's sometimes used to address a young woman who is no longer a ``child'' 
-- just like the generalized meaning of \corpus{girl} in English.

\subsubsection{\corpus{Sensei}}

\corpus{Sensei} (先生, せんせい) is a title used for people mastering a skill well, 
especially people who would instruct others concerning that skill:
teachers, doctors, lawyers, well recognized artists, politicians,
the head of a martial art training center, etc.

When the focus is ``high academic achievement'', \corpus{hakase} is a better choice.

\subsubsection{\corpus{Hakase}}

\corpus{Hakase} (博士, はかせ), as its kanji version directly reflects, means \translate{Ph.D.}

\section{Collective markers}

Though Japanese lacks the category of number,
these exist several collective markers.

\chapter{Particles in noun phrases}\label{chap:particle-in-np}

There are several systems of particles after \ac{np}s:
case particles 格助詞 (\prettyref{sec:case-particle}),
and adverbial particles 副助詞 (\prettyref{sec:adverbial-particle} 
-- the name is actually misleading, see the relevant section).

\begin{theorybox}{List of particles as a lookup table}{case-particles}
    I learn from \citet{jacques2021grammar} and organize all \ac{np}-final particles into one chapter 
    for quick lookup of the distributions of case, adverbial types, etc.
\end{theorybox}

The systems are not completely compatible (\prettyref{sec:particle-compatible}).
For example, a well known generalization is structural case markers 
-- the nominative \corpus{ga} and the accusative \corpus{o} -- 
are erased when \ac{np}s are topicalized,
while inherent case markers may be kept.

\section{Case particles}\label{sec:case-particle}

Here is a list of case particles:
\begin{itemize}
    \item Nominative: \corpus{ga}, 
    appearing in certain circumstances as the focus marker (\prettyref{sec:topic-subject}).
    \item Accusative: \corpus{o}
    \item Dative: \corpus{ni}: time and location 
    \item Genitive: \corpus{no} 
    \item Lative: \corpus{e}, used for destination direction (like in "to some place")
    \item Ablative: \corpus{kara}, used for source direction (like in "from some place")
    \item Instrumental/Locative: \corpus{de}
\end{itemize}

\subsection{The accusative case \corpus{o}}

\subsubsection{The object}

The accusative case is usually used to mark the object.
Note, however, that the semantically O argument may also be marked by \corpus{ga} 
and promoted to the initial of the clause, 
and thus there is no syntactic object in the clause 
(\prettyref{sec:multiple-ga}).

\subsubsection{The path}

The path argument (\translate{walk through/along/in \dots}) is marked as accusative.

\subsection{The possessive marker \corpus{no}}

\section{Adverbial particles, or miscellaneous}\label{sec:adverbial-particle}

The so-called adverbial particle class is a catch-all class for all particles appearing in the \ac{np}
but hard to classify.
They don't necessarily appear on peripheral arguments:
\corpus{dake}, for example, can appear on an object (\prettyref{sec:dake}),
and \corpus{hodo} can appear on a copular complement.

\subsection{\corpus{Dake}}\label{sec:dake}

\subsection{\corpus{Hodo}}

\subsubsection{Approximation}

The particle \corpus{hodo} may be attached to a countable \ac{np},
and takes the reading of \corpus{about}.
The \ac{np} may be a copular complement:
\begin{exe}
    \ex 1000円ほどです 
    \gll Sen en hodo desu \\
    1000 yen HODO \acs{copula}.\acs{polite}-\acs{present} \\
    \glt \translate{(It) is about 1,000 yen.}
\end{exe}

\section{Compatibility}\label{sec:particle-compatible}



\chapter{The structure of noun phrases}

\chapter{The verbal complex}\label{chap:verb-complex}

\begin{theorybox}{The notion of verbal complex}{verb-complex-def}
    In this note I use the term \term{verbal complex} to denote the \ac{blt} verb phrase,
    i.e. what fills the \ac{blt} predicate slot,
    i.e. the realization of the verbal functional hierarchy -- from \vP to CP.
    Periphrastic conjugation -- things similar to the English \corpus{is doing my project} -- 
    is also included as a part of the verbal complex.
\end{theorybox}

\section{Introduction}

\subsection{Typological information}

Japanese verbal complexes are headed by verbs or adjectives.
A verbal complex headed by an adjective is more limited 
in the suffix chain (TODO: ref; see below for the meaning of the term \term{suffix chain}).

Japanese is typologically agglutinative:
the morphemes in the verbal complex have relatively clear boundaries,
each morpheme representing a grammatical category.
Still, there are two important factors in Japanese that deviate away 
from the perfect agglutinative prototype.
The first is there is still some degree of fusion, 
in which historically analyzable morphemes arguably already form a single fused morpheme 
(\prettyref{sec:tense-polarity-polite}, TODO).
The second is most components in the verbal complex 
-- both the lexical head and most functional morphemes --
either have \emph{internal} morphology or alter the form of suffixes following them,
depending on how you analyze it (\prettyref{box:affix-conjugation-theory}),
and this is related to the analysis of basic forms of the verb (\prettyref{sec:conjugation-class}).

As is said before, Japanese is strongly modifier-first,
and hence productive functional morphemes in the verbal complex are predominantly after the main verb
(\prettyref{sec:suffix-chain}, TODO: periphrastic conjugation).
Prefixes are highly limited -- they are mainly used in derivations and honorifics (\prettyref{sec:prefixes}).
In the verbal complex, the lexical verb always comes the first,
followed by a chain of auxiliaries,
due to the modifier-head constituent order.

The so-called ``auxiliaries'' I just mentioned include 
pure inflectional suffixes and auxiliary verbs:
the latter may be seem as head verbs or parts of periphrastic conjugation forms
depending on their stage of grammaticalization,
but in either way they appear after the main verb because they are ``heads'' 
in the sense of ``head-final'' typological parameter of Japanese (\prettyref{box:head-modifier}).

\begin{theorybox}{About internal morphology of conjugation suffixes}{affix-conjugation-theory}
    Note that since they are both spellout of functional heads,
    the fact that auxiliary verbs and suffixes have almost no differences shouldn't surprise anyone.
    The real problem is the inner morphology of them 
    obviously has no correspondence in the syntactic tree.
    It's therefore a morphophonological phenomenon,
    not a morphosyntactic one.
    This is a mismatch between syntax and phonetic forms,
    and it's usually analyzed as a result of morphophonological readjustments.
    The English \corpus{have been being assaulted}
    would be generated by the following vocabulary insertion
    \begin{exe}
        \sn {[have-en]}_{\text{\acs{present}, \acs{perfect}}} 
        [be-ing]_{\text{\acs{progressive}}} [be-en]_{\text{\acs{passive}}} assault
    \end{exe}
    plus phonetic readjustment rules of ``affix lowering''.
    The same can happen in Japanese, in an exactly the same manner,
    except now higher functional heads are on the right 
    and the affix like \corpus{-ing} or \corpus{-en} 
    which is responsible for the inner structure of auxiliaries 
    is now passed from right to left.
    This makes things even easier, 
    because now the end of each conjugation suffix is directly connected to the 
    ``affix'' of the higher position conjugation suffix.
    Indeed, this seems to be the origin of the so-called irrealis form,
    which is completely unmotivated semantically (\prettyref{sec:irrealis-form}).

    There is another way to analyze English auxiliaries.
    For English \corpus{have been being done},
    we may posit the following rules.
    The auxiliary verb \corpus{being} is the default spellout of the progressive aspect:
    if \corpus{-ing} is close to a verb stem, then it goes after that verb stem,
    but in \corpus{being done}, the verb stem has already been incorporated into \corpus{done},
    so \corpus{-ing} is spelt out as \corpus{being}.
    Similarly we have \corpus{been}, which is the default spellout of the so-called 
    Asp$_{\text{\corpus{en}}}$ head \citep{ramchand2014152}.
    Thus, the ``stems'' of auxiliary verbs in the English verb phrase 
    are actually inserted later as a last resort.
    This works for English, but not for Japanese without necessary amendments:
    now it's the ``stems'' of auxiliaries that contains grammatical information,
    and now the continuative ending \corpus{-i} is inserted as a last resort,
    if we deem the ``stem'' of an auxiliary can't appear in the final output.

    The two analyses have their own pros and cons.
    For the analysis of the continuative form 
    -- which literally appears almost all of the case before a pure conjugation suffix
    (i.e. ``auxiliary verb'' in the School Grammar) -- 
    the best approach is of course to say 
    it comes from the last resort insertion, somehow as the default form.
    For the so-called irrealis form,
    the morphophonological readjustment is of course the best analysis.
    The rest of the forms are also to be discussed case by case (\prettyref{sec:internal-forms}).

    For a surface-oriented phenomenological analysis,
    we may say things like ``an auxiliary selects a continuative form or a \corpus{te}-form'',
    and we may also say ``an auxiliary actually has a separable part $\corpus{i}$ 
    that moves to the component before it'',
    and we may also say ``an auxiliary triggers insertion of a vowel before it''.
    The first wording is the most phenomenological,
    and the second and the third correspond to the above two approaches.
\end{theorybox}

\subsection{The auxiliary verb/suffix distinction}\label{sec:auxliary-verb-suffix}

The distinction between inflectional suffixes and auxiliary verbs -- 
or in other words, the distinction between canonical and periphrastic conjugations -- 
is subtle: they are both after the lexical word,
intervening between them and the lexical word is disallowed, 
they have similar internal morphology, etc.
The main criterion used to motivate a distinction between the two 
seems to be that the latter have lexical uses, while the former don't.
This may be the reason that the School Grammar directly calls them directly as ``verbs'' 
(\prettyref{sec:so-called-auxiliary-verb}).
That being said, a suffix-auxiliary verb distinction 
-- essentially a morphological v.s. periphrastic distinction -- 
can still be observed,
based on comparison between one-clause constructions 
and clause combining constructions:
what is obviously the first while still having a form resembling the latter is considered periphrastic.

Take the marking of aspect as an example:
the plain aspect is marked by nothing, and with the indicative mood (TODO: ref, and whether this concept holds water) and the plain aspect,
the auxiliaries (we are not sure about their nature yet, so let's use the vague name \term{auxiliary}) 
are added in a templatic manner, shown in \prettyref{fig:verb-complex-scheme}(a) 
(\prettyref{sec:suffix-chain}).
In a complement clause construction taking a \corpus{te}-form (TODO: ref),
visualized as \prettyref{fig:verb-complex-scheme}(b),
things are different: now the verbal complex of the complement clause ends in \corpus{te},
and the main verb of the matrix clause has its own list of auxiliaries with the same rigid order.
Now consider the progressive aspect (TODO: ref):
in the progressive aspect, the templatic chain of auxiliaries after the main verb ends with \corpus{te},
and then comes the \corpus{iru} (TODO: ref), followed by its own chain of auxiliaries 
which marks the tense, polarity and politeness.
Its form is close to \prettyref{fig:verb-complex-scheme}(b),
though we know the \corpus{-te i-} segment is actually the spellout of the progressive aspect head 
(see \prettyref{box:affix-conjugation-theory} again).
Thus, we say the progressive aspect is marked by periphrastic conjugation 
while the plain aspect is not,
and we say in the auxiliary chain found in 
\prettyref{fig:verb-complex-scheme}(a-c)
is made by suffixes: 
there is a template of the chain, each slot of which holds one suffix,
while the main verb-like ??? position in \prettyref{fig:verb-complex-scheme}(c) 
is filled by an auxiliary verb.
In \prettyref{fig:verb-complex-scheme}(b),
there are two verbal complexes:
one is in the subordinated clause,
and the other is in the matrix clause.
In \prettyref{fig:verb-complex-scheme}(c)
there is only one verbal complex,
but there are two suffix chains.

\begin{figure}[H]
    \centering
    


\begin{tikzpicture}[x=0.75pt,y=0.75pt,yscale=-1,xscale=1]
%uncomment if require: \path (0,507); %set diagram left start at 0, and has height of 507

%Shape: Rectangle [id:dp35161586898455655] 
\draw   (147,88.26) -- (234,88.26) -- (234,109.26) -- (147,109.26) -- cycle ;
%Shape: Rectangle [id:dp16503080821526184] 
\draw   (286,88.26) -- (241,88.26) -- (241,109.26) -- (286,109.26) -- cycle ;
%Shape: Rectangle [id:dp16255280450821386] 
\draw   (241,88.26) -- (234,88.26) -- (234,109.26) -- (241,109.26) -- cycle ;
%Shape: Rectangle [id:dp6398851060853092] 
\draw   (338,88.26) -- (293,88.26) -- (293,109.26) -- (338,109.26) -- cycle ;
%Shape: Rectangle [id:dp5923295827787749] 
\draw   (293,88.26) -- (286,88.26) -- (286,109.26) -- (293,109.26) -- cycle ;
%Straight Lines [id:da3757507657024388] 
\draw [color={rgb, 255:red, 80; green, 227; blue, 194 }  ,draw opacity=1 ][line width=2.25]    (241,125.26) -- (338,125.26) ;
%Straight Lines [id:da9972923973085535] 
\draw [color={rgb, 255:red, 80; green, 227; blue, 194 }  ,draw opacity=1 ][line width=2.25]    (239,210.26) -- (336,210.26) ;
%Straight Lines [id:da5465523467441933] 
\draw [color={rgb, 255:red, 80; green, 227; blue, 194 }  ,draw opacity=1 ][line width=2.25]    (449,210.26) -- (546,210.26) ;
%Straight Lines [id:da9103547136453445] 
\draw [color={rgb, 255:red, 144; green, 19; blue, 254 }  ,draw opacity=1 ][line width=2.25]    (149,240.26) -- (338,240.26) ;
%Straight Lines [id:da21214292532671486] 
\draw [color={rgb, 255:red, 144; green, 19; blue, 254 }  ,draw opacity=1 ][line width=2.25]    (149,270.26) -- (546,270.26) ;
%Straight Lines [id:da5062887834362] 
\draw [color={rgb, 255:red, 80; green, 227; blue, 194 }  ,draw opacity=1 ][line width=2.25]    (239,357.26) -- (336,357.26) ;
%Straight Lines [id:da23787107953327236] 
\draw [color={rgb, 255:red, 80; green, 227; blue, 194 }  ,draw opacity=1 ][line width=2.25]    (449,357.26) -- (546,357.26) ;
%Straight Lines [id:da7978827288524031] 
\draw [color={rgb, 255:red, 74; green, 144; blue, 226 }  ,draw opacity=1 ][line width=2.25]    (355,357.26) -- (449,357.26) ;
%Straight Lines [id:da7314914818351383] 
\draw [color={rgb, 255:red, 144; green, 19; blue, 254 }  ,draw opacity=1 ][line width=2.25]    (149,390.26) -- (546,390.26) ;
%Shape: Rectangle [id:dp8318830214100368] 
\draw   (146,172.26) -- (233,172.26) -- (233,193.26) -- (146,193.26) -- cycle ;
%Shape: Rectangle [id:dp9791823848488326] 
\draw   (285,172.26) -- (240,172.26) -- (240,193.26) -- (285,193.26) -- cycle ;
%Shape: Rectangle [id:dp7551650922097486] 
\draw   (240,172.26) -- (233,172.26) -- (233,193.26) -- (240,193.26) -- cycle ;
%Shape: Rectangle [id:dp9212991968444435] 
\draw   (337,172.26) -- (292,172.26) -- (292,193.26) -- (337,193.26) -- cycle ;
%Shape: Rectangle [id:dp6573200834937438] 
\draw   (292,172.26) -- (285,172.26) -- (285,193.26) -- (292,193.26) -- cycle ;
%Shape: Rectangle [id:dp8980580590320928] 
\draw   (354,172.26) -- (441,172.26) -- (441,193.26) -- (354,193.26) -- cycle ;
%Shape: Rectangle [id:dp9235071180644034] 
\draw   (493,172.26) -- (448,172.26) -- (448,193.26) -- (493,193.26) -- cycle ;
%Shape: Rectangle [id:dp0766475322070379] 
\draw   (448,172.26) -- (441,172.26) -- (441,193.26) -- (448,193.26) -- cycle ;
%Shape: Rectangle [id:dp5593612719238286] 
\draw   (545,172.26) -- (500,172.26) -- (500,193.26) -- (545,193.26) -- cycle ;
%Shape: Rectangle [id:dp9118325066537258] 
\draw   (500,172.26) -- (493,172.26) -- (493,193.26) -- (500,193.26) -- cycle ;
%Shape: Rectangle [id:dp1554128698162991] 
\draw   (146,320.26) -- (233,320.26) -- (233,341.26) -- (146,341.26) -- cycle ;
%Shape: Rectangle [id:dp1833810858919207] 
\draw   (285,320.26) -- (240,320.26) -- (240,341.26) -- (285,341.26) -- cycle ;
%Shape: Rectangle [id:dp8141735612127439] 
\draw   (240,320.26) -- (233,320.26) -- (233,341.26) -- (240,341.26) -- cycle ;
%Shape: Rectangle [id:dp26275584002034424] 
\draw   (337,320.26) -- (292,320.26) -- (292,341.26) -- (337,341.26) -- cycle ;
%Shape: Rectangle [id:dp4604350211757442] 
\draw   (292,320.26) -- (285,320.26) -- (285,341.26) -- (292,341.26) -- cycle ;
%Shape: Rectangle [id:dp8796206525042598] 
\draw   (354,320.26) -- (441,320.26) -- (441,341.26) -- (354,341.26) -- cycle ;
%Shape: Rectangle [id:dp7287630058740182] 
\draw   (493,320.26) -- (448,320.26) -- (448,341.26) -- (493,341.26) -- cycle ;
%Shape: Rectangle [id:dp4139726396039576] 
\draw   (448,320.26) -- (441,320.26) -- (441,341.26) -- (448,341.26) -- cycle ;
%Shape: Rectangle [id:dp18763826945232376] 
\draw   (545,320.26) -- (500,320.26) -- (500,341.26) -- (545,341.26) -- cycle ;
%Shape: Rectangle [id:dp936337555876763] 
\draw   (500,320.26) -- (493,320.26) -- (493,341.26) -- (500,341.26) -- cycle ;

% Text Node
\draw (190.5,98.76) node   [align=left] {main verb};
% Text Node
\draw (263.5,98.76) node   [align=left] {aux};
% Text Node
\draw (315.5,98.76) node   [align=left] {aux};
% Text Node
\draw (89.5,90) node [anchor=north west][inner sep=0.75pt]   [align=left] {(a)};
% Text Node
\draw (89,174) node [anchor=north west][inner sep=0.75pt]   [align=left] {(b)};
% Text Node
\draw (289.5,128.26) node [anchor=north] [inner sep=0.75pt]  [color={rgb, 255:red, 80; green, 227; blue, 194 }  ,opacity=1 ] [align=left] {suffix chain};
% Text Node
\draw (287.5,213.26) node [anchor=north] [inner sep=0.75pt]  [color={rgb, 255:red, 80; green, 227; blue, 194 }  ,opacity=1 ] [align=left] {suffix chain};
% Text Node
\draw (497.5,213.26) node [anchor=north] [inner sep=0.75pt]  [color={rgb, 255:red, 80; green, 227; blue, 194 }  ,opacity=1 ] [align=left] {suffix chain};
% Text Node
\draw (243.5,243.26) node [anchor=north] [inner sep=0.75pt]  [color={rgb, 255:red, 144; green, 19; blue, 254 }  ,opacity=1 ] [align=left] {complement clause};
% Text Node
\draw (347.5,273.26) node [anchor=north] [inner sep=0.75pt]  [color={rgb, 255:red, 144; green, 19; blue, 254 }  ,opacity=1 ] [align=left] {matrix clause};
% Text Node
\draw (90,320) node [anchor=north west][inner sep=0.75pt]   [align=left] {(c)};
% Text Node
\draw (287.5,360.26) node [anchor=north] [inner sep=0.75pt]  [color={rgb, 255:red, 80; green, 227; blue, 194 }  ,opacity=1 ] [align=left] {suffix chain};
% Text Node
\draw (497.5,360.26) node [anchor=north] [inner sep=0.75pt]  [color={rgb, 255:red, 80; green, 227; blue, 194 }  ,opacity=1 ] [align=left] {suffix chain};
% Text Node
\draw (402,360.26) node [anchor=north] [inner sep=0.75pt]  [color={rgb, 255:red, 144; green, 19; blue, 254 }  ,opacity=1 ] [align=left] {auxliary verb};
% Text Node
\draw (347.5,393.26) node [anchor=north] [inner sep=0.75pt]  [color={rgb, 255:red, 144; green, 19; blue, 254 }  ,opacity=1 ] [align=left] {clause};
% Text Node
\draw (189.5,182.76) node   [align=left] {main verb};
% Text Node
\draw (262.5,182.76) node   [align=left] {aux};
% Text Node
\draw (314.5,182.76) node   [align=left] {aux};
% Text Node
\draw (397.5,182.76) node   [align=left] {main verb};
% Text Node
\draw (470.5,182.76) node   [align=left] {aux};
% Text Node
\draw (522.5,182.76) node   [align=left] {aux};
% Text Node
\draw (189.5,330.76) node   [align=left] {main verb};
% Text Node
\draw (262.5,330.76) node   [align=left] {aux};
% Text Node
\draw (314.5,330.76) node   [align=left] {aux};
% Text Node
\draw (397.5,330.76) node   [align=left] {???};
% Text Node
\draw (470.5,330.76) node   [align=left] {aux};
% Text Node
\draw (522.5,330.76) node   [align=left] {aux};


\end{tikzpicture}

    \caption{Several schemes of verbal complexes (the spaces between elements indicate internal morphology). 
    (a) Templatic morphology (b) Complement clause (c) Periphrastic conjugation}
    \label{fig:verb-complex-scheme}
\end{figure}

Still, one may say the position of the auxiliary verb is also relatively rigid 
and therefore the auxiliary verb is also in the template of suffix chain,
though some late morphophonological processes somehow ``break'' the chain.
That's correct: after all, auxiliaries are realizations of the verbal functional hierarchy.
So probably the notion of auxiliary verb/suffix distinction is more of historical interest
(TODO: phonological word?).

\subsection{Organization this chapter}

This chapter is organized in a bottom-up manner.
First I talk about the internal morphology of verbal elements in the verbal complex,
that's to say, the internal morphology of verbs, adjectives and auxiliaries 
(\prettyref{sec:conjugation-class}).
Then I talk about the suffix and prefix chains,
with brief introduction of the grammatical categories marked by them
(\prettyref{sec:suffix-chain}, \prettyref{sec:prefixes}).
Auxiliary verbs, like the \corpus{iru} in the progressive aspect,
are \emph{not} discussed in \prettyref{sec:suffix-chain}:
the latter only deals with templatic auxiliaries, i.e. suffixes.
They -- together with grammatical constructions containing them -- 
are introduced in \prettyref{sec:periphrastic}.

Previous researches, mainly the system of the School Grammar and the Education Grammar,
often use seemingly incompatible though translatable terminologies 
to describe the verbal complex (\prettyref{sec:verb-complex-previous}).
Knowing both grammar systems is important for Japanese learners,
because (of course) dictionaries are edited by native speakers of Japanese
and they use the School Grammar system to carry out their work.
This note is a mixture of the two schools.

\begin{learnbox}{How to conjugate a verb}{how-to-conjugate}
    First have a look at \prettyref{sec:conjugation-class} to ,
    then 
\end{learnbox}

\section{Internal morphology}\label{sec:conjugation-class}

\subsection{Overview of conjugation classes}\label{sec:conjugation-class-overview}

Verbal internal morphology of a verbal element or 活用 
is decided by its position in the verbal complex 
and its conjugation class (活用形).
According to whether the final sound of the stem is a vowel or consonant,
regular Japanese verbs can be divided into \concept{c-stem verbs} and \concept{v-stem verbs}.
C-stem verbs are also called 五段動詞 or \concept{group-1 verbs},
because changing the conjugation ending means 
the final mora may appear in every column of the kana chart (\prettyref{tbl:hiragana-chart}), 
and v-stem verbs are also called 一段動詞 or \concept{group-2 verbs},
because the last or the second but last mora is always in the same column 
with the last mora in the dictionary form,
which is the present, positive, plain form of the verb (\prettyref{sec:tense-polarity-polite}).

一段動詞 can be further divided into 上一段動詞 or \corpus{iru}-verbs or \concept{group-2a verbs} 
and 下一段動詞 or \corpus{eru}-verbs or \concept{group-2b verbs}:
the final vowel of \corpus{iru}-verbs is \corpus{i},
and \corpus{-ru} is actually the conjugation ending of the terminal form,
and similarly the final vowel of \corpus{eru}-verbs is \corpus{e}.

There are two important irregular verbs: \corpus{suru} \translate{to do} 
and \corpus{kuru} \translate{to come}.
Since the verb category of Japanese is closed,
the two verbs are highly productive as light verbs when new verbal meanings are required.
They, together with their semi-conventionalized compound with \ac{np}s,
are collectively called \concept{group-3 verbs}.

Adjectives have their own conjugation pattern.

Japanese also has a few irregular verbs other than the group-3 verbs. TODO: more discussion

The internal morphology of each auxiliary verb also follows one of the patterns mentioned above,
and this is even true for morphosyntactically inflectional suffixes:
the latter are definitely suffixes in any morphosyntactic sense (TODO: ref),
but have the same internal morphological patterns with lexical verbs.
This makes the traditional School Grammar include them into the category of ``auxiliary verbs''
(\prettyref{sec:so-called-auxiliary-verb}).
It's rather weird to talk about ``stems'' of inflectional suffixes,
but for coherence of morphophonological description,
it seems necessary use the terms \term{c-stem} and \term{v-stem} for inflectional suffixes.
Some auxiliaries, like the negator \corpus{na-},
conjugate according to the paradigm of adjectives.

It should be noted that the concept of \term{stem} is important,
though at the first glance, the regular morphology of Japanese seems to 
mean you can find all forms of a verbal element from the dictionary form:
there are verbs ending in \corpus{iru} or \corpus{eru} but are c-stem verbs.
Also, the distinction between c-stem verbs and v-stem verbs is more purely phonological synchronically:
in some cases, the suffix has alternations that can't be attributed to contemporary phonology 
depending on the conjugation class of the verbal element before it
(\prettyref{sec:imperative-form}, TODO: more).

\subsection{The basic forms of c-stem and v-stem verbs}\label{sec:internal-forms}

In this section I list the basic forms of verbal internal morphology.
Not all of them are on the same plane.
The terminal form (\prettyref{sec:terminal-form}) is actually not a part of the internal morphology:
it's basically equivalent to the stem with a present tense marker \corpus{-(r)u},
and only appears at the end of the suffix chain,
though not necessarily the end of the verbal complex
(see the beginning of \prettyref{sec:indicative-paradigm}).
So is the case of the imperative form.

The irrealis form, the hypothetical form, 
the continuative form (\prettyref{sec:continuative-form}) TODO: list
are about internal morphology.
However, the irrealis form, the hypothetical form, TODO: list 
only appear before one suffix before the end of the suffix chain.
Only the continuative form appears everywhere in the suffix chain -- 
which is also what its name implies.
Also, it doesn't mean the irrealis form, the hypothetical form, etc. 
never appear at the end of the \emph{verbal complex}:
we can construct a \corpus{te}-form from the euphonic form 
and then use it to construct a progressive construction (\prettyref{sec:te-iru}). 

\begin{infobox}{Resources on the distribution of basic forms}{distribution-basic-form}
    See \citet[\citesec{3.2.4}]{gu2004} (and references in it) 
    for a comprehensive and kana-based description of 
    what can appear after each internal form.
\end{infobox}

\subsubsection{The irrealis form}\label{sec:irrealis-form}

The \concept{irrealis form} (未然形) can be obtained by using \corpus{-a} to the final of a c-stem,
and by doing nothing to a v-stem.
The irrealis form appears in some negative verbal complexes (\prettyref{sec:tense-polarity-polite})
as well as in the causative and passive constructions 
and the combination of the two (\prettyref{sec:valency-irrealis}). 

This narrow and semantically unmotivated distribution means 
the irrealis form seems to be a result of  
interaction between the verb stem and 
suffixes with \corpus{a} as the underlying starting vowel or with a starting mora \corpus{Ca}.
Relevant discussions can be found in the aforementioned sections.

\subsubsection{The continuative form}\label{sec:continuative-form}

The \concept{continuative form} is the default form for anything that conjugates
and is not at the end of a suffix chain (\prettyref{sec:suffix-chain}),
and hence the name.
Apart from that, the continuative form is also used in norminalization and non-finite clauses (TODO: ref).
The continuative form is also the historical origin of the euphonic form (\prettyref{sec:euphonic-form}).

\subsubsection{The terminal form}\label{sec:terminal-form}

The \concept{terminal form} (終止形) is obtained by attaching \corpus{-u} to a v-stem and \corpus{-ru} to a c-stem.
It appears clearly as a marker of the non-past tense (\prettyref{sec:tense-polarity-polite}),
and frequently appears at the end of the verbal complex, and hence the name;
expectedly, it's also the form appearing in dictionaries,
so we also call it the \concept{dictionary form} 辞书形.
The difference between the terminal forms of c-stem and v-stem verbs 
seems to be a result of contraction of successive consonants,
in which \corpus{C-ru} becomes \corpus{Cu} in the surface form
(\prettyref{sec:removing-rs}).

\subsubsection{The attributive form}

The attributive form, as its name suggests, appears mainly in relative clauses (TODO: ref).
Its form is exactly the same as the terminal form for regular verbs in modern Japanese,
though for adjectives there are differences (TODO: ref).

\subsubsection{The hypothetical form}

The \concept{hypothetical form} (仮定形) is obtained by adding \corpus{-e} to a c-stem and do nothing to a v-stem.
It appears in conditional clauses (TODO: ref).

\subsubsection{The imperative form}\label{sec:imperative-form}

The \concept{imperative form} (命令形) only appears in a ``real'' (i.e. not sugarcoated) command (TODO: ref: mood).
It's obtained by adding \corpus{-e} to the stem of c-stem verbs,
and \corpus{-ro} (or \corpus{-yo} as a written form) to v-stems.
It doesn't appear very frequently in public speaking, 
since as you can expect, 
it's rather rude.

\subsubsection{The volitional form}\label{sec:volition-form}

The \concept{volitional form} (意志形 or 推量形)
is obtained by adding \corpus{-\={o}} to a c-stem and \corpus{-y\={o}} to a v-stem.
It's historically the irrealis form,
and then sound changes happen.
Some books call it \term{the second irrealis form}.


\subsubsection{The euphonic form}\label{sec:euphonic-form}

The \concept{euphonic form} (音便形) is 
historically the continuative form when used together with any particle starting with \corpus{t-}
(\prettyref{sec:indicative-paradigm}, \prettyref{sec:te-form}, and TODO: one conditional form),
but due to the phonological rule of the contraction 
between the final consonant of the stem and the \corpus{-t-} sound (\prettyref{sec:t-euphonic}),
a new form is created. 

The creation of a new form doesn't summarize the consonant contraction rules.
After certain verbs, \corpus{-t-} becomes \corpus{-d-}.

\subsubsection{The potential form}\label{sec:potential-form}

The \concept{potential form} (可能形) is obtained by adding \corpus{-rare} to a v-stem 
and \corpus{-e} to a c-stem.
It's Historically related to the hypothetical form, TODO: how?
It's used within the suffix chain to mark the possibility of the event in question 
(\prettyref{sec:be-able-to-do}).

\subsection{The internal morphology of group-3 verbs}

\subsubsection{\corpus{Suru} \translate{do}}

The verb \corpus{suru} means \corpus{to do}. 

\subsubsection{\corpus{Kuru} \translate{come}}

The verb \corpus{kuru} (くる, 来る) means \translate{to come}.
It's internal morphology is shown in \prettyref{tbl:kuru-table}.

\begin{table}[H]
\caption{The internal morphology of \corpus{kuru}}
\label{tbl:kuru-table}
\centering
\begin{tabular}{llll}
\toprule
                & \multicolumn{3}{l}{internal morphological forms} \\ \midrule
Irrealis     & 来             & こ             & ko               \\
Continuative & 来             & き             & ki               \\
Terminal     & 来る            & くる            & kuru             \\
Attributive  & 来る            & くる            & kuru             \\
Hypothetical & 来れ            & くれ            & kure             \\
Imperative   & 来い            & こい            & koi              \\
Volitional   & 来よう           & こよう           & koyō             \\
Euphonic     & 来             & き             & ki               \\
Potential    & 来られる          & こられる          & korareru        \\ 
\bottomrule
\end{tabular}
\end{table}

\subsection{The ``to be'' verbs}

\subsubsection{\corpus{Da}}\label{sec:copula-da}

There is only one synchronically uncontroversial Japanese copula:
the irregular \corpus{da}.
Its mostly 

\subsection{Other irregular verbs}

\subsection{The internal morphology of adjectives}\label{sec:adjective-internal-form}

\subsubsection{The terminal form}\label{sec:adjective-terminal-form}

\section{The structure of the verbal complex}

\subsection{The suffix chain}\label{sec:suffix-chain}

The suffix chain can be roughly divided into an internal part (on the left side) 
and an external part (on the right side). 
The external part is mainly about 
the marking of tense, polarity and politeness,
shows certain degree of fusion,
and interacts strongly with external factors (\prettyref{sec:tense-polarity-polite}) 
The internal part is about valency changing and marking of modality,
the latter also involving certain degree of valency changing (\prettyref{sec:internal-suffix-chain}).

In most cases,
when an inflectional suffix is added,
if the previous form ends in a consonant,
an \corpus{i} is inserted after the consonant.
This results in the so-called continuative form (\prettyref{sec:continuative-form}).
However, the valency changing suffixes \corpus{-sase-} and \corpus{-rare-}
have their initial consonants removed when used together with a v-stem,
and therefore effectively make the preceding element in the verbal complex into the irrealis form 
(\prettyref{sec:valency-irrealis}).

The end of a suffix chain may be a complete external suffix chain 
(\prettyref{sec:tense-polarity-polite}).
But it's also possible to end a suffix chain with a partial (or null) external suffix chain,
and then to use it as a subordinate clause or as a part of a periphrastic conjugation.
In the latter cases, 
the last suffix in the suffix chain is also in the continuative form
(or the euphonic form).

\subsection{Prefixes}\label{sec:prefixes}

Prefixes to verbs and adjectives are rare in Japanese.
They are most likely to be honorifics prefixes or derivational prefixes.

\subsection{Periphrastic conjugation}\label{sec:periphrastic}

As is said in \prettyref{sec:auxliary-verb-suffix},
though in Japanese, 
due to the strong head-final tendency,
it's hard to differentiate
suffixes following a verb and auxiliaries following a verb,
by comparing relevant constructions with complement clause constructions, 
we can still draw a line between ordinary conjugations and periphrastic conjugations,
though this line is mainly of historical interest and bears little synchronic significance.
Japanese employs lots of periphrastic conjugations.
Actually, much of the agglutinative flavor of Japanese comes from periphrastic constructions,
while the purely suffixal part doesn't create incredibly long verbal complexes.
Most of aspectual marking (\prettyref{sec:aspect}), 
one cell in the indicative paradigm (\prettyref{sec:indicative-paradigm}), TODO 
are all periphrastic conjugations.

TODO: list of auxiliary verbs

TODO: \corpus{ni} particle at the end of verbal complex

\section{Stem and stem derivation}

\subsection{The transitive-intransitive correspondence}

TODO: verb derivation

\section{The internal suffix chain}\label{sec:internal-suffix-chain}

TODO: possible template: -sase-rare-nakaQ-u?

\subsection{The suffixes \corpus{-rare} and \corpus{-sase-}, and the irrealis form}\label{sec:valency-irrealis}

The passive suffix is \corpus{-rare-},
and the causative suffix is \corpus{-sase-}.
When applied upon c-stem verbs,
the \corpus{r} and \corpus{s} sounds are dropped (\prettyref{sec:removing-rs}),
so the passive of c-stem verbs as the irrealis form plus \corpus{-re-},
and similarly,
the causative of c-stem verbs is the irrealis form plus \corpus{-se-}.

It's also possible to apply the passive to the causative:
we get \translate{be made to do sth.}
In this case, the suffix is the expected 
\corpus{-saserare-},
called the passive causative.
Again, the \corpus{s} sound is dropped when used together with a c-stem verb.
There is no internal morphology on \corpus{-sase-} in this case,
because it's a v-stem auxiliary, but for consistency we may say
the \corpus{-sase-} is in the irrealis form,
because \corpus{-rare-} applies to the irrealis form of c-stem verbs 
and hence any v-stem verbs appearing before \corpus{-rare-} is also in the irrealis form.

\subsection{The potential \corpus{-rare-} or \corpus{-e-}}\label{sec:be-able-to-do}

Adding \corpus{-rare-} to c-stems and \corpus{-e-} to v-stems 
means the A argument has certain capacity or possibility to do something.
Essentially, this is to say to use the potential form (\prettyref{sec:potential-form}).

Note that it's not acceptable to use the potential \corpus{-rare-} and the passive \corpus{-rare-} together,
probably for phonological reasons: repetition sounds awkward \citep{kuno1978japanese}.
It's fine to just use a single \corpus{-rare-} to denote ``being able to be done''.

\subsection{The desiderative \corpus{-ta-}}

The suffix \corpus{-ta-} has the desiderative meaning.
Note that \corpus{-ta-} has adjectival internal morphology:
for example, its non-past plain form is \corpus{-ta-i}.

\subsection{The potential \corpus{-u-}}

The suffix \corpus{-u-} also serves as a potential marker.

\section{Moods, and marking of tense, polarity and politeness}\label{sec:tense-polarity-polite}

The marking of categories of tense, politeness and polarity 
already shows certain signs of fusion.
It's still worthwhile to analyze its inner structure, however.
In the \corpus{dar\={o}} form (\prettyref{sec:daro-form}),
politeness may be marked in a separate auxiliary following 
the tense, polarity and politeness complex of the main verb:
in this case, the former may or may not carry the ``polite'' feature
(with a \corpus{desh\={o}}/\corpus{dar\={o}} alternation),
while the latter is always in the plain form instead of the polite form.
In other cases -- essentially in other \emph{moods} -- the tense marker is forbidden to appear,
but the polarity and the politeness are well marked (TODO: te-form?).

\subsection{The indicative paradigm}\label{sec:indicative-paradigm}

This section introduces the complex of tense, polarity and politeness,
which appears in a fully developed way in the indicative mood
and is the last part of the verbal complex in the indicative mood
-- only \ac{sfp}s may follow them.
(But that's not to say the tense, polarity and politeness marking shown below
always appears at the end of the verbal complex at all times:
there is, for example, a periphrastic conjugation used in the presumptive mood, 
in which \corpus{dar\={o}} or \corpus{desho} is added after the tense, polarity and politeness marking.)
\prettyref{tbl:tense-polarity-politeness} shows the paradigm. 
Note again that words like ``terminal form'' indicate the internal morphology 
of the last verbal element, 
which may or may not be the head verb/adjective.
TODO: adjective

\begin{table}[H]
    \caption{The tense-polarity-politeness complex in the indicative mood}
    \label{tbl:tense-polarity-politeness}
    \begin{tabular}{llll}
        \toprule
        &                            & affirmative                                                       & negative                                                          \\
                              \midrule
    \multirow{2}{*}{non-past} & plain                      & terminal form                                                     & irrealis form plus \corpus{-nai}                                   \\
                              & polite                     & continuative form plus \corpus{-masu}                              & continuative form plus \corpus{-masen}                             \\ \midrule
    \multirow{2}{*}{past}     & \multicolumn{1}{l}{plain}  & \multicolumn{1}{l}{euphonic form plus \corpus{-ta} or \corpus{-da}} & \multicolumn{1}{l}{irrealis form plus \corpus{-nakatta}}           \\
                              & \multicolumn{1}{l}{polite} & \multicolumn{1}{l}{continuative form plus \corpus{-mashita}}       & \multicolumn{1}{l}{continuative form plus \corpus{-masen deshita}} \\
                              \bottomrule
    \end{tabular}
\end{table}

Some comments should be made regarding the inner structure of each cell of the paradigm.
It's easy to find the negative polite past form is a periphrastic one,
made by adding \corpus{deshita}, the the past polite form of the copula \corpus{da} 
(\prettyref{sec:copula-da}).
The past polite affirmative ending \corpus{-mashita} can be decomposed into 
\corpus{-[mas-i]-ta},
where \corpus{-ta} is the past tense marker,
which, as we see in \prettyref{tbl:tense-polarity-politeness},
is added to the euphonic form i.e. the continuative form when the stem ends in \corpus{-s}.
Comparing this with \corpus{-masu},
we find \corpus{-mas-} is a c-stem auxiliary,
and \corpus{-masu} can be composed into \corpus{-mas-u},
which is the terminal form of \corpus{-mas-}.
By comparing \corpus{-nai} and \corpus{-nakatta},
we find a negator \corpus{-na-} with adjectival internal morphology (\prettyref{sec:adjective-internal-form}),
so \corpus{-nai} is \corpus{-na-i} and \corpus{-nakatta} is \corpus{-nakaQ-ta}.
The ending \corpus{-masen} is a contraction of historical forms;
it's inner components aren't easily recognizable.

Thus, the endings mentioned in \prettyref{tbl:tense-polarity-politeness} can be glossed better as 
\prettyref{fig:analyzing-tpp-complex}.

\begin{figure}[H]
    \centering
    \begin{tikzpicture}[x=0.75pt,y=0.75pt,yscale=-1,xscale=1]
    %uncomment if require: \path (0,461); %set diagram left start at 0, and has height of 461
    
    %Straight Lines [id:da03896636779270812] 
    \draw    (209,169.59) -- (277.65,94.07) ;
    \draw [shift={(279,92.59)}, rotate = 132.27] [fill={rgb, 255:red, 0; green, 0; blue, 0 }  ][line width=0.08]  [draw opacity=0] (12,-3) -- (0,0) -- (12,3) -- cycle    ;
    %Straight Lines [id:da49291771335410184] 
    \draw    (229,181) -- (301,181.9) ;
    \draw [shift={(303,181.93)}, rotate = 180.72] [fill={rgb, 255:red, 0; green, 0; blue, 0 }  ][line width=0.08]  [draw opacity=0] (12,-3) -- (0,0) -- (12,3) -- cycle    ;
    %Straight Lines [id:da2285219783348671] 
    \draw    (324,156.93) -- (303.67,99.81) ;
    \draw [shift={(303,97.93)}, rotate = 70.41] [fill={rgb, 255:red, 0; green, 0; blue, 0 }  ][line width=0.08]  [draw opacity=0] (12,-3) -- (0,0) -- (12,3) -- cycle    ;
    %Straight Lines [id:da9602276837659014] 
    \draw    (204,194) -- (173.75,268.41) ;
    \draw [shift={(173,270.26)}, rotate = 292.12] [fill={rgb, 255:red, 0; green, 0; blue, 0 }  ][line width=0.08]  [draw opacity=0] (12,-3) -- (0,0) -- (12,3) -- cycle    ;
    %Straight Lines [id:da686002108813192] 
    \draw    (153,275) -- (119.74,191.78) ;
    \draw [shift={(119,189.93)}, rotate = 68.22] [fill={rgb, 255:red, 0; green, 0; blue, 0 }  ][line width=0.08]  [draw opacity=0] (12,-3) -- (0,0) -- (12,3) -- cycle    ;
    %Straight Lines [id:da10282744336315841] 
    \draw    (354,173.93) -- (403.02,167.84) ;
    \draw [shift={(405,167.59)}, rotate = 172.92] [fill={rgb, 255:red, 0; green, 0; blue, 0 }  ][line width=0.08]  [draw opacity=0] (12,-3) -- (0,0) -- (12,3) -- cycle    ;
    %Straight Lines [id:da8997891062176484] 
    \draw    (461,148.59) -- (489.68,116.09) ;
    \draw [shift={(491,114.59)}, rotate = 131.42] [fill={rgb, 255:red, 0; green, 0; blue, 0 }  ][line width=0.08]  [draw opacity=0] (12,-3) -- (0,0) -- (12,3) -- cycle    ;
    %Straight Lines [id:da9356897174621746] 
    \draw    (326,208) -- (317.45,244.65) ;
    \draw [shift={(317,246.59)}, rotate = 283.13] [fill={rgb, 255:red, 0; green, 0; blue, 0 }  ][line width=0.08]  [draw opacity=0] (12,-3) -- (0,0) -- (12,3) -- cycle    ;
    %Straight Lines [id:da7521493899687208] 
    \draw    (217,193.59) -- (281.54,254.22) ;
    \draw [shift={(283,255.59)}, rotate = 223.21] [fill={rgb, 255:red, 0; green, 0; blue, 0 }  ][line width=0.08]  [draw opacity=0] (12,-3) -- (0,0) -- (12,3) -- cycle    ;
    %Straight Lines [id:da12012679724572561] 
    \draw    (191,298.59) -- (274.01,288.83) ;
    \draw [shift={(276,288.59)}, rotate = 173.29] [fill={rgb, 255:red, 0; green, 0; blue, 0 }  ][line width=0.08]  [draw opacity=0] (12,-3) -- (0,0) -- (12,3) -- cycle    ;
    %Straight Lines [id:da813186309028515] 
    \draw    (350,272.59) -- (394.02,265.89) ;
    \draw [shift={(396,265.59)}, rotate = 171.35] [fill={rgb, 255:red, 0; green, 0; blue, 0 }  ][line width=0.08]  [draw opacity=0] (12,-3) -- (0,0) -- (12,3) -- cycle    ;
    %Straight Lines [id:da5810558850948748] 
    \draw    (489,97.59) -- (454.71,76.64) ;
    \draw [shift={(453,75.59)}, rotate = 31.43] [fill={rgb, 255:red, 0; green, 0; blue, 0 }  ][line width=0.08]  [draw opacity=0] (12,-3) -- (0,0) -- (12,3) -- cycle    ;
    
    % Text Node
    \draw (193,171.93) node [anchor=north west][inner sep=0.75pt]   [align=left] {stem};
    % Text Node
    \draw    (297, 71.5) circle [x radius= 25.71, y radius= 25.71]   ;
    \draw (276,63) node [anchor=north west][inner sep=0.75pt]   [align=left] {\mbox{-}ru/-u};
    % Text Node
    \draw    (426.5, 262.5) circle [x radius= 28.89, y radius= 28.89]   ;
    \draw (402,254) node [anchor=north west][inner sep=0.75pt]   [align=left] {\mbox{-}ta/-da};
    % Text Node
    \draw  [dash pattern={on 4.5pt off 4.5pt}]  (329, 181.5) circle [x radius= 23.94, y radius= 23.94]   ;
    \draw (310,173) node [anchor=north west][inner sep=0.75pt]   [align=left] {\mbox{-}mas-};
    % Text Node
    \draw  [dash pattern={on 4.5pt off 4.5pt}]  (310, 279.5) circle [x radius= 34, y radius= 34]   ;
    \draw (280,271) node [anchor=north west][inner sep=0.75pt]   [align=left] {euphonic};
    % Text Node
    \draw  [dash pattern={on 4.5pt off 4.5pt}]  (170, 290.5) circle [x radius= 19.7, y radius= 19.7]   ;
    \draw (156,282) node [anchor=north west][inner sep=0.75pt]   [align=left] {\mbox{-}na-};
    % Text Node
    \draw    (114.5, 175.5) circle [x radius= 13.73, y radius= 13.73]   ;
    \draw (109,167) node [anchor=north west][inner sep=0.75pt]   [align=left] {\mbox{-}i};
    % Text Node
    \draw    (504.5, 100.5) circle [x radius= 14.92, y radius= 14.92]   ;
    \draw (497,92) node [anchor=north west][inner sep=0.75pt]   [align=left] {\mbox{-}n};
    % Text Node
    \draw    (424.5, 60.5) circle [x radius= 28.89, y radius= 28.89]   ;
    \draw (400,52) node [anchor=north west][inner sep=0.75pt]   [align=left] {deshita};
    % Text Node
    \draw  [dash pattern={on 4.5pt off 4.5pt}]  (433, 165.5) circle [x radius= 27.52, y radius= 27.52]   ;
    \draw (410,157) node [anchor=north west][inner sep=0.75pt]   [align=left] {irrealis};
    
    
    \end{tikzpicture}
    
    \caption{The inner structure of the tense-polarity-politeness complex}
    \label{fig:analyzing-tpp-complex}
\end{figure}

\subsection{The \corpus{dar\={o}} (\corpus{desh\={o}}) form}\label{sec:daro-form}

There is no single presumptive mood in Japanese.
The \corpus{dar\={o}} form is a presumptive construction.

\section{Aspects}\label{sec:aspect}

\subsection{The lexical aspect}\label{sec:lexical-aspect}

https://www.japanesewithanime.com/2020/08/lexical-aspect.html

Verbs with different lexical aspect yield different meanings 
when placed in seemingly similar constructions.
In Japanese, there are roughly four kind of verbs with regard to lexical aspects:
continuous verbs, momentary verbs, stative verbs and adjectival or ``type-4'' verbs.
Stative verbs never appear in the \corpus{-te iru} construction (\prettyref{sec:te-iru}),
while adjectival verbs never appear not being in the construction.
And there are differences in the meaning of the \corpus{-te iru} construction 
between the continuous verbs and momentary verbs.
This four-fold classification is robust with several tests \citep[\citesec{3.1}]{gu2004},
shown in \prettyref{tbl:lexical-aspect}.

\begin{table}[H]
    \centering
    \caption{The four lexical aspects and their distributions and semantics}
    \label{tbl:lexical-aspect}
    \begin{tabular}{lcccc}
        \toprule
                     & \multicolumn{2}{c}{event} & \multicolumn{2}{c}{state} \\
                     & continuous    & momentary & stative    & adjectival   \\ \midrule
    \corpus{-te iru} & imperfective  & perfect   & -          & (always in this construction)       \\ 
    imperative       & +             &           & -          & -             \\  \bottomrule
    \end{tabular}
\end{table}

\subsection{The \corpus{-te iru} construction}\label{sec:te-iru}


\section{Non-finite forms and norminalization}

\subsection{The \corpus{te}-form or the so-called gerund}\label{sec:te-form}

The \corpus{te}-form is the euphonic form plus the particle \corpus{-te},
or in a more traditional account,
the continuative form plus \corpus{-te} plus 音便 rules (\prettyref{sec:t-euphonic}).
It's sometimes called the gerund,
though the name is kind of misleading,
because the term \term{gerund} hints at the possibility of a non-finite clause to fill argument positions,
while \corpus{te}-form is usually used to fill \emph{adverbial} positions.

\section{Notes about previous studies}\label{sec:verb-complex-previous}

\subsection{The School Grammar and ``auxiliary verbs''}\label{sec:so-called-auxiliary-verb}

Here is a little terminological confusion:
in the School Grammar inflectional suffixes,
sometimes also auxiliary verbs, are called 助動詞 \translate{auxiliary verb},
while auxiliary verbs are called 補助動詞 \translate{helping verbs}.
The confusion of the two is easily understood,
because both of them appear after the main verb
and are attached closely to the main verb 
and have internal morphology (\prettyref{sec:conjugation-class-overview}).
The fact that the accepted writing system for Japanese doesn't distinguish words 
(\prettyref{sec:writing-symbols}) -- whatever this term means -- 
also contributes to native speakers' decision to call suffixes (and sometimes auxiliary verbs) ``auxiliary verbs'',
and then they have to invent another name for more prototypical auxiliary verbs.

\prettyref{sec:internal-forms} is written in a School Grammar view.
The difference between \prettyref{sec:internal-forms} and traditional School Grammar 
is the euphonic form is included,
while traditional School Grammar rejects its status as a single form
and use phonological rules to cover it (TODO: ref).
Though this is also an acceptable analysis,
I find it inconvenient practically and problematic theoretically:
other ``forms'' of verbal internal morphology also seem to
be derivable from regular morphophonological rules,
and rejecting one of them is equivalent to rejecting all of them,
if we are to be very self-consistent.
So I include this form anyway.

\subsection{The Education Grammar}

The sections after \prettyref{sec:conjugation-class} are written in the spirit of the Education Grammar.
So-called ``forms'' in the Education Grammar are actually \emph{processes}.
Thus, the theory of verb conjugation in the Education Grammar 
is actually a mixture of the Item-and-Process approach and the Word-and-Paradigm approach.
The definition of each form is 
``removing the final kana of \dots and attaching \dots to its end''
(for example, instead of saying ``the \corpus{te}-form is traditionally built by 
the continuative form plus $\corpus{te}$'',
an Education Grammar book may use the following formulation:
``the \corpus{te}-form is obtained by removing \corpus{-masu} from the end of the verb 
and then add \corpus{-te}''),
which can be seen as a rule relating two cells in the conjugation paradigm.
The difference between the standard Word-and-Paradigm approach and the Education Grammar 
is the so-called ``forms'' can be applied for more than once:
we may talk about ``the $x$ form of the $y$ form of $z$''
and not just ``the $y$ form of $z$'',
so this is a typical Item-and-Process description.
By doing so, the School Grammar avoids 
the notion of internal morphology for each element in the verbal complex.
The price is advanced learners have to spend much more time to memorize the rules of all the different forms 
-- essentially internal conjugation of the last element plus one or more auxiliaries attached,
and the internal morphology ending may also be seen as a part of the auxiliaries 
(\prettyref{box:affix-conjugation-theory}).

\chapter{Arguments of verbs}\label{chap:arguments}

\section{\translate{I hit the wall \dots}: prototypical transitive and intransitive verbs}

\section{\translate{I think, I feel}: }

\chapter{Clausal constituent order and information packaging}

\section{Topic and subject}\label{sec:topic-subject}

The difference between the so-called topic marker \corpus{wa} and the subject marker \corpus{ga}
is a long problem in Japanese grammar.
This section provides a tentative summary of the function of each.

\subsection{The function of \corpus{wa}}\label{sec:wa-topic}

The meaning of \ac{np}-\corpus{wa} can be summarized as 
\translate{as for \ac{np}, I know \dots (the rest I'm not talking about)}.
Thus \corpus{wa} is a canonical marker of topic,
and the \ac{np} it's attached to must be somehow a ``known object''.

When two clauses containing \corpus{wa} are conjoined together,
we get the meaning of 
\translate{as for \dots, I know \dots; (but) as for \dots, \dots},
and this naturally has a contrastive meaning, as in (TODO).
Note that though it's possible that the \corpus{wa}-\ac{np}s in the two clauses are different,
because the above construction is contrasting two clauses, not two \ac{np}s.

\begin{exe}
    \ex TODO: example of contrastive  
\end{exe}

\subsection{The function of nominative \corpus{ga}, with or without \corpus{wa}} 

\subsection{Multiple \corpus{ga}}\label{sec:multiple-ga}

\subsection{Summary of constructions}\label{sec:wa-ga-template}

\section{Sentence final particles}\label{sec:sfp}

\ac{sfp}s are useful in oral communication.

\begin{infobox}{Resources on \ac{sfp}s}{sfp-source}
    This section is based on \citet[\citesec{6.4}]{akiyama2012japanese}.
\end{infobox}

\subsection{\corpus{Ka}}

The \ac{sfp} \corpus{ka} is the interrogative marker.
It marks both open and closed questions.
To make a question open,
just introduce interrogative pro-forms.

\subsection{\corpus{Ne}}

The \ac{sfp} \corpus{ne} invites the listener to confirm a claim:
\translate{\dots, don't you think so?}
We have the following examples:
\begin{exe}
    \ex \gll Atsui desu ne  \\
    hot {} NE  \\
    \glt \translate{It's hot, isn't it?}
\end{exe}

\subsection{\corpus{Yo}}

The \ac{sfp} \corpus{yo} is used to make a very strong assertion.

\chapter{Subordination and coordination}

As is often the case, there are fewer information packaging devices in subordinated clauses.
TODO: topicalization, 

\begin{theorybox}{Matrix clauses and subordinated clauses}{matrix-subordinated}
    From a generative perspective, both a matrix clause 
    (which is actually a sentence in the sense of \acs{blt}, 
    though it can still be embedded as an argument of a verb like \translate{say})
    and a subordinated clause are CPs.
    The structural differences between the two 
    are mainly about the lack of certain functional projections in the subordinated clause.
    Speaking of so-called Force projection(s),
    the subordinated clause usually only contains the marking of 
    imperative, declarative, etc.,
    without specification of detailed speech act,
    which may be marked by \ac{sfp}s in Japanese (and Chinese).
    Since topicalization is also a CP process,
    we can expect it's more restricted in subordinated clauses.
\end{theorybox}

\section{Adverbial clauses}

\subsection{Conditional clauses}

\section{Clauses linked by particles}\label{sec:particle-linking-clause}

TODO: Compatibility with \ac{sfp}s
This section is a more annotated version of \citet[\citesec{6.3}]{akiyama2012japanese}.

\subsection{\corpus{Ga}}

\subsubsection{The concessive construction}

The particle \corpus{ga} may be attached to the end of a clause 
and has the reading of \translate{despite the circumstance}.

A conventionalized construction in oral speech is to avoid the main clause 
and end the utterance with \corpus{ga}.
In this way, it's a marginal \ac{sfp}.

\begin{exe}
    \ex Ikitakatta desu ga \dots
\end{exe}

\subsubsection{Neutral linking}

We can also just use \corpus{ga} as \translate{and}.

\subsection{\corpus{Kara}}

\subsubsection{The causal construction}

\bibliographystyle{plainnat}
\bibliography{grammars,details,../methodology/famous-grammars,theory}

\end{document}