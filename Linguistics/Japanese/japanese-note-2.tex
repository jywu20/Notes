\documentclass[UTF8, a4paper, oneside, scheme=plain]{ctexrep}

\usepackage{geometry}
\usepackage{float}
\usepackage{titling}
\usepackage{titlesec}
\usepackage{paralist}
\usepackage{footnote}
\usepackage{enumerate}
\usepackage{amsmath, amssymb, amsthm}
\usepackage{gb4e}
\noautomath
\usepackage{bbm}
\usepackage{soul}
\usepackage{graphicx}
\usepackage{siunitx}
\usepackage[table,xcdraw]{xcolor}
\usepackage{tikz}
\usepackage[ruled, vlined, linesnumbered, noend]{algorithm2e}
\usepackage{xr-hyper}
\usepackage[colorlinks]{hyperref} % linkcolor=black, anchorcolor=black, citecolor=black, filecolor=black
\usepackage[most]{tcolorbox}
\usepackage{caption}
\usepackage{subcaption}
\usepackage{booktabs}
\usepackage{multirow}
\usepackage[figuresright]{rotating}
\usepackage{acro}
\usepackage[round]{natbib} 
\usepackage{nameref,zref-xr}
\zxrsetup{toltxlabel}
\zexternaldocument*[cgel-]{../English/cambridge}[cambridge.pdf]
\zexternaldocument*[alignment-]{../alignment/alignment}[alignment.pdf]
\zexternaldocument*[exercise1-]{../Exercise/2021-3}[2021-3.pdf]
\zexternaldocument*[method-]{../methodology/glossing}[glossing.pdf]
\usepackage{prettyref}

\geometry{left=3.18cm,right=3.18cm,top=2.54cm,bottom=2.54cm}
\titlespacing{\paragraph}{0pt}{1pt}{10pt}[20pt]
\setlength{\droptitle}{-5em}

\DeclareMathOperator{\timeorder}{\mathcal{T}}
\DeclareMathOperator{\diag}{diag}
\DeclareMathOperator{\legpoly}{P}
\DeclareMathOperator{\primevalue}{P}
\DeclareMathOperator{\sgn}{sgn}
\newcommand*{\ii}{\mathrm{i}}
\newcommand*{\ee}{\mathrm{e}}
\newcommand*{\const}{\mathrm{const}}
\newcommand*{\suchthat}{\quad \text{s.t.} \quad}
\newcommand*{\argmin}{\arg\min}
\newcommand*{\argmax}{\arg\max}
\newcommand*{\normalorder}[1]{: #1 :}
\newcommand*{\pair}[1]{\langle #1 \rangle}
\newcommand*{\fd}[1]{\mathcal{D} #1}

\newcommand*{\citesec}[1]{\S~{#1}}
\newcommand*{\citechap}[1]{chap.~{#1}}
\newcommand*{\citefig}[1]{Fig.~{#1}}
\newcommand*{\citetable}[1]{Table~{#1}}
\newcommand*{\citepage}[1]{pp.~{#1}}
\newcommand*{\citefootnote}[1]{fn.~{#1}}

\newrefformat{sec}{\citesec{\ref{#1}}}
\newrefformat{fig}{\citefig{\ref{#1}}}
\newrefformat{tbl}{\citetable{\ref{#1}}}
\newrefformat{chap}{\citechap{\ref{#1}}}
\newrefformat{fn}{\citefootnote{\ref{#1}}}

\usetikzlibrary{arrows,shapes,positioning}
\usetikzlibrary{arrows.meta}
\usetikzlibrary{decorations.markings}
\tikzstyle arrowstyle=[scale=1]
\tikzstyle directed=[postaction={decorate,decoration={markings,
    mark=at position .5 with {\arrow[arrowstyle]{stealth}}}}]
\tikzstyle ray=[directed, thick]
\tikzstyle dot=[anchor=base,fill,circle,inner sep=1pt]


\tcbuselibrary{skins, breakable, theorems}

\newtcbtheorem[number within=chapter]{infobox}{Box}%
{colback=blue!5,colframe=blue!65,fonttitle=\bfseries, breakable}{infobox}
\newtcbtheorem[number within=chapter]{theorybox}{Theoretical aspect}%
{colback=orange!5, colframe=orange!65, fonttitle=\bfseries, breakable}{theorybox}
\newtcbtheorem[number within=chapter]{learnbox}{Learning note}%
{colback=green!5,colframe=green!65,fonttitle=\bfseries, breakable}{learnbox}

\newcommand*{\concept}[1]{\textbf{#1}}
\newcommand*{\term}[1]{\emph{#1}}
\newcommand{\corpus}[1]{\emph{#1}}

\DeclareAcronym{blt}{short = BLT, long = Basic Linguistic Theory}
\DeclareAcronym{cgel}{short = CGEL, long = The Cambridge Grammar of the English Language}
\DeclareAcronym{dm}{short = DM, long = Distributed Morphology}
\DeclareAcronym{tag}{long = Tree-adjoining grammar, short = TAG}
\DeclareAcronym{sfp}{long = sentence final particle, short = SFP}
\DeclareAcronym{np}{long = noun phrase, short = NP}
\DeclareAcronym{vp}{long = verb phrase, short = VP}
\DeclareAcronym{pp}{long = preposition phrase, short = PP}
\DeclareAcronym{cls}{long = classifier, short = CLS}
\DeclareAcronym{dist}{long = distal, short = DIST}
\DeclareAcronym{prox}{long = proximate, short = PROX}
\DeclareAcronym{dem}{long = demonstrative, short = DEM}
\DeclareAcronym{dur}{long = durative, short = DUR}
\DeclareAcronym{neg}{long = negative, short = NEG}
\DeclareAcronym{cc}{long = copular complement, short = CC}
\DeclareAcronym{cs}{long = copular subject, short = CS}
\DeclareAcronym{tam}{long = {tense, aspect, mood}, short = TAM}

\newcommand*{\homo}[2]{#1$_{\text{#2}}$}

\newcommand{\cgel}{\href{../English/cambridge.pdf}{my notes about CGEL}}
\newcommand{\latin}{\href{../Latin/latin-notes.pdf}{my notes about Latin}}
\newcommand{\alignment}{\href{../alignment/alignment.pdf}{my notes about alignment}}
\newcommand{\exerciseone}{\href{../Exercise/2021-3.pdf}{this exercise}}
\newcommand{\method}{\href{../methodology/glossing.pdf}{this note about how descriptive grammars work}}

\newcommand{\ala}{à la}
\newcommand{\translate}[1]{`#1'}

% Make subsubsection labeled
\setcounter{secnumdepth}{3}
% reset example counter every chapter (but do not include the chapter number to the label)
\counterwithin{exx}{chapter} 

\title{Japanese grammar notes}
\author{Jinyuan Wu}

\begin{document}

\maketitle

This note is a more well-organized version of \href{./japanese-note-1.pdf}{this note}.
It's a reading note of \citet{akiyama2012japanese}, \citet{tsutsui1989dictionary},
as well as lots of books and articles listed in the reference.
The methodology followed is in \method,
i.e. ``largely generatively informed but surface-oriented and flat-tree in the appearance''.

\chapter{Introduction}

\section{The Japanese language and its history}

\section{Previous studies}

Japanese is a relatively well-documented language,
with a native grammar study tradition.

\section{Language and culture}

\chapter{Overview of Japanese grammar}

\section{Introduction}

Japanese has a strict modifier-first constituent order,
and here the term \term{modifier} includes 
arguments in a clause (``modifiers of the verb or the verbal adjective''),
and even \acs{np}s with respect to case particles.

\begin{theorybox}{the notion of head and modifier}{head-modifier}
    This notion of head and modifier is \acs{cgel}-like, 
    and is probably related to a strong head-final tendency in the linearization:
    if a so-called modifier is introduced as a specifier 
    in a functional projection with a root as the core,
    then obviously the root and the functional heads are realized into one unit (for example a verb complex) 
    and the ``modifier'' precedes the unit to ensure the (functional) head-final rule,
    and therefore in the surface-oriented analysis,
    we also get a modifier-head constituent order (where \term{head} means lexical heads).
    If there is no core root,
    then trivially the ``modifier'' is realized in a position before the spellout of functional heads,
    and the latter is regarded as somehow a head in the \acs{cgel} sense,
    and again we get a modifier-head constituent order,
    if we understand things like case particles as heads,
    which is the \ac{cgel} approach but not the \acs{blt} approach.
\end{theorybox}

\section{Phonology and the writing system}

\section{Parts of speech}

\subsection{Lexical words: nouns and verbs, and everything else}

Japanese has a clear noun-verb distinction.
This can be found by looking at the morphology:
nouns are subject to case marking,
which is basically adding a particle to the \acs{np} 
which can be dropped especially in casual speech,
while verbs always appear as one of the stem forms plus agglutinative endings.

There are two adjective classes:
the verbal adjectives (or \corpus{i}-adjectives)
and the nominal adjectives (or \corpus{na}-adjectives),
with different syntactic distribution 
(verbal adjectives may fill the predicate slot on their own; nominal adjectives never do so)
and morphological appearances
(verbal adjectives are more like verbs).

One rare property of modern Japanese is the verb class and the verbal adjective class 
are already closed classes:
they rarely accept new members (though not entirely impossible).
What makes Japanese rarer is despite being closed,
the verbal adjective class is large.

\subsection{Function items}

I specifically use the term \term{fucntion items} instead of \term{function words}
in the title of this section,
because the word-or-morpheme-or-phrase problem is especially serious in Japanese morphosyntax
(TODO: ref: school grammar, education grammar).
Function items in Japanese include particles, auxiliary verbs, TODO 

\begin{theorybox}{So-called category of functional items}{function-category}
    Though particles are in the grammar and do not really carry category labels like ``noun'' or ``verb''
    and it actually makes no sense to discuss the categories of them,
    the traditional practice to list all particles and classify them 
    is practically desirable, 
    as it provides a quick way to navigate across grammatical systems.
\end{theorybox}

Japanese lacks the prototypically pronoun class:
so-called pronouns are customized referential nouns like \translate{that girl},
and thus the pronoun class is not closed and strictly speaking is not a part of the grammar.
The article class is also not attested.

\section{The structure of this note}

\begin{theorybox}{The organization in reference grammars}{grammar-org}
    The structure of this note and the contents of chapters follow 
    the examples set by \citet{Friesen2017}, \citet{jacques2021grammar}, \citet{Grimm2021},
    the famous \acs{cgel} \citep{cgel}, and of course Dixon's three volumes of \acs{blt}.
    The nominal chapters (TODO: ref), 
    especially \prettyref{chap:case-particle}, 
    are organized in the same way as \citet{jacques2021grammar}.
    The notion of verb complex (\prettyref{chap:verb-complex}) is also found in 
    \citet{Friesen2017}.
\end{theorybox}

\chapter{Pronouns}

\chapter{Numerals}

\chapter{Case particles}\label{chap:case-particle}

\chapter{The structure of noun phrases}

\chapter{The verb complex}\label{chap:verb-complex}

\section{Introduction}

As is said before, Japanese is strongly modifier-first,
and hence productive functional morphemes in the verb complex are always suffixes.
Japanese is typologically agglutinative:
the morphemes have relatively clear boundaries,
each morpheme representing a grammatical category.
Still, there are two important factors in Japanese that deviate away 
from the perfect agglutinative prototype.
The first is most morpheme in the verb complex 
-- both the lexical head and most functional morphemes following it --
have \emph{internal} morphology.
The second is there is still some degree of fusion, 
in which historically analyzable morphemes seem to already form a single fused morpheme 
(TODO: tense-neg complex).

Previous researches, mainly the system of the School Grammar and the Education Grammar,
often use mutually incompatible terminologies to describe the verb complex.
TODO: ref

\section{The template of the suffix chain}

\begin{exe}
    \ex TODO: examples of Japanese verb complex
\end{exe}

\chapter{Arguments of verbs}



\chapter{Clausal constituent order and information packaging}

\section{Topic and subject}

The difference between the so-called topic marker \corpus{wa} and the subject marker \corpus{ga}
is a long problem in Japanese grammar.

\begin{exe}
    \ex TODO: example of Japanese clause
\end{exe}

\chapter{Subordination}

\bibliographystyle{plainnat}
\bibliography{grammars,details,../methodology/famous-grammars,theory}

\end{document}