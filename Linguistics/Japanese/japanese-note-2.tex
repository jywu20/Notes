\documentclass[UTF8, a4paper, oneside, scheme=plain]{ctexrep}

\usepackage{geometry}
\usepackage{float}
\usepackage{titling}
\usepackage{titlesec}
\usepackage{paralist}
\usepackage{footnote}
\usepackage{enumerate}
\usepackage{amsmath, amssymb, amsthm}
\usepackage{gb4e}
\noautomath
\usepackage{bbm}
\usepackage{soul}
\usepackage{graphicx}
\usepackage{siunitx}
\usepackage[table,xcdraw]{xcolor}
\usepackage{tikz}
\usepackage[ruled, vlined, linesnumbered, noend]{algorithm2e}
\usepackage{xr-hyper}
\usepackage[colorlinks]{hyperref} % linkcolor=black, anchorcolor=black, citecolor=black, filecolor=black
\usepackage[most]{tcolorbox}
\usepackage{caption}
\usepackage{subcaption}
\usepackage{booktabs}
\usepackage{multirow}
\usepackage[figuresright]{rotating}
\usepackage{acro}
\usepackage[round]{natbib} 
\usepackage{nameref,zref-xr}
\zxrsetup{toltxlabel}
\zexternaldocument*[cgel-]{../English/cambridge}[cambridge.pdf]
\zexternaldocument*[alignment-]{../alignment/alignment}[alignment.pdf]
\zexternaldocument*[exercise1-]{../Exercise/2021-3}[2021-3.pdf]
\zexternaldocument*[method-]{../methodology/glossing}[glossing.pdf]
\usepackage{prettyref}

\geometry{left=3.18cm,right=3.18cm,top=2.54cm,bottom=2.54cm}
\titlespacing{\paragraph}{0pt}{1pt}{10pt}[20pt]
\setlength{\droptitle}{-5em}

\DeclareMathOperator{\timeorder}{\mathcal{T}}
\DeclareMathOperator{\diag}{diag}
\DeclareMathOperator{\legpoly}{P}
\DeclareMathOperator{\primevalue}{P}
\DeclareMathOperator{\sgn}{sgn}
\newcommand*{\ii}{\mathrm{i}}
\newcommand*{\ee}{\mathrm{e}}
\newcommand*{\const}{\mathrm{const}}
\newcommand*{\suchthat}{\quad \text{s.t.} \quad}
\newcommand*{\argmin}{\arg\min}
\newcommand*{\argmax}{\arg\max}
\newcommand*{\normalorder}[1]{: #1 :}
\newcommand*{\pair}[1]{\langle #1 \rangle}
\newcommand*{\fd}[1]{\mathcal{D} #1}

\newcommand*{\citesec}[1]{\S~{#1}}
\newcommand*{\citechap}[1]{chap.~{#1}}
\newcommand*{\citefig}[1]{Fig.~{#1}}
\newcommand*{\citetable}[1]{Table~{#1}}
\newcommand*{\citepage}[1]{pp.~{#1}}
\newcommand*{\citefootnote}[1]{fn.~{#1}}

\newrefformat{sec}{\citesec{\ref{#1}}}
\newrefformat{fig}{\citefig{\ref{#1}}}
\newrefformat{tbl}{\citetable{\ref{#1}}}
\newrefformat{chap}{\citechap{\ref{#1}}}
\newrefformat{fn}{\citefootnote{\ref{#1}}}

\usetikzlibrary{arrows,shapes,positioning}
\usetikzlibrary{arrows.meta}
\usetikzlibrary{decorations.markings}
\tikzstyle arrowstyle=[scale=1]
\tikzstyle directed=[postaction={decorate,decoration={markings,
    mark=at position .5 with {\arrow[arrowstyle]{stealth}}}}]
\tikzstyle ray=[directed, thick]
\tikzstyle dot=[anchor=base,fill,circle,inner sep=1pt]


\tcbuselibrary{skins, breakable, theorems}

\newtcbtheorem[number within=chapter]{infobox}{Box}%
{colback=blue!5,colframe=blue!65,fonttitle=\bfseries, breakable}{infobox}
\newtcbtheorem[number within=chapter]{theorybox}{Theoretical aspect}%
{colback=orange!5, colframe=orange!65, fonttitle=\bfseries, breakable}{theorybox}
\newtcbtheorem[number within=chapter]{learnbox}{Learning note}%
{colback=green!5,colframe=green!65,fonttitle=\bfseries, breakable}{learnbox}

\newcommand*{\concept}[1]{\textbf{#1}}
\newcommand*{\term}[1]{\emph{#1}}
\newcommand{\corpus}[1]{\emph{#1}}

\DeclareAcronym{blt}{short = BLT, long = Basic Linguistic Theory}
\DeclareAcronym{cgel}{short = CGEL, long = The Cambridge Grammar of the English Language}
\DeclareAcronym{dm}{short = DM, long = Distributed Morphology}
\DeclareAcronym{tag}{long = Tree-adjoining grammar, short = TAG}
\DeclareAcronym{sfp}{long = sentence final particle, short = SFP}
\DeclareAcronym{np}{long = noun phrase, short = NP}
\DeclareAcronym{vp}{long = verb phrase, short = VP}
\DeclareAcronym{pp}{long = preposition phrase, short = PP}
\DeclareAcronym{cls}{long = classifier, short = CLS}
\DeclareAcronym{dist}{long = distal, short = DIST}
\DeclareAcronym{prox}{long = proximate, short = PROX}
\DeclareAcronym{dem}{long = demonstrative, short = DEM}
\DeclareAcronym{dur}{long = durative, short = DUR}
\DeclareAcronym{neg}{long = negative, short = NEG}
\DeclareAcronym{cc}{long = copular complement, short = CC}
\DeclareAcronym{cs}{long = copular subject, short = CS}
\DeclareAcronym{tam}{long = {tense, aspect, mood}, short = TAM}

\newcommand*{\homo}[2]{#1$_{\text{#2}}$}

\newcommand{\cgel}{\href{../English/cambridge.pdf}{my notes about CGEL}}
\newcommand{\latin}{\href{../Latin/latin-notes.pdf}{my notes about Latin}}
\newcommand{\alignment}{\href{../alignment/alignment.pdf}{my notes about alignment}}
\newcommand{\exerciseone}{\href{../Exercise/2021-3.pdf}{this exercise}}
\newcommand{\method}{\href{../methodology/glossing.pdf}{this note about how descriptive grammars work}}

\newcommand{\ala}{à la}
\newcommand{\translate}[1]{`#1'}

% Make subsubsection labeled
\setcounter{secnumdepth}{4}
\setcounter{tocdepth}{4}
% reset example counter every chapter (but do not include the chapter number to the label)
\counterwithin{exx}{chapter} 

\title{Japanese grammar notes}
\author{Jinyuan Wu}

\begin{document}

\maketitle

This note is a more well-organized version of \href{./japanese-note-1.pdf}{this note}.
It's a reading note of \citet{akiyama2012japanese}, \citet{tsutsui1989dictionary},
as well as lots of books and articles listed in the reference.
The methodology followed is in \method,
i.e. ``largely generatively informed but surface-oriented and flat-tree in the appearance''.

There is basically nothing new in the note. 
Sometimes you will find word-by-word copying of the books and papers in the reference.

\chapter{Introduction}

\section{The Japanese language and its history}

\section{Previous studies}

Japanese is a relatively well-documented language,
with a native grammar study tradition.

\section{Language and culture}

\chapter{Overview of Japanese grammar}


\section{Phonology and the writing system}

Japanese is roughly a mora-timing language,
though with some deviation from the prototypical ones.
Japanese has a rather trivial inventory of 5 vowel phonemes (\prettyref{sec:vowel}) 
and 15 consonant phonemes (\prettyref{sec:consonant}),
which can be combined into around 50 moras,
and there are three types of special moras -- 
the syllable-final nasal, vowel lengthening, and consonant germination 
(\prettyref{sec:mora-scheme}).

The above features results in a native writing system,
which includes two sets of mora-based syllabaries called kanas (\prettyref{sec:kana})
and Japanese Chinese characters, the kanji (\prettyref{sec:kanji}).
The spelling system used in this note is a romanization system (\prettyref{sec:romanization}).

\section{Parts of speech}

\subsection{Lexical words: nouns and verbs, and everything else}

Japanese has a clear noun-verb distinction.
This can be found by looking at the morphology:
nouns are subject to case marking,
which is basically adding a particle to the \acs{np} 
which can be dropped especially in casual speech,
while verbs always appear as one of the stem forms plus agglutinative endings.

The following things should be recorded about a verb:
its conjugation class (\prettyref{sec:conjugation-class}),
the argument structure (\prettyref{chap:arguments}),
which may also affect how the arguments are marked,
TODO: unaccusative and unergative

There are two adjective classes:
the verbal adjectives (or \corpus{i}-adjectives)
and the nominal adjectives (or \corpus{na}-adjectives),
with different syntactic distribution 
(verbal adjectives may fill the predicate slot on their own; nominal adjectives never do so)
and morphological appearances
(verbal adjectives are more like verbs).

One rare property of modern Japanese is the verb class and the verbal adjective class 
are already closed classes:
they rarely accept new members (though not entirely impossible).
What makes Japanese rarer is despite being closed,
the verbal adjective class is large.

The vocabulary of Japanese can be divided into three parts according to the etymology:
the native words,
Sino-Japanese words,
and recently borrowed words.
This distinction is sometimes of grammatical significance (TODO: politeness).

\subsection{Function items}

I specifically use the term \term{fucntion items} instead of \term{function words}
in the title of this section,
because the word-or-morpheme-or-phrase problem is especially serious in Japanese morphosyntax
(TODO: ref: school grammar, education grammar).
In functional items,
we have particles in \acs{np}s (\prettyref{chap:particle-in-np}),
\acs{sfp}s in matrix clauses (\prettyref{sec:sfp}),
particles used for clause linking (\prettyref{sec:particle-linking-clause}),
and TODO: 助动词.

\begin{theorybox}{So-called category of functional items}{function-category}
    Though particles are in the grammar and do not really carry category labels like ``noun'' or ``verb''
    and it actually makes no sense to discuss the categories of them,
    the traditional practice to list all particles and classify them 
    is practically desirable, 
    as it provides a quick way to navigate across grammatical systems.
\end{theorybox}

Japanese lacks the prototypically pronoun class:
so-called pronouns are customized referential nouns like \translate{that girl},
and thus the pronoun class is not closed and strictly speaking is not a part of the grammar.
The article class is also not attested.

\section{Noun phrases}

In Japanese \acs{np}s, gender and number are not marked.
The case is marked by an \ac{np}-final particle (\prettyref{sec:case-particle}).
The information structure also receive explicit marking by particles,
including the topic marker \corpus{wa}, 
and in some cases the nominative marker \corpus{ga} has the reading of focus marker
(\prettyref{sec:topic-subject}).

\section{The verb and the clause}

\section{Constituent order}\label{sec:constituent-order}

Japanese has a strict modifier-first constituent order,
and here the term \term{modifier} includes 
arguments in a clause (``modifiers of the verb or the verbal adjective''),
and even \acs{np}s with respect to case particles.

\begin{theorybox}{the notion of head and modifier}{head-modifier}
    This notion of head and modifier is \acs{cgel}-like, 
    and is probably related to a strong head-final tendency in the linearization:
    if a so-called modifier is introduced as a specifier 
    in a functional projection with a root as the core,
    then obviously the root and the functional heads are realized into one unit (for example a verb complex) 
    and the ``modifier'' precedes the unit to ensure the (functional) head-final rule,
    and therefore in the surface-oriented analysis,
    we also get a modifier-head constituent order (where \term{head} means lexical heads).
    If there is no core root,
    then trivially the ``modifier'' is realized in a position before the spellout of functional heads,
    and the latter is regarded as somehow a head in the \acs{cgel} sense,
    and again we get a modifier-head constituent order,
    if we understand things like case particles as heads,
    which is the \ac{cgel} approach but not the \acs{blt} approach.
\end{theorybox}

Despite the strict modifier-head constituent order,
in the clause, the order of core and peripheral arguments and adverbials is relatively flexible,
which usually reflects the information structure.
Relevant mechanisms include 
topicalization and the ordering between \corpus{wa}-\ac{np}s and \corpus{ga}-\ac{np}s 
(\prettyref{sec:wa-ga-template}, TODO: ref), scrambling (TODO: ref), 
TODO: others

\section{Clause combining}

\section{Remarkable features}

\subsection{Politeness}

Some languages, like Chinese, have a hierarchy of politeness coded in the lexicon.
In Japanese this kind of lexical politeness also exists,
but some components of the grammar are also about politeness
(\prettyref{sec:nominal-polite}, \prettyref{sec:tense-polarity-polite}, TODO).
Some parts of the grammar do not involve any category about politeness,
but using them is shunned if the speaker wants to be polite
(\prettyref{sec:personal-pronoun}, TODO).

\subsection{Gender of speaker in speech}

Japanese doesn't give any place to grammatical gender.
However, the gender of the \emph{speaker} is important:
there is one dialect for men,
and another for women.
Grammar points involving gender of the speaker include TODO

\section{The structure of this note}

\begin{theorybox}{The organization in reference grammars}{grammar-org}
    The structure of this note and the contents of chapters follow 
    the examples set by \citet{Friesen2017}, \citet{jacques2021grammar}, \citet{Grimm2021},
    the famous \acs{cgel} \citep{cgel}, and of course Dixon's three volumes of \acs{blt}.
    The nominal chapters (TODO: ref), 
    
    The notion of verb complex (\prettyref{chap:verb-complex}) is also found in 
    \citet{Friesen2017}.
\end{theorybox}

\chapter{Phonology and the writing system}

\section{Vowels}\label{sec:vowel}

\section{Consonants}\label{sec:consonant}

\section{Phonotactics}

\subsection{The scheme of moras}\label{sec:mora-scheme}

Japanese is usually analyzed as a mora-timing language:
each mora occupies one rhythmic unit.
This isn't strictly true:
moras with devoiced vowels may be shorter (TODO: ref),
and so is geminated consonants (see below).
The allowed types of moras include V, CV, jV, CjV, R, N, and Q.

Here the symbol j is the glide /j/,
which may appear after a non-glide consonant and before a vowel,
and the sequence of the three phonemes is still one mora.
The appearance of j is called 拗音 \corpus{yōon} in Japanese.
The glide is only compatible with /a/, /u/ and /o/.
The symbol N is a moraic nasal:
it constitutes a single mora, called 撥音 \corpus{hatsuon} 
and never appears at the initial of a word.
Thus, except for CjV, Japanese doesn't allow multiple consonants,
and except for (C)VN, Japanese syllables are always open.

The symbol R means a chroneme,
which prolongs the last vowel,
called 長音 \corpus{chōon}.
It's compatible with any vowels.
Q means geminating the following consonant.
It's called 促音 \corpus{sokuon},
and may be realized as ``pause for a mora'' before the consonant. 
The two abstract phonemes represent adjustment of vowel and consonant lengths,
which are all distinctive in Japanese phonology.

\section{Accent}

\section{Sound change}

\section{The writing system}

\subsection{Overview}

The mora-based phonology of Japanese results in two syllabaries used to write Japanese,
called \term{kanas} 仮名.
There are two kinds of kana in contemporary use:
one is hiragana 平仮名, the other is katakana 片仮名.
Hiragana is used to write grammatical items (like inflectional endings and particles)
and a subset of words with native etymology,
while katakana is used for newly borrowed words.
Ideophones are traditionally written in katakana,
though sometimes they are written in hiragana for a softened, adorable appearance.

There exists several romanization systems for Japanese,
which are called \corpus{rōmaji} ローマ字 \translate{Roman letters}.
The Hepburn romanization is designed for non-native speakers,
which roughly reflects the actually contemporary pronunciation.
There are other systems of romanization, which are discussed in \prettyref{sec:romanization}.

\subsection{Differences between romanizations}\label{sec:romanization}

Since I (and intended readers) of this note are all non-native speakers,
it's a good idea to first introduce the romanization systems,
and for the same reason,
the system used in this note is the revised Hepburn romanization,
which tells us more about the phonological structure of Japanese 
in the eyes of an outsider.

All systems of romanization use the same set of letters to represent the five vowels:
\corpus{a}, \corpus{i}, \corpus{u}, \corpus{e}, and \corpus{o}.
The letters for the consonants are also largely the same.
Consonants without any flavor of the glide j are represented by 
\corpus{k}, \corpus{g}, \corpus{s}, \corpus{z}, \corpus{t}, \corpus{d}, \corpus{n}, 
\corpus{h}, \corpus{b}, \corpus{p}, \corpus{m}, \corpus{r}, and \corpus{w}.



\subsection{Table of kanas}\label{sec:kana}

Below is a table enumerating all hiraganas and their romaji correspondence (TODO: which kind of romaji?), 
called \corpus{gojūon} 五十音 \translate{fifty sounds}:
\begin{table}[H]
    \caption{The \corpus{gojūon} 五十音}
    \label{tbl:hiragana-chart}
    \centering
    \begin{tabular}{cccccc}
        \toprule
         & a & i & u & e & o \\ 
        \midrule
        $\emptyset$ & あ \corpus{a} & い \corpus{i} & う \corpus{u} & え \corpus{e} & お \corpus{o} \\ 
        k & か \corpus{ka} & き \corpus{ki} & く \corpus{ku} & け \corpus{ke} & こ \corpus{ko} \\ 
        g & が \corpus{ga} & ぎ \corpus{gi} & ぐ \corpus{gu} & げ \corpus{ge} & ご \corpus{go} \\ 
        s & さ \corpus{sa} & し \corpus{si} & す \corpus{su} & せ \corpus{se} & そ \corpus{so} \\ 
        z & ざ & じ & ず & ぜ & ぞ \\ 
        t & た & ち & つ & て & と \\ 
        d & だ & ぢ & づ & で & ど \\ 
        n & な & に & ぬ & ね & の \\ 
        h & は & ひ & ふ & へ & ほ \\ 
        b & ば & び & ぶ & べ & ぼ \\ 
        p & ぱ & ぴ & ぷ & ぺ & ぽ \\ 
        m & ま & み & む & め & も \\ 
        y & や & ~ & ゆ & ~ & よ \\ 
        r & ら & り & る & れ & ろ \\ 
        w & わ & ゐ & ~ & ゑ & を \\ 
        \bottomrule
    \end{tabular}
\end{table}

Certain phonological rules can already be observed in the kana-romaji correspondence in the above table.

\begin{learnbox}{Remembering kanas}{kana-remember}
    Hiraganas can be remembered by recalling the spelling of grammatical items.
    The dictionary form ending of verbal adjectives is い.
    The genitive case particle is の.
    The \corpus{te}-form of verbs ends in て.
    The verb politeness marker ます \corpus{masu} appears frequently.
    TODO
\end{learnbox}


\subsection{The Kanji}\label{sec:kanji}


\subsection{Spelling conventions}

\chapter{Nominal categories}

\begin{theorybox}{How to enumerate nominal classes}{how-to-nominal-class}
    The organization of nominal categories is the same as \citet[\citechap{3}, \citechap{4}]{Friesen2017}.
\end{theorybox}

\section{Pronouns}

\subsection{Personal pronouns}\label{sec:personal-pronoun}

\subsection{Demonstratives}

\section{Numerals}

\chapter{Nominal morphology}

\section{Honorifics}\label{sec:nominal-polite}

\subsection{Prefixes for objects}

For Sino-Japanese nouns, the prefix \corpus{go-} is used to add politeness.
For native nouns, the prefix \corpus{o-} is used instead.
TODO: how to write in kanji?

\subsection{Suffixes of people}

TODO: a table

\section{Nominal derivations}

\chapter{Particles in noun phrases}\label{chap:particle-in-np}

There are several systems of particles after \ac{np}s:
case particles 格助詞 (\prettyref{sec:case-particle}),
and adverbial particles 副助詞 (\prettyref{sec:adverbial-particle} 
-- the name is actually misleading, see the relevant section).

\begin{theorybox}{List of particles as a lookup table}{case-particles}
    I learn from \citet{jacques2021grammar} and organize all \ac{np}-final particles into one chapter 
    for quick lookup of the distributions of case, adverbial types, etc.
\end{theorybox}

The systems are not completely compatible (\prettyref{sec:particle-compatible}).
For example, a well known generalization is structural case markers 
-- the nominative \corpus{ga} and the accusative \corpus{o} -- 
are erased when \ac{np}s are topicalized,
while inherent case markers may be kept.

\section{Case particles}\label{sec:case-particle}

Here is a list of case particles:
\begin{itemize}
    \item Nominative: \corpus{ga}, 
    appearing in certain circumstances as the focus marker (\prettyref{sec:topic-subject}).
    \item Accusative: \corpus{o}
    \item Dative: \corpus{ni}: time and location 
    \item Genitive: \corpus{no} 
    \item Lative: \corpus{e}, used for destination direction (like in "to some place")
    \item Ablative: \corpus{kara}, used for source direction (like in "from some place")
    \item Instrumental/Locative: \corpus{de}
\end{itemize}

\subsection{The accusative case \corpus{o}}

\subsubsection{The object}

The accusative case is usually used to mark the object.
Note, however, that the semantically O argument may also be marked by \corpus{ga} 
and promoted to the initial of the clause, 
and thus there is no syntactic object in the clause 
(\prettyref{sec:multiple-ga}).

\subsubsection{The path}

The path argument (\translate{walk through/along/in \dots}) is marked as accusative.

\subsection{The possessive marker \corpus{no}}

\section{Adverbial particles, or miscellaneous}\label{sec:adverbial-particle}

The so-called adverbial particle class is a catch-all class for all particles appearing in the \ac{np}
but hard to classify.
They don't necessarily appear on peripheral arguments:
\corpus{dake}, for example, can appear on an object (\prettyref{sec:dake}),
and \corpus{hodo} can appear on a copular complement.

\subsection{\corpus{Dake}}\label{sec:dake}

\subsection{\corpus{Hodo}}

\subsubsection{Approximation}

The particle \corpus{hodo} may be attached to a countable \ac{np},
and takes the reading of \corpus{about}.
The \ac{np} may be a copular complement:
\begin{exe}
    \ex 1000円ほどです 
    \gll Sen en hodo desu \\
    1000 yen HODO  \\
    \glt \translate{(It) is about 1,000 yen.}
\end{exe}

\section{Compatibility}\label{sec:particle-compatible}



\chapter{The structure of noun phrases}

\chapter{The verb complex}\label{chap:verb-complex}

\section{Introduction}

As is said before, Japanese is strongly modifier-first,
and hence productive functional morphemes in the verb complex are always suffixes.
Japanese is typologically agglutinative:
the morphemes have relatively clear boundaries,
each morpheme representing a grammatical category.
Still, there are two important factors in Japanese that deviate away 
from the perfect agglutinative prototype.
The first is most morphemes in the verb complex 
-- both the lexical head and most functional morphemes following it --
either have \emph{internal} morphology or alter the form of suffixes following them,
depending on how you analyze it (\prettyref{sec:verb-complex-previous}),
and this is related to the conjugation class of the verb (\prettyref{sec:conjugation-class}).
The second is there is still some degree of fusion, 
in which historically analyzable morphemes seem to already form a single fused morpheme 
(\prettyref{sec:tense-polarity-polite}).

Previous researches, mainly the system of the School Grammar and the Education Grammar,
often use mutually incompatible terminologies to describe the verb complex.
How to translate between these terms is discussed in \prettyref{sec:verb-complex-previous}.
This is important,
because (of course) dictionaries are edited by native speakers of Japanese
and they use the School Grammar system (\prettyref{sec:so-called-auxiliary-verb}) to carry out their work.

\begin{learnbox}{How to conjugate a verb}{how-to-conjugate}
    First have a look at \prettyref{sec:conjugation-class} to ,
    then 
\end{learnbox}

\section{Conjugation classes}\label{sec:conjugation-class}

TODO: finally decide one way to describe conjugation forms

According to whether the final sound of the stem is a vowel or consonant,
Regular Japanese verbs can be divided into \concept{c-stem verbs} and \concept{v-stem verbs}.
C-stem verbs are also called 五段動詞 or \concept{group-1 verbs},
because changing the conjugation ending means 
the final mora may appear in every column of the kana chart (\prettyref{tbl:hiragana-chart}), 
and v-stem verbs are also called 一段動詞 or \concept{group-2 verbs},
because the last or the second but last mora is always in the same column 
with the last mora in the dictionary form,
which is the present, positive, plain form of the verb (\prettyref{sec:final-conjugation-pattern}).

一段動詞 can be further divided into 上一段動詞 or \corpus{iru}-verbs or \concept{group-2a verbs} 
and 下一段動詞 or \corpus{eru}-verbs or \concept{group-2b verbs}:
the final vowel of \corpus{iru}-verbs is \corpus{i},
and \corpus{-ru} is actually the conjugation ending of the terminal form,
and similarly the final vowel of \corpus{eru}-verbs is \corpus{e}.

There are two important irregular verbs: \corpus{suru} \translate{to do} 
and \corpus{kuru} \translate{to come}.
Since the verb category of Japanese is closed,
the two verbs are highly productive as light verbs when new verbal meanings are required.
They, together with their semi-conventionalized compound with \ac{np}s,
are collectively called \concept{group-3 verbs}.

Auxiliary verbs also follow the same pattern,
and this is even true for morphosyntactically inflectional suffixes:
the latter are definitely suffixes in any morphosyntactic sense (TODO: ref),
but have the same internal morphological patterns with lexical verbs.
This makes the traditional School Grammar include them into the category of ``auxiliary verbs''
(\prettyref{sec:so-called-auxiliary-verb}).
Since it's rather weird to talk about ``stems'' of inflectional suffixes,
we may use \term{c-suffix} and \term{v-suffix} instead of 
\term{c-stem} and \term{v-stem}.

\section{The suffix chain}

Inflectional endings of Japanese verbs are solely suffixal, 
due to the modifier-head constituent order (\prettyref{sec:constituent-order}).
When an inflectional suffix is added,
if the previous form ends in a consonant,
an \corpus{i} is inserted (TODO: making clear of the distribution of this form).

\begin{exe}
    \ex TODO: examples of Japanese verban complex
\end{exe}

\section{Valency changing devices}

\section{The tense, polarity and politeness complex}\label{sec:tense-polarity-polite}

In this note,
the categories of tense, politeness and polarity 
are deemed as realized by a single, fused ending without analyzable inner structure.
This is adequate for descriptive usage,
since nothing is able to ``fine tuning'' the inner structure of the 
tense, politeness and polarity marking.

\subsection{The conjugation pattern}\label{sec:final-conjugation-pattern}



\subsection{Historical notes}

\section{Notes about previous studies}\label{sec:verb-complex-previous}

\subsection{The School Grammar and ``auxiliary verbs''}\label{sec:so-called-auxiliary-verb}



\chapter{Arguments of verbs}\label{chap:arguments}

\section{\translate{I hit the wall \dots}: prototypical transitive and intransitive verbs}

\section{\translate{I think, I feel}: }

\chapter{Clausal constituent order and information packaging}

\section{Topic and subject}\label{sec:topic-subject}

The difference between the so-called topic marker \corpus{wa} and the subject marker \corpus{ga}
is a long problem in Japanese grammar.
This section provides a tentative summary of the function of each.

\subsection{The function of \corpus{wa}}

The meaning of \ac{np}-\corpus{wa} can be summarized as 
\translate{as for \ac{np}, I know \dots (the rest I'm not talking about)}.
Thus \corpus{wa} is a canonical marker of topic,
and the \ac{np} it's attached to must be somehow a ``known object''.

When two clauses containing \corpus{wa} are conjoined together,
we get the meaning of 
\translate{as for \dots, I know \dots; (but) as for \dots, \dots},
and this naturally has a contrastive meaning, as in (TODO).
Note that though it's possible that the \corpus{wa}-\ac{np}s in the two clauses are different,
because the above construction is contrasting two clauses, not two \ac{np}s.

\begin{exe}
    \ex TODO: example of contrastive  
\end{exe}

\subsection{The function of nominative \corpus{ga}, with or without \corpus{wa}} 

\subsection{Multiple \corpus{ga}}\label{sec:multiple-ga}

\subsection{Summary of constructions}\label{sec:wa-ga-template}

\section{Sentence final particles}\label{sec:sfp}

\ac{sfp}s are useful in oral communication.
This section is based on \citet[\citesec{6.4}]{akiyama2012japanese}.

\subsection{\corpus{Ka}}

The \ac{sfp} \corpus{ka} is the interrogative marker.
It marks both open and closed questions.
To make a question open,
just introduce interrogative pro-forms.

\subsection{\corpus{Ne}}

The \ac{sfp} \corpus{ne} invites the listener to confirm a claim:
\translate{\dots, don't you think so?}
We have the following examples:
\begin{exe}
    \ex \gll Atsui desu ne  \\
    hot {} NE  \\
    \glt \translate{It's hot, isn't it?}
\end{exe}

\subsection{\corpus{Yo}}

The \ac{sfp} \corpus{yo} is used to make a very strong assertion.

\chapter{Subordination}

As is often the case, there are fewer information packaging devices in subordinated clauses.
TODO: topicalization, 

\begin{theorybox}{Matrix clauses and subordinated clauses}{matrix-subordinated}
    From a generative perspective, both a matrix clause 
    (which is actually a sentence in the sense of \acs{blt}, 
    though it can still be embedded as an argument of a verb like \translate{say})
    and a subordinated clause are CPs.
    The structural differences between the two 
    are mainly about the lack of certain functional projections in the subordinated clause.
    Speaking of so-called Force projection(s),
    the subordinated clause usually only contains the marking of 
    imperative, declarative, etc.,
    without specification of detailed speech act,
    which may be marked by \ac{sfp}s in Japanese (and Chinese).
    Since topicalization is also a CP process,
    we can expect it's more restricted in subordinated clauses.
\end{theorybox}

\section{Clauses linked by particles}\label{sec:particle-linking-clause}

TODO: Compatibility with \ac{sfp}s
This section is a more annotated version of \citet[\citesec{6.3}]{akiyama2012japanese}.

\subsection{\corpus{Ga}}

\subsubsection{The concessive construction}

The particle \corpus{ga} may be attached to the end of a clause 
and has the reading of \translate{despite the circumstance}.

A conventionalized construction in oral speech is to avoid the main clause 
and end the utterance with \corpus{ga}.
In this way, it's a marginal \ac{sfp}.

\begin{exe}
    \ex Ikitakatta desu ga \dots
\end{exe}

\subsubsection{Neutral linking}

We can also just use \corpus{ga} as \translate{and}.

\subsection{\corpus{Kara}}

\subsubsection{The causal construction}

\bibliographystyle{plainnat}
\bibliography{grammars,details,../methodology/famous-grammars,theory}

\end{document}