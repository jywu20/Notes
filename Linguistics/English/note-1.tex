\documentclass[UTF8, a4paper, oneside, scheme=plain]{ctexrep}

\usepackage{libertinus}
\usepackage{geometry}
\usepackage{float}
\usepackage{titling}
\usepackage{titlesec}
\usepackage{paralist}
\usepackage{footnote}
\usepackage{enumerate}
\usepackage{amsmath, amssymb, amsthm}
\usepackage{gb4e}
\noautomath
\usepackage{bbm}
\usepackage{textcomp}
\usepackage{soul}
\usepackage{graphicx}
\usepackage{siunitx}
\usepackage[table,xcdraw]{xcolor}
\usepackage{tikz}
\usepackage[ruled, vlined, linesnumbered, noend]{algorithm2e}
\usepackage{xr-hyper}
\usepackage[colorlinks, citecolor = purple]{hyperref} % linkcolor=black, anchorcolor=black, citecolor=black, filecolor=black
\usepackage[most]{tcolorbox}
\usepackage{caption}
\usepackage{subcaption}
\usepackage{booktabs}
\usepackage{multirow}
\usepackage[figuresright]{rotating}
\usepackage{acro}
\usepackage[round]{natbib} 
\usepackage{nameref,zref-xr}
\zxrsetup{toltxlabel}
\zexternaldocument*[cgel-]{../English/cambridge}[cambridge.pdf]
\zexternaldocument*[alignment-]{../alignment/alignment}[alignment.pdf]
\zexternaldocument*[exercise1-]{../Exercise/2021-3}[2021-3.pdf]
\zexternaldocument*[method-]{../methodology/glossing}[glossing.pdf]
\usepackage{prettyref}

\geometry{left=3.18cm,right=3.18cm,top=2.54cm,bottom=2.54cm}
\titlespacing{\paragraph}{0pt}{1pt}{10pt}[20pt]
\setlength{\droptitle}{-5em}

\DeclareMathOperator{\timeorder}{\mathcal{T}}
\DeclareMathOperator{\diag}{diag}
\DeclareMathOperator{\legpoly}{P}
\DeclareMathOperator{\primevalue}{P}
\DeclareMathOperator{\sgn}{sgn}
\newcommand*{\ii}{\mathrm{i}}
\newcommand*{\ee}{\mathrm{e}}
\newcommand*{\const}{\mathrm{const}}
\newcommand*{\suchthat}{\quad \text{s.t.} \quad}
\newcommand*{\argmin}{\arg\min}
\newcommand*{\argmax}{\arg\max}
\newcommand*{\normalorder}[1]{: #1 :}
\newcommand*{\pair}[1]{\langle #1 \rangle}
\newcommand*{\fd}[1]{\mathcal{D} #1}

\newcommand*{\citesec}[1]{\S~{#1}}
\newcommand*{\citechap}[1]{Ch~{#1}}
\newcommand*{\citefig}[1]{Fig.~{#1}}
\newcommand*{\citetable}[1]{Table~{#1}}
\newcommand*{\citepage}[1]{pp.~{#1}}
\newcommand*{\citefootnote}[1]{fn.~{#1}}

\newrefformat{sec}{\citesec{\ref{#1}}}
\newrefformat{fig}{\citefig{\ref{#1}}}
\newrefformat{tbl}{\citetable{\ref{#1}}}
\newrefformat{chap}{\citechap{\ref{#1}}}
\newrefformat{fn}{\citefootnote{\ref{#1}}}
\newrefformat{box}{Box~\ref{#1}}
\newrefformat{ex}{\ref{#1}}

% color boxes

\tcbuselibrary{skins, breakable, theorems}

\newtcbtheorem[number within=chapter]{infobox}{Box}{
    enhanced,
    boxrule=0pt,
    colback=blue!5,
    colframe=blue!5,
    coltitle=blue!50,
    borderline west={4pt}{0pt}{blue!65},
    sharp corners,
    fonttitle=\bfseries, 
    breakable,
    before upper={\parindent15pt\noindent}}{box}
\newtcbtheorem[number within=chapter, use counter from=infobox]{theorybox}{Box}{
    enhanced,
    boxrule=0pt,
    colback=orange!5, 
    colframe=orange!5, 
    coltitle=orange!50,
    borderline west={4pt}{0pt}{orange!65},
    sharp corners,
    fonttitle=\bfseries, 
    breakable,
    before upper={\parindent15pt\noindent}}{box}
\newtcbtheorem[number within=chapter, use counter from=infobox]{learnbox}{Box}{
    enhanced,
    boxrule=0pt,
    colback=green!5,
    colframe=green!5,
    coltitle=green!50,
    borderline west={4pt}{0pt}{green!65},
    sharp corners,
    fonttitle=\bfseries, 
    breakable,
    before upper={\parindent15pt\noindent}}{box}

\newcommand*{\concept}[1]{\textbf{#1}}
\newcommand*{\term}[1]{\emph{#1}}
\newcommand{\corpus}[1]{\emph{#1}}

\newcommand{\redp}{\textasciitilde}

\DeclareAcronym{blt}{short = BLT, long = Basic Linguistic Theory}
\DeclareAcronym{cgel}{short = CGEL, long = The Cambridge Grammar of the English Language}
\DeclareAcronym{dm}{short = DM, long = Distributed Morphology}
\DeclareAcronym{tag}{long = Tree-adjoining grammar, short = TAG}
\DeclareAcronym{sfp}{long = sentence-final particle, short = \textsc{sfp}}
\DeclareAcronym{np}{long = noun phrase, short = NP}
\DeclareAcronym{vp}{long = verb phrase, short = VP}
\DeclareAcronym{pp}{long = preposition phrase, short = PP}
\DeclareAcronym{cls}{long = classifier, short = CLS}
\DeclareAcronym{dist}{long = distal, short = DIST}
\DeclareAcronym{prox}{long = proximate, short = PROX}
\DeclareAcronym{dem}{long = demonstrative, short = DEM}
\DeclareAcronym{classify}{long = classifier, short = \textsc{cl}}
\DeclareAcronym{dur}{long = durative, short = DUR}
\DeclareAcronym{neg}{long = negative, short = \textsc{neg}}
\DeclareAcronym{cc}{long = copular complement, short = CC}
\DeclareAcronym{cs}{long = copular subject, short = CS}
\DeclareAcronym{tam}{long = {tense, aspect, and mood}, short = TAM}
\DeclareAcronym{past}{long = past, short = PST}
\DeclareAcronym{nonpast}{long = non-past, short = NPST}
\DeclareAcronym{present}{long = present, short = PRES}
\DeclareAcronym{progressive}{long = progressive, short = \textsc{poss}}
\DeclareAcronym{perfect}{long = perfect, short = \textsc{perf}}
\DeclareAcronym{passive}{long = passive, short = \textsc{pass}}
\DeclareAcronym{copula}{long = copula, short = COP}
\DeclareAcronym{possessive}{long = possessive, short = \textsc{poss}}

\newcommand{\asis}[1]{\textsc{#1}}
\newcommand{\oneof}[1]{{#1}}
\newcommand*{\homo}[2]{#1$_{\text{#2}}$}
\newcommand{\category}[1]{\textsc{#1}}

\newcommand{\alignment}{\href{../alignment/alignment.pdf}{my notes about alignment}}
\newcommand{\exerciseone}{\href{../Exercise/2021-3.pdf}{this exercise}}
\newcommand{\method}{\href{../methodology/glossing.pdf}{this note about my understanding of descriptive grammars}}

\newcommand{\ala}{à la}
\newcommand{\translate}[1]{`#1'}
\newcommand{\vP}{\textit{v}P}

% Make subsubsection labeled
\setcounter{secnumdepth}{4}
\setcounter{tocdepth}{4}
% reset example counter every chapter (but do not include the chapter number to the label)
\counterwithin{exx}{chapter} 

% Reference formats
\renewcommand{\bibname}{References}
\setcitestyle{aysep={}} 

\title{English notes}
\author{Jinyuan Wu}

\begin{document}
    
\maketitle

\automath

\chapter{Introduction}

\chapter{Grammatical overview}

\section{Parts of speech}

\section{Nouns and noun phrases}

\section{Verbal morphology and the clause}

\subsection{\acs{tam} categories}

English has two tenses: the past and the present.
The aspectual system is more complicated. 
The concept of composition 
-- whether the inner makeup of an event is important \citep[\citesec{19.10}]{dixon2012basic3} -- 
is marked by the so-called plain-progressive distinction,
though \citep{dixon2012basic3} calls it the imperfective-perfective distinction.
The English plain-perfect distinction arguably marks 
a distorted version of the completion concept 
-- whether the time of an event is before the time of narratives defined by tense 
\citep[\citesec{19.7}]{dixon2012basic3},
because sometimes the event time in the English \category{Perfect} 
is the starting time of the event in question,
not the finishing time,
and this disagrees with the term \term{completion}.
English also has several modal constructions.
The above categories interact freely (TODO: really?).

The category of tense is always realized morphologically 
on the main verb when there is no auxiliary 
or on the highest auxiliary verb (\prettyref{sec:verb-forms}).
The category of the two aspect categories are marked by auxiliary verbs, 
as well as modality (\prettyref{sec:auxiliaries}). 

There is no future tense in English:
The future time is marked by the auxiliary \corpus{will} or \corpus{would} 
or the \corpus{be going to do} construction (\prettyref{sec:future}).

Besides the regular \acs{tam} system,
there are also some periphrastic constructions marking specific \acs{tam} configurations,
like \corpus{used to} or \corpus{would rather} (\prettyref{sec:semi-auxiliary}).
Adverbs are also a important part in expressing \acs{tam} information in English (TODO: ref).

\subsection{Negation} 

Negation in English clauses is usually realized by the negator \corpus{not}.
Negative forms of auxiliaries and pronouns can also be used to express a negative idea.

\subsection{Verb valency and alignment}

English is a typical accusative language.
A subject can be identified syntactically 
according to constituent order, 

\subsection{Moods}

English has three types of finite main clauses with regard to the related speech act:
the imperative, the indicative and the interrogative.
No further morphosyntactic marking of sentential speech act (such as sentence-final particles)
exists in English.
The three moods are not marked morphologically.
The interrogative mood 

\begin{infobox}{Mood and modality}{mood}
    \citet{dixon2009basic1} firmly argues against using the term \term{mood} 
    for the syntactic marking of modality,
    while \citet{cgel} uses the term \term{mood} for the syntactic marking of modality
    and uses \term{clause type} to specifically refer to Dixon's \term{mood}.
    To avoid confusion (\term{clause type} is too vague),
    this note follows the definition of Dixon.

    The confusion seems to arise from traditional Latin grammar,
    in which there is no significant difference 
    between a declarative sentence and an interrogative sentence, 
    while there is significant difference
    between the verbal morphology in indicative and subjunctive clauses.
    On the other hand, in imperative clauses 
    there is no indicative-subjunctive distinction.
    Therefore the imperative-non-imperative distinction is fused with 
    the indicative-subjunctive distinction 
    and is named \term{mood}.
    This relies on the specificities of Latin grammar 
    and surely is not a universal category for all languages.
    English also has subjunctive clauses,
    but that's about modality, not mood.
\end{infobox}

\begin{theorybox}{Clause and sentence}{clause-sentence}
    Also, note that \emph{mood} is about \emph{sentences}, 
    and not necessarily anything that can be called a clause \citet[96]{dixon2009basic1}.
    (Here Dixon is trapped into another extreme by claiming 
    the clause linking procedure is flat,
    and the sentence is the ultimate product of grammar.
    But of course clause linking can be done recursively,
    with a tree-like order.)
    In generative syntax, 
    a TP or a low-level CP is already well qualified as a clause,
    but in order to construct a sentence -- a verbal constituent 
    that serves as a ``full'' utterance --
    we still need to include the full functional projections marking speech acts or speech ``forces''.
    In Mandarin Chinese, for example,
    a clause often needs sentence-final particles attached to it to be an acceptable independent sentence,
    while clauses without sentence-final particles appear regularly in 
    clause linking,
    and even more impoverished ``small clauses'' -- basically \vP{}s -- also appear in clause embedding.
    Thus, there are several types of clauses with different sizes.

    Such phenomena also appear in English, arguably in all languages.
    Non-embedded finite clauses are of course full CPs,
    but some embedded finite clauses,
    like the indirect quoted question in \corpus{I asked [why he was always late]},
    show different behaviors with their non-embedded counterparts:
    In the bracketed example, the subject-auxiliary inversion is absent.
    Participle constructions are likely to be TPs, as well as infinitives in control constructions
    \citep{pires2006minimalist},
    while infinitives are impoverished CPs.

    Still, \citet[\citepage{45}]{cgel} uses the term \term{sentence} 
    almost as a synonym of \term{utterance},
    and all discussions concerning the syntax in their account of English grammar are about clauses.
    In practice, covering all kinds of TPs and CPs with the catch-all term \term{clause}
    doesn't create much confusion,
    because non-embedded finite clauses, 
    embedded finite clauses, and nonfinite clauses are usually discussed in different places 
    and it's easy to infer whether the term \term{clause} 
    means a full CP, a defective CP or a TP. 
    So to say mood (or \term{clause type}, 
    with the specific meaning of the imperative/declarative/interrogative distinction)
    is marked on a finite clause doesn't create much confusion, 
    and nor do wordings like ``the clause type or the mood marks the speech force'',
    though the latter is not universally true
    (in Chinese there are several syntactic systems marking the speech force).
\end{theorybox}

\subsection{Valency changing}

\begin{theorybox}{Valency changing}{valency-changing}
    Valency changing involves the lexicon, the \vP{} layer, and the TP layer.
    The term \term{valency changing} is kind of misleading,
    because what actually happens is \emph{valency corresponding},
    and the transformational rules used to describe valency changing 
    are just phenomenological.

    Some kinds of valency changing is likely to be purely because of 
    the verb in question has two subcategorization frames.
    The clause \corpus{John and Mary will meet tomorrow} 
    has the same meaning of \corpus{John will meet with Mary tomorrow},
    but it's unlikely the relation between the two arises from some operations in the \vP{} layer:
    The case may just be that \corpus{meet} is compatible with two \vP{} structures,
    which turn out to have the same semantic interpretation.

    Sometimes, however, we can add a \term{v} head to an existing \vP{},
    and extract one of the arguments introduced in the latter into the specifier of the former
    or introduce a new argument.
    This is frequently seen in Old Chinese as well as its sisters,
    and is still the main way Modern Mandarin Chinese does valency changing.

    It's also possible to have two (or more) \vP{} structures that both work for a group of verbs,
    and for some reason (e.g. the agentive argument is assigned an inherent case, 
    so it's no longer visible for the A-movement to SpecTP),
    one of them disrupts the way TP usually works.
    This seems to be the way the English passive works.

    In more descriptive terms,
    the \vP-internal strategy lies on the blur line between derivation and inflection
    (and if the additional \term{v} head is realized as a word, 
    that it lines on the line between multi-verb predicates and auxiliary verb constructions),
    while the \vP-TP strategy involves the \emph{alignment}.%
    \footnote{
        This -- the agentive argument in a transitive clause being assigned an inherent case -- 
        also seems to be the source of morphological ergativity \citep{aldridge2008generative}.
        Syntactic ergativity, on the other hand, is caused by an early EPP feature 
        targeting the absolutive \acs{np}.
    }
    It's hard to draw a clear line between the first and the second strategy.
\end{theorybox}

\subsection{Nonfinite constructions}

\section{Clause combining}

All English clause combining devices are on the level of complete clauses:
There is no complex predicate or clause chaining.
Thus, TODO: types of clause combining

\section{Constituent order}

As is said above, 
English has highly rigid constituent orders. 
Moving of syntactic objects usually indicates 
non-trivial information structure (TODO) 
or is triggered by the syntactic environment (TODO: interrogative, etc.).

\chapter{The structure of the noun phrase}

\chapter{Verb inflection}

\section{The verb paradigm}

\subsection{Inflectional forms}\label{sec:verb-forms}

\subsubsection{Lexical verbs}

\begin{theorybox}{Distinguishing inflectional forms}{morphological-form}
    Traditional grammars usually have a large paradigm
    with its row and column headers being grammatical categories.
    (When there are too many categories 
    -- and in this case the language in question is usually agglutinative -- 
    the paradigm will be unbearably large, 
    and another way -- like the School Grammar of Japanese -- is needed to cover verb inflection.
    Still, partial paradigms are useful in this case.) 
    This is a morphosyntactic way to represent the inflection of a word, 
    but if we are talking purely about the \emph{morphological} part
    (i.e. how grammatical relations and categories are realized),
    then it's sometimes not necessary to recognize so many forms:
    If a verb appears exactly the same in two different syntactic environments,
    then we say there is only one \emph{inflectional form} of that verb.
    For languages like Latin, 
    the traditional large-paradigm way is handy,
    while for English, we can zip the paradigm severely \citep[\citesec{3.1.2}]{cgel}.
\end{theorybox}

Modern English has already lost most of its verb inflection.
Following the analysis of \citet[\citesec{3.1.1}]{cgel},
for lexical verbs,
there are six remaining inflectional forms: 
the past form, the plain present form, 
the 3sg present form,
the plain form, the \corpus{ing}-participle,
and the \corpus{ed}-participle.
The two present forms and the past form appear solely 
with trivial aspectual values and trivial modality.
They are \concept{primary} forms:
They already have all \acs{tam} categories marked on them.
The plain form and the two participles are \concept{secondary} forms:
They usually appear after auxiliaries 
in a periphrastic construction to have full \acs{tam} marking,
though a subjunctive clause may sometimes get rid of any auxiliary verb,
as in \corpus{he suggests that she [complete] this task first} (TODO: ref).

Examples of these forms are illustrated in \prettyref{tbl:lexical-inflection}. 
This is a copy of [1] in \citet[\citesec{1.1}]{cgel}.
It can be noticed that the plain form is usually the same as the plain present form.
However, since modal verbs (see below) have no plain form,
and that the syntactic environments of the plain form and the present plain form are too different,
if \prettyref{tbl:lexical-inflection} is to be regarded as a paradigm
-- that is, to be incorporated with morphosyntactic information -- 
then the two forms should occupy two cells.

\begin{table}[H]
    \caption{Paradigms of lexical verbs}
    \label{tbl:lexical-inflection}
    \centering
    \begin{tabular}{@{}llllll@{}}
    \toprule
    \multicolumn{1}{l}{}       &                               &       & \corpus{take}   & \corpus{want}    & \corpus{hit}     \\ \midrule
    \multirow{3}{*}{Primary}   & past form                     &       & \corpus{took}   & \corpus{wanted}  & \corpus{hit}     \\
                               & \multirow{2}{*}{present form} & 3sg   & \corpus{takes}  & \corpus{wants}   & \corpus{hits}    \\
                               &                               & plain & \corpus{take}   & \corpus{want}    & \corpus{hit}     \\ \midrule
    \multirow{3}{*}{Secondary} & plain form                    &       & \corpus{take}   & \corpus{want}    & \corpus{hit}     \\
                               & \corpus{ing}-participle       &       & \corpus{taking} & \corpus{wanting} & \corpus{hitting} \\
                               & \corpus{ed}-participle        &       & \corpus{taken}  & \corpus{wanted}  & \corpus{hit}     \\ \bottomrule
    \end{tabular}
\end{table}

\begin{infobox}{The name of the forms}{verb-form-name}
    Here I deviate from the practice in \citep[\citechap{3}]{cgel} 
    and pick up the more common names for some of the forms. 

    The \corpus{ing}-participle is frequently called the \term{gerund},
    because it now has the function of both a gerund and an active participle.
    \citep{cgel} calls it the \term{gerund-participle}.
    Some grammars use the term \term{present participle}.
    Since in Modern English,
    the \corpus{ing}-participle no longer carries any tense information,
    the historical term \term{present participle} is abandoned in this note.

    The traditional name \term{past participle} for the \corpus{ed}-participle makes more sense,
    because it's morphologically related to the past form for regular verbs 
    and it still has some sense of ``past'':
    It is strongly related to the \category{perfect} and therefore has some sense of the past,
    though it doesn't carry the past tense.
    A better term would be the one in Latin grammar: the \term{perfect passive participle},
    but this is in conflict with the name of the \corpus{having been done} construction.

    A usual name for the plain form is the infinitive form,
    which I reject here because the morphological marking of 
    the main verb after modal auxiliary verbs  
    (\corpus{would [like]}),
    the verb in a subjunctive clause 
    (\corpus{he suggests that she [complete] this task first}),
    and the verb in a real infinitive clause are all the same,
    and therefore it makes no sense to use the term \term{infinitive} 
    to cover the \emph{morphological} form of all the three.
\end{infobox}

The \corpus{ing}-participle is regularly formed by adding \corpus{-ing} to the end of the plain form
(TODO: -tt- in splitting).
The \corpus{ed}-participle and the past form are usually obtained 
by adding \corpus{-ed} to the end of the plain form,
but for irregular verbs they can't be inferred from the plain form.
Thus English verbs have three principal forms:
the plain form, the past form, and the \corpus{ed}-participle.

\subsubsection{Auxiliary verbs}

English also has a number of auxiliary verbs (\prettyref{sec:auxiliaries}).
All auxiliary verbs have tense-dependent forms,
because all of them may appear as the first word in an auxiliary chain,
and the tense category is to be marked on the highest i.e. the first of them (\prettyref{sec:auxiliary-chain}).
Thus, we say English auxiliaries also have primary forms.
Modal auxiliaries don't have a separate 3sg present form,
but \corpus{do}, \corpus{have} and \corpus{be} (when used as auxiliary verbs) do.
It should be noted that the past forms of many auxiliary verbs don't just appear in past clauses:
They may have distinct meanings (TODO: ref).

Modal auxiliaries don't have secondary forms,
probably because they never appear after another auxiliary verb 
or in nonfinite clauses,
but \corpus{do}, \corpus{have} and \corpus{be} do.

English auxiliary verbs also have negative forms,
which are obtained by attaching \corpus{-nt} to the end of auxiliary.
The \corpus{-nt} is historically a contraction form of the negator \corpus{not},
but in modern English the negative suffix moves together with the auxiliary in
subject-auxiliary inversion (\prettyref{sec:sai}).
Thus, it's recognized as a part of the auxiliary \citep[\citepage{91}]{cgel}.
All auxiliaries don't have secondary negative forms,
though \corpus{do}, \corpus{have} and \corpus{be} have primary negative forms.

Since auxiliary verbs are a part of the grammar,
here I list the paradigms TODO

\subsection{Periphrastic constructions with auxiliary verbs}\label{sec:auxiliaries}

\subsubsection{The regular auxiliary chain}\label{sec:auxiliary-chain}

In a canonical finite clause
(non-canonical clauses may undergo subject-auxiliary inversion (\prettyref{sec:sai}) or may not), 
the order of auxiliaries is constantly given by \prettyref{tbl:auxiliary-chain}.
The auxiliaries positions can be filled by the corresponding auxiliaries or be just left blank,
without creating ungrammatical constructions.
The \category{modal} slot may be filled by a modal auxiliary.
The \category{perfect} slot may be filled by the auxiliary version of \corpus{have} with the correct inflection,
and the \category{progressive} and \corpus{passive} slots 
may be filled by the auxiliary version of \corpus{be} with the correct inflection.

The rules of inflection are the follows.
The tense category is always marked on the first auxiliary,
and when there is no auxiliary,
it's marked on the main verb.
The progressive marking \corpus{be} is always followed by an \corpus{ing}-participle,
and the perfect marking \corpus{have} is always followed by an \corpus{ed}-participle,
and so is the passive marking \corpus{be}.

In nonfinite forms, the \category{modal} slot has to go;
the rest are still there.
Thus we have \corpus{to have been being taken} or \corpus{having been being taken}.

\begin{table}[H]
    \caption{The order of auxiliaries and some examples}
    \label{tbl:auxiliary-chain}
    \centering
    \begin{tabular}{@{}lllll@{}}
    \toprule
    \category{modal}      & \category{perfect}      & \category{progressive}        & \category{passive}            & main verb    \\ \midrule
    \corpus{}           & \corpus{}             & \corpus{}                   & \corpus{}                   & \corpus{takes}  \\
    \corpus{}           & \corpus{}             & \corpus{}                   & \corpus{am/are/is/was/were} & \corpus{taken}  \\
    \corpus{}           & \corpus{}             & \corpus{am/are/is/was/were} & \corpus{}                   & \corpus{taking} \\
    \corpus{}           & \corpus{have/has/had} & \corpus{}                   & \corpus{}                   & \corpus{taken}  \\
    \corpus{}           & \corpus{have/has/had} & \corpus{been}               & \corpus{being}              & \corpus{taken}  \\
    \corpus{will/would} & \corpus{have}         & \corpus{been}               & \corpus{being}              & \corpus{taken}  \\ \bottomrule
    \end{tabular}
\end{table}

\prettyref{tbl:auxiliary-chain} is a part of the larger picture of clause structure,
because adverbs and the negator may be inserted into somewhere between two auxiliaries.
The rule of negation is simple: 
If the negator \corpus{not} is used, it is always after the first auxiliary 
(\prettyref{ex:auxiliary-chain-breaking-2}),
otherwise the first auxiliary is in its negative form (\prettyref{ex:auxiliary-chain-breaking-3}).
TODO: rules for adverbs and where to place them

\begin{exe}
    \ex\label{ex:auxiliary-chain-breaking-1} 
    He [is]_{\text{\category{progressive}}} [vigorously]_{\text{TODO:}} [doing]_{\text{main verb}} [his job]_{\text{object}}. 
    \ex\label{ex:auxiliary-chain-breaking-2}
    He is [not]_{\text{negation}} vigorously doing his job.
    \ex\label{ex:auxiliary-chain-breaking-3}
    He isn't vigorously doing his job.
\end{exe}

\begin{theorybox}{The TP projections}{tp}
    From a generative perspective, 
    what happens here isn't surprising:
    What happens here is the span spellout of grammatical categories in the TP layer.
    The T feature is realized via affix lowering, 
    and it's attached locally to the nearest ``word'' after vocabulary insertion,
    which is, of course, the first auxiliary.
    Some adverbs are actually heads of functional projections,
    and they are by all means a part of the sequence in \prettyref{tbl:auxiliary-chain}.
    TODO: cartography of TP, especially the position of not
\end{theorybox}

\subsubsection{Subject-auxiliary inversion}\label{sec:sai}

In interrogative sentences, TODO: what else 
the first auxiliary in the chain undergoes leftward movement,
usually to the initial position but may be preceded by preposed constituents (TODO: ref). 
This is called \concept{subject-auxiliary inversion}.
If 

\subsection{Semi-auxiliaries}\label{sec:semi-auxiliary}

\subsection{Other semantic concepts}

Some concepts exist in English but are not marked in the verb paradigm.

\subsubsection{The future time}\label{sec:future}

\subsubsection{Evidentiality}

The usual idea is English doesn't have an evidentiality category.
The idea of evidentiality may be expressed by TODO: allegedly 
and by complement clause constructions about quoted speech (TODO: ref).

\section{The imperative mood}

\section{The verb in nonfinite constructions}

\section{The subjunctive mood}

\chapter{Verb valency}

\chapter{Peripheral arguments}

\chapter{Passivization}

\chapter{Simple clauses}

This chapter is mainly about the inner makeup of ``canonical'' clauses, 
i.e. clauses without information packaging (\prettyref{chap:information-packaging}).
The details of how a clause is embedded into another are not covered in this chapter
-- they are covered in \prettyref{chap:complement-clause}, \prettyref{chap:relative-clause} and TODO: adverbial clause.

\section{Canonical verbal clauses}

\section{Clauses containing copular complements}

\subsection{Copular clause}

\chapter{Information packaging}\label{chap:information-packaging}

\chapter{Complement clause constructions}\label{chap:complement-clause}

\section{Types of complement clauses}

\section{Complement-taking verbs}

\chapter{Relative clauses}\label{chap:relative-clause}

\bibliographystyle{plainnat}
\bibliography{cambridge}

\end{document}