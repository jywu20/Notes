\documentclass[UTF8, a4paper, oneside, scheme=plain]{ctexrep}

\usepackage{libertinus}
\usepackage{geometry}
\usepackage{float}
\usepackage{titling}
\usepackage{titlesec}
\usepackage{paralist}
\usepackage{footnote}
\usepackage[inline]{enumitem}
\usepackage{amsmath, amssymb, amsthm}
\usepackage{gb4e}
\noautomath
\usepackage{bbm}
\usepackage{textcomp}
\usepackage{soul}
\usepackage{graphicx}
\usepackage{siunitx}
\usepackage[table,xcdraw]{xcolor}
\usepackage{tikz}
\usepackage[ruled, vlined, linesnumbered, noend]{algorithm2e}
\usepackage{xr-hyper}
\usepackage[colorlinks, citecolor = purple]{hyperref} % linkcolor=black, anchorcolor=black, citecolor=black, filecolor=black
\usepackage[most]{tcolorbox}
\usepackage{caption}
\usepackage{subcaption}
\usepackage{booktabs}
\usepackage{multirow}
\usepackage[figuresright]{rotating}
\usepackage{acro}
\usepackage[round]{natbib} 
\usepackage{nameref,zref-xr}
\zxrsetup{toltxlabel}
\zexternaldocument*[alignment-]{../alignment/alignment}[alignment.pdf]
\zexternaldocument*[exercise1-]{../Exercise/2021-3}[2021-3.pdf]
\zexternaldocument*[method-]{../methodology/glossing}[glossing.pdf]
\usepackage{prettyref}

\geometry{left=3.18cm,right=3.18cm,top=2.54cm,bottom=2.54cm}
\titlespacing{\paragraph}{0pt}{1pt}{10pt}[20pt]
\setlength{\droptitle}{-5em}

\DeclareMathOperator{\timeorder}{\mathcal{T}}
\DeclareMathOperator{\diag}{diag}
\DeclareMathOperator{\legpoly}{P}
\DeclareMathOperator{\primevalue}{P}
\DeclareMathOperator{\sgn}{sgn}
\newcommand*{\ii}{\mathrm{i}}
\newcommand*{\ee}{\mathrm{e}}
\newcommand*{\const}{\mathrm{const}}
\newcommand*{\suchthat}{\quad \text{s.t.} \quad}
\newcommand*{\argmin}{\arg\min}
\newcommand*{\argmax}{\arg\max}
\newcommand*{\normalorder}[1]{: #1 :}
\newcommand*{\pair}[1]{\langle #1 \rangle}
\newcommand*{\fd}[1]{\mathcal{D} #1}

\newcommand*{\citesec}[1]{\S~{#1}}
\newcommand*{\citechap}[1]{Ch~{#1}}
\newcommand*{\citefig}[1]{Fig.~{#1}}
\newcommand*{\citetable}[1]{Table~{#1}}
\newcommand*{\citepage}[1]{p.~{#1}}
\newcommand*{\citepages}[1]{pp.~{#1}}
\newcommand*{\citefootnote}[1]{fn.~{#1}}
\newcommand*{\citechapsec}[2]{\citechap{#1}.\citesec{#2}}

\newrefformat{sec}{\citesec{\ref{#1}}}
\newrefformat{fig}{\citefig{\ref{#1}}}
\newrefformat{tbl}{\citetable{\ref{#1}}}
\newrefformat{chap}{\citechap{\ref{#1}}}
\newrefformat{fn}{\citefootnote{\ref{#1}}}
\newrefformat{box}{Box~\ref{#1}}
\newrefformat{ex}{\ref{#1}}

% color boxes

\tcbuselibrary{skins, breakable, theorems}

\newtcbtheorem[number within=chapter]{infobox}{Box}{
    enhanced,
    boxrule=0pt,
    colback=blue!5,
    colframe=blue!5,
    coltitle=blue!50,
    borderline west={4pt}{0pt}{blue!65},
    sharp corners,
    fonttitle=\bfseries, 
    breakable,
    before upper={\parindent15pt\noindent}}{box}
\newtcbtheorem[number within=chapter, use counter from=infobox]{theorybox}{Box}{
    enhanced,
    boxrule=0pt,
    colback=orange!5, 
    colframe=orange!5, 
    coltitle=orange!50,
    borderline west={4pt}{0pt}{orange!65},
    sharp corners,
    fonttitle=\bfseries, 
    breakable,
    before upper={\parindent15pt\noindent}}{box}
\newtcbtheorem[number within=chapter, use counter from=infobox]{learnbox}{Box}{
    enhanced,
    boxrule=0pt,
    colback=green!5,
    colframe=green!5,
    coltitle=green!50,
    borderline west={4pt}{0pt}{green!65},
    sharp corners,
    fonttitle=\bfseries, 
    breakable,
    before upper={\parindent15pt\noindent}}{box}

% Shorthands
\newcommand*{\concept}[1]{\textbf{#1}}
\newcommand*{\term}[1]{\emph{#1}}
\newcommand{\form}[1]{\emph{#1}}

\newcommand{\redp}{\textasciitilde}

\newcommand{\deictictime}{T$_{\text{d}}$}
\newcommand{\referredtime}{T$_{\text{r}}$}
\newcommand{\orientationtime}{T$_{\text{o}}$}

\DeclareAcronym{blt}{short = BLT, long = Basic Linguistic Theory}
\DeclareAcronym{cgel}{short = CGEL, long = The Cambridge Grammar of the English Language}
\DeclareAcronym{dm}{short = DM, long = Distributed Morphology}
\DeclareAcronym{tag}{long = Tree-adjoining grammar, short = TAG}
\DeclareAcronym{sfp}{long = sentence-final particle, short = \textsc{sfp}}
\DeclareAcronym{np}{long = noun phrase, short = NP}
\DeclareAcronym{vp}{long = verb phrase, short = VP}
\DeclareAcronym{pp}{long = preposition phrase, short = PP}
\DeclareAcronym{advp}{long = adverb phrase, short = AdvP}
\DeclareAcronym{cls}{long = classifier, short = CLS}
\DeclareAcronym{dist}{long = distal, short = DIST}
\DeclareAcronym{prox}{long = proximate, short = PROX}
\DeclareAcronym{dem}{long = demonstrative, short = DEM}
\DeclareAcronym{classify}{long = classifier, short = \textsc{cl}}
\DeclareAcronym{dur}{long = durative, short = DUR}
\DeclareAcronym{neg}{long = negative, short = \textsc{neg}}
\DeclareAcronym{cc}{long = copular complement, short = CC}
\DeclareAcronym{cs}{long = copular subject, short = CS}
\DeclareAcronym{tam}{long = {tense, aspect, and mood}, short = TAM}
\DeclareAcronym{past}{long = past, short = PST}
\DeclareAcronym{nonpast}{long = non-past, short = NPST}
\DeclareAcronym{present}{long = present, short = PRES}
\DeclareAcronym{progressive}{long = progressive, short = \textsc{poss}}
\DeclareAcronym{perfect}{long = perfect, short = \textsc{perf}}
\DeclareAcronym{passive}{long = passive, short = \textsc{pass}}
\DeclareAcronym{copula}{long = copula, short = COP}
\DeclareAcronym{possessive}{long = possessive, short = \textsc{poss}}
\DeclareAcronym{coca}{long = Corpus of Contemporary American English, short = COCA}

\newcommand{\asis}[1]{\textsc{#1}}
\newcommand{\oneof}[1]{{#1}}
\newcommand*{\homo}[2]{#1$_{\text{#2}}$}
\newcommand{\category}[1]{\textsc{#1}}
\newcommand{\formcat}[1]{\textsc{#1}}
\newcommand{\emptymorpheme}{$\emptyset$}
\newcommand*{\fromto}[2]{\langle {#1}, {#2} \rangle}

\newcommand{\alignment}{\href{../alignment/alignment.pdf}{my notes about alignment}}
\newcommand{\exerciseone}{\href{../Exercise/2021-3.pdf}{this exercise}}
\newcommand{\method}{\href{../methodology/glossing.pdf}{this note about my understanding of descriptive grammars}}

\newcommand{\ala}{à la}
\newcommand{\translate}[1]{`#1'}
\newcommand{\vP}{\textit{v}P}

% Make subsubsection labeled
\setcounter{secnumdepth}{4}
\setcounter{tocdepth}{4}
% reset example counter every chapter (but do not include the chapter number to the label)
\counterwithin{exx}{chapter} 

% Reference formats
\renewcommand{\bibname}{References}
\setcitestyle{aysep={}} 

% List format
\setlist[enumerate,1]{label=\alph*\upshape)}

\title{Aspects of English morphosyntax}
\author{Jinyuan Wu}

\begin{document}
    
\maketitle

\automath

\tableofcontents

\chapter{Introduction}

\section{The language and the speakers}

\section{Theoretical orientation}



\chapter{Parts of speech}

\chapter{Noun phrase}

\chapter{Verb phrase}

\section{The verbal complex}

\subsection{The structure of the regular verbal complex}

Now we can combine everything in the verbal complex together.
When there is no auxiliary needed,
the tense feature is lowered to the main verb. 
In other cases, the highest auxiliary -- the first auxiliary -- 
is lifted to the tense position,
before negation and the default position of many adverbs.

\subsection{Regular lexical verbs}

\begin{theorybox}{Inflectional forms are about realization and not underlying structure}{morphological-form}
    Traditional grammars usually have a large paradigm
    with its row and column headers being grammatical categories.
    (When there are too many categories 
    -- and in this case the language in question is usually agglutinative -- 
    the paradigm will be unbearably large, 
    and another way -- like the School Grammar of Japanese -- is needed to cover verb inflection.
    Still, partial paradigms are useful in this case.) 
    This is a morphosyntactic way to represent the inflection of a word, 
    but if we are talking purely about the \emph{morphological} part
    (i.e. how grammatical relations and categories are realized),
    then it's sometimes not necessary to recognize so many forms:
    If a verb appears exactly the same in two different syntactic environments,
    then we say there is only one \emph{inflectional form} of that verb.
    For languages like Latin, 
    the traditional large-paradigm way is handy,
    while for English, we can zip the paradigm severely \citep[\citechapsec{3}{1.2}]{cgel}.
\end{theorybox}

Modern English has already lost most of its verb inflection.
Following the analysis of \citet[\citechapsec{3}{1.1}]{cgel},
for lexical verbs,
there are six remaining inflectional forms: 
the past form, the plain present form, 
the 3sg present form,
the plain form, the \formcat{ing}-participle,
and the \formcat{ed}-participle.
The two present forms and the past form appear solely 
with trivial aspectual values and trivial modality.
They are \concept{primary} forms:
They already have all \acs{tam} categories marked on them.
The plain form and the two participles are \concept{secondary} forms:
They usually appear after auxiliaries 
in a periphrastic construction to have full \acs{tam} marking,
though a subjunctive clause may sometimes get rid of any auxiliary verb,
as in \form{he suggests that she [complete] this task first} (\prettyref{sec:complement.subjunctive}).

Examples of these forms are illustrated in \prettyref{tbl:lexical-inflection}. 
This is a copy of [1] in \citet[\citesec{1.1}]{cgel}.
It can be noticed that the plain form is usually the same as the plain present form.
However, since modal verbs (see below) have no plain form,
and that the syntactic environments of the plain form and the present plain form are too different,
if \prettyref{tbl:lexical-inflection} is to be regarded as a paradigm
-- that is, to be incorporated with morphosyntactic information -- 
then the two forms should occupy two cells.

\begin{table}[H]
    \caption{Paradigms of lexical verbs}
    \label{tbl:lexical-inflection}
    \centering
    \begin{tabular}{@{}llllll@{}}
    \toprule
    \multicolumn{1}{l}{}       &                               &       & \form{take}   & \form{want}    & \form{hit}     \\ \midrule
    \multirow{3}{*}{Primary}   & past form                     &       & \form{took}   & \form{wanted}  & \form{hit}     \\
                               & \multirow{2}{*}{present form} & 3sg   & \form{takes}  & \form{wants}   & \form{hits}    \\
                               &                               & plain & \form{take}   & \form{want}    & \form{hit}     \\ \midrule
    \multirow{3}{*}{Secondary} & plain form                    &       & \form{take}   & \form{want}    & \form{hit}     \\
                               & \formcat{ing}-participle       &       & \form{taking} & \form{wanting} & \form{hitting} \\
                               & \formcat{ed}-participle        &       & \form{taken}  & \form{wanted}  & \form{hit}     \\ \bottomrule
    \end{tabular}
\end{table}

\begin{infobox}{The name of the forms}{verb-form-name}
    Here I deviate from the practice in \citep[\citechap{3}]{cgel} 
    and pick up the more common names for some of the forms. 

    The \formcat{ing}-participle is frequently called the \term{gerund},
    because it now has the function of both a gerund and an active participle.
    \citet{cgel} call it the \term{gerund-participle}.
    Some grammars use the term \term{present participle}.
    Since in Modern English,
    the \formcat{ing}-participle no longer carries any tense information,
    the historical term \term{present participle} is abandoned in this note.

    The traditional name \term{past participle} for the \formcat{ed}-participle makes more sense,
    because it's morphologically related to the past form for regular verbs 
    and it still has some sense of ``past'':
    It is strongly related to the \category{perfect} and therefore has some sense of the past,
    though it doesn't carry the past tense.
    A better term would be the one in Latin grammar: the \term{perfect passive participle},
    but this is in conflict with the name of the \form{having been done} construction.

    A usual name for the plain form is the infinitive form,
    which I reject here because the morphological marking of 
    the main verb after modal auxiliary verbs  
    (\form{would [like]}),
    the verb in a subjunctive clause 
    (\form{he suggests that she [complete] this task first}),
    and the verb in a real infinitive clause are all the same,
    and therefore it makes no sense to use the term \term{infinitive} 
    to cover the \emph{morphological} form of all the three.
\end{infobox}

The \formcat{ing}-participle is regularly formed by adding \form{-ing} to the end of the plain form
(TODO: -tt- in splitting).
The \formcat{ed}-participle and the past form are usually obtained 
by adding \form{-ed} to the end of the plain form,
but for irregular verbs they can't be inferred from the plain form.
Thus English verbs have three \concept{principal forms}:
the plain form, the past form, and the \formcat{ed}-participle.
We may also say there are three stems in English:
the plain form, the past form, and the \formcat{ed}-participle,
with only the first one being productive for further morphological processes.

\subsection{Types of irregular verbs}

As is mentioned above, for a number of irregular verbs,
the \category{ed}-participle and the past form can't be inferred from the plain form.
Whether there are still some patterns between the three,
or in other words,
the formation of the principal parts,
is investigated in detail in \citet[\citepages{105-120}]{quirk1985}.

\subsection{Auxiliary verbs}\label{sec:verb-inflection.auxiliary}

English also has a number of auxiliary verbs (\prettyref{sec:auxiliaries}).
All auxiliary verbs have tense-dependent forms,
because all of them may appear as the first word in an auxiliary chain,
and the tense category is to be marked on the highest i.e. the first of them (\prettyref{sec:auxiliary-chain}).
Thus, we say English auxiliaries also have primary forms.
Modal auxiliaries don't have a separate 3sg present form,
but \form{do}, \form{have} and \form{be} (when used as auxiliary verbs) do.
It should be noted that the past forms of many auxiliary verbs don't just appear in past clauses:
They may have distinct meanings (\prettyref{sec:verb-inflection.modal-use}).

Modal auxiliaries don't have secondary forms,
probably because they never appear after another auxiliary verb 
or in nonfinite clauses,
but \form{do}, \form{have} and \form{be} do.

English auxiliary verbs also have negative forms,
which are obtained by attaching \form{-nt} to the end of auxiliary.
The \form{-nt} is a contraction form of the negator \form{not},
but in modern English the negative suffix moves together with the auxiliary in
subject-auxiliary inversion (\prettyref{sec:sai}).
Thus, it's recognized as a part of the auxiliary \citep[\citepage{91}]{cgel}.
This seems to be purely about phonetic realization:
There seems to be no large morphosyntactic differences 
between auxiliary-\form{not} and the negative auxiliary
besides subject-auxiliary inversion.
All auxiliaries don't have secondary negative forms,
though \form{do}, \form{have} and \form{be} have primary negative forms.

Since auxiliary verbs are a part of the grammar,
here I list the paradigms TODO

\begin{infobox}{Auxiliary constructions are single-clause ones}{auxiliary-single-clause}
    \citet{cgel} treat auxiliary verbs as verbs taking complement clauses 
    (as in, say, [11] in \citepage{782}).
    This is not the position of this note:
    Here I follow the standard practice in generative syntax (probably also American structuralism) 
    and assume auxiliary verb constructions are always single-clause constructions.
    \emph{Historically}, auxiliaries may origin from complement-taking verbs,
    but now \emph{synchronically}, they have the same function of inflectional affixations.
    Complement clause constructions may (or may not) have the same \emph{semantics} of 
    auxiliary verb constructions and inflectional affixations,
    but they never have the same \emph{structure}.

    The main reasons \citet{cgel} analyze auxiliary verbs as complement-taking verbs or
    \term{concatenative verbs} in their terms 
    are shown in their \citechapsec{14}{4.2.2}.
    However, these arguments are based on interpretation of constituency trees like
    \form{[would [like to do]]} as complement clause constructions,
    which doesn't necessarily hold.
    They also confuse lexical heads and (PF realization of) functional heads.
    They therefore bring in much inconsistency when they argue that 
    the complementizer \form{that} isn't a head.
    In this note, I follow the standard definition of (lexical) headhood in the descriptive literature
    while fully being aware of the generative functional head analysis.

    Evidences supporting my claim that auxiliary constructions are indeed single-clauses ones 
    can be obtained by observing how auxiliaries interact with clausal dependents.
    If \citet{cgel} are correct on their claim that English auxiliary verbs take bare infinitive clauses,
    then we expect the verbal phrase after an auxiliary verb to receive any modification 
    that's acceptable for a bare infinitive clause.
    However, as we see in \prettyref{sec:verb-inflection.adverb-auxiliary-chain},
    there is a strong tendency for adverbs to appear after the first auxiliary,
    which can be easily explained by assuming the first auxiliary 
    undergoes some kind of fronting (\prettyref{sec:verb-inflection.negation}),
    or after all auxiliaries and before the main verb,
    and the functions of adverbs in the two positions 
    have clear correlation with the positions.
    This pattern are hard to account for 
    when we assume auxiliary verb constructions are complement clause constructions,
    because nothing motivates it.
    If, on the other hand, 
    auxiliary verb constructions are single-clause constructions,
    then we can say the distribution of adverbs and auxiliaries
    show is just the surface reflection of a deep functional hierarchy,
    just like the subject is somehow higher than the object.
\end{infobox}


\subsection{Minimal auxiliary chain}\label{sec:auxiliary-chain}

In a declarative finite clause,  
the order of auxiliaries is constantly given by \prettyref{tbl:auxiliary-chain}.
\prettyref{tbl:auxiliary-chain} is a part of the larger picture of clause structure:
The auxiliary \form{do} (\prettyref{sec:verb-inflection.do}), 
adverbs (\prettyref{sec:verb-inflection.adverb-auxiliary-chain})
and the negator (\prettyref{sec:verb-inflection.negation})
may be inserted into somewhere between two auxiliaries.
Other types of clauses still largely follow the scheme but 
may undergo subject-auxiliary inversion (\prettyref{sec:sai}).

The auxiliaries positions can be filled by the corresponding auxiliaries or be just left blank,
without creating ungrammatical constructions.
The \category{modal} slot may be filled by a modal auxiliary.
The \category{perfect} slot may be filled by the auxiliary version of \form{have} with the correct inflection,
and the \category{progressive} and \form{passive} slots 
may be filled by the auxiliary version of \form{be} with the correct inflection.

The rules of inflection are the follows.
The tense category is always marked on the first auxiliary
(not necessarily one of the slots in \prettyref{tbl:auxiliary-chain}
-- it may be an inserted \form{do}),
and when there is no auxiliary,
it's marked on the main verb.
Note that it isn't true that if the first auxiliary is in the past form,
it always means a past event (\prettyref{sec:verb-inflection.modal-use}).
The modal auxiliary is always followed by a plain form,
and the progressive marking \form{be} is always followed by an \formcat{ing}-participle,
and the perfect marking \form{have} is always followed by an \formcat{ed}-participle,
and so is the passive marking \form{be}.
When the clause is finite and the tense is \category{present},
and the \category{modal} slot is empty,
if the subject is 3sg in number,
then the first non-empty slot in \prettyref{tbl:auxiliary-chain}
is in the 3sg present form,
which means for verbs other than \form{be}, the \form{-s} suffix is attached to it;
for \form{be} the correct form is \form{is}.
This is the only case subject-verb agreement happens in English 
other than the case of \form{be} (\prettyref{ex:verb-inflection.3sg-1}).
For \form{be}, the tense is still  TODO: subjunctive 

In nonfinite forms, the \category{modal} slot has to go;
the rest are still there, following the same inflectional pattern as is described above
(\prettyref{ex:verb-inflection.infinitive-1}).
Note that the subject-verb agreement is missing in all nonfinite clauses,
be it the third person singular \form{-s} or inflectional forms of \form{be}.

\begin{table}[H]
    \caption{The order of auxiliaries and some examples}
    \label{tbl:auxiliary-chain}
    \centering
    \begin{tabular}{@{}lllll@{}}
    \toprule
    \category{modal}      & \category{perfect}      & \category{progressive}        & \category{passive}            & main verb    \\ \midrule
    \form{}           & \form{}             & \form{}                   & \form{}                   & \form{takes}  \\
    \form{}           & \form{}             & \form{}                   & \form{am/are/is/was/were} & \form{taken}  \\
    \form{}           & \form{}             & \form{am/are/is/was/were} & \form{}                   & \form{taking} \\
    \form{}           & \form{have/has/had} & \form{}                   & \form{}                   & \form{taken}  \\
    \form{}           & \form{have/has/had} & \form{been}               & \form{being}              & \form{taken}  \\
    \form{will/would} & \form{have}         & \form{been}               & \form{being}              & \form{taken}  \\ \bottomrule
    \end{tabular}
\end{table}

\begin{exe}
    \ex\label{ex:verb-inflection.3sg-1} \begin{xlist}
        \ex I [like] this.
        \ex He [likes] this.
    \end{xlist}
    \ex\label{ex:verb-inflection.infinitive-1} 
    The award is reported [to have been being taken]_{\text{complement clause: \formcat{to}-infinitive}} 
    \ex\label{ex:verb-inflection.infinitive-2} 
\end{exe}



\subsection{\form{Do} insertion}\label{sec:verb-inflection.do}

\subsubsection{Obligatory \form{do} insertion}

\form{Do} insertion happens in two circumstances.
The first is we need an auxiliary but there isn't one.
This is the case when we negate a clause with no auxiliary verb
(\prettyref{sec:verb-inflection.negation}),
and the case when subject-auxiliary inversion happens but there is no auxiliary verb
(\prettyref{sec:sai}).
In both cases, \form{do} is inserted before the main verb,
and is regarded as an auxiliary,
which carries the tense feature and the subject-verb agreement information
and is inflected accordingly
(\prettyref{ex:verb-inflection.do-1}, \prettyref{ex:verb-inflection.do-2}).

We may say the \form{do} is the default realization of the tense category and the agreement 
when these can't find an appropriate host.
It's roughly in the same position of \category{modal} in \prettyref{tbl:auxiliary-chain}.
Then, expectedly, adverbs can be inserted between \form{do} and the main verb
(\prettyref{ex:verb-inflection.do-3}).

\begin{exe}
    \ex\label{ex:verb-inflection.do-1} I do not like the gift. I don't like the gift.
    \ex\label{ex:verb-inflection.do-2} Did he enter the room that night?
    \ex\label{ex:verb-inflection.do-3} I do not particularly like that kind of flower.
\end{exe}

\subsubsection{\form{Do} for emphasis} 

Unlike (\prettyref{ex:verb-inflection.do-1}, 
\prettyref{ex:verb-inflection.do-2}, 
\prettyref{ex:verb-inflection.do-3}),
we can also just insert \form{do} to emphasize on the action,
and in this case the inserted \form{do} receives stress.
The morphology of \form{do} is the same as the obligatory \form{do} insertion,
and so is the distribution of adverbs.

\begin{exe}
    \ex Your company [\emph{do}]_{\text{\form{do} insertion}} [have]_{\text{main verb}} lots of rules!
\end{exe}

\subsection{Adverbs in the auxiliary chain}\label{sec:verb-inflection.adverb-auxiliary-chain}

The adverbs mentioned in this section are 
manner-like adverbs, \acs{tam}-related adverbs and speech act-related adverbs 
(\prettyref{sec:valency.overview.dependents}),
instead of adverbial peripheral arguments.
Adverbs are never inserted between the first auxiliary (if any) and the negator.
TODO: what else?

\begin{exe}
    \ex\label{ex:auxiliary-chain-breaking-1} 
    He [is]_{\text{\category{progressive}}} [vigorously]_{\text{TODO:}} [doing]_{\text{main verb}} [his job]_{\text{object}}. 
\end{exe}

\subsection{Negation in the auxiliary chain}\label{sec:verb-inflection.negation}

The rule of the negator \form{not} is close to the rule of adverbs: 
If \form{not} is used, it is \emph{always} after the first auxiliary
(while adverbs can appear before the first auxiliary in marked cases), 
which may be the inserted \form{do} (\prettyref{ex:auxiliary-chain-breaking-2}).
Any auxiliary-\form{not} sequence may be replaced by the negative form of that auxiliary
if there is one (\prettyref{ex:auxiliary-chain-breaking-4}, \prettyref{ex:auxiliary-chain-breaking-3}).

\begin{exe}
    \ex\label{ex:auxiliary-chain-breaking-2}
    He [does]_{\text{\form{do} inserted, pres, 3sg}} [not]_{\text{negation}} love his job.
    \ex\label{ex:auxiliary-chain-breaking-4}
    He doesn't love his job.
    \ex\label{ex:auxiliary-chain-breaking-3}
    He isn't vigorously doing his job.
\end{exe}

It should be noted the surface position of the negator doesn't determine the scope of negation
\citep[\citepage{668}]{cgel}.
See, for example, the ambiguity of (\prettyref{ex:verb-inflection.negation-ambiguity-1}).
Here the ambiguity is an indicator that 
there are at least two available syntactic position of the reason clause (TODO: ref).
Another ambiguity arises when negation appears together with modality
(\prettyref{ex:verb-inflection.negation-ambiguity-2}, \prettyref{ex:verb-inflection.negation-ambiguity-3}).
This means the negator-after-first-auxiliary rule is about \emph{realization} 
and not about the underlying syntactic structure (\prettyref{sec:morphology-meaning}),
if we assume the semantic difference has structural significance.
This, together with the fact that auxiliaries have negative forms
and that the existence of \form{not} blocks subject-auxiliary inversion 
of the main verb,
may lead to the conclusion that the negator \form{not} is a quasi-verbal clitic
which is always attached after the highest verbal element.
We, however, shouldn't rush to such a conclusion,
because it's also possible that 
the rule is actually the highest verbal element is always moved \emph{before} the negator.
Note that 

TODO: the Tense - Negation - Modality - Perfect - ... sequence

\begin{exe} 
    \ex\label{ex:verb-inflection.negation-ambiguity-1} 
    I don't appoint him because he is my son. \\
    \translate{I appoint him, but because of his talent, not because his relation with me. / 
    I don't appoint him, because he's my son and I don't want to appoint him and  
    leave a bad impression on my colleagues.}
    \ex\label{ex:verb-inflection.negation-ambiguity-2}
    He shouldn't play football in the streets. \\
    \translate{It's required that he doesn't play football in the streets./
    *It's not required that he plays football in the streets,
    but he can if he wants to.} 
    \ex\label{ex:verb-inflection.negation-ambiguity-3}
    He can't play football. \\
    \translate{It's not possible/permitted that he plays football./
    *He can suppress the desire to play football.}
\end{exe}   


\subsection{Subject-auxiliary inversion}\label{sec:sai}

In interrogative sentences and in other cases (\prettyref{sec:clause-template}),  
the first auxiliary in the chain undergoes leftward movement,%
\footnote{
    Note that this is head movement and are often attributed to post-syntactic operations 
    in Distributed Morphology,
    making the operation kind of ``morphological''.
    See \prettyref{sec:morphology-meaning} for theoretical issues concerning this.
}
often to the initial position but may be preceded by preposed constituents (\prettyref{sec:clause-template}). 
This is called \concept{subject-auxiliary inversion}.
When there is no auxiliary, 
the correct form of \form{do} carrying the tense and agreement features is inserted.

\begin{exe}
    \ex {} [Do]_{\text{inverted auxiliary}} [you see my umbrella]_{\text{nucleus}}
    \ex Only then do we cook
\end{exe}


\subsection{Semi-auxiliaries}\label{sec:semi-auxiliary}

\subsection{Comparability with moods}\label{sec:tam-mood-compatibility}

\section{Clausal dependents and verb frames}

\subsection{Overview}

\subsection{Prepositional object}

Verb-preposition constructions and verb-particle constructions
can be classified according to the following parameters:
\begin{enumerate*}
    \item whether it's a transitive preposition or a particle 
    (an intransitive preposition, or something else),
    \item whether the construction can be interpreted in a compositional way or 
    has already gained an established (idiomatic) meaning,
    \item how the choice of preposition/particle is restricted by the verb,  
    \item the mobility of the preposition/particle 
    in, say, \formcat{wh}-movements, and 
    \item complement-related properties of the associated \acs{np} coming with the preposition/particle,
    like whether it can be passivized
\end{enumerate*}
\citep[\citepages{272-274}]{cgel}.

\begin{infobox}{Intransitive prepositions}{intrans-prep}
    The term \term{particle} here covers intransitive prepositions;
    the term \term{preposition} is used to cover transitive prepositions.
    Although strictly speaking, 
    this terminology confuses form and function 
    (prepositions are a word class, 
    and can be used intransitively in some cases),
    I choose to do so to keep the notation consistent with 
    the current grammar writing practice.
\end{infobox}

Concerning verbs coming with a single preposition,
trivially, if a verb doesn't specify the preposition following it, 
the preposition is always mobile.
Thus we have a three-fold classification:
\begin{enumerate*}
    \item verbs with non-specified prepositions,
    \item verbs with specified but mobile prepositions 
    (\concept{preposition verbs with mobile prepositions}; \citealp[\citepage{273}]{cgel}), and 
    \item verbs with specified and fixed prepositions 
    (\concept{fossilized preposition verbs}; \citealp[\citepage{277}]{cgel}).
\end{enumerate*}
Note that a non-specified prepositional phrase is still a complement \citep[\citepage{273}]{cgel}.

The parameters of established meaning and complement properties
are largely independent to the classification made above.
Passivization is completely not predictable 
from the classification made above \citep[\citepage{276} {[11]}]{cgel}.
Fossilized verb-preposition constructions are usually idioms,
but some, like \form{break with}, 
still have largely inferrable meaning;
the same applies for verbs with specified prepositions
(indeed, the presence of a specified preposition introduces a sense of 
directed volition \citep[\citepage{293}]{dixon2005semantic});
verbs with non-specified prepositions usually are less idiom-like,
but this is because if they are idiomatic enough,
we will recognize them as verbs with specified prepositions.

The classification of verb-preposition constructions,
therefore, is given in \prettyref{tbl:verb-prep-construction}.
The examples used in the table is based on \citet[\citepage{278}, {[17]}]{cgel}.

\begin{table}[H]
    \centering
    \caption{Classification of verb-single-preposition constructions}
    \label{tbl:verb-prep-construction}
    \begin{tabular}{lllll}
    \toprule
    &       &       & \multicolumn{2}{c}{specified preposition} \\ \cmidrule(l){4-5} 
    \multirow{-2}{*}{\begin{tabular}[c]{@{}l@{}}passivization of NP \\ after preposition\end{tabular}} & \multirow{-2}{*}{idiom} & \multirow{-2}{*}{\begin{tabular}[c]{@{}l@{}}non-specified \\ preposition\end{tabular}} & mobile             & fixed                \\ \midrule
                                                                                                       & yes                     & \cellcolor[HTML]{C0C0C0}\form{}                                                      & \form{call on?}          & \form{see to}      \\
    \multirow{-2}{*}{yes}                                                                              & no                      & \form{sleep in}                                                                      & \form{refer to}  & \form{fuss over}            \\
                                                                                                       & yes                     & \cellcolor[HTML]{C0C0C0}\form{}                                                      & \form{stand for} & \form{come across} \\
    \multirow{-2}{*}{no}                                                                               & no                      & \form{fly to/from}                                                                   & \form{feel for}          & \form{come into}            \\ \bottomrule
    \end{tabular}
\end{table}

Beside the classification given by \prettyref{tbl:verb-prep-construction},
another parameter is the origin of preposition verb constructions.
Some of them are similar to verbs licensing oblique cases
(found in languages with rich case morphology, like Latin),
like \form{refer to},
the verb parts of which rarely appear alone or with other prepositions.
For others, like \form{see to},
the verb part of the construction (usually a simple, monosyllabic one)
does appear alone or with other prepositions.
In the first case,
the ``idiom-or-not'' parameter is actually not so important,
because we can consider the preposition as a part of the verb lexeme,
while in the second case,
the parameter is important,
because \form{stand at the door} is of course not idiomatic,
while \form{stand for} has an established meaning.

The complement introduced by the preposition of a preposition verb 
is object-like (TODO: ref),
and therefore preposition verbs are transitive
(\citealt[\citepage{291}, \citepage{297}]{dixon2005semantic};
\citealt[\citepage{277}]{cgel}).

\section{Agreement}

If you take a closer look to how native speakers of English do subject-verb agreement,
you'll find some more subtle details than 
the textbook rule that when the tense is \acl{present}
and the subject is 3sg, 
\form{-s} is added to the first auxiliary or the main verb
\citep[\citechap{5}, \citesec{18}]{cgel}.

\chapter{Simple clause}

\bibliographystyle{plainnat}
\bibliography{cambridge}

\end{document}