\documentclass[UTF8, a4paper, oneside, scheme=plain, 12pt]{ctexrep}

\usepackage{libertinus}
\usepackage{geometry}
\usepackage{float}
\usepackage{titling}
\usepackage{titlesec}
\usepackage{paralist}
\usepackage{footnote}
\usepackage[inline]{enumitem}
\usepackage{amsmath, amssymb, amsthm}
\usepackage{gb4e}
\noautomath
\usepackage{bbm}
\usepackage{textcomp}
\usepackage{soul}
\usepackage{graphicx}
\usepackage{siunitx}
\usepackage[table,xcdraw]{xcolor}
\usepackage{tikz}
\usepackage[ruled, vlined, linesnumbered, noend]{algorithm2e}
\usepackage{xr-hyper}
\usepackage[colorlinks, citecolor = purple]{hyperref} % linkcolor=black, anchorcolor=black, citecolor=black, filecolor=black
\usepackage[most]{tcolorbox}
\usepackage{caption}
\usepackage{subcaption}
\usepackage{booktabs}
\usepackage{multirow}
\usepackage[figuresright]{rotating}
\usepackage{acro}
\usepackage[round]{natbib} 
\usepackage{nameref,zref-xr}
\zxrsetup{toltxlabel}
\zexternaldocument*[alignment-]{../alignment/alignment}[alignment.pdf]
\zexternaldocument*[exercise1-]{../Exercise/2021-3}[2021-3.pdf]
\zexternaldocument*[method-]{../methodology/glossing}[glossing.pdf]
\usepackage{prettyref}

\geometry{left=3.18cm,right=3.18cm,top=2.54cm,bottom=2.54cm}
\titlespacing{\paragraph}{0pt}{1pt}{10pt}[20pt]
\setlength{\droptitle}{-5em}

\DeclareMathOperator{\timeorder}{\mathcal{T}}
\DeclareMathOperator{\diag}{diag}
\DeclareMathOperator{\legpoly}{P}
\DeclareMathOperator{\primevalue}{P}
\DeclareMathOperator{\sgn}{sgn}
\newcommand*{\ii}{\mathrm{i}}
\newcommand*{\ee}{\mathrm{e}}
\newcommand*{\const}{\mathrm{const}}
\newcommand*{\suchthat}{\quad \text{s.t.} \quad}
\newcommand*{\argmin}{\arg\min}
\newcommand*{\argmax}{\arg\max}
\newcommand*{\normalorder}[1]{: #1 :}
\newcommand*{\pair}[1]{\langle #1 \rangle}
\newcommand*{\fd}[1]{\mathcal{D} #1}

\newcommand*{\citesec}[1]{\S~{#1}}
\newcommand*{\citechap}[1]{Ch~{#1}}
\newcommand*{\citefig}[1]{Fig.~{#1}}
\newcommand*{\citetable}[1]{Table~{#1}}
\newcommand*{\citepage}[1]{p.~{#1}}
\newcommand*{\citepages}[1]{pp.~{#1}}
\newcommand*{\citefootnote}[1]{fn.~{#1}}
\newcommand*{\citechapsec}[2]{\citechap{#1}.\citesec{#2}}

\newrefformat{sec}{\citesec{\ref{#1}}}
\newrefformat{fig}{\citefig{\ref{#1}}}
\newrefformat{tbl}{\citetable{\ref{#1}}}
\newrefformat{chap}{\citechap{\ref{#1}}}
\newrefformat{fn}{\citefootnote{\ref{#1}}}
\newrefformat{box}{Box~\ref{#1}}
\newrefformat{ex}{\ref{#1}}

% color boxes

\tcbuselibrary{skins, breakable, theorems}

\AtBeginEnvironment{infobox}{\small}
\AtBeginEnvironment{theorybox}{\small}

\newtcbtheorem[number within=chapter]{infobox}{Box}{
    enhanced,
    boxrule=0pt,
    colback=blue!5,
    colframe=blue!5,
    coltitle=blue!50,
    borderline west={4pt}{0pt}{blue!65},
    sharp corners,
    fonttitle=\bfseries, 
    breakable,
    before upper={\parindent15pt\noindent}}{box}
\newtcbtheorem[number within=chapter, use counter from=infobox]{theorybox}{Box}{
    enhanced,
    boxrule=0pt,
    colback=orange!5, 
    colframe=orange!5, 
    coltitle=orange!50,
    borderline west={4pt}{0pt}{orange!65},
    sharp corners,
    fonttitle=\bfseries, 
    breakable,
    before upper={\parindent15pt\noindent}}{box}
\newtcbtheorem[number within=chapter, use counter from=infobox]{learnbox}{Box}{
    enhanced,
    boxrule=0pt,
    colback=green!5,
    colframe=green!5,
    coltitle=green!50,
    borderline west={4pt}{0pt}{green!65},
    sharp corners,
    fonttitle=\bfseries, 
    breakable,
    before upper={\parindent15pt\noindent}}{box}

% Shorthands
\newcommand*{\concept}[1]{\textbf{#1}}
\newcommand*{\term}[1]{\emph{#1}}
\newcommand{\form}[1]{\emph{#1}}

\newcommand{\redp}{\textasciitilde}

\newcommand{\deictictime}{T$_{\text{d}}$}
\newcommand{\referredtime}{T$_{\text{r}}$}
\newcommand{\orientationtime}{T$_{\text{o}}$}

\DeclareAcronym{blt}{short = BLT, long = Basic Linguistic Theory}
\DeclareAcronym{cgel}{short = CGEL, long = The Cambridge Grammar of the English Language}
\DeclareAcronym{dm}{short = DM, long = Distributed Morphology}
\DeclareAcronym{tag}{long = Tree-adjoining grammar, short = TAG}
\DeclareAcronym{sfp}{long = sentence-final particle, short = \textsc{sfp}}
\DeclareAcronym{np}{long = noun phrase, short = NP}
\DeclareAcronym{vp}{long = verb phrase, short = VP}
\DeclareAcronym{pp}{long = preposition phrase, short = PP}
\DeclareAcronym{advp}{long = adverb phrase, short = AdvP}
\DeclareAcronym{cls}{long = classifier, short = CLS}
\DeclareAcronym{dist}{long = distal, short = DIST}
\DeclareAcronym{prox}{long = proximate, short = PROX}
\DeclareAcronym{dem}{long = demonstrative, short = DEM}
\DeclareAcronym{classify}{long = classifier, short = \textsc{cl}}
\DeclareAcronym{dur}{long = durative, short = DUR}
\DeclareAcronym{neg}{long = negative, short = \textsc{neg}}
\DeclareAcronym{cc}{long = copular complement, short = CC}
\DeclareAcronym{cs}{long = copular subject, short = CS}
\DeclareAcronym{tam}{long = {tense, aspect, and mood}, short = TAM}
\DeclareAcronym{past}{long = past, short = PST}
\DeclareAcronym{nonpast}{long = non-past, short = NPST}
\DeclareAcronym{present}{long = present, short = PRES}
\DeclareAcronym{progressive}{long = progressive, short = \textsc{poss}}
\DeclareAcronym{perfect}{long = perfect, short = \textsc{perf}}
\DeclareAcronym{passive}{long = passive, short = \textsc{pass}}
\DeclareAcronym{copula}{long = copula, short = COP}
\DeclareAcronym{possessive}{long = possessive, short = \textsc{poss}}
\DeclareAcronym{coca}{long = Corpus of Contemporary American English, short = COCA}

\newcommand{\asis}[1]{\textsc{#1}}
\newcommand{\oneof}[1]{{#1}}
\newcommand*{\homo}[2]{#1$_{\text{#2}}$}
\newcommand{\category}[1]{\textsc{#1}}
\newcommand{\formcat}[1]{\textsc{#1}}
\newcommand{\emptymorpheme}{$\emptyset$}
\newcommand*{\fromto}[2]{\langle {#1}, {#2} \rangle}

\newcommand{\alignment}{\href{../alignment/alignment.pdf}{my notes about alignment}}
\newcommand{\exerciseone}{\href{../Exercise/2021-3.pdf}{this exercise}}
\newcommand{\method}{\href{../methodology/glossing.pdf}{this note about my understanding of descriptive grammars}}

\newcommand{\ala}{à la}
\newcommand{\translate}[1]{`#1'}
\newcommand{\vP}{\textit{v}P}

% Make subsubsection labeled
\setcounter{secnumdepth}{4}
\setcounter{tocdepth}{4}
% reset example counter every chapter (but do not include the chapter number to the label)
\counterwithin{exx}{chapter} 

% Reference formats
\renewcommand{\bibname}{References}
\setcitestyle{aysep={}} 

% List format
\setlist[enumerate,1]{label=\alph*\upshape)}

\title{Aspects of English morphosyntax}
\author{Jinyuan Wu}

\begin{document}
    
\maketitle

\automath

\tableofcontents

\chapter{Introduction}

\section{The language and the speakers}

\section{Theoretical orientation}

{\small

In short, this note is based on Basic Linguistic Theory 
\citep{dixon2009basic1,dixon2010basic2,dixon2012basic3}
with generative flavor.
It would cost too much space to given an outline of the descriptive framework here; 
TODO: turn note-1 into a note about formalism

\subsection{Formalism}

\subsubsection{Constituency and dependency}

\subsubsection{Form and function}

Although we have already made clear terminological distinction between form and function 
in the framework we use in this note, 
in typological literature this is not always the case. 
For example, the labels in \prettyref{fig:np-template} are about function and not form;
but in typology, we use the symbols Dem, Num, A, and N to represent
their prototypical syntactic function in \acs{np}s:
Dem = the determiner-like region, 
A = the attributive region, etc. 
Saying English has a Dem Num A N constituent order 
doesn't mean the attributives are all adjective phrases.
This misuse of terminology may also be kind of justified,
since part of speech tags are defined by syntactic functions anyway
(\prettyref{sec:theory.pos}); 
although the ability to be an attributive isn't truly substantial 
for adjective-ness.

\subsubsection{Fossilization}

Some constructions are only semi-analyzable -- 
they are fossilized.
Fossilization has several stages:
a fossilized construction created by a fossilized process 
be totally historical and no longer analyzable in contemporary morphosyntax,
or analyzable in contemporary morphosyntax but already with an established meaning,
or compositionally analyzable but the relevant process is limited in productivity.
The parameter of morphosyntactic fossilization 
and the parameter of semantic establishment,
despite having strong connections to each other,
are still two parameters.

\subsection{Part of speech}

\subsubsection{The word}\label{sec:theory.word}

Wordhood can be divided into phonological and grammatical wordhood.
The latter can be divided into syntactic and morphological wordhood.
A syntactic word is just a mini-constituent,
like nP (as opposed to, say, a DP).
Thus, in this note I refrain from discussing whether a function item 
like \form{a}, \form{the}, etc. 
is a grammatical word or not in detail,
since it's the realization of one or more functional heads
and is not the realization of a maximal projection.
A function item licensed in a phrase
may be phonologically dependent -- we say it's a clitic,
while a functional item licensed in a grammatical word, 
like \form{-like} in \form{Distributed Morphology-like},
may sometimes has the status of an independent phonological unit.
Thus when talking about wordhood of a function item,
it's always better to specify whether we are talking about its 
licensing position or its phonological independence.

The syntactic wordhood has some vagueness:
the meaning of \term{constituent} is not determined.
Recall that some authors use the term to refer to a \emph{phase} in generative syntax,
and thus a verbal complex is a ``verb phrase'' 
and a inflected verb is a mini-constituent (i.e. an inflectional version of the verb phrase).
But if we insist on using the term \term{constituent} 
to refer to a constituent in the generative sense, 
then an inflected verb is not recognized as a word, 
but always a word plus some auxiliaries -- 
indeed this is the perspective taken in traditional Japanese grammar,
even though these auxiliaries are tightly attached to the verb root (see below)
and are unable to undergo, say, English auxiliary fronting.

Syntactic wordhood is not enough for wordhood definition:
If we only focus on syntactic wordhood 
we may say \form{has being eating} is a word,
since it's the exponent of the verbal complex;
to capture the intuition that \form{has being eating} has several words, 
we define the concept of morphological word:
if functional items pertaining to a root 
are tightly attached to that root, 
then they form a morphological word,
while in \form{has being eating},
we can insert an adverb between the auxiliaries and \form{eating},
so there are three morphological words.
But whether a function item is firmly attached to the root it modifies is sometimes hard to say 
and observes diachronic changes. 
An example is whether spoken French has polysynthetic structures.
Another example is about Japanese, 
in which historical auxiliaries seem to already fossilized into inflectional endings,
as an inflectional template seems to have already formed
(although these auxiliaries still have their own ``inflections'',
just like English auxiliaries).

Another criterion frequently invoked is that 
a word has an established meaning.
This however is better understood as a tendency:
some phrases also have established meaning.
It seems words have established meanings not because 
wordhood enjoys a special position a priori in grammar, 
but because a word is a small unit 
and small units tend to be fossilized both in meaning and in structure.

\subsubsection{The meaning of ``morphology''}

As is said above, the term \term{morphology}
involves both the internal structure of the word 
(``syntax within the word'')
and how this structure is realized.
Traditional grammars usually have a large paradigm
with its row and column headers being grammatical categories.
(When there are too many categories 
-- and in this case the language in question is usually agglutinative -- 
the paradigm will be unbearably large, 
and another way -- like the School Grammar of Japanese -- is needed to cover verb inflection.
Still, partial paradigms are useful in this case.) 
This is a morphosyntactic way to represent the inflection of a word, 
but if we are talking purely about the \emph{morphological} part
(i.e. how grammatical relations and categories are realized),
then it's sometimes not necessary to recognize so many forms:
If a verb appears exactly the same in two different syntactic environments,
then we say there is only one \emph{inflectional form} of that verb.
For languages like Latin, 
the traditional large-paradigm way is handy,
while for English, we can zip the paradigm severely \citep[\citechapsec{3}{1.2}]{cgel}.

Note that the fact that the realization of a mini-phrase 
may not absolutely transparently reflect its inner structure 
means that even though the morphology of a word 
seems quite ``concatenative'',
it's possible that, say, 
a grammatical category is realized by two or even more affixes, 
both of which are obligatory,
not because of any morphosyntactic reason, 
but because of a template of the word. 
In this case we say the language has template morphology;
on the other hand, 
when the affixes reflect the abstract structure of the word clearly 
and can be seen to be added one by one 
according to the functional hierarchy,
we say the language has layered morphology.
Still, it seems that template morphology
can be well captured by primitives that are already well known 
in languages with layered morphology \cite{oxford2019fission},
so the distinction between the two seems to be smaller than it appears to be.

\subsubsection{Part of speech tags}\label{sec:theory.pos}

The variance of parts of speech observed cross-linguistically
is often used to support the idea that 
concepts like ``noun'' or ``verb'' 
are all language-specific
(and also to attack the idea of language universals and ``formalists'').
Following the idea of lexical decomposition (a certain brand of ``formalist'' linguistics),
the position of this note is to regard ``nominal properties''
(i.e. being in the center of the nominal system (\prettyref{sec:theory.nominal}))
and ``verbal properties''
as universal concepts.
That's to say, I assume that we can take noun phrases and clauses
with functional hierarchies similar to well-known languages 
for granted;
the part of speech tags like ``noun'' or ``verb''
then are about how a root interacts with these syntactic environments,
like whether and how a root can be spelt out 
with the nominal categorizer or the verbal categorizer in Distributed Morphology. 
Part of speech classes then are epiphenomena 
of interaction between roots without part of speech labels  
and functional hierarchies.

Still, part of speech tags are useful in language description.
With respect to the parameters of 
openness, the content-function dichotomy,
and whether function items are similar to content items enough,
there are roughly four types of parts of speech:
\begin{itemize}
    \item Open content categories with clear part of speech labels like noun, verb or adjective.
    \item Close content categories with clear part of speech labels like noun, verb, or adjectives.
    \item Close grammatical categories which may still be seen as noun-like or verb-like. 
    The prototypical members of this type are pro-forms.
    \item Close grammatical categories like particles or affixes that don't really need part of speech labels.
    Grammatical words in this type -- the type without much resemblance to prototypical content words -- 
    don't really need part of speech tags,
    but since they are surface realizations of different grammatical categories,
    and exponents of different values of the same set of categories 
    usually have similar distributions,
    dividing these items into groups helps us to organize the grammar,
    though unlike labels like noun or verb,
    these labels have less substantiality in the mind of native speakers.
    For example, in a language with case particles,
    we may set up a word class called ``case particles'',
    which falls under the class of ``particles''.
    This is the case for traditional Japanese grammar.
    Then we can look up for an unfamiliar grammatical construction 
    in a ``grammar dictionary''.
\end{itemize}

The third and the fourth classes are usually primarily exponents of functional heads.
A word in the third class may contain some features making it look like a word from the first two classes,
like the categorizer feature or the person/number/case feature,
while a word in the fourth class may not.

Gradience occurs between the boundaries of the four types of parts of speech.
Fillers of a specifier position,
if limited to a few (like adverbs filling certain positions in TP),
may be bleached into realization of functional heads,
thus coming into the third and fourth types of parts of speech.
Competing analyses occur when this change is happening.
The boundary between the latter isn't clear,
and neither is the boundary between the first type and the second type
(because many so-called closed content word classes 
sporadically accept new members),
and the second type and the third type
(for example, when the number of verbs is so-limited -- say, only a handful -- 
then is it a better idea to regard the verb class as the exponents of different light verbs?),
and also the third type and the fourth type
(since the criteria of ``looking like a noun or a verb'' are never clear).

It's often said that nouns are about objects, 
verbs are about events,
and adjectives are about properties.
This mapping from word class to semantics is a coarse one 
and should be refined for more systematic description of languages.

It should be noted that actions and processes can be conceptualized as objects:
We have, for example, \form{his playing of the national anthem},
where \form{play} is nominalized into \form{playing}
(this is not an \formcat{ing}-participle -- see TODO: ref).
The boundary between objects and properties is also hard to draw:
Apart from pronouns or demonstratives 
that directly refer to the conversation context 
and pull out a specific object from the listener/reader's memory
(and therefore introduce a free variable in the semantic interpretation of the utterance),
nouns denote \emph{sets of objects},
and we know we can have a one-to-one mapping 
between a set $A$ and a predicate in the form of $\cdot \in A$:
A noun like \form{toothbrush} 
can be immediately mapped to an adjective, like \form{toothbrush-like},
and therefore whether \form{toothbrush} \emph{really} means 
a set or a predicate becomes an undecidable question.
And similarly a property can also be conceptualized as an object: 
as a set, or maybe as a kind of ``essence''
(compare \form{the ice harvesters are [manly]_{\text{about property}} men}
and \form{the ice harvesters have strong [manliness]_{\text{object?}}}).

The boundary between actions and properties is also not clear:
In traditional grammar they are all called ``predicates''.
When translated into logical expressions,
a clause about an event introduces an event argument,
but a clause about a property doesn't.
The difference between events and properties may be that
while a clause about the fact that an object has a property 
can be simply interpreted as \translate{property($x$)},
or if we want to reduce the number of logical predicates 
(to avoid the necessity of using higher-order logic),
\translate{$x \in \text{the-set-with-the-property}$}.
But of course having a property is temporal,
so \form{this is beautiful} is to be interpreted as 
\translate{$\exists e (\text{time}(e, \text{speech-time}) \land 
\text{$\in$-in-a-time}(x, \text{what-is-beatuful}, e) )$}.
This doesn't seem quite different from the meaning of a clause about an event,
although there may still be some subtle differences like 
whether the $e$ in clauses about properties can be the invisible topic, 
which seems to be the reason for the mysterious \form{wa}/\form{ga} alternation in Japanese
\citep{heycock2008}.

So in the end,
nouns are prototypically about objects,
and they may denote events, 
and whatever they denote, they can be thought as properties in semantic interpretation.
What makes a root a noun 
is essentially its \emph{syntactic environment}.
Intuitively, we say nouns are similar to pronouns, demonstratives, etc., 
which, however, can never be interpreted as properties.
That's because these words contain a determiner fused inside 
and therefore have clear and direct reference (\prettyref{sec:np.fused-head}),
and the reason we say nouns are like them 
is that a noun can be placed at the center of a DP, 
and the DP now has almost the same syntactic distribution with pronouns, etc. 

Semantically, verbs are about actions and properties,
but again, they are \emph{categorized} as verbs 
not because of inherent semantic properties, 
but by the syntactic environment
(being immersed in the \vP-TP-CP projections).
Thus, in principle we don't really need content words other than nouns,
which is indeed the case in some languages
with very limited verb classes.

The semantic function of adjectives can in principle always be realized by nouns or verbs.
The role of them is highly language-specific,
and they appear when a meaning is hard to convey using existing constructions concerning nouns or verbs.
For example, in English, when it comes to gradience of properties
(manifested in comparative constructions),
nouns are of limited use,
so adjectives are indeed necessary.
But we still have \form{he's more a scientist than a public health official},
and the adjective version \form{he's more scientist-like than public health official-like},
despite being grammatical, is awkward
and only appears in language games instead of natural, everyday speech.

Note that the above discussion may be well found in an introduction to, say, 
``Radical Construction Grammar'',
in which it's argued that grammatical categories only make sense 
in constructions.
This note, however, takes the stance that 
constructions are not routinized structureless strings,
but are made of building blocks that are subject to universal constraints.
It's my theoretical assumption that a root being categorized into a noun 
has nothing substantiality different from adding an article before a noun,
although when it comes to processing,
the first may be more ``automated'' and ``fossilized''.

\subsubsection{Derivation and inflection}

The accepted wisdom is ``derivation relates one lexeme to another lexeme,
while inflection relates one lexeme to its form in the final utterance''.
This definition however still has intrinsic vagueness.
It involves two parameters:
structure size
and fossilization.
Regarding structure size,
derivation is on the level of lexemes,
which are smaller than phrases,
and also smaller than ``finished words'' -- inflected words -- 
that involve influences from the external syntactic environment.
For verbs, for example, we may define derivation and inflection like the follows: 
derivation means everything in the verbal categorizer phrase,
while inflection means everything in the TP
(basically, the morphological counterpart of Dixon's verb phrase).
Regarding fossilization, 
derivation should be less synchronically active than inflection
(and therefore less productive).
There is certain amount of correlation between the two parameters:
small units are easier to lexicalize, 
so compound words are more likely to have established meanings,
and compounding is therefore prototypically derivational.
But of course the parameters may not always agree with each other,
and both parameters have vagueness.
The derivation-inflection distinction is therefore not appropriate 
in some cases \citep[\citepage{221}]{dixon2009basic1}.

Here are some examples of the vagueness.
Concerning structure size, in case stacking, 
should the inner case markers be considered as inflection?
And note that a valency changing device also doesn't apply to all verbs that seem to have 
an appropriate number of arguments --
indeed, \citet{jacques2021grammar} calls valency changing \term{derivation}.
A further subtlety is the parameter of fossilization should be further split into two:
some constructions may appear less frequently but still have largely compositional meaning,
while others -- like the Latin \form{com-} prefix -- 
appear everywhere but the meanings of resulting words
can hardly be inferred regularly.

\subsection{The nominal system}\label{sec:theory.nominal}

\subsubsection{The nominal or the NumP-like region}

There usually is a nominal part in the nominal system,
which is basically the central noun plus modifiers
and corresponds to the NumP domain.
The NumP itself is similar to the role of TP,
and the various adjectives are similar to specifiers of AdvPs in the TP domain,
and the complementation is similar to the VP layer
\citep{laenzlinger2017view}. 

\subsubsection{The determiner-like region}

The higher determiner-like region corresponds to the DP domain,
which contains D_{\text{det}} (the DP version of FinP; \prettyref{sec:np.det.definite}),
the quantifier Q position (\prettyref{sec:np.det.quantifier}),
which is placed over the determiner \citep{gianollo2021reference},
the DP version of topic (\prettyref{sec:np.det.specific}),
and the D_{\text{deixis}} position (\prettyref{sec:np.det.deixis}),
which corresponds to the ForceP in the CP domain
and is about whether the DP is referential, etc.,
quite similar to the way ForceP expresses what the clause is intended for
\citep{laenzlinger2005french}.
Note that in English we don't have \acs{np}-inside topics and focuses,
between D_{\text{deixis}} and D_{\text{det}},
although some Romance languages allow them.

Though quantifiers are often seen inside \acs{np}s,
their semantic scopes are definitely larger.
This is TODO

It should be noted that when studying the structure of DP,
we should distinguish semantics and syntax.
The meaning of \emph{determination} may be realized by something like the English articles,
but it may also be realized by something that looks very like an adjective,
as is the case in Latin.
What is uncontroversially universal is a set of atomic syntactic features 
and related semantic meanings, 
not how they are packaged into concrete constructions,
and whether \acs{np}s and clauses follow the same cross-linguistic template
is still a disputed problem 
(see the info box in \href{./note-1.pdf}{note 1}).

\subsection{The verbal system}

\subsubsection{The verbal complex}\label{sec:vp-tp-cp}

Roughly speaking, in \prettyref{fig:clause-template},
the verb phrase is the part of TP that is lower than the projection in which the subject is introduced.
The subject-predicate structure is roughly the complete TP.
Layer 3 and layer 4 are about CP.

TODO: argument structure, three step

}

\chapter{Parts of speech}


This note is organized in a bottom-up manner and I start with 
form classes -- or parts of speech, using terms in traditional grammar -- in English.
This chapter covers roots, derivational processes of them,  
and the phrases the roots project into.
In principle, the structures of phrases can be covered when discussing their head words,
so in some traditional grammars,
all morphosyntactic information 
-- i.e. things not about phonology, the writing system, cultural background, etc. -- 
falls under the part about ``parts of speech''.
The structures of the extended phrases of nouns and verbs however are too large 
to be placed into one chapter:
They are the main focus of the following several chapters.

Since characterization of parts of speech and their contexts unavoidably involves 
information about larger constructions like \acs{np}s or clauses,
forward references to the following chapters are to be expected in this chapter.

\section{Wordhood in English}\label{sec:pos.wordhood}

The orthographical tradition of English contains the notion of word.
Linguistically there are two types of wordhood:
the phonological wordhood (whether word boundaries exist, etc.)
and grammatical wordhood (\prettyref{sec:theory.word}), 
and the latter can be then divided into 
``morphological'' or ``realizational'' wordhood 
(i.e. existence of an obligatory inflectional template, etc.)
and ``syntactic'' wordhood
(i.e. ability to take modifications usually believed to belong to words, etc.).

The English orthographical word is 
a mixture of the phonological word in very slow speech
(i.e. the unit between two natural stops in very slow speech)
and also some factors in grammatical wordhood.
The phonological factors are dominant:
\form{Distributed Morphology-like} is not recognized as a orthographical word,
but at least two orthographical words,
although its outmost constituent is the derivational suffix \form{-like} 
and is not subject to any adjectival modification within the scope of \form{-like},
and therefore is a sub-phrasal unit  
that is smaller than a nominal (\prettyref{sec:np.template})
and is to be recognized as a grammatical word.
English orthography is therefore different from, say, 
German orthography,
where an orthography word usually covers a whole grammatical word 
even when the latter is very complicated.

The orthographical word is also not always necessarily the phonological word 
in more fluent and natural speech, 
where some orthographical words, 
like a personal pronoun immediately after a verb 
and the article \form{the}, 
are essentially clitics when not emphasized.

\section{Distinguishing parts of speech}

\subsection{Open content categories}

Open content categories with clear part of speech labels
include nouns, verbs, adjectives and adverbs.
Since English still has some inflectional morphology,
the class of \concept{countable nouns} can be easily told from others:
When we see the single/plural \form{-\emptymorpheme}/\form{-s} alternation, 
it has to be a countable noun.
The class of \concept{uncountable nouns} can be tell from 
the fact that they appear in similar syntactic environments 
to those of countable nouns,
including adjectival modification,
argument positions,
as well as their inability to be the predicator of a clause.

The verb class can be distinguished by its inflection (\prettyref{sec:verb-forms})
as well as syntactic distribution,
including being only able to receive adverb modification. 

There exist words with both verb and noun versions.
The noun \form{strike} and the verb \form{strike}, for example,
clearly have semantic relation.
There however lacks regular correspondence between the meanings 
the verb and noun versions of these words, 
and no ``verb-used-as-noun'' or similar rules have to be included 
in a synchronic grammar of English.

The English adjective class -- there is only one adjective class, not two or more,
which is the case in Japanese -- 
can be distinguished by 

\subsection{Close content categories}

Unlike Japanese, no close content categories with clear part of speech labels 
like noun, verb, or adjectives exists in English.
Still, some subclasses of verbs seem to be closed TODO: ref.

\subsection{Functional items}

Close function words that still bear clear part of speech tags  
include personal pronouns, TODO

Close grammatical categories like particles or affixes that don't really need part of speech labels
include prepositions (\prettyref{sec:pos.prep}) TODO: list

Division between the two classes 
is sometimes hard to make.
Blurred cases include 
prepositions that are adverb-like and may be regard as lexical heads, TODO

TODO: combining forms \citep[\citepage{1661}]{cgel}


\citet[\citepage{330}]{cgel} introduces the class of determinatives.
A full list of determinatives are given by \citet[\citepage{356}]{cgel}.


\section{History of the lexical inventory of English}

The sources of the lexicon inventory of English 
also helps understand English grammar.

A generalization is Romance affixes are usually added before Germanic affixes,
which seems expected,
because the Germanic affixes are integrated parts of the grammar 
and interact naturally with external environments
(TODO: coordination of affixes, etc.),
while Romance affixes are not,
so Romance affixes are unable to appear at the edge 
between the inner structures of the lexeme and the external morphosyntactic surroundings.

Derivational affixes are easier to borrow than inflectional ones, 
since when borrowing a mini-tree from one language to another, 
the smaller the tree is, 
the better the compatibility will be. 
Indeed, some affixes in English originate from inflectional affixes in other languages, 
but since then are reanalyzed into derivational devices.





\section{Nouns}

\section{Verbs}

\section{Adjectives}


\section{Pro-forms}

\subsection{Personal pronouns}

\subsection{Demonstratives}

\section{Numerals}

There are four types of numerals in English:
the cardinal numerals, 
the ordinal numerals,
the adverbial numerals,
and the multiplicative numerals.
There is no affixational derivation to show the rank or quality of something 
(which is attested in the Latin ordinal numeral plus \form{-āris} derivation):
The meaning is conveyed by 

Cardinal numerals prototypically appear in \acs{np}s,
possibly in a 

\section{Adverbs and adverb phrases}\label{sec:pos.adverb}

Derivation into adverbs is terminal:
No further derivation is possible after that.

\subsection{Distributions}

Adverbs appear at the initial (\prettyref{ex:overview.adverb-1}) or
at the end of a clause (\prettyref{ex:overview.adverb-2}, \prettyref{ex:overview.adverb-5}), 
or in the verbal complex (\prettyref{ex:overview.adverb-3}, \prettyref{ex:overview.adverb-4}).
The verbal complex positions can be further divided into 
the positions after the first auxiliary (\prettyref{ex:overview.adverb-3})
and the position before the main verb (\prettyref{ex:auxiliary-chain-breaking-4}).
Similarly -- though less apparently -- 
there are also two types of clause final adverbs,
one with a pause in utterance, usually shown by a comma (\prettyref{ex:overview.adverb-2}),
the other without a pause (\prettyref{ex:overview.adverb-5}).
All the five positions -- actually there are further subtypes -- have structural and meaning differences
(\prettyref{tbl:clausal-dependent}):
(\prettyref{ex:overview.adverb-1}, \prettyref{ex:overview.adverb-2}) 
are in the speech act-related positions,
(\prettyref{ex:overview.adverb-3}, \prettyref{ex:overview.adverb-5}) 
are in the manner-like positions,
and (\prettyref{ex:overview.adverb-4}) 
is in the \acs{tam}-related region.

\begin{exe}
    \ex\label{ex:overview.adverb-1} [Strikingly], he is a liar.
    \ex\label{ex:overview.adverb-2} He is quite smart, [frankly].
    \ex\label{ex:overview.adverb-5} He finished the task [quite cleverly].
    \ex\label{ex:overview.adverb-3} He might [now] be hoping to skip the test.
    \ex\label{ex:overview.adverb-4} If you were me, you would have [smartly] answered the question.
\end{exe}

\section{Prepositions}\label{sec:pos.prep}

Prepositions are usually regarded as extensions of the case system 
\citep[\citesec{5.4}]{dixon2009basic1}.
From this perspective, in a surface-based analysis,
we should do away with the term \term{preposition phrase},
because in this case prepositions aren't lexical heads
(they are realization of functional heads
and therefore are markers of grammatical relations 
in a surface-oriented analysis).
Some English prepositions can indeed be seen as such,
but there are more complicated cases.
Prepositions can take modifiers (\prettyref{ex:np.pp.ex-1}),
and may be stacked in limited cases (\prettyref{ex:np.pp.ex-2}).
The prepositions in the above cases also 
seem to have lexical meanings instead of mere grammatical functions.
Also, prepositions can be attached to non-\acs{np} constituents \citep[\citepage{609}]{cgel},
The prepositional constructions in English therefore deserve more careful treatment.

\begin{exe}
    \ex\label{ex:np.pp.ex-1} The spot is [[ten meters]_{\text{degree}} [behind]_{\text{preposition}} the house]_{\text{\acs{pp}}}
    \ex\label{ex:np.pp.ex-2} The sample is collected [from under] the glacier.
\end{exe}

TODO: case-like preposition, place preposition, path preposition \citet{spatialpp}

\subsection{Distributions}

TODO: copular PP and adverbial PP

\subsection{Comparison with adjectives and adverbs}

The word \form{worth} has an exceptional property
that makes it similar to prepositions:
It takes an \acs{np} complement directly.
The overall properties however are still adjectival \citep[\citepage{607}]{cgel}.

\begin{exe}
    \ex The paintings are worth thousands of dollars.
\end{exe}
\section{Derivational devices}


In English a relatively clear derivation-inflection distinction can be established,
partly because English inflection has already been simplified.
Some recent works argue that the \form{-ly} suffix for adjective-to-adverb derivation 
should better be considered as an inflection (TODO: ref).
In English clearly inflection always happens after derivation.
In \form{his holier-than-thou attitude} \citep[\citepage{1646}]{cgel}
we find uncontroversial inflection (the comparative \form{-er}) happens 
before uncontroversial derivation,
but since clearly \form{holier-than-thou} is created by a dephrasal derivation,
the inflection step and the derivation step 
are not done when building the same word,
and therefore this example doesn't violate the derivation-before-inflection generalization.

Here I adopt the terminology in \citet{cgel}
and call the most general term for units participating derivation a \term{base}.%
\footnote{
    The term \term{base} may refer to a \emph{form} of base elsewhere.
    In agglutinative languages like Japanese,
    attaching a suffix to an existing unit may slightly changes its tail
    (we may say each suffix carries a morphophonological command at the initial
    dictating this change,
    which reflects historical inflection of the suffixes as 
    morphologically independent auxiliaries),
    and there are a finite number of such changes.
    Thus, we may say ``this unit is conjugated into the 2nd base before accepts that suffix''.
    Here the term \term{base} in Japanese grammar
    means a particular type of ending 
    of a base in this note.
}
A root with no category is a primitive base,
and a fully derived, ready-to-inflect unit is a maximal base,
the latter sometimes being known as a \term{stem}.

Another tendency is compounding happens before derivation.
In constructions like \form{acceptability judgement},
compounding does happen after derivation,
but it's likely to come from the fact that 
compounding here is already in the nominal,
not the head noun.

Derivational affixes are not merely part of speech tags:
There are subtle (yet inferrable, not fossilized) meaning differences between 
\form{healthful} and \form{healthy}.

\subsection{The hierarchy of derivational morphology and order of processes}

Derivation frequently involves conversion of word class of a base.
Both ``symmetric'' and ``non-symmetric'' conversions exist in English.
A symmetric conversion is essentially 
alternation between two possible categorizations of the same category-free root,%
\footnote{
    Still, many roots participating in symmetric conversion 
    have preferred part of speech,
    so people may still informally talk about a \term{noun root} or a \term{verb root}
    even knowing it appears without a category label.
}
while asymmetric conversion involves two part of speech labels,
one added \emph{outside of} another one 
(\citealt[\citepage{1641}]{cgel}; \citealt[\citepage{62}, (15)]{siddiqi2009syntax}).

The symmetric/asymmetric distinction seems to be orthogonal to 
whether the conversion has zero marking.
A symmetric conversion can have explicit marking
(\form{speech} \formcat{n} \translate{the action of speaking} -- 
note that here I'm not talking about the meaning \translate{an event in which someone publicly speaks}), 
and a non-symmetric conversion can have zero marking
(\form{attempt} \formcat{n} \translate{the action of attempting}).



The compatibility between derivational devices involves several factors.
\begin{enumerate*}
    \item Most derivational devices select categorized bases,
    and wrong part of speech tags cause incompatibility.
    Specifically, some affixes are terminal ones:
    The part of speech tags carried by them are never accepted by any other derivational devices.
    Once they are added,
    no further derivation is possible. %Enumerating: all examples 
    \item Some combinations of affixes are not possible because of redundancy:
    the complex \form{??-ness-ful},
    for example, is extremely rare,
    because usually \form{-ness} is attached to an adjective,
    and the whole sequence \form{-ness-ful} therefore adds nothing new semantically.
    The fact that it does appear in manufactured examples, like \form{awkwardnessful},
    to show a sense of cumbersomeness,
    confirms the above claim that its rarity is largely semantically motivated.
    \item Some combinations of affixes are not possible because of realizational reasons:
    \form{*-ic-ly} is not possible 
    because this combination is somehow ``hard to pronounce'' in English,
    and there is no vowel insertion rule pertaining to this configuration to ease the problem.
    We have to add a \form{-al} between the two suffixes.
    On the other hand, although \form{-ize-tion} is also awkward,
    \form{-a-} is inserted, and the resulting \form{-ization} is completely fine.
    \item A variant of the realizational incompatibility problem is 
    not due to phonology, but (possibly hostorical) stylist reasons.
    The combination \form{*-ness-al} isn't good,
    because \form{-ness} is of Germanic origin,
    while \form{-al} is of Romance origin.
    Thus, usually the two affixes are not used together. TODO: why?
\end{enumerate*}

Certain -- although highly limited -- degree of recursion exists in English derivational morphology.
The sequence \form{-ize-tion-al} is an example.
The verb \form{renormalize} is a term in physics,
which means to systematically modify the effective values of constants in a theory 
when throwing away unneeded degrees of freedom 
(and a formally similar procedure used to solve divergence problems).
From it we have \form{renormalization},
and hence the adjective \form{renormalizational},
\translate{having something to do with renormalization,
or is expressed in the theoretical framework of renormalization}.
The form \form{*renormalizationalize} is not acceptable.
This however seems to come from the absence of a feasible meaning.
In a highly marked context,
where we talk about 
``how to make a theory that is usually not written down in the renormalizational framework
renormalizational'',
\form{renormalizationalize} is no longer completely unacceptable,
and hence we have \form{renormalizationalization} or even \form{renormalizationalizational}, etc.
The reason why people decide that such use of derivation is not authentic English is complicated:
It involves the tendency to reject long words, 
rejection of replication,
or lack of established meaning,
or some other factors.



It should also be noted that mismatch 
between the realization and the underlying structure 
is also possible in derivation.
The structure \form{[cross-language]-ic} is realized as 
\form{cross-linguistic}.
Here \form{linguistic} is not to be interpreted as 
``related linguistics'',
but merely a collective realization of the root \form{language} and the adjectivizer \form{-ic},
ignoring the real constituency structure.

Similarly, the validity of \form{Distributed Morphology-like}
means despite being two orthographical (and maybe phonological) words, 
\form{Distributed Morphology} has already collapsed into a noun base.

We however need to be cautious when doing this kind of analysis.
The structure of the form \form{grammatical judgement} is discussed above.
It's tempting to represent the structure of it as 
\form{[[grammati-cal] judge]-ment},
and thus the whole construction is to be recognized as a word.
The problem however is that \form{[[grammati-cal] judge]}
is never attested:
it's completely impossible as a phrase 
and is also not acceptable as a base.
The link between \form{grammatical} and \form{judge-} 
is likely to come from the fact 
that the verb \form{judge} has 
an manner adjunct 
(as in \form{she judges this article scientifically, 
not according to its literature value}),
and after nominalization,
the manner adverbial is 
replaced by a manner attributive,
comparable to the fact that the object of a verb 
becomes a prepositional phrase after nominalization of the verb. 
So \form{grammatical judgement} is not a single noun,
but a nominal created by the nominal attributive construction.
The fact that \form{grammatical judgement} has its established meaning 
of ``judgement of grammaticality''
instead of, say, ``a judgement written grammatically correctly'',
exemplifies the caveat in the theoretical preliminary 
that established meaning is not a deciding factor in wordhood definition.



Then we have marginal cases like \form{generative grammarian},
in which both the analysis \form{[[generative grammar]-ian]_{\text{NP}}} 
and the analysis in which 
the adjective \form{generative} is licensed 
because \form{grammar} licenses an adjunct about its content 
and this adjunct position is still valid with the suffix \form{-ian}.
It's likely that 
this ambiguity is not a theoretical artifact 
but a source of variance in English.

\subsection{Compounding}\label{sec:pos.overview.derivation.compound}

A compound is interpreted as an indication of ``aboutness''
when it is created for the first time.
The word \form{sickbed}, 
for example, of course means a bed related to sickness,
but most compounds have gained established meanings:
\form{sickbed} means hospital beds.


\chapter{Noun phrase}\label{chap:np}


\section{Overview}\label{sec:np.template}

English has a pretty rigid surface constituent order,
directly reflecting the inner structure of \acs{np}s.
Typologically speaking, the constituent order of English is Dem Num A N
(\prettyref{ex:overview.np.1}). 
A more refined template of the noun phrase is given in \prettyref{fig:np-template}.
Note that we may encounter fused-function constructions
(\prettyref{sec:np.fused-head}):
The demonstrative \form{these}, for example, 
may appear as a determiner, 
but may also appear in a fused-function construction,
covering the functions from the head to the determiner
(\prettyref{sec:np.fused-head.dem}).


English \acs{np}s may have -- sometimes must have -- 
one determiner, 
which may be a demonstrative (\prettyref{ex:overview.np.1}),
an article (\prettyref{ex:overview.np.2}),
or others, like a quantifier or a degree expression, 
and several determiners may appear together (\prettyref{sec:np.det}).
The region below the determiner-like region 
is the \concept{nominal} (\prettyref{sec:np.nominal}),
following the notation in \citet[\citepage{329}]{cgel}.
If we remove the determiner(s), 
the bare nominal
can still have limited distribution as an attributive 
(\prettyref{ex:overview.np.3}, \prettyref{sec:np.nominal}),
although not as a full \acs{np} (\prettyref{sec:np.nominal.nominal-attributive}).
A nominal plus its determiner(s) is a minimal \acs{np}.

\begin{exe}
    \ex\label{ex:overview.np.1} He was frightened by [[these]_{\text{Dem}} [three]_{\text{Num}} [ugly]_{\text{A}} [bears]_{\text{N}}].
    \ex\label{ex:overview.np.2} This is 
    [[a]_{\text{article}} book about learning Vim in [[the]_{\text{article}} difficult way]_{\text{\acs{np}}}]_{\text{\acs{np}}}.
    \ex\label{ex:overview.np.3} We plan to plant four more [Fuji apple]_{\text{nominal}} trees.
\end{exe}

\begin{figure}[H]
    \centering
    \begin{tikzpicture}[x=0.75pt,y=0.75pt,yscale=-1,xscale=1]
    %uncomment if require: \path (0,427); %set diagram left start at 0, and has height of 427
    
    %Straight Lines [id:da03551150377502288] 
    \draw [color={rgb, 255:red, 208; green, 2; blue, 27 }  ,draw opacity=1 ][fill={rgb, 255:red, 144; green, 19; blue, 254 }  ,fill opacity=1 ][line width=3]    (393,223.93) -- (393,317.93) ;
    %Straight Lines [id:da12974027648754127] 
    \draw [color={rgb, 255:red, 208; green, 63; blue, 2 }  ,draw opacity=1 ][fill={rgb, 255:red, 74; green, 144; blue, 226 }  ,fill opacity=1 ][line width=3]    (325,162.93) -- (325,317.93) ;
    %Straight Lines [id:da790063550615397] 
    \draw [color={rgb, 255:red, 208; green, 99; blue, 10 }  ,draw opacity=1 ][fill={rgb, 255:red, 74; green, 144; blue, 226 }  ,fill opacity=1 ][line width=3]    (274,134.26) -- (274,342.26) ;
    %Straight Lines [id:da5340823111898327] 
    \draw [color={rgb, 255:red, 245; green, 101; blue, 35 }  ,draw opacity=1 ][fill={rgb, 255:red, 74; green, 144; blue, 226 }  ,fill opacity=1 ][line width=3]    (223,105.26) -- (223,367.26) ;
    
    % Text Node
    \draw (374,190) node [anchor=north west][inner sep=0.75pt]  [color={rgb, 255:red, 208; green, 63; blue, 2 }  ,opacity=1 ] [align=left] {central determiner: article/demonstrative/possessive};
    % Text Node
    \draw (419,218) node [anchor=north west][inner sep=0.75pt]  [color={rgb, 255:red, 208; green, 2; blue, 27 }  ,opacity=1 ] [align=left] {cardinal numeral};
    % Text Node
    \draw (457,246) node [anchor=north west][inner sep=0.75pt]  [color={rgb, 255:red, 208; green, 2; blue, 27 }  ,opacity=1 ] [align=left] {complements and attributives};
    % Text Node
    \draw (496,274) node [anchor=north west][inner sep=0.75pt]  [color={rgb, 255:red, 208; green, 2; blue, 27 }  ,opacity=1 ] [align=left] {head noun};
    % Text Node
    \draw (341,161) node [anchor=north west][inner sep=0.75pt]  [color={rgb, 255:red, 208; green, 63; blue, 2 }  ,opacity=1 ] [align=left] {quantification/degree};
    % Text Node
    \draw (526,302) node [anchor=north west][inner sep=0.75pt]  [color={rgb, 255:red, 208; green, 2; blue, 27 }  ,opacity=1 ] [align=left] {complements and attributives};
    % Text Node
    \draw (373,315.93) node [anchor=south east] [inner sep=0.75pt]  [color={rgb, 255:red, 208; green, 2; blue, 27 }  ,opacity=1 ,rotate=-90] [align=left] {nominal};
    % Text Node
    \draw (322,315.93) node [anchor=north east] [inner sep=0.75pt]  [color={rgb, 255:red, 208; green, 63; blue, 2 }  ,opacity=1 ,rotate=-90] [align=left] {minimal noun phrase};
    % Text Node
    \draw (271,340.26) node [anchor=north east] [inner sep=0.75pt]  [color={rgb, 255:red, 208; green, 99; blue, 10 }  ,opacity=1 ,rotate=-90] [align=left] {peripheral modification};
    % Text Node
    \draw (294,130) node [anchor=north west][inner sep=0.75pt]  [color={rgb, 255:red, 208; green, 99; blue, 10 }  ,opacity=1 ] [align=left] {peripheral modifiers};
    % Text Node
    \draw (242,102) node [anchor=north west][inner sep=0.75pt]  [color={rgb, 255:red, 245; green, 101; blue, 35 }  ,opacity=1 ] [align=left] {preposition(s),peripheral modifiers};
    % Text Node
    \draw (220,365.26) node [anchor=north east] [inner sep=0.75pt]  [color={rgb, 255:red, 208; green, 99; blue, 10 }  ,opacity=1 ,rotate=-90] [align=left] {prepositional phrase};
    % Text Node
    \draw (552,329) node [anchor=north west][inner sep=0.75pt]  [color={rgb, 255:red, 208; green, 99; blue, 10 }  ,opacity=1 ] [align=left] {peripheral modifiers};
    % Text Node
    \draw (586,354) node [anchor=north west][inner sep=0.75pt]  [color={rgb, 255:red, 245; green, 101; blue, 35 }  ,opacity=1 ] [align=left] {possessive -s};
    
    
    \end{tikzpicture}
    
    \caption{The structure of English noun phrase (the indentation means linear order and not constituency relations)}
    \label{fig:np-template}
\end{figure}


Above the minimal \acs{np} region, 
we still have peripheral modifiers available, 
like \form{even} or \form{along} (TODO: ref).
Note that the peripheral modifier \form{along} appears 
on the right side of the modified minimal \acs{np}.
Above peripheral modifiers 
we have prepositions and the possessive \form{-s} clitic;
a part of them can be conceptualized as 
a periphrastic case system;
other prepositions license subjects
and should better be modeled as TODO .

Note that although traditionally, complement clauses are regarded as \term{nominal clauses},
this notion is given up in this note,
because the inner structure of complement clauses is too different from prototypical \acs{np}s,
and they are therefore excluded from the class of \acs{np}s.

The semantic interpretation of an \acs{np}
is not a one-to-one translation of the underlying syntactic structure:
When, for example, there is no syntactic quantifier,
the \acs{np} still receives semantic quantification,
and quantification, strictly speaking, 
is a phenomenon involving the semantics of the whole clause 
and is not restricted to the \acs{np}:
The (final, not immediately -- see \prettyref{box:quantifier-definite}) 
interpretation of an \acs{np} within a clause is like
\translate{$\forall/\exists x$ ($x$ is something $\land$ \dots $x$ \dots)},
with possible change of the conversational context;
the content of the \acs{np} more branches containing $x$ after the logical quantifier.
introduce a conjunctive branch like \translate{$x$ is a student}
introduce a quantifier to bind the variable,
Below is an example of this procedure.
\begin{enumerate}
    \item The sentence \form{[students] usually take [at least four courses] [each year]}
    -- which contains three \acs{np}s -- 
    is to be interpreted as 
    \translate{$\forall x (
        \text{is-student} (x) \land 
        x \in \text{context } \land$
        $x$ usually take at least four courses each year)}.
    \item The part \form{$x$ usually take at least four courses each year} 
    is in turn interpreted as 
    \translate{$\forall y (
        \text{is-year}(y) \land 
        \exists e (
            \text{time}(e, y) \land 
            \text{frequency} (e, \text{habitual}) \land 
            \text{$x$ takes at least four courses in the event $e$})
    )$}.
    \item The part \form{$x$ takes at least four courses in the event $e$} 
    can further be interpreted as 
    \translate{$
        \exists S (
            |S| \geq 4 \land 
            S \subseteq \text{all-courses} \land 
            \text{action}(e, \text{take}) \land 
            \text{agent}(e, x) \land 
            \text{patient}(e, S)
        )
    $}.
    The number 4 here can further be expanded into things like 
    \translate{$
        \exists c_1, c_2, c_3, c_4 (
            \text{$c_1, c_2, c_3, c_4$ are all different} \land 
            \cdots
        )
    $}.
\end{enumerate}
Fortunately the organization of the semantic concepts still largely follows 
the hierarchy in \prettyref{fig:np-template} 
and will be discussed in corresponding sections.

TODO: possibly related literature: 
\begin{itemize}
    \item Chapter 1 Reference and Quantification in Nominal Phrases: The Current Landscape and the Way Ahead in Determiners and Quantifiers
    \item Definiteness By Christopher Lyons
    \item Layers in the Determiner Phrase by Rob Zamparelli
    \item The English noun phrase in its sentential aspect by SP Abney
    \item \citet{laenzlinger2017view}: the end part of it!!
\end{itemize}


\section{The (extended) noun}\label{sec:np.head}

\subsection{Compound nouns}\label{sec:pos.noun.compound}

Compound nouns -- a compound where the two immediate constituents are all nouns -- 
can be divided into centered an non-centered ones
\citep[\citepages{1646-1648}]{cgel}.
The semantic relation between the two branches of a centered compound noun
is highly diverse; 
some compounding structures seem to be head-complement or head-modifier constructions,
while for others, what relates the two branches is merely ``aboutness'':
when an established meaning is absent,
the form A-B means ``B that has something to do with A''.
Non-centered compound nouns 
(or \concept{dvandva nouns}, which is the term in Sanskrit) 
are relatively rare
compared with the case in Sanskrit.

\subsubsection{Conditions for centered noun compouding}

The two constituents in a compound noun usually can't bear any inflection.
However, irregular plurals are sometimes permitted.
This means the threshold of noun compounding has a ``lightweight'' requirement:
The branch A \emph{has to} bear the ``category is noun'' feature,
but \emph{nothing more}.
Thus ordinary plurals have a number feature higher than the category feature 
and are excluded from engaging in noun compounding,
while in fused plurals like \form{mice},
the root, the noun category feature and the plural number feature 
are all realized into one unit,
so compounding is licensed \citep[\citesec{7.1}]{siddiqi2009syntax}

\subsection{Adjective-noun compounding}

An adjective-noun compound is usually a modifier-head structure, 
but not always: 
\form{sick-bed} receives an aboutness interpretation outlined in \prettyref{sec:pos.noun.compound},
where the adjective \form{sick} is used as a category-less root,
which means \form{sickness} here.

\subsection{Verb-centered compound nouns}\label{sec:pos.noun.compound.verb-centered}

\subsection{Deverbal nominalization}

Nominalization of a verb either gives a noun about the action or state described by the verb
\citep[\citepage{1700}]{cgel},
or gives a noun referring to an object or person 
that is a semantic argument of the verb \citep[\citepage{1697}]{cgel}.
The two kinds of nominalization are semantically similar to 
content clauses and relative clauses, respectively.
Thus, cross-linguistically, 
we often see nonfinite forms of a verb serving as 
its normalized forms,
either in the first meaning or in the second meaning mentioned above.
In English, the relative clause-like use of participles is rare, 
but the content clause-like use of participles is highly frequent
(\prettyref{ex:pos.noun.deverbal.1}, \prettyref{ex:pos.noun.deverbal.2}).

\begin{exe}
    \ex\label{ex:pos.noun.deverbal.1} {} [His playing the national anthem]_{\text{\formcat{ing}-participle complement clause}} amazed us.
    \ex\label{ex:pos.noun.deverbal.2} {} [His playing of the national anthem]_{\text{\formcat{ing}-nominalization}} is amazing.
\end{exe}

\subsubsection{Zero derivation}

the one doing something: \form{coach}, \form{spy}

the action of \form{read}, \form{go}, \form{attempt}

\subsection{The number category}

The number category is regularly realized as the suffix \form{-s},
attached to the end of the noun.
Some nouns have irregular plural forms, like 
\form{foot} (\category{sg}) v.s. \form{feet} (\category{pl}),

\section{The structure of nominals}\label{sec:np.nominal}

\subsection{Nominal attributives}\label{sec:np.nominal.nominal-attributive}

A \concept{nominal attributive} is an attributive that is a nominal.
There seems to be a tendency that 
a nominal attributive should normally be ``small'' enough:
It can contain adjectives or another nominal attributive,
but the modification construction usually has an established meaning
and is regarded ``as a word''. TODO: more precise definition

\begin{exe}
    \ex {} [The [Ministry of Defense]_{\text{nominal attributive}} officials]_{\text{\acs{np}}} are having a secret meeting in that room.
    \ex We planted [an [apple]_{\text{nominal attributive}} tree]_{\text{\acs{np}}} yesterday.
    \ex {} [The [[Fuji apple]_{\text{nominal attributive}}_{\text{nominal attributive}} tree] variety]_{\text{\acs{np}}} has a reddish-green color.
\end{exe}

\subsection{Adjectival attributives}

\subsection{Attributive possessives}

Possessive \acs{np}s can also serve as attributives. 

\begin{exe}
    \ex There are three teachers' books on the desk.
\end{exe}

\subsubsection{Interpretation of attributives: restrictive and non-restrictive}

If an attributive is removed in an \acs{np} 
and the reference of that \acs{np} remains the same,
we say it's non-restrictive.
Non-restrictive attributives are ``comments'' or ``afterthoughts'' concerning the reference of the \acs{np}.
In \form{members of the music club, 
[who have developed very close friendship], 
are all going to the same college},
the relative clause (TODO: ref) is not necessary to decide who the members of the music club are.

If the concept of definiteness were perceived according to Russell's theory of description 
(\prettyref{box:theory-of-description}),
then the distinction between restrictive and non-restrictive attributives 
would by definition be categorical:
Restrictive attributives are about uniqueness of the \acs{np}'s reference,
while non-restrictive attributives are not.
The position of this note is however more contextualist,
TODO: how we treat 
Of course, \emph{some} attributives are still bound to be restrictive 
or non-restrictive for various reasons:
If an \acs{np} is definite, 
TODO: why removing some attributives makes a definite NP unnatural??
(TODO: ref).

\subsection{Ordering and compatibility of attributives}

\subsection{Cardinal numerals}

There can be zero or one cardinal numeral in a nominal,
and it always appears over all attributives.

\begin{exe}
    \ex I need [ten [green lasers]_{\text{inside num.}}]_{\text{nominal}}.
\end{exe}

\section{The determiner and the like}\label{sec:np.det}

Above the nominal layer, 
we have a series of determiner-like grammatical functions,
filled by various forms.
In this note, 
I define the prototypical grammatical function of articles, demonstratives, 
and ``determining'' possessives as the \concept{central determiner}.
I also assume a quantification function over the determiner function
to account for expressions like \form{all the things},
where \form{all} is the syntactic quantifier and \form{the} the central determiner.

\begin{infobox}{About \term{determiners} and \term{predeterminers}}{predeterminers}
    \citet[\citepage{331}]{cgel} takes a slightly different definition of the term \term{determiner},
    and it also recognizes a syntactic function called the \term{predeterminer}.
    The rules seem to be the follows: 
    If there is no article or demonstrative or any other thing that is prototypically a determiner,
    then the lowest determinative is the determiner;
    otherwise the article or demonstrative or \dots is the determiner,
    and determinatives lower than it are modifiers.
    Thus, in \form{the three cute cats},
    \form{three} is a modifier, while \form{the} is the determiner 
    \citep[\citepage{356}, {[4ii]}]{cgel},
    although in \form{three cute cats}, \form{three} is the determiner 
    \citep[\citepage{355}, {[2ii]}; \prettyref{sec:np.det.num}]{cgel}.
    In \form{all vases}, \form{all} is the determiner,
    while in \form{all the vases}, \form{all} is the predeterminer
    \citep[\citepage{356}, {[4i]}]{cgel}.

    I find the term \term{predeterminer} unnecessarily complicate the matters,
    and to say something is a predeterminer 
    tells us nothing about its position in \prettyref{fig:np-template}
    or its possible semantic interpretations.
    Predeterminers in \citet[\citepage{433}]{cgel}
    all seem to be some kind of quantification,
    so I just skip the term \term{predeterminer} in \prettyref{fig:np-template},
    and explicitly inserts a quantifier position.

    So, a wise terminology is to replace the term \term{predeterminer} with \term{quantifier}.
    Since some grammars use the term \term{determiner} to cover all determiner-like grammatical functions,
    \citet[\citepage{253}]{quirk1985} call the position prototypically filled by 
    articles and demonstratives 
    the \term{central determiner}. TODO: whether to use this notation
\end{infobox}

One typological property of English is 
the prominence position of determiner,
as opposed to some other Indo-European languages, like Latin.
In Latin the determiner -- like a demonstrative -- 
looks just like an attributive:
morphologically speaking, it has the adjectival declension pattern
(there is no such thing as an article category),
and its surface position in the \acs{np} 
is as flexible as other attributives,
modulated by the information structure.
In English, on the other hand, we have prototypical fillers of the determiner position
-- articles \form{a} and \form{the} -- 
and 

\subsection{The central determiner and identifiability }\label{sec:np.det.definite}

The coverage of a nominal can be extremely huge.
to decide what are being talked about,
a simple way, which is used in the above example, 
is to add quantification on them.
but the speaker/writer may instead invite the listener/reader to 
identify the object(s) referred by the \acs{np} 
from the conversational context
and/or some uniqueness conditions.
The usual syntactic device to send such an invitation 
is \concept{definiteness} 
(in this note, it's the determiner function),
and the corresponding semantic concept is called \concept{identifiability}.

\begin{theorybox}{The weakness of the naive theory of description}{theory-of-description}
    The fact that definiteness implies identifiability has long been noticed in the study of semantics.
    Russell's theory of description interprets \form{the} 
    as a logical symbol $\iota$,
    and $Q(\iota x P(x))$
    means $\exists x (P(x) \land Q(x) \land \forall y (P(x) \land P(y) \rightarrow  x = y)) $.
    This is a neat approximation of the definiteness concept,
    but still has some subtle differences from definiteness in natural languages.
    When the uniqueness condition is broken, 
    usually we don't say the sentence has a false truth-value:
    We say it doesn't make sense at all.
    (Some may develop more delicate logic systems to handle this,
    but I feel this is not necessary:
    natural language sentences are more like 
    ``commands'' in imperative programming than logical expressions,
    and of course a subroutine can throw an error and give no return value.)
    The position of this note is 
    theory of description is not one hundred percent correct,
    and is better replaced by a contextual account \citep[\citepage{368}]{cgel},
    though the techniques used in the theory of description
    are of course of great importance:
    We may, for example, correct the description theory 
    by letting $P$ be the mix of the interpretation of a nominal and contextual information.
\end{theorybox}


\subsubsection{The articles \form{the}}

There are several degrees of identifiability that may be conveyed by \form{the};
with low-degree identifiability, 
the reference of an \acs{np} containing \form{the} is 
to be decided partially by its inner attributives and partially by the context.

Logical uniqueness is the strongest, 
but this is usually constrained to mathematical objects
(\prettyref{ex:np.det.math-1}).
Uniqueness from empirical observation or man-made rules is slightly weakened:
The article \form{the} in (\prettyref{ex:np.det.decline-to-comment})
is almost never replaced by \form{a},
because for almost all companies,
the CEO position is unique.

A further weakened version is uniqueness in the conversational context.
In \eqref{ex:np.det.conversation-1}, 
of course the speaker isn't implying that there is only one T-shirt in the world:
but if \emph{in the context of the conversation},
there is only one T-shirt,
then the sentence makes perfect sense.
This gives rise to the famous \form{a}-\form{the} alternation in discourses:
When a person or an object first appears,
\form{a} is used,
and then the entity is referred to with \form{the}.

But even this can be loosen:
It may be the case that there are several entities in the conversational context
that satisfy the conditions,
but it's OK to randomly pick up one,
and this is still a kind of identifiability.

\begin{exe}
    \ex\label{ex:np.det.math-1} [The set containing no element] is [the empty set].
    \ex\label{ex:np.det.decline-to-comment} The CEO of this company declined to comment.
    \ex\label{ex:np.det.conversation-1} Pass [the green T-shirt] to me.
\end{exe}

\subsubsection{The abstract generic usage of \form{the}}

\begin{exe}
    \ex I like playing the guitar.
    \ex The computer replaces the typewriter.
    \ex The kangaroo lives in Australia.
    \ex The dermatologist specializes in skin care.
\end{exe}

Modification also destroys the possibility of the abstract generic reading.
This is also a piece of evidence that 
there are more fine-grained structure within the nominal,
specifically, an ``extended noun'' (TODO: ref).

\begin{exe}
    \ex *The towel absorbs water.
    \ex I like playing the high-quality guitar. \\
    \translate{I like play that specific guitar that has high quality.
    / *I like play all high-quality guitars.}
\end{exe}

\subsubsection{The subject-determiner possessives}

Note that the possessive \acs{np} is also a complement or ``argument''
of the head noun,
and in some morphologically rich languages,
the possessive \acs{np} is indeed marked as an argument 
\citep[\citesec{5.1.2.1}]{jacques2021grammar}.
Thus we may say the possessive is the subject in the \acs{np},
and it's therefore the subject-determiner \citep[\citepage{467}]{cgel}.

\subsubsection{\form{We} and \form{you} as determiners}

\begin{infobox}{About \form{we the people}}{we-the-people}
    Not all personal pronouns appearing at the initial of an \acs{np}
    are determiners.
    In \form{we the people}, 
    \form{we} seems to be TODO: appositive
\end{infobox}

\subsubsection{Indefiniteness and numerals as determiners}

Indefiniteness is the opposite of definiteness,
and occurs whenever identifiability can't be established.

The function of \form{a} 
and the function of \form{one} when there is no other determinative in the \acs{np}
seems to be the same \citep[\citepage{372}]{cgel},
which may be the reason why \citet[\citepage{385}]{cgel} claims 
the numeral sometimes is the determiner. TODO: so what should be my analysis?

It should be noted that there is something asymmetric between definiteness and indefiniteness.
A definite \acs{np} directly refers to some objects \emph{on its own},
while an indefinite \acs{np} doesn't:
It may gain specificity from the context (TODO: ref), 
but itself doesn't have specific reference.
Thus, an indefinite \acs{np} \emph{always} introduces a bound variable when interpreted 
and involves quantification,
while a definite \acs{np} doesn't necessarily involve quantification.
The status of indefinite is not the same as the status of definite semantically,
and also possibly syntactically \citep{gianollo2021reference,klockmann2020article}

\begin{theorybox}{About the ``quantifier-less'' interpretation of definite \acs{np}s}{quantifier-definite}
    The end of \prettyref{box:theory-of-description}
    shows that even in the contextualist approach taken in this note,
    an existential quantifier can still bind the logical variable introduced by a definite \acs{np},
    so in the first glimpse, 
    definiteness and indefiniteness has nothing different concerning quantification.
    But for a definite \acs{np},
    in principle, 
    \emph{immediately after it is interpreted},
    it's possible that we don't see any quantifier introduced:
    The interpretation of a subject \acs{np} 
    may just be \translate{$P(\text{concept-to-set}(\text{interpretation-of-nominal}))$},
    where $P$ is the interpretation of the \acs{vp} (in the sense of this note).
    Of course, to \emph{finally} eliminate the function concept-to-set, 
    we still need to use the techniques in \prettyref{box:theory-of-description}
    and introduce quantifiers,
    but it's not the \emph{immediate} result of interpreting the definite \acs{np}.
    On the other hand, 
    for an indefinite \acs{np},
    its \emph{immediate} interpretation \emph{always} comes with a logical quantifier.
\end{theorybox}

\subsection{Syntactic quantifiers}\label{sec:np.det.quantifier}

Syntactic quantifiers in English are given by \citet[\citepage{361}, {[9]}]{cgel},
replicated here:

\subsubsection{Quantification}

In mathematics, quantification 
is about bound variables,
while definiteness 
is essentially a template which 
maps a predicate to a set (\prettyref{box:quantifier-definite}).
In natural languages, we need to note that 
syntactic marking of quantification can be applied on top of definiteness:
We have \form{all the things I've heard about} 
and \form{both the parents}:
Quantification can act as a ``filter'' 
to further filter what is retrieved from the conversational context.
The reverse is not possible, 
possibly for semantic reasons.

Apart from that, we get 
the familiar universal-existential distinction
(as in mathematics).
Note that in natural languages, 
usually syntactic $\forall$ implicates semantic $\exists$:
This may be motivated by the ``iterating-over'' meaning of $\forall$,
where iterating over an empty set throws an error.

\subsubsection{Collective or not?}

One important parameter -- although without explicit syntactic coding in English 
-- is the \concept{joint (or collective)-distributive distinction}.
The speaker/writer may only talk about a certain part of the objects retrieved
from the conversational context (universal), 
or he or she may talk about all of them (existential).
The object denoted by \acs{np} may jointly participate in the predication (joint)
so that it's not correct to say 
one object participate in the predicate,
or maybe individually (distributive).
Ambiguity may occur here.
Someone says the student selects four courses -- 
does it mean the student takes the four courses one by one in the course selection system
(distributive), 
or does it mean the student selects the four courses at once 
(by, say, using the worksheet of the course selection system)?
The ambiguity has to be settled by contextual information.

\subsubsection{The negative polarity items}

The \form{any} family

\begin{theorybox}{Is the semantico-pragmatic approach correct?}{npi-semantic}
    One problem is whether we are heading to the wrong direction here.
    It may just be the case that the distribution of \form{any}, \form{anyone}, etc. 
    is determined by syntactic factors.
    Indeed, usually people consider the misuse of these items 
    a \emph{grammatical} problem 
    instead of a logical or feasibility problem
    \citep[\citepage{812}]{zeijlstra2013}.
    \form{I know the current King of France} is of course wrong,
    but it's valid;
    \form{*I know anything about this topic} is grammatically wrong.
    This criticism however doesn't kill the semantico-pragmatic approach,
    because the latter focuses on the syntax-semantic interface 
    instead of purely semantic or pragmatic issues.
    Using the programming language metaphor,
    misusing negative polarity items doesn't break AST generation,
    but it does break the compiling procedure
    because the AST can't be mapped to machine codes.
    For ordinary people, the machine of grammar contains the syntax proper-LF interface.
    A comparable example is the English \form{wa}-\form{ga} alternation,
    which seems to be motivated pragmatically 
    but is usually regarded as a part of the grammar.
\end{theorybox}

\subsection{Specificity}\label{sec:np.det.specific}

It should be noted that definiteness doesn't necessarily imply \emph{specificity},
because while any kind of specificity implies identifiability, 
the weakest sense of identifiability 
is not specificity;
and specificity -- or more generally, identifiability -- 
also doesn't imply definiteness.
In \form{a man was sent to hospital after the shooting}, 
\form{a man} usually receives a identifiable reading 
because of scalar implicature: 
If I say there is one man sent to hospital,
then it's highly unlikely that I mean there are two men sent to hospital.
So there is sort of a uniqueness condition concerning \form{a man}.
But still, we use the indefinite article \form{a},
because the nominal \form{man} itself 
isn't enough for us to retrieve who is sent to hospital from the conversational context
(while, say, even if Tom's father is unknown to us,
we know there is someone -- and likely only one -- who is his father,
so \form{Tom's father} still fixes the reference).

\begin{infobox}{Specificity as a syntactic function}{syntax-specificity}
    Here in English, 
    specificity is purely semantic.
    Cross-linguistically, the semantic concept of specificity 
    may be realized by a syntactic function
    higher than the determiner,
    which also serves as the ``\acs{np}-inside topic'' position 
    \citep{ihsane2001specific},
    which is absent in English.
\end{infobox}


\subsection{Referentiality}\label{sec:np.det.deixis}

Another semantic parameter of \acs{np}s is whether it's referential,
i.e. whether its appearance introduces a new entry in the old information list 
that can be referred to by a following pronoun.
There is no explicit syntactic marking of referentiality in English.

Note that referentiality is not strongly coupled to determination or quantification:
Although the reference of an indefinite \acs{np} cab never be determined on its own 
(even with contextual information),
indefinite \acs{np}s are still referential.

\begin{exe}
    \ex {} [Teachers]_{\text{indefinite \acs{np}}, i} here are expected to be patient. 
    They_i shouldn't give up on a child too quickly.
\end{exe}

Non-referential usages of \acs{np}s are relatively limited.
The cases include negative \acs{np}s (TODO: ref), 
interrogatives (TODO), 
and meta-linguistic usages as in \form{[``Mary''] is a famous name for girls}
\citep[\citepage{400}]{cgel}.

\begin{infobox}{Referentiality as a syntactic function}{d-deixis}
    
\end{infobox}

\subsection{Is the numeral a kind of determiner?}\label{sec:np.det.num}

The analysis that the cardinal numeral has 
a determiner function besides its function as a modifier in the nominal
when there is no other determiner
\citep[\citepage{355}, {[2ii]}]{cgel}
does make some sense,
because it seems the article \form{a} is very similar to 
the numeral \form{one} without \form{the},
and therefore a cardinal numeral without a higher determiner 
appears to be an indefinite determiner.
An alternative analysis is to assume that 
the so-called article \form{a} in fact a specific cardinal numeral
\citep[\citesec{2.5}]{lyons1999definiteness}.

\subsection{Ordering and compatibility}

The order of all determiner-like elements is highly rigid (\prettyref{tbl:det-region-template}).
The compatibility between them however shows certain degree of variations.

\begin{table}[H]
    \caption{Possible values of the determiner-like region}
    \label{tbl:det-region-template}
    \centering
    \begin{tabular}{llll}
    \toprule
    Quant.       & Det.                     & Num.                        & Nominal            \\ \midrule
    \form{}    & \form{}                & \form{}                   & \form{things}    \\
    \form{}    & \form{}                & \form{a/one}              & \form{thing}     \\
    \form{}    & \form{}                & \form{two/three/four/...} & \form{things}    \\
    \form{all} & \form{}                & \form{}                   & \form{things}    \\
    \form{all} & \form{}                & \form{three/four/...}     & \form{points}    \\
    \form{all} & \form{the/these/those} & \form{}                   & \form{things}    \\
    \form{all} & \form{the/these/those} & \form{three/four/...}     & \form{things}    \\
    \form{}    & \form{we/you}          & \form{}                   & \form{engineers} \\ \midrule
    \end{tabular}
\end{table}

\section{Peripheral modifiers}

\section{Possessive constructions}

Possessive \acs{np}s appear in TODO

\section{Fused-head constructions}\label{sec:np.fused-head}

A fused-head \acs{np} is an \acs{np} in which the function of the main noun 
-- the (lexical, not functional) head -- 
is fulfilled by another constituent.
We know \form{all} in \form{[all] of these statements}
carries the function of head as well as its usual function of quantification,
because it selects complementation (\form{of these statements}):
Compare, say, \form{all [instances]_{\text{head}} of these statements}
(and also because of the theoretical orientation of this note 
that the ``noun category'' feature has to be realized by something).

Fused-head constructions allow less variations than ordinary \acs{np}s.
For example, when the quantification and the head is fused,
modifications are no longer possible.

\subsection{Personal pronouns}

\begin{exe}
    \ex\label{ex:np.fuse.negative-np-1} *[No car in the race]_i broke down and [it]_i had to be repaired.
\end{exe}

\begin{theorybox}{Pronouns are not gap fillers}{pronoun-not-gap-filler}
    Pronouns are not residue of \acs{np}s moved out,
    even with coreferential relations.
    In (\prettyref{ex:np.fuse.negative-np-1}),
    for example, if the pronoun \form{it} can be analyzed by the trace left by \form{no car in the race},
    then there is no reason for the unacceptability.
    This means coreferences are not always generated by movement.
\end{theorybox}

\subsection{Demonstratives}\label{sec:np.fused-head.dem}

\section{Preposition constructions}


TODO: \form{in spite of} is an established form but still has a synchronically meaningful constituency structure


\chapter{Verb phrase}

\section{The verbal complex}\label{sec:verbal-complex}

\subsection{The structure of the regular verbal complex}

Now we can combine everything in the verbal complex together.
When there is no auxiliary needed,
the tense feature is lowered to the main verb. 
In other cases, the highest auxiliary -- the first auxiliary -- 
is lifted to the tense position,
before negation and the default position of many adverbs.

\subsection{Regular lexical verbs}\label{sec:verb-forms}

Modern English has already lost most of its verb inflection.
Following the analysis of \citet[\citechapsec{3}{1.1}]{cgel},
for lexical verbs,
there are six remaining inflectional forms: 
the past form, the plain present form, 
the 3sg present form,
the plain form, the \formcat{ing}-participle,
and the \formcat{ed}-participle.
The two present forms and the past form appear solely 
with trivial aspectual values and trivial modality.
They are \concept{primary} forms:
They already have all \acs{tam} categories marked on them.
The plain form and the two participles are \concept{secondary} forms:
They usually appear after auxiliaries 
in a periphrastic construction to have full \acs{tam} marking,
though a subjunctive clause may sometimes get rid of any auxiliary verb,
as in \form{he suggests that she [complete] this task first} (\prettyref{sec:complement.subjunctive}).

Examples of these forms are illustrated in \prettyref{tbl:lexical-inflection}. 
This is a copy of [1] in \citet[\citesec{1.1}]{cgel}.
It can be noticed that the plain form is usually the same as the plain present form.
However, since modal verbs (see below) have no plain form,
and that the syntactic environments of the plain form and the present plain form are too different,
if \prettyref{tbl:lexical-inflection} is to be regarded as a paradigm
-- that is, to be incorporated with morphosyntactic information -- 
then the two forms should occupy two cells.

\begin{table}[H]
    \caption{Paradigms of lexical verbs}
    \label{tbl:lexical-inflection}
    \centering
    \begin{tabular}{@{}llllll@{}}
    \toprule
    \multicolumn{1}{l}{}       &                               &       & \form{take}   & \form{want}    & \form{hit}     \\ \midrule
    \multirow{3}{*}{Primary}   & past form                     &       & \form{took}   & \form{wanted}  & \form{hit}     \\
                               & \multirow{2}{*}{present form} & 3sg   & \form{takes}  & \form{wants}   & \form{hits}    \\
                               &                               & plain & \form{take}   & \form{want}    & \form{hit}     \\ \midrule
    \multirow{3}{*}{Secondary} & plain form                    &       & \form{take}   & \form{want}    & \form{hit}     \\
                               & \formcat{ing}-participle       &       & \form{taking} & \form{wanting} & \form{hitting} \\
                               & \formcat{ed}-participle        &       & \form{taken}  & \form{wanted}  & \form{hit}     \\ \bottomrule
    \end{tabular}
\end{table}

\begin{infobox}{The name of the forms}{verb-form-name}
    Here I deviate from the practice in \citep[\citechap{3}]{cgel} 
    and pick up the more common names for some of the forms. 

    The \formcat{ing}-participle is frequently called the \term{gerund},
    because it now has the function of both a gerund and an active participle.
    \citet{cgel} call it the \term{gerund-participle}.
    Some grammars use the term \term{present participle}.
    Since in Modern English,
    the \formcat{ing}-participle no longer carries any tense information,
    the historical term \term{present participle} is abandoned in this note.

    The traditional name \term{past participle} for the \formcat{ed}-participle makes more sense,
    because it's morphologically related to the past form for regular verbs 
    and it still has some sense of ``past'':
    It is strongly related to the \category{perfect} and therefore has some sense of the past,
    though it doesn't carry the past tense.
    A better term would be the one in Latin grammar: the \term{perfect passive participle},
    but this is in conflict with the name of the \form{having been done} construction.

    A usual name for the plain form is the infinitive form,
    which I reject here because the morphological marking of 
    the main verb after modal auxiliary verbs  
    (\form{would [like]}),
    the verb in a subjunctive clause 
    (\form{he suggests that she [complete] this task first}),
    and the verb in a real infinitive clause are all the same,
    and therefore it makes no sense to use the term \term{infinitive} 
    to cover the \emph{morphological} form of all the three.
\end{infobox}

The \formcat{ing}-participle is regularly formed by adding \form{-ing} to the end of the plain form
(TODO: -tt- in splitting).
The \formcat{ed}-participle and the past form are usually obtained 
by adding \form{-ed} to the end of the plain form,
but for irregular verbs they can't be inferred from the plain form.
Thus English verbs have three \concept{principal forms}:
the plain form, the past form, and the \formcat{ed}-participle.
We may also say there are three stems in English:
the plain form, the past form, and the \formcat{ed}-participle,
with only the first one being productive for further morphological processes.

\subsection{Types of irregular verbs}

As is mentioned above, for a number of irregular verbs,
the \category{ed}-participle and the past form can't be inferred from the plain form.
Whether there are still some patterns between the three,
or in other words,
the formation of the principal parts,
is investigated in detail in \citet[\citepages{105-120}]{quirk1985}.

\subsection{Auxiliary verbs}\label{sec:verb-inflection.auxiliary}

English also has a number of auxiliary verbs.
All auxiliary verbs have tense-dependent forms,
because all of them may appear as the first word in an auxiliary chain,
and the tense category is to be marked on the highest i.e. the first of them (\prettyref{sec:auxiliary-chain}).
Thus, we say English auxiliaries also have primary forms.
Modal auxiliaries don't have a separate 3sg present form,
but \form{do}, \form{have} and \form{be} (when used as auxiliary verbs) do.
It should be noted that the past forms of many auxiliary verbs don't just appear in past clauses:
They may have distinct meanings (\prettyref{sec:verb-inflection.modal-use}).

Modal auxiliaries don't have secondary forms,
probably because they never appear after another auxiliary verb 
or in nonfinite clauses,
but \form{do}, \form{have} and \form{be} do.

English auxiliary verbs also have negative forms,
which are obtained by attaching \form{-nt} to the end of auxiliary.
The \form{-nt} is a contraction form of the negator \form{not},
but in modern English the negative suffix moves together with the auxiliary in
subject-auxiliary inversion (\prettyref{sec:sai}).
Thus, it's recognized as a part of the auxiliary \citep[\citepage{91}]{cgel}.
This seems to be purely about phonetic realization:
There seems to be no large morphosyntactic differences 
between auxiliary-\form{not} and the negative auxiliary
besides subject-auxiliary inversion.
All auxiliaries don't have secondary negative forms,
though \form{do}, \form{have} and \form{be} have primary negative forms.

Since auxiliary verbs are a part of the grammar,
here I list the paradigms TODO

\begin{infobox}{Auxiliary constructions are single-clause ones}{auxiliary-single-clause}
    \citet{cgel} treat auxiliary verbs as verbs taking complement clauses 
    (as in, say, [11] in \citepage{782}).
    This is not the position of this note:
    Here I follow the standard practice in generative syntax (probably also American structuralism) 
    and assume auxiliary verb constructions are always single-clause constructions.
    \emph{Historically}, auxiliaries may origin from complement-taking verbs,
    but now \emph{synchronically}, they have the same function of inflectional affixations.
    Complement clause constructions may (or may not) have the same \emph{semantics} of 
    auxiliary verb constructions and inflectional affixations,
    but they never have the same \emph{structure}.

    The main reasons \citet{cgel} analyze auxiliary verbs as complement-taking verbs or
    \term{concatenative verbs} in their terms 
    are shown in their \citechapsec{14}{4.2.2}.
    They argue that we have constituency trees like
    \form{[would [like to do]]} -- which I also agree on.
    They then argue that this constituency structure means 
    \form{like to do} is a complement clause -- which then is not always true.
    This second part of their argument 
    seems to come from confusion between lexical heads and (PF realization of) functional heads.
    But then they are inconsistent when they argue that 
    the complementizer \form{that} isn't a head.
    In this note, I follow the standard definition of (lexical) headhood in the descriptive literature
    while fully being aware of the generative functional head analysis.

    Evidences supporting my claim that auxiliary constructions are indeed single-clauses ones 
    can be obtained by observing how auxiliaries interact with clausal dependents.
    If \citet{cgel} are correct on their claim that English auxiliary verbs take bare infinitive clauses,
    then we expect the verbal phrase after an auxiliary verb to receive any modification 
    that's acceptable for a bare infinitive clause.
    However, as we see in \prettyref{sec:verb-inflection.adverb-auxiliary-chain},
    there is a strong tendency for adverbs to appear after the first auxiliary,
    which can be easily explained by assuming the first auxiliary 
    undergoes some kind of fronting (\prettyref{sec:verb-inflection.negation}),
    or after all auxiliaries and before the main verb,
    and the functions of adverbs in the two positions 
    have clear correlation with the positions.
    This pattern are hard to account for 
    when we assume auxiliary verb constructions are complement clause constructions,
    because nothing motivates it.
    If, on the other hand, 
    auxiliary verb constructions are single-clause constructions,
    then we can say the distribution of adverbs and auxiliaries
    show is just the surface reflection of a deep functional hierarchy,
    just like the subject is somehow higher than the object.
\end{infobox}


\subsection{Minimal auxiliary chain}\label{sec:auxiliary-chain}

In a declarative finite clause,  
the order of auxiliaries is constantly given by \prettyref{tbl:auxiliary-chain}.
\prettyref{tbl:auxiliary-chain} is a part of the larger picture of clause structure:
The auxiliary \form{do} (\prettyref{sec:verb-inflection.do}), 
adverbs (\prettyref{sec:verb-inflection.adverb-auxiliary-chain})
and the negator (\prettyref{sec:verb-inflection.negation})
may be inserted into somewhere between two auxiliaries.
Other types of clauses still largely follow the scheme but 
may undergo subject-auxiliary inversion (\prettyref{sec:sai}).

The auxiliaries positions can be filled by the corresponding auxiliaries or be just left blank,
without creating ungrammatical constructions.
The \category{modal} slot may be filled by a modal auxiliary.
The \category{perfect} slot may be filled by the auxiliary version of \form{have} with the correct inflection,
and the \category{progressive} and \form{passive} slots 
may be filled by the auxiliary version of \form{be} with the correct inflection.

The rules of inflection are the follows.
The tense category is always marked on the first auxiliary
(not necessarily one of the slots in \prettyref{tbl:auxiliary-chain}
-- it may be an inserted \form{do}),
and when there is no auxiliary,
it's marked on the main verb.
Note that it isn't true that if the first auxiliary is in the past form,
it always means a past event (\prettyref{sec:verb-inflection.modal-use}).
The modal auxiliary is always followed by a plain form,
and the progressive marking \form{be} is always followed by an \formcat{ing}-participle,
and the perfect marking \form{have} is always followed by an \formcat{ed}-participle,
and so is the passive marking \form{be}.
When the clause is finite and the tense is \category{present},
and the \category{modal} slot is empty,
if the subject is 3sg in number,
then the first non-empty slot in \prettyref{tbl:auxiliary-chain}
is in the 3sg present form,
which means for verbs other than \form{be}, the \form{-s} suffix is attached to it;
for \form{be} the correct form is \form{is}.
This is the only case subject-verb agreement happens in English 
other than the case of \form{be} (\prettyref{ex:verb-inflection.3sg-1}).
For \form{be}, the tense is still  TODO: subjunctive 

In nonfinite forms, the \category{modal} slot has to go;
the rest are still there, following the same inflectional pattern as is described above
(\prettyref{ex:verb-inflection.infinitive-1}).
Note that the subject-verb agreement is missing in all nonfinite clauses,
be it the third person singular \form{-s} or inflectional forms of \form{be}.

\begin{table}[H]
    \caption{The order of auxiliaries and some examples}
    \label{tbl:auxiliary-chain}
    \centering
    \begin{tabular}{@{}lllll@{}}
    \toprule
    \category{modal}      & \category{perfect}      & \category{progressive}        & \category{passive}            & main verb    \\ \midrule
    \form{}           & \form{}             & \form{}                   & \form{}                   & \form{takes}  \\
    \form{}           & \form{}             & \form{}                   & \form{am/are/is/was/were} & \form{taken}  \\
    \form{}           & \form{}             & \form{am/are/is/was/were} & \form{}                   & \form{taking} \\
    \form{}           & \form{have/has/had} & \form{}                   & \form{}                   & \form{taken}  \\
    \form{}           & \form{have/has/had} & \form{been}               & \form{being}              & \form{taken}  \\
    \form{will/would} & \form{have}         & \form{been}               & \form{being}              & \form{taken}  \\ \bottomrule
    \end{tabular}
\end{table}

\begin{exe}
    \ex\label{ex:verb-inflection.3sg-1} \begin{xlist}
        \ex I [like] this.
        \ex He [likes] this.
    \end{xlist}
    \ex\label{ex:verb-inflection.infinitive-1} 
    The award is reported [to have been being taken]_{\text{complement clause: \formcat{to}-infinitive}} 
    \ex\label{ex:verb-inflection.infinitive-2} 
\end{exe}



\subsection{\form{Do} insertion}\label{sec:verb-inflection.do}

\subsubsection{Obligatory \form{do} insertion}

\form{Do} insertion happens in two circumstances.
The first is we need an auxiliary but there isn't one.
This is the case when we negate a clause with no auxiliary verb
(\prettyref{sec:verb-inflection.negation}),
and the case when subject-auxiliary inversion happens but there is no auxiliary verb
(\prettyref{sec:sai}).
In both cases, \form{do} is inserted before the main verb,
and is regarded as an auxiliary,
which carries the tense feature and the subject-verb agreement information
and is inflected accordingly
(\prettyref{ex:verb-inflection.do-1}, \prettyref{ex:verb-inflection.do-2}).

We may say the \form{do} is the default realization of the tense category and the agreement 
when these can't find an appropriate host.
It's roughly in the same position of \category{modal} in \prettyref{tbl:auxiliary-chain}.
Then, expectedly, adverbs can be inserted between \form{do} and the main verb
(\prettyref{ex:verb-inflection.do-3}).

\begin{exe}
    \ex\label{ex:verb-inflection.do-1} I do not like the gift. I don't like the gift.
    \ex\label{ex:verb-inflection.do-2} Did he enter the room that night?
    \ex\label{ex:verb-inflection.do-3} I do not particularly like that kind of flower.
\end{exe}

\subsubsection{\form{Do} for emphasis} 

Unlike (\prettyref{ex:verb-inflection.do-1}, 
\prettyref{ex:verb-inflection.do-2}, 
\prettyref{ex:verb-inflection.do-3}),
we can also just insert \form{do} to emphasize on the action,
and in this case the inserted \form{do} receives stress.
The morphology of \form{do} is the same as the obligatory \form{do} insertion,
and so is the distribution of adverbs.

\begin{exe}
    \ex Your company [\emph{do}]_{\text{\form{do} insertion}} [have]_{\text{main verb}} lots of rules!
\end{exe}

\subsection{Adverbs in the auxiliary chain}\label{sec:verb-inflection.adverb-auxiliary-chain}

The adverbs mentioned in this section are 
manner-like adverbs, \acs{tam}-related adverbs and speech act-related adverbs 
(\prettyref{sec:valency.overview.dependents}),
instead of adverbial peripheral arguments.
Adverbs are never inserted between the first auxiliary (if any) and the negator.
TODO: what else?

\begin{exe}
    \ex\label{ex:auxiliary-chain-breaking-1} 
    He [is]_{\text{\category{progressive}}} [vigorously]_{\text{TODO:}} [doing]_{\text{main verb}} [his job]_{\text{object}}. 
\end{exe}

\subsection{Negation in the auxiliary chain}\label{sec:verb-inflection.negation}

The rule of the negator \form{not} is close to the rule of adverbs: 
If \form{not} is used, it is \emph{always} after the first auxiliary
(while adverbs can appear before the first auxiliary in marked cases), 
which may be the inserted \form{do} (\prettyref{ex:auxiliary-chain-breaking-2}).
Any auxiliary-\form{not} sequence may be replaced by the negative form of that auxiliary
if there is one (\prettyref{ex:auxiliary-chain-breaking-4}, \prettyref{ex:auxiliary-chain-breaking-3}).

\begin{exe}
    \ex\label{ex:auxiliary-chain-breaking-2}
    He [does]_{\text{\form{do} inserted, pres, 3sg}} [not]_{\text{negation}} love his job.
    \ex\label{ex:auxiliary-chain-breaking-4}
    He doesn't love his job.
    \ex\label{ex:auxiliary-chain-breaking-3}
    He isn't vigorously doing his job.
\end{exe}

It should be noted the surface position of the negator doesn't determine the scope of negation
\citep[\citepage{668}]{cgel}.
See, for example, the ambiguity of (\prettyref{ex:verb-inflection.negation-ambiguity-1}).
Here the ambiguity is an indicator that 
there are at least two available syntactic position of the reason clause (TODO: ref).
Another ambiguity arises when negation appears together with modality
(\prettyref{ex:verb-inflection.negation-ambiguity-2}, \prettyref{ex:verb-inflection.negation-ambiguity-3}).
This means the negator-after-first-auxiliary rule is about \emph{realization} 
and not about the underlying syntactic structure (\prettyref{sec:pos.wordhood}),
if we assume the semantic difference has structural significance.
This, together with the fact that auxiliaries have negative forms
and that the existence of \form{not} blocks subject-auxiliary inversion 
of the main verb,
may lead to the conclusion that the negator \form{not} is a quasi-verbal clitic
which is always attached after the highest verbal element.
We, however, shouldn't rush to such a conclusion,
because it's also possible that 
the rule is actually the highest verbal element is always moved \emph{before} the negator.
Note that 

TODO: the Tense - Negation - Modality - Perfect - ... sequence

\begin{exe} 
    \ex\label{ex:verb-inflection.negation-ambiguity-1} 
    I don't appoint him because he is my son. \\
    \translate{I appoint him, but because of his talent, not because his relation with me. / 
    I don't appoint him, because he's my son and I don't want to appoint him and  
    leave a bad impression on my colleagues.}
    \ex\label{ex:verb-inflection.negation-ambiguity-2}
    He shouldn't play football in the streets. \\
    \translate{It's required that he doesn't play football in the streets./
    *It's not required that he plays football in the streets,
    but he can if he wants to.} 
    \ex\label{ex:verb-inflection.negation-ambiguity-3}
    He can't play football. \\
    \translate{It's not possible/permitted that he plays football./
    *He can suppress the desire to play football.}
\end{exe}   


\subsection{Subject-auxiliary inversion}\label{sec:sai}

In interrogative sentences and in other cases (\prettyref{sec:clause-template}),  
the first auxiliary in the chain undergoes leftward movement,
often to the initial position but may be preceded by preposed constituents (\prettyref{sec:clause-template}). 
This is called \concept{subject-auxiliary inversion}.
When there is no auxiliary, 
the correct form of \form{do} carrying the tense and agreement features is inserted.

\begin{theorybox}{Subject-auxiliary inversion and ordinary constituent movement}{sai-status}
    Note that this is head movement and are often attributed to post-syntactic operations 
    in Distributed Morphology,
    making the operation kind of ``realizational'',
    and is comparable to realizational factors in morphology,
    instead of the underlying structure (\prettyref{sec:pos.wordhood}),
    while ordinary constituent movement 
    like topicalization or wh-movement 
    is comparable to the latter.
\end{theorybox}

\begin{exe}
    \ex {} [Do]_{\text{inverted auxiliary}} [you see my umbrella]_{\text{nucleus}}
    \ex Only then do we cook
\end{exe}


\subsection{Semi-auxiliaries}\label{sec:semi-auxiliary}

\subsection{Comparability with moods}\label{sec:tam-mood-compatibility}

\section{Interpretation of \acs{tam} categories}

\subsection{Minor categories}\label{sec:verb-inflection.modal-use}

\subsubsection{\form{Would}}

\section{Clausal dependents and verb frames}\label{sec:valency.overview}

\subsection{Overview}

In this section I discuss the first three rows in 
lower clausal dependents in \prettyref{tbl:clausal-dependent}.

The passive \form{by}-phrase is also an argument, not an adjunct 
(\prettyref{sec:valency.overview.by-phrase}).



\subsection{The subject}\label{sec:subject}

\subsection{Prototypical transitive and intransitive verbs}



\subsection{Dative and beneficiary ditransitive constructions}

Most ditransitive verbs are about giving and receiving,
and therefore they have A, G, T arguments (G = goal-like, T = theme-like),
where the subject in active voice is the A argument.
Both G and T have some similarities with the monotransitive object. 
The question is whether G and T show relatively uniform syntactic behavior
among various verbs
and therefore are legit labels for syntactic clausal dependent positions as well;
also, it's necessary to investigate into how properties of the monotransitive object 
are distributed between the G and T arguments.

\begin{infobox}{Mismatch between semantics and syntax}{mismatch-semantics-syntax}
    On the other hand, 
    in English, the contrast between the experiencer role and the stimulus role:
    doesn't have much syntactic significance,
    since some verbs' experiencers work like some other verbs' stimulus. TODO: ref
\end{infobox}


\subsubsection{The extended argument approach}\label{sec:blt-e-argument}

\citet{dixon2009basic1} analyzes the English ditransitive construction
with labels O and E (i.e. the less object-like internal complement).
(6) and (7) in \citet[\citesec{3.3}]{dixon2009basic1}
with semantic role labels replaced by ones in \ac{cgel} \citesec{4.2.2} 
are shown in (\ref{ex:john-gave-goods-to-charity}, \ref{ex:john-gave-student-book}).
The \acs{np}s \form{all his goods} and \form{his facorite students} 
are assigned the label O,
and the \acs{pp} \form{to charity} and the \acs{np} \form{some books}
are named E arguments.
One reason for this analysis is the constituent order.
Another piece is passivization availability:
the first two cannot be promoted to the subject position in passivization,
while the latter two can.

The \acs{np} \form{all his goods} is a theme while \form{his favorite student} is a goal,
but they have similar syntactic properties.
So it seems the division between O and E is useful in English 
while the division between G and T is not:
no goal-like and theme-like semantic role classes with stable syntactic appearance 
can be established.

\begin{exe}
    \ex \label{ex:john-gave-goods-to-charity} 
    John gave [all his goods]_{\text{O, theme}} [to charity]_{\text{E, goal}}
    \ex \label{ex:john-gave-student-book} 
    John gave [his favorite student]_{\text{O, goal}} [some books]_{\text{E, theme}}
\end{exe}


% TODO: E argument是否单独出现?是否有He speaks with a very quick pace 这样,但是后面的PP是complement的动词? 
% TODO:He treats me with kindness中,kindness应该是E论元,因此V-O-E论元结构不只有一种,依照论元种类的不同,有多种clausal structure coding的方式

\subsubsection{The direct object/indirect object approach}\label{sec:direct-indirect}

However, there is also evidence for a stable G-T contrast
in syntax and not just semantics.
This is the approach taken in \citet{cgel}. 
Applying criteria in \citet{cgel} \citesec{4.4.3},
\eqref{ex:john-gave-goods-to-charity} and \eqref{ex:john-gave-student-book}
are to be labeled as (\ref{ex:john-gave-goods-to-charity-2}, \ref{ex:john-gave-student-book-2}).
\eqref{ex:john-gave-goods-to-charity-2} is labeled according to the rule 
that indirect objects are nowhere to be found in a clause without a direct object,
and \eqref{ex:john-gave-student-book-2} is labeled according to 
\citet[\citesec{4.3} {[8]}]{cgel}.
If these labeling rules stand, 
then a stable syntactic contrast between G and T can be established:
the T argument is always the direct object,
while the G argument is less object-like:
it is either an indirect object or a prepositional complement.

\begin{exe}
    \ex \label{ex:john-gave-goods-to-charity-2} 
    John gave [all his goods]_{\text{O^{\text{d}}, theme}} [to charity]_{\text{prepositional complement, goal}}
    \ex \label{ex:john-gave-student-book-2} 
    John gave [his favorite student]_{\text{O^{\text{i}}, goal}} [some books]_{\text{O^{\text{d}}, theme}}
\end{exe}

\citet[\citesec{4.4.3}]{cgel} provides ample argumentation for the analysis shown in 
\eqref{ex:john-gave-goods-to-charity-2} and \eqref{ex:john-gave-student-book-2}.
Though \form{all his goods} and \form{his favorite student} 
have the passivization and constituent order properties of typical monotransitive objects,
only \form{all his goods} enjoys the full range of monotransitive object properties.
In the ditransitive example \eqref{ex:john-gave-student-book-2},
\form{some books} behaves like the monotransitive object 
in object postposing and preposing, the predicative adjunct construction, and controlling.
Thus the position of \form{some books} has more object properties 
than the position of \form{his favorite student}.

\subsubsection{The typology of G and T}\label{sec:g-t-typology}

To summarize, no complement type can be merged with another  
in the three types of VPs having been considered now,
namely V-O, V-G-T, and V-T-pG,
except the T argument in a V-T-pG clause and the monotransitive object.
\prettyref{fig:english-object} visualizes the resemblance 
of these complement types with the prototypical monotransitive direct object 
according to the six criteria listed in \ac{cgel} \citesec{4.4.3}.
The T argument in V-T-pG has almost identical properties
with O in a V-O construction.
Both of them, therefore, are put under the category of monotransitive object.
For the rest three complement types,
G in the V-G-T construction is similar to the monotransitive object 
in terms of passivization and constituent order,
though when the G is the beneficiary,
passivization is marginally acceptable (\ac{cgel} \citesec{4.3.3}, [10ii]),
so in \prettyref{fig:english-object} I slightly lower the G in V-G-T point.
(And for the same reason, since T in V-G-T can be marginally passivized 
-- see \ac{cgel} \citesec{4.3.3}, [10iii] --
the T in V-G-T point is slightly raised.)
If passivization and constituent order are considered as the decisive factors to distinguish complement types,
then we get the classification in \prettyref{sec:blt-e-argument}.
On the other hand, if the rest four factors are emphasized,
then the classification in \prettyref{sec:direct-indirect} works,
where G arguments have uniform behaviors 
and so do T arguments,
and we find T arguments are largely similar to O arguments.
When all the six factors are considered,
the total distance between the monotransitive object (i.e. the O argument)
and T in V-G-T is smaller than 
the distance between the monotransitive object and G in V-G-T,
and the former two are clustered into the \term{direct object}.
Finally, the direct object is clustered with G in V-G-T 
-- or the \term{indirect object}, correspondingly -- 
and we get the final object concept.
G in V-T-pG is far from the rest four points 
and hence is placed out of the range of objects.

\begin{figure}[H]
    \centering
    

\tikzset{every picture/.style={line width=0.3pt}} %set default line width to 0.75pt        

\begin{tikzpicture}[x=0.75pt,y=0.75pt,yscale=-0.8,xscale=0.8]
%uncomment if require: \path (0,500); %set diagram left start at 0, and has height of 500

%Straight Lines [id:da5593585477377434] 
\draw    (107,391) -- (506,391) ;
\draw [shift={(508,391)}, rotate = 180] [fill={rgb, 255:red, 0; green, 0; blue, 0 }  ][line width=0.08]  [draw opacity=0] (12,-3) -- (0,0) -- (12,3) -- cycle    ;
%Straight Lines [id:da7245999585667466] 
\draw    (107,391) -- (107,97.59) ;
\draw [shift={(107,95.59)}, rotate = 90] [fill={rgb, 255:red, 0; green, 0; blue, 0 }  ][line width=0.08]  [draw opacity=0] (12,-3) -- (0,0) -- (12,3) -- cycle    ;
%Straight Lines [id:da8964896003323979] 
\draw  [dash pattern={on 4.5pt off 4.5pt}]  (48,250) -- (494,250) ;
%Straight Lines [id:da9625166392954305] 
\draw  [dash pattern={on 4.5pt off 4.5pt}]  (299,442.59) -- (299,120.59) ;
%Shape: Ellipse [id:dp7384360040390137] 
\draw  [color={rgb, 255:red, 155; green, 155; blue, 155 }  ,draw opacity=1 ][fill={rgb, 255:red, 155; green, 155; blue, 155 }  ,fill opacity=0.2 ] (343,133.43) .. controls (343,99.46) and (375.46,71.93) .. (415.5,71.93) .. controls (455.54,71.93) and (488,99.46) .. (488,133.43) .. controls (488,167.39) and (455.54,194.93) .. (415.5,194.93) .. controls (375.46,194.93) and (343,167.39) .. (343,133.43) -- cycle ;
%Shape: Ellipse [id:dp035938837693224146] 
\draw  [color={rgb, 255:red, 155; green, 155; blue, 155 }  ,draw opacity=1 ][fill={rgb, 255:red, 155; green, 155; blue, 155 }  ,fill opacity=0.2 ] (308,194.43) .. controls (308,120.14) and (355.46,59.93) .. (414,59.93) .. controls (472.54,59.93) and (520,120.14) .. (520,194.43) .. controls (520,268.71) and (472.54,328.93) .. (414,328.93) .. controls (355.46,328.93) and (308,268.71) .. (308,194.43) -- cycle ;
%Shape: Ellipse [id:dp04989243684342659] 
\draw  [color={rgb, 255:red, 155; green, 155; blue, 155 }  ,draw opacity=1 ][fill={rgb, 255:red, 155; green, 155; blue, 155 }  ,fill opacity=0.2 ] (142,194.93) .. controls (142,168.97) and (174.68,147.93) .. (215,147.93) .. controls (255.32,147.93) and (288,168.97) .. (288,194.93) .. controls (288,220.88) and (255.32,241.93) .. (215,241.93) .. controls (174.68,241.93) and (142,220.88) .. (142,194.93) -- cycle ;
%Shape: Polygon Curved [id:ds6446496039380643] 
\draw  [color={rgb, 255:red, 155; green, 155; blue, 155 }  ,draw opacity=1 ][fill={rgb, 255:red, 155; green, 155; blue, 155 }  ,fill opacity=0.2 ] (301,63.26) .. controls (349,45.26) and (450,5.26) .. (505,83.26) .. controls (560,161.26) and (568,275.59) .. (468,335.93) .. controls (368,396.26) and (124,281.26) .. (120,207.26) .. controls (116,133.26) and (253,81.26) .. (301,63.26) -- cycle ;

% Text Node
\draw (131,92.59) node [anchor=south] [inner sep=0.75pt]   [align=left] {passivization,\\constituent order};
% Text Node
\draw (510,391) node [anchor=west] [inner sep=0.75pt]   [align=left] {postposing,\\preposing,\\gap controlling,\\predicative adjunct};
% Text Node
\draw (176,172) node [anchor=north west][inner sep=0.75pt]   [align=left] {G in V-G-T};
% Text Node
\draw (174,330) node [anchor=north west][inner sep=0.75pt]   [align=left] {G in V-T-pG};
% Text Node
\draw (372,134) node [anchor=north west][inner sep=0.75pt]   [align=left] {O in V-O};
% Text Node
\draw (372,156) node [anchor=north west][inner sep=0.75pt]   [align=left] {T in V-T-pG};
% Text Node
\draw (366,287) node [anchor=north west][inner sep=0.75pt]   [align=left] {T in V-G-T};
% Text Node
\draw (215,424) node [anchor=north west][inner sep=0.75pt]  [color={rgb, 255:red, 208; green, 2; blue, 27 }  ,opacity=1 ] [align=left] {G};
% Text Node
\draw (397,424) node [anchor=north west][inner sep=0.75pt]  [color={rgb, 255:red, 208; green, 2; blue, 27 }  ,opacity=1 ] [align=left] {O=T};
% Text Node
\draw (89.5,170.5) node [anchor=north west][inner sep=0.75pt]  [color={rgb, 255:red, 245; green, 166; blue, 35 }  ,opacity=1 ,rotate=-90] [align=left] {O};
% Text Node
\draw (89.5,317.5) node [anchor=north west][inner sep=0.75pt]  [color={rgb, 255:red, 245; green, 166; blue, 35 }  ,opacity=1 ,rotate=-90] [align=left] {E};
% Text Node
\draw (32,143) node [anchor=north west][inner sep=0.75pt]  [color={rgb, 255:red, 245; green, 166; blue, 35 }  ,opacity=1 ,rotate=-90] [align=left] {The extended argument approach};
% Text Node
\draw (229,462) node [anchor=north west][inner sep=0.75pt]  [color={rgb, 255:red, 208; green, 2; blue, 27 }  ,opacity=1 ] [align=left] {The S, A, O, G, T approach};
% Text Node
\draw (363,89) node [anchor=north west][inner sep=0.75pt]  [color={rgb, 255:red, 155; green, 155; blue, 155 }  ,opacity=1 ] [align=left] {monotransitive \\object};
% Text Node
\draw (370,225) node [anchor=north west][inner sep=0.75pt]  [color={rgb, 255:red, 155; green, 155; blue, 155 }  ,opacity=1 ] [align=left] {direct object};
% Text Node
\draw (164,201) node [anchor=north west][inner sep=0.75pt]  [color={rgb, 255:red, 155; green, 155; blue, 155 }  ,opacity=1 ] [align=left] {indirect object};
% Text Node
\draw (240,111) node [anchor=north west][inner sep=0.75pt]  [color={rgb, 255:red, 155; green, 155; blue, 155 }  ,opacity=1 ] [align=left] {object};


\end{tikzpicture}

    \caption{Classification of internal complements in English V-O, V-G-T, and V-T-pG clauses.
    The orange labels are discussed in \prettyref{sec:direct-indirect},
    and the red labels are discussed in \prettyref{sec:blt-e-argument}.
    The grey blobs indicate clustering of the points.}
    \label{fig:english-object}
\end{figure}

So here we see why \term{object} is a useful concept in English grammar,
at least among V-O, V-G-T, and V-T-pG clauses:
it can be defined both via form (i.e. no preposition and not predicative)
and via function (the aforementioned six factors),
and the two definitions happen to include the same complement types.
If we try to doing away with the abstract concept of \term{object}
and only keep notions like monotransitive object,
then the coincidence has the following equivalence formulation:
a complement slot prototypically filled by non-predicative NPs
always share some or all grammatical properties
about passivization, canonical constituent order,
preposing and postposing, gap controlling, 
and being able to be modified by a predicative adjunct. 

There are other constructions in which an object position can be distinguished,
including PPs and clauses with both object and predicative complement. % TODO
Whether in these constructions the term \term{object} still hints 
certain resemblance with the prototypically monotransitive object 
is a question to be answered when these constructions are discussed about.


Typological significance for the two possible analyses is discussed in
\citefootnote{26} in \ac{cgel} \citechap{4}.
Some languages assign the direct object position to the goal-like argument.
Were English such a language, 
it would follow that \eqref{ex:john-gave-student-book} would be the correct analysis,
and since the subcategorization frame of \eqref{ex:john-gave-goods-to-charity} 
prohibits assignment of the object status to the goal argument
(the object cannot be oblique),
in \eqref{ex:john-gave-goods-to-charity},
\form{all his goods} would be the subject,
and hence both \eqref{ex:john-gave-goods-to-charity} and \eqref{ex:john-gave-student-book} 
would be correct,
and we see a split G role.
If, however, English belongs to the type of languages 
that assign the direct object position to the theme-like argument,
we immediately get \eqref{ex:john-gave-goods-to-charity-2} and \eqref{ex:john-gave-student-book-2}.
Since the position of \form{some books} has more object properties 
than the position of \form{his favorite student},
\eqref{ex:john-gave-goods-to-charity-2} and \eqref{ex:john-gave-student-book-2} are preferable,
but \eqref{ex:john-gave-goods-to-charity} and \eqref{ex:john-gave-student-book} also make sense 
to some extents.
This means English belongs to the class which identify T with the monotransitive O, 
but it is not a clear-cut member.

Clausal complementation of ditransitive verbs, therefore, is summarized in \prettyref{fig:ditransitive-gt}.
Note that in the diagram there are four -- not five -- internal complement types,
because the O in V-O is identified with the T in V-G-T.

\begin{figure}[H]
    \centering
    

\tikzset{every picture/.style={line width=0.3pt}} %set default line width to 0.75pt        

\begin{tikzpicture}[x=0.75pt,y=0.75pt,yscale=-0.85,xscale=0.85]
%uncomment if require: \path (0,513); %set diagram left start at 0, and has height of 513

%Straight Lines [id:da5590373900481334] 
\draw [color={rgb, 255:red, 74; green, 144; blue, 226 }  ,draw opacity=1 ][line width=3]    (132,62) -- (157.72,62) ;
%Straight Lines [id:da6034249079442651] 
\draw [color={rgb, 255:red, 80; green, 227; blue, 194 }  ,draw opacity=1 ][line width=3]    (191.72,62) -- (262.72,62) ;
%Straight Lines [id:da3340903560309998] 
\draw [color={rgb, 255:red, 184; green, 233; blue, 134 }  ,draw opacity=1 ][line width=3]    (297.72,62) -- (426.72,62) ;
%Curve Lines [id:da7559551700584779] 
\draw [color={rgb, 255:red, 80; green, 227; blue, 194 }  ,draw opacity=1 ]   (257.72,109.97) .. controls (258.72,168.97) and (231.72,155.97) .. (232.72,211.97) ;
%Curve Lines [id:da6470497609004202] 
\draw [color={rgb, 255:red, 80; green, 227; blue, 194 }  ,draw opacity=1 ]   (203.72,109.97) .. controls (204.72,168.97) and (92.72,250.94) .. (82.72,395.94) ;
%Curve Lines [id:da9535821756090674] 
\draw [color={rgb, 255:red, 74; green, 144; blue, 226 }  ,draw opacity=1 ]   (144.72,109.97) .. controls (145.72,168.97) and (77.72,291.94) .. (82.72,395.94) ;
%Curve Lines [id:da4943116446528697] 
\draw [color={rgb, 255:red, 184; green, 233; blue, 134 }  ,draw opacity=1 ]   (312.72,108.97) .. controls (313.72,167.97) and (117.72,178.83) .. (82.72,395.94) ;
%Curve Lines [id:da034641113672127855] 
\draw [color={rgb, 255:red, 248; green, 231; blue, 28 }  ,draw opacity=1 ]   (365.72,108.97) .. controls (366.72,167.97) and (226.72,169.97) .. (232.72,211.97) ;
%Curve Lines [id:da5196681080395307] 
\draw [color={rgb, 255:red, 248; green, 231; blue, 28 }  ,draw opacity=1 ]   (419.72,108.97) .. controls (420.72,167.97) and (594.72,167.97) .. (600.72,209.97) ;
%Straight Lines [id:da23388092728957277] 
\draw [color={rgb, 255:red, 248; green, 231; blue, 28 }  ,draw opacity=1 ][line width=3]    (297.72,54) -- (426.72,54) ;
%Curve Lines [id:da13093524389220645] 
\draw [color={rgb, 255:red, 184; green, 233; blue, 134 }  ,draw opacity=1 ]   (365.72,108.97) .. controls (366.72,167.97) and (359.72,154.97) .. (360.72,210.97) ;
%Curve Lines [id:da2405074684879065] 
\draw [color={rgb, 255:red, 184; green, 233; blue, 134 }  ,draw opacity=1 ]   (419.72,108.97) .. controls (420.72,167.97) and (485.72,154.97) .. (486.72,210.97) ;
%Curve Lines [id:da5399897307832593] 
\draw [color={rgb, 255:red, 155; green, 155; blue, 155 }  ,draw opacity=1 ]   (231,289) .. controls (229.72,317.57) and (309.72,293.57) .. (306.72,345.57) ;
%Curve Lines [id:da43486300562119107] 
\draw [color={rgb, 255:red, 155; green, 155; blue, 155 }  ,draw opacity=1 ]   (363.72,289.94) .. controls (362.43,318.5) and (309.72,293.57) .. (306.72,345.57) ;
%Curve Lines [id:da9115056023887791] 
\draw [color={rgb, 255:red, 155; green, 155; blue, 155 }  ,draw opacity=1 ]   (306,377) .. controls (306.72,400.94) and (390.72,380.94) .. (389.72,410.94) ;
%Curve Lines [id:da27935753858234835] 
\draw [color={rgb, 255:red, 155; green, 155; blue, 155 }  ,draw opacity=1 ]   (487.72,287.57) .. controls (491.72,337.57) and (391.72,335.94) .. (389.72,410.94) ;
%Curve Lines [id:da7115308344150204] 
\draw [color={rgb, 255:red, 155; green, 155; blue, 155 }  ,draw opacity=1 ] [dash pattern={on 4.5pt off 4.5pt}]  (231,289) .. controls (221.72,313.57) and (511.72,360.57) .. (508.72,412.57) ;
%Curve Lines [id:da31095802179111853] 
\draw [color={rgb, 255:red, 155; green, 155; blue, 155 }  ,draw opacity=1 ] [dash pattern={on 4.5pt off 4.5pt}]  (487.72,287.57) .. controls (486.43,316.13) and (507.72,356.57) .. (508.72,412.57) ;
%Curve Lines [id:da630655999182743] 
\draw [color={rgb, 255:red, 155; green, 155; blue, 155 }  ,draw opacity=1 ] [dash pattern={on 4.5pt off 4.5pt}]  (363.72,289.94) .. controls (362.43,318.5) and (627.72,357.57) .. (624.72,409.57) ;
%Curve Lines [id:da5122649297059649] 
\draw [color={rgb, 255:red, 155; green, 155; blue, 155 }  ,draw opacity=1 ] [dash pattern={on 4.5pt off 4.5pt}]  (601.72,290.57) .. controls (600.43,319.13) and (623.72,353.57) .. (624.72,409.57) ;

% Text Node
\draw (140,78) node [anchor=north west][inner sep=0.75pt]   [align=left] {S};
% Text Node
\draw (197,78) node [anchor=north west][inner sep=0.75pt]   [align=left] {A};
% Text Node
\draw (249,78) node [anchor=north west][inner sep=0.75pt]   [align=left] {O};
% Text Node
\draw (304,78) node [anchor=north west][inner sep=0.75pt]   [align=left] {A};
% Text Node
\draw (412,78) node [anchor=north west][inner sep=0.75pt]   [align=left] {G};
% Text Node
\draw (360,78) node [anchor=north west][inner sep=0.75pt]   [align=left] {T};
% Text Node
\draw (68,414.5) node [anchor=north west][inner sep=0.75pt]   [align=left] {subject};
% Text Node
\draw (182,230.75) node [anchor=north west][inner sep=0.75pt]   [align=left] {\begin{minipage}[lt]{68.99pt}\setlength\topsep{0pt}
\begin{center}
monotransitive\\object
\end{center}

\end{minipage}};
% Text Node
\draw (328,220.25) node [anchor=north west][inner sep=0.75pt]   [align=left] {\begin{minipage}[lt]{53.13pt}\setlength\topsep{0pt}
\begin{center}
ditransitive\\direct\\object
\end{center}

\end{minipage}};
% Text Node
\draw (462,230.75) node [anchor=north west][inner sep=0.75pt]   [align=left] {\begin{minipage}[lt]{36.67pt}\setlength\topsep{0pt}
\begin{center}
indirect\\object
\end{center}

\end{minipage}};
% Text Node
\draw (560,220.25) node [anchor=north west][inner sep=0.75pt]   [align=left] {\begin{minipage}[lt]{60.22pt}\setlength\topsep{0pt}
\begin{center}
specified\\prepositional\\complement
\end{center}

\end{minipage}};
% Text Node
\draw (260,355) node [anchor=north west][inner sep=0.75pt]   [align=left] {direct object};
% Text Node
\draw (374,414.5) node [anchor=north west][inner sep=0.75pt]   [align=left] {object};
% Text Node
\draw (469,415) node [anchor=north west][inner sep=0.75pt]   [align=left] {Dixon's O};
% Text Node
\draw (595,415) node [anchor=north west][inner sep=0.75pt]   [align=left] {Dixon's E};


\end{tikzpicture}

    \caption{English alignment concerning S, A, O, G, T. Each color of lines means one canonical clause which codes a type of argument structure.}
    \label{fig:ditransitive-gt}
\end{figure}

\subsection{Prepositional object}

Verb-preposition constructions and verb-particle%
\footnote{
    The term \term{particle} here covers intransitive prepositions;
    the term \term{preposition} is used to cover transitive prepositions.
    Although strictly speaking, 
    this terminology confuses form and function 
    (prepositions are a word class, 
    and can be used intransitively in some cases),
    I choose to do so to keep the notation consistent with 
    the current grammar writing practice.
}
constructions
can be classified according to the following parameters
\citep[\citepages{272-274}]{cgel}:
\begin{itemize}
    \item whether it's a transitive preposition or a particle 
    (an intransitive preposition, or something else),
    \item whether the construction can be interpreted in a compositional way or 
    has already gained an established (idiomatic) meaning,
    \item how the choice of preposition/particle is restricted by the verb,  
    \item the mobility of the preposition/particle 
    in, say, \formcat{wh}-movements, and 
    \item complement-related properties of the associated \acs{np} coming with the preposition/particle,
    like whether it can be passivized.
\end{itemize}


Concerning verbs coming with a single preposition,
it can be seen that if a verb doesn't specify the preposition following it, 
the preposition is always mobile.
Thus we have a three-fold classification:
\begin{itemize}
    \item verbs with non-specified prepositions,
    \item verbs with specified but mobile prepositions 
    (\concept{preposition verbs with mobile prepositions}; \citealp[\citepage{273}]{cgel}), and 
    \item verbs with specified and fixed prepositions 
    (\concept{fossilized preposition verbs}; \citealp[\citepage{277}]{cgel}).
\end{itemize}
Note that a non-specified prepositional phrase is still a complement \citep[\citepage{273}]{cgel}:
we say it's an oblique argument.

The parameters of established meaning and complement properties
are largely independent to the classification made above.
Passivization is completely not predictable 
from the classification made above \citep[\citepage{276} {[11]}]{cgel}.
Fossilized verb-preposition constructions are usually idioms,
but some, like \form{break with}, 
still have largely inferrable meaning;
the same applies for verbs with specified prepositions
(indeed, the presence of a specified preposition introduces a sense of 
directed volition \citep[\citepage{293}]{dixon2005semantic});
verbs with non-specified prepositions usually are less idiom-like,
but this is because if they are idiomatic enough,
we will recognize them as verbs with specified prepositions.

The classification of verb-preposition constructions,
therefore, is given in \prettyref{tbl:verb-prep-construction}.
The examples used in the table is based on \citet[\citepage{278}, {[17]}]{cgel}.

\begin{table}[H]
    \centering
    \caption{Classification of verb-single-preposition constructions}
    \label{tbl:verb-prep-construction}
    \begin{tabular}{lllll}
    \toprule
    &       &       & \multicolumn{2}{c}{specified preposition} \\ \cmidrule(l){4-5} 
    \multirow{-2}{*}{\begin{tabular}[c]{@{}l@{}}passivization of NP \\ after preposition\end{tabular}} & \multirow{-2}{*}{idiom} & \multirow{-2}{*}{\begin{tabular}[c]{@{}l@{}}non-specified \\ preposition\end{tabular}} & mobile             & fixed                \\ \midrule
                                                                                                       & yes                     & \cellcolor[HTML]{C0C0C0}\form{}                                                      & \form{call on?}          & \form{see to}      \\
    \multirow{-2}{*}{yes}                                                                              & no                      & \form{sleep in}                                                                      & \form{refer to}  & \form{fuss over}            \\
                                                                                                       & yes                     & \cellcolor[HTML]{C0C0C0}\form{}                                                      & \form{stand for} & \form{come across} \\
    \multirow{-2}{*}{no}                                                                               & no                      & \form{fly to/from}                                                                   & \form{feel for}          & \form{come into}            \\ \bottomrule
    \end{tabular}
\end{table}

Beside the classification given by \prettyref{tbl:verb-prep-construction},
another parameter is the origin of preposition verb constructions.
Some of them are similar to verbs licensing oblique cases \emph{only},
with no verb-object subcategorization
(found in languages with rich case morphology, like Latin),
like \form{refer to},
the verb parts of which rarely appear alone or with other prepositions.
For others, like \form{see to},
the verb part of the construction (usually a simple, monosyllabic one)
does appear alone or with other prepositions.
In the first case,
the ``idiom-or-not'' parameter is actually not so important,
because we can consider the preposition as a part of the verb lexeme,
while in the second case,
the parameter is important,
because \form{stand at the door} is of course not idiomatic,
while \form{stand for} has an established meaning.

By the criteria for object above (\prettyref{fig:english-object}),
we can find that the complement introduced by the preposition of a preposition verb 
is object-like, 
and therefore preposition verbs are transitive
(\citealt[\citepage{291}, \citepage{297}]{dixon2005semantic};
\citealt[\citepage{277}]{cgel});
hence the title of this section.

\subsubsection{The \form{by} complement in passivization}\label{sec:valency.overview.by-phrase}

\subsection{Verb-particle constructions}

English verb-particle constructions or \concept{phrasal verb}%
\footnote{
    There is ambiguity in what the term means.
    A verb selecting a preposition, 
    for example, may also be regarded as a phrasal verb.
    \citet{cgel} rejects the term \term{phrasal verb} 
    as what is a phrase in constituency analysis is not \form{play upon} 
    but the whole verb phrase \form{play upon the fact};
    since this note also recognizes the phrase-as-generative-phase conception of the term \term{phrase},
    the term \term{phrasal verb} will still be used.
}
has 

\section{Agreement}

If you take a closer look to how native speakers of English do subject-verb agreement,
you'll find some more subtle details than 
the textbook rule that when the tense is \acl{present}
and the subject is 3sg, 
\form{-s} is added to the first auxiliary or the main verb
\citep[\citechap{5}, \citesec{18}]{cgel}.

\section{Voice} 

One thing that happens in the verb phrase 
and strongly influences the structural building process is the category \emph{voice}.
English doesn't have a rich set of valency changing devices,
and the active-passive distinction is the only regular valency changing mechanism.
There are other alternations of verb valency,
but they are much more strongly determined by the lexicon. 

\subsection{Passivized argument}

Passivization depends on - though not in a very apparent way -- the properties of the main verb. 
TODO: passivized argument

\chapter{Adjectives and adverbs}

\section{The structure of the adverb}

\subsection{The \form{-ly} derivation}

TODO: terminal derivation; only \form{-c-al-ly} is acceptable,
except \form{puclicly}.

\section{Comparative constructions}

Comparative constructions appear to fill a ``degree'' position
of adverbs, adjectives, and sometimes, nouns and verbs.
TODO: compare with \form{how old} constructions  

In some languages, like Mandarin, 
the seemingly counterpart of \form{than} 
has a different role: 
the \acs{np} it introduces is not the thing or person that is directly compared, 
but the \emph{possessor} of the thing or person that is directly compared
(\ref{ex:modify.comparative.mandarin-1}).
Speakers of these languages may transfer this usage 
to the way they use \form{than},
leading to the rise of the ungrammatical form (\ref{ex:modify.comparative.mandarin-induced-1}).

\begin{exe}
    \ex\label{ex:modify.comparative.mandarin-1} \gll 我 的 成绩 [比] 你 好 \\
    1 \category{poss} score \category{prep} 2 good \\
    \glt{My score is better than yours. (lit. my score is better [than the counterpart of what's compared of]  you.)}
    \ex\label{ex:modify.comparative.mandarin-induced-1} *My score is better than you.
\end{exe}

\chapter{Simple clauses}\label{chap:simple-clause}

After several chapters about each part of the clause,
this chapter discusses how the parts are assembled into one.
The details of how a clause is embedded into another are not covered in this chapter
-- they are covered in \prettyref{chap:clause-combining}.

\section{Overview of clause structure}\label{sec:clause-template}

\subsection{The template}

The template of English clause structure is shown in \prettyref{fig:clause-template}.
The figure displays the four rough levels of clause structure.
Each layer in \prettyref{fig:clause-template} as well as justification of them,
if not described in chapters above, are described 
in the rest of this chapter.

The first layer contains the verb-argument (core or peripheral) grammatical relations,
\ac{tam} marking (by inflection, auxiliary construction, or adverbs), and negation.
In structuralist tradition as is described in \citet{cgel},
this layer is the \concept{verb phrase}.%
\footnote{
    Dixon argues against using the term \term{verb phrase} in the sense of this note;
    his \term{verb phrase} is \prettyref{tbl:auxiliary-chain}.
    The two definitions of \term{verb phrase} are all frequent in modern descriptive grammars.
    When the term \term{verb phrase} is used in the sense in \prettyref{fig:clause-template},
    Dixon's verb phrase is sometimes called the \term{verb complex} \citep{Friesen2017}.

    Another terminology issue is many people -- like Dixon -- use the term \term{predicate}
    for the syntactic function of the verb complex
    (i.e. the realization of functional heads),
    while others use it for the syntactic function of the verb phrase 
    (i.e. a lower part of the TP -- see \prettyref{sec:vp-tp-cp}).
    To avoid this endless confusion, 
    I will just avoid the notion of \term{predicate} as much as possible.
    I then confuse \emph{function} (predicate)
    with \emph{form} (verb phrase).
    However, English verb phrases -- roughly \vP{} after case assignment, etc. -- 
    almost never appear outside a clause,
    and it doesn't provide additional information 
    to introduce separate terms for form and function in \prettyref{fig:clause-template}.
    This is also the practice taken in most works adopting the notion of verb phrase,
    like \citet{Friesen2017}.
}
It contains the auxiliary chain and the main verb 
(\prettyref{sec:auxiliary-chain}),
internal complements (\prettyref{sec:valency.overview}),
and adverbials that prototypically appear in the clause-final position (TODO: ref).
Note that that these clause-final adverbials 
appear \emph{higher} than the internal complements 
in the constituency tree:
The former are after the latter for some other reasons (TODO).
The first layer has several sub-layers:
First the core argument structure,
then peripheral arguments, 
then auxiliary verbs and negation and also \acs{tam} marking by adverbs,
and also the category of voice. 

The second and the third layers shown in \prettyref{fig:clause-template}
are much slimmer than the first layer.
They are shown as separate layers mainly because 
the subject-predicate relation 
and the subject-auxiliary inversion 
traditionally gain more attention.
The second layer highlights the prominent status of the subject (\prettyref{sec:subject}).
A subject plus a verb phrase is a \concept{nucleus clause}.
A declarative clause without information packaging operations
can just be a nucleus clause without further syntactic operations.
The third layer is optional:
it arises when subject-auxiliary inversion happens (\prettyref{sec:sai}),
which is the case in %Enumerating: all cases in which SAI happens 
question formation (\prettyref{sec:simple-clause.interrogative.formation}).

The fourth layer is also optional and may have several preposed constituents,
each of which may be preposed by a different reason,
and interacts freely with the subject-auxiliary inversion.
This is also a fat layer:
There exist several types of preposing operations (\prettyref{sec:simple-clause.derivation.preposing}),
and the layer also contains some high-level adverbials
which are about speech force, etc., %Enumerating: high-level adverbials
like \form{frankly} (\prettyref{sec:pos.adverb}).

\begin{figure}[H]
    \centering
    {\small \begin{tikzpicture}[x=0.75pt,y=0.75pt,yscale=-1,xscale=1]
    %uncomment if require: \path (0,300); %set diagram left start at 0, and has height of 300
    
    %Straight Lines [id:da48517433952426825] 
    \draw [color={rgb, 255:red, 144; green, 19; blue, 254 }  ,draw opacity=1 ][fill={rgb, 255:red, 144; green, 19; blue, 254 }  ,fill opacity=1 ][line width=3]    (368,138.93) -- (368,232.93) ;
    %Straight Lines [id:da4265308922026523] 
    \draw [color={rgb, 255:red, 189; green, 16; blue, 224 }  ,draw opacity=1 ][fill={rgb, 255:red, 189; green, 16; blue, 224 }  ,fill opacity=1 ][line width=3]    (315,110.93) -- (315,232.93) ;
    %Straight Lines [id:da30928832159485653] 
    \draw [color={rgb, 255:red, 74; green, 144; blue, 226 }  ,draw opacity=1 ][fill={rgb, 255:red, 74; green, 144; blue, 226 }  ,fill opacity=1 ][line width=3]    (260,77.93) -- (260,232.93) ;
    %Straight Lines [id:da6215927356131927] 
    \draw [color={rgb, 255:red, 100; green, 157; blue, 224 }  ,draw opacity=1 ][fill={rgb, 255:red, 189; green, 16; blue, 224 }  ,fill opacity=1 ][line width=3]    (203,34.93) -- (203,232.93) ;
    
    % Text Node
    \draw (338,105) node [anchor=north west][inner sep=0.75pt]  [color={rgb, 255:red, 189; green, 16; blue, 224 }  ,opacity=1 ] [align=left] {subject};
    % Text Node
    \draw (394,133) node [anchor=north west][inner sep=0.75pt]  [color={rgb, 255:red, 144; green, 19; blue, 254 }  ,opacity=1 ] [align=left] {auxliary chain (and negator and/or adverbials)};
    % Text Node
    \draw (432,161) node [anchor=north west][inner sep=0.75pt]  [color={rgb, 255:red, 144; green, 19; blue, 254 }  ,opacity=1 ] [align=left] {main verb};
    % Text Node
    \draw (471,189) node [anchor=north west][inner sep=0.75pt]  [color={rgb, 255:red, 144; green, 19; blue, 254 }  ,opacity=1 ] [align=left] {internal complements};
    % Text Node
    \draw (288,75) node [anchor=north west][inner sep=0.75pt]  [color={rgb, 255:red, 74; green, 144; blue, 226 }  ,opacity=1 ] [align=left] {fronted auxiliary};
    % Text Node
    \draw (501,217) node [anchor=north west][inner sep=0.75pt]  [color={rgb, 255:red, 144; green, 19; blue, 254 }  ,opacity=1 ] [align=left] {adverbials};
    % Text Node
    \draw (232,30) node [anchor=north west][inner sep=0.75pt]  [color={rgb, 255:red, 100; green, 157; blue, 224 }  ,opacity=1 ] [align=left] {preposed constituents and high-level adverbials};
    % Text Node
    \draw (365,230.93) node [anchor=north east] [inner sep=0.75pt]  [color={rgb, 255:red, 74; green, 144; blue, 226 }  ,opacity=1 ,rotate=-90] [align=left] {verb phrase};
    % Text Node
    \draw (312,230.93) node [anchor=north east] [inner sep=0.75pt]  [color={rgb, 255:red, 189; green, 16; blue, 224 }  ,opacity=1 ,rotate=-90] [align=left] {nucleus clause};
    % Text Node
    \draw (257,230.93) node [anchor=north east] [inner sep=0.75pt]  [color={rgb, 255:red, 189; green, 16; blue, 224 }  ,opacity=1 ,rotate=-90] [align=left] {subject-auxiliary inversion};
    % Text Node
    \draw (200,230.93) node [anchor=north east] [inner sep=0.75pt]  [color={rgb, 255:red, 189; green, 16; blue, 224 }  ,opacity=1 ,rotate=-90] [align=left] {preposing};
    
    
    \end{tikzpicture}
    }
    \caption{English clause structure (the indentation means linear order and not constituency relations)}
    \label{fig:clause-template}
\end{figure}

\subsection{Clausal dependents}

As is shown in the figure,
putting purely grammatical items (like auxiliary verbs) aside,
clausal dependents are traditionally divided into arguments (or ``complements'') 
and adjuncts (i.e. adverbials).
Both types have many diverse subtypes,
and sometimes, 
a subtype of arguments and a subtype of adjuncts can be quite similar.
Similar to the case in other languages, 
clausal dependents in English can be summarized as \prettyref{tbl:clausal-dependent}.

The first three rows are already discussed in \prettyref{sec:valency.overview}. TODO: place to discuss peripheral arguments
Prototypical core argument roles, 
like agent, patient, theme, etc.,
are always clausal complements and never adjuncts.
Prototypical peripheral roles, like location or instrument, can also be arguments,
but they can also be adjuncts.
The concepts of manner, whether the action in question creates frustration 
(e.g. \form{I spent the whole day working on that problem [in vain]}), etc.
are sometimes grammaticalized as arguments,
as in \form{we were treated [badly]},
without which the clause is not grammatical,
but more frequently they are adjuncts.

So-called peripheral arguments
are adjuncts with prototypical peripheral semantic role
or adjuncts about manner or frustrative expressions like \form{in a stupid way}:
the latter can still be asked about (\prettyref{ex:simple-clause.dependents.ex-1}), 
just like prototypical peripheral arguments (\prettyref{ex:simple-clause.dependents.ex-2}). 
The cell corresponding to prototypical peripheral arguments
is colored blue in \prettyref{tbl:clausal-dependent};
the cell corresponding to less-prototypical peripheral arguments
is colored light blue in \prettyref{tbl:clausal-dependent}.

\begin{exe}
    \ex\label{ex:simple-clause.dependents.ex-1} - [How] did they treat you? - They treated us [quite badly].
    \ex\label{ex:simple-clause.dependents.ex-2} - [Where] did they detained you? - They detailed us [in a building near the sea].
\end{exe}

There also exists oblique arguments,
which are complements with oblique case marking, i.e. not nominative or accusative.
They can be found in cells that are colored dark, light, and pale green in \prettyref{tbl:clausal-dependent}.

\begin{exe}
    \ex\label{ex:simple-clause.dependents.ex-3} - How did you finish this article? 
    - I finished it with LaTeX.
\end{exe}

\acs{tam} adverbials are adverbials that mark the \acs{tam} categories 
in the way that can be also found in \prettyref{sec:verbal-complex}.
They are usually quite limited in variation,
resembling the tense or aspect system in the verbal complex.
Adverbials like \form{yesterday} or \form{in that very moment} (TODO: ref)
seem to be peripheral arguments, instead of a part of \acs{tam} marking devices,
because their syntactic functions allow too much variation 
and therefore can't be captured by that kind of feature combination 
(``S before R'' or ``S=R'', etc.)
usually seen in \acs{tam} devices. 


\begin{table}[H]
    \caption{English clausal dependents}
    \label{tbl:clausal-dependent}
    \centering
    {\small
    \begin{tabular}{ccc}
    \toprule
    \multirow{2}{*}{meaning}      & \multicolumn{2}{c}{syntactic position}                                                       \\ \cmidrule{2-3}
                                  & argument (i.e. complement)                             & adjunct                                             \\ \midrule
    prototypical core roles       & \cellcolor[HTML]{32CB00}\form{[I] loves [that apartment]}             & \\ 
    prototypical peripheral role  & \cellcolor[HTML]{34FF34}\form{She lives [in that apartment]} & \cellcolor[HTML]{34CDF9}\form{The machine is fixed with this new tool}    \\
    manner, frustrative, etc.     & \cellcolor[HTML]{DAE8FC}\cellcolor[HTML]{67FD9A}\form{We were treated [quite badly]} & \cellcolor[HTML]{DAE8FC}\form{He answered the question in a silly manner} \\
    \acs{tam}-related adverbials &                                        & \cellcolor[HTML]{ECF4FF}\form{I [always] feel tired}                      \\
    peripheral adverbials  &                                        & \form{[Frankly], I think you are fooled by them} \\ \bottomrule
    \end{tabular}}
\end{table}

\begin{infobox}{The term \term{argument} and the argument-adjunct distinction}{argument-terminology}
    Note that here is a terminological confusion:
    The term \term{argument} is used sometimes 
    as opposed to more grammatical clausal components like \acs{tam} adverbs
    (as in \term{peripheral argument}, 
    i.e. any specifier positions that allows large variation with regard to its content),
    and sometimes as opposed to \term{adjunct},
    i.e. an element that doesn't have very strong relation with the verb.
    Here we have two descriptive parameters when we talk about arguments:
    one is the ability of variation
    (core arguments, peripheral arguments, oblique arguments can be filled by diverse constituents,
    while \acs{tam} adverbials only allow a limited number of adverbs),
    and the other other is the closeness to the lexical head, which is the main verb here 
    (core arguments, oblique arguments are closely related to the main verb,
    while \acs{tam} adverbials and peripheral arguments are not).
    The parameter of closeness to the lexical head is the parameter used for argument-adjunct distinction
    in \prettyref{tbl:clausal-dependent}.

    \citet[\citepage{732}]{quirk1985} says we need gradient analysis 
    in cases like \form{we were treated [quite badly]}.
    This is correct, but doesn't say much about the essence of English grammar:
    At a given time, for a given speaker,
    we can still tell how close the constituent \form{quite badly} is to the verb.
    Here, the requirement of gradience 
    comes from the inherent deficiency of the terms \term{argument} and \term{adjunct}. 
    The emphasis of the authors on this kind of construction 
    seems to arises from confusion between form and function:
    \acs{advp}s ``should'' only occur as adjuncts and not complements,
    and when they actually appear to be complements,
    some make-up mechanisms are needed to maintain the generalization that \acs{advp}s are adjuncts.
\end{infobox}

Prototypical core arguments,
peripheral arguments and oblique arguments representing prototypical peripheral semantic roles 
allow quite diverse choices when it comes to how to fill them.
Manner-like phrases -- be them oblique arguments or peripheral arguments -- 
allow less variations.
\acs{tam} adjuncts are usually highly limited in their contents.
Peripheral adverbials, on the other hand,
allow much more variation (TODO: ref).

Some adverbials are even higher than the speech act-related adverbials 
mentioned in \prettyref{tbl:clausal-dependent},
but they are too high to be considered as clausal dependents:
many of them are clause linking devices (\prettyref{sec:clause-linking.subordination}).
Subordinated adverbial clauses may be about 
cause and result (\form{now }), 
concession, 
and condition (\form{if \dots then \dots}, with the ``reason'' clause being semantically irrealis).
There are also connective adjuncts like \form{moreover} or \form{alternatively},
which refer to \emph{discourse} structures, instead of syntactic structures.

There are of course subtleties between the classification of meanings in \prettyref{tbl:clausal-dependent}.
The instrument role, for example, 
may appear in the subject position or as a peripheral argument, 
while the manner expression is similar with the peripheral instrument argument 
in their forms (prepositional constructions or ``oblique cases'')
and the ability to be \form{wh}-extracted
(\prettyref{ex:simple-clause.dependents.ex-1}, \prettyref{ex:simple-clause.dependents.ex-3}).

\begin{infobox}{Adverbial classification in the literature}{adverbial-classification}
    Different authors have slightly different terminologies concerning adverbials.
    \citet[\citepage{576}]{cgel} put \acs{tam} adverbials (except modality adverbials) 
    and peripheral arguments
    under the class of \acs{vp}-oriented adverbials, 
    while modality adverbials and speech-act-like adverbials are called clause-oriented adverbials.
    \citet[\citepage{386}]{dixon2005semantic} on the other hand 
    put all \acs{tam} adverbials and speech-act-like adverbials 
    into the category of sentential adverbials (he calls them \term{adverbs})
    and the non-prototypical peripheral arguments are packaged into the class of manner adverbials,
    though some of them are not really about manner -- 
    for example it may be about degree \citet[\citepage{576}]{cgel}.
\end{infobox}

Clause combining may happen in each layer represented in \prettyref{fig:clause-template}. (TODO: ref)
Linked clauses may appear before or after the main clause.
Supplementation and subject-sharing coordination is also not shown in this figure 
(\prettyref{sec:clause-linking.coordination},
\prettyref{sec:clause-linking.supplementation}).
Nor is clausal derivation illustrated in the figure 
(\prettyref{sec:simple-clause.derivation}),
because arguably, some post-internal complement adverbials 
are likely to be the result of heavy constituent postponing (TODO: ref).
Apart from the above constructions,
the scheme illustrated in \prettyref{fig:clause-template} works for all clause types
(\prettyref{ex:simple-clause.structure-1}, 
\prettyref{ex:simple-clause.structure-2},
\prettyref{ex:simple-clause.structure-3}),
including nonfinite clauses,
though for the latter,
the properties of the subject 
and the allowed auxiliaries deviate from the finite case,
and this is also the same for allowed preposing constructions. %Enumerating: restriction on infinite clause derivation
(\prettyref{ex:simple-clause.structure-1}) is a fused relative clause,
in which there is \formcat{wh}-fronting 
but no subject-auxiliary inversion (TODO: ref).
In (\prettyref{ex:simple-clause.structure-2}) we see two preposing constructions,
one topicalization (TODO: ref)
and \formcat{wh}-movement for question formation (TODO: ref),
and the only verb -- the copula \form{is} -- is moved out of the verb phrase
because of subject-auxiliary inversion.
The 

\begin{exe}
    \ex\label{ex:simple-clause.structure-1} {} [%
        [What]_{i,\text{\formcat{wh}-preposed: \formcat{wh}-pronoun}} %
        [[Max]_{\text{subject:\acs{np}}} %
        [said Liz bought --_{i}]_{\text{verb phrase}}]_{\text{nucleus}}]_{\text{\formcat{wh}-preposing}}
    \ex\label{ex:simple-clause.structure-2} {} [%
        [In your opinion]_{\text{topicalized}}
        [[what]_{i,\text{\formcat{wh}-preposed}} %
            [is [--_{i} the most dangerous]_{\text{verb phrase}}]_{\text{SAI}}%
        ]_{\text{\formcat{wh}-preposing}} %
    ]_{\text{topic-preposing}}
    \ex\label{ex:simple-clause.structure-3} {} [%
        [what]_{i,\text{\formcat{wh}-preposed}} [to [do --_i]_{\text{verb phrase}}]_{\text{nucleus}}%
    ]_{\text{\formcat{wh}-preposing}}
\end{exe}


\subsection{Moods or clause types}\label{sec:moods}

The classification of verbal clauses in English is shown in \prettyref{tbl:verbal-clause}.
Traditionally, the distinction between finite and non-finite clauses 
is defined by the ability to be a sentence:%
\footnote{
    In this note, an \term{utterance} is a unit spoken by a speaker,
    while a \term{sentence} is a ``maximal'' clause that is an utterance.
    An utterance doesn't have to be a sentence:
    It can be an \acs{np}, as a concise reply to a question,
    or even a single word.

    Some people use the term \term{sentence} to cover all utterances.
    \citet[\citepage{45}, \citepage{853}]{cgel} uses the term \term{sentence} 
    almost as a synonym of \term{utterance},
    and all discussions concerning the syntax in their account of English grammar are about clauses.
} 
a finite clause is either able to be a sentence i.e. a clause that is an utterance on its own
(but of course is also able to be embedded),
or is close enough to a sentence,
while a nonfinite clause is almost always embedded,
and when it does appear as an utterance,
it behaves more like an \acs{np} that's used as an utterance.

English sentences may further be divided into the imperative mood (\ref{ex:clause.mood.imp-1})
and non-imperative moods (\ref{ex:clause.mood.dec-1}, 
\ref{ex:clause.mood.int-1}, 
\ref{ex:clause.mood.int-2},
\ref{ex:clause.mood.int-3}).%
\footnote{
    \citet{dixon2009basic1} firmly argues against using the term \term{mood} 
    for the syntactic marking of modality,
    while \citet{cgel} uses the term \term{mood} for the syntactic marking of modality
    and uses \term{clause type} to specifically refer to Dixon's \term{mood}.
    To avoid confusion (\term{clause type} is too vague),
    this note follows the definition of Dixon.

    The confusion between mood and modality seems to arise from traditional Latin grammar,
    in which there is no significant difference 
    between a declarative sentence and an interrogative sentence, 
    while there is significant difference
    between the verbal morphology in indicative and subjunctive clauses.
    On the other hand, in imperative clauses 
    there is no indicative-subjunctive distinction.
    Therefore the imperative-non-imperative distinction is fused with 
    the indicative-subjunctive distinction 
    and is named \term{mood}.
    This relies on the specificities of Latin grammar 
    and surely is not a universal category for all languages.
    English also has modal clauses with auxiliaries like \form{would} or \form{should},
    but that's about modality, not mood.
    There is indeed a subjunctive clause type in English,
    but it has already been restricted to complement clauses,
    and never appear as a full sentence. TODO: ref
}
The distinction between the two can be decided by compatibility with the auxiliary verbs.
The former doesn't allow any nontrivial modality and aspect,
and the tense is always present,
while non-imperative moods interact freely with all \acs{tam} categories.

The non-imperative finite mood
can be further divided into the declarative mood (\ref{ex:clause.mood.dec-1})
and several interrogative moods (\ref{ex:clause.mood.int-1}, 
\ref{ex:clause.mood.int-2},
\ref{ex:clause.mood.int-3}).
The formation of interrogative clauses 
only takes two (often skipped) syntactic steps 
(\prettyref{sec:simple-clause.interrogative.formation}),
which can be attributed to a focus construction which also happens in declarative clauses (TODO: ref).
Indeed, some, like \citet[\citepage{25}]{dixon2005semantic}, only recognize two moods.
This note still keeps the declarative-interrogative distinction 
for convenience.

As for embedded finite clauses,
the imperative mood is never embedded (except as direct reported speech),
while we do have embedded non-imperative clauses (\ref{ex:clause.mood.dec-cc-1}).
Beside declarative embedded clauses,
we have the category of subjunctive clause (\ref{ex:clause.mood.subj-1})
which is not nonfinite 
since its structure is too close to 
the embedded declarative clause \citep[\citepage{83}]{cgel}. 

Nonfinite clauses are deficient in \acs{tam} marking 
\citep[\citepage{1174}, {[5-7]}]{cgel},
and never appear as full sentences.
The class of participles contain \formcat{ed}-participles and \formcat{ing}-participles.
The class of infinitive clauses can be further divided into 
\formcat{to}-infinitives and bare infinitives \citet[\citechap{14}, \citesec{1.4.3}]{cgel}.
The class of \formcat{to}-infinitives have three superficial constituent order:
\form{to do sth.}, \form{sb. to do sth.}, and \form{for sb. to do sth.}
However, the \form{sb. to do sth.} sequence doesn't correspond to a separate type of infinitive clause:
the \form{sb.} position is always an object position licensed by the verb.
Indeed, we never find the \form{sb. to do sth.} sequence
in constructions beside various infinitive complement clause constructions.
Thus we only have two types of infinitive clauses:
the one without \form{for} and with a null subject,
and the one with \form{for} and a visible subject.

Nonfinite clauses prototypically appear as complement clauses, 
but they can also be relative clauses and adverbial clauses \citep[\citepage{1264}]{cgel}.

\begin{table}[H]
    \caption{Classification of verbal clauses based on independence and finiteness}
    \label{tbl:verbal-clause}
    \centering
    \begin{tabular}{@{}cccccc@{}}
        \toprule
        \multicolumn{2}{c}{independent}            & \multicolumn{4}{c}{ embedded}                                                                \\ \cmidrule(lr){1-2} \cmidrule(lr){3-6}
        \multicolumn{4}{c}{finite}                                                                & \multicolumn{2}{c}{nonfinite}                         \\ \cmidrule(lr){1-4} \cmidrule(lr){5-6}
        imperative               & \multicolumn{2}{c}{``normal'' or ``indicative''} & subjunctive & infinitive           & participle \\ \midrule
        (\ref{ex:clause.mood.imp-1})    &  
        (\ref{ex:clause.mood.dec-1}, \ref{ex:clause.mood.int-1}, \ref{ex:clause.mood.int-2}, \ref{ex:clause.mood.int-3})                 &   
        (\ref{ex:clause.mood.dec-cc-1})         &           
        (\ref{ex:clause.mood.subj-1})                & 
        (\ref{ex:clause.mood.inf-1})  &        
        (\ref{ex:clause.mood.part-1})                       \\ \bottomrule
    \end{tabular}
\end{table}

\begin{exe}
    \ex\label{ex:clause.mood.imp-1} [Go back immediately]_{\text{imperative}}!
    \ex\label{ex:clause.mood.dec-1} [Today is a love day]_{\text{declarative}}.
    \ex\label{ex:clause.mood.int-1} [Is it a lovely day]_{\text{yes-no question}}?
    \ex\label{ex:clause.mood.int-2} [What's the weather today]_{\text{content question}}?
    \ex\label{ex:clause.mood.int-3} [It's a lovely day, isn't it]_{\text{tag question}}?
    \ex\label{ex:clause.mood.dec-cc-1} The weather forecast says [that today should be a love day]_{\text{declarative complement clause}}.
    \ex\label{ex:clause.mood.subj-1} The doctor suggests [that he brush his teeth more carefully]_{\text{subjunctive complement clause}}.
    \ex\label{ex:clause.mood.inf-1} I want [to go hiking outside].
    \ex\label{ex:clause.mood.part-1} Do you mind [my opening the window]?
\end{exe}


Besides the verbal clauses shown above,
there is a further class of clauses -- the \concept{verbless clause} \citep[\citepage{1266}]{cgel} -- 
that may be placed into the nonfinite column,
but some think it's just a type of sub-clausal phrase.
Its distribution is also very different from other types of clauses.


Of course, the inner structure of nonfinite clauses are strongly related to their licensing environments,
which we discuss in the next chapter.

\subsection{Clausal derivations}\label{sec:simple-clause.derivation}

%Enumerating: clause derivation
(TODO: heavy NP shift, final adverbial with a pause)

\subsubsection{Preposing}\label{sec:simple-clause.derivation.preposing}

%Enumerating: preposing cases
(\prettyref{sec:simple-clause.information.topicalization})

\section{Minimal declarative clause}

\subsection{Clausal dependents}\label{sec:simple-clause.dependents}



\subsection{Clausal continuation}

TODO: \form{not even} construction, heavy NP shift, etc.


\subsection{Logic and default information structure}

\subsubsection{Descriptive parameters}

TODO: position of quantifiers for NPs with or without determiner,
and the difference between \form{all}, \form{every}, \form{some}, \form{any};
the position and scope of negation;
how the position of NPs (subject or object) influences quantification;
the relation with information structure
(if a subject NP is never given, 
it tends to be read as a representative of its kind and is therefore bound by $\forall$;
but it's never the case for objects)

\section{Topicalization}\label{sec:simple-clause.information.topicalization}

\section{Cleft constructions}

\subsection{\form{It}-cleft}

A \form{it}-cleft construction 
contains a dummy subject \form{it},
a finite form of \form{be}, 
a focused constituent,
and a \formcat{that}-clause in which there is a gap
(\prettyref{ex:simple-clause.cleft.it.1}, \prettyref{ex:simple-clause.cleft.it.2}).
Note that here \form{it} never changes into, say, \form{she} or \form{they},
and \form{is} doesn't show any agreement with the focused constituent.
This is the expected behavior: 
Note that it's pretty fine 
for the syntactic numbers of the \acs{np}s 
before and after \form{be} to be different
in non-cleft clauses, 
and \form{be} always agrees with the subject.

\begin{exe}
    \ex\label{ex:simple-clause.cleft.it.1} It is [him]_{i, \text{focused}} that he wanted to murder ---_i !
    \ex\label{ex:simple-clause.cleft.it.2} It is [by this new method]_i that we have achieved such success ---_i.
\end{exe}

The range of constituents able to be focused 
can be found in \citet[\citepages{1417-1419}]{cgel}.
We seems to have a generalization: 
If the focused constituent is an adverbial, 
it has to be able to be an answer to a \form{how} question -- 
and therefore can appear after \form{be}.

The \form{it}-cleft construction seems to be a clausal idiom:
The dummy \form{it} can be raised
when the \form{it}-cleft construction is in an infinitive form 
and embedded into a complement clause construction.
Note that the \form{it}-cleft construction can't be a verbless clause: 
This makes the focusing reading inaccessible.

TODO: \form{that} or \form{who}?? 

Thus, although the \form{it}-cleft construction 
is already an established construction with a fixed meaning (i.e. focusing),
it's still analyzable as a bi-clausal construction
following the usual syntactic constraints found elsewhere in English.

\subsection{\form{Wh}-cleft}

\section{Expletive subject constructions}

\subsection{\form{There be} existential construction}

\begin{exe}
    \ex 
\end{exe}

\subsection{\form{It seems that ...}}

\section{Focus constructions}

The English focus construction involves subject-auxiliary inversion 
and preposing of the focused constituent.
Note that the fronted constituent can be a copular complement,
a locative adverbial, TODO: list 
but never an object in a prototypical transitive construction.
This seems to be motivated for functional reasons,
because in the latter case it's impossible to correctly restore the meaning.

\begin{exe}
    \ex {} [On the top of the mountain]_{\text{locative \acs{pp}}} lies [a small church]_{\text{subject}}.
\end{exe}

\begin{infobox}{Focusing is not valency changing}{focus-not-voice}
    This note follows the analysis in \citet[\citepage{244}]{cgel}
    and regard this as a type of information packaging
    but not a voice construction.
    Some grammars, like \citet[\citepage{736}]{quirk1985},
    analyze the fronted constituent as the subject. 
    This is not the position of this note TODO: why?
\end{infobox}


\section{Interrogative moods}

\subsection{Overview}\label{sec:simple-clause.interrogative.formation}

There are two movements involved in forming a canonical interrogative clause:
the subject-auxiliary inversion,
and fronting of the \formcat{wh}-phrase, if any.
Both operations can be omitted in casual speech.

\subsection{Yes-no questions}

\subsection{Open questions}

\subsection{Tag questions}

\begin{exe}
    \ex The car is broken, isn't it?
\end{exe}

\section{The imperative mood}

The imperative mood is only compatible with active clauses.
Passive imperatives are never possible, 
and the intended meaning may be alternatively expressed by 

\section{The \formcat{to}-infinitive}

\chapter{Clause combining}\label{chap:clause-combining}

\section{Complement clause constructions}\label{sec:clause-combining.complement-clause}

Complement clauses or \term{content clauses} \citep{cgel} are clauses embedded as arguments of certain verbs.
English adverbial clauses have the same form of complement clauses,
and therefore \citet{cgel} use the term \term{content clause}.
Here I'll just stick to the more common terminology in linguistic description.

\subsection{Types of complement clauses}

According to \citet[\citesec{18.4}]{dixon2010basic2},
there are usually three types of complement clauses:
\begin{itemize}
    \item the Fact type, which looks like a full sentence
    and usually express a fact, 
    \item the Activity type, which looks like an \acs{np}
    but still keeps key features of clauses 
    and usually express an ongoing activity 
    without specifying the time (and thus with deficient \acs{tam} marking), and 
    \item the Potential type,
    which describes the potential or plan to do certain things 
    and also has deficient \acs{tam} marking,
    but is formally less similar to \acs{np}s.
\end{itemize}
In English, the three types correspond to precisely 
the finite complement clause,
the participle clause,
and the infinitive clause.
Note that the classification is pseudo-semantic:
It is similar to the ``generalized semantic role'' labels S, A, P, etc.,
which suggest their prototypical -- but not unique -- semantic function.
Indeed, a Fact clause -- a finite complement clause -- 
is able to express a potential,
like \form{that I would spend my summer in Paris},
and an Activity clause can express a fact 
(\form{[Being an engineer], he has a sharp mind when solving practical problems}).

\subsubsection{Subjunctive clauses}\label{sec:complement.subjunctive}

\subsection{Infinitive constructions}

Before going into details of each construction,
I list some parameters to classify infinitive complement clause constructions.

\subsubsection{Infinitive appearing in subject}

It's possible for an infinitive clause to appear in the subject position
(\prettyref{ex:complement.infinitive.1}).
However, this use of infinitive clauses 
has nothing particularly interesting:
The complement clause has almost parallel behaviors with an \acs{np} 
(\prettyref{ex:complement.infinitive.1p}).

\begin{exe}
    \ex\label{ex:complement.infinitive.1} 
    [To try your best] also includes to ask for help when it's necessary.
    \ex\label{ex:complement.infinitive.1p} [The skill to ask for help]_{\text{subject:\acs{np}}}
    is a strength.
\end{exe}

TODO: Is this control?
We may analyze this as (semi-)control:
The null subject of \form{to travel a lot} 
seeks reference and can only has coreference with the object.
However, this doesn't seem like a purely syntactic process
compared with raising:
Similar coreference requirements can be found in 
several constructions with little structural resemblance
(TODO: CGEL chapters).
The coreference here may be better analyzed as a semantic effect:
The only ``active'' \acs{np} in the clause is the object \form{him},
and therefore the null subject has to refer to the object.
This creates another problem: 
Whether the classical object control is also from the same mechanism.

\begin{exe}
    \ex To travel a lot annoys him.
\end{exe}
\begin{exe}
    \ex To travel a lot sometimes is annoying for him.
    \ex For her, to travel a lot is never a burden.
\end{exe}

TODO: see \citet[\citepage{1269}]{cgel}

However, the subject of an infinitive is not always coreferential with an argument within the clause:
\begin{exe}
    \ex To smoke around babies is dangerous.
\end{exe} 

If we try hard enough
(actually not that hard, because what is done below clearly has semantic motivation), 
the \form{annoy} clause can receive an analysis very similar to the object control:
Here \form{him} is the experiencer, 
and may be higher in the \vP structure, 
and thus in a certain stage of the syntactic derivation,
\form{him} indeed controls the null subject in \form{to travel a lot}.
This, however, can also be done at the syntax-semantic interface: 
We may say \form{him} \emph{semantically} controls 
the null subject of the infinitive clause
when interpreted.

Syntactic versus semantic control
S Wurmbrand - Linguistik aktuell/linguistics today, 2 is also useful

\subsubsection{Object and post-object infinitives}

Infinitive clauses in the \acs{vp},
on the other hand, 
have much richer behaviors.
This section discusses infinitive clause constructions with the constituent order 
shown in \prettyref{fig:complement.infinitive.template},
where the object position and the infinitive complement clause position 
are all internal complements shown in \prettyref{fig:clause-template},
and the object position may be absent.

The subject and the object of the matrix clause may originate from the subject of the infinitive.
If this is the case,
then there are two possibilities regarding the mechanism that connects the matrix subject/object 
and the infinitive subject:
control and raising \citet[\citepages{1194-1197}]{cgel}.
With the control mechanism,
the \acs{np} in question bears two argument roles,
one from the main verb of the matrix clause, and the other from the main verb of the infinitive clause.
With the raising mechanism, however,
the \acs{np} in question receives no argument role from the main verb of the matrix clause.
This results in several structural and semantic differences,
which will be talked about later.

\begin{figure}
    \centering
    {\small \begin{tikzpicture}[x=0.75pt,y=0.75pt,yscale=-1,xscale=1]
    %uncomment if require: \path (0,300); %set diagram left start at 0, and has height of 300
    
    %Straight Lines [id:da7702975216957955] 
    \draw [color={rgb, 255:red, 144; green, 19; blue, 254 }  ,draw opacity=1 ][fill={rgb, 255:red, 144; green, 19; blue, 254 }  ,fill opacity=1 ][line width=3]    (223,103.93) -- (223,211.59) ;
    %Straight Lines [id:da8677174949213389] 
    \draw [color={rgb, 255:red, 189; green, 16; blue, 224 }  ,draw opacity=1 ][fill={rgb, 255:red, 189; green, 16; blue, 224 }  ,fill opacity=1 ][line width=3]    (179,75.93) -- (179,210.59) ;
    
    % Text Node
    \draw (193,70) node [anchor=north west][inner sep=0.75pt]  [color={rgb, 255:red, 189; green, 16; blue, 224 }  ,opacity=1 ] [align=left] {subject};
    % Text Node
    \draw (242,105) node [anchor=north west][inner sep=0.75pt]  [color={rgb, 255:red, 144; green, 19; blue, 254 }  ,opacity=1 ] [align=left] {...};
    % Text Node
    \draw (267,120) node [anchor=north west][inner sep=0.75pt]  [color={rgb, 255:red, 144; green, 19; blue, 254 }  ,opacity=1 ] [align=left] {infinitive-taking main verb};
    % Text Node
    \draw (306,148) node [anchor=north west][inner sep=0.75pt]  [color={rgb, 255:red, 144; green, 19; blue, 254 }  ,opacity=1 ] [align=left] {(\acs{np} object(s))};
    % Text Node
    \draw (336,176) node [anchor=north west][inner sep=0.75pt]  [color={rgb, 255:red, 144; green, 19; blue, 254 }  ,opacity=1 ] [align=left] {infinitive complement clause};
    % Text Node
    \draw (220,209.59) node [anchor=north east] [inner sep=0.75pt]  [color={rgb, 255:red, 74; green, 144; blue, 226 }  ,opacity=1 ,rotate=-90] [align=left] {verb phrase};
    % Text Node
    \draw (176,208.59) node [anchor=north east] [inner sep=0.75pt]  [color={rgb, 255:red, 189; green, 16; blue, 224 }  ,opacity=1 ,rotate=-90] [align=left] {nucleus clause};
    % Text Node
    \draw (358,203) node [anchor=north west][inner sep=0.75pt]  [color={rgb, 255:red, 144; green, 19; blue, 254 }  ,opacity=1 ] [align=left] {...};
    
    
    \end{tikzpicture}
    }
    \caption{The template of all infinitive clause constructions}
    \label{fig:complement.infinitive.template}
\end{figure}

\begin{theorybox}{The control/raising distinction from a Minimalist perspective}{minimalist-control-raising}
    Essentially, the difference between control and raising is 
    whether the ``object'' bears more than one $\theta$-role,
    and historically people assume that a DP can only bear one $\theta$-role,
    so control can't be from movement.
    This however is disputed from a Minimalist lens,
    and the ``single $\theta$-role'' condition 
    can be loosen without overgeneration,
    and therefore both control and raising come from movement
    \citet{hornstein1999movement}. 

    The real problem, however, is not whether control \emph{can} be accounted for 
    by a movement theory, 
    but whether it's appropriate to do so:
    If doing so TODO: semantic theory of control fits better crosslinguistically??
\end{theorybox}

Some generalizations further narrow down 
the number of possible infinitive clause constructions.
English has a strong tendency to spell out only one copy of a moved constituent,
so if the subject of the main clause is linked to the subject of the infinitive,
then the latter is not visible,
and the object of the main clause, if any, has to be base-generated;
and if the object of the main clause is linked to the subject of the infinitive,
the latter is also not visible,
and the subject of the main clause has to be base-generated.

The subject can also be a dummy in other complement clause constructions.
When the complement clause is an infinitive, however, 
this seems impossible (\prettyref{ex:complement.infinitive.no-dummy-subject}).

\begin{exe}
    \ex\label{ex:complement.infinitive.no-dummy-subject}  \begin{xlist}
        \ex It seems that he is mad.
        \ex *It seems for he to be mad.
    \end{xlist}
\end{exe}

Subject raising (i.e. raising the subject of the infinitive to the subject of the matrix clause) 
is not compatible with the object position.
This seems to have a semantic motivation:
If a verb doesn't have an agentive role -- 
which is always true in the case of subject raising,
by the definition mentioned above -- 
then it also doesn't have a patientive role,
and therefore the object position is not licensed.
Subject control, on the other hand, allows the object position.

Thus, possible infinitive clause constructions with 
the infinitive in the \acs{vp} are summarized 
in \prettyref{tbl:infinitive-object}.

\begin{table}[H]
    \caption{Infinitive constructions with the infinitive being in \acs{vp}}
    \label{tbl:infinitive-object}
    \centering
    \begin{tabular}{@{}lll@{}}
    \toprule
    subject                                 & object                & example \\ \midrule
    subject raising                         & no object             & (\prettyref{ex:complement.infinitive.vp1})  \\ \midrule
    \multirow{2}{*}{subject control}        & no object             & (\prettyref{ex:complement.infinitive.vp2})  \\
                                            & object base generated & (\prettyref{ex:complement.infinitive.vp3}) \\ \midrule
    \multirow{4}{*}{subject base generated} & no object             & (\prettyref{ex:complement.infinitive.vp4}) \\
                                            & object base generated & (\prettyref{ex:complement.infinitive.vp7}) \\
                                            & object control        & (\prettyref{ex:complement.infinitive.vp5}) \\
                                            & object raising        & (\prettyref{ex:complement.infinitive.vp6}) \\
                                             \bottomrule
    \end{tabular}
\end{table}

\begin{exe}
    \ex\label{ex:complement.infinitive.vp1} 
    {} [The student]_{i,\text{subject}} seems [---_{i, \text{raised}} to be cheerful]_{\text{infinitive}}.
    \ex\label{ex:complement.infinitive.vp2}  
    {} [I]_{i,\text{subject}} want [---_{i,\text{controlled}} to join your group]_{\text{infinitive}}.
    \ex\label{ex:complement.infinitive.vp3}  
    {} [I]_{i, \text{subject}} promise [her]_{\text{base-generated object}} 
    [---_{i, \text{controlled}} to go away]_{\text{infinitive}}.
    \ex\label{ex:complement.infinitive.vp4}  
    {} I want [for her to complete this task tomorrow]_{\text{sealed infinitive}}.
    \ex\label{ex:complement.infinitive.vp7}  
    I promise [you]_{\text{base-generated object}} [for John to come here]_{\text{sealed infinitive}}.
    \ex\label{ex:complement.infinitive.vp5}  
    I want [him]_{i, \text{object}} 
    [---_{i, \text{controlled}} to complete this task tomorrow]_{\text{infinitive}}.
    \ex\label{ex:complement.infinitive.vp6}  
    I ask [you]_{i, \text{object}} [---_{i, \text{raised}} to do this tomorrow].
\end{exe}

The above classification has direct consequence in the form of the infinitive clause.
(\prettyref{ex:complement.infinitive.vp4},
\prettyref{ex:complement.infinitive.vp7})
have no essential difference with (\prettyref{ex:complement.infinitive.1}).
(\prettyref{ex:complement.infinitive.vp7}) is less frequent 
but is still attested \citep[\citepage{243}]{dixon2005semantic}.

The next step is to analyze the \emph{form} of these infinitive clauses
appearing in raising/control/no-raising-or-control environments.
Whenever raising or control appears, \form{for} is absent;
and if an infinitive clause isn't meant to be involved in raising or control,
then \form{for} is usually present,
because otherwise the construction 
is interpreted as a control or raising construction 
(\prettyref{ex:complement.infinitive.preference-control-reading}).
Thus, we conclude that \begin{enumerate*}
    \item The word \form{for} in an infinitive clause seals the clause 
    and turns it into an \acs{np}-like construction 
    with invisible inner structure in the eys of the syntactic environment, and 
    \item In non-control-or-raising infinitive constructions listed 
    in \prettyref{tbl:infinitive-object},
    the infinitive clauses are the \acs{np}-like infinitives just mentioned  
    and are put in object-like positions.
    (compare (\prettyref{ex:complement.infinitive.promise-np-object}) and 
    (\prettyref{ex:complement.infinitive.vp7}))
\end{enumerate*}

\begin{exe}
    \ex\label{ex:complement.infinitive.preference-control-reading} I want [to join your group]. \\
    \translate{*I want someone else to join your group.}
    \ex\label{ex:complement.infinitive.promise-np-object} The great powers promised the Jews [an independent nation].
\end{exe}

TODO: bare infinitives, let sb. to, make sb. to, see sb. do \citet[\citepage{1236},\citepage{1254}]{cgel}

\subsubsection{Semantic classification of infinitives}

Semantically, an infinitive clause either expresses a potential situation,
or a subjective judgement \citep[\citepage{245}]{dixon2005semantic}.
A judgement infinitive clause 
is always in a subject-raising construction or an object-raising construction,
probably for semantic reasons:
A verb taking a judgement infinitive clause 
takes a cognitor semantic argument,
TODO: but why can't there be a \form{for} infinitive clause? 
\form{*I find for this food to be bad}

\subsubsection{Interpretation of the null subject}

The \form{to do sth.} type of infinitive has null subject.
What the null subject refers to is sometimes decided by structural factors,
as in the cases of control and raising,
and sometimes by semantic and pragmatic feasibility.

TODO: semantic subject interpretation



\subsubsection{Subject-raising}

\begin{exe}
    \ex The boy seems unhappy.
\end{exe}

\subsubsection{Object raising}

\begin{exe}
    \ex I wanted them to start.
\end{exe}

\subsubsection{Control}

\begin{exe}
    \ex I ask them to be helpful.
\end{exe}

Although \citet[\citepage{15}]{dixon2005semantic}, \citet[\citepage{388}]{dixon2010basic2} argue that 
it's not necessary to introduce the concept of object raising in English, TODO

\subsection{Quoted speech}

It should be noted that direct quoted speech 
is not as simple as a sequence of sound:
It's possible for a \acs{vp} 
to appear as a quoted speech.

\begin{exe}
    \ex In closing, they said they ``stand ready and willing to help you win Michigan in 2024.''
\end{exe}

\section{Relative clauses}\label{sec:relative-clause}

The relative clause construction is formed by 

\subsection{Types of relative clauses}

It should be noted that the \formcat{wh}-movement in relative clauses 
is not structurally the same as the \formcat{wh}-movement in interrogative constructions.
Consider the pair in (\prettyref{ex:clause-combine.relative-question}):
It clearly demonstrates that the relative \formcat{wh}-phrase 
is structurally higher than the topic,
while the opposite is true for interrogative constructions.
This may have a semantic motivation \citet[\citepage{330}]{radford2009analysing}:
In question formation, the \formcat{wh}-movement is just a marking strategy of the \emph{focus},
which appears below the topic,
while in the formation of relative clauses,
\formcat{wh}-movement happens \emph{last},
marks the whole clause as a relative clause, 
and ``seals'' the whole relative clause,
separating its content and the matrix clause.

\begin{exe}
    \ex\label{ex:clause-combine.relative-question} \begin{xlist}
        \ex {} [In you opinion]_{\text{topic}}, 
        [what]_{\text{focus:\formcat{wh}}} [is]_{\text{fronted auxiliary}} our most urgent task right now?
        \ex {} [[What], [in his opinion]_{\text{topic}}, is our most urgent task right now]_{\text{relative clause}} still remains unknown for the listeners.
    \end{xlist}
\end{exe}

\subsection{Purpose relative clause}

A rare type of 

\begin{exe}
    \ex I need [a house [to live]_{\text{purpose}}]_{\text{object: \acs{np}}}
    \ex I need [a house [to live in]_{\text{purpose}}]_{\text{object: \acs{np}}}
\end{exe}

\section{Clause linking: subordination}\label{sec:clause-linking.subordination}



\section{Clause linking: coordination}\label{sec:clause-linking.coordination}

Parameters concerning variation of coordination 
include number of coordinates, 
number of coordinators,
and whether we have correlative items like \form{both} in the coordination construction.

TODO: \citet[\citepage{1276}]{cgel}

This section talks about FANBOY
TODO: subject extraction

\section{Supplementation}\label{sec:clause-linking.supplementation}


\chapter{Prosody, punctuation, and spelling conventions}

\begin{infobox}{The notion of \term{thought group}}{thought-group}
    It should be noted that so-called \term{speech groups}
    and \term{thought groups}
    are neither morphosyntactic or semantic concepts.
    When the subject is light (for example, when it's a personal pronoun),
    the subject and the verbal complex may be grouped into one ``thought group'',
    but that doesn't mean the subject and the verb constitute a constituency:
    it merely arises from prosodical considerations.
    In real world utterances, 
    we may also see subject-only sentences with the object omitted,
    but this may be about production of utterance:
    I already have some templates of clause structures in our mind
    (though with inner hierarchical structures, as in Tree-Adjoining Grammar),
    and after we fill the subject position of one of the templates,
    somehow -- possibly because I forget what to say -- 
    I just stop filling the template and pour the half-finished sentence 
    to the one I'm talking with.
    But still, though groups have something to do with stress allocation,
    which indeed is influenced syntactically \citep[\citepage{7}]{kahnemuyipour2009syntax}. 
\end{infobox}

\chapter{Notable variations}

\section{Early Modern English}

Dialects, etc.


\bibliographystyle{plainnat}
\bibliography{cambridge}

\end{document}