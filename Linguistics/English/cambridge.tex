\documentclass{article}

\usepackage{geometry}
\usepackage{titling}
\usepackage{titlesec}
\usepackage{paralist}
\usepackage{footnote}
\usepackage{enumerate}
\usepackage{amsmath, amssymb, amsthm}
\usepackage{gb4e}
\noautomath
\usepackage{bbm}
\usepackage{soul}
\usepackage{graphicx}
\usepackage{siunitx}
\usepackage[table,xcdraw]{xcolor}
\usepackage{tikz}
\usepackage[ruled, vlined, linesnumbered, noend]{algorithm2e}
\usepackage{xr-hyper}
\usepackage[colorlinks]{hyperref} % linkcolor=black, anchorcolor=black, citecolor=black, filecolor=black
\usepackage[most]{tcolorbox}
\usepackage{caption}
\usepackage{subcaption}
\usepackage{booktabs}
\usepackage{multirow}
\usepackage[figuresright]{rotating}
\usepackage{acro}
\usepackage[round]{natbib} 
\usepackage{prettyref}

\geometry{left=3.18cm,right=3.18cm,top=2.54cm,bottom=2.54cm}
\titlespacing{\paragraph}{0pt}{1pt}{10pt}[20pt]
\setlength{\droptitle}{-5em}

\DeclareMathOperator{\timeorder}{\mathcal{T}}
\DeclareMathOperator{\diag}{diag}
\DeclareMathOperator{\legpoly}{P}
\DeclareMathOperator{\primevalue}{P}
\DeclareMathOperator{\sgn}{sgn}
\newcommand*{\ii}{\mathrm{i}}
\newcommand*{\ee}{\mathrm{e}}
\newcommand*{\const}{\mathrm{const}}
\newcommand*{\suchthat}{\quad \text{s.t.} \quad}
\newcommand*{\argmin}{\arg\min}
\newcommand*{\argmax}{\arg\max}
\newcommand*{\normalorder}[1]{: #1 :}
\newcommand*{\pair}[1]{\langle #1 \rangle}
\newcommand*{\fd}[1]{\mathcal{D} #1}

\newcommand*{\citesec}[1]{\S~{#1}}
\newcommand*{\citechap}[1]{chap.~{#1}}
\newcommand*{\citefig}[1]{Fig.~{#1}}
\newcommand*{\citetable}[1]{Table~{#1}}
\newcommand*{\citefootnote}[1]{footnote~{#1}}

\newrefformat{sec}{\citesec{\ref{#1}}}
\newrefformat{fig}{\citefig{\ref{#1}}}
\newrefformat{tbl}{\citetable{\ref{#1}}}
\newrefformat{chap}{\citechap{\ref{#1}}}

\usetikzlibrary{arrows,shapes,positioning}
\usetikzlibrary{arrows.meta}
\usetikzlibrary{decorations.markings}
\tikzstyle arrowstyle=[scale=1]
\tikzstyle directed=[postaction={decorate,decoration={markings,
    mark=at position .5 with {\arrow[arrowstyle]{stealth}}}}]
\tikzstyle ray=[directed, thick]
\tikzstyle dot=[anchor=base,fill,circle,inner sep=1pt]


\tcbuselibrary{skins, breakable, theorems}

\newtcbtheorem[number within=chapter]{infobox}{Box}%
  {colback=blue!5,colframe=blue!65,fonttitle=\bfseries, breakable}{infobox}

\newcommand*{\concept}[1]{\textbf{#1}}
\newcommand*{\term}[1]{\emph{#1}}
\newcommand*{\corpus}[1]{\emph{#1}}

\newcommand*{\vP}{\textit{v}P}

\DeclareAcronym{blt}{short = BLT, long = Basic Linguistic Theory}
\DeclareAcronym{cgel}{short = CGEL, long = The Cambridge Grammar of the English Language}
\DeclareAcronym{dm}{short = DM, long = Distributed Morphology}
\DeclareAcronym{tag}{long = Tree-adjoining grammar, short = TAG}
\DeclareAcronym{sfp}{long = sentence final particle, short = SFP}
\DeclareAcronym{vp}{long = verb phrase, short = VP}
\DeclareAcronym{cls}{long = classifier, short = CLS}
\DeclareAcronym{dist}{long = distal, short = DIST}
\DeclareAcronym{prox}{long = proximate, short = PROX}
\DeclareAcronym{dem}{long = demonstrative, short = DEM}
\DeclareAcronym{dur}{long = durative, short = DUR}
\DeclareAcronym{neg}{long = negative, short = NEG}
\DeclareAcronym{tame}{long = {Tense, Aspect, Mood, Evidentiality}, short = TAME}

% Disable unsupported commands in bookmark titles 
\pdfstringdefDisableCommands{%
  \def\\{}%
  \def\texttt#1{<#1>}%
  \def\mathbb#1{#1}%
}
\pdfstringdefDisableCommands{\def\eqref#1{(\ref{#1})}}

\makeatletter
\pdfstringdefDisableCommands{\let\HyPsd@CatcodeWarning\@gobble}
\makeatother

\title{Reading notes of Cambridge Grammar of English Language}
\author{Jinyuan Wu}

\begin{document}

\maketitle

\automath

This note is a reconstruction of the contents \acl{cgel} (\citealt{cgel}, henceforth \acs{cgel}). 

\section{Theoretical preliminaries}

For the discussion on the underlying theoretical framework, 
see the relevant chapter in \href{../Chinese/main.pdf}{this note}.
Briefly speaking, the framework is coarse-grained Minimalism:
invisible functional heads are erased,
but dependency relations created by the functional projections are kept (\prettyref{fig:coarse-grained}(a)),
and the corresponding \ac{cgel} tree is obtained by replacing the name of dependency relations 
by syntactic function-form pairs (\prettyref{fig:coarse-grained}(b)).
The head is defined as the dominant \emph{lexical} word.
This makes \ac{cgel} look like old-fashioned X-bar theory, 
where a lexical word projects into a phrase (\ac{cgel} \citesec{5.2} [5]),
but there can be several ``maximal projections'' headed by the same lexical word (\ac{cgel} \citesec{5.2 [11]}).
This is actually consistent with Minimalism (\prettyref{fig:cgel-minimalism}).

The standard of ``lexical'', however, varies from one author to another.
A PP and a CaseP (in generative terms) are similar objects,
but in the coarse-grained \ac{cgel} framework,
the latter is definitely an NP, 
while the status of the first is kind of controversial:
should we recognize the preposition as a lexical word,
or as a functional word?
In the first analysis, a PP in generative terms is still a PP,
while in the second analysis, a PP in generative terms is an oblique NP.
Choosing between the two analyses is merely a problem of notation.
But it is good practice to choose a notation that hints the readers about 
certain properties of the construction in question.
For example, if a PP is analyzed as a PP, then probably the preposition has some predicative properties:
a PP is therefore more like a VP, not an NP with case marking 
(or ``CaseP'' in generative terms, especially in theories focusing on morphosyntax 
and do not make sharp contrast between syntax and morphology).
If, however, prepositions in a language work just like case markers,
the argumentation raised in \citet[\citesec{1.11}, \citesec{5.4}]{dixon2009basic1} is then attempting:
we should recognize the prepositions as case markers with standalone phonological realizations.
Which analysis to choose should be decided according to observed phenomena.
This is summarized by the slogan ``describing a language in its own terms''
-- but the slogan does not go into conflict with generativism.

\begin{figure}
    \centering
    

\tikzset{every picture/.style={line width=0.75pt}} %set default line width to 0.75pt        

\begin{tikzpicture}[x=0.75pt,y=0.75pt,yscale=-0.85,xscale=0.85]
%uncomment if require: \path (0,408); %set diagram left start at 0, and has height of 408

%Straight Lines [id:da2665478368972569] 
\draw [color={rgb, 255:red, 245; green, 166; blue, 35 }  ,draw opacity=1 ] [dash pattern={on 4.5pt off 4.5pt}]  (264,61) -- (544,61) ;
\draw [shift={(546,61)}, rotate = 180] [fill={rgb, 255:red, 245; green, 166; blue, 35 }  ,fill opacity=1 ][line width=0.08]  [draw opacity=0] (12,-3) -- (0,0) -- (12,3) -- cycle    ;
%Straight Lines [id:da6278442099042232] 
\draw [color={rgb, 255:red, 245; green, 166; blue, 35 }  ,draw opacity=1 ] [dash pattern={on 4.5pt off 4.5pt}]  (295,107) -- (622,107) ;
\draw [shift={(624,107)}, rotate = 180] [fill={rgb, 255:red, 245; green, 166; blue, 35 }  ,fill opacity=1 ][line width=0.08]  [draw opacity=0] (12,-3) -- (0,0) -- (12,3) -- cycle    ;
%Rounded Rect [id:dp9005470977487176] 
\draw  [draw opacity=0][fill={rgb, 255:red, 80; green, 227; blue, 194 }  ,fill opacity=0.2 ] (494,41.22) .. controls (494,32.48) and (501.08,25.4) .. (509.82,25.4) -- (741.18,25.4) .. controls (749.92,25.4) and (757,32.48) .. (757,41.22) -- (757,325.99) .. controls (757,334.73) and (749.92,341.81) .. (741.18,341.81) -- (509.82,341.81) .. controls (501.08,341.81) and (494,334.73) .. (494,325.99) -- cycle ;
%Straight Lines [id:da3555404365615693] 
\draw [color={rgb, 255:red, 155; green, 155; blue, 155 }  ,draw opacity=1 ]   (630.05,206.81) -- (630.05,301) ;
%Straight Lines [id:da5116688308518689] 
\draw [color={rgb, 255:red, 80; green, 227; blue, 194 }  ,draw opacity=1 ]   (703.05,158.83) -- (665.05,140.5) ;
%Straight Lines [id:da7532901881200198] 
\draw [color={rgb, 255:red, 80; green, 227; blue, 194 }  ,draw opacity=1 ]   (665.05,140.5) -- (630.05,158.83) ;
%Straight Lines [id:da3708728686228633] 
\draw [color={rgb, 255:red, 80; green, 227; blue, 194 }  ,draw opacity=1 ]   (658.65,97.63) -- (601.2,71.73) ;
%Straight Lines [id:da18650265857430104] 
\draw [color={rgb, 255:red, 80; green, 227; blue, 194 }  ,draw opacity=1 ]   (601.2,71.73) -- (548.65,97.63) ;
%Rounded Rect [id:dp9951745940140804] 
\draw  [draw opacity=0][fill={rgb, 255:red, 255; green, 255; blue, 255 }  ,fill opacity=1 ] (675,215.16) .. controls (675,210.29) and (678.95,206.33) .. (683.83,206.33) -- (722.17,206.33) .. controls (727.05,206.33) and (731,210.29) .. (731,215.16) -- (731,314.17) .. controls (731,319.05) and (727.05,323) .. (722.17,323) -- (683.83,323) .. controls (678.95,323) and (675,319.05) .. (675,314.17) -- cycle ;
%Rounded Rect [id:dp4970986462373306] 
\draw  [draw opacity=0][fill={rgb, 255:red, 255; green, 255; blue, 255 }  ,fill opacity=1 ] (521.24,153.76) .. controls (521.24,148.82) and (525.25,144.81) .. (530.19,144.81) -- (568.05,144.81) .. controls (572.99,144.81) and (577,148.82) .. (577,153.76) -- (577,314.05) .. controls (577,318.99) and (572.99,323) .. (568.05,323) -- (530.19,323) .. controls (525.25,323) and (521.24,318.99) .. (521.24,314.05) -- cycle ;
%Rounded Rect [id:dp4071663798075751] 
\draw  [draw opacity=0][fill={rgb, 255:red, 80; green, 227; blue, 194 }  ,fill opacity=0.2 ] (106,41.89) .. controls (106,33.39) and (112.89,26.5) .. (121.4,26.5) -- (346.6,26.5) .. controls (355.11,26.5) and (362,33.39) .. (362,41.89) -- (362,324.41) .. controls (362,332.92) and (355.11,339.81) .. (346.6,339.81) -- (121.4,339.81) .. controls (112.89,339.81) and (106,332.92) .. (106,324.41) -- cycle ;
%Straight Lines [id:da5088138181155522] 
\draw [color={rgb, 255:red, 155; green, 155; blue, 155 }  ,draw opacity=1 ]   (217.05,133.93) -- (217.05,299.1) ;
%Straight Lines [id:da998446713157706] 
\draw [color={rgb, 255:red, 80; green, 227; blue, 194 }  ,draw opacity=1 ]   (290.05,133.93) -- (252.05,115.6) ;
%Straight Lines [id:da5752866187220778] 
\draw [color={rgb, 255:red, 80; green, 227; blue, 194 }  ,draw opacity=1 ]   (252.05,115.6) -- (217.05,133.93) ;
%Straight Lines [id:da6021628337779128] 
\draw [color={rgb, 255:red, 80; green, 227; blue, 194 }  ,draw opacity=1 ]   (249.65,94.73) -- (200.2,68.83) ;
%Straight Lines [id:da8361003247119008] 
\draw [color={rgb, 255:red, 80; green, 227; blue, 194 }  ,draw opacity=1 ]   (200.2,68.83) -- (153.65,94.73) ;
%Rounded Rect [id:dp3858870486231194] 
\draw  [draw opacity=0][fill={rgb, 255:red, 255; green, 255; blue, 255 }  ,fill opacity=1 ] (257.24,151.34) .. controls (257.24,145.52) and (261.96,140.81) .. (267.77,140.81) -- (313.48,140.81) .. controls (319.29,140.81) and (324,145.52) .. (324,151.34) -- (324,309.05) .. controls (324,314.86) and (319.29,319.57) .. (313.48,319.57) -- (267.77,319.57) .. controls (261.96,319.57) and (257.24,314.86) .. (257.24,309.05) -- cycle ;
%Rounded Rect [id:dp2951833006567197] 
\draw  [draw opacity=0][fill={rgb, 255:red, 255; green, 255; blue, 255 }  ,fill opacity=1 ] (134.24,118.4) .. controls (134.24,113.11) and (138.54,108.81) .. (143.84,108.81) -- (184.41,108.81) .. controls (189.71,108.81) and (194,113.11) .. (194,118.4) -- (194,305.98) .. controls (194,311.28) and (189.71,315.57) .. (184.41,315.57) -- (143.84,315.57) .. controls (138.54,315.57) and (134.24,311.28) .. (134.24,305.98) -- cycle ;

% Text Node
\draw (630.76,321) node [anchor=south] [inner sep=0.75pt]   [align=left] {hurt};
% Text Node
\draw (665.05,137.5) node [anchor=south] [inner sep=0.75pt]   [align=left] {\begin{minipage}[lt]{89.53pt}\setlength\topsep{0pt}
\begin{center}
predicate:\\smaller verb phrase
\end{center}

\end{minipage}};
% Text Node
\draw (601.2,68.73) node [anchor=south] [inner sep=0.75pt]   [align=left] {verb phrase};
% Text Node
\draw (548.65,100.63) node [anchor=north] [inner sep=0.75pt]   [align=left] {\begin{minipage}[lt]{29.61pt}\setlength\topsep{0pt}
\begin{center}
agent:\\DP
\end{center}

\end{minipage}};
% Text Node
\draw (630.05,161.83) node [anchor=north] [inner sep=0.75pt]   [align=left] {\begin{minipage}[lt]{50.89pt}\setlength\topsep{0pt}
\begin{center}
predicator:\\V
\end{center}

\end{minipage}};
% Text Node
\draw (703.05,161.83) node [anchor=north] [inner sep=0.75pt]   [align=left] {\begin{minipage}[lt]{36.95pt}\setlength\topsep{0pt}
\begin{center}
patient:\\DP
\end{center}

\end{minipage}};
% Text Node
\draw (216.76,321.1) node [anchor=south] [inner sep=0.75pt]   [align=left] {hurt};
% Text Node
\draw (249.65,97.73) node [anchor=north] [inner sep=0.75pt]   [align=left] {Trans - DP};
% Text Node
\draw (179.8,50.81) node [anchor=north west][inner sep=0.75pt]   [align=left] {$\displaystyle v$ - DP};
% Text Node
\draw (211,383) node [anchor=north west][inner sep=0.75pt]   [align=left] {(a)};
% Text Node
\draw (636,383) node [anchor=north west][inner sep=0.75pt]   [align=left] {(b)};


\end{tikzpicture}

    \caption{From coarse-grained Minimalist derivational tree to \ac{cgel} tree}
    \label{fig:coarse-grained}
\end{figure}

\begin{figure}
    \centering
    

\tikzset{every picture/.style={line width=0.3pt}} %set default line width to 0.75pt        

\begin{tikzpicture}[x=0.75pt,y=0.75pt,yscale=-0.8,xscale=0.8]
%uncomment if require: \path (0,523); %set diagram left start at 0, and has height of 523

%Straight Lines [id:da6705457534949089] 
\draw    (313,92.15) -- (343,75.15) ;
%Straight Lines [id:da28013820417559354] 
\draw    (376,92.15) -- (343,75.15) ;
%Straight Lines [id:da26185532929398736] 
\draw    (345,123.15) -- (375,106.15) ;
%Straight Lines [id:da1670225517468129] 
\draw    (408,123.15) -- (375,106.15) ;
%Straight Lines [id:da7951370858439135] 
\draw    (279,62.15) -- (309,45.15) ;
%Straight Lines [id:da5307564248915579] 
\draw    (342,62.15) -- (309,45.15) ;
%Straight Lines [id:da6264444449679678] 
\draw [color={rgb, 255:red, 155; green, 155; blue, 155 }  ,draw opacity=1 ]   (321,164.15) -- (152.85,234.38) ;
\draw [shift={(151,235.15)}, rotate = 337.33] [fill={rgb, 255:red, 155; green, 155; blue, 155 }  ,fill opacity=1 ][line width=0.08]  [draw opacity=0] (12,-3) -- (0,0) -- (12,3) -- cycle    ;
%Straight Lines [id:da99450357998767] 
\draw [color={rgb, 255:red, 80; green, 227; blue, 194 }  ,draw opacity=1 ]   (204,363.15) -- (234,346.15) ;
%Straight Lines [id:da7516847541397156] 
\draw [color={rgb, 255:red, 80; green, 227; blue, 194 }  ,draw opacity=1 ]   (267,363.15) -- (234,346.15) ;
%Straight Lines [id:da7086721116542554] 
\draw [color={rgb, 255:red, 80; green, 227; blue, 194 }  ,draw opacity=1 ]   (236,403.15) -- (266,386.15) ;
%Straight Lines [id:da23963037522787767] 
\draw    (299,403.15) -- (266,386.15) ;
%Straight Lines [id:da8629804052760564] 
\draw [color={rgb, 255:red, 74; green, 144; blue, 226 }  ,draw opacity=1 ]   (139,296.15) -- (146.67,291.8) -- (169,279.15) ;
%Straight Lines [id:da21212301774701436] 
\draw [color={rgb, 255:red, 74; green, 144; blue, 226 }  ,draw opacity=1 ]   (202,296.15) -- (169,279.15) ;
%Straight Lines [id:da45399705070135465] 
\draw [color={rgb, 255:red, 155; green, 155; blue, 155 }  ,draw opacity=1 ]   (321,164.15) -- (511.12,234.45) ;
\draw [shift={(513,235.15)}, rotate = 200.29] [fill={rgb, 255:red, 155; green, 155; blue, 155 }  ,fill opacity=1 ][line width=0.08]  [draw opacity=0] (12,-3) -- (0,0) -- (12,3) -- cycle    ;
%Straight Lines [id:da06579349148912339] 
\draw [color={rgb, 255:red, 80; green, 227; blue, 194 }  ,draw opacity=1 ]   (552,327.15) -- (582,310.15) ;
%Straight Lines [id:da7813158499201878] 
\draw [color={rgb, 255:red, 80; green, 227; blue, 194 }  ,draw opacity=1 ]   (615,327.15) -- (582,310.15) ;
%Straight Lines [id:da6554597016072419] 
\draw [color={rgb, 255:red, 80; green, 227; blue, 194 }  ,draw opacity=1 ]   (585,398.15) -- (615,381.15) ;
%Straight Lines [id:da38771820652223643] 
\draw    (648,398.15) -- (615,381.15) ;
%Straight Lines [id:da7029269298051468] 
\draw [color={rgb, 255:red, 74; green, 144; blue, 226 }  ,draw opacity=1 ]   (521,265.15) -- (528.67,260.8) -- (551,248.15) ;
%Straight Lines [id:da3078830979711833] 
\draw [color={rgb, 255:red, 74; green, 144; blue, 226 }  ,draw opacity=1 ]   (584,265.15) -- (551,248.15) ;
%Straight Lines [id:da5015548112989887] 
\draw [color={rgb, 255:red, 80; green, 227; blue, 194 }  ,draw opacity=1 ]   (171,336.15) -- (201,319.15) ;
%Straight Lines [id:da5954251542916287] 
\draw [color={rgb, 255:red, 80; green, 227; blue, 194 }  ,draw opacity=1 ]   (234,336.15) -- (201,319.15) ;
%Straight Lines [id:da4688878650352748] 
\draw [color={rgb, 255:red, 74; green, 144; blue, 226 }  ,draw opacity=1 ]   (108,269.15) -- (115.67,264.8) -- (138,252.15) ;
%Straight Lines [id:da2607503270176317] 
\draw [color={rgb, 255:red, 74; green, 144; blue, 226 }  ,draw opacity=1 ]   (171,269.15) -- (138,252.15) ;
%Straight Lines [id:da8702130416826466] 
\draw [color={rgb, 255:red, 155; green, 155; blue, 155 }  ,draw opacity=1 ]   (238,450.15) -- (505,450.15) ;
\draw [shift={(507,450.15)}, rotate = 180] [fill={rgb, 255:red, 155; green, 155; blue, 155 }  ,fill opacity=1 ][line width=0.08]  [draw opacity=0] (12,-3) -- (0,0) -- (12,3) -- cycle    ;
\draw [shift={(236,450.15)}, rotate = 0] [fill={rgb, 255:red, 155; green, 155; blue, 155 }  ,fill opacity=1 ][line width=0.08]  [draw opacity=0] (12,-3) -- (0,0) -- (12,3) -- cycle    ;

% Text Node
\draw (313.03,96.09) node [anchor=north west][inner sep=0.75pt]  [rotate=-30]  {$\cdots $};
% Text Node
\draw (408,126.15) node [anchor=north] [inner sep=0.75pt]   [align=left] {lexical head};
% Text Node
\draw (279.03,66.09) node [anchor=north west][inner sep=0.75pt]  [rotate=-30]  {$\cdots $};
% Text Node
\draw (42,67) node [anchor=north west][inner sep=0.75pt]   [align=left] {Fact: functional \\domains exist};
% Text Node
\draw (251.89,95.5) node [anchor=north west][inner sep=0.75pt]  [rotate=-300] [align=left] {{\LARGE \{}};
% Text Node
\draw (294.89,121.5) node [anchor=north west][inner sep=0.75pt]  [rotate=-300] [align=left] {{\LARGE \{}};
% Text Node
\draw (180,103) node [anchor=north west][inner sep=0.75pt]   [align=left] {outer domain};
% Text Node
\draw (257,136) node [anchor=north west][inner sep=0.75pt]   [align=left] {inner domain};
% Text Node
\draw (299,406.15) node [anchor=north] [inner sep=0.75pt]   [align=left] {lexical root};
% Text Node
\draw (20,305) node [anchor=north west][inner sep=0.75pt]  [color={rgb, 255:red, 74; green, 144; blue, 226 }  ,opacity=1 ] [align=left] {outer domain};
% Text Node
\draw (109,382) node [anchor=north west][inner sep=0.75pt]  [color={rgb, 255:red, 80; green, 227; blue, 194 }  ,opacity=1 ] [align=left] {inner domain};
% Text Node
\draw (139,299.15) node [anchor=north] [inner sep=0.75pt]  [color={rgb, 255:red, 74; green, 144; blue, 226 }  ,opacity=1 ] [align=left] {F1};
% Text Node
\draw (204,366.15) node [anchor=north] [inner sep=0.75pt]  [color={rgb, 255:red, 80; green, 227; blue, 194 }  ,opacity=1 ] [align=left] {F2};
% Text Node
\draw (125,442.15) node [anchor=north west][inner sep=0.75pt]   [align=left] {Minimalism};
% Text Node
\draw (648,401.15) node [anchor=north] [inner sep=0.75pt]   [align=left] {lexical head};
% Text Node
\draw (374,300) node [anchor=north west][inner sep=0.75pt]  [color={rgb, 255:red, 74; green, 144; blue, 226 }  ,opacity=1 ] [align=left] {outer domain};
% Text Node
\draw (468,392) node [anchor=north west][inner sep=0.75pt]  [color={rgb, 255:red, 80; green, 227; blue, 194 }  ,opacity=1 ] [align=left] {inner domain};
% Text Node
\draw (521,268.15) node [anchor=north] [inner sep=0.75pt]  [color={rgb, 255:red, 74; green, 144; blue, 226 }  ,opacity=1 ] [align=left] {\begin{minipage}[lt]{57.07pt}\setlength\topsep{0pt}
\begin{center}
function 1:\\dependent 1\\
\end{center}

\end{minipage}};
% Text Node
\draw (552,330.15) node [anchor=north] [inner sep=0.75pt]  [color={rgb, 255:red, 80; green, 227; blue, 194 }  ,opacity=1 ] [align=left] {\begin{minipage}[lt]{57.07pt}\setlength\topsep{0pt}
\begin{center}
function 2:\\dependent 2
\end{center}

\end{minipage}};
% Text Node
\draw (171,339.15) node [anchor=north] [inner sep=0.75pt]  [color={rgb, 255:red, 80; green, 227; blue, 194 }  ,opacity=1 ] [align=left] {dependent 2};
% Text Node
\draw (115.67,267.8) node [anchor=north] [inner sep=0.75pt]  [color={rgb, 255:red, 74; green, 144; blue, 226 }  ,opacity=1 ] [align=left] {dependent 1};
% Text Node
\draw (202,299.15) node [anchor=north] [inner sep=0.75pt]  [color={rgb, 255:red, 74; green, 144; blue, 226 }  ,opacity=1 ] [align=left] {$\displaystyle \cdots $};
% Text Node
\draw (267,366.15) node [anchor=north] [inner sep=0.75pt]  [color={rgb, 255:red, 80; green, 227; blue, 194 }  ,opacity=1 ] [align=left] {$\displaystyle \cdots $};
% Text Node
\draw (584,281.15) node [anchor=north] [inner sep=0.75pt]  [color={rgb, 255:red, 74; green, 144; blue, 226 }  ,opacity=1 ] [align=left] {$\displaystyle \cdots $};
% Text Node
\draw (616,345.15) node [anchor=north] [inner sep=0.75pt]  [color={rgb, 255:red, 80; green, 227; blue, 194 }  ,opacity=1 ] [align=left] {$\displaystyle \cdots $};
% Text Node
\draw (535,442) node [anchor=north west][inner sep=0.75pt]   [align=left] {CGEL};
% Text Node
\draw (362,452.15) node [anchor=north west][inner sep=0.75pt]  [color={rgb, 255:red, 155; green, 155; blue, 155 }  ,opacity=1 ] [align=left] {dual};


\end{tikzpicture}

    \caption{Duality between \ac{cgel} and Minimalism}
    \label{fig:cgel-minimalism}
\end{figure}

Not all dependency relations can be expressed purely by the context-free phrase structure grammar 
\citep{pullum2008expressive}. 
Movements (or cross-serial dependencies in dependency terms) in Minimalism 
are represented by a gap (\ac{cgel} \citesec{2.2} [5]),
or by indirect dependency (\ac{cgel} \citesec{5.14.1}).

The close relation between \ac{cgel} and generative syntax apparently deviates 
from the common descriptive practice, 
especially the common descriptive framework extracted and summarized 
by \citet{dixon2009basic1,dixon2010basic2,dixon2012basic3} and named \ac{blt} by Dixon,
but this distinction is illusory:
while \ac{blt} assumes a flatter phrase structure,
information coded by the binary-branching phrase structure in \ac{cgel}
is coded by dependency arcs in \ac{blt} (\prettyref{fig:to-the-fat-man-blt}),
and the hierarchy ordering of the dependency relations 
are coded by notions like ``pipeline ordering'' and ``scope'',
e.g. ``a prototypical passive construction takes in an AVO argument structure 
and turns the deep A into the surface S, the deep O into the surface E''
(which in generative terms means VoiceP is higher than \vP),
and ``the scope of \corpus{the} in \prettyref{fig:to-the-fat-man-blt} is \corpus{fat man},
indicating that the latter is definite,
and the scope of \corpus{to} is \corpus{the fat man}''.

Though several disagreements with the mainstream generative syntax is raised in \ac{cgel},
and much more severe accusations are made in \ac{blt},
it can be seen all the three approaches are describing almost the same complexity class of grammars.%
\footnote{
    Well, strictly speaking, they may not,
    because when formalized, a grammatical theory often allows some strikingly weird productions,
    which arise from corner cases ignored in the formalization.
    But these corner cases can be, in principle, rules out by watching 
    how linguists use the formalism in everyday work:
    some derivational devices are never used, 
    weird interplay of features that trigger unexpected movements is never considered,
    some phrase structure rules in \ac{blt} are possible but never appear in actual reference grammars, etc.
    In a word, the complexity class of grammars supposed in a grammatical theory 
    is not just determined by the explicitly articulated formalism.
    What is fed into the formalism is equally important.
}

It should be noted that even though \ac{cgel} comes close with the lexical decomposition-style Minimalism,
sometimes binary branching is still not available in the surface-oriented analysis,
and some words about the underlying dependency relations beside the tree is needed,
just as in the case of \ac{blt}.
An example is ditransitive verbs. 
Even though in generative analysis we have plenty of reasons to assume a \vP{} structure like 
[somebody [something SHOW]] in the clause \corpus{I showed him the photo}
(possible reasons including binding effects, passivization preference, etc.),
the head movement (or its post-syntactic version) breaks the binary branching structure 
in the surface-oriented analysis of the clause,
and hence in \ac{cgel} we get a (quite infrequent) ternary tree (\ac{cgel} \citesec{12.3.1} [4]). 
Another example is supplementation.
This type of constructions shows significantly weaker dependency 
between the ``anchor'' and the inserted supplement.
In generative syntax, supplementation can be analyzed as a result of PF-level clause reformulation,
with major disruption of the constituent structure of the host clause, 
which tempts lots of people to analyze the phenomenon 
as a result of an idiosyncratic structure building mechanism beside Merge \citep{ott2014ellipsis}.
Therefore, in \ac{cgel}, supplementation is recognized as a construction type,
but the supplement and the anchor are not recognized as a constituent (\ac{cgel} \citesec{15.5.1} [12]).
It is practically not possible to insist on both binary branching and surface-oriented analysis.
The difference between \ac{cgel} and \ac{blt} is therefore more about quantity instead of quality.

\begin{figure}
    \centering
    

\tikzset{every picture/.style={line width=0.3pt}} %set default line width to 0.75pt        

\begin{tikzpicture}[x=0.75pt,y=0.75pt,yscale=-0.8,xscale=0.8]
%uncomment if require: \path (0,462); %set diagram left start at 0, and has height of 462

%Straight Lines [id:da7470525027519479] 
\draw    (65,82.81) -- (123,65.81) ;
%Straight Lines [id:da284026246045451] 
\draw    (188,82.81) -- (123,65.81) ;
%Straight Lines [id:da6793973159868627] 
\draw    (130,148.81) -- (188,131.81) ;
%Straight Lines [id:da7618139010556202] 
\draw    (253,148.81) -- (188,131.81) ;
%Straight Lines [id:da8777852448147017] 
\draw    (195,212.81) -- (253,195.81) ;
%Straight Lines [id:da7372022724021017] 
\draw    (318,212.81) -- (253,195.81) ;
%Straight Lines [id:da9619077986855507] 
\draw    (65,318.81) -- (65,134.81) ;
%Straight Lines [id:da5838222630790659] 
\draw    (130,318.81) -- (130,194.81) ;
%Straight Lines [id:da5932119473868447] 
\draw    (195,318.81) -- (195,254.81) ;
%Straight Lines [id:da9627969740205233] 
\draw    (318,318.81) -- (318,254.81) ;
%Straight Lines [id:da17223380454923265] 
\draw    (450,227.81) -- (450,163.81) ;
%Straight Lines [id:da6952728185192718] 
\draw    (547,227.81) -- (547,163.81) ;
%Straight Lines [id:da9330113293183067] 
\draw    (638,227.81) -- (638,163.81) ;
%Straight Lines [id:da8988276865905278] 
\draw    (716,227.81) -- (716,163.81) ;
%Straight Lines [id:da891102641862588] 
\draw    (452,141.48) -- (585,64.81) ;
%Straight Lines [id:da13577169125065036] 
\draw    (550,141.48) -- (585,64.81) ;
%Straight Lines [id:da027055110670632487] 
\draw    (638,141.81) -- (585,64.81) ;
%Straight Lines [id:da0804405283159666] 
\draw    (716,141.81) -- (585,64.81) ;
%Curve Lines [id:da8849752474371109] 
\draw    (716,254.81) .. controls (684,313.81) and (654,289.81) .. (637,254.81) ;
%Curve Lines [id:da33081916759136454] 
\draw    (720,254.81) .. controls (663,407.81) and (575,321.48) .. (549,253.48) ;
%Curve Lines [id:da5450701123076458] 
\draw    (726,254.81) .. controls (669,407.81) and (531,442.15) .. (450,255.48) ;

% Text Node
\draw (123,62.81) node [anchor=south] [inner sep=0.75pt]   [align=left] {PP};
% Text Node
\draw (65,85.81) node [anchor=north] [inner sep=0.75pt]   [align=left] {\begin{minipage}[lt]{52.34pt}\setlength\topsep{0pt}
\begin{center}
head:\\preposition
\end{center}

\end{minipage}};
% Text Node
\draw (65,321.81) node [anchor=north] [inner sep=0.75pt]   [align=left] {to};
% Text Node
\draw (188,85.81) node [anchor=north] [inner sep=0.75pt]   [align=left] {\begin{minipage}[lt]{59.05pt}\setlength\topsep{0pt}
\begin{center}
complement:\\NP
\end{center}

\end{minipage}};
% Text Node
\draw (130,151.81) node [anchor=north] [inner sep=0.75pt]   [align=left] {\begin{minipage}[lt]{53.71pt}\setlength\topsep{0pt}
\begin{center}
determiner:\\article
\end{center}

\end{minipage}};
% Text Node
\draw (253,151.81) node [anchor=north] [inner sep=0.75pt]   [align=left] {\begin{minipage}[lt]{38.39pt}\setlength\topsep{0pt}
\begin{center}
head:\\nominal
\end{center}

\end{minipage}};
% Text Node
\draw (200,214.81) node [anchor=north] [inner sep=0.75pt]   [align=left] {\begin{minipage}[lt]{42.1pt}\setlength\topsep{0pt}
\begin{center}
modifier:\\adjective
\end{center}

\end{minipage}};
% Text Node
\draw (318,215.81) node [anchor=north] [inner sep=0.75pt]   [align=left] {\begin{minipage}[lt]{26.5pt}\setlength\topsep{0pt}
\begin{center}
head:\\noun
\end{center}

\end{minipage}};
% Text Node
\draw (130,321.81) node [anchor=north] [inner sep=0.75pt]   [align=left] {the};
% Text Node
\draw (195,321.81) node [anchor=north] [inner sep=0.75pt]   [align=left] {fat};
% Text Node
\draw (318,321.81) node [anchor=north] [inner sep=0.75pt]   [align=left] {man};
% Text Node
\draw (173,427) node [anchor=north west][inner sep=0.75pt]   [align=left] {(a)};
% Text Node
\draw (450,230.81) node [anchor=north] [inner sep=0.75pt]   [align=left] {to};
% Text Node
\draw (547,230.81) node [anchor=north] [inner sep=0.75pt]   [align=left] {the};
% Text Node
\draw (637,230.81) node [anchor=north] [inner sep=0.75pt]   [align=left] {fat};
% Text Node
\draw (716,230.81) node [anchor=north] [inner sep=0.75pt]   [align=left] {man};
% Text Node
\draw (450,152.31) node   [align=left] {preposition};
% Text Node
\draw (547,152.31) node   [align=left] {determiner};
% Text Node
\draw (638,152.31) node   [align=left] {adjective};
% Text Node
\draw (716,152.31) node   [align=left] {noun};
% Text Node
\draw (585,62.81) node [anchor=south] [inner sep=0.75pt]   [align=left] {PP (or oblique NP)};
% Text Node
\draw (600,294) node [anchor=north west][inner sep=0.75pt]   [align=left] {modified by};
% Text Node
\draw (605,341) node [anchor=north west][inner sep=0.75pt]   [align=left] {definite};
% Text Node
\draw (556,384) node [anchor=north west][inner sep=0.75pt]   [align=left] {dative};
% Text Node
\draw (585,427) node [anchor=north west][inner sep=0.75pt]   [align=left] {(b)};


\end{tikzpicture}

    \caption{Comparison between the \ac{cgel} and \ac{blt} analyses of \corpus{to the fat man}}
    \label{fig:to-the-fat-man-blt}
\end{figure}

\section{Overview of clause structure}

\subsection{The pipeline of clause building}\label{sec:pipeline}

This section and several following sections are about the phrase structure of the clause.
I will discuss clause dependents, 
their forms and functions,
grammatical systems in the clause,
and how everything is put together.

Clausal grammars of nominative-accusative languages can fit more or less into the following paradigm:
\begin{enumerate}
    \item Clausal complements are fed into the argument structure%
    \footnote{
        This is not the term used in \ac{cgel}.
        For discussion on \term{argument} and \term{complmenet}, see \prettyref{sec:core-oblique}.
    } or in other words \vP. 
    They may be NPs, adverbials or subordinated clauses.
    \item Then several clausal grammatical systems are employed on the verb-complement complex or ``small clause'',
    including agreement (or argument indexation more generally), case marking%
    \footnote{
        Technically, case marking involves the argument structure.
        The nominative-accusative alignment, for example, 
        can be summarized as ``the patient-like argument of a transitive verb receives the accusative case,
        while anything promoted to the subject position receives the nominative case''.
        In more generative terms, we have 
        ``the transitive \vP{} (or some functional projection lower than Spec\vP) assigns the accusative case,
        while T assigns the nominative case''. 
        The first half of the generalization
        means the assignment of the accusative case is directly related to the argument structure.
        But we can change the wording of the above generalization easily.
        For example, an equivalent formulation may be 
        ``in the TP, any DP with a lower position than some DP else receives the accusative case, 
        while the subject receives the nominative case''.
        Yet we have a third expression: 
        ``in the TP, DPs receive the accusative case as the default case,
        while the subject receives the nominative case''.
        It is therefore acceptable to take case marking completely away from the argument structure.
    }
    and non-spatial settings (or \ac{tame} as people call them).
    The syntactic prominence of the subject is also introduced in this stage (commonly known as TP), 
    but without adjuncts it is hard to see the structural prominence of the subject,
    and without question formation, etc, it is hard to see the consequences of the prominence.
    The result of these steps, after including obligatory speech act marking,
    is a minimal canonical clause.
    \item Several adjunctions can be made to the minimal clause, 
    conveying information like time, position, manner, etc. 
    This step may be seen as an additional one,
    or it may be considered as occurring together with \ac{tame}.
    In the latter analysis, we need an explicit subject promoting analysis,
    which makes the subject in a higher position than any other complements.
    The result of these steps, after including obligatory speech act marking, 
    is a canonical clause.
    \item Speech act information 
    (or ``force'' in generative terms, or ``mood'' in \ac{blt}) 
    -- declarative, interrogative, imperative -- are added to the result of 
    \ac{tame} marked and possibly adjoined small clauses. 
    These stages occur in the CP domain.

    After these steps, the clause is now fully assembled. 
    If the speech act information marked meets the standards set in the language
    (e.g. the clause is finite, there is no subordination markers like a complementizer, etc.), 
    the clause is a qualified \emph{sentence}, 
    i.e. it can occur independently in utterance.
    But it can be embedded (or subordinated) into other clauses, too, probably with some additional marking.
    A clause without any subordinated clauses as complements or adjuncts is called a \concept{simple clause}.
    Otherwise it is a \concept{complex clause}.

    The declarative marking is often (but not always) zero, 
    and in this way there is no obligatory speech act marking for clauses. 

    \item The canonical clause may further undergo processes like topicalization or focusing.
    This makes it non-canonical.

    Topicalization and focusing may be analyzed as happening together 
    with speech act marking and/or subordination.
    In the classical generative analysis, the CP domain is roughly Force-Focus-Topic-Finiteness,
    with the left feature being introduced after the right feature.
    Subordination involves both 
    finiteness (gerund clauses can be subordinated, but they are never independent sentences) 
    and force (subordinated clauses often have limited force choices, 
    and the complementizer has to be introduced at a certain position).
    So in this analysis, there is no strict time ordering between topicalization etc. and subordination.

    But it is also possible to introduce topicalization and focusing with transformational rules,
    since most grammars are much more surface-oriented.
    In this case, it still makes sense to say one happens after another.

    Transformational rules are largely abandoned in contemporary generative syntax.
    In a surface-oriented grammar, 
    transformational rules does not say anything about the plausible generative derivational%
    \footnote{From then on, \term{derivational process}, etc. all mean Minimalist ones, i.e. Merge-based ones. 
    But \term{derived} may be associated with surface-oriented transformational rules.} process in our brain.
    If construction A and construction B are connected by a transformation,
    it merely means the derivational process of A and B have some similar stages 
    and there is a uniform correspondence between them. 
    \item There are usually certain kinds of valency-changing devices. 
    They may be implemented by a Latin-like voice system, 
    or by auxiliary verbs the subject of which undergoes an event,
    and the complement of which is a clause or small clause indicating the event,
    or by certain kind of \vP constructions.
    There are no strict boundary between these strategies.
    Combination of these devices, like \citet{collins2005pass}, is also possible.
    Just like the case of arguments and adjuncts, and the case of topicalization and speed act marking,
    passivization can be described with a derivational account as in contemporary generative syntax,
    as well as a transformational approach.
    \item Each stage of the above processes may be grammaticalized or be realized by a grammaticalized construction.
    The so-called Chinese passive construction, the \corpus{bei}-construction,
    is just a grammaticalized complement clause construction 
    with the verb \corpus{bei} expressing a meaning of ``suffer from''. 
    \item Finally, sentences and clauses can be coordinated and undergo supplementation. 
\end{enumerate}

The pipelines of the machine of English clause structure is also organized in this way. 
Valency is an enduring topic in English grammar.
Tense and aspect (usually denoted together as \term{tense}) appears 
in every introduction of the English language:
do something, did something, be doing something.
The subjunctive mood (or ``modality'' in \ac{blt}) is especially necessary in formal or archaic writing.
English adverbs are traditionally ignored but are more complicated than many will think.
The list can be very long.

\subsection{Canonical clauses and their inner grammatical relations}

\ac{cgel} uses the strategy to first describe a canonical clause 
and then use syntactic processes (or transformational rules in generative terms) 
to derive non-canonical ones.
Several syntactic processes may be used together to derive a clause with multiple non-canonical features.

It should be noted that when the definition of canonical clauses is narrow 
(a wise decision leading to easier starting),
it is possible that some non-canonical clauses cannot be obtained 
by applying well-recognized syntactic processes (\ac{cgel} \citesec{2.2} [3]).
This is shown in \prettyref{fig:canonical-clause}.

\begin{figure}
    \centering
    

\tikzset{every picture/.style={line width=0.3pt}} %set default line width to 0.75pt        

\begin{tikzpicture}[x=0.75pt,y=0.75pt,yscale=-1,xscale=1]
%uncomment if require: \path (0,386); %set diagram left start at 0, and has height of 386

%Straight Lines [id:da13680161853786577] 
\draw    (194,152.81) -- (313.26,84.8) ;
\draw [shift={(315,83.81)}, rotate = 150.31] [fill={rgb, 255:red, 0; green, 0; blue, 0 }  ][line width=0.08]  [draw opacity=0] (12,-3) -- (0,0) -- (12,3) -- cycle    ;
%Straight Lines [id:da16837784710650316] 
\draw    (194,152.81) -- (310.33,229.71) ;
\draw [shift={(312,230.81)}, rotate = 213.47] [fill={rgb, 255:red, 0; green, 0; blue, 0 }  ][line width=0.08]  [draw opacity=0] (12,-3) -- (0,0) -- (12,3) -- cycle    ;
%Straight Lines [id:da07747924239146564] 
\draw    (194,152.81) -- (306.85,314.17) ;
\draw [shift={(308,315.81)}, rotate = 235.03] [fill={rgb, 255:red, 0; green, 0; blue, 0 }  ][line width=0.08]  [draw opacity=0] (12,-3) -- (0,0) -- (12,3) -- cycle    ;
%Straight Lines [id:da42303822011922976] 
\draw  [dash pattern={on 4.5pt off 4.5pt}]  (358,201.81) -- (358,98.15) ;
\draw [shift={(358,203.81)}, rotate = 270] [fill={rgb, 255:red, 0; green, 0; blue, 0 }  ][line width=0.08]  [draw opacity=0] (12,-3) -- (0,0) -- (12,3) -- cycle    ;
%Straight Lines [id:da4793554387643766] 
\draw  [dash pattern={on 4.5pt off 4.5pt}]  (358,293.81) -- (358,266.81) ;
\draw [shift={(358,295.81)}, rotate = 270] [fill={rgb, 255:red, 0; green, 0; blue, 0 }  ][line width=0.08]  [draw opacity=0] (12,-3) -- (0,0) -- (12,3) -- cycle    ;

% Text Node
\draw (118,134) node [anchor=north west][inner sep=0.75pt]   [align=left] {\begin{minipage}[lt]{44.93pt}\setlength\topsep{0pt}
\begin{center}
argument\\structure
\end{center}

\end{minipage}};
% Text Node
\draw (200,36) node [anchor=north west][inner sep=0.75pt]   [align=left] {\begin{minipage}[lt]{39.53pt}\setlength\topsep{0pt}
\begin{center}
trivial\\polarity,\\voice,\\etc. 
\end{center}

\end{minipage}};
% Text Node
\draw (328,57) node [anchor=north west][inner sep=0.75pt]   [align=left] {\begin{minipage}[lt]{44.05pt}\setlength\topsep{0pt}
\begin{center}
canonical\\clauses
\end{center}

\end{minipage}};
% Text Node
\draw (313,213) node [anchor=north west][inner sep=0.75pt]   [align=left] {\begin{minipage}[lt]{63.87pt}\setlength\topsep{0pt}
\begin{center}
some\\non-canonical\\clauses
\end{center}

\end{minipage}};
% Text Node
\draw (315,296) node [anchor=north west][inner sep=0.75pt]   [align=left] {\begin{minipage}[lt]{63.87pt}\setlength\topsep{0pt}
\begin{center}
other\\non-canonical\\clauses
\end{center}

\end{minipage}};
% Text Node
\draw (181,240) node [anchor=north west][inner sep=0.75pt]   [align=left] {\begin{minipage}[lt]{45.76pt}\setlength\topsep{0pt}
\begin{center}
nontrivial\\polarity,\\voice,\\etc. 
\end{center}

\end{minipage}};
% Text Node
\draw (409,114) node [anchor=north west][inner sep=0.75pt]   [align=left] {\begin{minipage}[lt]{68.79pt}\setlength\topsep{0pt}
\begin{center}
transformation\\processes
\end{center}

\end{minipage}};
% Text Node
\draw (412,260) node [anchor=north west][inner sep=0.75pt]   [align=left] {\begin{minipage}[lt]{64.45pt}\setlength\topsep{0pt}
\begin{center}
generalization
\end{center}

\end{minipage}};


\end{tikzpicture}

    \caption{Building up canonical and non-canonical clauses}
    \label{fig:canonical-clause}
\end{figure}

It is already reviewed in \prettyref{sec:pipeline} that clauses, strictly speaking, 
are fully marked with speech act information and hence are CPs themselves.
Some people use the term \term{clause} for TPs and \term{sentence} for CPs.
This does not agree with the acceptable terminology in surface-oriented studies,
since a pure TP without CP layers rarely occurs.
Therefore, in \ac{cgel}, the pipelines of clause building is not made quite clear:
the approach accepted is to present grammatical relations as \emph{static} ones.

Here is a demonstration about this static approach. Recall \prettyref{fig:to-the-fat-man-blt}.
Though (a) has more embedding hierarchies than (b),
while (b) uses dependency relations in lieu of the binary branching hierarchies,
both of them are static ones:
(a) is presented in the grammar \emph{as a whole}, 
i.e. the grammar does not treat (a) as built by first merging \corpus{fat} and \corpus{man}
and then merging the result with \corpus{the} and then \corpus{to}.
Rather, once a PP is discussed, all grammatical relations involved
-- attributive modification, definiteness, dative construction -- 
are considered as completed.
Similarly, when talking about a clause, 
all grammatical relations in the clause is considered as already given in \ac{cgel}.
We \emph{do not} describe the structure in terms of Merging elements.
The discrepancy between \ac{cgel} and \ac{blt} is merely that 
the former uses hierarchy structure to show grammatical relations while the latter uses dependency arcs.

It should also be noted that though the \ac{cgel} approach is mostly static and not derivational,
its analyses are not idiomatically representational, either,
for the obvious deviation from approaches like HPSG.
It is best described as \ac{tag}-like.
Sometimes, the pipeline notion is of descriptive benefits, 
because some syntactic processes definitely work in order:
for example, subject-auxiliary inversion always happens before \term{wh}-fronting.
This is discussed in \ac{cgel} \citesec{2.2}: 
``Main clause interrogatives \dots will 
thus have one prenucleus + nucleus construction \dots functioning as nucleus within another \dots''

\subsubsection{Canonical clauses}

A \concept{canonical clause} is a standalone, declarative, active clause 
consisting of only the verb complex 
(\emph{do} or \emph{has done} or \emph{be doing} or \dots, i.e. the verb phrase in \ac{blt})
, complements and adjuncts. 
A \concept{minimal canonical clause} does not have adjuncts.

This definition means negative clauses are not canonical,
because \corpus{not} is an additional clausal dependent.
An interrogative clause is not canonical,
since it is not declarative and has prenucleus dependents 
-- the fronted verb and, if any, the \term{wh}-expression.
A passive clause is not canonical.
A subordinated clause is not canonical, 
because even though it is declarative and active,
it is not standalone.
(\ac{cgel} \citesec{2.2} [1])


\subsubsection{Clausal complements and the argument structure}

Though the starting point of the building process of all clauses is the argument structure,
\ac{cgel} starts \citechap{4} with discussion on clausal complements.
This is a wise decision for a largely surface-oriented grammar,
and also does not obscure the argument structure, 
because how the arguments (or non-argument complements) of the verb 
fills the \emph{clausal} complement positions
is largely based on the structural distribution of these arguments in the \vP{} structure.
In other words, the clause complement positions -- subject, direct and indirect objects, etc. --
are \emph{syntactic} marking of coarse-grained S, A, O, E arguments,
which reflect the relative positions of arguments in \vP.%
\footnote{
    O is the label of the patient-like or \term{patientive} argument in \ac{blt}.
    In many modern grammars, the symbol is replaced by P.
    In this note the symbol O is used to be consistent with the notation in \ac{cgel}, 
    though \ac{cgel} itself uses O as a symbol of clausal complements.
}

For discussion on various complement positions and their correlation with the argument structure,
see \prettyref{sec:types-of-complements}.

\subsubsection{The subject}

English is a subject-prominent language.
This is to say, in every clause there is a \concept{subject} 
(which may be omitted in limited cases) % TODO: what cases?
which is usually filled by an agent-like argument
and is somehow ``higher'' and external in the clause structure,
and therefore is subject to several types of extraction, 
including relativization and coordination pivot.

In typological terms, this means English is a syntactic nominative-accusative language,
because its syntactic obligatory topic 
(something somehow ``higher'' in the clause structure,
but lower than the topic in topicalization)
is identical to the agent-like argument.%
\footnote{In languages like Chinese, though when a semantically obvious agent is present,
it is always the syntactic obligatory topic, 
there are quite common clauses in which the subject is not agent-like at all,
like the famous \corpus{tai shang zuo zhe zhu xi tuan}.
But a deeper analysis will show that this can be attributed to 
the rich light verb inventory of Chinese,
which may be understood as ``mini-voice'' in more descriptive terms,
and therefore does not impose any threat to the nominative-accusative status of Chinese.}
In ergative languages the notion of \term{subject} is more complicated,
because in transitive clauses, the two do not coincide.
But this is not the case for English.

\begin{figure}
    \centering
    

\tikzset{every picture/.style={line width=0.3pt}} %set default line width to 0.75pt        

\begin{tikzpicture}[x=0.75pt,y=0.75pt,yscale=-1,xscale=1]
%uncomment if require: \path (0,300); %set diagram left start at 0, and has height of 300

%Straight Lines [id:da9834013734875582] 
\draw    (125,95.81) -- (183,78.81) ;
%Straight Lines [id:da10465096544133767] 
\draw    (248,95.81) -- (183,78.81) ;
%Straight Lines [id:da9284048202068891] 
\draw    (191,155.81) -- (249,138.81) ;
%Straight Lines [id:da46755571549890007] 
\draw    (314,155.81) -- (249,138.81) ;
%Straight Lines [id:da9217356601959923] 
\draw    (191,155.81) -- (314,155.81) ;

% Text Node
\draw (183,75.81) node [anchor=south] [inner sep=0.75pt]   [align=left] {Clause};
% Text Node
\draw (125,98.81) node [anchor=north] [inner sep=0.75pt]   [align=left] {\begin{minipage}[lt]{36.4pt}\setlength\topsep{0pt}
\begin{center}
Subject
\end{center}

\end{minipage}};
% Text Node
\draw (248,98.81) node [anchor=north] [inner sep=0.75pt]   [align=left] {\begin{minipage}[lt]{134.96pt}\setlength\topsep{0pt}
\begin{center}
Predicate:\\VP (or rarely something else)
\end{center}

\end{minipage}};
% Text Node
\draw (237,161.4) node [anchor=north west][inner sep=0.75pt]    {$\cdots $};


\end{tikzpicture}

    \caption{A minimal canonical clause is made up by a subject and a predicate}
    \label{fig:subject-predicate}
\end{figure}

The external property of subjects means we have the tree diagram \prettyref{fig:subject-predicate}.
The rest of the clause is named the \concept{predicate},
which is mostly a VP, but in some cases can be verbless (\ac{cgel} \citesec{2.14} [1]).
Since functional heads in the \vP-TP-CP hierarchy may be viewed as realized on the verb,
the ultimate head is the verb,
and in \prettyref{fig:subject-predicate}, the predicate is the head.

For details about the subject position, see \prettyref{sec:subject}.

\subsubsection{Agreement}\label{sec:agreement}

Agreement, if any, is only found between the subject and the highest verb in the predicate,
be it the main verb or an auxiliary verb. 
For details see \prettyref{sec:verb-agreement}.

\subsubsection{Objects, predicative complements, and others, and the structure of a minimal VP}

A VP always start with a verb. What follow the verb are inner complements.
In English possible complements include objects and predicative complements.
The complexity of dependency relations inside the VP blocks the possibility 
to represent them with a constituency tree without movement,
and hence \ac{cgel} presents the inner structure of the VP 
as a multiple-branching tree (\ac{cgel} \citesec{12.3.1} [4]).
The scheme of VPs is shown in \prettyref{fig:verb-phrase}.

\begin{figure}
    \centering
    

\tikzset{every picture/.style={line width=0.3pt}} %set default line width to 0.75pt        

\begin{tikzpicture}[x=0.75pt,y=0.75pt,yscale=-1,xscale=1]
%uncomment if require: \path (0,300); %set diagram left start at 0, and has height of 300

%Straight Lines [id:da9808512749731233] 
\draw    (131,102.81) -- (235,75.81) ;
%Straight Lines [id:da3092850381738288] 
\draw    (254,102.81) -- (235,75.81) ;
%Straight Lines [id:da017157875147768342] 
\draw    (197,162.81) -- (255,145.81) ;
%Straight Lines [id:da005860005871168195] 
\draw    (320,162.81) -- (255,145.81) ;
%Straight Lines [id:da5175071366232089] 
\draw    (197,162.81) -- (320,162.81) ;
%Straight Lines [id:da43669314291625905] 
\draw    (370,102.81) -- (235,75.81) ;

% Text Node
\draw (235,72.81) node [anchor=south] [inner sep=0.75pt]   [align=left] {VP};
% Text Node
\draw (131,105.81) node [anchor=north] [inner sep=0.75pt]   [align=left] {\begin{minipage}[lt]{28.49pt}\setlength\topsep{0pt}
\begin{center}
Head:\\V
\end{center}

\end{minipage}};
% Text Node
\draw (254,105.81) node [anchor=north] [inner sep=0.75pt]   [align=left] {\begin{minipage}[lt]{113.47pt}\setlength\topsep{0pt}
\begin{center}
Complement:\\NP, VP, PP, Clause, etc.
\end{center}

\end{minipage}};
% Text Node
\draw (243,168.4) node [anchor=north west][inner sep=0.75pt]    {$\cdots $};
% Text Node
\draw (370,106.21) node [anchor=north] [inner sep=0.75pt]    {$\cdots $};


\end{tikzpicture}

    \caption{The inner structure of a VP}
    \label{fig:verb-phrase}
\end{figure}

Note that complements are not necessarily referential, not even lexical.
Referential complements that are typically filled by NPs are \concept{objects},
and strongly predicative complements are \concept{predicate complements}.

There are also adjunct-like complements for some verbs:
the adverb \corpus{badly} in \corpus{he treats us badly},
the PP \corpus{to her article} in \corpus{he referred to her article}, etc.
The adverb complement can usually be replaced by a PP (\prettyref{sec:adjunct}).
Prepositional complements can similarly be divided into object-like ones 
and predicative complement-like ones 
(\ac{cgel} \citesec{2.4} [4]).

There is yet a final class of complements: \concept{particles}.
Particles are preposition-like but are able to appear after the NP connected with it.

\subsubsection{Case marking}

In 

\subsubsection{Tense, aspect and mood}\label{sec:tense-aspect}

Tense is the grammatical category about obligatory time marking.
Aspect is the grammatical category about obligatory marking of the temporal organization of the action.
Mood is the grammatical category about modality -- 
though \ac{blt} uses \term{modality} for \term{mood} in \ac{cgel} 
and \term{mood} for \term{clause type} in \ac{cgel}.

In \ac{cgel}, the perfect-imperfect contrast is placed under tense, not aspect.

\subsubsection{Adjuncts}\label{sec:adjunct}

A VP may have left or right adjuncts. 
\term{Adjunct} is the synonym of \term{clausal modifier},
or \term{adverbial} in some grammars.
Adjuncts may also be added to the left of a minimal canonical clause.
% TODO: 这一节的位置要移动,因为non-canonical clause也可以经历adjunction;但是也许也未必要移动,因为non-canonical clause的转换规则应该足够把adjunct的信息带进去了
It should be noted that adjuncts of different functions fill different positions
and are by no means \term{adjuncts} in early version of generative syntax.
They are actually peripheral arguments in \ac{blt}.
This is even the case for adverbs: 
adverbs can be almost straightforwardly 
(but not always: some additional thinking is needed to transform some adverbs to PPs) 
replaced by PPs which codes peripheral arguments 
that can be filled by NPs denoting abstract properties
(e.g. from \corpus{exceedingly} to \corpus{in an exceeding manner}).

\subsection{Non-canonical clauses without prenucleus}\label{sec:in-clause-transformation}

This section is about non-canonical clauses that largely keep the structure of canonical clauses,
so structural analyses of canonical clauses like \prettyref{fig:subject-predicate} still works. % TODO: other figures about the structure of the canonical clause

Below I collect non-canonical constructions, with the order from more internal ones to more external ones.

\subsubsection{Polarity}



\subsubsection{Voice}\label{sec:voice}

\subsubsection{Finiteness}\label{sec:finiteness}

A nonfinite clause is a clause that obligatorily lacks some TP and/or CP functional projections,
often resulting in inability to have a typical subject (since the TP layer is somehow quirky).
It may be considered as somehow nominalized and the nominalization degree is deeper than content clauses, %TODO: ref to content clause
since its main role is to fill argument slots,
and unlike finite content clauses that can be a sentence with a minimal transformational,
the nonfinite clause can never be a full sentence.
It can still constitute a short reply to a question, as in 
\corpus{-- What are you doing? -- Doing my midterm project.}
But NPs have this function, too.

In English, nonfinite clauses are either participles or infinitives.
None of them is able to have a typical, nominative (if realized as a pronoun) subject,
and all of them have distinct morphosyntactic marking.

It should be noted that participle differ from ``genuine'' nominalization. % TODO: ref
A participle takes object(s), 
while a nominalized verb does not. 

\subsubsection{Clause types}\label{sec:force}

In \ac{cgel}, \concept{clause type} means Force in generative syntax and mood in \ac{blt}.
In English there are the following clause types:
declarative, closed interrogative, open interrogative, exclamative, imperative.

The clause type is the syntactic coding of \concept{illocutionary force}:
an imperative clause expresses a direct illocutionary force.
But illocutionary force need not be coded as the clause type:
\corpus{would you please \dots} is formally interrogative, 
but it is often a polite directive.
In this case, the illocutionary force is indirect.

\subsubsection{Subordination}

\subsection{Prenucleus}\label{sec:prenucleus}

Some non-canonical clauses go further: something is promoted to a position even higher than the subject.
Such positions are called \concept{prenucleus} positions (\ac{cgel} \citesec{2.2} [5]), 
and we get a structure like \prettyref{fig:prenucleus}.

\begin{figure}
    \centering
    

\tikzset{every picture/.style={line width=0.3pt}} %set default line width to 0.75pt        

\begin{tikzpicture}[x=0.75pt,y=0.75pt,yscale=-1,xscale=1]
%uncomment if require: \path (0,330); %set diagram left start at 0, and has height of 330

%Straight Lines [id:da6926211514368799] 
\draw    (254,180.81) -- (312,163.81) ;
%Straight Lines [id:da09624855735969873] 
\draw    (377,180.81) -- (312,163.81) ;
%Straight Lines [id:da7698332578870457] 
\draw    (320,240.81) -- (378,223.81) ;
%Straight Lines [id:da14094495477352398] 
\draw    (443,240.81) -- (378,223.81) ;
%Straight Lines [id:da9272675571958595] 
\draw    (320,240.81) -- (443,240.81) ;
%Straight Lines [id:da2811370178161874] 
\draw    (186,114.81) -- (244,97.81) ;
%Straight Lines [id:da05920620141627109] 
\draw    (309,114.81) -- (244,97.81) ;
%Curve Lines [id:da1523420057005347] 
\draw    (378,263.81) .. controls (290,353.81) and (206,335.81) .. (186,161.81) ;
\draw [shift={(186,161.81)}, rotate = 83.44] [fill={rgb, 255:red, 0; green, 0; blue, 0 }  ][line width=0.08]  [draw opacity=0] (12,-3) -- (0,0) -- (12,3) -- cycle    ;

% Text Node
\draw (312,160.81) node [anchor=south] [inner sep=0.75pt]   [align=left] {\begin{minipage}[lt]{40.42pt}\setlength\topsep{0pt}
\begin{center}
Nucleus:\\Clause
\end{center}

\end{minipage}};
% Text Node
\draw (254,183.81) node [anchor=north] [inner sep=0.75pt]   [align=left] {\begin{minipage}[lt]{36.4pt}\setlength\topsep{0pt}
\begin{center}
Subject
\end{center}

\end{minipage}};
% Text Node
\draw (377,183.81) node [anchor=north] [inner sep=0.75pt]   [align=left] {\begin{minipage}[lt]{134.96pt}\setlength\topsep{0pt}
\begin{center}
Predicate:\\VP (or rarely something else)
\end{center}

\end{minipage}};
% Text Node
\draw (366,246.4) node [anchor=north west][inner sep=0.75pt]    {$\cdots $};
% Text Node
\draw (244,94.81) node [anchor=south] [inner sep=0.75pt]   [align=left] {Clause};
% Text Node
\draw (186,117.81) node [anchor=north] [inner sep=0.75pt]   [align=left] {\begin{minipage}[lt]{53.6pt}\setlength\topsep{0pt}
\begin{center}
Prenucleus:\\$\displaystyle \cdots $
\end{center}

\end{minipage}};


\end{tikzpicture}

    \caption{Prenucleus positions}
    \label{fig:prenucleus}
\end{figure}

\subsubsection{Subject-auxiliary inversion}

\subsubsection{Interrogative}

\subsection{Information packaging}

\prettyref{sec:in-clause-transformation} and \prettyref{sec:prenucleus} are about 
transformations on canonical clauses, 
about what feature triggers what transformation,
and relevant syntactic positions that are not observed in canonical clauses.
There are, however, constructions of which constituency analyses involve 
nothing more than \prettyref{sec:in-clause-transformation} and \prettyref{sec:prenucleus}
but are unable to show the information structure of these constructions.
The constituency tree of a \corpus{there be} construction is just 
an instance of the canonical clausal structure \prettyref{fig:subject-predicate},
with the subject being a dummy \corpus{there},
but from the constituency tree one cannot infer confidently the existential meaning.

Without considering the information structure, 
the way perfect aspect is realized in English
resembles the example of the \corpus{there be} existential construction
in that its meaning is not combinatory:
in \corpus{he has done this before},
the main verb may be analyzed as \corpus{has},
which agrees with the subject (\prettyref{sec:agreement})
and takes the past participle VP \corpus{done this before} as its complement.
In the VP \corpus{done this before}, \corpus{before} is an adjunct, 
and \corpus{this} is the object. 
The structure is crystal-clear.
The only problem is the true meaning of the clause cannot be inferred from the structure.
However, the perfect aspect is not marked with respect to the information structure,
so it is not introduced in the chapter about information packaging in \ac{cgel}.

\subsection{Coordination}

The English clausal coordination construction is largely symmetric, 
thought there are reasons to believe the second clause and the coordinator form a constituent.
Since the coordination head is realized as a single word but is by no means lexical,
it is impossible to assign the head status to any lexical word as in \prettyref{fig:cgel-minimalism},
and therefore in \ac{cgel}, coordination is deemed as a headless construction. % TODO: ref

\subsection{Supplementation}

\subsection{The organization of chapters in \ac{cgel}}

\section{Complements and adjuncts}

\subsection{Types of complements}\label{sec:types-of-complements}

Clausal (or verbal, since the clause is headed by the verb) complements 
may be NPs and PPs, and less frequently, adverbs 
(as in \corpus{He treated us [kindly]}). % TODO: adverbial clause: the name and classification 

This section lists some criteria of classification of complements.
They are all discussed in \ac{cgel} \citesec{4.1.1}.


How to tell a complement from adjuncts is a question addressed in \prettyref{sec:recognizing-complement-clause}.
\prettyref{sec:recognizing-complement-clause} is delayed after discussion on complements and adjuncts,
because we have to first show prototypical properties of the two 
and introduce necessary concepts like how semantic roles are coded as clausal dependents,
and only then can we draw an exact line between the two.

\subsubsection{Core v.s. oblique}\label{sec:core-oblique}

One classification standard is the make up of the complement.
A \concept{core} complement is a complement with similar morphosyntactic properties of NP complements.
A \concept{non-core} complement is a complement with similar morphosyntactic properties of PP complements.
If a non-core complement itself takes an NP complement (or something with similar morphosyntactic properties),
the latter is called an \concept{oblique}.

Note that in \ac{cgel}, the term \term{argument} is reserved for purely semantic objects.
A clausal complement is therefore the syntactic incarnation of an argument,
but itself is not an argument.
This is not the way \term{argument} is used in \ac{blt}.

It should also be noted that in \ac{cgel}, the terms \term{non-core} and \term{oblique} 
are reserved for clausal complements and the NP part of PP clausal complements.
They \emph{do not} include adjuncts with similar forms.
On the other hand, in \ac{blt}, the term \term{peripheral argument} covers both complements and adjuncts.
The term \term{oblique} is often associated with \term{oblique cases}.
In traditional Latin grammar, 
cases other than the nominative and the vocative cases are all called oblique cases.
In other usages, oblique cases exclude the accusative case.
In English the case system has largely collapsed,
and the name \term{oblique} does not have much morphological consequences:
an oblique complement is never nominative as we will see, and that is all.

The prototypical definition of core and oblique complements 
are based on syntactic forms instead of functions,
while the definition is extended by analog with respect to syntactic functions.
Whether these terms are useful is a question we need to wait and see.

\subsubsection{External and internal}



The subject is the \concept{external} complement for obvious reasons.
All other complements are \concept{internal}.

In English, internal complements include \concept{objects} and \concept{predicative complements}.
The objects split into \concept{direct objects} and \concept{indirect objects}.
Objects are prototypically NPs, 
while predicative complements are predicative.

\subsubsection{Relation with the argument structure}\label{sec:s-a-o-e}

In the typological perspective, 
both the coarse-grained argument structure and 
the coding strategy of the argument slots (or semantic roles) as clausal complements 
are quite straightforward in English.

The argument structure of any verb in English fits into the S, A, O, E paradigm in \ac{blt}.
Some typological studies have G and T abstract semantic roles, 
but in English no goal-like and theme-like semantic role classes with stable syntactic appearance 
can be established.
Consider, for example, the example in \ac{blt} \citesec{3.3} (6) and (7),
with semantic role labels replaced by ones in \ac{cgel} \citesec{4.2.2}:
\begin{exe}
    \ex \label{ex:john-gave-goods-to-charity} 
    John gave [all his goods]_{\text{O, theme}} [to charity]_{\text{E, goal}}
    \ex \label{ex:john-gave-student-book} 
    John gave [his favorite student]_{\text{O, goal}} [some books]_{\text{E, theme}}
\end{exe}
The PP \corpus{to charity} and NP \corpus{some books} have similar syntactic behaviors,
and in a similar manner, \corpus{all his goods} and \corpus{his facorite students} 
form another group of arguments with similar syntactic appearance.
The first two cannot be promoted to the subject position in passivization,
while the latter two can.
It is, therefore, reasonable to name the group containing 
\corpus{all his goods} and \corpus{his facorite students} 
with the label O,
and name the group containing
\corpus{to charity} and \corpus{some books}
with the label E.
Note, however, there are some heterogeneity in each group:
\corpus{all his goods} is a theme while \corpus{his favorite student} is a goal,
but they have similar syntactic properties.
So the division between O and E is useful in English 
(which is also shown in \citefootnote{26} in \ac{cgel} \citechap{4}),
while the division between G and T is not.

% TODO: E argument是否单独出现?是否有He speaks with a very quick pace 这样,但是后面的PP是complement的动词? 
% TODO:He treats me with kindness中,kindness应该是E论元,因此V-O-E论元结构不只有一种,依照论元种类的不同,有多种clausal structure coding的方式

In canonical clauses,
A and S arguments are consistently coded as the subject, both syntactically and, 
in the case of pronouns, morphologically.
No split of S arguments is easily observable. % TODO: any subtle details?
O arguments are uniformly coded as clausal objects.
E arguments may be coded as clausal objects (as in \eqref{ex:john-gave-student-book}) 
or obliques (as in \eqref{ex:john-gave-goods-to-charity}).
% TODO: direct and indirect object

\subsection{Types of minimal canonical clauses}

The contents of \prettyref{sec:clause-transitivity} and \prettyref{sec:clause-pc} 
are covered in \ac{cgel} \citesec{4.1.1}.

\subsubsection{Subject and object(s): transitivity and valency}\label{sec:clause-transitivity}

In English every clause has a subject, so the number of subjects is not a parameter.
The number of objects may be 0, 1, and 2,
which is denoted as \concept{intransitive}, \concept{transitve}, and \concept{ditransitive}.
This parameter is named as \concept{transitivity}.

\subsubsection{The number of predicative complements}\label{sec:clause-pc}

There may be 0 or 1 \concept{predicative complement (PC)}.
It is impossible for a ditransitive clause to have a PC.
So the parameters of transitivity and PC gives the classification 
of clauses with respect to their complements 
as in \ac{cgel} \citesec{4.1.1} [9].
Hence we have the notion of \concept{valency} (\ac{cgel} \citesec{4.1.1} [10]). 
The number of PC is not included into the valency.

\ac{cgel} \citesec{4.1.1} [9] is just the classification based on S, O, and PC.
It does not mean there are no other complements that do not fit perfectly in the paradigm.

\subsubsection{Five canonical clauses}

\subsection{The subject} \label{sec:subject}

\subsection{Direct and indirect objects}\label{sec:object}

\subsection{Types of adjuncts}\label{sec:adjuncts-classification}


\subsection{Semantic roles}

Once clausal complement positions are related to coarse-grained argument positions (\prettyref{sec:s-a-o-e}),
the question becomes how concrete semantic roles of verbs fit into the paradigm.

\subsubsection{The causer-like group}

\subsection{Distinguish complements from adjuncts}\label{sec:recognizing-complement-clause}

\subsection{Summary}



\section{The verb}

\subsection{Syncretism in English}

This section deals with subcategories in the verb category,
and how the verb changes its form according to the syntactic environment.
Relevant syntactic categories include 
finiteness (\prettyref{sec:finiteness}),
tense, aspect and mood (\prettyref{sec:tense-aspect}),
voice (\prettyref{sec:voice}),
and agreement with the subject (\prettyref{sec:agreement}).
These syntactic categories are all prototypical ones in Indo-European languages.
Note, however, in \ac{cgel} these features are not considered as the features of the verb itself,
but the clause. 
If these features are placed on the verb, 
then the verb \corpus{play} in a subjunctive clause is a homonym of 
the verb \corpus{play} in an indicative clauses,
but anyone will consider the two identical inflectional forms.

This does not mean these features do not play any role in distinguishing inflectional forms.
Another extreme of morphological analysis is to 
inserting all inflectional forms with the same surface appearance into one inflectional class.
This is also not the approach taken in \ac{cgel}.
To see the problem with this approach, consider the example of Latin fourth declension.
What is in common between the genitive singular and the nominative plural?
It definitely makes no sense to regard then the same.

The criteria to distinguish inflectional distinctions are shown in \ac{cgel} \citesec{5.1.2} [5].
It rejects the approach placing features like finiteness, tense, etc. on the verb,
which is essentially the traditional grammar approach%
\footnote{
    Note, however, the traditional inflectional paradigm is still useful though cumbersome:
    it visualizes possible environments verbs may appear in.
} (\citechap{3} \citefootnote{1}):
the uniformness of the finite part of the so-called inflectional paradigm of English verb
means this part should be compressed into just three inflectional forms.
It also rejects the approach positing minimal inflectional distinctions.
It is often said that English has five verbal inflectional forms:
\corpus{take}, \corpus{takes}, \corpus{took}, \corpus{taking}, \corpus{token}. 
But modal auxiliary verbs do not appear in infinitives:
the present tense form is therefore to be split into two
according to \ac{cgel} \citesec{5.1.2} [5],
one appears in a finite environment,
the other appears in a non-finite environment.
In the latter environment, modal auxiliary verbs are constantly absent,
therefore constituting a stable contrast between modal auxiliary verbs and lexical verbs,
while in the former environment there is no contrast between the two.
This split may be viewed as based on the finiteness category,
and not purely on the surface appearance of the verb.

\ac{cgel} \citesec{5.1.2} [5] accidentally rejects the analysis of Latin noun declension
that combines the genitive singular and the nominative plural,
because the identification of the two does not work for, say, the third declension,
so with the same logic that makes distinction between the finite and infinite \corpus{take}.
But suppose we have an imaginary language in which in all noun declension classes,
the genitive singular and the nominative plural have the same surface realization.
In this language, \ac{cgel} \citesec{5.1.2} [5] is not sufficient 
to reject temptation of analyzing the two as one inflectional form.
Fortunately, this is not the case in English, and I do not proceed more on 
inflectional paradigms.

\subsection{Transitivity and valency}

\subsection{Agreement with the subject}\label{sec:verb-agreement}

% TODO: classification by complements

\bibliographystyle{plainnat}
\bibliography{cambridge}

\end{document}