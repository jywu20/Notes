\documentclass{article}

\usepackage{geometry}
\usepackage{titling}
\usepackage{titlesec}
\usepackage{paralist}
\usepackage{footnote}
\usepackage{enumerate}
\usepackage{amsmath, amssymb, amsthm}
\usepackage{gb4e}
\noautomath
\usepackage{bbm}
\usepackage{soul}
\usepackage{graphicx}
\usepackage{siunitx}
\usepackage[table,xcdraw]{xcolor}
\usepackage{tikz}
\usepackage[ruled, vlined, linesnumbered, noend]{algorithm2e}
\usepackage{xr-hyper}
\usepackage[colorlinks]{hyperref} % linkcolor=black, anchorcolor=black, citecolor=black, filecolor=black
\usepackage[most]{tcolorbox}
\usepackage{caption}
\usepackage{subcaption}
\usepackage{booktabs}
\usepackage{multirow}
\usepackage[figuresright]{rotating}
\usepackage{acro}
\usepackage[round]{natbib} 
\usepackage{prettyref}

\geometry{left=3.18cm,right=3.18cm,top=2.54cm,bottom=2.54cm}
\titlespacing{\paragraph}{0pt}{1pt}{10pt}[20pt]
\setlength{\droptitle}{-5em}

\DeclareMathOperator{\timeorder}{\mathcal{T}}
\DeclareMathOperator{\diag}{diag}
\DeclareMathOperator{\legpoly}{P}
\DeclareMathOperator{\primevalue}{P}
\DeclareMathOperator{\sgn}{sgn}
\newcommand*{\ii}{\mathrm{i}}
\newcommand*{\ee}{\mathrm{e}}
\newcommand*{\const}{\mathrm{const}}
\newcommand*{\suchthat}{\quad \text{s.t.} \quad}
\newcommand*{\argmin}{\arg\min}
\newcommand*{\argmax}{\arg\max}
\newcommand*{\normalorder}[1]{: #1 :}
\newcommand*{\pair}[1]{\langle #1 \rangle}
\newcommand*{\fd}[1]{\mathcal{D} #1}

\newcommand*{\citesec}[1]{\S~{#1}}
\newcommand*{\citechap}[1]{chap.~{#1}}
\newcommand*{\citefig}[1]{Fig.~{#1}}
\newcommand*{\citetable}[1]{Table~{#1}}

\newrefformat{sec}{\citesec{\ref{#1}}}
\newrefformat{fig}{\citefig{\ref{#1}}}
\newrefformat{tbl}{\citetable{\ref{#1}}}
\newrefformat{chap}{\citechap{\ref{#1}}}

\usetikzlibrary{arrows,shapes,positioning}
\usetikzlibrary{arrows.meta}
\usetikzlibrary{decorations.markings}
\tikzstyle arrowstyle=[scale=1]
\tikzstyle directed=[postaction={decorate,decoration={markings,
    mark=at position .5 with {\arrow[arrowstyle]{stealth}}}}]
\tikzstyle ray=[directed, thick]
\tikzstyle dot=[anchor=base,fill,circle,inner sep=1pt]


\tcbuselibrary{skins, breakable, theorems}

\newtcbtheorem[number within=chapter]{infobox}{Box}%
  {colback=blue!5,colframe=blue!65,fonttitle=\bfseries, breakable}{infobox}

\newcommand*{\concept}[1]{\textbf{#1}}
\newcommand*{\term}[1]{\emph{#1}}
\newcommand*{\corpus}[1]{\emph{#1}}

\newcommand*{\vP}{\textit{v}P}

\DeclareAcronym{blt}{short = BLT, long = Basic Linguistic Theory}
\DeclareAcronym{cgel}{short = CGEL, long = The Cambridge Grammar of the English Language}
\DeclareAcronym{dm}{short = DM, long = Distributed Morphology}
\DeclareAcronym{tag}{long = Tree-adjoining grammar, short = TAG}
\DeclareAcronym{sfp}{long = sentence final particle, short = SFP}
\DeclareAcronym{vp}{long = verb phrase, short = VP}
\DeclareAcronym{cls}{long = classifier, short = CLS}
\DeclareAcronym{dist}{long = distal, short = DIST}
\DeclareAcronym{prox}{long = proximate, short = PROX}
\DeclareAcronym{dem}{long = demonstrative, short = DEM}
\DeclareAcronym{dur}{long = durative, short = DUR}
\DeclareAcronym{neg}{long = negative, short = NEG}

% Disable unsupported commands in bookmark titles 
\pdfstringdefDisableCommands{%
  \def\\{}%
  \def\texttt#1{<#1>}%
  \def\mathbb#1{#1}%
}
\pdfstringdefDisableCommands{\def\eqref#1{(\ref{#1})}}

\makeatletter
\pdfstringdefDisableCommands{\let\HyPsd@CatcodeWarning\@gobble}
\makeatother

\title{Reading notes of Cambridge Grammar of English Language}
\author{Jinyuan Wu}

\begin{document}

\maketitle

This note is a reconstruction of the contents \acl{cgel} (\citealt{cgel}, henceforth \acs{cgel}). 

\section{Theoretical preliminaries}

For the discussion on the underlying theoretical framework, 
see the relevant chapter in \href{../Chinese/main.pdf}{this note}.
Briefly speaking, the framework is coarse-grained Minimalism:
invisible functional heads are erased,
but dependency relations created by the functional projections are kept (\prettyref{fig:coarse-grained}(a)),
and the corresponding \ac{cgel} tree is obtained by replacing the name of dependency relations 
by syntactic function-form pairs (\prettyref{fig:coarse-grained}(b)).
The head is defined as the dominant \emph{lexical} word.
This makes \ac{cgel} look like old-fashioned X-bar theory, 
where a lexical word projects into a phrase (\ac{cgel} \citesec{5.2} [5]),
but there can be several ``maximal projections'' headed by the same lexical word (\ac{cgel} \citesec{5.2 [11]}).
This is actually consistent with Minimalism (\prettyref{fig:cgel-minimalism}).

The standard of ``lexical'', however, varies from one author to another.
A PP and a CaseP (in generative terms) are similar objects,
but in the coarse-grained \ac{cgel} framework,
the latter is definitely an NP, 
but the status of the first is kind of controversial:
should we recognize the preposition as a lexical word,
or as a functional word?
In the first analysis, a PP in generative terms is still a PP,
while in the second analysis, a PP in generative terms is an oblique NP.
Choosing between the two analyses is merely a problem of notation.
But it is good practice to choose a notation that hints the readers about 
certain properties of the construction in question.
For example, if a PP is analyzed as a PP, then probably the preposition has some predicative properties.

\begin{figure}
    \centering
    

\tikzset{every picture/.style={line width=0.75pt}} %set default line width to 0.75pt        

\begin{tikzpicture}[x=0.75pt,y=0.75pt,yscale=-0.85,xscale=0.85]
%uncomment if require: \path (0,408); %set diagram left start at 0, and has height of 408

%Straight Lines [id:da2665478368972569] 
\draw [color={rgb, 255:red, 245; green, 166; blue, 35 }  ,draw opacity=1 ] [dash pattern={on 4.5pt off 4.5pt}]  (264,61) -- (544,61) ;
\draw [shift={(546,61)}, rotate = 180] [fill={rgb, 255:red, 245; green, 166; blue, 35 }  ,fill opacity=1 ][line width=0.08]  [draw opacity=0] (12,-3) -- (0,0) -- (12,3) -- cycle    ;
%Straight Lines [id:da6278442099042232] 
\draw [color={rgb, 255:red, 245; green, 166; blue, 35 }  ,draw opacity=1 ] [dash pattern={on 4.5pt off 4.5pt}]  (295,107) -- (622,107) ;
\draw [shift={(624,107)}, rotate = 180] [fill={rgb, 255:red, 245; green, 166; blue, 35 }  ,fill opacity=1 ][line width=0.08]  [draw opacity=0] (12,-3) -- (0,0) -- (12,3) -- cycle    ;
%Rounded Rect [id:dp9005470977487176] 
\draw  [draw opacity=0][fill={rgb, 255:red, 80; green, 227; blue, 194 }  ,fill opacity=0.2 ] (494,41.22) .. controls (494,32.48) and (501.08,25.4) .. (509.82,25.4) -- (741.18,25.4) .. controls (749.92,25.4) and (757,32.48) .. (757,41.22) -- (757,325.99) .. controls (757,334.73) and (749.92,341.81) .. (741.18,341.81) -- (509.82,341.81) .. controls (501.08,341.81) and (494,334.73) .. (494,325.99) -- cycle ;
%Straight Lines [id:da3555404365615693] 
\draw [color={rgb, 255:red, 155; green, 155; blue, 155 }  ,draw opacity=1 ]   (630.05,206.81) -- (630.05,301) ;
%Straight Lines [id:da5116688308518689] 
\draw [color={rgb, 255:red, 80; green, 227; blue, 194 }  ,draw opacity=1 ]   (703.05,158.83) -- (665.05,140.5) ;
%Straight Lines [id:da7532901881200198] 
\draw [color={rgb, 255:red, 80; green, 227; blue, 194 }  ,draw opacity=1 ]   (665.05,140.5) -- (630.05,158.83) ;
%Straight Lines [id:da3708728686228633] 
\draw [color={rgb, 255:red, 80; green, 227; blue, 194 }  ,draw opacity=1 ]   (658.65,97.63) -- (601.2,71.73) ;
%Straight Lines [id:da18650265857430104] 
\draw [color={rgb, 255:red, 80; green, 227; blue, 194 }  ,draw opacity=1 ]   (601.2,71.73) -- (548.65,97.63) ;
%Rounded Rect [id:dp9951745940140804] 
\draw  [draw opacity=0][fill={rgb, 255:red, 255; green, 255; blue, 255 }  ,fill opacity=1 ] (675,215.16) .. controls (675,210.29) and (678.95,206.33) .. (683.83,206.33) -- (722.17,206.33) .. controls (727.05,206.33) and (731,210.29) .. (731,215.16) -- (731,314.17) .. controls (731,319.05) and (727.05,323) .. (722.17,323) -- (683.83,323) .. controls (678.95,323) and (675,319.05) .. (675,314.17) -- cycle ;
%Rounded Rect [id:dp4970986462373306] 
\draw  [draw opacity=0][fill={rgb, 255:red, 255; green, 255; blue, 255 }  ,fill opacity=1 ] (521.24,153.76) .. controls (521.24,148.82) and (525.25,144.81) .. (530.19,144.81) -- (568.05,144.81) .. controls (572.99,144.81) and (577,148.82) .. (577,153.76) -- (577,314.05) .. controls (577,318.99) and (572.99,323) .. (568.05,323) -- (530.19,323) .. controls (525.25,323) and (521.24,318.99) .. (521.24,314.05) -- cycle ;
%Rounded Rect [id:dp4071663798075751] 
\draw  [draw opacity=0][fill={rgb, 255:red, 80; green, 227; blue, 194 }  ,fill opacity=0.2 ] (106,41.89) .. controls (106,33.39) and (112.89,26.5) .. (121.4,26.5) -- (346.6,26.5) .. controls (355.11,26.5) and (362,33.39) .. (362,41.89) -- (362,324.41) .. controls (362,332.92) and (355.11,339.81) .. (346.6,339.81) -- (121.4,339.81) .. controls (112.89,339.81) and (106,332.92) .. (106,324.41) -- cycle ;
%Straight Lines [id:da5088138181155522] 
\draw [color={rgb, 255:red, 155; green, 155; blue, 155 }  ,draw opacity=1 ]   (217.05,133.93) -- (217.05,299.1) ;
%Straight Lines [id:da998446713157706] 
\draw [color={rgb, 255:red, 80; green, 227; blue, 194 }  ,draw opacity=1 ]   (290.05,133.93) -- (252.05,115.6) ;
%Straight Lines [id:da5752866187220778] 
\draw [color={rgb, 255:red, 80; green, 227; blue, 194 }  ,draw opacity=1 ]   (252.05,115.6) -- (217.05,133.93) ;
%Straight Lines [id:da6021628337779128] 
\draw [color={rgb, 255:red, 80; green, 227; blue, 194 }  ,draw opacity=1 ]   (249.65,94.73) -- (200.2,68.83) ;
%Straight Lines [id:da8361003247119008] 
\draw [color={rgb, 255:red, 80; green, 227; blue, 194 }  ,draw opacity=1 ]   (200.2,68.83) -- (153.65,94.73) ;
%Rounded Rect [id:dp3858870486231194] 
\draw  [draw opacity=0][fill={rgb, 255:red, 255; green, 255; blue, 255 }  ,fill opacity=1 ] (257.24,151.34) .. controls (257.24,145.52) and (261.96,140.81) .. (267.77,140.81) -- (313.48,140.81) .. controls (319.29,140.81) and (324,145.52) .. (324,151.34) -- (324,309.05) .. controls (324,314.86) and (319.29,319.57) .. (313.48,319.57) -- (267.77,319.57) .. controls (261.96,319.57) and (257.24,314.86) .. (257.24,309.05) -- cycle ;
%Rounded Rect [id:dp2951833006567197] 
\draw  [draw opacity=0][fill={rgb, 255:red, 255; green, 255; blue, 255 }  ,fill opacity=1 ] (134.24,118.4) .. controls (134.24,113.11) and (138.54,108.81) .. (143.84,108.81) -- (184.41,108.81) .. controls (189.71,108.81) and (194,113.11) .. (194,118.4) -- (194,305.98) .. controls (194,311.28) and (189.71,315.57) .. (184.41,315.57) -- (143.84,315.57) .. controls (138.54,315.57) and (134.24,311.28) .. (134.24,305.98) -- cycle ;

% Text Node
\draw (630.76,321) node [anchor=south] [inner sep=0.75pt]   [align=left] {hurt};
% Text Node
\draw (665.05,137.5) node [anchor=south] [inner sep=0.75pt]   [align=left] {\begin{minipage}[lt]{89.53pt}\setlength\topsep{0pt}
\begin{center}
predicate:\\smaller verb phrase
\end{center}

\end{minipage}};
% Text Node
\draw (601.2,68.73) node [anchor=south] [inner sep=0.75pt]   [align=left] {verb phrase};
% Text Node
\draw (548.65,100.63) node [anchor=north] [inner sep=0.75pt]   [align=left] {\begin{minipage}[lt]{29.61pt}\setlength\topsep{0pt}
\begin{center}
agent:\\DP
\end{center}

\end{minipage}};
% Text Node
\draw (630.05,161.83) node [anchor=north] [inner sep=0.75pt]   [align=left] {\begin{minipage}[lt]{50.89pt}\setlength\topsep{0pt}
\begin{center}
predicator:\\V
\end{center}

\end{minipage}};
% Text Node
\draw (703.05,161.83) node [anchor=north] [inner sep=0.75pt]   [align=left] {\begin{minipage}[lt]{36.95pt}\setlength\topsep{0pt}
\begin{center}
patient:\\DP
\end{center}

\end{minipage}};
% Text Node
\draw (216.76,321.1) node [anchor=south] [inner sep=0.75pt]   [align=left] {hurt};
% Text Node
\draw (249.65,97.73) node [anchor=north] [inner sep=0.75pt]   [align=left] {Trans - DP};
% Text Node
\draw (179.8,50.81) node [anchor=north west][inner sep=0.75pt]   [align=left] {$\displaystyle v$ - DP};
% Text Node
\draw (211,383) node [anchor=north west][inner sep=0.75pt]   [align=left] {(a)};
% Text Node
\draw (636,383) node [anchor=north west][inner sep=0.75pt]   [align=left] {(b)};


\end{tikzpicture}

    \caption{From coarse-grained Minimalist derivational tree to \ac{cgel} tree}
    \label{fig:coarse-grained}
\end{figure}

\begin{figure}
    \centering
    

\tikzset{every picture/.style={line width=0.3pt}} %set default line width to 0.75pt        

\begin{tikzpicture}[x=0.75pt,y=0.75pt,yscale=-0.8,xscale=0.8]
%uncomment if require: \path (0,523); %set diagram left start at 0, and has height of 523

%Straight Lines [id:da6705457534949089] 
\draw    (313,92.15) -- (343,75.15) ;
%Straight Lines [id:da28013820417559354] 
\draw    (376,92.15) -- (343,75.15) ;
%Straight Lines [id:da26185532929398736] 
\draw    (345,123.15) -- (375,106.15) ;
%Straight Lines [id:da1670225517468129] 
\draw    (408,123.15) -- (375,106.15) ;
%Straight Lines [id:da7951370858439135] 
\draw    (279,62.15) -- (309,45.15) ;
%Straight Lines [id:da5307564248915579] 
\draw    (342,62.15) -- (309,45.15) ;
%Straight Lines [id:da6264444449679678] 
\draw [color={rgb, 255:red, 155; green, 155; blue, 155 }  ,draw opacity=1 ]   (321,164.15) -- (152.85,234.38) ;
\draw [shift={(151,235.15)}, rotate = 337.33] [fill={rgb, 255:red, 155; green, 155; blue, 155 }  ,fill opacity=1 ][line width=0.08]  [draw opacity=0] (12,-3) -- (0,0) -- (12,3) -- cycle    ;
%Straight Lines [id:da99450357998767] 
\draw [color={rgb, 255:red, 80; green, 227; blue, 194 }  ,draw opacity=1 ]   (204,363.15) -- (234,346.15) ;
%Straight Lines [id:da7516847541397156] 
\draw [color={rgb, 255:red, 80; green, 227; blue, 194 }  ,draw opacity=1 ]   (267,363.15) -- (234,346.15) ;
%Straight Lines [id:da7086721116542554] 
\draw [color={rgb, 255:red, 80; green, 227; blue, 194 }  ,draw opacity=1 ]   (236,403.15) -- (266,386.15) ;
%Straight Lines [id:da23963037522787767] 
\draw    (299,403.15) -- (266,386.15) ;
%Straight Lines [id:da8629804052760564] 
\draw [color={rgb, 255:red, 74; green, 144; blue, 226 }  ,draw opacity=1 ]   (139,296.15) -- (146.67,291.8) -- (169,279.15) ;
%Straight Lines [id:da21212301774701436] 
\draw [color={rgb, 255:red, 74; green, 144; blue, 226 }  ,draw opacity=1 ]   (202,296.15) -- (169,279.15) ;
%Straight Lines [id:da45399705070135465] 
\draw [color={rgb, 255:red, 155; green, 155; blue, 155 }  ,draw opacity=1 ]   (321,164.15) -- (511.12,234.45) ;
\draw [shift={(513,235.15)}, rotate = 200.29] [fill={rgb, 255:red, 155; green, 155; blue, 155 }  ,fill opacity=1 ][line width=0.08]  [draw opacity=0] (12,-3) -- (0,0) -- (12,3) -- cycle    ;
%Straight Lines [id:da06579349148912339] 
\draw [color={rgb, 255:red, 80; green, 227; blue, 194 }  ,draw opacity=1 ]   (552,327.15) -- (582,310.15) ;
%Straight Lines [id:da7813158499201878] 
\draw [color={rgb, 255:red, 80; green, 227; blue, 194 }  ,draw opacity=1 ]   (615,327.15) -- (582,310.15) ;
%Straight Lines [id:da6554597016072419] 
\draw [color={rgb, 255:red, 80; green, 227; blue, 194 }  ,draw opacity=1 ]   (585,398.15) -- (615,381.15) ;
%Straight Lines [id:da38771820652223643] 
\draw    (648,398.15) -- (615,381.15) ;
%Straight Lines [id:da7029269298051468] 
\draw [color={rgb, 255:red, 74; green, 144; blue, 226 }  ,draw opacity=1 ]   (521,265.15) -- (528.67,260.8) -- (551,248.15) ;
%Straight Lines [id:da3078830979711833] 
\draw [color={rgb, 255:red, 74; green, 144; blue, 226 }  ,draw opacity=1 ]   (584,265.15) -- (551,248.15) ;
%Straight Lines [id:da5015548112989887] 
\draw [color={rgb, 255:red, 80; green, 227; blue, 194 }  ,draw opacity=1 ]   (171,336.15) -- (201,319.15) ;
%Straight Lines [id:da5954251542916287] 
\draw [color={rgb, 255:red, 80; green, 227; blue, 194 }  ,draw opacity=1 ]   (234,336.15) -- (201,319.15) ;
%Straight Lines [id:da4688878650352748] 
\draw [color={rgb, 255:red, 74; green, 144; blue, 226 }  ,draw opacity=1 ]   (108,269.15) -- (115.67,264.8) -- (138,252.15) ;
%Straight Lines [id:da2607503270176317] 
\draw [color={rgb, 255:red, 74; green, 144; blue, 226 }  ,draw opacity=1 ]   (171,269.15) -- (138,252.15) ;
%Straight Lines [id:da8702130416826466] 
\draw [color={rgb, 255:red, 155; green, 155; blue, 155 }  ,draw opacity=1 ]   (238,450.15) -- (505,450.15) ;
\draw [shift={(507,450.15)}, rotate = 180] [fill={rgb, 255:red, 155; green, 155; blue, 155 }  ,fill opacity=1 ][line width=0.08]  [draw opacity=0] (12,-3) -- (0,0) -- (12,3) -- cycle    ;
\draw [shift={(236,450.15)}, rotate = 0] [fill={rgb, 255:red, 155; green, 155; blue, 155 }  ,fill opacity=1 ][line width=0.08]  [draw opacity=0] (12,-3) -- (0,0) -- (12,3) -- cycle    ;

% Text Node
\draw (313.03,96.09) node [anchor=north west][inner sep=0.75pt]  [rotate=-30]  {$\cdots $};
% Text Node
\draw (408,126.15) node [anchor=north] [inner sep=0.75pt]   [align=left] {lexical head};
% Text Node
\draw (279.03,66.09) node [anchor=north west][inner sep=0.75pt]  [rotate=-30]  {$\cdots $};
% Text Node
\draw (42,67) node [anchor=north west][inner sep=0.75pt]   [align=left] {Fact: functional \\domains exist};
% Text Node
\draw (251.89,95.5) node [anchor=north west][inner sep=0.75pt]  [rotate=-300] [align=left] {{\LARGE \{}};
% Text Node
\draw (294.89,121.5) node [anchor=north west][inner sep=0.75pt]  [rotate=-300] [align=left] {{\LARGE \{}};
% Text Node
\draw (180,103) node [anchor=north west][inner sep=0.75pt]   [align=left] {outer domain};
% Text Node
\draw (257,136) node [anchor=north west][inner sep=0.75pt]   [align=left] {inner domain};
% Text Node
\draw (299,406.15) node [anchor=north] [inner sep=0.75pt]   [align=left] {lexical root};
% Text Node
\draw (20,305) node [anchor=north west][inner sep=0.75pt]  [color={rgb, 255:red, 74; green, 144; blue, 226 }  ,opacity=1 ] [align=left] {outer domain};
% Text Node
\draw (109,382) node [anchor=north west][inner sep=0.75pt]  [color={rgb, 255:red, 80; green, 227; blue, 194 }  ,opacity=1 ] [align=left] {inner domain};
% Text Node
\draw (139,299.15) node [anchor=north] [inner sep=0.75pt]  [color={rgb, 255:red, 74; green, 144; blue, 226 }  ,opacity=1 ] [align=left] {F1};
% Text Node
\draw (204,366.15) node [anchor=north] [inner sep=0.75pt]  [color={rgb, 255:red, 80; green, 227; blue, 194 }  ,opacity=1 ] [align=left] {F2};
% Text Node
\draw (125,442.15) node [anchor=north west][inner sep=0.75pt]   [align=left] {Minimalism};
% Text Node
\draw (648,401.15) node [anchor=north] [inner sep=0.75pt]   [align=left] {lexical head};
% Text Node
\draw (374,300) node [anchor=north west][inner sep=0.75pt]  [color={rgb, 255:red, 74; green, 144; blue, 226 }  ,opacity=1 ] [align=left] {outer domain};
% Text Node
\draw (468,392) node [anchor=north west][inner sep=0.75pt]  [color={rgb, 255:red, 80; green, 227; blue, 194 }  ,opacity=1 ] [align=left] {inner domain};
% Text Node
\draw (521,268.15) node [anchor=north] [inner sep=0.75pt]  [color={rgb, 255:red, 74; green, 144; blue, 226 }  ,opacity=1 ] [align=left] {\begin{minipage}[lt]{57.07pt}\setlength\topsep{0pt}
\begin{center}
function 1:\\dependent 1\\
\end{center}

\end{minipage}};
% Text Node
\draw (552,330.15) node [anchor=north] [inner sep=0.75pt]  [color={rgb, 255:red, 80; green, 227; blue, 194 }  ,opacity=1 ] [align=left] {\begin{minipage}[lt]{57.07pt}\setlength\topsep{0pt}
\begin{center}
function 2:\\dependent 2
\end{center}

\end{minipage}};
% Text Node
\draw (171,339.15) node [anchor=north] [inner sep=0.75pt]  [color={rgb, 255:red, 80; green, 227; blue, 194 }  ,opacity=1 ] [align=left] {dependent 2};
% Text Node
\draw (115.67,267.8) node [anchor=north] [inner sep=0.75pt]  [color={rgb, 255:red, 74; green, 144; blue, 226 }  ,opacity=1 ] [align=left] {dependent 1};
% Text Node
\draw (202,299.15) node [anchor=north] [inner sep=0.75pt]  [color={rgb, 255:red, 74; green, 144; blue, 226 }  ,opacity=1 ] [align=left] {$\displaystyle \cdots $};
% Text Node
\draw (267,366.15) node [anchor=north] [inner sep=0.75pt]  [color={rgb, 255:red, 80; green, 227; blue, 194 }  ,opacity=1 ] [align=left] {$\displaystyle \cdots $};
% Text Node
\draw (584,281.15) node [anchor=north] [inner sep=0.75pt]  [color={rgb, 255:red, 74; green, 144; blue, 226 }  ,opacity=1 ] [align=left] {$\displaystyle \cdots $};
% Text Node
\draw (616,345.15) node [anchor=north] [inner sep=0.75pt]  [color={rgb, 255:red, 80; green, 227; blue, 194 }  ,opacity=1 ] [align=left] {$\displaystyle \cdots $};
% Text Node
\draw (535,442) node [anchor=north west][inner sep=0.75pt]   [align=left] {CGEL};
% Text Node
\draw (362,452.15) node [anchor=north west][inner sep=0.75pt]  [color={rgb, 255:red, 155; green, 155; blue, 155 }  ,opacity=1 ] [align=left] {dual};


\end{tikzpicture}

    \caption{Duality between \ac{cgel} and Minimalism}
    \label{fig:cgel-minimalism}
\end{figure}

Not all dependency relations can be expressed purely by the context-free phrase structure grammar 
\citep{pullum2008expressive}. 
Movements (or cross-serial dependencies in dependency terms) in Minimalism 
are represented by a gap (\ac{cgel} \citesec{2.2} [5]),
or by indirect dependency (\ac{cgel} \citesec{5.14.1}).

The close relation between \ac{cgel} and generative syntax apparently deviates 
from the common descriptive practice, 
especially the common descriptive framework extracted and summarized 
by \citet{dixon2009basic1,dixon2010basic2,dixon2012basic3} and named as \ac{blt} by Dixon,
but this distinction is illusory:
while \ac{blt} assumes a flatter phrase structure,
information coded by the binary-branching phrase structure in \ac{cgel}
is coded by dependency arcs in \ac{blt} (\prettyref{fig:to-the-fat-man-blt}).

Though several disagreements with the mainstream generative syntax is raised in \ac{cgel},
and much more severe accusations are made in \ac{blt},
it can be seen all the three approaches are describing almost the same complexity class of grammars.

\begin{figure}
    \centering
    

\tikzset{every picture/.style={line width=0.3pt}} %set default line width to 0.75pt        

\begin{tikzpicture}[x=0.75pt,y=0.75pt,yscale=-0.8,xscale=0.8]
%uncomment if require: \path (0,462); %set diagram left start at 0, and has height of 462

%Straight Lines [id:da7470525027519479] 
\draw    (65,82.81) -- (123,65.81) ;
%Straight Lines [id:da284026246045451] 
\draw    (188,82.81) -- (123,65.81) ;
%Straight Lines [id:da6793973159868627] 
\draw    (130,148.81) -- (188,131.81) ;
%Straight Lines [id:da7618139010556202] 
\draw    (253,148.81) -- (188,131.81) ;
%Straight Lines [id:da8777852448147017] 
\draw    (195,212.81) -- (253,195.81) ;
%Straight Lines [id:da7372022724021017] 
\draw    (318,212.81) -- (253,195.81) ;
%Straight Lines [id:da9619077986855507] 
\draw    (65,318.81) -- (65,134.81) ;
%Straight Lines [id:da5838222630790659] 
\draw    (130,318.81) -- (130,194.81) ;
%Straight Lines [id:da5932119473868447] 
\draw    (195,318.81) -- (195,254.81) ;
%Straight Lines [id:da9627969740205233] 
\draw    (318,318.81) -- (318,254.81) ;
%Straight Lines [id:da17223380454923265] 
\draw    (450,227.81) -- (450,163.81) ;
%Straight Lines [id:da6952728185192718] 
\draw    (547,227.81) -- (547,163.81) ;
%Straight Lines [id:da9330113293183067] 
\draw    (638,227.81) -- (638,163.81) ;
%Straight Lines [id:da8988276865905278] 
\draw    (716,227.81) -- (716,163.81) ;
%Straight Lines [id:da891102641862588] 
\draw    (452,141.48) -- (585,64.81) ;
%Straight Lines [id:da13577169125065036] 
\draw    (550,141.48) -- (585,64.81) ;
%Straight Lines [id:da027055110670632487] 
\draw    (638,141.81) -- (585,64.81) ;
%Straight Lines [id:da0804405283159666] 
\draw    (716,141.81) -- (585,64.81) ;
%Curve Lines [id:da8849752474371109] 
\draw    (716,254.81) .. controls (684,313.81) and (654,289.81) .. (637,254.81) ;
%Curve Lines [id:da33081916759136454] 
\draw    (720,254.81) .. controls (663,407.81) and (575,321.48) .. (549,253.48) ;
%Curve Lines [id:da5450701123076458] 
\draw    (726,254.81) .. controls (669,407.81) and (531,442.15) .. (450,255.48) ;

% Text Node
\draw (123,62.81) node [anchor=south] [inner sep=0.75pt]   [align=left] {PP};
% Text Node
\draw (65,85.81) node [anchor=north] [inner sep=0.75pt]   [align=left] {\begin{minipage}[lt]{52.34pt}\setlength\topsep{0pt}
\begin{center}
head:\\preposition
\end{center}

\end{minipage}};
% Text Node
\draw (65,321.81) node [anchor=north] [inner sep=0.75pt]   [align=left] {to};
% Text Node
\draw (188,85.81) node [anchor=north] [inner sep=0.75pt]   [align=left] {\begin{minipage}[lt]{59.05pt}\setlength\topsep{0pt}
\begin{center}
complement:\\NP
\end{center}

\end{minipage}};
% Text Node
\draw (130,151.81) node [anchor=north] [inner sep=0.75pt]   [align=left] {\begin{minipage}[lt]{53.71pt}\setlength\topsep{0pt}
\begin{center}
determiner:\\article
\end{center}

\end{minipage}};
% Text Node
\draw (253,151.81) node [anchor=north] [inner sep=0.75pt]   [align=left] {\begin{minipage}[lt]{38.39pt}\setlength\topsep{0pt}
\begin{center}
head:\\nominal
\end{center}

\end{minipage}};
% Text Node
\draw (200,214.81) node [anchor=north] [inner sep=0.75pt]   [align=left] {\begin{minipage}[lt]{42.1pt}\setlength\topsep{0pt}
\begin{center}
modifier:\\adjective
\end{center}

\end{minipage}};
% Text Node
\draw (318,215.81) node [anchor=north] [inner sep=0.75pt]   [align=left] {\begin{minipage}[lt]{26.5pt}\setlength\topsep{0pt}
\begin{center}
head:\\noun
\end{center}

\end{minipage}};
% Text Node
\draw (130,321.81) node [anchor=north] [inner sep=0.75pt]   [align=left] {the};
% Text Node
\draw (195,321.81) node [anchor=north] [inner sep=0.75pt]   [align=left] {fat};
% Text Node
\draw (318,321.81) node [anchor=north] [inner sep=0.75pt]   [align=left] {man};
% Text Node
\draw (173,427) node [anchor=north west][inner sep=0.75pt]   [align=left] {(a)};
% Text Node
\draw (450,230.81) node [anchor=north] [inner sep=0.75pt]   [align=left] {to};
% Text Node
\draw (547,230.81) node [anchor=north] [inner sep=0.75pt]   [align=left] {the};
% Text Node
\draw (637,230.81) node [anchor=north] [inner sep=0.75pt]   [align=left] {fat};
% Text Node
\draw (716,230.81) node [anchor=north] [inner sep=0.75pt]   [align=left] {man};
% Text Node
\draw (450,152.31) node   [align=left] {preposition};
% Text Node
\draw (547,152.31) node   [align=left] {determiner};
% Text Node
\draw (638,152.31) node   [align=left] {adjective};
% Text Node
\draw (716,152.31) node   [align=left] {noun};
% Text Node
\draw (585,62.81) node [anchor=south] [inner sep=0.75pt]   [align=left] {PP (or oblique NP)};
% Text Node
\draw (600,294) node [anchor=north west][inner sep=0.75pt]   [align=left] {modified by};
% Text Node
\draw (605,341) node [anchor=north west][inner sep=0.75pt]   [align=left] {definite};
% Text Node
\draw (556,384) node [anchor=north west][inner sep=0.75pt]   [align=left] {dative};
% Text Node
\draw (585,427) node [anchor=north west][inner sep=0.75pt]   [align=left] {(b)};


\end{tikzpicture}

    \caption{Comparison between the \ac{cgel} and \ac{blt} analyses of \corpus{to the fat man}}
    \label{fig:to-the-fat-man-blt}
\end{figure}

\section{The clause structure}

This section is about the phrase structure of the clause.
I will discuss clause dependents, 
their forms and functions,
grammatical systems in the clause,
and how everything is put together.

\subsection{Types of complements}

Clausal (or verbal, since the clause is headed by the verb) complements 
may be NPs and PPs, and less frequently, adverbs 
(as in \corpus{He treated us [kindly]}). % TODO: adverbial clause: the name and classification 

This section lists some criteria of classification of complements.
They are all discussed in \ac{cgel} \citesec{4.1.1}.

\subsubsection{Core v.s. oblique}

One classification standard is the make up of the complement.
A \concept{core} complement is a complement with similar morphosyntactic properties of NP complements.
A \concept{non-core} complement is a complement with similar morphosyntactic properties of PP complements.
If a non-core complement itself takes an NP complement (or something with similar morphosyntactic properties),
the latter is called an \concept{oblique}.

Note that in \ac{cgel}, the term \term{argument} is reserved for purely semantic objects.
A clausal complement is therefore the syntactic incarnation of an argument,
but itself is not an argument.
This is \emph{not} the way \term{argument} is used in \ac{blt}.

It should also be noted that in \ac{cgel}, the terms \term{non-core} and \term{oblique} 
are reserved for clausal complements and the NP part of PP clausal complements.
They \emph{do not} include adjuncts with similar forms.
On the other hand, in \ac{blt}, the term \term{peripheral argument} covers both complements and adjuncts.

The prototypical definition of \emph{core} and \term{oblique} complements 
are based on syntactic forms instead of functions,
while the definition is extended by analog with respect to syntactic functions.
Whether these terms are useful is a question we need to wait and see.

\subsubsection{External and internal}



\section{The verb category}

% TODO: classification by complements

\bibliographystyle{plainnat}
\bibliography{cambridge}

\end{document}