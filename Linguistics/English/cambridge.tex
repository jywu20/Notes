\documentclass{article}

\usepackage{geometry}
\usepackage{titling}
\usepackage{titlesec}
\usepackage{paralist}
\usepackage{footnote}
\usepackage{enumerate}
\usepackage{amsmath, amssymb, amsthm}
\usepackage{gb4e}
\noautomath
\usepackage{bbm}
\usepackage{soul}
\usepackage{graphicx}
\usepackage{siunitx}
\usepackage[table,xcdraw]{xcolor}
\usepackage{tikz}
\usepackage[ruled, vlined, linesnumbered, noend]{algorithm2e}
\usepackage{xr-hyper}
\usepackage[colorlinks]{hyperref} % linkcolor=black, anchorcolor=black, citecolor=black, filecolor=black
\usepackage[most]{tcolorbox}
\usepackage{caption}
\usepackage{subcaption}
\usepackage{booktabs}
\usepackage{multirow}
\usepackage[figuresright]{rotating}
\usepackage{acro}
\usepackage[round]{natbib} 
\usepackage{prettyref}

\geometry{left=3.18cm,right=3.18cm,top=2.54cm,bottom=2.54cm}
\titlespacing{\paragraph}{0pt}{1pt}{10pt}[20pt]
\setlength{\droptitle}{-5em}

\DeclareMathOperator{\timeorder}{\mathcal{T}}
\DeclareMathOperator{\diag}{diag}
\DeclareMathOperator{\legpoly}{P}
\DeclareMathOperator{\primevalue}{P}
\DeclareMathOperator{\sgn}{sgn}
\newcommand*{\ii}{\mathrm{i}}
\newcommand*{\ee}{\mathrm{e}}
\newcommand*{\const}{\mathrm{const}}
\newcommand*{\suchthat}{\quad \text{s.t.} \quad}
\newcommand*{\argmin}{\arg\min}
\newcommand*{\argmax}{\arg\max}
\newcommand*{\normalorder}[1]{: #1 :}
\newcommand*{\pair}[1]{\langle #1 \rangle}
\newcommand*{\fd}[1]{\mathcal{D} #1}

\newcommand*{\citesec}[1]{\S~{#1}}
\newcommand*{\citechap}[1]{chap.~{#1}}
\newcommand*{\citefig}[1]{Fig.~{#1}}
\newcommand*{\citetable}[1]{Table~{#1}}
\newcommand*{\citefootnote}[1]{footnote~{#1}}

\newrefformat{sec}{\citesec{\ref{#1}}}
\newrefformat{fig}{\citefig{\ref{#1}}}
\newrefformat{tbl}{\citetable{\ref{#1}}}
\newrefformat{chap}{\citechap{\ref{#1}}}

\usetikzlibrary{arrows,shapes,positioning}
\usetikzlibrary{arrows.meta}
\usetikzlibrary{decorations.markings}
\tikzstyle arrowstyle=[scale=1]
\tikzstyle directed=[postaction={decorate,decoration={markings,
    mark=at position .5 with {\arrow[arrowstyle]{stealth}}}}]
\tikzstyle ray=[directed, thick]
\tikzstyle dot=[anchor=base,fill,circle,inner sep=1pt]


\tcbuselibrary{skins, breakable, theorems}

\newtcbtheorem[number within=chapter]{infobox}{Box}%
  {colback=blue!5,colframe=blue!65,fonttitle=\bfseries, breakable}{infobox}

\newcommand*{\concept}[1]{\textbf{#1}}
\newcommand*{\term}[1]{\emph{#1}}
\newcommand*{\corpus}[1]{\emph{#1}}

\newcommand*{\vP}{\textit{v}P}

\DeclareAcronym{blt}{short = BLT, long = Basic Linguistic Theory}
\DeclareAcronym{cgel}{short = CGEL, long = The Cambridge Grammar of the English Language}
\DeclareAcronym{dm}{short = DM, long = Distributed Morphology}
\DeclareAcronym{tag}{long = Tree-adjoining grammar, short = TAG}
\DeclareAcronym{sfp}{long = sentence final particle, short = SFP}
\DeclareAcronym{vp}{long = verb phrase, short = VP}
\DeclareAcronym{cls}{long = classifier, short = CLS}
\DeclareAcronym{dist}{long = distal, short = DIST}
\DeclareAcronym{prox}{long = proximate, short = PROX}
\DeclareAcronym{dem}{long = demonstrative, short = DEM}
\DeclareAcronym{dur}{long = durative, short = DUR}
\DeclareAcronym{neg}{long = negative, short = NEG}
\DeclareAcronym{tame}{long = {Tense, Aspect, Mood, Evidentiality}, short = TAME}

% Disable unsupported commands in bookmark titles 
\pdfstringdefDisableCommands{%
  \def\\{}%
  \def\texttt#1{<#1>}%
  \def\mathbb#1{#1}%
}
\pdfstringdefDisableCommands{\def\eqref#1{(\ref{#1})}}

\makeatletter
\pdfstringdefDisableCommands{\let\HyPsd@CatcodeWarning\@gobble}
\makeatother

\title{Reading notes of Cambridge Grammar of English Language}
\author{Jinyuan Wu}

\begin{document}

\maketitle

\automath

This note is a reconstruction of the contents \acl{cgel} (\citealt{cgel}, henceforth \acs{cgel}). 

\section{Theoretical preliminaries}\label{sec:theory}

For the discussion on the underlying theoretical framework, 
see the relevant chapter in \href{../Chinese/main.pdf}{this note}.
Briefly speaking, the framework is coarse-grained Minimalism:
invisible functional heads are erased,
but dependency relations created by the functional projections are kept (\prettyref{fig:coarse-grained}(a)),
and the corresponding \ac{cgel} tree is obtained by replacing the name of dependency relations 
by syntactic function-form pairs (\prettyref{fig:coarse-grained}(b)).
The head is defined as the dominant \emph{lexical} word.
This makes \ac{cgel} look like old-fashioned X-bar theory, 
where a lexical word projects into a phrase (\ac{cgel} \citesec{5.2} [5]),
but there can be several ``maximal projections'' headed by the same lexical word (\ac{cgel} \citesec{5.2 [11]}).
This is actually consistent with Minimalism (\prettyref{fig:cgel-minimalism}).

The standard of ``lexical'', however, varies from one author to another.
A PP and a CaseP (in generative terms) are similar objects,
but in the coarse-grained \ac{cgel} framework,
the latter is definitely an NP, 
while the status of the first is kind of controversial:
should we recognize the preposition as a lexical word,
or as a function word?
In the first analysis, a PP in generative terms is still a PP,
while in the second analysis, a PP in generative terms is an oblique NP.
Choosing between the two analyses is merely a problem of notation.
But it is good practice to choose a notation that hints the readers about 
certain properties of the construction in question.
For example, if a PP is analyzed as a PP, then probably the preposition has some predicative properties:
a PP is therefore more like a VP, not an NP with case marking 
(or ``CaseP'' in generative terms, especially in theories focusing on morphosyntax 
and do not make sharp contrast between syntax and morphology).
If, however, prepositions in a language work just like case markers,
the argumentation raised in \citet[\citesec{1.11}, \citesec{5.4}]{dixon2009basic1} is then attempting:
we should recognize the prepositions as case markers with standalone phonological realizations.
Which analysis to choose should be decided according to observed phenomena.
This is summarized by the slogan ``describing a language in its own terms''
-- but the slogan does not go into conflict with generativism.

\begin{figure}
    \centering
    

\tikzset{every picture/.style={line width=0.75pt}} %set default line width to 0.75pt        

\begin{tikzpicture}[x=0.75pt,y=0.75pt,yscale=-0.85,xscale=0.85]
%uncomment if require: \path (0,408); %set diagram left start at 0, and has height of 408

%Straight Lines [id:da2665478368972569] 
\draw [color={rgb, 255:red, 245; green, 166; blue, 35 }  ,draw opacity=1 ] [dash pattern={on 4.5pt off 4.5pt}]  (264,61) -- (544,61) ;
\draw [shift={(546,61)}, rotate = 180] [fill={rgb, 255:red, 245; green, 166; blue, 35 }  ,fill opacity=1 ][line width=0.08]  [draw opacity=0] (12,-3) -- (0,0) -- (12,3) -- cycle    ;
%Straight Lines [id:da6278442099042232] 
\draw [color={rgb, 255:red, 245; green, 166; blue, 35 }  ,draw opacity=1 ] [dash pattern={on 4.5pt off 4.5pt}]  (295,107) -- (622,107) ;
\draw [shift={(624,107)}, rotate = 180] [fill={rgb, 255:red, 245; green, 166; blue, 35 }  ,fill opacity=1 ][line width=0.08]  [draw opacity=0] (12,-3) -- (0,0) -- (12,3) -- cycle    ;
%Rounded Rect [id:dp9005470977487176] 
\draw  [draw opacity=0][fill={rgb, 255:red, 80; green, 227; blue, 194 }  ,fill opacity=0.2 ] (494,41.22) .. controls (494,32.48) and (501.08,25.4) .. (509.82,25.4) -- (741.18,25.4) .. controls (749.92,25.4) and (757,32.48) .. (757,41.22) -- (757,325.99) .. controls (757,334.73) and (749.92,341.81) .. (741.18,341.81) -- (509.82,341.81) .. controls (501.08,341.81) and (494,334.73) .. (494,325.99) -- cycle ;
%Straight Lines [id:da3555404365615693] 
\draw [color={rgb, 255:red, 155; green, 155; blue, 155 }  ,draw opacity=1 ]   (630.05,206.81) -- (630.05,301) ;
%Straight Lines [id:da5116688308518689] 
\draw [color={rgb, 255:red, 80; green, 227; blue, 194 }  ,draw opacity=1 ]   (703.05,158.83) -- (665.05,140.5) ;
%Straight Lines [id:da7532901881200198] 
\draw [color={rgb, 255:red, 80; green, 227; blue, 194 }  ,draw opacity=1 ]   (665.05,140.5) -- (630.05,158.83) ;
%Straight Lines [id:da3708728686228633] 
\draw [color={rgb, 255:red, 80; green, 227; blue, 194 }  ,draw opacity=1 ]   (658.65,97.63) -- (601.2,71.73) ;
%Straight Lines [id:da18650265857430104] 
\draw [color={rgb, 255:red, 80; green, 227; blue, 194 }  ,draw opacity=1 ]   (601.2,71.73) -- (548.65,97.63) ;
%Rounded Rect [id:dp9951745940140804] 
\draw  [draw opacity=0][fill={rgb, 255:red, 255; green, 255; blue, 255 }  ,fill opacity=1 ] (675,215.16) .. controls (675,210.29) and (678.95,206.33) .. (683.83,206.33) -- (722.17,206.33) .. controls (727.05,206.33) and (731,210.29) .. (731,215.16) -- (731,314.17) .. controls (731,319.05) and (727.05,323) .. (722.17,323) -- (683.83,323) .. controls (678.95,323) and (675,319.05) .. (675,314.17) -- cycle ;
%Rounded Rect [id:dp4970986462373306] 
\draw  [draw opacity=0][fill={rgb, 255:red, 255; green, 255; blue, 255 }  ,fill opacity=1 ] (521.24,153.76) .. controls (521.24,148.82) and (525.25,144.81) .. (530.19,144.81) -- (568.05,144.81) .. controls (572.99,144.81) and (577,148.82) .. (577,153.76) -- (577,314.05) .. controls (577,318.99) and (572.99,323) .. (568.05,323) -- (530.19,323) .. controls (525.25,323) and (521.24,318.99) .. (521.24,314.05) -- cycle ;
%Rounded Rect [id:dp4071663798075751] 
\draw  [draw opacity=0][fill={rgb, 255:red, 80; green, 227; blue, 194 }  ,fill opacity=0.2 ] (106,41.89) .. controls (106,33.39) and (112.89,26.5) .. (121.4,26.5) -- (346.6,26.5) .. controls (355.11,26.5) and (362,33.39) .. (362,41.89) -- (362,324.41) .. controls (362,332.92) and (355.11,339.81) .. (346.6,339.81) -- (121.4,339.81) .. controls (112.89,339.81) and (106,332.92) .. (106,324.41) -- cycle ;
%Straight Lines [id:da5088138181155522] 
\draw [color={rgb, 255:red, 155; green, 155; blue, 155 }  ,draw opacity=1 ]   (217.05,133.93) -- (217.05,299.1) ;
%Straight Lines [id:da998446713157706] 
\draw [color={rgb, 255:red, 80; green, 227; blue, 194 }  ,draw opacity=1 ]   (290.05,133.93) -- (252.05,115.6) ;
%Straight Lines [id:da5752866187220778] 
\draw [color={rgb, 255:red, 80; green, 227; blue, 194 }  ,draw opacity=1 ]   (252.05,115.6) -- (217.05,133.93) ;
%Straight Lines [id:da6021628337779128] 
\draw [color={rgb, 255:red, 80; green, 227; blue, 194 }  ,draw opacity=1 ]   (249.65,94.73) -- (200.2,68.83) ;
%Straight Lines [id:da8361003247119008] 
\draw [color={rgb, 255:red, 80; green, 227; blue, 194 }  ,draw opacity=1 ]   (200.2,68.83) -- (153.65,94.73) ;
%Rounded Rect [id:dp3858870486231194] 
\draw  [draw opacity=0][fill={rgb, 255:red, 255; green, 255; blue, 255 }  ,fill opacity=1 ] (257.24,151.34) .. controls (257.24,145.52) and (261.96,140.81) .. (267.77,140.81) -- (313.48,140.81) .. controls (319.29,140.81) and (324,145.52) .. (324,151.34) -- (324,309.05) .. controls (324,314.86) and (319.29,319.57) .. (313.48,319.57) -- (267.77,319.57) .. controls (261.96,319.57) and (257.24,314.86) .. (257.24,309.05) -- cycle ;
%Rounded Rect [id:dp2951833006567197] 
\draw  [draw opacity=0][fill={rgb, 255:red, 255; green, 255; blue, 255 }  ,fill opacity=1 ] (134.24,118.4) .. controls (134.24,113.11) and (138.54,108.81) .. (143.84,108.81) -- (184.41,108.81) .. controls (189.71,108.81) and (194,113.11) .. (194,118.4) -- (194,305.98) .. controls (194,311.28) and (189.71,315.57) .. (184.41,315.57) -- (143.84,315.57) .. controls (138.54,315.57) and (134.24,311.28) .. (134.24,305.98) -- cycle ;

% Text Node
\draw (630.76,321) node [anchor=south] [inner sep=0.75pt]   [align=left] {hurt};
% Text Node
\draw (665.05,137.5) node [anchor=south] [inner sep=0.75pt]   [align=left] {\begin{minipage}[lt]{89.53pt}\setlength\topsep{0pt}
\begin{center}
predicate:\\smaller verb phrase
\end{center}

\end{minipage}};
% Text Node
\draw (601.2,68.73) node [anchor=south] [inner sep=0.75pt]   [align=left] {verb phrase};
% Text Node
\draw (548.65,100.63) node [anchor=north] [inner sep=0.75pt]   [align=left] {\begin{minipage}[lt]{29.61pt}\setlength\topsep{0pt}
\begin{center}
agent:\\DP
\end{center}

\end{minipage}};
% Text Node
\draw (630.05,161.83) node [anchor=north] [inner sep=0.75pt]   [align=left] {\begin{minipage}[lt]{50.89pt}\setlength\topsep{0pt}
\begin{center}
predicator:\\V
\end{center}

\end{minipage}};
% Text Node
\draw (703.05,161.83) node [anchor=north] [inner sep=0.75pt]   [align=left] {\begin{minipage}[lt]{36.95pt}\setlength\topsep{0pt}
\begin{center}
patient:\\DP
\end{center}

\end{minipage}};
% Text Node
\draw (216.76,321.1) node [anchor=south] [inner sep=0.75pt]   [align=left] {hurt};
% Text Node
\draw (249.65,97.73) node [anchor=north] [inner sep=0.75pt]   [align=left] {Trans - DP};
% Text Node
\draw (179.8,50.81) node [anchor=north west][inner sep=0.75pt]   [align=left] {$\displaystyle v$ - DP};
% Text Node
\draw (211,383) node [anchor=north west][inner sep=0.75pt]   [align=left] {(a)};
% Text Node
\draw (636,383) node [anchor=north west][inner sep=0.75pt]   [align=left] {(b)};


\end{tikzpicture}

    \caption{From coarse-grained Minimalist derivational tree to \ac{cgel} tree}
    \label{fig:coarse-grained}
\end{figure}

\begin{figure}
    \centering
    

\tikzset{every picture/.style={line width=0.3pt}} %set default line width to 0.75pt        

\begin{tikzpicture}[x=0.75pt,y=0.75pt,yscale=-0.8,xscale=0.8]
%uncomment if require: \path (0,523); %set diagram left start at 0, and has height of 523

%Straight Lines [id:da6705457534949089] 
\draw    (313,92.15) -- (343,75.15) ;
%Straight Lines [id:da28013820417559354] 
\draw    (376,92.15) -- (343,75.15) ;
%Straight Lines [id:da26185532929398736] 
\draw    (345,123.15) -- (375,106.15) ;
%Straight Lines [id:da1670225517468129] 
\draw    (408,123.15) -- (375,106.15) ;
%Straight Lines [id:da7951370858439135] 
\draw    (279,62.15) -- (309,45.15) ;
%Straight Lines [id:da5307564248915579] 
\draw    (342,62.15) -- (309,45.15) ;
%Straight Lines [id:da6264444449679678] 
\draw [color={rgb, 255:red, 155; green, 155; blue, 155 }  ,draw opacity=1 ]   (321,164.15) -- (152.85,234.38) ;
\draw [shift={(151,235.15)}, rotate = 337.33] [fill={rgb, 255:red, 155; green, 155; blue, 155 }  ,fill opacity=1 ][line width=0.08]  [draw opacity=0] (12,-3) -- (0,0) -- (12,3) -- cycle    ;
%Straight Lines [id:da99450357998767] 
\draw [color={rgb, 255:red, 80; green, 227; blue, 194 }  ,draw opacity=1 ]   (204,363.15) -- (234,346.15) ;
%Straight Lines [id:da7516847541397156] 
\draw [color={rgb, 255:red, 80; green, 227; blue, 194 }  ,draw opacity=1 ]   (267,363.15) -- (234,346.15) ;
%Straight Lines [id:da7086721116542554] 
\draw [color={rgb, 255:red, 80; green, 227; blue, 194 }  ,draw opacity=1 ]   (236,403.15) -- (266,386.15) ;
%Straight Lines [id:da23963037522787767] 
\draw    (299,403.15) -- (266,386.15) ;
%Straight Lines [id:da8629804052760564] 
\draw [color={rgb, 255:red, 74; green, 144; blue, 226 }  ,draw opacity=1 ]   (139,296.15) -- (146.67,291.8) -- (169,279.15) ;
%Straight Lines [id:da21212301774701436] 
\draw [color={rgb, 255:red, 74; green, 144; blue, 226 }  ,draw opacity=1 ]   (202,296.15) -- (169,279.15) ;
%Straight Lines [id:da45399705070135465] 
\draw [color={rgb, 255:red, 155; green, 155; blue, 155 }  ,draw opacity=1 ]   (321,164.15) -- (511.12,234.45) ;
\draw [shift={(513,235.15)}, rotate = 200.29] [fill={rgb, 255:red, 155; green, 155; blue, 155 }  ,fill opacity=1 ][line width=0.08]  [draw opacity=0] (12,-3) -- (0,0) -- (12,3) -- cycle    ;
%Straight Lines [id:da06579349148912339] 
\draw [color={rgb, 255:red, 80; green, 227; blue, 194 }  ,draw opacity=1 ]   (552,327.15) -- (582,310.15) ;
%Straight Lines [id:da7813158499201878] 
\draw [color={rgb, 255:red, 80; green, 227; blue, 194 }  ,draw opacity=1 ]   (615,327.15) -- (582,310.15) ;
%Straight Lines [id:da6554597016072419] 
\draw [color={rgb, 255:red, 80; green, 227; blue, 194 }  ,draw opacity=1 ]   (585,398.15) -- (615,381.15) ;
%Straight Lines [id:da38771820652223643] 
\draw    (648,398.15) -- (615,381.15) ;
%Straight Lines [id:da7029269298051468] 
\draw [color={rgb, 255:red, 74; green, 144; blue, 226 }  ,draw opacity=1 ]   (521,265.15) -- (528.67,260.8) -- (551,248.15) ;
%Straight Lines [id:da3078830979711833] 
\draw [color={rgb, 255:red, 74; green, 144; blue, 226 }  ,draw opacity=1 ]   (584,265.15) -- (551,248.15) ;
%Straight Lines [id:da5015548112989887] 
\draw [color={rgb, 255:red, 80; green, 227; blue, 194 }  ,draw opacity=1 ]   (171,336.15) -- (201,319.15) ;
%Straight Lines [id:da5954251542916287] 
\draw [color={rgb, 255:red, 80; green, 227; blue, 194 }  ,draw opacity=1 ]   (234,336.15) -- (201,319.15) ;
%Straight Lines [id:da4688878650352748] 
\draw [color={rgb, 255:red, 74; green, 144; blue, 226 }  ,draw opacity=1 ]   (108,269.15) -- (115.67,264.8) -- (138,252.15) ;
%Straight Lines [id:da2607503270176317] 
\draw [color={rgb, 255:red, 74; green, 144; blue, 226 }  ,draw opacity=1 ]   (171,269.15) -- (138,252.15) ;
%Straight Lines [id:da8702130416826466] 
\draw [color={rgb, 255:red, 155; green, 155; blue, 155 }  ,draw opacity=1 ]   (238,450.15) -- (505,450.15) ;
\draw [shift={(507,450.15)}, rotate = 180] [fill={rgb, 255:red, 155; green, 155; blue, 155 }  ,fill opacity=1 ][line width=0.08]  [draw opacity=0] (12,-3) -- (0,0) -- (12,3) -- cycle    ;
\draw [shift={(236,450.15)}, rotate = 0] [fill={rgb, 255:red, 155; green, 155; blue, 155 }  ,fill opacity=1 ][line width=0.08]  [draw opacity=0] (12,-3) -- (0,0) -- (12,3) -- cycle    ;

% Text Node
\draw (313.03,96.09) node [anchor=north west][inner sep=0.75pt]  [rotate=-30]  {$\cdots $};
% Text Node
\draw (408,126.15) node [anchor=north] [inner sep=0.75pt]   [align=left] {lexical head};
% Text Node
\draw (279.03,66.09) node [anchor=north west][inner sep=0.75pt]  [rotate=-30]  {$\cdots $};
% Text Node
\draw (42,67) node [anchor=north west][inner sep=0.75pt]   [align=left] {Fact: functional \\domains exist};
% Text Node
\draw (251.89,95.5) node [anchor=north west][inner sep=0.75pt]  [rotate=-300] [align=left] {{\LARGE \{}};
% Text Node
\draw (294.89,121.5) node [anchor=north west][inner sep=0.75pt]  [rotate=-300] [align=left] {{\LARGE \{}};
% Text Node
\draw (180,103) node [anchor=north west][inner sep=0.75pt]   [align=left] {outer domain};
% Text Node
\draw (257,136) node [anchor=north west][inner sep=0.75pt]   [align=left] {inner domain};
% Text Node
\draw (299,406.15) node [anchor=north] [inner sep=0.75pt]   [align=left] {lexical root};
% Text Node
\draw (20,305) node [anchor=north west][inner sep=0.75pt]  [color={rgb, 255:red, 74; green, 144; blue, 226 }  ,opacity=1 ] [align=left] {outer domain};
% Text Node
\draw (109,382) node [anchor=north west][inner sep=0.75pt]  [color={rgb, 255:red, 80; green, 227; blue, 194 }  ,opacity=1 ] [align=left] {inner domain};
% Text Node
\draw (139,299.15) node [anchor=north] [inner sep=0.75pt]  [color={rgb, 255:red, 74; green, 144; blue, 226 }  ,opacity=1 ] [align=left] {F1};
% Text Node
\draw (204,366.15) node [anchor=north] [inner sep=0.75pt]  [color={rgb, 255:red, 80; green, 227; blue, 194 }  ,opacity=1 ] [align=left] {F2};
% Text Node
\draw (125,442.15) node [anchor=north west][inner sep=0.75pt]   [align=left] {Minimalism};
% Text Node
\draw (648,401.15) node [anchor=north] [inner sep=0.75pt]   [align=left] {lexical head};
% Text Node
\draw (374,300) node [anchor=north west][inner sep=0.75pt]  [color={rgb, 255:red, 74; green, 144; blue, 226 }  ,opacity=1 ] [align=left] {outer domain};
% Text Node
\draw (468,392) node [anchor=north west][inner sep=0.75pt]  [color={rgb, 255:red, 80; green, 227; blue, 194 }  ,opacity=1 ] [align=left] {inner domain};
% Text Node
\draw (521,268.15) node [anchor=north] [inner sep=0.75pt]  [color={rgb, 255:red, 74; green, 144; blue, 226 }  ,opacity=1 ] [align=left] {\begin{minipage}[lt]{57.07pt}\setlength\topsep{0pt}
\begin{center}
function 1:\\dependent 1\\
\end{center}

\end{minipage}};
% Text Node
\draw (552,330.15) node [anchor=north] [inner sep=0.75pt]  [color={rgb, 255:red, 80; green, 227; blue, 194 }  ,opacity=1 ] [align=left] {\begin{minipage}[lt]{57.07pt}\setlength\topsep{0pt}
\begin{center}
function 2:\\dependent 2
\end{center}

\end{minipage}};
% Text Node
\draw (171,339.15) node [anchor=north] [inner sep=0.75pt]  [color={rgb, 255:red, 80; green, 227; blue, 194 }  ,opacity=1 ] [align=left] {dependent 2};
% Text Node
\draw (115.67,267.8) node [anchor=north] [inner sep=0.75pt]  [color={rgb, 255:red, 74; green, 144; blue, 226 }  ,opacity=1 ] [align=left] {dependent 1};
% Text Node
\draw (202,299.15) node [anchor=north] [inner sep=0.75pt]  [color={rgb, 255:red, 74; green, 144; blue, 226 }  ,opacity=1 ] [align=left] {$\displaystyle \cdots $};
% Text Node
\draw (267,366.15) node [anchor=north] [inner sep=0.75pt]  [color={rgb, 255:red, 80; green, 227; blue, 194 }  ,opacity=1 ] [align=left] {$\displaystyle \cdots $};
% Text Node
\draw (584,281.15) node [anchor=north] [inner sep=0.75pt]  [color={rgb, 255:red, 74; green, 144; blue, 226 }  ,opacity=1 ] [align=left] {$\displaystyle \cdots $};
% Text Node
\draw (616,345.15) node [anchor=north] [inner sep=0.75pt]  [color={rgb, 255:red, 80; green, 227; blue, 194 }  ,opacity=1 ] [align=left] {$\displaystyle \cdots $};
% Text Node
\draw (535,442) node [anchor=north west][inner sep=0.75pt]   [align=left] {CGEL};
% Text Node
\draw (362,452.15) node [anchor=north west][inner sep=0.75pt]  [color={rgb, 255:red, 155; green, 155; blue, 155 }  ,opacity=1 ] [align=left] {dual};


\end{tikzpicture}

    \caption{Duality between \ac{cgel} and Minimalism}
    \label{fig:cgel-minimalism}
\end{figure}

Not all dependency relations can be expressed purely 
by our familiar CFG formalization of American Structuralism,
and movements are hence required.
(But note that at least in English,  
even after considering indirect dependencies, gaps, etc,
the expressive power is still context-free \citep{pullum2008expressive}.)
Movements (cross-serial dependencies in dependency terms) in Minimalism 
are represented by a gap (\ac{cgel} \citesec{2.2} [5]),
or by indirect dependency (\ac{cgel} \citesec{5.14.1}),
as \ac{cgel} chooses to shun the notion of movement altogether.

The close relation between \ac{cgel} and generative syntax apparently deviates 
from the common descriptive practice, 
especially the common descriptive framework extracted and summarized 
by \citet{dixon2009basic1,dixon2010basic2,dixon2012basic3} and named \ac{blt} by Dixon,
but this distinction is illusory:
while \ac{blt} assumes a flatter phrase structure,
information coded by the binary-branching phrase structure in \ac{cgel}
is coded by dependency arcs in \ac{blt} (\prettyref{fig:to-the-fat-man-blt}),
and the hierarchy ordering of the dependency relations 
are coded by notions like ``pipeline ordering'' and ``scope'',
e.g. ``a prototypical passive construction takes in an AVO argument structure 
and turns the deep A into the surface S, the deep O into the surface E''
(which in generative terms means VoiceP is higher than \vP),
and ``the scope of \corpus{the} in \prettyref{fig:to-the-fat-man-blt} is \corpus{fat man},
indicating that the latter is definite,
and the scope of \corpus{to} is \corpus{the fat man}''.
Human languages tend to use constituent order as a hint of scopes, 
and this tendency guarantees that
when \ac{blt} posits a flat structure but each function word in it has a scope,
the ``flat structure'' can then be turned into \ac{cgel}'s largely binary branching tree
according to the scopes,
and when scope factors are not important,
a flat \ac{blt} tree can by trivially turned into a largely binary branching \ac{cgel} tree.

Though several disagreements with the mainstream generative syntax is raised in \ac{cgel},
and much more severe accusations are made in \ac{blt},
it can be seen all the three approaches are describing almost the same complexity class of grammars.%
\footnote{
    Well, strictly speaking, they may not,
    because when formalized, a grammatical theory often allows some strikingly weird productions,
    which arise from corner cases ignored in the formalization.
    But these corner cases can be, in principle, rules out by watching 
    how linguists use the formalism in everyday work:
    some derivational devices are never used, 
    weird interplay of features that trigger unexpected movements is never considered,
    some phrase structure rules in \ac{blt} are possible but never appear in actual reference grammars, etc.
    In a word, the complexity class of grammars supposed in a grammatical theory 
    is not just determined by the explicitly articulated formalism.
    What is fed into the formalism is equally important.
}

It should be noted that even though \ac{cgel} comes close with the lexical decomposition-style Minimalism,
sometimes binary branching is still not available in the surface-oriented analysis,
and some words about the underlying dependency relations beside the tree is needed,
just as in the case of \ac{blt}.
An example is ditransitive verbs. 
Even though in generative analysis we have plenty of reasons to assume a \vP{} structure like 
[somebody [something SHOW]] in the clause \corpus{I showed him the photo}
(possible reasons including binding effects, passivization preference, etc.),
the head movement (or its post-syntactic version) breaks the binary branching structure 
in the surface-oriented analysis of the clause,
and hence in \ac{cgel} we get a (quite infrequent) ternary tree (\ac{cgel} \citesec{12.3.1} [4]). 
Another example is supplementation (\prettyref{sec:supplmentation}).
The difference between \ac{cgel} and \ac{blt} is therefore more about quantity instead of quality.

\begin{figure}
    \centering
    

\tikzset{every picture/.style={line width=0.3pt}} %set default line width to 0.75pt        

\begin{tikzpicture}[x=0.75pt,y=0.75pt,yscale=-0.8,xscale=0.8]
%uncomment if require: \path (0,462); %set diagram left start at 0, and has height of 462

%Straight Lines [id:da7470525027519479] 
\draw    (65,82.81) -- (123,65.81) ;
%Straight Lines [id:da284026246045451] 
\draw    (188,82.81) -- (123,65.81) ;
%Straight Lines [id:da6793973159868627] 
\draw    (130,148.81) -- (188,131.81) ;
%Straight Lines [id:da7618139010556202] 
\draw    (253,148.81) -- (188,131.81) ;
%Straight Lines [id:da8777852448147017] 
\draw    (195,212.81) -- (253,195.81) ;
%Straight Lines [id:da7372022724021017] 
\draw    (318,212.81) -- (253,195.81) ;
%Straight Lines [id:da9619077986855507] 
\draw    (65,318.81) -- (65,134.81) ;
%Straight Lines [id:da5838222630790659] 
\draw    (130,318.81) -- (130,194.81) ;
%Straight Lines [id:da5932119473868447] 
\draw    (195,318.81) -- (195,254.81) ;
%Straight Lines [id:da9627969740205233] 
\draw    (318,318.81) -- (318,254.81) ;
%Straight Lines [id:da17223380454923265] 
\draw    (450,227.81) -- (450,163.81) ;
%Straight Lines [id:da6952728185192718] 
\draw    (547,227.81) -- (547,163.81) ;
%Straight Lines [id:da9330113293183067] 
\draw    (638,227.81) -- (638,163.81) ;
%Straight Lines [id:da8988276865905278] 
\draw    (716,227.81) -- (716,163.81) ;
%Straight Lines [id:da891102641862588] 
\draw    (452,141.48) -- (585,64.81) ;
%Straight Lines [id:da13577169125065036] 
\draw    (550,141.48) -- (585,64.81) ;
%Straight Lines [id:da027055110670632487] 
\draw    (638,141.81) -- (585,64.81) ;
%Straight Lines [id:da0804405283159666] 
\draw    (716,141.81) -- (585,64.81) ;
%Curve Lines [id:da8849752474371109] 
\draw    (716,254.81) .. controls (684,313.81) and (654,289.81) .. (637,254.81) ;
%Curve Lines [id:da33081916759136454] 
\draw    (720,254.81) .. controls (663,407.81) and (575,321.48) .. (549,253.48) ;
%Curve Lines [id:da5450701123076458] 
\draw    (726,254.81) .. controls (669,407.81) and (531,442.15) .. (450,255.48) ;

% Text Node
\draw (123,62.81) node [anchor=south] [inner sep=0.75pt]   [align=left] {PP};
% Text Node
\draw (65,85.81) node [anchor=north] [inner sep=0.75pt]   [align=left] {\begin{minipage}[lt]{52.34pt}\setlength\topsep{0pt}
\begin{center}
head:\\preposition
\end{center}

\end{minipage}};
% Text Node
\draw (65,321.81) node [anchor=north] [inner sep=0.75pt]   [align=left] {to};
% Text Node
\draw (188,85.81) node [anchor=north] [inner sep=0.75pt]   [align=left] {\begin{minipage}[lt]{59.05pt}\setlength\topsep{0pt}
\begin{center}
complement:\\NP
\end{center}

\end{minipage}};
% Text Node
\draw (130,151.81) node [anchor=north] [inner sep=0.75pt]   [align=left] {\begin{minipage}[lt]{53.71pt}\setlength\topsep{0pt}
\begin{center}
determiner:\\article
\end{center}

\end{minipage}};
% Text Node
\draw (253,151.81) node [anchor=north] [inner sep=0.75pt]   [align=left] {\begin{minipage}[lt]{38.39pt}\setlength\topsep{0pt}
\begin{center}
head:\\nominal
\end{center}

\end{minipage}};
% Text Node
\draw (200,214.81) node [anchor=north] [inner sep=0.75pt]   [align=left] {\begin{minipage}[lt]{42.1pt}\setlength\topsep{0pt}
\begin{center}
modifier:\\adjective
\end{center}

\end{minipage}};
% Text Node
\draw (318,215.81) node [anchor=north] [inner sep=0.75pt]   [align=left] {\begin{minipage}[lt]{26.5pt}\setlength\topsep{0pt}
\begin{center}
head:\\noun
\end{center}

\end{minipage}};
% Text Node
\draw (130,321.81) node [anchor=north] [inner sep=0.75pt]   [align=left] {the};
% Text Node
\draw (195,321.81) node [anchor=north] [inner sep=0.75pt]   [align=left] {fat};
% Text Node
\draw (318,321.81) node [anchor=north] [inner sep=0.75pt]   [align=left] {man};
% Text Node
\draw (173,427) node [anchor=north west][inner sep=0.75pt]   [align=left] {(a)};
% Text Node
\draw (450,230.81) node [anchor=north] [inner sep=0.75pt]   [align=left] {to};
% Text Node
\draw (547,230.81) node [anchor=north] [inner sep=0.75pt]   [align=left] {the};
% Text Node
\draw (637,230.81) node [anchor=north] [inner sep=0.75pt]   [align=left] {fat};
% Text Node
\draw (716,230.81) node [anchor=north] [inner sep=0.75pt]   [align=left] {man};
% Text Node
\draw (450,152.31) node   [align=left] {preposition};
% Text Node
\draw (547,152.31) node   [align=left] {determiner};
% Text Node
\draw (638,152.31) node   [align=left] {adjective};
% Text Node
\draw (716,152.31) node   [align=left] {noun};
% Text Node
\draw (585,62.81) node [anchor=south] [inner sep=0.75pt]   [align=left] {PP (or oblique NP)};
% Text Node
\draw (600,294) node [anchor=north west][inner sep=0.75pt]   [align=left] {modified by};
% Text Node
\draw (605,341) node [anchor=north west][inner sep=0.75pt]   [align=left] {definite};
% Text Node
\draw (556,384) node [anchor=north west][inner sep=0.75pt]   [align=left] {dative};
% Text Node
\draw (585,427) node [anchor=north west][inner sep=0.75pt]   [align=left] {(b)};


\end{tikzpicture}

    \caption{Comparison between the \ac{cgel} and \ac{blt} analyses of \corpus{to the fat man}}
    \label{fig:to-the-fat-man-blt}
\end{figure}

\section{The organization of \ac{cgel}}

\ac{cgel} differs from another \ac{cgel}, i.e. the one by Quirk et al., in several aspects \citep{leech2004}.

\subsection{Chapters}

Authors have various strategies to carry out their grammars.
The organization of \ac{cgel} may appear confusing for a beginner,
but this is illusory:
\ac{cgel} is pretty clearly organized as long as 

This note is based on a different approach:
to describe everything in terms of dependency relations.
This makes it much easier to check the syntactic environment of a construction 
and how constructions are put together to form larger ones.
But it is quite painstaking and cumbersome to carry out a grammar in this way.
For example, traditional grammar discusses 
the types of clausal complements in sections about cases.
This makes it harder to check, say, whether there is a semi-object position for certain clauses.
The approach of this note does not have this problem, 
because in the section about complement types (\prettyref{sec:nucleus-dependents})
there will be an entry for the semi-object if any.
However, this also means all relevant syntactic phenomena involved 
have to be mentioned when a dependency relation is introduced,
for example, passivization properties should be mentioned for the direct object,
and the difference between the direct objects in intransitive clauses and transitive clauses 
have to be discussed in the section about objects,
not verb complementation pattern, where they most naturally occur.
In practice, therefore, no one takes the approach adopted in this note.
Dependency-based description can be seen as a consequence of surface-oriented equivalence of Minimalism,
but so is the \ac{tag}-like approach,
which, once the following facts are taken into account, is just the usually way grammars are written,
i.e. from words to phrase to clause:
when discussing words and morphology, a grammar usually covers corresponding grammatical relations,
which, essentially, creates treelets headed by the word in question,%
\footnote{
    In some grammars, there is a separate chapter named ``morphology''
    that covers all morphological devices attested
    regardless of the lexical category.
    Here I am talking about discussions on morphology of a specific lexical category 
    and not this morphology-device-enumeration chapter.
}
and the following discussions about phrases and clauses are just about how trees are built from the treelets.

\subsection{Terminology}

\begin{itemize}
    \item \term{Clause type}: this term specifically denotes Force in generative terms.
    The term \term{force} is avoided, 
    because it is too pragmatic,
    and may cause confusion between a pragmatic system and its syntactic coding.
    \item \term{Complement}: an \term{argument} in \ac{blt} terms. 
    Some authors use the term complement as an abbreviation of \term{complement clause},
    and this is not how this term is used in \ac{cgel}.
    \item \term{Complementation}: what complements (or arguments) a verb take, 
    not how to make a clause into a complement clause.
\end{itemize}

\section{Overview of clause structure}

\subsection{The pipeline of clause building}\label{sec:pipeline}

This section and several following sections are about the phrase structure of the clause.
I will discuss clause dependents, 
their forms and functions,
grammatical systems in the clause,
and how everything is put together.

Clausal grammars of nominative-accusative languages can fit more or less into the following paradigm:
\begin{enumerate}
    \item Clausal complements are fed into the argument structure%
    \footnote{
        This is not the term used in \ac{cgel}.
        For discussion on \term{argument} and \term{complmenet}, see \prettyref{sec:core-oblique}.
    } or in other words \vP. 
    They may be NPs, adverbials or subordinated clauses.
    \item Then several clausal grammatical systems are employed on the verb-complement complex or ``small clause'',
    including agreement (or argument indexation more generally), case marking%
    \footnote{
        Technically, case marking involves the argument structure.
        The nominative-accusative alignment, for example, 
        can be summarized as ``the patient-like argument of a transitive verb receives the accusative case,
        while anything promoted to the subject position receives the nominative case''.
        In more generative terms, we have 
        ``the transitive \vP{} (or some functional projection lower than Spec\vP) assigns the accusative case,
        while T assigns the nominative case''. 
        The first half of the generalization
        means the assignment of the accusative case is directly related to the argument structure.
        But we can change the wording of the above generalization easily.
        For example, an equivalent formulation may be 
        ``in the TP, any DP with a lower position than some DP else receives the accusative case, 
        while the subject receives the nominative case''.
        Yet we have a third expression: 
        ``in the TP, DPs receive the accusative case as the default case,
        while the subject receives the nominative case''.
        It is therefore acceptable to take case marking completely away from the argument structure.
    }
    and non-spatial settings (or \ac{tame} as people call them).
    The syntactic prominence of the subject is also introduced in this stage (commonly known as TP), 
    but without adjuncts it is hard to see the structural prominence of the subject,
    and without question formation, etc, it is hard to see the consequences of the prominence.
    The result of these steps, after including obligatory speech act marking,
    is a minimal canonical clause.
    \item Several adjunctions can be made to the minimal clause, 
    conveying information like time, position, manner, etc. 
    This step may be seen as an additional one,
    or it may be considered as occurring together with \ac{tame}.
    In the latter analysis, we need an explicit subject promoting analysis,
    which makes the subject in a higher position than any other complement.
    The result of these steps, after including obligatory speech act marking, 
    is a canonical clause.
    \item Speech act information 
    (or ``force'' in generative terms, or ``mood'' in \ac{blt}) 
    -- declarative, interrogative, imperative -- are added to the result of 
    \ac{tame} marked and possibly adjoined small clauses. 
    These stages occur in the CP domain.

    After these steps, the clause is now fully assembled. 
    If the speech act information marked meets the standards set in the language
    (e.g. the clause is finite, there is no subordination markers like a complementizer, etc.), 
    the clause is a qualified \emph{sentence}, 
    i.e. it can occur independently in utterance.
    But it can be embedded (or subordinated) into other clauses, too, probably with some additional marking.
    A clause without any subordinated clause as a complement or adjunct is called a \concept{simple clause}.
    Otherwise it is a \concept{complex clause}.

    The declarative marking is often (but not always) zero, 
    and in this way there is no obligatory speech act marking for clauses. 

    \item The canonical clause may further undergo processes like topicalization or focusing.
    This makes it non-canonical.

    Topicalization and focusing may be analyzed as happening together 
    with speech act marking and/or subordination.
    In the classical generative analysis, the CP domain is roughly Force-Focus-Topic-Finiteness,
    with the left feature being introduced after the right feature.
    Subordination involves both 
    finiteness (gerund clauses can be subordinated, but they are never independent sentences) 
    and force (subordinated clauses often have limited force choices, 
    and the complementizer has to be introduced at a certain position).
    So in this analysis, there is no strict time ordering between topicalization etc. and subordination.

    But it is also possible to introduce topicalization and focusing with transformational rules,
    since most grammars are much more surface-oriented.
    In this case, it still makes sense to say one happens after another.

    Transformational rules are largely abandoned in contemporary generative syntax.
    In a surface-oriented grammar, 
    transformational rules does not say anything about the plausible generative derivational%
    \footnote{From then on, \term{derivational process}, etc. all mean Minimalist ones, i.e. Merge-based ones. 
    But \term{derived} may be associated with surface-oriented transformational rules.} process in our brain.
    If construction A and construction B are connected by a transformation,
    it merely means the derivational process of A and B have some similar stages 
    and there is a uniform correspondence between them. 
    \item There are usually certain kinds of valency-changing devices. 
    They may be implemented by a Latin-like voice system, 
    or by auxiliary verbs the subject of which undergoes an event,
    and the complement of which is a clause or small clause indicating the event,
    or by certain kind of \vP constructions.
    There are no strict boundary between these strategies.
    Combination of these devices, like \citet{collins2005pass}, is also possible.
    Just like the case of arguments and adjuncts, and the case of topicalization and speed act marking,
    passivization can be described with a derivational account as in contemporary generative syntax,
    as well as a transformational approach.
    \item Each stage of the above processes may be grammaticalized or be realized by a grammaticalized construction.
    The so-called Chinese passive construction, the \corpus{bei}-construction,
    is just a grammaticalized complement clause construction 
    with the verb \corpus{bei} expressing a meaning of ``suffer from''. 
    \item Finally, sentences and clauses can be coordinated and undergo supplementation. 
\end{enumerate}

The pipelines of the machine of English clause structure is also organized in this way. 
Valency is an enduring topic in English grammar.
Tense and aspect (usually denoted together as \term{tense}) appears 
in every introduction of the English language:
do something, did something, be doing something.
The subjunctive mood (or ``modality'' in \ac{blt}) is especially necessary in formal or archaic writing.
English adverbs are traditionally ignored but are more complicated than many will think.
The list can be very long.

\subsection{Canonical clauses and their inner grammatical systems}

\ac{cgel} uses the strategy to first describe a canonical clause 
and then use syntactic processes (or transformational rules in generative terms) 
to derive non-canonical ones.
Several syntactic processes may be used together to derive a clause with multiple non-canonical features.

It should be noted that when the definition of canonical clauses is narrow 
(a wise decision leading to easier starting),
it is possible that some non-canonical clauses cannot be obtained 
by applying well-recognized syntactic processes (\ac{cgel} \citesec{2.2} [3]).
This is shown in \prettyref{fig:canonical-clause}.

\begin{figure}
    \centering
    

\tikzset{every picture/.style={line width=0.3pt}} %set default line width to 0.75pt        

\begin{tikzpicture}[x=0.75pt,y=0.75pt,yscale=-1,xscale=1]
%uncomment if require: \path (0,386); %set diagram left start at 0, and has height of 386

%Straight Lines [id:da13680161853786577] 
\draw    (194,152.81) -- (313.26,84.8) ;
\draw [shift={(315,83.81)}, rotate = 150.31] [fill={rgb, 255:red, 0; green, 0; blue, 0 }  ][line width=0.08]  [draw opacity=0] (12,-3) -- (0,0) -- (12,3) -- cycle    ;
%Straight Lines [id:da16837784710650316] 
\draw    (194,152.81) -- (310.33,229.71) ;
\draw [shift={(312,230.81)}, rotate = 213.47] [fill={rgb, 255:red, 0; green, 0; blue, 0 }  ][line width=0.08]  [draw opacity=0] (12,-3) -- (0,0) -- (12,3) -- cycle    ;
%Straight Lines [id:da07747924239146564] 
\draw    (194,152.81) -- (306.85,314.17) ;
\draw [shift={(308,315.81)}, rotate = 235.03] [fill={rgb, 255:red, 0; green, 0; blue, 0 }  ][line width=0.08]  [draw opacity=0] (12,-3) -- (0,0) -- (12,3) -- cycle    ;
%Straight Lines [id:da42303822011922976] 
\draw  [dash pattern={on 4.5pt off 4.5pt}]  (358,201.81) -- (358,98.15) ;
\draw [shift={(358,203.81)}, rotate = 270] [fill={rgb, 255:red, 0; green, 0; blue, 0 }  ][line width=0.08]  [draw opacity=0] (12,-3) -- (0,0) -- (12,3) -- cycle    ;
%Straight Lines [id:da4793554387643766] 
\draw  [dash pattern={on 4.5pt off 4.5pt}]  (358,293.81) -- (358,266.81) ;
\draw [shift={(358,295.81)}, rotate = 270] [fill={rgb, 255:red, 0; green, 0; blue, 0 }  ][line width=0.08]  [draw opacity=0] (12,-3) -- (0,0) -- (12,3) -- cycle    ;

% Text Node
\draw (118,134) node [anchor=north west][inner sep=0.75pt]   [align=left] {\begin{minipage}[lt]{44.93pt}\setlength\topsep{0pt}
\begin{center}
argument\\structure
\end{center}

\end{minipage}};
% Text Node
\draw (200,36) node [anchor=north west][inner sep=0.75pt]   [align=left] {\begin{minipage}[lt]{39.53pt}\setlength\topsep{0pt}
\begin{center}
trivial\\polarity,\\voice,\\etc. 
\end{center}

\end{minipage}};
% Text Node
\draw (328,57) node [anchor=north west][inner sep=0.75pt]   [align=left] {\begin{minipage}[lt]{44.05pt}\setlength\topsep{0pt}
\begin{center}
canonical\\clauses
\end{center}

\end{minipage}};
% Text Node
\draw (313,213) node [anchor=north west][inner sep=0.75pt]   [align=left] {\begin{minipage}[lt]{63.87pt}\setlength\topsep{0pt}
\begin{center}
some\\non-canonical\\clauses
\end{center}

\end{minipage}};
% Text Node
\draw (315,296) node [anchor=north west][inner sep=0.75pt]   [align=left] {\begin{minipage}[lt]{63.87pt}\setlength\topsep{0pt}
\begin{center}
other\\non-canonical\\clauses
\end{center}

\end{minipage}};
% Text Node
\draw (181,240) node [anchor=north west][inner sep=0.75pt]   [align=left] {\begin{minipage}[lt]{45.76pt}\setlength\topsep{0pt}
\begin{center}
nontrivial\\polarity,\\voice,\\etc. 
\end{center}

\end{minipage}};
% Text Node
\draw (409,114) node [anchor=north west][inner sep=0.75pt]   [align=left] {\begin{minipage}[lt]{68.79pt}\setlength\topsep{0pt}
\begin{center}
transformation\\processes
\end{center}

\end{minipage}};
% Text Node
\draw (412,260) node [anchor=north west][inner sep=0.75pt]   [align=left] {\begin{minipage}[lt]{64.45pt}\setlength\topsep{0pt}
\begin{center}
generalization
\end{center}

\end{minipage}};


\end{tikzpicture}

    \caption{Building up canonical and non-canonical clauses}
    \label{fig:canonical-clause}
\end{figure}

It is already reviewed in \prettyref{sec:pipeline} that clauses, strictly speaking, 
are fully marked with speech act information and hence are CPs themselves.
Some people use the term \term{clause} for TPs and \term{sentence} for CPs.
This does not agree with the acceptable terminology in surface-oriented studies,
since a pure TP without CP layers rarely occurs.
Therefore, in \ac{cgel}, the pipelines of clause building is not made quite clear:
the approach accepted is to present grammatical relations as \emph{static} ones.

Here is a demonstration about this static approach. Recall \prettyref{fig:to-the-fat-man-blt}.
Though (a) has more embedding hierarchies than (b),
while (b) uses dependency relations in lieu of the binary branching hierarchies,
both of them are static ones:
(a) is presented in the grammar \emph{as a whole}, 
i.e. the grammar does not treat (a) as built by first merging \corpus{fat} and \corpus{man}
and then merging the result with \corpus{the} and then \corpus{to}.
Rather, once a PP is discussed, all grammatical relations involved
-- attributive modification, definiteness, dative construction -- 
are considered as completed.
Similarly, when talking about a clause, 
all grammatical relations in the clause is considered as already given in \ac{cgel}.
We \emph{do not} describe the structure in terms of Merging elements.
The discrepancy between \ac{cgel} and \ac{blt} is merely that 
the former uses hierarchy structure to show grammatical relations while the latter uses dependency arcs.

It should also be noted that though the \ac{cgel} approach is mostly static and not derivational,
its analyses are not idiomatically representational, either,
for the obvious deviation from approaches like HPSG.
It is best described as \ac{tag}-like.
Sometimes, the pipeline notion is of descriptive benefits, 
because some syntactic processes definitely work in order:
for example, subject-auxiliary inversion always happens before \term{wh}-fronting.
This is discussed in \ac{cgel} \citesec{2.2}: 
``Main clause interrogatives \dots will 
thus have one prenucleus + nucleus construction \dots functioning as nucleus within another \dots''

\subsubsection{Canonical clauses}

A \concept{canonical clause} is a standalone, declarative, active clause 
consisting of only the verb complex 
(\emph{do} or \emph{has done} or \emph{be doing} or \dots, i.e. the verb phrase in \ac{blt}), 
complements and adjuncts. 
In \ac{cgel} the verbal complex is not recognized as a phrase.
Rather, it is a verbal idiom, and the head of the clause is the highest auxiliary verb. % TODO: ref
This means a canonical clause is made up solely by a verb and its complements and adjuncts.
A \concept{minimal canonical clause} does not have adjuncts.

This definition means negative clauses are not canonical,
because \corpus{not} is an additional clausal dependent.
An interrogative clause is not canonical,
since it is not declarative and has prenucleus dependents 
-- the fronted verb and, if any, the \term{wh}-expression.
A passive clause is not canonical.
A subordinated clause is not canonical, 
because even though it is declarative and active,
it is not standalone.
(\ac{cgel} \citesec{2.2} [1])


\subsubsection{Clausal complements and the argument structure}

Though the starting point of the building process of all clauses is the argument structure,
\ac{cgel} starts \citechap{4} with discussion on clausal complements.
This is a wise decision for a largely surface-oriented grammar,
and also does not obscure the argument structure, 
because how the arguments (or non-argument complements) of the verb 
fills the \emph{clausal} complement positions
is largely based on the structural distribution of these arguments in the \vP{} structure.
In other words, the clause complement positions -- subject, direct and indirect objects, etc. --
are \emph{syntactic} marking of coarse-grained S, A, O, E arguments,
which reflect the relative positions of arguments in \vP.%
\footnote{
    O is the label of the patient-like or \term{patientive} argument in \ac{blt}.
    In many modern grammars, the symbol is replaced by P.
    In this note the symbol O is used to be consistent with the notation in \ac{cgel}, 
    though \ac{cgel} itself uses O as a symbol of clausal complements.
}

For discussion on various complement positions and their correlation with the argument structure,
see \prettyref{sec:types-of-complements}.

\subsubsection{The subject}\label{sec:subject-predicate}

English is a subject-prominent language.
This is to say, in every clause there is a \concept{subject} 
(which may be omitted in limited cases) % TODO: what cases?
which is usually filled by an agentive argument
and is somehow ``higher'' and external in the clause structure,
and therefore is subject to several types of extraction, 
including relativization and coordination pivot.

In typological terms, this means English is a syntactic nominative-accusative language,
because its syntactic obligatory topic 
(something somehow ``higher'' in the clause structure,
but lower than the topic in topicalization)
is identical to the agentive argument.%
\footnote{In languages like Chinese, though when a semantically obvious agent is present,
it is always the syntactic obligatory topic, 
there are quite common clauses in which the subject is not agentive at all,
like the famous \corpus{tai shang zuo zhe zhu xi tuan}.
But a deeper analysis will show that this can be attributed to 
the rich light verb inventory of Chinese,
which may be understood as ``mini-voice'' in more descriptive terms,
and therefore does not impose any threat to the nominative-accusative status of Chinese.}
In ergative languages the notion of \term{subject} is more complicated,
because in transitive clauses, the two do not coincide.
But this is not the case for English.

\begin{figure}
    \centering
    

\tikzset{every picture/.style={line width=0.3pt}} %set default line width to 0.75pt        

\begin{tikzpicture}[x=0.75pt,y=0.75pt,yscale=-1,xscale=1]
%uncomment if require: \path (0,300); %set diagram left start at 0, and has height of 300

%Straight Lines [id:da9834013734875582] 
\draw    (125,95.81) -- (183,78.81) ;
%Straight Lines [id:da10465096544133767] 
\draw    (248,95.81) -- (183,78.81) ;
%Straight Lines [id:da9284048202068891] 
\draw    (191,155.81) -- (249,138.81) ;
%Straight Lines [id:da46755571549890007] 
\draw    (314,155.81) -- (249,138.81) ;
%Straight Lines [id:da9217356601959923] 
\draw    (191,155.81) -- (314,155.81) ;

% Text Node
\draw (183,75.81) node [anchor=south] [inner sep=0.75pt]   [align=left] {Clause};
% Text Node
\draw (125,98.81) node [anchor=north] [inner sep=0.75pt]   [align=left] {\begin{minipage}[lt]{36.4pt}\setlength\topsep{0pt}
\begin{center}
Subject
\end{center}

\end{minipage}};
% Text Node
\draw (248,98.81) node [anchor=north] [inner sep=0.75pt]   [align=left] {\begin{minipage}[lt]{134.96pt}\setlength\topsep{0pt}
\begin{center}
Predicate:\\VP (or rarely something else)
\end{center}

\end{minipage}};
% Text Node
\draw (237,161.4) node [anchor=north west][inner sep=0.75pt]    {$\cdots $};


\end{tikzpicture}

    \caption{A minimal canonical clause is made up by a subject and a predicate}
    \label{fig:subject-predicate}
\end{figure}

The external property of subjects means we have the tree diagram \prettyref{fig:subject-predicate}.
The rest of the clause is named the \concept{predicate},
which is mostly a VP, but in some cases can be verbless (\ac{cgel} \citesec{2.14} [1]).
Since functional heads in the \vP-TP-CP hierarchy may be viewed as realized on the verb,
the ultimate head is the verb,
and in \prettyref{fig:subject-predicate}, the predicate is the head.

For details about the subject position, see \prettyref{sec:subject}.

\subsubsection{Agreement}\label{sec:agreement}

Agreement, if any, is only found between the subject and the highest verb in the predicate,
be it the main verb or an auxiliary verb. 
For details see \prettyref{sec:verb-agreement}.

\subsubsection{Objects, predicative complements, and others, and the structure of a minimal VP}\label{sec:complementation-type}

A VP always start with a verb. What follow the verb are inner complements.
In English possible complements include objects and predicative complements.
The complexity of dependency relations inside the VP blocks the possibility 
to represent them with a constituency tree without movement,
and hence \ac{cgel} presents the inner structure of the VP 
as a multiple-branching tree (\ac{cgel} \citesec{12.3.1} [4]).
The scheme of VPs is shown in \prettyref{fig:verb-phrase}.

\begin{figure}
    \centering
    

\tikzset{every picture/.style={line width=0.3pt}} %set default line width to 0.75pt        

\begin{tikzpicture}[x=0.75pt,y=0.75pt,yscale=-1,xscale=1]
%uncomment if require: \path (0,300); %set diagram left start at 0, and has height of 300

%Straight Lines [id:da9808512749731233] 
\draw    (131,102.81) -- (235,75.81) ;
%Straight Lines [id:da3092850381738288] 
\draw    (254,102.81) -- (235,75.81) ;
%Straight Lines [id:da017157875147768342] 
\draw    (197,162.81) -- (255,145.81) ;
%Straight Lines [id:da005860005871168195] 
\draw    (320,162.81) -- (255,145.81) ;
%Straight Lines [id:da5175071366232089] 
\draw    (197,162.81) -- (320,162.81) ;
%Straight Lines [id:da43669314291625905] 
\draw    (370,102.81) -- (235,75.81) ;

% Text Node
\draw (235,72.81) node [anchor=south] [inner sep=0.75pt]   [align=left] {VP};
% Text Node
\draw (131,105.81) node [anchor=north] [inner sep=0.75pt]   [align=left] {\begin{minipage}[lt]{28.49pt}\setlength\topsep{0pt}
\begin{center}
Head:\\V
\end{center}

\end{minipage}};
% Text Node
\draw (254,105.81) node [anchor=north] [inner sep=0.75pt]   [align=left] {\begin{minipage}[lt]{113.47pt}\setlength\topsep{0pt}
\begin{center}
Complement:\\NP, VP, PP, Clause, etc.
\end{center}

\end{minipage}};
% Text Node
\draw (243,168.4) node [anchor=north west][inner sep=0.75pt]    {$\cdots $};
% Text Node
\draw (370,106.21) node [anchor=north] [inner sep=0.75pt]    {$\cdots $};


\end{tikzpicture}

    \caption{The inner structure of a VP}
    \label{fig:verb-phrase}
\end{figure}

Note that complements are not necessarily referential, not even lexical.
Referential complements that are typically filled by NPs are \concept{objects},
and strongly predicative complements are \concept{predicative complements}.
Note that predicative complements may also be NPs.

There are also adjunct-like complements for some verbs:
the adverb \corpus{badly} in \corpus{he treats us badly},
the PP \corpus{to her article} in \corpus{he referred to her article}, etc.
The adverb complement can usually be replaced by a PP (\prettyref{sec:adjunct}).
Prepositional complements have their own complements, 
which can similarly be divided into object-like ones 
and predicative complement-like ones 
(\ac{cgel} \citesec{2.4} [4]).

There is yet a final class of complements: \concept{particles}.
Particles are usually intransitive prepositions, but adjectives 
(e.g. \corpus{short} in \corpus{he cut short the debate})
are also possible.
A particle is able to appear after the NP connected with it, 
which convince us that the NP is directly a complement of the verb and not the particle.

\concept{Complementation pattern} of a verb decides what complements appear in the cause.
It is limited to several cases 
and is to be found in the dictionary (\prettyref{sec:dict-comp}).

Why the complement types make sense (i.e. whether each complement type has stable syntactic appearance)
is to be discussed in \prettyref{sec:nucleus-dependents}.

\subsubsection{Case marking}\label{sec:case-marking-clause}

In finite clauses, the subject receives the nominative case, 
and other NP complements receive the accusative case.
Morphological case marking is only available for pronouns.

The subject of an infinitive receives the accusative case.
The subject of a nonfinite clause headed by a participle 
receives the dependent genitive case.

\subsubsection{Tense, aspect and mood}\label{sec:tense-aspect}

Tense is the grammatical category about obligatory time marking.
Aspect is the grammatical category about obligatory marking of the temporal organization of the action.
Mood is the grammatical category about modality -- 
though \ac{blt} uses \term{modality} for \term{mood} in \ac{cgel} 
and \term{mood} for \term{clause type} in \ac{cgel}.

In \ac{cgel}, the perfect-imperfect contrast is placed under tense, not aspect.

\subsubsection{Adjuncts}\label{sec:adjunct}

A VP may have left or right adjuncts. 
\term{Adjunct} is the synonym of \term{clausal modifier},
or \term{adverbial} in some grammars.
Adjuncts may also be added to the left of a minimal canonical clause.
% TODO: 这一节的位置要移动,因为non-canonical clause也可以经历adjunction;但是也许也未必要移动,因为non-canonical clause的转换规则应该足够把adjunct的信息带进去了
It should be noted that adjuncts of different functions fill different positions
and are by no means \term{adjuncts} in early version of generative syntax.
They are actually peripheral arguments in \ac{blt}.
This is even the case for adverbs: 
adverbs can be almost straightforwardly 
(but not always: some additional thinking is needed to transform some adverbs to PPs) 
replaced by PPs which codes peripheral arguments 
that can be filled by NPs denoting abstract properties
(e.g. from \corpus{exceedingly} to \corpus{in an exceeding manner}).

\subsection{Non-canonical clauses without prenucleus}\label{sec:in-clause-transformation}

This section is about non-canonical clauses that largely keep the structure of canonical clauses,
so structural analyses of canonical clauses like \prettyref{fig:subject-predicate} still works. % TODO: other figures about the structure of the canonical clause

Below I collect non-canonical constructions, with the order from more internal ones to more external ones.

\subsubsection{Polarity}

A clause with clausal-level negative polarity is called a \concept{negative clause}.
Clausal negation is realized by the negative operator \corpus{not},
which always appears after the highest auxiliary verb and before the second highest verb, auxiliary or not.
If there is no auxiliary verb, 
\corpus{do}-insertion is invoked and \corpus{not} appears after \corpus{do} and before the main verb.
The negator \corpus{not} may also be attached to the highest auxiliary verb,
forming the negative form of the auxiliary verb.
The position of \corpus{not} indicates the scope of negation.
The relative position between \corpus{not} and adjuncts is therefore not fixed.

A negative clause has several aspects different with positive ones.
Three such aspects are illustrated in \ac{cgel} \citesec{2.9} [1], which are
license of \corpus{not even} supplementation 
(the \corpus{not even} phrase is a supplement according to \ac{cgel} \citesec{9.1.1}),
negative connective adjuncts \corpus{nor} and \corpus{neither} instead of the positive counterpart \corpus{so},
and differences in tag question formation.

Some terms (like \corpus{no longer}) are licensed in non-affirmative clauses. 
Negative clauses are non-affirmative, but so are interrogative ones.

Apart from clausal negation, 
there is also \concept{subclausal negation}.

\subsubsection{Voice}\label{sec:voice}

English has only one nontrivial voice: the passive voice.
In a passive clause, the main verb is replaced by its past participle,
the subject of the active counterpart is turned into an optional \corpus{by}-PP,
which is an internal complement of the past participle,
and the object of the active counterpart (or the indirect object in a ditransitive clause) % TODO: ref
is promoted to the subject position.
These operations do not alter auxiliary verbs, complements not involved and adjuncts.
In a finite passive clause, 
a proper form of \corpus{be} or \corpus{get} is inserted 
right before the past participle VP 
(so \corpus{be} or \corpus{get} is lower than the lowest auxiliary verb and higher than adjuncts in the VP).
In a nonfinite passive clause,
if there is any auxiliary verb, 
\corpus{be}-insertion happens,
but if there is not,
it is permitted to have a \emph{bare} passive clause.
Formally, then, finite passive clauses and nonfinite passive clauses with auxiliary verbs 
can be analyzed as auxiliary verb(s) + \corpus{be}/\corpus{get} + bare passive clause. 

\subsubsection{Finiteness}\label{sec:finiteness}

A nonfinite clause is a clause that obligatorily lacks some TP and/or CP functional projections,
often resulting in inability to have a typical subject (since the TP layer is somehow quirky).
It may be considered as somehow nominalized and the nominalization degree is deeper than content clauses, %TODO: ref to content clause
since its main role is to fill argument slots,
and unlike finite content clauses that can be a sentence with a minimal transformational,
the nonfinite clause can never be a full sentence.
It can still constitute a short reply to a question, as in 
\corpus{-- What are you doing? -- Doing my midterm project.}
But NPs have this function, too.

In English, nonfinite clauses are either participles or infinitives.
None of them is able to have a typical, nominative (if realized as a pronoun) subject,
and all of them have distinct morphosyntactic marking.

It should be noted that participle differ from ``genuine'' nominalization. % TODO: ref
A participle takes object(s), 
while a nominalized verb does not. 

Even more ``nonfinite'' (though the term is already not appropriate here, since the TP layer may have gone)
is the \concept{verbless clause} construction, 
where the predicate position is not filled.
In generative terms, maybe something like an invisible PredP is assumed in lieu of the copular, 
but from a more surface-oriented perspective, this is not necessary.
Verbless clauses in English, like typical nonfinite clauses, 
are bond to be subordinated clauses.

\subsubsection{Summary: nucleus clause}\label{sec:summary-nucleus}

Above are all syntactic categories involved in clauses 
that are either canonical or have structures identical to canonical clauses.
Such a clause plays the nucleus role in prenucleus constructions (\prettyref{sec:prenucleus}).
The structure can be summarized as generated by the following procedure,
which covers varieties created by transformational rules concerning 
negation and voice:
\begin{enumerate}
    \item First, build a VP with a non-catenative verb and its complements.
    The complementation pattern (\prettyref{sec:complementation-type}),
    case marking of the complements (\prettyref{sec:case-marking-clause}),
    and of course the argument structure 
    are grammatical categories involved in this step.
    If we are building a passive clause, 
    the voice category is also involved.
    \item Build the maximal VP in the clause, which is in the form of 
    auxiliary verbs (if any) -- 
    non-auxiliary catenative verbs (if any) --
    a non-catenative verb -- complements.
    Note that leftward verbs have higher syntactic positions.

    All verbs must be in correct inflectional forms.
    Modal auxiliary verbs do not have nonfinite forms.
    Voice (\prettyref{sec:voice}), TAM categories (\prettyref{sec:tense-aspect}),
    finiteness (\prettyref{sec:finiteness})
    are grammatical categories involved in this step.
    
    Acceptable sequences of auxiliary verbs are given in \ac{cgel} \citesec{3.2.3} [43].
    \item Add the subject. Make sure the subject is in the nominative case if it is a pronoun.
    Check the agreement between the subject and the verb.
    \item Add adjuncts. They can appear in all levels of the VP.
    \item Add clausal negation if necessary.
    Mind the scope.
    Check whether all complements and adjuncts fit in the polarity environment.
    The polarity category is involved in this step.
\end{enumerate}

The procedure above is certainly not the whole story of clause generation.
Recall the definition of canonical clauses,
the two keywords \term{standalone} and \term{declarative} are yet to come. 
They correspond to whether the clause is subordinated and the clause type, respectively.
Syntactic devices related to these two categories, however,
involve the prenucleus position, which is discussed in \prettyref{sec:prenucleus}.

\subsection{Prenucleus}\label{sec:prenucleus}

For declarative and standalone clauses with trivial information structure, 
\prettyref{sec:summary-nucleus} is the whole story.
Other clauses -- all non-canonical ones -- go further: 
something is promoted to a position even higher than the subject.
Such positions are called \concept{prenucleus} positions (\ac{cgel} \citesec{2.2} [5]), 
and we get a structure like \prettyref{fig:prenucleus}.

\begin{figure}
    \centering
    

\tikzset{every picture/.style={line width=0.3pt}} %set default line width to 0.75pt        

\begin{tikzpicture}[x=0.75pt,y=0.75pt,yscale=-1,xscale=1]
%uncomment if require: \path (0,330); %set diagram left start at 0, and has height of 330

%Straight Lines [id:da6926211514368799] 
\draw    (254,180.81) -- (312,163.81) ;
%Straight Lines [id:da09624855735969873] 
\draw    (377,180.81) -- (312,163.81) ;
%Straight Lines [id:da7698332578870457] 
\draw    (320,240.81) -- (378,223.81) ;
%Straight Lines [id:da14094495477352398] 
\draw    (443,240.81) -- (378,223.81) ;
%Straight Lines [id:da9272675571958595] 
\draw    (320,240.81) -- (443,240.81) ;
%Straight Lines [id:da2811370178161874] 
\draw    (186,114.81) -- (244,97.81) ;
%Straight Lines [id:da05920620141627109] 
\draw    (309,114.81) -- (244,97.81) ;
%Curve Lines [id:da1523420057005347] 
\draw    (378,263.81) .. controls (290,353.81) and (206,335.81) .. (186,161.81) ;
\draw [shift={(186,161.81)}, rotate = 83.44] [fill={rgb, 255:red, 0; green, 0; blue, 0 }  ][line width=0.08]  [draw opacity=0] (12,-3) -- (0,0) -- (12,3) -- cycle    ;

% Text Node
\draw (312,160.81) node [anchor=south] [inner sep=0.75pt]   [align=left] {\begin{minipage}[lt]{40.42pt}\setlength\topsep{0pt}
\begin{center}
Nucleus:\\Clause
\end{center}

\end{minipage}};
% Text Node
\draw (254,183.81) node [anchor=north] [inner sep=0.75pt]   [align=left] {\begin{minipage}[lt]{36.4pt}\setlength\topsep{0pt}
\begin{center}
Subject
\end{center}

\end{minipage}};
% Text Node
\draw (377,183.81) node [anchor=north] [inner sep=0.75pt]   [align=left] {\begin{minipage}[lt]{134.96pt}\setlength\topsep{0pt}
\begin{center}
Predicate:\\VP (or rarely something else)
\end{center}

\end{minipage}};
% Text Node
\draw (366,246.4) node [anchor=north west][inner sep=0.75pt]    {$\cdots $};
% Text Node
\draw (244,94.81) node [anchor=south] [inner sep=0.75pt]   [align=left] {Clause};
% Text Node
\draw (186,117.81) node [anchor=north] [inner sep=0.75pt]   [align=left] {\begin{minipage}[lt]{53.6pt}\setlength\topsep{0pt}
\begin{center}
Prenucleus:\\$\displaystyle \cdots $
\end{center}

\end{minipage}};


\end{tikzpicture}

    \caption{Prenucleus positions}
    \label{fig:prenucleus}
\end{figure}

\subsubsection{Subordination}

English has three major types subordinated clause:
\begin{itemize}
    \item \concept{Content clauses}, 
    which are often named as nominal clauses in more traditional works,
    but their status as NPs are rejected in \ac{cgel}. % TODO: ref
    \item \concept{Relative clauses}.
    \item \concept{Comparative clauses}, as in \corpus{it costs more than [we expected]}.
\end{itemize}
Subordinated clauses cannot be imperative.

\ac{cgel} also does not use the term \term{complement clause},
since \term{content clause} is about syntactic form and not function -- 
a content clause's main role is indeed filling a complement position,
but it can also be an adjunct, for example (\ac{cgel} \citesec{8.6.3} [18] and [19]).

Traditional grammar uses \term{adverbial clause} and \term{noun clause} 
to distinguish these two functions,
but the names mislead people to a content clause and an adjunct clause differ in their forms.
For some languages, distinction between (function labels) complement clauses, temporal adjunct clauses, etc. 
indeed directly results in distinction in forms
(with adjunct clauses being headed by \term{converbs}, for example),
so terms like \term{complement clause} may well be good terms for syntactic forms,
but this is not the case in English.
An adjunct clause can be a complement clause (\ac{cgel} \citesec{11.3.1} [4]),
as well as a fused relative clause (\ac{cgel} \citesec{12.6.4} [30, 31])
preceded by a word like \corpus{when},
which may be considered as the fusion between a preposition and a relative clause 
(\corpus{at the occation that \dots})
or a preposition that takes a complement clause.
Thus % TODO: 这一段的逻辑需要理一下 
Thus the relation between ``complement clauses'' and ``adjunct clauses'' is 
just the relation between NPs and PPs, 
and one certainly will not call a PP ``adjunct NP'',
since it can well appear as a complement,
in the same way an ``adjunct clause'' -- like a \corpus{when} clause --
may act as a complement.

A finite subordinated declarative clause may be labeled by an initial \corpus{that}.

\subsubsection{Clause types}\label{sec:force}

In \ac{cgel}, \concept{clause type} means Force in generative syntax and mood in \ac{blt}.
In English there are the following clause types:
declarative, closed interrogative, open interrogative, exclamative, imperative.

The clause type is the syntactic coding of \concept{illocutionary force}:
an imperative clause expresses a direct illocutionary force.
But illocutionary force need not be coded as the clause type:
\corpus{would you please \dots} is formally interrogative, 
but it is often a polite directive.
In this case, the illocutionary force is indirect.

Non-subordinated interrogative clauses always involve the prenucleus position,
except tag questions.
Non-subordinated closed interrogative clauses undergo subject-auxiliary inversion,
but the tag question construction (\prettyref{sec:question-tag}) is also a strategy.
Non-subordinated open interrogative clauses first undergo subject-auxiliary inversion
and then \corpus{wh}-fronting, if the \corpus{wh}-phrase is not the subject.
Otherwise only \corpus{wh}-fronting happens.
Subordinated interrogative clauses do not undergo subject-auxiliary inversion,
but \corpus{wh}-fronting happens for open interrogative ones.

Imperative clauses may have zero subjects.

\subsubsection{Subject-auxiliary inversion}

Subject-auxiliary inversion appears in several constructions that may have certain degree of focusing.
The scheme of subject-auxiliary inversion is shown in \prettyref{fig:sai},
which also explains its name.

\begin{figure}
    \centering
    

\tikzset{every picture/.style={line width=0.3pt}} %set default line width to 0.75pt        

\begin{tikzpicture}[x=0.75pt,y=0.75pt,yscale=-0.8,xscale=0.8]
%uncomment if require: \path (0,300); %set diagram left start at 0, and has height of 300

%Straight Lines [id:da9652722991543099] 
\draw    (257,113.81) -- (315,96.81) ;
%Straight Lines [id:da5648303688919729] 
\draw    (380,113.81) -- (315,96.81) ;
%Straight Lines [id:da8783102648672401] 
\draw    (189,55.81) -- (247,38.81) ;
%Straight Lines [id:da13941838633719073] 
\draw    (312,55.81) -- (247,38.81) ;
%Straight Lines [id:da813119782840229] 
\draw    (275,181.81) -- (379,154.81) ;
%Straight Lines [id:da42858587157058614] 
\draw    (398,181.81) -- (379,154.81) ;
%Straight Lines [id:da9718482817188618] 
\draw    (341,241.81) -- (399,224.81) ;
%Straight Lines [id:da19520051987662845] 
\draw    (464,241.81) -- (399,224.81) ;
%Straight Lines [id:da6335012219476788] 
\draw    (341,241.81) -- (464,241.81) ;
%Straight Lines [id:da5194311249882904] 
\draw    (514,181.81) -- (379,154.81) ;

% Text Node
\draw (312,58.81) node [anchor=north] [inner sep=0.75pt]   [align=left] {\begin{minipage}[lt]{40.39pt}\setlength\topsep{0pt}
\begin{center}
Nucleus:\\Clause
\end{center}

\end{minipage}};
% Text Node
\draw (257,116.81) node [anchor=north] [inner sep=0.75pt]   [align=left] {\begin{minipage}[lt]{36.38pt}\setlength\topsep{0pt}
\begin{center}
Subject
\end{center}

\end{minipage}};
% Text Node
\draw (380,116.81) node [anchor=north] [inner sep=0.75pt]   [align=left] {\begin{minipage}[lt]{47.57pt}\setlength\topsep{0pt}
\begin{center}
Predicate:\\VP
\end{center}

\end{minipage}};
% Text Node
\draw (247,35.81) node [anchor=south] [inner sep=0.75pt]   [align=left] {Clause};
% Text Node
\draw (189,58.81) node [anchor=north] [inner sep=0.75pt]   [align=left] {\begin{minipage}[lt]{53.57pt}\setlength\topsep{0pt}
\begin{center}
Prenucleus:\\V$\displaystyle _{i}$
\end{center}

\end{minipage}};
% Text Node
\draw (275,184.81) node [anchor=north] [inner sep=0.75pt]   [align=left] {\begin{minipage}[lt]{28.47pt}\setlength\topsep{0pt}
\begin{center}
Head:\\---$\displaystyle _{i}$
\end{center}

\end{minipage}};
% Text Node
\draw (398,184.81) node [anchor=north] [inner sep=0.75pt]   [align=left] {\begin{minipage}[lt]{113.4pt}\setlength\topsep{0pt}
\begin{center}
Complement:\\NP, VP, PP, Clause, etc.
\end{center}

\end{minipage}};
% Text Node
\draw (387,247.4) node [anchor=north west][inner sep=0.75pt]    {$\cdots $};
% Text Node
\draw (514,185.21) node [anchor=north] [inner sep=0.75pt]    {$\cdots $};


\end{tikzpicture}

    \caption{Subject-auxiliary inversion}
    \label{fig:sai}
\end{figure}

Subject-auxiliary inversion occurs in subjunctive conditional clauses, % TODO: ref
causes with connective adjuncts, % TODO: nor do I 这样的句子

After subject-auxiliary inversion, 

\subsubsection{Preposing}

\subsubsection{Adjuncts near the prenucleus positions}

Adjuncts have pre-clausal positions (\ac{cgel} \citesec{8.20.1}), 
and this also works when prenucleus objects appear.
Adjuncts appearing in pre-clausal positions are always before the fronted auxiliary verb.
% TODO: preposing?
On the other hand, 
there is no strict ordering between pre-clausal adjuncts and the fronted \corpus{wh}-phrase.
The pre-clausal adjuncts, therefore, may be consider as a limited topicalization device.

\subsection{Rightward movements and related controversies}

There are several constructions that involve a seemingly rightward movement.

\subsubsection{Postposing}

Postposing, also known as heavy-NP shift, 

\subsubsection{Delayed right constituent coordination}

The so-called postnucleus position in \ac{cgel} \citesec{15.4.4} [28],
therefore, is to be interpreted as a position created in PF,
which is consistent with the discussion related to this tree diagram 
that no functional label except the postnucleus label is assigned to the delayed right constituent,
and the delayed right constituent keeps its functions in the two coordinates, 
which may be different (as in the case of [28]):
since the postnucleus position is not strictly core-syntactic,
indeed no new dependency relations are introduced.

\subsection{Information packaging}

\prettyref{sec:in-clause-transformation} and \prettyref{sec:prenucleus} are about 
transformations on canonical clauses, 
about what feature triggers what transformation,
and relevant syntactic positions that are not observed in canonical clauses.
There are, however, constructions of which constituency analyses involve 
nothing more than \prettyref{sec:in-clause-transformation} and \prettyref{sec:prenucleus}
but are unable to show the information structure of these constructions.
The constituency tree of a \corpus{there be} construction is just 
an instance of the canonical clausal structure \prettyref{fig:subject-predicate},
with the subject being a dummy \corpus{there},
but from the constituency tree one cannot infer confidently the existential meaning.

Without considering the information structure, 
the way perfect aspect is realized in English
resembles the example of the \corpus{there be} existential construction
in that its meaning is not combinatory:
in \corpus{he has done this before},
the head may be analyzed as \corpus{has},
which agrees with the subject (\prettyref{sec:agreement})
and takes the past participle VP \corpus{done this before} as its complement.
In the VP \corpus{done this before}, \corpus{before} is an adjunct, 
and \corpus{this} is the object. 
The structure is crystal-clear.
The only problem is the true meaning of the clause cannot be inferred from the structure.
However, the perfect aspect is not marked with respect to the information structure,
so it is not introduced in the chapter about information packaging in \ac{cgel}.

\ac{cgel} \citesec{2.16} [1] lists all information packaging constructions.
The existential construction and the passive construction are available for question formation,
while others do not.

\subsubsection{Summary: what happens after \prettyref{sec:summary-nucleus}}

\subsection{Coordination}

The English clausal coordination construction is largely symmetric, 
thought there are reasons to believe the second clause and the coordinator form a constituent.
Since the coordination head is realized as a single word but is by no means lexical,
it is impossible to assign the head status to any lexical word as in \prettyref{fig:cgel-minimalism},
and therefore in \ac{cgel}, coordination is deemed as a headless construction. % TODO: ref

\begin{figure}
    \centering
    

\tikzset{every picture/.style={line width=0.3pt}} %set default line width to 0.75pt        

\begin{tikzpicture}[x=0.75pt,y=0.75pt,yscale=-1,xscale=1]
%uncomment if require: \path (0,300); %set diagram left start at 0, and has height of 300

%Straight Lines [id:da9034114884818367] 
\draw    (187,104.81) -- (245,87.81) ;
%Straight Lines [id:da5436516056536784] 
\draw    (310,104.81) -- (245,87.81) ;
%Straight Lines [id:da11793075192673963] 
\draw    (250,143.81) -- (308,126.81) ;
%Straight Lines [id:da045608599287051854] 
\draw    (373,143.81) -- (308,126.81) ;

% Text Node
\draw (245,84.81) node [anchor=south] [inner sep=0.75pt]   [align=left] {Coordination};
% Text Node
\draw (187,107.81) node [anchor=north] [inner sep=0.75pt]   [align=left] {Coordinate$\displaystyle _{1}$};
% Text Node
\draw (310,107.81) node [anchor=north] [inner sep=0.75pt]   [align=left] {Coordinate$\displaystyle _{2}$};
% Text Node
\draw (373,146.81) node [anchor=north] [inner sep=0.75pt]   [align=left] {Coordinate$\displaystyle _{2}$};
% Text Node
\draw (250,146.81) node [anchor=north] [inner sep=0.75pt]   [align=left] {Marker};


\end{tikzpicture}

    \caption{Scheme tree diagram of a simple coordination construction}
    \label{fig:simple-coord}
\end{figure}

\prettyref{fig:simple-coord} is a scheme of coordination like ``John and Paul''.
More complicated forms of coordination exist, though.

\subsection{Supplementation}\label{sec:supplmentation}

Supplementation, as in \corpus{her father -- [a man of quality] -- promises to help us},
deviates further from the headed tree structure.
This type of constructions shows significantly weaker dependency 
between the ``anchor'' and the inserted supplement.
In generative syntax, supplementation can be analyzed as a result of PF-level clause reformulation,
with major disruption of the constituent structure of the host clause, 
which tempts lots of people to analyze the phenomenon 
as a result of an idiosyncratic structure building mechanism beside Merge \citep{ott2014ellipsis}.
Therefore, in \ac{cgel}, supplementation is recognized as a construction type,
but the supplement and the anchor are not recognized as a constituent (\ac{cgel} \citesec{15.5.1} [12]).
It is practically not possible to insist on both binary branching and surface-oriented analysis.

\subsubsection{Question tags}\label{sec:question-tag}

\section{Complements and adjuncts}\label{sec:nucleus-dependents}

\subsection{Overview of complement types}\label{sec:types-of-complements}

Clausal (or verbal, since the clause is headed by the verb) complements 
may be NPs and PPs, and less frequently, adverbs 
(as in \corpus{He treated us [kindly]}). % TODO: adverbial clause: the name and classification 

This section lists some criteria of classification of complements.
They are all discussed in \ac{cgel} \citesec{4.1.1}.


How to tell a complement from adjuncts is a question addressed in \prettyref{sec:recognizing-complement-clause}.
\prettyref{sec:recognizing-complement-clause} is delayed after discussion on complements and adjuncts,
because we have to first show prototypical properties of the two 
and introduce necessary concepts like how semantic roles are coded as clausal dependents,
and only then can we draw an exact line between the two.

\subsubsection{Core v.s. oblique}\label{sec:core-oblique}

One classification standard is the make up of the complement.
A \concept{core} complement is a complement with similar morphosyntactic properties of NP complements.
A \concept{non-core} complement is a complement with similar morphosyntactic properties of PP complements.
If a non-core complement itself takes an NP complement (or something with similar morphosyntactic properties),
the latter is called an \concept{oblique}.

Note that in \ac{cgel}, the term \term{argument} is reserved for purely semantic objects.
A clausal complement is therefore the syntactic incarnation of an argument,
but itself is not an argument.
This is not the way \term{argument} is used in \ac{blt}.

It should also be noted that in \ac{cgel}, the terms \term{non-core} and \term{oblique} 
are reserved for clausal complements and the NP part of PP clausal complements.
They \emph{do not} include adjuncts with similar forms.
On the other hand, in \ac{blt}, the term \term{peripheral argument} covers both PP complements and adjuncts.
The term \term{oblique} is often associated with \term{oblique cases}.
In traditional Latin grammar, 
cases other than the nominative and the vocative cases are all called oblique cases.
In other usages, oblique cases exclude the accusative case.
In English the case system has largely collapsed,
and the name \term{oblique} does not have much morphological consequences:
an oblique complement is never nominative as we will see, and that is all.

The prototypical definition of core and oblique complements 
are based on syntactic forms instead of functions,
while the definition is extended by analog with respect to syntactic functions.
Whether these terms are useful is a question we need to wait and see (\prettyref{sec:complement-form-function}).

\subsubsection{External and internal}\label{sec:external-and-internal}

The subject is the \concept{external} complement for obvious reasons.
All other complements are \concept{internal}.

In English, internal complements include \concept{objects}, \concept{predicative complements},
adjunct-like complements, including prepositional complements and adverb complements,
and particles.
Objects and predicate complements are core complements,
while prepositional complements are non-core ones.
It does not make much sense to say whether adverb and particle complements are core or not.

The objects split into \concept{direct objects} and \concept{indirect objects}.
Objects are prototypically NPs, 
while predicative complements are predicative.
Adverbs can be replaced by appropriate PPs (\prettyref{sec:adjunct}), % TODO: more ref
and hence adverb complements may be considered as a variant of prepositional complements.
The complement of a prepositional complement may also be an object or a predicative complement.

The behavior of the object in a transitive clause differs from 
the behavior of objects in a ditransitive clause,
and the behaviors of the two objects in a ditransitive clause are also different.
Complement positions named ``objects'' are therefore not homogeneous.
Their behaviors are strongly dependent on the transitivity.
Nonetheless, the term \term{object} is still a handy one,
because behind it there is a substantial generalization:
complement slots that are prototypically filled by NPs 
always have certain shared properties with the prototypically monotransitive object
(\prettyref{sec:g-t-typology}).

Just like the definition of object, 
the definition of the term \term{prepositional complement} is also about the form and not the function.
Indeed, similar to the case of objects,
not all prepositional complements are in the same syntactic position.
Prepositional complements can be further divided into unspecified and specified ones.
Prepositional complements about manner or mean or (change of) location usually have less restricted prepositions:
both \corpus{he treats us in a bad manner} and \corpus{he treat the new protocol with disdain} are fine.
They are indeed complements (\ac{cgel} \citesec{4.6.1}),
but the preposition is not the key point.
Therefore, they are \concept{unspecified prepositional complements}.
But some verbs only license PPs with a predetermined preposition.
In this case, a PP complement licensed by the verb is a \concept{specified prepositional complement}.
Oblique objects in specified prepositional complements may be available to passivization, 
or may not be, without a general rule, 
and therefore whether passivization is possible is to be recorded in the dictionary (\prettyref{sec:verb-dict});
oblique objects in unspecified prepositional complements are usually unable to undergo passivization,
but locative prepositional complements are exceptional (\prettyref{sec:externalized-passive}).

The syntactic properties of these complements vary from construction to construction
(especially considering the definitions of objects, predicative complements and prepositional complements 
are either semantics-based or form-based, and never syntactic function-based).
It is a wise idea to first give a rough sketch of coarse-grained cartography of complements
-- which is what I am doing here --
and then analyze their behaviors in various constructions, 
and hence obtain more detailed classification of complements.

\subsubsection{Form-based definition has implications about function}\label{sec:complement-form-function}

The term \term{object} is defined as NP-like complements without a preposition.
The term \term{predicative complement} is defined as 
complements resembling the internal complement of a copular clause.
The meaning of the term \term{prepositional complement} is obvious.
These terms are all defined in terms of \emph{form},
but in \ac{cgel} they are used to define canonical clause constructions
(\prettyref{sec:minimal-canonical-clause}).
Obviously, form and function are strongly correlated,
or otherwise the traditional grammar approach that defines NPs according to argument positions 
will not be convincing for centuries -- 
and some argumentation has to be done to establish these terms.
Comparison between complement types is to be done in sections about canonical clauses,
and this section is a brief summary of the relevant discussion.

\begin{itemize}
    \item \emph{Constituent order}. In the default constituent order,
    objects always precede predicative complements and non-core complements. % TODO: ref
    \item \emph{Passivization}. Objects are easier to passivize. % TODO: ref
    Predicative complements are never possible to passivize.
    This is an important criterion to distinguish predicative complements from objects,
    when both of them are NPs
    (\ac{cgel} \citesec{4.5.1} [6]).
\end{itemize}

\subsubsection{Limited benefits of surface-oriented tree diagrams}

Though we have robust evidence for the subtle subdivision 
in objects, prepositional complements, and predicative complements, 
these subtleties are often hard to represent in the surface-oriented coarse-grained tree diagram.
How, for example, to show the reason 
why the indirect object is easier to passivize,
while the direct object is easier to prepose 
(\prettyref{sec:blt-e-argument}, \prettyref{sec:direct-indirect})?
Without a VP-shell theory or in other words, fine-grained analysis of \vP{} \citep{harley2017syntax},
it is rather hard to see the difference from the tree diagram.

\subsection{Types of minimal canonical clauses}\label{sec:minimal-canonical-clause}

As is mentioned in \prettyref{sec:external-and-internal},
strictly speaking, 
a generalized notion of grammatical relations like ``object'' 
is a catch-all term for a variety of similar but not identical grammatical relations 
in several constructions.
Since all types of complements are already listed in \prettyref{sec:types-of-complements},
it is time to list all constructions that contain them, 
and the next step is to find out the behavior of their complements.

\subsubsection{Five canonical clauses with only objects and predicative complements}

The contents of this section are covered in \ac{cgel} \citesec{4.1.1}.

%\paragraph{Subject and object(s): transitivity and valency}

In English every clause has a subject, so the number of subjects is not a parameter.
The number of objects may be 0, 1, and 2,
which is denoted as \concept{intransitive}, \concept{transitve}, and \concept{ditransitive}.
This parameter is named as \concept{transitivity}.

%\paragraph{The number of predicative complements}

Another parameter is the number of predicate complements.
There may be 0 or 1 predicative complement.
It is impossible for a ditransitive clause to have a PC.

So the parameters of transitivity and PC gives the classification 
of clauses with respect to their complements 
as in \ac{cgel} \citesec{4.1.1} [9].
Hence we have the notion of \concept{valency} (\ac{cgel} \citesec{4.1.1} [10]). 
The number of PC is included into the valency.
\ac{cgel} \citesec{4.1.1} [9] is just the classification based on S, O, and PC.
There are no other complements that do not fit perfectly in the paradigm,
namely prepositional complements, adverbs and particles.

We name the object in a monotransitive construction as a \concept{monotransitive direct object}.
In a S-P-O-O construction, 
the first object is named as a \concept{ditransitive indirect object},
while the second object is a \concept{ditransitive direct object}.
These three types of objects have some subtle differences in their properties. %TODO: ref

A predicative complement may be about the subject or the object 
(always a monotransitive direct one, since in ditransitive clauses there is no predicative complement).
What it is about is called the \concept{predicand}.
If the predicand is the subject, the predicative complement is a \concept{subject-oriented} one,
and otherwise it is an \concept{object-oriented} one.

\subsubsection{Canonical clauses headed by prepositional verbs}

A verb selecting a PP complement containing a specified preposition is a \concept{prepositional verb}.
Constructions containing prepositional verbs are listed in \ac{cgel} \citesec{6.1.2} [16].
It can be seen that these schemes are directly related to \ac{cgel} \citesec{4.1.1} [9]
with the following rules:
\begin{itemize}
    \item Replace all predicative complements with PPs containing predicative complements.
    \item In monotransitive schemes, replace the direct objects by PPs containing oblique objects; 
    but it is also okay to not do so.
    \item In ditransitive schemes, always replace the direct objects by PPs containing oblique objects;
    the indirect objects may be -- but not necessarily -- replaced by PPs contacting oblique objects.
\end{itemize}

\subsubsection{Verb-particle construction}

A 

Now, adjunct-like complements aside, 
the 

\subsubsection{Adjunct-like complements}

Unspecified prepositional complements are still not discussed at this point.
They are adjunct-like, and are dealt with in \ac{cgel} \citechap{8} instead of \citechap{4}.

Adjunct-like complements appear in SV (\corpus{he lives in London}) and 
SVO (\corpus{I put my shoes here}) constructions. % TODO: more?
Their positions in a canonical clause are just the same as typical adjuncts of the same semantic roles.

\subsubsection{Catenative complements}

Some verbs are there to tell us some ``properties'' of an action.
In \corpus{everyone believes Kim to be guilty},
the main verb \corpus{believes} takes \corpus{Kim to be guilty} as its complement
and tells us \corpus{Kim to be guilty} is a prevalent opinion.
Whether Kim \emph{is} guilty is not talked about.

Such a complement is a \concept{catenative complement} (\ac{cgel} \citesec{2.14, 3.2.2}).
A verb taking a catenative complement is called a \concept{catenative verb} in \ac{cgel} 
and a secondary verb in \ac{blt}.  
It seems in English (as well as in many other languages),
catenative verbs only take one internal complement 
-- the catenative complement, usually a clause, often nonfinite.
(In clauses like \corpus{I want you to complete this},
\corpus{you} may be analyzed as the object of the verb \corpus{want}, 
leaving a gap in the catenative complement \corpus{complete this},
and \corpus{you} controls this gap.)
The subject may be external to the catenative complement,
or may be inside it.
In the latter case the catenative verb is a raising verb. % TODO

All auxiliary verbs in English are catenative. 
But verbs like \corpus{believe}, \corpus{seem}, etc. are also catenative and certainly not auxiliary verbs.

Syntactic devices with similar semantics 
(and tree structures, if affixes are analyzed as being attached to the verb in a post-syntactic process)
include certain adverbs (\corpus{Kim is [allegedly] guilty})
and certain affixes in other languages.

\subsubsection{Summary}

% TODO: enumeration of all clause structures

\subsection{Semantic roles and the argument structure}

This section is mainly about the argument structure of non-catenative lexical verbs,
for the argument structures of catenative verbs are rather straightforward.
Verbs have verb-specific semantic roles (\ac{blt} \citetable{3.1}),
but these roles can be clustered into larger roles,
which have syntactic significance. % TODO: ref

For examples of these roles, see \ac{cgel} \citesec{4.2.2}.



\subsubsection{The causer-like group}

In the most generalized sense,
a \concept{causer} is an argument that causes an action. 
In a narrower sense, a causer is something causing an event but not intentionally or voluntarily.
A causer in the narrow sense is also called as a \concept{force}.
An \concept{agent}, in the opposite, is something causing an event consciously.
Agents are therefore usually animate.

An \concept{instrument} may be an adjunct (\ac{cgel} \citesec{4.2.2} [3]),
but it can also be coded as the subject, 
as in \corpus{the knife cut the lace}.
When no agent is using an instrument, 
whether the instrument should be analyzed as a causer in the narrow sense or an instrument
becomes not so clear.
The practice in \ac{cgel} is to use the term \term{instrument} only when 
there is indeed an (explicit or implicit) agent using the instrument.

\subsubsection{The patient and similar roles}

A patient always exists with an explicit or implicit causer, especially an agent.
The patient is \emph{influenced} by an externally originated action caused by the causer.
It does not merely undergo displacement or some transient mental change. 
Examples of prototypical patients are found in \ac{cgel} \citesec{4.2.2} [4i].

Sometimes other argument roles have rather similar syntactic and semantic properties with the patient.
For the verb \corpus{hear}, for example,
the canonical object is a stimulus rather than a patient,
but we have passive constructions like \corpus{my opinions are never heard},
in which the stimulus \corpus{my opinion} is just like 
a patient having undergone passivization and is now the subject.

\subsubsection{Experiencer and stimulus}

\concept{Experiencer} and \concept{stimulus} always appear together (though maybe one of them is implicit).
They appear in situations of emotional feeling or sensory perception, as well as cognition.

For some verbs, the experiencer is more agentive, while for others, the stimulus is more agentive.
This is easy to expect:
metaphorically, both the experiencer and the stimulus can be imagined to ``do something'' to the other.

\subsubsection{Themes}

The term \concept{theme} covers a large number of verb-specific roles.
Any argument possessing a (possibly change) property that is relevant in the clause is a theme.
The prototypical ``property'' is the location: 
in \corpus{she ran home}, \corpus{she} is the theme, and \corpus{home} the goal.
Another property, metaphorically related to the location, is (change of) possession.
Thus \corpus{the key} in \corpus{he gave me the key}, and \corpus{the key} in \corpus{the key is mine}
are all possessive themes.
Themes also exist in clauses about more abstract properties, like mental states, 
which often but not necessarily have predicative complements.
In this way, 
\corpus{she} in \corpus{she went mad}, 
\corpus{her} in \corpus{it made her angry},
\corpus{Kim's} and \corpus{Pat's} in \corpus{Kim's writing resembles Pat's}
are all themes.

Similar to the case in experiencers and stimulus,
a theme can be agentive or patient-like, depending to the semantics of the verb.

\subsubsection{(Change of) location}

For a process involving changing location,
we have \concept{path}, \concept{source} and \concept{goal}.
In a clause about a state instead of an event,
there is only a \concept{location} role.

The \concept{recipient} is a role related to verbs like \corpus{give}.
The \concept{beneficiary} is a role that something is done for,
as in \corpus{I've bought [you] a present}.
The two roles are subtypes of the goal.

\subsubsection{Peripheral arguments related to adjuncts}

Compared to complements, the relation between peripheral arguments and adjunct positions 
is much more clear. 
This can be expected, because most peripheral arguments % TODO: justification

Therefore, there is no need to list all peripheral argument roles here 
and then link them to adjunct positions: 
it is fine to introduce an adjunct position and at the same time introduce its semantic role,
because there is almost a one-to-one mapping between the two.

% TODO: he treated us badly中badly和adjunct的相互作用

\subsubsection{Types of argument structures}

It is possible to be two agents in a clause: 

\subsection{Subject, object, and the S, A, O typology}\label{sec:sao-typology}

\subsubsection{About the S, A, O notation}

Once we have gone over complement types and semantic roles,
the question becomes how they match, or in other words, the \concept{alignment}
(\ac{cgel} \citesec{4.2.3}).
Again, since there are so many clause types, 
I start from the basic intransitive clauses and monotransitive clauses,
which do not take complements other than a subject and at most one object, 
i.e. without predicative complements, 
non-core complements, etc.,
as a good starting point.

The typological literature largely uses the S, A, O terms.
These terms have dual roles:
they are coarse-grained argument positions (or semantic roles)
(for example, A may be an agent or an inanimate causer, 
or maybe stimulus for some verbs and experiencer for others),
and they are also clausal positions.
The two roles are not contradictory, 
because the latter is actually an important criterion 
in determining the correct way of coarse-graining in the former:
when two verb-specific argument positions are given the same S, A, O label,
they never appears as different clausal complements in a canonical clause.

An example demonstrates well this point.
In English, both \corpus{the door} in \corpus{the door opens}
and \corpus{the cat} in \corpus{the cat kills} 
are given the S label,
despite evidence from the two labile verbs suggests that 
\corpus{the door} is in a more patientive position (\corpus{I closed [the door]}), 
while \corpus{the cat} is in a more agentive position (\corpus{[the cat] killed a bird}),
because both \corpus{the cat} and \corpus{the door} are in the typical subject position 
in the finished intransitive clauses.
For split-S languages, however, there is no uniform S role,
and Sa (like \corpus{the cat} mentioned above)
and So (like \corpus{the door})
are distinguished, 
again because of the way these arguments fill clausal complements.

In generative terms, the S, A, O positions may be thought of as rough A-positions. 
The definition of S in a language without split-S, for example, may be 
``the only NP argument position in an intransitive \vP{}
(coarse-grained argument position, without specification on whether it moves to SpecTP)'',
and the definition of Sa in a language with split-S may be 
``the only NP argument position in an intransitive \vP{} similar to a DoP (coarse-grained argument position),
which undergoes a A-movement to SpecTP (clausal complement position)'',
while the definition of So may be 
``the only NP argument position in an intransitive \vP{} similar to a UndergoP,
which does not undergo the movement to SpecTP''.
An O argument may be defined as ``the lower argument position in a monotransitive \vP{} 
(coarse-grained argument position),
which usually stays in that position and can be analyzed as a part of the VP in a surface-oriented account
(clausal complement position)''.

Saying S and A are \term{marked in the same way} means S and A are the same with respect to clausal position.
A language in which S and A are marked in the same way (i.e. a nominative-accusative language)
has only one subject-like position, which is usually SpecTP.
A language in which S and O are marked in the same way (i.e. an absolutive-ergative language)
has only one obligatory topic-like position,
which may be SpecTP (in the case of Dyirbal) 
or a position lower than SpecTP which can appear even in nonfinite clauses.
In either case, we know in monotransitive and intransitive clauses,
there are only two complement types (i.e. \term{marking} or \term{coding} of argument,
with S and A \term{coded as the subject} in nominative-accusative languages).
If we know a language can be well described in terms of S, A and O,
we are still not quite sure about how many types of complements there are,
but at least the number should not be more than three.

This is shown in \prettyref{fig:sao-acc}.
By consideration of semantics 
(which frequently also hints at the structure of \vP,
which can be tested by binding effects),
A and O arguments can be proposed as possible coarse-graining labels.
A deeper check about syntactic properties assures 
that all As behave in a similar way, 
and all Os behave in another similar way,
and hence A and O labels can be established, 
which is shown in the first row of \prettyref{fig:sao-acc}.
Similarly, a S position can be established,
since all intransitive clauses have similar \vP{} structures, 
and the A-movement and A'-movement properties of the only argument are largely uniform.
(For split-S languages, this step of reasoning is broken,
and hence SpecUndergoP-like positions and SpecDoP-like positions 
are to be coarse-grained into So and Sa, respectively).
This gives the second row.
Finally, S and A can be identified as a single clausal position,
and hence we get two complement types: the subject 
(= S and A, or if S and A are to be regarded as more semantics-based, 
``S and A are coded as the subject'') 
and the object (= O).
By the same logic of coarse-graining of verb-specific argument positions,
we can coarse-grain S and A into S, 
and thus go back to the traditional grammar's (oversimplified) notion that 
``a verb takes a subject and possibly an object''.
For ergative languages, the story is obviously different.

\begin{figure}
    \centering
    

\tikzset{every picture/.style={line width=0.3pt}} %set default line width to 0.75pt        

\begin{tikzpicture}[x=0.75pt,y=0.75pt,yscale=-0.8,xscale=0.8]
%uncomment if require: \path (0,418); %set diagram left start at 0, and has height of 418

%Straight Lines [id:da5607379428644168] 
\draw    (319,184.57) -- (319,160) ;
\draw [shift={(319,158)}, rotate = 90] [fill={rgb, 255:red, 0; green, 0; blue, 0 }  ][line width=0.08]  [draw opacity=0] (12,-3) -- (0,0) -- (12,3) -- cycle    ;
%Straight Lines [id:da23260332063199396] 
\draw    (439,184.57) -- (439,158) ;
%Straight Lines [id:da2641033045888783] 
\draw    (319,184.57) -- (439,184.57) ;
%Straight Lines [id:da5131301965277419] 
\draw    (319,259.57) -- (319,235) ;
\draw [shift={(319,233)}, rotate = 90] [fill={rgb, 255:red, 0; green, 0; blue, 0 }  ][line width=0.08]  [draw opacity=0] (12,-3) -- (0,0) -- (12,3) -- cycle    ;
%Straight Lines [id:da2634775648555059] 
\draw    (439,259.57) -- (439,233) ;
%Straight Lines [id:da7092937779305173] 
\draw    (319,259.57) -- (439,259.57) ;
%Rounded Rect [id:dp18336678257663852] 
\draw  [draw opacity=0][fill={rgb, 255:red, 245; green, 166; blue, 35 }  ,fill opacity=0.2 ] (307,132.34) .. controls (307,129.39) and (309.39,127) .. (312.34,127) -- (328.37,127) .. controls (331.32,127) and (333.72,129.39) .. (333.72,132.34) -- (333.72,287.59) .. controls (333.72,290.55) and (331.32,292.94) .. (328.37,292.94) -- (312.34,292.94) .. controls (309.39,292.94) and (307,290.55) .. (307,287.59) -- cycle ;
%Rounded Rect [id:dp7831208634321845] 
\draw  [draw opacity=0][fill={rgb, 255:red, 80; green, 227; blue, 194 }  ,fill opacity=0.2 ] (543,132.34) .. controls (543,129.39) and (545.39,127) .. (548.34,127) -- (564.37,127) .. controls (567.32,127) and (569.72,129.39) .. (569.72,132.34) -- (569.72,287.59) .. controls (569.72,290.55) and (567.32,292.94) .. (564.37,292.94) -- (548.34,292.94) .. controls (545.39,292.94) and (543,290.55) .. (543,287.59) -- cycle ;

% Text Node
\draw (122,133.5) node [anchor=north west][inner sep=0.75pt]   [align=left] {transitive};
% Text Node
\draw (124,210.75) node [anchor=north west][inner sep=0.75pt]   [align=left] {intransitive};
% Text Node
\draw (263,131.9) node [anchor=north west][inner sep=0.75pt]    {$[_{\text{TP}}$};
% Text Node
\draw (367,131.9) node [anchor=north west][inner sep=0.75pt]    {$[_{\text{DoP/...}}$};
% Text Node
\draw (621,133.5) node [anchor=north west][inner sep=0.75pt]   [align=left] {]]]};
% Text Node
\draw (434,133.5) node [anchor=north west][inner sep=0.75pt]   [align=left] {$\displaystyle t$};
% Text Node
\draw (550,133.5) node [anchor=north west][inner sep=0.75pt]   [align=left] {O};
% Text Node
\draw (580,133.5) node [anchor=north west][inner sep=0.75pt]   [align=left] {$\displaystyle \cdots $};
% Text Node
\draw (263,209.15) node [anchor=north west][inner sep=0.75pt]    {$[_{\text{TP}}$};
% Text Node
\draw (312,133.5) node [anchor=north west][inner sep=0.75pt]   [align=left] {A};
% Text Node
\draw (312,210.75) node [anchor=north west][inner sep=0.75pt]   [align=left] {S};
% Text Node
\draw (468,131.9) node [anchor=north west][inner sep=0.75pt]    {$[_{\text{TransP/...}}$};
% Text Node
\draw (335,133.5) node [anchor=north west][inner sep=0.75pt]   [align=left] {$\displaystyle \cdots $};
% Text Node
\draw (620,210.75) node [anchor=north west][inner sep=0.75pt]   [align=left] {]]};
% Text Node
\draw (367,209.15) node [anchor=north west][inner sep=0.75pt]    {$[_{\text{DoP/...}}$};
% Text Node
\draw (434,210.75) node [anchor=north west][inner sep=0.75pt]   [align=left] {$\displaystyle t$};
% Text Node
\draw (580,210.75) node [anchor=north west][inner sep=0.75pt]   [align=left] {$\displaystyle \cdots $};
% Text Node
\draw (320.48,306) node [anchor=north] [inner sep=0.75pt]   [align=left] {\begin{minipage}[lt]{90.41pt}\setlength\topsep{0pt}
\begin{center}
complement type 1:\\subject
\end{center}

\end{minipage}};
% Text Node
\draw (568.48,305) node [anchor=north] [inner sep=0.75pt]   [align=left] {\begin{minipage}[lt]{90.41pt}\setlength\topsep{0pt}
\begin{center}
complement type 2:\\obejct
\end{center}

\end{minipage}};


\end{tikzpicture}

    \caption{Recognizing S, A, O arguments in a nominative-accusative language and complement types}
    \label{fig:sao-acc}
\end{figure}

As is mentioned in \prettyref{sec:subject-predicate},
English is a nominative-accusative language,
both syntactically and morphologically.
In canonical clauses,
A and S arguments are consistently coded as the subject, both syntactically and, 
in the case of pronouns, morphologically.
No split of S arguments is easily observable. % TODO: any subtle details?
The problem, then, is the classification of internal complements.
For intransitive clauses and transitive clauses,
as is shown in \prettyref{fig:sao-acc},
an object position can be distinguished.
A predicative complement position is also easy to find. % TODO: 和之前讲complement type的节串起来

\subsubsection{The subject} \label{sec:subject}

Subject properties are listed in \ac{cgel} \citesec{4.3.1}. 
Some of these properties are demonstrated in a canonical clause.
Some are reflected in certain syntactic transformation processes.
Here is a summarized version:
\begin{itemize}
    \item A subject is never oblique are is typically filled by an NP, though a content clause is also possible.
    \item Since constituency order in modern English is largely fixed,
    \prettyref{fig:subject-predicate} means the subject is before the predicate in a canonical clause.
    Various syntactic processes can alter this, though.
    \item The predicator agrees with the subject.
    \item Pronoun subjects receive the nominative case.
    \item  
\end{itemize}

\subsubsection{The object}

\subsection{Ditransitive clauses and the G, T typology}\label{sec:gt-typology}

\subsubsection{Problems about G and T}

The next question is what the same line of reasoning has to say about ditransitive clauses. 
From the semantic aspect,
most ditransitive verbs have A, G, T arguments (G = goal-like, T = theme-like).
The most agentive argument (usually the highest position in \vP) 
has stable cross-linguistic similarity with the transitive A argument, % TODO
and there is no need to posit a separate label for it: it is wise to use A,
just like in a nominative-accusative language, 
there is no much need to use the A label: it can be merged into S.
The real problem is about internal arguments G and T.
Are they legit labels, i.e. are their syntactic behavior stable with different verbs?
A good counterexample is the experiencer role and the stimulus role:
contrast between them do not have much syntactic significance,
since some verbs' experiencers work like some other verbs' stimulus,
and this is why we use A and O to benchmark the behavior of the experiencer and the stimulus,
and not inversely.

Another (orthogonal) question is 
whether G and T (stable syntactic notion or not) have any similarity with the object position. 

\subsubsection{The extended argument approach}\label{sec:blt-e-argument}

One way to deal with ditransitive verbs (and similar monotransitive verbs with a non-core complement)
is the S, A, O, E framework in \ac{blt}%
\footnote{
    It should be noted that the \ac{blt} descriptive framework does not require such an analysis.
    The notation of E argument is preferred by Dixon,
    which other linguists embracing \ac{blt} do not necessarily subscribe.
},
where the less monotransitive object-like argument is labeled as E.
Consider, for example, the example in \ac{blt} \citesec{3.3} (6) and (7),
with semantic role labels replaced by ones in \ac{cgel} \citesec{4.2.2}:
\begin{exe}
    \ex \label{ex:john-gave-goods-to-charity} 
    John gave [all his goods]_{\text{O, theme}} [to charity]_{\text{E, goal}}
    \ex \label{ex:john-gave-student-book} 
    John gave [his favorite student]_{\text{O, goal}} [some books]_{\text{E, theme}}
\end{exe}
The PP \corpus{to charity} and NP \corpus{some books} have similar syntactic behaviors,
and in a similar manner, \corpus{all his goods} and \corpus{his facorite students} 
form another group of arguments with similar syntactic appearance.
One piece of evidence is the constituent order.
Another piece is passivization availability:
the first two cannot be promoted to the subject position in passivization,
while the latter two can.
It is, therefore, reasonable to name the group containing 
\corpus{all his goods} and \corpus{his facorite students} 
with the label O,
and name the group containing
\corpus{to charity} and \corpus{some books}
with the label E.
Note, however, there are some heterogeneity in each group:
\corpus{all his goods} is a theme while \corpus{his favorite student} is a goal,
but they have similar syntactic properties.
So it seems the division between O and E is useful in English 
while the division between G and T is not:
no goal-like and theme-like semantic role classes with stable syntactic appearance 
can be established.

% TODO: E argument是否单独出现?是否有He speaks with a very quick pace 这样,但是后面的PP是complement的动词? 
% TODO:He treats me with kindness中,kindness应该是E论元,因此V-O-E论元结构不只有一种,依照论元种类的不同,有多种clausal structure coding的方式

\subsubsection{The direct object/indirect object approach}\label{sec:direct-indirect}

However, there is also evidence for a stable G-T contrast.
This, essentially, is the approach taken in \ac{cgel}. 
Applying criteria in \ac{cgel} \citesec{4.4.3},
\eqref{ex:john-gave-goods-to-charity} and \eqref{ex:john-gave-student-book}
are to be labeled as 
\begin{exe}
    \ex \label{ex:john-gave-goods-to-charity-2} 
    John gave [all his goods]_{\text{O^{\text{d}}, theme}} [to charity]_{\text{prepositional complement, goal}}
    \ex \label{ex:john-gave-student-book-2} 
    John gave [his favorite student]_{\text{O^{\text{i}}, goal}} [some books]_{\text{O^{\text{d}}, theme}}
\end{exe}
\eqref{ex:john-gave-goods-to-charity-2} is labeled according to the rule 
that indirect objects are nowhere to be found in a clause without a direct object,
and \eqref{ex:john-gave-student-book-2} is labeled according to 
\ac{cgel} \citesec{4.3} [8].
If these labeling rules stand, 
then a stable syntactic contrast between G and T can be established:
the T argument is always the direct object,
while the G argument is less object-like:
it is either an indirect object or a prepositional complement.

\ac{cgel} \citesec{4.4.3} provides ample argumentation for the analysis shown in 
\eqref{ex:john-gave-goods-to-charity-2} and \eqref{ex:john-gave-student-book-2}.
Though \corpus{all his goods} and \corpus{his favorite student} 
have the passivization and constituent order properties of typical monotransitive objects,
only \corpus{all his goods} enjoys the full range of monotransitive object properties.
In the ditransitive example \eqref{ex:john-gave-student-book-2},
\corpus{some books} behaves like the monotransitive object 
in object postposing and preposing, the predicative adjunct construction, and controlling.
Thus the position of \corpus{some books} has more object properties 
than the position of \corpus{his favorite student}.

\subsubsection{The typology of G and T}\label{sec:g-t-typology}

To summarize, no complement type can be merged with another  
in the three types of VPs having been considered now,
namely V-O, V-G-T, and V-T-pG,
except the T argument in a V-T-pG clause and the monotransitive object.
\prettyref{fig:english-object} visualizes the resemblance 
of these complement types with the prototypical monotransitive direct object 
according to the six criteria listed in \ac{cgel} \citesec{4.4.3}.
The T argument in V-T-pG has almost identical properties
with O in a V-O construction.
Both of them, therefore, are put under the category of monotransitive object.
For the rest three complement types,
G in the V-G-T construction is similar to the monotransitive object 
in terms of passivization and constituent order,
though when the G is the beneficiary,
passivization is marginally acceptable (\ac{cgel} \citesec{4.3.3}, [10ii]),
so in \prettyref{fig:english-object} I slightly lower the G in V-G-T point.
(And for the same reason, since T in V-G-T can be marginally passivized 
-- see \ac{cgel} \citesec{4.3.3}, [10iii] --
the T in V-G-T point is slightly raised.)
If passivization and constituent order are considered as the decisive factors to distinguish complement types,
then we get the classification in \prettyref{sec:blt-e-argument}.
On the other hand, if the rest four factors are emphasized,
then the classification in \prettyref{sec:direct-indirect} works,
where G arguments have uniform behaviors 
and so do T arguments,
and we find T arguments are largely similar to O arguments.
When all the six factors are considered,
the total distance between the monotransitive object (i.e. the O argument)
and T in V-G-T is smaller than 
the distance between the monotransitive object and G in V-G-T,
and the former two are clustered into the \term{direct object}.
Finally, the direct object is clustered with G in V-G-T 
-- or the \term{indirect object}, correspondingly -- 
and we get the final object concept.
G in V-T-pG is far from the rest four points 
and hence is placed out of the range of objects.

So here we see why \term{object} is a useful concept in English grammar,
at least among V-O, V-G-T, and V-T-pG clauses:
it can be defined both via form (i.e. no preposition and not predicative)
and via function (the aforementioned six factors),
and the two definitions happen to include the same complement types.
If we try to doing away with the abstract concept of \term{object}
and only keep notions like monotransitive object,
then the coincidence has the following equivalence formulation:
a complement slot prototypically filled by non-predicative NPs
always share some or all grammatical properties
about passivization, canonical constituent order,
preposing and postposing, gap controlling, 
and being able to be modified by a predicative adjunct. 

There are other constructions in which an object position can be distinguished,
including PPs and clauses with both object and predicative complement. % TODO
Whether in these constructions the term \term{object} still hints 
certain resemblance with the prototypically monotransitive object 
is a question to be answered when these constructions are discussed about.

\begin{figure}
    \centering
    

\tikzset{every picture/.style={line width=0.3pt}} %set default line width to 0.75pt        

\begin{tikzpicture}[x=0.75pt,y=0.75pt,yscale=-0.8,xscale=0.8]
%uncomment if require: \path (0,500); %set diagram left start at 0, and has height of 500

%Straight Lines [id:da5593585477377434] 
\draw    (107,391) -- (506,391) ;
\draw [shift={(508,391)}, rotate = 180] [fill={rgb, 255:red, 0; green, 0; blue, 0 }  ][line width=0.08]  [draw opacity=0] (12,-3) -- (0,0) -- (12,3) -- cycle    ;
%Straight Lines [id:da7245999585667466] 
\draw    (107,391) -- (107,97.59) ;
\draw [shift={(107,95.59)}, rotate = 90] [fill={rgb, 255:red, 0; green, 0; blue, 0 }  ][line width=0.08]  [draw opacity=0] (12,-3) -- (0,0) -- (12,3) -- cycle    ;
%Straight Lines [id:da8964896003323979] 
\draw  [dash pattern={on 4.5pt off 4.5pt}]  (48,250) -- (494,250) ;
%Straight Lines [id:da9625166392954305] 
\draw  [dash pattern={on 4.5pt off 4.5pt}]  (299,442.59) -- (299,120.59) ;
%Shape: Ellipse [id:dp7384360040390137] 
\draw  [color={rgb, 255:red, 155; green, 155; blue, 155 }  ,draw opacity=1 ][fill={rgb, 255:red, 155; green, 155; blue, 155 }  ,fill opacity=0.2 ] (343,133.43) .. controls (343,99.46) and (375.46,71.93) .. (415.5,71.93) .. controls (455.54,71.93) and (488,99.46) .. (488,133.43) .. controls (488,167.39) and (455.54,194.93) .. (415.5,194.93) .. controls (375.46,194.93) and (343,167.39) .. (343,133.43) -- cycle ;
%Shape: Ellipse [id:dp035938837693224146] 
\draw  [color={rgb, 255:red, 155; green, 155; blue, 155 }  ,draw opacity=1 ][fill={rgb, 255:red, 155; green, 155; blue, 155 }  ,fill opacity=0.2 ] (308,194.43) .. controls (308,120.14) and (355.46,59.93) .. (414,59.93) .. controls (472.54,59.93) and (520,120.14) .. (520,194.43) .. controls (520,268.71) and (472.54,328.93) .. (414,328.93) .. controls (355.46,328.93) and (308,268.71) .. (308,194.43) -- cycle ;
%Shape: Ellipse [id:dp04989243684342659] 
\draw  [color={rgb, 255:red, 155; green, 155; blue, 155 }  ,draw opacity=1 ][fill={rgb, 255:red, 155; green, 155; blue, 155 }  ,fill opacity=0.2 ] (142,194.93) .. controls (142,168.97) and (174.68,147.93) .. (215,147.93) .. controls (255.32,147.93) and (288,168.97) .. (288,194.93) .. controls (288,220.88) and (255.32,241.93) .. (215,241.93) .. controls (174.68,241.93) and (142,220.88) .. (142,194.93) -- cycle ;
%Shape: Polygon Curved [id:ds6446496039380643] 
\draw  [color={rgb, 255:red, 155; green, 155; blue, 155 }  ,draw opacity=1 ][fill={rgb, 255:red, 155; green, 155; blue, 155 }  ,fill opacity=0.2 ] (301,63.26) .. controls (349,45.26) and (450,5.26) .. (505,83.26) .. controls (560,161.26) and (568,275.59) .. (468,335.93) .. controls (368,396.26) and (124,281.26) .. (120,207.26) .. controls (116,133.26) and (253,81.26) .. (301,63.26) -- cycle ;

% Text Node
\draw (131,92.59) node [anchor=south] [inner sep=0.75pt]   [align=left] {passivization,\\constituent order};
% Text Node
\draw (510,391) node [anchor=west] [inner sep=0.75pt]   [align=left] {postposing,\\preposing,\\gap controlling,\\predicative adjunct};
% Text Node
\draw (176,172) node [anchor=north west][inner sep=0.75pt]   [align=left] {G in V-G-T};
% Text Node
\draw (174,330) node [anchor=north west][inner sep=0.75pt]   [align=left] {G in V-T-pG};
% Text Node
\draw (372,134) node [anchor=north west][inner sep=0.75pt]   [align=left] {O in V-O};
% Text Node
\draw (372,156) node [anchor=north west][inner sep=0.75pt]   [align=left] {T in V-T-pG};
% Text Node
\draw (366,287) node [anchor=north west][inner sep=0.75pt]   [align=left] {T in V-G-T};
% Text Node
\draw (215,424) node [anchor=north west][inner sep=0.75pt]  [color={rgb, 255:red, 208; green, 2; blue, 27 }  ,opacity=1 ] [align=left] {G};
% Text Node
\draw (397,424) node [anchor=north west][inner sep=0.75pt]  [color={rgb, 255:red, 208; green, 2; blue, 27 }  ,opacity=1 ] [align=left] {O=T};
% Text Node
\draw (89.5,170.5) node [anchor=north west][inner sep=0.75pt]  [color={rgb, 255:red, 245; green, 166; blue, 35 }  ,opacity=1 ,rotate=-90] [align=left] {O};
% Text Node
\draw (89.5,317.5) node [anchor=north west][inner sep=0.75pt]  [color={rgb, 255:red, 245; green, 166; blue, 35 }  ,opacity=1 ,rotate=-90] [align=left] {E};
% Text Node
\draw (32,143) node [anchor=north west][inner sep=0.75pt]  [color={rgb, 255:red, 245; green, 166; blue, 35 }  ,opacity=1 ,rotate=-90] [align=left] {The extended argument approach};
% Text Node
\draw (229,462) node [anchor=north west][inner sep=0.75pt]  [color={rgb, 255:red, 208; green, 2; blue, 27 }  ,opacity=1 ] [align=left] {The S, A, O, G, T approach};
% Text Node
\draw (363,89) node [anchor=north west][inner sep=0.75pt]  [color={rgb, 255:red, 155; green, 155; blue, 155 }  ,opacity=1 ] [align=left] {monotransitive \\object};
% Text Node
\draw (370,225) node [anchor=north west][inner sep=0.75pt]  [color={rgb, 255:red, 155; green, 155; blue, 155 }  ,opacity=1 ] [align=left] {direct object};
% Text Node
\draw (164,201) node [anchor=north west][inner sep=0.75pt]  [color={rgb, 255:red, 155; green, 155; blue, 155 }  ,opacity=1 ] [align=left] {indirect object};
% Text Node
\draw (240,111) node [anchor=north west][inner sep=0.75pt]  [color={rgb, 255:red, 155; green, 155; blue, 155 }  ,opacity=1 ] [align=left] {object};


\end{tikzpicture}

    \caption{Classification of internal complements in English V-O, V-G-T, and V-T-pG clauses.
    The orange labels are discussed in \prettyref{sec:direct-indirect},
    and the red labels are discussed in \prettyref{sec:blt-e-argument}.
    The grey blobs indicate clustering of the points.}
    \label{fig:english-object}
\end{figure}

Typological significance for the two possible analyses is discussed in
\citefootnote{26} in \ac{cgel} \citechap{4}.
Some languages assign the direct object position to the goal-like argument.
Were English such a language, 
it would follow that \eqref{ex:john-gave-student-book} would be the correct analysis,
and since the subcategorization frame of \eqref{ex:john-gave-goods-to-charity} 
prohibits assignment of the object status to the goal argument
(the object cannot be oblique),
in \eqref{ex:john-gave-goods-to-charity},
\corpus{all his goods} would be the subject,
and hence both \eqref{ex:john-gave-goods-to-charity} and \eqref{ex:john-gave-student-book} 
would be correct,
and we see a split G role.
If, however, English belongs to the type of languages 
that assign the direct object position to the theme-like argument,
we immediately get \eqref{ex:john-gave-goods-to-charity-2} and \eqref{ex:john-gave-student-book-2}.
Since the position of \corpus{some books} has more object properties 
than the position of \corpus{his favorite student},
\eqref{ex:john-gave-goods-to-charity-2} and \eqref{ex:john-gave-student-book-2} are preferable,
but \eqref{ex:john-gave-goods-to-charity} and \eqref{ex:john-gave-student-book} also make sense 
to some extents.
This means English belongs to the class which identify T with the monotransitive O, 
but it is not a clear-cut member.

Clausal complementation of ditransitive verbs, therefore, is summarized in \prettyref{fig:ditransitive-gt}.
Note that in the diagram there are four -- not five -- internal complement types,
because the O in V-O is identified with the T in V-G-T.

\begin{figure}
    \centering
    

\tikzset{every picture/.style={line width=0.3pt}} %set default line width to 0.75pt        

\begin{tikzpicture}[x=0.75pt,y=0.75pt,yscale=-0.85,xscale=0.85]
%uncomment if require: \path (0,513); %set diagram left start at 0, and has height of 513

%Straight Lines [id:da5590373900481334] 
\draw [color={rgb, 255:red, 74; green, 144; blue, 226 }  ,draw opacity=1 ][line width=3]    (132,62) -- (157.72,62) ;
%Straight Lines [id:da6034249079442651] 
\draw [color={rgb, 255:red, 80; green, 227; blue, 194 }  ,draw opacity=1 ][line width=3]    (191.72,62) -- (262.72,62) ;
%Straight Lines [id:da3340903560309998] 
\draw [color={rgb, 255:red, 184; green, 233; blue, 134 }  ,draw opacity=1 ][line width=3]    (297.72,62) -- (426.72,62) ;
%Curve Lines [id:da7559551700584779] 
\draw [color={rgb, 255:red, 80; green, 227; blue, 194 }  ,draw opacity=1 ]   (257.72,109.97) .. controls (258.72,168.97) and (231.72,155.97) .. (232.72,211.97) ;
%Curve Lines [id:da6470497609004202] 
\draw [color={rgb, 255:red, 80; green, 227; blue, 194 }  ,draw opacity=1 ]   (203.72,109.97) .. controls (204.72,168.97) and (92.72,250.94) .. (82.72,395.94) ;
%Curve Lines [id:da9535821756090674] 
\draw [color={rgb, 255:red, 74; green, 144; blue, 226 }  ,draw opacity=1 ]   (144.72,109.97) .. controls (145.72,168.97) and (77.72,291.94) .. (82.72,395.94) ;
%Curve Lines [id:da4943116446528697] 
\draw [color={rgb, 255:red, 184; green, 233; blue, 134 }  ,draw opacity=1 ]   (312.72,108.97) .. controls (313.72,167.97) and (117.72,178.83) .. (82.72,395.94) ;
%Curve Lines [id:da034641113672127855] 
\draw [color={rgb, 255:red, 248; green, 231; blue, 28 }  ,draw opacity=1 ]   (365.72,108.97) .. controls (366.72,167.97) and (226.72,169.97) .. (232.72,211.97) ;
%Curve Lines [id:da5196681080395307] 
\draw [color={rgb, 255:red, 248; green, 231; blue, 28 }  ,draw opacity=1 ]   (419.72,108.97) .. controls (420.72,167.97) and (594.72,167.97) .. (600.72,209.97) ;
%Straight Lines [id:da23388092728957277] 
\draw [color={rgb, 255:red, 248; green, 231; blue, 28 }  ,draw opacity=1 ][line width=3]    (297.72,54) -- (426.72,54) ;
%Curve Lines [id:da13093524389220645] 
\draw [color={rgb, 255:red, 184; green, 233; blue, 134 }  ,draw opacity=1 ]   (365.72,108.97) .. controls (366.72,167.97) and (359.72,154.97) .. (360.72,210.97) ;
%Curve Lines [id:da2405074684879065] 
\draw [color={rgb, 255:red, 184; green, 233; blue, 134 }  ,draw opacity=1 ]   (419.72,108.97) .. controls (420.72,167.97) and (485.72,154.97) .. (486.72,210.97) ;
%Curve Lines [id:da5399897307832593] 
\draw [color={rgb, 255:red, 155; green, 155; blue, 155 }  ,draw opacity=1 ]   (231,289) .. controls (229.72,317.57) and (309.72,293.57) .. (306.72,345.57) ;
%Curve Lines [id:da43486300562119107] 
\draw [color={rgb, 255:red, 155; green, 155; blue, 155 }  ,draw opacity=1 ]   (363.72,289.94) .. controls (362.43,318.5) and (309.72,293.57) .. (306.72,345.57) ;
%Curve Lines [id:da9115056023887791] 
\draw [color={rgb, 255:red, 155; green, 155; blue, 155 }  ,draw opacity=1 ]   (306,377) .. controls (306.72,400.94) and (390.72,380.94) .. (389.72,410.94) ;
%Curve Lines [id:da27935753858234835] 
\draw [color={rgb, 255:red, 155; green, 155; blue, 155 }  ,draw opacity=1 ]   (487.72,287.57) .. controls (491.72,337.57) and (391.72,335.94) .. (389.72,410.94) ;
%Curve Lines [id:da7115308344150204] 
\draw [color={rgb, 255:red, 155; green, 155; blue, 155 }  ,draw opacity=1 ] [dash pattern={on 4.5pt off 4.5pt}]  (231,289) .. controls (221.72,313.57) and (511.72,360.57) .. (508.72,412.57) ;
%Curve Lines [id:da31095802179111853] 
\draw [color={rgb, 255:red, 155; green, 155; blue, 155 }  ,draw opacity=1 ] [dash pattern={on 4.5pt off 4.5pt}]  (487.72,287.57) .. controls (486.43,316.13) and (507.72,356.57) .. (508.72,412.57) ;
%Curve Lines [id:da630655999182743] 
\draw [color={rgb, 255:red, 155; green, 155; blue, 155 }  ,draw opacity=1 ] [dash pattern={on 4.5pt off 4.5pt}]  (363.72,289.94) .. controls (362.43,318.5) and (627.72,357.57) .. (624.72,409.57) ;
%Curve Lines [id:da5122649297059649] 
\draw [color={rgb, 255:red, 155; green, 155; blue, 155 }  ,draw opacity=1 ] [dash pattern={on 4.5pt off 4.5pt}]  (601.72,290.57) .. controls (600.43,319.13) and (623.72,353.57) .. (624.72,409.57) ;

% Text Node
\draw (140,78) node [anchor=north west][inner sep=0.75pt]   [align=left] {S};
% Text Node
\draw (197,78) node [anchor=north west][inner sep=0.75pt]   [align=left] {A};
% Text Node
\draw (249,78) node [anchor=north west][inner sep=0.75pt]   [align=left] {O};
% Text Node
\draw (304,78) node [anchor=north west][inner sep=0.75pt]   [align=left] {A};
% Text Node
\draw (412,78) node [anchor=north west][inner sep=0.75pt]   [align=left] {G};
% Text Node
\draw (360,78) node [anchor=north west][inner sep=0.75pt]   [align=left] {T};
% Text Node
\draw (68,414.5) node [anchor=north west][inner sep=0.75pt]   [align=left] {subject};
% Text Node
\draw (182,230.75) node [anchor=north west][inner sep=0.75pt]   [align=left] {\begin{minipage}[lt]{68.99pt}\setlength\topsep{0pt}
\begin{center}
monotransitive\\object
\end{center}

\end{minipage}};
% Text Node
\draw (328,220.25) node [anchor=north west][inner sep=0.75pt]   [align=left] {\begin{minipage}[lt]{53.13pt}\setlength\topsep{0pt}
\begin{center}
ditransitive\\direct\\object
\end{center}

\end{minipage}};
% Text Node
\draw (462,230.75) node [anchor=north west][inner sep=0.75pt]   [align=left] {\begin{minipage}[lt]{36.67pt}\setlength\topsep{0pt}
\begin{center}
indirect\\object
\end{center}

\end{minipage}};
% Text Node
\draw (560,220.25) node [anchor=north west][inner sep=0.75pt]   [align=left] {\begin{minipage}[lt]{60.22pt}\setlength\topsep{0pt}
\begin{center}
specified\\prepositional\\complement
\end{center}

\end{minipage}};
% Text Node
\draw (260,355) node [anchor=north west][inner sep=0.75pt]   [align=left] {direct object};
% Text Node
\draw (374,414.5) node [anchor=north west][inner sep=0.75pt]   [align=left] {object};
% Text Node
\draw (469,415) node [anchor=north west][inner sep=0.75pt]   [align=left] {Dixon's O};
% Text Node
\draw (595,415) node [anchor=north west][inner sep=0.75pt]   [align=left] {Dixon's E};


\end{tikzpicture}

    \caption{English alignment concerning S, A, O, G, T. Each color of lines means one canonical clause which codes a type of argument structure.}
    \label{fig:ditransitive-gt}
\end{figure}

\subsection{Predicative complements and the CS, CC typology}

\prettyref{sec:sao-typology} and \prettyref{sec:gt-typology} 
have classified external and internal complements for verbs with one or two complements
which are filled prototypically by NPs,
and the result is shown in \prettyref{fig:ditransitive-gt}. 
Now it is time to start consider predicative complements.
Are there subclasses in the predicative complement class, 
like the subclassification of objects (\prettyref{fig:english-object})?


\subsection{Adjuncts and peripheral arguments}\label{sec:adjuncts-classification}

Adjuncts -- and adjunct-like complements -- are discussed in \ac{cgel} \citechap{8}.
% TODO: adjunct的位置:动词前,动词后,话题;话题化在哪里?
% 特别是pre-subject adjunct是不是在nucleus里面似乎没有定论——见8.20

\subsection{Distinguish complements from adjuncts}\label{sec:recognizing-complement-clause}

\subsubsection{Licensing and obligatoriness}

If something is obligatory, is is always a complement.
Complements can also be optional, though.


\subsubsection{Category}

\subsection{Verb idioms and fossilization}

\subsection{Summary}



\section{The verb}

Discussion on the verb is usually the beginning of the discussion on the clause structure in a grammar,
since in the minimal case, 
the predicate can be just a single verb.
This section deals with subcategories in the verb category,
and how the verb changes its form according to the syntactic environment.

\subsection{Syncretism in English}

Syntactic categories relevant to verb morphology in English include 
finiteness (\prettyref{sec:finiteness}),
tense, aspect and mood (\prettyref{sec:tense-aspect}),
voice (\prettyref{sec:voice}),
and agreement with the subject (\prettyref{sec:agreement}).
These syntactic categories are all prototypical ones in Indo-European languages.
Note, however, in \ac{cgel} these features are not considered as the features of the verb itself,
but the clause. 
If these features are placed on the verb, 
then the verb \corpus{play} in a subjunctive clause is a homonym of 
the verb \corpus{play} in an indicative clauses,
but anyone will consider the two identical inflectional forms.

This does not mean these features do not play any role in distinguishing inflectional forms.
Another extreme of morphological analysis is to 
inserting all inflectional forms with the same surface appearance into one inflectional class.
This is also not the approach taken in \ac{cgel}.
To see the problem with this approach, consider the example of Latin fourth declension.
What is in common between the genitive singular and the nominative plural?
It definitely makes no sense to regard then the same.

The criteria to distinguish inflectional distinctions are shown in \ac{cgel} \citesec{5.1.2} [5].
It rejects the approach placing features like finiteness, tense, etc. on the verb,
which is essentially the traditional grammar approach%
\footnote{
    Note, however, the traditional inflectional paradigm is still useful though cumbersome:
    it visualizes possible environments verbs may appear in.
} (\citechap{3} \citefootnote{1}):
the uniformness of the finite part of the so-called inflectional paradigm of English verb
means this part should be compressed into just three inflectional forms.
It also rejects the approach positing minimal inflectional distinctions.
It is often said that English has five verbal inflectional forms:
\corpus{take}, \corpus{takes}, \corpus{took}, \corpus{taking}, \corpus{token}. 
But modal auxiliary verbs do not appear in infinitives:
the present tense form is therefore to be split into two
according to \ac{cgel} \citesec{5.1.2} [5],
one appears in a finite environment,
the other appears in a non-finite environment.
In the latter environment, modal auxiliary verbs are constantly absent,
therefore constituting a stable contrast between modal auxiliary verbs and lexical verbs,
while in the former environment there is no contrast between the two.
This split may be viewed as based on the finiteness category,
and not purely on the surface appearance of the verb.

\ac{cgel} \citesec{5.1.2} [5] accidentally rejects the analysis of Latin noun declension
that combines the genitive singular and the nominative plural,
because the identification of the two does not work for, say, the third declension,
so with the same logic that makes distinction between the finite and infinite \corpus{take}.
But suppose we have an imaginary language in which in all noun declension classes,
the genitive singular and the nominative plural have the same surface realization.
In this language, \ac{cgel} \citesec{5.1.2} [5] is not sufficient 
to reject temptation of analyzing the two as one inflectional form.
Fortunately, this is not the case in English, and I do not proceed more on 
inflectional paradigms.

\subsection{Agreement with the subject}\label{sec:verb-agreement}

\subsection{What to see in the dictionary}\label{sec:verb-dict}

This section is about what to investigate about a verb when looking it up in a dictionary.
In other words, it is about how verbs are to be classified.

\subsubsection{Complementation}\label{sec:dict-comp}

One criterion of verb classification is the valency information.
In \ac{cgel}, \term{valency} is defined purely as the number of complements.
Some authors use \term{valency} to denote the whole complementation type,
including the syntactic function of each complement.

Prepositional verbs are classified in \ac{cgel} \citesec{4.6.1.2}.
Verbs with several patterns of complementation are discussed in \ac{cgel} \citesec{4.8}.

\section{Preposition}

Some prepositions have complementizer-like properties,
e.g. allowing prenucleus constructions (\ac{cgel} \citesec{7.4.4}).

\section{Information packaging}

\subsection{The passive voice}

\subsubsection{Complements externalized}\label{sec:externalized-passive}

\ac{cgel} \citesec{16.10.1.2}

\bibliographystyle{plainnat}
\bibliography{cambridge}

\end{document}