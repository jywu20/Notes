\PassOptionsToPackage{table}{xcolor}
\documentclass[a4paper, oneside, 12pt]{report}

\usepackage[T1]{fontenc}
\usepackage{libertinus}
\usepackage{geometry}
\usepackage{float}
\usepackage{titling}
\usepackage{titlesec}
\usepackage{paralist}
\usepackage{footnote}
\usepackage[colorinlistoftodos]{todonotes}
\usepackage[inline]{enumitem}
\usepackage{amsmath, amsthm}
\usepackage{gb4e}
\noautomath
\usepackage{bbm}
\usepackage{textcomp}
\usepackage{soul}
\usepackage{graphicx}
\usepackage{siunitx}
\usepackage{tikz}
\usepackage[ruled, vlined, linesnumbered, noend]{algorithm2e}
\usepackage[colorlinks, citecolor = purple, bookmarksnumbered]{hyperref} % linkcolor=black, anchorcolor=black, citecolor=black, filecolor=black
\usepackage[most]{tcolorbox}
\usepackage{caption}
\usepackage{subcaption}
\usepackage{booktabs}
\usepackage{multirow}
\usepackage[figuresright]{rotating}
\usepackage{acro}
\usepackage[round]{natbib} 
\usepackage{prettyref}

\geometry{left=3.18cm,right=3.18cm,top=2.54cm,bottom=2.54cm}
\titlespacing{\paragraph}{0pt}{1pt}{10pt}[20pt]
\setlength{\droptitle}{-5em}

\DeclareMathOperator{\timeorder}{\mathcal{T}}
\DeclareMathOperator{\diag}{diag}
\DeclareMathOperator{\legpoly}{P}
\DeclareMathOperator{\primevalue}{P}
\DeclareMathOperator{\sgn}{sgn}
\newcommand*{\ii}{\mathrm{i}}
\newcommand*{\ee}{\mathrm{e}}
\newcommand*{\const}{\mathrm{const}}
\newcommand*{\suchthat}{\quad \text{s.t.} \quad}
\newcommand*{\argmin}{\arg\min}
\newcommand*{\argmax}{\arg\max}
\newcommand*{\normalorder}[1]{: #1 :}
\newcommand*{\pair}[1]{\langle #1 \rangle}
\newcommand*{\fd}[1]{\mathcal{D} #1}

\newcommand*{\citesec}[1]{\S~{#1}}
\newcommand*{\citechap}[1]{Ch~{#1}}
\newcommand*{\citefig}[1]{Fig.~{#1}}
\newcommand*{\citetable}[1]{Table~{#1}}
\newcommand*{\citepage}[1]{p.~{#1}}
\newcommand*{\citepages}[1]{pp.~{#1}}
\newcommand*{\citefootnote}[1]{fn.~{#1}}
\newcommand*{\citechapsec}[2]{\citechap{#1}.\citesec{#2}}

\newrefformat{sec}{\citesec{\ref{#1}}}
\newrefformat{fig}{\citefig{\ref{#1}}}
\newrefformat{tbl}{\citetable{\ref{#1}}}
\newrefformat{chap}{\citechap{\ref{#1}}}
\newrefformat{fn}{\citefootnote{\ref{#1}}}
\newrefformat{box}{Box~\ref{#1}}
\newrefformat{ex}{\ref{#1}}


% color boxes

\tcbuselibrary{skins, breakable, theorems}

\AtBeginEnvironment{infobox}{\small}
\AtBeginEnvironment{theorybox}{\small}

\newtcbtheorem[number within=chapter]{infobox}{Box}{
    enhanced,
    boxrule=0pt,
    colback=blue!5,
    colframe=blue!5,
    coltitle=blue!50,
    borderline west={4pt}{0pt}{blue!65},
    sharp corners,
    fonttitle=\bfseries, 
    breakable,
    before upper={\parindent15pt\noindent}}{box}
\newtcbtheorem[number within=chapter, use counter from=infobox]{theorybox}{Box}{
    enhanced,
    boxrule=0pt,
    colback=orange!5, 
    colframe=orange!5, 
    coltitle=orange!50,
    borderline west={4pt}{0pt}{orange!65},
    sharp corners,
    fonttitle=\bfseries, 
    breakable,
    before upper={\parindent15pt\noindent}}{box}
\newtcbtheorem[number within=chapter, use counter from=infobox]{learnbox}{Box}{
    enhanced,
    boxrule=0pt,
    colback=green!5,
    colframe=green!5,
    coltitle=green!50,
    borderline west={4pt}{0pt}{green!65},
    sharp corners,
    fonttitle=\bfseries, 
    breakable,
    before upper={\parindent15pt\noindent}}{box}

% Shorthands
\newcommand*{\concept}[1]{\textbf{#1}}
\newcommand*{\term}[1]{\emph{#1}}
\newcommand{\form}[1]{\emph{#1}}

\newcommand{\redp}{\textasciitilde}

\newcommand{\deictictime}{T$_{\text{d}}$}
\newcommand{\referredtime}{T$_{\text{r}}$}
\newcommand{\orientationtime}{T$_{\text{o}}$}

\DeclareAcronym{blt}{short = BLT, long = Basic Linguistic Theory}
\DeclareAcronym{cgel}{short = CGEL, long = The Cambridge Grammar of the English Language}
\DeclareAcronym{dm}{short = DM, long = Distributed Morphology}
\DeclareAcronym{tag}{long = Tree-adjoining grammar, short = TAG}
\DeclareAcronym{sfp}{long = sentence-final particle, short = \textsc{sfp}}
\DeclareAcronym{np}{long = noun phrase, short = NP}
\DeclareAcronym{vp}{long = verb phrase, short = VP}
\DeclareAcronym{pp}{long = preposition phrase, short = PP}
\DeclareAcronym{advp}{long = adverb phrase, short = AdvP}
\DeclareAcronym{cls}{long = classifier, short = CLS}
\DeclareAcronym{dist}{long = distal, short = DIST}
\DeclareAcronym{prox}{long = proximate, short = PROX}
\DeclareAcronym{dem}{long = demonstrative, short = DEM}
\DeclareAcronym{classify}{long = classifier, short = \textsc{cl}}
\DeclareAcronym{dur}{long = durative, short = DUR}
\DeclareAcronym{neg}{long = negative, short = \textsc{neg}}
\DeclareAcronym{cc}{long = copular complement, short = CC}
\DeclareAcronym{cs}{long = copular subject, short = CS}
\DeclareAcronym{tame}{long = {tense, aspect, mood, evidentiality}, short = TAME}
\DeclareAcronym{past}{long = past, short = PST}
\DeclareAcronym{nonpast}{long = non-past, short = NPST}
\DeclareAcronym{present}{long = present, short = PRES}
\DeclareAcronym{progressive}{long = progressive, short = \textsc{poss}}
\DeclareAcronym{perfect}{long = perfect, short = \textsc{perf}}
\DeclareAcronym{passive}{long = passive, short = \textsc{pass}}
\DeclareAcronym{copula}{long = copula, short = COP}
\DeclareAcronym{possessive}{long = possessive, short = \textsc{poss}}
\DeclareAcronym{coca}{long = Corpus of Contemporary American English, short = COCA}

\newcommand{\asis}[1]{\textsc{#1}}
\newcommand{\oneof}[1]{{#1}}
\newcommand*{\homo}[2]{#1$_{\text{#2}}$}
\newcommand{\category}[1]{\textsc{#1}}
\newcommand{\formcat}[1]{\textsc{#1}}
\newcommand{\emptymorpheme}{$\emptyset$}
\newcommand*{\fromto}[2]{\langle {#1}, {#2} \rangle}

\newcommand{\alignment}{\href{../alignment/alignment.pdf}{my notes about alignment}}
\newcommand{\method}{\href{../methodology/glossing.pdf}{this note about my understanding of descriptive grammars}}

\newcommand{\ala}{à la}
\newcommand{\translate}[1]{`#1'}
\newcommand{\vP}{\textit{v}P}

% Make subsubsection labeled
\setcounter{secnumdepth}{4}
\setcounter{tocdepth}{4}
% reset example counter every chapter (but do not include the chapter number to the label)
\counterwithin{exx}{chapter} 

% Reference formats
\renewcommand{\bibname}{References}
\setcitestyle{aysep={}} 

% List format
\setlist[enumerate,1]{label=\alph*\upshape)}

% Source of examples 
\newcommand{\source}[1]{\hspace{\fill}\mbox{}\linebreak[0]\hspace*{\fill}\mbox{(\small #1)}}

\title{Notes about Moloko}
\author{Jinyuan Wu}

\begin{document}

\automath

\maketitle

\chapter{Noun phrase}

Just like what is seen in English and many other languages,
the Moloko noun phrase can be divided into two regions:
the inner region containing the head noun 
and modifiers which together with the head noun 
describe a certain \emph{property} 
(or, considering this, the constituents should be called \term{complements}, 
not \term{modifiers}),
and the outer region specifying the exact reference of the noun phrase.

In Moloko both regions seem very restricted in complexity.
the inner region consists of either one single noun or maybe a pronoun, 
or the so-called permanent attribution construction 
\citep[\citesec{5.4.2}]{friesen2017grammar},
in which two nouns -- or one noun and one normalized clause -- 
are placed together without any markers, 
and the second noun indicates something about the identity 
or some permanent attribute of the first noun.
The two are regarded as a whole by any constructions in the external region.

The external region either takes the form described by  
\citet[\citepage{145}, \citefig{5.1}]{friesen2017grammar},
where the ``head noun'' is to be understood as 
the aforementioned internal region,
or is a genitive construction \citep[\citesec{5.4.1}]{friesen2017grammar}.
All constituents in the external region follow the internal region.
The genitive construction is in conflict with all other constructions in the external region; 
therefore if any external region modifier appears after the genitive marker \form{a},
it has to modify the genitive noun, not the head noun
\citep[\citepage{159}]{friesen2017grammar}.
This is probably because the genitive noun phrase 
is regarded as a big fused-function determiner 
covering functions of all other modifiers in the external region.
\todo{
    Is the genitive NP also in conflict with the number clitic?
}
In the rest external field constructions, 
the quantifier is in conflict with the numeral 
or the derived adjective construction (see below),
again possibly because the former is a fused-function construction, 
which is assumed to automatically assign a specific reference to the noun phrase.

The derived adjective construction is a very interesting construction.
In the most general sense (as in \citet[\citefig{5.1}]{friesen2017grammar}),
it includes any noun phrase with \form{ga} in the end.
There are two usages of \form{ga}.
One is for a sense of definiteness and emphasis \citep[\citesec{5.3.2}]{friesen2017grammar}.
\todo{
    Is this usage compatible with genitive or quantifier?
}
In the other usage, 
another noun N_2 is placed before \form{ga} 
and after all other external region constituents
after the head noun N_1, 
and the N_2 \form{ga} sequence becomes an adjective. 
Only one such adjective position is allowed for each noun phrase; 
my guess would be that this construction is the semantically drifted version of 
the construction where \form{ga} 
licenses a position for N_2 to express
the is \translate{\emph{the} N_1 that has something to do with N_2}.
This usage is the derived adjective construction in the narrow sense 
\citep[\citepages{149-152}]{friesen2017grammar}.
N_2 appears to be another noun phrase in some cases, 
and ``the entire noun phrase is adjectivized''
\citep[\citepage{151, (45)}]{friesen2017grammar}; 
note however the \translate{monkeys and baboons} 
noun phrase is actually a nominal predicate, 
and it's likely that \form{ga} is merely emphasizing it, 
not adjectivize it; 
if in contemporary uses, \form{ga} \emph{can't} adjectivize 
a noun phrase, then the problem of number agreement (see below) 
is no longer a problem.

The number of the noun phrase modifier before \form{ga} 
should be consistent with the number of the head noun; 
whether it is due to semantic reasons  
(and can be violated in extreme conversation contexts),
or is due to morphosyntactic reasons is not clear.
The second explanation is 
kind of subtle if noun phrases can be adjectivized, 
but is still possible if the adjectivized noun phrase 
actually is just a ``nominal'' in the sense of \citet{cgel}
and needs to get its number feature from somewhere else; 
but it's unable to explain why in \citet[\citepage{151, (45)}]{friesen2017grammar},
the two branches in the \translate{monkeys and baboons} phrase are all in plural; 
this doesn't look like agreement.
But if we don't analyze this example as adjectivization, 
then the morphological agreement theory has nothing problematic, 
although the Moloko adjectivization construction is still somehow unusual 
in that the adjective formed still automatically contains a determinative meaning, 
and a noun phrase can't contain two adjectives -- 
or more exactly, two adjectival-determiners.
\todo{
    Whether there are further examples of adjectivization of complex NP.
}

The distinction between the adjectivization construction 
and the permanent attribution construction 
resembles the distinction between two genitive constructions in Semitic languages,
although the markers are completely different. 
Similar phenomena can be observed in Indo-European languages:
in English we have \form{a teacher's desk} (c.f. the permanent attribution construction)
v.s. \form{[that teacher]'s desk} 
(a desk that happens to be occupied by that teacher, 
but not necessarily a desk used by the teacher in classroom; 
c.f. the adjectivization construction); 
in Latin we have the distinction between \form{-icus} adjectives and 
ordinary genitive construction.

\chapter{Clause}

Manner-like or peripheral argument-like adverbs 
are attested \citep[\citepage{110}, (106-108)]{friesen2017grammar}.
There exist several seemingly TAM adverbs,
which only appear at the end of the clause 
and are not active in intensifying derivation,
while peripheral argument-like adverbs frequently undergo such derivation
\citep[\citepages{110-111}, (109-111)]{friesen2017grammar}.
For some reasons, TAM adverbs are considered verb phrase adverbs 
while temporal adverbs expressing time locations like today or tomorrow 
are seen as clause-level adverbs,
possibly because they are a part of the peripheral argument system 
and involve verb-adjunct relations and therefore are ``clause-wide''
while TAM adverbs belong to the verbal system only.
\citep[\citepage308]{friesen2017grammar} 
inconsistently call \form{apazan} \translate{yesterday} 
a temporal noun phrase; 
regardless of the name, 
\translate{today/yesterday/tomorrow}-like temporal adverbs 
seem to be the only part of speech that may appear before the subject,
apart from the \form{na}-introduced old information topics;
probably the two are both because of topicalization,
although for realizational reasons the temporal adverbs don't accept \form{na} following them.
The true order of the adverbs them has to be found 
by their relative positions at the end of the clause.
\todo{
    Find multi-adverb constructions.
}

\bibliographystyle{plainnat}
\bibliography{moloko.bib}

\end{document}