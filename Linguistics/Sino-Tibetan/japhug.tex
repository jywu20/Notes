\documentclass[a4paper, oneside, scheme=plain, 12pt]{report}

\usepackage[T1]{fontenc}
\usepackage{libertinus}
\usepackage{geometry}
\usepackage{float}
\usepackage{titling}
\usepackage{titlesec}
\usepackage{paralist}
\usepackage{footnote}
\usepackage[inline]{enumitem}
\usepackage{amsmath, amsthm}
\usepackage{gb4e}
\noautomath
\usepackage{bbm}
\usepackage{textcomp}
\usepackage{soul}
\usepackage{graphicx}
\usepackage{siunitx}
\usepackage[table,xcdraw]{xcolor}
\usepackage{tikz}
\usepackage[ruled, vlined, linesnumbered, noend]{algorithm2e}
\usepackage{xr-hyper}
\usepackage[colorlinks, citecolor = purple, bookmarksnumbered]{hyperref} % linkcolor=black, anchorcolor=black, citecolor=black, filecolor=black
\usepackage[most]{tcolorbox}
\usepackage{caption}
\usepackage{subcaption}
\usepackage{booktabs}
\usepackage{multirow}
\usepackage[figuresright]{rotating}
\usepackage{acro}
\usepackage[round]{natbib} 
\usepackage{nameref,zref-xr}
\zxrsetup{toltxlabel}
\zexternaldocument*[alignment-]{../alignment/alignment}[alignment.pdf]
\zexternaldocument*[exercise1-]{../Exercise/2021-3}[2021-3.pdf]
\zexternaldocument*[method-]{../methodology/glossing}[glossing.pdf]
\usepackage{prettyref}

\geometry{left=3.18cm,right=3.18cm,top=2.54cm,bottom=2.54cm}
\titlespacing{\paragraph}{0pt}{1pt}{10pt}[20pt]
\setlength{\droptitle}{-5em}

\DeclareMathOperator{\timeorder}{\mathcal{T}}
\DeclareMathOperator{\diag}{diag}
\DeclareMathOperator{\legpoly}{P}
\DeclareMathOperator{\primevalue}{P}
\DeclareMathOperator{\sgn}{sgn}
\newcommand*{\ii}{\mathrm{i}}
\newcommand*{\ee}{\mathrm{e}}
\newcommand*{\const}{\mathrm{const}}
\newcommand*{\suchthat}{\quad \text{s.t.} \quad}
\newcommand*{\argmin}{\arg\min}
\newcommand*{\argmax}{\arg\max}
\newcommand*{\normalorder}[1]{: #1 :}
\newcommand*{\pair}[1]{\langle #1 \rangle}
\newcommand*{\fd}[1]{\mathcal{D} #1}

\newcommand*{\citesec}[1]{\S~{#1}}
\newcommand*{\citechap}[1]{Ch~{#1}}
\newcommand*{\citefig}[1]{Fig.~{#1}}
\newcommand*{\citetable}[1]{Table~{#1}}
\newcommand*{\citepage}[1]{p.~{#1}}
\newcommand*{\citepages}[1]{pp.~{#1}}
\newcommand*{\citefootnote}[1]{fn.~{#1}}
\newcommand*{\citechapsec}[2]{\citechap{#1}.\citesec{#2}}

\newrefformat{sec}{\citesec{\ref{#1}}}
\newrefformat{fig}{\citefig{\ref{#1}}}
\newrefformat{tbl}{\citetable{\ref{#1}}}
\newrefformat{chap}{\citechap{\ref{#1}}}
\newrefformat{fn}{\citefootnote{\ref{#1}}}
\newrefformat{box}{Box~\ref{#1}}
\newrefformat{ex}{\ref{#1}}


% color boxes

\tcbuselibrary{skins, breakable, theorems}

\AtBeginEnvironment{infobox}{\small}
\AtBeginEnvironment{theorybox}{\small}

\newtcbtheorem[number within=chapter]{infobox}{Box}{
    enhanced,
    boxrule=0pt,
    colback=blue!5,
    colframe=blue!5,
    coltitle=blue!50,
    borderline west={4pt}{0pt}{blue!65},
    sharp corners,
    fonttitle=\bfseries, 
    breakable,
    before upper={\parindent15pt\noindent}}{box}
\newtcbtheorem[number within=chapter, use counter from=infobox]{theorybox}{Box}{
    enhanced,
    boxrule=0pt,
    colback=orange!5, 
    colframe=orange!5, 
    coltitle=orange!50,
    borderline west={4pt}{0pt}{orange!65},
    sharp corners,
    fonttitle=\bfseries, 
    breakable,
    before upper={\parindent15pt\noindent}}{box}
\newtcbtheorem[number within=chapter, use counter from=infobox]{learnbox}{Box}{
    enhanced,
    boxrule=0pt,
    colback=green!5,
    colframe=green!5,
    coltitle=green!50,
    borderline west={4pt}{0pt}{green!65},
    sharp corners,
    fonttitle=\bfseries, 
    breakable,
    before upper={\parindent15pt\noindent}}{box}

% Shorthands
\newcommand*{\concept}[1]{\textbf{#1}}
\newcommand*{\term}[1]{\emph{#1}}
\newcommand{\form}[1]{\emph{#1}}

\newcommand{\redp}{\textasciitilde}

\newcommand{\deictictime}{T$_{\text{d}}$}
\newcommand{\referredtime}{T$_{\text{r}}$}
\newcommand{\orientationtime}{T$_{\text{o}}$}

\DeclareAcronym{blt}{short = BLT, long = Basic Linguistic Theory}
\DeclareAcronym{cgel}{short = CGEL, long = The Cambridge Grammar of the English Language}
\DeclareAcronym{dm}{short = DM, long = Distributed Morphology}
\DeclareAcronym{tag}{long = Tree-adjoining grammar, short = TAG}
\DeclareAcronym{sfp}{long = sentence-final particle, short = \textsc{sfp}}
\DeclareAcronym{np}{long = noun phrase, short = NP}
\DeclareAcronym{vp}{long = verb phrase, short = VP}
\DeclareAcronym{pp}{long = preposition phrase, short = PP}
\DeclareAcronym{advp}{long = adverb phrase, short = AdvP}
\DeclareAcronym{cls}{long = classifier, short = CLS}
\DeclareAcronym{dist}{long = distal, short = DIST}
\DeclareAcronym{prox}{long = proximate, short = PROX}
\DeclareAcronym{dem}{long = demonstrative, short = DEM}
\DeclareAcronym{classify}{long = classifier, short = \textsc{cl}}
\DeclareAcronym{dur}{long = durative, short = DUR}
\DeclareAcronym{neg}{long = negative, short = \textsc{neg}}
\DeclareAcronym{cc}{long = copular complement, short = CC}
\DeclareAcronym{cs}{long = copular subject, short = CS}
\DeclareAcronym{tame}{long = {tense, aspect, mood, evidentiality}, short = TAME}
\DeclareAcronym{past}{long = past, short = PST}
\DeclareAcronym{nonpast}{long = non-past, short = NPST}
\DeclareAcronym{present}{long = present, short = PRES}
\DeclareAcronym{progressive}{long = progressive, short = \textsc{poss}}
\DeclareAcronym{perfect}{long = perfect, short = \textsc{perf}}
\DeclareAcronym{passive}{long = passive, short = \textsc{pass}}
\DeclareAcronym{copula}{long = copula, short = COP}
\DeclareAcronym{possessive}{long = possessive, short = \textsc{poss}}
\DeclareAcronym{coca}{long = Corpus of Contemporary American English, short = COCA}

\newcommand{\asis}[1]{\textsc{#1}}
\newcommand{\oneof}[1]{{#1}}
\newcommand*{\homo}[2]{#1$_{\text{#2}}$}
\newcommand{\category}[1]{\textsc{#1}}
\newcommand{\formcat}[1]{\textsc{#1}}
\newcommand{\emptymorpheme}{$\emptyset$}
\newcommand*{\fromto}[2]{\langle {#1}, {#2} \rangle}

\newcommand{\alignment}{\href{../alignment/alignment.pdf}{my notes about alignment}}
\newcommand{\exerciseone}{\href{../Exercise/2021-3.pdf}{this exercise}}
\newcommand{\method}{\href{../methodology/glossing.pdf}{this note about my understanding of descriptive grammars}}

\newcommand{\ala}{à la}
\newcommand{\translate}[1]{`#1'}
\newcommand{\vP}{\textit{v}P}

% Make subsubsection labeled
\setcounter{secnumdepth}{4}
\setcounter{tocdepth}{4}
% reset example counter every chapter (but do not include the chapter number to the label)
\counterwithin{exx}{chapter} 

% Reference formats
\renewcommand{\bibname}{References}
\setcitestyle{aysep={}} 

% List format
\setlist[enumerate,1]{label=\alph*\upshape)}

\title{Reading notes of A Grammar of Japhug}
\author{Jinyuan Wu}

\begin{document}

\maketitle

The theoretical orientation is already well-documented in my notes about English, Latin and Mandarin Chinese.

\chapter{Part of speech}

\section{Noun}

\section{Verb}

Verbs can be regularly formed by denominal derivations
\citet[\citechap{20}]{jacques2021grammar}.
Since an independent adjective class is absent, 
the only two kinds of denominal derivations 
are noun-to-adverb derivations and noun-to-verb derivations,
the former being relative marginal \citet[\citepage{1011}]{jacques2021grammar};
thus the term \term{denominal} can be used specifically 
to refer to noun-to-verb derivations.

\subsection{Grammatical categories}

Japhug has 11 morphologically marked (``primary'') \acs{tame} categories 
\citet[\citetable{21.1}]{jacques2021grammar},
as well as several periphrastic \acs{tame} categories
that are realized by combining 
the marking of one of the above primary \acs{tame} categories
and various forms of the copula \form{ŋu}
\citep[\citepage{1081}]{jacques2021grammar};
here a difference with English or Latin periphrastic conjugation
is what are used in these periphrastic \acs{tame} categories
are \emph{finite} verb forms,
possibly because the periphrastic \acs{tame} categories 
historically comes from finite complement clause constructions.
TODO: is there any constraints on the distribution of participle or infinitive?

Decomposition of these \acs{tame} categories 
in the same way English \form{he [is playing] football} 
is analyzed as ``present (imperfect) progressive'' is not necessary:
TODO: why

The morphological realization of these categories is remarkable
Their main exponents are the alternation of the orientation prefix.
some \acs{tame} categories insert a fixed prefix 
into the orientation prefix slot (TODO: regardless of the lexically determined orientation prefix 
or the semantically significant orientation prefix of a orientable prefix?);
others choose one of the four prefixes that have the same directional meaning 
in \citet[\citetable{15.1}]{jacques2021grammar}. 

\subsection{Wordhood}

\citet[\citechap{11}]{jacques2021grammar} gives 
the realization of the verbal system. 
Whether this complex is to be regarded as one \emph{morphological} or \emph{phonological} word 
is discussed in \citesec{11.6}
in the reference above.
Recognition of wordhood, expectedly, is not self-evident;
\citet{prins2011web} provides an analysis of another rGyalrong language, Jiaomuzu, 
and in this thesis the term \term{verb phrase} 
(i.e. verbal complex in this note) is used,
skipping the discussion on what is a word.
In \citet[\citetable{11.3}]{jacques2021grammar}
four domains are defined using various criteria.

Domain A is defined according to both syntactic and morphological reasons.
What's shown in 
\citetable{11.3} contains all formatives that are relevant to verb inflection,
and they have non-adjacent dependencies,
so strong dependencies exist between them:
these formatives are realized in the same batch 
in clause building.
Now syntactically, the formative \form{-ci} in slit +4 
is selected by some modal prefixes in slot -6,
so the two slots belong to the same system;
on the other hand, outside the +4 and -6 slots 
we only have clitics which clearly belong to systems with higher positions
\citep[\citesec{11.6.2}]{jacques2021grammar},
and thus all -- and only -- formatives in \citetable{11.3}
constitute a syntactic word,
with the same \emph{syntactic} status of a verb-plus-auxiliary verbal complex or a
``verb phrase'' in Dixon's definition (i.e. without internal complements). 
Morphologically, no element is able to intervene 
between two slots in the template, 
so we say this batch is realized as a single morphological word 
instead of a verbal complex.

Domain B is about \emph{obligatoriness}:
thus the +4 slot is not included.
Domain C is defined according to prosodic reasons.

\section{Ideophones}

The category of ideophone occupies mainly manner adverbial positions
\citep[\citesec{10.1.7}]{jacques2021grammar}.
Its main difference with the adverb class 
is its morphology \citep[\citesec{10.1.2}]{jacques2021grammar}
and phonology \citep[\citesec{10.1.5}]{jacques2021grammar}.

\section{Analyzed examples}

TODO: \begin{itemize}
    \item What's the sentence final \form{ŋu}? The dictionary says it's a stative verb.
        What's its argument structure?
        Complement clause construction?
\end{itemize}

The sentence final stative verb \form{ŋu} be.\category{fact} is 
listed as a stative verb in the dictionary
and seems to take the constituents before it 
as a finite complement clause (TODO: or report speech? see the condition on \citepage{1317}),
which is without any explicit complementizer. 
But also see \citepages{1081, }

\chapter{Noun phrase}

One interesting feature of the Japhug comitative 
is it's also considered when deciding the number of an \acs{np}
\citep[\citepage{332}]{jacques2021grammar};
but it's still not prototypically a conjunction \citep[\citepage{420}]{jacques2021grammar}:
the \acs{np} following the comitative marker 
may be omitted, 
agreeing with the fact 
that the head noun of an \acs{np}
can also be dropped \citep[\citepage{425}]{jacques2021grammar}.
(In English this is only possible for clauses:
in informal writing and speech people may start with a sentence with \form{and},
i.e. a conjunction construction 
without the first branch,
but they never do so to an \acs{np}.)
The \acs{np} after the comitative marker can also be relativized.
Thus the comitative suffix is still recognized as a type of 
modification.

\chapter{Clause structure}

\section{The overall structure}

Japhug clauses are verb-final:
core arguments and adverbials are before the verb, 
\citet{jacques2021grammar} doesn't mention verb phrase coordination,
but \citet[\citepage{549}]{prins2011web} mentions 
coordination of two verb phrases sharing the same subject
in a relative language Jiaomuzu,
and therefore the verb phrase layer should be kept??

\bibliographystyle{plainnat}
\bibliography{gyalrong.bib}

\end{document}