\documentclass[a4paper, oneside, 12pt]{report}

\usepackage[T1]{fontenc}
\usepackage{libertinus}
\usepackage{geometry}
\usepackage{float}
\usepackage{titling}
\usepackage{titlesec}
\usepackage{paralist}
\usepackage{footnote}
\usepackage[inline]{enumitem}
\usepackage{amsmath, amsthm}
\usepackage{gb4e}
\noautomath
\usepackage{bbm}
\usepackage{textcomp}
\usepackage{soul}
\usepackage{graphicx}
\usepackage{siunitx}
\usepackage[table,xcdraw]{xcolor}
\usepackage{tikz}
\usepackage[ruled, vlined, linesnumbered, noend]{algorithm2e}
\usepackage{xr-hyper}
\usepackage[colorlinks, citecolor = purple, bookmarksnumbered]{hyperref} % linkcolor=black, anchorcolor=black, citecolor=black, filecolor=black
\usepackage[most]{tcolorbox}
\usepackage{caption}
\usepackage{subcaption}
\usepackage{booktabs}
\usepackage{multirow}
\usepackage[figuresright]{rotating}
\usepackage{acro}
\usepackage[round]{natbib} 
\usepackage{nameref,zref-xr}
\zxrsetup{toltxlabel}
\zexternaldocument*[alignment-]{../alignment/alignment}[alignment.pdf]
\zexternaldocument*[exercise1-]{../Exercise/2021-3}[2021-3.pdf]
\zexternaldocument*[method-]{../methodology/glossing}[glossing.pdf]
\usepackage{prettyref}

\geometry{left=3.18cm,right=3.18cm,top=2.54cm,bottom=2.54cm}
\titlespacing{\paragraph}{0pt}{1pt}{10pt}[20pt]
\setlength{\droptitle}{-5em}

\DeclareMathOperator{\timeorder}{\mathcal{T}}
\DeclareMathOperator{\diag}{diag}
\DeclareMathOperator{\legpoly}{P}
\DeclareMathOperator{\primevalue}{P}
\DeclareMathOperator{\sgn}{sgn}
\newcommand*{\ii}{\mathrm{i}}
\newcommand*{\ee}{\mathrm{e}}
\newcommand*{\const}{\mathrm{const}}
\newcommand*{\suchthat}{\quad \text{s.t.} \quad}
\newcommand*{\argmin}{\arg\min}
\newcommand*{\argmax}{\arg\max}
\newcommand*{\normalorder}[1]{: #1 :}
\newcommand*{\pair}[1]{\langle #1 \rangle}
\newcommand*{\fd}[1]{\mathcal{D} #1}

\newcommand*{\citesec}[1]{\S~{#1}}
\newcommand*{\citechap}[1]{Ch~{#1}}
\newcommand*{\citefig}[1]{Fig.~{#1}}
\newcommand*{\citetable}[1]{Table~{#1}}
\newcommand*{\citepage}[1]{p.~{#1}}
\newcommand*{\citepages}[1]{pp.~{#1}}
\newcommand*{\citefootnote}[1]{fn.~{#1}}
\newcommand*{\citechapsec}[2]{\citechap{#1}.\citesec{#2}}

\newrefformat{sec}{\citesec{\ref{#1}}}
\newrefformat{fig}{\citefig{\ref{#1}}}
\newrefformat{tbl}{\citetable{\ref{#1}}}
\newrefformat{chap}{\citechap{\ref{#1}}}
\newrefformat{fn}{\citefootnote{\ref{#1}}}
\newrefformat{box}{Box~\ref{#1}}
\newrefformat{ex}{\ref{#1}}


% color boxes

\tcbuselibrary{skins, breakable, theorems}

\AtBeginEnvironment{infobox}{\small}
\AtBeginEnvironment{theorybox}{\small}

\newtcbtheorem[number within=chapter]{infobox}{Box}{
    enhanced,
    boxrule=0pt,
    colback=blue!5,
    colframe=blue!5,
    coltitle=blue!50,
    borderline west={4pt}{0pt}{blue!65},
    sharp corners,
    fonttitle=\bfseries, 
    breakable,
    before upper={\parindent15pt\noindent}}{box}
\newtcbtheorem[number within=chapter, use counter from=infobox]{theorybox}{Box}{
    enhanced,
    boxrule=0pt,
    colback=orange!5, 
    colframe=orange!5, 
    coltitle=orange!50,
    borderline west={4pt}{0pt}{orange!65},
    sharp corners,
    fonttitle=\bfseries, 
    breakable,
    before upper={\parindent15pt\noindent}}{box}
\newtcbtheorem[number within=chapter, use counter from=infobox]{learnbox}{Box}{
    enhanced,
    boxrule=0pt,
    colback=green!5,
    colframe=green!5,
    coltitle=green!50,
    borderline west={4pt}{0pt}{green!65},
    sharp corners,
    fonttitle=\bfseries, 
    breakable,
    before upper={\parindent15pt\noindent}}{box}

% Shorthands
\newcommand*{\concept}[1]{\textbf{#1}}
\newcommand*{\term}[1]{\emph{#1}}
\newcommand{\form}[1]{\emph{#1}}

\newcommand{\redp}{\textasciitilde}

\newcommand{\deictictime}{T$_{\text{d}}$}
\newcommand{\referredtime}{T$_{\text{r}}$}
\newcommand{\orientationtime}{T$_{\text{o}}$}

\DeclareAcronym{blt}{short = BLT, long = Basic Linguistic Theory}
\DeclareAcronym{cgel}{short = CGEL, long = The Cambridge Grammar of the English Language}
\DeclareAcronym{dm}{short = DM, long = Distributed Morphology}
\DeclareAcronym{tag}{long = Tree-adjoining grammar, short = TAG}
\DeclareAcronym{sfp}{long = sentence-final particle, short = \textsc{sfp}}
\DeclareAcronym{np}{long = noun phrase, short = NP}
\DeclareAcronym{vp}{long = verb phrase, short = VP}
\DeclareAcronym{pp}{long = preposition phrase, short = PP}
\DeclareAcronym{advp}{long = adverb phrase, short = AdvP}
\DeclareAcronym{cls}{long = classifier, short = CLS}
\DeclareAcronym{dist}{long = distal, short = DIST}
\DeclareAcronym{prox}{long = proximate, short = PROX}
\DeclareAcronym{dem}{long = demonstrative, short = DEM}
\DeclareAcronym{classify}{long = classifier, short = \textsc{cl}}
\DeclareAcronym{dur}{long = durative, short = DUR}
\DeclareAcronym{neg}{long = negative, short = \textsc{neg}}
\DeclareAcronym{cc}{long = copular complement, short = CC}
\DeclareAcronym{cs}{long = copular subject, short = CS}
\DeclareAcronym{tame}{long = {tense, aspect, mood, evidentiality}, short = TAME}
\DeclareAcronym{past}{long = past, short = PST}
\DeclareAcronym{nonpast}{long = non-past, short = NPST}
\DeclareAcronym{present}{long = present, short = PRES}
\DeclareAcronym{progressive}{long = progressive, short = \textsc{poss}}
\DeclareAcronym{perfect}{long = perfect, short = \textsc{perf}}
\DeclareAcronym{passive}{long = passive, short = \textsc{pass}}
\DeclareAcronym{copula}{long = copula, short = COP}
\DeclareAcronym{possessive}{long = possessive, short = \textsc{poss}}
\DeclareAcronym{coca}{long = Corpus of Contemporary American English, short = COCA}

\newcommand{\asis}[1]{\textsc{#1}}
\newcommand{\oneof}[1]{{#1}}
\newcommand*{\homo}[2]{#1$_{\text{#2}}$}
\newcommand{\category}[1]{\textsc{#1}}
\newcommand{\formcat}[1]{\textsc{#1}}
\newcommand{\emptymorpheme}{$\emptyset$}
\newcommand*{\fromto}[2]{\langle {#1}, {#2} \rangle}

\newcommand{\alignment}{\href{../alignment/alignment.pdf}{my notes about alignment}}
\newcommand{\exerciseone}{\href{../Exercise/2021-3.pdf}{this exercise}}
\newcommand{\method}{\href{../methodology/glossing.pdf}{this note about my understanding of descriptive grammars}}

\newcommand{\ala}{à la}
\newcommand{\translate}[1]{`#1'}
\newcommand{\vP}{\textit{v}P}

% Make subsubsection labeled
\setcounter{secnumdepth}{4}
\setcounter{tocdepth}{4}
% reset example counter every chapter (but do not include the chapter number to the label)
\counterwithin{exx}{chapter} 

% Reference formats
\renewcommand{\bibname}{References}
\setcitestyle{aysep={}} 

% List format
\setlist[enumerate,1]{label=\alph*\upshape)}

\title{Reading notes of A Grammar of Japhug}
\author{Jinyuan Wu}

\begin{document}

\maketitle

The theoretical orientation is already well-documented in my notes about English, Latin and Mandarin Chinese.

\chapter{Part of speech}

\section{Noun}

\section{Verb}

Verbs can be regularly formed by denominal derivations
\citet[\citechap{20}]{jacques2021grammar}.
Since an independent adjective class is absent, 
the only two kinds of denominal derivations 
are noun-to-adverb derivations and noun-to-verb derivations,
the former being relative marginal \citet[\citepage{1011}]{jacques2021grammar};
thus the term \term{denominal} can be used specifically 
to refer to noun-to-verb derivations.

\subsection{Grammatical categories}


Decomposition of these \acs{tame} categories 
in the same way English \form{he [is playing] football} 
is analyzed as ``present (imperfect) progressive'' is not necessary:
TODO: why

The morphological realization of these categories is remarkable
Their main exponents are the alternation of the orientation prefix.
some \acs{tame} categories insert a fixed prefix 
into the orientation prefix slot (TODO: regardless of the lexically determined orientation prefix 
or the semantically significant orientation prefix of a orientable prefix?);
others choose one of the four prefixes that have the same directional meaning 
in \citet[\citetable{15.1}]{jacques2021grammar}. 

\subsection{Wordhood}

\citet[\citechap{11}]{jacques2021grammar} gives 
the realization of the verbal system. 
Whether this complex is to be regarded as one \emph{morphological} or \emph{phonological} word 
is discussed in \citesec{11.6}
in the reference above.
Recognition of wordhood, expectedly, is not self-evident;
\citet{prins2011web} provides an analysis of another rGyalrong language, Jiaomuzu, 
and in this thesis the term \term{verb phrase} 
(i.e. verbal complex in this note) is used,
skipping the discussion on what is a word.
In \citet[\citetable{11.3}]{jacques2021grammar}
four domains are defined using various criteria.

Domain A is defined according to both syntactic and morphological reasons.
What's shown in 
\citetable{11.3} contains all formatives that are relevant to verb inflection,
and they have non-adjacent dependencies,
so strong dependencies exist between them:
these formatives are realized in the same batch 
in clause building.
Now syntactically, the formative \form{-ci} in slit +4 
is selected by some modal prefixes in slot -6,
so the two slots belong to the same system;
on the other hand, outside the +4 and -6 slots 
we only have clitics which clearly belong to systems with higher positions
\citep[\citesec{11.6.2}]{jacques2021grammar},
and thus all -- and only -- formatives in \citetable{11.3}
constitute a syntactic word,
with the same \emph{syntactic} status of a verb-plus-auxiliary verbal complex or a
``verb phrase'' in Dixon's definition (i.e. without internal complements). 
Morphologically, no element is able to intervene 
between two slots in the template, 
so we say this batch is realized as a single morphological word 
instead of a verbal complex.

Domain B is about \emph{obligatoriness}:
thus the +4 slot is not included.
Domain C is defined according to prosodic reasons.

\section{Ideophones}

The category of ideophone occupies mainly manner adverbial positions
\citep[\citesec{10.1.7}]{jacques2021grammar}.
Its main difference with the adverb class 
is its morphology \citep[\citesec{10.1.2}]{jacques2021grammar}
and phonology \citep[\citesec{10.1.5}]{jacques2021grammar}.

\section{Analyzed examples}

TODO: \begin{itemize}
    \item What's the sentence final \form{ŋu}? The dictionary says it's a stative verb.
        What's its argument structure?
        Complement clause construction?
\end{itemize}

The sentence final stative verb \form{ŋu} be.\category{fact} is 
listed as a stative verb in the dictionary
and seems to take the constituents before it 
as a finite complement clause (TODO: or report speech? see the condition on \citepage{1317}),
which is without any explicit complementizer. 
But also see \citepages{1081, }

\chapter{Noun phrase}

One interesting feature of the Japhug comitative 
is it's also considered when deciding the number of an \acs{np}
\citep[\citepage{332}]{jacques2021grammar};
but it's still not prototypically a conjunction \citep[\citepage{420}]{jacques2021grammar}:
the \acs{np} following the comitative marker 
may be omitted, 
agreeing with the fact 
that the head noun of an \acs{np}
can also be dropped \citep[\citepage{425}]{jacques2021grammar}.
(In English this is only possible for clauses:
in informal writing and speech people may start with a sentence with \form{and},
i.e. a conjunction construction 
without the first branch,
but they never do so to an \acs{np}.)
The \acs{np} after the comitative marker can also be relativized.
Thus the comitative suffix is still recognized as a type of 
modification.

\chapter{\acs{tame} marking}

The \acs{tame} categories in Japhug 
are introduced in \citet[\citechap{21}]{jacques2021grammar}.
Morphologically speaking, there are three systems
\citep[\citepage{516}]{jacques2019egophoric}:
\begin{itemize}
    \item The \category{primary} system, 
        the morphological marking of which 
        happens in the template of the verb and is listed in \citetable{21.1}.
        This system's main exponents are stem alternation, the orientation preverb,
        and the modal prefix.
    \item The \category{secondary} system, 
        which also happens in the inflection of the verb  
        but is less involved with the \acs{tame} parameters in the \category{primary} system.
    \item The \category{periphrastic} system, 
        which appears in the first glance to be similar to complement clause constructions 
        with the copula \form{ŋu}.
        The copula in periphrastic constructions 
        never takes any argument indexation markers \citep[\citepage{1090}]{jacques2021grammar},
        and if we are to analyze the constructions as complement clause constructions,
        then the literal reading will be something like  
        ``it's the case that an event happens'',
        with all the contents before the copula 
        being a complement clause of the copula.
        In the follows however it can be seen that 
        the \acs{tame} categories on the copula 
        is complementary with the lexical verb,
        and hence the periphrastic constructions are to be analyzed 
        as single-clause constructions.

        In periphrastic constructions 
        the main verb is often in \emph{finite} forms \citep[\citepage{1081}]{jacques2021grammar};
        Japhug periphrastic conjugation thus has 
        a difference with English or Latin periphrastic conjugation,
        where what are used in periphrastic \acs{tame} categories
        are \emph{non-finite} verb forms.
        The reason possibly is because the periphrastic \acs{tame} categories in Japhug
        historically comes from finite complement clause constructions.

        TODO: is there any constraints on the distribution of participle or infinitive?
\end{itemize}
The interaction of the three morphological systems makes Japhug \acs{tame} system extremely complicated;
some periphrastic categories seem to have identical semantics with 
primary and secondary \acs{tame} marking devices (\citepage{1092});
whether there are hidden nuances is still not clear.

Besides the verbal complex, \acs{tame} categories are also marked by 
sentential adverbs and sentence-final particles 
(\citealt[\citepage{518}]{jacques2019egophoric}; \citealt[\citesec{21.8}]{jacques2021grammar}).

Also, \acs{tame} categories interact strongly with 
the lexical aspect of the main verb,
which can be crudely divided into being stative 
and being dynamic (e.g. \citesec{21.3.1.2}),
the person of the subject, and  TODO: other properties

The following subcategories can be recognized in the primary system:
\begin{itemize}
    \item Subjective evaluation: some \acs{tame} categories can be used to express 
    the feeling of the speaker (\citesec{21.3.2.4}) -- but is this a grammatical category?
    \item Evidentiality.
    Japhug has a highly complicated evidentiality system 
    \citep[\citetable{31.4}]{jacques2015sketch}.
    The values of evidentiality attested include 
    the generic, the factual, the sensory, the egophoric, and the inferential.

    A three-fold distinction can be observed with the non-past tense
    (\citealt[\citesec{21.3.4}]{jacques2021grammar}; 
    \citealt[\citesec{517}]{jacques2019egophoric}): 
    the factual or common knowledge (\citesec{21.3.1.2}), 
    the sensory (\citesec{21.3.2.2}),
    and the egophoric.

    Actually there is a fourth, bleached ``generic'' evidentiality value 
    with the non-past tense.
    The generic non-past \acs{tame} configuration 
    with no other non-trivial \acs{tame} marking 
    is known as the \category{imperfective} \citep[\citesec{21.2}]{jacques2021grammar}.
    This however seems to be very infrequent in main clauses
    without periphrastic auxiliaries (\citepage{1087}),
    indicating a strong preference for Japhug 
    to include a non-trivial evidentiality value in non-past sentences.
    
    The inferential evidentiality value appears only with the past tense,
    possibly because of semantic reasons: 
    an event happening now usually doesn't need to be ``inferred'',
    and this rarity means even this category existed historically,
    it has long been eroded.
    With the past tense, 
    we have a dichotomy between the generic evidentiality and the inferential evidentiality.
    It's impossible to morphologically mark the sensory evidentiality
    with the past tense, 
    possibly again because of the infrequency of this configuration.  
    It should however be noted that 
    the sensory can still be combined with the past tense 
    by periphrastically attaching a sensory copula 
    to the \category{aorist} (i.e. \category{past perfective} -- see below) 
    and the \category{past imperfective}
    which have default evidentiality
    (\citealt[\citesec{21.5.1.8}, \citesec{21.5.3.5}]{jacques2021grammar}; 
    \citealt[\citepage{518}]{jacques2019egophoric}). 
    On the other hand, the factual evidentiality and the egophoric evidentiality
    are never seen together with the past tense.

    \item Primary tense. 
    The distinction between the past and the non-past
    can be clearly identified (\citesec{23.3}, \citesec{23.5}),
    partly from the interaction with evidentiality.
    The meaning of future
    is regularly expressed in the \category{factual} category (\citepage{1102}),
    and thus is sometimes recognized as the future tense 
    or ``factual evidentiality in the future tense'' 
    \citep[\citepage{518}]{jacques2019egophoric}.
    This however seems to be the natural extension of the meaning 
    of the present tense 
    (c.f. English \form{the next high tide is around 4 this afternoon}; 
    \citealt[\citepage{131}, {[20]}]{cgel}),
    and in \citet{jacques2021grammar}, 
    the future tense is not recognized as a grammatical tense in Japhug
    \citep[\citepage{1102}, (46)]{jacques2021grammar}.

    \item Modality. 
        In Japhug, once the irrealis situation occurs, 
        it seems other \acs{tame} categories are not available. 
        There are four types of irrealis modalities falling in this domain:
        the \category{irrealis}, the \category{dubitative}, 
        the \category{imperative}, and the \category{prohibitive}
        \citep[\citesec{21.4}]{jacques2021grammar}.

    \item It seems the anterior category
        (the \category{perfect} category in English)
        is absent in Japhug.

    \item Aspect: the imperfective-perfective distinction.
        The non-past categories are always inherently imperfective:
        no perfective aspectuality is seen with non-past tense 
        \citep[\citepage{517}]{jacques2019egophoric},
        again possibly because of semantic reasons,
        since the perfective may be semantically identified with the past.
        The imperative-perfective distinction can only be seen 
        with the past tense \citep[\citetable{21.1}, note that the \category{aorist} is also known as 
        the \category{past perfective}; \citepages{1135, 1143}]{jacques2021grammar}.
    \item Aspects TODO: terminative, continuative, etc. TODO: the position of the inchoative aspect 
        The progressive aspect (\citesec{21.6}; note that \citetable{21.8}: compatibility?) 
\end{itemize}

The composition between these categories is not orthogonal, 
and no independent morphological exponent can be identified for 
each separate \acs{tame} categories mentioned above
(but in periphrastic conjugation, 
distribution of these primitive \acs{tame} categories 
onto the main verb and the auxiliary copula 
can be observed; \citealt[\citepage{1089}, (7)]{jacques2021grammar});
these parameters are therefore deeply fused into each other.
By combining these categories and noticing the constraints listed above, 
we find the 11 primary \acs{tame} categories 
listed in \citet[\citepage{21.1}]{jacques2021grammar}.
The realis part is replicated in \prettyref{tbl:realis-tam}.
Note that the \category{aorist} is just the \category{past perfective} in
\citet[\citetable{31.4}]{jacques2015sketch}.
 
\begin{table}[H]
    \centering
    \caption{Analysis of Japhug realis \acs{tame} categories;
    the two sensory past cells may be filled by periphrastic conjugation}
    \label{tbl:realis-tam}
    \scriptsize
    \begin{tabular}{@{}lllllll@{}}
        \toprule
                                &                          & \multicolumn{5}{c}{evidentiality}                                                                                                                                                                        \\ \cmidrule(l){3-7} 
        \multirow{-2}{*}{tense} & \multirow{-2}{*}{aspect} & generic                           & factual                           & sensory                           & egophoric                         & inferential                                              \\ \midrule
        non-past                & imperfective             & \category{imperfective}             & \category{factual}                  & \category{sensory}                  & \category{egophoric present}        & \cellcolor[HTML]{C0C0C0}{\color[HTML]{C0C0C0} \category{}} \\ \midrule
                                & imperfective             & \category{past imperfective}        & \cellcolor[HTML]{C0C0C0}\category{} & \cellcolor[HTML]{C0C0C0}\category{} & \cellcolor[HTML]{C0C0C0}\category{} & \category{inferential}                                     \\ \cmidrule(l){2-7} 
        \multirow{-2}{*}{past}  & perfective               & \category{aorist} & \cellcolor[HTML]{C0C0C0}\category{} & \cellcolor[HTML]{C0C0C0}\category{} & \cellcolor[HTML]{C0C0C0}\category{} & \category{inferential imperfective}                        \\ \bottomrule
    \end{tabular}
\end{table}

Periphrastic conjugations 

\chapter{Clause structure}

\section{The overall structure}

Japhug clauses are verb-final:
core arguments and adverbials are before the verb, 
\citet{jacques2021grammar} doesn't mention verb phrase coordination,
but \citet[\citepage{549}]{prins2011web} mentions 
coordination of two verb phrases sharing the same subject
in a relative language Jiaomuzu,
and therefore the verb phrase layer should be kept??

\bibliographystyle{plainnat}
\bibliography{gyalrong.bib}

\end{document}