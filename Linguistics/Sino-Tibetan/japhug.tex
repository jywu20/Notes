\documentclass[a4paper, oneside, 12pt]{report}

\usepackage[T1]{fontenc}
\usepackage{libertinus}
\usepackage{underscore}
\usepackage{geometry}
\usepackage{float}
\usepackage{titling}
\usepackage{titlesec}
\usepackage{paralist}
\usepackage{footnote}
\usepackage[inline]{enumitem}
\usepackage{amsmath, amsthm}
\usepackage{gb4e}
\noautomath
\usepackage{bbm}
\usepackage{textcomp}
\usepackage{soul}
\usepackage{graphicx}
\usepackage{siunitx}
\usepackage[table,xcdraw]{xcolor}
\usepackage{tikz}
\usepackage[ruled, vlined, linesnumbered, noend]{algorithm2e}
\usepackage{xr-hyper}
\usepackage[colorlinks, citecolor = purple, bookmarksnumbered]{hyperref} % linkcolor=black, anchorcolor=black, citecolor=black, filecolor=black
\usepackage[most]{tcolorbox}
\usepackage{caption}
\usepackage{subcaption}
\usepackage{booktabs}
\usepackage{multirow}
\usepackage[figuresright]{rotating}
\usepackage{acro}
\usepackage[round]{natbib} 
\usepackage{nameref,zref-xr}
\zxrsetup{toltxlabel}
\zexternaldocument*[alignment-]{../alignment/alignment}[alignment.pdf]
\zexternaldocument*[exercise1-]{../Exercise/2021-3}[2021-3.pdf]
\zexternaldocument*[method-]{../methodology/glossing}[glossing.pdf]
\usepackage{prettyref}

\geometry{left=3.18cm,right=3.18cm,top=2.54cm,bottom=2.54cm}
\titlespacing{\paragraph}{0pt}{1pt}{10pt}[20pt]
\setlength{\droptitle}{-5em}

\DeclareMathOperator{\timeorder}{\mathcal{T}}
\DeclareMathOperator{\diag}{diag}
\DeclareMathOperator{\legpoly}{P}
\DeclareMathOperator{\primevalue}{P}
\DeclareMathOperator{\sgn}{sgn}
\newcommand*{\ii}{\mathrm{i}}
\newcommand*{\ee}{\mathrm{e}}
\newcommand*{\const}{\mathrm{const}}
\newcommand*{\suchthat}{\quad \text{s.t.} \quad}
\newcommand*{\argmin}{\arg\min}
\newcommand*{\argmax}{\arg\max}
\newcommand*{\normalorder}[1]{: #1 :}
\newcommand*{\pair}[1]{\langle #1 \rangle}
\newcommand*{\fd}[1]{\mathcal{D} #1}
\newcommand*{\textto}{$\to$}
\newcommand*{\textgt}{$>$ }

\newcommand*{\citesec}[1]{\S~{#1}}
\newcommand*{\citechap}[1]{Ch~{#1}}
\newcommand*{\citefig}[1]{Fig.~{#1}}
\newcommand*{\citetable}[1]{Table~{#1}}
\newcommand*{\citepage}[1]{p.~{#1}}
\newcommand*{\citepages}[1]{pp.~{#1}}
\newcommand*{\citefootnote}[1]{fn.~{#1}}
\newcommand*{\citechapsec}[2]{\citechap{#1}.\citesec{#2}}

\newrefformat{sec}{\citesec{\ref{#1}}}
\newrefformat{fig}{\citefig{\ref{#1}}}
\newrefformat{tbl}{\citetable{\ref{#1}}}
\newrefformat{chap}{\citechap{\ref{#1}}}
\newrefformat{fn}{\citefootnote{\ref{#1}}}
\newrefformat{box}{Box~\ref{#1}}
\newrefformat{ex}{\ref{#1}}


% color boxes

\tcbuselibrary{skins, breakable, theorems}

\AtBeginEnvironment{infobox}{\small}
\AtBeginEnvironment{theorybox}{\small}

\newtcbtheorem[number within=chapter]{infobox}{Box}{
    enhanced,
    boxrule=0pt,
    colback=blue!5,
    colframe=blue!5,
    coltitle=blue!50,
    borderline west={4pt}{0pt}{blue!65},
    sharp corners,
    fonttitle=\bfseries, 
    breakable,
    before upper={\parindent15pt\noindent}}{box}
\newtcbtheorem[number within=chapter, use counter from=infobox]{theorybox}{Box}{
    enhanced,
    boxrule=0pt,
    colback=orange!5, 
    colframe=orange!5, 
    coltitle=orange!50,
    borderline west={4pt}{0pt}{orange!65},
    sharp corners,
    fonttitle=\bfseries, 
    breakable,
    before upper={\parindent15pt\noindent}}{box}
\newtcbtheorem[number within=chapter, use counter from=infobox]{learnbox}{Box}{
    enhanced,
    boxrule=0pt,
    colback=green!5,
    colframe=green!5,
    coltitle=green!50,
    borderline west={4pt}{0pt}{green!65},
    sharp corners,
    fonttitle=\bfseries, 
    breakable,
    before upper={\parindent15pt\noindent}}{box}
\newtcbtheorem[number within=chapter, use counter from=infobox]{todobox}{Box}{
    enhanced,
    boxrule=0pt,
    colback=red!5,
    colframe=red!5,
    coltitle=red!50,
    borderline west={4pt}{0pt}{red!65},
    sharp corners,
    fonttitle=\bfseries, 
    breakable,
    before upper={\parindent15pt\noindent}}{box}

% Shorthands
\newcommand*{\concept}[1]{\textbf{#1}}
\newcommand*{\term}[1]{\emph{#1}}
\newcommand{\form}[1]{\emph{#1}}

\newcommand{\redp}{\textasciitilde}

\newcommand{\deictictime}{T$_{\text{d}}$}
\newcommand{\referredtime}{T$_{\text{r}}$}
\newcommand{\orientationtime}{T$_{\text{o}}$}

\DeclareAcronym{blt}{short = BLT, long = Basic Linguistic Theory}
\DeclareAcronym{cgel}{short = CGEL, long = The Cambridge Grammar of the English Language}
\DeclareAcronym{dm}{short = DM, long = Distributed Morphology}
\DeclareAcronym{tag}{long = Tree-adjoining grammar, short = TAG}
\DeclareAcronym{sfp}{long = sentence-final particle, short = \textsc{sfp}}
\DeclareAcronym{np}{long = noun phrase, short = NP}
\DeclareAcronym{vp}{long = verb phrase, short = VP}
\DeclareAcronym{pp}{long = preposition phrase, short = PP}
\DeclareAcronym{advp}{long = adverb phrase, short = AdvP}
\DeclareAcronym{cls}{long = classifier, short = CLS}
\DeclareAcronym{dist}{long = distal, short = DIST}
\DeclareAcronym{prox}{long = proximate, short = PROX}
\DeclareAcronym{dem}{long = demonstrative, short = DEM}
\DeclareAcronym{classify}{long = classifier, short = \textsc{cl}}
\DeclareAcronym{dur}{long = durative, short = DUR}
\DeclareAcronym{neg}{long = negative, short = \textsc{neg}}
\DeclareAcronym{cc}{long = copular complement, short = CC}
\DeclareAcronym{cs}{long = copular subject, short = CS}
\DeclareAcronym{tame}{long = {tense, aspect, mood, evidentiality}, short = TAME}
\DeclareAcronym{past}{long = past, short = PST}
\DeclareAcronym{nonpast}{long = non-past, short = NPST}
\DeclareAcronym{present}{long = present, short = PRES}
\DeclareAcronym{progressive}{long = progressive, short = \textsc{poss}}
\DeclareAcronym{perfect}{long = perfect, short = \textsc{perf}}
\DeclareAcronym{passive}{long = passive, short = \textsc{pass}}
\DeclareAcronym{copula}{long = copula, short = COP}
\DeclareAcronym{possessive}{long = possessive, short = \textsc{poss}}
\DeclareAcronym{coca}{long = Corpus of Contemporary American English, short = COCA}
\DeclareAcronym{sap}{long = speech act participant, short = SAP}

\newcommand{\asis}[1]{\textsc{#1}}
\newcommand{\oneof}[1]{{#1}}
\newcommand*{\homo}[2]{#1$_{\text{#2}}$}
\newcommand{\category}[1]{\textsc{#1}}
\newcommand{\formcat}[1]{\textsc{#1}}
\newcommand{\emptymorpheme}{$\emptyset$}
\newcommand*{\fromto}[2]{\langle {#1}, {#2} \rangle}

\newcommand{\alignment}{\href{../alignment/alignment.pdf}{my notes about alignment}}
\newcommand{\exerciseone}{\href{../Exercise/2021-3.pdf}{this exercise}}
\newcommand{\method}{\href{../methodology/glossing.pdf}{this note about my understanding of descriptive grammars}}

\newcommand{\ala}{à la}
\newcommand{\translate}[1]{`#1'}
\newcommand{\vP}{\textit{v}P}

% Make subsubsection labeled
\setcounter{secnumdepth}{4}
\setcounter{tocdepth}{4}
% reset example counter every chapter (but do not include the chapter number to the label)
\counterwithin{exx}{chapter} 

% Reference formats
\renewcommand{\bibname}{References}
\setcitestyle{aysep={}} 

% List format
\setlist[enumerate,1]{label=\alph*\upshape)}

\title{Reading notes of A Grammar of Japhug}
\author{Jinyuan Wu}

\begin{document}

\maketitle

The theoretical orientation is already well-documented in my notes about English, Latin and Mandarin Chinese.

\chapter{Introduction}

\section{Classification}

\begin{todobox}{Japhug and other Sino-Tibetan languages}{relation}
    \begin{itemize}
        \item Tangut: direct-inverse
        \item Direct-inverse in proto-ST
        \item Relation with Situ
        \item Dialects within Japhug
    \end{itemize}
\end{todobox}

\section{Sociolinguistic status}

\begin{todobox}{Sociolinguistic status}{sociolinguistic}
    \begin{itemize}
        \item Relation with Ando Tibetan
        \item Relation with Mandarin
    \end{itemize}
\end{todobox}

\chapter{Grammatical overview}

In this chapter we do a round-by-round survey of Japhug grammar.
We start with a very rough top-down anatomy of the structure of Japhug clauses
(\prettyref{sec:grammatical.clause.template}),
followed by a bottom-up examination of grammatical categories and relations in the clause
in the rest of \prettyref{sec:grammatical.clause}.
We then do the same for the noun phrase.
We finally list points to investigate in the lexicon of Japhug.

\begin{todobox}{Grammatical sketch TODO list}{grammatical-todo}
    \begin{itemize}
        \item NP
        \item POS list
    \end{itemize}
\end{todobox}

\section{Clause structure}\label{sec:grammatical.clause}

\subsection{The structure template}\label{sec:grammatical.clause.template}

\paragraph*{Clause combining}
We start our discussion on Japhug grammar 
by first dividing utterances into simple clauses.
Adverbial clauses, like temporal clauses or conditional clauses,
seem to always appear before the main clause \citep[\citechap{25}]{jacques2021grammar}.
Coordination may be marked by coordination linker \form{tɕe} or \form{qʰe} or simple parataxis
\citep[\citesec{25.1.6}]{jacques2021grammar}.

In both adverbial clause subordination and coordination,
clauses involved may share the auxiliary copula if there is any
\citep[\citepage{47}, (40); \citepage{1091}, (10)]{jacques2021grammar}.
It seems changing the subject in the middle is also possible
(\prettyref{box:periphrastic-vp-coordination}).

\paragraph*{The simple clause: the nucleus and information packaging}
\label{sec:grammatical.clause.template.nucleus-identification}
A simple clause, which may or may not be a part 
of aforementioned subordination or coordination constructions,
consists of a nucleus clause and possible information packaging devices.

We can say with confidence that the basic order of Japhug nucleus clauses is SOV.
Justification of concepts like subject or object is discussed in 
\prettyref{sec:grammatical.clause.internal} and \prettyref{sec:grammatical.clause.subject}.
All arguments -- and of course adjuncts -- of clauses can be omitted;
in this way it is possible that the nucleus clause consists of only verb.
The arguments are to be recovered from argument indexation
(\prettyref{sec:grammatical.clause.direct-inverse}).

Possible alternations of the SOV order and orders of adverbials outlines below 
can be explained by information packaging constructions,
including right dislocation for afterthought, disambiguation, or emphasis 
\citep[\citesec{22.1.3}]{jacques2021grammar},
left dislocation for topicalization 
\citep[\citepage{1189}]{jacques2021grammar},
or focalization \citep[\citepage{1190}]{jacques2021grammar}.%
\footnote{
    \citep[\citepage{1190}]{jacques2021grammar} has the OSV order,
    and Jacques mentions that the subject is focalized.
    It is possible that this example is a topic-focus-verb construction.
}
Topicalization in Japhug may result in a chain of nucleus clauses sharing one topic
\citep[\citepage{1190}, (11)]{jacques2021grammar}.

\paragraph*{Positions of adverbials}\label{sec:grammatical.clause.template.adverb}
Temporal expressions indicating the absolute time usually appear before the subject
\citep[\citepage{344}, (167); \citepage{283}, (123)]{jacques2021grammar}.
This probably is because the absolute time sets the stage for the event
and by default is topical,
and their syntactic position is comparable to that of adverbial clauses.%
\footnote{
    The situation is similar to that in English:
    temporal phrases like \form{last year} almost always appear at the margin of a clause.
}

On the other hand, \ac{tame} adverbs
appear after the subject and before the direct object
\citep[\citepages{1200-1201,1210}]{jacques2021grammar}.
It seems that locational phases also reside in roughly the same region
\citep[\citepage{302}]{jacques2021grammar}.
Example (46) in \citet[\citepage{1200}]{jacques2021grammar}
suggests that the aspectual adverb precedes the manner adverb.

In some examples the \emph{object} appears before a \ac{tame} adverb,
but this likely arises from topicalization as a pause can be observed after the object
\citep[\citepage{1210}, (82)]{jacques2021grammar}.

\begin{todobox}{Positions of adverbs}{adverb-position}
    Compare the positions of TAME and locational adverbials.
    
    Also, is it possible to have locational stage-setting adverbials?
\end{todobox}

\paragraph*{Auxiliary verb constructions}
In periphrastic conjugation constructions, 
the copula that carries the main \ac{tame} information 
is put at the final position, 
and the main verb precedes it; 
when there are several main verbs coordinated in a clause, 
only one final copula needs to appear \citep[\citepages{1090-1091}]{jacques2021grammar}.

\begin{todobox}{Periphrastic conjugation and verb phrase coordination}{periphrastic-vp-coordination} 
    Note that the subject seems to be changed in the middle 
    (the problem is the meaning of \form{ɯ-ŋgɯ} \translate{\category{3sg.poss}-inside}
    and the constituents introduced by it: 
    does it mean \translate{as three very beautiful girls}, 
    and therefore is an adjunct, or is it a new subject? 
    Also verb phrase coordination is related to syntactic ergativity -- 
    see the discussion on clause pivot)
    
    If the coordination construction indeed works on the level of clause and not VP
    and therefore allows changing the subject in the middle,
    another question arises:
    sharing TAME markers between two clauses is highly unusual.
    This can't be a purely realizational process
    because of the long-range nature of copula sharing.
    So this has to involve (purely) syntactic coordination,
    and the syntactic position of the TAME copula is then higher than the subject,
    which is unusual.
    Another analysis of the construction is that we are looking at a biclausal construction,
    and the coordination happens to a series of nonfinite clauses,
    But then TAME marking can't be biclausal. 
    
    It seems currently the best option is to analyze this construction as coordination of ``small clauses''
    whose \ac{tame} values are then specified by the end-of-sentence copula.
    Note that this construction seems to be only available for \category{imperfective} verbs,
    which appears frequently in periphrastic conjugation
    and likely only specifies the aspectual value. 
    The tense and evidentiality categories coded on the copula are then higher than the subject.
\end{todobox}

\paragraph*{Clausal structural template}
In summary, a Japhug simple clause is a nucleus clause with possible left- or right-dislocations
and/or temporal expressions expressing the absolute time of the event
(\prettyref{box:adverb-position}),
an the nucleus clause, from left to right,
consists of the subject,
tense and aspect adverbs,
manner and locational phrases,
core arguments, the main verb, and a possible auxiliary copula;
all arguments and adjuncts are not obligatory,
presumably because enough information has been coded in the verb.

\begin{todobox}{Other forms of the nucleus clause}{nucleus-clause-infrequent-types}
    Serial verb constructions and nominal predicates;
    also see fossilized N-V sequences. 
\end{todobox}

\paragraph*{Speech fillers}
One final comment about the clausal template:
in spontaneous speeches, Japhug speakers use several speech fillers and not a central vowel 
to mark pause or earn some time to think about what to say next.
The speech fillers have other functions like pronoun or topic marker
and should be ignored when reading Japhug texts
\citep[\citesec{10.3}]{jacques2021grammar}.

\subsection{The verb}

\begin{todobox}{Event structure and valency}{event-structure-valency}
    State change, action, etc. and relation with unergative verbs or unaccusative verbs.
\end{todobox}

Verbs can be regularly formed by denominal derivations
\citep[\citechap{20}]{jacques2021grammar}.
Since an independent adjective class is absent, 
the only two kinds of denominal derivations 
are noun-to-adverb derivations and noun-to-verb derivations,
the former being relative marginal \citep[\citepage{1011}]{jacques2021grammar};
thus the term \term{denominal} can be used specifically 
to refer to noun-to-verb derivations.

\subsection{Internal arguments}\label{sec:grammatical.clause.internal}

In the typological literature, macrorole notations like S, A, P 
are based on the premise that the arguments it represents
behave syntactically in similar ways.
The symbols then refer to the bundles of shared syntactic properties.

Roughly speaking, the properties may be about the deep argument structure
(control, reflexive, etc.) or clause-level grammatical phenomena
(relativization, subject-sharing coordination, etc.).
Case marking or in other words argument flagging 
is controlled by the latter (e.g. accusative alignment v.s. ergative alignment)
as well as the former (e.g. oblique arguments cannot be passivized),
and so is verb-argument agreement or in other words argument indexation.

The definition of macroroles in a specific language therefore should represent 
the aforementioned phenomena in that language.
A good definition of macroroles in a language
is equivalent to a good classification of its valency classes.

\paragraph*{Intransitive constructions}
In Japhug, the sole arguments of intransitive clauses largely behave in the same way:
we can stipulate a S macrorole and there is no Sa/So distinctions. 
There exist so-called semi-transitive verbs with additional arguments,
which are however not relevant to argument indexation at all
\citep[\citesec{14.2.5}]{jacques2021grammar}
and do not seem to be active in other grammatical phenomena.

A notable distinction within transitive clauses
is that in the passive of secundative verbs,
the subject does not trigger indexation
\citep[\citesec{18.1.4}]{jacques2021grammar}.

\begin{todobox}{Oblique arguments in Japhug}{obligque-argument}
    Compare oblique arguments with, say, directional particles.
\end{todobox}

\begin{todobox}{Labile verb}{labile-verb}
    Japhug does have S=O labile verbs,
    and it remains unclear whether the surface S in this case
    reflects some of the deep P properties. 
\end{todobox}

\paragraph*{Monotransitive constructions and objecthood}
In transitive clauses we can recognize A and P arguments.

The A argument is clearly more ``external'' or prominent.
Verbal polypersonal indexation always involves two argument,
and the A is always one of them, 
while the rest of the arguments compete to be the other argument involved 
(\prettyref{sec:grammatical.clause.direct-inverse.hierarchy}).
By default we can observe that the A 
appears at the start of the nucleus clause 
(\prettyref{sec:grammatical.clause.template.nucleus-identification}).
The comparison between A and S and hence the definition of \term{subject}
are discussed in \prettyref{sec:grammatical.clause.subject}.
We note however reflexive binding is absent in Japhug
\citep[\citepage{543}]{jacques2021grammar}
and cannot be used to prove the pivot status of the A argument
either in the deep argument structure or in the whole clause.

\paragraph*{Comparison between P and other internal arguments}
\label{sec:grammatical.clause.internal.object}
The P argument is also known as the \term{object}.
For something to be known as the object,
it usually needs to be an internal argument position
that is not restricted to the monotransitive construction
and syntactically more active than other internal arguments.
In English, for example,
an argument -- the P argument in the monotransitive construction 
and the T argument in \form{give sth. to sb.} --
always follows the main verb with almost no other constituents
being able to appear between the two,
and constituents like manner phrases, relative clauses, etc.
which have their scopes \emph{over} the core verb phrase, follow the object.
This seems to indicate that there is some sort of implicit fronting 
of both the verb and the object,
leaving a swamp of various constituents behind.
Therefore there is an object position after the main verb
that is attested in more than one verb frames. 

It is possible that the object position cannot be well defined in other verb frames. 
The split of object properties is observed in the English \form{give sb. sth.} verb frame,
where the recipient argument can be passivized but not topicalized,
and the theme argument can be topicalized but not passivized.
The reason possibly is that the recipient argument is in an object-like position
(and thus can be passivized)
but either faces realizational effects that lock it there 
\citep{oba2005double},
or is obligatorily focalized and hence topicalization causes conflicts in information structure
\citep{im2005alternative}.
In this case there does not exist a prototypical object.

The P argument of monotransitive verbs regularly triggers object-like indexation
\citep[\citesec{8.1.3}, \citepage{543}]{jacques2021grammar}.
On the other hand, some arguments, 
despite being similar to the monotransitive P in being the receiver of an event 
and possibly being the sole internal argument of the clause,
can never trigger indexation.
Hence the monotransitive P is the prototypical object
while the internal arguments that do not trigger indexation 
but are not in genitive, dative, etc. cases
are known to be \term{semi-objects}
\citep[\citesec{8.1.5}]{jacques2021grammar}.
So an object position can be defined for some two-places verbs.

The next question is whether in other verb frames,
there exists a position whose properties are close to the monotransitive P enough,
so that the concept of \term{object} can be defined not just for 
the prototypical monotransitive construction. 

The only kind of passivization allowed for ditransitive verbs
acts on secundative verbs,
and surprisingly turns the A\textto R\textto T argument structure
into a T\textto R one, although the R seems more object-like
\citep[\citepage{884}]{jacques2021grammar}.
It might be possible that the latter is homophonous to the passive construction
but has a more complicated structure.
For verbs derived from valency alternation constructions,
the passive cannot happen on top of synchronically analyzable causative or applicative 
\citep[\citepage{885}]{jacques2021grammar},
and its application on top of the tropative is also not attested
\citep[\citesec{17.5.4}]{jacques2021grammar}.
Therefore, if passivization is taken as the criterion,
then only the monotransitive P is the object.

In applicative constructions,
the promoted argument ends up in a position quite close to the prototypical P,
which acts just like the P in argument indexation
(e.g. \citealt[\citepage{863}, (102)]{jacques2021grammar}).

Finally, we note that if we are to take a broad definition of objecthood,
then relativization is a good criterion 
\citep{jacques2016subjects}.

\paragraph*{Ditransitive constructions}
Japhug has a clearly indirective ditransitive verb frame
(i.e. a verb frame where the theme is object-like)
and a secundative verb frame  
(i.e. a verb frame where the recipient is object-like).
In the indirective construction,
the recipient is dative or genitive \citep[\citesec{14.4.1}]{jacques2021grammar}
and it seems to have not many shared properties with prototypical subjects or objects,
while the theme can be seen as the object \citep{jacques2016subjects}.%
\footnote{
    \citet{jacques2016subjects} 
    uses R_1, T_1 to refer to the recipient and the theme of secundative verbs,
    and R_2, T_2 to refer to the recipient and the theme of indirective verbs. 
}

\begin{todobox}{Passivization of arguments}{passivization-ditransitive}
    Can T_2 be passivized?
\end{todobox}

As for what \citet[\citesec{14.4.2}]{jacques2021grammar} analyzes as 
the secundative verbs (the valency class where the recipient is more object-like),
the theme can be extracted in relativization \citep[\citepages{581}]{jacques2021grammar}
and be passivized \citep[\citesec{18.1.4}]{jacques2021grammar},
but the recipient participates in argument indexation
in the same way as the monotransitive object does.

What is interesting is that in the resulting verb,
the subject (which is the theme) does not trigger any indexation affixes.
This may be a case of quirky subject.

\paragraph*{Valency alternation} 
Japhug seems to have a construction comparable to
the ``passive'' construction in Mandarin;
in this pseudo-passive construction,
the deep, animate A argument does not appear, 
while the inanimate deep P argument is present, 
and the verb has an inverse marker 
despite semantically the event happens 
from an animate participant to an inanimate participant
and is therefore semantically in the direct configuration
\citep[\citepage{575}]{jacques2021grammar}.
The construction only appears in translation of sentences from Mandarin,
and yet a native speaker didn't consider them to be ungrammatical;
whether the 


\subsection{Syntactic pivot of the core clause, or the subject}\label{sec:grammatical.clause.subject}

Among arguments in multivalent clauses, we can easily recognize that 
A is the more or less external one
and the rest ones are more or less internal
(\prettyref{sec:grammatical.clause.internal}).
These facts however can all be explained by defining a pivot for the \emph{argument structure},
and are not direct evidence for a pivot of \emph{the whole clause}:
telling us nothing about the alignment type of Japhug:
in ergative languages, we still find that the A argument is the argument structure pivot,
but definitely the P argument shares various other properties with S.

The usual criteria for syntactic ergativity are not viable in Japhug,
as we do not have uncontroversial verb phrase-level coordination \citep{jacques2014clause}.
There are coordination constructions in which the two branches share one auxiliary,
but the branches may have different subjects
(see \prettyref{box:periphrastic-vp-coordination}).
This is likely to be a peculiarity of Japhug,
as \citet[\citepage{549}]{prins2011web} mentions 
coordination of two verb phrases sharing the same subject
in Jiaomuzu, a language close to Japhug.

If we restrict ourselves to Japhug,
there is one construction that at least partially resembles
shared subject coordination of verb phrases.
We observe that the A argument (with ergative marking) 
can be separated from the object and the verb 
by an intransitive clause \citep[\citepage{306}]{jacques2021grammar}
in which a gap coreferential with the A argument exists.
This may also be understood 
as the A argument of the main clause and the S argument of the embedded clause
being extracted and fronted,
and therefore neutralization of S and A as the clause-level pivot.

Subjecthood seems to be most clearly defined by relativization
(\citealt{jacques2016subjects};
see \prettyref{sec:grammatical.clause.internal.object} for the relation 
between relativization and objecthood).
Relativization happens only after the whole clause is finished
and a pivot position definable by relativization is definitely a clausal pivot
-- hence the \term{subject}.

Additionally, the fact that one argument may be separated from
the verb and other arguments by \ac{tame} adverbs
(\prettyref{sec:grammatical.clause.template.adverb})
may also be taken as evidence for subjecthood in Japhug.

With facts above, we find that there exists a well-defined clausal pivot in Japhug
and it is identical to the pivot of the argument structure.
Japhug therefore has a nominative-accusative pivot.

The subjecthood defined alone the lines above does not entail
anything about case marking.
In prototypical monotransitive constructions
we observe morphological ergativity.
The existence of a direct-inverse argument indexation system (\prettyref{sec:grammatical.clause.direct-inverse})
means rules about argument indexation cannot be summarized as 
``the verb agrees with the subject'',
and therefore the subject cannot be defined according to verb morphology \citep{jacques2016subjects},
and in certain circumstances even leads to neutralization of S and P
(\prettyref{sec:grammatical.clause.direct-inverse.hierarchy}).

\subsection{The direct-inverse system}\label{sec:grammatical.clause.direct-inverse}

\paragraph*{Polypersonal indexation and argument flagging}
The sole argument of a Japhug intransitive clause is indexed on the verb.
In transitive clauses, 
two -- not just one -- argument are indexed on the verb. 
The personal affixes on the verb are only about person and number, 
and tell us nothing about the argument position of the argument 
from which they originate \citep[\citepage{543}]{jacques2021grammar}.
What argument is indexed is controlled by whether the clause is in a direct configuration
or an indirect configuration.

\paragraph*{Inverse indexation v.s. inverse voice}
According to \citet{oxford2023tale},
what is known as an inverse construction may be 
(a) the change of clausal grammatical function of the arguments,
i.e. ``deep inverse'', in which the patient in a way or another gets some subject-related properties 
and hence some sort of ergativity is observed in inverse configurations,
with effects like alternations of the surface word order,
scope of \ac{np}s, and reflexive pronouns (e.g. see \citealt{bruening2005algonquian}), or
(b) a ``shallow inverse'' which is mostly about argument indexation, 
for example a requirement that only agreement with a \ac{sap} 
is morphologically permissible
leading to a direct-inverse system based on the distinction between 1/2 and 3.%
\footnote{
    The deep inverse is also known as ``syntactic'' inverse or ``inverse voice'',
    and the shallow inverse is also known as ``morphological'' inverse, or ``inverse alignment''.
    This terminology however is sometimes misleading.
    For example, if we stipulate that in Italian,
    person clitics are agreement formatives
    and first/second person clitics belong to an agreement system only targeting \ac{sap}s,
    while the third person clitic belongs to another agreement system,
    then the impossibility of co-appearance of a first/second person direct object clitic 
    and a third person indirect object clitic Italian 
    can be argued to be due to locality constraints
    \citep{bianchi2006syntax}.
    The existence of a speech-act only agreement system is comparable to ``morphological inverse'',
    but the phenomena related to this system usually will not be called morphology.

    On the other hand, ``syntactic inverse'' covers both syntactic ergativity and morphological ergativity.
    This confusion is seen in the case of Japhug:
    in the inverse configuration of Japhug we see morphological ergativity,
    but this is deep inverse and therefore ``syntactic inverse'' in \citet{oxford2023tale}.
}
The deep inverse can be optional and in this case it is essentially a voice construction;
the shallow inverse should be obligatory or otherwise verbal agreement is arbitrary.
The two of course can be combined and we get a particularly strong inverse system.

In Japhug we can observe certain phenomena that may be described as ``deep inverse''.
In the inverse configuration of Japhug, 
the A argument receives an ergative marker.
Properties other than case marking are however not observed
(\prettyref{sec:grammatical.clause.subject}).
Therefore, in the inverse configuration, we cannot say that
the patient is rendered the clausal pivot.
The ergative marking of the A argument in the inverse configuration should be analyzed as
inverse-triggered morphological ergativity,
in which the only subject-like property that the P argument gets
is that it has the same case marking with the intransitive subject. 

\paragraph*{The empathy hierarchy}\label{sec:grammatical.clause.direct-inverse.hierarchy} 
The relative positions of the A argument and the P argument in the empathy hierarchy
control the direct-inverse configuration.
Roughly, the hierarchy in Japhug is 
\ac{sap} \textgt human \textgt animal \textgt inanimate \textgt generic argument.
Note that this hierarchy means that
if the A argument of a clause is generic,
while the P argument is not,
the clause is in inverse configuration,
while when the P argument is generic the clause is in direct configuration.
This means neutralization of S and P can be observed in the generic indexation
(\citealt{jacques2012argument}; \citealt[\citesec{14.3.2.5}]{jacques2021grammar}).
This, of course, cannot be interpreted as ergativity.

The empathy hierarchy may be analyzed as a synchronic device. 
\citet[\citesec{7.4}]{wiltschko2014universal}, for example,
stipulates that the clause first decides its point of view in the empathy hierarchy,
and then this piece of information controls what argument is to agree with the verb,
and further notices that this process is formally comparable to
how the aspect value (c.f. the direct/inverse value) 
dictates what is the time that is to be compared with the speech time to decide the tense
(c.f. the argument indexed on the verb).

Alternatively, the empathy hierarchy may be treated as a \emph{diachronic tendency}
that languages with polypersonal indexation will likely be trapped into,
possibly because a direct-inverse system reduces misunderstanding in discourses
by highlighting less likely configurations.
The synchronic analysis of shallow inverse then is merely 
some morphological manipulations of the person features received by the verb,
and obligatory deep inverse can be analyzed as incompatibility between
direct verb morphology and the inverse voice construction,
probably because there simply is no verb form expressing both of them
and therefore co-appearance of the two is blocked.

It seems in Japhug, we do have a synchronic empathy hierarchy.
In the causative construction, the argument playing the role of P in indexation
can be the causee or the object.
Which argument is chosen depends on the persons of the arguments.
First, among internal arguments,
if one is first or second person and another is third person,
then the first/second person one is indexed.
Thus both 2\textto 3\textto 1 and 2\textto 1\textto 3 
are equivalent to 2\textto 1 in argument indexation
\citep[\citepage{584}]{jacques2021grammar},
and both 3\textto 3\textto 1 and 3\textto 1\textto 3 
are equivalent to 3\textto 1 in argument indexation
\citep[\citepage{310}]{jacques2021grammar}.
On the other hand, 3\textto 1\textto 2 becomes 3\textto 1,
and 3\textto 2\textto 1 becomes 3\textto 2,
which means if the causee and the object are in a local configuration,
the causee is indexed on the main verb.
We can therefore say that the argument playing the role of P in polypersonal indexation
is always the most salient argument,
either according the standard of speech act participation or according to the standard of agentivity.
The mechanism of argument indexation in the causative
therefore is comparable to the analysis of English tense and aspect 
by \citet[\citesec{7.4.1}]{wiltschko2014universal},
where the aspect value of a clause selects the start or the end or the totality of an event
as the reference time (c.f. the argument playing the role of P),
and then the reference time and the utterance time (c.f. the A argument)
are compared to decide the tense (c.f. the inverse marker).

The causative may be applied twice,
with the highest argument being the agent
and the second highest argument (introduced by the inner causative) being the instrument.
The meaning then is \translate{$X$ makes [[$Z$ do sth.] with $Y$]}.
The A-like argument in indexation seems to be the highest argument,
and the P-like argument seems to be the most salient in the rest of arguments:
hence a 1\textto 3\textto 2\textto 3 configuration
is morphologically the same as 1\textto 2
\citep[\citepage{848}, (67)]{jacques2021grammar}.
This further affirms the analysis outlined above.

An interesting phenomenon is the theme of secundative verbs
cannot be indexed and it cannot be first or second person.
This seems to again suggest some relation between
the ability to trigger argument indexation and animacy,
and again can be explained by stipulating that in secundative constructions,
we have a R\textto T structure,
and probably the inverse configuration in this structure is forbidden. 

\paragraph*{Deviations from the ideal inverse system}
The ability for an argument to trigger argument indexation
depends on the animacy of the argument.
Inanimate arguments rarely trigger number indexation in some intransitive constructions,
including the existential construction and some dynamic verbs
\citep[\citesec{14.6.1.1}]{jacques2021grammar}.


\subsection{\ac{tame} categories}


Decomposition of \acs{tame} categories 
in the same way English \form{he [is playing] football} 
is analyzed as ``present (imperfect) progressive'' is not necessary
if there is no need for cross-linguistic comparison:
although we are able to distinguish between e.g. the present tense and the past tense,
or the progressive aspectuality v.s. the non-progressive one,
not every combination of attested tense, aspect, modality and evidentiality values in Japhug 
can be morphologically realized.

The morphological realization of these categories is remarkable
Their main exponents are the alternation of the orientation prefix.
Some \acs{tame} categories insert a fixed prefix 
into the orientation prefix slot;
others choose one of the four prefixes that have the same directional meaning 
in \citet[\citetable{15.1}]{jacques2021grammar}. 

\begin{todobox}{Interaction between TAME and orientation}{tame-orientation-morphology}
    Does the TAME marking override the lexically determined orientation prefix 
    or the semantically significant orientation prefix of a orientable prefix?
\end{todobox}

\section{Noun phrase}

\paragraph*{Compounds} 

\begin{todobox}{Compounds: phrasal, or stem-level?}{compound-analysis}
    Japhug seems to have ``real'' compounds and not just
    the nominal attributive construction in English:%
    \footnote{
        As in \form{noun phrase}.
    }
    inalienable nouns, if appearing as the second element in a compound,
    have no possessive marker \citep[\citepage{15}]{jacques2021grammar}. 
    This however can be analyzed as a realizational effect as well:
    we need other proofs to show that the inalienable noun 
    loses its subcategorization in the compounding process
    and therefore appears purely as a stem.
\end{todobox}

\paragraph*{Counting words}

\begin{todobox}{Counted noun constructions: what's the head?}{counted-noun-headness}
    In \citet[\citepage{10}]{jacques2021grammar},
    it is mentioned that counted nouns correspond to classifiers in Chinese grammar.
    An interesting problem is in a counted noun construction in Japhug,
    which element is the head of the NP.
    In English, the head of the \form{one of} construction is likely \form{one},
    because it can undergo modifications:
    \form{this specific one of \dots},
    so the partitive reading is merely semantic.
    If no direct modification is possible to the counted noun,
    maybe the counted noun has already collapsed into a classifier in this usage.
\end{todobox}

\paragraph*{Noun valency class}
Japhug has the distinction between alienable and inalienable nouns.
An inalienable noun has to have a possessor
which is indexed on the noun,
and this possessor can be understood as a core argument of the head noun
\citep[\citepage{116}]{jacques2021grammar}.

\paragraph*{Coordination}
One interesting feature of the Japhug comitative 
is it's also considered when deciding the number of an \acs{np}
\citep[\citepage{332}]{jacques2021grammar};
but it's still not prototypically a conjunction \citep[\citepage{420}]{jacques2021grammar}:
the \acs{np} following the comitative marker 
may be omitted, 
agreeing with the fact 
that the head noun of an \acs{np}
can also be dropped \citep[\citepage{425}]{jacques2021grammar}.
(In English this is only possible for clauses:
in informal writing and speech people may start with a sentence with \form{and},
i.e. a conjunction construction 
without the first branch,
but they never do so to an \acs{np}.)
The \acs{np} after the comitative marker can also be relativized.
Thus the comitative suffix is still recognized as a type of 
modification.

\section{Ideophones}

The category of ideophone occupies mainly manner adverbial positions
\citep[\citesec{10.1.7}]{jacques2021grammar}.
Its main difference with the adverb class 
is its morphology \citep[\citesec{10.1.2}]{jacques2021grammar}
and phonology \citep[\citesec{10.1.5}]{jacques2021grammar}.

\section{Analyzed examples}


The sentence final stative verb \form{ŋu} be.\category{fact} is 
listed as a stative verb in the dictionary
and seems to take the constituents before it 
as a finite complement clause (TODO: or report speech? see the condition on \citepage{1317}),
which is without any explicit complementizer. 
But also see \citepages{1081, }

\chapter{The verb}

Japhug is a heavily inflected language, 
and most grammatical categories in the clause 
have something to do with the verb.
The structure of the verb can be divided into 
the outer prefix chain (\citealt[\citetable{11.1}]{jacques2021grammar}),
the extended stem, 
and the suffix chain (\citealt[\citesec{11.3}]{jacques2021grammar});
the extended stem contains the stem, 
which may undergo stem alternation (\citealt[\citechap{12}]{jacques2021grammar}),
and inner prefixes related to valence alternation
(\citealt[\citesec{11.2.2}]{jacques2021grammar}).

Whether this complex is to be regarded as one \emph{morphological} or \emph{phonological} word 
is discussed in \citesec{11.6}
in the reference above.
Recognition of wordhood, expectedly, is not self-evident;
\citet{prins2011web} provides an analysis of another rGyalrong language, Jiaomuzu, 
and in this thesis the term \term{verb phrase} 
(i.e. verbal complex in this note) is used,
skipping the discussion on what is a word.
In \citet[\citetable{11.3}]{jacques2021grammar}
four domains are defined using various criteria.

Domain A is defined according to both syntactic and morphological reasons.
What's shown in 
\citetable{11.3} contains all formatives that are relevant to verb inflection,
and they have non-adjacent dependencies,
so strong dependencies exist between them:
these formatives are realized in the same batch 
in clause building.
Now syntactically, the formative \form{-ci} in slit +4 
is selected by some modal prefixes in slot -6,
so the two slots belong to the same system;
on the other hand, outside the +4 and -6 slots 
we only have clitics which clearly belong to systems with higher positions
\citep[\citesec{11.6.2}]{jacques2021grammar},
and thus all -- and only -- formatives in \citetable{11.3}
constitute a syntactic word,
with the same \emph{syntactic} status of a verb-plus-auxiliary verbal complex or a
``verb phrase'' in Dixon's definition (i.e. without internal complements). 
Morphologically, no element is able to intervene 
between two slots in the template, 
so we say this batch is realized as a single morphological word 
instead of a verbal complex.

Domain B is about \emph{obligatoriness}:
thus the +4 slot is not included.
Domain C is defined according to prosodic reasons.

\chapter{Verb frames}

In this chapter we summarize basic verb frames in Japhug.
This is done in \citet[\citepage{14}]{jacques2021grammar},
which is mostly about the finite clause without any valency alternation operations.

\chapter{\acs{tame} marking}


\paragraph*{Realizational details}

The \acs{tame} categories in Japhug 
are introduced in \citet[\citechap{21}]{jacques2021grammar}.
Morphologically speaking, there are three systems
\citep[\citepage{516}]{jacques2019egophoric}:
\begin{itemize}
    \item The \category{primary} system, 
        whose main exponents are stem alternation, the orientation preverb,
        and the modal prefix;
        all of these happens in the template of the verb \citep[\citetable{21.1}]{jacques2021grammar}.
        The grammatical categories marked in this system are listed below.
    \item The \category{secondary} system, 
        which also happens in the inflection pattern of the verb  
        but is about aspectual and modal categories 
        largely orthogonal to the grammatical categories marked by the \category{primary} system 
        \citep[\citesec{21.6}, \citesec{21.7}]{jacques2021grammar}. 
    \item The \category{periphrastic} system, 
        whose surface form is similar to complement clause constructions 
        with the copula \form{ŋu}.
        The copula in periphrastic constructions 
        never takes any argument indexation markers \citep[\citepage{1090}]{jacques2021grammar},
        and if we are to analyze the constructions as complement clause constructions,
        then the literal reading will be something like  
        ``it's the case that an event happens'',
        with all the contents before the copula 
        being a complement clause of the copula.
        In the follows however it can be seen that 
        the \acs{tame} categories on the copula 
        is complementary with the lexical verb,
        and hence the periphrastic constructions are to be analyzed 
        as single-clause constructions.

        In periphrastic constructions 
        the main verb is often in \emph{finite} forms \citep[\citepage{1081}]{jacques2021grammar};
        Japhug periphrastic conjugation thus has 
        a difference with English or Latin periphrastic conjugation,
        where what are used in periphrastic \acs{tame} categories
        are \emph{non-finite} verb forms.
        The reason possibly is because the periphrastic \acs{tame} categories in Japhug
        historically comes from finite complement clause constructions.

        TODO: is there any constraints on the distribution of participle or infinitive?
\end{itemize}
The interaction of the three morphological systems makes Japhug \acs{tame} system extremely complicated;
some periphrastic categories seem to have identical semantics with 
\category{primary} and \category{secondary} \acs{tame} marking devices \citep[\citepage{1092}]{jacques2021grammar};
whether there are hidden nuances is still not clear.

Besides the verbal complex, \acs{tame} categories are also marked by 
sentential adverbs and sentence-final particles 
(\citealt[\citepage{518}]{jacques2019egophoric}; \citealt[\citesec{21.8}]{jacques2021grammar}).

\paragraph*{Interaction with other categories}

\acs{tame} categories interact strongly with 
the lexical aspect of the main verb,
which can be crudely divided into being stative 
and being dynamic (e.g. \citealt[\citesec{21.3.1.2}]{jacques2021grammar}),
the person of the subject, and  TODO: other properties

\paragraph*{Attested categories}

The following subcategories can be recognized in Japhug: 
\begin{itemize}
    \item Subjective evaluation: some \acs{tame} categories can be used to express 
    the feeling of the speaker (\citesec{21.3.2.4}) -- but is this a grammatical category?
    \item Evidentiality.
    Japhug has a highly complicated evidentiality system 
    \citep[\citetable{31.4}]{jacques2015sketch}.
    The values of evidentiality attested include 
    the generic, the factual, the sensory, the egophoric, and the inferential.

    A three-fold distinction can be observed with the non-past tense
    (\citealt[\citesec{21.3.4}]{jacques2021grammar}; 
    \citealt[\citepage{517}]{jacques2019egophoric}): 
    the factual or common knowledge (\citesec{21.3.1.2}), 
    the sensory (\citesec{21.3.2.2}),
    and the egophoric.

    Actually there is a fourth, bleached ``generic'' evidentiality value 
    with the non-past tense.
    The generic non-past \acs{tame} configuration 
    with no other non-trivial \acs{tame} marking 
    is known as the \category{imperfective} \citep[\citesec{21.2}]{jacques2021grammar}.
    This however seems to be very infrequent in main clauses
    without periphrastic auxiliaries (\citepage{1087}),
    indicating a strong preference for Japhug 
    to include a non-trivial evidentiality value in non-past sentences.
    
    The inferential evidentiality value appears only with the past tense,
    possibly because of semantic reasons: 
    an event happening now usually doesn't need to be ``inferred'',
    and this rarity means even this category existed historically,
    it has long been eroded.
    With the past tense, 
    we have a dichotomy between the generic evidentiality and the inferential evidentiality.
    It's impossible to morphologically mark the sensory evidentiality
    with the past tense, 
    possibly again because of the infrequency of this configuration.  
    It should however be noted that 
    the sensory can still be combined with the past tense 
    by periphrastically attaching a sensory copula 
    to the \category{aorist} (i.e. \category{past perfective} -- see below) 
    and the \category{past imperfective}
    which have default evidentiality
    (\citealt[\citesec{21.5.1.8}, \citesec{21.5.3.5}]{jacques2021grammar}; 
    \citealt[\citepage{518}]{jacques2019egophoric}). 
    On the other hand, the factual evidentiality and the egophoric evidentiality
    are never seen together with the past tense.

    \item Primary tense. 
    The distinction between the past and the non-past
    can be clearly identified (\citesec{23.3}, \citesec{23.5}),
    partly from the interaction with evidentiality.
    The meaning of future
    is regularly expressed in the \category{factual} category (\citepage{1102}),
    and thus is sometimes recognized as the future tense 
    or ``factual evidentiality in the future tense'' 
    \citep[\citepage{518}]{jacques2019egophoric}.
    This however seems to be the natural extension of the meaning 
    of the present tense 
    (c.f. English \form{the next high tide is around 4 this afternoon}; 
    \citealt[\citepage{131}, {[20]}]{cgel}),
    and in \citet{jacques2021grammar}, 
    the future tense is not recognized as a grammatical tense in Japhug
    \citep[\citepage{1102}, (46)]{jacques2021grammar}.

    \item Modality. 
        In Japhug, once the irrealis situation occurs, 
        it seems other \acs{tame} categories are not available. 
        There are four types of irrealis modalities falling in this domain:
        the \category{irrealis}, the \category{dubitative}, 
        the \category{imperative}, and the \category{prohibitive}
        \citep[\citesec{21.4}]{jacques2021grammar}.

    \item It seems the anterior category
        (the \category{perfect} category in English)
        is absent in Japhug.

    \item Aspect: the imperfective-perfective distinction.
        The non-past categories are always inherently imperfective:
        no perfective aspectuality is seen with non-past tense 
        \citep[\citepage{517}]{jacques2019egophoric},
        again possibly because of semantic reasons,
        since the perfective may be semantically identified with the past.
        The imperative-perfective distinction can only be seen 
        with the past tense \citep[\citetable{21.1}, note that the \category{aorist} is also known as 
        the \category{past perfective}; \citepages{1135, 1143}]{jacques2021grammar}.
    \item Aspects TODO: terminative, continuative, etc. TODO: the position of the inchoative aspect 
        The progressive aspect (\citesec{21.6}; note that \citetable{21.8}: compatibility?) 
\end{itemize}

The composition between these categories is not orthogonal, 
and no independent morphological exponent can be identified for 
each separate \acs{tame} categories mentioned above
(but in periphrastic conjugation, 
distribution of these primitive \acs{tame} categories 
onto the main verb and the auxiliary copula 
can be observed; \citealt[\citepage{1089}, (7)]{jacques2021grammar}).


\paragraph*{Primary categories}

By combining these categories and removing unattested combinations, 
we find the 11 \category{primary} \acs{tame} categories 
listed in \citet[\citepage{21.1}]{jacques2021grammar}.
The realis part is replicated in \prettyref{tbl:realis-tam}.
Note that the \category{aorist} is just the \category{past perfective} in
\citet[\citetable{31.4}]{jacques2015sketch}, 
and indeed \citet[\citepage{1135}]{jacques2021grammar}
makes it clear that the category expresses past perfective events.
It should be noted that the mapping from tense, aspect and evidentiality features 
to concrete \category{primary} categories 
is only unidirectional:
for some categories there are usages that can't be full described by \prettyref{tbl:realis-tam}.
For example, the \category{imperfective} category 
has hortative meanings sometimes \citep[\citesec{21.2.5}]{jacques2021grammar}; 
and the concrete \category{primary} categories also have non-trivial interaction with the lexical aspect of verbs 
\citep[\citesec{21.2.6}, \citesec{21.2.7}]{jacques2021grammar}.
 
\begin{table}[H]
    \centering
    \caption{Analysis of Japhug realis \acs{tame} categories;
    the two sensory past cells may be filled by 
    \category{periphrastic narrative} \citep[\citesec{21.5.1.8}]{jacques2021grammar}
    and \category{periphrastic imperfective narrative} \citep[\citepage{1157}]{jacques2021grammar},
    although the two constructions are only in use in a part of the population; 
    the meaning of the two constructions are also mostly similar to 
    the \category{inferential} and the \category{inferential imperfective}. }
    \label{tbl:realis-tam}
    \scriptsize
    \begin{tabular}{lllllll}
        \toprule
                                &                          & \multicolumn{5}{c}{evidentiality}                                                                                                                                                                        \\ \cmidrule(l){3-7} 
        \multirow{-2}{*}{tense} & \multirow{-2}{*}{aspect} & generic                           & factual                           & sensory                           & egophoric                         & inferential                                              \\ \midrule
        non-past                & imperfective             & \category{imperfective}             & \category{factual}                  & \category{sensory}                  & \category{egophoric present}        & \cellcolor[HTML]{C0C0C0}{\color[HTML]{C0C0C0} \category{}} \\ \midrule
                                & imperfective             & \category{past imperfective}        & \cellcolor[HTML]{C0C0C0}\category{} & \cellcolor[HTML]{C0C0C0}\category{} & \cellcolor[HTML]{C0C0C0}\category{} & \category{inferential}                                     \\ \cmidrule(l){2-7} 
        \multirow{-2}{*}{past}  & perfective               & \category{aorist} & \cellcolor[HTML]{C0C0C0}\category{} & \cellcolor[HTML]{C0C0C0}\category{} & \cellcolor[HTML]{C0C0C0}\category{} & \category{inferential imperfective}                        \\ \bottomrule
    \end{tabular}
\end{table}

\paragraph*{Periphrastic conjugations} 

In a periphrastic construction we have a main verb in one of the non-past forms 
and one copula carrying tense and aspect information.
The most prevalent periphrastic constructions 
are those formed by combining the copula and the \category{imperfective} 
\citep[\citepage{1089}]{jacques2021grammar},
but constructions with \category{factual} \citep[\citesec{21.3.1.4}]{jacques2021grammar}
and \category{sensory} verb forms are also possible. 

\chapter{Clause embedding}

\paragraph*{Manner serial verb construction}

The term \term{serial verb construction} is a cover-all terms for constructions 
where there are two verbs (or at least words that look like verbs) found in a clause 
but the clause is clearly not a complement clause construction or an auxiliary verb construction.
The underlying structure can be extremely heterogeneous:
control construction, coordination on the level of core verb phrase 
(hence the two verbs share the same \ac{tame} marking), 
manner adverbial construction where the modifier is a core verb phrase, 
non-prototypical auxiliary clause construction, and even more.
In Japhug attested serial verb constructions can all be placed under the 
category of manner adverbial construction \citep[\citesec{25.4.1}]{jacques2021grammar}.

In the Japhug serial verb construction,
the modified main verb is the second verb; 
the modifier precedes the main verb and may be one of the follows:
\begin{itemize}
    \item A deideophonic verb \citep[\citesec{25.4.1.1}]{jacques2021grammar};
    \item A similative verb phrase containing \form{fse} or \form{stu} 
        and their semi-object, 
        with the meaning of \translate{do like this};
    \item A verb phrase describing simultaneous action \citep[\citesec{25.4.1.4}]{jacques2021grammar},
    possibly followed by the emphasis marker \form{zo};
    \item Other verbs of manner. 
\end{itemize}
In all cases, the first verb (or verb phrase) is the modifier; 
discourse linker \form{tɕe} can appear between the modifier and the main verb
\citep[\citepage{1408}, (73)]{jacques2021grammar}, 
demonstrating that the two verbs are two morphological words. 

\paragraph*{Degree serial verb construction} 

Interestingly, a serial verb construction describing the degree of an action 
can also be found in Japhug; 
in this construction the verb describing the degree is the \emph{second} verb.
One way to analyze the historical origin of the construction 
is to treat everything before the stative degree verb as its complement,
and thus \citet[\citepage{1410}, (76)]{jacques2021grammar} 
may be analyzed as \translate{that the elders who knew [traditional stories] well die is finished.}
This however is against the observation that the agreement marker on the second verb 
agrees with the subject (and therefore is plural in the above case, 
not singular as expected for an impersonal verb),
which clearly says that the construction is indeed monoclausal.

Another possible historical origin of the construction 
is verb phrase-level coordination,  
something with the meaning of 
English \form{?The elders who knew traditional stories died and they finished.}
The English example here is awkward, 
partly because \form{die} is a state-change verb in English 
and therefore specifying its progress is semantically unacceptable.


\bibliographystyle{plainnat}
\bibliography{gyalrong.bib}

\end{document}