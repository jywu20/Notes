\documentclass[a4paper, oneside, 12pt]{report}

\usepackage[T1]{fontenc}
\usepackage{libertinus}
\usepackage{underscore}
\usepackage{geometry}
\usepackage{float}
\usepackage{titling}
\usepackage{titlesec}
\usepackage{paralist}
\usepackage{footnote}
\usepackage{enumitem}
\usepackage{amsmath, amsthm}
\usepackage{gb4e}
\noautomath
\usepackage{bbm}
\usepackage{textcomp}
\usepackage{soul}
\usepackage{graphicx}
\usepackage{siunitx}
\usepackage[table,xcdraw]{xcolor}
\usepackage{tikz}
\usepackage[ruled, vlined, linesnumbered, noend]{algorithm2e}
\usepackage{xr-hyper}
\usepackage[colorlinks, citecolor = purple, bookmarksnumbered]{hyperref} % linkcolor=black, anchorcolor=black, citecolor=black, filecolor=black
\usepackage[most]{tcolorbox}
\usepackage{caption}
\usepackage{subcaption}
\usepackage{booktabs}
\usepackage{multirow}
\usepackage[figuresright]{rotating}
\usepackage{acro}
\usepackage[round]{natbib} 
\usepackage{nameref,zref-xr}
\zxrsetup{toltxlabel}
\zexternaldocument*[alignment-]{../alignment/alignment}[alignment.pdf]
\zexternaldocument*[exercise1-]{../Exercise/2021-3}[2021-3.pdf]
\zexternaldocument*[method-]{../methodology/glossing}[glossing.pdf]
\usepackage{prettyref}

\geometry{left=3.18cm,right=3.18cm,top=2.54cm,bottom=2.54cm}
\titlespacing{\paragraph}{0pt}{1pt}{10pt}[20pt]
\setlength{\droptitle}{-5em}

\DeclareMathOperator{\timeorder}{\mathcal{T}}
\DeclareMathOperator{\diag}{diag}
\DeclareMathOperator{\legpoly}{P}
\DeclareMathOperator{\primevalue}{P}
\DeclareMathOperator{\sgn}{sgn}
\newcommand*{\ii}{\mathrm{i}}
\newcommand*{\ee}{\mathrm{e}}
\newcommand*{\const}{\mathrm{const}}
\newcommand*{\suchthat}{\quad \text{s.t.} \quad}
\newcommand*{\argmin}{\arg\min}
\newcommand*{\argmax}{\arg\max}
\newcommand*{\normalorder}[1]{: #1 :}
\newcommand*{\pair}[1]{\langle #1 \rangle}
\newcommand*{\fd}[1]{\mathcal{D} #1}
\newcommand*{\textto}{$\to$}
\newcommand*{\textgt}{$>$ }

\newcommand*{\citesec}[1]{\S~{#1}}
\newcommand*{\citechap}[1]{Ch~{#1}}
\newcommand*{\citefig}[1]{Fig.~{#1}}
\newcommand*{\citetable}[1]{Table~{#1}}
\newcommand*{\citepage}[1]{p.~{#1}}
\newcommand*{\citepages}[1]{pp.~{#1}}
\newcommand*{\citefootnote}[1]{fn.~{#1}}
\newcommand*{\citechapsec}[2]{\citechap{#1}.\citesec{#2}}

\newrefformat{sec}{\citesec{\ref{#1}}}
\newrefformat{fig}{\citefig{\ref{#1}}}
\newrefformat{tbl}{\citetable{\ref{#1}}}
\newrefformat{chap}{\citechap{\ref{#1}}}
\newrefformat{fn}{\citefootnote{\ref{#1}}}
\newrefformat{box}{Box~\ref{#1}}
\newrefformat{ex}{\ref{#1}}


% color boxes

\tcbuselibrary{skins, breakable, theorems}

\AtBeginEnvironment{infobox}{\small}
\AtBeginEnvironment{theorybox}{\small}

\newtcbtheorem[number within=chapter]{infobox}{Box}{
    enhanced,
    boxrule=0pt,
    colback=blue!5,
    colframe=blue!5,
    coltitle=blue!50,
    borderline west={4pt}{0pt}{blue!65},
    sharp corners,
    fonttitle=\bfseries, 
    breakable,
    before upper={\parindent15pt\noindent}}{box}
\newtcbtheorem[number within=chapter, use counter from=infobox]{theorybox}{Box}{
    enhanced,
    boxrule=0pt,
    colback=orange!5, 
    colframe=orange!5, 
    coltitle=orange!50,
    borderline west={4pt}{0pt}{orange!65},
    sharp corners,
    fonttitle=\bfseries, 
    breakable,
    before upper={\parindent15pt\noindent}}{box}
\newtcbtheorem[number within=chapter, use counter from=infobox]{learnbox}{Box}{
    enhanced,
    boxrule=0pt,
    colback=green!5,
    colframe=green!5,
    coltitle=green!50,
    borderline west={4pt}{0pt}{green!65},
    sharp corners,
    fonttitle=\bfseries, 
    breakable,
    before upper={\parindent15pt\noindent}}{box}
\newtcbtheorem[number within=chapter, use counter from=infobox]{todobox}{Box}{
    enhanced,
    boxrule=0pt,
    colback=red!5,
    colframe=red!5,
    coltitle=red!50,
    borderline west={4pt}{0pt}{red!65},
    sharp corners,
    fonttitle=\bfseries, 
    breakable,
    before upper={\parindent15pt\noindent}}{box}

% Shorthands
\newcommand*{\concept}[1]{\textbf{#1}}
\newcommand*{\term}[1]{\emph{#1}}
\newcommand{\form}[1]{\emph{#1}}

\newcommand{\redp}{\textasciitilde}

\newcommand{\deictictime}{T$_{\text{d}}$}
\newcommand{\referredtime}{T$_{\text{r}}$}
\newcommand{\orientationtime}{T$_{\text{o}}$}

\DeclareAcronym{blt}{short = BLT, long = Basic Linguistic Theory}
\DeclareAcronym{cgel}{short = CGEL, long = The Cambridge Grammar of the English Language}
\DeclareAcronym{dm}{short = DM, long = Distributed Morphology}
\DeclareAcronym{tag}{long = Tree-adjoining grammar, short = TAG}
\DeclareAcronym{sfp}{long = sentence-final particle, short = SFP}
\DeclareAcronym{np}{long = noun phrase, short = NP}
\DeclareAcronym{vp}{long = verb phrase, short = VP}
\DeclareAcronym{pp}{long = preposition phrase, short = PP}
\DeclareAcronym{advp}{long = adverb phrase, short = AdvP}
\DeclareAcronym{cls}{long = classifier, short = CLS}
\DeclareAcronym{dist}{long = distal, short = DIST}
\DeclareAcronym{prox}{long = proximate, short = PROX}
\DeclareAcronym{dem}{long = demonstrative, short = DEM}
\DeclareAcronym{classify}{long = classifier, short = \textsc{cl}}
\DeclareAcronym{dur}{long = durative, short = DUR}
\DeclareAcronym{neg}{long = negative, short = \textsc{neg}}
\DeclareAcronym{cc}{long = copular complement, short = CC}
\DeclareAcronym{cs}{long = copular subject, short = CS}
\DeclareAcronym{tame}{long = {tense, aspect, mood, evidentiality}, short = TAME}
\DeclareAcronym{past}{long = past, short = PST}
\DeclareAcronym{nonpast}{long = non-past, short = NPST}
\DeclareAcronym{present}{long = present, short = PRES}
\DeclareAcronym{progressive}{long = progressive, short = \textsc{poss}}
\DeclareAcronym{perfect}{long = perfect, short = \textsc{perf}}
\DeclareAcronym{passive}{long = passive, short = \textsc{pass}}
\DeclareAcronym{copula}{long = copula, short = COP}
\DeclareAcronym{possessive}{long = possessive, short = \textsc{poss}}
\DeclareAcronym{coca}{long = Corpus of Contemporary American English, short = COCA}
\DeclareAcronym{sap}{long = speech act participant, short = SAP}

\newcommand{\asis}[1]{\textsc{#1}}
\newcommand{\oneof}[1]{{#1}}
\newcommand*{\homo}[2]{#1$_{\text{#2}}$}
\newcommand{\category}[1]{\textsc{#1}}
\newcommand{\formcat}[1]{\textsc{#1}}
\newcommand{\emptymorpheme}{$\emptyset$}
\newcommand*{\fromto}[2]{\langle {#1}, {#2} \rangle}

\newcommand{\alignment}{\href{../alignment/alignment.pdf}{my notes about alignment}}
\newcommand{\exerciseone}{\href{../Exercise/2021-3.pdf}{this exercise}}
\newcommand{\method}{\href{../methodology/glossing.pdf}{this note about my understanding of descriptive grammars}}

\newcommand{\ala}{à la}
\newcommand{\translate}[1]{`#1'}
\newcommand{\vP}{\textit{v}P}

% Make subsubsection labeled
\setcounter{secnumdepth}{4}
\setcounter{tocdepth}{4}
% reset example counter every chapter (but do not include the chapter number to the label)
\counterwithin{exx}{chapter} 

% Reference formats
\renewcommand{\bibname}{References}
\setcitestyle{aysep={}} 

% List format
\setlist[enumerate,1]{label=\alph*\upshape)}

\title{Reading notes of A Grammar of Japhug}
\author{Jinyuan Wu}

\begin{document}

\maketitle

The theoretical orientation is already well-documented in my notes about English, Latin and Mandarin Chinese.

\chapter{Introduction}

\section{Classification}

\begin{todobox}{Japhug and other Sino-Tibetan languages}{relation}
    \begin{itemize}
        \item Tangut: direct-inverse
        \item Direct-inverse in proto-ST
        \item Relation with Situ
        \item Dialects within Japhug
    \end{itemize}
\end{todobox}

\section{Sociolinguistic status}

\begin{todobox}{Sociolinguistic status}{sociolinguistic}
    \begin{itemize}
        \item Relation with Ando Tibetan
        \item Relation with Mandarin
    \end{itemize}
\end{todobox}

\chapter{Typological overview}

In this chapter we do a round-by-round survey of Japhug grammar.
We start with a very rough top-down anatomy of the structure of Japhug clauses
(\prettyref{sec:grammatical.clause.template}),
followed by a bottom-up examination of grammatical categories and relations in the clause
in the rest of \prettyref{sec:grammatical.clause}.
We then do the same for the noun phrase.
We finally list points to investigate in the lexicon of Japhug.

\begin{todobox}{Grammatical sketch TODO list}{grammatical-todo}
    \begin{itemize}
        \item NP
        \item POS list
    \end{itemize}
\end{todobox}

\section{Clause structure}\label{sec:grammatical.clause}

\subsection{Structural template}\label{sec:grammatical.clause.template}

\paragraph*{Clause in syntax}
We start our discussion on Japhug grammar 
by first dividing utterances into simple clauses.
Subordinated clauses,%
\footnote{
    The term \term{subordination} here is used to refer to
    embedding a clause into the peripheral system of another clause.
    Subordination in the general sense is referred to simply as \term{clause combining} in this note.
}
like temporal clauses or conditional clauses,
seem to always appear before the main clause \citep[\citechap{25}]{jacques2021grammar}.
Coordination may be marked by coordination linker \form{tɕe} or \form{qʰe} or simple parataxis
\citep[\citesec{25.1.6}]{jacques2021grammar}.

In both subordination and coordination,
clauses involved may share the auxiliary copula if there is any
\citep[\citepage{47}, (40); \citepage{1091}, (10)]{jacques2021grammar}.
It seems changing the subject in the middle is also possible
(\prettyref{box:periphrastic-vp-coordination}).

\paragraph*{Speech-act marking and information packaging}
\label{sec:grammatical.clause.template.nucleus-identification}
A simple clause is a nucleus clause with top-level categories,
namely speech-act and information packaging.
We start with the latter.

Complete sentences (i.e. clauses that are utterances on their own)
and reported speech \citep[\citesec{24.2.5.1}]{jacques2021grammar} allow \acp{sfp},

\begin{todobox}{The position of orientation system}{orientation}
    \begin{itemize}
        \item What's the syntactic position of the orientational system?
        \item And what's the position of associated motion?
    \end{itemize}
\end{todobox}

Information packaging is mostly reflected by alternation of 
the default SOV order and orders of adverbials outlines below.
Information packaging constructions include
right dislocation for afterthought, disambiguation, or emphasis 
\citep[\citesec{22.1.3}]{jacques2021grammar},
left dislocation for topicalization 
\citep[\citepage{1189}]{jacques2021grammar},
or focalization \citep[\citepage{1190}]{jacques2021grammar}.%
\footnote{
    \citep[\citepage{1190}]{jacques2021grammar} has the OSV order,
    and Jacques mentions that the subject is focalized.
    It is possible that this example is a topic-focus-verb construction.
}
Topicalization in Japhug may result in a chain of nucleus clauses sharing one topic
\citep[\citepage{1190}, (11)]{jacques2021grammar}.

Japhug \category{finite} clauses are clauses 
allowing the full morphological complexity of the main verb.
Sentential clauses and reported speeches are \category{finite}.
Other embedded finite constructions do not allow \acp{sfp}. 
Japhug also has several nonfinite constructions,
including infinitives, participles, and coverbs,
with limited verbal morphological complexity and \ac{tame} marking.  

\begin{todobox}{Information packaing in embedded clauses}{information-packaging-embedded}
    Topicalization, etc. possible in complement or relative clause?
\end{todobox}

\paragraph*{Arguments in the nucleus clause}\label{sec:grammatical.clause.template.arguments}
We can say with confidence that the basic order of Japhug nucleus clauses is SOV.
Justification of concepts like subject or object is discussed in 
\prettyref{sec:grammatical.clause.internal} and \prettyref{sec:grammatical.clause.subject}.
All arguments -- and of course adjuncts -- of clauses can be omitted;
in this way it is possible that the nucleus clause consists of only verb.
The arguments are to be recovered from argument indexation
(\prettyref{sec:grammatical.clause.direct-inverse}).
It should be noted that pragmatic argument omission or ``pro-drop''
is different from argument omission that's related to verb frame alternation 
(\prettyref{sec:grammatical.clause.internal.optional-core}).
Finally, the meaning of the argument-adjunct distinction is discussed in 
\prettyref{sec:grammatical.clause.internal}.

\paragraph*{Adjuncts in the nucleus clause}\label{sec:grammatical.clause.template.adverb}
Temporal expressions indicating the absolute time usually appear before the subject
\citep[\citepage{344}, (167); \citepage{283}, (123)]{jacques2021grammar}.
This probably is because the absolute time sets the stage for the event
and by default is topical,
and their syntactic position is comparable to that of subordinated clauses.%
\footnote{
    The situation is similar to that in English:
    temporal phrases like \form{last year} almost always appear at the margin of a clause.
}

On the other hand, \ac{tame} adverbs
appear after the subject and before the direct object
\citep[\citepages{1200-1201,1210}]{jacques2021grammar}.
Example (46) in \citet[\citepage{1200}]{jacques2021grammar}
suggests that the aspectual adverb precedes the manner adverb.

In some examples the \emph{object} appears before a \ac{tame} adverb,
but this likely arises from topicalization as a pause can be observed after the object
\citep[\citepage{1210}, (82)]{jacques2021grammar}.

\begin{todobox}{Positions of adverbs}{adverb-position}
    Compare the positions of TAME and locational adverbials.
    
    Also, is it possible to have locational stage-setting adverbials?
\end{todobox}

\paragraph*{Auxiliary verb constructions}
In periphrastic conjugation constructions, 
the copula that carries the main \ac{tame} information 
is put at the final position, 
and the main verb precedes it; 
when there are several main verbs coordinated in a clause, 
only one final copula needs to appear \citep[\citepages{1090-1091}]{jacques2021grammar}.

\begin{todobox}{Periphrastic conjugation and verb phrase coordination}{periphrastic-vp-coordination} 
    Note that the subject seems to be changed in the middle 
    (the problem is the meaning of \form{ɯ-ŋgɯ} \translate{\category{3sg.poss}-inside}
    and the constituents introduced by it: 
    does it mean \translate{as three very beautiful girls}, 
    and therefore is an adjunct, or is it a new subject? 
    Also verb phrase coordination is related to syntactic ergativity -- 
    see the discussion on clause pivot)
    
    If the coordination construction indeed works on the level of clause and not VP
    and therefore allows changing the subject in the middle,
    another question arises:
    sharing TAME markers between two clauses is highly unusual.
    This can't be a purely realizational process
    because of the long-range nature of copula sharing.
    So this has to involve (purely) syntactic coordination,
    and the syntactic position of the TAME copula is then higher than the subject,
    which is unusual.
    Another analysis of the construction is that we are looking at a biclausal construction,
    and the coordination happens to a series of nonfinite clauses,
    But then TAME marking can't be biclausal. 
    
    It seems currently the best option is to analyze this construction as coordination of ``small clauses''
    whose \ac{tame} values are then specified by the end-of-sentence copula.
    Note that this construction seems to be only available for \category{imperfective} verbs,
    which appears frequently in periphrastic conjugation
    and likely only specifies the aspectual value. 
    The tense and evidentiality categories coded on the copula are then higher than the subject.
\end{todobox}

\paragraph*{Clausal structural template}
In summary, a Japhug simple clause is a nucleus clause with possible left- or right-dislocations
and/or temporal expressions expressing the absolute time of the event
(\prettyref{box:adverb-position}),
an the nucleus clause, from left to right,
consists of the subject,
tense and aspect adverbs,
manner and locational phrases,
core arguments, the main verb, and a possible auxiliary copula;
all arguments and adjuncts are not obligatory,
presumably because enough information has been coded in the verb.

\begin{todobox}{Other forms of the nucleus clause}{nucleus-clause-infrequent-types}
    Serial verb constructions and nominal predicates;
    also see fossilized N-V sequences. 
\end{todobox}

\paragraph*{Speech fillers}
One final comment about the clausal template:
in spontaneous speeches, Japhug speakers use several speech fillers and not a central vowel 
to mark pause or earn some time to think about what to say next.
The speech fillers have other functions like pronoun or topic marker
and should be ignored when reading Japhug texts
\citep[\citesec{10.3}]{jacques2021grammar}.


\subsection{Syntactic pivot of the core clause, or the subject}\label{sec:grammatical.clause.subject}

\paragraph*{Pivot of the argument structure}
\label{sec:grammatical.clause.subject.argument-pivot}
Among arguments in multivalent clauses, we can easily recognize that 
A is the more or less external one
and the rest ones are more or less internal
(for instance, according to how the direct-inverse system works; \prettyref{sec:grammatical.clause.direct-inverse}).
These facts however can all be explained by defining a pivot for the \emph{argument structure},
and are not direct evidence for a pivot of \emph{the whole clause}:
it tells us nothing about the alignment type of Japhug:
in ergative languages, we still find that the A argument is the argument structure pivot,
but definitely the P argument shares various other properties with S.

\paragraph*{Diagnoses not available in Japhug}
The usual criteria for syntactic ergativity are not viable in Japhug,
as we do not have uncontroversial verb phrase-level coordination \citep{jacques2014clause}.
There are coordination constructions in which the two branches share one auxiliary,
but the branches may have different subjects
(see \prettyref{box:periphrastic-vp-coordination}).
This is likely to be a peculiarity of Japhug,
as \citet[\citepage{549}]{prins2011web} mentions 
coordination of two verb phrases sharing the same subject
in Jiaomuzu, a language close to Japhug.

\paragraph*{Several more indirect diagnoses}
If we restrict ourselves to Japhug,
there is one construction that at least partially resembles
shared subject coordination of verb phrases.
We observe that the A argument (with ergative marking) 
can be separated from the object and the verb 
by an intransitive clause \citep[\citepage{306}]{jacques2021grammar}
in which a gap coreferential with the A argument exists.
This may also be understood 
as the A argument of the main clause and the S argument of the embedded clause
being extracted and fronted,
and therefore neutralization of S and A as the clause-level pivot.

Subjecthood seems to be most clearly defined by relativization
(\citealt{jacques2016subjects};
see \prettyref{sec:grammatical.clause.internal.object} for the relation 
between relativization and objecthood).
Relativization happens only after the whole clause is finished
and a pivot position definable by relativization is definitely a clausal pivot
-- hence the \term{subject}.

Additionally, the fact that one argument may be separated from
the verb and other arguments by \ac{tame} adverbs
(\prettyref{sec:grammatical.clause.template.adverb})
may also be taken as evidence for subjecthood in Japhug.

With facts above, we find that there exists a well-defined clausal pivot in Japhug
and it is identical to the pivot of the argument structure.
Japhug therefore has a nominative-accusative pivot.

\paragraph*{Alignment in case marking}
The subjecthood defined alone the lines above does not entail
anything about case marking.
In prototypical monotransitive constructions
we observe morphological ergativity.
The existence of a direct-inverse argument indexation system (\prettyref{sec:grammatical.clause.direct-inverse})
means rules about argument indexation cannot be summarized as 
``the verb agrees with the subject'',
and therefore the subject cannot be defined according to verb morphology \citep{jacques2016subjects},
and in certain circumstances even leads to neutralization of S and P
(\prettyref{sec:grammatical.clause.direct-inverse.hierarchy}).

\subsection{The direct-inverse system}\label{sec:grammatical.clause.direct-inverse}

\paragraph*{Polypersonal indexation and argument flagging}
The sole argument of a Japhug intransitive clause is indexed on the verb.
In transitive clauses, 
two -- not just one -- argument are indexed on the verb. 
The personal affixes on the verb are only about person and number, 
and tell us nothing about the argument position of the argument 
from which they originate \citep[\citepage{543}]{jacques2021grammar}.
What argument is indexed is controlled by whether the clause is in a direct configuration
or an inverse configuration:
if the A is higher in the empathy hierarchy,
then we have a direct configuration,
and if the P is higher we have an inverse configuration.

\paragraph*{Inverse indexation v.s. inverse voice}
According to \citet{oxford2023tale},
what is known as an inverse construction may be 
(a) the change of clausal grammatical function of the arguments,
i.e. ``deep inverse'', in which the patient in a way or another gets some subject-related properties 
and hence some sort of ergativity is observed in inverse configurations,
with effects like alternations of the surface word order,
scope of \acp{np}, and reflexive pronouns (e.g. see \citealt{bruening2005algonquian}), or
(b) a ``shallow inverse'' which is mostly about argument indexation, 
for example a requirement that only agreement with a \ac{sap} 
is morphologically permissible
leading to a direct-inverse system based on the distinction between 1/2 and 3.%
\footnote{
    The deep inverse is also known as ``syntactic'' inverse or ``inverse voice'',
    and the shallow inverse is also known as ``morphological'' inverse, or ``inverse alignment''.
    This terminology however is sometimes misleading.
    For example, if we stipulate that in Italian,
    person clitics are agreement formatives
    and first/second person clitics belong to an agreement system only targeting \ac{sap}s,
    while the third person clitic belongs to another agreement system,
    then the impossibility of co-appearance of a first/second person direct object clitic 
    and a third person indirect object clitic Italian 
    can be argued to be due to locality constraints
    \citep{bianchi2006syntax}.
    The existence of a speech-act only agreement system is comparable to ``morphological inverse'',
    but the phenomena related to this system usually will not be called morphology.

    On the other hand, ``syntactic inverse'' covers both syntactic ergativity and morphological ergativity.
    This confusion is seen in the case of Japhug:
    in the inverse configuration of Japhug we see morphological ergativity,
    but this is deep inverse and therefore ``syntactic inverse'' in \citet{oxford2023tale}.
}
The deep inverse can be optional and in this case it is essentially a voice construction;
the shallow inverse should be obligatory or otherwise verbal agreement is arbitrary.
The two of course can be combined and we get a particularly strong inverse system.

In Japhug we can observe certain phenomena that may be described as ``deep inverse''.
In the inverse configuration of Japhug, 
the A argument receives an ergative marker.
Properties other than case marking are however not observed
(\prettyref{sec:grammatical.clause.subject}).
Therefore, in the inverse configuration, we cannot say that
the patient is rendered the clausal pivot.
The ergative marking of the A argument in the inverse configuration should be analyzed as
inverse-triggered morphological ergativity,
in which the only subject-like property that the P argument gets
is that it has the same case marking with the intransitive subject. 

\paragraph*{The empathy hierarchy}\label{sec:grammatical.clause.direct-inverse.hierarchy} 
The relative positions of the A argument and the P argument in the empathy hierarchy
control the direct-inverse configuration.
Roughly, the hierarchy in Japhug is 
\ac{sap} \textgt human \textgt animal \textgt inanimate \textgt generic argument.
Note that this hierarchy means that
if the A argument of a clause is generic,
while the P argument is not,
the clause is in inverse configuration,
while when the P argument is generic the clause is in direct configuration.
This means neutralization of S and P can be observed in the generic indexation
(\citealt{jacques2012argument}; \citealt[\citesec{14.3.2.5}]{jacques2021grammar}).
This, of course, cannot be interpreted as ergativity.

It's tempting to analyze shallow inverse as 
a part of the realizational morphology:
it may just be due to rules like ``3sg.A + 1sg.P = \category{inv}''. 
Obligatory deep inverse can be analyzed as realizational%
\footnote{
    That's to say, there simply is no verb form expressing both of them
    and therefore co-appearance of the two is blocked:
    the explanation involves nothing about the underlying abstract syntactic structure 
    and therefore is \emph{realizational}.
}
incompatibility between
direct verb morphology and the inverse voice construction.
The empathy hierarchy then is a \emph{diachronic tendency}:
to make communication more efficient,
languages develop polypersonal indexation, 
and since inverse configurations are less frequent they tend to be marked differently.

Alternatively, the empathy hierarchy may be analyzed as a synchronic device. 
\citet[\citesec{7.4}]{wiltschko2014universal}, for example,
stipulates that the clause first decides its point of view in the empathy hierarchy,
and then this piece of information controls what argument is to agree with the verb,
and further notices that this process is formally comparable to
how the aspect value (c.f. the direct/inverse value) 
dictates what is the time that is to be compared with the speech time to decide the tense
(c.f. the argument indexed on the verb).

It seems the viewpoint-based direct-inverse system is indeed the case in Japhug. 
This is most clearly demonstrated by argument indexation 
in the double causative construction,
where the internal argument that plays the role of P in argument indexation
itself seems to be chosen by its prominence on the empathy hierarchy
(\prettyref{sec:voice.causative.indexation}).
Another demonstration of the synchronic nature of the empathy hierarchy
is the hybrid indirect speech,
which basically is an indirect speech construction
but argument indexation in the complement clause is determined
according to the viewpoint of the A of the \emph{main clause}
(\prettyref{sec:complement.indirect.hybrid}).

Another interesting phenomenon is the theme of secundative verbs
cannot be indexed and it cannot be first or second person.
The first fact may be due to the relation between the ability to trigger argument indexation and animacy.
The second fact can be explained by stipulating that in secundative constructions,
we have a R\textto T structure,
and probably the inverse configuration in this structure is forbidden. 

\paragraph*{Deviations from the ideal inverse system}
The ability for an argument to trigger argument indexation
depends on the animacy of the argument.
Inanimate arguments rarely trigger number indexation in some intransitive constructions,
including the existential construction and some dynamic verbs
\citep[\citesec{14.6.1.1}]{jacques2021grammar}.


\subsection{\ac{tame} categories}\label{sec:grammatical.clause.tame}

Following the blueprint in \citet{cinque1999adverbs},
the \ac{tame} parameters that are clearly syntactically relevant in Japhug
are summarized below.
The absence of some parameters is discussed in \prettyref{sec:tame.missing}.

The composition between these categories is not orthogonal:
not every combination of attested tense, aspect, modality and evidentiality values in Japhug 
can be morphologically realized
(\prettyref{sec:tame.primary}).
No independent morphological exponent can be identified for 
each separate \acs{tame} categories mentioned above,
and decomposition of \acs{tame} categories 
in the same way English \form{he [is playing] football} 
is analyzed as ``present (imperfect) progressive'' is not possible. 
In periphrastic conjugation, though, 
distribution of these primitive \acs{tame} categories 
onto the main verb and the auxiliary copula 
can be observed \citep[\citepage{1089}, (7)]{jacques2021grammar}.


\paragraph*{Mood, or clause type-related parameters}
The interrogative force, the rhetorical interrogative force,
the positive and negative imperative force
are related to \ac{tame} marking in Japhug
\citep[\citesec{21.4}, \citesec{21.7.3}, \citesec{21.7.4}]{jacques2021grammar}.

\paragraph*{Subjective evaluation}
expression of surprise is linked to the \category{sensory} category 
\citep[\citesec{21.3.2.4}]{jacques2021grammar}.
The opposite of the subjective evaluative adverb \form{hopefully} in English,
with a meaning of \translate{unfortunately, it is likely that \dots}
is also a modal category in Japhug \citep[\citesec{21.7.1}]{jacques2021grammar}.

\paragraph*{Evidentiality}
Japhug has a highly complicated evidentiality system 
\citep[\citetable{31.4}]{jacques2015sketch}.
The values of evidentiality attested include 
the generic, the factual, the sensory, the egophoric, and the inferential.

\paragraph*{Epistemic modality}
Japhug has a probabilitative modality value 
(\translate{probably, \dots})
\citep[\citesec{21.7.2}]{jacques2021grammar},
as well as a dubitative value:
\translate{this \emph{may} be the case (but I don't know)}
\citep[\citesec{21.4.4}]{jacques2021grammar}.

\paragraph*{Primary tense}
The distinction between the past and the non-past
can be clearly identified (\citesec{23.3}, \citesec{23.5}),
partly from the interaction with evidentiality.
They are closely related to the evidentiality values
and the imperfective/perfective aspectual values
to form \category{primary} categories.

\paragraph*{Deontic and dynamic modality}
We have a realis/irrealis distinction \citep[\citesec{21.4.1}]{jacques2021grammar}.

\paragraph*{Anterior (i.e. secondary tense)}
This is the \category{perfect} tense in English,
which is missing in Japhug.

\paragraph*{Viewpoint aspect}
Japhug has an imperfective-perfective distinction.
There is a \category{progressive} aspect 
which seems to be compatible with some perfective categories
\citep[\citetable{21.8}]{jacques2021grammar}. 
The proximative aspect (\translate{soon I will\dots})
and the prospective aspect (\translate{I almost\dots})
are expressed by the same morphological exponent, the \category{proximative}.
    
\paragraph*{Situational aspect}
Japhug verbs may be dynamic or stative. 
This can be observed in the interplay between \ac{tame} categories
and the main verb (e.g. \citealt[\citepage{1102}]{jacques2021grammar}).

\subsection{Classification of internal dependents}\label{sec:grammatical.clause.internal}

If we ignore \ac{tame} categories,
we get what could be called the \term{extended argument structure},
which consists of the predicator plus its core and peripheral arguments,
the latter being also known as adjuncts
(and correspondingly, core arguments are often simply known as arguments,
leading to certain ambiguities). 

Arguments and adjuncts are to be classified according to the follows:
\begin{itemize}
    \item \emph{Structural closeness to the verb}.
    This is usually accurately reflected by the semantic scope of the argument:
    a peripheral argument modifies the core information about a situation as a whole.
    This also strongly affects constituent order.
    Note that this parameter is accompanied by lots of subtleties.
    For example, there is mini-information structure in the argument structure \citep{devine2006latin}.
    Also, arguments -- like the subject -- may be promoted to higher positions.
    This problem however can always be solved by analyzing the relation 
    between the promoted constituent and the argument structure
    (e.g. \prettyref{sec:grammatical.clause.subject.argument-pivot}).

    \item \emph{Lexical obligatoriness}.
    Some verb roots won't be phonologically realized without a transitivity feature accompanying it
    -- this is known as \term{subcategorization}.
    The lexicon can also specify more about the verb root,
    leading to idiomization.%
    \footnote{
        We need to tell this from \emph{syntactic} fossilization, i.e. grammaticalization:
        the latter changes the constituency-dependency structure.
        idiomization does have syntactic consequences other than subcategorization:
        movements out of idioms are harder or otherwise interpretation of the idiom as a whole is disrupted.
    }
    For a radical pro-drop language like Japhug,
    definiteness can be used as a test,
    as an obligatory but dropped argument usually should have already appeared in the context.
    But insertion of an indefinite empty pro-form is also sometimes possible (see below).
    Note that due to locality of nature language syntax,
    usually an argument selected by the verb can't be structurally too far from the verb.
    
    \item \emph{Semantic or pragmatic obligatoriness}.
    Some constituents are needed simply because the clause needs to say something:
    this is sometimes hard to be distinguished from lexical subcategorization.
    In \form{he treated us badly}, it's hard to say whether \form{badly} is lexically selected.

    \item \emph{Internal makeup}.
    For \acp{np} this usually means case marking:
    peripheral arguments usually bear semantically meaningful or ``inherent'' cases,%
    \footnote{
        In the case in Latin, there may be two uses of the accusative,
        one without any semantic value,
        another with a meaning of \translate{goal}.
        In Japhug the absolutive case seems to have split uses as well
        (\prettyref{sec:grammatical.clause.internal.monotransitive}). 
    }
    while core arguments may bear semantically empty or ``structural'' cases
    (the nominative case in many languages, for example,
    tells us nothing about the role of the \ac{np}).
    For embedded clauses, verbal morphological differences may appear.
    These markings are however extremely complicated and are highly realization-dependent,
    so they can't be used as direct evidence for argument-adjunct distinction.
\end{itemize}

The exact definition of the terms \term{argument} and \term{adjunct}
therefore is not always certain:
they are to be made based on language-specific information regarding the above three parameters.
Based on these parameters we recognize the following types 
of constituents in the extended argument structure.

\paragraph*{Prototypical core arguments}
They structurally close to the verb
(but may be promoted to syntactic pivot positions, like the subject),
obligatory, bearing semantically empty cases.
Note that these arguments may still be dropped in actual discourses
because their references should be inferrable from the context,
and core arguments dropped in discourse then usually should have a definite reading
(unlike peripheral arguments).

In Japhug we can recognize a subject position mentioned in \prettyref{sec:grammatical.clause.subject}.
The concept of object is discussed in \prettyref{sec:grammatical.clause.internal.object}.

\paragraph*{Oblique core arguments}
They are structurally close to the verb,
obligatory, bearing \emph{semantic} cases,
e.g. \form{you are referred [to a specialist]}.
They are structurally closer to the verb than adjuncts:
\form{he constantly sit [in his chair]_{\text{oblique argument}} [in this episode]_{\text{adjunct}}}.
Sometimes they are licensed together with other core arguments,
as in \form{he moved [the device]_{\text{object}} [to a larger lab]_{\text{oblique argument (goal)}}}.
Depending on the properties of the preposition construction
(e.g. whether passivization is possible)
and idiomization (which may prohibit topicalization of the argument),
these arguments need to be classified into finer categories.

\begin{todobox}{Intransitive verbs' arguments}{argument-hierarchy-intransitive}
    In intransitive clauses, a tendency oblique\textgt degree\textgt essive is observed.
    Does this mean in Japhug, oblique arguments are actually peripheral?
    What about the case in transitive clauses? 
\end{todobox}

\paragraph*{``Optional'' core arguments}\label{sec:grammatical.clause.internal.optional-core}
They are structurally close to the verb and bear semantic cases or semantically empty cases,
because of labile alternation,
e.g. \form{the cat kills [a mouse]} v.s. \form{the cat kills}.
These arguments are still dictated by the lexical entry of the verb stem
(and are still obligatory in this sense),
and removing one ``optional''  core argument
may change the meaning of the verb in a (slightly or greatly) irregular way:
\form{I fly} can be interpreted in various ways,
but \form{I flied to Boston} usually means I took a plane to Boston.
In other cases, a correspondence between the two meanings can still be observed:
\form{The cat kills} in English, for example, 
means \form{the cat has a habit of killing other animals},
roughly equivalent to placing an indefinite pronoun in the object position.

Whether interpreting the omission of the argument
as pro-drop with an invisible indefinite pronoun
or as valency alternation is purely terminological:
when omission of the argument is prevalent we may want to 
denote the phenomenon as insertion of a indefinite empty pronoun,
and in this case the argument has nothing different with a peripheral argument with respect to lexical selection,
and if the possibility and semantics of the omission depends on the verb root,
it's better analyzed as valency alternation.

As is mentioned in \prettyref{sec:grammatical.clause.internal.optional-core},
Japhug is strongly pro-drop,
so when an argument is dropped it's not always easy to see whether it's an optional core argument.
Clear examples include dropping of the recipient argument in the indirect verb frame.

The three types of arguments above are structurally close to the main verb,
and their relations with the main verb are discussed in \prettyref{sec:grammatical.clause.frames}.
A detailed discussion on the standard of core arguments in Japhug
is done in \prettyref{sec:verb-frame.standard},
where we also discuss the differences between discourse-related omission of arguments
and the absence of adjuncts or labile alternation.

\paragraph*{Prototypical peripheral arguments}
They are structurally far from the verb,
optional, bearing semantic cases (or being adverbs).
Constituents with roles comparable to peripheral arguments 
but are not necessarily intuitively ``arguments'' (like adverbs)
are \term{adjuncts}.
Note that arguments (core or peripheral) that are structurally far from the main verb
usually will not take semantically empty cases like the nominative or the accusative,%

Among Japhug circumstantial peripheral arguments,
i.e. peripheral arguments specifying the locational and temporal position of the situation,
phrases specifying absolute time are almost always fronted
(\prettyref{sec:grammatical.clause.template.adverb}),
which may suggest that they are not truly temporal peripheral arguments 
but based-generated frames.

We can recognize a type of locational phrases that modify the whole situation,
which seem to be not lexically selected by the verb
\citep[\citepage{304}, (33); \citepage{387}, (66); \citepage{639}, (31)]{jacques2021grammar}.
In \citet[\citepage{387}, (66)]{jacques2021grammar},
the locational phrase is also fronted before the subject,
likely to a position comparable to the position of absolute time expressions.
Interestingly, this peripheral type of locational phrases
rarely appear after the subject:
\citep[\citepage{1220}, (115)]{jacques2021grammar} seems to be an example,
but I'm not sure if \form{sleep in} involves a locational argument or adjunct in Japhug. 
Therefore, like the absolute time expression,
either the peripheral locational phrases are actually base-generated frames,
or there is a rule in Japhug that requires obligatory topicalization of them. 

It's not clear whether the locational phrase with existential verbs 
is core or peripheral structurally.
Topicalization is also frequent in this case \citealt[\citepage{369}, (9)]{jacques2021grammar}.

\begin{todobox}{What about location, manner, and instrument?}{location-position}
    \begin{itemize}        
        \item Stacking of locational phrase:
        \citet[\citepage{284}, (125)]{jacques2021grammar},
        a frame plus a locational phrase in the existential construction.
        \item Manner
        \item Degree
        \item Instrument; note that sometimes it's a core argument, as in the double causative
    \end{itemize}    
\end{todobox}

\begin{todobox}{Locative constituents}{locative-classification}
    Difference?
    \begin{itemize}
        \item 8.1.8-9: goal and location
        \item 8.2.4: Locative
        \item 8.3.4: locative relator nouns
    \end{itemize}
\end{todobox}

\paragraph*{Argument-like adjuncts i.e. obligatory peripheral arguments}
They are structurally far from the verb
and bear semantic cases or are even adverbs,
but are obligatory:
the most notable example may be \form{he treats us [badly]}.

\paragraph*{The role of the verb root}
In certain constructions,
the verb root may actually be the manner expression
(e.g. \citealt{acedo2013satellite}, \citealt{punske2013three}).
Depending on the type of the constructions and realizational verbal morphology,
this is related to the satellite-frame/verb-frame distinction
light verb constructions, complex predication, etc.

\paragraph*{Valency-alternation}\label{sec:grammatical.clause.internal.valency}
Japhug has a varieties of valency alternation constructions.
They're quite different from the Indo-European type of \term{voice}: 
for example, the passive does not seem to work blindly on an object position
without any selection of the verb frame
(\prettyref{sec:voice.passive.input}).
This is reasonable,
as for a language in which valency alternation constructions 
working according to a single, ubiquitous object position,
only an Indo-European-like active-(middle)-passive distinction is possible.
Japhug, on the other hand, has an abundance of valency alternation constructions
that can be stacked to each other.

\subsection{Verb frames}\label{sec:grammatical.clause.frames}

\paragraph*{Overview of core arguments}
In the typological literature, macrorole notations like S, A, P 
are based on the premise that the arguments it represents
behave syntactically in similar ways.
The symbols then refer to the bundles of shared syntactic properties.

Roughly speaking, the properties may be about deep properties, 
i.e. properties originating from the deep argument structure
(control, reflexive, etc.), 
or the shallow properties, i.e. clause-level grammatical phenomena
(relativization, subject-sharing coordination, etc.).
Case marking or in other words argument flagging 
is related to the latter (e.g. accusative alignment v.s. ergative alignment)
as well as the former (e.g. oblique arguments cannot be passivized),
and so is verb-argument agreement or in other words argument indexation.

As languages differ in both the deep argument structures
and how complete clauses are built around them,
the definition of macroroles in a specific language therefore should represent 
the aforementioned phenomena in that language.
A good definition of macroroles in a language
is equivalent to a good classification of its valency classes.

\paragraph*{Intransitive constructions}
In Japhug, the sole arguments of intransitive clauses largely behave in the same way:
we can stipulate a S macrorole and there is no Sa/So distinctions. 
There exist so-called semi-transitive verbs with additional arguments,
which are however not relevant to argument indexation at all
\citep[\citesec{14.2.5}]{jacques2021grammar}
and do not seem to be active in other grammatical phenomena.

A notable distinction within transitive clauses
is that in the passive of secundative verbs,
the subject does not trigger indexation
\citep[\citesec{18.1.4}]{jacques2021grammar}.

\begin{todobox}{Oblique arguments in Japhug}{obligque-argument}
    Compare oblique arguments with, say, directional particles.
\end{todobox}

\begin{todobox}{Labile verb}{labile-verb}
    Japhug does have S=O labile verbs,
    and it remains unclear whether the surface S in this case
    reflects some of the deep P properties. 
\end{todobox}

\paragraph*{Monotransitive constructions}
\label{sec:grammatical.clause.internal.monotransitive}
In transitive clauses we can recognize A and P arguments.

The A argument is clearly more ``external'' or prominent.
Verbal polypersonal indexation always involves two argument,
and the A is always one of them, 
while the rest of the arguments compete to be the other argument involved 
(\prettyref{sec:grammatical.clause.direct-inverse.hierarchy}).
By default we can observe that the A 
appears at the start of the nucleus clause 
(\prettyref{sec:grammatical.clause.template.nucleus-identification}).
The comparison between A and S and hence the definition of \term{subject}
are discussed in \prettyref{sec:grammatical.clause.subject}.
We note however reflexive binding is absent in Japhug
\citep[\citepage{543}]{jacques2021grammar}
and cannot be used to prove the pivot status of the A argument
either in the deep argument structure or in the whole clause.

The P position should be split into the prototypical P
and the semi-object, referred to as P' in \citet{jacques2016subjects}.
The prototypical P regularly triggers so-called object-like indexation
\citep[\citesec{8.1.3}, \citepage{543}]{jacques2021grammar}.
For some verbs, the internal argument P',
despite being similar to the monotransitive P in being the receiver of an event 
and possibly being the sole internal argument of the clause,
can never trigger indexation.

\paragraph*{\term{Object} as a shallow grammatical relation}
\label{sec:grammatical.clause.internal.object}
The P argument in the monotransitive clause is also known as the \term{object}.
Usually we expect the term \term{object} to refer to a position 
which is syntactically more active than other internal arguments
and whose distribution is not restricted to the monotransitive construction
(otherwise \term{object} and P are interchangeable).
In English, for example,
an argument -- the P argument in the monotransitive construction 
and the T argument in \form{give sth. to sb.} --
always follows the main verb with almost no other constituents
being able to appear between the two,
and constituents like manner phrases, relative clauses, etc.
which have their scopes \emph{over} the core verb phrase, follow the object.
This seems to indicate that there is some sort of implicit fronting 
of both the verb and the object,
leaving a swamp of various constituents behind.
Therefore an object position after the main verb
that is attested in more than one verb frames is well-defined
as a shallow grammatical concept.
It is possible that the object position cannot be well defined in other verb frames. 
The split of object properties is observed in the English \form{give sb. sth.} verb frame,
where the recipient argument can be passivized but not topicalized,
and the theme argument can be topicalized but not passivized.
The reason possibly is that the recipient argument is in an object-like position
(and thus can be passivized)
but either faces realizational effects that lock it there 
\citep{oba2005double},
or is obligatorily focalized and hence topicalization causes conflicts in information structure
\citep{im2005alternative}.
In this case there does not exist a prototypical object.

In Japhug, we consider the following criteria for objecthood:
\begin{itemize}
    \item \emph{Agreement}. The monotransitive P is the prototypical object
    while the internal arguments that do not trigger indexation 
    but are not in genitive, dative, etc. cases
    are known to be \term{semi-objects}
    \citep[\citesec{8.1.5}]{jacques2021grammar}.
    So if objecthood is to be defined according to argument indexation,
    it can be defined for some two-places verbs but not all.
    
    \item \emph{Voice}. Valency changing is not a good criterion for objecthood
    because it doesn't work on the concept of objecthood (\prettyref{sec:grammatical.clause.internal.valency}).

    \item \emph{Relativization}. There is one construction where the monotransitive P is neutralized with
    arguments in other verb frames, providing a broad definition of objecthood:
    relativization \citep{jacques2016subjects}.%
    \footnote{
        Note that the term \term{pivot} in this paper refers to 
        any macro-macro-role that is a collection of macroroles with shared properties,
        and not necessarily something subject-like.
    }
\end{itemize}

\begin{todobox}{Position of the ``object'' defined in argument indexation}{argument-indexation-word-order}
    When multiple arguments compete to be the P,
    does this influence their word orders?
    If so, we have another piece of evidence for stipulating objecthood.
\end{todobox}



\paragraph*{Ditransitive constructions}
Japhug has a clearly indirective ditransitive verb frame
(i.e. a verb frame where the theme is object-like)
and a secundative verb frame  
(i.e. a verb frame where the recipient is object-like).

In the indirective construction,
the recipient is dative or genitive \citep[\citesec{14.4.1}]{jacques2021grammar}
and it seems to have not many shared properties with prototypical subjects or objects,
while the theme can be seen as the object \citep{jacques2016subjects}.%
\footnote{
    \citet{jacques2016subjects} 
    uses R_1, T_1 to refer to the recipient and the theme of secundative verbs,
    and R_2, T_2 to refer to the recipient and the theme of indirective verbs. 
}
This verb frame can be analyzed as the main verb controlling a ``transferring'' small clause:
the small clause has the form of T\textto R,
and T then becomes the P of the main verb.
The recipient can be omitted with either a indefinite or a definite meaning
\citep[\citesec{22.1.2.2}, (29-30)]{jacques2021grammar},
which satisfies the definition of optional core arguments.

\begin{todobox}{Passivization of arguments}{passivization-ditransitive}
    Can T_2 be passivized?
\end{todobox}

As for what \citet[\citesec{14.4.2}]{jacques2021grammar} analyzes as 
the secundative verbs (the valency class where the recipient is more object-like),
the theme can be extracted in relativization \citep[\citepages{581}]{jacques2021grammar}
and be passivized \citep[\citesec{18.1.4}]{jacques2021grammar},
but the recipient participates in argument indexation
in the same way as the monotransitive object does.
What is interesting is that in the resulting verb,
the subject (which is the theme) also does not trigger any indexation affixes.
It's possible that the theme bears some sort of inherent case 
that prevents it from triggering any argument indexation. 

\paragraph*{Copular complement}
Japhug lacks adjectival copular complements;
an \term{essive} construction however can be recognized
\citep[\citesec{8.1.7}]{jacques2021grammar}.

\paragraph*{Directional construction in verb frame}

A directional small clause may be incorporated into a verb frame 
to express the consequence or intention of the event.
In Japhug this is attested in e.g. \citet[\citepage{324}, (101)]{jacques2021grammar}
and \citet[\citepage{407}, (132)]{jacques2021grammar}.
\citet[\citepage{407}, (132)]{jacques2021grammar} seems to 
contain the coordination of two directional small clauses.
Alternatively it can be analyzed as verb omission,
but the semantics -- the two locational phrases do not modify the whole event,
but the object -- strongly indicate that we are dealing with oblique core arguments.
A complete path-goal construction can be found in \citet[\citepage{751}, (115)]{jacques2021grammar}.%
\footnote{
    \citet[\citepage{407}]{jacques2021grammar} calls the two locational phrases locative adjuncts.
    It's not clear how the author makes the argument/adjunct distinction here.
    In \citet[\citesec{22.1.2.2}]{jacques2021grammar},
    goals are uniformly called oblique arguments,
    so it seems he uniformly treats (static) locations as adjuncts.
    We have noted above that locational phrases may modify the whole event
    or an specific core argument,
    and therefore they differ in their structural closeness to the verb,
    so calling them uniformly adjuncts seems oversimplified. 
}

\begin{todobox}{Directional}{directional-construction}
    \begin{itemize}
        \item Type 1: directional particle comes from directional construction.
        \item Type 2: directional particle is a part of the stem.
        \item Is it possible to distinguish between the two?
        
        \item The orientational adverb: if it originates from the directional construction.
        it should modify the motion of the object when the clause is transitive?
        It seems the orientational preverb should be the path?
        \item Note that in Japhug, orientational adverbs can also appear with \translate{plant it down}
        \citep[\citepage{1211}, (85)]{jacques2021grammar}.
        \item Hierarchy of path, ground, etc.
    \end{itemize}
\end{todobox}

\section{Noun phrase}

\subsection{Case marking}

Japhug grammatical functions can be marked by zero marking (the so-called absolutive),
postpositions, or relator nouns.

In locational and directional constructions,
stacking of markers is possible
\citep[\citepage{831}, (20); \citepage{361}, (1)]{jacques2021grammar}.

\subsection{Noun valency class}
Japhug has the distinction between alienable and inalienable nouns.
An inalienable noun has to have a possessor
which is indexed on the noun,
and this possessor can be understood as a core argument of the head noun
\citep[\citepage{116}]{jacques2021grammar}.

\subsection{Compounds} 

\begin{todobox}{Compounds: phrasal, or stem-level?}{compound-analysis}
    Japhug seems to have ``real'' compounds and not just
    the nominal attributive construction in English:%
    \footnote{
        As in \form{noun phrase}.
    }
    inalienable nouns, if appearing as the second element in a compound,
    have no possessive marker \citep[\citepage{15}]{jacques2021grammar}. 
    This however can be analyzed as a realizational effect as well:
    we need other proofs to show that the inalienable noun 
    loses its subcategorization in the compounding process
    and therefore appears purely as a stem.
\end{todobox}

\subsection{Coordination}
One interesting feature of the Japhug comitative 
is it's also considered when deciding the number of an \acs{np}
\citep[\citepage{332}]{jacques2021grammar};
but it's still not prototypically a conjunction \citep[\citepage{420}]{jacques2021grammar}:
the \acs{np} following the comitative marker 
may be omitted, 
agreeing with the fact 
that the head noun of an \acs{np}
can also be dropped \citep[\citepage{425}]{jacques2021grammar}.
(In English this is only possible for clauses:
in informal writing and speech people may start with a sentence with \form{and},
i.e. a conjunction construction 
without the first branch,
but they never do so to an \acs{np}.)
The \acs{np} after the comitative marker can also be relativized.
Thus the comitative suffix is still recognized as a type of 
modification.

\section{Clause combining}

\paragraph*{Control} In Japhug, the argument in a controlled velar infinitive
that is co-referential with the argument in the matrix clause
can be S, A, P, and also themes in ditransitive constructions
and the recipient of secundative verbs \citep{jacques2016subjects},
or even the possessor of these arguments \citep[\citepage{1366}]{jacques2021grammar}.

The neutralization of S and P may hint at \emph{syntactic} ergativity.
I however suspect that this may be related to the pseudo-passive construction,
where the deep P argument appears as the surface A,
and as the velar infinitive does not have polypersonal indexation,
the inverse marking is absent.

The coreference between the matrix clause subject and a possessor in the infinitive 
is more interesting.
This is definitely an external possession construction,
but we do not know what mechanisms are under it
\citep{deal2017external}.

\section{Parts of speech}

\subsection{The verb}

\begin{todobox}{Event structure and valency}{event-structure-valency}
    State change, action, etc. and relation with unergative verbs or unaccusative verbs.
\end{todobox}

Verbs can be regularly formed by denominal derivations
\citep[\citechap{20}]{jacques2021grammar}.
Since an independent adjective class is absent, 
the only two kinds of denominal derivations 
are noun-to-adverb derivations and noun-to-verb derivations,
the former being relative marginal \citep[\citepage{1011}]{jacques2021grammar};
thus the term \term{denominal} can be used specifically 
to refer to noun-to-verb derivations.

\subsection{Nouns}

\begin{todobox}{Counted noun constructions: what's the head?}{counted-noun-headness}
    In \citet[\citepage{10}]{jacques2021grammar},
    it is mentioned that counted nouns correspond to classifiers in Chinese grammar.
    An interesting problem is in a counted noun construction in Japhug,
    which element is the head of the NP.
    In English, the head of the \form{one of} construction is likely \form{one},
    because it can undergo modifications:
    \form{this specific one of \dots},
    so the partitive reading is merely semantic.
    If no direct modification is possible to the counted noun,
    maybe the counted noun has already collapsed into a classifier in this usage.
\end{todobox}

\subsection{Ideophones}

The category of ideophone occupies mainly manner adverbial positions
\citep[\citesec{10.1.7}]{jacques2021grammar}.
Its main difference with the adverb class 
is its morphology \citep[\citesec{10.1.2}]{jacques2021grammar}
and phonology \citep[\citesec{10.1.5}]{jacques2021grammar}.



\chapter{The verb}

Japhug is a heavily inflected language, 
and most grammatical categories in the clause 
have something to do with the verb.
The structure of the verb can be divided into 
the outer prefix chain (\citealt[\citetable{11.1}]{jacques2021grammar}),
the extended stem, 
and the suffix chain (\citealt[\citesec{11.3}]{jacques2021grammar});
the extended stem contains the stem, 
which may undergo stem alternation (\citealt[\citechap{12}]{jacques2021grammar}),
and inner prefixes related to valence alternation
(\citealt[\citesec{11.2.2}]{jacques2021grammar}).

\paragraph*{Wordhood}
Whether this complex is to be regarded as one \emph{morphological} or \emph{phonological} word 
is discussed in \citesec{11.6}
in the reference above.
Recognition of wordhood, expectedly, is not self-evident;
\citet{prins2011web} provides an analysis of another rGyalrong language, Jiaomuzu, 
and in this thesis the term \term{verb phrase} 
(i.e. verbal complex in this note) is used,
skipping the discussion on what is a word.
In \citet[\citetable{11.3}]{jacques2021grammar}
four domains are defined using various criteria.

Domain A is defined according to both syntactic and morphological reasons.
What's shown in 
\citetable{11.3} contains all formatives that are relevant to verb inflection,
and they have non-adjacent dependencies,
so strong dependencies exist between them:
these formatives are realized in the same batch 
in clause building.
Now syntactically, the formative \form{-ci} in slit +4 
is selected by some modal prefixes in slot -6,
so the two slots belong to the same system;
on the other hand, outside the +4 and -6 slots 
we only have clitics which clearly belong to systems with higher positions
\citep[\citesec{11.6.2}]{jacques2021grammar},
and thus all -- and only -- formatives in \citetable{11.3}
constitute a syntactic word,
with the same \emph{syntactic} status of a verb-plus-auxiliary verbal complex or a
``verb phrase'' in Dixon's definition (i.e. without internal complements). 
Morphologically, no element is able to intervene 
between two slots in the template, 
so we say this batch is realized as a single morphological word 
instead of a verbal complex.

Domain B is about \emph{obligatoriness}:
thus the +4 slot is not included.
Domain C is defined according to prosodic reasons.

\paragraph*{Morphological marking of \ac{tame}}
The morphological realization of \ac{tame} is remarkable.
Their main exponents are the alternation of the orientation prefix.
Some \acs{tame} categories insert a fixed prefix 
into the orientation prefix slot;
others choose one of the four prefixes that have the same directional meaning 
in \citet[\citetable{15.1}]{jacques2021grammar}. 

\begin{todobox}{Interaction between TAME and orientation}{tame-orientation-morphology}
    Does the TAME marking override the lexically determined orientation prefix 
    or the semantically significant orientation prefix of a orientable prefix?
\end{todobox}

\chapter{Verb frames}

In this chapter we summarize basic verb frames in Japhug.
This is done in \citet[\citepage{14}]{jacques2021grammar},
which is mostly about the finite clause without any valency alternation operations.

\section{Descriptive parameters}

\subsection{Recognition of core arguments}\label{sec:verb-frame.standard}

We first need to classify arguments and adjuncts appearing in Japhug clauses
along the line of \prettyref{sec:grammatical.clause.internal},
as discussions on verb frames should focus on core arguments.
Here we treat obligatoriness first,
as the criteria involves are easy.
We then move to study various verb frames and the syntactic properties of their core arguments.

\paragraph*{Lexical selection and obligatoriness}
Since every clause constituent in Japhug can be omitted,
whether a constituent is lexically selected by the main verb
cannot be established using superficial obligatoriness.

One possible criterion in this case is definiteness:
if an argument is first present and only later omitted, 
its omission should have discourse-motivated reasons,
and usually the reason is that the argument has appeared in the context before:
if the argument is indefinite then most of the time, we expect an indefinite pronoun to appear.
Hence, we expect non-overt but obligatory arguments or \term{core arguments}
(if we define the term this way)
to be almost always definite \citep[\citesec{22.1.2.1}]{jacques2021grammar}.
On the other hand, non-obligatory arguments 
-- or \term{adjuncts} or \term{peripheral arguments} --
should be either definite or indefinite
\citep[\citesec{22.1.2.2}]{jacques2021grammar}.

This criterion is invoked in \citet[\citesec{22.1.2.2}]{jacques2021grammar}:
oblique arguments, when not present, are always possibly indefinite.
Jacques thus argues that this, instead of indexibility, 
is the main difference between core and oblique arguments:
oblique arguments are less lexically bound to the main verb 
compared to core arguments.

The problem of this account however is that 
the possibility of the indefinite reading is still related to indexibility:
if a core argument is omitted with an indefinite reading,
its absence has to trigger intransitive indexation on the verb,
so the omission is analyzed as labile alternation
\citep[\citesec{22.1.2.1}]{jacques2021grammar},
and the fact that the possibility of labile alternation 
is controlled by the verb root is then easily noticeable. 
but an oblique argument by definition contributes nothing to argument indexation,
and if its omission is indeed controlled by the verb
(and hence its omission should better be seen as valency alternation;
\prettyref{sec:grammatical.clause.internal}),
it will not be easy to notice. 

One piece of evidence of the omission of an oblique argument being controlled by lexical properties of the verb
is that after the omission, the meaning of the verb changes, often irregularly.
Examples presented in \citet[\citesec{22.1.2.2}]{jacques2021grammar}
however do not clearly demonstrate change of the meaning.
In (30), the verb means \translate{shoot arrows to see whose arrow will reach the farthest},
which deviates from \translate{to shoot arrows to an target},
but this is an imperative clause so the clause type is sufficient to explain the change of the meaning.
In (31), we see an alternation between
\translate{say} and \translate{make a sound},
but the deviation between the two is not large enough. 

In conclusion, it seems that the only clearly identifiable type of ``oblique core arguments'' 
in \prettyref{sec:grammatical.clause.internal}
is the semiobject,
which is classified as core arguments in \citet{jacques2021grammar}.
The distinction between other oblique arguments and adjuncts i.e. peripheral arguments
is more elusive, and may be non-existent.

\begin{todobox}{Non-core arguments and adjuncts}{argument-adjunct-distinction}
Still, we can see wordings like ``non-core arguments and adjuncts''
(e.g. pp. 751).
\begin{itemize}
    \item On \citepage{534}, it's mentioned that oblique arguments are selected by verbs.
    It seems we are dealing with directional constructions here,
    which usually should be arguments.
    \item Is there any evidence suggesting that the oblique arguments are structurally far from the verb?
    Or maybe they are structurally close to the verb?
    \item We turn to find evidence for any specific syntactic properties
    shared by some goal- or path-like arguments but not others.
    Applicative happens to a broad range of arguments 
    \citep[\citepage{859}]{jacques2021grammar}.
    One interesting phenomenon is that for two verbs,
    the locative can be relativized using the negative object participle 
    \citep[\citesec{23.5.5.2}]{jacques2021grammar}.
\end{itemize}
\end{todobox}

\begin{todobox}{Descriptive parameters in argument structure}{argument-description}
    \begin{itemize}
        \item Note that valency alternation is not discussed in this chapter
    \end{itemize}
\end{todobox}

\chapter{Valency alternation}

\section{Causative}

\paragraph*{Argument indexation}\label{sec:voice.causative.indexation}
In the causative construction, the argument playing the role of P in indexation
can be the causee or the object.
Which argument is chosen depends on the persons of the arguments.
First, among internal arguments,
if one is first or second person and another is third person,
then the first/second person one is indexed.
Thus both 2\textto 3\textto 1 and 2\textto 1\textto 3 
are equivalent to 2\textto 1 in argument indexation
\citep[\citepage{584}]{jacques2021grammar},
and both 3\textto 3\textto 1 and 3\textto 1\textto 3 
are equivalent to 3\textto 1 in argument indexation
\citep[\citepage{310}]{jacques2021grammar}.
On the other hand, 3\textto 1\textto 2 becomes 3\textto 1,
and 3\textto 2\textto 1 becomes 3\textto 2,
which means if the causee and the object are in a local configuration,
the causee is indexed on the main verb.
We can therefore say that the argument playing the role of P in polypersonal indexation
is always the most salient argument,
either according the standard of speech act participation or according to the standard of agentivity.

The mechanism of argument indexation in the causative
therefore is comparable to the analysis of English tense and aspect 
by \citet[\citesec{7.4.1}]{wiltschko2014universal},
where the aspect value of a clause selects the start or the end or the totality of an event
as the reference time (c.f. the argument playing the role of P),
and then the reference time and the utterance time (c.f. the A argument)
are compared to decide the tense (c.f. the inverse marker).
This supports the hypothesis that in Japhug,
the empathy hierarchy is a synchronic concept
(\prettyref{sec:grammatical.clause.direct-inverse.hierarchy}).

\paragraph*{Objecthood in the causative}
In the applicative construction,
the promoted argument ends up in a position quite close to the prototypical P,
which acts just like the P in argument indexation
(e.g. \citealt[\citepage{863}, (102)]{jacques2021grammar}).
Another phenomenon relevant to argument indexation is that
all arguments except the highest causer compete on the empathy hierarchy
to be the argument whose role in indexation is similar to the P in monotransitive clauses,
and therefore the winner may be taken to be the object
(\prettyref{sec:grammatical.clause.direct-inverse.hierarchy}).
These indexation-related phenomena however can be analyzed without stipulating 
any promotion to an object position:
we can simply assume that all arguments without oblique case marking
implicitly participate in argument indexation 
according to the rules in \prettyref{sec:grammatical.clause.direct-inverse.hierarchy}.

\paragraph*{Double causative}
The causative may be applied twice,
with the highest argument being the agent
and the second highest argument (introduced by the inner causative) being the instrument.
The meaning then is \translate{$X$ makes [[$Z$ do sth.] with $Y$]}.
The A-like argument in indexation seems to be the highest argument,
and the P-like argument seems to be the most salient in the rest of arguments:
hence a 1\textto 3\textto 2\textto 3 configuration
is morphologically the same as 1\textto 2
\citep[\citepage{848}, (67)]{jacques2021grammar}.
This further affirms the analysis outlined above.

We also note that the fact that causative can only be applied twice and
when it is applied twice, the highest argument is the agent
and the second highest argument is the instrument
probably reveals that Japhug causative is not productive as it seems:
even when we are dealing with seemingly recursive use of the causative,
we are still dealing with the usual agent-instrument-internal arguments hierarchy,
and it is impossible to employ the causative truly recursively.

\section{Passive}

\paragraph*{Verb frame compatibility}\label{sec:voice.passive.input}
The only kind of passivization allowed for ditransitive verbs
acts on secundative verbs,
and surprisingly turns the A\textto R\textto T argument structure
into a T\textto R one, although the R seems more object-like
\citep[\citepage{884}]{jacques2021grammar}.
It might be possible that the latter is homophonous to the passive construction
but has a more complicated structure.
For verbs derived from valency alternation constructions,
the passive cannot happen on top of synchronically analyzable causative or applicative 
\citep[\citepage{885}]{jacques2021grammar},
and its application on top of the tropative is also not attested
\citep[\citesec{17.5.4}]{jacques2021grammar}.

Therefore, if passivization is taken as a criterion of objecthood,
then only the monotransitive P is the object;
alternatively we may say passivization in Japhug
is structurally different from the English passive
and only works when the action has a clear ``initiator-receiver'' structure,
and the P argument is not a semi-object:
hence objecthood in broader contexts, if it exists,
does not depend on passivization.

\section{Pseudo-passive}

Japhug seems to have a construction comparable to
the ``passive'' construction in Mandarin;
in this pseudo-passive construction,
the deep, animate A argument does not appear, 
while the inanimate deep P argument is present, 
and the verb has an inverse marker 
despite semantically the event happens 
from an animate participant to an inanimate participant
and is therefore semantically in the direct configuration
\citep[\citepage{575}]{jacques2021grammar}.
The construction only appears in translation of sentences from Mandarin,
and yet a native speaker didn't consider them to be ungrammatical;
whether the 

\chapter{\acs{tame} marking}

\section{Introduction}


The \acs{tame} categories in Japhug 
are introduced in \citet[\citechap{21}]{jacques2021grammar}.
Morphologically speaking, there are three systems
\citep[\citepage{516}]{jacques2019egophoric}:
\begin{itemize}
    \item The \category{primary} system, 
        whose main exponents are stem alternation, the orientation preverb,
        and the modal prefix;
        all of these happens in the template of the verb \citep[\citetable{21.1}]{jacques2021grammar}.
        The grammatical categories marked in this system are listed below.
    \item The \category{secondary} system, 
        which also happens in the inflection pattern of the verb  
        but is about aspectual and modal categories 
        largely orthogonal to the grammatical categories marked by the \category{primary} system 
        \citep[\citesec{21.6}, \citesec{21.7}]{jacques2021grammar}. 
    \item The \category{periphrastic} system, 
        whose surface form is similar to complement clause constructions 
        with the copula \form{ŋu}.
        The copula in periphrastic constructions 
        never takes any argument indexation markers \citep[\citepage{1090}]{jacques2021grammar},
        and if we are to analyze the constructions as complement clause constructions,
        then the literal reading will be something like  
        ``it's the case that an event happens'',
        with all the contents before the copula 
        being a complement clause of the copula.
        In the follows however it can be seen that 
        the \acs{tame} categories on the copula 
        is complementary with the lexical verb,
        and hence the periphrastic constructions are to be analyzed 
        as single-clause constructions.

        In periphrastic constructions 
        the main verb is often in \emph{finite} forms \citep[\citepage{1081}]{jacques2021grammar};
        Japhug periphrastic conjugation thus has 
        a difference with English or Latin periphrastic conjugation,
        where what are used in periphrastic \acs{tame} categories
        are \emph{non-finite} verb forms.
        The reason possibly is because the periphrastic \acs{tame} categories in Japhug
        historically comes from finite complement clause constructions.

        TODO: is there any constraints on the distribution of participle or infinitive?
\end{itemize}
The interaction of the three morphological systems makes Japhug \acs{tame} system extremely complicated;
some periphrastic categories seem to have identical semantics with 
\category{primary} and \category{secondary} \acs{tame} marking devices \citep[\citepage{1092}]{jacques2021grammar};
whether there are hidden nuances is still not clear.

Besides the verbal complex, \acs{tame} categories are also marked by 
sentential adverbs and sentence-final particles 
(\citealt[\citepage{518}]{jacques2019egophoric}; \citealt[\citesec{21.8}]{jacques2021grammar}).


\section{Primary categories}
\label{sec:tame.primary}

\paragraph*{The realis paradigm}
The combination of parameters in \prettyref{sec:grammatical.clause.tame} is not trivial.
Since the irrealis modality is only compatible with the perfective/imperfective aspectuality
and not tense or evidentiality
\citep[\citepage{1119}]{jacques2021grammar},
Japhug \ac{tame} paradigm can be first divided into the realis part and the irrealis part.
The realis part can then be divided into the non-past part and the past part.

A three-fold distinction can be observed with the non-past tense
(\citealt[\citesec{21.3.4}]{jacques2021grammar}; 
\citealt[\citepage{517}]{jacques2019egophoric}): 
the factual or common knowledge (\citesec{21.3.1.2}), 
the sensory (\citesec{21.3.2.2}),
and the egophoric.

Actually there is a fourth, bleached ``generic'' evidentiality value 
with the non-past tense.
The generic non-past \acs{tame} configuration 
with no other non-trivial \acs{tame} marking 
is known as the \category{imperfective} \citep[\citesec{21.2}]{jacques2021grammar}.
This however seems to be very infrequent in main clauses
without periphrastic auxiliaries (\citepage{1087}),
indicating a strong preference for Japhug 
to include a non-trivial evidentiality value in non-past sentences.

The non-past categories are always inherently imperfective:
no perfective aspectuality is seen with non-past tense 
\citep[\citepage{517}]{jacques2019egophoric},
again possibly because of semantic reasons,
since the perfective may be semantically identified with the past.
The imperative-perfective distinction can only be seen 
with the past tense \citep[\citetable{21.1}, note that the \category{aorist} is also known as 
the \category{past perfective}; \citepages{1135, 1143}]{jacques2021grammar}.

Now we turn to the past paradigm.
The inferential evidentiality value appears only with the past tense,
possibly because of semantic reasons: 
an event happening now usually doesn't need to be ``inferred'',
and this rarity means even this category existed historically,
it has long been eroded.
With the past tense, 
we have a dichotomy between the generic evidentiality and the inferential evidentiality.
It's impossible to morphologically mark the sensory evidentiality
with the past tense, 
possibly again because of the infrequency of this configuration.  
The sensory can still be combined with the past tense 
by periphrastically attaching a sensory copula 
to the \category{aorist} (i.e. \category{past perfective} -- see below) 
and the \category{past imperfective}
which have default evidentiality
(\citealt[\citesec{21.5.1.8}, \citesec{21.5.3.5}]{jacques2021grammar}; 
\citealt[\citepage{518}]{jacques2019egophoric}). 
On the other hand, the factual evidentiality and the egophoric evidentiality
are never seen together with the past tense.

Now we map the combinations of the \ac{tame} values in \prettyref{sec:grammatical.clause.tame}
to the 11 \category{primary} \acs{tame} categories 
listed in \citet[\citesec{21.1}]{jacques2021grammar}.
The realis part is replicated in \prettyref{tbl:realis-tam}.
Note that the \category{aorist} is just the \category{past perfective} in
\citet[\citetable{31.4}]{jacques2015sketch}, 
and indeed \citet[\citepage{1135}]{jacques2021grammar}
makes it clear that the category expresses past perfective events.

\begin{table}[H]
    \centering
    \caption{Japhug realis \category{primary} \acs{tame} categories}
    \label{tbl:realis-tam}
    \scriptsize
    \begin{tabular}{lllllll}
        \toprule
                                &                          & \multicolumn{5}{c}{evidentiality}                                                                                                                                                                        \\ \cmidrule(l){3-7} 
        \multirow{-2}{*}{tense} & \multirow{-2}{*}{aspect} & generic                           & factual                           & sensory                           & egophoric                         & inferential                                              \\ \midrule
        non-past                & imperfective             & \category{imperfective}             & \category{factual}                  & \category{sensory}                  & \category{egophoric present}        & \cellcolor[HTML]{C0C0C0}{\color[HTML]{C0C0C0} \category{}} \\ \midrule
                                & imperfective             & \category{past imperfective}        & \cellcolor[HTML]{C0C0C0}\category{} & \cellcolor[HTML]{C0C0C0}\category{} & \cellcolor[HTML]{C0C0C0}\category{} & \category{inferential imperfective}                                     \\ \cmidrule(l){2-7} 
        \multirow{-2}{*}{past}  & perfective               & \category{aorist} & \cellcolor[HTML]{C0C0C0}\category{} & \cellcolor[HTML]{C0C0C0}\category{} & \cellcolor[HTML]{C0C0C0}\category{} & \category{inferential }                        \\ \bottomrule
    \end{tabular}
\end{table}

It should be noted that \prettyref{tbl:realis-tam} does not exhaust 
\ac{tame} meanings of the morphological forms listed.
For example, the \category{imperfective} category 
has hortative meanings sometimes \citep[\citesec{21.2.5}]{jacques2021grammar}.

The \ac{tame} meaning of \category{primary} categories in \prettyref{tbl:realis-tam} 
may also be fine-tuned by the lexical aspect of verbs 
\citep[\citesec{21.2.6}, \citesec{21.2.7}]{jacques2021grammar}.


\section{Secondary aspects}

There are two so-called \category{secondary aspects} in Japhug,
which are less intertwined (but are still intertwined) 
with \ac{tame} parameters expressed by \category{primary} categories.

\paragraph*{The \category{progressive}}
Imperfective categories and some perfective categories
are compatible with the \category{progressive},
which emphasizes that the event is ongoing or habitual,
and prohibits the common knowledge reading
\citep[\citepage{1161}]{jacques2021grammar}. 

\paragraph*{The \category{proximative}}
\label{sec:tame.secondary.proximative}
An interesting fact is that the \category{proximative},
when used with perfective categories,
means the action is almost finished but is not completed,
while when it is used with the \category{factual},
it means the action is about to start.
One possible analysis is that 
\category{factual} may refer to an ongoing event or an imminent event,
and if the former interpretation is taken,
then there exists no boundary of the event
that is close to the utterance time,
and if the second interpretation is taken,
the boundary of the event is the starting point of the event,
and the \category{proximative} category
then is interpreted as \translate{the utterance time is close to the starting point of the event},
or in other words, \translate{the event is going to happen}.

\section{Periphrastic conjugation} 

In a periphrastic construction we have a main verb in a finite form 
and a copula carrying the majority of \ac{tame} information.

\paragraph*{Periphrastic constructions involving the \category{imperfective}}
The most prevalent periphrastic constructions 
are those formed by combining the copula and the \category{imperfective} 
\citep[\citepage{1089}]{jacques2021grammar}.
Periphrastic constructions involving the \category{imperfective}
can also be placed into \prettyref{tbl:realis-tam} or the irrealis paradigm.

\begin{todobox}{Range of periphrastic constructions}{imperfective-based-periphrastic}
    Can the imperative speech force also be marked by a periphrastic construction?
    And what about \category{secondary modal} prefixes?
\end{todobox}

By putting a \category{factual} or \category{sensory} copula
behind the \category{imperfective} main verb,
we have a evidential contrast between the sensory and non-sensory meanings.
There is no periphrastic way to mark egophoricity, though.
\category{past imperfective}, \category{inferential imperfective}, and \category{irrealis} copulas
can also be attached to the \category{imperfective}
to form the \category{periphrastic past imperfective, inferential imperfective} 
and \category{irrealis} \citep[\citesec{21.2.2}]{jacques2021grammar}.

\begin{todobox}{Missing periphrastic forms}{missing-periphrastic-realis}
    What about periphrastic aorist and inferential?
\end{todobox}

The meanings of periphrastic constructions seem to be indistinguishable
from their synthetic counterparts, if there is any
\citep[\citepage{1089, 1092}]{jacques2021grammar}.
We tentatively assume that there is no syntactic parameter realized 
by the periphrastic/synthetic distinction.

\paragraph*{Periphrastic constructions involving past categories}
The two sensory past cells in \prettyref{tbl:realis-tam} may be filled by 
\category{periphrastic narrative} \citep[\citesec{21.5.1.8}]{jacques2021grammar}
and \category{periphrastic imperfective narrative} \citep[\citepage{1157}]{jacques2021grammar},
although the two constructions are only in use in a part of the population; 
the meaning of the two constructions are also mostly similar to 
the \category{inferential} and the \category{inferential imperfective}.

The reason for the \category{aorist} and \category{past imperfective} forms to be used is clear:
there is no \category{sensory aorist} copula,
so the past tense plus perfective aspectuality has to be marked on the main verb.

\paragraph*{Periphrastic constructions involving the \category{factual}}
The \category{factual} may be used with 
various copulas to form several \category{periphrastic proximative} constructions
\citep[\citesec{21.3.1.4}]{jacques2021grammar}.

It seems the non-past tense value of the \category{factual} is breached in
\category{periphrastic proximative} constructions:
the \ac{tame} value contributed by the \category{factual}
is the fact that one of its uses indicates a definite staring point of the event.
Therefore, \category{periphrastic proximative} constructions
can always refer to an event going to happen.
This effect also appears in the \category{proximative} 
\prettyref{sec:tame.secondary.proximative}.

The \category{periphrastic proximative} construction
with the past tense \category{past imperfective} or \category{inferential imperfective} copula,
therefore, means \translate{was about to do $X$}.
The \translate{almost did $X$} meaning may be a semantic extension.

When the copula is \category{aorist},
the tense value of the whole clause seems to vanish,
as the clause can refer to both past and future events.
This may be comparable to the English \category{past of irrealis},
i.e. \form{were I you, \dots}

When a \category{sensory} copula is used,
the meaning of the \category{periphrastic proximative} construction 
becomes \translate{is about to $X$}.
Again, we see that the copula brings a tense value.

\section{Missing \ac{tame} concepts}
\label{sec:tame.missing}

\paragraph*{Future}
The meaning of future
is regularly expressed in the \category{factual} category (\citepage{1102}),
and thus is sometimes recognized as the future tense 
or ``factual evidentiality in the future tense'' 
\citep[\citepage{518}]{jacques2019egophoric}.
This however seems to be the natural extension of the meaning 
of the present tense 
(c.f. English \form{the next high tide is around 4 this afternoon}; 
\citealt[\citepage{131}, {[20]}]{cgel}),
and in \citet{jacques2021grammar}, 
the future tense is not recognized as a grammatical tense in Japhug
\citep[\citepage{1102}, (46)]{jacques2021grammar}.

\chapter{Clause embedding}

\paragraph*{Manner serial verb construction}

The term \term{serial verb construction} is a cover-all terms for constructions 
where there are two verbs (or at least words that look like verbs) found in a clause 
but the clause is clearly not a complement clause construction or an auxiliary verb construction.
The underlying structure can be extremely heterogeneous:
control construction, coordination on the level of core verb phrase 
(hence the two verbs share the same \ac{tame} marking), 
manner adverbial construction where the modifier is a core verb phrase, 
non-prototypical auxiliary clause construction, and even more.
In Japhug attested serial verb constructions can all be placed under the 
category of manner adverbial construction \citep[\citesec{25.4.1}]{jacques2021grammar}.

In the Japhug serial verb construction,
the modified main verb is the second verb; 
the modifier precedes the main verb and may be one of the follows:
\begin{itemize}
    \item A deideophonic verb \citep[\citesec{25.4.1.1}]{jacques2021grammar};
    \item A similative verb phrase containing \form{fse} or \form{stu} 
        and their semi-object, 
        with the meaning of \translate{do like this};
    \item A verb phrase describing simultaneous action \citep[\citesec{25.4.1.4}]{jacques2021grammar},
    possibly followed by the emphasis marker \form{zo};
    \item Other verbs of manner. 
\end{itemize}
In all cases, the first verb (or verb phrase) is the modifier; 
discourse linker \form{tɕe} can appear between the modifier and the main verb
\citep[\citepage{1408}, (73)]{jacques2021grammar}, 
demonstrating that the two verbs are two morphological words. 

\paragraph*{Degree serial verb construction} 

Interestingly, a serial verb construction describing the degree of an action 
can also be found in Japhug; 
in this construction the verb describing the degree is the \emph{second} verb.
One way to analyze the historical origin of the construction 
is to treat everything before the stative degree verb as its complement,
and thus \citet[\citepage{1410}, (76)]{jacques2021grammar} 
may be analyzed as \translate{that the elders who knew [traditional stories] well die is finished.}
This however is against the observation that the agreement marker on the second verb 
agrees with the subject (and therefore is plural in the above case, 
not singular as expected for an impersonal verb),
which clearly says that the construction is indeed monoclausal.

Another possible historical origin of the construction 
is verb phrase-level coordination,  
something with the meaning of 
English \form{?The elders who knew traditional stories died and they finished.}
The English example here is awkward, 
partly because \form{die} is a state-change verb in English 
and therefore specifying its progress is semantically unacceptable.

\chapter{Complement clause}

\section{Indirect speech}

\subsection{Hybrid indirect speech}\label{sec:complement.indirect.hybrid}

This means the person feature of the main clause
can sometimes infiltrate into the quoted speech
and becomes its global personal viewpoint.

\begin{todobox}{Hybrid indirect speech}{hybrid-indirect}
    \begin{itemize}
        \item Possessive marker
    \end{itemize}
\end{todobox}

\chapter{Analyzed examples}

The sentence final stative verb \form{ŋu} be.\category{fact} is 
listed as a stative verb in the dictionary
and seems to take the constituents before it 
as a finite complement clause (TODO: or report speech? see the condition on \citepage{1317}),
which is without any explicit complementizer. 
But also see \citepages{1081, }

\bibliographystyle{plainnat}
\bibliography{gyalrong.bib}

\end{document}