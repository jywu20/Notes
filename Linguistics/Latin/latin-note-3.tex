\documentclass[a4paper, oneside]{report}

\usepackage{geometry}
\usepackage{float}
\usepackage{titling}
\usepackage{titlesec}
\usepackage{paralist}
\usepackage{footnote}
\usepackage{enumerate}
\usepackage{amsmath, amssymb, amsthm}
\usepackage{gb4e}
\noautomath
\usepackage{bbm}
\usepackage{soul}
\usepackage{graphicx}
\usepackage{siunitx}
\usepackage[table,xcdraw]{xcolor}
\usepackage{tikz}
\usepackage[ruled, vlined, linesnumbered, noend]{algorithm2e}
\usepackage{xr-hyper}
\usepackage[colorlinks,citecolor=purple]{hyperref} % linkcolor=black, anchorcolor=black, citecolor=black, filecolor=black
\usepackage[most]{tcolorbox}
\usepackage{caption}
\usepackage{subcaption}
\usepackage{booktabs}
\usepackage{multirow}
\usepackage[figuresright]{rotating}
\usepackage{acro}
\usepackage[round]{natbib} 
\usepackage{nameref,zref-xr}
\zxrsetup{toltxlabel}
\zexternaldocument*[cgel-]{../English/cambridge}[cambridge.pdf]
\zexternaldocument*[alignment-]{../alignment/alignment}[alignment.pdf]
\zexternaldocument*[exercise1-]{../Exercise/2021-3}[2021-3.pdf]
\zexternaldocument*[general-]{../methodology/glossing}[glossing.pdf]
\usepackage{prettyref}

\geometry{left=3.18cm,right=3.18cm,top=2.54cm,bottom=2.54cm}
\titlespacing{\paragraph}{0pt}{1pt}{10pt}[20pt]
\setlength{\droptitle}{-5em}

\DeclareMathOperator{\timeorder}{\mathcal{T}}
\DeclareMathOperator{\diag}{diag}
\DeclareMathOperator{\legpoly}{P}
\DeclareMathOperator{\primevalue}{P}
\DeclareMathOperator{\sgn}{sgn}
\newcommand*{\ii}{\mathrm{i}}
\newcommand*{\ee}{\mathrm{e}}
\newcommand*{\const}{\mathrm{const}}
\newcommand*{\suchthat}{\quad \text{s.t.} \quad}
\newcommand*{\argmin}{\arg\min}
\newcommand*{\argmax}{\arg\max}
\newcommand*{\normalorder}[1]{: #1 :}
\newcommand*{\pair}[1]{\langle #1 \rangle}
\newcommand*{\fd}[1]{\mathcal{D} #1}

\newcommand*{\citesec}[1]{\S~{#1}}
\newcommand*{\citechap}[1]{chap.~{#1}}
\newcommand*{\citefig}[1]{Fig.~{#1}}
\newcommand*{\citetable}[1]{Table~{#1}}
\newcommand*{\citepage}[1]{pp.~{#1}}
\newcommand*{\citefootnote}[1]{fn.~{#1}}

\newrefformat{sec}{\citesec{\ref{#1}}}
\newrefformat{fig}{\citefig{\ref{#1}}}
\newrefformat{tbl}{\citetable{\ref{#1}}}
\newrefformat{chap}{\citechap{\ref{#1}}}
\newrefformat{fn}{\citefootnote{\ref{#1}}}
\newrefformat{box}{Box~\ref{#1}}

\tcbuselibrary{skins, breakable, theorems}

\newtcbtheorem[number within=chapter]{infobox}{Box}{
    enhanced,
    boxrule=0pt,
    colback=blue!5,
    colframe=blue!5,
    coltitle=blue!50,
    borderline west={4pt}{0pt}{blue!65},
    sharp corners,
    fonttitle=\bfseries, 
    breakable,
    before upper={\parindent15pt\noindent}}{box}
\newtcbtheorem[number within=chapter, use counter from=infobox]{theorybox}{Box}{
    enhanced,
    boxrule=0pt,
    colback=orange!5, 
    colframe=orange!5, 
    coltitle=orange!50,
    borderline west={4pt}{0pt}{orange!65},
    sharp corners,
    fonttitle=\bfseries, 
    breakable,
    before upper={\parindent15pt\noindent}}{box}
\newtcbtheorem[number within=chapter, use counter from=infobox]{learnbox}{Box}{
    enhanced,
    boxrule=0pt,
    colback=green!5,
    colframe=green!5,
    coltitle=green!50,
    borderline west={4pt}{0pt}{green!65},
    sharp corners,
    fonttitle=\bfseries, 
    breakable,
    before upper={\parindent15pt\noindent}}{box}

\newcommand*{\concept}[1]{\textbf{#1}}
\newcommand*{\term}[1]{\emph{#1}}
\newcommand{\corpus}[1]{\emph{#1}}

%region Acronym

% Theory
\DeclareAcronym{blt}{short = BLT, long = Basic Linguistic Theory}
\DeclareAcronym{cgel}{short = CGEL, long = The Cambridge Grammar of the English Language}
\DeclareAcronym{dm}{short = DM, long = Distributed Morphology}
\DeclareAcronym{tag}{long = Tree-adjoining grammar, short = TAG}

% History

\DeclareAcronym{pie}{long = proto-Indo-European, short = PIE}

% Roles

\DeclareAcronym{sfp}{long = sentence final particle, short = SFP}
\DeclareAcronym{np}{long = noun phrase, short = NP}
\DeclareAcronym{vp}{long = verb phrase, short = VP}
\DeclareAcronym{pp}{long = preposition phrase, short = PP}
\DeclareAcronym{adjp}{long = adjective phrase, short = AdjP}
\DeclareAcronym{advp}{long = adverb phrase, short = AdvP}
\DeclareAcronym{cc}{long = copular complement, short = CC}
\DeclareAcronym{cs}{long = copular subject, short = CS}
\DeclareAcronym{tam}{long = {tense, aspect, mood}, short = TAM}
\DeclareAcronym{tame}{long = {Tense, Aspect, Mood, Evidentiality}, short = TAME}
\DeclareAcronym{copula}{long = copula, short = COP}

% Pronouns 

\DeclareAcronym{dist}{long = distal, short = \textsc{dist}}
\DeclareAcronym{prox}{long = proximate, short = \textsc{prox}}
\DeclareAcronym{dem}{long = demonstrative, short = \textsc{dem}}

% TAME, negative

\DeclareAcronym{neg}{long = negative, short = \textsc{neg}}
\DeclareAcronym{past}{long = \textsc{past}, short = \textsc{pst}}
\DeclareAcronym{imperfect}{long = \textsc{imperfect}, short = \textsc{impf}}
\DeclareAcronym{present}{long = \textsc{present}, short = \textsc{pres}}
\DeclareAcronym{perfect}{long = \textsc{perfect}, short = \textsc{perf}}
\DeclareAcronym{future}{long = \textsc{future}, short = \textsc{fut}}
\DeclareAcronym{pluperfect}{long = \textsc{pluperfect}, short = \textsc{plup}}
\DeclareAcronym{future perfect}{long = \textsc{future perfect}, short = \textsc{fut.perf}}
\DeclareAcronym{passive}{long = passive, short = PASS}
\DeclareAcronym{indicative}{long = \textsc{indicative}, short = \textsc{ind}}
\DeclareAcronym{subjunctive}{long = \textsc{subjunctive}, short = \textsc{sjv}}
\DeclareAcronym{imperative}{long = \textsc{imperative}, short = \textsc{imp}}


%endregion

\newcommand*{\homo}[2]{#1$_{\text{#2}}$}

\newcommand{\cgel}{\href{../English/cambridge.pdf}{my notes about CGEL}}
\newcommand{\latin}{\href{../Latin/latin-notes.pdf}{my notes about Latin}}
\newcommand{\alignment}{\href{../alignment/alignment.pdf}{my notes about alignment}}
\newcommand{\exerciseone}{\href{../Exercise/2021-3.pdf}{this exercise}}
\newcommand{\general}{\href{../methodology/glossing.pdf}{this note}}

\newcommand{\ala}{à la}
\newcommand{\translate}[1]{`#1'}
\newcommand{\vP}{\textit{v}P}

\newcommand{\classify}[1]{{\textsc{#1}}}

% Make subsubsection labeled
\setcounter{secnumdepth}{4}
\setcounter{tocdepth}{4}
% reset example counter every chapter (but do not include the chapter number to the label)
\counterwithin{exx}{chapter} 

\renewcommand{\bibname}{References}

\title{Note on Latin Grammar}
\author{Jinyuan Wu}

\begin{document}

\maketitle

\chapter{Overview}

\section{Historical notes}

This note is about Classical Latin and Ecclesiastical Latin.
That's to say Old Latin, vulgar Latin (with prototypes of Romance articles), etc.
are not discussed.

\section{Phonology and the writing system}

\section{Parts of speech}

\subsection{Classification}

Latin word classes can be defined easily via morphology,
and these classes prove to have morphosyntactic significance.
Traditionally speaking, 
word classes with none or poor morphology are called \concept{particles},
and non-particle words can be divided into two large classes:
those with similar morphology of prototypical nouns (i.e. \concept{declension}) are \concept{nominals},
while words with similar morphology of prototypical verbs (i.e. \concept{conjugation})
form a uniform class rightfully called \concept{verbs}.
Nominals include \concept{nouns} and \concept{adjectives},
the distinction between the two can also be defined morphologically.

Latin particles include \concept{prepositions}, \concept{adverbs},
\concept{interjections}, and \concept{conjunctions}.
The adverb class and the preposition class have a large overlap:
often a preposition has an intransitive counterpart,
which is similar to a prototypical adverb.
Conjunctions may be seen as ``prepositions for clauses''.
The functions and etymologies of particles are highly diverse.

Latin nouns, verbs, and adjectives are all open categories.
They are able to head constituents,
and so are correlatives (though correlatives can be listed in the grammar).
The preposition class is closed and is a part of the grammar,
just like conjunctions.
However, conjunctions are purely functional,
while certain prepositions may be argued to head attributive expressions:
though prepositions are often said to be markers of a periphrastic case system,
the semantics carried by certain Latin prepositions are too complicated for a case system.
This is also the case of adverbs:
some adverbs seem to be periphrastic markers of \acs{tame} categories
and therefore may be considered as a part of the grammar,
while others seem to carry ``real'' meanings.
\prettyref{fig:latin-word-class} is a visualization of the classification of Latin word classes.

\begin{theorybox}{Lexical and function classes}{word-class-classification}
    In \citesec{\ref{general-sec:lexical-function-distinction}} in \general, 
    by words with ``real category labels'',
    I mean words that have``real'' meanings
    and serve as lexical heads of constituents
    (i.e. being surrounded by function words and dependents).
    Certain adverbs and prepositions have ``real category labels'',
    and they appear at the left side in \prettyref{fig:latin-word-class}.
    Prepositions can be enumerated and therefore are considered as a part of the grammar,
    so they are always at the lower side in \prettyref{fig:latin-word-class}.
    Other adverbs and prepositions are light in their semantic
    and are purely functional,
    so they appear in the southeast corner of \prettyref{fig:latin-word-class}.

    Note that this is not the standard terminology. 
    Linguists use their own notion of \term{lexical class} and \term{function class}
    to cover what I say here. 
\end{theorybox}

\begin{sidewaysfigure}
    \centering
    

\tikzset{every picture/.style={line width=0.3pt}} %set default line width to 0.75pt        

\begin{tikzpicture}[x=0.75pt,y=0.75pt,yscale=-0.8,xscale=0.8]
%uncomment if require: \path (0,674); %set diagram left start at 0, and has height of 674

%Straight Lines [id:da0819119912251447] 
\draw [color={rgb, 255:red, 155; green, 155; blue, 155 }  ,draw opacity=0.2 ]   (109.33,384.91) -- (793.33,384.91) ;
%Rounded Rect [id:dp9835773367494156] 
\draw  [color={rgb, 255:red, 155; green, 155; blue, 155 }  ,draw opacity=1 ][fill={rgb, 255:red, 155; green, 155; blue, 155 }  ,fill opacity=0.2 ] (141.33,256.64) .. controls (141.33,246.48) and (149.57,238.24) .. (159.73,238.24) -- (214.93,238.24) .. controls (225.1,238.24) and (233.33,246.48) .. (233.33,256.64) -- (233.33,518.84) .. controls (233.33,529) and (225.1,537.24) .. (214.93,537.24) -- (159.73,537.24) .. controls (149.57,537.24) and (141.33,529) .. (141.33,518.84) -- cycle ;
%Rounded Rect [id:dp12690257166641228] 
\draw  [color={rgb, 255:red, 155; green, 155; blue, 155 }  ,draw opacity=1 ][fill={rgb, 255:red, 155; green, 155; blue, 155 }  ,fill opacity=0.2 ] (233.33,256.64) .. controls (233.33,246.48) and (241.57,238.24) .. (251.73,238.24) -- (306.93,238.24) .. controls (317.1,238.24) and (325.33,246.48) .. (325.33,256.64) -- (325.33,518.84) .. controls (325.33,529) and (317.1,537.24) .. (306.93,537.24) -- (251.73,537.24) .. controls (241.57,537.24) and (233.33,529) .. (233.33,518.84) -- cycle ;
%Rounded Rect [id:dp4571683339095527] 
\draw  [color={rgb, 255:red, 155; green, 155; blue, 155 }  ,draw opacity=1 ][fill={rgb, 255:red, 155; green, 155; blue, 155 }  ,fill opacity=0.2 ] (349.33,256.57) .. controls (349.33,248.1) and (356.2,241.24) .. (364.67,241.24) -- (426,241.24) .. controls (434.47,241.24) and (441.33,248.1) .. (441.33,256.57) -- (441.33,302.57) .. controls (441.33,311.04) and (434.47,317.91) .. (426,317.91) -- (364.67,317.91) .. controls (356.2,317.91) and (349.33,311.04) .. (349.33,302.57) -- cycle ;
%Rounded Rect [id:dp13945556495695044] 
\draw  [color={rgb, 255:red, 155; green, 155; blue, 155 }  ,draw opacity=1 ][fill={rgb, 255:red, 155; green, 155; blue, 155 }  ,fill opacity=0.2 ] (122.93,223.64) .. controls (122.93,211.82) and (132.51,202.24) .. (144.33,202.24) -- (316.93,202.24) .. controls (328.75,202.24) and (338.33,211.82) .. (338.33,223.64) -- (338.33,537.84) .. controls (338.33,549.66) and (328.75,559.24) .. (316.93,559.24) -- (144.33,559.24) .. controls (132.51,559.24) and (122.93,549.66) .. (122.93,537.84) -- cycle ;
%Straight Lines [id:da9866800561526903] 
\draw [color={rgb, 255:red, 155; green, 155; blue, 155 }  ,draw opacity=0.2 ]   (595.33,577.91) -- (595.33,184.91) ;
%Rounded Rect [id:dp9919662328975185] 
\draw  [color={rgb, 255:red, 155; green, 155; blue, 155 }  ,draw opacity=1 ][fill={rgb, 255:red, 155; green, 155; blue, 155 }  ,fill opacity=0.2 ] (435,431.44) .. controls (435,427.46) and (438.22,424.24) .. (442.2,424.24) -- (688.8,424.24) .. controls (692.78,424.24) and (696,427.46) .. (696,431.44) -- (696,453.04) .. controls (696,457.02) and (692.78,460.24) .. (688.8,460.24) -- (442.2,460.24) .. controls (438.22,460.24) and (435,457.02) .. (435,453.04) -- cycle ;
%Rounded Rect [id:dp04492990660366303] 
\draw  [color={rgb, 255:red, 155; green, 155; blue, 155 }  ,draw opacity=1 ][fill={rgb, 255:red, 155; green, 155; blue, 155 }  ,fill opacity=0.2 ] (147,482.31) .. controls (147,476.75) and (151.51,472.24) .. (157.07,472.24) -- (579.93,472.24) .. controls (585.49,472.24) and (590,476.75) .. (590,482.31) -- (590,512.51) .. controls (590,518.07) and (585.49,522.57) .. (579.93,522.57) -- (157.07,522.57) .. controls (151.51,522.57) and (147,518.07) .. (147,512.51) -- cycle ;
%Shape: Path Data [id:dp3209513602067542] 
\draw  [color={rgb, 255:red, 155; green, 155; blue, 155 }  ,draw opacity=1 ][fill={rgb, 255:red, 155; green, 155; blue, 155 }  ,fill opacity=0.2 ] (478.97,255.24) -- (552.94,255.24) .. controls (557.16,255.24) and (560.58,258.46) .. (560.58,262.43) -- (560.58,367.51) .. controls (560.58,372.79) and (565.13,377.07) .. (570.74,377.07) -- (630.37,377.07) .. controls (630.59,377.07) and (630.82,377.07) .. (631.04,377.05) -- (712.31,377.05) .. controls (717.01,377.05) and (720.81,380.63) .. (720.81,385.05) -- (720.81,434.24) .. controls (720.81,438.66) and (717.01,442.24) .. (712.31,442.24) -- (478.97,442.24) .. controls (474.75,442.24) and (471.33,439.02) .. (471.33,435.05) -- (471.33,262.43) .. controls (471.33,258.46) and (474.75,255.24) .. (478.97,255.24) -- cycle ;
%Rounded Rect [id:dp9478936078417666] 
\draw  [color={rgb, 255:red, 155; green, 155; blue, 155 }  ,draw opacity=1 ][fill={rgb, 255:red, 155; green, 155; blue, 155 }  ,fill opacity=0.2 ] (451.33,237.22) .. controls (451.33,228.95) and (458.04,222.24) .. (466.31,222.24) -- (759.35,222.24) .. controls (767.63,222.24) and (774.33,228.95) .. (774.33,237.22) -- (774.33,552.26) .. controls (774.33,560.53) and (767.63,567.24) .. (759.35,567.24) -- (466.31,567.24) .. controls (458.04,567.24) and (451.33,560.53) .. (451.33,552.26) -- cycle ;

% Text Node
\draw (169,267) node [anchor=north west][inner sep=0.75pt]   [align=left] {noun};
% Text Node
\draw (252,267) node [anchor=north west][inner sep=0.75pt]   [align=left] {adjective};
% Text Node
\draw (158,479) node [anchor=north west][inner sep=0.75pt]   [align=left] {personal \\pronoun};
% Text Node
\draw (244,479) node [anchor=north west][inner sep=0.75pt]   [align=left] {correlative \\pronoun};
% Text Node
\draw (368,266) node [anchor=north west][inner sep=0.75pt]   [align=left] {verb};
% Text Node
\draw (565.5,442.24) node   [align=left] {preposition};
% Text Node
\draw (674,517) node [anchor=north west][inner sep=0.75pt]   [align=left] {conjunction};
% Text Node
\draw (492,264) node [anchor=north west][inner sep=0.75pt]   [align=left] {adverb};
% Text Node
\draw (674,471) node [anchor=north west][inner sep=0.75pt]   [align=left] {interjection};
% Text Node
\draw (151,361) node [anchor=north west][inner sep=0.75pt]  [color={rgb, 255:red, 155; green, 155; blue, 155 }  ,opacity=1 ] [align=left] {noun \\morphology};
% Text Node
\draw (238.14,361) node [anchor=north west][inner sep=0.75pt]  [color={rgb, 255:red, 155; green, 155; blue, 155 }  ,opacity=1 ] [align=left] {adjective \\morphology};
% Text Node
\draw (107.33,384.91) node [anchor=east] [inner sep=0.75pt]  [color={rgb, 255:red, 155; green, 155; blue, 155 }  ,opacity=1 ] [align=left] {with real \\category \\label};
% Text Node
\draw (795.33,384.91) node [anchor=west] [inner sep=0.75pt]  [color={rgb, 255:red, 155; green, 155; blue, 155 }  ,opacity=1 ] [align=left] {with \\no real \\category \\label};
% Text Node
\draw (141,212.24) node [anchor=north west][inner sep=0.75pt]  [color={rgb, 255:red, 155; green, 155; blue, 155 }  ,opacity=1 ] [align=left] {nominal};
% Text Node
\draw (595.33,181.91) node [anchor=south] [inner sep=0.75pt]  [color={rgb, 255:red, 155; green, 155; blue, 155 }  ,opacity=1 ] [align=left] {not a part\\of grammar};
% Text Node
\draw (595.33,580.91) node [anchor=north] [inner sep=0.75pt]  [color={rgb, 255:red, 155; green, 155; blue, 155 }  ,opacity=1 ] [align=left] {a part of\\grammar};
% Text Node
\draw (499,489.5) node [anchor=north west][inner sep=0.75pt]   [align=left] {pro-adverb};
% Text Node
\draw (379,489.5) node [anchor=north west][inner sep=0.75pt]  [color={rgb, 255:red, 155; green, 155; blue, 155 }  ,opacity=1 ] [align=left] {pro-forms};
% Text Node
\draw (661,241.5) node [anchor=north west][inner sep=0.75pt]  [color={rgb, 255:red, 155; green, 155; blue, 155 }  ,opacity=1 ] [align=left] {particle};


\end{tikzpicture}

    \caption{Latin word classes}
    \label{fig:latin-word-class}
\end{sidewaysfigure}

Articles (English \corpus{a} or \corpus{the}), 
despite prevalent in other Indo-European languages,
are missing in Latin.
This, together with the fact that Classical Sanskrit and Old Persian didn't have articles 
and the Slavic languages still don't,
is a strong indicator that \ac{pie} didn't have articles. 

\section{Morphology}

Latin has rich morphology,
which enables a rather free -- but still not completely arbitrary -- constituent order.
Latin has a clear inflection-derivation distinction.
Despite its richness, 
Latin derivation is largely historical,
with meanings of derived forms 
having shifted and no longer regularly inferrable.
Latin inflection is always suffixal,
while derivation is predominantly prefixal.
Concatenative morphology (affixation and compounding) 
is prominent but isn't the only morphological device:
the following non-concatenative mechanisms are all attested:
\begin{itemize}
    \item \emph{Reduplication}: formation of the perfect stem (TODO: ref)
    \item \emph{Subtraction}: dropping of first-conjugation stem-final vowel (\prettyref{sec:tense-mood-marking}).
    \item \emph{Infixation}:   TODO: ref 
    The imperfect \corpus{-ba-} is sometimes said to be an infix 
    (as well as its counterparts like \corpus{-bi-}),
    though it fits in a concatenative picture of verbal morphology.
\end{itemize}
These mechanisms, however, are largely historical,
just like their concatenative counterparts.



\section{Noun phrases and nominal morphology}

\section{Verbal morphology and clause structure}\label{sec:verb-inflection-abs}

Most clausal grammatical categories are marked on the verbal morphology.
Sometimes a grammatical category is there but is not reflected in the morphology.
For example, in English we have infinitive clauses,
but strictly speaking, there is no such thing as ``infinitive verb'':
the head verb of an infinitive clause 
has exactly the same form of a non-third person singular present tense verb.
This is not the case in Latin.
For example, the head verb of a infinitive clause in Latin 
indeed has a separate position in the paradigm.
Thus, grammatical categories of the clause are listed in this section.

\subsection{The finite paradigm}

\subsubsection{Voice}

Latin doesn't have rich valency changing devices:
there is only one clause-wide valency decreasing device -- passivization -- 
and there is no valency increasing device.
Causative constructions are realized by complement clauses,
not any change in the argument structure.
Whether passivization happens is recorded by the category of \concept{voice}.
A verb (and hence the clause headed by it) is therefore either in \concept{active voice},
or in \concept{passive voice}.

\begin{theorybox}{Valency changing}{valency-changing}
    See \citesec{\ref{general-sec:valency-changing-theory}} in \general.
    From a generative perspective, some languages realize valency changing 
    by a series of \vP{} structures, and then the case assignment of the arguments is trivial.
    Some languages use non-trivial structural case assignment mechanisms
    to achieve valency changing 
    (``suppressing the agent argument, 
    and leave the nominative probe to find the subject;
    the probe then has to choose the patient argument'').
    Of course, \vP{} changes in the second type are still there,
    which may be a likely source of relevant verb morphology.
    Naturally, the second group of languages have more restricted valency changing devices;
    this is the case of Latin.
\end{theorybox}

\subsubsection{\acs{tame} categories}

Latin has fused tense and aspect:
the composition of three tense values and three aspect values 
gives nine options,
but in Latin, there are only six morphologically distinguished options,
as is shown in \prettyref{tbl:latin-tense-aspect}. 
When people talk about \concept{tense} in Latin (and in many other Indo-European languages),
they are often taking about things like the six options,
instead of the past/present/future system.

\begin{table}[H]
    \caption{Latin tense and aspect}
    \label{tbl:latin-tense-aspect}
    \centering
    \begin{tabular}{@{}cccc@{}}
    \toprule
              & past       & present                  & future                  \\ \midrule
    imperfect & imperfect  & \multirow{2}{*}{present} & \multirow{2}{*}{future} \\
    simple    & perfect    &                          &                         \\
    perfect   & pluperfect & perfect                  & future perfect          \\ \bottomrule
\end{tabular}    
\end{table}

Similar fusion between categories is shown in the category of \concept{mood}.
It's the fusion of morphologically marked clause type 
(declarative and imperative)
and morphologically marked modality.
The verb morphology of interrogative clauses is exactly the same as declarative clauses:
the interrogative clause type is marked by the existence of interrogative \term{pro}-forms.
Thus, there are three moods in finite clauses in Latin:
\acl{indicative}, \acl{subjunctive}, and \acl{imperative}.
The \acl{indicative} is the fusion of 
the declarative/interrogative clause type and the realis modality.
The \acl{subjunctive} mood is the fusion of 
the declarative/interrogative clause type and the irrealis modality.
The \acl{imperative} is basically the imperative clause type:
it doesn't allow modality marking.
Sometimes people say the infinitive is the fourth mood,
though it's a non-finite clause.

\begin{infobox}{The term \term{mood}}{mood}
    \acs{blt} only calls the first category \term{mood}.
    Different linguists use the term \term{mood} and \term{modality} in radically different ways.
    In this note I just focus on the common practice in Latin grammar study.
\end{infobox}

\begin{theorybox}{Mismathc between \ac{tame} constructions and fine-grained categories}{tame}
    Atomic \ac{tame} features and packaged \ac{tame} marking constructions
    often show certain degree of discrepancy.
    As we see in \prettyref{tbl:latin-tense-aspect},
    the \acl{perfect} construction may have simple aspect and past tense.
    Following the example in \citet{grimm2021grammar},
    in this note, I use small capitals for the names of attested surface realizations of \ac{tame}
    and the default font for \ac{tame} values.
    (Some other grammars, like \citet{jacques2021grammar,friesen2017grammar}, 
    use initial capitals for the former.)
\end{theorybox}

\subsubsection{Agreement}\label{sec:agreement-abs}

Latin is a typical nominative-accusative language,
both morphologically and syntactically.
In finite clauses, 
there is subject-verb agreement:
the number and person of the subject is marked on the main verb.
In the case of periphrastic conjugation,
the features are marked on the copula.

\subsubsection{Compatability of categories}

There is no \acl{future} tense and \acl{future perfect} tense in subjunctive clauses,
probably for the semantic reason
that the future tense already contains certain sense of modality
(an event predicted to happen),
and thus is not compatible with the \acl{subjunctive} mood.
The \acl{imperative} mood is not compatible with other \ac{tame} markings
except the \acl{present} tense and the \acl{future} tense.
It's still compatible with the voice category,
and allowed persons are 
second person singular/plural with the \acl{present} tense,
and second/third person singular/plural with the \acl{future} tense.
The absence of first person is also probably from semantic origin.

In conclusion, the categories involved in the finite verb paradigm of Latin 
are shown in \prettyref{fig:paradigm-finite-verb}.
Here mood and tense are realized in one morpheme,
and voice, person and number are realized in one morpheme.
The paradigm is realized synthetically in all circumstances 
except in passive voice and perfect tense.
In that case, the verb conjugation is realized like the English passive,
i.e. via a copula and the perfect passive participle.
The realization of the paradigm is divided into four conjugation classes (\prettyref{sec:finite-paradigm}),
and there are also deponent verbs (\prettyref{sec:deponent-verbs}) and 
irregular verbs (\prettyref{sec:irregular-verbs}).

\begin{theorybox}{About the number of verb forms}{conjugation-form}
    Different people use the term \term{verb forms} -- and count them -- in different ways.
    The most generous -- and the most syntactically relevant -- way 
    is to view the realization of every possible CP-TP-\vP{} projection 
    as a form of the main verb -- the verb root at the core of the CP-TP-\vP{} domains.
    This results in a paradigm in traditional grammar, 
    essentially the traditional way to enumerate Latin verb forms 
    (``the indicative active present second person form of a verb is \dots'').

    The problem with this approach is sometimes two cells in the paradigm are always identical.
    In this way, morphosyntactically there are indeed two different paradigm cells,
    but morphophonologically there is only one verb form.
    Take English as an example: 
    a traditional grammar may say 
    ``the present subjunctive first person singular of English \corpus{take} is \corpus{take}''. 
    The problem here is the present subjunctive first person singular \emph{clause}
    always contains the same form of the \emph{verb}
    with a present indicative first person singular clause,
    so it makes no sense to talk about ``the present subjunctive first person singular \emph{verb form}''.
    A stingy linguist may then stipulate that conjugation forms are literally about \emph{forms},
    and thus there is no such thing as ``the subjunctive form'' of English verbs,
    because in subject \emph{clauses}, 
    the main verb always has the same form as the infinitive
    \citep[\citepage{76}]{cgel}.

    Another problem with this approach occurs 
    when dealing with languages like Japanese.
    There are so many suffix slots,
    and the boundaries between suffixes are relatively clear,
    so the paradigm is too big to be displayed as a whole
    and too regular to be enumerated cell by cell.

    The analysis of conjugation forms of the verb, theoretically speaking,
    is more about vocabulary insertion and readjustment rules,
    instead of the syntax proper.
    This is an instance of the \emph{separation principle}:
    morphophonological features can be separated from morphosyntactic features
    \citep{embick2000features}.
    Distinguishing between verb forms and clause categories isn't just a game about wording:
    in periphrastic conjugation,
    we have auxiliary verb(s) plus a non-finite verb form,
    but here the non-finite verb form is just the spellout of several features together with the verb root 
    and is definitely not thea head of a non-finite \emph{clause}:
    what we have here is one clause, not clause embedding.
    Thus it makes no sense to say ``we use a non-finite verb form in a periphrastic construction'',
    because finiteness is a category of a clause,
    and here is no clause combining.
    This is also relevant for surface-oriented descriptive linguistics:
    \citet[\citepage{74,83}]{cgel} rejects the notion of the \term{infinitive form} of the verb,
    and replace the term by \term{default form},
    because the so-called infinitive form also appears in the subjunctive mood
    or the imperative mood. 
    Despite this, to respect the tradition, 
    I will still use the term \term{non-finite verbs} 
    or wordings like ``the perfect passives are formed by attaching forms of copula to the perfect participle''.

    The generous paradigm-cell-as-verb-form approach fortunately works in Latin 
    because Latin is morphologically rich
    and thanks to historical changes,
    the boundaries between suffixes marking each grammatical category 
    are already vague enough, so the Japanese School Grammar approach is also not applicable. 
    So it does make sense to talk about 
    ``the indicative active perfect second person singular form'' of a verb.
    Similarly, we also talk about non-finite verb forms (\prettyref{sec:non-finite-abs}),
    though strictly speaking, finiteness is a category of the clause.
\end{theorybox}

\begin{figure}[H]
    \centering
    

\tikzset{every picture/.style={line width=0.75pt}} %set default line width to 0.75pt        

\begin{tikzpicture}[x=0.75pt,y=0.75pt,yscale=-0.8,xscale=0.8]
%uncomment if require: \path (0,560); %set diagram left start at 0, and has height of 560

%Straight Lines [id:da4771979505475994] 
\draw [color={rgb, 255:red, 208; green, 2; blue, 27 }  ,draw opacity=1 ][line width=2.25]    (72,476.59) -- (165,476.59) ;
%Shape: Rectangle [id:dp4651809120211663] 
\draw  [draw opacity=0][fill={rgb, 255:red, 208; green, 2; blue, 27 }  ,fill opacity=0.1 ] (72,51.93) -- (165,51.93) -- (165,195.59) -- (72,195.59) -- cycle ;
%Shape: Rectangle [id:dp8674436914111818] 
\draw  [draw opacity=0][fill={rgb, 255:red, 245; green, 166; blue, 35 }  ,fill opacity=0.1 ] (208,51.93) -- (324,51.93) -- (324,195.59) -- (208,195.59) -- cycle ;
%Straight Lines [id:da9097932489630769] 
\draw [color={rgb, 255:red, 245; green, 166; blue, 35 }  ,draw opacity=1 ][line width=2.25]    (210,476.59) -- (324,476.59) ;
%Shape: Rectangle [id:dp6458311390526075] 
\draw  [draw opacity=0][fill={rgb, 255:red, 245; green, 166; blue, 35 }  ,fill opacity=0.1 ] (208,213.93) -- (324,213.93) -- (324,317.59) -- (208,317.59) -- cycle ;
%Shape: Rectangle [id:dp4418485190624626] 
\draw  [draw opacity=0][fill={rgb, 255:red, 208; green, 2; blue, 27 }  ,fill opacity=0.1 ] (73,213.93) -- (166,213.93) -- (166,318.26) -- (73,318.26) -- cycle ;
%Shape: Rectangle [id:dp7155013561532602] 
\draw  [draw opacity=0][fill={rgb, 255:red, 245; green, 166; blue, 35 }  ,fill opacity=0.1 ] (208,341.59) -- (324,341.59) -- (324,373.59) -- (208,373.59) -- cycle ;
%Shape: Rectangle [id:dp5109073549880265] 
\draw  [draw opacity=0][fill={rgb, 255:red, 245; green, 166; blue, 35 }  ,fill opacity=0.1 ] (208,387.59) -- (324,387.59) -- (324,419.59) -- (208,419.59) -- cycle ;
%Shape: Rectangle [id:dp4978185928249894] 
\draw  [draw opacity=0][fill={rgb, 255:red, 208; green, 2; blue, 27 }  ,fill opacity=0.1 ] (73,341.59) -- (166,341.59) -- (166,420.93) -- (73,420.93) -- cycle ;
%Shape: Rectangle [id:dp7974586880324295] 
\draw  [draw opacity=0][fill={rgb, 255:red, 126; green, 211; blue, 33 }  ,fill opacity=0.1 ] (364,52.93) -- (471.33,52.93) -- (471.33,419.59) -- (364,419.59) -- cycle ;
%Straight Lines [id:da40576537799721035] 
\draw [color={rgb, 255:red, 126; green, 211; blue, 33 }  ,draw opacity=1 ][line width=2.25]    (364,476.59) -- (470,476.59) ;
%Shape: Rectangle [id:dp37304865952309374] 
\draw  [draw opacity=0][fill={rgb, 255:red, 80; green, 227; blue, 194 }  ,fill opacity=0.1 ] (515,51.93) -- (631,51.93) -- (631,318.93) -- (515,318.93) -- cycle ;
%Shape: Rectangle [id:dp09424442237743991] 
\draw  [draw opacity=0][fill={rgb, 255:red, 80; green, 227; blue, 194 }  ,fill opacity=0.1 ] (514,341.59) -- (630,341.59) -- (630,373.59) -- (514,373.59) -- cycle ;
%Shape: Rectangle [id:dp7879043441622926] 
\draw  [draw opacity=0][fill={rgb, 255:red, 80; green, 227; blue, 194 }  ,fill opacity=0.1 ] (514,387.59) -- (630,387.59) -- (630,419.59) -- (514,419.59) -- cycle ;
%Straight Lines [id:da6694037096709569] 
\draw [color={rgb, 255:red, 80; green, 227; blue, 194 }  ,draw opacity=1 ][line width=2.25]    (514,476.59) -- (628,476.59) ;
%Shape: Rectangle [id:dp8356914669291959] 
\draw  [draw opacity=0][fill={rgb, 255:red, 74; green, 144; blue, 226 }  ,fill opacity=0.1 ] (675,52.93) -- (782.33,52.93) -- (782.33,419.59) -- (675,419.59) -- cycle ;
%Straight Lines [id:da18180972606397305] 
\draw [color={rgb, 255:red, 74; green, 144; blue, 226 }  ,draw opacity=1 ][line width=2.25]    (676.33,476.59) -- (782.33,476.59) ;

% Text Node
\draw (118.5,123.76) node   [align=left] {indicative};
% Text Node
\draw (119.5,266.09) node   [align=left] {subjunctive};
% Text Node
\draw (266,123.76) node   [align=left] {present/\\imperfect/\\future/\\perfect/\\pluperfect/\\future perfect};
% Text Node
\draw (266,265.76) node   [align=left] {present/\\imperfect/\\perfect/\\pluperfect};
% Text Node
\draw (417.67,236.26) node   [align=left] {active/\\passive};
% Text Node
\draw (573,185.43) node   [align=left] {1/\\2/\\3};
% Text Node
\draw (728.67,236.26) node  [color={rgb, 255:red, 0; green, 0; blue, 0 }  ,opacity=1 ] [align=left] {single/\\plural};
% Text Node
\draw (119.5,381.26) node   [align=left] {imperative};
% Text Node
\draw (266,357.59) node   [align=left] {present};
% Text Node
\draw (266,403.59) node   [align=left] {future};
% Text Node
\draw (572,357.59) node   [align=left] {2};
% Text Node
\draw (572,403.59) node   [align=left] {2/3};
% Text Node
\draw (118.5,479.59) node [anchor=north] [inner sep=0.75pt]  [color={rgb, 255:red, 208; green, 2; blue, 27 }  ,opacity=1 ] [align=left] {mood};
% Text Node
\draw (267,479.59) node [anchor=north] [inner sep=0.75pt]  [color={rgb, 255:red, 245; green, 166; blue, 35 }  ,opacity=1 ] [align=left] {\textcolor[rgb]{0.96,0.65,0.14}{tense}};
% Text Node
\draw (417,479.59) node [anchor=north] [inner sep=0.75pt]  [color={rgb, 255:red, 126; green, 211; blue, 33 }  ,opacity=1 ] [align=left] {voice};
% Text Node
\draw (571,479.59) node [anchor=north] [inner sep=0.75pt]  [color={rgb, 255:red, 80; green, 227; blue, 194 }  ,opacity=1 ] [align=left] {\textcolor[rgb]{0.31,0.89,0.76}{person}};
% Text Node
\draw (729.33,479.59) node [anchor=north] [inner sep=0.75pt]  [color={rgb, 255:red, 74; green, 144; blue, 226 }  ,opacity=1 ] [align=left] {number};


\end{tikzpicture}

    \caption{Categories in the finite paradigm}
    \label{fig:paradigm-finite-verb}
\end{figure}


\subsection{Non-finite forms}\label{sec:non-finite-abs}

According to the morphology,
Latin non-finite verb forms can be classified into the infinitives (\prettyref{sec:infinitives})
and the nominal forms (\prettyref{sec:nominal-form}),
the latter having noun-like or adjective-like morphology (\prettyref{sec:gerund-participle-morphology}).
Non-finite verb forms don't agree with the subjects they take,
so there is no number or person category marked on them in the same way as \prettyref{fig:paradigm-finite-verb},
though for nominal verb forms there are number and person categories 
marked in the same way as the nominal morphology (\prettyref{sec:gerund-participle-morphology}).

The infinitives include present active, present passive, perfect active, 
perfect passive, future active, and future passive infinitives.
The latter three are realized periphrastically (\prettyref{fig:stem-to-form}).

The nominal verb forms include 
simple active, perfect passive (often just called the perfect participle) and future active participles,
a gerund, a gerundive which may be also called as the future passive participle, 
and two supine forms.
The first supine is identical in the form to the singular neutral accusative perfect participle,
without any reference to the number of any argument in the clause.
The second supine is identical to the singular neutral ablative or dative past participle,
also without any reference to the number of any argument in the clause.
Though they are identical to other forms in the 

\begin{infobox}{Whether to keep supine as a verb form}{supine-verb-form}
    The idea of the stingy linguist may lead one to reject the notion of supine in Latin grammar,
    but since sometimes a verb lacks TODO: argumentation for a separate supine form,
    for the same reason the infinitive 
    (or the ``plain form'', since the infinitive is actually a label of clauses 
    -- see the discussion and the separation principle in \prettyref{box:conjugation-form}) 
    is recognized as a form 
    independent from the present form in English in \citet[\citepage{74}]{cgel},
    the status of supine as a separate form is recognized in this note.
\end{infobox}

Latin non-finite verb forms are 

\subsection{Core, oblique, and peripheral arguments}

\subsubsection{Alignment}

Latin is a clear nominative-accusative language.
Similar to what is documented in \acs{cgel},
Latin core arguments are coded as subject, object(s),
and copular complements at the level of alignment.
They can be distinguished by the semantic roles,
case marking, possible contents, and transformational properties 
(\prettyref{sec:core-argument-marking}).
In 

\begin{theorybox}{The term \term{complement}}{complement}
    In \ac{cgel} the term \term{complement} 
    means the grammatical status of anything that originates from the argument structure of the main verb.
    Thus the subject, several kinds of objects,
    the copular complement (\ac{cgel} calls it \term{predicative complement}) are all complements,
    and they are also function labels 
    (roughly corresponding to ``SpecTP'' or ``what is in \vP{} 
    and receives the object case from a high light verb'').
    In traditional Latin grammar \term{complement} means the copular complement. 

    This note follows the terminology used in most descriptive grammars,
    so use the term \term{copular complement} to refer to the \ac{cgel} \term{predicative complement}.
    Also, the term \term{complement-taking verb}, despite being confusing,
    is used to denote a verb that take a complement clause as one of its arguments.

    The mapping from the argument structure of a verb 
    to clausal complement types is called alignment.
    This is not trivial: in generative terms,
    the argument structure is the structure of the ``canonical'' \vP containing a verb root,
    while the clausal complement types are largely decided by what happens after TP is finished.
    Lots of things can happen between the two.
\end{theorybox}

\subsubsection{Peripheral arguments}

There is no serial verb constructions in Latin (\prettyref{sec:clause-combine-abs}),
and thus semantic functions like location or instrument 
are always realized by typical peripheral arguments
attached to the core argument structure.
These peripheral argument positions sometimes can be filled by adverbs,
which also reveals an origin of adverbs: fossilized case forms.

\section{Clause combining}\label{sec:clause-combine-abs}

\begin{theorybox}{Types of coordination and subordination}{clause-combining}
    It's hard to draw a line between coordination and asymmetric (i.e. subordinating) clause linking 
    (like concessive clauses).
    Theoretically, this is because any clause combining construction follows the X-bar scheme:
    one clause is the Specifier, 
    and another clause is the Complement,
    and certain asymmetry has to be introduced.
    In English, the FANBOYS 
    -- \corpus{for}, \corpus{and}, \corpus{nor}, \corpus{but}, 
    \corpus{or}, \corpus{yet}, \corpus{so} -- are usually regarded as coordinating conjunctions.
    But what's the essential difference between \corpus{although} and \corpus{but}?

    On the other hand, adverbial clause constructions 
    are uncontroversially asymmetric and can in theory be distinguished from clause linking:
    in clause linking, the less important clause 
    is base-generated in one Specifier position in the CP layer of the main clause,
    so the two combined units are of roughly the same structure,
    while adverbial clauses appear in the TP layer,
    so the two combined units are of different structures:
    the adverbial clause is a CP,
    while the main clause, when the adverbial clause enters derivation,
    is a TP.
    Complement clauses, on the other hand, are first introduced in the \vP{} layer:
    they are TPs or CPs,
    while the main clause, when complement clauses enter the derivation,
    are \vP s.
    But there are still certain subtleties regarding the boundaries of \vP{}, TP, and CP.

    Relative clauses are introduced in DPs, 
    so the probability to confuse a relative clause construction 
    with a complement clause construction is small -- but still not zero.
    It can be expected that \corpus{I like the man dancing} and \corpus{I like the dancing man} 
    are realized in quite similar ways.
    Besides, some languages lack prototypical complement clause constructions 
    but have complementation strategies.
    That is, when they talk about \corpus{I like the dancing man},
    a speaker of such a language may be implying that he or she actually likes the man's dancing,
    though not the man's personality.
    Now comes the question:
    when there are vague evidences indicating the grammaticalization of this construction,
    should we now claim the language has already developed a complement clause construction?

    It's still possible to do the same thing 
    -- largely symmetric coordination and certainly asymmetric subordination -- 
    completely with \vP s.
    The former results in clause chaining \citep{nonato2014clause},
    while the latter results in serial verb constructions.
    These construction types, however, are absent in Latin, and I will not go deep into them in this note.
\end{theorybox}

In Latin there is no serial verb constructions.
Subordination strategies can be neatly summarized into 
complement clauses, relative clauses and adverbial clauses.

\section{Constituent order}\label{sec:constituent-order-abs}

\begin{theorybox}{Constituency deemphasized in Latin grammar}{constituency-in-grammar}
    The largely free constituent order 
    means description of Latin grammar is mostly dependency-relation based or \acs{blt}-based,
    because surface-based constituents other than \acs{np}s and clauses are hard to define.
    Still, generative (constituency-based, 
    though the introduction of movements and the structure of Cinque hierarchy
    gives it certain flavor of dependency grammars) approaches exist for Latin constituent order.
    There is evidence suggesting Latin is configurational, 
    i.e. has phrase structures \citep{danckaert2017development}.
    This is probably not surprising because
    even the most non-configurational languages show certain degree of configurationality 
    \citep[among others]{niedzielski2017clausal,morris2018evidence,legate2002warlpiri}.
    Then, \emph{how} non-configurational Latin is is a question needing addressing.
    Is it closer to a typical non-configurational language, say Warlpiri, 
    or is it closer to Japanese where we have more localized scrambling?
    I will address this question in TODO: ref ,
    though unfortunately, we still does not have a very clear answer.
\end{theorybox}

\chapter{Phonology and the writing system}

\chapter{Nominal inflection}

\section{Declension of regular nouns}\label{sec:regular-noun-declension}

\subsection{The paradigms}

\subsection{Distributions of case forms}

\subsubsection{The accusative}\label{sec:accusative-distribution}

\section{Declension of regular adjectives}

\section{Declension of pronouns}

\section{The gerund and participles}\label{sec:gerund-participle-morphology}

\subsection{The gerund}\label{sec:gerund-morphology}

Note that the nominative case is missing -- 
when a non-finite clause is required in the subject position,
it's always an infinitive.

\chapter{Nominal derivations}

\chapter{The structure of the noun phrase}

\section{Attributives}

This section only discusses adjective or numeral attributives in detail.
For in-depth discussion of relative clauses, see \prettyref{sec:relative-clause}.

\section{Arguments of adjectives}

TODO: case forms with adjectives

\section{The possessive construction}

\section{Numerals in the noun phrase}

\section{Prepositions}

\chapter{Verb inflection}

\section{The structure of Latin verbs}

The traditional analysis of the structure of a Latin verb 
can be roughly represented by \prettyref{fig:latin-verb}.
Derivation in Latin is predominantly preverbal,
and hence the conjugation is mostly about the final lexical morpheme in the verb stem.
The stem-final vowel -- also known as the \concept{thematic vowel} -- 
is sometimes considered as a part of the stem,
and sometimes as a part of the verb ending.
It is the residue of a \ac{pie} stem suffix, 
which is after the core stem and before the conjugation ending.
The uncontroversial components of the verb ending include 
the \concept{tense and mood marker},
and the person, number and voice marker,
which is called the \concept{personal ending} here, 
following the terminology in \citet[\citesec{165}]{allen1903allen}.

\begin{figure}[H]
    \centering
    

\tikzset{every picture/.style={line width=0.3pt}} %set default line width to 0.75pt        

\begin{tikzpicture}[x=0.75pt,y=0.75pt,yscale=-0.85,xscale=0.85]
%uncomment if require: \path (0,414); %set diagram left start at 0, and has height of 414

%Shape: Rectangle [id:dp7661505767193904] 
\draw  [color={rgb, 255:red, 74; green, 144; blue, 226 }  ,draw opacity=1 ] (200,155) -- (278.01,155) -- (278.01,201.48) -- (200,201.48) -- cycle ;

%Shape: Rectangle [id:dp9235304615883395] 
\draw  [color={rgb, 255:red, 74; green, 144; blue, 226 }  ,draw opacity=1 ] (51,155) -- (187.01,155) -- (187.01,201.48) -- (51,201.48) -- cycle ;

%Shape: Rectangle [id:dp4025769445021785] 
\draw  [color={rgb, 255:red, 74; green, 144; blue, 226 }  ,draw opacity=1 ] (290,155) -- (416.01,155) -- (416.01,201.48) -- (290,201.48) -- cycle ;

%Shape: Rectangle [id:dp9180138973853116] 
\draw  [color={rgb, 255:red, 80; green, 227; blue, 194 }  ,draw opacity=1 ] (428,155) -- (564.01,155) -- (564.01,201.48) -- (428,201.48) -- cycle ;

%Shape: Rectangle [id:dp2406875282163703] 
\draw  [color={rgb, 255:red, 80; green, 227; blue, 194 }  ,draw opacity=1 ] (575,155) -- (722.01,155) -- (722.01,201.48) -- (575,201.48) -- cycle ;


% Text Node
\draw (239.01,178.24) node   [align=left] {core stem};
% Text Node
\draw (119.01,178.24) node   [align=left] {\begin{minipage}[lt]{87.06pt}\setlength\topsep{0pt}
\begin{center}
derivation \\prefix/compouding
\end{center}

\end{minipage}};
% Text Node
\draw (353.01,178.24) node   [align=left] {stem-final vowel};
% Text Node
\draw (496.01,178.24) node   [align=left] {\begin{minipage}[lt]{83.46pt}\setlength\topsep{0pt}
\begin{center}
tense and mood \\marking
\end{center}

\end{minipage}};
% Text Node
\draw (648.51,178.24) node   [align=left] {\begin{minipage}[lt]{84.5pt}\setlength\topsep{0pt}
\begin{center}
person, number, \\and voice marking
\end{center}

\end{minipage}};
% Text Node
\draw (181,228) node [anchor=north west][inner sep=0.75pt]  [color={rgb, 255:red, 74; green, 144; blue, 226 }  ,opacity=1 ] [align=left] {verb stem};
% Text Node
\draw (532.01,228) node [anchor=north west][inner sep=0.75pt]  [color={rgb, 255:red, 80; green, 227; blue, 194 }  ,opacity=1 ] [align=left] {verb ending};


\end{tikzpicture}

    \caption{The template of Latin verbs}
    \label{fig:latin-verb}
\end{figure}

The core stem assumes semi-regular alternations,
marked by affixation and certain kind of .
It has morphosyntactic alternation 
and undergoes phonological ones in Latin. % TODO: ref
But the tense and mood marker is influenced by the personal ending:
the same tense and mood may be marked by one marker under one person and one number
but by another under another case.
The inverse is also true.
There is also phonological interaction between the two components of the ending.
TODO: explain the \corpus{-mus} and \corpus{-imus} alternation





\section{Formation of stems}

\subsection{The three verb stems}\label{sec:three-latin-stem}

\begin{theorybox}{About the concept of stem}{stem}
    Prototypically, the verb conjugation in a language is described by 
    a series of morphological devices that take \emph{the} verb stem as the input,
    and give conjugated verb forms as the final product.
    This is indeed the case for Latin nouns (\prettyref{sec:regular-noun-declension})
    and for English regular verbs:
    the infinitive form is taken in,
    and third-person singular \corpus{-s}, past tense \corpus{-ed}, 
    past participle \corpus{-ed}, and the gerund-participle \corpus{-ing}
    are attached according to the syntactic environment.
    Sometimes the process is a little more irregular but not that irregular:
    \emph{several} stems can be identified, each of which is fed into different morphosyntactic machines.
    In other words, we have irregular stem alternation.
    Again, for English irregular verbs,
    there are three stems: the infinitive stem (e.g. \corpus{go}), 
    the preterite stem (e.g. \corpus{went})
    and the past participle stem (e.g. \corpus{gone}).
    The step to feed stems into morphosyntactic machine is irregular,
    but everything else is regular:
    irregular, in this case, does appear, but it appear \emph{regularly}:
    it only appears in certain parts.

    This phenomenon -- that a verb has more than one stem, i.e. irregular stem alternation
    -- is frequent cross-linguistically
    (\citealt{jacques2021grammar} \citesec{12.2}, \citealt{forker2020grammar} \citesec{11.2}, among others).
    Usually, certain correlation between the stem varieties can still be recognized,
    and verbs can be grouped accordingly,
    which, if the linguist truly will, 
    can be (though tediously) summarized as more fine-grained conjugation classes.
    This is also the case for Latin.
    The irregularity of stem alternation is so prevalent
    that if the conjugation paradigm of a verb can be described with a few stems,
    the verb is deemed as regular, 
    despite the fact that such verbs are obviously irregular by the standard of English.

    The notion of \concept{stems} isn't really essential in the description of morphosyntactic:
    it can well be modeled by environment-dependent vocabulary insertion rules 
    and/or post-syntactic operations.
    When certain correlations can still be built between so-called suppletive forms,
    what happens may be analyzed as in \citet{embick2005status},
    where certain stems receive morphophonological readjustment
    (according to the aforementioned hyper fine-grained conjugation subclasses).
    Thus, it's not true suppletion:
    it's just a corner case of non-concatenative morphology.
    When these readjustment rules are fossilized,
    suppletion -- like the English \corpus{good}/\corpus{better} -- 
    may just be the result of conditional insertion,
    as is outlined in \citet{siddiqi2009syntax}.

    A general tendency about suppletion
    is truly suppletive verbs are usually light verbs 
    (in the surface-oriented sense),
    with meanings like \translate{do}, \translate{come}, etc.
    This may come from the fact that conditional realization of the root 
    is somehow ``heavy''.
    When we do away with this,
    e.e. under \citet{embick2005status}'s assumption,
    unrestricted suppletion can only be the result of 
    vocabulary insertion rules of functional heads;
    certain degree of suppletion can still be realized by readjustment rules,
    which however are restricted in their computational capacity.
    Thus, real lexical verbs are highly unlikely to have truly irregular suppletion;
    if a verb is truly suppletive,
    then it's highly likely to be 
    the spellout of \vP{} functional heads.
\end{theorybox}

Latin shows prevalent and not completely predictable stem alternation.
All forms mentioned in \prettyref{sec:verb-inflection-abs}
can be obtained by three stems \citep[\citesec{164}]{allen1903allen},
if the verb is regular:
\begin{itemize}
    \item The \concept{present stem}, which, after attached with proper endings, forms
    \begin{itemize}
        \item The \acl{present}, \acl{imperfect}, and future forms, indicative or subjunctive,
        active or passive. (There is no future or future perfect subjunctive).
        \item All the imperatives.
        \item The present infinitives, active and passive.
        \item The present participle, the gerundive, and the gerund.
    \end{itemize}
    \item The \concept{perfect stem}, which, after attached with proper endings, forms 
    \begin{itemize}
        \item The perfect, pluperfect, and future perfect active, indicative or subjunctive.
        Again, there is no future or future perfect subjunctive.
        Note that the passives are \emph{not} formed by the perfect stem.
        \item The perfect active infinitive. 
        (Or the perfective infinitive active, since infinitive is considered as a mood by some people.)
    \end{itemize}
    Note that the perfect passive participle is \emph{not} obtained from the perfect stem.
    \item The \concept{supine stem}, 
    which, after attached with proper endings or used together with proper forms of \corpus{sum},
    forms 
    \begin{itemize}
        \item The perfect passive participle, which, by being used with proper forms of \corpus{sum}, forms
        \begin{itemize}
            \item The perfect, pluperfect, and future perfect passive forms, indicative or subjunctive.
            Again, there is no future or future perfect subjunctive.
            This is periphrastic conjugation: it is done by using proper forms of \corpus{sum}
            with the perfect passive participle.
            \item The perfect infinitive passive.
        \end{itemize}
        \item The future active participle, which, used together with \corpus{esse},
        makes the future active infinitive.
        \item The future passive infinitive, by being used together with \corpus{īrī}.
    \end{itemize}
\end{itemize}
This process is summarized in \prettyref{fig:stem-to-form}.

Note that in Medieval Latin, often,
instead of \corpus{iri} plus the first supine,
\corpus{fore} plus the perfect participle is used to form the future passive infinitive.
TODO: find a reference https://www.nationalarchives.gov.uk/latin/stage-2-latin/lessons/lesson-24-infinitives-accusative-and-infinitive-clause/

\begin{sidewaysfigure}
    \centering
    

\tikzset{every picture/.style={line width=0.3pt}} %set default line width to 0.75pt        

\begin{tikzpicture}[x=0.75pt,y=0.75pt,yscale=-0.8,xscale=0.8]
%uncomment if require: \path (0,697); %set diagram left start at 0, and has height of 697

%Curve Lines [id:da8600925548352094] 
\draw [color={rgb, 255:red, 208; green, 2; blue, 27 }  ,draw opacity=1 ]   (289.01,269.33) .. controls (329.01,239.33) and (422.01,193.33) .. (676.01,187.33) ;
\draw [shift={(676.01,187.33)}, rotate = 178.65] [fill={rgb, 255:red, 208; green, 2; blue, 27 }  ,fill opacity=1 ][line width=0.08]  [draw opacity=0] (12,-3) -- (0,0) -- (12,3) -- cycle    ;
%Curve Lines [id:da7478415205524331] 
\draw [color={rgb, 255:red, 208; green, 2; blue, 27 }  ,draw opacity=1 ]   (275.01,266.33) .. controls (256.2,201.98) and (244.25,196.43) .. (205.2,166.25) ;
\draw [shift={(204.01,165.33)}, rotate = 37.78] [fill={rgb, 255:red, 208; green, 2; blue, 27 }  ,fill opacity=1 ][line width=0.08]  [draw opacity=0] (12,-3) -- (0,0) -- (12,3) -- cycle    ;
%Curve Lines [id:da22329313492102099] 
\draw [color={rgb, 255:red, 208; green, 2; blue, 27 }  ,draw opacity=1 ]   (241.01,279.33) .. controls (203.2,213.66) and (164.4,199.47) .. (101.95,158.94) ;
\draw [shift={(101.01,158.33)}, rotate = 33.06] [fill={rgb, 255:red, 208; green, 2; blue, 27 }  ,fill opacity=1 ][line width=0.08]  [draw opacity=0] (12,-3) -- (0,0) -- (12,3) -- cycle    ;
%Curve Lines [id:da9720425555739067] 
\draw [color={rgb, 255:red, 208; green, 2; blue, 27 }  ,draw opacity=1 ]   (250.01,319.22) .. controls (250.01,411.38) and (358.91,534.94) .. (466.39,595.28) ;
\draw [shift={(468.01,596.18)}, rotate = 209.05] [fill={rgb, 255:red, 208; green, 2; blue, 27 }  ,fill opacity=1 ][line width=0.08]  [draw opacity=0] (12,-3) -- (0,0) -- (12,3) -- cycle    ;
%Curve Lines [id:da9116874614549022] 
\draw [color={rgb, 255:red, 208; green, 2; blue, 27 }  ,draw opacity=1 ]   (265.01,312.33) .. controls (276.01,374.96) and (344.01,487.96) .. (487.01,511.96) ;
\draw [shift={(487.01,511.96)}, rotate = 189.53] [fill={rgb, 255:red, 208; green, 2; blue, 27 }  ,fill opacity=1 ][line width=0.08]  [draw opacity=0] (12,-3) -- (0,0) -- (12,3) -- cycle    ;
%Curve Lines [id:da9399420006349646] 
\draw [color={rgb, 255:red, 208; green, 2; blue, 27 }  ,draw opacity=1 ]   (277.01,310.33) .. controls (320.79,382.6) and (345.76,424.54) .. (456.34,434.81) ;
\draw [shift={(458.01,434.96)}, rotate = 185.1] [fill={rgb, 255:red, 208; green, 2; blue, 27 }  ,fill opacity=1 ][line width=0.08]  [draw opacity=0] (12,-3) -- (0,0) -- (12,3) -- cycle    ;
%Curve Lines [id:da863359914759456] 
\draw [color={rgb, 255:red, 248; green, 231; blue, 28 }  ,draw opacity=1 ]   (404.01,268.33) .. controls (396.09,243.58) and (393.07,211.97) .. (403.69,164.76) ;
\draw [shift={(404.01,163.33)}, rotate = 102.91] [fill={rgb, 255:red, 248; green, 231; blue, 28 }  ,fill opacity=1 ][line width=0.08]  [draw opacity=0] (12,-3) -- (0,0) -- (12,3) -- cycle    ;
%Curve Lines [id:da44942461080998] 
\draw [color={rgb, 255:red, 248; green, 231; blue, 28 }  ,draw opacity=1 ]   (418.01,269.33) .. controls (457.81,239.48) and (590.67,220.74) .. (688.54,241.24) ;
\draw [shift={(690.01,241.55)}, rotate = 192.09] [fill={rgb, 255:red, 248; green, 231; blue, 28 }  ,fill opacity=1 ][line width=0.08]  [draw opacity=0] (12,-3) -- (0,0) -- (12,3) -- cycle    ;
%Curve Lines [id:da6857007576480554] 
\draw [color={rgb, 255:red, 245; green, 166; blue, 35 }  ,draw opacity=1 ]   (543.01,333) .. controls (619.63,368.19) and (641.79,399.53) .. (690.28,478.65) ;
\draw [shift={(691.01,479.85)}, rotate = 238.51] [fill={rgb, 255:red, 245; green, 166; blue, 35 }  ,fill opacity=1 ][line width=0.08]  [draw opacity=0] (12,-3) -- (0,0) -- (12,3) -- cycle    ;
%Curve Lines [id:da10519824034130609] 
\draw [color={rgb, 255:red, 245; green, 166; blue, 35 }  ,draw opacity=1 ]   (548.01,313.33) .. controls (610.7,314.32) and (660.51,321.26) .. (718.14,390.28) ;
\draw [shift={(719.01,391.33)}, rotate = 230.36] [fill={rgb, 255:red, 245; green, 166; blue, 35 }  ,fill opacity=1 ][line width=0.08]  [draw opacity=0] (12,-3) -- (0,0) -- (12,3) -- cycle    ;
%Curve Lines [id:da05946917136382068] 
\draw [color={rgb, 255:red, 245; green, 166; blue, 35 }  ,draw opacity=1 ]   (764.01,427.33) .. controls (931.01,427.33) and (930.01,58.33) .. (587.01,121.33) ;
\draw [shift={(587.01,121.33)}, rotate = 349.59] [fill={rgb, 255:red, 245; green, 166; blue, 35 }  ,fill opacity=1 ][line width=0.08]  [draw opacity=0] (12,-3) -- (0,0) -- (12,3) -- cycle    ;
%Curve Lines [id:da40444377743555227] 
\draw [color={rgb, 255:red, 245; green, 166; blue, 35 }  ,draw opacity=1 ]   (760.01,418.33) .. controls (782.67,400.12) and (788.83,381.88) .. (787.1,343.11) ;
\draw [shift={(787.01,341.33)}, rotate = 87.14] [fill={rgb, 255:red, 245; green, 166; blue, 35 }  ,fill opacity=1 ][line width=0.08]  [draw opacity=0] (12,-3) -- (0,0) -- (12,3) -- cycle    ;
%Curve Lines [id:da6236126078150841] 
\draw [color={rgb, 255:red, 245; green, 166; blue, 35 }  ,draw opacity=1 ]   (776.01,515.33) .. controls (822.31,499.24) and (848.23,480.56) .. (870.97,451.34) ;
\draw [shift={(872.01,450)}, rotate = 127.48] [fill={rgb, 255:red, 245; green, 166; blue, 35 }  ,fill opacity=1 ][line width=0.08]  [draw opacity=0] (12,-3) -- (0,0) -- (12,3) -- cycle    ;
%Curve Lines [id:da6624205121022633] 
\draw [color={rgb, 255:red, 245; green, 166; blue, 35 }  ,draw opacity=1 ]   (526.01,337) .. controls (548.9,359.55) and (593.56,477.99) .. (600.9,583.59) ;
\draw [shift={(601.01,585.18)}, rotate = 266.22] [fill={rgb, 255:red, 245; green, 166; blue, 35 }  ,fill opacity=1 ][line width=0.08]  [draw opacity=0] (12,-3) -- (0,0) -- (12,3) -- cycle    ;
%Curve Lines [id:da8468021797759513] 
\draw [color={rgb, 255:red, 245; green, 166; blue, 35 }  ,draw opacity=1 ]   (630.01,594.18) .. controls (720.56,591.2) and (773.48,577.14) .. (847.89,560.25) ;
\draw [shift={(849.01,560)}, rotate = 167.23] [fill={rgb, 255:red, 245; green, 166; blue, 35 }  ,fill opacity=1 ][line width=0.08]  [draw opacity=0] (12,-3) -- (0,0) -- (12,3) -- cycle    ;
%Shape: Ellipse [id:dp3626499664001579] 
\draw  [color={rgb, 255:red, 74; green, 144; blue, 226 }  ,draw opacity=1 ][fill={rgb, 255:red, 74; green, 144; blue, 226 }  ,fill opacity=0.1 ] (39,126.52) .. controls (39,83.15) and (167.94,48) .. (327.01,48) .. controls (486.07,48) and (615.01,83.15) .. (615.01,126.52) .. controls (615.01,169.88) and (486.07,205.03) .. (327.01,205.03) .. controls (167.94,205.03) and (39,169.88) .. (39,126.52) -- cycle ;
%Curve Lines [id:da19915198987505445] 
\draw [color={rgb, 255:red, 80; green, 227; blue, 194 }  ,draw opacity=1 ][fill={rgb, 255:red, 80; green, 227; blue, 194 }  ,fill opacity=0.2 ]   (435.01,413.18) .. controls (475.01,383.18) and (680.01,340.18) .. (752.01,382.18) .. controls (824.01,424.18) and (801.01,553.18) .. (717.01,571.18) .. controls (633.01,589.18) and (657.01,503.18) .. (615.01,482.18) .. controls (573.01,461.18) and (401.02,474.22) .. (435.01,413.18) -- cycle ;
%Shape: Ellipse [id:dp7283210175576866] 
\draw  [color={rgb, 255:red, 126; green, 211; blue, 33 }  ,draw opacity=1 ][fill={rgb, 255:red, 126; green, 211; blue, 33 }  ,fill opacity=0.1 ] (423,565.16) .. controls (423,524.84) and (473.15,492.14) .. (535.01,492.14) .. controls (596.86,492.14) and (647.01,524.84) .. (647.01,565.16) .. controls (647.01,605.49) and (596.86,638.18) .. (535.01,638.18) .. controls (473.15,638.18) and (423,605.49) .. (423,565.16) -- cycle ;
%Curve Lines [id:da8675081082501883] 
\draw [color={rgb, 255:red, 80; green, 227; blue, 194 }  ,draw opacity=1 ][fill={rgb, 255:red, 80; green, 227; blue, 194 }  ,fill opacity=0.1 ]   (363.01,378.55) .. controls (387.01,323.55) and (691.01,321.55) .. (763.01,363.55) .. controls (835.01,405.55) and (805.01,594.55) .. (721.01,612.55) .. controls (637.01,630.55) and (649.01,514.55) .. (540.01,525.55) .. controls (431.01,536.55) and (329.02,439.59) .. (363.01,378.55) -- cycle ;
%Shape: Polygon Curved [id:ds050164225991979006] 
\draw  [color={rgb, 255:red, 184; green, 233; blue, 134 }  ,draw opacity=1 ][fill={rgb, 255:red, 184; green, 233; blue, 134 }  ,fill opacity=0.1 ] (683.01,164.55) .. controls (770.01,100.55) and (871.01,192.55) .. (904.01,306.55) .. controls (937.01,420.55) and (940.76,404.18) .. (965.01,472.55) .. controls (989.26,540.93) and (945.01,634.55) .. (872.01,627.55) .. controls (799.01,620.55) and (851.01,517.55) .. (838.01,459.55) .. controls (825.01,401.55) and (798.01,367.55) .. (738.01,329.55) .. controls (678.01,291.55) and (596.01,228.55) .. (683.01,164.55) -- cycle ;

% Text Node
\draw (489,83) node [anchor=north west][inner sep=0.75pt]   [align=left] {perfect,\\pluperfect,\\future perfect\\passive};
% Text Node
\draw (359,74) node [anchor=north west][inner sep=0.75pt]   [align=left] {perfect,\\pluperfect,\\future perfect\\active};
% Text Node
\draw (175,100) node [anchor=north west][inner sep=0.75pt]   [align=left] {present,\\imperfect,\\future};
% Text Node
\draw (385,269.07) node [anchor=north west][inner sep=0.75pt]  [color={rgb, 255:red, 0; green, 0; blue, 0 }  ,opacity=1 ] [align=left] {perfect\\stem};
% Text Node
\draw (248,268.07) node [anchor=north west][inner sep=0.75pt]  [color={rgb, 255:red, 0; green, 0; blue, 0 }  ,opacity=1 ] [align=left] {present\\stem};
% Text Node
\draw (470,587.07) node [anchor=north west][inner sep=0.75pt]   [align=left] {gerund};
% Text Node
\draw (496,501.07) node [anchor=north west][inner sep=0.75pt]   [align=left] {gerundive};
% Text Node
\draw (59,130) node [anchor=north west][inner sep=0.75pt]   [align=left] {imperative};
% Text Node
\draw (688,166) node [anchor=north west][inner sep=0.75pt]   [align=left] {present\\infinitives};
% Text Node
\draw (467,412) node [anchor=north west][inner sep=0.75pt]   [align=left] {present\\participle};
% Text Node
\draw (501,292.07) node [anchor=north west][inner sep=0.75pt]   [align=left] {supine\\stem};
% Text Node
\draw (696,218) node [anchor=north west][inner sep=0.75pt]   [align=left] {perfect\\active\\infinitive};
% Text Node
\draw (579,586.07) node [anchor=north west][inner sep=0.75pt]   [align=left] {supine};
% Text Node
\draw (697,395) node [anchor=north west][inner sep=0.75pt]   [align=left] {perfect\\passive\\participle};
% Text Node
\draw (765,278) node [anchor=north west][inner sep=0.75pt]   [align=left] {perfect\\passive\\infinitive};
% Text Node
\draw (695,482) node [anchor=north west][inner sep=0.75pt]   [align=left] {future\\active\\participle};
% Text Node
\draw (872,382) node [anchor=north west][inner sep=0.75pt]   [align=left] {future\\active\\infinitive};
% Text Node
\draw (864,521) node [anchor=north west][inner sep=0.75pt]   [align=left] {future\\passive\\infinitive};
% Text Node
\draw (200,60) node [anchor=north west][inner sep=0.75pt]  [color={rgb, 255:red, 74; green, 144; blue, 226 }  ,opacity=1 ] [align=left] {finite forms};
% Text Node
\draw (594,405.14) node [anchor=north west][inner sep=0.75pt]  [color={rgb, 255:red, 80; green, 227; blue, 194 }  ,opacity=1 ] [align=left] {particle\\in narrow\\sense};
% Text Node
\draw (519,532.51) node [anchor=north west][inner sep=0.75pt]  [color={rgb, 255:red, 126; green, 211; blue, 33 }  ,opacity=1 ] [align=left] {"nominal"\\nonfinite\\forms};
% Text Node
\draw (389,356.14) node [anchor=north west][inner sep=0.75pt]  [color={rgb, 255:red, 80; green, 227; blue, 194 }  ,opacity=1 ] [align=left] {particle in broad sense};
% Text Node
\draw (775,228) node [anchor=north west][inner sep=0.75pt]  [color={rgb, 255:red, 184; green, 233; blue, 134 }  ,opacity=1 ] [align=left] {infinitives};


\end{tikzpicture}

    \caption{How to get all conjugation forms from the three stems}
    \label{fig:stem-to-form}
\end{sidewaysfigure}

\subsection{Rules of stem alternation}

\section{The finite paradigm}\label{sec:finite-paradigm}

\subsection{Tense and mood}\label{sec:tense-mood-marking}

\subsection{The personal ending}

\subsection{Periphrastic conjugations}

\section{Non-finite forms}

\subsection{The infinitives}\label{sec:infinitives}

\subsection{The gerund and participles}\label{sec:nominal-form}

The morphology of the gerund and participles are described in \prettyref{sec:gerund-participle-morphology}.
This section is about their meanings and distributions.

\section{Deponent verbs}\label{sec:deponent-verbs}

\begin{theorybox}{Distributed Morphology of deponent verbs}{deponent}
    \citet{embick2000features} analyzes deponent verbs as roots carrying a passive feature themselves.
    When functional heads higher than the roots are realized,
    the passive feature -- which may come from the root or from the passive light verb -- 
    guides the realization of the person and number categories.
\end{theorybox}



\section{Irregular verbs}\label{sec:irregular-verbs}

\subsection{The verb \corpus{sum}}\label{sec:sum-morphology}

The verb \corpus{sum} has lots of uses in Latin grammar (\prettyref{sec:sum}).

\chapter{Verb derivations}

\section{Preverbs}

\chapter{Arguments and verb valency classes}

\section{Overview of marking of core arguments}\label{sec:core-argument-marking}

This section gives an overview of the statuses of arguments as clausal dependents,
or in other words, how they're marked in the clause.
Testing standards of types of clausal dependents include 
their correspondence with semantic roles in verb frames,
what's their cases, 
agreement, 
their behaviors in passivization, and
what constructions are able to fill them. 

\begin{infobox}{How complement types are discussed}{complement-discussion}
    The traditional practice of Latin grammar research
    is to classify clausal complement and adjunct types according to their case marking.
    This strategy is also found in modern grammars.
    Some introduce clausal complement types just in chapters about case marking 
    \citep[\citechap{8}]{jacques2021grammar},
    while other grammars, despite giving a brief description of the context of case marking,
    spare some time to discuss complement types in the chapters about valency and clause structure 
    \citep[\citesec{3.4}, \citechap{19}, \citechap{22}]{forker2020grammar}.
    From a \ac{tag} perspective, 
    the two extremes are different in how they treat function labels:
    in the former, the function label of a construction appears together with its category label on the root node,
    while in the latter, the function is described separately from the form of what fills that position.
    The former is more bottom-up, 
    while the latter is more top-down.
    The choice between the two, however, is usually language-dependent:
    grammars for analytic languages, of course, have to lean even further to the 
    ``complement type as clause slot'' extreme 
    and away from the ``complement type as case-form context'' extreme.
    \citet{allen1903allen} uses a hybrid method:
    the discussion about case marking (\citesec{39}) is separated from 
    the discussion about complement and adjunct types (\citesec{338}),
    so the top-down approach seems to be adopted,
    but the latter is still arranged in terms of case.
    This arises both from the distinct features of Latin and the intended readers:
    the relation between complement types and cases is regular enough in Latin,
    and what is most important for Latinists is to understand, at least sketchily, ancient writings, 
    so a parsing-oriented grammar is much handier.
\end{infobox}

Allowed combinations of clausal dependents are determined 
by the valency class of the verb and valency changing devices.
The valency classes are discussed in the rest of this chapter.
I will generally follow the classification in \citet[\citechap{4}]{Pinkster1};
some verb classes enumerated by Pinkster are discussed in more details 
in \prettyref{chap:complement-clause-construct},
since a full account of their behaviors is closely related to the structure of complement clauses.

\begin{infobox}{Typology of verb valency classes}{valency-class}
    To find a summary of frequent verb valency classes,
    see \citet[Part B]{dixon2005semantic},
    \citet[\citesec{18.5}]{dixon2010basic2},
    and \citet[\citesec{3.3}]{dixon2009basic1}.
    Dixon classifies the verb class into three subgroups:
    the first is Primary-A, which contains verbs that don't take complement clauses;
    the second is Primary-B, which contains complement-taking lexical verbs,
    with meanings more complicated then what can be expressed by the grammar;
    the third is Secondary, members of which have the same \emph{meaning} 
    of certain constructions in the \vP{}-TP-CP projections,
    but possibly not the same syntactic properties
    (for example, they may just take complement clauses instead of being auxiliary verbs).

    Dixon gives mutually exclusive tags for semantic roles 
    between two semantic classes of verbs.
    For a more unified terminology on semantic roles,
    see \citet[\citesec{4.2}]{cgel}.
\end{infobox}

This chapter is mainly about verbs that don't take complement clauses as arguments.
The phenomena discussed in this chapter mostly apply to complement clause constructions as well,
but complement clause constructions have their own peculiarities 
(\prettyref{sec:complement-clause-construct-overview}).

\subsection{The subject}

Latin is an accusative language.
A \concept{subject} can be identified for all clauses, 
though it is frequently omitted.
Grammatical behaviors of the subject are summarized in the following list: 
\begin{itemize}
    \item \emph{Coding of semantic role}: In an active clause, 
    the subject is always the most agentive argument,
    i.e. the S argument in a prototypical intransitive argument structure 
    and the A argument in a prototypical transitive argument structure.
    In a passive clause, 
    the subject corresponds to the ``promoted argument'' (\prettyref{sec:passive}).  
    \item \emph{Case marking}: 
    Subjects are always nominative for finite clauses,
    whenever the case system is in action,
    i.e. whenever the subject is an \ac{np} or a gerund. 
    Nonfinite clauses may be argued to be subjectless in the surface form 
    (a reasonable claim, since they have deficient TP layers, 
    and hence it is possible that no canonical subject position exists),
    but in accusative-infinitive constructions, % TODO: control or ECM or ... ?
    the accusative may be seen as the non-canonical subject of the infinitive.
    \item \emph{Agreement}: 
    the number and person features on the subject leave marking on the verb complex.
    Latin does not have verbal agreement with arguments other than the agreement with the subject.
    \item \emph{Category}: a subject is an \ac{np}  
    or a complement clause (\prettyref{sec:complement-clause-construct-overview}), 
    usually an infinitive but never a gerund (\prettyref{sec:gerund-morphology}).
    This constraint isn't seen in any other clausal complement types.
\end{itemize}


\subsection{The direct object}

Here is a list of grammatical properties of the direct object:
\begin{itemize}
    \item \emph{Coding of semantic role}: In a prototypical transitive argument structure, 
    the direct object is the O argument, i.e. the most patientive argument. 

    \item \emph{Case marking}: Direct objects are always accusative when it makes sense to talk about case -- 
    but not all accusative arguments are direct objects (\prettyref{sec:accusative-distribution}).
    \item \emph{Passivization}: If an argument is coded as the direct object, 
    then it can regularly be promoted to the subject position in a passive clause (\prettyref{sec:passive}). 
    Secondary objects are less frequently promoted in passivization (\prettyref{sec:accusative-distribution}).
\end{itemize}

\subsection{The indirect object and the secondary object}\label{sec:indirect-object}

Latin also has two complement positions named as object:
the indirect object and the secondary object.
The indirect object is distinguished by the following grammatical properties:
\begin{itemize}
    \item \emph{Coding of semantic role}: in a AGT-type argument structure, 
    the indirect object is usually the G argument.
    Intransitive clauses sometimes also have indirect objects, 
    and an indirect object, in this case, is also a G argument.
    \item \emph{Case marking}: indirect objects are always dative.
    \item \emph{Passivization}: indirect objects are always retained in passive clauses. 
    They are never promoted to subjects in passivization.
    % TODO: category
\end{itemize}

The secondary object is distinguished by the following grammatical properties:
\begin{itemize}
    \item \emph{Coding of semantic role}: in an AGT-type argument structure
    that is always about information flowing,
    the T argument (i.e. the thing asked about or taught about) is the secondary object.
    The G argument (i.e. the person who is asked or taught) is the direct object.
    Sometimes the G argument is ablative, and in this case, 
    there is only one accusative argument: the secondary object.
    Another place where secondary objects appear is 
    clauses headed by a verb with a compounded accusative preposition. % TODO: SAO typology
    \item \emph{Case marking}: secondary objects are always accusative.
    \item \emph{Passivization}: secondary objects can be passivized, but much more rarely than direct objects.
    \item % TODO: category
\end{itemize}

The distributions of the secondary object and the indirect object 
are mutually exclusive.
This means for ditransitive verbs of type \classify{giving}, 
Latin shows a clear and strong tendency to identify the T argument with the monotransitive O,
while for ditransitive verbs about teaching,
the inverse is true.

\begin{infobox}{Comparison with English}{english-indirect-object}
    It can be found that the Latin indirect object has more similarity with the English \corpus{to}-PP,
    which is also called the indirect object in some grammars, but not \ac{cgel}.
    The Latin indirect object differs from the English (accusative) indirect object in passivization.
    Since in Latin, verbs with AGT-type argument structure do not have alternation of complementation pattern
    -- in English we have \corpus{give sth. to sb.} and \corpus{give sb. sth.}, 
    while in Latin there is only the former one, but \corpus{to sb.} is replaced by a dative,
    and in Latin datives never have prepositions --
    the G argument is identified with the E argument,
    and the T argument is identified with the O argument.
    In other words, in Latin, there is only pattern \eqref{cgel-ex:john-gave-goods-to-charity} in \cgel.
    
    The secondary object, therefore, is like \eqref{cgel-ex:john-gave-student-book} in \cgel.
    Therefore, for typical ditransitive verbs, i.e. verbs like \corpus{give}, 
    Latin shows a clear and strong tendency to identify the T argument with the monotransitive O,
    which is more typical than English%
    \footnote{
        In English, in the \corpus{give sb. sth.} construction, it is the person i.e. the G argument that is passivized,
        while the T argument i.e. \corpus{sth.} cannot, though the latter is identified with monotransitive O
        according to other criteria. 
    },
    but for verbs with meaning of \corpus{teach} or \corpus{ask},
    there is also a clear and strong tendency to identify the G argument with the monotransitive O.
\end{infobox}

\subsection{Copular complements}

Latin also has copular complements.
A copular complement, just like its counterpart in English,
basically can be viewed as a displaced attributive or appositive 
(and hence is prototypically filled by an \ac{np} or an \acs{adjp})
but is a little more peripheral (manner, state, factitive, etc.) 
in its meaning than an attributive or appositive.
Latin has nominative predicate and accusative predicate:
as hinted by their names, 
the nominative predicate is roughly about the subject and agrees with it,
and the accusative predicate is roughly about the direct object and agrees with it.
In passivization of the direct object,
the accusative predicate becomes the nominative predicate.


\section{Peripheral arguments}

Beside the subject, various types of objects, and copular complements,
there are more peripheral clause dependents
(or ``adjuncts'' in the terminology of \citet{cgel}),
like purpose, direction, location, etc.
Besides clauses and \ac{np}s (with or without prepositions),
their categories also include \ac{advp}s,
which is an etymological origin of \ac{advp}s.

Latin peripheral arguments do not necessarily have prepositions.
This fact, together with the fact that Latin is highly free-ordered, 
highly \term{pro}-drop,
means telling core arguments from peripheral arguments difficult.
The criteria of category, position, and argumenthood in \citet[\citesec{4.1.2}]{cgel} 
all fail to work,
and since the problem is how to tell adjuncts from adjunct-like complements,
the criterion about role also fails.
Latin doesn't have systematic way to replace the core predicate (i.e. without adjuncts) by an anaphora,
and that criterion does not work, either.
The remaining criteria are about selection, licensing, and obligatoriness.

\subsection{}

\section{Prototypical intransitive verbs}\label{sec:prototypical-intransitive}

\subsection{The \classify{motion} type}


\section{Prototypical transitive verbs}\label{sec:prototypical-transitive}

With complement-taking verbs temporarily excluded,
a prototypical transitive verb is more or less close to the \classify{affect} type,
with an A argument which is the causer of the event 

\section{Copular verbs}

\subsection{The verb \corpus{sum}}\label{sec:sum}

\chapter{Valency changing}

\section{The passive}\label{sec:passive}

\section{Preverbs}

preverb

\chapter{Clause structure}

\section{Constituent order and the information structure}

\chapter{Clause linking and conjunctions}



\chapter{Complement clause constructions}\label{chap:complement-clause-construct}

\section{Overview}\label{sec:complement-clause-construct-overview}

\chapter{Relative constructions}\label{sec:relative-clause}

\chapter{Examples of texts}

\bibliographystyle{plainnat}
\bibliography{latin,theory}

\end{document}