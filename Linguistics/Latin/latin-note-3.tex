\documentclass[UTF8, a4paper, oneside]{report}

\usepackage{geometry}
\usepackage{float}
\usepackage{titling}
\usepackage{titlesec}
\usepackage{paralist}
\usepackage{footnote}
\usepackage{enumerate}
\usepackage{amsmath, amssymb, amsthm}
\usepackage{gb4e}
\noautomath
\usepackage{bbm}
\usepackage{soul}
\usepackage{graphicx}
\usepackage{siunitx}
\usepackage[table,xcdraw]{xcolor}
\usepackage{tikz}
\usepackage[ruled, vlined, linesnumbered, noend]{algorithm2e}
\usepackage{xr-hyper}
\usepackage[colorlinks]{hyperref} % linkcolor=black, anchorcolor=black, citecolor=black, filecolor=black
\usepackage[most]{tcolorbox}
\usepackage{caption}
\usepackage{subcaption}
\usepackage{booktabs}
\usepackage{multirow}
\usepackage[figuresright]{rotating}
\usepackage{acro}
\usepackage[round]{natbib} 
\usepackage{nameref,zref-xr}
\zxrsetup{toltxlabel}
\zexternaldocument*[cgel-]{../English/cambridge}[cambridge.pdf]
\zexternaldocument*[alignment-]{../alignment/alignment}[alignment.pdf]
\zexternaldocument*[exercise1-]{../Exercise/2021-3}[2021-3.pdf]
\zexternaldocument*[general-]{../methodology/glossing}[glossing.pdf]
\usepackage{prettyref}

\geometry{left=3.18cm,right=3.18cm,top=2.54cm,bottom=2.54cm}
\titlespacing{\paragraph}{0pt}{1pt}{10pt}[20pt]
\setlength{\droptitle}{-5em}

\DeclareMathOperator{\timeorder}{\mathcal{T}}
\DeclareMathOperator{\diag}{diag}
\DeclareMathOperator{\legpoly}{P}
\DeclareMathOperator{\primevalue}{P}
\DeclareMathOperator{\sgn}{sgn}
\newcommand*{\ii}{\mathrm{i}}
\newcommand*{\ee}{\mathrm{e}}
\newcommand*{\const}{\mathrm{const}}
\newcommand*{\suchthat}{\quad \text{s.t.} \quad}
\newcommand*{\argmin}{\arg\min}
\newcommand*{\argmax}{\arg\max}
\newcommand*{\normalorder}[1]{: #1 :}
\newcommand*{\pair}[1]{\langle #1 \rangle}
\newcommand*{\fd}[1]{\mathcal{D} #1}

\newcommand*{\citesec}[1]{\S~{#1}}
\newcommand*{\citechap}[1]{chap.~{#1}}
\newcommand*{\citefig}[1]{Fig.~{#1}}
\newcommand*{\citetable}[1]{Table~{#1}}
\newcommand*{\citepage}[1]{pp.~{#1}}
\newcommand*{\citefootnote}[1]{fn.~{#1}}

\newrefformat{sec}{\citesec{\ref{#1}}}
\newrefformat{fig}{\citefig{\ref{#1}}}
\newrefformat{tbl}{\citetable{\ref{#1}}}
\newrefformat{chap}{\citechap{\ref{#1}}}
\newrefformat{fn}{\citefootnote{\ref{#1}}}
\newrefformat{box}{Box~\ref{#1}}

\usetikzlibrary{arrows,shapes,positioning}
\usetikzlibrary{arrows.meta}
\usetikzlibrary{decorations.markings}
\tikzstyle arrowstyle=[scale=1]
\tikzstyle directed=[postaction={decorate,decoration={markings,
    mark=at position .5 with {\arrow[arrowstyle]{stealth}}}}]
\tikzstyle ray=[directed, thick]
\tikzstyle dot=[anchor=base,fill,circle,inner sep=1pt]


\tcbuselibrary{skins, breakable, theorems}

\newtcbtheorem[number within=chapter]{infobox}{Box}{
    enhanced,
    boxrule=0pt,
    colback=blue!5,
    colframe=blue!5,
    coltitle=blue!50,
    borderline west={4pt}{0pt}{blue!65},
    sharp corners,
    fonttitle=\bfseries, 
    breakable,
    before upper={\parindent15pt\noindent}}{box}
\newtcbtheorem[number within=chapter, use counter from=infobox]{theorybox}{Box}{
    enhanced,
    boxrule=0pt,
    colback=orange!5, 
    colframe=orange!5, 
    coltitle=orange!50,
    borderline west={4pt}{0pt}{orange!65},
    sharp corners,
    fonttitle=\bfseries, 
    breakable,
    before upper={\parindent15pt\noindent}}{box}
\newtcbtheorem[number within=chapter, use counter from=infobox]{learnbox}{Box}{
    enhanced,
    boxrule=0pt,
    colback=green!5,
    colframe=green!5,
    coltitle=green!50,
    borderline west={4pt}{0pt}{green!65},
    sharp corners,
    fonttitle=\bfseries, 
    breakable,
    before upper={\parindent15pt\noindent}}{box}

\newcommand*{\concept}[1]{\textbf{#1}}
\newcommand*{\term}[1]{\emph{#1}}
\newcommand{\corpus}[1]{\emph{#1}}

%region Acronym

% Theory
\DeclareAcronym{blt}{short = BLT, long = Basic Linguistic Theory}
\DeclareAcronym{cgel}{short = CGEL, long = The Cambridge Grammar of the English Language}
\DeclareAcronym{dm}{short = DM, long = Distributed Morphology}
\DeclareAcronym{tag}{long = Tree-adjoining grammar, short = TAG}

% History

\DeclareAcronym{pie}{long = proto-Indo-European, short = PIE}

% Roles

\DeclareAcronym{sfp}{long = sentence final particle, short = SFP}
\DeclareAcronym{np}{long = noun phrase, short = NP}
\DeclareAcronym{vp}{long = verb phrase, short = VP}
\DeclareAcronym{pp}{long = preposition phrase, short = PP}
\DeclareAcronym{cc}{long = copular complement, short = CC}
\DeclareAcronym{cs}{long = copular subject, short = CS}
\DeclareAcronym{tam}{long = {tense, aspect, mood}, short = TAM}
\DeclareAcronym{tame}{long = {Tense, Aspect, Mood, Evidentiality}, short = TAME}
\DeclareAcronym{copula}{long = copula, short = COP}

% Pronouns 

\DeclareAcronym{dist}{long = distal, short = \textsc{dist}}
\DeclareAcronym{prox}{long = proximate, short = \textsc{prox}}
\DeclareAcronym{dem}{long = demonstrative, short = \textsc{dem}}

% TAME, negative

\DeclareAcronym{neg}{long = negative, short = \textsc{neg}}
\DeclareAcronym{past}{long = \textsc{past}, short = \textsc{pst}}
\DeclareAcronym{imperfect}{long = \textsc{imperfect}, short = \textsc{impf}}
\DeclareAcronym{present}{long = \textsc{present}, short = \textsc{pres}}
\DeclareAcronym{perfect}{long = \textsc{perfect}, short = \textsc{perf}}
\DeclareAcronym{future}{long = \textsc{future}, short = \textsc{fut}}
\DeclareAcronym{pluperfect}{long = \textsc{pluperfect}, short = \textsc{plup}}
\DeclareAcronym{future perfect}{long = \textsc{future perfect}, short = \textsc{fut.perf}}
\DeclareAcronym{passive}{long = passive, short = PASS}
\DeclareAcronym{indicative}{long = \textsc{indicative}, short = \textsc{ind}}
\DeclareAcronym{subjunctive}{long = \textsc{subjunctive}, short = \textsc{sjv}}
\DeclareAcronym{imperative}{long = \textsc{imperative}, short = \textsc{imp}}


%endregion

\newcommand*{\homo}[2]{#1$_{\text{#2}}$}

\newcommand{\cgel}{\href{../English/cambridge.pdf}{my notes about CGEL}}
\newcommand{\latin}{\href{../Latin/latin-notes.pdf}{my notes about Latin}}
\newcommand{\alignment}{\href{../alignment/alignment.pdf}{my notes about alignment}}
\newcommand{\exerciseone}{\href{../Exercise/2021-3.pdf}{this exercise}}
\newcommand{\general}{\href{../methodology/glossing.pdf}{this note}}

\newcommand{\ala}{à la}
\newcommand{\translate}[1]{`#1'}
\newcommand{\vP}{\textit{v}P}

% Make subsubsection labeled
\setcounter{secnumdepth}{4}
\setcounter{tocdepth}{4}
% reset example counter every chapter (but do not include the chapter number to the label)
\counterwithin{exx}{chapter} 


\title{Note on Latin Grammar}
\author{Jinyuan Wu}

\begin{document}

\maketitle

\chapter{Overview}

\section{Historical notes}

This note is about Classical Latin and Ecclesiastical Latin.
That's to say Old Latin, vulgar Latin (with prototypes of Romance articles), etc.
are not discussed.

\section{Phonology and the writing system}

\section{Parts of speech}

\subsection{Classification}

Latin word classes can be defined easily via morphology,
and these classes prove to have morphosyntactic significance.
Traditionally speaking, 
word classes with none or poor morphology are called \concept{particles},
and non-particle words can be divided into two large classes:
those with similar morphology of prototypical nouns (i.e. \concept{declension}) are \concept{nominals},
while words with similar morphology of prototypical verbs (i.e. \concept{conjugation})
form a uniform class rightfully called \concept{verbs}.
Nominals include \concept{nouns} and \concept{adjectives},
the distinction between the two can also be defined morphologically.

Latin particles include \concept{prepositions}, \concept{adverbs},
\concept{interjections}, and \concept{conjunctions}.
The adverb class and the preposition class have a large overlap:
often a preposition has an intransitive counterpart,
which is similar to a prototypical adverb.
Conjunctions may be seen as ``prepositions for clauses''.
The functions and etymologies of particles are highly diverse.

Latin nouns, verbs, and adjectives are all open categories.
They are able to head constituents,
and so are correlatives (though correlatives can be listed in the grammar).
The preposition class is closed and is a part of the grammar,
just like conjunctions.
However, conjunctions are purely functional,
while certain prepositions may be argued to head attributive expressions:
though prepositions are often said to be markers of a periphrastic case system,
the semantics carried by certain Latin prepositions are too complicated for a case system.
This is also the case of adverbs:
some adverbs seem to be periphrastic markers of \acs{tame} categories
and therefore may be considered as a part of the grammar,
while others seem to carry ``real'' meanings.
\prettyref{fig:latin-word-class} is a visualization of the classification of Latin word classes.

\begin{theorybox}{Lexical and function classes}{word-class-classification}
    In \citesec{\ref{general-sec:lexical-function-distinction}} in \general, 
    by words with ``real category labels'',
    I mean words that have``real'' meanings
    and serve as lexical heads of constituents
    (i.e. being surrounded by function words and dependents).
    Certain adverbs and prepositions have ``real category labels'',
    and they appear at the left side in \prettyref{fig:latin-word-class}.
    Prepositions can be enumerated and therefore are considered as a part of the grammar,
    so they are always at the lower side in \prettyref{fig:latin-word-class}.
    Other adverbs and prepositions are light in their semantic
    and are purely functional,
    so they appear in the southeast corner of \prettyref{fig:latin-word-class}.
\end{theorybox}

\begin{sidewaysfigure}
    \centering
    

\tikzset{every picture/.style={line width=0.3pt}} %set default line width to 0.75pt        

\begin{tikzpicture}[x=0.75pt,y=0.75pt,yscale=-0.8,xscale=0.8]
%uncomment if require: \path (0,674); %set diagram left start at 0, and has height of 674

%Straight Lines [id:da0819119912251447] 
\draw [color={rgb, 255:red, 155; green, 155; blue, 155 }  ,draw opacity=0.2 ]   (109.33,384.91) -- (793.33,384.91) ;
%Rounded Rect [id:dp9835773367494156] 
\draw  [color={rgb, 255:red, 155; green, 155; blue, 155 }  ,draw opacity=1 ][fill={rgb, 255:red, 155; green, 155; blue, 155 }  ,fill opacity=0.2 ] (141.33,256.64) .. controls (141.33,246.48) and (149.57,238.24) .. (159.73,238.24) -- (214.93,238.24) .. controls (225.1,238.24) and (233.33,246.48) .. (233.33,256.64) -- (233.33,518.84) .. controls (233.33,529) and (225.1,537.24) .. (214.93,537.24) -- (159.73,537.24) .. controls (149.57,537.24) and (141.33,529) .. (141.33,518.84) -- cycle ;
%Rounded Rect [id:dp12690257166641228] 
\draw  [color={rgb, 255:red, 155; green, 155; blue, 155 }  ,draw opacity=1 ][fill={rgb, 255:red, 155; green, 155; blue, 155 }  ,fill opacity=0.2 ] (233.33,256.64) .. controls (233.33,246.48) and (241.57,238.24) .. (251.73,238.24) -- (306.93,238.24) .. controls (317.1,238.24) and (325.33,246.48) .. (325.33,256.64) -- (325.33,518.84) .. controls (325.33,529) and (317.1,537.24) .. (306.93,537.24) -- (251.73,537.24) .. controls (241.57,537.24) and (233.33,529) .. (233.33,518.84) -- cycle ;
%Rounded Rect [id:dp4571683339095527] 
\draw  [color={rgb, 255:red, 155; green, 155; blue, 155 }  ,draw opacity=1 ][fill={rgb, 255:red, 155; green, 155; blue, 155 }  ,fill opacity=0.2 ] (349.33,256.57) .. controls (349.33,248.1) and (356.2,241.24) .. (364.67,241.24) -- (426,241.24) .. controls (434.47,241.24) and (441.33,248.1) .. (441.33,256.57) -- (441.33,302.57) .. controls (441.33,311.04) and (434.47,317.91) .. (426,317.91) -- (364.67,317.91) .. controls (356.2,317.91) and (349.33,311.04) .. (349.33,302.57) -- cycle ;
%Rounded Rect [id:dp13945556495695044] 
\draw  [color={rgb, 255:red, 155; green, 155; blue, 155 }  ,draw opacity=1 ][fill={rgb, 255:red, 155; green, 155; blue, 155 }  ,fill opacity=0.2 ] (122.93,223.64) .. controls (122.93,211.82) and (132.51,202.24) .. (144.33,202.24) -- (316.93,202.24) .. controls (328.75,202.24) and (338.33,211.82) .. (338.33,223.64) -- (338.33,537.84) .. controls (338.33,549.66) and (328.75,559.24) .. (316.93,559.24) -- (144.33,559.24) .. controls (132.51,559.24) and (122.93,549.66) .. (122.93,537.84) -- cycle ;
%Straight Lines [id:da9866800561526903] 
\draw [color={rgb, 255:red, 155; green, 155; blue, 155 }  ,draw opacity=0.2 ]   (595.33,577.91) -- (595.33,184.91) ;
%Rounded Rect [id:dp9919662328975185] 
\draw  [color={rgb, 255:red, 155; green, 155; blue, 155 }  ,draw opacity=1 ][fill={rgb, 255:red, 155; green, 155; blue, 155 }  ,fill opacity=0.2 ] (435,431.44) .. controls (435,427.46) and (438.22,424.24) .. (442.2,424.24) -- (688.8,424.24) .. controls (692.78,424.24) and (696,427.46) .. (696,431.44) -- (696,453.04) .. controls (696,457.02) and (692.78,460.24) .. (688.8,460.24) -- (442.2,460.24) .. controls (438.22,460.24) and (435,457.02) .. (435,453.04) -- cycle ;
%Rounded Rect [id:dp04492990660366303] 
\draw  [color={rgb, 255:red, 155; green, 155; blue, 155 }  ,draw opacity=1 ][fill={rgb, 255:red, 155; green, 155; blue, 155 }  ,fill opacity=0.2 ] (147,482.31) .. controls (147,476.75) and (151.51,472.24) .. (157.07,472.24) -- (579.93,472.24) .. controls (585.49,472.24) and (590,476.75) .. (590,482.31) -- (590,512.51) .. controls (590,518.07) and (585.49,522.57) .. (579.93,522.57) -- (157.07,522.57) .. controls (151.51,522.57) and (147,518.07) .. (147,512.51) -- cycle ;
%Shape: Path Data [id:dp3209513602067542] 
\draw  [color={rgb, 255:red, 155; green, 155; blue, 155 }  ,draw opacity=1 ][fill={rgb, 255:red, 155; green, 155; blue, 155 }  ,fill opacity=0.2 ] (478.97,255.24) -- (552.94,255.24) .. controls (557.16,255.24) and (560.58,258.46) .. (560.58,262.43) -- (560.58,367.51) .. controls (560.58,372.79) and (565.13,377.07) .. (570.74,377.07) -- (630.37,377.07) .. controls (630.59,377.07) and (630.82,377.07) .. (631.04,377.05) -- (712.31,377.05) .. controls (717.01,377.05) and (720.81,380.63) .. (720.81,385.05) -- (720.81,434.24) .. controls (720.81,438.66) and (717.01,442.24) .. (712.31,442.24) -- (478.97,442.24) .. controls (474.75,442.24) and (471.33,439.02) .. (471.33,435.05) -- (471.33,262.43) .. controls (471.33,258.46) and (474.75,255.24) .. (478.97,255.24) -- cycle ;
%Rounded Rect [id:dp9478936078417666] 
\draw  [color={rgb, 255:red, 155; green, 155; blue, 155 }  ,draw opacity=1 ][fill={rgb, 255:red, 155; green, 155; blue, 155 }  ,fill opacity=0.2 ] (451.33,237.22) .. controls (451.33,228.95) and (458.04,222.24) .. (466.31,222.24) -- (759.35,222.24) .. controls (767.63,222.24) and (774.33,228.95) .. (774.33,237.22) -- (774.33,552.26) .. controls (774.33,560.53) and (767.63,567.24) .. (759.35,567.24) -- (466.31,567.24) .. controls (458.04,567.24) and (451.33,560.53) .. (451.33,552.26) -- cycle ;

% Text Node
\draw (169,267) node [anchor=north west][inner sep=0.75pt]   [align=left] {noun};
% Text Node
\draw (252,267) node [anchor=north west][inner sep=0.75pt]   [align=left] {adjective};
% Text Node
\draw (158,479) node [anchor=north west][inner sep=0.75pt]   [align=left] {personal \\pronoun};
% Text Node
\draw (244,479) node [anchor=north west][inner sep=0.75pt]   [align=left] {correlative \\pronoun};
% Text Node
\draw (368,266) node [anchor=north west][inner sep=0.75pt]   [align=left] {verb};
% Text Node
\draw (565.5,442.24) node   [align=left] {preposition};
% Text Node
\draw (674,517) node [anchor=north west][inner sep=0.75pt]   [align=left] {conjunction};
% Text Node
\draw (492,264) node [anchor=north west][inner sep=0.75pt]   [align=left] {adverb};
% Text Node
\draw (674,471) node [anchor=north west][inner sep=0.75pt]   [align=left] {interjection};
% Text Node
\draw (151,361) node [anchor=north west][inner sep=0.75pt]  [color={rgb, 255:red, 155; green, 155; blue, 155 }  ,opacity=1 ] [align=left] {noun \\morphology};
% Text Node
\draw (238.14,361) node [anchor=north west][inner sep=0.75pt]  [color={rgb, 255:red, 155; green, 155; blue, 155 }  ,opacity=1 ] [align=left] {adjective \\morphology};
% Text Node
\draw (107.33,384.91) node [anchor=east] [inner sep=0.75pt]  [color={rgb, 255:red, 155; green, 155; blue, 155 }  ,opacity=1 ] [align=left] {with real \\category \\label};
% Text Node
\draw (795.33,384.91) node [anchor=west] [inner sep=0.75pt]  [color={rgb, 255:red, 155; green, 155; blue, 155 }  ,opacity=1 ] [align=left] {with \\no real \\category \\label};
% Text Node
\draw (141,212.24) node [anchor=north west][inner sep=0.75pt]  [color={rgb, 255:red, 155; green, 155; blue, 155 }  ,opacity=1 ] [align=left] {nominal};
% Text Node
\draw (595.33,181.91) node [anchor=south] [inner sep=0.75pt]  [color={rgb, 255:red, 155; green, 155; blue, 155 }  ,opacity=1 ] [align=left] {not a part\\of grammar};
% Text Node
\draw (595.33,580.91) node [anchor=north] [inner sep=0.75pt]  [color={rgb, 255:red, 155; green, 155; blue, 155 }  ,opacity=1 ] [align=left] {a part of\\grammar};
% Text Node
\draw (499,489.5) node [anchor=north west][inner sep=0.75pt]   [align=left] {pro-adverb};
% Text Node
\draw (379,489.5) node [anchor=north west][inner sep=0.75pt]  [color={rgb, 255:red, 155; green, 155; blue, 155 }  ,opacity=1 ] [align=left] {pro-forms};
% Text Node
\draw (661,241.5) node [anchor=north west][inner sep=0.75pt]  [color={rgb, 255:red, 155; green, 155; blue, 155 }  ,opacity=1 ] [align=left] {particle};


\end{tikzpicture}

    \caption{Latin word classes}
    \label{fig:latin-word-class}
\end{sidewaysfigure}

Articles (English \corpus{a} or \corpus{the}), 
despite prevalent in other Indo-European languages,
are missing in Latin.
This, together with the fact that Classical Sanskrit and Old Persian didn't have articles 
and the Slavic languages still don't,
is a strong indicator that \ac{pie} didn't have articles. 

\section{Morphology}

Latin has rich morphology,
which enables a rather free -- but still not completely arbitrary -- constituent order.
Latin has a clear inflection-derivation distinction.
Despite its richness, 
Latin derivation is largely historical,
with meanings of derived forms 
having shifted and no longer regularly inferrable.
Latin inflection is always suffixal,
while derivation is predominantly prefixal.
Concatenative morphology (affixation and compounding) 
is prominent but isn't the only morphological device:
the following non-concatenative mechanisms are all attested:
\begin{itemize}
    \item \emph{Reduplication}: formation of the perfect stem (TODO: ref)
    \item \emph{Subtraction}: dropping of first-conjugation stem-final vowel (\prettyref{sec:tense-mood-marking}).
    \item \emph{Infixation}:   TODO: ref 
    The imperfect \corpus{-ba-} is sometimes said to be an infix 
    (as well as its counterparts like \corpus{-bi-}),
    though it fits in a concatenative picture of verbal morphology.
\end{itemize}
These mechanisms, however, are largely historical,
just like their concatenative counterparts.

\begin{theorybox}{Constituency deemphasized in Latin grammar}{constituency-in-grammar}
    The largely free constituent order 
    means description of Latin grammar is mostly dependency-relation based or \acs{blt}-based,
    because surface-based constituents other than \acs{np}s and clauses are hard to define.
    Still, generative (constituency-based, 
    though the introduction of movements and the structure of Cinque hierarchy
    gives it certain flavor of dependency grammars) approaches exist for Latin constituent order.
    There is evidence suggesting Latin is configurational, 
    i.e. has phrase structures \citep{danckaert2017development}.
    This is probably not surprising because
    even the most non-configurational languages show certain degree of configurationality 
    \citep[among others]{niedzielski2017clausal,morris2018evidence,legate2002warlpiri}.
    Then, \emph{how} non-configurational Latin is is a question needing addressing.
    Is it closer to a typical non-configurational language, say Warlpiri, 
    or is it closer to Japanese where we have more localized scrambling?
    I will address this question in TODO: ref ,
    though unfortunately, we still does not have a very clear answer.
\end{theorybox}

\section{Noun phrases and nominal morphology}

\section{Verbal morphology and clause structure}

Most clausal grammatical categories are marked on the verbal morphology.
Sometimes a grammatical category is there but is not reflected in the morphology.
For example, in English we have infinitive clauses,
but strictly speaking, there is no such thing as ``infinitive verb'':
the head verb of an infinitive clause 
has exactly the same form of a non-third person singular present tense verb.
This is not the case in Latin.
For example, the head verb of a infinitive clause in Latin 
indeed has a separate position in the paradigm.
Thus, grammatical categories of the clause are listed in this section.

\subsection{The finite paradigm}

\subsubsection{Voice}

Latin doesn't have rich valency changing devices:
there is only one clause-wide valency decreasing device -- passivization -- 
and there is no valency increasing device.
Causative constructions are realized by complement clauses,
not any change in the argument structure.
Whether passivization happens is recorded by the category of \concept{voice}.
A verb (and hence the clause headed by it) is therefore either in \concept{active voice},
or in \concept{passive voice}.

\begin{theorybox}{Valency changing}{valency-changing}
    See \citesec{\ref{general-sec:valency-changing-theory}} in \general.
    From a generative perspective, some languages realize valency changing 
    by a series of \vP{} structures, and then the case assignment of the arguments is trivial.
    Some languages use non-trivial cases of 
    the structural case assignment mechanism 
    to achieve valency changing 
    (``suppressing the agent argument, and then the nominative probe has to choose the patient argument'').
    Of course, \vP{} changes in the second type are still there,
    which may be a likely source of relevant verb morphology.
    Naturally, the second group of languages have more restricted valency changing devices;
    this is the case of Latin.
\end{theorybox}

\subsubsection{\acs{tame} categories}

Latin has fused tense and aspect:
the composition of three tense values and three aspect values 
gives nine options,
but in Latin, there are only six morphologically distinguished options,
as is shown in \prettyref{tbl:latin-tense-aspect}. 
When people talk about \concept{tense} in Latin (and in many other Indo-European languages),
they are often taking about things like the six options,
instead of the past/present/future system.

\begin{table}[H]
    \caption{Latin tense and aspect}
    \label{tbl:latin-tense-aspect}
    \centering
    \begin{tabular}{@{}cccc@{}}
    \toprule
              & past       & present                  & future                  \\ \midrule
    imperfect & \acl{imperfect}  & \multirow{2}{*}{\acl{present}} & \multirow{2}{*}{\acl{future}} \\
    simple    & \acl{perfect}    &                          &                         \\
    perfect   & \acl{pluperfect} & \acl{perfect}                  & \acl{future perfect}          \\ \bottomrule
\end{tabular}    
\end{table}

Similar fusion between categories is shown in the category of \concept{mood}.
It's the fusion of morphologically marked clause type 
(declarative and imperative)
and morphologically marked modality.
The verb morphology of interrogative clauses is exactly the same as declarative clauses:
the interrogative clause type is marked by the existence of interrogative \term{pro}-forms.
Thus, there are three moods in finite clauses in Latin:
\acl{indicative}, \acl{subjunctive}, and \acl{imperative}.
The \acl{indicative} is the fusion of 
the declarative/interrogative clause type and the realis modality.
The \acl{subjunctive} mood is the fusion of 
the declarative/interrogative clause type and the irrealis modality.
The \acl{imperative} is basically the imperative clause type:
it doesn't allow modality marking.
Sometimes people say the infinitive is the fourth mood,
though it's a non-finite clause.

\begin{theorybox}{The term \term{mood}}{mood}
    \acs{blt} only calls the first category \term{mood}.
    Different linguists use the term \term{mood} and \term{modality} in radically different ways.
    In this note I just focus on the common practice in Latin grammar study.
\end{theorybox}

\begin{theorybox}{Mismathc between \ac{tame} constructions and fine-grained categories}{tame}
    Atomic \ac{tame} features and packaged \ac{tame} marking constructions
    often show certain degree of discrepancy.
    As we see in \prettyref{tbl:latin-tense-aspect},
    the \acl{perfect} construction may have simple aspect and past tense.
    Following the example in \citet{grimm2021grammar},
    in this note, I use small capitals for the names of attested surface realizations of \ac{tame}
    and the default font for \ac{tame} values.
    (Some other grammars, like \citet{jacques2021grammar,friesen2017grammar}, 
    use initial capitals for the former.)
\end{theorybox}

\subsubsection{Agreement}

Latin is a typical nominative-accusative language,
both morphologically and syntactically.
In finite clauses, 
there is subject-verb agreement:
the number and person of the subject is marked on the main verb
(in the case of periphrastic conjugation,
the features are marked on the copula).

\subsubsection{Compatability of categories}

There are compatibility problems of these categories.
There is no \acl{future} tense and \acl{future perfect} tense in subjunctive clauses,
probably for the semantic reason
that the future tense already contains certain sense of modality
(an event predicted to happen),
and thus is not compatible with the \acl{subjunctive} mood.
The \acl{imperative} mood is not compatible with other \ac{tame} markings
except the \acl{present} tense and the \acl{future} tense.
It's still compatible with the voice category,
and allowed persons are 
second person singular/plural with the \acl{present} tense,
and second/third person singular/plural with the \acl{future} tense.
The absence of first person is also probably from semantic origin.

In conclusion, the scheme of the finite verb paradigm of Latin 
is shown in \prettyref{fig:paradigm-finite-verb}.
The exact realization is divided into four conjugation classes,
and the details are too complex to show here.

\begin{figure}[H]
    \centering
    

\tikzset{every picture/.style={line width=0.75pt}} %set default line width to 0.75pt        

\begin{tikzpicture}[x=0.75pt,y=0.75pt,yscale=-0.8,xscale=0.8]
%uncomment if require: \path (0,560); %set diagram left start at 0, and has height of 560

%Straight Lines [id:da4771979505475994] 
\draw [color={rgb, 255:red, 208; green, 2; blue, 27 }  ,draw opacity=1 ][line width=2.25]    (72,476.59) -- (165,476.59) ;
%Shape: Rectangle [id:dp4651809120211663] 
\draw  [draw opacity=0][fill={rgb, 255:red, 208; green, 2; blue, 27 }  ,fill opacity=0.1 ] (72,51.93) -- (165,51.93) -- (165,195.59) -- (72,195.59) -- cycle ;
%Shape: Rectangle [id:dp8674436914111818] 
\draw  [draw opacity=0][fill={rgb, 255:red, 245; green, 166; blue, 35 }  ,fill opacity=0.1 ] (208,51.93) -- (324,51.93) -- (324,195.59) -- (208,195.59) -- cycle ;
%Straight Lines [id:da9097932489630769] 
\draw [color={rgb, 255:red, 245; green, 166; blue, 35 }  ,draw opacity=1 ][line width=2.25]    (210,476.59) -- (324,476.59) ;
%Shape: Rectangle [id:dp6458311390526075] 
\draw  [draw opacity=0][fill={rgb, 255:red, 245; green, 166; blue, 35 }  ,fill opacity=0.1 ] (208,213.93) -- (324,213.93) -- (324,317.59) -- (208,317.59) -- cycle ;
%Shape: Rectangle [id:dp4418485190624626] 
\draw  [draw opacity=0][fill={rgb, 255:red, 208; green, 2; blue, 27 }  ,fill opacity=0.1 ] (73,213.93) -- (166,213.93) -- (166,318.26) -- (73,318.26) -- cycle ;
%Shape: Rectangle [id:dp7155013561532602] 
\draw  [draw opacity=0][fill={rgb, 255:red, 245; green, 166; blue, 35 }  ,fill opacity=0.1 ] (208,341.59) -- (324,341.59) -- (324,373.59) -- (208,373.59) -- cycle ;
%Shape: Rectangle [id:dp5109073549880265] 
\draw  [draw opacity=0][fill={rgb, 255:red, 245; green, 166; blue, 35 }  ,fill opacity=0.1 ] (208,387.59) -- (324,387.59) -- (324,419.59) -- (208,419.59) -- cycle ;
%Shape: Rectangle [id:dp4978185928249894] 
\draw  [draw opacity=0][fill={rgb, 255:red, 208; green, 2; blue, 27 }  ,fill opacity=0.1 ] (73,341.59) -- (166,341.59) -- (166,420.93) -- (73,420.93) -- cycle ;
%Shape: Rectangle [id:dp7974586880324295] 
\draw  [draw opacity=0][fill={rgb, 255:red, 126; green, 211; blue, 33 }  ,fill opacity=0.1 ] (364,52.93) -- (471.33,52.93) -- (471.33,419.59) -- (364,419.59) -- cycle ;
%Straight Lines [id:da40576537799721035] 
\draw [color={rgb, 255:red, 126; green, 211; blue, 33 }  ,draw opacity=1 ][line width=2.25]    (364,476.59) -- (470,476.59) ;
%Shape: Rectangle [id:dp37304865952309374] 
\draw  [draw opacity=0][fill={rgb, 255:red, 80; green, 227; blue, 194 }  ,fill opacity=0.1 ] (515,51.93) -- (631,51.93) -- (631,318.93) -- (515,318.93) -- cycle ;
%Shape: Rectangle [id:dp09424442237743991] 
\draw  [draw opacity=0][fill={rgb, 255:red, 80; green, 227; blue, 194 }  ,fill opacity=0.1 ] (514,341.59) -- (630,341.59) -- (630,373.59) -- (514,373.59) -- cycle ;
%Shape: Rectangle [id:dp7879043441622926] 
\draw  [draw opacity=0][fill={rgb, 255:red, 80; green, 227; blue, 194 }  ,fill opacity=0.1 ] (514,387.59) -- (630,387.59) -- (630,419.59) -- (514,419.59) -- cycle ;
%Straight Lines [id:da6694037096709569] 
\draw [color={rgb, 255:red, 80; green, 227; blue, 194 }  ,draw opacity=1 ][line width=2.25]    (514,476.59) -- (628,476.59) ;
%Shape: Rectangle [id:dp8356914669291959] 
\draw  [draw opacity=0][fill={rgb, 255:red, 74; green, 144; blue, 226 }  ,fill opacity=0.1 ] (675,52.93) -- (782.33,52.93) -- (782.33,419.59) -- (675,419.59) -- cycle ;
%Straight Lines [id:da18180972606397305] 
\draw [color={rgb, 255:red, 74; green, 144; blue, 226 }  ,draw opacity=1 ][line width=2.25]    (676.33,476.59) -- (782.33,476.59) ;

% Text Node
\draw (118.5,123.76) node   [align=left] {indicative};
% Text Node
\draw (119.5,266.09) node   [align=left] {subjunctive};
% Text Node
\draw (266,123.76) node   [align=left] {present/\\imperfect/\\future/\\perfect/\\pluperfect/\\future perfect};
% Text Node
\draw (266,265.76) node   [align=left] {present/\\imperfect/\\perfect/\\pluperfect};
% Text Node
\draw (417.67,236.26) node   [align=left] {active/\\passive};
% Text Node
\draw (573,185.43) node   [align=left] {1/\\2/\\3};
% Text Node
\draw (728.67,236.26) node  [color={rgb, 255:red, 0; green, 0; blue, 0 }  ,opacity=1 ] [align=left] {single/\\plural};
% Text Node
\draw (119.5,381.26) node   [align=left] {imperative};
% Text Node
\draw (266,357.59) node   [align=left] {present};
% Text Node
\draw (266,403.59) node   [align=left] {future};
% Text Node
\draw (572,357.59) node   [align=left] {2};
% Text Node
\draw (572,403.59) node   [align=left] {2/3};
% Text Node
\draw (118.5,479.59) node [anchor=north] [inner sep=0.75pt]  [color={rgb, 255:red, 208; green, 2; blue, 27 }  ,opacity=1 ] [align=left] {mood};
% Text Node
\draw (267,479.59) node [anchor=north] [inner sep=0.75pt]  [color={rgb, 255:red, 245; green, 166; blue, 35 }  ,opacity=1 ] [align=left] {\textcolor[rgb]{0.96,0.65,0.14}{tense}};
% Text Node
\draw (417,479.59) node [anchor=north] [inner sep=0.75pt]  [color={rgb, 255:red, 126; green, 211; blue, 33 }  ,opacity=1 ] [align=left] {voice};
% Text Node
\draw (571,479.59) node [anchor=north] [inner sep=0.75pt]  [color={rgb, 255:red, 80; green, 227; blue, 194 }  ,opacity=1 ] [align=left] {\textcolor[rgb]{0.31,0.89,0.76}{person}};
% Text Node
\draw (729.33,479.59) node [anchor=north] [inner sep=0.75pt]  [color={rgb, 255:red, 74; green, 144; blue, 226 }  ,opacity=1 ] [align=left] {number};


\end{tikzpicture}

    \caption{The scheme of the finite paradigm}
    \label{fig:paradigm-finite-verb}
\end{figure}

\subsection{Non-finite forms}

\subsection{Core, oblique, and peripheral arguments}

\section{Clause combining}

\section{Constituent order}\label{sec:constituent-order-abs}


\chapter{Phonology and the writing system}

\chapter{Nominal morphology}

\begin{infobox}{Testing}{test}
    test
\end{infobox}

\chapter{Verb morphology}

\begin{theorybox}{About the number of conjugation forms}{conjugation-form}
    Different people use the term \term{conjugation forms} -- and count them -- in different ways.
    The most generous -- and the most syntactically relevant -- way 
    is to view the realization of every possible CP-TP-\vP{} projection 
    as a form of the main verb -- the verb root at the core of the CP-TP-\vP{} domains.
    This results in a paradigm in traditional grammar.
    The problem with this approach is sometimes two cells in the paradigm are always identical,
    so recognizing them as two morphological forms is weird.
    (Also, this is not a good idea when dealing with languages like Japanese.)
    A stingy linguist may then stipulate that conjugation forms are literally about \emph{forms},
    and thus there is no such thing as ``the subjunctive form'' of English verbs,
    because in subject \emph{clauses}, 
    the main verb always has the same form as the infinitive.

    The generous approach fortunately works in Latin 
    because Latin is morphologically rich.
    The idea of the stingy linguist may lead one to reject the notion of supine in Latin grammar,
    but since sometimes a verb lacks TODO: argumentation for a separate supine form,
    for the same reason the infinitive is recognized as a form 
    independent from the ``default form'' in English in \acs{cgel},
    the status of supine as a separate form is recognized in this note.

    The analysis of conjugation forms of the verb, theoretically speaking,
    is more about vocabulary insertion and readjustment rules,
    instead of the syntax proper.
\end{theorybox}

\begin{theorybox}{About the concept of stem}{stem}
    The notion of \concept{stems} isn't really essential in the description of morphosyntactic:
    it can well be modeled by environment-dependent vocabulary insertion rules 
    and/or post-syntactic operations.
    When suppletion rules are still synchronic,
    what happens may be analyzed as in \citet{embick2005status},
    where certain stems receive morphophonological readjustment.
    When these readjustment rules are fossilized,
    suppletion -- like the English \corpus{good}/\corpus{better} -- 
    may just be the result of conditional insertion,
    as is outlined in \citet{siddiqi2009syntax}.

    This in return gives an predication about suppletion:
    conditional vocabulary insertion can create unlimited allomorphs,
    while readjustment rules are restricted in their computational capacity,
    so real lexical verbs are highly unlikely to have truly irregular suppletion;
    that is, if a verb is truly suppletive,
    then it's likely to be 
    As is said in ,
\end{theorybox}

\chapter{Valency classes}

\chapter{Examples of texts}

\bibliographystyle{plainnat}
\bibliography{latin,theory}

\end{document}