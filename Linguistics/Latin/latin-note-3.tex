\documentclass[a4paper, oneside]{report}

\usepackage{libertinus}
\usepackage{geometry}
\usepackage{float}
\usepackage{titling}
\usepackage{titlesec}
\usepackage{paralist}
\usepackage{footnote}
\usepackage[inline]{enumitem}
\usepackage{amsmath, amssymb, amsthm}
\usepackage{gb4e}
\noautomath
\usepackage{bbm}
\usepackage{soul}
\usepackage{graphicx}
\usepackage{siunitx}
\usepackage[table,xcdraw]{xcolor}
\usepackage{tikz}
\usepackage[ruled, vlined, linesnumbered, noend]{algorithm2e}
\usepackage{xr-hyper}
\usepackage[colorlinks,citecolor=purple]{hyperref} % linkcolor=black, anchorcolor=black, citecolor=black, filecolor=black
\usepackage[most]{tcolorbox}
\usepackage{caption}
\usepackage{subcaption}
\usepackage{booktabs}
\usepackage{multirow}
\usepackage[figuresright]{rotating}
\usepackage{acro}
\usepackage[round]{natbib} 
\usepackage{nameref,zref-xr}
\zxrsetup{toltxlabel}
\zexternaldocument*[cgel-]{../English/cambridge}[cambridge.pdf]
\zexternaldocument*[alignment-]{../alignment/alignment}[alignment.pdf]
\zexternaldocument*[exercise1-]{../Exercise/2021-3}[2021-3.pdf]
\zexternaldocument*[general-]{../methodology/glossing}[glossing.pdf]
\usepackage{prettyref}

\geometry{left=3.18cm,right=3.18cm,top=2.54cm,bottom=2.54cm}
\titlespacing{\paragraph}{0pt}{1pt}{10pt}[20pt]
\setlength{\droptitle}{-5em}

\DeclareMathOperator{\timeorder}{\mathcal{T}}
\DeclareMathOperator{\diag}{diag}
\DeclareMathOperator{\legpoly}{P}
\DeclareMathOperator{\primevalue}{P}
\DeclareMathOperator{\sgn}{sgn}
\newcommand*{\ii}{\mathrm{i}}
\newcommand*{\ee}{\mathrm{e}}
\newcommand*{\const}{\mathrm{const}}
\newcommand*{\suchthat}{\quad \text{s.t.} \quad}
\newcommand*{\argmin}{\arg\min}
\newcommand*{\argmax}{\arg\max}
\newcommand*{\normalorder}[1]{: #1 :}
\newcommand*{\pair}[1]{\langle #1 \rangle}
\newcommand*{\fd}[1]{\mathcal{D} #1}

\newcommand*{\citesec}[1]{\S~{#1}}
\newcommand*{\citechap}[1]{chap.~{#1}}
\newcommand*{\citefig}[1]{Fig.~{#1}}
\newcommand*{\citetable}[1]{Table~{#1}}
\newcommand*{\citepage}[1]{p.~{#1}}
\newcommand*{\citepages}[1]{pp.~{#1}}
\newcommand*{\citefootnote}[1]{fn.~{#1}}

\newrefformat{sec}{\citesec{\ref{#1}}}
\newrefformat{fig}{\citefig{\ref{#1}}}
\newrefformat{tbl}{\citetable{\ref{#1}}}
\newrefformat{chap}{\citechap{\ref{#1}}}
\newrefformat{fn}{\citefootnote{\ref{#1}}}
\newrefformat{box}{Box~\ref{#1}}
\newrefformat{ex}{\ref{#1}}

\tcbuselibrary{skins, breakable, theorems}

\newtcbtheorem[number within=chapter]{infobox}{Box}{
    enhanced,
    boxrule=0pt,
    colback=blue!5,
    colframe=blue!5,
    coltitle=blue!50,
    borderline west={4pt}{0pt}{blue!65},
    sharp corners,
    fonttitle=\bfseries, 
    breakable,
    before upper={\parindent15pt\noindent}}{box}
\newtcbtheorem[number within=chapter, use counter from=infobox]{theorybox}{Box}{
    enhanced,
    boxrule=0pt,
    colback=orange!5, 
    colframe=orange!5, 
    coltitle=orange!50,
    borderline west={4pt}{0pt}{orange!65},
    sharp corners,
    fonttitle=\bfseries, 
    breakable,
    before upper={\parindent15pt\noindent}}{box}
\newtcbtheorem[number within=chapter, use counter from=infobox]{learnbox}{Box}{
    enhanced,
    boxrule=0pt,
    colback=green!5,
    colframe=green!5,
    coltitle=green!50,
    borderline west={4pt}{0pt}{green!65},
    sharp corners,
    fonttitle=\bfseries, 
    breakable,
    before upper={\parindent15pt\noindent}}{box}

\newcommand*{\concept}[1]{\textbf{#1}}
\newcommand*{\term}[1]{\emph{#1}}
\newcommand{\form}[1]{\emph{#1}}
\newcommand*{\category}[1]{\textsc{#1}}

%region Acronym

% Theory
\DeclareAcronym{blt}{short = BLT, long = Basic Linguistic Theory}
\DeclareAcronym{cgel}{short = CGEL, long = The Cambridge Grammar of the English Language}
\DeclareAcronym{dm}{short = DM, long = Distributed Morphology}
\DeclareAcronym{tag}{long = Tree-adjoining grammar, short = TAG}

% History

\DeclareAcronym{pie}{long = proto-Indo-European, short = PIE}

% Roles

\DeclareAcronym{sfp}{long = sentence final particle, short = SFP}
\DeclareAcronym{np}{long = noun phrase, short = NP}
\DeclareAcronym{vp}{long = verb phrase, short = VP}
\DeclareAcronym{pp}{long = preposition phrase, short = PP}
\DeclareAcronym{adjp}{long = adjective phrase, short = AdjP}
\DeclareAcronym{advp}{long = adverb phrase, short = AdvP}
\DeclareAcronym{cc}{long = copular complement, short = CC}
\DeclareAcronym{cs}{long = copular subject, short = CS}
\DeclareAcronym{tam}{long = {tense, aspect, mood}, short = TAM}
\DeclareAcronym{tame}{long = {Tense, Aspect, Mood, Evidentiality}, short = TAME}
\DeclareAcronym{copula}{long = copula, short = COP}

% Pronouns 

\DeclareAcronym{dist}{long = distal, short = \textsc{dist}}
\DeclareAcronym{prox}{long = proximate, short = \textsc{prox}}
\DeclareAcronym{dem}{long = demonstrative, short = \textsc{dem}}

% TAME, negative

\DeclareAcronym{neg}{long = negative, short = \textsc{neg}}
\DeclareAcronym{past}{long = \textsc{past}, short = \textsc{pst}}
\DeclareAcronym{imperfect}{long = \textsc{imperfect}, short = \textsc{impf}}
\DeclareAcronym{present}{long = \textsc{present}, short = \textsc{pres}}
\DeclareAcronym{perfect}{long = \textsc{perfect}, short = \textsc{perf}}
\DeclareAcronym{future}{long = \textsc{future}, short = \textsc{fut}}
\DeclareAcronym{pluperfect}{long = \textsc{pluperfect}, short = \textsc{plup}}
\DeclareAcronym{future perfect}{long = \textsc{future perfect}, short = \textsc{fut.perf}}
\DeclareAcronym{passive}{long = passive, short = PASS}
\DeclareAcronym{indicative}{long = \textsc{indicative}, short = \textsc{ind}}
\DeclareAcronym{subjunctive}{long = \textsc{subjunctive}, short = \textsc{sjv}}
\DeclareAcronym{imperative}{long = \textsc{imperative}, short = \textsc{imp}}


%endregion

\newcommand*{\homo}[2]{#1$_{\text{#2}}$}

\newcommand{\cgel}{\href{../English/cambridge.pdf}{my notes about CGEL}}
\newcommand{\latin}{\href{../Latin/latin-notes.pdf}{my notes about Latin}}
\newcommand{\alignment}{\href{../alignment/alignment.pdf}{my notes about alignment}}
\newcommand{\exerciseone}{\href{../Exercise/2021-3.pdf}{this exercise}}
\newcommand{\general}{\href{../methodology/glossing.pdf}{this note}}

\newcommand{\ala}{à la}
\newcommand{\translate}[1]{`#1'}
\newcommand{\vP}{\textit{v}P}

\newcommand{\classify}[1]{{\textsc{#1}}}

% Make subsubsection labeled
\setcounter{secnumdepth}{4}
\setcounter{tocdepth}{4}
% reset example counter every chapter (but do not include the chapter number to the label)
\counterwithin{exx}{chapter} 

\renewcommand{\bibname}{References}

\title{Note on Latin Grammar}
\author{Jinyuan Wu}

\begin{document}

\automath

\maketitle

\chapter{Introduction}

\section{The language and the speaker}

\subsection{Latin as a classical language}

Latin was the language of the Romans
and the official language of both the Roman Republic and the Roman Empire,
and hence the official language of the Catholic Church, 
which was \emph{the} church for the Western Roman Empire. 
The international nature of the Roman Empire made Latin 
the international language around the Mediterranean Sea at that time -- 
indeed, \form{Mare Nostrum} \translate{our sea} in Latin,
and its importance in science, arts, law, religion, and literature 
lent it more than one thousand of years of life 
as a common literary language and a sacred language in western Europe
after the collapse of the Western Roman Empire 
and the emergence of the Romance language family.

As recently as the nineteen century,
Latin was still fluently used by scholars and in the Catholic Mass. 
A decline in the popularity of Latin was observed after that.
The rapid development of English (at first, also French and German and sometimes Russian)
as the language of science 
largely replaced the status of Latin as a scholar language.
After Vatican allowed vernacular languages being used in liturgies, 
Latin also largely lose its position in the daily use in the Catholic church. 

This note is about \concept{Classical Latin} -- the Latin of classical Latin writers -- 
and \concept{Ecclesiastical Latin} -- the kind of Latin of the Catholic church.
That's to say Old Latin, vulgar Latin (with prototypes of Romance articles), etc.
are not discussed in detail in this note.
Still, some historical knowledge is important for us to understand 
why Latin is the way it is. 

\subsection{Latin in ancient Mediterranean world}

The historical and contemporary importance of Latin 
of course doesn't endorse it as a inherently superior language. 
Indeed, Latin used to be TODO: other languages were more important

We only have a handful of Latin texts before 600BC; 
as a comparison, there are about 150 pre-600BC Etruscan texts.
Even in the period between 600BC and 100BC, 
during which we have around 3000 Latin texts,
we have about 9000 Etruscan texts, 
which is three times as many as their Latin counterparts. 


\section{Previous studies}

\section{Theoretical orientation}\label{sec:theoretical-orientation}

In a word, the theoretical framework of this note 
is \ac{blt}\citep{dixon2009basic1,dixon2010basic2,dixon2012basic3}
with generative flavors.
Here by \term{generative} I mean 
Minimalism plus Distributed Morphology plus Syntactic Cartography 
(but the Antisymmetry theory is not strictly followed here; 
I only use the idea of a semi-rigid template of functional projections).
Although generativism is harshly criticized by Dixon, 
I believe this is largely due to notational reasons;
some criticisms, like ``generativism sticks to the universal notation of words''
or ``generativists blindly believe in a 
noun-verb distinction \emph{exactly} the same as English'',
are invalid in the aforementioned framework. 
To connect the aforementioned school of generativism and \ac{blt}, 
I list some observations:
\begin{itemize}
    \item First, note that ``functional heads'' are just 
        an alternative way to say ``grammatical categories'' or ``grammatical relations''
        in a constituency-based framework doing away with dependency relations.
        The constituency-dependency correspondence 
        has long been discovered 
        \citep{schneider1998linguistic,osborne2011bare,kahane2015syntactic,nefdt2023notational}. 
        On the other hand, 
        ``lexical heads'' -- nouns, verbs, etc. -- 
        lie at the \emph{bottom} of an ``extended \ac{np}'' (i.e. the DP projections) 
        or an ``extended \ac{vp}'' (i.e. the CP projections).
        This settles the issue raised by many descriptive linguists: 
        the term \term{head} is no longer used in the same way 
        as it was in contemporary generativism.
        The \term{head} of descriptive linguists is essentially the \term{root} in Distributed Morphology.
    \item It's possible to ``zip'' the Minimalist constituency (or dependency) structure: 
        removing invisible functional projections, 
        replacing labels like SpecTP with ``subject'',
        using the term \term{head} to refer to the lexical head, etc.
        Thus, we are able to automatically 
        obtain more traditional constituency analysis (as in \citet{cgel})
        or dependency analysis 
        from generative trees. 
        The counterpart of c-command relations in the dependency analysis 
        is how ``tight'' a dependency relation is:
        that the relation between the verb and the object is tighter 
        than the relation between the verb and the subject 
        is equivalent to that the subject has a higher position in the syntactic tree.
    \item One implicit message hidden in the idea that 
        \acs{np}s and clauses are the only two types of constituents 
        in \citet{dixon2009basic1}  
        is that when we finish building up an \acs{np} 
        and insert it into an argument slot in a clause, 
        the syntactic processing enters a new stage;
        on the other hand, the difference between a half-finished \acs{np}
        and a completely finished \acs{np}
        is not that huge.
        Now if we use constituency analysis all the way down, 
        we are in the risk of losing this piece of information.
        This is settled by the concept of \emph{phase} in modern generativism:
        when typologists argue for recognizing only noun phrases and clauses as constituents, 
        they are essentially referring to phases. 
    \item The phase theory also explains why some have the intuition that 
        the \term{verb phrase} should exclude the object: 
        because when the CP is being built, 
        the arguments are already ``frozen'', 
        and what are manipulated and realized together 
        are verbal functional heads -- that's exactly \emph{their} verb phrase. 
        The similar thing happens for a \term{word} (see the next point). 
    \item Some people (many functionalists, but also some formal grammarians) 
        really don't like the idea that 
        differences in constituent order have their roots in 
        the constituent structure and especially in movements. 
        They say constituent order \emph{directly} reflects 
        grammatical relations and categories like topicalization, 
        instead of the mainstream generative idea that 
        constituent order reflects constituency relations,
        which then codes things like topicalization. 
        The exact meaning of ``directly reflect'' however is rather hard to tell 
        from an empirical perspective.
        What we already know is that
        quantitative researches suggest that 
        at least a semi-configurational approach 
        (i.e. a linear template with fixed constituent slots in it)
        is needed to fully capture Latin constituent order, 
        because the diachronic change of the frequency of OVAux
        looks very different from the diachronic change of the overall OV frequency,
        which includes, say, OAuxV \citep{danckaert2015studying}.
        But after we accept the semi-configurational approach, 
        we can then do tests like, say, Principles A, B, and C, 
        coordination and ellipsis tests, etc.,
        on slots of these templates, 
        and usually a hierarchy of 
        relative ``strength '' of dependency relations
        can be established 
        \citep[\citesec{1.6}]{danckaert2017development}.
        Then, by the duality between constituency and dependency, 
        usually we will find 
        that a constituency-based analysis is \emph{accurate} for 
        a so-called non-configurational language, 
        although it may not be \emph{convenient} 
        for its documentation.
    \item A word is just a mini-phrase 
        (in the above phase-as-descriptivist-phrase sense, 
        possibly a mini-tree, possibly a collective realization of a span of functional heads).
        The syntax all the way down analysis taken in this note 
        therefore explains why we always have controversies 
        concerning whether a unit is a word or a phrase 
        (like \form{American history teacher} 
        -- note that its inner parts don't actively participate in other syntactic processes):
        this is no objective standard for drawing a line between the two. 
        What are objective are the morphosyntactic units recognized: 
        \form{American history teacher} is a nominal compounding structure 
        (a certain kind of FP within the NumP projections),
        regardless of whether you say it's a word. 

        According to Cartography syntax, 
        cross-linguistically, 
        we should find similar patterns of functional heads,
        so we should expect to see a morphosyntactic unit 
        of a size similar to what we usually call words in English 
        in another language, 
        although the native speakers may not find this unit important in their society 
        (that ``word'' unit may not be the unit for measuring the length of an article or for writing).
    \item Finally, the hierarchy of functional heads -- or in other words, 
        grammatical relations and categories -- 
        can be ``routinized'' and packaged,
        and how they are stored in the actual brain 
        may have more resemblance to \ac{tag}. 
        Most inflection patterns, for example, 
        seem to be packaged, 
        which explains why sometimes they seem to be psychologically different from syntax, 
        though Distributed Morphology has shown it's possible to treat the grammar 
        as syntax all the way down. 
        This is just what people call \term{construction}.
        However, it seems a construction is still not a packaged \emph{linear} sequence: 
        its inner structure still observes the usual rules for syntax 
        and may (although of course sometimes may not) engage 
        actively with productive syntactic elements. 
        Thus a structuralist -- as opposed to canonical constructivism -- 
        analysis is still valuable.
\end{itemize}

Elements of a language, in the perspective of Distributed Morphology,
contain Lists A, B, and C, 
which are a list of roots and features, 
a list of how List A is phonetically realized 
(and also covers some syntactic selections that don't have semantic motivation: 
``if A and B meet, they never get appropriately spelt out and the derivation crashes -- no why''),
and the meaning of idiomatic phrases. 
With functional heads being kicked out in a practical language description project, 
the same amount of information needs to be shown in a different matter. 
What we need now are: 
\begin{itemize}
    \item Abstract syntax and ``abstract morphology'':
        facts like that Latin has six cases, 
        that Latin has well-defined subjects and objects, etc. 
        The grammatical concepts are knitted into subcategorization frames,
        each of which waits for a lexical head 
        and some arguments being filled in. 
        This covers the List A and the abstract part of List B 
        (i.e. what structures can be spelt out, 
        without considering what \emph{is} the realization)
    \item How the above is realized: 
        actual nominal and verbal paradigms, 
        productive derivation rules, 
        constituent order, which may involve ``transformation rules''.
        This covers 
        the concrete part of List B. 

        The term \term{transformation rule} is kind of misleading 
        because it's possible -- and even frequent -- 
        that a marked construction can't be obtained by transformation 
        of the canonical construction
        in a natural way, 
        because what really happens is 
        that the two constructions undergo shared stages of syntactic structure building,
        and then diverge from each other,
        and the transformational rule linking the two 
        is only a coarse phenomenological description of the relation between the two 
        (\prettyref{fig:transformation-rule}).
    \item A dictionary, 
        containing roots, how roots are placed into subcategorization frames, 
        and established meanings of complex constructions (all things in List C). 
        The discussion on (synchronic) roots may also involve 
        historical morphology,
        the products of which are syntactically independent roots.
        The last includes some semi-fossilized derivations 
        idioms in the everyday sense, formulaic speech, etc. 
        Usually dictionaries don't really record roots: 
        they record already inflected principal parts,
        from which the whole paradigm can be found.
\end{itemize}
The first two are about the grammar, 
while the third is about the dictionary
or the so-called \concept{lexicon}.

\begin{figure}[H]
    \centering
    \begin{tikzpicture}[x=0.75pt,y=0.75pt,yscale=-0.8,xscale=0.8]
        %uncomment if require: \path (0,386); %set diagram left start at 0, and has height of 386
        
        %Straight Lines [id:da13680161853786577] 
        \draw    (194,152.81) -- (313.26,84.8) ;
        \draw [shift={(315,83.81)}, rotate = 150.31] [fill={rgb, 255:red, 0; green, 0; blue, 0 }  ][line width=0.08]  [draw opacity=0] (12,-3) -- (0,0) -- (12,3) -- cycle    ;
        %Straight Lines [id:da16837784710650316] 
        \draw    (194,152.81) -- (310.33,229.71) ;
        \draw [shift={(312,230.81)}, rotate = 213.47] [fill={rgb, 255:red, 0; green, 0; blue, 0 }  ][line width=0.08]  [draw opacity=0] (12,-3) -- (0,0) -- (12,3) -- cycle    ;
        %Straight Lines [id:da07747924239146564] 
        \draw    (194,152.81) -- (306.85,314.17) ;
        \draw [shift={(308,315.81)}, rotate = 235.03] [fill={rgb, 255:red, 0; green, 0; blue, 0 }  ][line width=0.08]  [draw opacity=0] (12,-3) -- (0,0) -- (12,3) -- cycle    ;
        %Straight Lines [id:da42303822011922976] 
        \draw  [dash pattern={on 4.5pt off 4.5pt}]  (358,201.81) -- (358,98.15) ;
        \draw [shift={(358,203.81)}, rotate = 270] [fill={rgb, 255:red, 0; green, 0; blue, 0 }  ][line width=0.08]  [draw opacity=0] (12,-3) -- (0,0) -- (12,3) -- cycle    ;
        %Straight Lines [id:da4793554387643766] 
        \draw  [dash pattern={on 4.5pt off 4.5pt}]  (358,293.81) -- (358,266.81) ;
        \draw [shift={(358,295.81)}, rotate = 270] [fill={rgb, 255:red, 0; green, 0; blue, 0 }  ][line width=0.08]  [draw opacity=0] (12,-3) -- (0,0) -- (12,3) -- cycle    ;
        
        % Text Node
        \draw (118,134) node [anchor=north west][inner sep=0.75pt]   [align=left] {\begin{minipage}[lt]{44.93pt}\setlength\topsep{0pt}
        \begin{center}
        argument\\structure
        \end{center}
        
        \end{minipage}};
        % Text Node
        \draw (200,36) node [anchor=north west][inner sep=0.75pt]   [align=left] {\begin{minipage}[lt]{39.53pt}\setlength\topsep{0pt}
        \begin{center}
        trivial\\polarity,\\voice,\\etc. 
        \end{center}
        
        \end{minipage}};
        % Text Node
        \draw (328,57) node [anchor=north west][inner sep=0.75pt]   [align=left] {\begin{minipage}[lt]{44.05pt}\setlength\topsep{0pt}
        \begin{center}
        canonical\\clauses
        \end{center}
        
        \end{minipage}};
        % Text Node
        \draw (313,213) node [anchor=north west][inner sep=0.75pt]   [align=left] {\begin{minipage}[lt]{63.87pt}\setlength\topsep{0pt}
        \begin{center}
        some\\non-canonical\\clauses
        \end{center}
        
        \end{minipage}};
        % Text Node
        \draw (315,296) node [anchor=north west][inner sep=0.75pt]   [align=left] {\begin{minipage}[lt]{63.87pt}\setlength\topsep{0pt}
        \begin{center}
        other\\non-canonical\\clauses
        \end{center}
        
        \end{minipage}};
        % Text Node
        \draw (181,240) node [anchor=north west][inner sep=0.75pt]   [align=left] {\begin{minipage}[lt]{45.76pt}\setlength\topsep{0pt}
        \begin{center}
        nontrivial\\polarity,\\voice,\\etc. 
        \end{center}
        
        \end{minipage}};
        % Text Node
        \draw (409,114) node [anchor=north west][inner sep=0.75pt]   [align=left] {\begin{minipage}[lt]{68.79pt}\setlength\topsep{0pt}
        \begin{center}
        transformation\\processes
        \end{center}
        
        \end{minipage}};
        % Text Node
        \draw (412,260) node [anchor=north west][inner sep=0.75pt]   [align=left] {\begin{minipage}[lt]{64.45pt}\setlength\topsep{0pt}
        \begin{center}
        generalization
        \end{center}
        
        \end{minipage}};
        
        
        \end{tikzpicture}
    \caption{What is a transformation rule}
    \label{fig:transformation-rule}        
\end{figure}

\section{About this note}

This note roughly follows the example of 

Of course, it's still possible to carry out grammatical description in different ways.  
For example, generally we shouldn't separate morphology and syntax categorically.
For a heavily inflected language like Latin, however, 
it's an appealing idea 
to start a chapter named ``verbal morphology''
and cover all \acs{tame} marking in it, 
both the abstract concepts and the concrete paradigms. 

\section{Texts}

TODO: classical writers

\section{Remarkable features}

Some typological parameters and peculiar points of Latin are listed here for the impatient.
For a bottom-up overview, 
see \prettyref{chap:pos}.

\subsection{Morphology}

Latin is well known for its rich morphology,
which enables a rather free -- but still not completely arbitrary -- constituent order.
Latin has a clear inflection-derivation distinction.
Despite its richness, 
Latin derivation is largely historical,
with meanings of derived forms 
having shifted and no longer regularly inferrable.
Latin inflection is always suffixal,
while derivation is predominantly prefixal.
Concatenative morphology (affixation and compounding) 
is prominent but isn't the only morphological device:
the following non-concatenative mechanisms are all attested:
\begin{itemize}
    \item \emph{Reduplication}: formation of the perfect stem (TODO: ref)
    \item \emph{Subtraction}: dropping of first-conjugation stem-final vowel (\prettyref{sec:tense-mood-marking}).
    \item \emph{Infixation}:   TODO: ref 
    The imperfect \form{-ba-} is sometimes said to be an infix 
    (as well as its counterparts like \form{-bi-}),
    though it fits in a concatenative picture of verbal morphology.
\end{itemize}
These mechanisms, however, are largely historical,
just like their concatenative counterparts.

Most clausal grammatical categories are marked on the verbal morphology.
Sometimes a grammatical category is there but is not reflected in the morphology.
For example, in English we have infinitive clauses,
but strictly speaking, there is no such thing as ``infinitive verb'':
the head verb of an infinitive clause 
has exactly the same form of a non-third person singular present tense verb.
This is not the case in Latin.
For example, the head verb of a infinitive clause in Latin 
indeed has a separate position in the paradigm.
Thus, grammatical categories of the clause are listed in this section.

\begin{learnbox}{Advices when reading}{reading-advice}
    The morphological richness (and the scrambled constituent order)
    makes Latin hard to read 
    especially for people whose first languages are, say, 
    English or Mandarin. 
    Whenever unsure about a sentence, do the follows: 
    \begin{enumerate}
        \item Skim over the words and label the stems that can be easily recognized.  
        \item Skim over and circle uncontroversial grammatical items,  
            like inflectional endings and prepositions. 
            It's OK to be unable to interpret them immediately (and we need the steps below).
        \item Choose a grammatical item and tentatively give a list of possible features it carries.  
            For example, seeing \form{-v-} in a verb usually means
            it's based on the perfect stem
            (\prettyref{sec:verb-inflection.stem.perfect});
            \form{-um} may be second declension accusative,
            but there are other possibilities
            (\prettyref{tbl:declension-ending-nouns-list}). 
        \item Use constraints like 
            ``the preposition \form{in} licenses the accusative case or the ablative case'' 
            to narrow the possibilities identified above.
        \item Draw unfinished dependency arrows:
            for a verb, draw arrows pointing to the subject and/or the object; 
            for a nominative adjective, draw an arrow pointing to the modified head noun. 
            But note that it's possible that the subject is dropped, 
            or there is no head noun (compare English \form{the poor}).
            Then try to pair the arrows.
    \end{enumerate}
    Repeat the above procedure and finally the sentence can be understood. 
    This procedure is demonstrated in \prettyref{sec:text.vulgate.john}.
\end{learnbox}

\subsection{Lack of determiner}

TODO: other IE languages

\subsection{Alignment}

Latin is a clear nominative-accusative language.
Similar to what is documented in \acs{cgel},
Latin core arguments are coded as subject, object(s),
and copular complements at the level of alignment.
They can be distinguished by the semantic roles,
case marking, possible contents, and transformational properties 
(\prettyref{sec:core-argument-marking}).
In 

\subsection{Peripheral arguments}

There is no serial verb constructions in Latin (\prettyref{sec:clause-combine-abs}),
and thus semantic functions like location or instrument 
are always realized by typical peripheral arguments
attached to the core argument structure.
These peripheral argument positions sometimes can be filled by adverbs,
which also reveals an origin of adverbs: fossilized case forms.

\subsection{Constituent order}\label{sec:constituent-order-abs}

Although traditional Latin grammars focus on the dependency relations 
introduced by morphology,
deeper examination of Latin grammar reveals that 
the dependency relations have a role in determining the constituent order, 
and thus Latin is better described as a discourse-configurational language,
with multiple topicalization and focusing structures.
There are several examples of the link between constituent order and dependency relations.
The negator \form{non} usually appears  before the verb (TODO: or aux?),
and apparent violations seem to be constituent negation 
as opposed to sentential negation \citep[\citepage{43}]{danckaert2017development};
if we subtract sentences in which the OV/VO alternation 
can be alternatively analyzed as topicalization 
and only keep sentences in which the object stays in the \acs{vp} 
(supposing there is a well-defined \acs{vp}), 
then VO frequency no longer shows significant change 
as time went by \citep[\citesec{1.5}, \citepage{29}]{danckaert2017development};
constituency tests also robustly hint at an \acs{vp}.

\begin{theorybox}{Constituency deemphasized in Latin grammar}{constituency-in-grammar}
    This is probably not surprising because
    even the most non-configurational languages show configurationality 
    under scrutiny
    \citep[among others]{niedzielski2017clausal,morris2018evidence,legate2002warlpiri}
    and therefore a thorough disruption 
    of the existing framework of generative syntax seems unnecessary.
    The question then becomes \emph{how} ``free-order'' Latin is.
    Is it closer to a prototypical ``non-configurational'' language, say Warlpiri, 
    or is it closer to Japanese where we have more localized scrambling?
    I will address this question in TODO: ref ,

    For the sake of convenience, even though we know mainstream generative (constituency-based, 
    though the introduction of movements and the structure of Cinque hierarchy
    gives it certain flavor of dependency grammars) approaches make perfect sense for Latin,
    a systematic and thorough description of Latin grammar 
    would be better carried out in a dependency-relation based or \acs{blt}-based way.
    This is of course mostly notational change:
    for example, we only recognize the most ``salient'' types of constituents 
    like \acs{np}s and clauses as constituents in our description,
    and the existence of fine-grained functional projections is covered by 
    additional information like the ``height'' or ``closeness'' of dependency arcs
    corresponding to these functional projections
    (\prettyref{sec:theoretical-orientation}).
\end{theorybox}

\chapter{Phonology and the writing system}

\section{The alphabet}

The most accepted writing system of Latin developed into 
what we call Latin letters -- or the Roman alphabet -- today, 
which is the most widely used writing system in the world.
\concept{Old Italic scripts},
used by Early Old Latin inscriptions 
as well as neighbor languages,
show a larger degree of variation, 
which clearly derived from Greek letters.
The standard Latin alphabet derived from old Italic scripts, 


\chapter{Parts of speech}\label{chap:pos}

\section{Overview}

Latin wordhood is easily defined using phonological criteria,
or, to be more accurate, orthographical criteria: 
what was documented by ancient Romans as a word 
is recognized as a word.
Latin is heavily inflectional,
and the grammatical categories seen in nominal and verbal morphology 
already reflect the most salient grammatical categories
in \acs{np}s and clauses;
words without inflectional morphology (so-called \concept{particles}) 
usually lack any synchronically active internal structures at all.
Thus, we can also say the traditional notion of Latin words 
has morphosyntactic significance,
and this note inherits this traditional and uncontroversial notion of Latin wordhood.

\begin{infobox}{Wordhood}{wordhood}
    Three aspects are involved in wordhood determination:
    syntactic status as a mini constituent 
    (in the terminology of Distributed Morphology, ``categorizer phrase'');
    morphological (or ``post-syntactic'') status 
    as a unit with a fixed template;
    status as a phonological unit surrounded with word boundaries.
    The first aspect largely overlooks 
    case or \acs{tame} affixes:
    they are technically categories within the \acs{np} or the clause, 
    and not categories within the syntactic word.
    The variance of the morphological aspect is 
    constrained by syntactic factors 
    (what is embedded into the morphological realization of the verb 
    can't be grammatical categories in the \acs{np}, etc.).
    The phonological word in principle is orthogonal to 
    the morphological word and the purely syntactic word,
    but in reality they of course have to have correlation
    or otherwise the language will be hard to use in practice.
    
    In some languages like Mandarin, 
    the morphological word almost completely faithfully represents 
    the underlying abstract syntactic word, 
    while the phonological word may have complicated relations with the former.
    In Latin it's quite the opposite: 
    Latin is heavy in inflection, 
    so many grammatical categories with scopes over much larger constituents, 
    like case or tense, 
    ``condense'' unto nouns and verbs, 
    so what a morphological word reflects 
    is the structure of the \acs{np} or clause that surround it;
    but the difference between phonological wordhood 
    and morphological wordhood is small.
    
    Also, note, however, that if we change the definition of a constituent 
    into ``the current \acs{np} or clause minus its already finished complements'',
    as in \prettyref{sec:theoretical-orientation},
    then a Latin morphological word is indeed a mini-constituent.
    So indeed we can say Latin inflected words,
    despite being defined using phonological and morphological standards,
    do bear syntactic significance.
\end{infobox}

Latin word classes can be defined easily via morphology
and these classes prove to have uniform morphosyntactic behaviors.
Non-particle words can be divided into two large classes:
those with similar morphology of prototypical nouns (i.e. \concept{declension}) are \concept{nominals},
while words with similar morphology of prototypical verbs (i.e. \concept{conjugation})
form a uniform class rightfully called \concept{verbs}.
Nominals include \concept{nouns} and \concept{adjectives},
the distinction between the two can also be defined morphologically.

Latin particles include \concept{prepositions}, \concept{adverbs},
\concept{interjections}, and \concept{conjunctions}.
The adverb class and the preposition class have a large overlap:
often a preposition has an intransitive counterpart,
which is similar to a prototypical adverb.
Conjunctions may be seen as ``prepositions for clauses''.
The functions and etymologies of particles are highly diverse.

Latin nouns, verbs, and adjectives are all open categories.
They are able to head constituents,
and so are correlatives (though correlatives can be listed in the grammar).
The preposition class is closed and is a part of the grammar,
just like conjunctions.
However, conjunctions are purely functional,
while certain prepositions may be argued to head attributive expressions:
though prepositions are often said to be markers of a periphrastic case system,
the semantics carried by certain Latin prepositions are too complicated for a case system.
This is also the case of adverbs:
some adverbs seem to be periphrastic markers of \acs{tame} categories
and therefore may be considered as a part of the grammar,
while others seem to carry ``real'' meanings.
\prettyref{fig:latin-word-class} is a visualization of the classification of Latin word classes.

\begin{theorybox}{Lexical and function classes}{word-class-classification}
    By words with ``real category labels'',
    I mean words that have``real'' meanings
    and serve as lexical heads of constituents
    (i.e. being surrounded by function words and dependents).
    Certain adverbs and prepositions have ``real category labels'',
    and they appear at the left side in \prettyref{fig:latin-word-class}.
    Prepositions can be enumerated and therefore are considered as a part of the grammar,
    so they are always at the lower side in \prettyref{fig:latin-word-class}.
    Other adverbs and prepositions are light in their semantic
    and are purely functional,
    so they appear in the southeast corner of \prettyref{fig:latin-word-class}.

    This is not the standard terminology. 
    Linguists use their own notion of \term{lexical class} and \term{function class}
    to cover what I say here. 
\end{theorybox}

\begin{sidewaysfigure}
    \centering
    

\tikzset{every picture/.style={line width=0.3pt}} %set default line width to 0.75pt        

\begin{tikzpicture}[x=0.75pt,y=0.75pt,yscale=-0.8,xscale=0.8]
%uncomment if require: \path (0,674); %set diagram left start at 0, and has height of 674

%Straight Lines [id:da0819119912251447] 
\draw [color={rgb, 255:red, 155; green, 155; blue, 155 }  ,draw opacity=0.2 ]   (109.33,384.91) -- (793.33,384.91) ;
%Rounded Rect [id:dp9835773367494156] 
\draw  [color={rgb, 255:red, 155; green, 155; blue, 155 }  ,draw opacity=1 ][fill={rgb, 255:red, 155; green, 155; blue, 155 }  ,fill opacity=0.2 ] (141.33,256.64) .. controls (141.33,246.48) and (149.57,238.24) .. (159.73,238.24) -- (214.93,238.24) .. controls (225.1,238.24) and (233.33,246.48) .. (233.33,256.64) -- (233.33,518.84) .. controls (233.33,529) and (225.1,537.24) .. (214.93,537.24) -- (159.73,537.24) .. controls (149.57,537.24) and (141.33,529) .. (141.33,518.84) -- cycle ;
%Rounded Rect [id:dp12690257166641228] 
\draw  [color={rgb, 255:red, 155; green, 155; blue, 155 }  ,draw opacity=1 ][fill={rgb, 255:red, 155; green, 155; blue, 155 }  ,fill opacity=0.2 ] (233.33,256.64) .. controls (233.33,246.48) and (241.57,238.24) .. (251.73,238.24) -- (306.93,238.24) .. controls (317.1,238.24) and (325.33,246.48) .. (325.33,256.64) -- (325.33,518.84) .. controls (325.33,529) and (317.1,537.24) .. (306.93,537.24) -- (251.73,537.24) .. controls (241.57,537.24) and (233.33,529) .. (233.33,518.84) -- cycle ;
%Rounded Rect [id:dp4571683339095527] 
\draw  [color={rgb, 255:red, 155; green, 155; blue, 155 }  ,draw opacity=1 ][fill={rgb, 255:red, 155; green, 155; blue, 155 }  ,fill opacity=0.2 ] (349.33,256.57) .. controls (349.33,248.1) and (356.2,241.24) .. (364.67,241.24) -- (426,241.24) .. controls (434.47,241.24) and (441.33,248.1) .. (441.33,256.57) -- (441.33,302.57) .. controls (441.33,311.04) and (434.47,317.91) .. (426,317.91) -- (364.67,317.91) .. controls (356.2,317.91) and (349.33,311.04) .. (349.33,302.57) -- cycle ;
%Rounded Rect [id:dp13945556495695044] 
\draw  [color={rgb, 255:red, 155; green, 155; blue, 155 }  ,draw opacity=1 ][fill={rgb, 255:red, 155; green, 155; blue, 155 }  ,fill opacity=0.2 ] (122.93,223.64) .. controls (122.93,211.82) and (132.51,202.24) .. (144.33,202.24) -- (316.93,202.24) .. controls (328.75,202.24) and (338.33,211.82) .. (338.33,223.64) -- (338.33,537.84) .. controls (338.33,549.66) and (328.75,559.24) .. (316.93,559.24) -- (144.33,559.24) .. controls (132.51,559.24) and (122.93,549.66) .. (122.93,537.84) -- cycle ;
%Straight Lines [id:da9866800561526903] 
\draw [color={rgb, 255:red, 155; green, 155; blue, 155 }  ,draw opacity=0.2 ]   (595.33,577.91) -- (595.33,184.91) ;
%Rounded Rect [id:dp9919662328975185] 
\draw  [color={rgb, 255:red, 155; green, 155; blue, 155 }  ,draw opacity=1 ][fill={rgb, 255:red, 155; green, 155; blue, 155 }  ,fill opacity=0.2 ] (435,431.44) .. controls (435,427.46) and (438.22,424.24) .. (442.2,424.24) -- (688.8,424.24) .. controls (692.78,424.24) and (696,427.46) .. (696,431.44) -- (696,453.04) .. controls (696,457.02) and (692.78,460.24) .. (688.8,460.24) -- (442.2,460.24) .. controls (438.22,460.24) and (435,457.02) .. (435,453.04) -- cycle ;
%Rounded Rect [id:dp04492990660366303] 
\draw  [color={rgb, 255:red, 155; green, 155; blue, 155 }  ,draw opacity=1 ][fill={rgb, 255:red, 155; green, 155; blue, 155 }  ,fill opacity=0.2 ] (147,482.31) .. controls (147,476.75) and (151.51,472.24) .. (157.07,472.24) -- (579.93,472.24) .. controls (585.49,472.24) and (590,476.75) .. (590,482.31) -- (590,512.51) .. controls (590,518.07) and (585.49,522.57) .. (579.93,522.57) -- (157.07,522.57) .. controls (151.51,522.57) and (147,518.07) .. (147,512.51) -- cycle ;
%Shape: Path Data [id:dp3209513602067542] 
\draw  [color={rgb, 255:red, 155; green, 155; blue, 155 }  ,draw opacity=1 ][fill={rgb, 255:red, 155; green, 155; blue, 155 }  ,fill opacity=0.2 ] (478.97,255.24) -- (552.94,255.24) .. controls (557.16,255.24) and (560.58,258.46) .. (560.58,262.43) -- (560.58,367.51) .. controls (560.58,372.79) and (565.13,377.07) .. (570.74,377.07) -- (630.37,377.07) .. controls (630.59,377.07) and (630.82,377.07) .. (631.04,377.05) -- (712.31,377.05) .. controls (717.01,377.05) and (720.81,380.63) .. (720.81,385.05) -- (720.81,434.24) .. controls (720.81,438.66) and (717.01,442.24) .. (712.31,442.24) -- (478.97,442.24) .. controls (474.75,442.24) and (471.33,439.02) .. (471.33,435.05) -- (471.33,262.43) .. controls (471.33,258.46) and (474.75,255.24) .. (478.97,255.24) -- cycle ;
%Rounded Rect [id:dp9478936078417666] 
\draw  [color={rgb, 255:red, 155; green, 155; blue, 155 }  ,draw opacity=1 ][fill={rgb, 255:red, 155; green, 155; blue, 155 }  ,fill opacity=0.2 ] (451.33,237.22) .. controls (451.33,228.95) and (458.04,222.24) .. (466.31,222.24) -- (759.35,222.24) .. controls (767.63,222.24) and (774.33,228.95) .. (774.33,237.22) -- (774.33,552.26) .. controls (774.33,560.53) and (767.63,567.24) .. (759.35,567.24) -- (466.31,567.24) .. controls (458.04,567.24) and (451.33,560.53) .. (451.33,552.26) -- cycle ;

% Text Node
\draw (169,267) node [anchor=north west][inner sep=0.75pt]   [align=left] {noun};
% Text Node
\draw (252,267) node [anchor=north west][inner sep=0.75pt]   [align=left] {adjective};
% Text Node
\draw (158,479) node [anchor=north west][inner sep=0.75pt]   [align=left] {personal \\pronoun};
% Text Node
\draw (244,479) node [anchor=north west][inner sep=0.75pt]   [align=left] {correlative \\pronoun};
% Text Node
\draw (368,266) node [anchor=north west][inner sep=0.75pt]   [align=left] {verb};
% Text Node
\draw (565.5,442.24) node   [align=left] {preposition};
% Text Node
\draw (674,517) node [anchor=north west][inner sep=0.75pt]   [align=left] {conjunction};
% Text Node
\draw (492,264) node [anchor=north west][inner sep=0.75pt]   [align=left] {adverb};
% Text Node
\draw (674,471) node [anchor=north west][inner sep=0.75pt]   [align=left] {interjection};
% Text Node
\draw (151,361) node [anchor=north west][inner sep=0.75pt]  [color={rgb, 255:red, 155; green, 155; blue, 155 }  ,opacity=1 ] [align=left] {noun \\morphology};
% Text Node
\draw (238.14,361) node [anchor=north west][inner sep=0.75pt]  [color={rgb, 255:red, 155; green, 155; blue, 155 }  ,opacity=1 ] [align=left] {adjective \\morphology};
% Text Node
\draw (107.33,384.91) node [anchor=east] [inner sep=0.75pt]  [color={rgb, 255:red, 155; green, 155; blue, 155 }  ,opacity=1 ] [align=left] {with real \\category \\label};
% Text Node
\draw (795.33,384.91) node [anchor=west] [inner sep=0.75pt]  [color={rgb, 255:red, 155; green, 155; blue, 155 }  ,opacity=1 ] [align=left] {with \\no real \\category \\label};
% Text Node
\draw (141,212.24) node [anchor=north west][inner sep=0.75pt]  [color={rgb, 255:red, 155; green, 155; blue, 155 }  ,opacity=1 ] [align=left] {nominal};
% Text Node
\draw (595.33,181.91) node [anchor=south] [inner sep=0.75pt]  [color={rgb, 255:red, 155; green, 155; blue, 155 }  ,opacity=1 ] [align=left] {not a part\\of grammar};
% Text Node
\draw (595.33,580.91) node [anchor=north] [inner sep=0.75pt]  [color={rgb, 255:red, 155; green, 155; blue, 155 }  ,opacity=1 ] [align=left] {a part of\\grammar};
% Text Node
\draw (499,489.5) node [anchor=north west][inner sep=0.75pt]   [align=left] {pro-adverb};
% Text Node
\draw (379,489.5) node [anchor=north west][inner sep=0.75pt]  [color={rgb, 255:red, 155; green, 155; blue, 155 }  ,opacity=1 ] [align=left] {pro-forms};
% Text Node
\draw (661,241.5) node [anchor=north west][inner sep=0.75pt]  [color={rgb, 255:red, 155; green, 155; blue, 155 }  ,opacity=1 ] [align=left] {particle};


\end{tikzpicture}

    \caption{Latin word classes}
    \label{fig:latin-word-class}
\end{sidewaysfigure}

Articles (English \form{a} or \form{the}), 
despite prevalent in other Indo-European languages,
are missing in Latin.
This, together with the fact that Classical Sanskrit and Old Persian didn't have articles 
and the Slavic languages still don't,
is a strong indicator that \ac{pie} didn't have articles. 
Note that the fact that Latin lacks articles 
doesn't mean the determiner syntactic function doesn't exist:
there are evidences suggesting certain aspects of the behavior of Latin \acs{np}s 
are just like English \citep{giusti2014split}.

\section{Nouns}

The structure of nouns and their phrases is introduced in \prettyref{sec:np}.
They are declined for case and number (\prettyref{sec:regular-noun-declension}),
and the features also spread to other nominals in the \acs{np} by agreement.
According to their meanings and ability to license \acs{np} dependents, 
Latin nouns can be classified into TODO 

\chapter{Noun and noun phrase}\label{sec:np}

\section{The structure of noun}

\section{Declension of regular nouns}\label{sec:regular-noun-declension}

\subsection{The paradigms}

Latin nouns are declined for case and number,
which is agreed upon by other nominal words in the \acs{np}.
There are five declension classes in Latin.
The paradigms of each is well documented in \citet[\citepage{17}]{allen1903allen}.
The full list of attested noun endings is \prettyref{tbl:declension-ending-nouns-list}.
The list is still not the full picture of Latin nominal inflection:
stem alternation is seen in the third declension
(\prettyref{sec:np.inflection.3}).

\begin{table}[H]
    \caption{Declension endings; Roman numerals are declension classes}
    \label{tbl:declension-ending-nouns-list}
    \centering
    \begin{tabular}{ll}
    \toprule
    ending & declension \\ \midrule
    -a & I, \category{sg}.\category{nom}, \category{sg}.\category{voc}; IIN, \category{pl}.\category{nom}, \category{pl}.\category{acc}, \category{pl}.\category{voc}; IIIN, \category{pl}.\category{nom}, \category{pl}.\category{acc}, \category{pl}.\category{voc} \\ 
    -ā & I, \category{sg}.\category{abs} \\ 
    -ae & I, \category{sg}.\category{gen}, \category{sg}.\category{dat}, \category{pl}.\category{nom}, \category{pl}.\category{voc};  \\ 
    -am & I, \category{sg}.\category{acc} \\ 
    -ārum & I, \category{pl}.\category{gen} \\ 
    -ās & I, \category{pl}.\category{acc} \\ 
    -e & IIM, \category{sg}.\category{voc}; IIIFMN, \category{sg}.\category{abs} \\ 
    -ē & V, \category{sg}.\category{abs} \\ 
    -ei/-ēi & V, \category{sg}.\category{gen}, \category{sg}.\category{dat} \\ 
    -em & IIIFM, \category{sg}.\category{acc}; V, \category{sg}.\category{acc} \\ 
    -ēbus & V, \category{pl}.\category{dat}, \category{pl}.\category{abs} \\ 
    -ērum & V, \category{pl}.\category{gen} \\ 
    -ēs & IIIFM, \category{pl}.\category{nom}, \category{pl}.\category{acc}, \category{pl}.\category{voc}; V, \category{sg}.\category{nom}, \category{sg}.\category{voc}, \category{pl}.\category{nom}, \category{pl}.\category{acc}, \category{pl}.\category{voc} \\ 
    -ī & IIM, \category{sg}.\category{gen}, \category{sg}.\category{voc}, \category{pl}.\category{nom}, \category{pl}.\category{voc}; IIN, \category{sg}.\category{gen}; IIIFMN, \category{sg}.\category{dat} \\ 
    -ibus & IIIFMN, \category{pl}.\category{dat}, \category{pl}.\category{abs}; IVFMN, \category{pl}.\category{dat}, \category{pl}.\category{abs} \\ 
    -is & IIIFMN, \category{sg}.\category{gen} \\ 
    -īs & I, \category{pl}.\category{dat}, \category{pl}.\category{abs}; IIMN, \category{pl}.\category{dat}, \category{pl}.\category{abs} \\ 
    -ō & IIMN, \category{sg}.\category{dat}, \category{sg}.\category{abs} \\ 
    -ōs & IIM, \category{pl}.\category{acc} \\ 
    -ōrum & IIMN, \category{pl}.\category{gen} \\ 
    -r & IIM, \category{sg}.\category{nom}, \category{sg}.\category{voc} \\ 
    -ū & IVFM, \category{sg}.\category{abs}; IVN, \category{sg}.\category{nom}, \category{sg}.\category{dat}, \category{sg}.\category{acc}, \category{sg}.\category{abs}, \category{sg}.\category{voc} \\ 
    -ua & IVN, \category{pl}.\category{nom}, \category{pl}.\category{acc}, \category{pl}.\category{voc} \\ 
    -uī & IVFM, \category{sg}.\category{dat} \\ 
    -um & IIMN, \category{sg}.\category{acc}; IIN, \category{sg}.\category{nom}, \category{sg}.\category{voc}; IIIFMN, \category{pl}.\category{gen}; IVFM, \category{sg}.\category{acc} \\ 
    -us & IIM, \category{sg}.\category{nom}; IVFM, \category{sg}.\category{nom}, \category{sg}.\category{voc} \\ 
    -ūs & IVFM, \category{sg}.\category{gen}, \category{pl}.\category{nom}, \category{pl}.\category{acc}, \category{pl}.\category{voc}; IVN, \category{sg}.\category{gen} \\ 
    -uum & IVFMN, \category{pl}.\category{gen} \\ \bottomrule
\end{tabular}
\end{table}

\begin{learnbox}{Frequent confusions when analyzing noun endings}{noun-ending-confusion}
    The ending sequence \form{-io} can'be be found in \prettyref{tbl:declension-ending-nouns-list}
    and we may hurry to the conclusion 
    that it's the third declension abstract noun ending \form{-io} 
    in the nominative or accusative case.
    Not necessarily -- 
    it can also be \form{-ium} in the dative or ablative case 
    (when the macron symbol for long vowels are not used).
\end{learnbox}

\subsubsection{The third declension}\label{sec:np.inflection.3}

The third declension is a big tent containing several subclasses.
Nominative singular endings attested in the third declension
include \form{-s}, \form{-t}, \form{-x} (i.e. \form{-cs}) \citep[\citesec{53}]{allen1903allen}.

\section{The structure of the noun phrase}

Although Latin lacks the article 

\subsection{Attributives}

This section only discusses adjective or numeral attributives in detail.
For in-depth discussion of relative clauses, see \prettyref{sec:relative-clause}.


\subsection{The possessive construction}

\subsection{Numerals in the noun phrase}

\section{Pronouns}

Latin pronouns are complete \acs{np} themselves:
no attributives should be attached to them. 
Pronouns are declined for case, gender and number,
and they also can be governed by prepositions.

\section{Case and preposition}

After discussing the internal structure of \acs{np}s, 
we investigate the interaction of \acs{np}s 
with their syntactic functions.
This is done by two systems: 
the case system and the preposition system.


\subsection{Distribution of the cases}\label{sec:np.case-distribution}

The roles of the five cases are not symmetric.
Being nominative simply means being the subject in a finite clause or something agreeing to it 
and nothing else: 
the subject may be in a passive clause and is not agentive at all.
The nominative case and the accusative case received by the direct object 
are \emph{structural} cases: 
they are purely decided by the syntactic environment 
and don't have much semantic significance.

On the other hand, the rest cases are \emph{inherent} cases:
they are similar to prepositions, having semantic interpretations
-- ``source'' or ``target'' or \dots -- themselves,
and once an inherent case is assigned to an \acs{np},
the latter is ``sealed'' just like a prepositional phrase:
the change of the outside syntactic environment 
doesn't change anything inside.

\subsubsection{The nominative}

\subsubsection{The accusative}\label{sec:accusative-distribution}

\subsubsection{The dative}

The dative case is assigned to the indirect object (\prettyref{sec:vp.complement.indirect-object})

\subsection{Prepositions}



\chapter{Adjectives}

The adjective class and the adverb class are linked together by several factors:
the adjective phrase and the adverb phrase are both prototypical modifiers,
often with parallel structures;
they both have the category of degree; 
adverbs can be formed regularly from adjectives.

\section{Declension of regular adjectives}

Peripheral arguments may also be regarded as adverbials.
This chapter, however, is mainly about mean, TODO

\section{Arguments of adjectives}

TODO: case forms with adjectives


\form{tam quam}

TODO: structure of some prepositions


\chapter{Verb inflection}

\section{The finite paradigm}

\subsection{The verb template}

Some inflected Latin verbs and their parts are shown in \prettyref{fig:latin-verb}.
Traditionally, the verb is divided 
into the \concept{stem} and the \concept{ending}.
Derivation in Latin is predominantly preverbal,
and hence the conjugation is mostly about the final lexical morpheme in the verb stem,
which is represented as the root in \prettyref{fig:latin-verb}.
There may be a perfect suffix after the root.
Components of the verb ending include 
the \concept{tense and mood marker},
and the person, number and voice marker,
which is called the \concept{personal ending} here, 
following the terminology in \citet[\citesec{165}]{allen1903allen}.

\begin{figure}[H]
    \centering
    

\tikzset{every picture/.style={line width=0.3pt}} %set default line width to 0.75pt        

\begin{tikzpicture}[x=0.75pt,y=0.75pt,yscale=-0.85,xscale=0.85]
%uncomment if require: \path (0,414); %set diagram left start at 0, and has height of 414

%Shape: Rectangle [id:dp7661505767193904] 
\draw  [color={rgb, 255:red, 74; green, 144; blue, 226 }  ,draw opacity=1 ] (200,155) -- (278.01,155) -- (278.01,201.48) -- (200,201.48) -- cycle ;

%Shape: Rectangle [id:dp9235304615883395] 
\draw  [color={rgb, 255:red, 74; green, 144; blue, 226 }  ,draw opacity=1 ] (51,155) -- (187.01,155) -- (187.01,201.48) -- (51,201.48) -- cycle ;

%Shape: Rectangle [id:dp4025769445021785] 
\draw  [color={rgb, 255:red, 74; green, 144; blue, 226 }  ,draw opacity=1 ] (290,155) -- (416.01,155) -- (416.01,201.48) -- (290,201.48) -- cycle ;

%Shape: Rectangle [id:dp9180138973853116] 
\draw  [color={rgb, 255:red, 80; green, 227; blue, 194 }  ,draw opacity=1 ] (428,155) -- (564.01,155) -- (564.01,201.48) -- (428,201.48) -- cycle ;

%Shape: Rectangle [id:dp2406875282163703] 
\draw  [color={rgb, 255:red, 80; green, 227; blue, 194 }  ,draw opacity=1 ] (575,155) -- (722.01,155) -- (722.01,201.48) -- (575,201.48) -- cycle ;


% Text Node
\draw (239.01,178.24) node   [align=left] {core stem};
% Text Node
\draw (119.01,178.24) node   [align=left] {\begin{minipage}[lt]{87.06pt}\setlength\topsep{0pt}
\begin{center}
derivation \\prefix/compouding
\end{center}

\end{minipage}};
% Text Node
\draw (353.01,178.24) node   [align=left] {stem-final vowel};
% Text Node
\draw (496.01,178.24) node   [align=left] {\begin{minipage}[lt]{83.46pt}\setlength\topsep{0pt}
\begin{center}
tense and mood \\marking
\end{center}

\end{minipage}};
% Text Node
\draw (648.51,178.24) node   [align=left] {\begin{minipage}[lt]{84.5pt}\setlength\topsep{0pt}
\begin{center}
person, number, \\and voice marking
\end{center}

\end{minipage}};
% Text Node
\draw (181,228) node [anchor=north west][inner sep=0.75pt]  [color={rgb, 255:red, 74; green, 144; blue, 226 }  ,opacity=1 ] [align=left] {verb stem};
% Text Node
\draw (532.01,228) node [anchor=north west][inner sep=0.75pt]  [color={rgb, 255:red, 80; green, 227; blue, 194 }  ,opacity=1 ] [align=left] {verb ending};


\end{tikzpicture}

    \caption{The template of Latin verbs.
    Indentation means linear order and not necessarily constituency structure.}
    \label{fig:latin-verb}
\end{figure}

The core stem assumes semi-regular alternations.
The perfect marker is often but not always \form{-v-} (\prettyref{sec:verb-inflection.stem.perfect}).
Although the aspect marker is a part of the \acs{tam} marking
(\prettyref{sec:verb-inflection.finite-template.tame}),
it somehow is much more closely attached to the root 
in the morphophonological realization of the verbal system:
how it is realized is not completely predictable,
and therefore the two finite stem forms -- 
the present stem and the perfect stem -- 
are both required for complete characterization of the paradigm of a regular verb.
To fully decide the paradigm 
we still need a further stem, the supine stem
(\prettyref{sec:three-latin-stem}).

Contextual allomorphs also exist for other morphemes in \prettyref{fig:latin-verb}
\citep[\citepage{11}]{embick2005status},
as is shown in \prettyref{tbl:latin-finite-verbs}.
The most salient change is that 
a vowel may be inserted between the root and the aspect suffix,
which is also known as the \concept{thematic vowel}.
It is the residue of the \ac{pie} ablaut
and labels the conjugation class.
The tense and mood marker is influenced by the existence of the perfect suffix.
The personal ending is also influenced by what precedes it,
as in the the \form{-mus} and \form{-imus} alternation.

\begin{sidewaystable}
    \centering
    \caption{Examples of Latin finite verbs}
    \label{tbl:latin-finite-verbs}
    \begin{tabular}{lllllll}
    \toprule
    \multirow{3}{*}{verb form} & \multicolumn{3}{c}{stem}                                           & \multicolumn{1}{c}{\multirow{3}{*}{tense and mood}} & \multicolumn{2}{c}{\multirow{2}{*}{personal ending}} \\ \cline{2-4}
                               & \multicolumn{1}{c}{extended root} &                & aspect marker & \multicolumn{1}{c}{}                                & \multicolumn{2}{c}{}                                 \\ \cline{6-7}
                               & extended root                     & thematic vowel &               & \multicolumn{1}{c}{}                                & ``thematic vowel''         & personal ending         \\ \midrule
    \corpus{amō}               & \corpus{am}                       & \corpus{}      & \corpus{}     & \corpus{}                                           & \corpus{}                  & \corpus{ō}              \\
    \corpus{laudāmus}          & \corpus{laud}                     & \corpus{ā}     & \corpus{}     & \corpus{}                                           & \corpus{}                  & \corpus{mus}            \\
    \corpus{olēvimus}          & \corpus{ol}                       & \corpus{ē}     & \corpus{v}    & \corpus{}                                           & \corpus{i}                 & \corpus{mus}            \\
    \corpus{amāveris}          & \corpus{am}                       & \corpus{ā}     & \corpus{v}    & \corpus{eri}                                        & \corpus{}                  & \corpus{s}              \\ \bottomrule
    \end{tabular}    
\end{sidewaystable}

Below I discuss the subsystems in \prettyref{fig:latin-verb}.

\subsection{Voice}

Latin doesn't have rich valency changing devices:
there is only one clause-wide valency decreasing device -- passivization -- 
and there is no valency increasing device.
Causative constructions are realized by complement clauses,
not any change in the argument structure.
Whether passivization happens is recorded by the category of \concept{voice}.
A verb (and hence the clause headed by it) is therefore either in \concept{active voice},
or in \concept{passive voice}.

\begin{theorybox}{Valency changing}{valency-changing}
    From a generative perspective, some languages realize valency changing 
    by a series of \vP{} structures, and then the case assignment of the arguments is trivial.
    Some languages use non-trivial structural case assignment mechanisms
    to achieve valency changing 
    (``suppressing the agent argument, 
    and leave the nominative probe to find the subject;
    the probe then has to choose the patient argument'').
    Of course, \vP{} changes in the second type are still there,
    which may be a likely source of relevant verb morphology.
    Naturally, the second group of languages have more restricted valency changing devices;
    this is the case of Latin.
\end{theorybox}

\subsection{\acs{tame} categories}\label{sec:verb-inflection.finite-template.tame}

Latin has fused tense and aspect:
the composition of three tense values and three aspect values 
gives nine options,
but in Latin, there are only six morphologically distinguished options,
as is shown in \prettyref{tbl:latin-tense-aspect}. 
When people talk about \concept{tense} in Latin (and in many other Indo-European languages),
they are often taking about things like the six options,
instead of the past/present/future system.
The imperfective/perfective distinction 
(for example the \form{be doing} construction in English)
is not syntactically coded in Latin.

\begin{table}[H]
    \caption{Latin tense and aspect}
    \label{tbl:latin-tense-aspect}
    \centering
    \begin{tabular}{@{}cccc@{}}
    \toprule
              & past       & present                  & future                  \\ \midrule
    imperfect & imperfect  & \multirow{2}{*}{present} & \multirow{2}{*}{future} \\
    simple    & perfect    &                          &                         \\
    perfect   & pluperfect & perfect                  & future perfect          \\ \bottomrule
\end{tabular}    
\end{table}

\begin{infobox}{Mismathc between \ac{tame} constructions and fine-grained categories}{tame}
    Note that semantic \ac{tame} features are different from 
    syntactic \ac{tame} features,
    and the two are in turn different from packaged \ac{tame} marking constructions
    that can be easily identified in surface-orientated analyses.
    This is illustrated in \prettyref{tbl:latin-tense-aspect}.
    Following the example in \citet{grimm2021grammar},
    in this note, I use small capitals for the names of attested surface realizations of \ac{tame}
    and the default font for \ac{tame} values.
    (Some other grammars, like \citet{jacques2021grammar,friesen2017grammar}, 
    use initial capitals for the former.)
\end{infobox}

Similar fusion between categories is shown in the category of \concept{mood}.
It's the fusion of morphologically marked clause type 
(declarative and imperative)
and morphologically marked modality.
The verb morphology of interrogative clauses is exactly the same as declarative clauses:
the interrogative clause type is marked by the existence of interrogative \term{pro}-forms.
Thus, there are three moods in finite clauses in Latin:
\acl{indicative}, \acl{subjunctive}, and \acl{imperative}.
The \acl{indicative} is the fusion of 
the declarative/interrogative clause type and the realis modality.
The \acl{subjunctive} mood is the fusion of 
the declarative/interrogative clause type and the irrealis modality.
The \acl{imperative} is basically the imperative clause type:
it doesn't allow modality marking.
Sometimes people say the infinitive is the fourth mood,
though it's a non-finite clause.

\begin{infobox}{The term \term{mood}}{mood}
    \acs{blt} only calls the first category \term{mood}.
    Different linguists use the term \term{mood} and \term{modality} in radically different ways.
    In this note I just focus on the common practice in Latin grammar study.
\end{infobox}

\subsection{Agreement}\label{sec:agreement-abs}

Latin is a typical nominative-accusative language,
both morphologically and syntactically.
In finite clauses, 
there is subject-verb agreement:
the number and person of the subject is marked on the main verb.
In the case of periphrastic conjugation,
the features are marked on the copula.

\subsection{Compatability of categories}

There is no \acl{future} tense and \acl{future perfect} tense in subjunctive clauses,
probably for the semantic reason
that the future tense already contains certain sense of modality
(an event predicted to happen),
and thus is not compatible with the \acl{subjunctive} mood.
The \acl{imperative} mood is not compatible with other \ac{tame} markings
except the \acl{present} tense and the \acl{future} tense.
It's still compatible with the voice category,
and allowed persons are 
second person singular/plural with the \acl{present} tense,
and second/third person singular/plural with the \acl{future} tense.
The absence of first person is also probably from semantic origin.

In conclusion, the categories involved in the finite verb paradigm of Latin 
are shown in \prettyref{fig:paradigm-finite-verb}.
Here mood and tense are realized in one morpheme,
and voice, person and number are realized in one morpheme.
The paradigm is realized synthetically in all circumstances 
except in passive voice and perfect tense.
In that case, the verb conjugation is realized like the English passive,
i.e. via a copula and the perfect passive participle.

\begin{figure}[H]
    \centering
    

\tikzset{every picture/.style={line width=0.75pt}} %set default line width to 0.75pt        

\begin{tikzpicture}[x=0.75pt,y=0.75pt,yscale=-0.8,xscale=0.8]
%uncomment if require: \path (0,560); %set diagram left start at 0, and has height of 560

%Straight Lines [id:da4771979505475994] 
\draw [color={rgb, 255:red, 208; green, 2; blue, 27 }  ,draw opacity=1 ][line width=2.25]    (72,476.59) -- (165,476.59) ;
%Shape: Rectangle [id:dp4651809120211663] 
\draw  [draw opacity=0][fill={rgb, 255:red, 208; green, 2; blue, 27 }  ,fill opacity=0.1 ] (72,51.93) -- (165,51.93) -- (165,195.59) -- (72,195.59) -- cycle ;
%Shape: Rectangle [id:dp8674436914111818] 
\draw  [draw opacity=0][fill={rgb, 255:red, 245; green, 166; blue, 35 }  ,fill opacity=0.1 ] (208,51.93) -- (324,51.93) -- (324,195.59) -- (208,195.59) -- cycle ;
%Straight Lines [id:da9097932489630769] 
\draw [color={rgb, 255:red, 245; green, 166; blue, 35 }  ,draw opacity=1 ][line width=2.25]    (210,476.59) -- (324,476.59) ;
%Shape: Rectangle [id:dp6458311390526075] 
\draw  [draw opacity=0][fill={rgb, 255:red, 245; green, 166; blue, 35 }  ,fill opacity=0.1 ] (208,213.93) -- (324,213.93) -- (324,317.59) -- (208,317.59) -- cycle ;
%Shape: Rectangle [id:dp4418485190624626] 
\draw  [draw opacity=0][fill={rgb, 255:red, 208; green, 2; blue, 27 }  ,fill opacity=0.1 ] (73,213.93) -- (166,213.93) -- (166,318.26) -- (73,318.26) -- cycle ;
%Shape: Rectangle [id:dp7155013561532602] 
\draw  [draw opacity=0][fill={rgb, 255:red, 245; green, 166; blue, 35 }  ,fill opacity=0.1 ] (208,341.59) -- (324,341.59) -- (324,373.59) -- (208,373.59) -- cycle ;
%Shape: Rectangle [id:dp5109073549880265] 
\draw  [draw opacity=0][fill={rgb, 255:red, 245; green, 166; blue, 35 }  ,fill opacity=0.1 ] (208,387.59) -- (324,387.59) -- (324,419.59) -- (208,419.59) -- cycle ;
%Shape: Rectangle [id:dp4978185928249894] 
\draw  [draw opacity=0][fill={rgb, 255:red, 208; green, 2; blue, 27 }  ,fill opacity=0.1 ] (73,341.59) -- (166,341.59) -- (166,420.93) -- (73,420.93) -- cycle ;
%Shape: Rectangle [id:dp7974586880324295] 
\draw  [draw opacity=0][fill={rgb, 255:red, 126; green, 211; blue, 33 }  ,fill opacity=0.1 ] (364,52.93) -- (471.33,52.93) -- (471.33,419.59) -- (364,419.59) -- cycle ;
%Straight Lines [id:da40576537799721035] 
\draw [color={rgb, 255:red, 126; green, 211; blue, 33 }  ,draw opacity=1 ][line width=2.25]    (364,476.59) -- (470,476.59) ;
%Shape: Rectangle [id:dp37304865952309374] 
\draw  [draw opacity=0][fill={rgb, 255:red, 80; green, 227; blue, 194 }  ,fill opacity=0.1 ] (515,51.93) -- (631,51.93) -- (631,318.93) -- (515,318.93) -- cycle ;
%Shape: Rectangle [id:dp09424442237743991] 
\draw  [draw opacity=0][fill={rgb, 255:red, 80; green, 227; blue, 194 }  ,fill opacity=0.1 ] (514,341.59) -- (630,341.59) -- (630,373.59) -- (514,373.59) -- cycle ;
%Shape: Rectangle [id:dp7879043441622926] 
\draw  [draw opacity=0][fill={rgb, 255:red, 80; green, 227; blue, 194 }  ,fill opacity=0.1 ] (514,387.59) -- (630,387.59) -- (630,419.59) -- (514,419.59) -- cycle ;
%Straight Lines [id:da6694037096709569] 
\draw [color={rgb, 255:red, 80; green, 227; blue, 194 }  ,draw opacity=1 ][line width=2.25]    (514,476.59) -- (628,476.59) ;
%Shape: Rectangle [id:dp8356914669291959] 
\draw  [draw opacity=0][fill={rgb, 255:red, 74; green, 144; blue, 226 }  ,fill opacity=0.1 ] (675,52.93) -- (782.33,52.93) -- (782.33,419.59) -- (675,419.59) -- cycle ;
%Straight Lines [id:da18180972606397305] 
\draw [color={rgb, 255:red, 74; green, 144; blue, 226 }  ,draw opacity=1 ][line width=2.25]    (676.33,476.59) -- (782.33,476.59) ;

% Text Node
\draw (118.5,123.76) node   [align=left] {indicative};
% Text Node
\draw (119.5,266.09) node   [align=left] {subjunctive};
% Text Node
\draw (266,123.76) node   [align=left] {present/\\imperfect/\\future/\\perfect/\\pluperfect/\\future perfect};
% Text Node
\draw (266,265.76) node   [align=left] {present/\\imperfect/\\perfect/\\pluperfect};
% Text Node
\draw (417.67,236.26) node   [align=left] {active/\\passive};
% Text Node
\draw (573,185.43) node   [align=left] {1/\\2/\\3};
% Text Node
\draw (728.67,236.26) node  [color={rgb, 255:red, 0; green, 0; blue, 0 }  ,opacity=1 ] [align=left] {single/\\plural};
% Text Node
\draw (119.5,381.26) node   [align=left] {imperative};
% Text Node
\draw (266,357.59) node   [align=left] {present};
% Text Node
\draw (266,403.59) node   [align=left] {future};
% Text Node
\draw (572,357.59) node   [align=left] {2};
% Text Node
\draw (572,403.59) node   [align=left] {2/3};
% Text Node
\draw (118.5,479.59) node [anchor=north] [inner sep=0.75pt]  [color={rgb, 255:red, 208; green, 2; blue, 27 }  ,opacity=1 ] [align=left] {mood};
% Text Node
\draw (267,479.59) node [anchor=north] [inner sep=0.75pt]  [color={rgb, 255:red, 245; green, 166; blue, 35 }  ,opacity=1 ] [align=left] {\textcolor[rgb]{0.96,0.65,0.14}{tense}};
% Text Node
\draw (417,479.59) node [anchor=north] [inner sep=0.75pt]  [color={rgb, 255:red, 126; green, 211; blue, 33 }  ,opacity=1 ] [align=left] {voice};
% Text Node
\draw (571,479.59) node [anchor=north] [inner sep=0.75pt]  [color={rgb, 255:red, 80; green, 227; blue, 194 }  ,opacity=1 ] [align=left] {\textcolor[rgb]{0.31,0.89,0.76}{person}};
% Text Node
\draw (729.33,479.59) node [anchor=north] [inner sep=0.75pt]  [color={rgb, 255:red, 74; green, 144; blue, 226 }  ,opacity=1 ] [align=left] {number};


\end{tikzpicture}

    \caption{Categories in the finite paradigm}
    \label{fig:paradigm-finite-verb}
\end{figure}

\begin{infobox}{Recording verb inflectional forms}{conjugation-form}
    Different people use the term \term{verb forms} -- and count them -- in different ways.
    The most generous -- and the most syntactically relevant -- way 
    is to view the realization of every possible CP-TP-\vP{} projection 
    as a form of the main verb -- the verb root at the core of the CP-TP-\vP{} domains.
    This results in a paradigm in traditional grammar, 
    essentially the traditional way to enumerate Latin verb forms 
    (``the indicative active present second person form of a verb is \dots'').

    The problem with this approach is sometimes two cells in the paradigm are always identical.
    In this way, morphosyntactically there are indeed two different paradigm cells,
    but morphophonologically there is only one verb form.
    Take English as an example: 
    a traditional grammar may say 
    ``the present subjunctive first person singular of English \form{take} is \form{take}''. 
    The problem here is the present subjunctive first person singular \emph{clause}
    always contains the same form of the \emph{verb}
    with a present indicative first person singular clause,
    so it makes no sense to talk about ``the present subjunctive first person singular \emph{verb form}''.
    A linguist stingy with the number of verb forms 
    may then stipulate that conjugation forms are literally about \emph{forms},
    and thus there is no such thing as ``the subjunctive form'' of English verbs,
    because in subject \emph{clauses}, 
    the main verb always has the same form as the infinitive
    \citep[\citepage{76}]{cgel}.

    Another problem with this approach occurs 
    when dealing with languages like Japanese.
    There are so many suffix slots,
    and the boundaries between suffixes are relatively clear,
    so the paradigm is too big to be displayed as a whole
    and too regular to be enumerated cell by cell.
    In this case, recording suffixes may be a better choice.

    The analysis of conjugation forms of the verb, theoretically speaking,
    is more about vocabulary insertion and readjustment rules,
    instead of the syntax proper.
    This is an instance of the \emph{separation principle}:
    morphophonological features can be separated from morphosyntactic features
    \citep{embick2000features}.
    Distinguishing between verb forms and clause categories isn't just a game about wording:
    in periphrastic conjugation,
    we have auxiliary verb(s) plus a non-finite verb form,
    but here the non-finite verb form is just the spellout of several features together with the verb root 
    and is definitely not thea head of a non-finite \emph{clause}:
    what we have here is one clause, not clause embedding.
    Thus it makes no sense to say ``we use a non-finite verb form in a periphrastic construction'',
    because finiteness is a category of a clause,
    and here is no clause combining.
    This is also relevant for surface-oriented descriptive linguistics:
    \citet[\citepage{74,83}]{cgel} rejects the notion of the \term{infinitive form} of the verb,
    and replace the term by \term{default form},
    because the so-called infinitive form also appears in the subjunctive mood
    or the imperative mood. 
    Despite this, to respect the tradition, 
    I will still use the term \term{non-finite verbs} 
    or wordings like ``the perfect passives are formed by attaching forms of copula to the perfect participle''.

    The generous paradigm-cell-as-verb-form approach fortunately works in Latin 
    because Latin is morphologically rich
    and thanks to historical changes,
    the boundaries between suffixes marking each grammatical category 
    are already vague enough, so the Japanese School Grammar approach is also not applicable. 
    So it does make sense to talk about 
    ``the indicative active perfect second person singular form'' of a verb.
    Similarly, we also talk about non-finite verb forms (\prettyref{sec:non-finite-abs}),
    though strictly speaking, finiteness is a category of the clause.
\end{infobox}

\subsection{Overview of conjugation classes}

The way realization of the paradigm for a verb
may be divided into four conjugation classes (\prettyref{sec:finite-paradigm}),
and there are also deponent verbs (\prettyref{sec:deponent-verbs}) and 
irregular verbs (\prettyref{sec:irregular-verbs}).


\section{The non-finite paradigm}\label{sec:non-finite-abs}

According to the morphology,
Latin non-finite verb forms can be classified into the infinitives (\prettyref{sec:infinitives})
and the nominal forms (\prettyref{sec:nominal-form}),
the latter having noun-like or adjective-like morphology.
Non-finite verb forms don't agree with the subjects they take,
so there is no number or person category marked on them in the same way as \prettyref{fig:paradigm-finite-verb},
though for nominal verb forms there are number and person categories 
marked in the same way as the nominal morphology.

The infinitives include present active, present passive, perfect active, 
perfect passive, future active, and future passive infinitives.
The latter three are realized periphrastically (\prettyref{fig:stem-to-form}).

The nominal verb forms include 
the \concept{simple active}, 
the \concept{perfect passive} (often just called the perfect participle), 
and the \concept{future active} participles,
the \concept{gerund}, 
the \concept{gerundive} which is also known as the \concept{future passive} participle, 
and two supine forms.
The \concept{first supine} is identical in the form to the singular neutral accusative perfect participle,
without any reference to the number category of any argument it takes.
The \concept{second supine} is identical to the singular neutral ablative or dative past participle,
also with no inflection with respect to the number category of any argument it takes.

\begin{infobox}{Whether to keep supine as a verb form}{supine-verb-form}
    The idea of the stingy linguist may lead one to reject the notion of supine in Latin grammar
    (\prettyref{box:conjugation-form}).
    However, for the same reason the infinitive 
    (or the ``plain form'', since the infinitive is actually a label of clauses 
    -- see the discussion and the separation principle in \prettyref{box:conjugation-form}) 
    is recognized as a form 
    independent from the present form in English in \citet[\citepage{74}]{cgel},
    the status of supine as a separate form is recognized in this note.
    The reasons include TODO
\end{infobox}

In Classical Latin, the gerund and participle forms are significantly more noun-like 
than their counterparts in English,
and this also justifies the term \term{nominal form},
because they are not far from prototypically nominalization:
although they are still modified by adverbs,
they are unable to take arguments.
In Ecclesiastical Latin, 
the so-called nominal forms are more verb-like (TODO: ref),
being able to take arguments,
and are therefore no longer ``nominal''.

\section{Formation of stems}\label{sec:verb-inflection.stem}

\subsection{The three verb stems}\label{sec:three-latin-stem}

\begin{theorybox}{About the concept of stem}{stem}
    \paragraph*{What's a stem} Prototypically, the verb conjugation in a language is described by 
    a series of morphological devices that take \emph{the} verb stem as the input,
    and give conjugated verb forms as the final product.
    This is indeed the case for Latin nouns (\prettyref{sec:regular-noun-declension})
    and for English regular verbs:
    the infinitive form is taken in,
    and third-person singular \form{-s}, past tense \form{-ed}, 
    past participle \form{-ed}, and the gerund-participle \form{-ing}
    are attached according to the syntactic environment.
    Sometimes the process is a little more irregular but not that irregular:
    \emph{several} stems can be identified, each of which is fed into different morphosyntactic machines.
    In other words, we have irregular stem alternation.
    Again, for English irregular verbs,
    there are three stems: the infinitive stem (e.g. \form{go}), 
    the preterite stem (e.g. \form{went})
    and the past participle stem (e.g. \form{gone}).
    The step to feed stems into morphosyntactic machine is irregular,
    but everything else is regular:
    irregular, in this case, does appear, but it appear \emph{regularly}:
    it only appears in certain parts.
    The irregularity of stem alternation is so prevalent
    that if the conjugation paradigm of a verb can be described with a few stems,
    the verb is deemed as regular, 
    despite the fact that such verbs are obviously irregular by the standard of English.

    \paragraph*{Stem alternation as mini conjugation classes} 
    This phenomenon -- that a verb has more than one stem, i.e. irregular stem alternation
    -- is frequent cross-linguistically
    (\citealt{jacques2021grammar} \citesec{12.2}, \citealt{forker2020grammar} \citesec{11.2}, among others).
    Usually, certain correlation between the stem varieties can still be recognized,
    and verbs can be grouped accordingly,
    which, if the linguist truly will, 
    can be (though tediously) summarized as more fine-grained conjugation classes.
    This is also the case for Latin (\prettyref{sec:verb-inflection.stem}).

    \paragraph*{Stem excluded from primary concepts in morphosyntax} 
    The notion of \concept{stems} isn't really essential in the description of morphosyntactic:
    it can well be modeled by environment-dependent vocabulary insertion rules 
    and/or post-syntactic operations.
    When certain correlations can still be built between so-called suppletive forms,
    what happens may be analyzed as in \citet{embick2005status},
    where certain stems receive morphophonological readjustment
    (according to the aforementioned hyper fine-grained conjugation subclasses).
    Thus, it's not true suppletion:
    it's just a corner case of non-concatenative morphology.
    When these readjustment rules are fossilized,
    suppletion -- like the English \form{good}/\form{better} -- 
    may just be the result of conditional insertion,
    as is outlined in \citet{siddiqi2009syntax}.

    \paragraph*{Strong irregularity in stem alternation usually restricted to light verbs} 
    A general tendency about suppletion
    is truly suppletive verbs are usually light verbs 
    (in the surface-oriented sense),
    with meanings like \translate{do}, \translate{come}, etc.
    This may come from the fact that conditional realization of the root -- 
    as opposed to grammatical items --
    is somehow ``heavy'' and not favorable.
    When we forbid conditional realization of roots,
    \citep{embick2005status},
    unrestricted suppletion can only be the result of 
    vocabulary insertion rules of functional heads;
    certain degree of suppletion can still be realized by readjustment rules,
    which however are restricted in their computational capacity.
    Thus, real lexical verbs are highly unlikely to have truly irregular suppletion;
    if a verb is truly suppletive,
    then it's highly likely to be 
    the spellout of \vP{} functional heads.
\end{theorybox}

Latin shows not completely predictable stem alternation.
All forms can be obtained by three stems \citep[\citesec{164}]{allen1903allen},
if the verb is regular:
\begin{itemize}
    \item The \concept{present stem}, which, after attached with proper endings, forms
    \begin{itemize}
        \item The \acl{present}, \acl{imperfect}, and future forms, 
        regardless of whether they are indicative or subjunctive,
        active or passive. (There is no future or future perfect subjunctive).
        \item All the imperatives.
        \item The present infinitives, active and passive.
        \item The present participle, the gerundive, and the gerund.
    \end{itemize}
    \item The \concept{perfect stem}, which, after attached with proper endings, forms 
    \begin{itemize}
        \item The perfect, pluperfect, and future perfect active, indicative or subjunctive.
        Again, there is no future or future perfect subjunctive.
        Note that the passives are \emph{not} formed by the perfect stem.
        \item The perfect active infinitive. 
        (Or the perfective infinitive active, since infinitive is considered as a mood by some people.)
    \end{itemize}
    Note that the perfect passive participle is \emph{not} obtained from the perfect stem.
    \item The \concept{supine stem}, 
    which, after attached with proper endings or used together with proper forms of \form{sum},
    forms 
    \begin{itemize}
        \item The perfect passive participle, which, by being used with proper forms of \form{sum}, forms
        \begin{itemize}
            \item The perfect, pluperfect, and future perfect passive forms, indicative or subjunctive.
            Again, there is no future or future perfect subjunctive.
            This is periphrastic conjugation: it is done by using proper forms of \form{sum}
            with the perfect passive participle.
            \item The perfect infinitive passive.
        \end{itemize}
        \item The future active participle, which, used together with \form{esse},
        makes the future active infinitive.
        \item The future passive infinitive, by being used together with \form{īrī}.
    \end{itemize}
\end{itemize}
This process is summarized in \prettyref{fig:stem-to-form}.
In a dictionary, 
typically the stems are not directly given 
-- which are given are representative verb forms,
from which the stems and the conjugation class can be inferred 
(\prettyref{sec:verb-inflection.parsing}).

Note that in Medieval Latin, often,
instead of \form{iri} plus the first supine,
\form{fore} plus the perfect participle is used to form the future passive infinitive.
TODO: find a reference https://www.nationalarchives.gov.uk/latin/stage-2-latin/lessons/lesson-24-infinitives-accusative-and-infinitive-clause/

\begin{sidewaysfigure}
    \centering
    

\tikzset{every picture/.style={line width=0.3pt}} %set default line width to 0.75pt        

\begin{tikzpicture}[x=0.75pt,y=0.75pt,yscale=-0.8,xscale=0.8]
%uncomment if require: \path (0,697); %set diagram left start at 0, and has height of 697

%Curve Lines [id:da8600925548352094] 
\draw [color={rgb, 255:red, 208; green, 2; blue, 27 }  ,draw opacity=1 ]   (289.01,269.33) .. controls (329.01,239.33) and (422.01,193.33) .. (676.01,187.33) ;
\draw [shift={(676.01,187.33)}, rotate = 178.65] [fill={rgb, 255:red, 208; green, 2; blue, 27 }  ,fill opacity=1 ][line width=0.08]  [draw opacity=0] (12,-3) -- (0,0) -- (12,3) -- cycle    ;
%Curve Lines [id:da7478415205524331] 
\draw [color={rgb, 255:red, 208; green, 2; blue, 27 }  ,draw opacity=1 ]   (275.01,266.33) .. controls (256.2,201.98) and (244.25,196.43) .. (205.2,166.25) ;
\draw [shift={(204.01,165.33)}, rotate = 37.78] [fill={rgb, 255:red, 208; green, 2; blue, 27 }  ,fill opacity=1 ][line width=0.08]  [draw opacity=0] (12,-3) -- (0,0) -- (12,3) -- cycle    ;
%Curve Lines [id:da22329313492102099] 
\draw [color={rgb, 255:red, 208; green, 2; blue, 27 }  ,draw opacity=1 ]   (241.01,279.33) .. controls (203.2,213.66) and (164.4,199.47) .. (101.95,158.94) ;
\draw [shift={(101.01,158.33)}, rotate = 33.06] [fill={rgb, 255:red, 208; green, 2; blue, 27 }  ,fill opacity=1 ][line width=0.08]  [draw opacity=0] (12,-3) -- (0,0) -- (12,3) -- cycle    ;
%Curve Lines [id:da9720425555739067] 
\draw [color={rgb, 255:red, 208; green, 2; blue, 27 }  ,draw opacity=1 ]   (250.01,319.22) .. controls (250.01,411.38) and (358.91,534.94) .. (466.39,595.28) ;
\draw [shift={(468.01,596.18)}, rotate = 209.05] [fill={rgb, 255:red, 208; green, 2; blue, 27 }  ,fill opacity=1 ][line width=0.08]  [draw opacity=0] (12,-3) -- (0,0) -- (12,3) -- cycle    ;
%Curve Lines [id:da9116874614549022] 
\draw [color={rgb, 255:red, 208; green, 2; blue, 27 }  ,draw opacity=1 ]   (265.01,312.33) .. controls (276.01,374.96) and (344.01,487.96) .. (487.01,511.96) ;
\draw [shift={(487.01,511.96)}, rotate = 189.53] [fill={rgb, 255:red, 208; green, 2; blue, 27 }  ,fill opacity=1 ][line width=0.08]  [draw opacity=0] (12,-3) -- (0,0) -- (12,3) -- cycle    ;
%Curve Lines [id:da9399420006349646] 
\draw [color={rgb, 255:red, 208; green, 2; blue, 27 }  ,draw opacity=1 ]   (277.01,310.33) .. controls (320.79,382.6) and (345.76,424.54) .. (456.34,434.81) ;
\draw [shift={(458.01,434.96)}, rotate = 185.1] [fill={rgb, 255:red, 208; green, 2; blue, 27 }  ,fill opacity=1 ][line width=0.08]  [draw opacity=0] (12,-3) -- (0,0) -- (12,3) -- cycle    ;
%Curve Lines [id:da863359914759456] 
\draw [color={rgb, 255:red, 248; green, 231; blue, 28 }  ,draw opacity=1 ]   (404.01,268.33) .. controls (396.09,243.58) and (393.07,211.97) .. (403.69,164.76) ;
\draw [shift={(404.01,163.33)}, rotate = 102.91] [fill={rgb, 255:red, 248; green, 231; blue, 28 }  ,fill opacity=1 ][line width=0.08]  [draw opacity=0] (12,-3) -- (0,0) -- (12,3) -- cycle    ;
%Curve Lines [id:da44942461080998] 
\draw [color={rgb, 255:red, 248; green, 231; blue, 28 }  ,draw opacity=1 ]   (418.01,269.33) .. controls (457.81,239.48) and (590.67,220.74) .. (688.54,241.24) ;
\draw [shift={(690.01,241.55)}, rotate = 192.09] [fill={rgb, 255:red, 248; green, 231; blue, 28 }  ,fill opacity=1 ][line width=0.08]  [draw opacity=0] (12,-3) -- (0,0) -- (12,3) -- cycle    ;
%Curve Lines [id:da6857007576480554] 
\draw [color={rgb, 255:red, 245; green, 166; blue, 35 }  ,draw opacity=1 ]   (543.01,333) .. controls (619.63,368.19) and (641.79,399.53) .. (690.28,478.65) ;
\draw [shift={(691.01,479.85)}, rotate = 238.51] [fill={rgb, 255:red, 245; green, 166; blue, 35 }  ,fill opacity=1 ][line width=0.08]  [draw opacity=0] (12,-3) -- (0,0) -- (12,3) -- cycle    ;
%Curve Lines [id:da10519824034130609] 
\draw [color={rgb, 255:red, 245; green, 166; blue, 35 }  ,draw opacity=1 ]   (548.01,313.33) .. controls (610.7,314.32) and (660.51,321.26) .. (718.14,390.28) ;
\draw [shift={(719.01,391.33)}, rotate = 230.36] [fill={rgb, 255:red, 245; green, 166; blue, 35 }  ,fill opacity=1 ][line width=0.08]  [draw opacity=0] (12,-3) -- (0,0) -- (12,3) -- cycle    ;
%Curve Lines [id:da05946917136382068] 
\draw [color={rgb, 255:red, 245; green, 166; blue, 35 }  ,draw opacity=1 ]   (764.01,427.33) .. controls (931.01,427.33) and (930.01,58.33) .. (587.01,121.33) ;
\draw [shift={(587.01,121.33)}, rotate = 349.59] [fill={rgb, 255:red, 245; green, 166; blue, 35 }  ,fill opacity=1 ][line width=0.08]  [draw opacity=0] (12,-3) -- (0,0) -- (12,3) -- cycle    ;
%Curve Lines [id:da40444377743555227] 
\draw [color={rgb, 255:red, 245; green, 166; blue, 35 }  ,draw opacity=1 ]   (760.01,418.33) .. controls (782.67,400.12) and (788.83,381.88) .. (787.1,343.11) ;
\draw [shift={(787.01,341.33)}, rotate = 87.14] [fill={rgb, 255:red, 245; green, 166; blue, 35 }  ,fill opacity=1 ][line width=0.08]  [draw opacity=0] (12,-3) -- (0,0) -- (12,3) -- cycle    ;
%Curve Lines [id:da6236126078150841] 
\draw [color={rgb, 255:red, 245; green, 166; blue, 35 }  ,draw opacity=1 ]   (776.01,515.33) .. controls (822.31,499.24) and (848.23,480.56) .. (870.97,451.34) ;
\draw [shift={(872.01,450)}, rotate = 127.48] [fill={rgb, 255:red, 245; green, 166; blue, 35 }  ,fill opacity=1 ][line width=0.08]  [draw opacity=0] (12,-3) -- (0,0) -- (12,3) -- cycle    ;
%Curve Lines [id:da6624205121022633] 
\draw [color={rgb, 255:red, 245; green, 166; blue, 35 }  ,draw opacity=1 ]   (526.01,337) .. controls (548.9,359.55) and (593.56,477.99) .. (600.9,583.59) ;
\draw [shift={(601.01,585.18)}, rotate = 266.22] [fill={rgb, 255:red, 245; green, 166; blue, 35 }  ,fill opacity=1 ][line width=0.08]  [draw opacity=0] (12,-3) -- (0,0) -- (12,3) -- cycle    ;
%Curve Lines [id:da8468021797759513] 
\draw [color={rgb, 255:red, 245; green, 166; blue, 35 }  ,draw opacity=1 ]   (630.01,594.18) .. controls (720.56,591.2) and (773.48,577.14) .. (847.89,560.25) ;
\draw [shift={(849.01,560)}, rotate = 167.23] [fill={rgb, 255:red, 245; green, 166; blue, 35 }  ,fill opacity=1 ][line width=0.08]  [draw opacity=0] (12,-3) -- (0,0) -- (12,3) -- cycle    ;
%Shape: Ellipse [id:dp3626499664001579] 
\draw  [color={rgb, 255:red, 74; green, 144; blue, 226 }  ,draw opacity=1 ][fill={rgb, 255:red, 74; green, 144; blue, 226 }  ,fill opacity=0.1 ] (39,126.52) .. controls (39,83.15) and (167.94,48) .. (327.01,48) .. controls (486.07,48) and (615.01,83.15) .. (615.01,126.52) .. controls (615.01,169.88) and (486.07,205.03) .. (327.01,205.03) .. controls (167.94,205.03) and (39,169.88) .. (39,126.52) -- cycle ;
%Curve Lines [id:da19915198987505445] 
\draw [color={rgb, 255:red, 80; green, 227; blue, 194 }  ,draw opacity=1 ][fill={rgb, 255:red, 80; green, 227; blue, 194 }  ,fill opacity=0.2 ]   (435.01,413.18) .. controls (475.01,383.18) and (680.01,340.18) .. (752.01,382.18) .. controls (824.01,424.18) and (801.01,553.18) .. (717.01,571.18) .. controls (633.01,589.18) and (657.01,503.18) .. (615.01,482.18) .. controls (573.01,461.18) and (401.02,474.22) .. (435.01,413.18) -- cycle ;
%Shape: Ellipse [id:dp7283210175576866] 
\draw  [color={rgb, 255:red, 126; green, 211; blue, 33 }  ,draw opacity=1 ][fill={rgb, 255:red, 126; green, 211; blue, 33 }  ,fill opacity=0.1 ] (423,565.16) .. controls (423,524.84) and (473.15,492.14) .. (535.01,492.14) .. controls (596.86,492.14) and (647.01,524.84) .. (647.01,565.16) .. controls (647.01,605.49) and (596.86,638.18) .. (535.01,638.18) .. controls (473.15,638.18) and (423,605.49) .. (423,565.16) -- cycle ;
%Curve Lines [id:da8675081082501883] 
\draw [color={rgb, 255:red, 80; green, 227; blue, 194 }  ,draw opacity=1 ][fill={rgb, 255:red, 80; green, 227; blue, 194 }  ,fill opacity=0.1 ]   (363.01,378.55) .. controls (387.01,323.55) and (691.01,321.55) .. (763.01,363.55) .. controls (835.01,405.55) and (805.01,594.55) .. (721.01,612.55) .. controls (637.01,630.55) and (649.01,514.55) .. (540.01,525.55) .. controls (431.01,536.55) and (329.02,439.59) .. (363.01,378.55) -- cycle ;
%Shape: Polygon Curved [id:ds050164225991979006] 
\draw  [color={rgb, 255:red, 184; green, 233; blue, 134 }  ,draw opacity=1 ][fill={rgb, 255:red, 184; green, 233; blue, 134 }  ,fill opacity=0.1 ] (683.01,164.55) .. controls (770.01,100.55) and (871.01,192.55) .. (904.01,306.55) .. controls (937.01,420.55) and (940.76,404.18) .. (965.01,472.55) .. controls (989.26,540.93) and (945.01,634.55) .. (872.01,627.55) .. controls (799.01,620.55) and (851.01,517.55) .. (838.01,459.55) .. controls (825.01,401.55) and (798.01,367.55) .. (738.01,329.55) .. controls (678.01,291.55) and (596.01,228.55) .. (683.01,164.55) -- cycle ;

% Text Node
\draw (489,83) node [anchor=north west][inner sep=0.75pt]   [align=left] {perfect,\\pluperfect,\\future perfect\\passive};
% Text Node
\draw (359,74) node [anchor=north west][inner sep=0.75pt]   [align=left] {perfect,\\pluperfect,\\future perfect\\active};
% Text Node
\draw (175,100) node [anchor=north west][inner sep=0.75pt]   [align=left] {present,\\imperfect,\\future};
% Text Node
\draw (385,269.07) node [anchor=north west][inner sep=0.75pt]  [color={rgb, 255:red, 0; green, 0; blue, 0 }  ,opacity=1 ] [align=left] {perfect\\stem};
% Text Node
\draw (248,268.07) node [anchor=north west][inner sep=0.75pt]  [color={rgb, 255:red, 0; green, 0; blue, 0 }  ,opacity=1 ] [align=left] {present\\stem};
% Text Node
\draw (470,587.07) node [anchor=north west][inner sep=0.75pt]   [align=left] {gerund};
% Text Node
\draw (496,501.07) node [anchor=north west][inner sep=0.75pt]   [align=left] {gerundive};
% Text Node
\draw (59,130) node [anchor=north west][inner sep=0.75pt]   [align=left] {imperative};
% Text Node
\draw (688,166) node [anchor=north west][inner sep=0.75pt]   [align=left] {present\\infinitives};
% Text Node
\draw (467,412) node [anchor=north west][inner sep=0.75pt]   [align=left] {present\\participle};
% Text Node
\draw (501,292.07) node [anchor=north west][inner sep=0.75pt]   [align=left] {supine\\stem};
% Text Node
\draw (696,218) node [anchor=north west][inner sep=0.75pt]   [align=left] {perfect\\active\\infinitive};
% Text Node
\draw (579,586.07) node [anchor=north west][inner sep=0.75pt]   [align=left] {supine};
% Text Node
\draw (697,395) node [anchor=north west][inner sep=0.75pt]   [align=left] {perfect\\passive\\participle};
% Text Node
\draw (765,278) node [anchor=north west][inner sep=0.75pt]   [align=left] {perfect\\passive\\infinitive};
% Text Node
\draw (695,482) node [anchor=north west][inner sep=0.75pt]   [align=left] {future\\active\\participle};
% Text Node
\draw (872,382) node [anchor=north west][inner sep=0.75pt]   [align=left] {future\\active\\infinitive};
% Text Node
\draw (864,521) node [anchor=north west][inner sep=0.75pt]   [align=left] {future\\passive\\infinitive};
% Text Node
\draw (200,60) node [anchor=north west][inner sep=0.75pt]  [color={rgb, 255:red, 74; green, 144; blue, 226 }  ,opacity=1 ] [align=left] {finite forms};
% Text Node
\draw (594,405.14) node [anchor=north west][inner sep=0.75pt]  [color={rgb, 255:red, 80; green, 227; blue, 194 }  ,opacity=1 ] [align=left] {particle\\in narrow\\sense};
% Text Node
\draw (519,532.51) node [anchor=north west][inner sep=0.75pt]  [color={rgb, 255:red, 126; green, 211; blue, 33 }  ,opacity=1 ] [align=left] {"nominal"\\nonfinite\\forms};
% Text Node
\draw (389,356.14) node [anchor=north west][inner sep=0.75pt]  [color={rgb, 255:red, 80; green, 227; blue, 194 }  ,opacity=1 ] [align=left] {particle in broad sense};
% Text Node
\draw (775,228) node [anchor=north west][inner sep=0.75pt]  [color={rgb, 255:red, 184; green, 233; blue, 134 }  ,opacity=1 ] [align=left] {infinitives};


\end{tikzpicture}

    \caption{How to get all conjugation forms from the three stems}
    \label{fig:stem-to-form}
\end{sidewaysfigure}

\subsection{Formation of the present stem}



\subsection{Formation of the perfect stem}\label{sec:verb-inflection.stem.perfect}

\section{The finite paradigm}\label{sec:finite-paradigm}

\subsection{Marking of tense and mood}\label{sec:tense-mood-marking}

The contextual alternation of the tense and mood marker is listed below.
Note that the tense and mood marker is also subject to phonological rules 
(diachronic or synchronic).
The following rule is the most important one that apples in verb conjugation:
\emph{a long vowel is shortened before \form{-m}, \form{-r}, \form{-t}, \form{-nt}, \form{-ntur}}.
Considering \form{b\={a}ris},
this doesn't look like a synchronic rule?? TODO

\begin{itemize}
    \item The indicative:
    \begin{itemize}
        \item Present: zero suffixation, but there is change on the stem-final vowel:
        \begin{itemize}
            \item For first conjugation verbs, the final \form{\={a}} is dropped.
            \item For second conjugation verbs, the final \form{\={e}} $\to$ \form{e}.
            \item  
        \end{itemize}
        \item Imperfect: \form{-b\={a}-}, possibly shortened.
        \item Future: 
        \begin{itemize}
            \item For first and second conjugation verbs, 
            the tense-mood morpheme is \form{-bi-}, except for 
            first-person singular (which is \form{-b-})
            and third-person plural (which is \form{-bu-}).
            \item For third and fourth conjugation verbs, change stem-final stem.
        \end{itemize}
        \item Perfect: 
        \begin{itemize}
            \item \form{-ī-}: first-person singular, third-person singular, first-person plural.
            Shortened for the latter two.
            \item \form{-is-}: second-person singular, second-person plural.
            \item \form{-\={e}ru-}: third-person plural.
        \end{itemize}
        \item Pluperfect: \form{-er\={a}-}, possibly shortened.
        \item Future perfect: 
        \begin{itemize}
            \item \form{-eri-} for all cases except first person singular.
            \item \form{-er-} for first-person singular.
        \end{itemize}
    \end{itemize}
    \item The subjunctive:
    \begin{itemize}
        \item Present: no suffixation, but there is regular change on the stem-final vowel:
        \begin{itemize}
            \item For first conjugation verbs, \form{\={a}-} $\to$ \form{\={e}-}.
            \item For second conjugation verbs, \form{} $\to$ \form{e\={a}-}.
            \item For third conjugation verbs, $\to$ \form{\={a}-}.
            \item For fourth conjugation verbs, $\to$ \form{i\={a}-}.
        \end{itemize}
        These alternations apply for both active and passive verbs,
        so they have nothing to do with polarity, and this is why I put them in this section.
        \item Imperfect: \form{-r\={e}-}, possibly shortened.
        \item Perfect: \form{-eri-} for all non-periphrastic cases, i.e. active.
        \item Pluperfect: \form{-iss\={e}-} for all non-periphrastic cases, i.e. active.
        Possibly shortened.
    \end{itemize}
\end{itemize}

\begin{infobox}{Shortening or prolonging?}{shorten-or-prolong-ba}
    At the first glance, it may be attempting as well to consider \form{-ba-} as 
    the indicative imperfect suffix,
    since \form{-b\={a}-} does not outnumber it.
    However, note that TODO: justification of analyzing \form{-b\={a}-} as the fundamental form.
    The same line of argumentation can be applied to justify the status of \form{-r\={e}-}
    as the somehow canonical subjunctive imperfect suffix.
\end{infobox}

\subsection{The personal ending}

Here I list possible personal endings for verbs that are indicative or subjunctive. 

\begin{itemize}
    \item The active:
    \begin{itemize}
        \item First-person singular: 
        \begin{itemize}
            \item \form{-\={o}}: present indicative, 
            future indicative (first and second conjugations only), 
            future perfect indicative.
            \item \form{-m}: imperfect indicative, 
            future indicative (third and fourth conjugations only),
            pluperfect indicative,
            subjunctive regardless of tense.
            \item \form{-ī}: perfect indicative.
        \end{itemize}
        \item Second-person singular:
        \begin{itemize}
            \item \form{-s}: compatible with all tenses and moods, except the perfect indicative.
            \item \form{-tī}: perfect indicative.
        \end{itemize}
        \item Third-person singular: \form{-t}, with all tenses and moods.
        \item First-person plural: \form{-mus}, with all tenses and moods.
        \item Second-person plural: \form{-tis}, with all tenses and moods.
        \item Third-person plural: \form{-nt}, with all tenses and moods.
    \end{itemize}
    \item The passive:
    \begin{itemize}
        \item First-person singular: 
        \begin{itemize}
            \item \form{-r}: compatible with all tenses and moods, except the present indicative.
            \item \form{-or}: present indicative.
            Also, note that the future indicative (first and second conjugations only) ending is \form{-bor},
            which may be analyzed as \form{-b-or}.
        \end{itemize}
        \item Second-person singular:
        \begin{itemize}
            \item \form{-ris}: compatible with all non-periphrastic tenses and moods.
            \item \form{-re}: alternative form of second-person singular compatible 
            with all non-periphrastic tenses and moods.
            If this personal ending is used, then the tense and mood marking is none.
            Note that the resulting verb form is the same as the infinitive participle part.
        \end{itemize}
        \item Third-person singular: \form{-tur}, with all non-periphrastic tenses and moods.
        \item First-person plural: \form{-mur}, with all non-periphrastic tenses and moods.
        \item Second-person plural: \form{-minī}, with all non-periphrastic tense and moods.
        \item Third-person plural: \form{-ntur}, with all non-periphrastic tenses and moods.
    \end{itemize}
\end{itemize}

\subsection{Periphrastic conjugations}

An auxiliary verb construction is a structure 
that contains one or more auxiliaries apart from the main verb, 
and yet is mono-clausal and is not a complement clause construction,
and the auxiliaries are realizations of the verbal system 
surrounding the main verb, and not independent lexical verbs.
Auxiliary verb constructions realizing grammatical categories 
that are usually realized by inflectional endings in a paradigm
should be seen as a part of that paradigm,
and therefore are known as periphrastic conjugations.

\section{Non-finite forms}

\subsection{The infinitives}\label{sec:infinitives}

\subsection{The gerund and participles}\label{sec:nominal-form}

\subsubsection{The gerund}\label{sec:gerund-morphology}

Note that the nominative case is missing -- 
when a non-finite clause is required in the subject position,
it's always an infinitive.

\section{Deponent verbs}\label{sec:deponent-verbs}

\begin{theorybox}{Distributed Morphology of deponent verbs}{deponent}
    \citet{embick2000features} analyzes deponent verbs as roots carrying a passive feature themselves.
    When functional heads higher than the roots are realized,
    the passive feature -- which may come from the root or from the passive light verb -- 
    guides the realization of the person and number categories.
\end{theorybox}



\section{Irregular verbs}\label{sec:irregular-verbs}

\subsection{The verb \form{sum}}\label{sec:sum-morphology}

\subsubsection{Overview}

The verb \form{sum} has lots of uses in Latin grammar (\prettyref{sec:sum}),
and its inflection is (unfortunately but expectedly) highly irregular.
It's also defective: 
it has no passive forms, either finite or nonfinite.
The principal parts (\prettyref{sec:verb-inflection.parsing.principal-part}) are 
\form{sum}, \form{esse}, \form{fuī}, 
with the supine form being absent -- usually replaced by the future active participle \form{futūrus}.

From the principal parts, 
we find the perfect stem is \form{fu-}, 
and the supine stem -- if we insist on defining it -- 
is the same, 
although the perfect passive participle is absent and so is the supine,
and therefore the supine stem only appears in the future active participle.

The present stem is not well-defined:
the second principal form \form{esse}
doesn't have the regular infinitive ending \form{-re},
though we can roughly recognize something like \form{es-} or \form{e-};
the first principal form \form{sum} gives \form{su-} or \form{s-}.
The two stems appear in the finite paradigm in an unpredictable manner, 
also with irregular though still recognizable endings.
Besides \form{s-} and \form{es-},
there is also \form{fo-} seen in one variant of the future active infinitive
(\prettyref{sec:verb-inflection.irregular.sum.nonfinite}),
which also appears in variants in
the subjunctive active imperfect part of the finite paradigm.

\subsubsection{The nonfinite paradigm}\label{sec:verb-inflection.irregular.sum.nonfinite}

The only nominal form is the future active participle \form{futūrus}.
The three active infinitives forms are all attested.
The present active infinitive is \form{esse}.
The perfect active infinitive is \form{fuisse}, 
regularly formed by the perfect stem \form{fu-}.

The future active infinitive can be regularly formed by adding \form{esse} 
to the future active participle,
and therefore is \form{futūrum esse}.
There is also a free variant \form{fore}.

\subsubsection{The perfect system}

The perfect forms -- finite forms and the perfect active infinitive -- 
of \form{sum} can be formed regularly (\prettyref{sec:finite-paradigm})
according to the perfect stem \form{fu-}.

\subsubsection{The imperative system}

The present imperative system, 
which is known for reflecting the present stem,
is formed regularly using \form{es-}:
the singular second person present imperative is \form{es}
and the plural second person present imperative is \form{este}.

\subsubsection{The present system}

The imperfect forms of \form{sum} are highly irregular,
though patterns can still be found.
In the indicative part (\prettyref{tbl:indicative-sum}): 
\begin{itemize}
    \item The \category{present} forms show no pattern
    except the personal endings.
    Note that here \form{-m} instead of \form{-\={o}}
    is used for the first person singular form.
    \item The \category{imperfect} forms are formed 
    by adding the standard personal endings 
    (\form{-m}, \form{-s}, \form{-t}, 
    \form{-mus}, \form{-tis}, \form{-nt})
    to \form{er\={a}},
    the vowel \form{\={a}} of which 
    undergoes shortening according to rules in \prettyref{sec:tense-mood-marking}.
    \item The \category{future} forms are formed by 
    the same personal endings seen in the first and the second conjugations,
    although the tense marker isn't the same: 
    the stem-tense marker complex is \form{er-}
    instead of the stem plus \form{-b-}.
\end{itemize}

\begin{table}[H]
    \caption{The indicative paradigm of \form{sum}}
    \label{tbl:indicative-sum}
    \centering
    \begin{tabular}{lll}
    \toprule
    \category{present} & \category{imperfect}  & \category{future}  \\
    \midrule
    \form{sum}     & \form{eram}       & \form{er\={o}} \\
    \form{es}      & \form{er\={a}s}   & \form{eris}    \\
    \form{est}     & \form{erat}       & \form{erit}    \\
    \form{sumus}   & \form{er\={a}mus} & \form{erimus}  \\
    \form{estis}   & \form{er\={a}tis} & \form{eritis}  \\
    \form{sunt}    & \form{erant}      & \form{erunt}   \\ \bottomrule
    \end{tabular}
\end{table}

In the subjunctive paradigm (\prettyref{tbl:subjunctive-sum}),
we find that in the \category{present} system, 
the stem-tense marker complex is fused into \form{sī-},
and in the \category{imperfect} system,
the stem-tense marker complex is fused into \form{ess\={e}-} or \form{for\={e}-},
both of which are then attached to the standard \form{-m}, \form{-s}, etc. 
personal endings, 
and the vowel shortening rule in \prettyref{sec:tense-mood-marking} works.

\begin{table}[H]
    \caption{The subjunctive paradigm of \form{sum}}
    \centering
    \label{tbl:subjunctive-sum}
    \begin{tabular}{ll}
    \toprule
    \category{present}   & \category{imperfect}   \\ \midrule
    \form{sim}       & \form{essem}, \form{forem}       \\
    \form{sīs}   & \form{ess\={e}s}, \form{for\={e}s}   \\
    \form{sit}       & \form{esset}, \form{foret}       \\
    \form{sīmus} & \form{ess\={e}mus}, \form{for\={e}mus} \\
    \form{sītis} & \form{ess\={e}tis}, \form{for\={e}tis} \\
    \form{sint}      & \form{essent}, \form{forent}     \\ \bottomrule
    \end{tabular}
\end{table}

\subsection{The verb \form{faci\={o}}}

The verb \form{faci\={o}} looks pretty regular
regarding the endings, 
except for one thing: 
its \emph{stem} alternates according to the voice.

\section{Guide for parsing and using Latin verbs}\label{sec:verb-inflection.parsing}

\subsection{Principal forms and stems}\label{sec:verb-inflection.parsing.principal-part}

In practice, the three stems aren't what stored in the dictionary,
for two reasons.
First, for fluent users,
recording actually attested word forms is easier
compared with the morpheme-based and ``anatomized'' approach.
Second, Latin has four conjugation types,
and hence the three stems themselves aren't sufficient to decide how to conjugate the verb:
more information is needed, 
and by storing already conjugated verb forms,
the conjugation class can be decided by observing the endings.
What are stored are the following \concept{principal forms},
from which the three stems and the conjugation class can be solved out
\citep[\citesec{172}]{allen1903allen}:
\begin{enumerate}
    \item \emph{The first-person present active indicative}: formed from the present stem.
    \item \emph{The present infinitive}: formed from the present stem. 
    By observing its ending, the conjugation class can be decided,
    and by comparing with the first principal form, 
    the present stem is obtained.
    \item \emph{The first-person perfect active indicative}: showing the perfect stem.
    \item \emph{The neutral accusative past participle}, i.e. the form of supine: showing the supine stem.
\end{enumerate}
The ways to obtain the stems from the principal forms are:
\begin{itemize}
    \item \emph{The present stem} can be found by dropping \form{-re} in the 
    \category{present infinitive}
    \citep[\citesec{175}]{allen1903allen}.
    \item \emph{The perfect stem} can be found from the third principal part:
    just remove \form{-ī}.
    \item \emph{The supine stem} can be found by dropping \form{-um} in the supine
    i.e. the fourth principal form
    \citep[\citesec{178}]{allen1903allen}.
\end{itemize}

\subsection{Constructing non-finite forms}

\subsubsection{Nominal forms}

\begin{itemize}
    \item \emph{The present active participle (i.e. the present participle)}: 
    replace the \form{-re} ending of the present active infinitive by \form{-nt}
    (or in other words, add \form{-nt} to the present stem)
    and the result is the nominative. % TODO: gender, and third declension
    \item \emph{The perfect passive participle (i.e. the perfect participle or the past participle)}:
    this can be found by declining the neutral accusative past participle, 
    i.e. the fourth principal part.
    \item \emph{The future active participle (i.e. the future participle)}:
     add \form{-turus} to the supine stem.
\end{itemize}


\section{Auxiliary verb constructions}

Latin is usually perceived as a language with few analytic properties;
the only example of periphrastic conjugation 
being the \form{sum} plus perfect passive participate construction.
A deeper look however reveals 
there might be more 

\chapter{Verb phrase}

\section{Introduction}\label{sec:core-argument-marking}

This chapter gives an overview of clausal dependents,
especially about the mapping from purely semantic argument roles to clause dependent slots.
This chapter is mainly about verbs that don't take complement clauses as arguments.
The phenomena discussed in this chapter mostly apply to complement clause constructions as well,
but complement clause constructions have their own peculiarities 
(\prettyref{sec:complement-clause-construct-overview}).

A list of semantic classification of verbs -- 
and hence valency classes -- can be found in 
\citet[Part B]{dixon2005semantic},
\citet[\citesec{18.5}]{dixon2010basic2},
and \citet[\citesec{3.3}]{dixon2009basic1}.
Dixon classifies the verb class into three subgroups:
\begin{enumerate*}
    \item Primary-A, which contains verbs that 
    don't take arguments with meanings similar to those of complement clauses,
    \item Primary-B, which are semantically \concept{complement-taking} (\prettyref{box:complement}) 
    and \concept{lexical},
    which have arguments that are semantically equivalent to complement clauses 
    (but not necessarily syntactically coded as complement clauses)
    and have meanings more complicated then what's expected for grammatical items, and 
    \item Secondary, members of which have the same \emph{meaning} 
    of certain grammatical constructions in the verbal system,
    but not the same syntactic properties
    (for example, they may just take complement clauses instead of being auxiliary verbs).
\end{enumerate*}

Allowed combinations of clausal dependents are determined 
by the valency class of the verb and how it engages with valency changing devices,
which is strongly related to but is not determined by the semantics of the verb.
In English, for example, \form{I like this} 
where \form{this} can represent an event 
is semantically complement-taking 
but involves no complement clause construction syntactically;
and it's possible to leave out a semantic argument in the syntactic frame of a verb.
Thus, research mainly focused on \emph{syntactic} properties of arguments is also needed.

\begin{theorybox}{A-positions}{a-position}
    In generative syntax, 
    we say the positions in the syntactic frame of a verb are \concept{A-positions}.
    To achieve a more disciplined analysis, 
    we may adopt a multiple-step analysis of A-positions:
    at least two steps -- the \vP{} step and the TP step 
    -- are to be distinguished,
    and in both steps there are sub-steps.

    \paragraph*{A-positions and semantics} 
    I should focus again that arguments with same semantic roles 
    may appear in different A-positions, 
    and one A-positions may host various semantic roles.
    A Secondary verb is different from, say, an auxiliary verb in the verbal system
    in the eye of syntax 
    and at the syntax-semantic interface:
    the lexical verb \form{start}, for example, 
    introduces a new event besides the event that the agent is start to do, 
    while an inchoative aspect, 
    if in the grammatical aspect region and not the lexical aspect region,
    reflects the speaker's attitude
    (possibly by shifting the time referred to 
    to the initial part of the whole situation),
    but they are of course equivalent to each other,
    although their interpretations 
    immediately at the syntax-semantic interface 
    are different.
    On the other hand, the subject A-position may host 
    the participant in an event that initiates the event 
    or it may just be an experiencer.

    One thing to note, however, is that if we do coarse-graining to the 
    various verb-specific semantic roles 
    and compare them with positions within the \vP, 
    we \emph{do} find a strong correlation, 
    known as the Uniformity of Theta-Assignment Hypothesis (UTAH).
    This means the coarse-grained semantic roles 
    (things like ``agent'', ``theme'', etc., 
    as opposed to ``Manipulator'' or ``Target'')
    are syntactic objects as well,
    as they label the positions in \vP;
    this creates a confusion of the meaning of the term \term{semantic role};
    in this note both types of semantic roles are discussed, 
    but usually the meaning of the term \term{semantic role} can be inferred from the context.

    \paragraph*{A-positions in an accusative language} 
    The first step is roughly a ``translation'' of the semantic argument roles 
    into their syntactic counterparts,
    according to the so-called Uniformity of Theta Assignment Hypothesis.
    (There are subtleties in topics like whether 
    the experiencer roles in psych-verbs as in 
    % source: A Note on the Sematic Role “Stimulus” of Like and Please*
    % TODO: better source
    \form{he fears the police} and \form{the police frightens him}
    are actually different in the \vP{} step,
    or they are the same in the \vP{} step but something else decides 
    which argument becomes the subject.)
    Assignment of the accusative case is said to be done within the \vP region.
    Then comes the TP step, in which
    usually the highest argument position in the \vP{} step -- also the most agentive one --
    becomes SpecTP, 
    which is better known as the subject.

    \paragraph*{Non-trivial correspondence between derivational steps in clausal syntax}%
    \footnote{
        Here \term{derivational} means what it means in modern generative grammar, 
        that grammatical structures are built by successive applications of (possibly internal) Merge.
        It \emph{doesn't} mean \emph{transformational operations}
        in early versions of generative grammar.
    }
    The \concept{argument structure} 
    (argument labels in which are ``agent'', ``patient'', etc., 
    i.e. the type of semantic roles bearing direct syntactic significance 
    discussed above)
    is the structure of the ``canonical'' \vP{} containing a verb root.
    For roles in the argument structure, 
    see, for example, \citet[\citesec{4.2}]{cgel}, 
    which lies around the syntax-semantic interface 
    and are actually syntactic categories.
    On the other hand, 
    the \concept{clausal complement types} 
    like subject, object, etc.
    are largely decided by what happens after TP is finished.
    Lots of things can happen between the two.
    The most well known case is passivization,
    where although we have evidences that the agent argument still occupies a higher position,
    it receives an inherent case (quite similar to how DPs are licensed by prepositions)
    and is thus unable to move to the subject position.
    It's therefore worthwhile to talk about ``deep'' roles and ``surface'' roles
    -- although they are defined as S, A, P, etc.
    and are known as generalized semantic roles, 
    these are \emph{syntactic} labels, not semantic ones.
    For unaccusative verbs, 
    the agentive argument may be absent, 
    and the deep P argument becomes the surface S argument.\footnote{
        Note that unaccusativity has nothing to do with alignment:
        we can have unaccusative verbs in a typical accusative language, like English.
    }
    Thus, in a truly well organized grammar, 
    we need to first study deep syntactic argument slots 
    and then surface syntactic argument slots.

    \paragraph*{Ergativity and split of grammatical relation labels}
    It should be noted that labels for A-positions, like \term{subject},
    implicitly means there is an argument which is spontaneously in several syntactic functions
    (in the case of \term{subject}, 
    it's the external argument which governs all internal complements, 
    the receiver of the nominative case, and, say, SpecDoP, 
    where Do or something else is the highest light verb in the \vP{} field).
    When these syntactic functions are disassembled with each other, 
    the corresponding collective label no longer makes sense.
    Thus in a morphological ergative language, 
    the ``nominative case receiver'' syntactic function is absent, 
    and the A argument of a transitive verb receives an inherent case, 
    so the A argument loses the morphological resemblance with the 
    S argument for intransitive verbs, 
    but it keeps the syntactic functions of 
    the highest argument; 
    for syntactic ergativity, 
    the external argument property is given to the P argument \citep{aldridge2008generative}.
    In both cases, grammatical relations condensed into the label \term{subject} 
    need to be taken out one by one and redistributed to new collective terms.
\end{theorybox}

\section{Core, oblique and peripheral arguments}

This section examines complement and adjunct positions in Latin clauses. 
Parameters used in the classification include 
their correspondence with semantic roles 
(agent, patient, source, etc.),
their internal structures (case, preposition, \acs{np} or clause, etc.), 
agreement, and
their behaviors in valency alternation.

\begin{infobox}{The term \term{complement}}{complement}
    In \citet{cgel} the term \term{complement} 
    means A-positions mentioned in \prettyref{box:a-position}.
    Thus the subject, several kinds of objects,
    the copular complement (\cite{cgel} calls it \term{predicative complement}) are all complements,
    and they are labels implying several grammatical relations:
    the label \term{subject} roughly corresponding to ``SpecTP'' or ``what is in \vP{} 
    and receives the object case from a high light verb''.

    In traditional Latin grammar, however, \term{complement} means the copular complement. 
    This note follows the terminology used in most descriptive grammars,
    so use the term \term{copular complement} to refer to the \ac{cgel} \term{predicative complement}.
    Also, as is seen in \prettyref{sec:core-argument-marking}, 
    the term \term{complement-taking verb}, despite being confusing,
    is used to refer to a verb that take a complement clause 
    or something semantically equivalent to a complement clause
    as one of its arguments.
\end{infobox}

Beside the subject, various types of objects, and copular complements,
there are more clause dependents
corresponding to less frequently seen semantic roles 
like purpose, direction, location, etc.
They may be licensed or even required by the verb 
(\concept{oblique argument}),
or they may be modifying the whole clause and therefore are usually optional 
(\concept{peripheral argument}, or ``adjuncts'').
Besides clauses and \ac{np}s (with or without prepositions),
their categories also include \ac{advp}s,
a majority of the latter originating from case forms.

A clear complement-adjunct distinction 
-- telling peripheral arguments from core arguments or oblique arguments --
is hard to establish in Latin.
Latin peripheral arguments do not necessarily have prepositions.
Latin is highly free-ordered and therefore all clause dependents 
can leave their base positions.
Latin is also highly \term{pro}-drop,
and even uncontroversial core arguments can be omitted.
Oblique arguments are frequent in Latin,
as is the case in English 
(consider \form{run away from} or \form{get into}).
Thus criteria of category, position, and argument in \citet[\citesec{4.1.2}]{cgel} 
all fail to work.
Latin doesn't have systematic way to replace the core predicate (i.e. without adjuncts) by an anaphora,
and that criterion does not work, either. TODO: really?
The remaining criteria are about selection, licensing, and obligatoriness;
these criteria are however hard to use for a classical language. 
Thus, despite I'm fully aware that  
clausal dependents concerning place, instrument, mean, etc. 
may be licensed by both the argument structure of the verb 
and by clausal adjunct positions 
and may have clear structural differences in other languages 
(as in English), 
currently no distinction between the two cases is made.


\begin{infobox}{How to document complement types}{complement-discussion}
    The traditional practice of Latin grammar research
    is to classify clausal complement and adjunct types according to their case marking.
    This strategy is also found in modern grammars.
    Some introduce clausal complement types just in chapters about case marking 
    \citep[\citechap{8}]{jacques2021grammar},
    while other grammars, despite giving a brief description of the context of case marking,
    spare some time to discuss complement types in the chapters about valency and clause structure 
    \citep[\citesec{3.4}, \citechap{19}, \citechap{22}]{forker2020grammar}.
    From a \ac{tag} perspective (\prettyref{sec:theoretical-orientation}), 
    the two extremes are different in how they treat function labels:
    in the former, the function label of a construction appears together with its category label on the root node,
    while in the latter, the function is described separately from the form of what fills that position.
    The former is more bottom-up, 
    while the latter is more top-down.
    The choice between the two, however, is usually language-dependent:
    grammars for analytic languages, of course, have to lean even further to the 
    ``complement type as clause slot'' extreme 
    and away from the ``complement type as case-form context'' extreme.
    \citet{allen1903allen} uses a hybrid method:
    the discussion about case marking (\citesec{39}) is separated from 
    the discussion about complement and adjunct types (\citesec{338}),
    so the top-down approach seems to be adopted,
    but the latter is still arranged in terms of case.
    This arises both from the distinct features of Latin and the intended readers:
    the relation between complement types and cases is regular enough in Latin,
    and what is most important for Latinists is to understand, at least sketchily, ancient writings, 
    so a parsing-oriented grammar is much handier.
\end{infobox}

\subsection{The subject}

Latin is an accusative language.
A \concept{subject} can be identified for all clauses, 
though it is frequently omitted.
Grammatical behaviors of the subject are summarized in the following list: 
\begin{itemize}
    \item \emph{Coding of semantic role}: In an active clause, 
    the subject is always the most agentive argument,
    i.e. the S argument in a prototypical intransitive argument structure 
    and the A argument in a prototypical transitive argument structure.
    In a passive clause, 
    the subject corresponds to the ``promoted argument'' (\prettyref{sec:passive}).  
    \item \emph{Case marking}: 
    Subjects are always nominative for finite clauses,
    whenever the case system is in action,
    i.e. whenever the subject is an \ac{np} or a gerund. 
    Nonfinite clauses may be argued to be subjectless in the surface form 
    (a reasonable claim, since they have deficient TP layers, 
    and hence it is possible that no canonical subject position exists),
    but in accusative-infinitive constructions, % TODO: control or ECM or ... ?
    the accusative may be seen as the non-canonical subject of the infinitive.
    \item \emph{Agreement}: 
    the number and person features on the subject leave marking on the verb complex.
    Latin does not have verbal agreement with arguments other than the agreement with the subject.
    \item \emph{Category}: a subject is an \ac{np}  
    or a complement clause (\prettyref{sec:complement-clause-construct-overview}), 
    usually an infinitive but never a gerund (\prettyref{sec:gerund-morphology}).
    This constraint isn't seen in any other clausal complement types.
\end{itemize}


\subsection{The direct object}\label{sec:vp.complement.direct-object}

Here is a list of grammatical properties of the direct object:
\begin{itemize}
    \item \emph{Coding of semantic role}: In a prototypical transitive argument structure, 
    the direct object is the P argument, i.e. the most patientive argument. 

    \item \emph{Case marking}: Direct objects are always accusative when it makes sense to talk about case -- 
    but not all accusative arguments are direct objects (\prettyref{sec:accusative-distribution}).
    \item \emph{Passivization}: If an argument is coded as the direct object, 
    then it can regularly be promoted to the subject position in a passive clause (\prettyref{sec:passive}). 
    Secondary objects are less frequently promoted in passivization (\prettyref{sec:accusative-distribution}).
\end{itemize}

\subsection{The indirect object and the secondary object}\label{sec:vp.complement.indirect-object}

Latin also has two complement positions named as object:
the indirect object and the secondary object.
The indirect object is distinguished by the following grammatical properties:
\begin{itemize}
    \item \emph{Coding of semantic role}: in a AGT-type argument structure, 
    the indirect object is usually the G argument.
    Intransitive clauses sometimes also have indirect objects, 
    and an indirect object, in this case, is also a G argument.
    \item \emph{Case marking}: indirect objects are always dative.
    \item \emph{Passivization}: indirect objects are always retained in passive clauses. 
    They are never promoted to subjects in passivization.
    % TODO: category
\end{itemize}

The secondary object is distinguished by the following grammatical properties:
\begin{itemize}
    \item \emph{Coding of semantic role}: in an AGT-type argument structure
    that is always about information flowing,
    the T argument (i.e. the thing asked about or taught about) is the secondary object.
    The G argument (i.e. the person who is asked or taught) is the direct object.
    Sometimes the G argument is ablative, and in this case, 
    there is only one accusative argument: the secondary object.
    Another place where secondary objects appear is 
    clauses headed by a verb with a compounded accusative preposition. % TODO: SAO typology
    \item \emph{Case marking}: secondary objects are always accusative.
    \item \emph{Passivization}: secondary objects can be passivized, but much more rarely than direct objects.
    \item % TODO: category
\end{itemize}

The distributions of the secondary object and the indirect object 
are mutually exclusive.
This means for ditransitive verbs of type \classify{giving}, 
Latin shows a clear and strong tendency to identify the T argument with the monotransitive O,
while for ditransitive verbs about teaching,
the inverse is true.

\begin{infobox}{Comparison with English}{english-indirect-object}
    It can be found that the Latin indirect object has more similarity with the English \form{to}-PP,
    which is also called the indirect object in some grammars, but not \ac{cgel}.
    The Latin indirect object differs from the English (accusative) indirect object in passivization.
    Since in Latin, verbs with AGT-type argument structure do not have alternation of complementation pattern
    -- in English we have \form{give sth. to sb.} and \form{give sb. sth.}, 
    while in Latin there is only the former one, but \form{to sb.} is replaced by a dative,
    (always with no preposition) --
    the G argument is identified with the E argument,
    and the T argument is identified with the P argument.
    In other words, in Latin, there is only 
    the \form{John gave [goods]_{\text{T}} to [charity]_{\text{G}}} pattern:
    the double-object \form{John gave charity goods} pattern is absent.
    
    Therefore, for typical ditransitive verbs, i.e. verbs like \form{give}, 
    Latin shows a clear and strong tendency to identify the T argument with the monotransitive O,
    which is more typical than English%
    \footnote{
        In English, in the \form{give sb. sth.} construction, it is the person i.e. the G argument that is passivized,
        while the T argument i.e. \form{sth.} cannot, though the latter is identified with monotransitive O
        according to other criteria. 
    },
    but for verbs with meaning of \category{teach} or \category{ask},
    there is also a clear and strong tendency to identify the G argument with the monotransitive O.
    The term \term{secondary object} is coined to cover this grammatical position.
\end{infobox}

\subsection{Copular complements}

Latin also has copular complements.
A copular complement, just like its counterpart in English,
basically can be viewed as a displaced attributive or appositive 
(and hence is prototypically filled by an \ac{np} or an \acs{adjp})
but is a little more peripheral (manner, state, factitive, etc.) 
in its meaning than an attributive or appositive.

Latin has nominative predicate and accusative predicate:
as hinted by their names, 
the nominative predicate gives a property of the subject and agrees with it,
and the accusative predicate gives a property of the direct object and agrees with it.
In passivization of the direct object,
the accusative predicate becomes the nominative predicate.

Other types of copular complements without agreeing with the subject exist.
TODO: ablative of quality, price, etc. 
The syntactic status of copular complements here are closer to \acs{pp}s:
we may say they receive \emph{inherent cases},
while the nominative and accusative copular complements 
receive \emph{structural cases} 
(\prettyref{sec:np.case-distribution}).

\subsection{(Change of) location}

TODO: considering moving this section to the case section

Various semantic roles can be summarized as \category{source}, 
and the source clausal dependents -- adjunct or complement -- have the following properties.
Note that we are dealing with a \emph{group} of clausal dependents.
\begin{itemize}
    \item \emph{Coding of semantic role}: 
        A source argument
        can be the position from which an object moves 
        (\concept{ablative of source})
        or the source in a separation event 
        (\concept{ablative of separation}: \translate{remove}, \translate{deprive}),
        or the place where something comes into being 
        (\concept{ablative of material}, \translate{birth}, \translate{origin}), 
        or the cause of something (``the source of the event'', \concept{ablative of cause});
        the agent in the passive voice possibly 
        comes from one of the figurative use of the ablative as well.
    \item \emph{Case marking}: a source argument is in the ablative case.
        It may come together with the prepositions \form{ex} or \form{ab}.
    \item \emph{Passivization}: not available.
\end{itemize}

\subsection{Others}

There are other clausal dependents with semantic roles and case/preposition markings 
different from any other type mentioned above;
they are skipped here for brevity.
A full list of these clausal dependent slots can be found by checking the usage of each preposition. 

\section{Passivization}\label{sec:passive}

\section{Preverbs and other verbal derivations}

\section{Verb frames}

I will generally follow the classification in \citet[\citechap{4}]{Pinkster1};
some verb classes enumerated by Pinkster are discussed in more details 
in \prettyref{sec:complement-clause-construct},
since a full account of their behaviors is closely related to the structure of complement clauses.

\subsection{Prototypical intransitive verbs}\label{sec:prototypical-intransitive}

\subsubsection{The \classify{motion} type}


\subsection{Prototypical transitive verbs}\label{sec:prototypical-transitive}

With complement-taking verbs temporarily excluded,
a prototypical transitive verb is more or less close to the \classify{affect} type,
with an A argument which is the causer of the event 

\subsection{Copular verbs}

\subsubsection{The verb \form{sum}}\label{sec:sum}

It's also possible to use \form{sum} with an indirect object, 
and the meaning because \translate{something be to [someone]_{\text{indirect object}}}.
In this case we get the possessive dative construction
\citep[\citesec{373}]{allen1903allen}.

\chapter{Clause structure}

\section{Small clauses}

accusative in \form{Deo gratias}

\section{Constituent order and the information structure}

Constituency tests reveal there is a \acs{vp} unit in Latin 
(\citealt[\citesec{1.6}]{danckaert2017development}; TODO: ref to my own analysis in conjunction),
and the \form{non}-before-auxiliary constraint 
(\citealt[\citesec{1.5}]{danckaert2017development}; TODO: my own ref)
also means there are 
This means Latin 

\chapter{Clause combining}

\begin{theorybox}{Types of coordination and subordination}{clause-combining}
    It's hard to draw a line between coordination and asymmetric (i.e. subordinating) clause linking 
    (like concessive clauses).
    Theoretically, this is because any clause combining construction follows the X-bar scheme:
    one clause is the Specifier, 
    and another clause is the Complement,
    and certain asymmetry has to be introduced.
    In English, the FANBOYS 
    -- \form{for}, \form{and}, \form{nor}, \form{but}, 
    \form{or}, \form{yet}, \form{so} -- are usually regarded as coordinating conjunctions.
    But what's the essential difference between \form{although} and \form{but}?

    On the other hand, adverbial clause constructions 
    are uncontroversially asymmetric and can in theory be distinguished from clause linking:
    in clause linking, the less important clause 
    is base-generated in one Specifier position in the CP layer of the main clause,
    so the two combined units are of roughly the same structure,
    while adverbial clauses appear in the TP layer,
    so the two combined units are of different structures:
    the adverbial clause is a CP,
    while the main clause, when the adverbial clause enters derivation,
    is a TP.
    Complement clauses, on the other hand, are first introduced in the \vP{} layer:
    they are TPs or CPs,
    while the main clause, when complement clauses enter the derivation,
    are \vP s.
    But there are still certain subtleties regarding the boundaries of \vP{}, TP, and CP.

    Relative clauses are introduced in DPs, 
    so the probability to confuse a relative clause construction 
    with a complement clause construction is small -- but still not zero.
    It can be expected that \form{I like the man dancing} and \form{I like the dancing man} 
    are realized in quite similar ways.
    Besides, some languages lack prototypical complement clause constructions 
    but have complementation strategies.
    That is, when they talk about \form{I like the dancing man},
    a speaker of such a language may be implying that he or she actually likes the man's dancing,
    though not the man's personality.
    Now comes the question:
    when there are vague evidences indicating the grammaticalization of this construction,
    should we now claim the language has already developed a complement clause construction?

    It's still possible to do the same thing 
    -- largely symmetric coordination and certainly asymmetric subordination -- 
    completely with \vP s.
    The former results in clause chaining \citep{nonato2014clause},
    while the latter results in serial verb constructions.
    These construction types, however, are absent in Latin, and I will not go deep into them in this note.
\end{theorybox}

In Latin there is no serial verb constructions.
Subordination strategies can be neatly summarized into 
complement clauses, relative clauses and adverbial clauses.

\section{Clause linking and conjunctions}

\section{Adverbial clauses}



\section{Complement clause constructions}\label{sec:complement-clause-construct}

\subsection{Overview}\label{sec:complement-clause-construct-overview}

\section{Relative constructions}\label{sec:relative-clause}

\subsection{Agreement properties}\label{sec:relative-clause.overview.agreement}

The case of a relative pronoun is determined 
by its syntactic position in the relative clause, 
and \emph{not} the case of the antecedent,
though the number and gender categories 
are determined by agreement with the antecedent.

\chapter{Examples of texts}

Below are some examples of Latin texts, 
in an order from the easiest to the hardest,
with remarks on their vocabulary and grammar. 

\section{Liturgy texts}

\subsection{Short formulae in the Roman Mass}

Examples in this section are short formulae found in the Roman Mass
in the order of their appearance.
In (\prettyref{ex:text.mass.1}, \prettyref{ex:text.mass.2}),
\form{nomine} and \form{patris} are third third declension nouns, 
while \form{spiritus} is a fourth declension noun. 

\begin{exe}
    \ex\label{ex:text.mass.1} \gll In Nomine Patris, et Filii, et Spiritus Sancti. \\
    in name-\category{sg}.\category{abl} Father-\category{sg}.\category{gen} 
    and Son-\category{sg}.\category{gen} 
    and \category{spirit}(\category{m})-\category{sg}.\category{gen}
    holy-\category{m}.\category{sg}.\category{gen} \\
    \glt \translate{In the name of the Father, and of the Son, and of the Holy Spirit.}

    \ex\label{ex:text.mass.2} \gll -- Dominus vobiscum. -- Et cum spiritu tuo. \\
    {} Lord(\category{m})-\category{sg}.\category{nom} 
    \category{2pl}.\category{abl} 
    {} and with spirit(\category{m})-\category{sg}.\category{abl} 
    your-\category{m}.\category{sg}.\category{abl} \\
    \glt \translate{-- The Lord be with you. -- And with your spirit.}
    
    \ex 
\end{exe}

\subsection{Nicene Creed}

\begin{exe}
    \ex \gll Credo in unum Deum, Patrem omnipotentem, \\
    believe-\category{ind}.\category{pres}.\category{1sg} in 
    one-\category{m}.\category{sg}.\category{acc} 
    God(\category{m})-\category{sg}.\category{acc} 
    father(\category{m})-\category{sg}.\category{acc}
    omnipotent-\category{m}.\category{sg}.\category{acc} \\
    \glt \translate{I believe in one God, (the) omnipotent Father,} 
    \ex \gll factorem caeli et terrae, visibilium omnium et \\ 
    maker- \\
    \glt \translate{maker of}
\end{exe}

\section{Vulgate bible}

\subsection{Excerpts in John 1}\label{sec:text.vulgate.john}

\begin{exe}
    \ex\label{ex:text.john.1.1} 
    \gll in principio erat Verbum et Verbum erat apud Deum et Deus erat Verbum \\
    in {} be.\acs{imperfect}  \\
    \glt \translate{In the beginning} (John 1:1)
    
    \ex\label{ex:text.john.1.3}
    \gll omnia per ipsum facta sunt 
    et sine ipso factum est nihil quod factum est \\
    all-\category{n}.\category{pl}.\category{nom} through \category{dem}-\category{acc}
    make.\category{pprt}-\category{n}.\category{pl}.\category{nom} 
    be.\category{ind}.\acs{present}.\category{3pl} 
    and without \category{dem}.\category{abl} 
    make.\category{pprt}-\category{n}.\category{sg}.\category{nom} 
    be.\category{ind}.\acs{present}.\category{3sg}
    nothing.\category{nom}
    \category{rel}.\category{n}.\category{3sg}
    make.\category{pprt}-\category{n}.\category{sg}.\category{nom} 
    be.\category{ind}.\acs{present}.\category{3sg} \\
    \glt \translate{All have been made through exactly this (i.e. the Word of Lord),
    and without exactly this, nothing that has been made has been made.}
\end{exe}

As an example, 
below I show how
(\prettyref{ex:text.john.1.3}) can be parsed.
First we can see a \form{et} dividing the sentence into two branches.
\begin{enumerate} 
    \item For the first branch, 
        we know \form{omni-} is a quantifier meaning \form{all},
        and morphologically it's a twin-termination third declension adjective; 
        then from \prettyref{tbl:declension-ending-nouns-list}
        and the fact that we are dealing with a third declension word, 
        the ending \form{-a} means neutral and \category{pl}.\category{nom}/\category{acc}/\category{voc}. 
        The vocative case is of course impossible here. 
    \item The word \form{per} is a preposition taking an accusative object. 
        \form{Ipsum} is a basic identity demonstrative, 
        with the meaning of ``exactly this''. 
        Since it follows \form{per}, 
        the ending \form{-um} here seems to be the accusative case marker, 
        instead of a neutral nominative case marker. 
    \item The sequence \form{facta sunt} contains 
        the indicative perfect 3pl copula \form{sunt}, 
        and in \form{facta}, we see the supine stem \form{fact-} 
        of the verb \form{faci\={o}} \translate{make}. 
        The second fact means 
        \form{facta} should be the perfect passive participle in a certain inflection form.
        Then \form{facta sunt}, collectively, 
        is the indicative passive perfect 3pl form of \form{faci\={o}}.
        (Here we are fortunate: 
        it's possible that \form{facta} and \form{sunt} get scattered to different places.)
        The \form{-a} ending can again be looked up for in \prettyref{tbl:declension-ending-nouns-list}:
        the possibilities are \category{pl}.\category{nom}/\category{acc}/\category{voc} -- 
        note that the first declension singular possibilities 
        are excluded by the fact that \form{sunt} is in plural form. 
        We expect \form{facta} to be nominative 
        because it has to agree with the subject, which is always nominative 
        and it turns to be possible. 
        \item Now we should link things together. 
            The open ends are: 
            the case of \form{omnia}, 
            and the (3pl) subject of \form{facta sunt}.
            Then quite obviously, 
            we find \form{omnia} should be in the subject position, 
            and therefore everything works well. 
        \item We can also check gender agreements to make sure our reading is correct.
\end{enumerate}
The second half is done in similar manners. The structure of the text looks like this: 
\begin{exe}
    \sn {} [[omnia]_{\text{subject}} [per ipsum]_{\text{instrument:\acs{pp}}} [facta sunt]_{\text{verbal complex}}]_{\text{coord}} 
    et [[sine ipso]_{\text{adverbial:\acs{pp}}} [factum est]_{\text{verbal complex}} [nihil [quod factum est]_{\text{rel}}]_{\text{subject}}]_{\text{coord}}
\end{exe}

\section{Aeneid}

\subsection{Introduction}



\begin{exe}
    \ex\label{ex:text.aeneid.1.1} \gll Arma virumque cano, Troiae qui primus ab oris
    Italiam, fato profugus, Laviniaque venit litora, 
    multum ille et terris iactatus et alto
    vi superum saevae memorem Iunonis ob iram;
    multa quoque et bello passus, dum conderet urbem,  
    inferretque deos Latio, genus unde Latinum,
    Albanique patres, atque altae moenia Romae. \\
    weapon(\category{n})-\category{pl}.\category{acc} 
    man(\category{m})-\category{sg}.\category{acc}=and 
    sing-\category{ind}.\category{pres}.\category{1sg}
    Troy-\category{sc}.\category{gen} 
    \category{rel}.\category{m}.\category{sg}.\category{nom}
    first-\category{m}.\category{sg}.\category{nom} 
    from shore(\category{f})-\category{pl}.\category{abl} 
    Italy(\category{f})-\category{sg}.\category{acc} 
    fate(\category{n})-\category{sg}.\category{abl} 
    exiled-\category{m}.\category{sg}.\category{nom} 
    Lavinia-TODO=and  
    go.to-\category{ind}.\category{pres}.\category{3sg} 
    shore(\category{n})-\category{pl}.\category{acc} \\
    \glt \translate{I sing weapons and a man, 
    who was the first from the shores of Troy to Italy, 
    was by fate exiled, 
    and }
\end{exe}

In (\prettyref{ex:text.aeneid.1.1}),
it should be noted that \form{arma} is in plural only.
The \form{qui} clause is an example of the rule 
that the relative pronoun doesn't agree in case 
with the antecedent (\prettyref{sec:relative-clause.overview.agreement}).
The copula is omitted in the \form{qui} clause.


\bibliographystyle{plainnat}
\bibliography{latin,theory}

\end{document}