\documentclass{article}

\usepackage{geometry}
\usepackage{titling}
\usepackage{titlesec}
\usepackage{paralist}
\usepackage{footnote}
\usepackage{enumerate}
\usepackage{amsmath, amssymb, amsthm}
\usepackage{gb4e}
\noautomath
\usepackage{bbm}
\usepackage{soul}
\usepackage{graphicx}
\usepackage{siunitx}
\usepackage[table,xcdraw]{xcolor}
\usepackage{tikz}
\usepackage[ruled, vlined, linesnumbered, noend]{algorithm2e}
\usepackage{xr-hyper}
\usepackage[colorlinks]{hyperref} % linkcolor=black, anchorcolor=black, citecolor=black, filecolor=black
\usepackage[most]{tcolorbox}
\usepackage{caption}
\usepackage{subcaption}
\usepackage{booktabs}
\usepackage{multirow}
\usepackage[figuresright]{rotating}
\usepackage{acro}
\usepackage[round]{natbib} 
\usepackage{nameref,zref-xr}
\zxrsetup{toltxlabel}
\zexternaldocument*[cgel-]{../English/cambridge}[cambridge.pdf]
\usepackage{prettyref}

\geometry{left=3.18cm,right=3.18cm,top=2.54cm,bottom=2.54cm}
\titlespacing{\paragraph}{0pt}{1pt}{10pt}[20pt]
\setlength{\droptitle}{-5em}

\DeclareMathOperator{\timeorder}{\mathcal{T}}
\DeclareMathOperator{\diag}{diag}
\DeclareMathOperator{\legpoly}{P}
\DeclareMathOperator{\primevalue}{P}
\DeclareMathOperator{\sgn}{sgn}
\newcommand*{\ii}{\mathrm{i}}
\newcommand*{\ee}{\mathrm{e}}
\newcommand*{\const}{\mathrm{const}}
\newcommand*{\suchthat}{\quad \text{s.t.} \quad}
\newcommand*{\argmin}{\arg\min}
\newcommand*{\argmax}{\arg\max}
\newcommand*{\normalorder}[1]{: #1 :}
\newcommand*{\pair}[1]{\langle #1 \rangle}
\newcommand*{\fd}[1]{\mathcal{D} #1}

\newcommand*{\citesec}[1]{\S~{#1}}
\newcommand*{\citechap}[1]{chap.~{#1}}
\newcommand*{\citefig}[1]{Fig.~{#1}}
\newcommand*{\citetable}[1]{Table~{#1}}
\newcommand*{\citefootnote}[1]{footnote~{#1}}

\newrefformat{sec}{\citesec{\ref{#1}}}
\newrefformat{fig}{\citefig{\ref{#1}}}
\newrefformat{tbl}{\citetable{\ref{#1}}}
\newrefformat{chap}{\citechap{\ref{#1}}}

\usetikzlibrary{arrows,shapes,positioning}
\usetikzlibrary{arrows.meta}
\usetikzlibrary{decorations.markings}
\tikzstyle arrowstyle=[scale=1]
\tikzstyle directed=[postaction={decorate,decoration={markings,
    mark=at position .5 with {\arrow[arrowstyle]{stealth}}}}]
\tikzstyle ray=[directed, thick]
\tikzstyle dot=[anchor=base,fill,circle,inner sep=1pt]


\tcbuselibrary{skins, breakable, theorems}

\newtcbtheorem[number within=chapter]{infobox}{Box}%
  {colback=blue!5,colframe=blue!65,fonttitle=\bfseries, breakable}{infobox}

\newcommand*{\concept}[1]{\textbf{#1}}
\newcommand*{\term}[1]{\emph{#1}}
\newcommand*{\corpus}[1]{\emph{#1}}

\newcommand*{\vP}{\textit{v}P}

\DeclareAcronym{blt}{short = BLT, long = Basic Linguistic Theory}
\DeclareAcronym{cgel}{short = CGEL, long = The Cambridge Grammar of the English Language}
\DeclareAcronym{dm}{short = DM, long = Distributed Morphology}
\DeclareAcronym{tag}{long = Tree-adjoining grammar, short = TAG}
\DeclareAcronym{sfp}{long = sentence final particle, short = SFP}
\DeclareAcronym{vp}{long = verb phrase, short = VP}
\DeclareAcronym{np}{long = noun phrase, short = NP}
\DeclareAcronym{cls}{long = classifier, short = CLS}
\DeclareAcronym{dist}{long = distal, short = DIST}
\DeclareAcronym{prox}{long = proximate, short = PROX}
\DeclareAcronym{dem}{long = demonstrative, short = DEM}
\DeclareAcronym{dur}{long = durative, short = DUR}
\DeclareAcronym{neg}{long = negative, short = NEG}
\DeclareAcronym{tam}{long = {Tense, Aspect, Mood}, short = TAM}

% Disable unsupported commands in bookmark titles 
\pdfstringdefDisableCommands{%
  \def\\{}%
  \def\texttt#1{<#1>}%
  \def\mathbb#1{#1}%
}
\pdfstringdefDisableCommands{\def\eqref#1{(\ref{#1})}}

\makeatletter
\pdfstringdefDisableCommands{\let\HyPsd@CatcodeWarning\@gobble}
\makeatother

\newcommand{\cgel}{\href{../English/cambridge.pdf}{my notes about CGEL}}

\title{Notes about Latin grammar}
\author{Jinyuan Wu}

\begin{document}

\maketitle

\automath

\section{Theoretical framework and descriptive framework}

\subsection{Tree diagrams}

The theoretical orientation can be found in \cgel, especially \citesec{\ref{cgel-sec:theory}}.
There is one barrier to adopt a surface-oriented binary-branching tree diagram analysis,
since Latin has largely arbitrary constituent order 
Many works have been attributed to this topic 
(\citealt{danckaert2011left,devine2006latin}, among others),
but whether they are enough to account for the variations observed in Latin is questioned
\citep{spevak2007latin},
and all of them (unsurprisingly) involve lots of movements,
which are not acceptable for a surface-oriented analysis --
indeed, translation between contemporary Minimalist theory 
and the more surface-oriented typologically oriented theory (the so-called \ac{blt})
is not trivial and needs special attention \citep{clausetypology}.
Note that, however, that argumentation pertaining to c-command relations, e.g. binding,
has been studied in the generative literature \citep{mateu2017latin},
and hence for the surface-oriented descriptive target in this note,
positing a \acs{cgel}-like surface-oriented framework 
plus a (possibly pragmatic) scrambling operation is still a good idea:
the former visualizes relevant grammatical relations and categories,
while the latter respects the free-order property.

\subsection{Distinction between form and function}

% TODO: 关于分词是“名词”之类的说法

\subsection{Terminology}

\ac{np}

\section{Morphosyntactic overview and topological information}

In the following sections,
I discuss grammatical systems contained in all levels of Latin grammar,
their realizations and typological information.

\subsection{Note about section organization}\label{sec:organization}

I intentionally do not employ the section division 
an experienced grammar writer will use to describe a language with rich morphology,
for example \citet[\citechap{2}]{jacques2021grammar},
to present Latin in a more ``content-free'' way,
dealing with only abstract (and typologically comparative) aspects of the grammar.
Should the organization of \citet[\citechap{2}]{jacques2021grammar} be used
and hence constructions are described in a bottom-up way,
with grammatical categories being introduced in discussions about smallest constructions that show them,
\prettyref{sec:alignment-abs} would be placed in a ``core and oblique arguments'' section,
% TODO: complete the list

It can be seen that describing a language purely in terms of 
``what happens in what construction''
is usually cumbersome.
\prettyref{sec:alignment-abs} is obviously too long, and deserves at least a subsection.
But since it all happens within the clause,
it has to be placed under the subsection about clause structure.
\citet{jacques2021grammar}, on the other hand, introduces categories in a bottom-up manner,
and thus there is no section devoted to the clause structure,
so contents in \prettyref{sec:alignment-abs} and the rest of \prettyref{sec:clause-structure-abs}
can be well split into two sections, 
which are \citesec{2.4} and \citesec{2.5} in his book.
On the other hand, 
sections from \prettyref{sec:voice-abs} to \prettyref{sec:force-abs} are too short.
They can be all inserted into \prettyref{sec:verb-inflection-abs}.

\subsection{Parts of speech}

\subsection{Clause structure}\label{sec:clause-structure-abs}

\subsubsection{Argument structure and alignment}\label{sec:alignment-abs}

Latin is a typical nominative-accusative language, both syntactically and morphologically.
A subject can be identified for all clauses, though it is frequently omitted.
Grammatical behaviors restricted to the subject are summarized in the following list: 
\begin{itemize}
    \item \emph{Case marking} (\prettyref{sec:case-abs}): 
    nominative for finite clauses, accusative for infinitives.
    \item \emph{}
\end{itemize}

A minimal clause -- without any adjuncts, negation, etc. -- can therefore be analyzed as 
a subject plus a predicate,
with the predicate headed by the predicator (which can only be filled by a verb in Latin) 
and its internal arguments
(the subject is the external argument).

\subsubsection{Voice}\label{sec:voice-abs}

Latin only has active (the canonical one) and passive voices.
Passivization can be done morphologically in all circumstances 
except the case of the passive perfective,
which is realized resembling the English passive,
i.e. via a copula and the perfect passive participle.

% TODO: ditransitive

\subsubsection{Polarity}\label{sec:polarity-abs}

Unlike languages like Japanese, polarity is not marked morphologically in Latin.
Latin realizes negation in a largely regular syntactic way: 
the negation operator \corpus{n\={o}n} can be placed into the clause.
% TODO: linear order, scope, etc.

\subsubsection{\ac{tam}}\label{sec:tam-abs}

Latin has the realis/irrealis contrast in mood (or modality in \ac{blt} terms), 
which are much more frequently called as the indicative/subjective distinction.

\subsubsection{Finiteness}\label{sec:finite-abs}

Besides the usual canonical clauses, Latin has several nonfinite clauses -- much more than English.


\subsubsection{Clause types and illocutionary forces}\label{sec:force-abs}

The term \term{clause type} is reserved for the syntactic marking of illocutionary force in \ac{cgel}.
Latin has three syntactically coded clause types:
declarative, interrogative and imperative.
The interrogative clause does not have much syntactic markedness,
and hence is usually not considered as a separate clause type in Latin grammars.
The imperative is restricted in TAM features:
the realis/irrealis distinction vanishes,
and the aspect feature is unavailable.

\subsubsection{Verb inflection}\label{sec:verb-inflection-abs}

Traditional grammars place the labels \term{voice}, \term{tense}, \term{mood}, etc. on the verb,
and hence the big, big conjugation table.
From a modern perspective, this is not quite correct:
labels like \term{voice}, \term{tense}, \term{mood} etc. are introduced by functional heads 
in the derivational process of the clause 
and should be considered, in an abstract and cross-linguistic way, 
as labels attached to the clause, not the predicator (usually filled by a verb).
Specifically, from a cross-linguistic way,
these labels are not to be considered as labels of verb conjugation:
the number of conjugation forms and how to name them should be decided 
solely with syntactic distributional tests. 
If in clausal environment 1 and clausal environment 2,
the verb has exactly the same appearance,
then we say there is only one verbal conjugation form in these two environments, not two.
It is thus in principle not appropriate to label verb conjugation forms according to 
clausal grammatical categories.
However, from a more practical (and describe-a-language-in-its-own-terms) perspective,
since in Latin (and languages with rich verbal morphology, like Japhug), 
these categories are marked on the verb and nowhere else,
and there are never two forms identical when changing these clausal categories,
naming verb conjugation forms with respect to the clausal environment
is well justified.

It also follows from the modern linguistic view taken here 
that in Latin, the passive perfect forms of a lexical verb, 
strictly speaking, do not exist, since what changes form is the copula and not the lexical verb,
the latter being always in the perfect passive participle form.
Nonetheless, showing these so-called ``conjugation'' form on the conjugation table 
is more user-friendly, and this is still the standard practice taken when studying and teaching Latin.



\subsubsection{Verbal derivation}\label{sec:verb-derivation-abs}

\subsubsection{Constituent order in the clause}

\subsection{\Acl{np}s and prepositions}

\subsubsection{Unattested categories}

Unlike the case in English, a NP in Latin does not have syntactic marking of definiteness,
and hence the determiner function (and related parts of speech, like the article) is completely absent.
Definiteness can be indirectly marked by demonstratives,
which is also how descendants of Latin obtained articles: by grammaticalized demonstratives.

\subsubsection{The head}

\subsubsection{Case}\label{sec:case-abs}

\subsubsection{Prepositional phrases}



\subsection{Subordination}

\section{Existing grammars, reference and pedagogical}

\subsection{Learning Latin}

\subsection{Reference grammars}

Most traditional Latin grammars organize grammatical systems according to their morphological realizations.
The reasons are summarized in \prettyref{sec:organization}.

\section{Nouns}

\section{\Acl{np}s}

\section{Verb}

\bibliographystyle{plainnat}
\bibliography{latin-notes}

\end{document}