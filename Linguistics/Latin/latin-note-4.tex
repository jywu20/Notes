\documentclass[a4paper, oneside, 12pt]{report}

\usepackage{libertinus}
\usepackage{geometry}
\usepackage{float}
\usepackage{titling}
\usepackage{titlesec}
\usepackage{paralist}
\usepackage{footnote}
\usepackage[inline]{enumitem}
\usepackage{amsmath, amsthm}
\usepackage{gb4e}
\noautomath
\usepackage{bbm}
\usepackage{soul}
\usepackage{graphicx}
\usepackage{siunitx}
\usepackage[table,xcdraw]{xcolor}
\usepackage{tikz}
\usepackage[ruled, vlined, linesnumbered, noend]{algorithm2e}
\usepackage{xr-hyper}
\usepackage[colorlinks,citecolor=purple]{hyperref} % linkcolor=black, anchorcolor=black, citecolor=black, filecolor=black
\usepackage[most]{tcolorbox}
\usepackage{caption}
\usepackage{subcaption}
\usepackage{booktabs}
\usepackage{multirow}
\usepackage[figuresright]{rotating}
\usepackage{acro}
\usepackage[round]{natbib} 
\usepackage{nameref,zref-xr}
\zxrsetup{toltxlabel}
\zexternaldocument*[alignment-]{../alignment/alignment}[alignment.pdf]
\zexternaldocument*[exercise1-]{../Exercise/2021-3}[2021-3.pdf]
\zexternaldocument*[general-]{../methodology/glossing}[glossing.pdf]
\usepackage{prettyref}

\geometry{left=3.18cm,right=3.18cm,top=2.54cm,bottom=2.54cm}
\titlespacing{\paragraph}{0pt}{1pt}{10pt}[20pt]
\setlength{\droptitle}{-5em}

\DeclareMathOperator{\timeorder}{\mathcal{T}}
\DeclareMathOperator{\diag}{diag}
\DeclareMathOperator{\legpoly}{P}
\DeclareMathOperator{\primevalue}{P}
\DeclareMathOperator{\sgn}{sgn}
\newcommand*{\ii}{\mathrm{i}}
\newcommand*{\ee}{\mathrm{e}}
\newcommand*{\const}{\mathrm{const}}
\newcommand*{\suchthat}{\quad \text{s.t.} \quad}
\newcommand*{\argmin}{\arg\min}
\newcommand*{\argmax}{\arg\max}
\newcommand*{\normalorder}[1]{: #1 :}
\newcommand*{\pair}[1]{\langle #1 \rangle}
\newcommand*{\fd}[1]{\mathcal{D} #1}

\newcommand*{\citesec}[1]{\S~{#1}}
\newcommand*{\citechap}[1]{chap.~{#1}}
\newcommand*{\citefig}[1]{Fig.~{#1}}
\newcommand*{\citetable}[1]{Table~{#1}}
\newcommand*{\citepage}[1]{p.~{#1}}
\newcommand*{\citepages}[1]{pp.~{#1}}
\newcommand*{\citefootnote}[1]{fn.~{#1}}

\newrefformat{sec}{\citesec{\ref{#1}}}
\newrefformat{fig}{\citefig{\ref{#1}}}
\newrefformat{tbl}{\citetable{\ref{#1}}}
\newrefformat{chap}{\citechap{\ref{#1}}}
\newrefformat{fn}{\citefootnote{\ref{#1}}}
\newrefformat{box}{Box~\ref{#1}}
\newrefformat{ex}{\ref{#1}}

\tcbuselibrary{skins, breakable, theorems}

\AtBeginEnvironment{infobox}{\small}
\AtBeginEnvironment{theorybox}{\small}

\newtcbtheorem[number within=chapter]{infobox}{Box}{
    enhanced,
    boxrule=0pt,
    colback=blue!5,
    colframe=blue!5,
    coltitle=blue!50,
    borderline west={4pt}{0pt}{blue!65},
    sharp corners,
    fonttitle=\bfseries, 
    breakable,
    before upper={\parindent15pt\noindent}}{box}
\newtcbtheorem[number within=chapter, use counter from=infobox]{theorybox}{Box}{
    enhanced,
    boxrule=0pt,
    colback=orange!5, 
    colframe=orange!5, 
    coltitle=orange!50,
    borderline west={4pt}{0pt}{orange!65},
    sharp corners,
    fonttitle=\bfseries, 
    breakable,
    before upper={\parindent15pt\noindent}}{box}
\newtcbtheorem[number within=chapter, use counter from=infobox]{learnbox}{Box}{
    enhanced,
    boxrule=0pt,
    colback=green!5,
    colframe=green!5,
    coltitle=green!50,
    borderline west={4pt}{0pt}{green!65},
    sharp corners,
    fonttitle=\bfseries, 
    breakable,
    before upper={\parindent15pt\noindent}}{box}
\newtcbtheorem[number within=chapter, use counter from=infobox]{todobox}{Box}{
    enhanced,
    boxrule=0pt,
    colback=red!5,
    colframe=red!5,
    coltitle=red!50,
    borderline west={4pt}{0pt}{red!65},
    sharp corners,
    fonttitle=\bfseries, 
    breakable,
    before upper={\parindent15pt\noindent}}{box}

\newcommand*{\concept}[1]{\textbf{#1}}
\newcommand*{\term}[1]{\emph{#1}}
\newcommand{\form}[1]{\emph{#1}}
\newcommand*{\category}[1]{\textsc{#1}}

%region Acronym

% Theory
\DeclareAcronym{blt}{short = BLT, long = Basic Linguistic Theory}
\DeclareAcronym{cgel}{short = CGEL, long = The Cambridge Grammar of the English Language}
\DeclareAcronym{dm}{short = DM, long = Distributed Morphology}
\DeclareAcronym{tag}{long = Tree-adjoining grammar, short = TAG}

% History

\DeclareAcronym{pie}{long = proto-Indo-European, short = PIE}

% Roles

\DeclareAcronym{sfp}{long = sentence final particle, short = SFP}
\DeclareAcronym{np}{long = noun phrase, short = NP}
\DeclareAcronym{vp}{long = verb phrase, short = VP}
\DeclareAcronym{pp}{long = preposition phrase, short = PP}
\DeclareAcronym{adjp}{long = adjective phrase, short = AdjP}
\DeclareAcronym{advp}{long = adverb phrase, short = AdvP}
\DeclareAcronym{cc}{long = copular complement, short = CC}
\DeclareAcronym{cs}{long = copular subject, short = CS}
\DeclareAcronym{tam}{long = {tense, aspect, mood}, short = TAM}
\DeclareAcronym{copula}{long = copula, short = COP}

% Pronouns 

\DeclareAcronym{dist}{long = distal, short = \textsc{dist}}
\DeclareAcronym{prox}{long = proximate, short = \textsc{prox}}
\DeclareAcronym{dem}{long = demonstrative, short = \textsc{dem}}

% TAME, negative

\DeclareAcronym{neg}{long = negative, short = \textsc{neg}}
\DeclareAcronym{past}{long = \textsc{past}, short = \textsc{pst}}
\DeclareAcronym{imperfect}{long = \textsc{imperfect}, short = \textsc{impf}}
\DeclareAcronym{present}{long = \textsc{present}, short = \textsc{pres}}
\DeclareAcronym{perfect}{long = \textsc{perfect}, short = \textsc{perf}}
\DeclareAcronym{future}{long = \textsc{future}, short = \textsc{fut}}
\DeclareAcronym{pluperfect}{long = \textsc{pluperfect}, short = \textsc{plup}}
\DeclareAcronym{future perfect}{long = \textsc{future perfect}, short = \textsc{fut.perf}}
\DeclareAcronym{passive}{long = passive, short = PASS}
\DeclareAcronym{indicative}{long = \textsc{indicative}, short = \textsc{ind}}
\DeclareAcronym{subjunctive}{long = \textsc{subjunctive}, short = \textsc{sjv}}
\DeclareAcronym{imperative}{long = \textsc{imperative}, short = \textsc{imp}}


%endregion

\newcommand*{\homo}[2]{#1$_{\text{#2}}$}

\newcommand{\cgel}{\href{../English/cambridge.pdf}{my notes about CGEL}}
\newcommand{\latin}{\href{../Latin/latin-notes.pdf}{my notes about Latin}}
\newcommand{\alignment}{\href{../alignment/alignment.pdf}{my notes about alignment}}
\newcommand{\exerciseone}{\href{../Exercise/2021-3.pdf}{this exercise}}
\newcommand{\general}{\href{../methodology/glossing.pdf}{this note}}

\newcommand{\ala}{à la}
\newcommand{\translate}[1]{`#1'}
\newcommand{\vP}{\textit{v}P}

\newcommand{\classify}[1]{{\textsc{#1}}}
\newcommand{\literature}[1]{\textit{#1}}

% Make subsubsection labeled
\setcounter{secnumdepth}{4}
\setcounter{tocdepth}{4}
% reset example counter every chapter (but do not include the chapter number to the label)
\counterwithin{exx}{chapter} 

\renewcommand{\bibname}{References}

\title{Note on Latin Grammar}
\author{Jinyuan Wu}

\begin{document}

\automath

\maketitle

\chapter{Introduction}

\section{The language and the speakers}

Latin was the language of the Romans
and the official language of both the Roman Republic and the Roman Empire,
and hence the official language of the Catholic Church, 
which was \emph{the} church for the Western Roman Empire. 
The international nature of the Roman Empire made Latin 
the international language around the Mediterranean Sea at that time -- 
indeed, \form{Mare Nostrum} \translate{our sea} in Latin,
and its importance in science, arts, law, religion, and literature 
lent it more than one thousand of years of life 
as a common literary language and a sacred language in western Europe
after the collapse of the Western Roman Empire 
and the emergence of the Romance language family.

As recently as the nineteen century,
Latin was still fluently used by scholars and in the Catholic Mass. 
A decline in the popularity of Latin was observed after that,
consistent with the rapid development of English
(initially also French and German and sometimes Russian)
as the language of rapidly developing natural sciences,
largely replacing the status of Latin as a scholar language.
After Vatican allowed vernacular languages being used in liturgies, 
Latin also largely lose its position in the daily use in the Catholic church. 

Latin today is still revered as \emph{a} classical language
that has deeply shaped at least the Western world,
but it has long lost the prestige of \emph{the} classical language
that every decent individual is to learn once in their life.
Still, as an important silhouette of what early Indo-European languages looked like,
as one of the earliest languages receiving systematic grammatical analysis,
and as a language that considerable deviates from its modern descendants in its grammar,
Latin deserves its status as a model organism in modern linguistics.

\section{Classification and history}\label{sec:introduction.history}

\subsection{Latin in Indo-European family}

Latin belongs to the Indo-European (IE) family.
Languages most closely genetically related to Latin include Faliscan,
a language attested from seventh-sixth century BC,
Lanuvian, and Praenestinian.
These languages form the Latino-Faliscan family.

The Sabellic family, which consists of Umbrian, Oscan, and South Picene,
was also spoken in ancient Italy.
The exact relation between Sabellian and Latino-Faliscan is not completely settled:
many postulate an Italic language family
with Latino-Faliscan and Sabellian as its two branches.

Above the level of the Italic family,
it has been proposed that Latin is genetically related to the Celtic branch. 

\subsection{Old Latin}

Pre-Classical Latin had gone dramatic evolution.
The earliest texts are still not fully deciphered today
and the Romans themselves find it hard to read old agreements they signed with the Carthaginians
\citep[\citepage{52}]{leonhardt2013latin}.
Indeed, late Old Latin texts had already shown changes that are also present in Vulgar Latin,
which are however absent in Classical texts,
suggesting archaisms in the standardization process
which makes the Classical texts looking \emph{older} than they actually are,
while late Old Latin texts are \emph{later} in the sense that they reflect 
innovations in the living language
(\prettyref{sec:introduction.history.classical.standard}).

\begin{todobox}{List of Old Latin traits}{old-latin-features}
    \begin{itemize}
        \item \form{-os} as ending
        \item texts?
    \end{itemize}
\end{todobox}

\subsection{Emergence of Classical Latin}

\paragraph*{The road to standardization}\label{sec:introduction.history.classical.standard}
The formation of what is known today as ``normal'' Latin,
i.e. the kind of Latin usually taught in Latin classes,
follows the general pattern of linguistic standardization.
The motivation of the process was closely connected to the formation of new identities,
which, in the case of standardization of Latin,
was the awareness that the Romans should focus on their own cultures
and got some sort of spiritual independence from the Greeks:
thus Virgil sought to replace \literature{Odyssey} by his \literature{Aeneid},
and not just to translate the former
\citep[\citepage{66}]{leonhardt2013latin}.
The focus on the new identify triggered the so-called ``first congress'' phenomenon,
in which broad discussion on what variety of a language is to be used
and eventually the creation of a new literary language.
In Rome, this was done by leading members of the upper society:
Cicero mentioned linguistic problems in his discussions on rhetorics,
Caesar was worried about irregularities of Latin inflectional morphology
and proposed some spelling reforms to make homophonous endings more identifiable,
and several other figures composed grammars of Latin
\citep[\citepages{60-61}]{leonhardt2013latin}.

Certain traits of Classical Latin are also recognizably deliberately refined.
The situation is the most telling when a feature is present in 
both pre-Classical texts (especially dramas, where the language is conceivably colloquial) 
and post-Classical texts but absent in Classical Latin,
meaning that this feature was artificially suppressed in standardization
(but likely had always stayed in oral uses).
Instances include missing word final \form{-m}
and \form{de} + noun phrase in place of the genitive case
\citep[\citesec{2.4.4}, \citesec{8.3.1}]{alkire2010romance}.

\paragraph*{Classical Latin as a spoken language}
Although Classical Latin underwent deliberate refinements,
it was still a true living language in its time,
namely around the first century.
Evidence for this claim comes from both letters and oration,
known to reflect the oral language according to literal evidence.

Cicero, for instance, was equally Classical in his popular oratory and in his written works,
and the fact that public oration was to be understood by as many as possible
clearly shows that Cicero's language was a real language that was actually spoken 
by at least Roman elites
\citep[\citepage{71}]{leonhardt2013latin},
which is even true for the language of legal procedures,
commonly expected to be rather technical:
although there were legal experts in ancient Rome,
the heads of tribunals were often not from their ranks,
and to capture their attentions the everyday language had to be used
\citep[\citepages{385-386,464-465}]{clackson2011companion}.
As for letters, he explicitly mentioned in his letter to her close friend that 
``as for letters, we weave them out of the language of everyday''
\citep[\citepage{428}]{clackson2011companion}.

Of course, both letters and public speeches, in all cultures, undergo deliberate refinements,
and are not direct representations of the oral language
\citep[\citepages{397-399,434-438}]{clackson2011companion}.
The problem we face here is comparable to the difficulties we face
when trying to reconstruct the oral sub-elite Latin language
from so-called sub-elite texts (\prettyref{sec:introduction.history.sub-elite.elusive}),
although we don't have the problem of (scribal) formulaic regularization.
What we are confident is that the oral, informal written, and formal written languages in Rome
between 1st century BC and 3rd century AD form a continuum,
which is rightfully known as Classical Latin,
under which several genres exist.

\paragraph*{Canonization and fixing of Latin}
After the Classical period,
Latin in official writings stopped evolution,
which is the destiny of all authorial languages
and also happened to Greek \citep[\citepages{71-72}]{leonhardt2013latin}.
Indeed, if a language first develops a writing tradition
and then undergoes considerable change,
usually this will leads to the old literature being forgotten by most (if not nearly all) readers
\citep[\citepage{144}]{leonhardt2013latin}. 
The study on what happened after a Classical Latin literary canon had appeared
therefore had to rely on less known and more indirect sources,
as is briefly discussed in \prettyref{sec:introduction.history.sub-elite}.

\subsection{Sub-elite Latin}\label{sec:introduction.history.sub-elite}

\paragraph*{Vulgar Latin as ancestor of Romance languages}
\label{sec:introduction.history.sub-elite.proto-romance}
Before we discuss Roman-era sources for non-Classical forms of Latin,
let us first look at the eventual daughters of Latin: Romance languages.
Comparison between Romance languages reveals several interesting commonalities:
neutralization of Latin long \form{\={e}} and short \form{i}
in their modern reflexes
and words commonly used in classical Latin works like \form{loquor} or \form{pulcher}
being completely missing.
With these facts, we conclude that there exists a proto-Romance,
which is a kind of Latin but deviates from Classical Latin,
in which there is no distinction between \form{\={e}} and \form{i}, 
and \form{loquor}, \form{pulcher}, etc. has been replaced
\citep[\citepages{1-3}]{herman2010vulgar}.

Now we go back to earlier sources.
Besides reconstruction based on modern Romance languages,
data on non-standard Latin can also be found in
``mistakes'' or ``barbarisms'', 
which are likely reflections of aspects of the oral language 
which has deviated from the classical works, 
in texts in late Roman Empire.
These ``mistakes'' seems to be consistent with the aforementioned proto-Romance:
for example, neutralization of \form{\={e}} and \form{i} is indeed constantly observed in these texts
\citep[\citepage{4}]{herman2010vulgar}.

Non-compliance to Classical Latin standards can be observed even in texts
that are supposed to be official and formal,
especially after the collapse of Western Roman Empire.
In \literature{Chronicel of Fredegar} (mid 7th century), for example, we observe
aforementioned Romance-like mixing of vowels
and incorrect endings which suggest they had been weakened in the oral language
(as in later Romance languages).
What is particularly telling is hypercorrections
which reveal that the author really couldn't tell some Classical Latin sounds apart
but was under the vague impression that
some sound distinctions were needed in Classical Latin
\citep[\citepages{158-159}]{leonhardt2013latin}.

\paragraph*{Unity of Vulgar Latin}
The next question is whether we can say there is one single Vulgar Latin.
Regional variance of Latin can be expected to exist,
according our experiences with contemporary languages,
and do seem to exist according to Latin authors' accounts
\citep[\citepage{116}]{herman2010vulgar}.
We are however still confident to say that the regional variance 
was not of very strong significance, for two reasons.

First, non-literary written texts was largely uniform across the whole Empire
\citep[\citepage{117}]{herman2010vulgar}:
this is partly due to the existence of 
many (Classical-like) formulaic texts that can be used everywhere,
but when deviations from Classical Latin appeared,
they too appeared everywhere in the empire
\citep[\citepage{235}]{clackson2011blackwell},
though the propagation may take some time
(e.g. see \prettyref{sec:noun.case.nominative.vulgar}),
and the statistical frequencies of non-standard features (attested ubiquitous) vary regionally
\citep[\citepage{118}]{herman2010vulgar}.

Second, Roman grammarians were keen to correct ``corrupt'' usages of Latin 
and were known to record local uses, 
like Africans did not distinguish between long and short vowels.
But they reported nothing more fundamental than that:
the most notable differences seemed to be pronunciation
\citep[\citepage{118}]{herman2010vulgar}.

Based on the available evidence,
we can define \term{Vulgar Latin} as the spoken variety of Latin
used by people who were without very formal Classical Latin-oriented school education,
which deviates from Classical Latin in a more or less homogeneous way across the Empire.

\paragraph*{Elusiveness of Vulgar Latin}\label{sec:introduction.history.sub-elite.elusive}
Although Vulgar Latin, as is mentioned above, is relative uniform,
it would be futile to attempt to compose a grammar of Vulgar Latin comparable
to a grammar of Classical Latin.
The first reason is the fact that
although regional variance in Vulgar Latin is mostly ignorable,
the language does have chronological development, 
and stylist variances in different communities and spheres are to be expected.
Conscious and unconscious prescriptive pressures can also cancel existing variances. 

The second reason is about the nature of materials we have about Vulgar Latin:
as is mentioned above, they were all ``Classicalized'' when written.
Two extreme perceptions of Vulgar Latin are both incorrect.
Vulgar Latin is \emph{not} an unattested language
that is quite different from Classical Latin and can only be reconstructed by the comparative method,
as we have texts.
But Vulgar Latin is also not another literary language besides Classical Latin, 
because these texts are strictly not truly ``Vulgar Latin texts'',
as Romans did not have field linguists to record Vulgar Latin speeches as they were,
and all ``Vulgar Latin texts'' inevitably have both Classical and Vulgar traits
\citep[\citepages{8,26}]{herman2010vulgar}.
If a feature that's absent in Classical Latin appears consistently,
it has to be a Vulgar,
but if certain deviations from Classical Latin is absent,
we do not know if it's because it's because Vulgar Latin (in the given time period)
was indeed consistent with Classical Latin
or if it's because relevant deviations have been regularized by scribes.

This is probably most obviously demonstrated in \literature{Gaius Nouius Enus},
the permission of extension of a loan written on a wax tablet dated to 39 AD.
Some consistently appearing spelling ``mistakes'' clearly show that
the spoken language of the author is not precisely the same as Classical Latin
and a large portion of the deviations are in accord with spoken Latin revealed by Pompeian graffiti.
However, some sub-standard features which are observed in \emph{earlier} periods
are absent in the text,
likely because they were stigmatized and the author didn't want to sound ``rural''.
Despite the sound changes in the phonology,
the text is still pretty comprehensible within the grammatical framework of Classical Latin,
which is not surprising given its formulaic nature.
Interestingly, some archaisms can be observed in the text,
clearly demonstrating that the author was very serious about it,
so any spelling mistake has to be because 
the author did not have access to the ``right'' pronunciations,
not because he had a sloppy attitude.
So now what we have is a mixture of some intended archaic spellings,
the Classical grammar and a large portion of Classical words,
and some sound variances cross validated by other sub-standard texts,
while other changes, attested elsewhere, are not present
\citep[\citesec{7.4.1}]{clackson2011blackwell}.

A final remark is that 
saying ``Romans spoke Vulgar Latin and wrote Classical Latin'' 
still does not accurately reflect the language situation of Roman Empire:
besides Latin, we have Greek in the eastern regions,
and bilingualism and language mixing are common
\citep[\citesec{7.2}]{clackson2011blackwell}.

\paragraph*{Christian Latin}
Now we turn to various sources of Vulgar Latin texts.
In Christian texts, a tendency is that in the Empire period,
early texts reflect \emph{more} spoken features than later texts do:
early Christians were often poorly educated,
while later Church Fathers had had a good literary education.
The same pattern can also be seen in how Neo-Latin replaced Medieval Latin.
Many works of the latter however are still Vulgar
in the sense that their purpose was to reach the widest audience possible,
and definitely some features of spoken syntax and vocabulary were mixed into the texts
\citep[\citepage{24}]{herman2010vulgar}.

\subsection{End of Latin as a living language}

\paragraph*{Evolution into Romance languages}
At some point, deviation of Vulgar Latin described in \prettyref{sec:introduction.history.sub-elite} 
finally reached a breaking point
such that it's better to stop calling it Vulgar Latin at all:
if speakers wanted to learn Classical Latin,
they could only learn it as something brand new,
not as some prescriptive rules that ``purify'' their mother tongues 
(``don't forget to use the ablative here!''),
and a speaker without any literary training would be unable to comprehend 
a Classical Latin article read to them.
The problem is when was that point.

It seems that until the 7th century AD, the spoken language was still some sort of Latin:
saints' lives were read directly to the congregations in Latin,
with the expectation that they could understand them.
In Gaul, the first piece of evidence of lack of intelligibility
was King Charlemagne's observation that even the clergy 
could not always understand the meaning of Lord's Prayer.
Later we have the famous phrase \form{lingua romana rustica}
mentioned at the council at Tour in 813
\citep[\citepage{114}]{herman2010vulgar}.
The Romance part of Strasbourg oaths give us a glance into what \form{lingua romana rustica} looked like:
the text can still be mapped to Classical Latin word-to-word,
but the forms of the words had changed so much
that intelligibility between the language reflected in the oath and Classical Latin
had been greatly weakened \citep[\citepages{300-301}]{clackson2011blackwell}.

In other areas, evidence for evolution of Latin into Romance was one or two centuries later.
And even after Classical Latin became largely unrecognizable
to speakers of Romance languages,
the memory that the local vernacular and Classical Latin have a shared origin
may still be maintained.
In Italy, as late as the fifteen century,
Latin and Italian were respectively known as \form{grammatica} and \form{volgare},
i.e. two varieties of the same language
\citep[\citepage{146}]{leonhardt2013latin}.

\paragraph*{The fate of spoken (Classical-like) Latin}
The birth of Romance languages does not necessarily mean 
the death of (Classical-like) Latin as an oral language.
In Rome this was preserved at least as late as 1000 AD,
as is exemplified by the epitaph of Pope Gregory V,
which mentions that he spoke Latin besides a possible Romance language 
\citep[\citepage{268}]{clackson2011blackwell}.

\subsection{Coverage of this note}

This note is about \concept{Classical Latin} -- the Latin of classical Latin writers -- 
and \concept{Ecclesiastical Latin} -- the kind of Latin of the Catholic church.
That's to say Old Latin, vulgar Latin (with prototypes of Romance articles), etc.
are not discussed in detail in this note.
Still, some historical knowledge is important for us to understand 
why Latin is the way it is. 

\section{Sociolinguistic status}

\subsection{Latin before and during early Republic}

\paragraph*{Etruscan: an early supraregional language}
The historical and contemporary importance of Latin 
of course doesn't endorse it as a inherently superior language. 
Indeed, we only have a handful of Latin texts before 600 BC; 
as a comparison, there are about 150 pre-600 BC Etruscan texts.
Even in the period between 600 BC and 100 BC, 
during which we have around 3000 Latin texts,
we have about 9000 Etruscan texts, 
which is three times as many as their Latin counterparts. 
Early Roman elites were partly educated in Etruscan,
which lasted to as late as 4th century BC \citep[\citepage{43}]{leonhardt2013latin}.

\paragraph*{Oscan: a competitor of Latin}
Oscan was a supraregional language before the rise of Latin,
and it seems people speaking the language were recognized by Romans as \form{Osci},
a concept based not on ethnicity but on language.
It's possible that Oscan had a literary tradition (see the discussion below about Greek),
which was not preserved in inscriptions,
probably for the same reason the great works in Latin were not carved into stones either
and that this tradition was possibly semi-oral \citep[\citepage{51}]{leonhardt2013latin}.

\paragraph*{Greek: the model of literature language}
The most notable language that was more prestigious than Latin and highly influenced Latin literature was Greek.
It is well known that the Roman written culture derived its form from the Greek.

\begin{todobox}{What type of Greek?}{greek-type}
    Koine, or Attic, or something else?
\end{todobox}

Perhaps Romans were not the only people impressed by Greek culture:
Greek tragedies were translated into Etruscan and played,
and Lucanians -- likely Oscan speakers -- appropriated Greek art for tomb paintings.
The whole Italy was deeply influenced by Greek literature.
Indeed, many early Latin poets were not native speakers of Latin: 
they were multilinguistic and lived on adapting Greek themes and genres and mixing them into Latin.
It's more than likely that they would do the same for various Italic languages.
Ennius, for example, explicitly said he had three harts, a Greek, an Oscan, and a Latin.
Maybe Oscan had its own literature as well,
which unfortunately wasn't pass to us because only Oscan inscriptions survived.
The influence of Greek culture to Romans was therefore both direct and indirect:
some Greek influences were adapted in the whole ancient Italy,
and then Romans adopted them
\citep[\citepages{49-50}]{leonhardt2013latin}.



\subsection{Latin in late Republic and early Empire}

The rise of Latin seems closely related to the rise of Romans as a politic superpower
in Italian Peninsula.
The superiority of Greek however had lasted at least to the first century AC,
when Latin literature only had its audience in Rome
but Greek works was welcome everywhere
\citep[\citepage{52}]{leonhardt2013latin}. 

\begin{todobox}{When did the status of Greek die away in Western Europe?}{greek-influence}
    When the empire collapsed?
\end{todobox}

\subsection{Latin in Medieval times}

\paragraph*{Before Carolingian Renaissance}
The prestige of Latin never vanished during the Middle Ages:
in former Roman areas, since ``barbarians'' had long been incorporated into Roman society, 
the Latin heritage was just accepted as a part of their culture 
\citep[\citepage{137}]{leonhardt2013latin},
and since missionaries played an important role in the culture of non-Roman areas, 
Latin works were highly regarded because they were essential 
for the training of future Roman Catholic clergies
\citep[\citepage{138}]{leonhardt2013latin}.

This doesn't mean authors were always capable to produce Ciceronian Latin works.
After the collapse of Western Roman Empire follows
a period when truly Classical Latin works ceased to be produced:
writings from this age were usually deeply influenced by Vulgar features 
(\prettyref{sec:introduction.history.sub-elite.proto-romance}).
It is likely that there was almost no language communities
that can use Latin for some sort of communication and maintain the diglossia status
\citep[\citepage{160}]{leonhardt2013latin}.

\paragraph*{Carolingian Renaissance}

\begin{todobox}{Carolingian Renaissance}{carolingian}
    \begin{itemize}
        \item Existing Latin grammars 
        \item Irish scholars
        \item Virgil and Cicero, for example, were read solely for the purpose of learning what's good Latin
        \citep[\citepage{170}]{leonhardt2013latin}:
        can we say the way Carolingian Renaissance revives Latin
        depended on available Latin texts at that time?
    \end{itemize}
\end{todobox}

\paragraph*{A living second language}
In the rest of the Middle Ages,
the status of Latin was somewhere between a truly dead language and a normal living language
\citep[\citepage{145-150}]{leonhardt2013latin}.
Latin was still live in the sense that
people learned Latin the same way modern people learn languages like Spanish or Mandarin,
that's to say, they learned Latin by memorizing a limited set of essential grammatical markers
(in the case of Mandarin, they are ``empty words'',
and in the case of Latin they are inflectional paradigms)
and engaging with an existing community under practical situations:
the overwhelmingly rule-based learning style was a much later invention.
On the other hand, Latin differed from a true living language
because what was acquired by the learner was always a ``learner's language'',
i.e. a possible mental grammar that is \emph{one} explanation 
of the Latin text that the learner encountered,
but probably not \emph{the} explanation that a Roman had in their mind.
When learning a true living language, on the other hand,
the learner's language is only auxiliary and is always subject to correction by native speakers.
Latin therefore was a living second language,
a language that was supposed to be living but no one spoke as their mother tongue.
The situation is not unlike English outside the core Anglosphere,
where people use English, the primary goal is often not to communicate with native speakers 
(often there is none)
\citep[\citepages{150-154}]{leonhardt2013latin}.

\paragraph*{Latin's competition with vernaculars}
In some senses, the status of Latin as \emph{the} true literary language
protected -- not eliminated -- our understanding of early European vernaculars,
because if one vernacular -- say Frankish -- was chosen by Charlemagne as the official language,
it soon would undergo the first congress phenomenon aforementioned
and then developed a high variety,
and the low varieties would then be poorly documented,
as in the case of Vulgar Latin
\citep[\citepages{143-144}]{leonhardt2013latin}.

Of course, if the status of Latin was so strong,
no written materials would be produced in vernaculars at all.
The interaction between Latin and the vernaculars therefore is an interesting topic.
A general observation is that written vernaculars appeared first in non-Roman areas,
and then in formerly Roman areas.
This could partially be explained by a residue of diglossia in formerly Roman areas:
since Classical Latin could still be partially (although highly limitedly) comprehended 
by speakers of local vernaculars,
no attempts would be make to write anything using local vernaculars,
just like how modern Greeks once rejected translating the Bible into Demotic Greek.

This mechanism however can only explain the \emph{absence} of vernacular literature in formerly Roman areas,
but the \emph{presence} of vernacular literature in non-Roman area 
does not automatically come into being because of lack of Latin capacity:
it's definitely possible that these areas were so culturally impoverished 
that no texts -- vernacular or Latin -- were published at all.
This however is definitely not the case 
because we actually observe a \emph{positive} correlation 
between vernacular literature and Latin literature
\citep[\citepages{167-168}]{leonhardt2013latin}.

\paragraph*{Latin's competition with Greek and Arabic}

Unlike Greek, Latin had never gained much interest outside Western Europe.

\begin{infobox}{Latin literature outside Western Europe}{latin-western-europe}
    \begin{itemize}
        \item Arabs' attitude 
        \item Byzantine
    \end{itemize}
\end{infobox}

\section{Previous studies}

\begin{todobox}{Topics in previous studies of Latin}{latin-existing-study}
    \begin{itemize}
        \item Grammars in ancient Rome 
        \item In Medieval times
        \item Latin and functional-typological tradition
    \end{itemize}
\end{todobox}

\section{Texts}

A brief summary of genres of Latin texts has already been provided in \prettyref{sec:introduction.history}.
Here we elaborate more on resources

\begin{todobox}{List of Latin texts}{latin-texts}
 \begin{itemize}
    \item Old Latin texts: archaeology
    \item Old Latin texts: Roman authors
    \item The road to standardization
    \item The Classical canon: Cicero, etc.
    \item Legal and technical Latin, and Latin historiography
    \item No one read Classical Latin after Christianization, except in Ireland
    \item Latin inscriptions, curse tables, etc. in Empire period
    \item Medieval Latin, Neo Latin
\end{itemize}
   
\end{todobox}

\chapter{Phonology and the writing system}

\section{Phonemes and the alphabet}

\paragraph*{The writing system}
Although the phoneme inventory of a language 
often is not accurately reflected by its preferred writing system, 
since Latin is a classical language 
and no ancient Roman is alive today, 
its phonology has to be inferred from known texts. 

The most accepted writing system of Latin developed into 
what we call \concept{Latin letters} -- or the \concept{Roman alphabet} -- today, 
which is the most widely used writing system in the world.
\concept{Old Italic scripts},
used by Early Old Latin inscriptions 
as well as neighbor languages,
show a larger degree of variation, 
which clearly derived from Greek letters.
The standard Latin alphabet derived from old Italic scripts.
Note that ancient Romans only used the big letters;
the small letters was invented during the era of Charlemange.

The letter \form{J} was not used by ancient Romans, 
although we sometimes see \form{I} appearing at the start of a word 
and therefore possibly represents the semivowel /j/.
The letters \form{U} and \form{W} are also not used.
Similar to the case of \form{I},
the letter \form{V} is used to represent 
what appears to be the semivowel /w/ 
as well as the vowel /u/. 
The letter \form{K} is an archaic one 
and only appears before \form{A} in a small number of words
\citep[\citechap{2}]{oniga2014latin}.
The letters \form{Y} and \form{Z} are used to spell Greek words that 
include TODO: Y and the voiced dental affricate, respectively.

\paragraph*{Consonant inventory}
The Latin consonant inventory is therefore given by \citet[\citetable{3.1}]{oniga2014latin}.
Note that the letter \form{X} is a double consonant: 
it means /ks/ or /gs/.

\paragraph*{Vowel inventory}
Two semivowels -- /j/ and /w/ -- can be recognized,
which appear as \form{i} or \form{u}.

The vowels are given by \citet[\citetable{3.2}]{oniga2014latin}.
Each vowel has a long variety and a short variety.

\paragraph*{Historical evolution of Latin vowels}


\section{Prosody}



\section{Morphophonological rules}\label{sec:phonology.morphological}

Some phonological rules in Latin are sensitive to morpheme boundaries.
We can therefore assert that at least in a historical stage, 
Latin speakers had a clear sense of morphemes 
as real phonological objects,
instead of mere theoretical models.


\subsection{Vowel deletion}\label{sec:phonology.rule.deletion}

Short vowels \form{a}, \form{o} and \form{e} 
become zero before a morpheme boundary or another vowel
\citep[\citesec{8.3}]{oniga2014latin}.
This rule is exemplified by the absence 
of the thematic vowel in both declension (TODO: rosis)
and conjugation (TODO).

\subsection{Vowel shortening}\label{sec:phonology.rule.shortening}

A long vowel before another vowel or morpheme boundary 
is not deleted, but shortened.
Again this is exemplified (\prettyref{sec:tense-mood-marking}, TODO: ref)

Also, in the final syllable of a phonological word,
a long vowel before a consonant except \form{s} is also shortened.
Counterexamples when the vowel is not in the final syllable exist,
like \form{b\={a}ris} \translate{a type of flat-bottomed freighter used on the Nile in Ancient Egypt}.

A long vowel is generally shortened before a sequence 
containing a liquid or nasal and a following stop consonant,
like \form{nt}.
This rule comes from an older Indo-European sound law: 
the Osthoff's Law
\citep[\citepage{55}]{oniga2014latin}.

\subsection{Vowel weakening}

When a short syllable is in a medial, open syllable,
and a morpheme boundary occurs immediately before, within or after the syllable,
it becomes \form{i} 
(\citealt[\citepage{55}]{oniga2014latin}; TODO: ref).

\subsection{Vowel lengthening}

A vowel is always \emph{lengthened} before \form{nf} and \form{ns}
\citep[\citepage{55}]{oniga2014latin}.

\chapter{Syntactic overview}

We carry out this report round by round.
In the first round we distinguish words and morphemes
(\prettyref{sec:grammatical.word}),
sketching how abstract linguistic structures are linearized into utterances in Latin.

In the second round, this note takes a syntax-first approach in the second round.
Unlike traditional Latin grammars that start with morphological properties of words
and then introduce syntax to impose some constraints on how words are bound together
by constituency and dependency,
this note starts with structural possibilities of clauses (\prettyref{sec:grammatical.clause})
and noun phrases (\prettyref{sec:grammatical.np}),
and modifications to them that aren't themselves noun phrases or clauses 
(\prettyref{sec:grammatical.modify}).
Then, according to how roots fit into these environment,
we define parts of speech (\prettyref{sec:grammatical.pos}). 

This means, kindly of strangely, we first define \term{noun phrases},
and then define \term{nouns}.
Note that there are several senses of terms like \term{noun}:
their meanings as purely syntactic concepts, as morphological concepts
and as parts of speech labels in dictionaries
may not be identical
\prettyref{sec:grammatical.pos.caveat}.

One advantage of the traditional approach is that it's parsing oriented;
suggestions for parsing Latin texts are given in \prettyref{sec:grammatical.parse}.

The third round of our description
consists of chapters following this chapter focus on different topics touched in this chapter.

\section{Morphological typology}\label{sec:grammatical.word}

\paragraph*{Wordhood}

Latin is well known for its rich morphology,
which enables a rather free -- but still not completely arbitrary (\prettyref{chap:order})
-- constituent order.
The grammatical categories seen in nominal and verbal morphology 
already reflect the most salient grammatical categories
in \acs{np}s and clauses;
thus the concept of morphological wordhood can be defined according to 
the boundaries of inflectional templates;
and the traditional wordhood definition largely agrees with this definition. 
Thus, in this note I follow the traditional definition of Latin wordhood,
which is easily done using phonological criteria,
or, to be more accurate, orthographical criteria: 
what was documented by ancient Romans as a word 
is recognized as a word.
For words without inflectional morphology,%
\footnote{
    The traditional name is \term{particles};
    see below.
} 
wordhood can still be defined according to phonological wordhood.
The two criteria of wordhood also largely agrees with the traditional word.

Also we can talk about various purely syntactic ways to define wordhood.
For example a noun phrase minus large complements or adjuncts should be recognized as a noun,
or a clause minus large complements or adjuncts should be recognized as verb.
In this regard Latin has split verbs in periphrastic conjugation,
and in the preverb construction, 
the prefixed verb, a morphologically single word, 
contains the verb proper and an adverbial root that 
syntactically should be a part of the locative complement of the verb. 
This line of wordhood definition helps us understand the structure of Latin noun phrases or clauses 
but goes against any conventional definition of wordhood.

The only subtlety in wordhood definition seems to be due to clitics, 
the most important one being the coordination \form{-que}: 
they are not a part of inflectional morphology of any stem 
but have to be attached after another word.
For them, the orthographical standard simply dictates that 
they are to be recognized as a part of the words they follow. 
In conclusion, we can say the traditional notion of Latin words 
is (expectedly) linguistically sound 
and should be kept in use.

\paragraph*{Stems and endings in nominal and verbal morphology}

A stem can be well defined for both verbal and nominal morphology in Latin,
enabling a clear derivation-inflection distinction:
prototypical derivational processes that are considered 
to be a part of the head noun/verb 
and not the surrounding \acs{np}/clause
are morphologically realized strictly 
before prototypical inflection processes,
forming the stem 
(\prettyref{sec:noun.paradigm.introduction},
\prettyref{sec:verb.finite.paradigm}) to which 
inflection processes are applied,
with possible contextual allomorphs
(\prettyref{sec:phonology.morphological}). 
Latin inflection is always suffixal,
while derivation is predominantly prefixal.
Despite its richness, 
a large portion of instances of Latin derivation are historical,
with meanings of derived forms 
having significantly shifted and no longer regularly inferrable.

A general tendency in the stem structure 
is modification affixes usually are prefixes,
while affixes bearing grammatical information 
(like change of part of speech) 
are usually suffixes
(\prettyref{sec:np.noun}, TODO: ref).
Apart from affixation,
Latin does use compounding, as in \form{aequilibrium} 
(\form{aequ-i-libr-ium}, \translate{even-balance-\category{sg.nom}, equilibrium};
\citealt[\citesec{92}]{smith2016greek}),
but compounding was already less productive -- if not completely obsolete --
in the Classical period.

\paragraph*{Non-concatenative morphology}

Concatenative morphology (affixation and compounding) 
is prominent but isn't the only morphological device.
Reduplication is attested in 
formation of the perfect stem (TODO: ref);
this however is largely historical 
and is no longer productive.
Dropping of first-conjugation stem-final vowel (\prettyref{sec:tense-mood-marking})
may be analyzed as subtraction,
although it can be seen as due to morphophonological rules.
The imperfect \form{-ba-} is sometimes said to be an infix 
(as well as its counterparts like \form{-bi-}),
though it fits in a concatenative picture of verbal morphology.

Residues of the ablaut system in PIE can be observed in Latin.
In the third declension, for example,
we have the following alternation:
\form{gen-us} \translate{family-\category{nom.sg}}/
\form{gen-er-is} \translate{family-\category{gen.sg}},
and since \form{-us} is \form{-os} in earlier stages of the language,
we observe an alternation between \form{o} and \form{e}.
Another instance is the alternation between \form{toa} and \form{teg\={o}} \translate{cover}.
It's also possible that no vowel exists at the place where we find \form{o} and \form{e}.
We therefore find a vowel gradation system with three variants:
∅, \form{o}, and \form{e}.
Note that what controls the vowel gradation system is grammatical categories like case, etc.,
and therefore ablaut is a part of morphology, not just phonology
\citep[\citepage{46}]{weiss2009outline}.

\begin{todobox}{Ablaut example in Latin}{ablaut-example}
    \begin{itemize}
        \item In noun inflection
        \item In verb inflection?
    \end{itemize}
\end{todobox}

\section{Clausal syntax}\label{sec:grammatical.clause}

\subsection{Introduction}

\paragraph*{Clauses in utterances}\label{sec:grammatical.clause.introduction.utterance}
We start our systematic discussion on Latin syntax from the verbal system, i.e. the clause. 
A clause may appear as a single utterance called a \concept{sentence},%
\footnote{
    Some authors (including some Latinists) use the term \term{sentence}
    as a synonym of \term{utterance};
    it is not followed in this note.
}
which is either declarative or interrogative,
An embedded clauses may be a complement of a verb (complement clause construction),
or a high-level modifier in the verbal system (subordination),%
\footnote{
    The term \term{subordination} sometimes is used as a synonym of \term{clause combining}.
    Here we use \term{subordination} in a narrower meaning.
}
and as a relative clause 
(i.e. modifier in the nominal system).

\begin{todobox}{Distribution of clauses}{clause-distribution}
    List where clauses can be found in other structures.
\end{todobox}

\paragraph*{Clause types}
Clauses types in Latin can be roughly divided into finite and non-finite ones,
which can be told from verbal morphology.

In sentential clauses, 
the category of speech force is marked by e.g. fronting of the interrogative pronoun.
No other grammatical markers are dedicated to clause type:
there are, for example, no sentence-final particles 
that can only be observed in \emph{sentential} clauses
and not embedded finite clauses.

Non-finite constructions include \category{infinitives} and \category{participles}, and the \category{gerund}.
These constructions can be distinguished by morphology only:
in English we have infinitive clauses,
but strictly speaking, there is no such thing as ``infinitive verb'',
as the head verb of an infinitive clause 
has exactly the same form of a non-third person singular present tense verb;
in Latin however, the head verb of a infinitive clause in Latin 
indeed has a separate position in the paradigm.

\begin{todobox}{Implicit marking of speech force categories}{speech-force-category}
    Are there any e.g. obligatory fronting or contraction that may be related to the speech force of a clause?
    
    We also need to investigate the licensing environment
    and the inside \form{wh}-forms if the clause is a relative clause.
\end{todobox}

\begin{todobox}{Clause-level subordination}{clause-level-subordination}
    \begin{itemize}
        \item Coordination
        \item Conditional
        \item Concessive
    \end{itemize}
\end{todobox}

\paragraph*{Internal makeup of a clause}\label{sec:grammatical.clause.overview.internal}
The external grammatical function of the clause
may be reflected by conjunctions and/or the morphological \category{mood}.
The complexity of the clausal structure is strongly influenced by the clause type,
which, as is said above, is reflected in the verbal morphology.
The maximal complexity possibility of a clause 
consists of a nucleus clause with possible information structure-related order alternations 
and/or clause-level subordination constructions.

At the surface level, the nucleus clause contains a (omissible) subject
(\prettyref{sec:grammatical.clause.subject}),
a main verb with \category{tense} marker and a possible auxiliary verb (\prettyref{chap:verb}),
adjuncts and internal arguments.\footnote{
    In some works on English syntax, e.g. \citet{cgel}, the term \term{complement} means a 
    dependent position which, according to various standards,
    is more closely related to the lexical head;
    in the case of clauses, a complement is just an argument.

    In traditional Latin grammar, however, \term{complement} means the copular complement. 
    This note follows the terminology used in most descriptive grammars,
    so use the term \term{copular complement} to refer to the \term{predicative complement} in \citet{cgel}.
    
    A further confusion, as is seen in \prettyref{sec:verb-phrase.arguments.compatibility}, 
    is due to the term \term{complement-taking verb}, 
    i.e. a verb that take a complement clause 
    or something semantically equivalent to a complement clause
    as one of its arguments.
}
The main verb shows regular agreement with the subject.
The inflectional ending of the main verb, 
together with some adjuncts and an optional auxiliary verb,
marks \ac{tam} grammatical categories (\prettyref{sec:grammatical.clause.tam}).
The distinction between what are known as peripheral arguments
-- another type of adjuncts -- and core arguments
is discussed in \prettyref{sec:grammatical.clause.peripheral};
we expect most so-called peripheral arguments 
to in some sense have scopes over core arguments.

The core arguments -- and to some extents the peripheral arguments --
are ultimately determined by the \concept{argument structure} of the main verb.
Arguments have ``deep'' and ``surface'' positions.
The deep positions are defined by properties like  
how reflexives work, mapping of semantics to argument slots,%
\footnote{
    We should however note that 
    two arguments being denoted by the same semantic role label 
    (e.g. \term{stimulus})
    do not mean they are truly semantically equivalent,
    and clearly they are necessarily syntactically comparable.
    In \citet[\citesec{4.2}]{dixon2005semantic}, for example,
    what bears the label \term{target}
    is almost prototypically patient-like in verb frame I
    and is the goal of a directional construction in verb frame II,
    and of course there are syntactic as well as semantic differences between the two frames.
}
the semantics difference between S=A and S=P valency alternations 
(\prettyref{sec:grammatical.clause.core.transitive}), etc.
The mapping from deep positions to surface positions
(like \term{subject}) is influenced by
\begin{itemize}
    \item \emph{Voice}. Latin has a passive \concept{voice},
    which is marked on verbal morphology (\prettyref{chap:verb})
    and seems to work solely on the concept of \concept{object};
    no other highly regular voice constructions of such kind are observed
    (\prettyref{sec:grammatical.clause.voice}).
    It's still possible to have verb frame alternations
    that highly depends on the properties of the verb stem.
    \item \emph{Alignment typology}. 
    Regarding \concept{alignment typology},
    Latin is a nominative-accusative language
    with clear \category{S}=\category{A} neutralization 
    and therefore a well-defined \concept{subject} (\prettyref{sec:grammatical.clause.subject}).
\end{itemize}


Details of deep argument structures and their reflections in the surface argument structure 
are specified in possible \concept{verb frames} in a language.
The argument structure is often related to the \concept{event structure} of the verb,
and therefore the verb frame may also influence the \ac{tam} marking
(\prettyref{chap:verb-frame}).


\paragraph*{Constituent order}
The flexible constituent orders of Latin 
lead someone to characterize it as a free-order language.
However, a closer examination reveals that 
dependency relations in Latin clauses do control the constituent order
and a neutral order roughly described by \prettyref{fig:latin-neutral-clause} can be established.
This order is due to both default information structure 
and argument structure,
and order alternations on top of it
due to information packaging are also well attested 
(\prettyref{sec:grammatical.clause.information}).
Prosody also plays a role in determining the constituent order.

\begin{figure}[H]
    \centering
    Subject - direct object - oblique argument - adjuncts - source/goal - cognate object - verb
    \caption{Latin neutral constituent order}
    \label{fig:latin-neutral-clause}
\end{figure}

\subsection{Subject}\label{sec:grammatical.clause.subject} 

\paragraph*{Clausal pivot}
Latin is a nominative-accusative language.
This means we have a well-defined clause-level pivot 
which, in active constructions,
can be identified with the structurally most agent-like argument%
\footnote{
    What it takes to be agent-like is discussed below.
}
in the argument structure.
This pivot is called the \concept{subject}.

Existence of a clause-level pivot can be clearly identified from the following facts.
\begin{itemize}
    \item \emph{Extraction in coordination}. 
    When two clauses with a shared subject are coordinated,
    the subject usually appears at the beginning of the coordination construction 
    \citep[\citesec{19.5}]{pinkster2}.
    This demonstrates that the subject is somehow structurally 
    promoted to a higher position in this kind of coordination.%
    \footnote{
        Omission of the object in coordination, on the other hand,
        shows more diversity in constituent orders \citep[\citesec{19.6}]{pinkster2},
        suggesting that the exact structural analysis may also be more complicated.
    }

    \item \emph{Case marking}.
    Subjects are always nominative for finite clauses,
    whenever case marking is available
    (i.e. whenever the subject is an \ac{np} or a gerund). 
    On the other hand, 
    nonfinite clauses are either subjectless  
    or have an accusative clausal dependent functioning as the subject
    (\prettyref{sec:complement-clause.infinitive.aci}):
    we may argue that by being nonfinite,
    they have deficient nucleus clause structures 
    and cannot afford a canonical subject.

    \item \emph{Agreement}.
    The number and person features on the subject leave marking on the verb complex.
    Latin does not have verbal agreement with arguments other than the agreement with the subject.
    
    \item \emph{Constituent order}. In clauses with neutral information structure,
    the subject usually appears at the beginning of the clause.
    Violations to this generalization are countless 
    but they can all be analyzed as focalization or topicalization of other constituents 
    and postponing of the subject (\prettyref{sec:clause-order.postpone}). 
\end{itemize}

Latin subjects also have a quirky features.
A subject is an \ac{np}  
or a complement clause, 
usually an infinitive but never a gerund (\prettyref{sec:gerund-morphology}).
This constraint isn't seen in any other clausal complement types.

Since we have already identified a clausal pivot in Latin,
we can call the rest of the nucleus clause the \concept{verb phrase}.
We will in turn separate \ac{tam} markers from it (\prettyref{sec:grammatical.clause.tam}),
getting the extended argument structure,
and then separate peripheral arguments from it (\prettyref{sec:grammatical.clause.peripheral})
and get the core argument structure (\prettyref{sec:grammatical.clause.core}).
Because of the flexible constituent order, however,
the term \term{verb phrase} doesn't have much value for surface-orientated analysis.

\paragraph*{Neutralization of argument structure pivot and clausal pivot}
In Latin we can also recognize an argument structure-level pivot,
or in other words, an ``agent-like'' argument in each clause
(\prettyref{sec:grammatical.clause.core.transitive}).
The fact that in active clauses, an agent-like argument is always the subject 
means Latin is indeed a nominative-accusative language.

\paragraph*{Impersonal constructions}
In certain cases, the subject position can be left empty.
The verb then takes the default 3sg agreement marking.

\subsection{Tense, aspect, modality, and so on}\label{sec:grammatical.clause.tam}

Tense, aspect and modality is most evidently marked by verbal inflection in Latin 
(\prettyref{chap:verb}).
Other \ac{tam} categories however are regularly marked by adverbials.

\begin{todobox}{Latin nonfinite constructions}{tame-nonfinite}
    Nonfinite constructions lack \emph{morphological TAM marking},
    but are TAME adverbials possible for them?
    Is nonfiniteness a morphological or syntactic property?
\end{todobox}

\paragraph*{Speaker-oriented categories}
Sentential clauses, i.e. clauses appearing as a complete utterance (\prettyref{sec:grammatical.clause.introduction.utterance}),
can first be classified into declarative, interrogative, and imperative ones:
the interrogative clause is marked by fronting of \category{wh}-words,
and the imperative clause is marked by verbal morphology.

Latin does not have morphological marking of evidentiality.
There are however hypotheses that the alternation between
\form{Accusativus cum Participio} and \form{Accusativus cum Infinitivo}
marks evidentiality \citep{greco2013latin}.

Epistemic modality is marked by the \category{mood} in declarative and interrogative clauses 
\citep[\citepage{388}]{Pinkster1}.

\paragraph*{Primary tense}
The primary tense establishes a \emph{reference time}
and compares its relation with the \emph{speech time}.
The ``speech time'', of course, can also be the time of writing,
or even the time of reading: \form{now readers may wonder why \dots}
Latin has a past/present/future tense system:
this can be clearly observed by staring at the finite paradigm.
The past future tense seen in English (\form{he [would] take part in in if he knew it})
is absent in Latin.

The reference time sets the stage of the situation of affair being described;
the exact time when the situation happens is not necessarily identical to the reference time:
see the category of anteriority. 

\paragraph*{Denontic modality}
The morphological \category{mood} in the imperative construction
seems to mark the denontic modality:
there is no epistemic modality to mark anyway
\citep[\citepage{388}]{Pinkster1}.

\paragraph*{Habituality and similar categories}
In Latin, just like in English,
the habitual aspectuality is not marked by verbal morphology,
but often the \category{present} category has a habitual reading.

We note that crosslinguistically, 
habituality and anteriority and point of view aspectuality (see below) can co-exist:
in English, for example, \form{already} can be seen as a weak version of anteriority,
and we have examples like \form{what we feel has [always]_{\text{habitual}} [already]_{\text{anteriority}} been felt, again and again}.
We can take a step further and observe co-existence of habituality, anteriority, and imperfectivity:
\form{this life has [always]_{\text{habitual}} [already]_{\text{anteriority}} [been happening]_{\text{progressive}}}.
The scopes of the three categories seem to follow
the order of habituality \textgreater{} anteriority \textgreater{} point of view aspectuality.

\begin{todobox}{Co-existence of habituality and anteriority}{habit-anterior}
In Latin is it possible?
\end{todobox}

\paragraph*{Anteriority}
Anteriority, i.e. whether the state of affair being described
happens before or after a reference time 
(whose relation with the speech time is determined by primary tense),
is sometimes recognized as tense and sometimes aspect.
In Latin we can recognize a clear perfect/simultaneous distinction from verbal inflection
because we have a \category{perfect} stem and a \category{imperfect} stem
and also different endings for the two system.

It seems we also have a posteriority value in Latin,
because even when the primary tense is absent 
(as in the two \category{future} participles)
or marked otherwise (as in two periphrastic verb forms),
a meaning of ``future'' -- not necessarily defined according to the speech time --
is still available,
and the \category{future} participles rarely appear
with \form{sum} in \category{perfect} (which encodes anteriority) forms
\citep[\citetable{7.3}]{Pinkster1}.

\paragraph*{Point of view aspectuality}
A category that is often known as the grammatical aspect 
is the \emph{point of view} of the speaker to the situation being described,
or, in other words, the position of the ``event time'' 
(which is compared with the reference time in anteriority)
within the whole situation:
the situation may be described as a whole (\concept{perfective}),
or the clause only describes a fraction of a continuous event 
(\concept{imperfective} or \concept{progressive}).
The distinction between the two can be subjective:
closing a door is an almost instant event,
but we still have \form{I'm closing the door}.

In Latin we do not have a clear marker of the perfective/imperfective distinction.
It is however generally believed that at least the \category{imperfect} verb form regularly has an imperfective meaning.

The relation between the point of view and the situation can be more complicated:
for example we have terminative, inchoative or retrospective aspects.
The Latin \form{-sc-} infix is an inchoative marker.

\paragraph*{Lexical aspect}
All the categories above are either about the temporal location of a situation
or the speaker's subject judgments of the situation.
A situation also has inherent properties,
known as its lexical aspect.

\subsection{Peripheral arguments}\label{sec:grammatical.clause.peripheral}

The term \concept{peripheral arguments} is used in \citet{dixon2009basic1}
to refer to what others call (non-\ac{tam}) adjuncts
and specify information like the spatial and temporal location of the event,
the instrument, the manner, etc.
The distinction between peripheral arguments and
\concept{oblique arguments} -- core arguments which are also known as instruments, manners, locations, etc. --
is inherently not clear;
grammatical phenomena traditionally attached to the distinction involves 
\citep{mcinnerney2022argument}:
\begin{itemize}
    \item \emph{Structural closeness to the main verb}.
    The prototypical core argument structure is a structure expressing 
    \translate{someone did something to \dots}.
    Constituents like so-called circumstantial constituents modify the core argument structure as a whole,
    so they are prototypically peripheral. 
    The syntactic status of other constituents are not so clear (see below).

    \item \emph{Idiomization and subcategorization},
    i.e. whether lexical properties of the main verb blocks removal of the argument.
    These factors are syntactic aspects of verb frames.
    English prepositional verbs are a good example of this.
    A argument licensed by the subcategorization pattern may still be omitted
    because of pro-drop,
    but in this case usually it has a definite reading
    because when omitted, it's expected refers to an entity specified in the context.
    Indefinite implicit pronouns are still possible in some cases;
    whether this is recognized as valency alternation or pro-drop
    is purely terminological.

    \item \emph{Semantic and pragmatic requirement}.
    A clause should ``say something'', and hence we have semantic obligatoriness of certain arguments.
    This does not necessarily dictate subcategorization.
    An argument that's semantically not obligatory may still be syntactically obligatory: \form{slog}.
    A semantically obligatory argument is usually syntactically projected,
    but it can be filled by an indefinite null pro-form:
    \form{I ate} (\prettyref{sec:grammatical.clause.core.transitive}).
\end{itemize}
Distribution patterns of arguments may sometimes be explainable
by both the second and the third factors. 
For example, the clause \form{it rains} will almost never have an instrument phrase,
which could be explained by semantic incompatibility between
a weather event and an instrument phrase,
or by subcategorization. 
Whether in extreme semantic cases an instrument argument can be licensed
may vary across speakers,
and often subcategorization patterns may just be syntactic fossilization of semantic factors.

Despite the theoretical independence of the three parameters,
for clear functional reasons, they are often correlated,
and most of the time the core/peripheral distinction is still meaningful.

In Latin, arguments that are not structurally close the the main verb
seem to appear at the center of a clause,
indicating that they do not undergo any information packaging operations by default 
(\prettyref{fig:latin-neutral-clause}, \prettyref{sec:grammatical.clause.information}).

\paragraph*{Circumstantial adjuncts}
Circumstantial adjuncts are clausal constituents that
modify the event as a whole and specify when and where it happened.
They likely are the most peripheral constituents in the extended argument structure,
modifying the core argument structure about \translate{someone did something to \dots}
\citep[\citepage{29}]{cinque1999adverbs}.

\paragraph*{Manner}
There seems to be two manner positions:
one has a position similar to that of the instrument and mean phrases,
and in the broad focus constituent order resides \emph{after} the direct object
\citep[\citepage{71-75}]{devine2006latin}.
Another position seems to scope over the core argument structure as a whole,
and therefore is more peripheral and appears before the object
in the default order
\cite[\citepages{101-109}]{devine2006latin}. 

\paragraph*{Instrument and mean}
\label{sec:grammatical.clause.peripheral.instrument}
The exact syntactic status of the instrument and the like is controversial.
Crosslinguistically, it seems some instrument phrases are a part of the core argument structure
\citep{pascual2001syntactic},
and some instruments, even the pure instrumental ones, 
can be regularly promoted to the subject position
under proper pragmatic environments \citep{alexiadou2008instrument},
indicating that they belong to \translate{do something to \dots} part of the core argument structure,
and they are not as peripheral as circumstantial adjuncts. 


\begin{todobox}{Instrument argument}{instrument-argument}
    In Latin it seems an argument/adjunct distinction exists for instrumental arguments:
    \citep[\citepages{57-58}, \citepage{65-67}]{devine2006latin}.
    Does this mean that when both appear, the argument instrument appears before the adjunct instrument?

    If we want to decide the position of English \form{with}:
    \begin{itemize}
        \item \form{he covered the hole with a blanket} is not the same as \form{he covered the hole with his toolkit}:
        the first can be transformed to \form{a blanket covered the hole}
        while the second can't.
        See The Reversible Core of ObjExp, Location, and Govern-Type Verbs
        \item Instrument-subject alternation: \form{the knife killed him}.
        This also leads to an agent-causer distinction.
        See Instrument Subjects Are Agents or Causers
        
        
    \end{itemize}
\end{todobox}

\paragraph*{Internal makeup of peripheral arguments in Latin}
The internal makeup of the argument -- case marking and prepositions --
is weakly correlated to the core/peripheral distinction in Latin.
Nominative arguments are never peripheral,
and accusative arguments are rarely peripheral,
but nothing could be said for the rest of the cases and prepositions.

\subsection{Core arguments}\label{sec:grammatical.clause.core}

\paragraph*{Grammatical relations, deep and shallow}

As is said in \prettyref{sec:grammatical.clause.overview.internal},
core arguments have deep and surface (i.e. clause-level) status in the clause.
In this section, we list several prototypes of the deep argument structure
and discuss how the arguments are assigned surface grammatical relations.
The most uncontroversial surface grammatical relation,
the subject, has already been described in \prettyref{sec:grammatical.clause.subject},
and here we consider other surface grammatical relations. 

The \concept{direct object} is the surface grammatical function of the P argument
in the prototypical transitive construction.
Here we list its properties and see whether arguments in other verb frames 
can also be described as direct objects:
\begin{itemize}
    \item \emph{Case marking}: Direct objects are always accusative when it makes sense to talk about case -- 
    but not all accusative arguments are direct objects. 

    \item \emph{Passivization}: If an argument is coded as the direct object, 
    then it can regularly be promoted to the subject position in a passive clause
    (\prettyref{sec:grammatical.clause.voice}). 
    Secondary objects are less frequently promoted in passivization. 
\end{itemize}

As is said in \prettyref{sec:grammatical.clause.peripheral},
oblique constituents in the clause are not necessarily peripheral.
The most prototypical examples are seen in directional constructions and dative constructions:
if we acknowledge that the object is a core argument,
then the locational/dative phrase has to be a core argument as well,
because the two are licensed together.
It's also possible to have manner and instrument phrases as core arguments.
Traditionally the recipient argument of a verb with a meaning of giving
is known as the \concept{indirect object}.
We do note that this ``indirect object'' isn't very different from 
other oblique constituents in the core argument structure.
The term \term{indirect object} therefore is used as a synonym of \term{recipient argument} in this note.

The instrument expression is structurally further from the main verb
but not as peripheral as circumstantial phrases
(\prettyref{sec:grammatical.clause.peripheral.instrument}).
In some verb phrases it can also be considered as a 

\begin{todobox}{\form{full of}}{genitive-content}
    Is it necessary to stipulate an argument structure of \form{full of}?
    consider a three-place verb where the third argument is genitive.
    This seems to be related to be judicial verbs:
    \form{accuse of}.
\end{todobox}

\paragraph*{Agent and patient in the prototypical transitive construction}
\label{sec:grammatical.clause.core.transitive}
The most prototypical transitive construction
involves an agent (A), which initiates the event
and exists independently of the event,
and a patient (P), which is affected by the event,
sometimes would not exist without the event.
In active clauses, the A becomes the subject
and the P becomes 
In Latin often the transitive construction can be detransitivized
by suppressing the P argument
(we can also understand this as filling the P position by an empty indefinite pronoun),
resulting in what is known as the \concept{absolute use} of the verb
\citep[\citesec{9.16}]{Pinkster1}.
This is also observed in English: \form{the dogs bited me} v.s. \form{the dog bites}.
This leads to a S=A valency alternation. 

There exists another class of transitive constructions,
in which the internal argument participates in a situation of affair,
and the external argument causes this.
We can say internal argument is P-like and the external argument is A-like.
It's possible that the external argument can be removed
and we still have a well-formed core argument structure:
this leads to S=P valency alternation.
In principle, the argument structure that contains only the internal argument
may be unaccusative or unergative (see below)
and the transitive use of the verb can be viewed as a causative construction.

\begin{todobox}{Detransitive derivation}{Detransitive}
    In Latin, do all S=P verbs have unaccusative intransitive counterparts?
\end{todobox}

\paragraph*{The sole argument of intransitive verbs}
\label{sec:grammatical.clause.core.intransitive}
The sole argument of intransitive verbs -- referred to as S in the typological literature --
either starts an event,
or passively undergoes an event:
consider \form{Simon jumped off} and \form{Simon fell}.
That's to say, the status of S in the deep argument structure 
may be comparable to either A or P in the prototypical transitive construction;
the intransitive verbs are correspondingly known as \term{unergative} and \term{unaccusative} verbs,
although the names are misleading as we are discussing about the deep argument structure
and not alignment yet.

In Latin, it seems that unaccusative intransitive verbs are all deponent
\citep[\citepages{308-309}]{oniga2014latin}.
The passive voice therefore is used whenever an agent is absent.
unergative intransitive verbs do not have \category{perfect passive participles},
because the passive voice is not compatible with them
\citep{giusti2019psychological}.



\paragraph*{Psych-verbs}
Verbs about psychological activities may display unusual behaviors.
The argument structure may be 
a causative construction where the stimulus causes the experiencer to have psychological activities,
and the stimulus is A-like,
or a construction where the experiencer directs their emotion to a target,
where the experiencer is A-like.
Plus, for at least some psych-verb, the stimulus-causers originate from the stimulus-target,
so that at least for some speakers,
reverse binding as in \form{this rumor about herself frightened Mary} is acceptable
\citep{hornstein2002psych}.

The above two verb frames also have their intransitive versions,
where the internal target argument or the internal experiencer argument is oblique.
The intransitive constructions can be unaccusative (\form{I'm anxious about \dots})
or unergative (\form{I rejoice because of \dots}).
We further have a possible impersonal construction where the experiencer and the stimulus
form a small clause,
and the subject is filled by a dummy pronoun.
All the constructions have been attested in Latin \citep{giusti2019psychological}.

\begin{todobox}{Passivization in verbs about teaching}{verb-teaching-passive}
    Can verbs taking accusative argument in \citep[\citesec{4.36}]{Pinkster1}
    be passivized?
\end{todobox}

\paragraph*{Comitative in argument structure}
\label{sec:grammatical.clause.core.comitative}
Sometimes, the argument structure licenses a comitative construction,
whose head becomes the subject or the object,
and the rest of the construction becomes an oblique argument \citep{zhang2007syntax}.
In Latin this is observed, with the comitative marker being the preposition \form{cum}.

\paragraph*{Dative and benefactive constructions}
The verb frame of verbs meaning \translate{giving} in Latin regularly
contains a dative construction,
which is quite similar to a directional construction (see below),
where the thing being given -- the \term{theme} -- goes to the \term{recipient}.
The case of the latter is always dative. 
In Latin, the dative construction comes together with
the causer of the event of giving,
which expectedly functions as the subject of the whole verb frame,
and the theme, without any semantically loaded case marking, becomes the object.

\begin{todobox}{Benefactive ditransitive}{ditransitive-benefactive}
    Do we also have \form{buy a bottle for me} construction in Latin?
    It seems Latin only has benefactive adjuncts. 
\end{todobox}

It's possible to keep only the dative recipient in the benefactive construction:
this leads to two-place verbs governing a dative argument 
\citep[\citesec{4.24}]{Pinkster1}. 
We note that the benefactive construction differs from 
the prototypical dative construction of giving,
and the latter is never attested in two-place verb frames
\citep[\citesec{12.7}]{Pinkster1}.

\paragraph*{Directional constructions}
Some Latin verbs with a preverb seem to have a directional small clause
(\prettyref{sec:grammatical.clause.small}) hidden in their frames:
the case marking of the argument(s)
are sometimes never attested without the preverb,
and the meaning of the verb may also vary depending on whether the preposition is attached to the verb,
indicating that it's the preverb that licenses the argument
\citep{mare2018issues}.

The preverb always describes the \emph{path} in the directional construction:
the type of the \emph{location} is either marked by the case of the location argument
(often hence dictated by the preverb
as if the preverb is used as a standalone preposition
and the location is governed by it),
or is in the default accusative case.
Passivization of accusative location arguments is sometimes possible
\citep[\citesec{4.22}]{Pinkster1}.
These phenomena seem to be comparable to English preposition verbs and verb-particle constructions,
where the preposition may be somehow defunct
and the location argument it governs syntactically is the object of the verb,
but it's also possible that the prepositional phrase is sealed and nothing can be extracted from it.

It is also possible for the location argument to appear in dative
\citep[\citesec{4.25}]{Pinkster1}.
The location argument in this case can always be understood as the goal of the action,
and the dative case seems to mark this fact.
This indicates that the requirement that 
the \ac{np} governed by prepositions has to be accusative or ablative
is likely a quirky realizational feature of Latin:
the fact that an argument is the target in a directional construction
can only be marked by the accusative case
when the directional construction is realized as an independent prepositional phrase,
but the dative case is available in verb frames.

\begin{todobox}{The subject of the directional construction}{subject-direction}
    The subject of the directional construction may appear as the subject of the clause 
    or the object of the clause?
\end{todobox}

\paragraph*{Copular complements}
Latin also has copular complements.
A copular complement, just like its counterpart in English,
basically can be viewed as a displaced attributive or appositive 
(and hence is prototypically filled by an \ac{np} or an \acs{adjp})
but is a little more peripheral (manner, state, factitive, etc.) 
in its meaning than an attributive or appositive.

Latin has nominative predicate and accusative predicate:
as hinted by their names, 
the nominative predicate gives a property of the subject and agrees with it,
and the accusative predicate gives a property of the direct object and agrees with it.
In passivization of the direct object,
the accusative predicate becomes the nominative predicate.

\begin{infobox}{Oblique copular complements}{oblique-copular}
    Other types of copular complements without agreeing with the subject exist.
    TODO: ablative of quality, price, etc. 
    The syntactic status of copular complements here are closer to \acs{pp}s:
    we may say they receive \emph{inherent cases},
    while the nominative and accusative copular complements 
    receive \emph{structural cases} 
    (\prettyref{sec:np.case-distribution}).
\end{infobox}

\paragraph*{Cognate objects}
A \concept{cognate object} is a clausal dependent that is obligatorily selected by the main verb
and usually overlaps semantically with the content of the main verb.
In English we have \form{Ben died [a long and painful death]}
and \form{Joe smiled [a smile]_i and it_i is a creepy one}.


\subsection{Voice}\label{sec:grammatical.clause.voice}

\paragraph*{Prototypical passive}
Latin has a prototypical passive voice:
in this construction, the A-like argument -- usually the most external argument --
is either suppressed or becomes an instrument-like argument
(we may say that when it's suppressed,
the instrument-like argument is replaced by an empty indefinite pronoun).
After this, the second highest argument that can go to the subject position
-- the direct object --
is promoted to the subject position.

\paragraph*{Deponent verbs and impersonal passive}
In Latin, the passive morphological marking of the verb
also generalizes to all verbs without a prototypical A.
This happens to unaccusative verbs
(\prettyref{sec:grammatical.clause.core.intransitive}),
and also in so-called impersonal constructions,
where the A is suppressed but nothing is promoted to the subject position.

\subsection{Information packaging}\label{sec:grammatical.clause.information}

Information packaging is observed both on the clausal level
and inside the extended argument structure.
The relevant information packaging devices can be roughly classified into three types:
\begin{itemize}
    \item \emph{Information packaging within argument structure}.
    Information structure-neutral clauses have the whole verb phrase as the focus
    (so-called \concept{broad focus} clauses),
    and within this focus we can identify several weak topic and focus positions.
    By default, the direct object and certain oblique arguments 
    (like the dative recipient)
    are weak-topicalized,
    meaning that they precede adjuncts in the extended argument structure 
    \citep[\citesec{1.5}]{devine2006latin}.
    On the other hand, directional phrases like goal or source are usually not weak-topicalized,
    which probably is because they are not conversational participants
    and are usually not old information.
    Cognate objects are always indefinite and they can't be topicalized,
    and because objects are close to the main verb,
    they are to the right of directional phrases.

    The two competing factors -- the peripheral nature of peripheral arguments,
    and direct object and certain oblique arguments being focused while others are not --
    lead to the neutral order in \prettyref{fig:latin-neutral-clause}. 

    \item \emph{Scrambling for presupposition}.
    This means fronting a constituent to imply that it is presupposed 
    (this is a weak form of topicalization) 
    or weakly focalized \citep[\citepages{102,108-109}]{devine2006latin}.
    When information-neutral manner adverbs are present
    (whether the adverb undergoes some movement can be tested by its scope \citep[\citepage{100}]{devine2006latin}),
    this move constituents to its left.

    \item \emph{Prototypical topicalization and focalization}.  
    Prototypical topic and focus constructions completely change the broad focus information structure,
    and may -- but not necessarily, in-situ and scrambled constituents can still receive strong focus -- 
    move the focalized or topicalized constituent to a higher position,
    sometimes even before the complementizer if there is any.
    Without information like stress or conversational contexts,
    it's sometimes hard to tell prototypical focalization and topicalization from scrambling
    but constructions like \form{Y non X} or interrogative clauses
    undoubtedly involve the former
    \citep[\citechap{3}]{devine2006latin}.
\end{itemize}
These operations render the constituent orders of Latin clauses extremely variable.

\subsection{Clause combining}

In Latin there is no serial verb constructions.
Subordination strategies can be neatly summarized into 
complement clauses, relative clauses and adverbial clauses.

\subsection{``Small clauses''}\label{sec:grammatical.clause.small}

Some constructions may be analyzed as ``small clauses'', 
which are smaller than usual clauses 
and lack real \acs{tam} marking;
they are therefore not ``real'' clauses, 
but still appear frequently in Latin texts.
(\ref{ex:clause.small-clause.1}) is a formula used frequently in Catholic Church 
and is an example of a small clause 
as a single and complete utterance.
The fact that \form{gratias} is not in the nominative or the vocative case 
means (\ref{ex:clause.small-clause.1}) is 
neither a finite construction nor a single-\acs{np} utterance.


\begin{exe}
    \ex \gll De-o grati-as \\
    God-\category{sg}.\category{dat} thank-\category{pl}.\category{acc} \\
    \glt \translate{(May) thanks be to God.}
    \label{ex:clause.small-clause.1}
\end{exe}

Since English -- and a lot of other languages -- 
bans small clauses appearing as complete utterances,
translation of these constructions may map them into full clauses in the target language. 

\begin{infobox}{Why small clauses}{small-clause}
    Alternatively, small clause constructions can be analyzed 
    as a full clause (for example, a copular clause) with the verb deleted.
    The main problem of this analysis is if this is true, 
    we need to explain why prototypical transitive clauses 
    never see their verbs omitted.
    It's therefore better to say 
    that a copular clause is a small clause \emph{plus} the copular,
    and that a giving or receiving ditransitive verb 
    takes a small clause as its internal complement,
    which doesn't involve a questionable verb deleting process
    and also captures the intuition that 
    a small clause is a full clause minus the verb.
\end{infobox}



\section{The noun phrase}\label{sec:grammatical.np}

\subsection{Introduction}

\paragraph*{Noun phrases in utterances}

\begin{todobox}{NP in utterance}{np-utterance}
    \begin{itemize}
        \item Response as a piece of utterance
        \item Argument and adjunct
    \end{itemize}
\end{todobox}

\paragraph*{Internal makeup of noun phrases}
The hierarchical structure of the noun phrase is often ignored in traditional grammar,
and yet it has delicate formal parallelism with the structure of the clause.
As Latin is a classical language,
detailed tests about these structures are not always available
and are likely subject to individual variances.
In this section we list some points worth noting that are repeatedly attested in other languages.

First, noun phrases are labeled for their grammatical functions,
which in Latin is mostly done by the morphological case system
and also sometimes the preposition system 
(\prettyref{sec:grammatical.np.case}).

The noun phrase can then be divided into determiners
(\prettyref{sec:grammatical.np.determiner}) and the inner parts (the nucleus noun phrase),
just like how we first note that a clause is a nucleus clause plus 
information packaging and/or marking of clause types and the grammatical function of the clause 
(\prettyref{sec:grammatical.clause.overview.internal}).

The nucleus noun phrase%
\footnote{
    This is known as a \term{nominal} in \citet{cgel};
    the term however is usually used to refer to morphologically nominal parts of speech in 
    the Latin grammatical tradition.
}
can further be divided into a core argument structure surrounding the head noun
(\prettyref{sec:grammatical.np.core}),
and a series of more peripheral dependents modifying the former.

Some languages, like English, have a genitive construction 
in which a genitive phrase is raised out of the nucleus noun phrase
and acts just like the subject in a clause;
Latin doesn't have this kind of constructions.

\begin{todobox}{The ``subject'' in Latin NP}{latin-np-subject}
    Does Latin have constructions like \form{his play of the national anthem}?
\end{todobox}

In the modifier series surrounding the core argument structure, 
modifiers that are ``direct modifications'' and structurally close to the head noun modify it as a whole,
just like clausal peripheral arguments and lexical aspect do,
and are often semantically non-restrictive%
\footnote{
    Semantically, some interpretations are also restrictive,
    as \form{the beautiful dancers were admitted into the academy}
    can be understood as \translate{the aforementioned dancers who were beautiful in their dancing were admitted}.
    This restrictive meaning however can also be derived without postulating more structural complexity:
    \form{beautiful dancer} is a property,
    and \form{the beautiful dancers} simply invites the listener to recall 
    a group of people with the property of being very well-qualified dancers from the conservational context.
    If the conversational context contains a group of dancers who were not all good at dancing,
    then \form{the beautiful dancers} gains a restrictive meaning,
    but this is not dictated by the syntactic structure. 
    On the other hand, the interpretation of \form{the dancers that were beautiful}
    has to be restrictive (\translate{recall the aforementioned dancers -- now I'm talking about some of them who were beautiful}).
}
and more about eternal properties of the things referred to.
Direct modifications usually form a hierarchy in terms of scopes and frequently constituent orders
(\citealt[\citepage{453}, {[10]}]{cgel}; \citealt[\citepages{25-34}]{cinque2010syntax}),
and are easier to form idioms with nouns they modify 
\citep[\citepage{108}]{cinque2010syntax}.

Structurally higher modifiers, on the other hand, restrict the reference of the noun phrase,
just as structurally higher \ac{tam} categories do
\citep[\citepages{23,25}]{cinque2010syntax},
and they generally look like reduced relative clauses,
and are therefore predictive
(\citealt[\citechap{5}]{cinque2010syntax})%
\footnote{
    Note that there may be several types of \term{predicative} dependents,
    for example in English postnominal and prenominal predictive constituents have subtle meaning differences.
}
are usually semantically intersective.%
\footnote{
    For example, \form{beautiful} in \form{a beautiful dancer}
    may refer to the manner of the dancer dances,
    or merely be a property of the specific dancer being referred to
    (in the latter case, its meaning is close to the noun \form{beauty}).
    The latter reading is intersective in the sense
    that the reference of the noun phrase is given by the intersection
    of the property of being a dancer and the property of being beautiful. 
    The second property has nothing to do with the first,
    and is equivalent to what a relative clause would contribute to the meaning of the noun phrase.
}
We note that the innermost direct modification constituents are also intersective in some sense 
\citep[\citepage{476}]{devine2006latin}.

Core/peripheral distinction in the argument structure of a noun phrase can be determined 
by the three parameters listed in \prettyref{sec:grammatical.clause.peripheral},
and again the three parameters, despite being occasionally independent to each other,
are often correlated,
giving us an impression of a unified parameter of core/peripheral distinction
\citep[\citepages{439-443}]{cgel}.
The internal makeup of these arguments may vary cross-linguistically:
in English they can be noun phrases, nominals,%
\footnote{
    In the sense in \citet{cgel}, not in the Latinist tradition.
}
or even adjectives considering \citep[\citepages{439-440}]{cgel}: thus we have \form{a [London]_{\text{location direction modification}} [legal]_{\text{argument}} expert}.
Just as the manner adverb is closer to the main verb than \ac{tam} adverbs,
so are these arguments of the head noun closer to the head noun
than the direct modification constituents \citep[\citepage{452}, {[2]}]{cgel}.%
\footnote{
    It seems \citet{cgel} doesn't consider the possibility of peripheral adjective arguments,
    which are still to the right of direct modification adjectives.
}

Finally, at the center of the noun phrase lies the noun stem,
which may be a result of derivational morphology.
Compounding in nominal derivation should be distinguished from
phrasal nominal modification/complementation
by tests like coordination and subcategorization
-- a nominal compound should be more restricted in coordination
(\form{back- and tooth-ache} for example is marginal in English)
and may have a subcategorization frame that's totally not predictable
from its branches.
In Classical Latin real compounding is no longer productive.

Long noun phrases that fully demonstrate the aforementioned complexity are hard to find in nature texts.
In English we can construct this example:
\form{[the company's]_{\text{subject-genitive}} [three]_{\text{num}} [EXTREMELY AUWFUL]_{\text{prenominal RC}} [``attractive_{\text{evaluative}} young_{\text{age}}]_{\text{direct modification}} [movie]_{argument} [superstar'']_{\text{head: compound}} [aboard]_{\text{postnominal RC}}}.
In Latin such elicitation is not possible.

\paragraph*{Constituent order}
Similar to the case in clause structure (\prettyref{chap:order}),
the internal constituent order of \acs{np}s reflect
the hierarchical closeness of the dependents to the head noun,
but are also strongly influenced by information structure
(\citealt[\citesec{11.75}]{Pinkster1}; \citealt[\citepage{484}]{devine2006latin}).
How these factors are reflected in the exact constituent order
varies from author to author and from time to time \citep[\citepage{484}]{devine2006latin}.
We further note that even in English,
constituent orders do not transparently reflect the relative scopes
of noun phrase dependents:
participle modifiers, for example, seem to have higher scopes,
but they regularly appear after the head noun,
without reflecting their syntactic positions in the pre-head linear order
(\citealt[\citechap{5}]{cinque2010syntax}).

The genitive argument of a deverbalized noun corresponding to the subject 
usually appears to the left of the genitive argument corresponding to the object
\citep[\citepage{316}]{devine2006latin},
and indeed we observe that among seemingly direct modifications,
``inner'' ones, i.e. ones corresponding to the material, etc.,
tend to appear to the right of ``outer'' ones 
i.e. ones that require information from the conversational context 
and are not themselves properties and non-intersective
\citep[\citepages{481-482}]{devine2006latin}.
Determiners tend to appear at the leftmost position.
Therefore the aforementioned noun phrase structure is indeed reflected on the surface linear order.

The head noun is frequently weakly topicalized 
(scrambling; c.f. \prettyref{sec:grammatical.clause.information}).
We know this is an information structure operation,
because after the head noun has already appeared in the utterance,
it is often no longer fronted \citep[\citepage{484}]{devine2006latin}.
Scrambling makes the head noun a ``little subject'' in the noun phrase,
which is supposed to be more or less independent 
of the semantics of what follows it,
and this may hinder the interpretation of non-intersective modifiers,
and therefore the head noun often appears between
the non-intersective direct modifiers and the intersective direct modifiers
\citep[\citepage{485}]{devine2006latin}.
This analysis is corroborated by the fact that 
semantically empty nouns easily appear to the left edge of the noun phrase:
\form{beautiful} in \form{a beautiful person}
can never have a non-intersective meaning as it has in \form{a beautiful dancer},
and hence nothing blocks the leftward movement of the head noun.

The default constituent order is attributive-head noun
\citep[\citepage{396}]{allen1903allen}.


\subsection{Case endings and prepositions}\label{sec:grammatical.np.case}

\paragraph*{Structural cases in clauses}
All cases are not born equal. 
In Latin we have two \emph{structural} cases licensed in clauses: the nominative and the accusative,
which are purely decided by the syntactic environment 
and don't have much semantic significance.
Being nominative simply means being the subject in a finite clause or something agreeing to it 
and nothing else: 
the subject may be in a passive clause and is not agentive at all.
The structural use of the accusative case labels the object.

\paragraph*{Structural case in noun phrases}
A complete overview of its distribution reveals that 
the genitive case is a structural case 
representing \emph{any} type of nominal dependency 
and not just possession.
Some of the nominal dependencies can be 
metaphorically understood as possession, 
but others can't \citep[\citepage{244}]{oniga2014latin}:
it's hard to explain genitive of quality 
in terms of possession.
This is comparable to English \form{of}
in \form{a man of great talent} and \form{it's so kind of you},
which seems to be a purely structural phenomenon
and is not based on metaphor;
and indeed what is considered as prototypical possession 
also has strong cross-linguistic variation 
\citep[\citepages{262-263}]{dixon2010basic2}.

\paragraph*{Directional and locational constructions}
Directional and locational constructions appear in both clauses and noun phrases,
and we can observe several inherent uses of cases in them.
In directional constructions, the ablative case marks the source.
The accusative case has an inherent case usage:
it marks the \term{goal} in a directional construction
\citep[\citepage{238}]{oniga2014latin}.
The dative case also marks the goal,
but this is only possible when the path is marked as a preverb.

In (semantically more static) locational constructions,
the location argument is always ablative.

It seems we have a clear division of labor in directional constructions in Latin:
morphological cases are for marking the type of the location,
while prepositions are for marking the path
(\prettyref{sec:modifiy.prep.grammar.directional}).

\paragraph*{Inherent case in clauses}
Besides the uses of cases in directional and locational constructions in clauses,
there are several other important clausal inherent cases,
which marks arguments mentioned in \prettyref{sec:grammatical.clause.core}:
the instrument or similar arguments is marked by the ablative,
the psychological causer, i.e. the stimulus, is marked by the genitive,
the target or benefactive or experiencer is marked by the dative.

Prepositions may also be used to explicitly mark the type of the argument structure.
The preposition \form{cum} for example marks the comitative construction
(\prettyref{sec:grammatical.clause.core.comitative},
\prettyref{sec:modifiy.prep.grammar.comitative}).
In these cases the morphological cases seem to be mechanically dictated by the preposition
and may be purely realizational, without labeling any independent syntactic information.

\subsection{Determiners}\label{sec:grammatical.np.determiner}

\begin{todobox}{Determiners}{determiner}
    \begin{itemize}
        \item Demonstratives
        \item Quantifiers
        \item Identity
    \end{itemize}
\end{todobox}

\subsection{Numerals}

\subsection{Attributives}

The problem of distinguishing core and peripheral constituents
in the noun phrase is as hard as it in the clause 
(\prettyref{sec:grammatical.clause.core}).
This section only discusses adjective or numeral attributives in detail.
For in-depth discussion of relative clauses, see \prettyref{chap:relative-clause}.


\subsection{Core argument structure}\label{sec:grammatical.np.core}



\section{Adjectives and adverbs}\label{sec:grammatical.modify}

Verb-like and 

\section{Parts of speech division}\label{sec:grammatical.pos}

\subsection{Theoretical caveats and typological information}\label{sec:grammatical.pos.caveat}

\paragraph*{Difference between syntactic, morphological and lexical part of speech labels}
We have four prototypical syntactic positions for roots:
the noun as the head of a noun phrase,
the verb as the head of a clause,
the adjective as the head of a direct modification to a noun phrase
\citep[\citepage{23}]{cinque2010syntax},
and the adverb as the head of a direct modification to a clause.
They are defined by virtual
of appearing at the centers of clauses or noun phrases above
and are about \emph{syntactic positions} of roots,
which are usually quite universal.
We can also define other syntactic positions based on language-specific non-universal syntactic properties.
A nominalization construction, for example, is a type of noun phrases that
admits a root categorized as a verb at its center.
and we can then say that what appears at its center is a \term{verb used as a noun} or something similar.
These syntactic labels are literally \term{parts of speech}.
Note however parts of speech tags defined in this way are \emph{not} properties of \emph{the root}:
for example a root may both appear as a regular noun and as a regular verb in the sense defined above,
and it's impossible to discuss whether the root is a noun or a verb without the syntactic context.

There are other ways to define \term{parts of speech}.
A second type of way to define these concepts is realizational morphology.
Adjectives (defined in the first sense) may have two ways of inflection
(as in, say, Japanese),
or they may inflect just like nouns (defined in the first sense; this is the case in Latin).
In Latin we have nominal and verbal morphology (\prettyref{sec:grammatical.pos.inflection})
as well as so-called \category{particles},
but these is no specific morphological template for adjectives in the first sense.

How a root enter grammatical constructions and be morphological realized 
(specified by the first and the second sense of parts of speech above)
is guided by labels in the \emph{lexicon}.
This is the third way to define \term{part of speech labels}:
a part of speech tag in this sense should give us some \emph{lexical} information
about the possibility of the root's parts of speech in the first and the second senses;
it's about how the roots are \emph{lexically categorized}.
This is the job of the lexicographer,
and this is how the term \term{parts of speech} is used in most cases:
it's essentially about how to write a dictionary
so that you can look for a root and use it wisely to make sure 
its part of speech in the first and second senses is not messed up.
In Latin, therefore, the tag \category{first conjugation verb} tells us a bundle of information:
the root in question here regularly appears at the centers of clauses,
may participate in certain nominalization constructions,
in its morphological realization agrees with the subject of the clause in number and person,
has certain endings, etc.

(A fourth definition of parts of speech tags is semantic:
what refer to objects are noun-like, etc.
But often this is not very useful for grammatical descriptions.
The distinction between nouns for objects and nouns for events may have grammatical significance
but usually it can also be established according to more syntactic criteria,
like argument structures.)

\paragraph*{Lexicon as repository of ``idioms''}
The third sense of parts of speech can be understood as idiomization:
therefore a \category{first conjugation transitive verb}
is to be understood as the complex of a root categorized as a verb (in the first i.e. syntactic sense)
in a certain type of clause construction (i.e. transitive subcategorization)
and placed into a certain morphological template.
Recording this form and \emph{only} this form into the lexicon implicitly forbids
placements of the root into other syntactic and morphological parts of speech.

In this way the lexicon may contain construction with different sizes.
Larger constructions, like the verb with a directional prefix,
or even a whole verb phrase (e.g. \form{kick the bucket} in English)
may have developed their own meanings,
and they are to be recorded in the lexicon as well.
We will expect something like \form{treatment},
which is the root \form{treat} plus the nominalizer \term{-ment}
and the subcategorization frame introduced by it,
to be recorded as an entry in a dictionary.
Note that there is no guarantee that parts of speech labels in the third senses
that are defined for roots still work for these larger constructions:
in Mandarin Chinese, for example, reduplicated adjectives seem to form a distinct lexical class
\citep[\citesec{5.3}]{paul2014new}.

A syntactic unit with a fixed meaning recorded in the lexicon is \emph{lexicalized}.
This sometimes may lead to unit to be \emph{fossilized}:
certain syntactic operations may hinder its non-compositional interpretation 
and hence be unacceptable.
This kind of fossilization may eventually lead to \emph{syntactic fossilization},
in which the internal structure of the syntactic unit collapses,
altering its behaviors
(a typical example being some English prepositional verbs allowing passivization),
and \emph{grammaticalization},
where the syntactic unit becomes a part of the grammar 
(e.g. complement-taking verbs become auxiliary verbs,
which appear in parts of the conjugational paradigm and have to appear in a given order).

\paragraph*{Openness and content/function distinction}
Putting together the degree of grammaticalization,
the openness of a part of speech,
and the syntactic position of the units,
we have the following classification of what may appear in a dictionary:
\begin{enumerate}
    \item prototypically lexical 
    (open to new members, large in size, being able to head noun phrase, etc.), or
    \item closed but still lexical 
    (a class with no new member allowed but still large in size, being able to head noun phrases, clauses, etc.),
    \item a group of pronoun-like things, which appear in syntactic environments similar to those of nouns, verbs, etc., or
    \item a finite list of members that are pure markers of grammatical categories.
\end{enumerate}
The first two types are usually considered lexical.
The first three types are usually considered to have 
``real'' part of speech labels: 
a member of the fourth type is always attached to a construction headed by 
something in the first three types of parts of speech
and doesn't really need its own part of speech label and is prototypically functional, not lexical.

But really, in the third sense of parts of speech,
only the first two types of items have real parts of speech:
grammaticalized syntactic items 
they are a part of the grammar by definition
and strictly speaking should not be included as a part of the lexicon.
But frequently they are for convenience,
and also because the lexical/functional distinction can't always be practically established,
as said below.

The distinctions between the four type of course are not categorical.
Regarding openness, 
the most important issue is that some so-called closed classes still admit new members albeit slowly.
The distinction between lexical and functional categories is hard for a classical language,
because this distinction needs native speakers' judgments on subtle cases 
and even psycholinguistic data (in acquisition, agrammatism, etc.),
which is of course not available to Latinists.
Additionally, grammaticalization is not synchronic among speakers,
so even if we do have access to fine details of acceptability judgments, 
what we find will likely be author-specific and do not lead to any generalization to Latin as a whole.
The distinction between the third and the fourth types 
may be problematic for, say, demonstratives, 
which can be a determinative marker as well as a whole \acs{np} on its own.
Finally, as is mentioned above, prototypical lexical words
can be seen as roots plus ``categorizers,''
and in languages with lots of etymologically related nouns and verbs,
we can even argue that the noun and verb endings are of the third type.
This however is merely a terminological issue.

\paragraph*{Summary of lexicographic methodology}
We have seen that parts of speech tags have multiple senses,
and the decision to define parts of speech tags that appear \emph{in a dictionary} 
should consider both syntactic and morphological properties
(which may also be abbreviated into parts of speech tags),
which needs language-specific analyses
and require a good observation of constructions available in the language.
The lexicon contains are various idioms in the language,
and we place parts of speech tags in the lexical sense beside each entry
to convey their syntactic and morphological properties in a brief way.
Although grammatical markers in theory are not a part of the lexicon,
because the content/functional distinction shows 
diachronic and populational variances,
they are often included into dictionaries for convenience.

Depending on the peculiarities of the grammar of a language,
the dictionary may be organized in different ways.
Arabic doesn't have many compounds and has very regular derivations,
so Arabic dictionaries are always organized according to (uncategoriezed) roots.
Dictionaries for Mandarin Chinese are based on ``characters''
because Sinitic roots are mostly monosyllabic
while both ``true'' compounding and nominal attributive constructions
(c.f. English forms like \form{data processing}) are very, very frequent, 
so giving a long list of words and lexicalized noun phrases is probably not a good idea.
In Classical Latin, compounding is shunned,
and although affixation is frequent,
the meanings of the derivatives are often obscured
(for example, the meaning of the prefix \form{re-} is not always clear),
so synchronically many derived words are nothing different from roots.
Categorization of a root into several different syntactic parts of speech
is indeed possible in Latin but is not as frequent as in e.g. Arabic or Chinese
and derivation morphology can be reasonably well described based on notions like 
``this word turns into another word'',
without mentioning too much about roots.
Therefore Latin dictionaries are never based on roots but on \emph{words}. 
The same can be said for English.

In the rest of this section we discuss lexical categories (i.e. parts of speech defined in the third sense)
attested in Latin.

\subsection{Lexical categories with inflectional paradigms}\label{sec:grammatical.pos.inflection}

Some Latin word classes can be defined easily via morphology
and these classes prove to have uniform morphosyntactic behaviors.
Words with large inflection paradigms can be divided into two large classes:
those with similar morphology of prototypical nouns (i.e. \category{declension}) are \category{nominals},
while words with similar morphology of prototypical verbs (i.e. \category{conjugation})
form a uniform class rightfully called \category{verbs}.
Nominals include \category{nouns} and \category{adjectives},
the distinction between the two can also be defined morphologically.

Latin nouns, verbs, and adjectives are all prototypical lexical categories;
there seems to be no closed lexical categories (like Japanese verbal adjectives) in Latin.

\paragraph*{Nouns}\label{sec:grammatical.pos.lexical.noun}

Nouns prototypically head noun phrases.
They are declined for case and number (\prettyref{sec:regular-noun-declension}),
and the features also spread to other nominals in the \acs{np} by agreement.

According to their meanings and ability to license \acs{np} dependents, 
descriptive parameters of Latin nouns include 
inflection class, valency, compatibility with single or plural number, and TODO

\paragraph*{Verbs}

Classification parameters of the Latin verb include 
the conjugation class (\prettyref{chap:verb}),
the syntactic and semantic argument structure
(\prettyref{chap:verb-frame}), 
the event structure (TODO: Compatability with \acs{tam}),

\subsection{Latin ``particles''}

Parts of speech without much inflectional morphology include \category{prepositions}, \category{adverbs},
\category{interjections}, and \category{conjunctions},
traditional known as \category{particles}, 
although in modern usage the term \term{particle}
usually refer to grammatical items, 
and therefore at least some adverbs should be excluded from that class;  
also, some adverbs have inflection with respect to \emph{grade} 
(comparative and superlative).
It's often said that adverb class and the preposition class have a large overlap:
often a preposition has an intransitive counterpart,
which is similar to a prototypical adverb -- 
but see the discussion below about the lexical/functional distinction in prepositions.
Conjunctions may be seen as ``prepositions for clauses''.
The functions and etymologies of particles are highly diverse.

\paragraph*{Adverbs}
The category \category{adverb} is often posited as a catch-all category
for any lexical words that are not nouns, verbs or adjectives.

\begin{todobox}{Definition of adverbs}{adverb-def}
    How to define the \category{adverb} class in Latin?
\end{todobox}

A majority of adverbs historically originating from fossilized case forms.

\paragraph*{Prepositions}
\label{sec:grammatical.pos.preposition}

The preposition class usually shows great diversity  
both within a language and among different languages,
and what are known as prepositions may be lexical words or function words 
(e.g. \citealt{garzonio2021functional}).
In certain languages, some prepositions are obligatorily selected in certain grammatical constructions
and have to form a hierarchy when two or more of them appear together,
and hence may be seen as a case particle:
they are phonological realizations of grammatical relations, not lexical words.
Other prepositions are just adverbs with complements.
Several case particle-like prepositions have been discussed above;
a detail lexical/functional classification can be found in 
\prettyref{sec:modify.prep.grammar}.

\subsection{Latin pro-forms}

For categories in the third type,
Latin has pronouns and pro-adverbs; 
depending whether we treat prepositions as heads:
though prepositions are often said to be markers of a periphrastic case system,
the semantics carried by certain Latin prepositions are too complicated for a case system,
and these prepositions can be understood as adverbs selecting a complement.
TODO: adverb modifying PP  
This is also the case of adverbs:
some adverbs seem to be periphrastic markers of \acs{tam} categories
and therefore may be considered as a part of the grammar,
while others seem to carry ``real'' meanings.

\subsection{Diagram of Latin word classes} 
\prettyref{fig:latin-word-class} is a visualization of the classification of Latin word classes.

\begin{sidewaysfigure}
    \centering
    \small
    

\tikzset{every picture/.style={line width=0.3pt}} %set default line width to 0.75pt        

\begin{tikzpicture}[x=0.75pt,y=0.75pt,yscale=-0.8,xscale=0.8]
%uncomment if require: \path (0,674); %set diagram left start at 0, and has height of 674

%Straight Lines [id:da0819119912251447] 
\draw [color={rgb, 255:red, 155; green, 155; blue, 155 }  ,draw opacity=0.2 ]   (109.33,384.91) -- (793.33,384.91) ;
%Rounded Rect [id:dp9835773367494156] 
\draw  [color={rgb, 255:red, 155; green, 155; blue, 155 }  ,draw opacity=1 ][fill={rgb, 255:red, 155; green, 155; blue, 155 }  ,fill opacity=0.2 ] (141.33,256.64) .. controls (141.33,246.48) and (149.57,238.24) .. (159.73,238.24) -- (214.93,238.24) .. controls (225.1,238.24) and (233.33,246.48) .. (233.33,256.64) -- (233.33,518.84) .. controls (233.33,529) and (225.1,537.24) .. (214.93,537.24) -- (159.73,537.24) .. controls (149.57,537.24) and (141.33,529) .. (141.33,518.84) -- cycle ;
%Rounded Rect [id:dp12690257166641228] 
\draw  [color={rgb, 255:red, 155; green, 155; blue, 155 }  ,draw opacity=1 ][fill={rgb, 255:red, 155; green, 155; blue, 155 }  ,fill opacity=0.2 ] (233.33,256.64) .. controls (233.33,246.48) and (241.57,238.24) .. (251.73,238.24) -- (306.93,238.24) .. controls (317.1,238.24) and (325.33,246.48) .. (325.33,256.64) -- (325.33,518.84) .. controls (325.33,529) and (317.1,537.24) .. (306.93,537.24) -- (251.73,537.24) .. controls (241.57,537.24) and (233.33,529) .. (233.33,518.84) -- cycle ;
%Rounded Rect [id:dp4571683339095527] 
\draw  [color={rgb, 255:red, 155; green, 155; blue, 155 }  ,draw opacity=1 ][fill={rgb, 255:red, 155; green, 155; blue, 155 }  ,fill opacity=0.2 ] (349.33,256.57) .. controls (349.33,248.1) and (356.2,241.24) .. (364.67,241.24) -- (426,241.24) .. controls (434.47,241.24) and (441.33,248.1) .. (441.33,256.57) -- (441.33,302.57) .. controls (441.33,311.04) and (434.47,317.91) .. (426,317.91) -- (364.67,317.91) .. controls (356.2,317.91) and (349.33,311.04) .. (349.33,302.57) -- cycle ;
%Rounded Rect [id:dp13945556495695044] 
\draw  [color={rgb, 255:red, 155; green, 155; blue, 155 }  ,draw opacity=1 ][fill={rgb, 255:red, 155; green, 155; blue, 155 }  ,fill opacity=0.2 ] (122.93,223.64) .. controls (122.93,211.82) and (132.51,202.24) .. (144.33,202.24) -- (316.93,202.24) .. controls (328.75,202.24) and (338.33,211.82) .. (338.33,223.64) -- (338.33,537.84) .. controls (338.33,549.66) and (328.75,559.24) .. (316.93,559.24) -- (144.33,559.24) .. controls (132.51,559.24) and (122.93,549.66) .. (122.93,537.84) -- cycle ;
%Straight Lines [id:da9866800561526903] 
\draw [color={rgb, 255:red, 155; green, 155; blue, 155 }  ,draw opacity=0.2 ]   (595.33,577.91) -- (595.33,184.91) ;
%Rounded Rect [id:dp9919662328975185] 
\draw  [color={rgb, 255:red, 155; green, 155; blue, 155 }  ,draw opacity=1 ][fill={rgb, 255:red, 155; green, 155; blue, 155 }  ,fill opacity=0.2 ] (435,431.44) .. controls (435,427.46) and (438.22,424.24) .. (442.2,424.24) -- (688.8,424.24) .. controls (692.78,424.24) and (696,427.46) .. (696,431.44) -- (696,453.04) .. controls (696,457.02) and (692.78,460.24) .. (688.8,460.24) -- (442.2,460.24) .. controls (438.22,460.24) and (435,457.02) .. (435,453.04) -- cycle ;
%Rounded Rect [id:dp04492990660366303] 
\draw  [color={rgb, 255:red, 155; green, 155; blue, 155 }  ,draw opacity=1 ][fill={rgb, 255:red, 155; green, 155; blue, 155 }  ,fill opacity=0.2 ] (147,482.31) .. controls (147,476.75) and (151.51,472.24) .. (157.07,472.24) -- (579.93,472.24) .. controls (585.49,472.24) and (590,476.75) .. (590,482.31) -- (590,512.51) .. controls (590,518.07) and (585.49,522.57) .. (579.93,522.57) -- (157.07,522.57) .. controls (151.51,522.57) and (147,518.07) .. (147,512.51) -- cycle ;
%Shape: Path Data [id:dp3209513602067542] 
\draw  [color={rgb, 255:red, 155; green, 155; blue, 155 }  ,draw opacity=1 ][fill={rgb, 255:red, 155; green, 155; blue, 155 }  ,fill opacity=0.2 ] (478.97,255.24) -- (552.94,255.24) .. controls (557.16,255.24) and (560.58,258.46) .. (560.58,262.43) -- (560.58,367.51) .. controls (560.58,372.79) and (565.13,377.07) .. (570.74,377.07) -- (630.37,377.07) .. controls (630.59,377.07) and (630.82,377.07) .. (631.04,377.05) -- (712.31,377.05) .. controls (717.01,377.05) and (720.81,380.63) .. (720.81,385.05) -- (720.81,434.24) .. controls (720.81,438.66) and (717.01,442.24) .. (712.31,442.24) -- (478.97,442.24) .. controls (474.75,442.24) and (471.33,439.02) .. (471.33,435.05) -- (471.33,262.43) .. controls (471.33,258.46) and (474.75,255.24) .. (478.97,255.24) -- cycle ;
%Rounded Rect [id:dp9478936078417666] 
\draw  [color={rgb, 255:red, 155; green, 155; blue, 155 }  ,draw opacity=1 ][fill={rgb, 255:red, 155; green, 155; blue, 155 }  ,fill opacity=0.2 ] (451.33,237.22) .. controls (451.33,228.95) and (458.04,222.24) .. (466.31,222.24) -- (759.35,222.24) .. controls (767.63,222.24) and (774.33,228.95) .. (774.33,237.22) -- (774.33,552.26) .. controls (774.33,560.53) and (767.63,567.24) .. (759.35,567.24) -- (466.31,567.24) .. controls (458.04,567.24) and (451.33,560.53) .. (451.33,552.26) -- cycle ;

% Text Node
\draw (169,267) node [anchor=north west][inner sep=0.75pt]   [align=left] {noun};
% Text Node
\draw (252,267) node [anchor=north west][inner sep=0.75pt]   [align=left] {adjective};
% Text Node
\draw (158,479) node [anchor=north west][inner sep=0.75pt]   [align=left] {personal \\pronoun};
% Text Node
\draw (244,479) node [anchor=north west][inner sep=0.75pt]   [align=left] {correlative \\pronoun};
% Text Node
\draw (368,266) node [anchor=north west][inner sep=0.75pt]   [align=left] {verb};
% Text Node
\draw (565.5,442.24) node   [align=left] {preposition};
% Text Node
\draw (674,517) node [anchor=north west][inner sep=0.75pt]   [align=left] {conjunction};
% Text Node
\draw (492,264) node [anchor=north west][inner sep=0.75pt]   [align=left] {adverb};
% Text Node
\draw (674,471) node [anchor=north west][inner sep=0.75pt]   [align=left] {interjection};
% Text Node
\draw (151,361) node [anchor=north west][inner sep=0.75pt]  [color={rgb, 255:red, 155; green, 155; blue, 155 }  ,opacity=1 ] [align=left] {noun \\morphology};
% Text Node
\draw (238.14,361) node [anchor=north west][inner sep=0.75pt]  [color={rgb, 255:red, 155; green, 155; blue, 155 }  ,opacity=1 ] [align=left] {adjective \\morphology};
% Text Node
\draw (107.33,384.91) node [anchor=east] [inner sep=0.75pt]  [color={rgb, 255:red, 155; green, 155; blue, 155 }  ,opacity=1 ] [align=left] {with real \\category \\label};
% Text Node
\draw (795.33,384.91) node [anchor=west] [inner sep=0.75pt]  [color={rgb, 255:red, 155; green, 155; blue, 155 }  ,opacity=1 ] [align=left] {with \\no real \\category \\label};
% Text Node
\draw (141,212.24) node [anchor=north west][inner sep=0.75pt]  [color={rgb, 255:red, 155; green, 155; blue, 155 }  ,opacity=1 ] [align=left] {nominal};
% Text Node
\draw (595.33,181.91) node [anchor=south] [inner sep=0.75pt]  [color={rgb, 255:red, 155; green, 155; blue, 155 }  ,opacity=1 ] [align=left] {not a part\\of grammar};
% Text Node
\draw (595.33,580.91) node [anchor=north] [inner sep=0.75pt]  [color={rgb, 255:red, 155; green, 155; blue, 155 }  ,opacity=1 ] [align=left] {a part of\\grammar};
% Text Node
\draw (499,489.5) node [anchor=north west][inner sep=0.75pt]   [align=left] {pro-adverb};
% Text Node
\draw (379,489.5) node [anchor=north west][inner sep=0.75pt]  [color={rgb, 255:red, 155; green, 155; blue, 155 }  ,opacity=1 ] [align=left] {pro-forms};
% Text Node
\draw (661,241.5) node [anchor=north west][inner sep=0.75pt]  [color={rgb, 255:red, 155; green, 155; blue, 155 }  ,opacity=1 ] [align=left] {particle};


\end{tikzpicture}

    \caption{Latin word classes}
    \label{fig:latin-word-class}
\end{sidewaysfigure}

\paragraph*{Parts of speech absent in Latin} 
Articles (English \form{a} or \form{the}), 
despite prevalent in other Indo-European languages,
are missing in Latin.
This, together with the fact that Classical Sanskrit and Old Persian didn't have articles 
and the Slavic languages still don't,
is a strong indicator that \ac{pie} didn't have articles. 
Note that the fact that Latin lacks articles 
doesn't mean the determiner syntactic function doesn't exist:
there are evidences suggesting certain aspects of the behavior of Latin \acs{np}s 
are just like English \citep{giusti2014split}.

\section{Reading Latin}\label{sec:grammatical.parse}

The morphological richness (and the scrambled constituent order)
makes Latin hard to read 
especially for people whose first languages are, say, 
English or Mandarin. 
Whenever unsure about a sentence, do the follows: 
\begin{enumerate}
    \item Skim over the words and label the stems that can be easily recognized.  
    \item Skim over and circle uncontroversial grammatical items,  
        like inflectional endings and prepositions. 
        It's OK to be unable to interpret them immediately (and we need the steps below).
    \item According to these grammatical items, 
        segment the sentence into large parts:
        noun phrases, relative or complement or adverbial clauses, 
        verbal system, etc.
    \item Choose a grammatical item and tentatively give a list of possible features it carries.  
        For example, seeing \form{-v-} in a verb usually means
        it's based on the perfect stem
        (\prettyref{sec:verb-inflection.stem.perfect});
        \form{-um} may be second declension accusative,
        but there are other possibilities
        (\prettyref{tbl:declension-ending-nouns-list}). 
    \item Use constraints like 
        ``the preposition \form{in} licenses the accusative case or the ablative case'' 
        to narrow the possibilities identified above.
    \item Draw unfinished dependency arrows:
        for a verb, draw arrows pointing to the subject and/or the object; 
        for a nominative adjective, draw an arrow pointing to the modified head noun. 
        But note that it's possible that the subject is dropped, 
        or there is no head noun (compare English \form{the poor}).
        Then try to pair the arrows.
\end{enumerate}
Repeat the above procedure and finally the sentence can be understood. 
This procedure is demonstrated in \prettyref{sec:text.vulgate.john}.


\chapter{Pro-forms}\label{chap:pro-forms}

\section{Overview}

Pro-forms can be divided into pronouns and correlatives.


\section{Personal pronouns}

Latin pronouns are complete \acs{np} themselves:
no attributives should be attached to them. 
Pronouns are declined for case, gender and number,
and they also can be governed by prepositions.

It should be noted that what counts as third person pronouns 
is not uncontroversial among reference books regarding Latin; 
some place the reflexive \form{sui} series into the third person column 
in the table of personal pronouns, 
citing the fact that the \form{is, ea, id} series mark the gender category, 
while the first and second person pronouns do not.


\section{Reflexive pronouns}

It can be seen that \form{se} lacks nominative forms.
In some circumstances, 
the reflexive \form{se} looks like the subject;
but these cases are more appropriately analyzed 
as reflexive usages of transitive verbs
(\ref{ex:np.minor.reflexive.1}).

\begin{exe}
    \ex\label{ex:np.minor.reflexive.1} \gll \dots necessarium est primo investigare de ipsa sacra doctrina, 
    qualis sit, et ad quae [se extendat]  \\
    \\
    \glt \translate{It is necessary to first investigate into this sacred doctrine, 
    of what kind it would be, 
    and what it would extend itself to (i.e. its nature and extent).} (\literature{Summa}, I q. 1 pr.)
\end{exe}

\section{The ``basic'' correlatives}

Latin correlatives can be classified in \prettyref{tbl:correlatives};
the rows correspond to their immediate roles in the clause 
(``basic'' i.e. \ac{np} heads, 
place, source, time, etc. that are otherwise expressed by prepositional phrases),
while the columns correspond to their meaning and/or 
relation with a more precedent in a higher position,
like whether the word is a demonstrative 
or maybe a relative pronoun. 

\begin{sidewaystable}
    \centering
    \caption{Classification of Latin correlatives}
    \label{tbl:correlatives}
    {\small \begin{tabular}{l l l l l l l l l l l l}
    \toprule
                              &       & \multicolumn{3}{c}{``question''}                 & \multicolumn{3}{c}{demonstrative} &          &            &            &          \\ \cmidrule(lr){3-5} \cmidrule(lr){6-8}
                              &       & interrogative & relative & indefinite relative & proximal    & medial   & distal   & identity & indefinite & collective & negative \\ \midrule
\multirow{2}{*}{nominal head} & basic &               &          &                     &             &          &          &          &            &            &          \\
                              & dual  &               &          &                     &             &          &          &          &            &            &          \\
                              &       &               &          &                     &             &          &          &          &            &            &         \\
                              \bottomrule
\end{tabular}}
\end{sidewaystable}

The form of correlatives in the ``basic'' can be summarized as the follows:
\begin{itemize}
    \item \emph{Indefinite}. They may be homonyms of \form{qui} or \form{quis},
    or they may be formed by adding \form{ali-} or \form{-dam} to \form{qui} or \form{quis}
    (with possible internal alternations), 
    or maybe they are formed in the format of \form{ali-quis-X-quam}.
    \item \emph{Indefinite relative}. Formed by reduplication of \form{quis} forms.
    \item \emph{Other}. Always take a form similar to \form{alter} or \form{alius}.  
\end{itemize}

Here is a guide for quickly determining the meaning of a ``basic'' pro-form:
\begin{itemize}
    \item What starts with \form{ill-} is a distal demonstrative.
    \item What starts with \form{ist-} is a medial demonstrative.
    \item What starts with \form{h-} is a proximal demonstrative.
    \item What starts with \form{s-} is a reflexive pronoun, 
    or maybe a reflexive possessive determiner like \form{suus}.
    \item What starts with \form{qu-} is an interrogative pronoun 
    or a relative pronoun or an indefinite pronoun 
    or an indefinite relative.
    \item What starts with \form{null-} is a negative pronoun.
    \item What starts with \form{omn-} is a collective pronoun.
    \item What starts with \form{ips-} is an identity pronoun (\translate{this exactly \dots}).
    \item What starts with \form{alter-} is an ``other'' pronoun.
    \item What starts with \form{ali-} may be an ``other'' pronoun, or an indefinite pronoun.
    \item What \emph{ends} with \form{-dem} and starts with a third person pronoun
        is an identity pronoun; sound changes are possible. 
        For example \form{īdem} or \form{eandem}. 
    \item What \emph{ends} with \form{-dam} and starts with \form{q-} is an indefinite pronoun.
\end{itemize} 

Some forms can be easily confused with correlatives.
They include:
\begin{itemize}
    \item \form{quod} used to introduce a finite complement clause.
    \item \form{quia} as a conjunction of a reason clause.
\end{itemize}

\subsection{Demonstratives}

Latin has proximate, medial, and distal demonstratives:
the first refers to something close to the speaker, 
the last refers to something far away,
and the second refers to something in between,
probably near the listener.

\subsection{Interrogative and relative pronouns}

\section{Degree}

The \form{tam quam} pair may be seen as a pro-form that takes the place of the degree expression of a clause.

\chapter{Nominal morphology}

\section{The structure of noun}\label{sec:np.noun}

\subsection{The inflectional endings}\label{sec:noun.paradigm.introduction}

Latin nouns are declined for case and number,
which is agreed upon by other nominal words in the \acs{np}.
The structure of the Latin noun is just
the stem plus inflectional ending 
with the case and number categories fused into one suffix
(\prettyref{sec:regular-noun-declension}).
There are five declension classes in Latin.

There exists a discrepancy in recognizing the stem of a noun.
The traditional approach is to find the common part 
of all case forms of one nominal lexeme;
thus \form{rosam} \translate{rose-\category{sg}.\category{acc}}
is analyzed as \form{ros-am}, 
with \form{ros-} being the stem.
\citet[\citepage{17}]{allen1903allen} documents the paradigms of all the five declension classes 
in this approach.
The full list of attested noun endings is \prettyref{tbl:declension-ending-nouns-list}.
The list is still not the full picture of Latin nominal inflection:
stem alternation is seen in the third declension
(\prettyref{sec:np.inflection.3}),
and this alternation is frequently directly exposed at the end of a nominative and accusative noun.

\begin{table}[H]
    \caption{Declension endings; Roman numerals are declension classes}
    \label{tbl:declension-ending-nouns-list}
    \centering
    \begin{tabular}{ll}
    \toprule
    ending & declension \\ \midrule
    -a & I, \category{sg}.\category{nom}, \category{sg}.\category{voc}; IIN, \category{pl}.\category{nom}, \category{pl}.\category{acc}, \category{pl}.\category{voc}; IIIN, \category{pl}.\category{nom}, \category{pl}.\category{acc}, \category{pl}.\category{voc} \\ 
    -ā & I, \category{sg}.\category{abs} \\ 
    -ae & I, \category{sg}.\category{gen}, \category{sg}.\category{dat}, \category{pl}.\category{nom}, \category{pl}.\category{voc};  \\ 
    -am & I, \category{sg}.\category{acc} \\ 
    -ārum & I, \category{pl}.\category{gen} \\ 
    -ās & I, \category{pl}.\category{acc} \\ 
    -e & IIM, \category{sg}.\category{voc}; IIIFMN, \category{sg}.\category{abs} \\ 
    -ē & V, \category{sg}.\category{abs} \\ 
    -ei/-ēi & V, \category{sg}.\category{gen}, \category{sg}.\category{dat} \\ 
    -em & IIIFM, \category{sg}.\category{acc}; V, \category{sg}.\category{acc} \\ 
    -ēbus & V, \category{pl}.\category{dat}, \category{pl}.\category{abs} \\ 
    -ērum & V, \category{pl}.\category{gen} \\ 
    -ēs & IIIFM, \category{pl}.\category{nom}, \category{pl}.\category{acc}, \category{pl}.\category{voc}; V, \category{sg}.\category{nom}, \category{sg}.\category{voc}, \category{pl}.\category{nom}, \category{pl}.\category{acc}, \category{pl}.\category{voc} \\ 
    -ī & IIM, \category{sg}.\category{gen}, \category{sg}.\category{voc}, \category{pl}.\category{nom}, \category{pl}.\category{voc}; IIN, \category{sg}.\category{gen}; IIIFMN, \category{sg}.\category{dat} \\ 
    -ibus & IIIFMN, \category{pl}.\category{dat}, \category{pl}.\category{abs}; IVFMN, \category{pl}.\category{dat}, \category{pl}.\category{abs} \\ 
    -is & IIIFMN, \category{sg}.\category{gen} \\ 
    -īs & I, \category{pl}.\category{dat}, \category{pl}.\category{abs}; IIMN, \category{pl}.\category{dat}, \category{pl}.\category{abs} \\ 
    -ō & IIMN, \category{sg}.\category{dat}, \category{sg}.\category{abs} \\ 
    -ōs & IIM, \category{pl}.\category{acc} \\ 
    -ōrum & IIMN, \category{pl}.\category{gen} \\ 
    -r & IIM, \category{sg}.\category{nom}, \category{sg}.\category{voc} \\ 
    -ū & IVFM, \category{sg}.\category{abs}; IVN, \category{sg}.\category{nom}, \category{sg}.\category{dat}, \category{sg}.\category{acc}, \category{sg}.\category{abs}, \category{sg}.\category{voc} \\ 
    -ua & IVN, \category{pl}.\category{nom}, \category{pl}.\category{acc}, \category{pl}.\category{voc} \\ 
    -uī & IVFM, \category{sg}.\category{dat} \\ 
    -um & IIMN, \category{sg}.\category{acc}; IIN, \category{sg}.\category{nom}, \category{sg}.\category{voc}; IIIFMN, \category{pl}.\category{gen}; IVFM, \category{sg}.\category{acc} \\ 
    -us & IIM, \category{sg}.\category{nom}; IVFM, \category{sg}.\category{nom}, \category{sg}.\category{voc} \\ 
    -ūs & IVFM, \category{sg}.\category{gen}, \category{pl}.\category{nom}, \category{pl}.\category{acc}, \category{pl}.\category{voc}; IVN, \category{sg}.\category{gen} \\ 
    -uum & IVFMN, \category{pl}.\category{gen} \\ \bottomrule
\end{tabular}
\end{table}

\begin{infobox}{Things that look like case endings but are not}{case-ending-disambiguity}
    There are several possible cases where something seeming to be a noun case ending 
    turns out to be something else.
    To name a view:
    \begin{itemize}
        \item The ending \form{-as} is not always the first declension plural accusative ending;
        \form{-a-} could be a part of the stem of a third declension noun.
        Consider \form{vēritās}.
        \item The ending sequence \form{-io} can'be be found in \prettyref{tbl:declension-ending-nouns-list}
        and we may hurry to the conclusion 
        that it's the third declension abstract noun ending \form{-io} 
        in the nominative or accusative case.
        Not necessarily -- 
        it can also be \form{-ium} in the dative or ablative case 
        (when the macron symbol for long vowels are not used).
        \item A word ending with \form{-us} could be an adverb:  
            \form{-tus} is an adverb suffix.
    \end{itemize}
\end{infobox}

Another approach -- informed by historical comparison with other Indo-European languages -- 
make use of the thematic vowel 
in declension endings that can also be found in
\prettyref{tbl:declension-ending-nouns-list}
\citep[\citepages{45, 63}]{oniga2014latin}.
Thus, the first conjugation singular accusative ending \form{-am} 
is \form{-a-m}, 
with \form{-a-} being the thematic vowel;
we may also attach the thematic vowel back to the stem in the narrow sense \form{ros-} 
and redefine the complex \form{ros-a-} as the stem of \form{rosam}.
In this way, the declension endings can be summarized in a neater way.
A comparison between reconstructed PIE noun and pronoun endings and early Latin endings
is given in \citet[\citepage{14}]{clackson2011blackwell}.

\begin{table}[H]
    \caption{PIE nominal case endings}
    \centering
    \begin{tabular}{lll}
        \toprule
                           & singular     & plural      \\
        \midrule                       
        \category{nom.m/f} & \form{*-s}   & \multirow{2}{*}{\form{*-es}} \\
        \category{nom.n}   & \form{*-m}/∅ &  \\
        \category{acc}     & \form{*-m}   & \form{*-ms} \\
        \category{gen}     & \form{*-es}/\form{*-os} & \form{*-om} \\
        \category{dat}     & \form{*-ei}  & \multirow{2}{*}{\form{*-b^hos}/\form{*-\={o}is}} \\
        \category{abl}     & \form{*-\={o}d}/∅ & \\
        \bottomrule
    \end{tabular}
\end{table}

\begin{todobox}{PIE ending}{pie-noun-declension}
    \begin{itemize}
        \item What's the neutral plural ending?
        \item Stem's selection of endings in PIE
    \end{itemize}
\end{todobox}

\subsection{Derivational morphology}

Inside the stem we find a list of 

\section{Declension of regular nouns}\label{sec:regular-noun-declension}

\subsection{The first declension}



\subsection{The second declension}

The stem-final thematic vowel in the second declension is historically \form{-o-}.
Because of sound changes, the \category{nom.sg} ending becomes \form{-us},

\subsection{The third declension}\label{sec:np.inflection.3}

The third declension is a big tent containing several subclasses.
Nominative singular endings attested in the third declension
include \form{-s}, \form{-t}, \form{-x} (i.e. \form{-cs}) \citep[\citesec{53}]{allen1903allen}.

\section{Uses of cases}\label{sec:np.case-distribution}

The roles of the five cases are not symmetric.


On the other hand, the rest cases are \emph{inherent} cases:
they have direct semantic interpretations
-- ``source'' or ``target'' or \dots -- themselves,
and once an inherent case is assigned to an \acs{np},
the latter is ``sealed'' just like a prepositional phrase:
the change of the outside syntactic environment 
doesn't change anything inside.
Sometimes inherent cases are realized as prepositions.

\paragraph*{Historical evolution of case morphology}
From PIE to Romance languages, the case system in Latin has progressively eroded.
There are two ways to simplify a case system:
it's possible that two morphological cases are merged altogether,
which means the relevant grammatical relations are no longer realized in nominal morphology
(e.g. \prettyref{sec:noun.case.accusative.vulgar});
it's also possible that the structure of the case paradigm is retained,
although for some declension classes,
some of the case forms are regularized
(e.g. \prettyref{sec:noun.case.nominative.vulgar}).


\subsection{The nominative}

\paragraph*{Historical evolution in Vulgar Latin}
\label{sec:noun.case.nominative.vulgar}
The first declension plural nominative ending \form{-ae}
started to be replaced by the accusative \form{-as} starting at no later than first century A.D.
but arrived at Gaul only after the fifth century.
Note that this replacement is not equivalent to elimination of
morphological marking of the subject/object distinction:
the \form{-us}/\form{-um} alternation is still kept
\citep[\citepage{55}]{herman2010vulgar}.
The motivation may be analogy of the neutralization 
of the nominative and accusative endings in the third declension,
and/or analogy of Oscan and Umbrian nominative plural ending \form{-as}.

\subsection{The accusative}

\paragraph*{In Vulgar Latin}
\label{sec:noun.case.accusative.vulgar}
A general tendency in Vulgar Latin is the ablative in prepositional constructions 
being replaced by the accusative \citep[\citepage{53}]{herman2010vulgar}.
Similarly, internal arguments which should be genitive or dative or ablative
tend to be accusative in Vulgar Latin
\citep[\citepage{54}]{herman2010vulgar}.
This suggests the accusative being recognized as the ``default case''.

\subsection{The dative}\label{sec:dative-distribution}

The dative case is an inherent case assigned to the benefactive, the experiencer, and the purpose
\citep[\citepage{251}]{oniga2014latin};
thus it's the case assigned to the indirect object (\prettyref{sec:vp.complement.indirect-object})
as well as many adjuncts TODO

\subsection{Ablative}

so its distribution is similar to the dative case: 
it appears in source and instrument indirect objects (TODO) 
as well as various adjuncts.

A source argument
can be the position from which something moves 
(\concept{ablative of source})
or the source in a separation event 
(\concept{ablative of separation}: \translate{remove}, \translate{deprive}),
or the place where something comes into being 
(\concept{ablative of material}, \translate{birth}, \translate{origin}), 
or the cause of something (``the source of the event'', \concept{ablative of cause});
the agent in the passive voice possibly 
comes from one of the figurative use of the ablative as well.

\chapter{Adjectives and adverbs}

The adjective class and the adverb class are linked together by several factors:
the adjective phrase and the adverb phrase are both prototypical modifiers,
often with parallel structures;
they both have the category of degree; 
adverbs can be formed regularly from adjectives.

\section{Declension of regular adjectives}

Peripheral arguments may also be regarded as adverbials.
This chapter, however, is mainly about mean, TODO

\section{Arguments of adjectives}

\subsection{Dative}

\begin{exe}
    \ex \gll \dots {} consubstantialem Patri \dots \\
    {} consubstantial-\category{sg.acc} Father-\category{sg.dat} \\
    \glt \translate{consubstantial with Father} (Nicene Creed)
\end{exe}

\section{Comparative construction}

\section{Prepositions}

\subsection{Are prepositions function words or content words?}\label{sec:modify.prep.grammar}

As is mentioned in \prettyref{sec:grammatical.pos.preposition},
various degrees of grammaticalization may happen to prepositions.
In this section,
we discuss Latin prepositions which are better analyzed as
grammatical markers instead of transitive adverbs.

\begin{todobox}{Latin preposition and adverbs}{preposition-adverb}
    Can \form{ex} or \form{in} be analyzed as adverbs?
    If so they should be able to be modified by other adverbs?
\end{todobox}

\paragraph*{Preposition in valency alternation}
English \form{of} is used to introduce the object in nominalization 
and \form{by} is used to introduce the deep A argument in passivization.
We can therefore say that the first is an analytic genitive case marker,
and the second is an analytic instrument case marker.

In Latin, one use of \form{ab} falls in this category:
personal agents in passive clauses are always 
preceded by \form{ab}.
The ablative case governed by \form{ab}, in this use, marks nothing:
this resembles the fact that English perfect auxiliary \form{has} selects a past participle 
or the Japanese verbal conjugation suffixes add \category{continuative} or \category{gerund} endings to the verbal complex before them,
which can be explained by diachronic reasons.

\paragraph*{Preposition sequence in directional constructions}
\label{sec:modifiy.prep.grammar.directional}
In directional constructions,
stacking of case markers and/or prepositions in a fixed order 
and with a finite number of possibilities
indicate that the prepositions have already been grammaticalized.
This is the case in Finnish or, to a lesser degree, in English 
(\citealt{spatialpp}; consider English \form{from behind the wall}).

In Classical Latin there is no clear example of this type:
no preposition stacking is allowed in Latin,
so coexistence patterns cannot be used to establish a lexical/functional distinction.
We however note that directional prepositions in Latin
seem to always encode the path
and never the properties of the location.
Semantically, \form{ex} encodes a path going away from a given region, 
while \form{in} encodes a path going into a given region,
and the status of the location as the goal or the source
is marked by its case.
Tentatively we can analyze \form{ex} and \form{in}
as grammaticalized markers of the type of the path in the directional construction.

In early Christian Latin, the form \form{abante faciem domini}
\translate{from in front of the face of Lord} is attested
\citep[\citepage{24}]{herman2010vulgar},
and the path-figure-ground hierarchy is perfectly demonstrated here.
This shows that at least in some sub-elite varieties of Latin,
\form{ab} and \form{ante} are grammaticalized directional markers.

\paragraph*{Preposition as modifier in another prepositional phrase}
It is also possible that a preposition is an adverbial modifier of an existing prepositional construction
(\citealt{botwinik2008greek};
c.f.English \form{the boat drifted [\textbf{down} from \textbf{up} above the dam]}).
This case seems absent in Latin.

\paragraph*{Comitative}
\label{sec:modifiy.prep.grammar.comitative}
The preposition \form{cum} regularly marks the comitative construction,
both in verb frames and in comitative \acp{np}.

\subsection{Prepositions as transitive adverbs}

Some Latin prepositions are likely as lexical as other lexical words.
They are ``transitive adverbs''.

The form \form{circum} appears in several contexts
and means a (static) location in space or a dynamic path 
depending on whether a path is implied by the verb;
in all these cases, however, the meaning \translate{around sth.} is always kept.
This is very unusual if it is understood as a purely grammatical orientation particle:
homophonous grammatical particles appearing in two contexts in principle can have rather different meanings.
It is likely a root surrounded by noun phrases expressing the \category{ground} and the \category{figure} 
in a spatial relation,
resulting in a locational expression \citet{mare2018issues}.

\subsection{Intransitive prepositions}

Many prepositions can also be used without any complements 
and may thus classified as an adverb; 
they can also appear as preverbs.

The preposition/adverb distinction

\chapter{Verb morphology}\label{chap:verb}

\section{The structure of the inflectional paradigm}

\subsection{The verb template of the finite paradigm}\label{sec:verb.finite.paradigm}

For a heavily inflected language like Latin
it's an appealing idea 
to start the discussion on clause structure 
with a surface-orientated analysis of verb structure.
The morphological template is shown in \prettyref{fig:latin-verb}; 
some inflected Latin verbs and their parts are shown in \prettyref{tbl:latin-finite-verbs}.
Traditionally, the verb is divided 
into the \concept{stem} and the \concept{ending}.
Derivation in Latin is predominantly preverbal,
and hence the conjugation is mostly about the final lexical morpheme in the verb stem,
which is represented as the root in \prettyref{fig:latin-verb}.
There may be a perfect suffix after the root.
Components of the verb ending include 
the \concept{tense and mood suffix} (also known as the \concept{tense suffix}),
and the person, number and voice marker,
which is called the \concept{personal ending} here, 
following the terminology in \citet[\citesec{165}]{allen1903allen}.

\begin{figure}[H]
    \centering
    

\tikzset{every picture/.style={line width=0.3pt}} %set default line width to 0.75pt        

\begin{tikzpicture}[x=0.75pt,y=0.75pt,yscale=-0.85,xscale=0.85]
%uncomment if require: \path (0,414); %set diagram left start at 0, and has height of 414

%Shape: Rectangle [id:dp7661505767193904] 
\draw  [color={rgb, 255:red, 74; green, 144; blue, 226 }  ,draw opacity=1 ] (200,155) -- (278.01,155) -- (278.01,201.48) -- (200,201.48) -- cycle ;

%Shape: Rectangle [id:dp9235304615883395] 
\draw  [color={rgb, 255:red, 74; green, 144; blue, 226 }  ,draw opacity=1 ] (51,155) -- (187.01,155) -- (187.01,201.48) -- (51,201.48) -- cycle ;

%Shape: Rectangle [id:dp4025769445021785] 
\draw  [color={rgb, 255:red, 74; green, 144; blue, 226 }  ,draw opacity=1 ] (290,155) -- (416.01,155) -- (416.01,201.48) -- (290,201.48) -- cycle ;

%Shape: Rectangle [id:dp9180138973853116] 
\draw  [color={rgb, 255:red, 80; green, 227; blue, 194 }  ,draw opacity=1 ] (428,155) -- (564.01,155) -- (564.01,201.48) -- (428,201.48) -- cycle ;

%Shape: Rectangle [id:dp2406875282163703] 
\draw  [color={rgb, 255:red, 80; green, 227; blue, 194 }  ,draw opacity=1 ] (575,155) -- (722.01,155) -- (722.01,201.48) -- (575,201.48) -- cycle ;


% Text Node
\draw (239.01,178.24) node   [align=left] {core stem};
% Text Node
\draw (119.01,178.24) node   [align=left] {\begin{minipage}[lt]{87.06pt}\setlength\topsep{0pt}
\begin{center}
derivation \\prefix/compouding
\end{center}

\end{minipage}};
% Text Node
\draw (353.01,178.24) node   [align=left] {stem-final vowel};
% Text Node
\draw (496.01,178.24) node   [align=left] {\begin{minipage}[lt]{83.46pt}\setlength\topsep{0pt}
\begin{center}
tense and mood \\marking
\end{center}

\end{minipage}};
% Text Node
\draw (648.51,178.24) node   [align=left] {\begin{minipage}[lt]{84.5pt}\setlength\topsep{0pt}
\begin{center}
person, number, \\and voice marking
\end{center}

\end{minipage}};
% Text Node
\draw (181,228) node [anchor=north west][inner sep=0.75pt]  [color={rgb, 255:red, 74; green, 144; blue, 226 }  ,opacity=1 ] [align=left] {verb stem};
% Text Node
\draw (532.01,228) node [anchor=north west][inner sep=0.75pt]  [color={rgb, 255:red, 80; green, 227; blue, 194 }  ,opacity=1 ] [align=left] {verb ending};


\end{tikzpicture}

    \caption{The template of Latin verbs.
    Indentation means linear order and not necessarily constituency structure.}
    \label{fig:latin-verb}
\end{figure}

The core stem assumes semi-regular alternations.
The perfect marker is often but not always \form{-v-} (\prettyref{sec:verb-inflection.stem.perfect}).
Although the aspect marker is a part of the \acs{tam} marking
(\prettyref{sec:verb-inflection.finite-template.tame}),
it somehow is much more closely attached to the root 
in the morphophonological realization of the verbal system:
how it is realized is not completely predictable,
and therefore the two finite stem forms -- 
the present stem and the perfect stem -- 
are both required for complete characterization of the paradigm of a regular verb.
To fully decide the paradigm 
we still need a further stem, the supine stem
(\prettyref{sec:three-latin-stem}).

Contextual allomorphs also exist for other morphemes in \prettyref{fig:latin-verb}
\citep[\citepage{11}]{embick2005status},
as can be observed in \prettyref{tbl:latin-finite-verbs}.
The most salient change is that 
an alternating vowel may appear between the root and the aspect suffix (if any),
which is also known as the \concept{thematic vowel}.
It is the residue of the \ac{pie} ablaut
and labels the conjugation class
(\prettyref{sec:verb-morphology.stem.conjugation}),
and is subject to morphophonological contextual alternation
(\prettyref{sec:tense-mood-marking}).
The tense affix is determined by ``tense'' in the sense of traditional Latin grammar, 
i.e. both tense and aspect (\prettyref{tbl:tense-suffix}),
and has dependence on the conjugation class (TODO: ref).
The \category{perfect} tense also has its own personal endings (\prettyref{tbl:personal-ending-perfect}).

\begin{infobox}{More than one thematic vowel?}{thematic-vowel}
    Phenomena similar to the thematic vowel -- vowel alternation with no morphosyntactic significance 
    between two morphemes -- can be observed 
    between the tense suffix and the personal ending 
    (as in \form{am\={a}b\textbf{u}ntur}; \prettyref{sec:verb-morphology.1c}) as well.
    We may go further and say that
    the \form{-imus} ending seen in first person active indicative perfect,
    as opposed to \form{-mus} observed in other cases, 
    contains a thematic vowel \form{i} after the perfect marker \form{-v-}.
    Some generalize the concept of thematic vowel and 
    stipulate a thematic vowel position after \emph{every} morpheme
    \cite{embick2003latin}.
    This is also motivated by evidences from some descendants of Latin 
    \citep{oltra1999notion,oltra2005stress}.
    Whether this is needed for describing Latin however is 
\end{infobox}

\begin{table}[H]
    \centering
    \caption{Examples of Latin finite verbs}
    \label{tbl:latin-finite-verbs}
    \begin{tabular}{lllllll}
    \toprule
    \multirow{3}{*}{verb form} & \multicolumn{3}{c}{stem}                                           & \multicolumn{1}{c}{\multirow{3}{*}{tense and mood}} & \multicolumn{2}{c}{\multirow{2}{*}{personal ending}} \\ \cline{2-4}
                               & \multicolumn{1}{c}{extended root} &                & aspect marker & \multicolumn{1}{c}{}                                & \multicolumn{2}{c}{}                                 \\ \cline{6-7}
                               & extended root                     & thematic vowel &               & \multicolumn{1}{c}{}                                & ``thematic vowel''         & personal ending         \\ \midrule
    \corpus{amō}               & \corpus{am}                       & \corpus{}      & \corpus{}     & \corpus{}                                           & \corpus{}                  & \corpus{ō}              \\
    \corpus{laudāmus}          & \corpus{laud}                     & \corpus{ā}     & \corpus{}     & \corpus{}                                           & \corpus{}                  & \corpus{mus}            \\
    \corpus{olēvimus}          & \corpus{ol}                       & \corpus{ē}     & \corpus{v}    & \corpus{}                                           & \corpus{i}                 & \corpus{mus}            \\
    \corpus{amāveris}          & \corpus{am}                       & \corpus{ā}     & \corpus{v}    & \corpus{eri}                                        & \corpus{}                  & \corpus{s}              \\ \bottomrule
    \end{tabular}    
\end{table}

Below I discuss the subsystems in \prettyref{fig:latin-verb}.


\subsection{\acs{tam} categories}\label{sec:verb-inflection.finite-template.tame}

Two subsystems can be observed in the tense and mood suffix in \prettyref{fig:latin-verb},
which are of corse, \category{tense} and \category{mood}.
Six \category{tense} morphological categories can be recognized in total:
\category{present}, \category{imperfect}, \category{perfect}, \category{pluperfect}, \category{future}, \category{future perfect}.
There are two \category{mood} morphological categories:
\category{indicative} and \category{subjunctive};
sometimes the \category{imperative} is recognized as the third \category{mood}.
These categories can be established based on sole morphological basis. 

We need to note that the morphological marking strategies of \ac{tam}
do not transparently reflect the underlying abstract \ac{tam} features
(as is illustrated in \prettyref{tbl:latin-tense-aspect}).
Following the example in \citet{grimm2021grammar},
in this note, I use small capitals for the names of attested surface \emph{morphological realizations} of \ac{tam},
and I use default fonts for abstract \ac{tam} values.
Some other grammars, like \citet{jacques2021grammar,friesen2017grammar}, 
use initial capitals for the former.

Furthermore, two constructions with \emph{syntactically} different abstract \ac{tam} features
may have comparable \emph{semantics}.
Examples include that if readers of a text know it's about historic events,
using present or past tenses in the text will not influence its meaning,
and that the distinction between simple past and present perfect isn't always considerable.
Sometimes this leads to morphological merger of \ac{tam} values
(\prettyref{tbl:latin-tense-aspect}).
Choosing one of several constructions with almost identical meaning
sometimes follow stylist conventions
(\prettyref{sec:verb.tense-usage}).

\paragraph*{Marking of primary and secondary tenses}
Abstract TAM categories marked by \category{tense} morphology includes
at least primary tense (past, present, future) and
secondary tense (anterior i.e. perfect v.s. simultaneous i.e. plain):
the \category{pluperfect}, the \category{perfect}, and the \category{future perfect}
all have a perfect meaning.

\paragraph*{Point of view aspect and its combinatorics with secondary tense}
It seems the \category{perfect} can also be used to refer to any incident in the past,
while the \category{imperfect} can also describe something happening in the past.
The distinction between the two is likely that point of view aspect (imperfective i.e. progressive v.s. perfective):
the \category{imperfect} has a progressive meaning.

After enumerating possible combinations of the imperfective-perfective distinction and the secondary tense, 
we find that among four possible options,
three are directly morphologically marked: 
plain (simultaneous, perfective), 
imperfective (simultaneous, imperfective), 
perfective (anterior, perfective). 
The three combinations are often known as the three \term{aspects} in Latin linguistics.
In the past paradigm they correspond to
the \category{perfect}, the \category{imperfect}, and the \category{pluperfect}, respectively.

The fourth combination, the combination of the imperfective and the anterior aspectual values, 
i.e. ``progressive perfect'' (\form{have/had been doing} in English),
does not have an independent morphological marker in Latin.
This may originate from the fact that in most situations, 
the perfect secondary tense presumably has a perfective meaning,
as observing an event from the outside (perfect i.e. anterior secondary tense)
usually means the internal makeup of the event does not matter (perfective aspect);
indeed, English \form{have/had been doing}
is mostly used to convey the meaning that the situation being described is still going on
at the reference time,
although this is an implication and not a proper part of the meaning of the perfect imperfective
(e.g. in \form{we've being working on this project for at least two years;
we finished it today, and today, we proudly present it to you.},
the project development was already finished at the speech time).

\paragraph*{The \ac{tam} values of morphological \category{tenses}}
\label{sec:verb.paradigm.tam.tense-combination}
The composition of present, past and future tenses
and plain, imperfect and perfect ``aspects''
gives nine options;
these options are further reduced to six because of several additional mergers.

First, the past simple and the present perfect are identified with each other,
possibly because of obvious semantic closeness.
Also, the imperfective and simple ``aspects'' are fused 
when the primary tense is present or future,
or in other words, the point-of-view imperfective/perfective distinction is not reflected
in the present or future paradigm:
only the spontaneous/anterior (i.e. simple/perfect) values of the secondary tense is morphologically relevant.

The six remaining \category{tense} forms and their \ac{tam} values
are shown in \prettyref{tbl:latin-tense-aspect}.
In the Latin grammatical tradition, 
the term \concept{tense} 
usually refers to these six options,
instead of the past/present/future system.

\begin{table}[H]
    \caption{Latin \category{tenses}}
    \label{tbl:latin-tense-aspect}
    \centering
    \begin{tabular}{@{}cccc@{}}
    \toprule
              & past       & present                  & future                  \\ \midrule
    imperfect & imperfect  & \multirow{2}{*}{present} & \multirow{2}{*}{future} \\
    simple    & perfect    &                          &                         \\
    perfect   & pluperfect & perfect                  & future perfect          \\ \bottomrule
\end{tabular}    
\end{table}

The usage of the six \category{tense} categories have peculiarities
that cannot be completely summarized into \prettyref{tbl:latin-tense-aspect};
these are discussed in \prettyref{sec:verb.tense-usage}.

\paragraph*{The function of \category{mood} markers}
Similar fusion between categories is shown in the category of \category{mood}.%
\footnote{
    \citet{dixon2009basic1} only calls the category of clause type \term{mood}.
    \citet{cgel}, on the other hand, 
    calls syntactic modality mood 
    and uses the term \term{modality} for pure semantics.
    Different linguists use the term \term{mood} and \term{modality} in radically different ways.
    In this note I just focus on the common practice in Latin grammar study.
    In terms of general linguistics, 
    the Latin \term{mood} is a mixture of clause type (the real mood)
    and modality.
}
It's the fusion of morphologically marked clause type 
(declarative and imperative)
and morphologically marked modality.
The verb morphology of interrogative clauses is exactly the same as declarative clauses:
the interrogative clause type is marked by the existence of interrogative \term{pro}-forms.
Thus, there are three moods in finite clauses in Latin:
\acl{indicative}, \acl{subjunctive}, and \acl{imperative}.
The \acl{indicative} is the fusion of 
the declarative/interrogative clause type and the realis modality.
The \acl{subjunctive} mood is the fusion of 
the declarative/interrogative clause type and the irrealis modality.
The \acl{imperative} is basically the imperative clause type:
it doesn't allow modality marking.
Sometimes people say the infinitive is the fourth mood,
though it's a non-finite clause.


\subsection{Agreement}\label{sec:agreement-abs}

Latin is a typical nominative-accusative language,
both morphologically and syntactically.
In finite clauses, 
there is subject-verb agreement:
the number and person of the subject is marked on the main verb.
In the case of periphrastic conjugation,
the features are marked on the copula.

\subsection{Compatability of categories}

There is no \acl{future} tense and \acl{future perfect} tense in subjunctive clauses,
probably for the semantic reason
that the future tense already contains certain sense of modality
(an event predicted to happen),
and thus is not compatible with the \acl{subjunctive} mood.
The \acl{imperative} mood is not compatible with other \ac{tam} markings
except the \acl{present} tense and the \acl{future} tense.
It's still compatible with the voice category,
and allowed persons are 
second person singular/plural with the \acl{present} tense,
and second/third person singular/plural with the \acl{future} tense.
The absence of first person is also probably from semantic origin.

In conclusion, the categories involved in the finite verb paradigm of Latin 
are shown in \prettyref{fig:paradigm-finite-verb}.
Here mood and tense are realized in one morpheme,
and voice, person and number are realized in one morpheme.
The paradigm is realized synthetically in all circumstances 
except in passive voice and perfect tense.
In that case, the verb conjugation is realized like the English passive,
i.e. via a copula and the perfect passive participle.

\begin{figure}[H]
    \centering
    

\tikzset{every picture/.style={line width=0.75pt}} %set default line width to 0.75pt        

\begin{tikzpicture}[x=0.75pt,y=0.75pt,yscale=-0.8,xscale=0.8]
%uncomment if require: \path (0,560); %set diagram left start at 0, and has height of 560

%Straight Lines [id:da4771979505475994] 
\draw [color={rgb, 255:red, 208; green, 2; blue, 27 }  ,draw opacity=1 ][line width=2.25]    (72,476.59) -- (165,476.59) ;
%Shape: Rectangle [id:dp4651809120211663] 
\draw  [draw opacity=0][fill={rgb, 255:red, 208; green, 2; blue, 27 }  ,fill opacity=0.1 ] (72,51.93) -- (165,51.93) -- (165,195.59) -- (72,195.59) -- cycle ;
%Shape: Rectangle [id:dp8674436914111818] 
\draw  [draw opacity=0][fill={rgb, 255:red, 245; green, 166; blue, 35 }  ,fill opacity=0.1 ] (208,51.93) -- (324,51.93) -- (324,195.59) -- (208,195.59) -- cycle ;
%Straight Lines [id:da9097932489630769] 
\draw [color={rgb, 255:red, 245; green, 166; blue, 35 }  ,draw opacity=1 ][line width=2.25]    (210,476.59) -- (324,476.59) ;
%Shape: Rectangle [id:dp6458311390526075] 
\draw  [draw opacity=0][fill={rgb, 255:red, 245; green, 166; blue, 35 }  ,fill opacity=0.1 ] (208,213.93) -- (324,213.93) -- (324,317.59) -- (208,317.59) -- cycle ;
%Shape: Rectangle [id:dp4418485190624626] 
\draw  [draw opacity=0][fill={rgb, 255:red, 208; green, 2; blue, 27 }  ,fill opacity=0.1 ] (73,213.93) -- (166,213.93) -- (166,318.26) -- (73,318.26) -- cycle ;
%Shape: Rectangle [id:dp7155013561532602] 
\draw  [draw opacity=0][fill={rgb, 255:red, 245; green, 166; blue, 35 }  ,fill opacity=0.1 ] (208,341.59) -- (324,341.59) -- (324,373.59) -- (208,373.59) -- cycle ;
%Shape: Rectangle [id:dp5109073549880265] 
\draw  [draw opacity=0][fill={rgb, 255:red, 245; green, 166; blue, 35 }  ,fill opacity=0.1 ] (208,387.59) -- (324,387.59) -- (324,419.59) -- (208,419.59) -- cycle ;
%Shape: Rectangle [id:dp4978185928249894] 
\draw  [draw opacity=0][fill={rgb, 255:red, 208; green, 2; blue, 27 }  ,fill opacity=0.1 ] (73,341.59) -- (166,341.59) -- (166,420.93) -- (73,420.93) -- cycle ;
%Shape: Rectangle [id:dp7974586880324295] 
\draw  [draw opacity=0][fill={rgb, 255:red, 126; green, 211; blue, 33 }  ,fill opacity=0.1 ] (364,52.93) -- (471.33,52.93) -- (471.33,419.59) -- (364,419.59) -- cycle ;
%Straight Lines [id:da40576537799721035] 
\draw [color={rgb, 255:red, 126; green, 211; blue, 33 }  ,draw opacity=1 ][line width=2.25]    (364,476.59) -- (470,476.59) ;
%Shape: Rectangle [id:dp37304865952309374] 
\draw  [draw opacity=0][fill={rgb, 255:red, 80; green, 227; blue, 194 }  ,fill opacity=0.1 ] (515,51.93) -- (631,51.93) -- (631,318.93) -- (515,318.93) -- cycle ;
%Shape: Rectangle [id:dp09424442237743991] 
\draw  [draw opacity=0][fill={rgb, 255:red, 80; green, 227; blue, 194 }  ,fill opacity=0.1 ] (514,341.59) -- (630,341.59) -- (630,373.59) -- (514,373.59) -- cycle ;
%Shape: Rectangle [id:dp7879043441622926] 
\draw  [draw opacity=0][fill={rgb, 255:red, 80; green, 227; blue, 194 }  ,fill opacity=0.1 ] (514,387.59) -- (630,387.59) -- (630,419.59) -- (514,419.59) -- cycle ;
%Straight Lines [id:da6694037096709569] 
\draw [color={rgb, 255:red, 80; green, 227; blue, 194 }  ,draw opacity=1 ][line width=2.25]    (514,476.59) -- (628,476.59) ;
%Shape: Rectangle [id:dp8356914669291959] 
\draw  [draw opacity=0][fill={rgb, 255:red, 74; green, 144; blue, 226 }  ,fill opacity=0.1 ] (675,52.93) -- (782.33,52.93) -- (782.33,419.59) -- (675,419.59) -- cycle ;
%Straight Lines [id:da18180972606397305] 
\draw [color={rgb, 255:red, 74; green, 144; blue, 226 }  ,draw opacity=1 ][line width=2.25]    (676.33,476.59) -- (782.33,476.59) ;

% Text Node
\draw (118.5,123.76) node   [align=left] {indicative};
% Text Node
\draw (119.5,266.09) node   [align=left] {subjunctive};
% Text Node
\draw (266,123.76) node   [align=left] {present/\\imperfect/\\future/\\perfect/\\pluperfect/\\future perfect};
% Text Node
\draw (266,265.76) node   [align=left] {present/\\imperfect/\\perfect/\\pluperfect};
% Text Node
\draw (417.67,236.26) node   [align=left] {active/\\passive};
% Text Node
\draw (573,185.43) node   [align=left] {1/\\2/\\3};
% Text Node
\draw (728.67,236.26) node  [color={rgb, 255:red, 0; green, 0; blue, 0 }  ,opacity=1 ] [align=left] {single/\\plural};
% Text Node
\draw (119.5,381.26) node   [align=left] {imperative};
% Text Node
\draw (266,357.59) node   [align=left] {present};
% Text Node
\draw (266,403.59) node   [align=left] {future};
% Text Node
\draw (572,357.59) node   [align=left] {2};
% Text Node
\draw (572,403.59) node   [align=left] {2/3};
% Text Node
\draw (118.5,479.59) node [anchor=north] [inner sep=0.75pt]  [color={rgb, 255:red, 208; green, 2; blue, 27 }  ,opacity=1 ] [align=left] {mood};
% Text Node
\draw (267,479.59) node [anchor=north] [inner sep=0.75pt]  [color={rgb, 255:red, 245; green, 166; blue, 35 }  ,opacity=1 ] [align=left] {\textcolor[rgb]{0.96,0.65,0.14}{tense}};
% Text Node
\draw (417,479.59) node [anchor=north] [inner sep=0.75pt]  [color={rgb, 255:red, 126; green, 211; blue, 33 }  ,opacity=1 ] [align=left] {voice};
% Text Node
\draw (571,479.59) node [anchor=north] [inner sep=0.75pt]  [color={rgb, 255:red, 80; green, 227; blue, 194 }  ,opacity=1 ] [align=left] {\textcolor[rgb]{0.31,0.89,0.76}{person}};
% Text Node
\draw (729.33,479.59) node [anchor=north] [inner sep=0.75pt]  [color={rgb, 255:red, 74; green, 144; blue, 226 }  ,opacity=1 ] [align=left] {number};


\end{tikzpicture}

    \caption{Categories in the finite paradigm}
    \label{fig:paradigm-finite-verb}
\end{figure}


\subsection{The non-finite paradigm}\label{sec:non-finite-abs}

According to the morphology,
Latin non-finite verb forms can be classified into the infinitives (\prettyref{sec:infinitives})
and the nominal forms (\prettyref{sec:nominal-form}),
the latter having noun-like or adjective-like morphology.
Non-finite verb forms don't agree with the subjects they take,
so there is no number or person category marked on them in the same way as \prettyref{fig:paradigm-finite-verb},
though for nominal verb forms there are number and person categories 
marked in the same way as the nominal morphology.

The infinitives include present active, present passive, perfect active, 
perfect passive, future active, and future passive infinitives.
The latter three are realized periphrastically (\prettyref{fig:stem-to-form}).

The nominal verb forms include 
the \concept{simple active}, 
the \concept{perfect passive} (often just called the perfect participle), 
and the \concept{future active} participles,
the \concept{gerund}, 
the \concept{gerundive} which is also known as the \concept{future passive} participle, 
and two supine forms.
The \concept{first supine} is identical in the form to the singular neutral accusative perfect participle,
without any reference to the number category of any argument it takes.
The \concept{second supine} is identical to the singular neutral ablative or dative past participle,
also with no inflection with respect to the number category of any argument it takes.
The gerund is morphologically the singular neutral of the gerundive
(\prettyref{sec:verb-inflection.non-finite.future-passive}).

I keep the traditional notion that 
the two supines are better recognized as separate morphological inflection forms,
instead of two specific usages of the perfect participle and the past participle:
although the two supines are identical to two other inflection forms in the surface form,
that they don't have number agreement or case inflection 
means they are products of morphological devices that differ from the participle constructions.%
\footnote{
    Similarly, the infinitive 
    (or the ``plain form'', since the infinitive is actually a label of clauses)
    is recognized as a form 
    independent from the present form in English in \citet[\citepage{74}]{cgel}.
}

In Classical Latin, the gerund and participle forms are significantly more noun-like 
than their counterparts in English,
and this also justifies the term \term{nominal form},
because they are not far from prototypically nominalization:
although they are still modified by adverbs,
they are unable to take arguments.
In Ecclesiastical Latin, 
the so-called nominal forms are more verb-like (TODO: ref),
being able to take arguments,
and are therefore no longer ``nominal''.

\subsection{The two periphrastic conjugations}

There are two additional periphrastic constructions 
beside the periphrastic forms in the perfect, passive part of the paradigm.
In the first construction, 
the copula \form{sum} with arbitrary person, number and \category{tense} inflection 
(\emph{including} the perfect \category{tenses})
is placed together with the future active participle,
to express the meaning of \translate{I/you/he/she/it/they are going to do something}, 
TODO: a inchoative feeling?
This construction is always active in voice.
In the second construction the future active participle is replaced by the future passive participle;
the meaning is \translate{something should be done}.

The present active participle is almost never used in periphrastic conjugation.
It can be seen that in periphrastic conjugation,
the roles of future active participle and future passive participle 
and the role of passive perfect participle 
are not comparable:
the latter dictates the secondary tense, i.e. it's perfect, 
as well as the voice;
the former dictate voice and a ``lower'' aspect/modality category 
(inchoative or obligation),
but not the secondary tense;
and in particular, the term \term{future} is kind of (although not really) misleading:
the two future participles carry aspect and modality values with them, 
which are semantically related to the concept of future; 
but the future primary tense is not involved.

\section{Formation of stems}\label{sec:verb-inflection.stem}

\subsection{The three verb stems}\label{sec:three-latin-stem}

Latin shows stem alternation that is not completely predictable.
All verb forms can be obtained by three stems \citep[\citesec{164}]{allen1903allen},
if the verb is regular:
\begin{itemize}
    \item The \concept{present stem}, which, after attached with proper endings, forms
    \begin{itemize}
        \item The \acl{present}, \acl{imperfect}, and future forms, 
        regardless of whether they are indicative or subjunctive,
        active or passive. (There is no future or future perfect subjunctive).
        \item All the imperatives.
        \item The present infinitives, active and passive.
        \item The present participle, the gerundive, and the gerund.
    \end{itemize}
    \item The \concept{perfect stem}, which, after attached with proper endings, forms 
    \begin{itemize}
        \item The perfect, pluperfect, and future perfect active, indicative or subjunctive.
        Again, there is no future or future perfect subjunctive.
        Note that the passives are \emph{not} formed by the perfect stem.
        \item The perfect active infinitive. 
        (Or the perfective infinitive active, since infinitive is considered as a mood by some people.)
    \end{itemize}
    Note that the perfect passive participle is \emph{not} obtained from the perfect stem.
    \item The \concept{supine stem}, 
    which, after attached with proper endings or used together with proper forms of \form{sum},
    forms 
    \begin{itemize}
        \item The perfect passive participle, which, by being used with proper forms of \form{sum}, forms
        \begin{itemize}
            \item The perfect, pluperfect, and future perfect passive forms, indicative or subjunctive.
            Again, there is no future or future perfect subjunctive.
            This is periphrastic conjugation: it is done by using proper forms of \form{sum}
            with the perfect passive participle.
            \item The perfect infinitive passive.
        \end{itemize}
        \item The future active participle, which, used together with \form{esse},
        makes the future active infinitive.
        \item The future passive infinitive, by being used together with \form{īrī}.
    \end{itemize}
\end{itemize}
This process is summarized in \prettyref{fig:stem-to-form}.

In a dictionary, 
typically the stems are not directly given 
-- which are given are representative verb forms,
from which the stems and the conjugation class can be inferred.
The reasons are the follows.
First, for fluent users,
recording actually attested word forms is easier
compared with the morpheme-based ``anatomized'' approach.
Second, Latin has four conjugation types,
and hence the three stems themselves aren't sufficient to decide how to conjugate the verb:
more information is needed, 
and by storing already conjugated verb forms,
the conjugation class can be decided by observing the endings.
What are stored are the following \concept{principal forms},
from which the three stems and the conjugation class can be solved out
\citep[\citesec{172}]{allen1903allen}:
\begin{enumerate}
    \item \emph{The first-person present active indicative}: formed from the present stem.
    \item \emph{The present infinitive}: formed from the present stem. 
    By observing its ending, the conjugation class can be decided,
    and by comparing with the first principal form, 
    the present stem is obtained.
    \item \emph{The first-person perfect active indicative}: showing the perfect stem.
    \item \emph{The neutral accusative past participle}, i.e. the form of supine: showing the supine stem.
\end{enumerate}
The ways to obtain the stems from the principal forms are:
\begin{itemize}
    \item \emph{The present stem} can be found by dropping \form{-re} in the 
    \category{present infinitive}
    \citep[\citesec{175}]{allen1903allen}.
    \item \emph{The perfect stem} can be found from the third principal part:
    just remove \form{-ī}.
    \item \emph{The supine stem} can be found by dropping \form{-um} in the supine
    i.e. the fourth principal form
    \citep[\citesec{178}]{allen1903allen}.
\end{itemize}

Note that in Medieval Latin, often,
instead of \form{iri} plus the first supine,
\form{fore} plus the perfect participle is used to form the future passive infinitive.
TODO: find a reference https://www.nationalarchives.gov.uk/latin/stage-2-latin/lessons/lesson-24-infinitives-accusative-and-infinitive-clause/

\begin{sidewaysfigure}
    \centering
    {\small 

\tikzset{every picture/.style={line width=0.3pt}} %set default line width to 0.75pt        

\begin{tikzpicture}[x=0.75pt,y=0.75pt,yscale=-0.8,xscale=0.8]
%uncomment if require: \path (0,697); %set diagram left start at 0, and has height of 697

%Curve Lines [id:da8600925548352094] 
\draw [color={rgb, 255:red, 208; green, 2; blue, 27 }  ,draw opacity=1 ]   (289.01,269.33) .. controls (329.01,239.33) and (422.01,193.33) .. (676.01,187.33) ;
\draw [shift={(676.01,187.33)}, rotate = 178.65] [fill={rgb, 255:red, 208; green, 2; blue, 27 }  ,fill opacity=1 ][line width=0.08]  [draw opacity=0] (12,-3) -- (0,0) -- (12,3) -- cycle    ;
%Curve Lines [id:da7478415205524331] 
\draw [color={rgb, 255:red, 208; green, 2; blue, 27 }  ,draw opacity=1 ]   (275.01,266.33) .. controls (256.2,201.98) and (244.25,196.43) .. (205.2,166.25) ;
\draw [shift={(204.01,165.33)}, rotate = 37.78] [fill={rgb, 255:red, 208; green, 2; blue, 27 }  ,fill opacity=1 ][line width=0.08]  [draw opacity=0] (12,-3) -- (0,0) -- (12,3) -- cycle    ;
%Curve Lines [id:da22329313492102099] 
\draw [color={rgb, 255:red, 208; green, 2; blue, 27 }  ,draw opacity=1 ]   (241.01,279.33) .. controls (203.2,213.66) and (164.4,199.47) .. (101.95,158.94) ;
\draw [shift={(101.01,158.33)}, rotate = 33.06] [fill={rgb, 255:red, 208; green, 2; blue, 27 }  ,fill opacity=1 ][line width=0.08]  [draw opacity=0] (12,-3) -- (0,0) -- (12,3) -- cycle    ;
%Curve Lines [id:da9720425555739067] 
\draw [color={rgb, 255:red, 208; green, 2; blue, 27 }  ,draw opacity=1 ]   (250.01,319.22) .. controls (250.01,411.38) and (358.91,534.94) .. (466.39,595.28) ;
\draw [shift={(468.01,596.18)}, rotate = 209.05] [fill={rgb, 255:red, 208; green, 2; blue, 27 }  ,fill opacity=1 ][line width=0.08]  [draw opacity=0] (12,-3) -- (0,0) -- (12,3) -- cycle    ;
%Curve Lines [id:da9116874614549022] 
\draw [color={rgb, 255:red, 208; green, 2; blue, 27 }  ,draw opacity=1 ]   (265.01,312.33) .. controls (276.01,374.96) and (344.01,487.96) .. (487.01,511.96) ;
\draw [shift={(487.01,511.96)}, rotate = 189.53] [fill={rgb, 255:red, 208; green, 2; blue, 27 }  ,fill opacity=1 ][line width=0.08]  [draw opacity=0] (12,-3) -- (0,0) -- (12,3) -- cycle    ;
%Curve Lines [id:da9399420006349646] 
\draw [color={rgb, 255:red, 208; green, 2; blue, 27 }  ,draw opacity=1 ]   (277.01,310.33) .. controls (320.79,382.6) and (345.76,424.54) .. (456.34,434.81) ;
\draw [shift={(458.01,434.96)}, rotate = 185.1] [fill={rgb, 255:red, 208; green, 2; blue, 27 }  ,fill opacity=1 ][line width=0.08]  [draw opacity=0] (12,-3) -- (0,0) -- (12,3) -- cycle    ;
%Curve Lines [id:da863359914759456] 
\draw [color={rgb, 255:red, 248; green, 231; blue, 28 }  ,draw opacity=1 ]   (404.01,268.33) .. controls (396.09,243.58) and (393.07,211.97) .. (403.69,164.76) ;
\draw [shift={(404.01,163.33)}, rotate = 102.91] [fill={rgb, 255:red, 248; green, 231; blue, 28 }  ,fill opacity=1 ][line width=0.08]  [draw opacity=0] (12,-3) -- (0,0) -- (12,3) -- cycle    ;
%Curve Lines [id:da44942461080998] 
\draw [color={rgb, 255:red, 248; green, 231; blue, 28 }  ,draw opacity=1 ]   (418.01,269.33) .. controls (457.81,239.48) and (590.67,220.74) .. (688.54,241.24) ;
\draw [shift={(690.01,241.55)}, rotate = 192.09] [fill={rgb, 255:red, 248; green, 231; blue, 28 }  ,fill opacity=1 ][line width=0.08]  [draw opacity=0] (12,-3) -- (0,0) -- (12,3) -- cycle    ;
%Curve Lines [id:da6857007576480554] 
\draw [color={rgb, 255:red, 245; green, 166; blue, 35 }  ,draw opacity=1 ]   (543.01,333) .. controls (619.63,368.19) and (641.79,399.53) .. (690.28,478.65) ;
\draw [shift={(691.01,479.85)}, rotate = 238.51] [fill={rgb, 255:red, 245; green, 166; blue, 35 }  ,fill opacity=1 ][line width=0.08]  [draw opacity=0] (12,-3) -- (0,0) -- (12,3) -- cycle    ;
%Curve Lines [id:da10519824034130609] 
\draw [color={rgb, 255:red, 245; green, 166; blue, 35 }  ,draw opacity=1 ]   (548.01,313.33) .. controls (610.7,314.32) and (660.51,321.26) .. (718.14,390.28) ;
\draw [shift={(719.01,391.33)}, rotate = 230.36] [fill={rgb, 255:red, 245; green, 166; blue, 35 }  ,fill opacity=1 ][line width=0.08]  [draw opacity=0] (12,-3) -- (0,0) -- (12,3) -- cycle    ;
%Curve Lines [id:da05946917136382068] 
\draw [color={rgb, 255:red, 245; green, 166; blue, 35 }  ,draw opacity=1 ]   (764.01,427.33) .. controls (931.01,427.33) and (930.01,58.33) .. (587.01,121.33) ;
\draw [shift={(587.01,121.33)}, rotate = 349.59] [fill={rgb, 255:red, 245; green, 166; blue, 35 }  ,fill opacity=1 ][line width=0.08]  [draw opacity=0] (12,-3) -- (0,0) -- (12,3) -- cycle    ;
%Curve Lines [id:da40444377743555227] 
\draw [color={rgb, 255:red, 245; green, 166; blue, 35 }  ,draw opacity=1 ]   (760.01,418.33) .. controls (782.67,400.12) and (788.83,381.88) .. (787.1,343.11) ;
\draw [shift={(787.01,341.33)}, rotate = 87.14] [fill={rgb, 255:red, 245; green, 166; blue, 35 }  ,fill opacity=1 ][line width=0.08]  [draw opacity=0] (12,-3) -- (0,0) -- (12,3) -- cycle    ;
%Curve Lines [id:da6236126078150841] 
\draw [color={rgb, 255:red, 245; green, 166; blue, 35 }  ,draw opacity=1 ]   (776.01,515.33) .. controls (822.31,499.24) and (848.23,480.56) .. (870.97,451.34) ;
\draw [shift={(872.01,450)}, rotate = 127.48] [fill={rgb, 255:red, 245; green, 166; blue, 35 }  ,fill opacity=1 ][line width=0.08]  [draw opacity=0] (12,-3) -- (0,0) -- (12,3) -- cycle    ;
%Curve Lines [id:da6624205121022633] 
\draw [color={rgb, 255:red, 245; green, 166; blue, 35 }  ,draw opacity=1 ]   (526.01,337) .. controls (548.9,359.55) and (593.56,477.99) .. (600.9,583.59) ;
\draw [shift={(601.01,585.18)}, rotate = 266.22] [fill={rgb, 255:red, 245; green, 166; blue, 35 }  ,fill opacity=1 ][line width=0.08]  [draw opacity=0] (12,-3) -- (0,0) -- (12,3) -- cycle    ;
%Curve Lines [id:da8468021797759513] 
\draw [color={rgb, 255:red, 245; green, 166; blue, 35 }  ,draw opacity=1 ]   (630.01,594.18) .. controls (720.56,591.2) and (773.48,577.14) .. (847.89,560.25) ;
\draw [shift={(849.01,560)}, rotate = 167.23] [fill={rgb, 255:red, 245; green, 166; blue, 35 }  ,fill opacity=1 ][line width=0.08]  [draw opacity=0] (12,-3) -- (0,0) -- (12,3) -- cycle    ;
%Shape: Ellipse [id:dp3626499664001579] 
\draw  [color={rgb, 255:red, 74; green, 144; blue, 226 }  ,draw opacity=1 ][fill={rgb, 255:red, 74; green, 144; blue, 226 }  ,fill opacity=0.1 ] (39,126.52) .. controls (39,83.15) and (167.94,48) .. (327.01,48) .. controls (486.07,48) and (615.01,83.15) .. (615.01,126.52) .. controls (615.01,169.88) and (486.07,205.03) .. (327.01,205.03) .. controls (167.94,205.03) and (39,169.88) .. (39,126.52) -- cycle ;
%Curve Lines [id:da19915198987505445] 
\draw [color={rgb, 255:red, 80; green, 227; blue, 194 }  ,draw opacity=1 ][fill={rgb, 255:red, 80; green, 227; blue, 194 }  ,fill opacity=0.2 ]   (435.01,413.18) .. controls (475.01,383.18) and (680.01,340.18) .. (752.01,382.18) .. controls (824.01,424.18) and (801.01,553.18) .. (717.01,571.18) .. controls (633.01,589.18) and (657.01,503.18) .. (615.01,482.18) .. controls (573.01,461.18) and (401.02,474.22) .. (435.01,413.18) -- cycle ;
%Shape: Ellipse [id:dp7283210175576866] 
\draw  [color={rgb, 255:red, 126; green, 211; blue, 33 }  ,draw opacity=1 ][fill={rgb, 255:red, 126; green, 211; blue, 33 }  ,fill opacity=0.1 ] (423,565.16) .. controls (423,524.84) and (473.15,492.14) .. (535.01,492.14) .. controls (596.86,492.14) and (647.01,524.84) .. (647.01,565.16) .. controls (647.01,605.49) and (596.86,638.18) .. (535.01,638.18) .. controls (473.15,638.18) and (423,605.49) .. (423,565.16) -- cycle ;
%Curve Lines [id:da8675081082501883] 
\draw [color={rgb, 255:red, 80; green, 227; blue, 194 }  ,draw opacity=1 ][fill={rgb, 255:red, 80; green, 227; blue, 194 }  ,fill opacity=0.1 ]   (363.01,378.55) .. controls (387.01,323.55) and (691.01,321.55) .. (763.01,363.55) .. controls (835.01,405.55) and (805.01,594.55) .. (721.01,612.55) .. controls (637.01,630.55) and (649.01,514.55) .. (540.01,525.55) .. controls (431.01,536.55) and (329.02,439.59) .. (363.01,378.55) -- cycle ;
%Shape: Polygon Curved [id:ds050164225991979006] 
\draw  [color={rgb, 255:red, 184; green, 233; blue, 134 }  ,draw opacity=1 ][fill={rgb, 255:red, 184; green, 233; blue, 134 }  ,fill opacity=0.1 ] (683.01,164.55) .. controls (770.01,100.55) and (871.01,192.55) .. (904.01,306.55) .. controls (937.01,420.55) and (940.76,404.18) .. (965.01,472.55) .. controls (989.26,540.93) and (945.01,634.55) .. (872.01,627.55) .. controls (799.01,620.55) and (851.01,517.55) .. (838.01,459.55) .. controls (825.01,401.55) and (798.01,367.55) .. (738.01,329.55) .. controls (678.01,291.55) and (596.01,228.55) .. (683.01,164.55) -- cycle ;

% Text Node
\draw (489,83) node [anchor=north west][inner sep=0.75pt]   [align=left] {perfect,\\pluperfect,\\future perfect\\passive};
% Text Node
\draw (359,74) node [anchor=north west][inner sep=0.75pt]   [align=left] {perfect,\\pluperfect,\\future perfect\\active};
% Text Node
\draw (175,100) node [anchor=north west][inner sep=0.75pt]   [align=left] {present,\\imperfect,\\future};
% Text Node
\draw (385,269.07) node [anchor=north west][inner sep=0.75pt]  [color={rgb, 255:red, 0; green, 0; blue, 0 }  ,opacity=1 ] [align=left] {perfect\\stem};
% Text Node
\draw (248,268.07) node [anchor=north west][inner sep=0.75pt]  [color={rgb, 255:red, 0; green, 0; blue, 0 }  ,opacity=1 ] [align=left] {present\\stem};
% Text Node
\draw (470,587.07) node [anchor=north west][inner sep=0.75pt]   [align=left] {gerund};
% Text Node
\draw (496,501.07) node [anchor=north west][inner sep=0.75pt]   [align=left] {gerundive};
% Text Node
\draw (59,130) node [anchor=north west][inner sep=0.75pt]   [align=left] {imperative};
% Text Node
\draw (688,166) node [anchor=north west][inner sep=0.75pt]   [align=left] {present\\infinitives};
% Text Node
\draw (467,412) node [anchor=north west][inner sep=0.75pt]   [align=left] {present\\participle};
% Text Node
\draw (501,292.07) node [anchor=north west][inner sep=0.75pt]   [align=left] {supine\\stem};
% Text Node
\draw (696,218) node [anchor=north west][inner sep=0.75pt]   [align=left] {perfect\\active\\infinitive};
% Text Node
\draw (579,586.07) node [anchor=north west][inner sep=0.75pt]   [align=left] {supine};
% Text Node
\draw (697,395) node [anchor=north west][inner sep=0.75pt]   [align=left] {perfect\\passive\\participle};
% Text Node
\draw (765,278) node [anchor=north west][inner sep=0.75pt]   [align=left] {perfect\\passive\\infinitive};
% Text Node
\draw (695,482) node [anchor=north west][inner sep=0.75pt]   [align=left] {future\\active\\participle};
% Text Node
\draw (872,382) node [anchor=north west][inner sep=0.75pt]   [align=left] {future\\active\\infinitive};
% Text Node
\draw (864,521) node [anchor=north west][inner sep=0.75pt]   [align=left] {future\\passive\\infinitive};
% Text Node
\draw (200,60) node [anchor=north west][inner sep=0.75pt]  [color={rgb, 255:red, 74; green, 144; blue, 226 }  ,opacity=1 ] [align=left] {finite forms};
% Text Node
\draw (594,405.14) node [anchor=north west][inner sep=0.75pt]  [color={rgb, 255:red, 80; green, 227; blue, 194 }  ,opacity=1 ] [align=left] {particle\\in narrow\\sense};
% Text Node
\draw (519,532.51) node [anchor=north west][inner sep=0.75pt]  [color={rgb, 255:red, 126; green, 211; blue, 33 }  ,opacity=1 ] [align=left] {"nominal"\\nonfinite\\forms};
% Text Node
\draw (389,356.14) node [anchor=north west][inner sep=0.75pt]  [color={rgb, 255:red, 80; green, 227; blue, 194 }  ,opacity=1 ] [align=left] {particle in broad sense};
% Text Node
\draw (775,228) node [anchor=north west][inner sep=0.75pt]  [color={rgb, 255:red, 184; green, 233; blue, 134 }  ,opacity=1 ] [align=left] {infinitives};


\end{tikzpicture}
}
    \caption{How to get all conjugation forms from the three stems}
    \label{fig:stem-to-form}
\end{sidewaysfigure}

\subsection{Formation of the present stem}


\subsection{Formation of the perfect stem}\label{sec:verb-inflection.stem.perfect}

The \category{perfect} stem is not completely predictable from the \category{present} stem. 
Frequently observed alternations are listed below.

\paragraph*{\form{-v-} suffix} 
The etymological source of this suffix is not clear.

\paragraph*{\form{-s-} suffix}

\paragraph*{Prolonged vowel}

\subsection{Conjugation classes}\label{sec:verb-morphology.stem.conjugation}

Depending on the way realization of the paradigm for a verb,
Latin verbs are traditionally divided into 
four conjugations classes
according to the thematic vowel of the stem:
if the stem ends in \form{a}, 
it's a first conjugation verb (\prettyref{sec:verb-morphology.1c});
if it ends in \form{e} 
then we have a second conjugation verb (\prettyref{sec:verb-morphology.2c});
TODO.
The conjugation classes however can't be reduced to 
the thematic vowel, 
since there are some contextual allomorphs that can't be explained synchronically
in this way
(\prettyref{sec:tense-mood-marking}, \prettyref{tbl:tense-suffix}).
There are a handful of irregular verbs (\prettyref{sec:irregular-verbs})
that can't be inflected using rules pertaining to the four regular conjugations.

Another aspect of the inflectional behavior of the verb
is whether it's deponent;
this also has some implications on 
the argument structure (\prettyref{sec:deponent-verbs}).

\section{The finite paradigm}\label{sec:finite-paradigm}

Due to morphophonological rules, 
morpheme division inevitably involves controversies.
This section follows the analysis in \citet[\citechap{14}]{oniga2014latin}.

\subsection{Marking of tense and mood}\label{sec:tense-mood-marking}

The contextual alternation of the tense and mood marker is listed in
\prettyref{tbl:tense-suffix}.
They are subject to phonological rules 
and have allomorphs in different conjugation classes. 
A long vowel after the suffix,
if any, is shortened before \form{-m}, \form{-r}, \form{-t}, and \form{-nt}
due to vowel shortening in the final syllable before a consonant,
and also before \form{-ntur} due to Osthoff's Law 
(\prettyref{sec:phonology.rule.shortening}; 
\citealt[\citepage{130}]{oniga2014latin}).
The present indicative doesn't add a suffix after the stem, 
so the thematic vowel at the end of the stem 
is directly exposed to the personal ending,
and vowel changes in \prettyref{sec:phonology.morphological} happen.

The perfect and pluperfect subjunctive suffixes 
are only used for the active voice.
For the passive voice,
periphrastic conjugation with the perfect passive participle is used.

\begin{table}[H]
    \caption{Tense and mood suffixes}
    \label{tbl:tense-suffix}
    \centering
    \begin{tabular}{lll}
        \toprule
        Suffix & Mood and tense & Note \\
        \midrule
        $\emptyset$         & Present indicative        & With change on thematic vowel\\
        \form{-b\={a}-}, \form{-eb\={a}-}    & Imperfect indicative      & Possibly shortened \\
        \form{-be-}         & Future indicative         & For 1c, 2c; 
        Allomorphs: \form{-b-}, \form{-bi-}, \form{-bu-} \\
        \midrule
        $\emptyset$         & Perfect indicative        & With its own personal endings \\ 
        \form{-er\={a}-}    & Pluperfect indicative     & Possibly shortened \\
        \form{-eri-}        & Future perfect indicative & Allomorph: \form{-er-} (\category{1sg}) \\ 
        \midrule
        \form{-\={e}-}      & Present subjunctive       & Allomorphs: \form{-a-}, \form{-\={a}-},
        \form{-e-} \\ 
        \form{-r\={e}-}     & Imperfect subjunctive     & Possibly shortened \\
        \midrule
        \form{-eri-}        & Perfect subjunctive       & For active only \\
        \form{-iss\={e}-}   & Pluperfect subjunctive    & For active only; possibly shortened  \\ \bottomrule
        \end{tabular}
\end{table}

\begin{infobox}{Shortening or lengthening?}{shorten-or-prolong-ba}
    If we just restrict ourselves to the verbal paradigm, 
    it may be attempting as well to consider \form{-ba-} as 
    the indicative imperfect suffix,
    since \form{-b\={a}-} does not outnumber it.
    But prolonging is rare in phonology:
    shortening, due to physiological motion control factors
    (some sound sequences are easier to pronounce), 
    is more frequent.
    The same line of argumentation can be applied to justify the status of \form{-r\={e}-}
    as the somehow canonical subjunctive imperfect suffix.
\end{infobox}

Perfect indicative active verb forms have their own set of personal endings
(\prettyref{tbl:personal-ending-perfect}).
The plural part of the personal endings still contains 
the usual \form{-mus}, \form{-tis}, \form{-nt} endings, 
and thus alternatively, we may analyze the verb endings for the \category{perfect} tense 
as the follows: 
the tense suffix is \form{-ī-}
for first-person singular, third-person singular and first-person plural,
\form{-is-} for second-person singular and second-person plural,
and \form{-\={e}ru-} for third-person plural;
the first person singular ending is empty,
and the second person singular ending is \form{-tī},
and the rest of personal endings are the same as those of other tenses.

\begin{table}[H]
    \centering
    \caption{Personal endings for the \category{perfect}}
    \label{tbl:personal-ending-perfect}
    \begin{tabular}{ll}
        \toprule
        Suffix & Person and number \\
        \midrule
        \form{-ī    } & \category{1sg} \\
        \form{-istī } & \category{2sg} \\
        \form{-it   } & \category{3sg} \\
        \form{-imus } & \category{1pl} \\
        \form{-istis} & \category{2pl} \\
        \form{-\={e}runt} & \category{3pl} \\ 
        \bottomrule
    \end{tabular}
\end{table}

\begin{itemize}
    \item The indicative:
    \begin{itemize}
        \item Future: 
        \begin{itemize}
            \item For first and second conjugation verbs, 
            the tense-mood morpheme is \form{-bi-}, except for 
            first-person singular (which is \form{-b-})
            and third-person plural (which is \form{-bu-}).
            \item For third and fourth conjugation verbs, change stem-final stem.
        \end{itemize}
        \item Future perfect: 
        \begin{itemize}
            \item \form{-eri-} for all cases except first person singular.
            \item \form{-er-} for first-person singular.
        \end{itemize}
    \end{itemize}
    \item The subjunctive:
    \begin{itemize}
        \item Present: no suffixation, but there is regular change on the stem-final vowel:
        \begin{itemize}
            \item For first conjugation verbs, .
            \item For second conjugation verbs, \form{} $\to$ \form{e\={a}-}.
            \item For third conjugation verbs, $\to$ \form{\={a}-}.
            \item For fourth conjugation verbs, $\to$ \form{i\={a}-}.
        \end{itemize}
        These alternations apply for both active and passive verbs,
        so they have nothing to do with polarity, and this is why I put them in this section.
        \item Imperfect: \form{-r\={e}-}, possibly shortened.
    \end{itemize}
\end{itemize}

\subsection{The personal ending}

Possible personal endings are listed in \prettyref{tbl:personal-ending}.
In the active voice, 
\form{-\={o}} is used with present indicative, 
future indicative (first and second conjugations only), 
future perfect indicative,
and \form{-m} is used with imperfect indicative, 
future indicative (third and fourth conjugations only),
pluperfect indicative,
and the subjunctive mood regardless of tense.
The \category{perfect} tense has its own personal endings:
in the alternative analysis outlined in the last section,
the first person perfect indicative ends with \form{-ī},
and the second person perfect indicative ends with \form{-tī}.

The ending \form{-re} is alternative form of second-person singular compatible 
with all non-periphrastic tenses and moods.
If this personal ending is used, then the tense and mood marking is none.
Note that the resulting verb form is the same as the infinitive principal part
(\prettyref{sec:three-latin-stem}).

The \form{-or} version of the passive first person ending 
is seen in present indicative and future indicative
in the first and second conjugations;
the latter has the deep form \form{-be-or},
then then is realized as \form{-bor} because of vowel deletion 
(\prettyref{sec:phonology.rule.deletion}).

\begin{table}[H]
    \caption{Personal endings}
    \label{tbl:personal-ending}
    \centering
    \begin{tabular}{lll}
        \toprule
        Suffix & Person and voice & Note \\
        \midrule
        \form{-\={o}}, \form{-m} & active, \category{1sg} & Depend on tense \\ 
        \form{-s}                & active, \category{2sg} & \\ 
        \form{-t}                & active, \category{3sg} & \\ 
        \form{-mus}              & active, \category{1pl} & \\
        \form{-tis}              & active, \category{2pl} & \\
        \form{-nt}               & active, \category{3pl} & \\ \midrule 
        \form{-r}, \form{-or}    & passive, \category{1sg} & \form{-or} for present and future indicative only \\ 
        \form{-ris}              & passive, \category{2sg} & Allomorph: \form{-re} (without tense suffix) \\
        \form{-tur}              & passive, \category{3sg} & Non-perfect tenses \\
        \form{-mur}              & passive, \category{1pl} & Non-perfect tenses \\
        \form{-minī}             & passive, \category{2pl} & Non-perfect tenses \\
        \form{-ntur}             & passive, \category{3pl} & Non-perfect tenses \\ \bottomrule
    \end{tabular}
\end{table}

\begin{itemize}
    \item The passive:
    \begin{itemize}
        \item First-person singular: 
        \begin{itemize}
            \item \form{-r}: compatible with all tenses and moods, except the present indicative.
            \item \form{-or}: present indicative.
            Also, note that the future indicative (first and second conjugations only) ending is \form{-bor},
            which may be analyzed as .
        \end{itemize}
        \item Second-person singular:
        \begin{itemize}
            \item \form{-ris}: compatible with all non-periphrastic tenses and moods.
            \item
        \end{itemize}
    \end{itemize}
\end{itemize}

\begin{infobox}{Parsing a finite verb}{finite-parsing}
    Follow these steps to parse a finite verb:
    \begin{itemize}
        \item First see whether the aspect is perfect (by looking at the stem part)
            and the personal ending.
        \item Then TODO 
    \end{itemize}
\end{infobox}

\subsection{The first conjugation}\label{sec:verb-morphology.1c}

Attested thematic vowel alternations include:
\begin{itemize}
    \item In first person singular present forms, 
    the final \form{\={a}} is dropped, 
    because the personal ending is \form{-\={o}} or \form{-or}
    and the thematic vowel is subject to vowel deletion.
    \item In active subjunctive  \form{\={a}-} $\to$ \form{\={e}-}.
\end{itemize}

\subsection{The second conjugation}\label{sec:verb-morphology.2c}

Attested thematic vowel alternations include:
\begin{itemize}
    \item In active indicative present forms, \form{\={e}} $\to$ \form{e}.
\end{itemize}

\subsection{The third conjugation}

The third conjugation has two subclasses:
the thematic vowel-less case, 
and the \form{i}-thematic case.

\subsection{Periphrastic conjugations}

An auxiliary verb construction is a structure 
that contains one or more auxiliaries apart from the main verb, 
and yet is mono-clausal and is not a complement clause construction,
and the auxiliaries are realizations of the verbal system 
surrounding the main verb, and not independent lexical verbs.
Auxiliary verb constructions realizing grammatical categories 
that are usually realized by inflectional endings in a paradigm
should be seen as a part of that paradigm,
and therefore are known as periphrastic conjugations.

\section{Uses of tenses and moods}\label{sec:verb.tense-usage}

\subsection{The \category{present}}

The

\section{Non-finite forms}

\subsection{The infinitives}\label{sec:infinitives}

\subsection{The gerund and participles}\label{sec:nominal-form}

\subsubsection{The gerund}\label{sec:gerund-morphology}

The gerund is morphologically a neutral singular noun.
The stem is formed by 
adding \form{-nd-} to the present stem;
in other words, 
the accusative form of the gerund of the verb 
is obtained by removing the final \form{s} 
of the present active participle 
and adding \form{dum}.
Note that the nominative case is missing -- 
when a non-finite clause is required in the subject position,
it's always an infinitive.

\subsubsection{The present active participle}

The present active participle (i.e. the present participle) 
is morphologically a third declension adjective (TODO: gender).
The stem of the present active participle 
is obtained by adding \form{-nt} to the present stem.
Equivalently, the nominative singular form -- the citation form --
is obtained by replace the \form{-re} ending of the present active infinitive by \form{-ns}
(or in other words, add \form{-ns} to the present stem).

\subsubsection{The perfect passive participle}

The perfect passive participle (i.e. the perfect participle or the past participle)
can be found by declining the neutral accusative past participle, 
i.e. the fourth principal part.

\subsubsection{The future active participle}

To get the future active participle (i.e. the future participle), add \form{-turus} to the supine stem.

\subsubsection{The future passive participle}\label{sec:verb-inflection.non-finite.future-passive}

The future passive participle, 
or in other words the gerundive,
can be obtained by declining the gerund 
as if it's an adjective (TODO: declension class);
in other words the gerund is morphologically the singular neutral of the gerundive.

\section{Deponent verbs}\label{sec:deponent-verbs}

\concept{Deponent verbs} are verbs that are in the passive voice realizationally,
either with the corresponding morphological marking (\ref{ex:verb-morphology.deponnet.2})
or periphrastic marking (\ref{ex:verb-morphology.deponent.1}),
but still have active meaning.
An observed tendency is that 
so-called ``unaccusative'' verbs
seem to be deponent verbs in Latin,
which means the passive voice essentially is the 
``non-agentive'' voice (\prettyref{sec:verb-phrase.arguments.compatibility}).

\begin{exe}
    \ex\label{ex:verb-morphology.deponnet.2} \gll 
    [Confiteor] unum baptisma in remissionem peccatorum \dots \\
    confess-\category{ind}.\category{pres}.\category{pass}.\category{1sg}
    one-\category{sg.acc} baptism-\category{sg.acc}
    into/towards/about remission-\category{sg.acc}
    sin-\category{pl.gen} \\
    \glt{\translate{I confess one baptism for the remission of sins \dots} (Nicene Creed)}
    
    \ex\label{ex:verb-morphology.deponent.1} \gll \dots {} qui [locutus est] per prophetas \dots \\
    {} who speek.\category{aprt}-\category{sg.nom} 
    \category{be}.\category{ind}.\category{pres}.\category{act}.\category{3sg}
    through prophet-\category{pl.acc} \\
    \glt{\translate{\dots who has spoken through the prophets \dots} (Nicene Creed)}
\end{exe}

\section{Irregular verbs}\label{sec:irregular-verbs}

\subsection{The verb \form{sum}}\label{sec:sum-morphology}

\subsubsection{Overview}

The verb \form{sum} has lots of uses in Latin grammar (\prettyref{sec:sum}),
and its inflection is (unfortunately but expectedly) highly irregular.
It's also defective: 
it has no passive forms, either finite or nonfinite.
The principal parts (\prettyref{sec:three-latin-stem}) are 
\form{sum}, \form{esse}, \form{fuī}, 
with the supine form being absent -- usually replaced by the future active participle \form{futūrus}.

From the principal parts, 
we find the perfect stem is \form{fu-}, 
and the supine stem -- if we insist on defining it -- 
is the same, 
although the perfect passive participle is absent and so is the supine,
and therefore the supine stem only appears in the future active participle.

The present stem is not well-defined:
the second principal form \form{esse}
doesn't have the regular infinitive ending \form{-re},
though we can roughly recognize something like \form{es-} or \form{e-};
the first principal form \form{sum} gives \form{su-} or \form{s-}.
The two stems appear in the finite paradigm in an unpredictable manner, 
also with irregular though still recognizable endings.
Besides \form{s-} and \form{es-},
there is also \form{fo-} seen in one variant of the future active infinitive
(\prettyref{sec:verb-inflection.irregular.sum.nonfinite}),
which also appears in variants in
the subjunctive active imperfect part of the finite paradigm.

\subsubsection{The nonfinite paradigm}\label{sec:verb-inflection.irregular.sum.nonfinite}

The only nominal form is the future active participle \form{futūrus}.
The three active infinitives forms are all attested.
The present active infinitive is \form{esse}.
The perfect active infinitive is \form{fuisse}, 
regularly formed by the perfect stem \form{fu-}.

The future active infinitive can be regularly formed by adding \form{esse} 
to the future active participle,
and therefore is \form{futūrum esse}.
There is also a free variant \form{fore}.

\subsubsection{The perfect system}

The perfect forms -- finite forms and the perfect active infinitive -- 
of \form{sum} can be formed regularly (\prettyref{sec:finite-paradigm})
according to the perfect stem \form{fu-}.

\subsubsection{The imperative system}

The present imperative system, 
which is known for reflecting the present stem,
is formed regularly using \form{es-}:
the singular second person present imperative is \form{es}
and the plural second person present imperative is \form{este}.

\subsubsection{The present system}

The imperfect forms of \form{sum} are highly irregular,
though patterns can still be found.
In the indicative part (\prettyref{tbl:indicative-sum}): 
\begin{itemize}
    \item The \category{present} forms show no pattern
    except the personal endings.
    Note that here \form{-m} instead of \form{-\={o}}
    is used for the first person singular form.
    \item The \category{imperfect} forms are formed 
    by adding the standard personal endings 
    (\form{-m}, \form{-s}, \form{-t}, 
    \form{-mus}, \form{-tis}, \form{-nt})
    to \form{er\={a}},
    the vowel \form{\={a}} of which 
    undergoes shortening according to rules in \prettyref{sec:tense-mood-marking}.
    \item The \category{future} forms are formed by 
    the same personal endings seen in the first and the second conjugations,
    although the tense marker isn't the same: 
    the stem-tense marker complex is \form{er-}
    instead of the stem plus \form{-b-}.
\end{itemize}

\begin{table}[H]
    \caption{The indicative paradigm of \form{sum}}
    \label{tbl:indicative-sum}
    \centering
    \begin{tabular}{lll}
    \toprule
    \category{present} & \category{imperfect}  & \category{future}  \\
    \midrule
    \form{sum}     & \form{eram}       & \form{er\={o}} \\
    \form{es}      & \form{er\={a}s}   & \form{eris}    \\
    \form{est}     & \form{erat}       & \form{erit}    \\
    \form{sumus}   & \form{er\={a}mus} & \form{erimus}  \\
    \form{estis}   & \form{er\={a}tis} & \form{eritis}  \\
    \form{sunt}    & \form{erant}      & \form{erunt}   \\ \bottomrule
    \end{tabular}
\end{table}

In the subjunctive paradigm (\prettyref{tbl:subjunctive-sum}),
we find that in the \category{present} system, 
the stem-tense marker complex is fused into \form{sī-},
and in the \category{imperfect} system,
the stem-tense marker complex is fused into \form{ess\={e}-} or \form{for\={e}-},
both of which are then attached to the standard \form{-m}, \form{-s}, etc. 
personal endings, 
and the vowel shortening rule in \prettyref{sec:tense-mood-marking} works.

\begin{table}[H]
    \caption{The subjunctive paradigm of \form{sum}}
    \centering
    \label{tbl:subjunctive-sum}
    \begin{tabular}{ll}
    \toprule
    \category{present}   & \category{imperfect}   \\ \midrule
    \form{sim}       & \form{essem}, \form{forem}       \\
    \form{sīs}   & \form{ess\={e}s}, \form{for\={e}s}   \\
    \form{sit}       & \form{esset}, \form{foret}       \\
    \form{sīmus} & \form{ess\={e}mus}, \form{for\={e}mus} \\
    \form{sītis} & \form{ess\={e}tis}, \form{for\={e}tis} \\
    \form{sint}      & \form{essent}, \form{forent}     \\ \bottomrule
    \end{tabular}
\end{table}

\subsection{The verb \form{faci\={o}}}

The verb \form{faci\={o}} looks pretty regular
regarding the endings, 
except for one thing: 
its \emph{stem} alternates according to the voice.

\section{Auxiliary verb constructions}

An auxiliary verb, despite morphologically being a verb, 
is embedded into the \ac{tam} system of a language.
The following aspects can be used to identify 
an auxiliary verb from a lexical one:
\begin{itemize}
    \item \emph{Standalone or not.} A lexical verb may appear on its own;
    an auxiliary verb never does this.
    \item \emph{Scope and hierarchy compared with other \ac{tam} marking.} 
    The relative order of complement-taking lexical verbs can be arbitrary;
    this is not true for auxiliary verbs. 
    \item \emph{Interplay with argument structure: voice}.
    \item \emph{Interplay with argument structure: complement types.} 
    A lexical verb usually imposes semantic confinements on the complement 
    \item \emph{Interplay with clause type.} 
    Some auxiliary verbs never appear in certain types of non-finite constructions, 
    because the \ac{tam} categories labeled by them 
    are absent in these non-finite clauses.
\end{itemize}


Latin is usually perceived as a language with few analytic properties.
Traditionally only \form{sum} and its forms are considered auxiliary verbs
when they are used with the perfect passive participle to form
perfect \ac{tam} constructions in passive voice.
Some however argue that there are more auxiliary verbs.

\chapter{Verb frames}\label{chap:verb-frame}

This chapter is mainly about verbs that don't take complement clauses as arguments.
The phenomena discussed in this chapter mostly apply to complement clause constructions as well,
but complement clause constructions have their own peculiarities 
(\prettyref{chap:complement-clause}).

\section{Descriptive parameters}

\subsection{The argument-adjunct distinction}

A clear complement-adjunct distinction 
-- telling peripheral arguments from core arguments or oblique arguments --
is hard to establish in Latin.
Below I assess several parameters usually used to draw the distinction
listed in \citet[\citesec{4.1.2}]{cgel}:
\begin{itemize}
    \item \emph{Content of clausal dependent.} Latin peripheral arguments do not necessarily have prepositions.
    \item \emph{Easiness of topicalization, etc.} Latin is highly free-ordered and therefore all clause dependents 
    can leave their base positions.
    \item \emph{Obligatoriness.} Latin is also highly \term{pro}-drop,
    and even uncontroversial core arguments can be omitted.
    \item \emph{Licensing by verb.} Oblique arguments are frequent in Latin,
    as is the case in English 
    (consider \form{run away from} or \form{get into}).
    \item \emph{Anaphora referring to core verb phrase.} Latin doesn't have systematic way to replace the core predicate (i.e. without adjuncts) by an anaphora.
\end{itemize}
Selection, licensing, and obligatoriness tests 
can still be used in more subtle ways
(e.g. reference in discourse to identify omitted argument)

these criteria are however hard to use for a classical language. 

\begin{todobox}{Argument and adjunct: directional construction}{direction-argument}
    If a directional phrase obligatorily has the subject/object as its subject,
    then it has to be a complement and not an adjunct.

    Thus, despite I'm fully aware that  
    clausal dependents concerning place, instrument, mean, etc. 
    may be licensed by both the argument structure of the verb 
    and by clausal adjunct positions 
    and may have clear structural differences in other languages 
    (as in English), 
    currently no distinction between the two cases is made.
    Following the traditional notion,
    the subject, several kinds of objects,
    the copular complement (\citet{cgel} calls it \term{predicative complement}) 
    are identified as complements.
\end{todobox}

\subsection{Argument as entity or situation}\label{sec:verb-phrase.arguments.compatibility}

Verbs can be classified semantically into three subgroups,
according to what their arguments refer to
(\citealt[Part B]{dixon2005semantic};
\citealt[\citesec{18.5}]{dixon2010basic2};
and \citealt[\citesec{3.3}]{dixon2009basic1}):
\begin{enumerate}
    \item Primary-A, which contains verbs that 
    don't take arguments with meanings similar to those of complement clauses,
    \item Primary-B, which are semantically \concept{complement-taking}%  
    \footnote{
        Here the term means ``semantically equivalent to a complement clause construction''.
    }
    and \concept{lexical},
    which have arguments that are semantically equivalent to complement clauses 
    (but not necessarily syntactically coded as complement clauses)
    and have meanings more complicated then what's expected for grammatical items, and 
    \item Secondary, members of which have the same \emph{meaning} 
    of certain grammatical constructions in the verbal system,
    but not the same syntactic properties
    (for example, they may just take complement clauses instead of being auxiliary verbs).
\end{enumerate}
The valency class of the verb is strongly related to but is not determined by the semantics of the verb.


\section{Prototypical intransitive verbs}\label{sec:prototypical-intransitive}

\subsection{The \classify{motion} type}


\section{Prototypical transitive verbs}\label{sec:prototypical-transitive}

With complement-taking verbs temporarily excluded,
a prototypical transitive verb is more or less close to the \classify{affect} type,
with an A argument which is the causer of the event 

\section{Copular verbs}

\subsection{The verb \form{sum}}\label{sec:sum}

It's also possible to use \form{sum} with an indirect object, 
and the meaning because \translate{something be to [someone]_{\text{indirect object}}}.
In this case we get the possessive dative construction
\citep[\citesec{373}]{allen1903allen}.

TODO: 

\begin{exe}
    \ex \gll Nono, utrum uti debeat metaphoricis vel symbolicis locutionibus. \\
    tenth whether at.any.rate owe-\category{subj}.\category{pres}-\category{3sg}.\category{act}  
    metaphoric-\category{pl.dat/abl} or symbolic-\category{pl.dat/abl} 
    speech-\category{pl.dat/abl} \\
    \glt \translate{Whether (it) at any rate uses metaphoric or symbolic speech.} (\literature{Summa}, I q. 1 pr.)
\end{exe}

\section{Two place verbs}

The case marking of the stimulus argument has some relations to its animacy
\citep[\citesec{4.34}]{Pinkster1}.

\section{Preverb}

Roughly speaking, the preverbs are adverbs or preposition incorporated into the verb stem.
The licensing condition of the preverb is highly restricted:
it only takes place when a \category{path} category is licensed by the verb 
(which in Latin requires the event to be telic, 
and therefore the \category{imperfect} tense and the like may be a problem for prefixation)

\subsection{The prototypical dative construction}\label{sec:vp.complement.indirect-object}

Latin also has two complement positions named as object:
the indirect object and the secondary object.
The indirect object is distinguished by the following grammatical properties:
\begin{itemize}
    \item \emph{Coding of semantic role}: in a AGT-type argument structure, 
    the indirect object is usually the G argument.
    Intransitive clauses sometimes also have indirect objects, 
    and an indirect object, in this case, is also a G argument.
    \item \emph{Case marking}: indirect objects are always dative
    (\prettyref{sec:dative-distribution}).
    \item \emph{Passivization}: indirect objects are always retained in passive clauses. 
    They are never promoted to subjects in passivization.
    % TODO: category
\end{itemize}

\section{Teaching and secondary object}

The secondary object is distinguished by the following grammatical properties:
\begin{itemize}
    \item \emph{Coding of semantic role}: in an AGT-type argument structure
    that is always about information flowing,
    the T argument (i.e. the thing asked about or taught about) is the secondary object.
    The G argument (i.e. the person who is asked or taught) is the direct object.
    Sometimes the G argument is ablative, and in this case, 
    there is only one accusative argument: the secondary object.
    Another place where secondary objects appear is 
    clauses headed by a verb with a compounded accusative preposition. % TODO: SAO typology
    \item \emph{Case marking}: secondary objects are always accusative.
    \item \emph{Passivization}: secondary objects can be passivized, but much more rarely than direct objects.
    \item % TODO: category
\end{itemize}

The distributions of the secondary object and the indirect object 
are mutually exclusive.
This means for ditransitive verbs of type \classify{giving}, 
Latin shows a clear and strong tendency to identify the T argument with the monotransitive O,
while for ditransitive verbs about teaching,
the inverse is true.

\begin{infobox}{Comparison with English}{english-indirect-object}
    It can be found that the Latin indirect object has more similarity with the English \form{to}-PP,
    which is also called the indirect object in some grammars, but not in \citet{cgel}.
    The Latin indirect object differs from the English (accusative) indirect object in passivization.
    Since in Latin, verbs with AGT-type argument structure do not have alternation of complementation pattern
    -- in English we have \form{give sth. to sb.} and \form{give sb. sth.}, 
    while in Latin there is only the former one, but \form{to sb.} is replaced by a dative,
    (always with no preposition) --
    the G argument is identified with the E argument,
    and the T argument is identified with the P argument.
    In other words, in Latin, there is only 
    the \form{John gave [goods]_{\text{T}} to [charity]_{\text{G}}} pattern:
    the double-object \form{John gave charity goods} pattern is absent.
    
    Therefore, for typical ditransitive verbs, i.e. verbs like \form{give}, 
    Latin shows a clear and strong tendency to identify the T argument with the monotransitive O,
    which is more typical than English%
    \footnote{
        In English, in the \form{give sb. sth.} construction, it is the person i.e. the G argument that is passivized,
        while the T argument i.e. \form{sth.} cannot, though the latter is identified with monotransitive O
        according to other criteria. 
    },
    but for verbs with meaning of \category{teach} or \category{ask},
    there is also a clear and strong tendency to identify the G argument with the monotransitive O.
    The term \term{secondary object} is coined to cover this grammatical position.
\end{infobox}



\section{(Change of) location}

TODO: considering moving this section to the case section

Various semantic roles can be summarized as \category{source}, 
and the source clausal dependents -- adjunct or complement -- have the following properties.
Note that we are dealing with a \emph{group} of clausal dependents.
\begin{itemize}
    \item \emph{Coding of semantic role}: 
        
    \item \emph{Case marking}: a source argument is in the ablative case.
        It may come together with the prepositions \form{ex} or \form{ab}.
    \item \emph{Passivization}: not available.
\end{itemize}


\chapter{Constituent order}\label{chap:order}

\section{Configurationality in Latin}

Patterns in Latin constituent order are often overlook in traditional grammar.
Still, fine-grained constituency is demonstrated by 
the relation between \form{non} and the auxiliary 
(\prettyref{sec:constituent-order.aux-neg}),
radical change of VO/OV frequency when structural ambiguity is controlled
(\prettyref{sec:constituent-order.history}),
and usual constituency tests 
(\citealt[\citesec{1.6}]{danckaert2017development}).
The \form{non}-before-auxiliary condition implies 
a auxiliary hierarchy just like the one in English, 
although it's not as developed as the latter 
since Latin is inflectionally rich.
The constituency tests hint on 
at least the subject-\acs{vp} binary branching.
The fact that superficial VO/OV orders may have structural ambiguity 
means it's likely that some of the constituent orders 
are comparable to
English poetry in imitation of Latin \citep[\citesec{600}]{allen1903allen},
Japanese scrambling and topicalization (TODO: ref).

Available evidence supports the tradition in existing secondary literatures that
Latin is thus better described as a discourse-configurational language,
with multiple topicalization and focalization structures
(\citealt[\citepage{189}]{oniga2014latin}; 
\citealt[\citepage{77}]{danckaert2017development}; 
\citealt{devine2006latin}, among others).
Initial positions in clauses clearly bear information structure functions. 
Constituents that are able to move to the positions 
include almost everything: 
arguments, adverbials, the negator \form{non}, 
and also the verb (TODO: aux) 
\citep[\citesec{598}]{allen1903allen}. 
TODO: is it possible for the main verb to move to the initial point only?
Note that fronting of the verb may be used 
to focus the verb root or the \emph{tense}
\citep[\citepage{397}]{allen1903allen};
this means preposing of the verb is comparable to 
stressing the verb in English.

Apart from the information structure, 
prosody is also an important factor TODO

\section{Broad focus clauses and the neutral order}

Pragmatically unmarked sentences have the arguments as pragmatic topics 
and the rest of the \acs{vp} are the focus:
\translate{we already know [Baebius], [his army], and [Pinarius] (topics);
the piece of new information is that 
[a transferring process happens involving the three] (focus)}.
This kind of sentences, known as ``broad scope focus'' sentences 
in \citet[\citepage{15}]{devine2006latin}
because the scope of focalization is broad
and not restricted to a single argument,
demonstrates a constituent order which may be referred as the \concept{neutral order} in Latin
\citep[\citepage{79}]{devine2006latin},
although its frequency -- without controlling the information structure -- 
isn't significant higher than other constituent orders.
This order clearly isn't a faithful representation 
of the argument structure, 
since the direct object -- the argument that is supposed to be the closest one to the main verb -- 
appears far from the main verb. 
The reason is likely to be that the ``neutral order'' also marks 
the aforementioned ``unmarked'' information structure, 
in which the arguments are by default topicalized.

Deviation from the ``neutral'' order may be divided into the following levels:

The usual typological classification of Latin as a SOV language, 
despite being misleading in suggesting a rigid base order, 
still makes sense in pointing out that indeed constituents that are ``higher'' are moved leftwards.

In periphrastic conjugation,
the constituent order is subject + object + verb + \form{sum}.
This may also show a mismatch between the dependency structure and the linear order,
since if we consider the auxiliary \form{sum} to be 
the analytic counterpart of the inflectional suffix 
and has a higher position compared to the main verb (i.e. the participle),
then since the subject and topicalized constituents usually appear on the left side, 
it also should appear on the left side compared with the participle under it.
But the case may just be that the participle is focused and is moved leftward by default, 
just like the topicalized direct object,
so the auxiliary then appears at the end of the clause in the neutral order.

\section{Positioning of arguments}

\section{Positioning of the verb (without auxiliary)}

\begin{infobox}{Position of the verb}{verb-position}
    The position of the verb involves a theoretical question:
    is its appearance away from the unmarked clause-final position
    due to phrasal movement 
    (the verb root being moved to a new position, 
    carrying all suffixal realizations of 
    TP functional heads together with it),
    or is it due to being attracted by some sort of functional head
    (in this case the verb root is just like a head in head movement;
    similar mechanisms appear in, say, 
    \form{on the top of the mountain \emph{lies} a small village})?
    Since this distinction is hard to test, 
    I refrain from picking up one explanation.
\end{infobox}

\section{Positioning of auxiliary and negation}\label{sec:constituent-order.aux-neg}



One piece of evidence suggesting the grammatical status of \form{sum} 
is somehow different from a lexical verb 
is that its position has non-trivial interaction 
with the position of \form{non}.
The negator \form{non} usually appears before the verb 
(\citealt[\citesec{1.5}]{danckaert2017development}, TODO: or aux?),
and apparent violations seem to be constituent negation 
as opposed to sentential negation \citep[\citepage{43}]{danckaert2017development}.

\section{On so-called postposing constructions}\label{sec:clause-order.postpone}

Whether postposing exists as an information structure marking device 
is still not completely clear. 
It's said that postposing is never used for emphasis \citep[\citepage{395}]{allen1903allen},
and apparent counterexamples are all ``afterthoughts'' TODO;
but 

The postponed subject is likely to be an afterthought, 
coindexed with a zero pronominal 
in the rest of the sentence before it
\citet[\citepage{87}]{devine2006latin},
comparable to English \form{it kills three people, the wandering puma}.

\section{Notes on some typologically rare constituent orders}


\section{Historical evolution}\label{sec:constituent-order.history}

Without sentences in which the OV/VO alternation 
can be alternatively analyzed as topicalization,
VO frequency no longer shows significant change as time went by,
indicating a well-defined extended verb phrase \citep[\citesec{1.5}, \citepage{29}]{danckaert2017development}.

\section{Information packaging constructions}

\section{Existential clause}\label{sec:clause.exist}

In the existential construction,
the \form{sum} verb always appears first
\citep[\citepage{396}]{allen1903allen}.

\section{Cleft construction}

nequitia est quae te non sinit esse senem

\section{Question formation}



\chapter{Complement clause constructions}\label{chap:complement-clause}


\section{Infinitives}

\subsection{\form{Accusativus cum infinitivo}, or the autonomous infinitive}\label{sec:complement-clause.infinitive.aci}

Despite the superficial resemblance to the English object raising constructions,
Latin complement infinitives with accusative subjects 
are licensed even after complete nominalization
of the complement-taking verb.
Therefore the accusative subject of the complement clause 
can't be seen as an object of the complement-taking verb, 
since the nominalized verb no longer takes object in Latin.
Therefore, the \form{accusativus cum infinitivo} construction 
is comparable to English \form{for sb. to do sth.},
where the subject of the complement clause is autonomous;
the accusative case here is the case assigned to 
the subject of an non-finite clause where the nominative case is not available.
This is not completely unexpected, 
since even absence of an explicit complementizer 
is observed in the \form{ut} clause as well
\citep[\citepages{290-292}]{oniga2014latin}.

\chapter{Relative constructions}\label{chap:relative-clause}

\section{General comments}

\paragraph*{Agreement properties}\label{sec:relative-clause.overview.agreement}

The case of a relative pronoun is determined 
by its syntactic position in the relative clause, 
and \emph{not} the case of the antecedent,
though the number and gender categories 
are determined by agreement with the antecedent.

\chapter{Coordination and subordination}

Latin coordination in the nominal system and the verbal system 
shows strong correspondence,
with most conjunction words being shared
by the two systems.



\chapter{Texts}

Below are some examples of Latin texts, 
in an order from the easiest to the hardest,
with remarks on their vocabulary and grammar. 

\section{\literature{Aeneid}}

\subsection{Introduction}



\begin{exe}
    \ex\label{ex:text.aeneid.1.1} \gll Arma virumque cano, Troiae qui primus ab oris
    Italiam, fato profugus, Laviniaque venit litora, 
    multum ille et terris iactatus et alto
    vi superum saevae memorem Iunonis ob iram;
    multa quoque et bello passus, dum conderet urbem,  
    inferretque deos Latio, genus unde Latinum,
    Albanique patres, atque altae moenia Romae. \\
    weapon(\category{n})-\category{pl}.\category{acc} 
    man(\category{m})-\category{sg}.\category{acc}=and 
    sing-\category{ind}.\category{pres}.\category{1sg}
    Troy-\category{sc}.\category{gen} 
    \category{rel}.\category{m}.\category{sg}.\category{nom}
    first-\category{m}.\category{sg}.\category{nom} 
    from shore(\category{f})-\category{pl}.\category{abl} 
    Italy(\category{f})-\category{sg}.\category{acc} 
    fate(\category{n})-\category{sg}.\category{abl} 
    exiled-\category{m}.\category{sg}.\category{nom} 
    Lavinia-TODO=and  
    go.to-\category{ind}.\category{pres}.\category{3sg} 
    shore(\category{n})-\category{pl}.\category{acc} \\
    \glt \translate{I sing weapons and a man, 
    who was the first from the shores of Troy to Italy, 
    was by fate exiled, 
    and }
\end{exe}

In (\prettyref{ex:text.aeneid.1.1}),
it should be noted that \form{arma} is in plural only.
The \form{qui} clause is an example of the rule 
that the relative pronoun doesn't agree in case 
with the antecedent (\prettyref{sec:relative-clause.overview.agreement}).
The copula is omitted in the \form{qui} clause.


\section{Liturgy texts}

\subsection{Short formulae in the Roman Mass}

Examples in this section are short formulae found in the Roman Mass
in the order of their appearance.
In (\prettyref{ex:text.mass.1}, \prettyref{ex:text.mass.2}),
\form{nomine} and \form{patris} are third declension nouns, 
while \form{spiritus} is a fourth declension noun. 

\begin{exe}
    \ex\label{ex:text.mass.1} \gll In Nomine Patris, et Filii, et Spiritus Sancti. \\
    in name-\category{sg}.\category{abl} Father-\category{sg}.\category{gen} 
    and Son-\category{sg}.\category{gen} 
    and \category{spirit}(\category{m})-\category{sg}.\category{gen}
    holy-\category{m}.\category{sg}.\category{gen} \\
    \glt \translate{In the name of the Father, and of the Son, and of the Holy Spirit.}

    \ex\label{ex:text.mass.2} \gll -- Dominus vobiscum. -- Et cum spiritu tuo. \\
    {} Lord(\category{m})-\category{sg}.\category{nom} 
    \category{2pl}.\category{abl} 
    {} and with spirit(\category{m})-\category{sg}.\category{abl} 
    your-\category{m}.\category{sg}.\category{abl} \\
    \glt \translate{-- The Lord be with you. -- And with your spirit.}
    
    \ex 
\end{exe}

\subsection{Nicene Creed}

\begin{exe}
    \ex \gll Credo in unum Deum, Patrem omnipotentem, \\
    believe-\category{ind}.\category{pres}.\category{1sg} in 
    one-\category{m}.\category{sg}.\category{acc} 
    God(\category{m})-\category{sg}.\category{acc} 
    father(\category{m})-\category{sg}.\category{acc}
    omnipotent-\category{m}.\category{sg}.\category{acc} \\
    \glt \translate{I believe in one God, (the) omnipotent Father,} 
    \ex \gll factorem caeli et terrae, visibilium omnium et \\ 
    maker- \\
    \glt \translate{maker of}
\end{exe}

\section{Vulgate bible}

\subsection{Excerpts in John 1}\label{sec:text.vulgate.john}

\begin{exe}
    \ex\label{ex:text.john.1.1} 
    \gll in principio erat Verbum et Verbum erat apud Deum et Deus erat Verbum \\
    in {} be.\acs{imperfect}  \\
    \glt \translate{In the beginning} (John 1:1)
    
    \ex\label{ex:text.john.1.3}
    \gll omnia per ipsum facta sunt 
    et sine ipso factum est nihil quod factum est \\
    all-\category{n}.\category{pl}.\category{nom} through \category{dem}-\category{acc}
    make.\category{pprt}-\category{n}.\category{pl}.\category{nom} 
    be.\category{ind}.\acs{present}.\category{3pl} 
    and without \category{dem}.\category{abl} 
    make.\category{pprt}-\category{n}.\category{sg}.\category{nom} 
    be.\category{ind}.\acs{present}.\category{3sg}
    nothing.\category{nom}
    \category{rel}.\category{n}.\category{3sg}
    make.\category{pprt}-\category{n}.\category{sg}.\category{nom} 
    be.\category{ind}.\acs{present}.\category{3sg} \\
    \glt \translate{All have been made through exactly this (i.e. the Word of Lord),
    and without exactly this, nothing that has been made has been made.} (John 1:3)
\end{exe}

As an example, 
below I show how
(\prettyref{ex:text.john.1.3}) can be parsed.
First we can see a \form{et} dividing the sentence into two branches.
\begin{enumerate} 
    \item For the first branch, 
        we know \form{omni-} is a quantifier meaning \form{all},
        and morphologically it's a twin-termination third declension adjective; 
        then from \prettyref{tbl:declension-ending-nouns-list}
        and the fact that we are dealing with a third declension word, 
        the ending \form{-a} means neutral and \category{pl}.\category{nom}/\category{acc}/\category{voc}. 
        The vocative case is of course impossible here. 
    \item The word \form{per} is a preposition taking an accusative object. 
        \form{Ipsum} is a basic identity demonstrative, 
        with the meaning of ``exactly this''. 
        Since it follows \form{per}, 
        the ending \form{-um} here seems to be the accusative case marker, 
        instead of a neutral nominative case marker. 
    \item The sequence \form{facta sunt} contains 
        the indicative perfect 3pl copula \form{sunt}, 
        and in \form{facta}, we see the supine stem \form{fact-} 
        of the verb \form{faci\={o}} \translate{make}. 
        The second fact means 
        \form{facta} should be the perfect passive participle in a certain inflection form.
        Then \form{facta sunt}, collectively, 
        is the indicative passive perfect 3pl form of \form{faci\={o}}.
        (Here we are fortunate: 
        it's possible that \form{facta} and \form{sunt} get scattered to different places.)
        The \form{-a} ending can again be looked up for in \prettyref{tbl:declension-ending-nouns-list}:
        the possibilities are \category{pl}.\category{nom}/\category{acc}/\category{voc} -- 
        note that the first declension singular possibilities 
        are excluded by the fact that \form{sunt} is in plural form. 
        We expect \form{facta} to be nominative 
        because it has to agree with the subject, which is always nominative 
        and it turns to be possible. 
        \item Now we should link things together. 
            The open ends are: 
            the case of \form{omnia}, 
            and the (3pl) subject of \form{facta sunt}.
            Then quite obviously, 
            we find \form{omnia} should be in the subject position, 
            and therefore everything works well. 
        \item We can also check gender agreements to make sure our reading is correct.
\end{enumerate}
The second half is done in similar manners. The structure of the text looks like this: 
\begin{exe}
    \sn {} [[omnia]_{\text{subject}} [per ipsum]_{\text{instrument:\acs{pp}}} [facta sunt]_{\text{verbal complex}}]_{\text{coord}} 
    et [[sine ipso]_{\text{adverbial:\acs{pp}}} [factum est]_{\text{verbal complex}} [nihil [quod factum est]_{\text{rel}}]_{\text{subject}}]_{\text{coord}}
\end{exe}


\section{\literature{Summa Theologiae}}

\subsection{Introduction}

Thomas Aquinas, a theologian and philosopher, 
was recognized by the Catholic Church 
as one of the Doctors of the Church,
usually known as \form{Doctor Angelicus} 
(doctor-\category{sg.nom} angle-\category{adj}-\category{sg.nom}, 
\translate{Angelic Doctor}).
\literature{Summa Theologiae}
(summary-\category{sg.nom} theology-\category{sg.gen}, \translate{Summary of Theology}) 
or \literature{Summa Theologica} 
(summary-\category{sg.nom} theology-\category{adj}-\category{sg.nom.f}),
usually abbreviated as \literature{Summa},
is probably his most known work.
As its name implies, the book is a summary of Catholic theology,
containing necessary information for 
beginning theological students 
and for arguing against frequently seen heresies. 
Apart from its great value within and out of Christianity, 
the book is a good example of 
what Medieval Latin looks like.

One salient feature of \literature{Summa} is 
it's written in a rather strict and even dull format.
The work is divided into three Parts,
each containing many Questions.
Each Question (i.e. a topic, like ``the nature and extent of this sacred doctrine'') 
is divided into several Articles 
(i.e. a specific question, like whether the sacred doctrine is a science).
Each Article follows the following scheme: 
\begin{enumerate}
    \item First, a sentence like (\ref{ex:text.summa.intro.1}) 
    indicating index of the Article within the Question it belongs to.
    \item Then several Objections are raised to support an unorthodox idea,
    usually in \category{subjunctive}.
    \item Then the accepted doctrine contrary to the above heretical claims is given
    (\ref{ex:text.summa.intro.2}).
    \item Finally, a comment by Thomas on this Article (\ref{ex:text.summa.intro.3}).
\end{enumerate}

\begin{exe}
    \ex\label{ex:text.summa.intro.1} 
    \gll Ad primum sic proceditur. \\
    to first thus proceed-\category{3sg.pass} \\
    \translate{(Our topic) goes to the first issue. (lit. It proceeds to the first.)}
    
    \ex\label{ex:text.summa.intro.2} 
    \gll Sed contra est quod dicitur II ad Tim. III \dots \\ 
    but contrarily be.\category{ind}.\category{pres}.\category{act}.\category{3sg} 
    what.\category{r}.\category{sg.n} 
    say-\category{ind.pres}-\category{3sg.pass} two to(TODO) \category{name} three \\
    \glt \translate{But on the contrary there is what is said in 2 Timothy 3 \dots} 
    (\literature{Summa}, I q. 1 s.c.)

    \ex\label{ex:text.summa.intro.3} 
    \gll Respondeo dicendum quod \dots \\
    respond-\category{ind.pres}-\category{1sg.act} \\
    \glt \translate{I respond to what is said that \dots}
\end{exe}

Here we present a glossed Article in \literature{Summa}: I q. 1 a. 9.


\bibliographystyle{plainnat}
\bibliography{latin,theory}

\end{document}