\documentclass{article}

\usepackage{geometry}
\usepackage{titling}
\usepackage{titlesec}
\usepackage{paralist}
\usepackage{float}
\usepackage{footnote}
\usepackage{enumerate}
\usepackage{amsmath, amssymb, amsthm}
\usepackage{gb4e}
\noautomath
\usepackage{bbm}
\usepackage{soul}
\usepackage{graphicx}
\usepackage{siunitx}
\usepackage[table,xcdraw]{xcolor}
\usepackage{tikz}
\usepackage[ruled, vlined, linesnumbered, noend]{algorithm2e}
\usepackage{xr-hyper}
\usepackage[colorlinks]{hyperref} % linkcolor=black, anchorcolor=black, citecolor=black, filecolor=black
\usepackage[most]{tcolorbox}
\usepackage{caption}
\usepackage{subcaption}
\usepackage{booktabs}
\usepackage{multirow}
\usepackage[figuresright]{rotating}
\usepackage{acro}
\usepackage[round]{natbib} 
\usepackage{nameref,zref-xr}
\zxrsetup{toltxlabel}
\zexternaldocument*[cgel-]{../English/cambridge}[cambridge.pdf]
\zexternaldocument*[general-]{../methodology/glossing}[glossing.pdf]
\usepackage{prettyref}

\geometry{left=3.18cm,right=3.18cm,top=2.54cm,bottom=2.54cm}
\titlespacing{\paragraph}{0pt}{1pt}{10pt}[20pt]
\setlength{\droptitle}{-5em}

\DeclareMathOperator{\timeorder}{\mathcal{T}}
\DeclareMathOperator{\diag}{diag}
\DeclareMathOperator{\legpoly}{P}
\DeclareMathOperator{\primevalue}{P}
\DeclareMathOperator{\sgn}{sgn}
\newcommand*{\ii}{\mathrm{i}}
\newcommand*{\ee}{\mathrm{e}}
\newcommand*{\const}{\mathrm{const}}
\newcommand*{\suchthat}{\quad \text{s.t.} \quad}
\newcommand*{\argmin}{\arg\min}
\newcommand*{\argmax}{\arg\max}
\newcommand*{\normalorder}[1]{: #1 :}
\newcommand*{\pair}[1]{\langle #1 \rangle}
\newcommand*{\fd}[1]{\mathcal{D} #1}

\newcommand*{\citesec}[1]{\S~{#1}}
\newcommand*{\citechap}[1]{chap.~{#1}}
\newcommand*{\citefig}[1]{Fig.~{#1}}
\newcommand*{\citetable}[1]{Table~{#1}}
\newcommand*{\citefootnote}[1]{footnote~{#1}}

\newrefformat{sec}{\citesec{\ref{#1}}}
\newrefformat{fig}{\citefig{\ref{#1}}}
\newrefformat{tbl}{\citetable{\ref{#1}}}
\newrefformat{chap}{\citechap{\ref{#1}}}

\usetikzlibrary{arrows,shapes,positioning}
\usetikzlibrary{arrows.meta}
\usetikzlibrary{decorations.markings}
\tikzstyle arrowstyle=[scale=1]
\tikzstyle directed=[postaction={decorate,decoration={markings,
    mark=at position .5 with {\arrow[arrowstyle]{stealth}}}}]
\tikzstyle ray=[directed, thick]
\tikzstyle dot=[anchor=base,fill,circle,inner sep=1pt]


\tcbuselibrary{skins, breakable, theorems}

\newtcbtheorem[number within=chapter]{infobox}{Box}%
  {colback=blue!5,colframe=blue!65,fonttitle=\bfseries, breakable}{infobox}

\newcommand*{\concept}[1]{\textbf{#1}}
\newcommand*{\term}[1]{\emph{#1}}
\newcommand*{\corpus}[1]{\emph{#1}}

\newcommand*{\vP}{\textit{v}P}

\DeclareAcronym{blt}{short = BLT, long = Basic Linguistic Theory}
\DeclareAcronym{cgel}{short = CGEL, long = The Cambridge Grammar of the English Language}
\DeclareAcronym{dm}{short = DM, long = Distributed Morphology}
\DeclareAcronym{tag}{long = Tree-adjoining grammar, short = TAG}
\DeclareAcronym{sfp}{long = sentence final particle, short = SFP}
\DeclareAcronym{vp}{long = verb phrase, short = VP}
\DeclareAcronym{np}{long = noun phrase, short = NP}
\DeclareAcronym{adjp}{long = adjective phrase, short = AdjP}
\DeclareAcronym{advp}{long = adverb phrase, short = AdvP}
\DeclareAcronym{pp}{long = preposition phrase, short = PP}
\DeclareAcronym{cls}{long = classifier, short = CLS}
\DeclareAcronym{dist}{long = distal, short = DIST}
\DeclareAcronym{prox}{long = proximate, short = PROX}
\DeclareAcronym{dem}{long = demonstrative, short = DEM}
\DeclareAcronym{dur}{long = durative, short = DUR}
\DeclareAcronym{neg}{long = negative, short = NEG}
\DeclareAcronym{tam}{long = {Tense, Aspect, Mood}, short = TAM}
\DeclareAcronym{tame}{long = {Tense, Aspect, Mood, Evidentiality}, short = TAME}
\DeclareAcronym{pie}{long = Proto-Indo-European, short = PIE}

% Disable unsupported commands in bookmark titles 
\pdfstringdefDisableCommands{%
  \def\\{}%
  \def\texttt#1{<#1>}%
  \def\mathbb#1{#1}%
}
\pdfstringdefDisableCommands{\def\eqref#1{(\ref{#1})}}

\makeatletter
\pdfstringdefDisableCommands{\let\HyPsd@CatcodeWarning\@gobble}
\makeatother

\newcommand{\cgel}{\href{../English/cambridge.pdf}{my notes about CGEL}}
\newcommand{\general}{\href{../methodology/glossing.pdf}{this note}}

\title{Notes about Latin grammar}
\author{Jinyuan Wu}

\begin{document}

\maketitle

This is a note about Latin grammar, 
based on the framework outlined in \general.

\section{Overview}

\subsection{Historical remarks}

This note is about Classical Latin and Ecclesiastical Latin.
That's to say languages like Old Latin, vulgar Latin (with prototypes of Romance articles)
are not discussed.

% TODO: existing works, how to use the dictionary

\subsection{Word classes}

Latin word classes can be defined easily via morphology,
and these classes prove to have morphosyntactic significance.
Traditionally speaking, 
word classes with none or poor morphology are called \concept{particles},
and non-particle words can be divided into two large classes:
those with similar morphology of prototypical nouns (i.e. \concept{declension}) are \concept{nominals},
while words with similar morphology of prototypical verbs (i.e. \concept{conjugation})
form a uniform class rightfully called \concept{verbs}.
Nominals include \concept{nouns} and \concept{adjectives},
the distinction between the two can also be defined morphologically.

Latin particles include \concept{prepositions}, \concept{adverbs},
\concept{interjections}, and \concept{conjunctions}.
The adverb class and the preposition class have a large overlap:
often a preposition has an intransitive counterpart,
which is similar to a prototypical adverb.
Conjunctions may be seen as ``prepositions for clauses''.
The functions and etymologies of particles are highly diverse.

Latin nouns, verbs, and adjectives are all open categories.
They are able to head constituents,
and so are correlatives (though correlatives can be listed in the grammar).
The preposition class is closed and is a part of the grammar,
just like conjunctions.
However, conjunctions are purely functional,
while certain prepositions may be argued to head attributive expressions:
though prepositions are often said to be markers of a periphrastic case system,
the semantics carried by certain Latin prepositions are too complicated for a case system.
This is also the case of adverbs:
some adverbs seem to be periphrastic markers of \acs{tame} categories
and therefore may be considered as a part of the grammar,
while others seem to carry ``real'' meanings.%
\footnote{
    In \citesec{\ref{general-sec:lexical-function-distinction}} in \general, 
    I say words with ``real'' meanings and head constituents 
    words with ``real category labels''.
    Thus, certain adverbs and prepositions have ``real category labels'',
    because they actually head peripheral arguments or attributives,
    so they are type-3 words,
    while other adverbs and prepositions are purely functional and are type-4 words.
}

\begin{figure}
    \centering
    

\tikzset{every picture/.style={line width=0.3pt}} %set default line width to 0.75pt        

\begin{tikzpicture}[x=0.75pt,y=0.75pt,yscale=-0.8,xscale=0.8]
%uncomment if require: \path (0,674); %set diagram left start at 0, and has height of 674

%Straight Lines [id:da0819119912251447] 
\draw [color={rgb, 255:red, 155; green, 155; blue, 155 }  ,draw opacity=0.2 ]   (109.33,384.91) -- (793.33,384.91) ;
%Rounded Rect [id:dp9835773367494156] 
\draw  [color={rgb, 255:red, 155; green, 155; blue, 155 }  ,draw opacity=1 ][fill={rgb, 255:red, 155; green, 155; blue, 155 }  ,fill opacity=0.2 ] (141.33,256.64) .. controls (141.33,246.48) and (149.57,238.24) .. (159.73,238.24) -- (214.93,238.24) .. controls (225.1,238.24) and (233.33,246.48) .. (233.33,256.64) -- (233.33,518.84) .. controls (233.33,529) and (225.1,537.24) .. (214.93,537.24) -- (159.73,537.24) .. controls (149.57,537.24) and (141.33,529) .. (141.33,518.84) -- cycle ;
%Rounded Rect [id:dp12690257166641228] 
\draw  [color={rgb, 255:red, 155; green, 155; blue, 155 }  ,draw opacity=1 ][fill={rgb, 255:red, 155; green, 155; blue, 155 }  ,fill opacity=0.2 ] (233.33,256.64) .. controls (233.33,246.48) and (241.57,238.24) .. (251.73,238.24) -- (306.93,238.24) .. controls (317.1,238.24) and (325.33,246.48) .. (325.33,256.64) -- (325.33,518.84) .. controls (325.33,529) and (317.1,537.24) .. (306.93,537.24) -- (251.73,537.24) .. controls (241.57,537.24) and (233.33,529) .. (233.33,518.84) -- cycle ;
%Rounded Rect [id:dp4571683339095527] 
\draw  [color={rgb, 255:red, 155; green, 155; blue, 155 }  ,draw opacity=1 ][fill={rgb, 255:red, 155; green, 155; blue, 155 }  ,fill opacity=0.2 ] (349.33,256.57) .. controls (349.33,248.1) and (356.2,241.24) .. (364.67,241.24) -- (426,241.24) .. controls (434.47,241.24) and (441.33,248.1) .. (441.33,256.57) -- (441.33,302.57) .. controls (441.33,311.04) and (434.47,317.91) .. (426,317.91) -- (364.67,317.91) .. controls (356.2,317.91) and (349.33,311.04) .. (349.33,302.57) -- cycle ;
%Rounded Rect [id:dp13945556495695044] 
\draw  [color={rgb, 255:red, 155; green, 155; blue, 155 }  ,draw opacity=1 ][fill={rgb, 255:red, 155; green, 155; blue, 155 }  ,fill opacity=0.2 ] (122.93,223.64) .. controls (122.93,211.82) and (132.51,202.24) .. (144.33,202.24) -- (316.93,202.24) .. controls (328.75,202.24) and (338.33,211.82) .. (338.33,223.64) -- (338.33,537.84) .. controls (338.33,549.66) and (328.75,559.24) .. (316.93,559.24) -- (144.33,559.24) .. controls (132.51,559.24) and (122.93,549.66) .. (122.93,537.84) -- cycle ;
%Straight Lines [id:da9866800561526903] 
\draw [color={rgb, 255:red, 155; green, 155; blue, 155 }  ,draw opacity=0.2 ]   (595.33,577.91) -- (595.33,184.91) ;
%Rounded Rect [id:dp9919662328975185] 
\draw  [color={rgb, 255:red, 155; green, 155; blue, 155 }  ,draw opacity=1 ][fill={rgb, 255:red, 155; green, 155; blue, 155 }  ,fill opacity=0.2 ] (435,431.44) .. controls (435,427.46) and (438.22,424.24) .. (442.2,424.24) -- (688.8,424.24) .. controls (692.78,424.24) and (696,427.46) .. (696,431.44) -- (696,453.04) .. controls (696,457.02) and (692.78,460.24) .. (688.8,460.24) -- (442.2,460.24) .. controls (438.22,460.24) and (435,457.02) .. (435,453.04) -- cycle ;
%Rounded Rect [id:dp04492990660366303] 
\draw  [color={rgb, 255:red, 155; green, 155; blue, 155 }  ,draw opacity=1 ][fill={rgb, 255:red, 155; green, 155; blue, 155 }  ,fill opacity=0.2 ] (147,482.31) .. controls (147,476.75) and (151.51,472.24) .. (157.07,472.24) -- (579.93,472.24) .. controls (585.49,472.24) and (590,476.75) .. (590,482.31) -- (590,512.51) .. controls (590,518.07) and (585.49,522.57) .. (579.93,522.57) -- (157.07,522.57) .. controls (151.51,522.57) and (147,518.07) .. (147,512.51) -- cycle ;
%Shape: Path Data [id:dp3209513602067542] 
\draw  [color={rgb, 255:red, 155; green, 155; blue, 155 }  ,draw opacity=1 ][fill={rgb, 255:red, 155; green, 155; blue, 155 }  ,fill opacity=0.2 ] (478.97,255.24) -- (552.94,255.24) .. controls (557.16,255.24) and (560.58,258.46) .. (560.58,262.43) -- (560.58,367.51) .. controls (560.58,372.79) and (565.13,377.07) .. (570.74,377.07) -- (630.37,377.07) .. controls (630.59,377.07) and (630.82,377.07) .. (631.04,377.05) -- (712.31,377.05) .. controls (717.01,377.05) and (720.81,380.63) .. (720.81,385.05) -- (720.81,434.24) .. controls (720.81,438.66) and (717.01,442.24) .. (712.31,442.24) -- (478.97,442.24) .. controls (474.75,442.24) and (471.33,439.02) .. (471.33,435.05) -- (471.33,262.43) .. controls (471.33,258.46) and (474.75,255.24) .. (478.97,255.24) -- cycle ;
%Rounded Rect [id:dp9478936078417666] 
\draw  [color={rgb, 255:red, 155; green, 155; blue, 155 }  ,draw opacity=1 ][fill={rgb, 255:red, 155; green, 155; blue, 155 }  ,fill opacity=0.2 ] (451.33,237.22) .. controls (451.33,228.95) and (458.04,222.24) .. (466.31,222.24) -- (759.35,222.24) .. controls (767.63,222.24) and (774.33,228.95) .. (774.33,237.22) -- (774.33,552.26) .. controls (774.33,560.53) and (767.63,567.24) .. (759.35,567.24) -- (466.31,567.24) .. controls (458.04,567.24) and (451.33,560.53) .. (451.33,552.26) -- cycle ;

% Text Node
\draw (169,267) node [anchor=north west][inner sep=0.75pt]   [align=left] {noun};
% Text Node
\draw (252,267) node [anchor=north west][inner sep=0.75pt]   [align=left] {adjective};
% Text Node
\draw (158,479) node [anchor=north west][inner sep=0.75pt]   [align=left] {personal \\pronoun};
% Text Node
\draw (244,479) node [anchor=north west][inner sep=0.75pt]   [align=left] {correlative \\pronoun};
% Text Node
\draw (368,266) node [anchor=north west][inner sep=0.75pt]   [align=left] {verb};
% Text Node
\draw (565.5,442.24) node   [align=left] {preposition};
% Text Node
\draw (674,517) node [anchor=north west][inner sep=0.75pt]   [align=left] {conjunction};
% Text Node
\draw (492,264) node [anchor=north west][inner sep=0.75pt]   [align=left] {adverb};
% Text Node
\draw (674,471) node [anchor=north west][inner sep=0.75pt]   [align=left] {interjection};
% Text Node
\draw (151,361) node [anchor=north west][inner sep=0.75pt]  [color={rgb, 255:red, 155; green, 155; blue, 155 }  ,opacity=1 ] [align=left] {noun \\morphology};
% Text Node
\draw (238.14,361) node [anchor=north west][inner sep=0.75pt]  [color={rgb, 255:red, 155; green, 155; blue, 155 }  ,opacity=1 ] [align=left] {adjective \\morphology};
% Text Node
\draw (107.33,384.91) node [anchor=east] [inner sep=0.75pt]  [color={rgb, 255:red, 155; green, 155; blue, 155 }  ,opacity=1 ] [align=left] {with real \\category \\label};
% Text Node
\draw (795.33,384.91) node [anchor=west] [inner sep=0.75pt]  [color={rgb, 255:red, 155; green, 155; blue, 155 }  ,opacity=1 ] [align=left] {with \\no real \\category \\label};
% Text Node
\draw (141,212.24) node [anchor=north west][inner sep=0.75pt]  [color={rgb, 255:red, 155; green, 155; blue, 155 }  ,opacity=1 ] [align=left] {nominal};
% Text Node
\draw (595.33,181.91) node [anchor=south] [inner sep=0.75pt]  [color={rgb, 255:red, 155; green, 155; blue, 155 }  ,opacity=1 ] [align=left] {not a part\\of grammar};
% Text Node
\draw (595.33,580.91) node [anchor=north] [inner sep=0.75pt]  [color={rgb, 255:red, 155; green, 155; blue, 155 }  ,opacity=1 ] [align=left] {a part of\\grammar};
% Text Node
\draw (499,489.5) node [anchor=north west][inner sep=0.75pt]   [align=left] {pro-adverb};
% Text Node
\draw (379,489.5) node [anchor=north west][inner sep=0.75pt]  [color={rgb, 255:red, 155; green, 155; blue, 155 }  ,opacity=1 ] [align=left] {pro-forms};
% Text Node
\draw (661,241.5) node [anchor=north west][inner sep=0.75pt]  [color={rgb, 255:red, 155; green, 155; blue, 155 }  ,opacity=1 ] [align=left] {particle};


\end{tikzpicture}

    \caption{Latin word classes}
    \label{fig:latin-word-class}
\end{figure}

\prettyref{fig:latin-word-class} is a visualization of the classification of Latin word classes.

Unattested word classes in Latin include 
articles (\corpus{a} or \corpus{the}), % TODO:

\subsection{Morphology}

Latin has rich morphology,
which enables a rather free -- but still not completely arbitrary -- constituent order.
This means description of Latin grammar is mostly dependency-relation based 
(\prettyref{sec:constituent-order-abs}),
or \acs{blt}-based,
because surface-based constituents other than \acs{np}s and clauses are hard to define.
Still, generative (constituency-based, 
though the introduction of movements and the structure of Cinque hierarchy
gives it certain flavor of dependency grammars) approaches exist for Latin constituent order.

Latin has a clear inflection-derivation distinction.
Despite its richness, 
Latin derivation is largely historical,
with meanings of derived forms 
having shifted and no longer regularly inferrable.
Latin inflection is always suffixal,
while derivation is predominantly prefixal.
Concatenative morphology (affixation and compounding) 
is prominent but isn't the only morphological device:
the following non-concatenative mechanisms are all attested:
\begin{itemize}
    \item \emph{Subtraction}: dropping of first-conjugation stem-final vowel (\prettyref{sec:tense-mood-marking}).
    \item \emph{Infixation}:  % TODO: ref
\end{itemize}
These mechanisms, however, are largely historical,
just like their concatenative counterparts.

\subsection{Noun phrases and nominal morphology}

\ac{np}s can be headed by nouns and adjectives.
Due to the rather flexible constituent order,
it's hard to tell, say, determiners from prototypical attributives (usually filled by \acs{adjp}s),
and indeed, Latin demonstratives are long classified as adjectives,
though the former are unable to take modification or complements.


\subsubsection{Noun morphology}\label{sec:nominal-inflection-abs}

\subsubsection{Pronouns}\label{sec:pro-abs}

\subsection{Clauses and verbal morphology}

\subsubsection{Categories marked by finite verb morphology}\label{sec:verb-inflection-abs}

Most clausal grammatical categories are marked on the verbal morphology.
Sometimes a grammatical category is there but is not reflected in the morphology.
For example, in English we have infinitive clauses,
but strictly speaking, there is no such thing as ``infinitive verb'':
the head verb of an infinitive clause 
has exactly the same form of a non-third person singular present tense verb.
This is not the case in Latin.
For example, the head verb of a infinitive clause in Latin 
indeed has a separate position in the paradigm.
Thus, in this section,
grammatical categories of the clause are listed in this section.

Latin doesn't have rich valency changing devices:
there is only one clause-wide valency decreasing device -- passivization -- 
and there is no valency increasing device.
Causative constructions are realized by complement clauses,
not any change in the argument structure.
Whether passivization happens is recorded by the category of \concept{voice}.%
\footnote{
    See \citesec{\ref{general-sec:valency-changing-theory}} in \general.
}
A verb (and hence the clause headed by it) is therefore either in \concept{active voice},
or in \concept{passive voice}.

Latin has fused tense and aspect:
the composition of three tense values and three aspect values 
gives nine options,
but in Latin, there are only six morphologically distinguished options,
as is shown in \prettyref{tbl:latin-tense-aspect}. 
When people talk about \concept{tense} in Latin (and in many other Indo-European languages),
they are often taking about things like the six options,
instead of the past/present/future system.

\begin{table}
    \caption{Latin tense and aspect}
    \label{tbl:latin-tense-aspect}
    \centering
    \begin{tabular}{@{}cccc@{}}
    \toprule
              & past       & present                  & future                  \\ \midrule
    imperfect & imperfect  & \multirow{2}{*}{present} & \multirow{2}{*}{future} \\
    simple    & perfect    &                          &                         \\
    perfect   & pluperfect & perfect                  & future perfect          \\ \bottomrule
\end{tabular}    
\end{table}

Similar fusion between categories is shown in the category of \concept{mood}.
It's the fusion of morphologically marked clause type 
(declarative and imperative)
and morphologically marked modality.
\acs{blt} only calls the first category \term{mood}.
The verb morphology of interrogative clauses is exactly the same as declarative clauses:
the interrogative clause type is marked by the existence of interrogative \term{pro}-forms.
Thus, there are three moods in finite clauses in Latin:
\concept{indicative}, \concept{subjunctive}, and \concept{imperative}.
The indicative mood is the composition of 
the declarative/interrogative clause type and the realis modality.
The subjunctive mood is the composition of 
the declarative/interrogative clause type and the irrealis modality.
The imperative mood is basically the imperative clause type:
it doesn't allow modality marking.
Sometimes people say the infinitive is the fourth mood,
though it's a non-finite clause.

Latin is a typical nominative-accusative language,
both morphologically and syntactically.
In finite clauses, 
there is subject-verb agreement:
the number and person of the subject is marked on the main verb
(in the case of periphrastic conjugation,
the features are marked on the copula).

Finally I discuss the compatibility of these categories.
There is no future tense and future perfect tense in subjunctive clauses,
probably for the semantic reason
that the future tense already contains certain sense of modality
(an event predicted to happen),
and thus is not compatible with the subjunctive mood.
The imperative mood is not compatible with other \ac{tame} markings
except the present tense and the future tense.
It's still compatible with the voice category,
and allowed persons are 
second person singular/plural with the present tense,
and second/third person singular/plural with the future tense.
The absence of first person is also probably from semantic origin.

In conclusion, the verb paradigm of Latin is shown in \prettyref{fig:paradigm-finite-verb}.
The exact realization is divided into four conjugation classes,
and the details are too complex to show here.

\begin{figure}
    \centering
    

\tikzset{every picture/.style={line width=0.75pt}} %set default line width to 0.75pt        

\begin{tikzpicture}[x=0.75pt,y=0.75pt,yscale=-0.8,xscale=0.8]
%uncomment if require: \path (0,560); %set diagram left start at 0, and has height of 560

%Straight Lines [id:da4771979505475994] 
\draw [color={rgb, 255:red, 208; green, 2; blue, 27 }  ,draw opacity=1 ][line width=2.25]    (72,476.59) -- (165,476.59) ;
%Shape: Rectangle [id:dp4651809120211663] 
\draw  [draw opacity=0][fill={rgb, 255:red, 208; green, 2; blue, 27 }  ,fill opacity=0.1 ] (72,51.93) -- (165,51.93) -- (165,195.59) -- (72,195.59) -- cycle ;
%Shape: Rectangle [id:dp8674436914111818] 
\draw  [draw opacity=0][fill={rgb, 255:red, 245; green, 166; blue, 35 }  ,fill opacity=0.1 ] (208,51.93) -- (324,51.93) -- (324,195.59) -- (208,195.59) -- cycle ;
%Straight Lines [id:da9097932489630769] 
\draw [color={rgb, 255:red, 245; green, 166; blue, 35 }  ,draw opacity=1 ][line width=2.25]    (210,476.59) -- (324,476.59) ;
%Shape: Rectangle [id:dp6458311390526075] 
\draw  [draw opacity=0][fill={rgb, 255:red, 245; green, 166; blue, 35 }  ,fill opacity=0.1 ] (208,213.93) -- (324,213.93) -- (324,317.59) -- (208,317.59) -- cycle ;
%Shape: Rectangle [id:dp4418485190624626] 
\draw  [draw opacity=0][fill={rgb, 255:red, 208; green, 2; blue, 27 }  ,fill opacity=0.1 ] (73,213.93) -- (166,213.93) -- (166,318.26) -- (73,318.26) -- cycle ;
%Shape: Rectangle [id:dp7155013561532602] 
\draw  [draw opacity=0][fill={rgb, 255:red, 245; green, 166; blue, 35 }  ,fill opacity=0.1 ] (208,341.59) -- (324,341.59) -- (324,373.59) -- (208,373.59) -- cycle ;
%Shape: Rectangle [id:dp5109073549880265] 
\draw  [draw opacity=0][fill={rgb, 255:red, 245; green, 166; blue, 35 }  ,fill opacity=0.1 ] (208,387.59) -- (324,387.59) -- (324,419.59) -- (208,419.59) -- cycle ;
%Shape: Rectangle [id:dp4978185928249894] 
\draw  [draw opacity=0][fill={rgb, 255:red, 208; green, 2; blue, 27 }  ,fill opacity=0.1 ] (73,341.59) -- (166,341.59) -- (166,420.93) -- (73,420.93) -- cycle ;
%Shape: Rectangle [id:dp7974586880324295] 
\draw  [draw opacity=0][fill={rgb, 255:red, 126; green, 211; blue, 33 }  ,fill opacity=0.1 ] (364,52.93) -- (471.33,52.93) -- (471.33,419.59) -- (364,419.59) -- cycle ;
%Straight Lines [id:da40576537799721035] 
\draw [color={rgb, 255:red, 126; green, 211; blue, 33 }  ,draw opacity=1 ][line width=2.25]    (364,476.59) -- (470,476.59) ;
%Shape: Rectangle [id:dp37304865952309374] 
\draw  [draw opacity=0][fill={rgb, 255:red, 80; green, 227; blue, 194 }  ,fill opacity=0.1 ] (515,51.93) -- (631,51.93) -- (631,318.93) -- (515,318.93) -- cycle ;
%Shape: Rectangle [id:dp09424442237743991] 
\draw  [draw opacity=0][fill={rgb, 255:red, 80; green, 227; blue, 194 }  ,fill opacity=0.1 ] (514,341.59) -- (630,341.59) -- (630,373.59) -- (514,373.59) -- cycle ;
%Shape: Rectangle [id:dp7879043441622926] 
\draw  [draw opacity=0][fill={rgb, 255:red, 80; green, 227; blue, 194 }  ,fill opacity=0.1 ] (514,387.59) -- (630,387.59) -- (630,419.59) -- (514,419.59) -- cycle ;
%Straight Lines [id:da6694037096709569] 
\draw [color={rgb, 255:red, 80; green, 227; blue, 194 }  ,draw opacity=1 ][line width=2.25]    (514,476.59) -- (628,476.59) ;
%Shape: Rectangle [id:dp8356914669291959] 
\draw  [draw opacity=0][fill={rgb, 255:red, 74; green, 144; blue, 226 }  ,fill opacity=0.1 ] (675,52.93) -- (782.33,52.93) -- (782.33,419.59) -- (675,419.59) -- cycle ;
%Straight Lines [id:da18180972606397305] 
\draw [color={rgb, 255:red, 74; green, 144; blue, 226 }  ,draw opacity=1 ][line width=2.25]    (676.33,476.59) -- (782.33,476.59) ;

% Text Node
\draw (118.5,123.76) node   [align=left] {indicative};
% Text Node
\draw (119.5,266.09) node   [align=left] {subjunctive};
% Text Node
\draw (266,123.76) node   [align=left] {present/\\imperfect/\\future/\\perfect/\\pluperfect/\\future perfect};
% Text Node
\draw (266,265.76) node   [align=left] {present/\\imperfect/\\perfect/\\pluperfect};
% Text Node
\draw (417.67,236.26) node   [align=left] {active/\\passive};
% Text Node
\draw (573,185.43) node   [align=left] {1/\\2/\\3};
% Text Node
\draw (728.67,236.26) node  [color={rgb, 255:red, 0; green, 0; blue, 0 }  ,opacity=1 ] [align=left] {single/\\plural};
% Text Node
\draw (119.5,381.26) node   [align=left] {imperative};
% Text Node
\draw (266,357.59) node   [align=left] {present};
% Text Node
\draw (266,403.59) node   [align=left] {future};
% Text Node
\draw (572,357.59) node   [align=left] {2};
% Text Node
\draw (572,403.59) node   [align=left] {2/3};
% Text Node
\draw (118.5,479.59) node [anchor=north] [inner sep=0.75pt]  [color={rgb, 255:red, 208; green, 2; blue, 27 }  ,opacity=1 ] [align=left] {mood};
% Text Node
\draw (267,479.59) node [anchor=north] [inner sep=0.75pt]  [color={rgb, 255:red, 245; green, 166; blue, 35 }  ,opacity=1 ] [align=left] {\textcolor[rgb]{0.96,0.65,0.14}{tense}};
% Text Node
\draw (417,479.59) node [anchor=north] [inner sep=0.75pt]  [color={rgb, 255:red, 126; green, 211; blue, 33 }  ,opacity=1 ] [align=left] {voice};
% Text Node
\draw (571,479.59) node [anchor=north] [inner sep=0.75pt]  [color={rgb, 255:red, 80; green, 227; blue, 194 }  ,opacity=1 ] [align=left] {\textcolor[rgb]{0.31,0.89,0.76}{person}};
% Text Node
\draw (729.33,479.59) node [anchor=north] [inner sep=0.75pt]  [color={rgb, 255:red, 74; green, 144; blue, 226 }  ,opacity=1 ] [align=left] {number};


\end{tikzpicture}

    \caption{The paradigm of finite verb forms}
    \label{fig:paradigm-finite-verb}
\end{figure}

\subsubsection{Non-finite verb forms}

Non-finite verb forms -- which morphologically mark non-finite clause types --
include the infinitives, the participles, the gerund, the gerundive, and the supine.
The gerundive is also a kind of participles. 
The infinitives, generally speaking, are more ``verb-like'' than the rest of non-finite forms.

\subsubsection{Argument positions: core and argument}

There is no serial verb constructions in Latin (\prettyref{sec:clause-combine-abs}),
and thus semantic functions like location or instrument 
are always realized by typical peripheral arguments
attached to the core argument structure.
These peripheral argument positions sometimes can be filled by adverbs,
which also reveals an origin of adverbs.

\subsection{Constituent order}\label{sec:constituent-order-abs}

\subsection{Clause combining}\label{sec:clause-combine-abs}

In Latin there is no serial verb constructions.
Subordination strategies can be neatly summarized into 
complement clauses, relative clauses and adverbial clauses.

\section{Nominal morphology}\label{sec:nominal-morphology}

This section is about the morphology of nominal lexical categories, i.e. nouns and adjectives.
Their morphologies are clearly related.

\subsection{Overview}

Latin has five noun declension classes.
Every introductory book or reference grammar covers all of them and it makes no sense to repeat them here.
Relevant grammatical categories are gender, case, and number (\prettyref{sec:nominal-inflection-abs}), 
and for personal pronouns, person (\prettyref{sec:pro-abs}).
The person category is always idiosyncratic:
there is no regular person ending that can be distinguished.
The gender category is idiosyncratic for nouns but not so for adjectives:
adjectives agree with the related % TODO: ref to agreement relations

\subsubsection{Cases (and where to find them)}\label{sec:case}

\begin{itemize}
    \item \emph{Nominative}: % TODO: ref
    \item \emph{Accusative}: 
    \begin{itemize}
        \item The object.
    \end{itemize}
    \item \emph{Genitive}
    \begin{itemize}
        \item 
    \end{itemize}
    \item \emph{Dative}
    \begin{itemize}
        \item \concept{Dative of agent}, which is the agent in the periphrastic passive construction 
        (\prettyref{sec:passive-periphrastic}). 
    \end{itemize}
\end{itemize}

\subsubsection{Gender}



\subsection{The five regular noun declensions}\label{sec:regular-noun-declension}

Unlike the case in verb conjugation (\prettyref{sec:three-latin-stem}), 
there is only one stem for Latin noun declension.
An additional piece of information about declension class 
is also required to have the correct declined forms.
Therefore, the way nouns are stored in the dictionary is not storing the stem and the declension class,
but storing the nominative singular form and the genitive singular form,
the latter bearing a different ending for each declension.

\subsubsection{The first declension}

The stem of first declension nouns ends in \corpus{\={a}-}.
First declension nouns are mostly feminine.
Exceptions include family or personal names,

\subsubsection{The fourth declension}

\begin{itemize}
    \item \corpus{Et cum spiritu tuo}: from the fact that \corpus{tuo} is ablative,
    it can be inferred that \corpus{-u} is the ablative singular ending.
\end{itemize}

\subsection{Declensions of adjectives}\label{sec:regular-adjective-declension}

\subsection{Pronouns}\label{sec:pronoun}

Note that \term{pro}-adverbs or interrogative adverbs are not included in this section.
They are to be discussed in 

\subsubsection{Personal pronouns and reflexive pronouns}

How to remember the second person pronouns:
\begin{itemize}
    \item From \corpus{qui propter nos homines \dots}, we find \corpus{nos} is the accusative plural.
\end{itemize}

How to remember the third person reflexive pronouns:
\begin{itemize}
    \item The singular forms are identical to the plural forms.
    \item From \corpus{per s\={e}} and the fact that \corpus{per} is an accusative preposition,
    it can be seen that \corpus{s\={e}} is the accusative form. 
    \item \corpus{s\={e}cum}: \corpus{cum} is an ablative preposition, 
    and hence \corpus{s\={e}} is ablative.
\end{itemize}

\subsubsection{Possessive pronouns}

Possessive pronouns are much more adjectival than personal pronouns,
and their declensions are also much more regular.

\subsection{Participles and gerunds}\label{sec:participle-gerund}

\begin{itemize}
    \item \emph{The present active participle (i.e. the present participle)}: 
    replace the \corpus{-re} ending of the present active infinitive by \corpus{-nt}
    (or in other words, add \corpus{-nt} to the present stem)
    and the result is the nominative. % TODO: gender, and third declension
    \item \emph{The perfect passive participle (i.e. the perfect participle or the past participle)}:
    this can be found by decline the neutral accusative past participle, 
    i.e. the fourth principal part.
    \item \emph{The future active participle (i.e. the future participle)}:
    
    \item \emph{}
\end{itemize}

\subsection{Derivation} % TODO: from noun to noun

\subsubsection{Diminutive}

\section{Noun phrases}

\section{Verb morphology}



\subsection{The template of Latin verb}\label{sec:verb-template}

\begin{figure}
    \centering
    

\tikzset{every picture/.style={line width=0.3pt}} %set default line width to 0.75pt        

\begin{tikzpicture}[x=0.75pt,y=0.75pt,yscale=-0.85,xscale=0.85]
%uncomment if require: \path (0,414); %set diagram left start at 0, and has height of 414

%Shape: Rectangle [id:dp7661505767193904] 
\draw  [color={rgb, 255:red, 74; green, 144; blue, 226 }  ,draw opacity=1 ] (200,155) -- (278.01,155) -- (278.01,201.48) -- (200,201.48) -- cycle ;

%Shape: Rectangle [id:dp9235304615883395] 
\draw  [color={rgb, 255:red, 74; green, 144; blue, 226 }  ,draw opacity=1 ] (51,155) -- (187.01,155) -- (187.01,201.48) -- (51,201.48) -- cycle ;

%Shape: Rectangle [id:dp4025769445021785] 
\draw  [color={rgb, 255:red, 74; green, 144; blue, 226 }  ,draw opacity=1 ] (290,155) -- (416.01,155) -- (416.01,201.48) -- (290,201.48) -- cycle ;

%Shape: Rectangle [id:dp9180138973853116] 
\draw  [color={rgb, 255:red, 80; green, 227; blue, 194 }  ,draw opacity=1 ] (428,155) -- (564.01,155) -- (564.01,201.48) -- (428,201.48) -- cycle ;

%Shape: Rectangle [id:dp2406875282163703] 
\draw  [color={rgb, 255:red, 80; green, 227; blue, 194 }  ,draw opacity=1 ] (575,155) -- (722.01,155) -- (722.01,201.48) -- (575,201.48) -- cycle ;


% Text Node
\draw (239.01,178.24) node   [align=left] {core stem};
% Text Node
\draw (119.01,178.24) node   [align=left] {\begin{minipage}[lt]{87.06pt}\setlength\topsep{0pt}
\begin{center}
derivation \\prefix/compouding
\end{center}

\end{minipage}};
% Text Node
\draw (353.01,178.24) node   [align=left] {stem-final vowel};
% Text Node
\draw (496.01,178.24) node   [align=left] {\begin{minipage}[lt]{83.46pt}\setlength\topsep{0pt}
\begin{center}
tense and mood \\marking
\end{center}

\end{minipage}};
% Text Node
\draw (648.51,178.24) node   [align=left] {\begin{minipage}[lt]{84.5pt}\setlength\topsep{0pt}
\begin{center}
person, number, \\and voice marking
\end{center}

\end{minipage}};
% Text Node
\draw (181,228) node [anchor=north west][inner sep=0.75pt]  [color={rgb, 255:red, 74; green, 144; blue, 226 }  ,opacity=1 ] [align=left] {verb stem};
% Text Node
\draw (532.01,228) node [anchor=north west][inner sep=0.75pt]  [color={rgb, 255:red, 80; green, 227; blue, 194 }  ,opacity=1 ] [align=left] {verb ending};


\end{tikzpicture}

    \caption{The template of Latin verbs}
    \label{fig:latin-verb}
\end{figure}

The structure of a Latin verb can be roughly represented by \prettyref{fig:latin-verb}.
Derivation in Latin is predominantly preverbal,
and hence the conjugation is mostly about the final lexical morpheme in the verb stem.
The stem-final vowel is sometimes considered as a part of the stem,
and sometimes as a part of the verb ending.
It is the residue of the \ac{pie} stem suffix, which is after the core stem and before the conjugation ending,
and still has morphosyntactic alternation as well as phonological ones in Latin. % TODO: ref
The uncontroversial components of the verb ending include 
the \concept{tense and mood marker},
and the person, number and voice marker 
(here after the \concept{personal ending}, 
following the terminology in \citet[\citesec{165}]{allen1903allen}).
But the tense and mood marker is influenced by the personal ending:
the same tense and mood may be marked by one marker under one person and one number
but by another under another case.
The inverse is also true.
There is also phonological interaction between the two components of the ending.

\subsection{The structure of verb stems}

\subsubsection{The three verb stems}\label{sec:three-latin-stem}

Prototypically, the verb conjugation in a language is described by 
a series of morphological devices that take \emph{the} verb stem as the input,
and give conjugated verb forms as the final product.
This is indeed the case for Latin nouns (\prettyref{sec:regular-noun-declension})
and for English regular verbs:
the infinitive form is taken in,
and third-person singular \corpus{-s}, past tense \corpus{-ed}, 
past participle \corpus{-ed}, and the gerund-participle \corpus{-ing}
are attached according to the syntactic environment.
Sometimes the process is a little more irregular but not that irregular:
\emph{several} stems can be identified, each of which is fed into different morphosyntactic machines.
In other words, we have irregular stem alternation.
Again, for English irregular verbs,
there are three stems: the infinitive stem (e.g. \corpus{go}), 
the preterite stem (e.g. \corpus{went})
and the past participle stem (e.g. \corpus{gone}).
The step to feed stems into morphosyntactic machine is irregular,
but everything else is regular:
irregular, in this case, does appear, but it appear \emph{regularly}:
it only appears in certain parts.

This phenomenon -- that a verb has more than one stem, i.e. irregular stem alternation
-- is frequent cross-linguistically
(\citealt{jacques2021grammar} \citesec{12.2}, \citealt{forker2020grammar} \citesec{11.2}, among others).
Usually, certain correlation between the stem varieties can still be recognized,
and verbs can be grouped accordingly,
which, if the linguist truly will, can be (tediously) summarized as more fine-grained conjugation classes.

This is also the case for Latin verbs.
The irregularity of stem alternation is so prevalent
that if the conjugation paradigm of a verb can be described with a few stems,
the verb is deemed as regular, 
despite the fact that such verbs are obviously irregular by the standard of English.
All forms mentioned in \prettyref{sec:verb-inflection-abs}
can be obtained by three stems \citep[\citesec{164}]{allen1903allen},
if the verb is regular:
\begin{itemize}
    \item \emph{The present stem}, which, after attached with proper endings, forms
    \begin{itemize}
        \item The present, imperfect, and future forms, indicative or subjunctive,
        active or passive. (There is no future or future perfect subjunctive).
        \item All the imperatives.
        \item The present infinitives, active and passive.
        \item The present participle, the gerundive, and the gerund.
    \end{itemize}
    \item \emph{The perfect stem}, which, after attached with proper endings, forms 
    \begin{itemize}
        \item The perfect, pluperfect, and future perfect active, indicative or subjunctive.
        Again, there is no future or future perfect subjunctive.
        Note that the passives are \emph{not} formed by the perfect stem.
        \item The perfect active infinitive. 
        (Or the perfective infinitive active, since infinitive is considered as a mood by some people.)
    \end{itemize}
    Note that the perfect passive participle is \emph{not} obtained from the perfect stem.
    \item \emph{The supine stem}, 
    which, after attached with proper endings or used together with proper forms of \corpus{sum},
    forms 
    \begin{itemize}
        \item The perfect passive participle, which, by being used with proper forms of \corpus{sum}, forms
        \begin{itemize}
            \item The perfect, pluperfect, and future perfect passive forms, indicative or subjunctive.
            Again, there is no future or future perfect subjunctive.
            This is periphrastic conjugation: it is done by using proper forms of \corpus{sum}
            with the perfect passive participle.
            \item The perfect infinitive passive.
        \end{itemize}
        \item The future active participle, which, used together with \corpus{esse},
        makes the future active infinitive.
        \item The future passive infinitive, by being used together with \corpus{īrī}.
    \end{itemize}
\end{itemize}
This process is summarized in \prettyref{fig:stem-to-form}.

In practice, the three stems aren't what stored in the dictionary,
for two reasons.
First, fluent users of a language often 
tend to \emph{not} anatomize the language in detail,
since ``everything is so natural'', 
and recording actually attested word forms is hence easier to do
compared to the morpheme-based approach.
Second, Latin has four conjugation types,
and hence the three stems themselves aren't sufficient to decide how to conjugate the verb:
more information is needed, 
and by storing already conjugated verb forms,
the conjugation class can be decided by observing the endings.
What are stored are the following \concept{principal forms},
from which the three stems and the conjugation class can be solved out
\citep[\citesec{172}]{allen1903allen}:
\begin{enumerate}
    \item \emph{The first-person present active indicative}: formed from the present stem.
    \item \emph{The present infinitive}: formed from the present stem. 
    By observing its ending, the conjugation class can be decided,
    and by comparing with the first principal form, 
    the present stem is obtained.
    \item \emph{The first-person perfect active indicative}: showing the perfect stem.
    \item \emph{The neutral accusative past participle}, i.e. the form of supine: showing the supine stem.
\end{enumerate}

\begin{sidewaysfigure}
    \centering
    

\tikzset{every picture/.style={line width=0.3pt}} %set default line width to 0.75pt        

\begin{tikzpicture}[x=0.75pt,y=0.75pt,yscale=-0.8,xscale=0.8]
%uncomment if require: \path (0,697); %set diagram left start at 0, and has height of 697

%Curve Lines [id:da8600925548352094] 
\draw [color={rgb, 255:red, 208; green, 2; blue, 27 }  ,draw opacity=1 ]   (289.01,269.33) .. controls (329.01,239.33) and (422.01,193.33) .. (676.01,187.33) ;
\draw [shift={(676.01,187.33)}, rotate = 178.65] [fill={rgb, 255:red, 208; green, 2; blue, 27 }  ,fill opacity=1 ][line width=0.08]  [draw opacity=0] (12,-3) -- (0,0) -- (12,3) -- cycle    ;
%Curve Lines [id:da7478415205524331] 
\draw [color={rgb, 255:red, 208; green, 2; blue, 27 }  ,draw opacity=1 ]   (275.01,266.33) .. controls (256.2,201.98) and (244.25,196.43) .. (205.2,166.25) ;
\draw [shift={(204.01,165.33)}, rotate = 37.78] [fill={rgb, 255:red, 208; green, 2; blue, 27 }  ,fill opacity=1 ][line width=0.08]  [draw opacity=0] (12,-3) -- (0,0) -- (12,3) -- cycle    ;
%Curve Lines [id:da22329313492102099] 
\draw [color={rgb, 255:red, 208; green, 2; blue, 27 }  ,draw opacity=1 ]   (241.01,279.33) .. controls (203.2,213.66) and (164.4,199.47) .. (101.95,158.94) ;
\draw [shift={(101.01,158.33)}, rotate = 33.06] [fill={rgb, 255:red, 208; green, 2; blue, 27 }  ,fill opacity=1 ][line width=0.08]  [draw opacity=0] (12,-3) -- (0,0) -- (12,3) -- cycle    ;
%Curve Lines [id:da9720425555739067] 
\draw [color={rgb, 255:red, 208; green, 2; blue, 27 }  ,draw opacity=1 ]   (250.01,319.22) .. controls (250.01,411.38) and (358.91,534.94) .. (466.39,595.28) ;
\draw [shift={(468.01,596.18)}, rotate = 209.05] [fill={rgb, 255:red, 208; green, 2; blue, 27 }  ,fill opacity=1 ][line width=0.08]  [draw opacity=0] (12,-3) -- (0,0) -- (12,3) -- cycle    ;
%Curve Lines [id:da9116874614549022] 
\draw [color={rgb, 255:red, 208; green, 2; blue, 27 }  ,draw opacity=1 ]   (265.01,312.33) .. controls (276.01,374.96) and (344.01,487.96) .. (487.01,511.96) ;
\draw [shift={(487.01,511.96)}, rotate = 189.53] [fill={rgb, 255:red, 208; green, 2; blue, 27 }  ,fill opacity=1 ][line width=0.08]  [draw opacity=0] (12,-3) -- (0,0) -- (12,3) -- cycle    ;
%Curve Lines [id:da9399420006349646] 
\draw [color={rgb, 255:red, 208; green, 2; blue, 27 }  ,draw opacity=1 ]   (277.01,310.33) .. controls (320.79,382.6) and (345.76,424.54) .. (456.34,434.81) ;
\draw [shift={(458.01,434.96)}, rotate = 185.1] [fill={rgb, 255:red, 208; green, 2; blue, 27 }  ,fill opacity=1 ][line width=0.08]  [draw opacity=0] (12,-3) -- (0,0) -- (12,3) -- cycle    ;
%Curve Lines [id:da863359914759456] 
\draw [color={rgb, 255:red, 248; green, 231; blue, 28 }  ,draw opacity=1 ]   (404.01,268.33) .. controls (396.09,243.58) and (393.07,211.97) .. (403.69,164.76) ;
\draw [shift={(404.01,163.33)}, rotate = 102.91] [fill={rgb, 255:red, 248; green, 231; blue, 28 }  ,fill opacity=1 ][line width=0.08]  [draw opacity=0] (12,-3) -- (0,0) -- (12,3) -- cycle    ;
%Curve Lines [id:da44942461080998] 
\draw [color={rgb, 255:red, 248; green, 231; blue, 28 }  ,draw opacity=1 ]   (418.01,269.33) .. controls (457.81,239.48) and (590.67,220.74) .. (688.54,241.24) ;
\draw [shift={(690.01,241.55)}, rotate = 192.09] [fill={rgb, 255:red, 248; green, 231; blue, 28 }  ,fill opacity=1 ][line width=0.08]  [draw opacity=0] (12,-3) -- (0,0) -- (12,3) -- cycle    ;
%Curve Lines [id:da6857007576480554] 
\draw [color={rgb, 255:red, 245; green, 166; blue, 35 }  ,draw opacity=1 ]   (543.01,333) .. controls (619.63,368.19) and (641.79,399.53) .. (690.28,478.65) ;
\draw [shift={(691.01,479.85)}, rotate = 238.51] [fill={rgb, 255:red, 245; green, 166; blue, 35 }  ,fill opacity=1 ][line width=0.08]  [draw opacity=0] (12,-3) -- (0,0) -- (12,3) -- cycle    ;
%Curve Lines [id:da10519824034130609] 
\draw [color={rgb, 255:red, 245; green, 166; blue, 35 }  ,draw opacity=1 ]   (548.01,313.33) .. controls (610.7,314.32) and (660.51,321.26) .. (718.14,390.28) ;
\draw [shift={(719.01,391.33)}, rotate = 230.36] [fill={rgb, 255:red, 245; green, 166; blue, 35 }  ,fill opacity=1 ][line width=0.08]  [draw opacity=0] (12,-3) -- (0,0) -- (12,3) -- cycle    ;
%Curve Lines [id:da05946917136382068] 
\draw [color={rgb, 255:red, 245; green, 166; blue, 35 }  ,draw opacity=1 ]   (764.01,427.33) .. controls (931.01,427.33) and (930.01,58.33) .. (587.01,121.33) ;
\draw [shift={(587.01,121.33)}, rotate = 349.59] [fill={rgb, 255:red, 245; green, 166; blue, 35 }  ,fill opacity=1 ][line width=0.08]  [draw opacity=0] (12,-3) -- (0,0) -- (12,3) -- cycle    ;
%Curve Lines [id:da40444377743555227] 
\draw [color={rgb, 255:red, 245; green, 166; blue, 35 }  ,draw opacity=1 ]   (760.01,418.33) .. controls (782.67,400.12) and (788.83,381.88) .. (787.1,343.11) ;
\draw [shift={(787.01,341.33)}, rotate = 87.14] [fill={rgb, 255:red, 245; green, 166; blue, 35 }  ,fill opacity=1 ][line width=0.08]  [draw opacity=0] (12,-3) -- (0,0) -- (12,3) -- cycle    ;
%Curve Lines [id:da6236126078150841] 
\draw [color={rgb, 255:red, 245; green, 166; blue, 35 }  ,draw opacity=1 ]   (776.01,515.33) .. controls (822.31,499.24) and (848.23,480.56) .. (870.97,451.34) ;
\draw [shift={(872.01,450)}, rotate = 127.48] [fill={rgb, 255:red, 245; green, 166; blue, 35 }  ,fill opacity=1 ][line width=0.08]  [draw opacity=0] (12,-3) -- (0,0) -- (12,3) -- cycle    ;
%Curve Lines [id:da6624205121022633] 
\draw [color={rgb, 255:red, 245; green, 166; blue, 35 }  ,draw opacity=1 ]   (526.01,337) .. controls (548.9,359.55) and (593.56,477.99) .. (600.9,583.59) ;
\draw [shift={(601.01,585.18)}, rotate = 266.22] [fill={rgb, 255:red, 245; green, 166; blue, 35 }  ,fill opacity=1 ][line width=0.08]  [draw opacity=0] (12,-3) -- (0,0) -- (12,3) -- cycle    ;
%Curve Lines [id:da8468021797759513] 
\draw [color={rgb, 255:red, 245; green, 166; blue, 35 }  ,draw opacity=1 ]   (630.01,594.18) .. controls (720.56,591.2) and (773.48,577.14) .. (847.89,560.25) ;
\draw [shift={(849.01,560)}, rotate = 167.23] [fill={rgb, 255:red, 245; green, 166; blue, 35 }  ,fill opacity=1 ][line width=0.08]  [draw opacity=0] (12,-3) -- (0,0) -- (12,3) -- cycle    ;
%Shape: Ellipse [id:dp3626499664001579] 
\draw  [color={rgb, 255:red, 74; green, 144; blue, 226 }  ,draw opacity=1 ][fill={rgb, 255:red, 74; green, 144; blue, 226 }  ,fill opacity=0.1 ] (39,126.52) .. controls (39,83.15) and (167.94,48) .. (327.01,48) .. controls (486.07,48) and (615.01,83.15) .. (615.01,126.52) .. controls (615.01,169.88) and (486.07,205.03) .. (327.01,205.03) .. controls (167.94,205.03) and (39,169.88) .. (39,126.52) -- cycle ;
%Curve Lines [id:da19915198987505445] 
\draw [color={rgb, 255:red, 80; green, 227; blue, 194 }  ,draw opacity=1 ][fill={rgb, 255:red, 80; green, 227; blue, 194 }  ,fill opacity=0.2 ]   (435.01,413.18) .. controls (475.01,383.18) and (680.01,340.18) .. (752.01,382.18) .. controls (824.01,424.18) and (801.01,553.18) .. (717.01,571.18) .. controls (633.01,589.18) and (657.01,503.18) .. (615.01,482.18) .. controls (573.01,461.18) and (401.02,474.22) .. (435.01,413.18) -- cycle ;
%Shape: Ellipse [id:dp7283210175576866] 
\draw  [color={rgb, 255:red, 126; green, 211; blue, 33 }  ,draw opacity=1 ][fill={rgb, 255:red, 126; green, 211; blue, 33 }  ,fill opacity=0.1 ] (423,565.16) .. controls (423,524.84) and (473.15,492.14) .. (535.01,492.14) .. controls (596.86,492.14) and (647.01,524.84) .. (647.01,565.16) .. controls (647.01,605.49) and (596.86,638.18) .. (535.01,638.18) .. controls (473.15,638.18) and (423,605.49) .. (423,565.16) -- cycle ;
%Curve Lines [id:da8675081082501883] 
\draw [color={rgb, 255:red, 80; green, 227; blue, 194 }  ,draw opacity=1 ][fill={rgb, 255:red, 80; green, 227; blue, 194 }  ,fill opacity=0.1 ]   (363.01,378.55) .. controls (387.01,323.55) and (691.01,321.55) .. (763.01,363.55) .. controls (835.01,405.55) and (805.01,594.55) .. (721.01,612.55) .. controls (637.01,630.55) and (649.01,514.55) .. (540.01,525.55) .. controls (431.01,536.55) and (329.02,439.59) .. (363.01,378.55) -- cycle ;
%Shape: Polygon Curved [id:ds050164225991979006] 
\draw  [color={rgb, 255:red, 184; green, 233; blue, 134 }  ,draw opacity=1 ][fill={rgb, 255:red, 184; green, 233; blue, 134 }  ,fill opacity=0.1 ] (683.01,164.55) .. controls (770.01,100.55) and (871.01,192.55) .. (904.01,306.55) .. controls (937.01,420.55) and (940.76,404.18) .. (965.01,472.55) .. controls (989.26,540.93) and (945.01,634.55) .. (872.01,627.55) .. controls (799.01,620.55) and (851.01,517.55) .. (838.01,459.55) .. controls (825.01,401.55) and (798.01,367.55) .. (738.01,329.55) .. controls (678.01,291.55) and (596.01,228.55) .. (683.01,164.55) -- cycle ;

% Text Node
\draw (489,83) node [anchor=north west][inner sep=0.75pt]   [align=left] {perfect,\\pluperfect,\\future perfect\\passive};
% Text Node
\draw (359,74) node [anchor=north west][inner sep=0.75pt]   [align=left] {perfect,\\pluperfect,\\future perfect\\active};
% Text Node
\draw (175,100) node [anchor=north west][inner sep=0.75pt]   [align=left] {present,\\imperfect,\\future};
% Text Node
\draw (385,269.07) node [anchor=north west][inner sep=0.75pt]  [color={rgb, 255:red, 0; green, 0; blue, 0 }  ,opacity=1 ] [align=left] {perfect\\stem};
% Text Node
\draw (248,268.07) node [anchor=north west][inner sep=0.75pt]  [color={rgb, 255:red, 0; green, 0; blue, 0 }  ,opacity=1 ] [align=left] {present\\stem};
% Text Node
\draw (470,587.07) node [anchor=north west][inner sep=0.75pt]   [align=left] {gerund};
% Text Node
\draw (496,501.07) node [anchor=north west][inner sep=0.75pt]   [align=left] {gerundive};
% Text Node
\draw (59,130) node [anchor=north west][inner sep=0.75pt]   [align=left] {imperative};
% Text Node
\draw (688,166) node [anchor=north west][inner sep=0.75pt]   [align=left] {present\\infinitives};
% Text Node
\draw (467,412) node [anchor=north west][inner sep=0.75pt]   [align=left] {present\\participle};
% Text Node
\draw (501,292.07) node [anchor=north west][inner sep=0.75pt]   [align=left] {supine\\stem};
% Text Node
\draw (696,218) node [anchor=north west][inner sep=0.75pt]   [align=left] {perfect\\active\\infinitive};
% Text Node
\draw (579,586.07) node [anchor=north west][inner sep=0.75pt]   [align=left] {supine};
% Text Node
\draw (697,395) node [anchor=north west][inner sep=0.75pt]   [align=left] {perfect\\passive\\participle};
% Text Node
\draw (765,278) node [anchor=north west][inner sep=0.75pt]   [align=left] {perfect\\passive\\infinitive};
% Text Node
\draw (695,482) node [anchor=north west][inner sep=0.75pt]   [align=left] {future\\active\\participle};
% Text Node
\draw (872,382) node [anchor=north west][inner sep=0.75pt]   [align=left] {future\\active\\infinitive};
% Text Node
\draw (864,521) node [anchor=north west][inner sep=0.75pt]   [align=left] {future\\passive\\infinitive};
% Text Node
\draw (200,60) node [anchor=north west][inner sep=0.75pt]  [color={rgb, 255:red, 74; green, 144; blue, 226 }  ,opacity=1 ] [align=left] {finite forms};
% Text Node
\draw (594,405.14) node [anchor=north west][inner sep=0.75pt]  [color={rgb, 255:red, 80; green, 227; blue, 194 }  ,opacity=1 ] [align=left] {particle\\in narrow\\sense};
% Text Node
\draw (519,532.51) node [anchor=north west][inner sep=0.75pt]  [color={rgb, 255:red, 126; green, 211; blue, 33 }  ,opacity=1 ] [align=left] {"nominal"\\nonfinite\\forms};
% Text Node
\draw (389,356.14) node [anchor=north west][inner sep=0.75pt]  [color={rgb, 255:red, 80; green, 227; blue, 194 }  ,opacity=1 ] [align=left] {particle in broad sense};
% Text Node
\draw (775,228) node [anchor=north west][inner sep=0.75pt]  [color={rgb, 255:red, 184; green, 233; blue, 134 }  ,opacity=1 ] [align=left] {infinitives};


\end{tikzpicture}

    \caption{How to get all conjugation forms from the three stems}
    \label{fig:stem-to-form}
\end{sidewaysfigure}

The scheme the verb paradigm can be found in \citep[\citesec{166}]{allen1903allen}.
The following sections in the book discusses how to do local phonological adjustments 
to get the correct word form.
How each conjugation form is formulated can be found in \citep[\citesec{180}]{allen1903allen}.

\subsubsection{The present stem}

The present stem may be found by dropping \corpus{-re} in the present infinitive, 
i.e. the second participle part.
The structure of 

\subsubsection{Forming the perfect stem}

% TODO: the final vowel???

\subsubsection{Forming the supine stem}

\subsection{The four regular conjugations: morphemes}

Factors involved in conjugation have been introduced in \prettyref{sec:verb-inflection-abs}.
This section is about how conjugation is actually realized.
\prettyref{sec:tense-mood-marking} and \prettyref{sec:personal-marking}
discuss how to map the five factors determining verb conjugation 
to appropriate verb endings. 
For the inverse, i.e. mapping verb endings to the values of the five grammatical categories,
see \citet[\citesec{166}]{allen1903allen}.

\subsubsection{Marking of person, number, and voice in non-imperative finite forms}\label{sec:personal-marking}

Here I list possible personal endings for verbs that are indicative or subjunctive. 

\begin{itemize}
    \item The active:
    \begin{itemize}
        \item First-person singular: 
        \begin{itemize}
            \item \corpus{-\={o}}: present indicative, 
            future indicative (first and second conjugations only), 
            future perfect indicative.
            \item \corpus{-m}: imperfect indicative, 
            future indicative (third and fourth conjugations only),
            pluperfect indicative,
            subjunctive regardless of tense.
            \item \corpus{-ī}: perfect indicative.
        \end{itemize}
        \item Second-person singular:
        \begin{itemize}
            \item \corpus{-s}: compatible with all tenses and moods, except the perfect indicative.
            \item \corpus{-tī}: perfect indicative.
        \end{itemize}
        \item Third-person singular: \corpus{-t}, with all tenses and moods.
        \item First-person plural: \corpus{-mus}, with all tenses and moods.
        \item Second-person plural: \corpus{-tis}, with all tenses and moods.
        \item Third-person plural: \corpus{-nt}, with all tenses and moods.
    \end{itemize}
    \item The passive:
    \begin{itemize}
        \item First-person singular: 
        \begin{itemize}
            \item \corpus{-r}: compatible with all tenses and moods, except the present indicative.
            \item \corpus{-or}: present indicative.
            Also, note that the future indicative (first and second conjugations only) ending is \corpus{-bor},
            which may be analyzed as \corpus{-b-or}.
        \end{itemize}
        \item Second-person singular:
        \begin{itemize}
            \item \corpus{-ris}: compatible with all non-periphrastic tenses and moods.
            \item \corpus{-re}: alternative form of second-person singular compatible 
            with all non-periphrastic tenses and moods.
            If this personal ending is used, then the tense and mood marking is none.
            Note that the resulting verb form is the same as the infinitive participle part.
        \end{itemize}
        \item Third-person singular: \corpus{-tur}, with all non-periphrastic tenses and moods.
        \item First-person plural: \corpus{-mur}, with all non-periphrastic tenses and moods.
        \item Second-person plural: \corpus{-minī}, with all non-periphrastic tense and moods.
        \item Third-person plural: \corpus{-ntur}, with all non-periphrastic tenses and moods.
    \end{itemize}
\end{itemize}

Here are some tips to remember these endings:
\begin{itemize}
    \item 
\end{itemize}

\subsubsection{Marking of tense and mood in non-imperative finite forms}\label{sec:tense-mood-marking}

Tense and mood is marked by the corresponding morpheme in \prettyref{fig:latin-verb}
as well as certain vowel changes before the personal ending.

Note that the tense and mood marker is also subject to phonological rules % TODO: 考虑到b\={a}ris,这个肯定不是音系规则
(diachronic or synchronic).
The following rule is the most important one that apples in verb conjugation:
\begin{exe}
    \ex\label{ex:vowel-shortening} 
    A long vowel is shortened before \corpus{-m}, \corpus{-r}, \corpus{-t}, \corpus{-nt}, \corpus{-ntur}.
\end{exe}

\begin{itemize}
    \item The indicative:
    \begin{itemize}
        \item Present: zero suffixation, but there is change on the stem-final vowel:
        \begin{itemize}
            \item For first conjugation verbs, the final \corpus{\={a}} is dropped.
            \item For second conjugation verbs, the final \corpus{\={e}} $\to$ \corpus{e}.
            \item  
        \end{itemize}
        \item Imperfect: \corpus{-b\={a}-}, possibly shortened by \eqref{ex:vowel-shortening}.
        \item Future: 
        \begin{itemize}
            \item For first and second conjugation verbs, 
            the tense-mood morpheme is \corpus{-bi-}, except for 
            first-person singular (which is \corpus{-b-})
            and third-person plural (which is \corpus{-bu-}).
            \item For third and fourth conjugation verbs, change stem-final stem.
        \end{itemize}
        \item Perfect: 
        \begin{itemize}
            \item \corpus{-ī-}: first-person singular, third-person singular, first-person plural.
            Shortened by \eqref{ex:vowel-shortening} for the latter two.
            \item \corpus{-is-}: second-person singular, second-person plural.
            \item \corpus{-\={e}ru-}: third-person plural.
        \end{itemize}
        \item Pluperfect: \corpus{-er\={a}-}, possibly shortened by \eqref{ex:vowel-shortening}.
        \item Future perfect: 
        \begin{itemize}
            \item \corpus{-eri-} for all cases except first person singular.
            \item \corpus{-er-} for first-person singular.
        \end{itemize}
    \end{itemize}
    \item The subjunctive:
    \begin{itemize}
        \item Present: no suffixation, but there is regular change on the stem-final vowel:
        \begin{itemize}
            \item For first conjugation verbs, \corpus{\={a}-} $\to$ \corpus{\={e}-}.
            \item For second conjugation verbs, \corpus{} $\to$ \corpus{e\={a}-}.
            \item For third conjugation verbs, $\to$ \corpus{\={a}-}.
            \item For fourth conjugation verbs, $\to$ \corpus{i\={a}-}.
        \end{itemize}
        These alternations apply for both active and passive verbs,
        so they have nothing to do with polarity, and this is why I put them in this section.
        \item Imperfect: \corpus{-r\={e}-}, possibly shortened by \eqref{ex:vowel-shortening}.
        \item Perfect: \corpus{-eri-} for all non-periphrastic cases, i.e. active.
        \item Pluperfect: \corpus{-iss\={e}-} for all non-periphrastic cases, i.e. active.
        Possibly shortened by \eqref{ex:vowel-shortening}.
    \end{itemize}
\end{itemize}

At the first glance, it may be attempting as well to consider \corpus{-ba-} as 
the indicative imperfect suffix,
since \corpus{-b\={a}-} does not outnumber it.
However, note that 
The same line of argumentation can be applied to justify the status of \corpus{-r\={e}-}
as the somehow canonical subjunctive imperfect suffix.

\subsubsection{The imperative}\label{sec:imperative-morphology}

The imperative verb endings do not show the clear pattern 
in \prettyref{sec:personal-marking} and \prettyref{sec:tense-mood-marking}.
They are discussed here.

Latin imperatives are limited to second-person and third-person
for obvious semantic reasons.

\begin{itemize}
    \item The active:
    the present forms are
    \begin{itemize}
        \item Second-person singular present: zero suffixation. 
        This means the second-person present imperative coincide with the verb stem 
        (with stem-final vowel)
        -- but this isn't the generally accepted way to cite a verb, for some reasons.
        \item Second-person plural present: \corpus{-te}.
    \end{itemize}
    There is no third-person present imperative.

    The future forms are
    \begin{itemize}
        \item Second-person singular future: \corpus{-t\={o}}.
        \item Third-person singular future: \corpus{-t\={o}}.
        \item Second-person plural future: \corpus{-t\={o}te}.
        \item Third-person plural future: \corpus{-nt\={o}}.
    \end{itemize}
    \item The passive: 
    the present forms are
    \begin{itemize}
        \item Second-person singular present: \corpus{-re}. 
        This coincides with the second-person singular \corpus{-re},
        as well as the infinitive ending.
        \item Second-person plural present: \corpus{-minī}, 
        which coincides the passive second-person plural personal ending.
    \end{itemize}
    There is no third-person present imperative.

    The future forms are
    \begin{itemize}
        \item Second-person singular future: \corpus{-tor}.
        \item Third-person singular future: \corpus{-tor}.
        \item Third-person plural future: \corpus{-ntor}.
    \end{itemize}
    There is no second-personal plural future imperative.
\end{itemize}

A \emph{prohibition}, i.e. a command of \emph{not} doing something,
is usually not expressed by a negative imperative clause,
but rather, by a % TODO: 450节

\subsubsection{The infinitives}

\subsection{Peculiarities of the four conjugations}

\subsubsection{The first conjugation}

The 

\subsubsection{The second conjugation}

\subsubsection{The third conjugation}

\subsubsection{Guide to parsing a regular verb}

When parsing a verb, the following rules may be followed:
\begin{enumerate}
    \item Check whether one of the following verb endings appears:
    \begin{itemize}
        \item \corpus{-ī}: first-person singular perfect active indicative.
        \item \corpus{-istī}: second-person singular perfect active indicative.
        \item \corpus{-m}: first-person singular perfect active indicative, 
        or % TODO:
    \end{itemize}
    If one of them appears, then the values of all grammatical categories can be identified
    and the parsing is finished.
    \item Find the personal ending (\prettyref{sec:personal-marking}).
    If no ending can be found, 
    check whether the verb is imperative (\prettyref{sec:imperative-morphology}).
    If the verb is imperative then the parsing is also finished.
\end{enumerate}

\subsection{Important irregular verbs}\label{sec:irregular-verb}

Certain irregular verbs require some additional remarks,
since they are frequently found as the last lexical root of derived verbs 
(i.e. the core stem in \prettyref{fig:latin-verb}),
and since Latin verb conjugation is mainly decided by the core stem (\prettyref{sec:verb-template}),
any verb derived from these verbs is irregular.

\subsubsection{The verb \corpus{sum}}\label{sec:sum-verb}

The verb \corpus{sum} `be' is useful in several constructions,
including % TODO
It is defective for obvious semantic reasons
and is highly irregular,
as copulas usually are cross-linguistically.

\subsubsection{The verb \corpus{e\={o}}}

\subsection{Periphrastic conjugations}

\subsubsection{The passive perfect, pluperfect and future perfect}

\subsubsection{The periphrastic passive construction}\label{sec:passive-periphrastic}

\subsection{Derivation by compounding with particles}

\bibliographystyle{plainnat}
\bibliography{latin-notes}

\end{document}