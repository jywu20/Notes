\documentclass[a4paper]{article}

\usepackage{geometry}
\usepackage{caption}
\usepackage{footnote}
\usepackage{marginnote}
\usepackage{subcaption}
\usepackage{abstract}
% \usepackage{paralist}
\usepackage{amsmath, amssymb}
\usepackage{qtree}
\usepackage{gb4e}
\usepackage[colorlinks, urlcolor=cyan]{hyperref}
\usepackage{prettyref}

\geometry{left=3.18cm,right=3.18cm,top=2.54cm,bottom=2.54cm}

\title{The Structure of the extended NP}
\author{Jinyuan Wu}

\begin{document}

\maketitle

This article is about the cartography of the extended nominal projection, or in the words of more descriptive and 
functional approaches, the structure of the extended \emph{NP}. Under the context of generative syntax, NP is often 
used to denote a certain projection of a \emph{bare} noun phase. 

\section{Demonstratives}

Here we list some key facts in \cite{bruge2002positions}. First, in English we have 
\begin{exe}
    \ex\label{ex:english-dem-adj-n} this cute (*this) girl  
\end{exe}
so it seems like that demonstratives, as something resembling pronouns, are lower than any adjective,
and we need to note the fact that in English demonstratives have roles resembling determiners. However, in 
Romance languages, we can find a pattern which is illegal in English. For example, in Romanian, \marginnote{(2) in \cite{bruge2002positions}}
we have:
\begin{exe}
    \ex 
    \begin{xlist}
        \ex \label{ex:romanian-d-dem}
        \gll băiatul acesta (frumos) \\  
        boy-the this (nice) \\
        \ex \label{ex:romanian-dem-adj-n-adj}
        \gll acest (frumos) băiat (frumos) \\
        this (nice) boy (nice) \\
        \ex 
        \gll frumosul (*acesta) băiat \\
        nice-the (*this) boy \\ 
    \end{xlist}
\end{exe}
\eqref{ex:romanian-dem-adj-n-adj} has a similar structure with \eqref{ex:english-dem-adj-n} in English,
where the demonstrative takes the role of the determiner. However, in \eqref{ex:romanian-d-dem}, both 
the demonstrative and the determiner occur, which is illegal in English. Moreover, in \eqref{ex:romanian-d-dem},
we find that it seems the demonstrative is \emph{closer} to the core noun than the adjective. 



\bibliographystyle{plain}
\bibliography{cartography}

\end{document}