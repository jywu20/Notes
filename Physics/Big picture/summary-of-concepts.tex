\documentclass[a4paper]{article}

\usepackage{geometry}
\usepackage{titling}
\usepackage{titlesec}
\usepackage{paralist}
\usepackage{footnote}
\usepackage{enumerate}
\usepackage{amsmath, amssymb, amsthm}
\usepackage{cite}
\usepackage{graphicx}
\usepackage{subfigure}
\usepackage{physics}
\usepackage{tikz}
\usepackage[colorlinks, linkcolor=black, anchorcolor=black, citecolor=black]{hyperref}

\geometry{left=3.18cm,right=3.18cm,top=2.54cm,bottom=2.54cm}
\titlespacing{\paragraph}{0pt}{1pt}{10pt}[20pt]
\setlength{\droptitle}{-5em}
\preauthor{\vspace{-10pt}\begin{center}}
\postauthor{\par\end{center}}

\DeclareMathOperator{\timeorder}{T}
\newcommand*{\ii}{\mathrm{i}}
\newcommand*{\diff}{\mathop{}\!\mathrm{d}}
\newcommand*{\st}{\quad \text{s.t.} \quad}
\newcommand*{\const}{\mathrm{const}}
\newcommand*{\comment}{\paragraph{Comment}}

\newenvironment{bigcase}{\left\{\quad\begin{aligned}}{\end{aligned}\right.}

\title{A Brief Summary of Basic Concepts of Physics}
\author{Wu Jinyuan}

\begin{document}

\maketitle

\begin{abstract}
    Basic concepts of physics, from a modern - yet not too mathematical or technical - perspective.
\end{abstract}

When we are talking about physics, we are talking about a theory that 

There are two main systems to describe a physics system. We can 

\section{Mathematics}

\section{Lagrangian dynamics}

\section{Hamiltonian dynamics}

\section{Statics and coarse graining}

\end{document}