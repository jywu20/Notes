空间平移群是李群,它在$\mathcal{H}_{1\text{d}}$上的幺正表示$\hat{Q}(a)$满足以下特征:
\begin{itemize}
    \item $\hat{Q}(a) \ket{x} = \ket{x+a}$;
    \item $\hat{Q}(a+b) = \hat{Q}(a) \hat{Q}(b)$;
\end{itemize}
$\hat{Q}(a)$是幺正的意味着$\ii \dd{\hat{Q}} / \dd{a}$是厄米的,因此对应一个可观察量。
$\hat{Q}(a+b)=\hat{Q}(a) \hat{Q}(b)$则意味着$\dd{\hat{Q}} / \dd{a}$与群参数$a$无关,即变换是均匀的。
于是设
\begin{equation}
    \hat{Q}(\dd{x}) = \hat{I} + \frac{1}{\ii} \dd{x} \hat{p},
\end{equation}
其中$\hat{p}$是一个不显含任何参量的厄米算符。容易看出它具有动量量纲。
注意到
\[
    \begin{split}
        \hat{x} \hat{Q}(\dd{x'}) \ket{x'} = \hat{x} \ket{x' + \dd{x'}} = (x' + \dd{x'}) \ket{x' + \dd{x'}}, \\
        \hat{Q}(\dd{x'}) \hat{x} \ket{x'} = \hat{Q}(\dd{x'}) x' \ket{x'} = x' \hat{Q}(\dd{x'}) \ket{x'} = x' \ket{x' + \dd{x'}},
    \end{split}
\]
就有
\[
    [\hat{x}, \hat{Q}(\dd{x'})] \ket{x'} = \dd{x'} \ket{x' + \dd{x'}} \approx \dd{x'} \ket{x'}.
\]
考虑到$\ket{x'}$的任意性,我们得到
\[
    [\hat{x}, \hat{Q}(\dd{x'})] = \left[\hat{x}, \hat{I} + \frac{1}{\ii} \dd{x} \hat{p}\right] = \hat{I},
\]
从而
\begin{equation}
    [\hat{x}, \hat{p}] = \ii . 
    \label{eq:x-p-commutator-1d}   
\end{equation}
对易关系\eqref{eq:x-p-commutator-1d}完全确定了$\hat{x}$和$\hat{p}$的李代数结构。

下面推导$\hat{x}, \hat{p}$和任意物理量的对易关系。
设能够将物理量$\hat{F}$展开为$\hat{x}, \hat{p}$的多项式$\hat{F} = F(\hat{x}, \hat{p})$。
对其中的每一项,都可以使用对易关系\eqref{eq:x-p-commutator-1d}把$\hat{x}$挪到最前面而把$\hat{p}$挪到后面,
因此展开式最后就可以写成若干个$a \hat{x}^m \hat{p}^n$形式的项之和。
现在分析其中的一项:
\[
    [\hat{x}, \hat{x}^m \hat{p}^n] = \hat{x}^m [\hat{x}, \hat{p}^n] + [\hat{x}, \hat{x}^m] \hat{p}^n = \hat{x}^m [\hat{x}, \hat{p}^n],
\]
而
\[
    [\hat{x}, \hat{p}^n] = [\hat{x}, \hat{p} \hat{p}^{n-1}] = 
    \hat{p} [\hat{x}, \hat{p}^{n-1}] + [\hat{x}, \hat{p}] \hat{p}^{n-1} = \hat{p} [\hat{x}, \hat{p}^{n-1}] + \ii \hat{p}^{n-1}
\]
于是递推得到
\[
    [\hat{x}, \hat{p}^n] = \ii n \hat{p}^{n-1},
\]
因此
\[
    [\hat{x}, \hat{x}^m \hat{p}^n] = \ii n \hat{x}^m \hat{p}^{n-1}.
\]
这样就可以写出
\begin{equation}
    [\hat{x}, \hat{F}(\hat{x}, \hat{p})] = \ii \pdv{p} \hat{F}(\hat{x}, \hat{p}),
\end{equation}
在作用偏微分符号之前需要先把$F$中的每一项都变形成$\hat{x}$在前$\hat{p}$在后的形式。
使用同样的方法还可以导出
\begin{equation}
    [\hat{p}, \hat{F}(\hat{x}, \hat{p})] = - \ii \pdv{x} \hat{F}(\hat{x}, \hat{p}),
\end{equation}
同样,作用偏微分符号之前需要先把$F$中的每一项都变形成$\hat{x}$在前$\hat{p}$在后的形式。

在海森堡绘景下
\[
    \dv{\hat{A}}{t} = \frac{1}{\ii} [\hat{A}, H] + \pdv{\hat{A}}{t},
\]
于是
\[
    \dv{\hat{x}}{t} = \frac{1}{\ii} [\hat{x}, H] = \pdv{p} \hat{H}(\hat{x}, \hat{p}), \quad
    \dv{\hat{p}}{t} = \frac{1}{\ii} [\hat{p}, H] = -\pdv{x} \hat{H}(\hat{x}, \hat{p})
\]
当$\hbar \to 0$时,上式仍然成立,而此时$\hat{x}$和$\hat{p}$已经是对易的了,因此它们退化为了可以直接使用实数表示的情况,我们也就过渡到了经典力学。
这就是要求$\hat{p}$具有动量量纲的原因——它的确是经典动量的推广。

三维空间平移群在态空间上的幺正表示为$\hat{Q}(\vb*{a})$,同样还是满足与一维情况类似的几个条件:
\begin{itemize}
    \item 平移将位置算符的一个本征态变换为另一个:
    \begin{equation}
        \hat{Q}(\vb*{a}) \ket{\vb*{x}} = \ket{\vb*{x} + \vb*{a}}.
    \end{equation}
    \item 一次完成平移和多次完成是一样的:
    \begin{equation}
        \hat{Q}(\vb*{a}) \hat{Q}(\vb*{b}) = \hat{Q}(\vb*{a} + \vb*{b}).
        \label{eq:movement-plusable}
    \end{equation}
\end{itemize}
同样,\eqref{eq:movement-plusable}意味着生成元不显含群参数。
三维空间平移群本身是三维的,也就是说有三个生成元。不妨设这三个生成元是满足
\begin{equation}
    \begin{split}
        \hat{Q}(\vb*{e}_1 \dd{x_1}) = \hat{I} + \frac{1}{\ii \hbar} \hat{p}_1, \\
        \hat{Q}(\vb*{e}_2 \dd{x_2}) = \hat{I} + \frac{1}{\ii \hbar} \hat{p}_2, \\
        \hat{Q}(\vb*{e}_3 \dd{x_3}) = \hat{I} + \frac{1}{\ii \hbar} \hat{p}_3
    \end{split} 
\end{equation}
的不显含群参量的算符。显然它们是厄米的。
我们可以重复一维的操作步骤来获得三维空间平移群的生成元及其对易关系。

%%%%%%%%%%%%%%%%%%%%%%%%%%%%%%%%%%%%%%%%%%%%%%%%

记态密度为 % TODO:能级实际上总是离散的
\begin{equation}
    d = \dv{\Omega}{E}.
\end{equation}
由于\eqref{eq:total-energy},我们有
\[
    \dd{E_s} \dd{E_r} = \dd{E_s} (\dd{E} - \dd{E_s}) = \dd{E_s} \dd{E},
\]
于是
\[
    \begin{aligned}
        d_T (E_T) &= \int_{E_s + E_r = E_T} \dd{E_s} \dd{E_r} d_r (E_r) d_s (E_s) \\
        &= \int \dd{E_s} \dd{E} d_r (E_T - E_s) d_s (E_s),
    \end{aligned}
\]
从而
\[
    d_T (E_T) = \int \dd{E_s} d_r (E_T - E_s) d_s (E_s).
\]
由于热库远大于系统,$d_s$只有在$E_s$远小于$E_T$时才有比较大的值,因此
\begin{equation}
    d_T (E_T) = d_r (E_T).
\end{equation}
% 这个公式有什么用吗?

我们来分析系统取特定能量的概率。%
\footnote{请注意由于系统可以和热库有相互作用,系统的能量并不是固定的。当然,到平衡态时,系统的能量的期望是固定的。平衡态系统的能量在其期望附加有上下涨落。}%
我们称系统处于某个能量上的所有态的总和为\textbf{能级}。
首先假定系统的能级没有简并,或者各个能级的简并情况是相同的(也就是说,每个能级上的简并数相同)。%
\footnote{等价地说就是,系统的态空间可以写成能量空间直积上别的某些自由度。另一种等价的说法是,系统的哈密顿算符$\hat{H}$和别的某几个算符可以构成系统的一组CSCO。}%
由于系统和热库总能量恒定,我们有
\[
    P(E_s) \propto d_r (E_T - E_s),
\]
从而
\[
    \frac{P(E_s)}{P(E_{s'})} = \frac{d_r (E_T - E_s)}{d_r (E_T - E_{s'})}.
\]
等式右边是一个关于$E_T, E_s, E_{s'}$的表达式。因此我们成功地确定了一个事实:$E_s$的变动导致的概率的变动仅仅依赖于$E_s, E_{s'}$和$E_T$。
但实际上,$E_T$并不会改变概率比值。只要满足热库远大于系统,将环境中的什么部分算作热库是完全任意的,因此完全可以将环境中和原热库和系统没有相互作用的一部分算作热库,这增大了$E_T$,却对系统的行为没有任何影响。这表明我们有
\[
    \frac{P(E_s)}{P(E_{s'})} \propto f(E_s, E_{s'}).
\]
最后,注意到系统的能量可以整体地加上一个常数而不影响其行为%
\footnote{我们就是在这里用到了系统无简并或者各个能级的简并情况相同这一假设。如果系统不同能级上的简并数不同,那当我们整体地将能量加上一个常数时还需要调整对应能级的简并数才能够让系统行为保持不变。
例如,设能级$E_0$上的简并数为$n_1$,$E_0 + \Delta E$上的简并数为$n_2$,现在如果将能量整体地加上常数$\Delta E$,那需要重新规定能级$E_0 + \Delta E$上的简并数为$n_1$。}%
,因此上式右边只应该关于$E_s$和$E_{s'}$的差值,从而
\[
    \frac{P(E_s)}{P(E_{s'})} \propto f(E_s - E_{s'}).
\]
满足这个关系的唯一可能就是
\begin{equation}
    P(E_s) \propto \ee^{- \beta E_s}.
    \label{eq:gibbs-distribution-without-degenerate}
\end{equation}
这就是所谓的\textbf{玻尔兹曼分布}或者\textbf{吉布斯分布}。与环境交换物质能量的系统在平衡时一定会落在这个分布上。

%%%%%%%%%%%%%%%%%%%%%%%%%%%%%%%%%%%%%%%%%%%%%%%55

进一步,假定系统能级无简并。这样可以很容易地对\eqref{eq:gibbs-distribution-without-degenerate}做归一化,就有
\begin{equation}
    P(E_s) = \frac{1}{Z} \ee^{ - \beta E_s},
    \label{eq:distribution-of-energy-without-degenerate}
\end{equation}
其中
\begin{equation}
    Z = \sum_{E_s} \ee^{ - \beta E_s}
    \label{eq:partition-function-without-degenerate}
\end{equation}
就是在引入未归一化的密度算符时提到的配分函数。相应的,密度算符为
\begin{equation}
    \hat{\rho} = \sum_{E_s} P(E_s) \dyad{E_s} = \sum_{E_s} \ee^{-\beta E_s} \dyad{E_s} = \sum_n \ee^{-\beta E_n} \dyad{n}.
    \label{eq:density-operator-without-degenerate}
\end{equation}
其中$n$是能量本征态的标记,$E_n$指的是$\ket{n}$的本征值。

如果系统的能级有简并,这说明能量$\hat{H}$不足以成为描述系统的CSCO;还需要一些其它的算符来完整地描述系统。我们把它们打包设为$\hat{A}$。
于是,可以先在哈密顿量中引入微小的关于$\hat{A}$的扰动,即
\[
    \hat{H}' = \hat{H} + \lambda \hat{H}_1 (\hat{A}), \quad \lambda \ll 1,
\]
让简并消失,但与此同时不对态空间产生很大影响,此时\eqref{eq:density-operator-without-degenerate}就适用了。现在让$\lambda$趋于零,我们注意到\eqref{eq:density-operator-without-degenerate}中关于$E_s$的所有表达式在$\lambda$取0时会出现突变,因为此时本来能够分开的$E_s$突然变得不能分开了,但是关于$\ket{n}$的表达式的变化却是可微的。
因此有简并时的密度算符为%
\footnote{设算符$\hat{A}$可被谱展开为
\[
    \hat{A} = \sum_i A_i \dyad{i},
\]
则可以验证,一个解析函数作用在$\hat{A}$上的结果为
\[
    f(\hat{A}) = \sum_i f(A_i) \dyad{i}.
\]
因此即使函数$f$的性质不那么好,我们也规定上式成立。
}
\begin{equation}
    \hat{\rho} = \sum_n \ee^{-\beta E_n} \dyad{n} = \ee^{-\beta \hat{H}}.
    \label{eq:gibbs-density-operator}
\end{equation}
相应的,可以读出
\begin{equation}
    P(E_i) = \frac{1}{Z} \sum_{E_n = E_i} \mel{n}{\hat{\rho}}{n} = \frac{1}{Z} g_i \ee^{-\beta E_i},
\end{equation}
其中$E_i$指的是第$i$个能级的能量,$g_i$指的是第$i$个能级的简并度。相应的配分函数为
\begin{equation}
    Z = \sum_i g_i \ee^{- \beta E_i} = \sum_\text{eigenstate $\ket{n}$} \ee^{- \beta E_n} .
\end{equation}
第二个等号表示对所有能量本征态求和,当然,此时没有必要使用简并数因子,因为所有态,无论简并不简并,都已被考虑了。