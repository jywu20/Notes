空间平移群是李群,它在$\mathcal{H}_{1\text{d}}$上的幺正表示$\hat{Q}(a)$满足以下特征:
\begin{itemize}
    \item $\hat{Q}(a) \ket{x} = \ket{x+a}$;
    \item $\hat{Q}(a+b) = \hat{Q}(a) \hat{Q}(b)$;
\end{itemize}
$\hat{Q}(a)$是幺正的意味着$\ii \dd{\hat{Q}} / \dd{a}$是厄米的,因此对应一个可观察量。
$\hat{Q}(a+b)=\hat{Q}(a) \hat{Q}(b)$则意味着$\dd{\hat{Q}} / \dd{a}$与群参数$a$无关,即变换是均匀的。
于是设
\begin{equation}
    \hat{Q}(\dd{x}) = \hat{I} + \frac{1}{\ii} \dd{x} \hat{p},
\end{equation}
其中$\hat{p}$是一个不显含任何参量的厄米算符。容易看出它具有动量量纲。
注意到
\[
    \begin{split}
        \hat{x} \hat{Q}(\dd{x'}) \ket{x'} = \hat{x} \ket{x' + \dd{x'}} = (x' + \dd{x'}) \ket{x' + \dd{x'}}, \\
        \hat{Q}(\dd{x'}) \hat{x} \ket{x'} = \hat{Q}(\dd{x'}) x' \ket{x'} = x' \hat{Q}(\dd{x'}) \ket{x'} = x' \ket{x' + \dd{x'}},
    \end{split}
\]
就有
\[
    [\hat{x}, \hat{Q}(\dd{x'})] \ket{x'} = \dd{x'} \ket{x' + \dd{x'}} \approx \dd{x'} \ket{x'}.
\]
考虑到$\ket{x'}$的任意性,我们得到
\[
    [\hat{x}, \hat{Q}(\dd{x'})] = \left[\hat{x}, \hat{I} + \frac{1}{\ii} \dd{x} \hat{p}\right] = \hat{I},
\]
从而
\begin{equation}
    [\hat{x}, \hat{p}] = \ii . 
    \label{eq:x-p-commutator-1d}   
\end{equation}
对易关系\eqref{eq:x-p-commutator-1d}完全确定了$\hat{x}$和$\hat{p}$的李代数结构。

下面推导$\hat{x}, \hat{p}$和任意物理量的对易关系。
设能够将物理量$\hat{F}$展开为$\hat{x}, \hat{p}$的多项式$\hat{F} = F(\hat{x}, \hat{p})$。
对其中的每一项,都可以使用对易关系\eqref{eq:x-p-commutator-1d}把$\hat{x}$挪到最前面而把$\hat{p}$挪到后面,
因此展开式最后就可以写成若干个$a \hat{x}^m \hat{p}^n$形式的项之和。
现在分析其中的一项:
\[
    [\hat{x}, \hat{x}^m \hat{p}^n] = \hat{x}^m [\hat{x}, \hat{p}^n] + [\hat{x}, \hat{x}^m] \hat{p}^n = \hat{x}^m [\hat{x}, \hat{p}^n],
\]
而
\[
    [\hat{x}, \hat{p}^n] = [\hat{x}, \hat{p} \hat{p}^{n-1}] = 
    \hat{p} [\hat{x}, \hat{p}^{n-1}] + [\hat{x}, \hat{p}] \hat{p}^{n-1} = \hat{p} [\hat{x}, \hat{p}^{n-1}] + \ii \hat{p}^{n-1}
\]
于是递推得到
\[
    [\hat{x}, \hat{p}^n] = \ii n \hat{p}^{n-1},
\]
因此
\[
    [\hat{x}, \hat{x}^m \hat{p}^n] = \ii n \hat{x}^m \hat{p}^{n-1}.
\]
这样就可以写出
\begin{equation}
    [\hat{x}, \hat{F}(\hat{x}, \hat{p})] = \ii \pdv{p} \hat{F}(\hat{x}, \hat{p}),
\end{equation}
在作用偏微分符号之前需要先把$F$中的每一项都变形成$\hat{x}$在前$\hat{p}$在后的形式。
使用同样的方法还可以导出
\begin{equation}
    [\hat{p}, \hat{F}(\hat{x}, \hat{p})] = - \ii \pdv{x} \hat{F}(\hat{x}, \hat{p}),
\end{equation}
同样,作用偏微分符号之前需要先把$F$中的每一项都变形成$\hat{x}$在前$\hat{p}$在后的形式。

在海森堡绘景下
\[
    \dv{\hat{A}}{t} = \frac{1}{\ii} [\hat{A}, H] + \pdv{\hat{A}}{t},
\]
于是
\[
    \dv{\hat{x}}{t} = \frac{1}{\ii} [\hat{x}, H] = \pdv{p} \hat{H}(\hat{x}, \hat{p}), \quad
    \dv{\hat{p}}{t} = \frac{1}{\ii} [\hat{p}, H] = -\pdv{x} \hat{H}(\hat{x}, \hat{p})
\]
当$\hbar \to 0$时,上式仍然成立,而此时$\hat{x}$和$\hat{p}$已经是对易的了,因此它们退化为了可以直接使用实数表示的情况,我们也就过渡到了经典力学。
这就是要求$\hat{p}$具有动量量纲的原因——它的确是经典动量的推广。

三维空间平移群在态空间上的幺正表示为$\hat{Q}(\vb*{a})$,同样还是满足与一维情况类似的几个条件:
\begin{itemize}
    \item 平移将位置算符的一个本征态变换为另一个:
    \begin{equation}
        \hat{Q}(\vb*{a}) \ket{\vb*{x}} = \ket{\vb*{x} + \vb*{a}}.
    \end{equation}
    \item 一次完成平移和多次完成是一样的:
    \begin{equation}
        \hat{Q}(\vb*{a}) \hat{Q}(\vb*{b}) = \hat{Q}(\vb*{a} + \vb*{b}).
        \label{eq:movement-plusable}
    \end{equation}
\end{itemize}
同样,\eqref{eq:movement-plusable}意味着生成元不显含群参数。
三维空间平移群本身是三维的,也就是说有三个生成元。不妨设这三个生成元是满足
\begin{equation}
    \begin{split}
        \hat{Q}(\vb*{e}_1 \dd{x_1}) = \hat{I} + \frac{1}{\ii \hbar} \hat{p}_1, \\
        \hat{Q}(\vb*{e}_2 \dd{x_2}) = \hat{I} + \frac{1}{\ii \hbar} \hat{p}_2, \\
        \hat{Q}(\vb*{e}_3 \dd{x_3}) = \hat{I} + \frac{1}{\ii \hbar} \hat{p}_3
    \end{split} 
\end{equation}
的不显含群参量的算符。显然它们是厄米的。
我们可以重复一维的操作步骤来获得三维空间平移群的生成元及其对易关系。

%%%%%%%%%%%%%%%%%%%%%%%%%%%%%%%%%%%%%%%%%%%%%%%%

记态密度为 % TODO:能级实际上总是离散的
\begin{equation}
    d = \dv{\Omega}{E}.
\end{equation}
由于\eqref{eq:total-energy},我们有
\[
    \dd{E_s} \dd{E_r} = \dd{E_s} (\dd{E} - \dd{E_s}) = \dd{E_s} \dd{E},
\]
于是
\[
    \begin{aligned}
        d_T (E_T) &= \int_{E_s + E_r = E_T} \dd{E_s} \dd{E_r} d_r (E_r) d_s (E_s) \\
        &= \int \dd{E_s} \dd{E} d_r (E_T - E_s) d_s (E_s),
    \end{aligned}
\]
从而
\[
    d_T (E_T) = \int \dd{E_s} d_r (E_T - E_s) d_s (E_s).
\]
由于热库远大于系统,$d_s$只有在$E_s$远小于$E_T$时才有比较大的值,因此
\begin{equation}
    d_T (E_T) = d_r (E_T).
\end{equation}
% 这个公式有什么用吗?

我们来分析系统取特定能量的概率。%
\footnote{请注意由于系统可以和热库有相互作用,系统的能量并不是固定的。当然,到平衡态时,系统的能量的期望是固定的。平衡态系统的能量在其期望附加有上下涨落。}%
我们称系统处于某个能量上的所有态的总和为\textbf{能级}。
首先假定系统的能级没有简并,或者各个能级的简并情况是相同的(也就是说,每个能级上的简并数相同)。%
\footnote{等价地说就是,系统的态空间可以写成能量空间直积上别的某些自由度。另一种等价的说法是,系统的哈密顿算符$\hat{H}$和别的某几个算符可以构成系统的一组CSCO。}%
由于系统和热库总能量恒定,我们有
\[
    P(E_s) \propto d_r (E_T - E_s),
\]
从而
\[
    \frac{P(E_s)}{P(E_{s'})} = \frac{d_r (E_T - E_s)}{d_r (E_T - E_{s'})}.
\]
等式右边是一个关于$E_T, E_s, E_{s'}$的表达式。因此我们成功地确定了一个事实:$E_s$的变动导致的概率的变动仅仅依赖于$E_s, E_{s'}$和$E_T$。
但实际上,$E_T$并不会改变概率比值。只要满足热库远大于系统,将环境中的什么部分算作热库是完全任意的,因此完全可以将环境中和原热库和系统没有相互作用的一部分算作热库,这增大了$E_T$,却对系统的行为没有任何影响。这表明我们有
\[
    \frac{P(E_s)}{P(E_{s'})} \propto f(E_s, E_{s'}).
\]
最后,注意到系统的能量可以整体地加上一个常数而不影响其行为%
\footnote{我们就是在这里用到了系统无简并或者各个能级的简并情况相同这一假设。如果系统不同能级上的简并数不同,那当我们整体地将能量加上一个常数时还需要调整对应能级的简并数才能够让系统行为保持不变。
例如,设能级$E_0$上的简并数为$n_1$,$E_0 + \Delta E$上的简并数为$n_2$,现在如果将能量整体地加上常数$\Delta E$,那需要重新规定能级$E_0 + \Delta E$上的简并数为$n_1$。}%
,因此上式右边只应该关于$E_s$和$E_{s'}$的差值,从而
\[
    \frac{P(E_s)}{P(E_{s'})} \propto f(E_s - E_{s'}).
\]
满足这个关系的唯一可能就是
\begin{equation}
    P(E_s) \propto \ee^{- \beta E_s}.
    \label{eq:gibbs-distribution-without-degenerate}
\end{equation}
这就是所谓的\textbf{玻尔兹曼分布}或者\textbf{吉布斯分布}。与环境交换物质能量的系统在平衡时一定会落在这个分布上。

%%%%%%%%%%%%%%%%%%%%%%%%%%%%%%%%%%%%%%%%%%%%%%%55

进一步,假定系统能级无简并。这样可以很容易地对\eqref{eq:gibbs-distribution-without-degenerate}做归一化,就有
\begin{equation}
    P(E_s) = \frac{1}{Z} \ee^{ - \beta E_s},
    \label{eq:distribution-of-energy-without-degenerate}
\end{equation}
其中
\begin{equation}
    Z = \sum_{E_s} \ee^{ - \beta E_s}
    \label{eq:partition-function-without-degenerate}
\end{equation}
就是在引入未归一化的密度算符时提到的配分函数。相应的,密度算符为
\begin{equation}
    \hat{\rho} = \sum_{E_s} P(E_s) \dyad{E_s} = \sum_{E_s} \ee^{-\beta E_s} \dyad{E_s} = \sum_n \ee^{-\beta E_n} \dyad{n}.
    \label{eq:density-operator-without-degenerate}
\end{equation}
其中$n$是能量本征态的标记,$E_n$指的是$\ket{n}$的本征值。

如果系统的能级有简并,这说明能量$\hat{H}$不足以成为描述系统的CSCO;还需要一些其它的算符来完整地描述系统。我们把它们打包设为$\hat{A}$。
于是,可以先在哈密顿量中引入微小的关于$\hat{A}$的扰动,即
\[
    \hat{H}' = \hat{H} + \lambda \hat{H}_1 (\hat{A}), \quad \lambda \ll 1,
\]
让简并消失,但与此同时不对态空间产生很大影响,此时\eqref{eq:density-operator-without-degenerate}就适用了。现在让$\lambda$趋于零,我们注意到\eqref{eq:density-operator-without-degenerate}中关于$E_s$的所有表达式在$\lambda$取0时会出现突变,因为此时本来能够分开的$E_s$突然变得不能分开了,但是关于$\ket{n}$的表达式的变化却是可微的。
因此有简并时的密度算符为%
\footnote{设算符$\hat{A}$可被谱展开为
\[
    \hat{A} = \sum_i A_i \dyad{i},
\]
则可以验证,一个解析函数作用在$\hat{A}$上的结果为
\[
    f(\hat{A}) = \sum_i f(A_i) \dyad{i}.
\]
因此即使函数$f$的性质不那么好,我们也规定上式成立。
}
\begin{equation}
    \hat{\rho} = \sum_n \ee^{-\beta E_n} \dyad{n} = \ee^{-\beta \hat{H}}.
    \label{eq:gibbs-density-operator}
\end{equation}
相应的,可以读出
\begin{equation}
    P(E_i) = \frac{1}{Z} \sum_{E_n = E_i} \mel{n}{\hat{\rho}}{n} = \frac{1}{Z} g_i \ee^{-\beta E_i},
\end{equation}
其中$E_i$指的是第$i$个能级的能量,$g_i$指的是第$i$个能级的简并度。相应的配分函数为
\begin{equation}
    Z = \sum_i g_i \ee^{- \beta E_i} = \sum_\text{eigenstate $\ket{n}$} \ee^{- \beta E_n} .
\end{equation}
第二个等号表示对所有能量本征态求和,当然,此时没有必要使用简并数因子,因为所有态,无论简并不简并,都已被考虑了。

设$\hat{H}_0$是一个不显含时间的算符,我们将$\hat{H}^H$做如下的分解:
\begin{equation}
    \hat{H}^H = \hat{H}_0 + \hat{H}^H_i,
\end{equation}
并且令$\hat{A}^I$的演化方程为
\begin{equation}
    \dv{\hat{A}^I}{t} = \frac{1}{\ii \hbar} \comm*{\hat{A}^I}{\hat{H}_0},
\end{equation}
从而我们有
\begin{equation}
    \hat{A}^I (t) = \exp \left( \frac{\ii}{\hbar} \hat{H}_0 t \right) \hat{A}^I (0) \exp \left( - \frac{\ii}{\hbar} \hat{H}_0 t \right).
\end{equation}
与$\hat{A}^H$相对比,我们有
\[
    \hat{A}^I (t) = \exp\left(\frac{\ii}{\hbar}\hat{H}_0 t\right) T \exp \left( \frac{\ii}{\hbar} \int_0^t \dd{t} \hat{H}^H(t) \right)^\dagger \hat{A}^H(0) T \exp \left( \frac{\ii}{\hbar} \int_0^t \dd{t} \hat{H}^H(t) \right) \exp\left(-\frac{\ii}{\hbar}\hat{H}_0 t\right).
\]
与\eqref{eq:picture-trans}相比较有
\[
    \hat{Q} = T \exp \left( \frac{\ii}{\hbar} \int_{t_0}^t \dd{t} \hat{H}^H_i (t) \right),
\]
这样
\[
    \ket{\psi^H} = \hat{Q} \ket{\psi^I (t)} = T \exp \left( \frac{\ii}{\hbar} \int_{t_0}^t \dd{t} \hat{H}^H_i (t) \right) \ket{\psi^I (t)}.
\]
于是我们发现
\[
    \ket{\psi^I (0)} = \ket{\psi^H},
\]
且与推导薛定谔绘景类似地可以推导出
\[
    \ii \hbar \dv{t} \ket{\psi^I (t)} = T \exp \left( \frac{\ii}{\hbar} \int_{t_0}^t \dd{t} \hat{H}^H_i (t) \right)^\dagger \hat{H}_i^H (t) T \exp \left( \frac{\ii}{\hbar} \int_{t_0}^t \dd{t} \hat{H}^H_i (t) \right) \ket{\psi^I(t)},
\]
做一个绘景变换就得到
\begin{equation}
    \ii \hbar \dv{t} \ket{\psi^I (t)} = \hat{H}^I_i (t) \ket{\psi^I (T)}.
    
\end{equation}
相互作用绘景中算符随时间的演化使用的$\hat{H}_0$由于不含时,无论在海森堡绘景下还是在相互作用绘景下都是一样的,无需特殊的变换。
在$\hat{H}_i^H$在各个时间点的值彼此对易时,
\[
    \comm{T \exp \left( \frac{\ii}{\hbar} \int_{t_0}^t \dd{t} \hat{H}^H_i (t) \right)}{\hat{H}_i^H(t')} = 0,
\]
于是可以将$\hat{H}^I_i$和$\hat{H}^H_i$的关系式中的时间演化算符移动到一起,让它们抵消,从而证明$\hat{H}^I_i(t)$和$\hat{H}^H_i(t)$相等。

%%%%%%%%%%%%%%%%%%%%%%%%%%%%%%%%%%%%%%%%%%%%%%%%%%%%%%%%%%%%%%

\subsection{线性响应}\label{sec:linear-response}

\subsubsection{推迟格林函数}

设我们在体系中加入一个可观察量微扰$\hat{A}$:
\begin{equation}
    \hat{H}' = \hat{H} - h(t) \hat{A},
\end{equation}
其中$h(t)$是一个含时的系数。%
\footnote{如果我们使用$+ h(t) \hat{A}$形式的哈密顿量,计算得到的响应就会差一个负号。}%
我们要求$h(t)$被缓慢地施加,而又缓慢地被撤去,从而系统初态(也即,$t=-\infty$时的状态)可以看成平衡态。
记密度算符$\hat{\rho}$在$t=-\infty$时的状态为$\hat{\rho}_0$。
显然微扰会改变系统的行为。计算受到微扰的系统中某不含时的可观察量$\hat{B}$的期望偏离平衡态的程度(下标0表示这是对没有加过微扰的系统取平均,也就是按照$\hat{\rho}_0$取平均):
\[
    y(t) = \expval*{\hat{B}}(t) - \expval*{\hat{B}}_0 = \trace\left((\hat{\rho} - \hat{\rho}_0)\hat{B}\right).
\]
前面提到这是微扰,因此$y(t)$和$h(t)$的关系近似是线性的(这就是我们正在讨论的理论称为\textbf{线性响应}理论的原因),从而我们可以使用一个响应函数联系两者,即
\[
    y(t) = \int \dd{t'} G(t, t') h(t').
\]
我们要求系统具有因果律,这意味着,在$t<t'$时应有$G(t,t')=0$,否则某一时刻的$h(t')$将会影响过去时间的$y(t)$。因此我们有
\[
    G(t,t') \propto \Theta(t-t'),
\]
且积分可以写成
\[
    y(t) = \int_{-\infty}^t \dd{t'} G(t, t') h(t').
\]

切换到相互作用绘景下。取$\hat{H}$为自由哈密顿量,$-h(t)\hat{A}$为相互作用哈密顿量,那么演化方程就是
\[
    \dv{\hat{\rho}^I}{t} = \frac{\ii}{\hbar} \comm*{\hat{\rho}^I}{-h(t)\hat{A}^I(t)} = \frac{\ii}{\hbar} h(t) \comm*{\hat{A}^I(t)}{\hat{\rho}^I}.
\]
上式等价于积分方程
\[
    \hat{\rho}^I(t) = \hat{\rho}_0^I + \frac{\ii}{\hbar}  \int_{-\infty}^t \dd{t'} h(t') \comm*{\hat{A}^I(t')}{\hat{\rho}^I(t')}.
\]
事实上,$\hat{\rho}_0^I$就是$\hat{\rho}_0$,因为$\hat{\rho}_0$为
\[
    \hat{\rho}_0 = \frac{1}{Z} \ee^{-\beta \hat{H}},
\]
从而
\[
    \hat{\rho}_0^I = \ee^{-\frac{\ii}{\hbar}t \hat{H}} \hat{\rho}_0 \ee^{\frac{\ii}{\hbar}t \hat{H}} = \hat{\rho}_0.
\]
由于$h(t)$非常小,$\hat{\rho}^I$的变化不是特别大,于是取以上方程的一阶近似,得到
\[
    \hat{\rho}^I(t) = \hat{\rho}_0 + \frac{\ii}{\hbar}  \int_{-\infty}^t \dd{t'} h(t') \comm*{\hat{A}^I(t')}{\hat{\rho}_0}.
\]
于是可以计算出$y(t)$:%
\footnote{请注意迹运算无论是在相互作用绘景还是在薛定谔绘景下都是一样的。}
\[
    \begin{aligned}
        y(t) &= \trace \left\{ (\hat{\rho}^I(t) - \hat{\rho}^I_0) \hat{B}^I(t) \right\} \\
        &= \frac{\ii}{\hbar} \int_{-\infty}^t \dd{t'} h(t') \trace \left\{ \comm*{\hat{A}^I(t')}{\hat{\rho}_0} \hat{B}^I(t) \right\},
    \end{aligned}
\]
利用迹运算的轮换性,得到
\[
    \trace \left\{ \comm*{\hat{A}^I(t')}{\hat{\rho}_0} \hat{B}^I(t) \right\} = \trace \left\{ \hat{\rho}_0 \comm*{\hat{B}^I(t)}{\hat{A}^I(t')} \right\},
\]
于是最终得到
\[
    \begin{aligned}
        y(t) &= \frac{\ii}{\hbar} \int_{-\infty}^t \dd{t'} h(t') \trace \left\{ \hat{\rho}_0 \comm*{\hat{B}^I(t)}{\hat{A}^I(t')} \right\} \\
        &= \frac{\ii}{\hbar} \int_{-\infty}^t \dd{t'} h(t') \expval*{\comm*{\hat{B}^I(t)}{\hat{A}^I(t')} }_0.
    \end{aligned}
\]
最后,注意到以$\hat{H}$为自由哈密顿量的相互作用绘景中的算符实际上就是以$\hat{H}$为哈密顿量的海森堡绘景中的算符,于是我们写出$y(t)$和$h(t')$之间的响应函数:
\[
    G(t,t') = \frac{\ii}{\hbar} \Theta(t-t') \expval*{\comm*{\hat{B}^H(t)}{\hat{A}^H(t')}}_0.
\]
为了更加明确,通常使用下面的记号:
\begin{equation}
    G(t,t')^\text{ret}_{BA} = \frac{\ii}{\hbar} \Theta(t-t') \expval*{\comm*{\hat{B}(t)}{\hat{A}(t')}}.
    \label{eq:retarded-green-function}
\end{equation}
上标ret表示这是\textbf{推迟格林函数}(retarded),下标表示扰动和响应。所谓“推迟”不代表该格林函数在时间上是非局域的,而是表示向后的因果性。
默认体系不受到微扰,且算符$\hat{A}$和$\hat{B}$默认处于无微扰的海森堡绘景中,于是略去期望值的下标0和算符的上标。

实际我们做实验分析一个系统的性质时,都是对系统施加某个输入(敲它以下、加上一个磁场、通电,等等),然后测量对应的响应,因此只需要计算出我们关心的过程(例如,外加电场会导致通电,即外加$\vb*{E}$导致$\vb*{j}$)的推迟格林函数,就确定了这个过程的性质。只需要推迟格林函数就够了。于是接下来来分析推迟格林函数的性质。

\subsubsection{涨落耗散定理}

本节展示推迟格林函数的一个非常重要的性质。采取海森堡绘景,定义两个算符的\textbf{关联函数}为
\begin{equation}
    S_{BA}(t,t') = \expval*{\hat{B}(t)\hat{A}(t')} - \expval*{\hat{B}(t)}\expval*{\hat{A}(t')}.
\end{equation}
我们可以不失一般性地对各个算符做一个平移,从而让它们的期望值为零,即做变换
\[
    \hat{A} \longrightarrow \var{\hat{A}} = \hat{A} - \expval*{\hat{A}},
\]
于是不失一般性地认为所有算符的期望值都为零,从而
\begin{equation}
    S_{BA}(t,t') = \expval*{\hat{B}(t)\hat{A}(t')}.
\end{equation}

由于哈密顿量不含时,我们有时间平移对称性,于是
\begin{equation}
    G_{BA}^\text{ret}(t,t') = G_{BA}^\text{ret}(t-t'), \quad S_{BA}(t,t') = S_{BA}(t-t').
\end{equation}
于是可以定义一个单变量傅里叶变换:
\begin{equation}
    S_{BA}(\omega) = \int_{-\infty}^\infty \dd{t} S_{BA}(t) \ee^{\ii \omega t}, \quad G_{BA}^\text{ret}(\omega) = \int_{-\infty}^\infty \dd{t} G_{BA}^\text{ret}(t) \ee^{\ii \omega t}.
\end{equation}
实际上由\eqref{eq:retarded-green-function},$t<0$处推迟格林函数为零,从而
\[
    G_{BA}^\text{ret}(\omega) = \int_0^\infty \dd{t} G_{BA}^\text{ret}(t) \ee^{\ii \omega t}.
\]
那么,我们有\textbf{涨落耗散定理}:%
\footnote{频域上的响应函数的虚部代表了耗散,而关联函数则代表了两个物理量共同涨落的强弱。}
\begin{equation}
    \Im G_{BA}^\text{ret}(\omega) = \frac{1 - \ee^{-\beta\hbar\omega}}{2\hbar} S_{BA}(\omega).
\end{equation}

下面我们来证明这一点。首先注意到,$h(t)$和$y(t)$都是实数,从而联系它们的推迟格林函数也一定是实数,这样就有
\[
    \begin{aligned}
        \Im G_{BA}^\text{ret}(\omega) &= \frac{1}{2\ii} \left( G_{BA}^\text{ret}(\omega) -( G_{BA}^\text{ret}(\omega))^* \right) \\
        &= \frac{1}{2\ii} \left( \int_0^\infty \dd{t} \ee^{\ii \omega t} G_{BA}^\text{ret}(t) -\int_0^\infty \dd{t} \ee^{- \ii \omega t} G_{BA}^\text{ret}(t) \right) \\
        &= \frac{1}{2\ii} \int_{-\infty}^\infty \dd{t} \ee^{\ii \omega t} G_{BA}^\text{ret}(t) \\
        &= \frac{1}{2\hbar} \int_{-\infty}^\infty \dd{t} \ee^{\ii \omega t} \expval{\comm{\hat{B}(t)}{\hat{A}(0)}} \\
        &= \frac{1}{2\hbar} \left( \int_{-\infty}^\infty \dd{t} \ee^{\ii \omega t} \hat{B}(t) \hat{A}(0) - \int_{-\infty}^\infty \dd{t} \ee^{\ii \omega t} \hat{A}(0) \hat{B}(t) \right).
    \end{aligned}
\]
第一项正是关联函数。对第二项,
\[
    \begin{aligned}
        \int_{-\infty}^\infty \dd{t} \ee^{\ii \omega t} \hat{A}(0) \hat{B}(t) &= \int_{-\infty}^\infty \dd{t} \ee^{\ii \omega t} \frac{1}{Z} \trace \left( \ee^{-\beta \hat{H}} \hat{A}(0) \ee^{\frac{\ii}{\hbar} \hat{H} t} \hat{B}(0) \ee^{ - \frac{\ii}{\hbar} \hat{H} t} \right) \\
        &= \int_{-\infty}^\infty \dd{t} \ee^{\ii \omega t} \frac{1}{Z} \trace \left( \ee^{-\beta\hat{H}} \hat{A}(0) \ee^{-\beta \hat{H}} \ee^{\beta \hat{H}} \ee^{\frac{\ii}{\hbar} \hat{H} t} \hat{B}(0) \ee^{ - \frac{\ii}{\hbar} \hat{H} t} \right) \\
        &= \int_{-\infty}^\infty \dd{t} \ee^{\ii \omega t} \frac{1}{Z} \trace \left( \ee^{-\beta \hat{H}} \ee^{\beta \hat{H}} \ee^{\frac{\ii}{\hbar} \hat{H} t} \hat{B}(0) \ee^{ - \frac{\ii}{\hbar} \hat{H} t} \ee^{-\beta\hat{H}} \hat{A}(0) \right) \\
        &= \int_{-\infty}^\infty \dd{t} \ee^{\ii \omega t} \frac{1}{Z} \trace \left( \ee^{-\beta \hat{H}} \ee^{\frac{\ii}{\hbar} \hat{H} (t - \ii \hbar \beta)} \hat{B}(0) \ee^{ - \frac{\ii}{\hbar} \hat{H} (t - \ii \hbar \beta)} \hat{A}(0) \right) \\
        &= \int_{-\infty+\ii\hbar\beta}^{\infty+\ii\hbar\beta} \dd{t} \ee^{\ii \omega (t + \ii \hbar \beta)} \frac{1}{Z} \trace \left( \ee^{-\beta \hat{H}} \ee^{\frac{\ii}{\hbar} \hat{H} t} \hat{B}(0) \ee^{ - \frac{\ii}{\hbar} \hat{H} t} \hat{A}(0) \right).
    \end{aligned}
\]
被积函数在区域$0 < \Im t < \hbar \beta$内是解析的,从而就有
\[
    \begin{aligned}
        \int_{-\infty}^\infty \dd{t} \ee^{\ii \omega t} \hat{A}(0) \hat{B}(t)  &= \ee^{-\omega \hbar \beta} \int_{-\infty+\ii\hbar\beta}^{\infty+\ii\hbar\beta} \dd{t} \ee^{-\ii \omega t} \frac{1}{Z} \trace \left( \ee^{-\beta \hat{H}} \ee^{\frac{\ii}{\hbar} \hat{H} t} \hat{B}(0) \ee^{ - \frac{\ii}{\hbar} \hat{H} t} \hat{A}(0) \right) \\
        &= \ee^{-\omega \hbar \beta} \int_{-\infty}^{\infty} \dd{t} \ee^{-\ii \omega t} \frac{1}{Z} \trace \left( \ee^{-\beta \hat{H}} \ee^{\frac{\ii}{\hbar} \hat{H} t} \hat{B}(0) \ee^{ - \frac{\ii}{\hbar} \hat{H} t} \hat{A}(0) \right),
    \end{aligned}
\]
于是就证明了涨落耗散定理。

\subsubsection{Kramers-Kronig关系}

实际上,推迟格林函数的实部和虚部也有关系。要看出这个关系只需要利用推迟格林函数的两个性质:
\begin{itemize}
    \item 因果性,即$G_{BA}^\text{ret}(t)$在$t<0$时为零;
    \item 频谱至少衰减得和$1/\omega$一样快,这个条件实际上需要额外的确认,但通常是成立的,因为当
\end{itemize}

第一个条件,也就是,因果律,意味着$G_{BA}^\text{ret}(\omega)$在上半平面上是解析的。这是因为
\[
    G_{BA}^\text{ret}(t) = \frac{1}{2\pi} \int_{-\infty}^\infty \dd{t} G_{BA}^\text{ret}(\omega) \ee^{-\ii\omega t},
\]
在$t<0$时能够保证在上半平面上,$\omega\to\inf$时$\ee^{-\ii \omega t}$快速衰减,于是
\[
    G_{BA}^\text{ret}(t) = \frac{1}{2\pi} \cdot 2 \pi \sum_\text{upper plane}  \Res G_{BA}^\text{ret}(\omega) \ee^{-\ii\omega t},
\]
在$t<0$时上式一定是零,从而上半平面上必定没有奇点,从而$G_{BA}^\text{ret}(\omega)$在上半平面上是解析的。
相应的,如果系统不是平凡的,那么下半平面一定有奇点,因为$t<0$时应当在下半平面取留数。

现在考虑积分
\[
    \int_{-\infty}^\infty \dd{\omega'} \frac{G_{BA}^\text{ret}(\omega')}{\omega' - \omega + \ii 0^+},
\]
被积函数仅有的奇点位于下半平面,因此它在上半平面和实轴上处处解析,从而
\[
    \oint \dd{\omega'} \frac{G_{BA}^\text{ret}(\omega')}{\omega' - \omega + \ii 0^+} = 0.
\]
另一方面,设$C$是上半平面上的辐角从$0$到$\pi$的大圆弧,由于$G_{BA}^\text{ret}(\omega)$衰减得很快,由大圆弧引理,
\[
    \int_C \dd{\omega} \frac{G_{BA}^\text{ret}(\omega')}{\omega' - \omega + \ii 0^+} = 0.
\]
那么,取实轴和大圆弧组成一个闭合回路,在这个闭合回路上的积分是零,在大圆弧上的积分还是零,于是实轴上的积分也是零,
\[
    \int_{-\infty}^\infty \dd{\omega'} \frac{G_{BA}^\text{ret}(\omega')}{\omega' - \omega + \ii 0^+} = 0.
\]
而这个积分可以通过经典的“将奇点移动到实轴而改变积分路径”的方法计算出来,或者等价地,使用公式
\[
    \frac{1}{\omega'-\omega+\ii 0^+} = \primevalue \frac{1}{\omega'-\omega} - \pi \ii \delta(\omega'-\omega),
\]
其中$\primevalue$表示柯西积分主值,就得到
\[
    0 = \int_{-\infty}^\infty \dd{\omega'} \left( \Re G_{BA}^\text{ret}(\omega') + \ii \Im G_{BA}^\text{ret}(\omega') \right) \left( \primevalue \frac{1}{\omega'-\omega} - \pi \ii \delta(\omega'-\omega) \right),
\]
分别取实部和虚部,就得到
\begin{equation}
    \begin{bigcase}
        \Re G_{BA}^\text{ret}(\omega) &= \frac{1}{\pi} \primevalue \int_{-\infty}^\infty \dd{\omega'} \frac{\Im G_{BA}^\text{ret}(\omega')}{\omega' - \omega}, \\
        \Im G_{BA}^\text{ret}(\omega) &= - \frac{1}{\pi} \primevalue \int_{-\infty}^\infty \dd{\omega'} \frac{\Re G_{BA}^\text{ret}(\omega')}{\omega' - \omega}.
    \end{bigcase}
\end{equation}
因此,推迟格林函数的实部和虚部之间可以相互换算。

总之,在频域上,推迟格林函数的实部、虚部和关联函数这三者是一一对应的,因此实际的自由度只有一个。

\subsubsection{谱函数}\label{sec:spectral-function}

定义
\begin{equation}
    A_{BA}(\omega) = \frac{1}{\pi} \Im G_{BA}^\text{ret}(\omega) = \frac{1}{2 \pi \hbar} \int_{-\infty}^\infty \dd{t} \expval*{\comm*{\hat{B}(t)}{\hat{A}(0)}} \ee^{\ii \omega t}
\end{equation}
为\textbf{谱函数}。当然,推迟格林函数可以很容易地使用谱函数表示出来:
\begin{equation}
    \begin{bigcase}
        \Re G_{BA}^\text{ret}(\omega) &= \primevalue \int_{-\infty}^\infty \dd{\omega'} \frac{A_{BA}(\omega)}{\omega'-\omega} , \\
        \Im G_{BA}^\text{ret}(\omega) &= \pi A_{BA}(\omega).
    \end{bigcase}
\end{equation}
或者,考虑到
\[
    \frac{1}{\omega'-\omega-\ii 0^+} = \primevalue \frac{1}{\omega'-\omega} + \ii \pi \delta(\omega'-\omega),
\]
就是
\begin{equation}
    G_{BA}^\text{ret}(\omega) = \int_{-\infty}^\infty \dd{\omega'} \frac{A_{BA}(\omega')}{\omega' - \omega - \ii 0^+}.
\end{equation}
这表明谱函数给出了推迟格林函数在不同频率上的分布情况。当$\omega\to 0$时,我们有
\[
    G_{BA}^\text{ret}(\omega) \sim \int_{-\infty}^\infty \dd{\omega'} \frac{A_{BA}(\omega')}{\omega' - \ii 0^+},
\]
而当$\omega\to \infty$时,我们有
\[
    G_{BA}^\text{ret}(\omega) \sim - \int_{-\infty}^\infty \dd{\omega'} \frac{A_{BA}(\omega')}{\omega},
\]
这表明外加驱动频率很低时和外加驱动频率很高时产生的响应的正负号是反的。这是谐振子对外加驱动的响应的一种推广:当驱动频率很小时,体系能够很好地跟上外加驱动,当驱动频率特别大时,体系几乎总是落在外加驱动后面。

到现在为止谱函数仅仅是纯形式的记号。现在我们要说明怎样通过哈密顿量得到谱函数,这个过程称为\textbf{谱表示}。由于通常并不能解出哈密顿量,谱表示的理论意义大于实际计算意义。
设哈密顿量被对角化为
\[
    \hat{H} \ket{n} = E_n \ket{n},
\]
则
\[
    \begin{aligned}
        A_{BA} (\omega) &= \frac{1}{\pi} \Im G_{BA}(\omega) \\
        &= \frac{1}{2 \pi \hbar} \int_{-\infty}^\infty \dd{t} \expval*{\comm*{\hat{B}(t)}{\hat{A}(0)}} \ee^{\ii \omega t} \\
        &= \frac{1}{2 \pi \hbar} \int_{-\infty}^\infty \dd{t} \ee^{\ii \omega t} \frac{1}{Z} \\
        & \quad \quad \times \sum_m \left( \mel{m}{\ee^{-\beta E_m} \ee^{\frac{\ii}{\hbar} E_m t} \hat{B}(0) \ee^{- \frac{\ii}{\hbar} \hat{H} t} \hat{A}(0) }{m} - \mel{m}{\ee^{-\beta E_m} \hat{A}(0) \ee^{\frac{\ii}{\hbar} \hat{H} t} \hat{B}(0) \ee^{- \frac{\ii}{\hbar} E_m t} }{m} \right).
    \end{aligned}
\]
上式中我们已经将紧邻左矢或右矢的哈密顿算符写成了本征值的形式。通过在每一个期望值中间再插入一组完备正交基,我们得到
\[
    \begin{aligned}
        &\quad \mel{m}{\ee^{-\beta E_m} \ee^{\frac{\ii}{\hbar} E_m t} \hat{B}(0) \ee^{- \frac{\ii}{\hbar} \hat{H} t} \hat{A}(0) }{m} \\
        &= \sum_n \mel{m}{\ee^{-\beta E_m} \ee^{\frac{\ii}{\hbar} E_m t} \hat{B}(0)}{n} \mel{n}{\ee^{- \frac{\ii}{\hbar} E_n t \hat{A}(0) }}{m} \\
        &= \ee^{- \beta E_m} \ee^{\frac{\ii}{\hbar} (E_m - E_n)} \sum_n \mel{m}{\hat{B}(0)}{n} \mel{n}{\hat{A}(0)}{m},
    \end{aligned}
\]
同理
\[
    \begin{aligned}
        &\quad \mel{m}{\ee^{-\beta E_m} \hat{A}(0) \ee^{\frac{\ii}{\hbar} \hat{H} t} \hat{B}(0) \ee^{- \frac{\ii}{\hbar} E_m t} }{m} \\
        &= \sum_n \mel{m}{\ee^{-\beta E_m} \hat{A}(0) \ee^{\frac{\ii}{\hbar} E_n t}}{n} \mel{n}{\hat{B}(0) \ee^{- \frac{\ii}{\hbar} E_m t}}{m} \\
        &= \ee^{-\beta E_m} \ee^{\frac{\ii}{\hbar} (E_n - E_m) t} \sum_n \mel{m}{\hat{A}(0)}{n} \mel{n}{\hat{B}(0)}{m}.
    \end{aligned}
\]
通过交换第二项中的$m$和$n$,并计算有关积分,就得到
\begin{equation}
    A_{BA}(\omega) = \frac{1}{\hbar Z} \sum_{m,n} \left( \ee^{-\beta E_m} - \ee^{-\beta E_n} \right) \mel{m}{\hat{B}}{n} \mel{n}{\hat{A}}{m} \delta\left( \omega + \frac{E_m - E_n}{\hbar} \right).
\end{equation}
这就是谱表示的具体表达式。考虑到狄拉克函数的性质,可以写出一个不那么对称的表达式:
\begin{equation}
    A_{BA}(\omega) = \frac{1}{\hbar Z} \sum_{m,n} \ee^{-\beta E_m} (1 - \ee^{- \beta \hbar \omega}) \mel{m}{\hat{B}}{n} \mel{n}{\hat{A}}{m} \delta\left( \omega + \frac{E_m - E_n}{\hbar} \right). 
\end{equation}

从谱表示出发可以获得涨落耗散定理的另一个证明,因为我们可以证明关联函数的形式实际上和谱表示只差了一个常数因子。

%%%%%%%%%%%%%%%%%%%%%%%%%%%%%%%%%%%%%%%%%%%%%55

系统为平衡态时可以使用\eqref{eq:correlation-function-from-partition-function}把所有量都计算出来,但现在要计算的东西涉及时间演化,因此不能够简单地使用这个公式。
但我们马上可以看到,对配分函数的形式做一个小的变动就能够从配分函数计算出松原格林函数。
注意到$\hat{H}$不显含时间,所以我们有
\[
    \beta \hat{H} = \int_0^\beta \dd{\tau} \hat{H},
\]
这样配分函数就是
\[
    Z = \sum_n \ee^{- \int_0^\beta \dd{\tau} E_n},
\]
既然现在使用场论的观点,记$\hat{\phi}_1, \hat{\phi}_2, \ldots, \hat{\phi}_n$是一组彼此对易的场,且这些场完全刻画了系统,则它们的共同本征态就构成了哈密顿量的本征态,%
\footnote{这些条件没有那么容易达到。例如就克莱因-高登实标量场而言,有
\[
    H = \frac{1}{2} \dot{\phi}^2 - \frac{1}{2} \left( \grad{\phi} \right)^2 - \frac{1}{2} m^2 \phi^2,
\]
仅仅依靠等时对易关系不能够计算出$\phi$和$\dot{\phi}$的对易关系,所以必须把它们看成不同的场;但是它们不对易。因此不能使用$\phi$为表象做下面的计算。
此外,对复场,应当把它的实部和虚部当成独立的场,或者等价地,将它本身和它的共轭转置当成独立的场。
}%
从而我们有
\begin{equation}
    Z = \int \fd{\phi_1} \fd{\phi_2} \cdots \fd{\phi_n}  \exp \left( - \int_0^\beta \dd{\tau} \int \dd[3]{\vb*{r}} \mathcal{H}[\phi_1(\vb*{r}), \phi_2(\vb*{r}), \ldots, \phi_n(\vb*{r})] \right),
\end{equation}
其中$\mathcal{H}[\phi_1, \phi_2, \ldots, \phi_n]$指的是$\hat{\mathcal{H}}$对应于$\ket{\phi_1, \phi_2, \ldots, \phi_n}$的能量本征值
(使用方括号是因为$\mathcal{H}$可能涉及对场求梯度等运算,而不仅仅是它的代数函数),显然
\[
    \eval{\mathcal{H}(\phi_1, \phi_2, \ldots, \phi_n)}_{\phi_i \longrightarrow \hat{\phi}_i} = \hat{\mathcal{H}}, \quad i = 1, 2, \ldots, n.
\]
由于我们总是可以把场$\hat{\phi}_1, \hat{\phi}_2, \ldots, \hat{\phi}_n$打包成一个多分量场$\hat{\phi}$,以下使用简便写法
\begin{equation}
    Z = \int \fd{\phi } \exp \left( - \int_0^\beta \dd{\tau} \int \dd[3]{\vb*{r}} \mathcal{H}[\phi(\vb*{r})] \right),
    \label{eq:partition-function-matsubara}
\end{equation}
或者如果使用四维矢量记号,就是
\begin{equation}
    Z = \int \fd{\phi } \exp \left( - \int \dd[4]{x} \mathcal{H}[\phi(\vb*{r})] \right).
\end{equation}
\eqref{eq:partition-function-matsubara}中的场当然也可以是不是定义在坐标空间中的,而是定义在动量空间中的,或者定义在别的什么空间中。
例如,在动量空间中就有
\begin{equation}
    Z = \int \fd{\phi} \exp \left( - \int_0^\beta \dd{\tau} \int \dd[3]{\vb*{k}} \mathcal{H}[\phi(\vb*{k})] \right).
\end{equation}

以上推导均为绘景无关的(见\autoref{sec:density-operator-canonical-ensemble}),如果我们使用虚时间的海森堡绘景,就有%
\footnote{由于哈密顿量不含时,海森堡绘景中的哈密顿量和薛定谔绘景中的哈密顿量是相等的,且两者关于各自的绘景中的场算符的表达式也是相等的。
更明确地说,设
\[
    \hat{H}^S[\hat{\phi}^S] = \int \dd[3]{\vb*{r}} f(\hat{\phi}^S, \grad{\hat{\phi}^S}),
\]
则
\[
    \hat{H}^H[\hat{\phi}^H] = \int \dd[3]{\vb*{r}} f(\hat{\phi}^H, \grad{\hat{\phi}^H}).
\]
这也就是为什么对不含时系统,我们从来不区分哈密顿量在哪个绘景中,也不区分不同绘景中哈密顿量的表达式。
同样,记$\phi$为能量本征态$\ket{\phi}$相对场算符的本征值,$H$为它相对哈密顿量的本征值,则有
\[
    H = \int \dd[3]{\vb*{r}} f(\phi(\vb*{r}), \grad{\phi(\vb*{r})}),
\]
而与绘景无关,因为绘景变换不改变$\phi$和$H$。
本文的剩余部分使用更加通常的记号,将$f$记作$\mathcal{H}$。
}%
\[
    Z = \int \fd{\phi} \exp \left( - \int_0^\beta \dd{\tau} \int \dd[3]{\vb*{r}} \mathcal{H}[\phi(\vb*{r}, \tau)] \right). 
\]
这正是典型的路径积分,它与高能物理中的路径积分唯一的区别在于它不在闵可夫斯基时空中,并且指数函数的宗量中没有因子$\ii$(但这一点差异在做变量代换之后就消失了)。
对它做扰动,得到
\begin{equation}
    Z[J] = \int \fd{\phi} \exp ( - S[\phi, J] ),
\end{equation}
其中
\begin{equation}
    S[\phi, J] = \int_0^\beta \dd{\tau} \int \dd[3]{\vb*{r}} (\mathcal{H}[\phi(\vb*{r}, \tau)] + J(\vb*{r}, \tau) \phi(\vb*{r}, \tau)),
\end{equation}
其中$J$也是一个多分量场,代表外来扰动的强度或者说“载荷”。

%%%%%%%%%%%%%%%%%%%%%%%%%%%%%%%%%%%%%%%%%%%%%%%%%%%%%%%%%%%

\subsection{从拉格朗日力学出发}

之前所有的讨论都没有对拉氏量的具体形式做出任何假定。在这一节,我们将以更加物理的眼光处理问题。
我们称系统中的一个\textbf{粒子}为由数目固定为$D$个广义坐标可以完全描写的一个子系统。
系统中除了若干个(数目通常确定的)粒子以外可能还会有一些别的东西。

首先考虑一个只有一个\textbf{自由粒子}的体系,这个自由粒子可以使用$D$个广义坐标描写。
所谓的自由粒子实际上并没有良好定义:例如,一个受到一个线性回复力的粒子算不算自由的?
实际上,什么是自由粒子取决于我们认为体系遵从怎样的对称性。
例如如果我们认为对自由粒子而言空间是均匀各向同性的,那么受到线性回复力的粒子就不是自由粒子,但如果没有这个要求,那么受到线性回复力的粒子可以被认为是自由的。
设一个自由粒子的拉氏量为
\begin{equation}
    L = m L_0(q_1, \ldots, q_D, \dot{q}_1, \ldots, \dot{q}_D, t)
    \label{eq:single-particle}
\end{equation}
然后就能够使用\eqref{eq:el}导出自由粒子的运动方程。其中$m$是一个常数。
当然在系统中只有一个粒子的时候$m$可以取任何非零的值。具体$m$为什么要存在,接下来会看到。

如果体系中有多个自由粒子,它们之间没有相互作用(也就是说,改变其中一个粒子的坐标不会对其它粒子的坐标产生任何影响),
那么整个系统的拉氏量就可以写成
\[
    L = \sum_i m_i L_0(q^{(i)}, \dot{q}^{(i)}, t)
\]
其中$q^{(i)}$表示第$i$个粒子的所有坐标。由于拉氏量中各项相互独立,$m_i$的取值同样无关紧要。

在\textbf{有相互作用}时,体系的拉氏量为
\[
    L = \sum_i m_i L_0(q^{(i)}, \dot{q}^{(i)}, t) - U(q^{\text{par}}, \dot{q}^{\text{par}}, t) + L^{\text{res}}
\]
其中$q^\text{par}$代表所有关于粒子的坐标,$L^\text{res}$代表不能表示成粒子相互作用的一些机制。
记
\begin{equation}
    L^{(i)}_0 = L_0(q^{(i)}, \dot{q}^{(i)}, t)
\end{equation}
则
\begin{equation}
    L = \sum_i m_i L_0^{(i)} - U(q^{\text{par}}, \dot{q}^{\text{par}}, t) + L^{\text{res}}
    \label{eq:interaction}
\end{equation}
如果$U=0$且$L^\text{res}=0$,\eqref{eq:interaction}就退化回了没有相互作用的情况。
现在$m_i$的取值不再无关紧要了!我们看到,加上$m$只是为了保证\eqref{eq:single-particle}在有相互作用时具有\textbf{可加性}。

实际上,只要系统中存在可以编组的广义坐标,就总是能够将拉氏量写成\eqref{eq:interaction}的形式。

通常我们对$L^\text{res}$项的具体形式不感兴趣,只是想知道它对粒子运动的影响,于是通过\ref{sec:non-conservative-system}节中的方法把它变成一个“外力项”。
使用\eqref{eq:generalized-el},得到运动方程
\begin{equation}
    m_i \dv{t} \pdv{L_0^{(i)}}{\dot{q}^{(i)}} - m_i \pdv{L^{(i)}_0}{q^{(i)}} = - \pdv{U}{q^{(i)}} + \dv{t} \pdv{U}{\dot{q}^{(i)}} + Q^{(i)}
    \label{eq:few-body}
\end{equation}
其中$Q^{(i)}$来自$L^\text{res}$,具体方法见\ref{sec:non-conservative-system}节。
这样,对由一些粒子组成的系统,\eqref{eq:few-body}完全确定了它的演化。

当系统中的粒子很多时,TODO

%%%%%%%%%%%%%%%%%%%%%%%%%%%%%%%%%%%%%%%%%%%%%%%%%%5

另一种做实际计算的方法是,考虑$\hat{H}$的本征态$\ket{\psi_n}$,其能量为$E_n$,则我们有
\[
    \hat{H} \ket{\psi_n} = E_n \ket{\psi_n}.
\]
有效哈密顿量只需要指导$\mathcal{H}_1$中的态的运行即可,因此它需要满足
\[
    \hat{H}_\text{eff} \ket{\psi_n} = E_n \ket{\psi_n}, \quad \text{for $\ket{\psi_n} \in \mathcal{H}_1$}.
\]
由于$\hat{H}_\text{eff}$是$\mathcal{H}_1$中的算符,以上方程的展开式为
\begin{equation}
    \sum_{\text{span}\{\ket{m}\} = \mathcal{H}_1} (\mel{l}{\hat{H}_\text{eff}}{m} - E_n \delta_{lm}) \braket{m}{\psi_n} = 0.
    \label{eq:effective-hamiltonian-eq-unfolded}
\end{equation}
这里我们使用一组任意的正交归一化基底$\{\ket{m}\}$,它们未必就是哈密顿量的本征态。相应的,$\hat{H}$满足
\begin{equation}
    \sum_m (\mel{l}{\hat{H}}{m} - E_n \delta_{lm}) \braket{m}{\psi_n} = 0, \quad \text{for $\ket{l} \in \mathcal{H}_1$}.
    \label{eq:hamiltonian-eq-unfolded}
\end{equation}
显然,\eqref{eq:hamiltonian-eq-unfolded}必须能够推导出\eqref{eq:effective-hamiltonian-eq-unfolded},线性代数上的结论告诉我们,记那些张成$\mathcal{H}_1$的基矢量的编号组成的集合为$W$,其余基矢量的编号组成的集合为$U$,则有
\[
    \mel{l}{\hat{H}_\text{eff}}{m} = \mel{l}{\hat{H}}{m} - \sum_{\alpha \in U} \frac{\mel{l}{H}{\alpha} \mel{\alpha}{H}{m}} {D_\alpha^W} + \sum_{\alpha \neq \beta \in U} \frac{\mel{l}{H}{\alpha} \mel{\alpha}{H}{\beta} \mel{\beta}{H}{m}}{D^W_{\alpha \beta}} + \cdots,
\]
其中,记$S$为一系列编号组成的集合,单脚标的$D$函数定义为
\[
    D_\alpha^S = H_{\alpha \alpha} - E_n - \sum_{\beta \in U, \beta \notin S} \frac{\mel{\alpha}{H}{\beta} \mel{\beta}{H}{\alpha}}{D_\alpha^S} + \sum_{\beta \neq \gamma \in U, \; \beta, \gamma \notin S} \frac{\mel{l}{H}{\alpha} \mel{\alpha}{H}{\beta} \mel{\beta}{H}{m}}{D_{\alpha \beta}^S} + \cdots,
\]
多脚标的$D$函数递归定义为
\[
    D_{\alpha \beta}^S = D_\alpha^S D_{\beta}^{S, \alpha}, \quad D_{\alpha, \beta, \gamma}^S = D_{\alpha}^S D_{\beta}^{S, \alpha} D_{\gamma}^{S, \alpha, \beta}, \ldots
\]
% TODO:和通常使用的微扰论有何关系???

%%%%%%%%%%%%%%%%%%%%%%%%%%%%%%%%%%%%%%%%%%%%%%%%%%%%%%55
为了便于分析问题,我们使用正则系综和巨正则系综的等价性,即做变换
\[
    \hat{H} - \sum_u \mu_i \hat{N}_i \longrightarrow \hat{H},
\]
从而将$\{\mu_i\}$——或者等价的$\{N_i\}$——看成哈密顿量的参数。这样就有
\[
    \dd{U} = \trace (\dd{\hat{\rho}} \hat{H}) + \sum_i \trace \left( \hat{\rho} \pdv{\hat{H}}{q_i} \right) \dd{q_i},
\]
再恢复原来的哈密顿量,就有
\[
    \dd{U} - \sum_i (\dd{\mu_i} N_i + \mu_i \dd{N_i}) = \trace \left(\dd{\hat{\rho}} \left(\hat{H} - \sum_i \mu_i \hat{N_i} \right) \right) + \sum_i \trace \left( \hat{\rho} \pdv{\hat{H}}{q_i} \right) \dd{q_i},
\]
这里我们使用了$\{\mu_i\}$和$\{N_i\}$均和$\{q_i\}$无关这一事实。

%%%%%%%%%%%%%%%%%%%%%%%%%%%%%%%%%%%%%%%%%%%%%%%%%%%%%%%%%%%%5

\section{量子动力学(正则表述)}

\subsection{系统演化方程与哈密顿算符}

现在讨论怎样让系统动起来。在经典理论中有哈密顿力学,其动力学部分为
\begin{equation}
    \dv{A}{t} = [H, A] + \pdv{A}{t}.
\end{equation}
其中$[H, A]$是泊松括号。在量子理论中我们也采取一个类似的动力学方程:
\begin{equation}
    \pdv{\hat{A}}{t} = \frac{1}{\ii \hbar} [\hat{H}, \hat{A}] + \pdv{\hat{A}}{t}.
\end{equation}
方程中的$\ii \hbar$无关紧要——这只是重新定义$[\cdot, \cdot]$导致的常数而已。为什么会需要这个常数在后面会看到。
在量子理论中$[\cdot, \cdot]$的定义为
\begin{equation}
    [A, B] = AB - BA
\end{equation}
容易看出这导致诸算符构成了一个李代数。

算符$\hat{H}$称为\textbf{哈密顿算符},与经典理论中的哈密顿量一致。

\subsection{绘景}

由于态矢量和算符可以整体做一个酉变换而不改变所描述的系统,我们可以假定算符的变化完全来自它自身的定义而将$[A, H]$部分造成的时间演化都归结到$\ket{\psi}$的变化上,从而得到\textbf{薛定谔绘景}(反之,以\eqref{eq:canonical-time-evolution}为算符演化、假定态矢量不变的绘景称为海森堡绘景)

算符$A$做酉演化,则设
\begin{equation}
    U(t, t_0) A^H(t) U^\dagger(t, t_0) = A^H(t_0)
\end{equation}
设
\begin{equation}
    A^S = A^H(t_0), \quad \ket{\psi^S(t)} = U(t, t_0) \ket{\psi^H}
\end{equation}
就得到TODO:从海森堡过渡到薛定谔。重点在于将$A^S$的变化全部归结到$A^H$的含时部分上面。
两种表象之间的转换方程:
\begin{equation}
    \quad U = U(t, t_0), \quad H^H = U^\dagger H^S U, \quad A^H = U^\dagger A^S U, \quad \pdv{U(t, t_0)}{t} = \frac{1}{\ii \hbar} H^S(t)
\end{equation}

相互作用表象
\[
    H^S = H^S_0 + H^S+i
\]
定义
\[
    \pdv{U_0(t,t_0)}{t} = \frac{1}{\ii \hbar} H^S_0(t)
\]
变换关系:
\begin{equation}
    \left\{ \quad
        \begin{aligned}
            U_0 = U_0(t,t_0), \quad \pdv{U_0(t,t_0)}{t} = \frac{1}{\ii \hbar} H^S_0(t), \\
            A^I = U_0^\dagger A^S U_0, \quad H_i^I = U_0^\dagger H_i^S U_0, \quad \quad H_0^I = U_0^\dagger H_0^S U_0
        \end{aligned}
    \right.
\end{equation}
运动方程:
\begin{equation}
    \dv{t} \ket{\psi^I(t)} = \frac{1}{\ii \hbar} H_i^I(t) \ket{\psi^I(t)}, \quad 
    \dv{t} A^I = \frac{1}{\ii \hbar} [A^I, H_0^I] + \pdv{A^I}{t}
\end{equation}
