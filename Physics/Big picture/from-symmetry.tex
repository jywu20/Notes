\documentclass[hyperref, UTF8, a4paper]{ctexart}

\usepackage{geometry}
\usepackage{titling}
\usepackage{titlesec}
\usepackage{paralist}
\usepackage{footnote}
\usepackage{enumerate}
\usepackage{amsmath, amssymb, amsthm}
\usepackage{simplewick}
\usepackage{cite}
\usepackage{graphicx}
\usepackage{subfigure}
\usepackage{physics}
\usepackage{centernot}
\usepackage{tikz}
\usepackage[colorlinks, linkcolor=black, anchorcolor=black, citecolor=black]{hyperref}
\usepackage{prettyref}

\geometry{left=3.18cm,right=3.18cm,top=2.54cm,bottom=2.54cm}
\titlespacing{\paragraph}{0pt}{1pt}{10pt}[20pt]
\setlength{\droptitle}{-5em}
\preauthor{\vspace{-10pt}\begin{center}}
\postauthor{\par\end{center}}

\DeclareMathOperator{\timeorder}{T}
\DeclareMathOperator{\diag}{diag}
\newcommand*{\ii}{\mathrm{i}}
\newcommand*{\ee}{\mathrm{e}}
\newcommand*{\const}{\mathrm{const}}
\newcommand*{\comment}{\paragraph{注记}}
\newcommand{\fsl}[1]{{\centernot{#1}}}
\newcommand*{\reals}{\mathbb{R}}
\newcommand*{\complexes}{\mathbb{C}}

\newrefformat{sec}{第\ref{#1}节}
\newrefformat{note}{注\ref{#1}}
\renewcommand{\autoref}{\prettyref}

\newenvironment{bigcase}{\left\{\quad\begin{aligned}}{\end{aligned}\right.}

\allowdisplaybreaks[4]

\title{从对称性出发看物理}
\author{wujinq}

\begin{document}

\maketitle

\vspace{2em}

% TODO:可能的相互作用过程需要且只需要受到对称性的约束(例如如果只有空间平移不变性,那么一切可能的相互作用过程就是满足动量守恒的过程,如果同时有时间和空间平移不变性那么一切可能的相互作用过程就是满足能量和动量守恒的过程)——为什么“只需要”?为什么满足对称性约束的所有过程都实际有可能发生?
% 此外,什么是“模式”?
% TODO:相互作用的时间延迟实际上就是格林函数中$r - ct$中的c。显然,延迟的大小依赖于相互作用哈密顿量的某个时间尺度。在考虑时间延迟时我们实际上把时间和空间分开来处理了;在相对论协变的理论中时间和空间必须一起被scaling,因此它们之间自然会有一个比例系数,这个系数的单位是速度,它就是光速的尺度;如果我们认为时间和空间可以分开scaling,就必须认为理论中存在一个速度尺度,否则会错误地认为相对论理论中可以有瞬时相互作用(既然没有任何时间尺度,相互作用时间延迟必定是零,否则不满足标度不变性)。
% TODO:是否能够商掉一些量子数?比如说商掉自旋,在没有自旋轨道耦合的情况下?

如无特殊说明,本文所谓的本征态指的都是归一化的本征矢量。
以下希腊字母的指标跑遍所有时空维度,而拉丁字母的指标仅仅跑遍空间维度,也就是$\mu, \nu, \ldots = 0, 1, 2, 3$而$i, j, \ldots = 1, 2, 3$。
常规斜体字母$x, y, p$等若经说明为多分量对象,默认为四维矢量,相应的,$\vb*{x}, \vb*{y}, \vb*{p}$等为它们的空间部分。
指标$a,b,\ldots$也有可能指多分量对象的指标,未必正好取$1, 2, 3$。
$T$表示编时算符,它是一个元算符,将一系列和时间有关的算符排列成左边时间大、右边时间小的形式。对玻色子而言重新排序后无需改变系数,对费米子而言重排之后的结果还要乘上$-1$的重排次数次方。
$N$表示正规序算符,它也是一个元算符,用于将一系列产生湮灭算符排列成产生算符在左、湮灭算符在右的形式。对玻色子而言重新排序后无需改变系数,对费米子而言重排之后的结果还要乘上$-1$的重排次数次方。
元算符的括号使用方括号。

\section{抽象代数}

\subsection{指标升降}

符号约定:$\vb*{A}^2$代表$\vb*{A}$的模长平方,$A^2$则表示分量。

\subsection{算符和态}

\[
    \comm{\hat{A}}{\hat{B}}^\dagger = \comm{\hat{B}}{\hat{A}},
\]
因此两个算符对易当且仅当它们的共轭转置对易。

% TODO:分析三种绘景下的态
% 两个态表示了同样的物理状态,当且仅当,
% $\ket{\psi}$和$\hat{A}$组成的系统和$\hat{Q}\ket{\psi}$和$\hat{Q} \hat{A} \hat{Q}^\dagger$组成的系统等价,其中$\hat{Q}$是一个幺正算符;反之,如果两个长度等价的向量描述等价的系统,
% TODO: 设算符$\hat{A}$是CSCO,且它在幺正变换$\hat{P}$下不变,那么对任何一个本征值$A_i$,有一个单位复数,使得$\hat{P} \ket{A_i} = c \ket{A_i}$.
% TODO:虽然描写一个态空间可以需要不止一个算符(或者说这个空间的CSCO的大小不为1),但往往可以将这些CSCO拼凑成一个:
% \hat{A} \ket{a_1 a_2 \cdots} = \pmqty{a_1 & a_2 & \cdots} \ket{a_1 a_2 \cdots}
% 只要推导中不涉及本征值的乘除,这样做就没有任何问题。
% 因此下文中将常常这么写。
% 谱结构和对易关系之间有什么联系?

\textbf{表象}指的是态空间的一组正交完备基。由于通常这样一组基是某个CSCO$\hat{M}$的本征态,我们通常使用对应的$CSCO$来标记表象。例如,我们有坐标表象、动量表象,等等。
表象变换公式
\begin{equation}
    \braket{A_i}{\psi} = \sum_j \braket{A_i}{B_j} \braket{B_j}{\psi}
\end{equation}
是基的完备性的推论。

变换
\[
    \braket{A_i}{\psi} \longrightarrow \mel{A_i}{\hat{B}}{\psi}
\]
称为算符$\hat{B}$在$A$表象下的表示。显然,$\hat{A}$在$A$表象下的表示就是
\[
    \braket{A_i}{\psi} \longrightarrow A_i \braket{A_i}{\psi}.
\]

在离散谱的情况下,归一化条件相当简单:
\[
    \braket{A_i}{A_j} = \delta_{ij}
\]
在连续谱的情况下,需要使用积分代替求和,使用$\delta$函数代替$\delta$符号。
多分量算符的本征值有可能不是按照$\reals^n$的方式分布的,而是分布在一个弯曲的空间上(例如,分布在一个球面上)。此时通常需要使用类似于
\[
    \int \dd[n]{x} \delta(f(x))
\]
这样的测度,其中$f(x)$为描述弯曲空间的方程。其结果是,即使两个表象中的态矢量能够做到一一对应,由于使用的测度不同,它们仍然可以差一个模长不为1的系数。
换而言之,相同的态在不同的表象中会以不同的内积被归一化。

\subsection{李群和李代数,以及它们的表示}
% TODO:形如$\exp(\phi_1 G_1 + \phi_2 G_2 + \ldots)$的映射是不是一定可以写成$\exp (\phi_1' G_1) \exp (\phi_2' G_2) \ldots$?
在讨论对称性和守恒量的联系的时候

\subsubsection{从李群到李代数}

本文中我们将不对李群的流形结构进行正式的分析,而仅仅满足于使用一定的群参数把一个李群完整地表示出来。
一个李群中的成员可以一般地写成
\begin{equation}
    g = \exp(\ii \theta_i \sigma_i) \equiv \exp (\ii \theta^i \sigma_i) = \exp (\ii \vb*{\theta} \vb*{\sigma}),
    \label{eq:lie-group-element}
\end{equation}
其中$\theta_i$指的是群参数,而$\sigma_i$指的是生成元。
通常要求群参数为实数。
$\ii$是一个无关紧要的系数,加上它和不加上它唯一的区别就是$\sigma$需不需要乘上一个$\ii$。
为了方便,常常将诸$\theta$排成行向量,$\sigma$排成列向量。由于没有度规,无需区分上下指标。
对应的,设$\theta$是一个群参数,对应的生成元为
\begin{equation}
    \sigma = \frac{1}{\ii} \dv{g}{\theta}.
\end{equation}
需注意\eqref{eq:lie-group-element}假定了
\[
    g(\theta_1) g(\theta_2) = g(\theta_1 + \theta_2),
\]
这又等价于,无论$\theta$取什么值,$g$对$\theta$求导都会得到完全相同的结果。
在大多数情况下可以不失一般性地要求这个性质成立,因为群参数到底是什么并不重要
——我们总是可以巧妙地定义$\theta$使得$g$对$\theta$求导的结果与$\theta$无关%
\footnote{这是来自常微分方程的基本结论:设$X$是一个生成元,那么必定可以找到李群的一个单参数子群$c(t)$,使得
\[
    \dv{t} c(t) = c(t) \cdot X,
\]
从而可以定义指数映射。这是解析映射,因此可以使用诸如求导等运算。},
% 但是真的如此吗?时间演化一定构成李群吗?
% 一种可能的质疑是:在球面上随意画一条闭合轨迹,它显然描述了起点位于球心,终点位于球上面的矢量的一个连续变换,
% 然而它却不能使用$\exp (\alpha G)$的形式表示出来。
% 但这个质疑本身不成立,因为通常的李群总是可以作用在线性空间上的,然而上述变换显然没有线性性。
% 感觉还是很奇怪。
但是有一个重要的例外:时间演化。
我们关注的是“正常人眼中的时间”,而不能随意定义时间流逝的速率,
因此并没有什么能够保证不同$t$处时间演化算符对$t$求导的结果都是$t=0$(也就是恒等映射附近)时间演化算符对$t$求导的结果。
记$U(t, t_0)$为从$t_0$演化到$t$的算符,也即
\[
    U(t, t_0) U(t_0) = U(t),
\]
由于$t$不再能够任意选取,我们不能够写出\eqref{eq:lie-group-element}这样的指数映射,但是可以证明,一定存在一个$H(t)$使得
\begin{equation}
    U(t, t_0) = T \exp \left( \int_{t_0}^t \dd{t} H(t) \right).
    \label{eq:time-dependent-lie-group}
\end{equation}
这里我们略去了\autoref{sec:time-evolution}中的公式中的因子$- \ii /\hbar$,不过这无关紧要。$T$为编时算符。
在不同时刻的$H(t)$彼此对易的情况下可以把$T$去掉,因为此时重排各算符顺序不会产生任何影响。

\eqref{eq:lie-group-element}和\eqref{eq:time-dependent-lie-group}的区别体现在很多地方。
\eqref{eq:lie-group-element}意味着
\[
    g^{-1}(\theta) = g(-\theta),
\]
或者说
\[
    \left( \exp(\theta \sigma) \right)^{-1} = \exp(- \theta \sigma),
\]
但是在不同时刻的$H(t)$彼此不对易时,
\[
    T \exp(\int \dd{t} H(t))^{-1} \neq T \exp(- \int \dd{t} H(t)).
\]
相应的,
\[
    \dv{t} \left(T \exp(\int \dd{t} H(t))^{-1}\right) \neq -H.
\]
这就是\autoref{sec:time-evolution}中做绘景变换时不同绘景下的哈密顿算符不相等的根本原因。

李代数是李群在单位元附近的切空间,也就是说,是$g$在$\theta=0$附近沿着任意方向对$\theta$求导之后得到的结果组成的代数。
接下来我们将讨论\eqref{eq:lie-group-element}的李群,因为“不同点处求导结果不同”基本上只会在处理时间演化时用到,
而此时只有一个生成元(就是哈密顿量),没有必要讨论李代数。
由于李代数的封闭性,设$g_1, g_2, \ldots$是一组相互独立的生成元,它们中任意两个的李括号$\comm*{g_1}{g_2}$一定也是一个生成元,
这意味着它可以使用$g_1, g_2, \ldots$线性表示。
从而我们有
\begin{equation}
    \comm*{g_i}{g_j} = f_{ij}^k g_k.
    \label{eq:structure-of-lie-algebra}
\end{equation}
如果我们只讨论抽象的李代数的性质而不考虑它作用在某些对象上产生的结果,那么\eqref{eq:structure-of-lie-algebra}就完全刻画了一个李代数的结构。
因此,称$f_{ij}^k$为\textbf{结构常数}。

\subsubsection{李代数的具体计算}

% TODO:把前面用到这一节的内容的部分写得更加简洁一些
若
\[
    \comm*{\hat{q}}{\hat{p}} = c,
\]
则
\[
    \comm*{\hat{q}}{\hat{p}^n} = n c \hat{p}^{n-1}.
\]

\subsubsection{表示论}\label{sec:rep-th}

接下来需要讨论李群和李代数的表示。
通常考虑两种表示,其一是李群和李代数在向量空间上的作用,
也就是说,我们在李群、李代数和向量空间上的算符组成的群(以算符的复合为乘法)之间建立一个同态,
一旦建立起这个同态,我们实际上就得到了李群或李代数的一个表示。
比较方便的做法是,先讨论李代数在特定向量空间上的表示,然后使用指数映射获得对应的李群的表示。
第二种表示是,李群和李代数在向量空间上的算符构成的向量空间上的作用。
这种表示和第一种表示是紧密相关的。
设李群$G$在向量空间$V$上的表示为$G_V$,则$G_V \subset GL(V)$。这就自然地诱导出了李群在$GL(V)$上的表示。
算符$\hat{B} \in GL(V)$建立起了这样的关系:
\[
    \phi = \hat{B} \psi,
\]
现在我们把$\hat{A} \in G_V$作用在$\phi$和$\psi$上面,就得到
\[
    \phi' = \hat{A} \phi, \quad \psi' = \hat{A} \psi,
\]
如果我们还是希望在$\phi'$和$\psi'$之间建立关系
\[
    \phi' = \hat{B}' \psi',
\]
应该怎么选取$\hat{B}$?
考虑到$\phi$和$\psi$的任意性,容易看出,
\[
    \hat{B}' = \hat{A} \hat{B} \hat{A}^{-1}.
\]
我们没有规定$\hat{B}$是什么——它是完全任意选取的。这样一来,$G_V$中的每一个元素$\hat{A}$都对应到下面的映射:
\begin{equation}
    \hat{B} \longrightarrow \hat{A} \hat{B} \hat{A}^{-1},
    \label{eq:group-action-on-operators}
\end{equation}
\eqref{eq:group-action-on-operators}是一个从$GL(V)$到$GL(V)$的映射,也就是满足封闭性。
请注意该映射是$GL(GL(V))$的成员,而不是$GL(V)$的成员——它作用在$V$上的算符上而不是$V$中的向量上。
因此,我们通常只讨论简单的向量空间上的群表示,因为这些向量空间上的算符组成的向量空间上的群表示可以使用前者按照\eqref{eq:group-action-on-operators}写出。
另外注意,不同的$\hat{A}$可能对应着同一个\eqref{eq:group-action-on-operators}型的从算符到算符的映射。
这一点在\autoref{sec:rotation}中体现得很明显。

李群和李代数通常被作用在几类向量空间上。
首先是有有限个分量的向量空间。李群在其上的作用形如
\[
    v \longrightarrow v', \quad (v')^\mu = R_{\nu}^\mu (\Lambda) v^\nu.
\]
其中$\Lambda$指抽象的李群。
在有限维向量空间$V$上的表示可能有不变子空间,也就是说,存在$V$的一个子空间$V'$,使得李群中的任何一个成员作用在$v \in V'$上之后得到的结果都还是在$V'$中。当然,$V$以及$\{0\}$一定是不变子空间。
如果一个表示有不是这两个空间的不变子空间,那么这就是一个\textbf{可约表示},反之则为\textbf{不可约表示}。
可以证明,任何一个可约表示都可以写成一系列不可约表示的直和。因此对有限维表示而言,只需要讨论不可约表示就可以了,因为可约表示可以使用不可约表示组装出来。
现在讨论不可约有限维表示。
首先可以证明,任何李群的生成元至少有一个(当然也可以有很多个)可以相似变换为对角矩阵。
% TODO:是不是每一个生成元都可以?
这些被对角化的生成元的集合称为Cartan子代数,它是对应的李群的李代数的表示的子代数。
Cartan子代数中的诸算符共享一组可以张成整个$V$的本征矢量,对应的各生成元的本征值——也就是对角矩阵的对角元——可以用来标记这个不可约表示。
要找到一组Cartan子代数并不难:只需要从李群中找到一个交换子代数,然后尝试对角化这个交换子代数中的某一个成员就可以了。
% TODO:李代数在怎样的程度上决定了对应的算符的谱结构?
非奇异矩阵表示一定可以通过相似变换而变成幺正表示(就是所有矩阵都是幺正的表示)。
这也就是我们频繁地讨论幺正表示的原因。但有许多重要的群——例如洛伦兹群——都不是紧致的(或者说群对应的流形无界),因此它们实际上并没有有限维的幺正表示。就洛伦兹群而言,我们将会看到,其推动生成元的有限维表示不是厄米的,因此整个群也没有幺正的有限维表示。

容易验证,设$\hat{X}$是厄米算符,且
\begin{equation}
    \comm*{\hat{a}^\dagger}{\hat{X}} = c \hat{a}^\dagger,
    \label{eq:raising-operator}
\end{equation}
那么
\[
    \hat{a}^\dagger \ket{X} \propto \ket{X+c},
\]
相应的,
\[
    \hat{a} \ket{X} \propto \ket{X-c}.
\]
因此称$\hat{a}^\dagger$为$\hat{X}$的\textbf{升算符},$\hat{a}$为$\hat{X}$的\textbf{降算符}。
数学上可以证明,在李代数的有限维表示上可以定义内积
\begin{equation}
    \langle \hat{A}, \hat{B} \rangle = \trace \hat{A} \hat{B},
\end{equation}
通过合适的线性组合,能够写出一组正交归一化的生成元。
此时非Cartan子代数的生成元中的每一个都是Cartan子代数中的每一个成员的升降算符,
并且任意两个非Cartan子代数的生成元的对易子都可以使用Cartan子代数的成员线性表示。
% TODO:看起来Cartan子代数似乎构成它的不可约表示空间的一个CSCO
% Symmetry and the Standard Model, p108
因此对一个一般的、没有正交归一化的李代数的有限维表示,我们总是可以从李代数的成员构造出一个升算符。设$\hat{X}$为$g_i$,且
\[
    \hat{a}^\dagger = \lambda^j g_j,
\]
则\eqref{eq:raising-operator}等价于
\[
    \comm*{\lambda^j g_j}{g_i} = c \lambda^j g_j,
\]
代入\eqref{eq:structure-of-lie-algebra},上式又等价于
\begin{equation}
    \left( f^k_{ji} - c \delta_j^k \right) \lambda^j = 0,
    \label{eq:determine-ladder-operators}
\end{equation}
于是通过求解
\begin{equation}
    \det \left( f^k_{ji} - c \delta_j^k \right) = 0
    \label{eq:possible-c}
\end{equation}
就可以得到所有可能的$c$,然后将它们代入\eqref{eq:determine-ladder-operators}就能够得到所有能够被非Cartan子代数表示出来的升降算符。
最后,由于是有限维表示,通过以上手法得到的升降算符实际上就是全部可能的升降算符,因此从一个本征态出发,通过它们可以构造出所有的本征态。
有限维表示还意味着,设$\hat{a}^\dagger$是某个升算符,那么对充分大的$N$,$(\hat{a}^\dagger)^N = 0$,$\hat{a}^N=0$,因为本征态的个数有限。
这些条件可用于确定什么样的不可约表示是被允许的。
% TODO:数学证明,不过多半鸽了
这些操作的一个典型的例子见\autoref{sec:rotation}。

现在我们分析一种比较特殊的情况。以上我们都是在“李代数可以分解成一个Cartan子代数和非Cartan元素,后者构成前者的升降算符”的框架下分析问题,那么如果李代数中所有元素都对易,那此时它会有怎样的表示?
由于没有非Cartan元素,这样的一个李代数——从而它的李群——不会有有限维的不可约表示。
通常这样的李群对应着某种平移操作,详情见\autoref{sec:translation}。

% TODO:连续谱的情况
以上讨论的不可约表示都是有限维的。无限维表示——这里指的是函数空间的表示——则需要一套不同的框架。设$\hat{q}$具有连续谱,且
\begin{equation}
    \comm*{\hat{q}}{\hat{p}} = \ii,
\end{equation}
则
\begin{equation}
    \exp \left( \ii \lambda \hat{p} \right) \ket{q} = \ket{q + \lambda}.
\end{equation}
也就是说$\exp (\ii \lambda \hat{p})$是让$\hat{q}$的本征矢对应的本征值上升$\lambda$的升算符。

由于空间坐标无非是一种向量,李群和李代数也可以被作用在坐标上。
作用在坐标上的有限维表示又诱导出了作用在函数上的无限维表示%
\footnote{在有限维表示中,上下标$\mu$标记向量的诸分量;在函数空间中,坐标$x^\mu$标记“向量”——也就是函数——的诸“分量”——也就是函数在这一点的值。
李群在有限维向量空间上的表示通常是某个矩阵群,它将不同分量混合在一起,即
\[
    \psi^\mu \longrightarrow R^\mu_\nu \psi^\nu.    
\]
李群在无限维向量空间上的表示通常是“改变坐标$x^\mu$”。
}%
。设$f=f(x)$,若李群在坐标上的表示为
\[
    x \longrightarrow x', \quad (x')^\mu = R_\nu^\mu (\Lambda) x^\nu,
\]
则它在关于坐标的函数——也就是“场”——组成的无限维向量空间上的表示就是
\[
    f \longrightarrow f', \quad f(x) = f'(x') = f'(R(\Lambda) x),
\]
或者等价的,
\begin{equation}
    (x \mapsto f(x)) \longrightarrow (x \mapsto f'(x) = f(R(\Lambda)^{-1} x)).
    \label{eq:infinite-dim-rep}
\end{equation}
换而言之,坐标变动“牵引”了从坐标到场值的映射。
考虑到$f$可能是某个多分量对象(比如矢量、矢量的张量积,或者接下来要看到的旋量)的分量,
李群在此多分量场上的作用还包括通常的有限维表示,也就是
\[
    \psi^a \longrightarrow M(\Lambda)^a_b \psi^b.
\]
需注意此处我们使用了另外一个表示$M^a_b$而不是$R^\mu_\nu$,因为不能够保证$\Lambda$在多分量场$\psi$上的作用和它在坐标向量上的作用来自同一个有限维表示。
由于大部分情况下我们都是从一个群在通常意义上的矢量的作用出发讨论其结构的,可以将$R(\Lambda) x$简记为$\Lambda x$,也就是群元$\Lambda$在$x$上的作用。
这样上式就可以简洁地写成
\begin{equation}
    \psi^a(x) \longrightarrow {\psi'}^a (x) = M^a_b (\Lambda) \psi^b (\Lambda^{-1} x).
    \label{eq:wigner-transform}
\end{equation}
这种同时考虑了多分量场在李群作用下各分量重新混合(这是一个有限维表示)和李群作用下坐标拖曳而改变场(这对坐标而言是另一个有限维表示,对场而言是一个无限维表示)的李群的表示就是\textbf{场表示}。
需要注意的是,不同的$\Lambda$作用到坐标上可能会得出同样的结果,而它们对应的$M$作用到场上却有不同的结果,正如$SU(2)$和$SO(3)$的关系告诉我们的那样。

\eqref{eq:wigner-transform}给出的是李群的场表示的一般形式,但此时我们还只有形式上的变换而没有显式的表达式。
我们来分析其李代数。取%
\footnote{虽然可以任意地调整群参数,从而让生成元前面的系数随意变动,但是通常对有限维表示和无限维表示我们总是采用同样的群参数。这就意味着,在有限维表示确定之后不能随意调节无限维表示的生成元前面的系数,不能随意加一个$\ii$或者改变正负号。这也就是我们在场表示中一并处理有限维表示和无限维表示的原因,因为此时两者的群参数自动地就是相同的。

下式中的$g$的定义可以是\[
    g = \frac{1}{\ii} \pdv{G}{g},
\]
但也可以是像我们定义旋转生成元时的那样,取
\[
    g = \ii \pdv{G}{g},
\]
只需要将$\epsilon$取为负值就可以了。无论$g$是怎么定义的,下式都是成立的。}%
\[
    \Lambda = I + \ii \epsilon g,
\]
其中$g$是一个生成元,我们就有
\[
    \begin{aligned}
        \psi^a \longrightarrow {\psi'}^a &= M^a_b (\Lambda) \psi^b (\Lambda^{-1} x) \\
        &= (I + \ii \epsilon M^a_b(g)) \psi^b (x - \ii \epsilon g x) \\
        &= (I + \ii \epsilon M^a_b(g)) (\psi^b - \ii \epsilon g x \cdot \grad{\psi^b}) \\
        &= \psi^b + \ii \epsilon M^a_b(g) \psi^b - \ii \epsilon g x \cdot \grad{\psi^b},
    \end{aligned}
\]
于是
\[
    {\psi'}^a = (I + \ii \epsilon  (M^a_b(g) - g x \cdot \grad)) \psi^a,
\]
于是场表示的生成元可以写成
\begin{equation}
    M_\text{field} = M_\text{fin} + M_\text{inf}, \quad M_\text{fin} = M^a_b(g), \quad M_\text{inf} = - (g x) \cdot \grad.
    \label{eq:fin-and-inf-rep}
\end{equation}
其中$M_\text{fin}$就是我们所熟悉的李群在有限维向量空间上的矩阵表示,而$M_\text{inf}$则是李群作用在坐标上,拖曳坐标而对场产生的影响。
显然,它们和$g$之间能够建立同态关系。$gx$和$\Lambda x$一样,都是“$g$在坐标空间上的有限维矩阵表示作用于$x$”的简写。
与通常物理中的记号不同,此处的梯度算符作用在所有坐标上,不仅仅是空间坐标,还包括时间坐标。

在以上讨论的基础上我们讨论态矢量。我们总是使用李群在希尔伯特空间上的幺正表示,因为需要保证变换前后的态矢量都是物理的,也就是说,都是正交归一化的。
我们刚才讨论了李群的场表示,这个场表示当然可以被作用在算符场上。但是注意到算符场是态空间上的算符,因此按照\eqref{eq:group-action-on-operators},李群的场表示自然地如下导出了李群在希尔伯特空间上的表示:
\begin{equation}
    \hat{U}(\Lambda) \hat{\psi}^b(\vb*{x}) \hat{U}^{-1}(\Lambda) = M^a_b (\Lambda) \psi^b (\Lambda^{-1} x).
    \label{eq:field-rep-and-state-rep-lie-group}
\end{equation}
由于对$\hat{\phi}$的变换等价于对其本征值做变换,这又等价于保持本征值不变而重新安排本征态,按照上式诱导出的在希尔伯特空间上的李群表示$\hat{U}$也是幺正的。
\eqref{eq:field-rep-and-state-rep-gen}看起来有些奇怪:为什么等式左边是$U \psi U^{-1}$的形式,而等式右边是$M \psi$的形式?
实际上,等式左边的$U$是普通的算符,它作用于希尔伯特空间上,而右边的$M$实际上起到了“元算符”的作用,它直接改变了算符$\psi$本身。
\eqref{eq:field-rep-and-state-rep-gen}实际上给出了把元算符作用在算符上的结果显式地写出来的方式。

相应的,\eqref{eq:field-rep-and-state-rep-lie-group}也导致了对应的李代数在希尔伯特空间上的表示。对\eqref{eq:field-rep-and-state-rep-lie-group}取微元,得到
\[
    (1 + \ii \epsilon M_\text{state}) \hat{\psi}^b (1 - \ii \epsilon M_\text{state}) = \ii \epsilon M_\text{field} \hat{\psi},
\]
从而
\begin{equation}
    \comm*{M_\text{state}}{\psi} = M_\text{field} \psi.
    \label{eq:field-rep-and-state-rep-gen}
\end{equation}
实际上,时间演化方程\eqref{eq:quantum-evolution}就是一个例子:时间平移群在希尔伯特空间上的表示是哈密顿算符$\hat{H}$,在场——这里是任何一种物理量——上的表示是$\frac{1}{\ii} \dv{t}$,那么
\[
    \comm*{\hat{H}}{\hat{A}} = \frac{1}{\ii} \dv{t},
\]
这就是时间演化方程。

考虑一个简单的单粒子量子力学的例子:$\hat{x} + a$是将大小为$a$的平移作用在$\hat{x}$上的结果,而考虑被$\hat{x}$完全描述的一个希尔伯特空间,在其上有
\[
    \hat{x} + a = \int \dd{x} x \dyad{x} + a \int \dd{x} \dyad{x} 
    = \int \dd{x} (x + a) \dyad{x} = \int \dd{x'} x' \dyad{x'-a},
\]
因此作用在$\hat{x}$上的大小为$a$的平移就等价于作用在态空间基矢量上的大小为$-a$的平移。
更一般的,将某一个李群$Q(a)$作用在某一算符上就相当于将这一李群的群参数倒转过来得到新的李群$Q'$,
也就是定义$Q'(a) = Q(a)^{-1}$(由于是群,$Q'$和$Q$同构),然后将$Q'(a)$作用在态空间的基矢量上。
由于$Q'$和$Q$同构,两者的区别仅仅是重新规定了群参数,因此它们对应着同样的对称性。
% TODO:以上说法的推广
总之,我们既可以直接从某种李群的场表示出发,推导它允许的算符场有哪些,然后使用二次量子化的有关知识导出其对应的单粒子态,%
\footnote{关于何为“粒子”需要说明:一般把能够使用一个不很复杂的CSCO描述的量子系统称为粒子,例如可以使用$\hat{\vb*{x}}$或$\hat{\vb*{p}}$描述一个粒子。但按照这种定义,原子能级也可以算粒子了——实际上这并不是胡思乱想,在处理量子光学等领域的一些问题时确实可以将能级看成一种粒子,定义其产生湮灭算符,得到费米场,等等——因此,何为粒子更多的是一种约定的说法。实际上任何一个哈密顿量都可以对角化,写出能级之后将不同能级看成不同粒子,然后使用二次量子化的语言描述它。}%
也可以从李群在希尔伯特空间上的表示出发,直接得到单粒子态然后构造算符场。
两种方法是完全一致的。舒尔引理告诉我们,卡西米尔算符(和所有生成元都对易)在不可约表示中一定是恒等算符的常数倍。这个常数可以用来标记相应的不可约表示;事实上这一类常数往往会出现在相应的表示描写的场/粒子的运动方程中,因为运动方程中会出现卡西米尔算符的场表示。
相对而言,在推导运动方程的时候,使用场的观点更加方便,因为相对论情况下粒子数通常是不确定的,因此使用单粒子态难以写出哈密顿量。

概括以下我们至今得到的结果:李群和李代数的表示有下面几种,它们彼此之间有非常密切的关系。
首先,李群和李代数在有限维向量空间上的表示是矩阵,它们或者是可约表示,或者不可约,前者可以通过直和运算由后者组装出来。
不可约有限维表示的结构可以通过使用李代数中的非Cartan元素构造Cartan子代数的升降算符来确定。
通过将有限维表示作用在坐标上,我们得到了作用在关于坐标的函数组成的向量空间上的无限维表示。
将作用在多分量对象上的有限维表示和作用在坐标函数上的无限维表示结合起来,就得到了场表示。
李群在向量空间上的表示很自然地就诱导出了李群在作用在向量空间上的算符上的表示。

$S_n$ n阶排序群,有$n!$个元素
$A_n$ n阶置换(偶排列)群,有$n!/2$个元素?
$Z_n$ n阶离散旋转群,可以使用复数$\ee^{\ii 2\pi j / n}$,有$n$个元素

$SL(n, \reals)$ $n$维实矩阵中行列式为$1$的矩阵组成的群
$SL(n, \complexes)$ $n$维实矩阵中行列式为$1$的矩阵组成的群
$U(n)$ $n$维酉变换组成的群
$O(n)$ $n$维正交变换组成的群
$SO(n)$ $n$维旋转组成的群

\section{动力学}

% 似乎拉格朗日动力学中含有虚部的场要看成两个场,而哈密顿动力学中含有虚部的场只需要看成一个场。
在进一步展开下面的叙述之前,我们先回顾现代物理的数学框架。总的来说,有两套可用的框架,
其一是拉格朗日动力学,路径积分方法是它的量子版本;其二是哈密顿动力学,正则量子化是它的量子版本。
尽管这两个框架在数学上是独立的,我们仍然可以找到它们之间非常深厚的联系。

本节首先从经典拉氏量出发,然后得到经典哈密顿量,然后再过渡到量子形式。
常见的物理问题涉及$3+1$维闵可夫斯基时空中或$0+1$维时空,而后者可以看成前者的一个退化情况,
于是我们将局限在$3+1$维闵可夫斯基时空中,
虽然无论是拉格朗日动力学还是哈密顿力学都适用于比这广得多的体系。
所谓闵可夫斯基时空指的是度规可以化为
\begin{equation}
    \eta_{\mu\nu} = \diag (1, -1, -1, -1)
\end{equation}
的四维几何。通常使用$t, x, y, z$或者$x^0, x^1, x^2, x^3$来依次标记这4个坐标。
容易看出$x, y, z$或者说$x^1, x^2, x^3$就构成一个三维欧氏几何,它们是\textbf{空间维}。
$x^0$则是\textbf{时间维}。

我们还将假设,所有场量在无穷远处的值都是零。
我们将要分析的对象是时空中的场,它是从闵可夫斯基时空到某一线性空间的光滑映射。

\subsection{拉格朗日动力学}

所谓\textbf{拉氏量密度}$\mathcal{L}$——在场论中简称为\textbf{拉氏量}——是这样一个量,它是场的局域泛函,
这就是说,它可以写成$\phi, \partial_\mu \phi, \ldots$以及时空坐标的函数。
本文假定所有的拉氏量仅含有一阶导数,这是为了避免含有高阶导数的拉氏量产生“可以无穷下降的能量”等反直觉现象,并且简化计算。
幸运的是,已有的实验数据并不要求我们考虑更高阶的拉氏量。
我们还假定物理规律在时空上是均匀的,因此我们不认为拉氏量中显含时空坐标。%
\footnote{
    需要注意的是在系统中有相互作用且其中一部分的运动状态已知的情况下,另一部分的等效拉氏量中是有可能出现时空坐标的,
    例如粒子在势场中的运动就是一个典型例子,在那里由于产生势场的物理机制远远比粒子本身要强,因此势场可以看成是给定的,
    于是粒子具有的等效拉氏量就显含了空间坐标。}%
从而我们有
\begin{equation}
    \mathcal{L} = \mathcal{L}(\phi, \partial_\mu \phi).
    \label{eq:lagrangian}
\end{equation}
需要注意的是\eqref{eq:lagrangian}中的$\phi$可以代表任何一个“从时空坐标到数量”的映射,
它可能是一个标量场也可能是一个矢量场的分量,或者是别的什么东西。
\textbf{作用量}是拉氏量在整个闵可夫斯基时空上的积分。

现在我们将一个任意的无穷小变换作用在泛函$S$上,观察其无穷小变动。
需要注意的是无穷小变换同时作用在$\phi$的场值和坐标上,从而$\phi$完整的变化%
\footnote{在实际计算时往往更加容易求出$\var{\phi}$,因为一旦把$\phi'(x')$完全写出,只需要计算$\phi'(x')-\phi(x)$ 即可。}%
同时包含两部分:
\begin{equation}
    \var{\phi} = \bar{\var} \phi + \partial_\mu \phi \var{x}^\mu,
    \label{eq:variance-of-phi}
\end{equation}
其中第一项指的是场值本身的变化%
\footnote{这个变化又有可能来自两个方面。
其一是“场的平移”,也就是我们手动把场$\phi$加减特定值;
其二是“场的旋转”,当$\phi$实际上是某个更大的对象(如矢量)的某个分量时,基矢量的旋转会导致不同的分量混在一起。
通常我们使用一样的基矢量来书写场的分量和坐标的分量,因此除了坐标平移外,坐标变换也伴随着非零的$\bar{\var}{\phi}$。}%
,第二项指的是坐标变换的“拖曳”作用。
坐标的变化还会导致导数算符和积分测度发生变化。这两个几何效应的具体表达式为
\begin{equation}
    \begin{bigcase}
        \partial_{\mu'} = \partial_\mu - \partial_\mu \var{x^\nu} \partial_\nu, \\
        \dd[4]{x'} = (1 + \partial_\mu \var{x^\mu}) \dd[4]{x}.
    \end{bigcase}
\end{equation}
由于$\partial_\mu$算符随着坐标变换会发生变化,我们发现$\partial_\mu \phi$的变化量的形式和$\phi$不完全一致:
\begin{equation}
    \var{\partial_\mu \phi} = \partial_\mu \bar{\var}{\phi} + \partial_\mu \partial_\nu \phi \var{x^\nu}.
\end{equation}
这样一来我们可以计算出
\begin{equation}
    \var{S} = \int \dd[4]{x} \left(
        \left( \pdv{\mathcal{L}}{\phi} - \partial_\mu \pdv{\mathcal{L}}{\partial_\mu \phi} \right) \bar{\var}{\phi} + 
        \partial_\mu \left( \mathcal{L} \var{x^\mu} + \pdv{\mathcal{L}}{\partial_\mu \phi} \bar{\var}{\phi} \right)
    \right).
    \label{eq:variance-of-s}
\end{equation}
在推导\eqref{eq:variance-of-s}时我们没有使用任何关于$\var{\phi}$和$\var{x}$的假设,因此它给出的是最一般的$\var{S}$形式。

实际的场的动力学由保持时空坐标$x$不变且$\phi$在无穷远处固定为零(从而无穷远处$\bar{\var}{\phi}$为零)的情况下的泛函极值问题
\begin{equation}
    \var{S} = \var{\int \dd[4]x \mathcal{L}(\phi, \partial_\mu \phi)}
    \label{eq:min-action}
\end{equation}
给出。
显然这个泛函极值问题的解就是
\begin{equation}
    \pdv{\mathcal{L}}{\phi} - \partial_\mu \pdv{\mathcal{L}}{\partial_\mu \phi} = 0.
    \label{eq:el-eq}
\end{equation}
这就是欧拉-拉格朗日方程。
由于推导欧拉-拉格朗日方程时用到了$\var{\phi}$的任意性,这意味着$\phi$被假定是一个实的场。
如果某些场有虚部,那么在使用\eqref{eq:el-eq}以及相关结论的时候需要把它的实部和虚部分开,当成两个场来处理。
并且,容易证明,设复场$\phi$的实部和虚部分别是$\phi_1$和$\phi_2$,且
\[
    \pmqty{\psi_1 \\ \psi_2} = \pmqty{a & b \\ c & d} \pmqty{\phi_1 \\ \phi_2},
\]
其中$a,b,c,d$为复常数,则$\psi_1$和$\psi_2$的运动方程也可以从\eqref{eq:el-eq}得出。
常见的选择包括取
\[
    \psi_1 = \phi, \psi_2 = \phi^\dagger,
\]
或者如果$\phi$是多分量场,设有一系列复矩阵(不必都是复矩阵,有一个是复的就可以)$\gamma^\mu$,取
\[
    \psi_1 = \phi, \psi_2 = \gamma^\mu \phi_\mu.
\]

需要注意如果两个拉氏量的形式不同,这并不意味着它们描述了不同的物理过程。
实际上容易看出,两个拉氏量描述了相同的物理过程,
当且仅当,它们给出的作用量$S$只相差一个相对于$\dd[4]{x}$的零测集上的积分(这样的积分不影响泛函极值问题的求解,因为它“太小”),
这又等价于这两个拉氏量相差一个散度项,即存在一个$\Lambda^\mu$使得
\begin{equation}
\mathcal{L}' = \mathcal{L} + \partial_\mu \Lambda^\mu.
\end{equation}

当场量$\phi$是物理解的时候,将$\phi$代入到$S$中,然后再做一个无穷小变换,此时\eqref{eq:variance-of-s}中的第一项为零,
于是我们有
\[
    \var{S} = \int \dd[4]{x} \partial_\mu \left( \mathcal{L} \var{x}^\mu + \pdv{\mathcal{L}}{\partial_\mu \phi} \bar{\var}\phi \right).
\]
如果这个无穷小变换实际上不改变系统的动力学,也就是说系统在这个无穷小变化下是对称的,
那么$\var{S}$就应该能够写成一个表面积分,于是我们得到
\begin{equation}
    \partial_\mu \left(\pdv{\mathcal{L}}{\partial_\mu \phi} \bar{\var}\phi + \mathcal{L} \var{x^\mu} + \Lambda^\mu\right) = 0.
    \label{eq:noether}
\end{equation}
当然,如果无穷小变换更进一步不改变拉氏量,那么$\Lambda=0$。

如果无穷小变换是一个李群的李代数的表示,那么$\bar{\var}{\phi},\var{x^\mu}, \Lambda^\mu$都是完全确定的。可以使用小量近似将$\bar{\var}{\phi}$写成小量$ \ii \epsilon$乘以李代数的场表示\eqref{eq:fin-and-inf-rep},$\var{x^\mu}$写成小量$\ii \epsilon$乘以李代数的四维矢量表示,
于是我们在\eqref{eq:noether}中除去一个$\epsilon$,就得到了一个守恒流。
于是\eqref{eq:noether}的括号中的内容能够完全写成坐标的函数。
这就是\textbf{诺特定理}:系统的无穷小对称性诱导出一个守恒流。
由于是四维闵可夫斯基时空,四维的一个守恒流
\begin{equation}
    \partial_\mu j^\mu = 0
\end{equation}
就意味着三维的一个输运方程
\begin{equation}
    \partial_t j^0 + \partial_a j^a = 0.
\end{equation}
从而,
\begin{equation}
    Q = \int \dd[3]x j^0
\end{equation}
就是一个\textbf{守恒荷}。如果其积分范围是一个有限的区域,那么它就是一个局域守恒量,也就是
\[
    \dv{t} Q = - \int \dd{\vb*{S}} \cdot \vb*{j},
\]
而如果其积分范围是全空间,那么它就是守恒的。

我们来检查一下常见的对称性导致的守恒量(这些对称性的数学结构参见\autoref{sec:symmetry})。%
\footnote{表面上看,下面的讨论在体系并不非常对称的情况下并无意义,而不非常对称的体系占了多数。
不对称性带来的后果是,我们不再有完美的守恒流方程,取而代之的是一个有源的输运方程
\[
    \partial_\mu j^\mu = \text{something},
\]
由于对称性分析无助于找到源的具体形式,使用对称性诱导出特定的物理量似乎并没有什么意义。
然而,我们相信,最基本的物理定律总应该是对称的,因此大部分体系的不对称性可以归结为我们人为地将它从环境中隔离出来进行研究,从而导致类似下面的方程:
\[
    \partial_\mu (j^\mu_\text{sys} + j^\mu_\text{env}), \quad \partial j^\mu_\text{sys} = - j^\mu_\text{env}
\]
第二个方程给出了我们想要的含源的输运方程。因此在分析基本的物理框架时我们可以不讨论“不对称”的情况,
而是导出了基本的方程之后再通过“隔离出一部分系统”来引入不对称性。
}%
假定拉氏量在变换下不变。下面处理的问题都只含有一个场,不过由拉氏量的叠加性,在拉氏量含有多个场的时候只需要把各部分加起来即可。
首先是最简单的平移。处理平移时假定场是标量场,这无损一般性,因为平移没有有限维表示,因此不会导致场分量发生混合。
平移变换作用于场上得到的结果是:
\[
    \begin{split}
        x^\mu \longrightarrow x^{\mu'} = x^\mu + a^\mu, \\
        \var{\phi} = \phi'(x') - \phi(x) = 0.
    \end{split}
\]
% TODO:群作用怎么取
按照\eqref{eq:variance-of-phi},可以计算出
\[
    \bar{\var}{\phi} = - \partial_\mu \phi \var{a^\mu},
\]
或者,由于场在坐标拖曳下的变动实际上就是平移变换的无限维表示,可以直接使用\eqref{eq:transition-inf-rep}得到上式。
于是对应的守恒流为
\[
    0 = \partial_\mu \left( - \pdv{\mathcal{L}}{\partial_\mu \phi} \partial_\nu \phi \var{a^\nu} + \mathcal{L} \var{a^\mu} \right) 
    = \partial_\mu \left( - \pdv{\mathcal{L}}{\partial_\mu \phi} \partial_\nu \phi + \mathcal{L} \delta^\mu_\nu \right) \var{a^\nu},
\]
考虑到$\var{a^\mu}$的任意性,我们有
\begin{equation}
    T_\mu^\nu = \pdv{\mathcal{L}}{\partial_\nu \phi} \partial_\mu \phi - \mathcal{L} \delta^\nu_\mu, \quad \partial_\nu T_\mu^\nu = 0.
\end{equation}
我们称$T^\nu_\mu$为\textbf{能动张量}。它给出了4个守恒荷,其中一个是来自时间平移不变性的\textbf{能量}
\begin{equation}
    E = \int \dd[3]{x} T^0_0 = \int \dd[3]{x} \left( \pdv{\mathcal{L}}{\partial_0 \phi} \partial_0 \phi - \mathcal{L} \right) ,
    \label{eq:field-energy}
\end{equation}
另外三个是来自空间平移不变性的\textbf{动量}
\begin{equation}
    P_i = \int \dd[3]{x} T^0_i = \int \dd[3]{x} \pdv{\mathcal{L}}{\partial_0 \phi} \partial_i \phi .
    \label{eq:field-momentum}
\end{equation}
能动张量的纯空间部分是能量和动量的输运流,因此就是\textbf{应力张量}。%
\footnote{在非相对论连续介质力学中这些结果也是成立的,因为时间和空间平移同时出现在伽利略群和庞加莱群中。}
相应的,
\begin{equation}
    \mathcal{P}_\mu = \pdv{\mathcal{L}}{\partial_0 \phi} \partial_\mu \phi - g_\mu^0 \mathcal{L}
\end{equation}
为四维动量$(E, \vb*{p})$的密度。
在计算场的三维动量时要注意一点:由于闵可夫斯基度规为$(+, -, -, -)$,闵可夫斯基时空中空间部分的基矢量实际上是指向空间坐标减少的方向的。从而,
\[
    \begin{aligned}
        \vb*{P} &= \int \dd[3]{x} \pdv{\mathcal{L}}{\partial_0 \phi} \partial_i \phi \vb*{g}^i \\
        &= - \int \dd[3]{x} \pdv{\mathcal{L}}{\partial_0 \phi} \partial_i \phi \vb*{g}^i_{\text{3dim}},
    \end{aligned}
\]
也即
\begin{equation}
    \vb*{P} = - \int \dd[3]{\vb*{x}} \pi \grad{\phi}.
\end{equation}

接下来是旋转对称性。%
\footnote{同样,这个对称性无论是在相对论性场论还是非相对论性场论中都是成立的。}%
旋转对称性不涉及时间维,于是我们有
\[
    \var{x^i} = \epsilon^i_{\ jk}  x^j \theta^k,
\]
相应的
\[
    \bar{\var}{\phi^a} = \ii (J_i)^a_{\ b} \theta^i \phi^b - \epsilon^i_{\ jk}  x^j \theta^k \partial_i \phi^a.
\]
其中指标$a,b$跑遍$\phi$的所有分量,$J$指的是旋转生成元在$\phi$所属的向量空间上的表示,具体表达式见\autoref{sec:rotation}。
我们使用了$\epsilon^i_{jk}  x^j \theta^k$这样的记号是因为要与\eqref{eq:generators-of-so3}保持一致,因为若取$J$为\eqref{eq:generators-of-so3},正好就有
\[
    \epsilon^i_{\ jk}  x^j \theta^k = (\ii \theta^k J_k x)^i.
\]
则诺特定理导致的守恒流为
\[
    \begin{aligned}
        0 &= \partial_0 \left( \pdv{\mathcal{L}}{\partial_0 \phi^a} \left( \ii (J_k)^a_b \theta^k \phi^b - \epsilon^i_{\ jk}  x^j \theta^k \partial_i \phi^a \right) \right) + \partial_i (\text{something}) \\
        &= \theta^k \partial_0 \left( \ii (J_k)^a_b \phi^b \pdv{\mathcal{L}}{\partial_0 \phi^a} - \epsilon^i_{\ jk} x^j \mathcal{P}_i \right) \\
        &= \theta^k \partial_0 \left( \ii (J_k)^a_b \phi^b \pdv{\mathcal{L}}{\partial_0 \phi^a} + \epsilon^i_{\ jk} x^j \mathcal{P}_i \right),
    \end{aligned}
\]
于是我们就得到了一个守恒量 % TODO:正负号
\begin{equation}
    L_k = \int \dd[3]{\vb*{x}} \left( \ii (J_k)^a_b \phi^b \pdv{\mathcal{L}}{\partial_0 \phi^a} + \epsilon^i_{\ jk} x^j \mathcal{P}_i \right).
    \label{eq:field-angular-momentum}
\end{equation}
其中的
\begin{equation}
    \mathcal{M}_k = \epsilon^i_{\ jk} x^j \mathcal{P}_i 
\end{equation}
部分和质点的角动量形式一致,称为\textbf{轨道角动量},另一部分
\begin{equation}
    \mathcal{S}_k = \ii (J_k)^a_b \phi^b \pdv{\mathcal{L}}{\partial_0 \phi^a}
    \label{eq:spin-angular-momentum}
\end{equation}
则不会出现在没有内部结构的质点的角动量内部。由于其形式和有自转的粒子除去轨道角动量以外的角动量一致,称其为\textbf{自旋角动量}。
正如其名称暗示的那样,自旋角动量源自场的内禀旋转自由度,也就是说,旋转群在场上的表示不仅仅含有无穷维的表示,还含有一个有限维的表示的时候就会出现自旋角动量。
如果场是标量场,那么旋转群在其上的有限维表示就是平凡的,此时$J_i$全部为零,就没有自旋角动量。

然后我们分析场的内禀对称性带来的守恒量。容易看出场的平移,也就是
\[
    \bar{\var}{\phi} = a, \; \var{x} = 0
\]
对应着守恒流
\[
    \partial_\mu \pdv{\mathcal{L}}{\partial_\mu \phi } a = 0,
\]
其守恒荷为
\begin{equation}
    \Pi = \int \dd[3]x \pdv{\mathcal{L}}{\partial_0 \phi}.
\end{equation}
这称为$\phi$的\textbf{共轭动量},相应的其密度
\begin{equation}
    \pi = \pdv{\mathcal{L}}{\partial_0 \phi}
    \label{eq:def-pi}
\end{equation}
就是\textbf{共轭动量密度}。
需注意此“动量”的名称只是类比而得,它未必和$P_i$有特别紧密的联系。

最后,我们对拉氏量的形式做出一些限定。我们称导出的方程完全线性、且其中的每一项的次数不超过2的拉氏量为\textbf{自由场拉氏量},并认为一个涉及多个场的拉氏量一定形如每个场的自由场拉氏量之和加上一系列相互作用项,其中相互作用项不含有导数。
在实际问题中,相互作用项可以含有导数,但因为本文探讨的是最基本的物理机制,我们倾向于认为这些机制是足够简单的。
% TODO:相互作用真的没有导数吗?
这就意味着,虽然相互作用的存在让\eqref{eq:field-momentum}和\eqref{eq:field-angular-momentum}不再是守恒量,但如果整个系统的拉氏量$\mathcal{L}$满足空间平移不变性或者旋转不变性,对每一个场计算\eqref{eq:field-momentum}和\eqref{eq:field-angular-momentum}并相加,得到的量仍然是守恒的。
并且由于满足旋转不变性的自由场拉氏量只含有二次项(一次项如果是旋转不变的,就可以通过重新定义场来消去;如果不是旋转不变的,那就不满足条件;\autoref{sec:qft-free-dynamics}给出了典型例子),场的动量和角动量也只含有二次项。
另一方面,相互作用会引入额外的能量,因此不能简单地将各个场使用\eqref{eq:field-energy}计算出的能量加起来。

\subsection{哈密顿动力学}

\subsubsection{经典哈密顿动力学}
原本可以直接从拉氏量通过一个勒让德变换得到哈密顿动力学,但当底流形有多个坐标时我们需要选择合适的一个或几个坐标来充当“时间”,也就是哈密顿系统的参数。
共轭动量使我们有了一个很好的选择。本文取$t=x^0$为哈密顿系统的单参数。接下来我们要观察共轭动量的变化情况,从而凑出一个哈密顿系统。

容易看出$\Pi$的运动方程为%
\footnote{本节的结果也不仅仅适用于相对论性场论。任何能够良定义场的平移并且保证场平移不改变拉氏量的拉格朗日动力学场论都可以使用本节的方法构造对应的哈密顿表述,因为本节只用到了场的内禀平移不变性诱导出的结构。}
\[
    \dv{\Pi}{t} = \int \dd[3]x \partial_0 \pdv{\mathcal{L}}{\partial_0 \phi} = \int \dd[3]x \left( \pdv{\mathcal{L}}{\phi} - \partial_i \pdv{\mathcal{L}}{\partial_i \phi} \right).
\]
被积函数是$\int \dd[3]x \mathcal{L}$在将$x^0$当成常数后对$\phi$泛函求导的结果。于是定义%
\footnote{在$\phi$是多分量场的时候,我们把它看成列向量,记号$\partial \mathcal{L} / \partial \phi$定义为一个行向量,从而所有公式形式上仍然成立。例如,
\[
    \pdv{\vb*{a} \cdot \vb*{x}}{\vb*{x}} = \vb*{a}^\top.
\]
% TODO:需要使用度规吗?
}%
\begin{equation}
    H = \int \dd[3]x \mathcal{H} 
    = \int \dd[3]x \eval{\left( \pdv{\mathcal{L}}{\partial_0 \phi} \partial_0 \phi - \mathcal{L} \right)}_{\partial_0 \phi \to \pi} 
    = \int \dd[3]x \eval{\left( \pi \partial_0 \phi - \mathcal{L} \right)}_{\partial_0 \phi \to \pi}.
    \label{eq:lagrangian-to-hamitonian}
\end{equation}
我们通过将$\partial_0 \phi$用$\pi$表示使得任何对$H$的泛函求导都不会将$\partial_0 \phi$当成变量。
看出,$H$对$\phi$泛函求导就是$-\int \dd[3]x \mathcal{L}$对$\phi$泛函求导,于是我们有
\[
    \dv{\Pi}{t} = - \int \dd[3]x \fdv{H}{\phi}.
\]
另一方面由于$H$不显含任何$\pi$的导数,我们有
\[
    \begin{aligned}
        \fdv{H}{\pi} &= \pdv{\pi} \eval{\left( \pi \partial_0 \phi - \mathcal{L} \right)}_{\partial_0 \phi \to \pi} 
        = \partial_0 \phi + \pi \pdv{\partial_0 \phi}{\pi} - \pdv{\pi} \eval{\mathcal{L}}_{\partial_0 \to \pi} \\
        &= \partial_0 \phi + \pi \pdv{\partial_0 \phi}{\pi} - \pdv{\mathcal{L}}{\partial_0} \pdv{\partial_0}{\pi} = \partial_0 \phi.
    \end{aligned}
\]
于是就得到了3+1维场论的哈密顿表述:
\begin{equation}
    \dv{\pi}{t} = - \fdv{H}{\phi}, \quad \dv{\phi}{t} = \fdv{H}{\pi}.
    \label{eq:hamitonian-eq}
\end{equation}
其中$H$仅仅是$\phi, \partial_i \phi$和$\pi$的函数。
方程中的全导数也可以写成偏导数,我们把它写成全导数是因为我们通常只在一个固定的空间点观察场的变化,也就是说在\eqref{eq:hamitonian-eq}中我们只把时间看成变量而将空间坐标看成“标签”(见\autoref{note:spacial-label})。
由于我们讨论的基本上是场论问题,常常使用下面的记号:%
\footnote{在一些上下文中,场的时间全导数常常被定义为某个位置会随时间发生变化的场点处的场的导数,也就是
\[
    \dot{\phi} = \dv{t} \phi(\vb*{x}, t) = \pdv{\phi}{t} + \dot{\vb*{x}} \cdot \grad{\phi} = \partial_0 A = \partial^0 A.
\]
本文不涉及这样的问题,因此不使用这个记号。
}%
\[
    \dot{A} = \dv{A}{t} = \pdv{A(\vb*{x}, t)}{t}.
\]
在同一个场有多个分量的情况下,我们记各场为$\phi^i$,如果还是希望维持形式上的指标升降关系,$\pi$就可以写成$\pi_i$。

总之,使用拉氏量描述的3+1维经典场也能够使用一个哈密顿动力学描述,这个哈密顿动力学的演化参数为$x^0$也就是时间维,而使用空间维作为连续的“正则坐标”的标记。%
\footnote{也就是说,空间坐标$x^1, x^2, x^3$对应离散情况下的场量标签,
如$\phi^1(x, y, z)$指的是以$1, x, y, z$为标签的一个正则坐标,正如离散时的$q^{1}$代表以$1$为标签的一个正则坐标。
注意到这种哈密顿表述并没有以统一的方式对待时间和空间。\label{note:spacial-label}}%
任何物理量都是$\phi$和$\partial_\mu \phi$的函数,因此它们能够写成$\phi, \partial_i \phi$和$\pi$的函数,从而它们的演化都可以使用\eqref{eq:hamitonian-eq}确定,因为
\begin{equation}
    \dv{A}{t} = \pdv{A}{\phi} \dv{\phi}{t} + \pdv{A}{\partial_i \phi} \partial_i \dv{\phi}{t} + \pdv{A}{\pi} \dv{\pi}{t}.
    \label{eq:evolution-of-any-quantity}
\end{equation}

哈密顿动力学(无论是经典哈密顿动力学还是下一节讨论的正则量子化)中如果场是复的,仍然可以使用\eqref{eq:lagrangian-to-hamitonian}从拉氏量得到哈氏量,但此时不能够保证$\phi$、$\phi^\dagger$、$\pi$、$\pi^\dagger$彼此独立。
% TODO

\subsubsection{正则量子化}\label{sec:canonical-quantization}

下面我们转而讨论量子情况下的哈密顿动力学。这种使用哈密顿动力学建立量子理论的方法称为\textbf{正则量子化}。
完整地描述一个量子系统的状态和演化情况需要一个三元组:
首先是一个希尔伯特空间,称为\textbf{态空间},其中的矢量称为\textbf{态矢量},它们表示了系统的状态,
并且我们认为只差了一个倍数的态矢量等价,从而我们可以仅使用单位长度的态矢量描述任何的系统;
其次是一组\textbf{可观察量},它们是希尔伯特空间上的厄米算符,这意味着它们可以被幺正对角化,并且本征值都是实数%
\footnote{后面会提到,如果一个理论在正则量子化时必须选择反对易子的量子化方案,那么实际上它描写的场算符的本征值是格拉斯曼数。
但是如果我们在正则量子化的框架下工作,就从来不关注这种理论对应的场算符的本征值到底是多少,因此没有必要特意讨论它们。
在路径积分量子化的框架中,由于需要讨论费米子的经典场,格拉斯曼数是比较重要的。}%
;
最后是一个\textbf{哈密顿量}或者说\textbf{哈密顿算符},它自身也是一个可观察量(在经典极限下就是经典哈密顿量),且它指示了系统的演化方式。
经典哈密顿理论中同样有对应的三元组,
只不过态空间并不是一个可以做线性叠加的向量空间,从而可观察量也只是从态到实数的映射而不是希尔伯特空间上的算符。
由于所谓的场量$\phi$需要使用算符$\hat{\phi}$代替,因此不再能够良好地定义$\mathcal{H}$对各场量的偏导数,
从而我们也不能良好地定义$\var{H}/\var{\phi}$,等等。
现在动力学方程由
\begin{equation}
    \dv{\hat{A}}{t} = \frac{1}{\ii \hbar} [\hat{A}, \hat{H}]
    \label{eq:quantum-evolution}
\end{equation}
确定。%
\footnote{我们不讨论其定义显含时间的算符,因为它们不会出现在基本的物理规律中。}%
此时有意义的物理量虽然是算符,但在正则量子化之下仍然能够写成场算符$\phi, \partial_i \phi$和$\pi$的函数,
因此一旦$\phi$和$\pi$的演化确定了,\eqref{eq:evolution-of-any-quantity}就以一种和经典情况完全一致的方式确定了所有物理量的演化。
换而言之,\eqref{eq:hamitonian-eq-quantum}完全等价于
\begin{equation}
    \dv{\hat{\phi}}{t} = \frac{1}{\ii \hbar} [\hat{\phi}, \hat{H}], 
    \quad \dv{\hat{\pi}}{t} = \frac{1}{\ii \hbar} [\hat{\pi}, \hat{H}].
    \label{eq:hamitonian-eq-quantum}
\end{equation}

运动方程\eqref{eq:quantum-evolution}意味着,哈密顿量就是与时间平移不变性对应的守恒量。我们称其本征值为\textbf{能量},相应的,其本征态就是\textbf{能量本征态}。

要确定系统的动力学,只需要讨论$[\hat{\phi}, \hat{H}]$和$[\hat{\pi}, \hat{H}]$就可以,
而要讨论这两者又只需要讨论所有有关的场之间的对易关系即可,因为我们总是可以把$H$写成这些场的多项式。
(下文中讨论量子化方案时有对这一点的形象说明)
因此,取对易子为李括号,一个理论中涉及的所有算符就构成了一个李代数,而基本的场之间的对易关系又完全确定了这个李代数的结构。

仅仅有一个抽象的李代数并不能获得完整的理论。
例如,单粒子体系中$\hat{\vb*{x}}$和$\hat{\vb*{p}}$之间的李代数和多粒子体系中每一个粒子的$\hat{\vb*{x}}$和$\hat{\vb*{p}}$之间的李代数具有完全一样的结构,但是显然单粒子体系不是多粒子体系。
例如单粒子体系中$\vb*{x}$的谱没有简并而多粒子体系中$\vb*{x}$的谱有简并。
要获得完整的理论,我们还需要讨论态空间的结构。
我们将不讨论完整的数学,而只是对物理上常用的一些操作做一些说明。

当我们选定一个希尔伯特空间并且将(抽象的李代数中的)算符作用于其上时,实际上是对这个算符做了一个幺正表示。
进一步,当我们说一个系统的希尔伯特空间$H$能够被一组相互对易的算符$S$完全描述时,
我们实际上是说,算符集合$S$在$H$上的(幺正)表示组成了$H$上的一个完备相容算符集合,也即,$S$中各个算符在$H$上的表示共享的本征矢量构成$H$的一组基。
可以证明,如果$S_1,S_2$是$S$的一个划分,且$S_1$完全描述了$H_1$而$S_2$完全描述了$H_2$,那么就有$H$和$H_1 \otimes H_2$同构。
因此我们把$H$完全分解成了若干空间的直积,这些空间中的每一个都由完整描述系统需要的算符中的其中一个完全刻画。

一旦同时知道了各算符的对易关系(从而建立起它们的李代数),以及完整描述系统需要的完备相容算符集合,
我们就可以完整地推导出这个系统每一时刻的状态以及其演化方式了。
实际上,我们真正关注的是完备相容算符集合中各算符的谱结构。

根据上下文,我们可以容易地分辨作为抽象的李代数成员的算符,以及它们在各个希尔伯特空间上的表示,
因此为方便陈述,以下不再对这些略有不同的对象做详细的区分。
对算符而言这样做是合理的,因为从某个表示中得到的代数关系只要不涉及具体的表示的细节,就在抽象的李代数中也成立。
例如如果我们在某个表示中推导出$[\hat{x}, \hat{p}] = \ii \hbar \hat{I}$,那么在抽象的李代数中必定也有这个式子成立,
因为其中只牵扯到算符而没有牵扯到态矢量。
同样,可以比较容易地分辨各个希尔伯特空间中的态矢量,因此在不引起混淆的情况下我们也不刻意区分它们。

现在我们要做的是,分析$\phi$和$\pi$之间要具有什么样的代数关系%
\footnote{实际上只需要分析同一个时间$t$下$\phi(\vb*{x}, t)$和$\pi(\vb*{y}, t)$之间的关系就可以了,
因为\eqref{eq:hamitonian-eq-quantum}中从来不会出现不同时间的量之间的对易子。
这是量子版本的哈密顿动力学不适宜用于分析洛伦兹协变性的一个例子。
另外请注意这套理论并不能原封不动地适用于广义相对论时空,因为那里会需要讨论“不同时间处的量之间的关系”。},
才能够让\eqref{eq:hamitonian-eq-quantum}在$\hbar \to 0$时退化到经典情况\eqref{eq:hamitonian-eq}。
选定这样一个代数关系就称为选取一种\textbf{量子化方案},因为一旦给定了这样的代数关系,我们就把\eqref{eq:hamitonian-eq}推广到了量子理论中。
我们将不试图穷举所有可以的量子化方案,而只是举两个行之有效的例子——也就是说,实验数据要求使用这样的量子化方案。%
\footnote{我们说“穷举所有情况”意味着,面对同一个经典哈密顿量密度$\mathcal{H}$,
有不止一种指定$\hat{\phi}$和$\hat{\pi}$的方式,
使得我们能够得到一个量子动力学\eqref{eq:hamitonian-eq-quantum},
并且在$\hbar \to 0$的极限情况下回退到经典动力学\eqref{eq:hamitonian-eq}。
这是可以预期的,因为“取$\hbar\to 0$的极限”这个操作显然不是一一对应的,或者说量子化方案可以不止一种。
}

第一个方案是指定对易子为
\begin{equation}
    [\hat{\phi}^i(\vb*{x}, t), \hat{\pi}_j(\vb*{y}, t)] = \ii \hbar \delta^i_j \delta^3(\vb*{x} - \vb*{y}), 
    \quad [\hat{\phi}^i(\vb*{x}, t), \hat{\phi}^j(\vb*{y}, t)] = [\hat{\pi}_i(\vb*{x}, t), \hat{\pi}_j(\vb*{y}, t)] = 0.
    \label{eq:symmetry-commutator}
\end{equation}
从中可以容易地看出
\begin{equation}
    [\partial_\mu \hat{\phi}^i(\vb*{x}, t), \hat{\phi}^j(\vb*{y}, t)] = 0, 
    \quad [\grad{\hat{\phi}^i(\vb*{x}, t)}, \hat{\pi}_j(\vb*{y}, t)] = \ii \hbar \delta^i_j \grad{\delta^3(\vb*{x} - \vb*{y})}.
    \label{eq:symmetry-partial-mu-commutator}
\end{equation}
只需要将导数看成非常接近的两个量之差,然后利用对易子的线性性即可导出上式。

现在我们来推导$\hbar\to 0$时\eqref{eq:hamitonian-eq-quantum}的极限。
为方便起见,推导过程中假定$\hat{\phi}$是标量场。这并不会有损一般性,因为推导过程中没有用到任何关于坐标变换导致场变化的知识,从而我们可以把同一个多分量场的不同分量看成不同的标量场。
使用\eqref{eq:symmetry-commutator},已知$\hat{\phi} \hat{\pi}$就能够写出$\hat{\pi} \hat{\phi}$。
将$\hat{\mathcal{H}}$写成关于$\hat{\phi}, \partial_i \hat{\phi}$和$\pi$的多项式,
我们可以使用\eqref{eq:symmetry-commutator}和\eqref{eq:symmetry-partial-mu-commutator}
将形如$\hat{\pi}\hat{\phi}$、$\hat{\pi}\partial_i \hat{\phi}$这样的式子改写为形如$\hat{\phi}\hat{\pi}$、$\hat{\partial_i \hat{\phi}}\hat{\pi}$这样的式子。这称为取\textbf{正规积序}:我们总是可以使用对易关系把一个算符多项式转化为一个与之恒等的多项式,后者中的每一项中的算符排列顺序都满足一定的要求。
因此不失一般性地,我们认为$\hat{\mathcal{H}}$的多项式表达式中的每一项都形如$\hat{\phi}^l (\partial_i \hat{\phi})^m \pi^n$。
我们有(因为时间$t$都一样,以下略去$t$变量)%
\footnote{注意下面的$\delta(\vb*{x} - \vb*{x}')$是系数,因此可以自由地移动;写在它们左边的算符不会作用在它们上面!}
\[
    \begin{aligned}
        &\quad\comm{\hat{\phi}(\vb*{x}')}{\int \dd[3]{x} \hat{\phi}^l(\vb*{x}) (\partial_i \hat{\phi})^m (\vb*{x}) \pi^n(\vb*{x})} \\
        &= \int \dd[3]{x} \comm{\hat{\phi}(\vb*{x}')}{\hat{\phi}^l (\vb*{x}) (\partial_i \hat{\phi})^m (\vb*{x}) \pi^n (\vb*{x})} \\
        &= \int \dd[3]{x} \left( \hat{\phi}^l (\vb*{x}) \comm{\hat{\phi}(\vb*{x}')}{(\partial_i \hat{\phi})^m (\vb*{x}) \pi^n (\vb*{x})} + [\hat{\phi}(\vb*{x}'), \hat{\phi}^l (\vb*{x})] \partial_i \hat{\phi}^m (\vb*{x}) \pi^n(\vb*{x}) \right) \\
        &= \int \dd[3]{x} \hat{\phi}^l (\vb*{x}) \left( (\partial_i \hat{\phi})^m (\vb*{x}) \comm{\hat{\phi} (\vb*{x}')}{\hat{\pi}^n (\vb*{x})} + \comm{\hat{\phi} (\vb*{x}')}{(\partial_i \hat{\phi})^m (\vb*{x})} \pi^n (\vb*{x}) \right) \\
        &= \int \dd[3]{x} \hat{\phi}^l (\vb*{x}) (\partial_i \hat{\phi})^m (\vb*{x}) \comm{\hat{\phi} (\vb*{x}')}{\hat{\pi}^n (\vb*{x})} \\
        &= \int \dd[3]{x} \hat{\phi}^l (\vb*{x}) (\partial_i \hat{\phi})^m (\vb*{x}) \ii \hbar \delta^3(\vb*{x'} - \vb*{x}) n \hat{\pi}^{n-1}(\vb*{x}) \\
        &= \ii \hbar n \hat{\phi}^l (\vb*{x}') (\partial_i \hat{\phi})^m (\vb*{x}') \hat{\pi}^{n-1}(\vb*{x}'), 
    \end{aligned}
\]
这正是求导公式。当$\hbar$接近零的时候$\phi$和$\pi$可以交换,于是$\mathcal{H}$可以写成普通的、字母顺序无关紧要的函数,此时
我们有
\[
    \begin{aligned}
        \dv{\hat{\phi}}{t} = \frac{1}{\ii \hbar} [\hat{\phi}(\vb*{x}), \hat{H}] &= \frac{1}{\ii \hbar} \sum_\text{terms} \comm{\hat{\phi}(\vb*{x}')}{\int \dd[3]{x} \hat{\phi}^l(\vb*{x}) (\partial_i \hat{\phi})^m (\vb*{x}) \pi^n(\vb*{x})} \\
        &= \frac{1}{\ii \hbar} \sum_\text{terms} \ii \hbar n \hat{\phi}^l (\vb*{x}') (\partial_i \hat{\phi})^m (\vb*{x}') \hat{\pi}^{n-1}(\vb*{x}') \\
        &= \pdv{\hat{\mathcal{H}}}{\hat{\pi}} = \fdv{\hat{H}}{\hat{\pi}},
    \end{aligned}
\]
这就意味着$\hbar\to 0$时关于$\hat{\phi}$的方程能够回退到经典版本。
同样也有
\[
    \begin{aligned}
        &\quad \comm{\hat{\pi}(\vb*{x}')}{\int \dd[3]{x} \hat{\phi}^l(\vb*{x}) (\partial_i \hat{\phi})^m (\vb*{x}) \pi^n(\vb*{x})} \\ 
        &= \int \dd[3]{x} \comm{\hat{\pi}(\vb*{x}')}{\hat{\phi}^l(\vb*{x}) (\partial_i \hat{\phi})^m (\vb*{x}) \pi^n(\vb*{x})} \\
        &= \int \dd[3]{x} \left( \hat{\phi}^l (\vb*{x}) (\partial_i \hat{\phi} )^m (\vb*{x}) \comm{\hat{\pi}(\vb*{x}')}{\hat{\pi}^n(\vb*{x})} + \hat{\phi}^l (\vb*{x}) \comm{\hat{\pi}(\vb*{x}')}{(\partial_i \hat{\phi})^m (\vb*{x})} \hat{\pi}^n (\vb*{x}) + \comm{\hat{\pi}(\vb*{x}')}{\hat{\phi}^l (\vb*{x})} (\partial_i \hat{\phi})^m (\vb*{x}) \hat{\pi}^n (\vb*{x}) \right) \\
        &= \int \dd[3]{x} \left( - \hat{\phi}^l (\vb*{x}) \ii \hbar \grad{\delta^3 (\vb*{x} - \vb*{x}')} m (\partial_i \hat{\phi})^{m-1} (\vb*{x}) \hat{\pi}^n (\vb*{x}) - \ii \hbar \delta^3(\vb*{x} - \vb*{x}') l \hat{\phi}^{l-1} (\vb*{x}) (\partial_i \hat{\phi})^m (\vb*{x}) \hat{\pi}^n (\vb*{x}) \right) \\
        &= \ii \hbar \partial_i \left(m (\partial_i \hat{\phi})^{m-1} (\vb*{x}') \hat{\pi}^n (\vb*{x}')\right) - \ii \hbar l \hat{\phi}^{l-1} (\vb*{x}') (\partial_i \hat{\phi})^m (\vb*{x}') \hat{\pi}^n (\vb*{x}').
    \end{aligned}
\]
当$\hbar \to 0$时
\[
    \begin{aligned}
        \dv{\hat{\pi}}{t} &= \frac{1}{\ii \hbar} [\hat{\pi}(\vb*{x}), H] \\
        &= \frac{1}{\ii \hbar} \sum_\text{terms} \left(\ii \hbar \partial_i \left(m (\partial_i \hat{\phi})^{m-1} (\vb*{x}') \hat{\pi}^n (\vb*{x}')\right) - \ii \hbar l \hat{\phi}^{l-1} (\vb*{x}') (\partial_i \hat{\phi})^m (\vb*{x}') \hat{\pi}^n (\vb*{x}')\right) \\
        &= - \sum_\text{terms} \left(l \hat{\phi}^{l-1} (\vb*{x}') (\partial_i \hat{\phi})^m (\vb*{x}') \hat{\pi}^n (\vb*{x}') - \partial_i \left(m (\partial_i \hat{\phi})^{m-1} (\vb*{x}') \hat{\pi}^n (\vb*{x}')\right) \right) \\
        &= - \left( \pdv{\hat{\mathcal{H}}}{\hat{\phi}} - \partial_i \pdv{\hat{\mathcal{H}}}{\partial_i \hat{\phi}} \right) = - \fdv{\hat{H}}{\hat{\phi}},
    \end{aligned}
\]
因此关于$\pi$的方程也回退到经典情况。
这表明\eqref{eq:symmetry-commutator}是一个可行的量子化方案。

第二个方案是指定反对易子——而不是对易子——为%
\footnote{彼此无关的场,无论它们自己服从\eqref{eq:symmetry-commutator}还是\eqref{eq:antisymmetry-commutator},相互之间总是对易的。在\eqref{eq:symmetry-commutator}中这是显然的,因为可以将无关的场看成某个多分量场的分量,然后因为它们是不同的分量,它们自然对易。但在\eqref{eq:antisymmetry-commutator}方案下需要额外增加一个规定:
\[
    \comm*{\hat{\phi}(\vb*{x}, t)}{\hat{\psi}(\vb*{y}, t)} = 0.
\]
}
\begin{equation}
    \{\hat{\phi}(\vb*{x}, t), \hat{\pi}(\vb*{y}, t)\} = \ii \hbar \delta^3(\vb*{x} - \vb*{y}), \quad \{\hat{\phi}(\vb*{x}, t), \hat{\phi}(\vb*{y}, t)\} = \{\hat{\pi}(\vb*{x}, t), \hat{\pi}(\vb*{y}, t)\} = 0.
    \label{eq:antisymmetry-commutator}
\end{equation}
同样,我们可以将哈密顿量写成若干个$\hat{\phi}^l (\partial_i \hat{\phi})^m \pi^n$形式的项的和。
需要注意的是\eqref{eq:antisymmetry-commutator}直接导出
\[
    \hat{\phi}(\vb*{x})^2 = 0, \quad \hat{\pi}(\vb*{x})^2 = 0, \quad (\partial_i \hat{\phi})^2(\vb*{x}) = 0,
\]
因此哈密顿量中$l, m, n \leq 1$。
这意味着这个量子化方案并不适用于所有的场,而是只适用于能够保证在任何情况下哈密顿量中的每一项都满足$l, m, n \leq 1$的场。
对于正常的实数/复数值场,这是一个不可能的事情。
事实上,设$\hat{\phi}$在$\vb*{x}$处的值为$\int \dd \phi(\vb*{x}) \dyad{\phi}$,
在$\vb*{y}$处的值为$\int \dd \phi(\vb*{y}) \dyad{\phi}$,则通过反对易关系能够得到
\[
    \phi(\vb*{x}) \phi(\vb*{y}) = - \phi(\vb*{y}) \phi(\vb*{x}).
\]
因此,反对易子意味着对应的场算符的本征值——也就是其经典极限——实际上并不是实数,甚至也不是复数,而是格拉斯曼数。
在复数域中满足反对易关系的场算符不能被对角化。
在路径积分量子化中,格拉斯曼数非常重要,因为路径积分量子化会分析经典场值的演化路径。
在正则量子化中只需要把这些格拉斯曼数看成算符(准确地说,是产生算符)就可以了——我们并不会用到它的微积分,因此也无需将它们看成数。

为了看出反对易方案的不同寻常,我们指出如下事实:一个通过\eqref{eq:antisymmetry-commutator}量子化的场不可能是厄米的。
我们有
\[
    \hat{\phi}(\vb*{x}, t) \hat{\pi} (\vb*{y}, t) + \hat{\pi} (\vb*{y}, t) \hat{\phi} (\vb*{x}, t) = \ii \hbar \delta(\vb*{x} - \vb*{y}),
\]
于是
\[
    \left(\hat{\phi}(\vb*{x}, t) \hat{\pi} (\vb*{y}, t) + \hat{\pi} (\vb*{y}, t) \hat{\phi} (\vb*{x}, t)\right)^\dagger = - \ii \hbar \delta(\vb*{x} - \vb*{y}),
\]
如果场是厄米的,那么就有
\[
    \hat{\pi} (\vb*{y}, t) \hat{\phi} (\vb*{x}, t) + \hat{\phi}(\vb*{x}, t) \hat{\pi} (\vb*{y}, t) = - \ii \hbar \delta(\vb*{x} - \vb*{y}).
\]
于是我们得到了一个矛盾。因此,使用\eqref{eq:antisymmetry-commutator}量子化的场应该分解成非零的厄米和反厄米部分,即
\begin{equation}
    \hat{\phi} = \hat{\phi}_1 + \ii \hat{\phi}_2,
\end{equation}
其中$\hat{\phi}_1$和$\hat{\phi}_2$分别是两个厄米算符。
对应的,描述它的拉氏量当中的场有实部和虚部,需要把它们——或者它们的线性组合——看成两个独立的场来列写\eqref{eq:el-eq}。

此外,$\mathcal{H}$中各项阶数的限制还意味着由此导出的运动方程在时间上只能是一阶的。
从而,$\pi$和$\phi$不是彼此独立的。这样,在哈密顿量只关于$\phi$和$\pi$时,我们总是可以适当地调节拉氏量和哈密顿量,或者对$\phi$和$\pi$做一些线性变换,使得$\pi$和$\phi$之间有线性关系。
这个关系显然不能是“乘以某个倍数”,否则将不能够区分这两个变量。
因此两者之间的关系涉及复共轭。通常取
\begin{equation}
    \pi = \ii \phi^\dagger.
\end{equation}
这也表明了取$\phi$为复场的重要性——否则将不能够区分$\phi$和$\pi$,从而难以建立哈密顿动力学。
% TODO:这段还是有问题

需要注意的是,无论是\eqref{eq:symmetry-commutator}还是\eqref{eq:antisymmetry-commutator},实际上都假定了$\phi^i$和$\pi_i$在时空变换下是协变的。
在场具有某些附加结构——例如,有某些外加约束以消除非物理的自由度——的时候,如果我们直接把独立的自由度拿出来写成$\phi^i$,就不能保证它们的协变性(虽然把原来的场恢复出来之后它仍是协变的),此时不能直接套用\eqref{eq:symmetry-commutator}或\eqref{eq:antisymmetry-commutator},而需要使用带约束的场论的有关知识。

此外,虽然本节通过表明指定对易子或者反对易子能够得到经典哈密顿动力学来论证量子化方案\eqref{eq:symmetry-commutator}和\eqref{eq:antisymmetry-commutator}的合理性,但是实际上这两个方案在本文展示的经典哈密顿动力学以外仍然适用。例如,如果哈氏量中出现了广义动量的导数,那么\eqref{eq:hamitonian-eq}需要做出修正,但是\eqref{eq:hamitonian-eq-quantum}仍然适用。换而言之,本节展示的量子动力学实际上才是最根本的理论。
% TODO:真的吗?

从正则量子化得到的算符运动方程就是经典的场运动方程算符化的结果,而后者又等价于通过最小作用量原理求出的运动方程。
这就产生了一个问题:路径积分量子化告诉我们,最小作用量原理只是路径积分的最速下降近似而已,
为什么在正则量子化中精确的运动方程却可以从最小作用量原理求出?
其原因在于,算符在演化过程中不同的本征态会混在一起(一个经典情况下不可能出现的现象),正是这一点构成了量子和经典的区别,
正则量子化中的本征态混合正好对应于路径积分量子化中非经典的路径。

\subsubsection{时间演化和绘景}\label{sec:time-evolution}
% TODO: 表征了相同的物理状态的态矢量之间只差了一个复数常数。好量子数就是守恒量(这是一个算符!)的值,它可以用来标记态。
在\autoref{sec:canonical-quantization}中我们仅仅将态矢量当成一个可以让场算符作用上去的对象。
但实际上如果我们想要的话,也可以让态矢量动起来而对算符做对应的修改,使得算符的谱结构始终不变(本征矢量重数一一对应、彼此对应的本征矢的内积相同),并且本征值不变。
只要算符的谱结构不变、对应的各个本征值不变,算符就正确地描述了系统。
两个算符的谱结构一致、对应的本征值相同的充要条件是它们酉相似(相似矩阵可以随时间变化)。
需要注意的是两个描述了同一个系统的算符会给出不同的基矢量,所以切换绘景的时候还需要改变态矢量。
综上,绘景变换公式为
\begin{equation}
    \hat{A}' = \hat{Q} \hat{A} \hat{Q}^\dagger, \quad \ket{\psi'} = \hat{Q} \ket{\psi},
    \label{eq:picture-trans}
\end{equation}
其中$\hat{Q}$为一个幺正算符,它可以显含时间。
对易子在绘景变换之下会发生下面的改变:
\begin{equation}
    \comm{\hat{A}}{\hat{B}} \longrightarrow \hat{Q} \comm{A}{B} \hat{Q}^\dagger = \comm{\hat{A}'}{\hat{B}'}.
\end{equation}

在\autoref{sec:canonical-quantization}中我们已经讨论了态矢量固定不动时怎么确定系统的动力学。
这种让态矢量固定、算符变动的方案称为\textbf{海森堡绘景}。以下我们使用上标$H$代表海森堡绘景下的量。
我们要证明的第一件事是,不同时间点上的同一个可观察量的值彼此酉相似。
要看清楚这是为什么,我们将酉相似的方程
\begin{equation}
    \hat{A}^H (t) = \hat{U}^H(t, t_0) \hat{A}^H (t_0) (\hat{U}^H)^\dagger(t, t_0)
    \label{eq:quantum-evolution-hes-u-operator}
\end{equation}
做一个等价变换,看看它等价于什么。%
\eqref{eq:quantum-evolution-hes-u-operator}中的$U$在$t=t_0$时必定为恒等变换,因为此时$\hat{A}^H (t) = \hat{A}^H (t_0)$;同时容易看出$\hat{U}^H(\tau)$实际上构成一个李群。这样我们就能够写出其生成元,记之为$\hat{G}(t)$:
\[
    \hat{U}^H(t+\dd{t}, t) = \hat{I} + \frac{\ii}{\hbar} \hat{G}(t) \dd{t}.
\]
$\hat{U}^H$是幺正的等价于$\hat{G}$是厄米的。
于是就能够写出\eqref{eq:quantum-evolution-hes-u-operator}的无穷小等价形式:
\[
    \hat{A}^H (t_0) + \dd{\hat{A}^H}(t_0) = \left( \hat{I} + \frac{\ii}{\hbar} \hat{G}(t) \dd{t} \right) \hat{A}^H (t_0) \left( \hat{I} - \frac{\ii}{\hbar} \hat{G}(t) \right) = \hat{A}^H (t_0) + \frac{\dd{t}}{\ii \hbar} \comm{\hat{A}^H}{\hat{G}(t)}.
\]
我们发现这就是\eqref{eq:quantum-evolution},只需要把$\hat{G}(t)$换成$\hat{H}(t)$;并且正则量子化的时候已经要求$\hat{H}$是厄米的了,因此$\hat{G}$的确是厄米的,从而$\hat{U}^H$是幺正的。
于是我们得出结论:海森堡绘景中的算符演化实际上是在做幺正变换,或者等价地说,海森堡绘景中各算符的本征态在做幺正变换。算符的变换式为\eqref{eq:quantum-evolution-hes-u-operator},相应的,本征态的变换式为
\begin{equation}
    \ket{a(t)} = \hat{U}^H(t, t_0) \ket{a(t_0)}.
\end{equation}
于是我们称$\hat{U}^H$为海森堡绘景下的时间演化算符。
$\hat{U}^H$可以写出显式表达式
\begin{equation}
    \hat{U}^H(t, t_0) = T \exp \left( \frac{\ii}{\hbar} \int_{t_0}^t \dd{t} \hat{H}^H (t) \right).
\end{equation}
注意\eqref{eq:quantum-evolution-hes-u-operator}保证了,一个可观察量在经过时间演化之后仍然是可观察量。

现在我们尝试使用\eqref{eq:picture-trans}来把时间演化完全转移到态矢量上面。
因此,我们希望在新的绘景中,$\hat{A}$始终不变。我们称这新的绘景为\textbf{薛定谔绘景}。
按照\eqref{eq:quantum-evolution-hes-u-operator},有
\[
    \hat{A}^H(t) = \hat{U}^H(t, t_0) \hat{A}^H (t_0) (\hat{U}^H)^\dagger(t, t_0) = \hat{U}^H(t, t_0) \hat{A}^S( \hat{U}^H)^\dagger(t, t_0),
\]
不失一般性地我们取$t=0$时的$\hat{A}^H$为$\hat{A}^S$,那么我们有
\[
    \hat{A}^H (t) = \hat{U}^H(t, 0) \hat{A}^S( \hat{U}^H)^\dagger(t, 0).
\]
将这个方程和\eqref{eq:picture-trans}对比可以看出
\[
    \hat{Q} = (\hat{U}^H)^\dagger(t, 0),
\]
于是得到薛定谔绘景下的态矢量演化公式
\[
    \ket{\psi^S(t)} = \hat{Q} \ket{\psi^H} = (\hat{U}^H)^\dagger (t, 0) \ket{\text{a constant}},
\]
考虑到$t=0$时$\hat{U}^H (t, 0)$就是恒等算符,上式又等价于
\[
    \ket{\psi^S(t)} = (\hat{U}^H)^\dagger (t, 0) \ket{\psi^S (0)},
\]
也即,薛定谔绘景下的时间演化算符和海森堡绘景下的时间演化算符互为逆。
这个方程还告诉我们,
\[
    \ket{\psi^H} = \ket{\psi^S(t_0)}.
\]
现在推导时间演化方程的微分形式。我们有
\[
    \begin{aligned}
        \ket{\psi^S (t + \dd{t})} &= \left( \hat{U}^H (t + \dd{t}, t) \hat{U}^H (t, 0)  \right)^\dagger \ket{\psi^S(0)} \\
        &= \left( (\hat{I} + \frac{\ii}{\hbar} \hat{H}(t) \dd{t})   \hat{U}^H (t, 0) \right)^\dagger \ket{\psi^S (0)} \\
        &= (\hat{U}^H)^\dagger (t, 0) \ket{\psi^S (0)} + \frac{\dd{t}}{\ii \hbar} (\hat{U}^H)^\dagger (t, 0) \hat{H}(t) \ket{\psi^S (0)} \\
        &= \ket{\psi^S (t)} + \frac{\dd{t}}{\ii \hbar} (\hat{U}^H)^\dagger (t, 0) \hat{H}(t) \hat{U}^H (t, 0) \ket{\psi^S (t)},
    \end{aligned}
\]
从而
\[
    \ii \hbar \dv{t} \ket{\psi^S (t)} = (\hat{U}^H)^\dagger (t, 0) \hat{H}(t) \hat{U}^H (t, 0) \ket{\psi^S (t)}.
\]
为了方便区分,我们将海森堡绘景中的$\hat{H}$记作$\hat{H}^H$,则它对应的薛定谔绘景中的算符为
\[
    \hat{H}^S = \hat{Q} \hat{H}^H \hat{Q}^\dagger = (\hat{U}^H)^\dagger (t, 0) \hat{H}^H(t) \hat{U}^H (t, 0), 
\]
这正是薛定谔绘景中态矢量的运动方程中出现的那个量,因此就获得了薛定谔绘景中的运动方程:
\begin{equation}
    \ii \hbar \dv{t} \ket{\psi^S(t)} = \hat{H}^S (t) \ket{\psi^S(t)}.
\end{equation}
% TODO:证明$\hat{H}^I_i$确实是$\hat{H}_i^H$在相互作用绘景下的
设$\hat{U}^S(t, t_0)$是薛定谔绘景下的时间演化算符,则容易证明$\hat{H}^S$是它的生成元,既然$\hat{H}^H$是厄米的,$\hat{H}^S$也是厄米的,从而$\hat{U}^S$是幺正的。%
\footnote{注意$\hat{H}^H$是$\hat{U}^H$的生成元而$\hat{H}^S$是$(\hat{U}^H)^\dagger$的生成元;由于$\hat{H}^H$可能含时,一般情况下
\[
    T \exp(\int \hat{H}^H (t) \dd{t})^\dagger \neq T \exp(- \int \hat{H}^H (t) \dd{t}),
\]
也就是说$\hat{H}^H$和$\hat{H}^S$之间没有简单的关系,而必须使用绘景变换公式联系两者。
}
因此薛定谔绘景中时间演化始终保持态矢量的幺正性。
时间演化算符的显式表达式为
\begin{equation}
    \hat{U}^S(t, t_0) = T \exp \left( - \frac{\ii}{\hbar} \int_{t_0}^t \dd{t} \hat{H}^S(t) \right),
\end{equation}
其中$T$为编时算符。

为了明显起见,我们将薛定谔绘景和海森堡绘景中哈密顿量相互换算的关系重复如下:
\begin{equation}
    \begin{aligned}
        \hat{H}^H(t) = T \exp \left( \frac{\ii}{\hbar} \int_{t_0}^t \dd{t} \hat{H}^H(t) \right) \hat{H}^S(t_0) \left(T \exp \left( \frac{\ii}{\hbar} \int_{t_0}^t \dd{t} \hat{H}^H(t) \right)\right)^\dagger, \\
        \hat{H}^S(t_0) = T \exp \left( - \frac{\ii}{\hbar} \int_{t_0}^t \dd{t} \hat{H}^S \right) \hat{H}^H(t) \left( T \exp \left( - \frac{\ii}{\hbar} \int_{t_0}^t \dd{t} \hat{H}^S \right)\right)^\dagger,
    \end{aligned}
\end{equation}
在$\hat{H}^H$在各个时间点的值彼此对易时,$\hat{U}^H$无非是$\hat{H}^H$的级数,因此它们对易,从而$\hat{H}^S$和$\hat{H}^H$相等。
这也等价于$\hat{H}^S$在各个时间点的值彼此对易。

事实上,虽然我们是从海森堡绘景出发建立我们的理论框架的,但\autoref{sec:back-to-classical}告诉我们,和经典力学中的系统状态直接对应的实际上就是态矢量,而不是算符,因此很多文献是从薛定谔绘景出发建立理论的。
此外,在哈密顿量不含时时,海森堡绘景中的算符的本征态以薛定谔绘景中的态矢量的方式演化。

现在我们已经讨论了“让可观察量变动”和让基矢量变动“两种方案的不同了。我们还可以把哈密顿算符分解成一个比较简单的不含时部分和一个含时的部分,并要求这两者均为厄米算符,然后分别用两者让算符和态矢量都动起来。这样的方案称为\textbf{相互作用绘景}。
为方便起见,考虑从薛定谔绘景到相互作用绘景的变换。当然也可以从海森堡绘景出发推导相互作用绘景,但实际上这样会很不自然。对薛定谔绘景下的哈密顿量做分解
\begin{equation}
    \hat{H}^S = \hat{H}_0^S + \hat{H}_i^S,
\end{equation}
称前者为\textbf{自由哈密顿量}(通常我们要求它不显含时间),后者为\textbf{相互作用哈密顿量},并指定
\begin{equation}
    \ket{\psi^I(t)} = \hat{U}_0^\dagger(t,t_0) \ket{\psi^S(t)},
\end{equation}
其中
\begin{equation}
    \hat{U}_0 = T \exp \left( - \frac{\ii}{\hbar} \int_{t_0}^t \dd{t} \hat{H}_0^S(t) \right).
\end{equation}
于是可观察量的绘景变换为
\begin{equation}
    \hat{A}^I(t) = \hat{U}_0^\dagger(t,t_0) \hat{A}^S(t) \hat{U}_0(t,t_0).
    \label{eq:operator-from-schodinger-to-interaction}
\end{equation}
通过求导,分别可以计算出态矢量和可观察量的时间演化方程为
\begin{equation}
    \ii \hbar \dv{t} \ket{\psi^I(t)} = \hat{H}^I_i(t) \ket{\psi^I(t)},
    \label{eq:time-evolution-in-interation-picture}
\end{equation}
以及
\begin{equation}
    \dv{t} \hat{A}^I(t) = \frac{1}{\ii \hbar} \comm*{\hat{A}^I(t)}{\hat{H}_0^I}.
\end{equation}
其中$\hat{H}_0^I$和$\hat{H}_i^I$正是对$\hat{H}_0^S$和$\hat{H}_i^S$做绘景变换\eqref{eq:operator-from-schodinger-to-interaction}得到的结果。
这样我们就成功地让时间演化分别由态矢量和可观察量各自承担一部分。
此外,由于
\[
    \ket{\psi^I(t)} = T \exp \left( - \frac{\ii}{\hbar} \int \dd{t} \hat{H}_i^I(t) \right) \ket{\psi^H},
\]
而
\[
    \ket{\psi^I(t)} = \hat{U}_0^\dagger (t, t_0) T \exp \left( - \frac{\ii}{\hbar} \dd{t} \hat{H}^S(t) \right) \ket{\psi^H},
\]
就得到
\begin{equation}
    T \exp \left( - \frac{\ii}{\hbar} \dd{t} \hat{H}^S(t) \right) = T \exp \left( - \frac{\ii}{\hbar} \int \dd{t} \hat{H}^S_0(t) \right) T \exp \left( - \frac{\ii}{\hbar} \int \dd{t} \hat{H}_i^I(t) \right).
\end{equation}

如果我们在海森堡绘景中工作,要怎么样切换到相互作用绘景中呢?最一般的公式非常复杂。
但是,实际上,如果哈密顿量含时,通常直接在薛定谔绘景中工作;如果哈密顿量不含时,那么薛定谔绘景和海森堡绘景下的哈密顿量是一样的,那么只需要选择一个较简单的可观察量$\hat{H}_0$,指定它为$\hat{H}_0^S$,就可以切换到相互作用绘景。需注意整个过程并没有用到$\hat{H}_0^H$,一般来说,它和$\hat{H}_0^S$可能会有区别,但是我们从来不关注这个区别。

相互作用绘景在微扰量子场论计算中起到了非常重要的作用,因为正如\autoref{sec:qft-free-dynamics}中展示的那样,通过对称性分析可以直接得到自由场的哈密顿量密度和演化方程,因此我们可以将相互作用项——也就是不同场之间的耦合——独立考虑,从而大大简化计算。
更加重要的是,此时相互作用绘景可以为我们提供有关量子场的态空间的结构的信息,实际上我们将在\autoref{sec:from-qft-to-many-body}中看到,如果假定态空间中有一个唯一的真空态——也就是所有场都是零的态——那么量子场的态空间就是多粒子态福克空间,在此基础上我们可以很自然地处理粒子创生和湮灭的过程。这就为我们展示了量子理论的另一面:波动看起来就像粒子一样。%
\footnote{需要注意的是,在处理相对论性量子场论的时候其实并不能完全放心地使用相互作用绘景。如果我们取$\hat{H}_i=0$,那么相互作用绘景就退化为了自由场的海森堡绘景;这样我们就看到了$\hat{H}_i$项的作用:它把带相互作用的场的态(也就是$\ket{\psi^I(t)}$)和自由场的态($\ket{\psi^I(0)}$,因为如果$\hat{H}_i=0$那么态就不会变化)使用一个幺正算符联系了起来,而且这个幺正算符是唯一的。然而Haag定理说,含相互作用的场有无数个不等价的幺正表示,因此我们并不能唯一地将带相互作用的场的态和自由场的态使用一个唯一的幺正算符联系起来。特别的,由于我们要求自由场和相互作用场的态空间都满足一定的物理条件(如有稳定的真空态,等等),自由场的态空间和相互作用场的满足这些条件的态空间一般来说并不幺正等价。这意味着类似于$\int \dd{t} \hat{H}^H_i$之类的表达式实际上并不收敛,于是相互作用绘景就失效了。但是有很多手段可以绕过这个定理的限制——例如因为我们从来只讨论一定能标下的物理现象而不把相对论性量子场论当成终极理论,实际上我们可以把空间格点化,这样量子场论就变成了有限自由度的量子力学,于是就可以使用相互作用绘景了。}

此外容易验证,各种形式的时间演化算符都满足以下公式:
\begin{equation}
    \hat{U}(t_3,t_2) \hat{U}(t_2,t_1) = \hat{U}(t_3,t_1),
\end{equation}
以及
\begin{equation}
    \hat{U}^\dagger (t_2, t_1) = \hat{U} (t_1, t_2).
\end{equation}

\subsubsection{对称性与守恒量}

对系统的变换大致可以分成两种,一种是对系统状态做的变换,即态矢量的变换;一种是对系统的动力学做的变换,即对哈密顿量的变换。(为了保证物理意义,变换一定是幺正的。)
容易看出两者是等价的,因为对态矢量做变换
\[
    \ket{\psi} \longrightarrow \hat{A} \ket{\psi}
\]
等价于对哈密顿量做变换
\[
    \hat{H} \longrightarrow \hat{A} \hat{H} \hat{A}^{-1}
\]
并与此同时对基矢量做一个幺正变换,而由于只有谱结构是重要的,这个基矢量的幺正变换可以略去。

系统在一个群下不变等价于对这个群的生成元$\hat{A}$,有
\[
    \hat{A} \hat{H} \hat{A}^{-1} = \hat{H},
\]
这等价于
\[
    \comm*{\hat{A}}{\hat{H}} = 0,
\]
这又等价于$\hat{A}$是一个守恒量。
如果群是连续的,那么$\hat{A}$是无穷小生成元,这样我们就找到了一个厄米的守恒量;如果群是离散的,那么所有群元本身就是守恒量。

\subsubsection{测量}\label{sec:measure}

\textbf{测量}指的是这样一个过程:两个系统(分别称为\textbf{待测系统}和\textbf{仪器})发生相对剧烈而时间短促的相互作用,相互作用后待测系统的态发生很大改变,而仪器的态则体现了相互作用前待测系统的某些信息。
采用相互作用绘景,设$\hat{q}$完全描述了仪器的态空间,$\hat{a}$是关于待测系统的某个算符,它和另一个算符$\hat{b}$共同描述了待测系统的态空间。(被测量的量$\hat{a}$未必能够完整描述待测系统。下文中需要将待测系统的态做展开,因此引入$\hat{b}$)
由于相互作用非常剧烈而时间短促,仪器和待测系统的相互作用哈密顿量可以写成
\begin{equation}
    H_\text{int} = - \gamma(t-t_0) \hat{a} \otimes \hat{p},
\end{equation}
% 为什么偏偏就是这个形式?为什么所有量都是一次项?
其中$\hat{p}$是$\hat{q}$对应的共轭动量,也就是说
\[
    \comm*{\hat{q}}{\hat{p}} = \ii \hbar,
\]
$\gamma$是一个函数,它是一个$t_0$附近的尖峰。
极限情况下,$\gamma(t) = g \delta(t)$,这称为\textbf{冯诺依曼测量}或者\textbf{标准量子测量},
我们在相互作用绘景下分析问题。系统初态为
\[
    \ket{i} = \ket{\psi_i} \ket{D} = \int \dd{q} \sum_{k, n} \braket{q}{D} \braket{a_k, b_n}{\psi_i} \ket{a_k} \ket{b_n} \ket{q},
\]
其中$\ket{\psi_i}$和$\ket{D}$分别为待测系统和仪器的初态,本征态$\ket{a}_k$,$\ket{b}_n$和$\ket{q}$是$t_0$时刻对应算符的本征态(下同)。%
\footnote{提醒:算符本征态反映的是算符的代数结构,它们的时间演化是由自由哈密顿量而不是相互作用哈密顿量指导的。}%
我们要求$\hat{q}$是连续谱,而$\hat{a}$和$\hat{b}$可以是离散谱也可以是连续谱。要求$\hat{q}$是连续谱的原因很快就可以看到。
系统的末态为
\[
    \begin{aligned}
        \ket{f} &= T \exp \left( - \frac{\ii}{\hbar} \int \dd{t} H_\text{int} \right) \ket{i} \\
        &= T \exp \left( \frac{\ii}{\hbar} g \int \dd{t} \delta(t-t_0) \hat{a}(t) \otimes \hat{p}(t) \right) \ket{i} \\
        &= \exp \left( \frac{\ii}{\hbar} g \hat{a}(t_0) \otimes \hat{p}(t_0) \right) \ket{i} \\
        &= \sum_{n=0}^\infty \frac{1}{n!} \left(\frac{\ii}{\hbar} g\right)^n \hat{a}(t_0)^n \hat{p}(t_0)^n \int \dd{q} \sum_{k, l} \braket{q}{D} \braket{a_k, b_l}{\psi_i} \ket{a_k} \ket{b_l} \ket{q} \\
        &= \int \dd{q} \sum_{k, l} \braket{q}{D} \braket{a_k, b_l}{\psi_i} \sum_{n=0}^\infty \frac{1}{n!} \left(\frac{\ii}{\hbar} g\right)^n \hat{a}(t_0)^n \ket{a_k} \ket{b_l} \hat{p}(t_0)^n \ket{q} \\
        &= \int \dd{q} \sum_{k, l} \braket{q}{D} \braket{a_k, b_l}{\psi_i} \sum_{n=0}^\infty \frac{1}{n!} \left(\frac{\ii}{\hbar} g\right)^n a_k^n \ket{a_k} \ket{b_l} \hat{p}(t_0)^n \ket{q} \\
        &= \int \dd{q} \sum_{k, l} \braket{q}{D} \braket{a_k, b_l}{\psi_i} \ket{a_k} \ket{b_l} \sum_{n=0}^\infty \frac{1}{n!} \left(\frac{\ii}{\hbar} g a_k \hat{p}(t_0) \right)^n \ket{q} \\
        &= \int \dd{q} \sum_{k, l} \braket{q}{D} \braket{a_k, b_l}{\psi_i} \ket{a_k} \ket{b_l} \exp \left( \frac{\ii}{\hbar} g a_k \hat{p}(t_0) \right) \ket{q} \\
        &= \int \dd{q} \sum_{k, l} \braket{q}{D} \braket{a_k, b_l}{\psi_i} \ket{a_k} \ket{b_l} \ket{q + g a_k} \\
        &= \int \dd{q} \sum_{k, l} \braket{q - g a_k}{D} \braket{a_k, b_l}{\psi_i} \ket{a_k} \ket{b_l} \ket{q} .
    \end{aligned}
\]
总之我们得到经典测量前后态的变化公式
\begin{equation}
    \ket{f} = \int \dd{q} \sum_{k, l} \braket{q - g a_k}{D} \braket{a_k, b_l}{\psi_i} \ket{a_k} \ket{b_l} \ket{q}.
    \label{eq:standard-measurement}
\end{equation}
需要注意的是由于我们采取的是相互作用绘景,算符$\hat{a}$和$\hat{b}$一直会发生变化。
然而,由于自由哈密顿量不显含时间,\eqref{eq:standard-measurement}中$\ket{a_k, b_l}$的时间演化和$\bra{a_k, b_l}$的时间演化抵消了,等等,从而$\ket{f}$在相互作用结束后没有时间演化——正如我们预期的那样,因为相互作用结束之后相互作用哈密顿量就是零。

\eqref{eq:standard-measurement}看起来仍然十分复杂。
然而,在很多情况下(具体是什么情况我们很快会看到)仪器的初始态非常接近$\hat{q}$的本征态,也就是说$\braket{q}{D}$只有在$q$和某一个$q_0$非常接近的时候才有较大的值,其余时候都接近零,因此实际上是一个$\delta$函数。
这样的情况称为\textbf{理想测量}。我们现在可以看到为什么要求$\hat{q}$具有连续谱了,因为要实施一次理想测量必须允许仪器有连续分布的状态。此时\eqref{eq:standard-measurement}近似为
\begin{equation}
    \ket{f} = \sum_{k, l} \braket{a_k, b_l}{\psi_i} \ket{a_k} \ket{b_l} \ket{q = q_0 + g a_k}.
    \label{eq:ideal-measurement}
\end{equation}
我们这样就得到了一个典型的纠缠态,其中每一个分量中,仪器和待测系统在测量之后都处于完全对应的状态。
总之,如果待测系统和仪器组成的系统和外界毫无相互作用,那么测量就是如下所示的过程:
\[
    \ket{i} = \left(\sum_{k, l} \braket{a_k, b_l}{\psi_i} \ket{a_k} \ket{b_l} \right) \ket{D} \longrightarrow \ket{f} = \sum_{k, l} \braket{a_k, b_l}{\psi_i} \ket{a_k} \ket{b_l} \ket{q = q_0 + g a_k},
\]
也就是待测系统将其信息复制到了仪器当中。
然而,假如仪器足够大,那么待测系统和仪器组成的系统和外界将会有大量的相互作用。
例如,仪器可能被放置在灯光下来方便我们读取其示数,这就意味着它要不停地受到四面八方的光子的轰击。
这就意味着\eqref{eq:ideal-measurement}会很快发生退相干,最后终结于$\hat{a} \otimes \hat{b} \otimes \hat{q}$的某个本征态上,因此最后仪器停留在某个$q=q_0 + g a_k$附近,且待测系统的态也转化为$\ket{a_k}$。
将待测系统和仪器组成的系统以及所有可能的环境变量放在一起就得到了一个系综;系综中,待测系统和仪器组成的系统在退相干之后停留在本征态$\ket{a_k} \ket{b_l} \ket{q = q_0 + g a_k}$的概率正是$\abs{\braket{a_k, b_l}{\psi_i}}^2$,
也就是说,在时刻$t$测量$\hat{a}$得到$a_k$(同时将待测系统的态转化为$\ket{a_k}$)的概率就是
\begin{equation}
    P_t(a_k) = \sum_l \abs{\braket{a_k, b_l(t)}{\psi_i}}^2,
    \label{eq:probablity-of-measurement}
\end{equation}
由\eqref{eq:probablity-of-measurement}出发容易证明,待测系统为$\ket{\psi_i}$态时做测量,测量值的期望为
\begin{equation}
    \expval{\hat{a}}(t) = \mel{\psi_i}{\hat{a}(t)}{\psi_i}.
\end{equation}

实际上,我们可以把四面八方的光子或者空气分子或者这一类的干扰看成是一个巨型仪器:它和待测系统的相互作用使待测系统和它的态按照\eqref{eq:ideal-measurement}纠缠在一起,而由于这是开放体系,退相干快速发生,这就意味着在充满干扰的环境中实际上很难真的展示出待测系统的量子特性:待测系统几乎总是出现在其偏好本征态附近,因为它没完没了地受到测量。
这也是理想测量很容易就能够实现的原因:真的会用来做测量的仪器总是被做得很大,因此它们自身可以看成不停地被空气、杂散光或者别的什么东西不断测量的系统,因此它们的态总是出现在其偏好本征态附近。

在$\hat{a}$本身是待测系统的一个CSCO,从而不需要$\hat{b}$的情况下,测量$\hat{a}$得到$a_k$的概率为
\begin{equation}
    P(a_k) = \abs{\braket{a_k}{\psi_i}}^2.
\end{equation}
这表明,假如我们有一个正交归一化基$\{\ket{a_k}\}_k$,就可以使用一组不同的实数$a_k$构造算符
\[
    \hat{a} = \sum_k a_k \dyad{a_k},
\]
使用这个算符对系统做测量,则测量结束之后系统位于态$\ket{a_k}$的概率就是
\begin{equation}
    P(\ket{a_k}) = \abs{\braket{a_k}{\psi_i}}^2.
\end{equation}
注意到这个表达式只和$\ket{a_k}$有关。因此,对态矢量为$\ket{\psi}$的系统做一次测量,发现系统测量后处于态$\ket{\phi}$的概率为
\begin{equation}
    P(\ket{\phi}) = \abs{\braket{\phi}{\psi}}^2,
\end{equation}
于是我们称$\braket{\phi}{\psi}$为\textbf{概率振幅}。

需注意以上讨论建立在几个关键假设上:其一,仪器和待测系统的相互作用非常强而短促;其二,仪器和环境有杂乱无章的相互作用。
这意味着合理地构造不怎么受外界干扰而又不会严重地扰动待测系统的仪器,我们就能够得到关于待测系统状态的不完整信息而与此同时不让待测系统的态塌缩到某个本征态上。
这称为\textbf{弱测量}。

\subsubsection{有效哈密顿量}

有时,一个物理系统的哈密顿量涉及大量复杂的过程,而特定的初始条件意味着这个系统的态基本上只会出现在态空间的一小部分当中。
但这并不意味着态空间的其它部分就不会对系统的动力学造成影响。
例如,设想一个三能级系统,其中一个能级的能量远远高于另外两个能级,这意味着系统基本上不可能出现在这个能级上,但如果其余两个能级和这个高能量能级有耦合,那么这个高能量能级就可能成为另外两个能级相互转换的渠道。

设投影算符$\hat{P}$选择出了我们关注的那部分态空间,而且这部分态空间的定义不随时间变化而变化;设$\hat{Q}$是与之互补的投影算符,则
\[
    \hat{P} + \hat{Q} = 1, \quad \hat{P}^2 = \hat{P}, \quad \hat{Q}^2 = \hat{Q}.
\]
考虑薛定谔绘景,运动方程为
\[
    \hat{H} \ket{\psi} = \ii \hbar \dv{t} \ket{\psi},
\]
将投影算符作用于其上得到
\[
    \begin{aligned}
        \hat{P} \hat{H}(\hat{P}+\hat{Q}) \ket{\psi} = \ii \hbar \dv{t} \hat{P} \ket{\psi}, \\
        \hat{Q} \hat{H}(\hat{P}+\hat{Q}) \ket{\psi} = \ii \hbar \dv{t} \hat{Q} \ket{\psi}.
    \end{aligned}
\]
哈密顿量可以分成四部分,一部分完全位于$\hat{P}$筛选出来的空间中,一部分完全位于$\hat{Q}$筛选出来的空间中,另外两部分从其中一个空间跳跃到另一个空间,这四部分分别是
\[
    \hat{H}_{PP} = \hat{P} \hat{H} \hat{P}, \quad \hat{H}_{QQ} = \hat{Q} \hat{H} \hat{Q}, \quad \hat{H}_{PQ} = \hat{P} \hat{H} \hat{Q}, \quad \hat{H}_{QP} = \hat{Q} \hat{H} \hat{P}.
\]
使用投影算符的性质可以写出
\[
    \begin{aligned}
        \hat{H}_{PP} \hat{P} \ket{\psi} + \hat{H}_{PQ} \hat{Q} \ket{\psi} = \ii \hbar \dv{t} \hat{P} \ket{\psi}, \\
        \hat{H}_{QP} \hat{P} \ket{\psi} + \hat{H}_{QQ} \hat{Q} \ket{\psi} = \ii \hbar \dv{t} \hat{Q} \ket{\psi},
    \end{aligned}
\]
从后一个方程可以解出
\[
    \hat{Q} \ket{\psi} = \frac{1}{\ii \hbar \dv{t} - \hat{H}_{QQ}} \hat{H}_{QP} \hat{P} \ket{\psi},
\]
代入前一个方程就得到
\[
    \ii \hbar \dv{t} \hat{P} \ket{\psi} = \left(\hat{H}_{PP} + \hat{H}_{PQ} \frac{1}{\ii \hbar \dv{t} - \hat{H}_{QQ}} \hat{H}_{QP}\right) \hat{P} \ket{\psi}.
\]
因此我们发现,我们关注的那一部分态的时间演化由等效哈密顿量
\begin{equation}
    \hat{H}_\text{eff} = \hat{H}_{PP} + \hat{H}_{PQ} \frac{1}{\ii \hbar \dv{t} - \hat{H}_{QQ}} \hat{H}_{QP}
    \label{eq:effective-hamiltonian-original}
\end{equation}
指导,而且由$\hat{H}_{PP}, \hat{H}_{PQ}, \hat{H}_{QP}$的定义,该等效哈密顿量是$\hat{P}$筛选出的空间中的算符。
\eqref{eq:effective-hamiltonian-original}非常符合我们的直觉:时间演化可以仅仅涉及$\hat{H}_{PP}$,也可以以$\hat{H}_{QQ}$为中介。

一种特殊的情况是,态空间可以写成两个空间(记为$\mathcal{H}_1$和$\mathcal{H}_2$)的直积,系统的初始条件决定了大部分有意义的过程都发生在$\mathcal{H}_1$中,但由于耦合,不能简单地将$\mathcal{H}_2$排除掉。
这时可以构造算符$\hat{P}$使之筛选出只在$\mathcal{H}_1$中有显著活动的态,计算出有效哈密顿量;$\hat{P}$筛选出的态均形如$\ket{\psi}_1 \otimes \ket{0}_2$,由于有效哈密顿量仅涉及$\ket{\psi}_1$,不会出现两个空间之间的耦合,于是可以直接将$\mathcal{H}_2$去掉,使用$\mathcal{H}_1$和$\hat{H}_\text{eff}$来描述系统。

然而,\eqref{eq:effective-hamiltonian-original}显含一个时间求导算符的倒数,这意味着$\hat{H}_\text{eff}$实际上显含时间,而且还显含关于时间的算符,也即我们实际上是手动把关于$\mathcal{H}_2$的时间演化放进了有效哈密顿量当中,这是不便计算的。在高能自由度和低能自由度的耦合并不明显时,高能自由度的存在与否对$\mathcal{H}_1$中的能量本征态只有不大的影响,这时可以以原哈密顿量中仅包含低能自由度的部分的本征值和本征态为起点,以低能自由度和高能自由度的耦合以及高能自由度的哈密顿量为微扰,求解出$\hat{H}$在$\mathcal{H}_1$中的本征态$\{\ket{n}\}$和本征值$\{E_n\}$。由于是本征态,它们和高能自由度没有耦合,于是低能自由度的运动完全由
\begin{equation}
    \hat{H}_\text{eff} = \sum_{\ket{n} \in \mathcal{H}_1} E_n \dyad{n}
\end{equation}
确定,我们也就得到了有效哈密顿量。
因此求解有效哈密顿量的问题就转化为了对角化微扰之后的哈密顿量的问题。

设我们考虑的过程的能量近似在$E_r$水平上,则对$\mathcal{H}_1$空间中的态,近似有
\[
    \ii \hbar \dv{t} \sim E_r,
\]
于是
\[
    \hat{H}_\text{eff} \sim \hat{H}_{PP} + \hat{H}_{PQ} \frac{1}{E_r - \hat{H}_{QQ}} \hat{H}_{QP}.
\]
对$\mathcal{H}_1$中$\hat{H}$的本征态而言,上式严格成立,我们得到自洽方程
\begin{equation}
    \left( \hat{H}_{PP} + \hat{H}_{PQ} \frac{1}{E - \hat{H}_{QQ}} \hat{H}_{QP} \right) \ket{\psi} = E \ket{\psi}.
\end{equation}
从这个方程求解出$E$(它们是微扰之后的哈密顿量的本征值!),我们就得到了$\hat{H}_\text{eff}$在$\mathcal{H}_1$的一组基上的作用结果,于是也就完全确定下了$\hat{H}_\text{eff}$。
换而言之,完全精确求解的有效哈密顿量保留了原哈密顿量在我们关注的空间上的全部能谱。

然而,即使上述自洽方程也难以求解,因为这是要对角化微扰之后的哈密顿量,也即对角化一个非常大的矩阵。为此通常使用微扰展开的方法。
设原哈密顿量中$\mathcal{H}_1$与$\mathcal{H}_2$没有耦合的部分为$\hat{H}_0$,其余部分为$\hat{H}'$,也即,以$\mathcal{H}_1$和$\mathcal{H}_2$为子空间将算符做分块,则$\hat{H}_0$包含对角部分,$\hat{H}'$包含非对角部分,则
\[
    \hat{H}_\text{eff} = \hat{H}_{0} + \hat{H}'_{PQ} \frac{1}{E - \hat{H}_{QQ}} \hat{H}'_{QP},
\]
做本征值和本征态的微扰展开,
% 注:无论怎么展开,都会涉及到哈密顿量的自乘,有自乘就有费曼图。这就是费曼图可以用来计算有效哈密顿量的原因。

\begin{equation}
    \mel{m}{\hat{H}_\text{eff}}{n} = E_m \delta_{mn} + \mel{m}{\hat{H}'}{n} + \frac{1}{2} \sum_{\text{$l$ in $\mathcal{H}_2$}} \left( \frac{\mel{m}{\hat{H}'}{l} \mel{l}{\hat{H}'}{n}}{E_m - E_l} + \frac{\mel{m}{\hat{H}'}{l} \mel{l}{\hat{H}'}{n}}{E_n - E_l} \right) + \cdots.
\end{equation}
其中$E$是微扰前的哈密顿量的本征值。

\subsubsection{退化到经典情况}\label{sec:back-to-classical}

前面提到,$\hbar \to 0$时,海森堡绘景下的量子时间演化方程\eqref{eq:quantum-evolution}退化为经典的时间演化方程\eqref{eq:evolution-of-any-quantity}。
但需要注意的是,在$\hbar\to 0$时由\eqref{eq:quantum-evolution}退化得到的方程仍然是一个算符方程。
要获得通常的关于物理量的方程,还需要做一些操作。$\hbar\to 0$时得到的演化方程是
\[
    \dv{\hat{A}}{t} = \pdv{\hat{A}}{\hat{\phi}} \dv{\hat{\phi}}{t} + \pdv{\hat{A}}{\partial_i \hat{\phi}} \partial_i \dv{\hat{\phi}}{t} + \pdv{\hat{A}}{\hat{\pi}} \dv{\hat{\pi}}{t},
\]
这个方程仅在海森堡绘景下成立。记系统的态矢量为$\ket{\psi}$,我们就得到
\[
    \dv{t} \mel{\psi}{\hat{A}}{\psi} =  \mel{\psi}{\pdv{\hat{A}}{\hat{\phi}} \dv{\hat{\phi}}{t}}{\psi} + \mel{\psi}{\pdv{\hat{A}}{\partial_i \hat{\phi}} \partial_i \dv{\hat{\phi}}{t}}{\psi} + \mel{\psi}{\pdv{\hat{A}}{\hat{\pi}} \dv{\hat{\pi}}{t}}{\psi}.
\]
在$\hbar\to 0$时,所有算符都近似是对易的,从而它们全部可以在同一组基下对角化。设这一组基为$\{\ket{n}\}$,则
% TODO:似乎$\ket{\psi}$总是几乎是这组基中的一个,为什么?
\[
    \begin{aligned}
        \mel{\psi}{\hat{A}\hat{B}}{\psi} &= \sum_{m,n} \braket{\psi}{m} \mel{m}{\hat{A}\hat{B}}{n} \braket{n}{\psi} \\
        &= \sum_{n} \braket{\psi}{n} \mel{n}{\hat{A}\hat{B}}{n} \braket{n}{\psi} \\
        &= 
    \end{aligned}
\]
% TODO
% 总之核心思想是,算符在$\hbar\to 0$时也不是实数物理量,真正的实数物理量的表达式必定会牵扯到态矢量。这也就是场算符的傅里叶分量看起来似乎是固定的值一样的原因,因为场算符本身包含了所有可能的经典场的取值,在$\hbar\to 0$时经典场的取值是多少不是场算符决定的而是态矢量决定的。

\subsection{关于单位制的注记}

到现在为止我们的理论还带有一些常数。用以标记我们的理论多大程度上偏离了经典情况的$\hbar$是一个重要的常数,同时标记了时间和空间的换算关系的光速$c$是另外一个。
通过做变换
\[
    t \longrightarrow t' = ct,
\]
我们可以让光速$c$从所有的公式中消失。相应的,时间导数算符发生了
\[
    \partial_t \longrightarrow \partial_{t'} = \frac{1}{c} \partial_t
\]
的变换。
$\hbar$在计算对易子的时候出现。做变换
\[
    \pi \longrightarrow \pi' = \frac{\pi}{\hbar}
\]
也可以完全消去这个常数。由于$\pi$是通过$\mathcal{L}$对$\partial_0 \phi$求导计算出来的,这个变换实际上就是对拉氏量做了变换
\[
    \mathcal{L} \longrightarrow \mathcal{L}' = \frac{\mathcal{L}}{\hbar},
\]
而这当然不影响实际的物理。事实上它改变的是能量和动量的单位。

从本节开始,在本文的剩余部分我们将使用自然单位制,那就是说,取消时间和空间的单位差异,并且取$\hbar = 1$。
从自然单位制恢复到国际单位制就是把上面的变换反过来,也就是做变换
\[
    \begin{aligned}
        \mathcal{L}_\text{nat} &\longrightarrow \mathcal{L}_\text{int} = \hbar \mathcal{L}_\text{nat}, \\
        E_\text{nat} &\longrightarrow E_\text{int} = \hbar E_\text{nat} , \\
        \vb*{p}_\text{nat} &\longrightarrow \vb*{p}_\text{int} = \hbar \vb*{p}_\text{nat}, \\
        t_\text{nat} &\longrightarrow t_\text{int} = c t_\text{nat}.
    \end{aligned}
\]
与此同时保持各个公式的形式不变。

\subsection{单粒子情况}

在已经知道了3+1维场论的理论之后,单粒子情况实际上就是一个退化情况,因为它实际上是0+1维场论。
在单粒子情况下底流形就是时间轴,其上定义有各种物理量$\hat{A}(t)$。单粒子情况下几乎不需要使用反对易量子化方案\eqref{eq:antisymmetry-commutator},物理量和它的共轭动量之间的关系可以全部取
\begin{equation}
    \comm*{\hat{x}}{\hat{p}} = \ii.
\end{equation}
下面推导$\hat{x}, \hat{p}$和任意物理量的对易关系。
设能够将物理量$\hat{F}$展开为$\hat{x}, \hat{p}$的多项式$\hat{F} = F(\hat{x}, \hat{p})$。
对其中的每一项,都可以使用对易关系\eqref{eq:x-p-commutator-1d}把$\hat{x}$挪到最前面而把$\hat{p}$挪到后面,
因此展开式最后就可以写成若干个$a \hat{x}^m \hat{p}^n$形式的项之和。
现在分析其中的一项:
\[
    [\hat{x}, \hat{x}^m \hat{p}^n] = \hat{x}^m [\hat{x}, \hat{p}^n] + [\hat{x}, \hat{x}^m] \hat{p}^n = \hat{x}^m [\hat{x}, \hat{p}^n],
\]
而
\[
    [\hat{x}, \hat{p}^n] = [\hat{x}, \hat{p} \hat{p}^{n-1}] = 
    \hat{p} [\hat{x}, \hat{p}^{n-1}] + [\hat{x}, \hat{p}] \hat{p}^{n-1} = \hat{p} [\hat{x}, \hat{p}^{n-1}] + \ii \hat{p}^{n-1}
\]
于是递推得到
\[
    [\hat{x}, \hat{p}^n] = \ii n \hat{p}^{n-1},
\]
因此
\[
    [\hat{x}, \hat{x}^m \hat{p}^n] = \ii n \hat{x}^m \hat{p}^{n-1}.
\]
这样就可以写出
\begin{equation}
    [\hat{x}, \hat{F}(\hat{x}, \hat{p})] = \ii \pdv{p} \hat{F}(\hat{x}, \hat{p}),
\end{equation}
在作用偏微分符号之前需要先把$F$中的每一项都变形成$\hat{x}$在前$\hat{p}$在后的形式。
使用同样的方法还可以导出
\begin{equation}
    [\hat{p}, \hat{F}(\hat{x}, \hat{p})] = - \ii \pdv{x} \hat{F}(\hat{x}, \hat{p}),
\end{equation}
同样,作用偏微分符号之前需要先把$F$中的每一项都变形成$\hat{x}$在前$\hat{p}$在后的形式。

在海森堡绘景下
\[
    \dv{\hat{A}}{t} = \frac{1}{\ii} [\hat{A}, H] + \pdv{\hat{A}}{t},
\]
于是
\[
    \dv{\hat{x}}{t} = \frac{1}{\ii} [\hat{x}, H] = \pdv{p} \hat{H}(\hat{x}, \hat{p}), \quad
    \dv{\hat{p}}{t} = \frac{1}{\ii} [\hat{p}, H] = -\pdv{x} \hat{H}(\hat{x}, \hat{p})
\]
当$\hbar \to 0$时,上式仍然成立,而此时$\hat{x}$和$\hat{p}$已经是对易的了,因此它们退化为了可以直接使用实数表示的情况,我们也就过渡到了经典力学。

\section{二次量子化}\label{sec:second-quantization}

在单粒子量子力学中我们发现要完整描述系统需要分析$\hat{\vb*{x}}$的各个本征态,
从而经典情况下描述$\vb*{x}$的轨迹方程在量子情况下不再适用,
而要改用关于\textbf{波函数}$\braket{\vb*{x}}{\phi} = \phi(\vb*{x})$的偏微分方程。
因此,量子理论中对粒子的描述涉及一个场。
本节将讨论相反的问题:量子理论中的场实际上能够自然地诱导出“粒子”的概念。
我们还将发现,单粒子量子力学中的波函数实际上真的相当于像电磁场这样的一个场。
将经典场量子化、并且从量子化的场当中发现粒子性的操作统称为\textbf{二次量子化}。
相应的,完全使用单粒子图像做的分析称为\textbf{一次量子化}。
有一些问题使用单粒子图像难以处理,因此二次量子化实际上是比一次量子化更加基本的理论,但很多时候我们先获得了一个一次量子化的理论,然后再构造一个能够推导出这个一次量子化的理论的场论以方便计算,从而似乎我们是在一次量子化理论上又做了一次量子化。
这并不是事实,虽然这种感觉正是“二次量子化”一词的来源。

% TODO:本节以及别的很多地方我们都认为使用一个算符就能够完全描述体系;但是实际上这个说法是不确切的,例如一个单粒子的状态就同时需要使用$\vb*{x}$和自旋来描述;但是能够将这两者直积起来,得到的算符的本征值是$\pmqty{ x^1 & x^2 & x^3 & S }$,于是它一个算符就成为了整个体系的CSCO。
% TODO:二次量子化场满足的方程和单粒子量子力学的波动方程之间的关系
\subsection{多粒子态空间}\label{sec:many-body-state}

本节我们将从两个方向分析多粒子态。
首先我们将从一系列完全相同的单粒子希尔伯特空间构造多粒子福克空间,
然后我们将说明,通常使用的这种多粒子福克空间实际上可以使用一对产生湮灭算符和一个唯一的真空态干脆利落地构造出来。
% 例如,同样是自由哈密顿量,玻色子可以取同样的量子态而费米子不行,为什么哈密顿量相同而竟然有不同的物理现象?关键在于当我们的问题涉及“两个费米子相撞”时根本就不能使用“自由哈密顿量”描写这个物理过程!
% 因此这里的关键在于需要在完全没有

\subsubsection{$n$粒子希尔伯特空间}\label{sec:n-particle-space}

考虑一个有$n$个粒子构成的体系。我们使用$\hat{M}_i$表示完全描述了第$i$号粒子的单粒子算符,也就是说它无简并。
既然不同的$i$对应的$\hat{M}_i$作用于不同的希尔伯特空间,它们当然就是对易的,
从而$M_1, M_2, \ldots, M_n$组成了一个完备对易算符集,整个体系的哈密顿量$\hat{H}$是这个算符集的函数。
设每个粒子的希尔伯特空间为$H_i$,那么整个体系的希尔伯特空间就是
\[
    H = H_1 \otimes H_2 \otimes \cdots \otimes H_n,
\]
且每个$H_i$都彼此同构。空间$H$的本征态可以写成
\[
    \ket{\text{eigenstate}} = \prod_i \ket{\text{one of $\hat{M}_i$'s eigenstates}}.
\]

我们说这些粒子是全同粒子,当且仅当
\begin{enumerate}
    \item 诸$M_i$具有相同的谱结构;从而,诸态空间$H_i, \; i=1, 2, \ldots$幺正等价,于是我们可以不失一般性地认为诸粒子的态空间都是相同的,且$M_1, M_2, \ldots$实际上可以看成某个抽象的李代数的元素$\hat{M}$在一系列幺正等价的态空间$H_1, H_2, \ldots$上的表示;
    \item 交换任意两个粒子之后系统的动力学不变,或者说交换两个粒子之后系统的哈密顿量不变。%
    \footnote{如果系统中粒子数恒等,这就是说系统的哈密顿量中不同粒子具有同等的地位;不过我们在这里并不处理这样的$n$粒子哈密顿量,因为在很多情况下粒子数会发生变化,从而哈密顿量中不可能直接含有各个粒子的算符。我们在\autoref{sec:from-qft-to-many-body}中会看到,此时哈密顿量中应该显含某种场算符,而单粒子算符是可以使用场算符弄出来的。}
\end{enumerate}
这两个要求缺一不可;前者要求诸粒子是同样的对象,后者要求交换粒子不改变系统的动力学。

如\autoref{sec:single-particle-quantity}所述,如果系统中的粒子在$\hbar \to 0$时对应到经典粒子,那么通常可以取$\hat{M}$为位置$\hat{\vb*{x}}$或者动量$\hat{\vb*{p}}$描述它们;不过这里我们首先处理更加一般性的问题而暂时不代入真实的物理。

下面我们来讨论交换粒子意味着什么。形如
\[
    \hat{P} (\ket{\psi_1}\ket{\psi_2}\cdots) = \ket{\psi_{k_1}}\ket{\psi_{k_2}}\cdots, \quad \{1, 2, \ldots\} = \{k_1, k_2, \ldots\}
\]
的算符$\hat{P}$称为\textbf{粒子交换算符}。显然,任何一个粒子交换算符都可以写成一系列交换两个粒子的算符的乘积。设算符$\hat{P}$交换了第$i$个粒子和第$j$个粒子,那么我们有
\[
    \hat{P} \left( \ket{\psi_1} \ket{\psi_2} \cdots \ket{\psi_i} \cdots \ket{\psi_j} \cdots \ket{\psi_n} \right) = \ket{\psi_1} \ket{\psi_2} \cdots \ket{\psi_j} \cdots \ket{\psi_i} \cdots \ket{\psi_n},
\]
既然$H$是各粒子态空间的直积而各个粒子的态空间实际上是完全一样的,算符$\hat{P}$实际上是从$H$到$H$的算符。
此外注意到$\hat{P}$并不改变态矢量的模,因此它还是一个幺正算符。
对一个全同粒子体系,交换两个粒子的信息不会改变系统的动力学,也就是说$\hat{H}$在$\hat{P}$的作用下不变,
因此$\hat{P}$和$\hat{H}$是对易的,这又意味着$\hat{P}$不会有时间演化。
于是可以迭代说明,任何一种粒子交换算符都没有时间演化。

考虑到交换任意两个$M_i$和$M_j$不改变系统,任何有意义的%
% TODO:“改变系统”和“改变系统的动力学”
% 还有,上面的说法是需要数学上的严格说明的
关于此$n$体系统的可观察量(它们是诸$M$的函数)在交换$M_i$和$M_j$之后形式保持不变。这就带来了一个结果:设$\hat{P}$是一个粒子交换算符(交换了哪些粒子随意),且$\hat{\phi}$是一个关于整个$n$体系统的可观察量%
\footnote{正如这个符号暗示的那样,$\hat{\phi}$通常可以被理解为某种量子化的场。我们将在\autoref{sec:from-qft-to-many-body}中看到这样的例子。},
那么$\hat{P}$和$\hat{\phi}$对易,于是
\[
    \hat{\phi} \left(\hat{P} \ket{\phi}\right) = \hat{P} \hat{\phi} \ket{\phi} = \hat{P} \phi \ket{\phi} = \phi \hat{P} \ket{\phi},
\]
这意味着任何一个本征态的置换都是另一个本征态,且本征值相同。这意味着$H$上任何有意义的可观察量以及它们的组合都不能够成为一组CSCO:任何有意义的可观察量都一定有简并!这样的简并称为\textbf{交换简并}。
这样,如果我们只关注系统的$\phi$属性的值,那么使用$\ket{\text{system}}$做计算和使用归一化的态$\sum_i \hat{P}_i \ket{\text{system}}$做计算得到的结果完全一样。
% TODO:对这一句话的严格解释:做什么样的计算?
因此$H$实际上是过大的——我们应该讨论某种比它的维度更低的空间,但其上的算符的结构不应该有变化。
也就是说,描写实际的全同粒子体系只需要$H$的一个子代数%
\footnote{当然,不构造子代数原则上也是可以的,但是有意义的算符都不构成CSCO这件事会大大加大处理问题的难度。
需注意虽然我们是把$H$的一部分孤立出来讨论,但这和构造混合态时“只考虑系统的一部分”是完全不同的。
全同粒子体系的态不会是不满足对称或者反对称条件的态矢量,因此从$H$缩小到$H$的对称化或反对称化子代数不会损失任何信息。
反之,构造混合态时被忽视的那部分系统仍然携带了信息。因此构造全同粒子体系不会产生混合态。},
任何一个$H$中的态都可以对应到这个子代数中的一个态,并且$\hat{\phi}$构成这个子代数的一个CSCO
——由于$\hat{\phi}$在粒子交换下不变,这又意味着,将交换算符作用在这个子代数的某一基矢量之后态的改变应该只是乘上了一个复数因子。
而由于粒子交换算符是幺正的,这个复数因子的模长一定是1。
% TODO:在我们只关注某些算符提供的信息时对希尔伯特空间的简化
% 一个可能的思路:记$f$将$H$中的态$\ket{\psi}$映射到了$H$的一个子代数上,且$f$是一个满射,且等价的态被$f$作用之后仍然是等价的;由于子代数中我们关注的算符构成CSCO,$H$中等价的态被$f$作用之后的结果只差了一个系数。
% TODO:定义什么叫做态“等价”,或者说“表示同一个系统”

具体这个复数因子是多少,不同的基矢量对应的复数因子是不是都相同,都值得讨论;
但是实际上实验数据暗示着,自然界中仅取这因子为$\pm 1$。这是有原因的——实际上这个因子的值和对应的粒子的自旋有关。
详细情况见\autoref{sec:from-qft-to-many-body}。
总之,现在我们需要构造$H$的对称化和反对称化子代数。
最简单的方法是,使用$H$的诸基矢量构造一组对称化与反对称化基矢量,这样无需对$H$上的算符做任何修改。

\subsubsection{对称与反对称基矢量}

这一节我们需要分析具体的粒子交换,因此引入一些记号。
$n$元组有$n!$种排列方式;我们将这$n!$种排列方式从$1$到$n!$编上号,并设$\hat{P}_s$为第$s$种排列方式对应的粒子交换算符,$p_s$是第$s$种排列方式对应的交换数。

设$M^{(1)}, M^{(2)}, \ldots$是$\hat{M}$的本征值。这里我们把$\hat{M}$看成离散谱的,但是可以取其极限得到连续谱的情况。
这样$H$的基矢量就全部可以写成
\[
    \ket{M^{(k_1)}}_1 \ket{M^{(k_2)}}_2 \cdots \ket{M^{(k_n)}}_n = \ket{M^{(k_1)} M^{(k_2)} \cdots M^{(k_n)}}
\]
的形式。其中下标$1, 2, \ldots$指的是这个基矢量在空间$H_1, H_2, \ldots$中。
设$k_1, k_2, k_3, \ldots$中有$n_1$个$1$,$n_2$个$2$,等等,称这些$n_1, n_2, \ldots$为\textbf{占据数};不重复的$k_i$——从而不重复的$M^{(k_i)}$——总共有$m$个,并且
\begin{equation}
    \sum_{i=1}^m n_i = n.
\end{equation}

现在尝试构造对称化子代数和反对称化子代数的基矢量。设$\ket{\psi}$是$H_S$或$H_A$中的一个态。当然,它也是$H$中的态,从而
\[
    \ket{\psi} = \sum_{k_1, k_2, \ldots, k_n} c_{k_1 k_2 \cdots k_n} \ket{M^{(k_1)}}_1 \ket{M^{(k_2)}}_2 \cdots \ket{M^{(k_n)}}_n,
\]
且
\[
    \hat{P}_s \ket{\psi} = (\pm 1)^{p_s} \ket{\psi},
\]
若$\ket{\psi}$是对称化的,则取$+1$,若$\ket{\psi}$是反对称化的,取$-1$。
然后就有
\[
    \sum_s (\pm 1)^{p_s} \hat{P}_s \ket{\psi} = \sum_s (\pm 1)^{p_s} \hat{P}_s \sum_{k_1, k_2, \ldots, k_n} c_{k_1 k_2 \cdots k_n} \ket{M^{(k_1)}}_1 \ket{M^{(k_2)}}_2 \cdots \ket{M^{(k_n)}}_n,
\]
此方程的左边是
\[
    \sum_s (\pm 1)^{p_s} \hat{P}_s \ket{\psi} = \sum_s (\pm 1)^{p_s} (\pm 1)^{p_s} \ket{\psi} = \sum_s \ket{\psi} = n! \ket{\psi},
\]
右边是
\[
    \begin{aligned}
        &\quad \sum_s (\pm 1)^{p_s} \hat{P}_s \sum_{k_1, k_2, \ldots, k_n} c_{k_1 k_2 \cdots k_n} \ket{M^{(k_1)}}_1 \ket{M^{(k_2)}}_2 \cdots \ket{M^{(k_n)}}_n \\
        &= \sum_{k_1, k_2, \ldots, k_n} c_{k_1 k_2 \cdots k_n} \sum_s (\pm 1)^{p_s} \hat{P}_s \ket{M^{(k_1)}}_1 \ket{M^{(k_2)}}_2 \cdots \ket{M^{(k_n)}}_n,
    \end{aligned}
\]
因此$\ket{\psi}$可以完全被形如
\[
    \sum_s (\pm 1)^{p_s} \hat{P}_s \ket{M^{(k_1)}}_1 \ket{M^{(k_2)}}_2 \cdots \ket{M^{(k_n)}}_n
\]
的矢量线性表示。
如果我们能够归一化这些矢量,并且证明它们的正交性,那么它们归一化之后就是$H_S$和$H_A$的基矢量。

首先我们讨论对称化子代数$H_S$。我们取$H$的形如下式的对称化基矢量:
\[
    \ket{n; M^{(k_1)} M^{(k_2)} \cdots M^{(k_n)}}_S \propto \sum_s \hat{P}_s \ket{M^{(k_1)}}_1 \ket{M^{(k_2)}}_2 \cdots \ket{M^{(k_n)}}_n,
\]
我们要计算其归一化系数。首先
\[
    \begin{aligned}
        &\quad \left(\sum_s \hat{P}_s \ket{M^{(k_1)}}_1 \ket{M^{(k_2)}}_2 \cdots \ket{M^{(k_n)}}_n\right)^\dagger \sum_s \hat{P}_s \ket{M^{(k_1)}}_1 \ket{M^{(k_2)}}_2 \cdots \ket{M^{(k_n)}}_n \\
        &= \sum_{s, s'} \bra{M^{(k_1)}}_1 \bra{M^{(k_2)}}_2 \cdots \bra{M^{(k_n)}}_n \hat{P}_{s'}^\dagger \hat{P}_s \ket{M^{(k_1)}}_1 \ket{M^{(k_2)}}_2 \cdots \ket{M^{(k_n)}}_n, 
    \end{aligned}
\]
注意到由于$\hat{P}$的幺正性和可逆性,$\hat{P}_{s'}^\dagger \hat{P}_s$也是一个粒子交换算符,
并且固定$s$不动,不同的$s'$会让$\hat{P}_{s'}^\dagger \hat{P}_s$取不同的值;
从而,对每个$s$,$\hat{P}_{s'}^\dagger \hat{P}_s$都有$n!$个值,
也即固定$s$不动而让$s'$从$1$取到$n!$,$\hat{P}_{s'}^\dagger \hat{P}_s$的值不重复地取遍所有共计$n!$个粒子交换算符;
从而当$s$和$s'$都从$1$计数到$n!$时,$\hat{P}_{s'}^\dagger \hat{P}_s$的值取遍所有粒子交换算符,且每个重复$n!$次,从而
\[
    \begin{aligned}
        &\quad \sum_{s, s'} \bra{M^{(k_1)}}_1 \bra{M^{(k_2)}}_2 \cdots \bra{M^{(k_n)}}_n \hat{P}_{s'}^\dagger \hat{P}_s \ket{M^{(k_1)}}_1 \ket{M^{(k_2)}}_2 \cdots \ket{M^{(k_n)}}_n \\
        &= n! \sum_s \bra{M^{(k_1)}}_1 \bra{M^{(k_2)}}_2 \cdots \bra{M^{(k_n)}}_n \hat{P}_s \ket{M^{(k_1)}}_1 \ket{M^{(k_2)}}_2 \cdots \ket{M^{(k_n)}}_n.
    \end{aligned}
\]
我们知道$\hat{M}$的各个本征态是正交的,因此上式中的内积只有在$\hat{P}$作用在$\ket{M^{(k_1)}}_1 \ket{M^{(k_2)}}_2 \cdots \ket{M^{(k_n)}}_n$得到的结果和作用前一样时才能取$1$,否则均取零。这样的排列方式总共有$n_1!n_2!\cdots n_m!$个,也就是说上式右边的求和号中值为$1$的项共有$n_1!n_2!\cdots n_m!$个,其余均为零,所以我们得到
\[
    \begin{aligned}
        &\quad \left(\sum_s \hat{P}_s \ket{M^{(k_1)}}_1 \ket{M^{(k_2)}}_2 \cdots \ket{M^{(k_n)}}_n\right)^\dagger \sum_s \hat{P}_s \ket{M^{(k_1)}}_1 \ket{M^{(k_2)}}_2 \cdots \ket{M^{(k_n)}}_n \\
        &= n! \sum_s \bra{M^{(k_1)}}_1 \bra{M^{(k_2)}}_2 \cdots \bra{M^{(k_n)}}_n \hat{P}_s \ket{M^{(k_1)}}_1 \ket{M^{(k_2)}}_2 \cdots \ket{M^{(k_n)}}_n \\
        &= n! n_1 ! n_2! \cdots ,
    \end{aligned}
\]
于是得到归一化对称基矢量
\begin{equation}
    \ket{n; M^{(k_1)} M^{(k_2)} \cdots M^{(k_n)}}_S = \frac{1}{\sqrt{n! n_1 ! n_2! \cdots}} \sum_s \hat{P}_s \ket{M^{(k_1)}}_1 \ket{M^{(k_2)}}_2 \cdots \ket{M^{(k_n)}}_n.
    \label{eq:sym-basis}
\end{equation}

同样的,我们考虑形如
\[
    \ket{n; M^{(k_1)} M^{(k_2)} \cdots M^{(k_n)}}_A \propto \sum_s (-1)^{p_s} \hat{P}_s \ket{M^{(k_1)}}_1 \ket{M^{(k_2)}}_2 \cdots \ket{M^{(k_n)}}_n.
\]
的反对称化基矢量。
使用和对称化基矢量同样的方法可以归一化这个基矢量。
在动手之前,注意到如果$\ket{M^{(k_1)}}_1$,$\ket{M^{(k_2)}}_2$,..., $\ket{M^{(k_n)}}_n$中有重复的态,
那么必定会导致相应的$\ket{n; M^{(k_1)} M^{(k_2)} \cdots M^{(k_n)}}_A$为零。
这是因为设$\hat{P}$交换了两个重复的态,那么就有
\[
    \hat{P} \ket{n; M^{(k_1)} M^{(k_2)} \cdots M^{(k_n)}}_A = \ket{n; M^{(k_1)} M^{(k_2)} \cdots M^{(k_n)}}_A,
\]
而由于这是反对称化基矢量,我们又有
\[
    \hat{P} \ket{n; M^{(k_1)} M^{(k_2)} \cdots M^{(k_n)}}_A = - \ket{n; M^{(k_1)} M^{(k_2)} \cdots M^{(k_n)}}_A,
\]
于是相应的反对称化基矢量就是零。因此我们只需要讨论其中所有单粒子态都不重复的反对称化基矢量。
如果单粒子态不重复,那么求和号中的态彼此正交,于是
\[
    \begin{aligned}
        &\quad \left(\sum_s (-1)^{p_s} \hat{P}_s \ket{M^{(k_1)}}_1 \ket{M^{(k_2)}}_2 \cdots \ket{M^{(k_n)}}_n\right)^\dagger \sum_s (-1)^{p_s} \hat{P}_s \ket{M^{(k_1)}}_1 \ket{M^{(k_2)}}_2 \cdots \ket{M^{(k_n)}}_n \\
        &= \sum_s (-1)^{2p_s} \left(\hat{P}_s \ket{M^{(k_1)}}_1 \ket{M^{(k_2)}}_2 \cdots \ket{M^{(k_n)}}_n\right)^\dagger \hat{P}_s \ket{M^{(k_1)}}_1 \ket{M^{(k_2)}}_2 \cdots \ket{M^{(k_n)}}_n \\
        &= \sum_s 1 = n!,
    \end{aligned}
\]
从而
\begin{equation}
    \ket{n; M^{(k_1)} M^{(k_2)} \cdots M^{(k_n)}}_A 
    = \frac{1}{\sqrt{n!}} \sum_s (-1)^{p_s} \hat{P}_s \ket{M^{(k_1)}}_1 \ket{M^{(k_2)}}_2 \cdots \ket{M^{(k_n)}}_n.
    \label{eq:asym-basis}
\end{equation}
当然,由于此时$n_1 = n_2 = \cdots = 1$,\eqref{eq:asym-basis}也可以写成\eqref{eq:sym-basis}的形式。

然后我们来讨论\eqref{eq:sym-basis}和\eqref{eq:asym-basis}的正交归一性。
归一化性已经通过计算归一化系数完成了,我们接下来讨论正交性。
无论作用怎么样的$\hat{P}_s$,都不会改变一个态中的占据数$n_1, n_2, \ldots$,
因此两个\eqref{eq:sym-basis}中占据数不同的态的求和号中出现的所有态都不相同%
\footnote{需要注意的是占据数相同的态也有可能不同。在$H_S$中占据数相同的态完全相同,因为它们之间差了有限次粒子交换;
$H_A$中占据数相同的态如果差了奇数次粒子交换,那么它们就差了一个负号,差了偶数次粒子交换则相同。},
因此两个\eqref{eq:sym-basis}中占据数不同的态的内积是零;
同样的思路也说明\eqref{eq:asym-basis}中占据数不同的态的内积是零。
在已经做了归一化之后,我们确认,\eqref{eq:sym-basis}和\eqref{eq:asym-basis}都满足正交归一化条件。

至此我们发现,实际的全同粒子系统只需要使用$H$的对称化子代数$H_S$或者反对称化子代数$H_A$描述,
\eqref{eq:sym-basis}是$H_S$的基底,\eqref{eq:asym-basis}是$H_A$的基底。
我们还知道,$H_A$中的基矢量中不会有两个粒子处于同样的态上,
从这个结论容易推导出,$H_A$中任何一个态中都不会有两个粒子的单粒子态完全相同。

\subsubsection{福克空间与产生湮灭算符}
% TODO:将“粒子数算符”的名称改为“占据数算符”
现在对任何一个正整数$n$,我们都已经建立起了$n$粒子全同粒子系统的态空间——那就是说,对称化和反对称化希尔伯特空间。
而我们也可以指定$n=0$时的“全同粒子系统”的态空间是平凡的向量空间$\{0\}$,并记其中唯一一个矢量为$\ket{0}$。
于是我们记$n$粒子对称化或反对称化希尔伯特空间为$H_S^{(n)}$和$H_A^{(n)}$,且$H_S^{(0)}$和$H_A^{(0)}$就是$\{\ket{0}\}$。
本节我们转而考虑这样的问题:在这一系列空间之间有什么联系?

在维数不同、互不等价的线性空间之间操作是非常麻烦的,因此我们尝试把诸$H_S^{(n)}$和$H_A^{(n)}$放在一个更大的空间中讨论,
这个更大的空间包含且仅包含诸$H_S^{(n)}$和$H_A^{(n)}$中的矢量。
构造这种“更大的空间”的方法有很多。
我们需要的操作绝对不是直积,因为两个空间的直积中含有原来的两个空间没有的向量。
因此尝试使用直和操作。
的确,并没有特殊的证据要求我们一定要使用直和,
但是如果我们讨论的系统中不涉及粒子数变动,那么具体使用的是直和还是别的什么操作无关紧要,
因为此时诸$H_S^{(n)}$和$H_A^{(n)}$直和出来的空间并没有物理意义
——我们只会把其中的粒子数和我们讨论的系统的粒子数相同的部分拿来做动力学计算。
而在\autoref{sec:from-qft-to-many-body}中会看到,使用直和是正确的。
总之,定义\textbf{对称的福克空间}
\[
    F_S = H_S^{(0)} \oplus H_S^{(1)} \oplus H_S^{(2)} \oplus \cdots,
\]
以及\textbf{反对称的福克空间}
\[
    F_A = H_A^{(0)} \oplus H_A^{(1)} \oplus H_A^{(2)} \oplus \cdots.
\]
每个空间的基矢量为
\[
    \ket{0}, \; \ket{1;M^{(k_1)}}, \; \ket{2;M^{(k_1)} M^{(k_2)}}, \; \ldots,
\]
其中$k_1, k_2, \ldots$跑遍所有可能的本征值。

现在我们定义\textbf{产生湮灭算符}。%
\footnote{产生湮灭算符未必有动力学上的意义。例如,在粒子数固定的动力学中它们就只是一种新的观点。但是在粒子数会变的情况下它们很重要。}%
以下我们略去了下角标$A$和$S$,因为这些定义与对称还是反对称无关。
首先取\textbf{产生算符}$a^\dagger$,使之满足
\begin{equation}
    \hat{a}^\dagger (M^{(i)}) \ket{n; M^{(k_1)} M^{(k_2)} \cdots M^{(k_n)}} = \sqrt{n_i + 1} \ket{n+1; M^{(i)} M^{(k_1)} M^{(k_2)} \cdots M^{(k_n)}}.
    \label{eq:creation-operator}
\end{equation}
其中${n_i}$指的是$\ket{n; M^{(k_1)} M^{(k_2)}}$中$M^{(i)}$态的个数。
给定了相应的系数,上式就完全地确定了一个福克空间(无论对称还是反对称)上的算符。%
\footnote{至于为什么要取这个$\sqrt{n_i + 1}$的系数,我们将在定义粒子数算符的时候看到。}
产生算符将$n$粒子态转化为$n+1$粒子态。
现在要问:有没有一个算符能够将$n+1$粒子态转化为$n$粒子态?
实际上由共轭转置的定义(注意到\eqref{eq:creation-operator}只涉及基矢量),
$\hat{a}^\dagger$的共轭转置,也就是$\hat{a}$就是满足这个条件的算符,
因为由共轭转置的定义可以导出
\[
    \hat{a} (M^{(i)}) \ket{n+1; M^{(i)} M^{(k_1)} M^{(k_2)} \cdots M^{(k_n)}} = \sqrt{n_i + 1} \ket{n; M^{(k_1)} M^{(k_2)} \cdots M^{(k_n)}},
\]
其中$n_i$指的是$\ket{n;M^{(i)} M^{(k_1)} M^{(k_2)} \cdots M^{(k_n)}}$中$M^{(i)}$态的个数,
于是我们称$\hat{a}$为\textbf{湮灭算符}。
这里还有一个微妙的细节。注意到$\hat{a}^\dagger(M^{(i)})$作用在任何一个态上面都不可能产生一个不含有$M^{(i)}$态的态,因此不能仅仅依靠\eqref{eq:creation-operator}得到$\hat{a}(M^{(i)})$作用在一个不含$M^{(i)}$态的态上的结果。
于是我们额外规定:$\hat{a}(M^{(i)})$作用在一个不含$M^{(i)}$态的态上得到长度为零的向量%
\footnote{不是得到$\ket{0}$——$\ket{0}$是真实的物理态,而长度为零的向量不是。},
从而$\hat{a}$和$\hat{a}^\dagger$在整个福克空间上都定义好了。%
\footnote{刚才提到的这种情况实际上来自数学上的一个结论:在希尔伯特空间中,$(\hat{a}^\dagger)^\dagger$未必就是$\hat{a}$。当然,我们在这里的处理方法并没有什么逻辑漏洞——当我们要求\eqref{eq:creation-operator}成立时我们只是要求存在某个算符$\hat{a}$(而不是产生算符)使得此方程成立,而接下来要求$\hat{a}(M^{(i)})$作用在一个不含$M^{(i)}$态的态上得到长度为零的向量就唯一确定了$\hat{a}$。}
这样如果重新定义$n_i$为$\ket{n+1; M^{(i)} M^{(k_1)} M^{(k_2)} \cdots M^{(k_n)}}$中$M_i$态的数量,那么就可以使用一个式子完全刻画$\hat{a}(M^{(i)})$的行为:
\begin{equation}
    \hat{a} (M^{(i)}) \ket{n+1; M^{(i)} M^{(k_1)} M^{(k_2)} \cdots M^{(k_n)}} = \sqrt{n_i} \ket{n; M^{(k_1)} M^{(k_2)} \cdots M^{(k_n)}}.
    \label{eq:annihitation-operator}
\end{equation}
于是\eqref{eq:creation-operator}和\eqref{eq:annihitation-operator}就给出了产生湮灭算符的定义。
请注意这两个式子中的$n_i$是不同的;它们分别是两个式子中被算符作用前的态中$M^{(i)}$态的个数。

现在指出一个事实:任何一个\eqref{eq:sym-basis}和\eqref{eq:asym-basis}中的态都可以通过产生算符和真空态$\ket{0}$推导出来。原因很简单:重复使用\eqref{eq:creation-operator},我们有
\begin{equation}
    \ket*{n; \underbrace{M^{(1)} M^{(1)} \cdots}_{n_1} \underbrace{M^{(2)} M^{(2)} \cdots}_{n_2} \cdots} = \frac{1}{\sqrt{n_1! n_2! \cdots}} \left(\hat{a}^\dagger (M^{(1)})\right)^{n_1} \left(\hat{a}^\dagger (M^{(2)})\right)^{n_2} \cdots \ket{0},
    \label{eq:creation-basis}
\end{equation}
而通过排列算符,\eqref{eq:sym-basis}和\eqref{eq:asym-basis}中的每一个基矢量都可以写成上式左边的形式(可能差一个负号),
因此所有的基矢量都可以通过产生算符构造出来。我们称形如\eqref{eq:creation-basis}这样的态也即,像\eqref{eq:sym-basis}和\eqref{eq:asym-basis}这样的态,为\textbf{乘积态},因为它们可以写成一系列产生算符乘积作用在真空态上的形式。

定义产生湮灭算符之后可以定义
\begin{equation}
    \hat{n}(M^{(i)}) = \hat{a}^\dagger (M^{(i)}) \hat{a} (M^{(i)})
    \label{eq:number-operator}
\end{equation}
为\textbf{占据数算符}。它叫做占据数算符是因为,按照\eqref{eq:creation-operator}和\eqref{eq:annihitation-operator}可以验证,我们有
\begin{equation}
    \hat{n}(M^{(i)}) \ket{n; M^{(k_1)} M^{(k_2)} \cdots M^{(k_n)}} = n_i \ket{n; M^{(k_1)} M^{(k_2)} \cdots M^{(k_n)}},
    \label{eq:number-eigenstate}
\end{equation}
其中$n_i$指$\ket{n; M^{(k_1)} M^{(k_2)} \cdots M^{(k_n)}}$中$M^{(i)}$态的个数。
\eqref{eq:number-eigenstate}意味着\eqref{eq:sym-basis}和\eqref{eq:asym-basis}都是$\hat{n}(M^{(i)})$的本征态,本征值就是$M^{(i)}$态的数目。
这就是占据数算符一词的来源。
相应的
\begin{equation}
    \hat{N} = \sum_i \hat{n}(M^{(i)})
    \label{eq:total-number-operator}
\end{equation}
就是\textbf{总粒子数算符},它的本征态也是\eqref{eq:sym-basis}和\eqref{eq:asym-basis},本征值为对应的态的总粒子数。

\subsubsection{产生湮灭算符的代数结构}\label{sec:algebra-ca-op}

% TODO:然后问题就来了:诸粒子数算符能不能完整描述被场算符完整描述的态空间?
在前面几节中,我们通过$M$表象下的单粒子空间构造出了全同粒子体系的希尔伯特空间,
通过全同性得到了有物理意义的对称化和反对称化希尔伯特空间,然后将它们直和起来得到福克空间,最后在其上定义了产生湮灭算符。
本节我们将推导产生湮灭算符的几个代数性质,然后接着我们会发现,通过这几个性质可以反过来,
从产生湮灭算符和一个真空态$\ket{0}$能够得到对称和反对称福克空间,然后退化到单粒子态。

首先我们推导产生湮灭算符的对易关系。我们有
\[
    \begin{split}
        \ket{n+2;M^{(i)}M^{(j)}M^{(k_1)}M^{(k_2)} \cdots} = \hat{a}^\dagger (M^{(i)}) \hat{a}^\dagger (M^{(j)}) \ket{n; M^{(k_1)}M^{(k_2)} \cdots}, \\ 
        \quad \ket{n+2;M^{(i)}M^{(j)}M^{(k_1)}M^{(k_2)} \cdots} = \hat{a}^\dagger (M^{(i)}) \hat{a}^\dagger (M^{(j)}) \ket{n; M^{(k_1)}M^{(k_2)} \cdots},
    \end{split}
\]
从而对$F_S$和$F_A$分别有
\[
    \hat{a}^\dagger (M^{(i)}) \hat{a}^\dagger (M^{(j)}) \ket{n; M^{(k_1)}M^{(k_2)} \cdots} = \pm \hat{a}^\dagger (M^{(i)}) \hat{a}^\dagger (M^{(j)}) \ket{n; M^{(k_1)}M^{(k_2)} \cdots}
\]
$F_S$取正号,$F_A$取负号。考虑到被它们作用的矢量的任意性,得到
\[
    \hat{a}^\dagger (M^{(i)}) \hat{a}^\dagger = \pm \hat{a}^\dagger (M^{(i)}) \hat{a}^\dagger (M^{(j)}).
\]
于是对$F_S$有
\[
    [\hat{a}^\dagger (M^{(i)}), \hat{a}^\dagger (M^{(j)})] = 0,
\]
对$F_A$有
\[
    \acomm{\hat{a}^\dagger (M^{(i)})}{\hat{a}^\dagger (M^{(j)})} = 0.
\]
特别的,
\[
    \hat{a}^\dagger (M^{(i)}) \hat{a}^\dagger (M^{(i)}) = 0.
\]
对以上公式取共轭转置,可以推出,对$F_S$有
\[
    \comm{\hat{a}(M^{(i)})}{\hat{a}(M^{(j)})} = 0,
\]
对$F_A$有
\[
    \acomm{\hat{a}(M^{(i)})}{\hat{a}(M^{(j)})} = 0.
\]
然后再考虑一个产生算符和一个湮灭算符的对易关系。
使用和上面类似的方式,可以推导出
\[
    \hat{a}(M^{(i)}) \hat{a}^\dagger (M^{(j)}) \pm \hat{a}^\dagger (M^{(j)}) \hat{a} (M^{(i)}) = \delta_{ij} \hat{I}.
\]
综上,在$F_S$中有
\begin{equation}
    \comm{\hat{a}(M^{(i)})}{\hat{a}(M^{(j)})} = \comm{\hat{a}^\dagger(M^{(i)})}{\hat{a}^\dagger(M^{(j)})} = 0, \quad \comm{\hat{a} (M^{(i)})}{\hat{a}^\dagger (M^{(j)})} = \delta_{ij},
    \label{eq:commutator-of-ca-op}
\end{equation}
在$F_A$中有
\begin{equation}
    \acomm{\hat{a}(M^{(i)})}{\hat{a}(M^{(j)})} = \acomm{\hat{a}^\dagger(M^{(i)})}{\hat{a}^\dagger(M^{(j)})} = 0, \quad \acomm{\hat{a} (M^{(i)})}{\hat{a}^\dagger (M^{(j)})} = \delta_{ij}.
    \label{eq:anticommutator-of-ca-op}
\end{equation}
从\eqref{eq:anticommutator-of-ca-op}可以看出,在$n_i > 1$时
\[
    \left( \hat{a}^\dagger (M^{(i)}) \right)^{n_i} = 0,
\]
结合\eqref{eq:creation-basis},我们再次发现反对称情况下不可能有重复出现的单粒子态。

得到了\eqref{eq:commutator-of-ca-op}和\eqref{eq:anticommutator-of-ca-op}之后,我们再来分析粒子数算符的性质。
首先可以计算出对易关系
\begin{equation}
    \begin{aligned}
        \comm{\hat{n}(M^{(i)})}{\hat{a}(M^{(i)})} = - \hat{a}(M^{(i)}), \quad \comm{\hat{n}(M^{(i)})}{\hat{a}^\dagger(M^{(i)})} = \hat{a}^\dagger(M^{(i)}), \\
        \comm{\hat{n}(M^{(i)})}{\hat{a}(M^{(j)})} = \comm{\hat{n}(M^{(i)})}{\hat{a}^\dagger (M^{(j)})} = 0
        \label{eq:commutator-n-a}
    \end{aligned}
\end{equation}
无论对$F_S$还是$F_A$均成立,这表明产生算符是与它关于同一个态的粒子数算符的升算符,湮灭算符是与它关于同一个态的粒子数算符的降算符。
将此对易关系线性叠加,得到
\begin{equation}
    \comm{\hat{n}}{\hat{a}^\dagger (M^{(i)})} =  \hat{a}^\dagger (M^{(i)}), \quad \comm{\hat{n}}{\hat{a} (M^{(i)})} =  -\hat{a} (M^{(i)}),
    \label{eq:commutator-ntotal-a}
\end{equation}
因此任何$M^{(i)}$标记的产生湮灭算符都是总粒子数算符的升降算符。
此外还有
\begin{equation}
    \comm{\hat{n}(M^{(i)})}{\hat{n}(M^{(j)})} = 0,
    \label{eq:commutator-n-and-n}
\end{equation}
无论$i$和$j$是不是相等、是$F_S$还是$F_A$。
另一方面,从基矢量\eqref{eq:sym-basis}和\eqref{eq:asym-basis}中任取一个,记它当中$M^{(1)}$态有$n_1$个,$M^{(2)}$态有$n_2$个,等等,则它和
\[
    \ket*{n; \underbrace{M^{(1)} M^{(1)} \cdots}_{n_1} \underbrace{M^{(2)} M^{(2)} \cdots}_{n_2} \cdots} 
\]
最多差一个负号,由于$n_1, n_2, \ldots$可以自由地变动而彼此无影响,诸粒子数算符$\hat{n}(M^{(1)}), \hat{n}(M^{(2)}), \ldots$足够完全描述诸态矢量了,从而也足够描述诸态空间了。
因此全体粒子数算符组成的集合是福克空间的一个CSCO。
既然我们使用的本征矢\eqref{eq:sym-basis}和\eqref{eq:asym-basis}是诸粒子数算符的共同本征矢(见\eqref{eq:number-eigenstate}),我们可以将使用诸粒子数算符为CSCO、使用\eqref{eq:sym-basis}或\eqref{eq:asym-basis}为基矢量的福克空间的表象称为\textbf{粒子数表象}或者\textbf{占据数表象}(因为我们讨论占据数$n_1$,$n_2$,等等)。
相应地,可以记基矢量为$\ket{n(M^{(1)}), n(M^{(2)}), \ldots}$。
% TODO:使用小n而不是大N来表示占据数

现在我们选取一条反过来的思路。首先设我们找到了一族算符$\hat{a}(M^{(i)})$,$i=1, 2, \ldots$,并且这些算符满足\eqref{eq:commutator-of-ca-op}或\eqref{eq:anticommutator-of-ca-op}中的其中一个,并满足
\begin{equation}
    \hat{a}(M^{(i)}) \ket{0} = 0,
    \label{eq:annihitation-on-vaccum}
\end{equation}
以及一个唯一的真空态$\ket{0}$。
一旦确认了这一点,按照\eqref{eq:number-operator}和\eqref{eq:total-number-operator}定义粒子数算符,立刻可以推导出\eqref{eq:commutator-n-a},从而\eqref{eq:commutator-ntotal-a}和\eqref{eq:commutator-n-and-n}。
使用\eqref{eq:creation-basis}计算出一系列矢量,这些矢量满足\eqref{eq:creation-operator}和\eqref{eq:annihitation-operator},且容易证明这些矢量张成一个对称(如果使用\eqref{eq:commutator-of-ca-op})或者反对称(如果使用\eqref{eq:anticommutator-of-ca-op})福克空间,且粒子数算符组成的集合构成这个空间上的一个CSCO。
和前面从多粒子态上的产生算符导出湮灭算符时一样,单单依靠\eqref{eq:creation-basis}不能够确定湮灭算符作用在$\ket{0}$上的结果。但既然我们已经有了\eqref{eq:annihitation-on-vaccum},通过对易或反对易关系很容易就能够证明$\hat{a}(M^{(i)})$作用在一个不含$M^{(i)}$态的态上给出$0$。
最后,该福克空间上的单粒子态为
\begin{equation}
    \ket{1;M^{(i)}} = \hat{a}^\dagger (M^{(i)}) \ket{0}
\end{equation}
张成的希尔伯特空间,其上可以定义单粒子算符
\begin{equation}
    \hat{M} = \sum_i M^{(i)} \dyad{M^{(i)}}.
\end{equation}
这表明,一个唯一的真空态,和一组满足\eqref{eq:commutator-of-ca-op}或\eqref{eq:anticommutator-of-ca-op}中的其中一个的算符,就能够完全刻画一个福克空间。

但是很快我们就会注意到一件事:我们从单粒子态出发定义福克空间的时候是有一个内积结构的,而使用一套产生湮灭算符和一个真空态构造出来的福克空间上的内积结构应该怎么定义呢?
实际上并不需要为产生湮灭算符引入额外的结构来描述福克空间的内积。
要看出这是为什么,注意到两个态的内积无非形如
\[
    \begin{aligned}
        \braket{\psi_1}{\psi_2} &\propto \left( \hat{a}(M^{(i_1)})^\dagger \hat{a}(M^{(i_2)})^\dagger \cdots \ket{0} \right)^\dagger \left( \hat{a}(M^{(j_1)})^\dagger \hat{a}(M^{(j_2)})^\dagger \cdots \ket{0} \right) \\
        &= \mel{0}{\cdots \hat{a}(M^{(i_2)}) \hat{a}(M^{(i_1)}) \hat{a}^\dagger(M^{(j_1)}) \hat{a}^\dagger (M^{(j_2)}) \cdots }{0},
    \end{aligned}
\]
推导上式只使用了\eqref{eq:creation-basis},因此在通过产生湮灭算符构造出来的福克空间和使用单粒子态构造出来的福克空间中都成立。
我们总是可以使用对易关系\eqref{eq:commutator-of-ca-op}或者反对易关系\eqref{eq:anticommutator-of-ca-op},把形如$\hat{a}_i \hat{a}^\dagger_j$的表达式写成形如$\hat{a}^\dagger_j \hat{a}_i$的表达式和一个常数之和,不断重复这个步骤,整个
\[
    \cdots \hat{a}(M^{(i_2)}) \hat{a}(M^{(i_1)}) \hat{a}^\dagger(M^{(j_1)}) \hat{a}^\dagger (M^{(j_2)}) \cdots
\]
就能够写成几种\textbf{正规序列}——也就是所有产生算符都在所有湮灭算符左边的算符连乘积,还有若干常数的线性组合。线性组合的系数和线性组合中出现的常数完全由\eqref{eq:commutator-of-ca-op}或\eqref{eq:anticommutator-of-ca-op}确定。
因此,只要使用产生湮灭算符的代数关系求出正规序列的真空态期望,就能够使用产生湮灭算符的代数结构刻画福克空间的内积。
如果正规序列中有湮灭算符,那么其真空态期望就是零,因为
\[
    \mel{0}{\cdots \hat{a}(M^{(i)})}{0} = \left( \cdots \ket{0} \right)^\dagger \hat{a}(M^{(i)}) \ket{0} = 0.
\]
而如果正规序列中只有产生算符,那么由于
\[
    \mel{0}{\hat{a}^\dagger(M^{(i)}) \hat{a}^\dagger(M^{(j)}) \cdots}{0} = \mel{0}{\cdots \hat{a}(M^{(j)}) \hat{a}(M^{(i)})}{0}^*,
\]
其真空态期望照样是零。因此正规序列的真空态期望一定是零。
因此,福克空间的内积结构也被产生湮灭算符描述了。

最后我们讨论表象变换。以上所有的讨论使用的都是$M$表象。现在如果我们使用$G$表象的单粒子态构造出了另外一套福克空间,这两个空间之间有什么样的联系?
综合使用\eqref{eq:sym-basis}和\eqref{eq:asym-basis}以及\eqref{eq:creation-basis},还有我们熟悉的单粒子态的表象变换公式
\[
    \ket{M^{(i)}} = \sum_j \ket{G^{(j)}} \braket{G^{(j)}}{M^{(i)}},
\]
可以发现,设$\hat{a}(M^{(i)})$是用$M$表象建立起来的福克空间的湮灭算符,$\hat{a}(G^{(i)})$是用$G$表象建立起来的福克空间的湮灭算符,且两个空间同时使对称空间或者同时是反对称空间,那么
\begin{equation}
    \begin{bigcase}
        \hat{a}^\dagger (M^{(i)}) = \sum_j \braket{G^{(j)}}{M^{(i)}} \hat{a}^\dagger (G^{(j)}), \\
        \hat{a} (M^{(i)}) = \sum_j \braket{M^{(i)}}{G^{(j)}} \hat{a} (G^{(j)}).
    \end{bigcase}
    \label{eq:creation-and-annihitation-trans}
\end{equation}
反之,在已有一套$\hat{a}(M^{(i)})$满足\eqref{eq:creation-and-annihitation-trans}且有真空态,从而能够使用这套算符构造一个$M$表象的对称福克空间时,使用\eqref{eq:creation-and-annihitation-trans}可以得到另一套$\hat{a}(G^{(i)})$,使用它们可以构造出一个$G$表象的对称福克空间;反对称同理。

\subsubsection{二次量子化形式的算符}

在\autoref{sec:n-particle-space}中已经说明,多粒子系统中有意义的可观察量在粒子交换下不变。
本节讨论所有这样的有意义的可观察量的形式。
我们首先考虑能够在多粒子态基底\eqref{eq:sym-basis}或\eqref{eq:asym-basis}下被对角化的算符,这就是说,关于单粒子、二粒子、三粒子……空间的算符。设$\hat{A}$是一个这样的算符,则
\[
    \hat{A} = \sum_i \sum_{j_1, j_2, \ldots, j_i} \hat{A}^{(i)}_{j_1 j_2 \ldots j_i},
\]
上标$(i)$表示这是$i$粒子算符,下标$j_1, j_2, \ldots$表示这是作用在第$j_1, j_2, \ldots$号粒子上的算符。
如果序列$j_1, j_2, \ldots$中有重复的序号,那么相应的$\hat{A}^{(i)}_{j_1 j_2 \ldots j_i}$实际上完全可以使用只涉及$i-1$粒子空间甚至涉及更少的粒子的空间上的算符来表示,因此不失一般性地,我们要求
% TODO:为什么?两个处于同一量子态的粒子之间不能有相互作用吗?
\[
    \hat{A} = \sum_i \sum_{j_1 \neq j_2 \neq \ldots \neq j_i} \hat{A}^{(i)}_{j_1 j_2 \ldots j_i}.
\]
考虑到$\hat{A}$是有意义的可观察量,它和任何一个粒子交换算符都是对易的。
这又意味着,所有的$\hat{A}^{(i)}_{j_1 j_2 \ldots j_i}$都是厄米算符,且它们和粒子交换算符都是对易的。
因此如果两个序列$j_1, j_2, \ldots, j_i$ \\
和$j'_1, j'_2, \ldots, j'_i$中出现的数完全相同,它们对应的$\hat{A}^{(i)}_{j_1 j_2 \ldots j_i}$和$\hat{A}^{(i)}_{j'_1 j'_2 \ldots j'_i}$就是完全相同的算符。
因此通过调整每一项所含的算符前面的系数我们可以写出
\begin{equation}
    \hat{A} = \sum_i \sum_{j_1 < j_2 < \ldots < j_i} \hat{A}^{(i)}_{j_1 j_2 \ldots j_i},
    \label{eq:fock-observable-original}
\end{equation}
求和号中包含的所有项都是厄米的,且它们各不相同。如果处理严格递增的指标比较麻烦,可以等价地写出
\begin{equation}
    \hat{A} = \sum_i \frac{1}{i!} \sum_{j_1 \neq j_2 \neq \ldots \neq j_i} \hat{A}^{(i)}_{j_1 j_2 \ldots j_i}.
    \label{eq:fock-observable}
\end{equation}
\eqref{eq:fock-observable}和\eqref{eq:fock-observable-original}中的$\hat{A}^{(i)}_{j_1 j_2 \ldots j_i}$完全相同;\eqref{eq:fock-observable}中的阶乘系数是为了消除下标重复计数,因为含有$i$个各不相同的数的序列有$i!$种排列方式,每种排列方式对应的$\hat{A}^{(i)}_{j_1 j_2 \ldots j_i}$都完全一样,且都对应着\eqref{eq:fock-observable-original}中要求$j_1 < j_2 < \ldots$的那个$\hat{A}^{(i)}_{j_1 j_2 \ldots j_i}$。
再次,由于$\hat{A}$是有意义的可观察量,\eqref{eq:fock-observable}中的$\hat{A}^{(i)}_{j_1 j_2 \ldots j_i}$和$\hat{A}^{(i)}_{j'_1 j'_2 \ldots j'_i}$只相差了一个粒子交换变换,即使它们涉及编号不同的粒子。
反之,如果任何两个$\hat{A}^{(i)}_{j_1 j_2 \ldots j_i}$和$\hat{A}^{(i)}_{j'_1 j'_2 \ldots j'_i}$只相差一个粒子交换,% TODO:两个算符之间相差一个“变换”,两个态之间相差一个算符
且对彼此互为重排的两个序列
\[
    j_1, j_2, \ldots, j_i, \quad j'_1, j'_2, \ldots, j'_i
\]
它们对应的$\hat{A}^{(i)}$完全相同,
那么通过\eqref{eq:fock-observable}或等价的\eqref{eq:fock-observable-original}给出的$\hat{A}$一定是福克空间上的可观察量。由于\eqref{eq:fock-observable}中第二个求和号涉及的任何一个$j_1, j_2, \ldots$这样的序列都不含有重复的元素,诸$\hat{A}^{(i)}_{j_1 j_2 \ldots j_i}$实际上可以通过将一个作用在$i$粒子对称化或反对称化希尔伯特空间上的算符作用在$j_1, j_2, \ldots$号粒子上得到,也就是说,\eqref{eq:fock-observable}中对$i$的求和号中的每一项都是使用同一个$i$粒子算符作用在不同的粒子上得出的。
因此我们有最后的结论:算符$\hat{A}$是福克空间上可使用\eqref{eq:sym-basis}或\eqref{eq:asym-basis}对角化的可观察量,当且仅当,可以找到一系列$i$粒子可观察量$\hat{A}^{(i)}$,使得\eqref{eq:fock-observable}或等价的\eqref{eq:fock-observable-original}成立,其中$\hat{A}^{(i)}_{j_1 j_2 \ldots j_i}$指的是$\hat{A}^{(i)}$在$j_1, j_2, \ldots, j_i$号粒子上的作用。

现在再尝试使用产生湮灭算符写出\eqref{eq:fock-observable}中的每一项。
对粒子的编号完全是任意的(当然也可以是任意的,既然交换粒子之后什么都没有变),因此我们将随时采用最方便的安排方式。
设$\hat{A}^{(i)}$是一个$i$粒子算符,且它与粒子交换算符对易。
% 下面的说法仍然有很大的直觉考量因素。能否从数学上严格处理?
% 在$n < i$时,
% \[
%     \hat{A}^{(i)}_{j_1 j_2 \ldots j_i} \ket{n;M^{(k_1)} M^{(k_2)} \cdots M^{(k_n)}} = 0,
% \]
% 而在$n \geq i$时,
% \[
%     \begin{aligned}
%         &\quad \hat{A}^{(i)}_{j_1 j_2 \ldots j_i} \ket{n; M^{(k_1)} M^{(k_2)} \cdots M^{(k_n)}} \\
%         &= \begin{bigcase}
%             \sum_{l_1, l_2, \ldots, l_i} \ket{n; M^{(k'_1) M^{(k'_2)}} \cdots M^{(k'_n)}} \mel{M^{(l_1)} M^{(l_2)} \cdots M^{(l_i)}}{\hat{A}^{(i)}}{M^{(k_{j_1})} M^{(k_{j_2})} \cdots M^{(k_{j_i})}}, \\ \text{with no repetition in $M^{(k_1)}, M^{(k_2)}, \ldots$}, \\
%             0, \quad \text{with repetition.}
%         \end{bigcase},
%     \end{aligned}
% \]
% 其中
% \[
%     k'_a = \begin{cases}
%         k_a, \quad a \neq j_1, j_2, \ldots, j_i, \\
%         l_b, \quad a = j_b, \; b = 1, 2, \ldots, i.
%     \end{cases}
% \]
% TODO:实际上以上两个式子就是“\hat{A}^{(i)}作用在n粒子态上”的定义,为什么可以这么定义?
那么,$\hat{A}^{(i)}_{j_1 j_2 \ldots j_i}$就是
\[
    \begin{aligned}
        &\hat{A}^{(i)}_{j_1 j_2 \ldots j_i} 
        = \sum_{k_1, k_2, \ldots, k_i} \sum_{l_1, l_2, \ldots, l_i} \\
        &\ket{M^{(k_1)}}_{j_1} \ket{M^{(k_2)}}_{j_2} \cdots \ket{M^{(k_i)}}_{j_i} 
        \bra{M^{(l_1)}}_{j_1} \bra{M^{(l_2)}}_{j_2} \cdots \bra{M^{(l_i)}}_{j_i} \\ 
        &\mel{M^{(k_1)} M^{(k_2)} \cdots M^{(k_i)}}{\hat{A}^{(i)}}{M^{(l_1)} M^{(l_2)} \cdots M^{(l_i)}}.
    \end{aligned}
\]
我们把$\hat{A}^{(i)}_{j_1 j_2 \ldots j_i}$作用在一个任意粒子数的基矢量上,其结果是将上式中的右矢和多粒子态基矢量做张量缩并的产物。
%\[
%    \begin{aligned}
%        &\quad \hat{A}^{(i)}_{j_1 j_2 \ldots j_i} \ket{n; M^{(k_1)} M^{(k_2)} \cdots M^{(k_n)}} \\
%        &= 
%    \end{aligned}
%\]
% TODO:这个严格的数学证明,不过我懒得写
可以看出,
\[
    \begin{aligned}
        &\quad \sum_{j_1 \neq j_2 \neq \ldots \neq j_i} \ket{M^{(k_1)}}_{j_1} \ket{M^{(k_2)}}_{j_2} \cdots \ket{M^{(k_i)}}_{j_i} 
        \bra{M^{(l_1)}}_{j_1} \bra{M^{(l_2)}}_{j_2} \cdots \bra{M^{(l_i)}}_{j_i} \\
        &= \hat{a}^\dagger (M^{(k_1)}) \hat{a}^\dagger (M^{(k_2)}) \cdots \hat{a}^\dagger (M^{(k_i)}) \hat{a} (M^{(l_i)}) \cdots \hat{a} (M^{(l_2)}) \hat{a} (M^{(l_1)})  
    \end{aligned}
\]
结合以上各式,我们得出:
\begin{equation}
    \begin{aligned}
        \hat{A} = \sum_i \frac{1}{i!} &\sum_{k_1, k_2, \ldots, k_i} \sum_{l_1, l_2, \ldots, l_i} \\
        &\mel{M^{(k_1)} M^{(k_2)} \cdots M^{(k_i)}}{\hat{A}^{(i)}}{M^{(l_1)} M^{(l_2)} \cdots M^{(l_i)}} \\
        &\hat{a}^\dagger (M^{(k_1)}) \hat{a}^\dagger (M^{(k_2)}) \cdots \hat{a}^\dagger (M^{(k_i)}) \hat{a} (M^{(l_i)}) \cdots \hat{a} (M^{(l_2)}) \hat{a} (M^{(l_1)}).
        \label{eq:general-form-of-fock-observable}
    \end{aligned}
\end{equation}
\eqref{eq:general-form-of-fock-observable}给出了福克空间上的可使用多粒子态基矢量对角化的可观察量的一般形式。
特别的,一个$n$粒子可观察量一定具有形式
\begin{equation}
    \begin{aligned}
        \hat{A} = &\frac{1}{n!} \sum_{k_1, k_2, \ldots, k_n} \sum_{l_1, l_2, \ldots, l_n} A_{k_1 k_2 \cdots k_n l_1 l_2 \cdots l_n} \\
        &\hat{a}^\dagger (M^{(k_1)}) \hat{a}^\dagger (M^{(k_2)}) \cdots \hat{a}^\dagger (M^{(k_n)}) \hat{a} (M^{(l_n)}) \cdots \hat{a} (M^{(l_2)}) \hat{a} (M^{(l_1)}),
    \end{aligned}
    \label{eq:n-particles-observable}
\end{equation}
其中系数$A_{k_1 k_2 \cdots k_n l_1 l_2 \cdots l_n}$在交换$k$和$l$以后应该和原来的值相差一个共轭变换,因为它是一个$n$粒子厄米算符的分量矩阵。%
\footnote{注意区分福克空间上的$n$粒子算符和$n$个单粒子态空间的张量积上的算符!}%
占据数算符是一种特殊的单粒子算符。

同样本节的论述可以双向地使用:给定一系列$\hat{A}^{(1)}, \hat{A}^{(2)}, \ldots$,总是可以使用\eqref{eq:general-form-of-fock-observable}构造出福克空间上的可观察量;反之,假定我们从一套产生湮灭算符和一个真空态构造出了一个福克空间,并且确认$\hat{A}$是这个福克空间上可用多粒子态基矢量对角化的可观察量,那么它一定可以展开成\eqref{eq:general-form-of-fock-observable}的形式。

以上讨论了可以使用\eqref{eq:sym-basis}或\eqref{eq:asym-basis}展开的可观察量。
还有一类可观察量不能使用多粒子态基矢量展开。在$n$粒子空间中不可能定义这样的可观察量,但在福克空间中可以,因此这一类可观察量不能和$n$个单粒子态空间的张量积上的算符建立类似于\eqref{eq:fock-observable}这样的联系。
% TODO:形式化的描述,总之就是同一项内产生算符的数目和湮灭算符不一样
不能使用多粒子态矢量展开的可观察量举例:
\[
    \hat{a}(M^{(i)}) + \hat{a}^\dagger(M^{(i)})
\]

\subsubsection{算符在乘积态基矢量下的展开}

% TODO:总之似乎是$\mel{12}{\hat{H}}{34}$对应着$\hat{a}_2 \hat{a}_1 \hat{a}_3^\dagger \hat{a}_4^\dagger$项,等等,但是实际上$\mel{512}{\hat{H}}{345}$也可以提取出同样的项。

\subsubsection{关于连续谱的注记}

以上讨论都建立在单粒子态可以使用离散谱描述的基础上。连续谱的情况基本上是一样的,只不过要将“对所有基的求和”改成积分,并且将$\delta_{ij}$改成$\delta(x - x')$。
需要注意的是可能会使用不同的积分测度,一般的,若有
\[
    \sum_i \longrightarrow \int \dd[n]{x} f(x),
\]
则有
\[
    \delta_{ij} \longrightarrow \frac{1}{f(x)} \delta^n(x - x').
\]
另外,由于连续谱情况下,福克空间中的多粒子态中两个粒子具有完全一样的状态的概率是零(点在连续谱中是零测的),所以可以忽略这种情况,而有
\begin{equation}
    \ket{\vb*{x}_1, \vb*{x}_2, \ldots, \vb*{x}_n} = \hat{a}^\dagger(\vb*{x}_1) \hat{a}^\dagger(\vb*{x}_2) \cdots \hat{a}^\dagger (\vb*{x}_n ) \ket{0}.
\end{equation}

在连续谱的情况中还有另一个问题需要注意。一般的,一个单粒子算符可以展开为
\[
    \mel{x}{\hat{A}}{\psi} = \mathcal{A} \braket{x}{\psi},
\]
如果算符$\hat{A}$是对角化的,那么$\mathcal{A}$就是一个数,否则它是一个作用在一个经典场上的算符。傅里叶变换可以证明即使$\hat{A}$不能对角化,$\hat{A}$的二次量子化形式也是
\[
    \hat{A}_\text{sq} = \int \dd{x} \hat{a}^\dagger (x) \mathcal{A} \hat{a}(x), 
\]
$\mathcal{A}$作用在湮灭算符$\hat{a}(x)$上。

\subsection{多体系统的动力学}\label{sec:many-body-dynamics}

% TODO:首先我们有二次量子化的哈密顿量,然后不同的守恒荷的值将相空间分成了不同的部分,每个部分内部的运动可以使用一个与二次量子化哈密顿量不同的哈密顿量描述。
% 需要做的则是讨论这两者的关系。
% 例如,考虑一个薛定谔场和静电场耦合的系统,无论系统中有多少粒子,关于场算符的哈密顿量也就是所谓的“二次量子化哈密顿量”都是完全一样的;另一方面,我们可以写出$n$粒子的这种系统的哈密顿量,这完全就是一个粒子数确定的单粒子量子力学的哈密顿量。前者是二次量子化哈密顿量,后者是一次量子化哈密顿量;对各守恒荷给定的态,一次量子化哈密顿量和二次量子化哈密顿量作用于其上得到完全一样的结果。
% 可以考虑将“多粒子态算符”分为一次量子化算符和二次量子化算符

% TODO:如果两组同类型粒子彼此没有相互作用,那么计算其中一群时可以忽略另一群,举例:
\[
    \frac{1}{\sqrt{6}} \left( \ket{1}\ket{2}\ket{3} + \ket{1}\ket{3}\ket{2} + \ket{2}\ket{1}\ket{3} + \ket{2}\ket{3}\ket{1} + \ket{3}\ket{1}\ket{2} + \ket{3}\ket{2}\ket{1}
    \right)
\]
如果1、2有相互作用而和3没有相互作用(注意如果12和3都和另一个系统有相互作用,那么它们之间还是有相互作用;判断两个系统之间有没有相互作用要把别的自由度都积掉只剩下这两个系统),那么在只考虑3时可以把$\ket{3}$略去,这样就剩下
\[
    \ket{1}\ket{2} + \ket{2}\ket{1}
\]
在计算只涉及1和2的振幅时,不管使用含有123的态还是只含有12的态都会得到一样的结果。

\autoref{sec:many-body-state}节描述了描述全同粒子系统态的基本框架,现在我们来让这些态动起来。
由于需要考虑粒子数可变的系统(到现在为止我们还从来没有讨论过粒子数怎么就会发生变化,或者说什么样的动力学能够让粒子数发生变化,但从福克空间的构造来看,至少粒子数发生变化在我们的理论框架内能够被恰当地描写),哈氏量中不应该出现任何单粒子算符。
由于产生湮灭算符足够描写福克空间,我们可以使用产生湮灭算符构造哈氏量(它本身应该是一个可观察量!)。
从而,只含有一种粒子的哈氏量的基本形式为
\begin{equation}
    \begin{split}
        \hat{H} = \sum_i c_i \hat{a}^\dagger (M^{(i)}) \hat{a} (M^{(i)}) + \underbrace{\sum_{m, n} S_{mn} \hat{a}^\dagger (M^{(m)}) \hat{a} (M^{(n)}) }_\text{one particle terms} + \\
        \underbrace{\frac{1}{2} \sum_{m, n, p, q} D_{mnpq} \hat{a}^\dagger (M^{(m)}) \hat{a}^\dagger (M^{(n)}) \hat{a} (M^{(q)}) \hat{a} (M^{(p)}) }_\text{two particles terms} + \cdots \\
        + \text{creation and annihilation terms},
    \end{split}
    \label{eq:one-kind-many-body-hamitonian}
\end{equation}
其中单粒子项、二粒子项等项均取\eqref{eq:general-form-of-fock-observable}的形式,产生湮灭项指的是不能使用多粒子基矢量\eqref{eq:sym-basis}或\eqref{eq:asym-basis}对角化的项,也就是不能和某个粒子数固定的希尔伯特空间上的可观察量建立类似于\eqref{eq:fock-observable}这样的对应的那些项,或者说每一项内产生算符的数目和湮灭算符不一样的项。我们称它们为产生湮灭项的原因是,系统粒子数守恒的充要条件是
\begin{equation}
    0 = \comm*{\hat{N}}{\hat{H}} = \sum_i \comm*{\hat{a}^\dagger (M^{(i)}) \hat{a} (M^{(i)})}{\hat{H}},
\end{equation}
而如果没有所谓的产生湮灭项,那么哈密顿量可以写成一系列形如
\[
    \hat{a}^\dagger (M^{(k_1)}) \hat{a}^\dagger (M^{(k_2)}) \cdots \hat{a}^\dagger (M^{(k_i)}) \hat{a} (M^{(l_i)}) \cdots \hat{a} (M^{(l_2)}) \hat{a} (M^{(l_1)})
\]
的算符的线性组合。无论是费米子还是玻色子,都有
\[
    \comm*{\hat{a}^\dagger (M^{(k_1)}) \hat{a}^\dagger (M^{(k_2)}) \cdots \hat{a}^\dagger (M^{(k_i)}) \hat{a} (M^{(l_i)}) \cdots \hat{a} (M^{(l_2)}) \hat{a} (M^{(l_1)})}{\hat{N}} = 0,
\]
因此在没有产生湮灭项时粒子数一定是守恒的。
没有产生湮灭项等价于粒子数守恒;事实上,这还等价于系统具有$U(1)$对称性。这是因为,对系统做$U(1)$变换就是做变换
\[
    \hat{a}^\dagger(M^{(k)}) \longrightarrow \ee^{\ii \alpha} \hat{a}^\dagger(M^{(k)}), \quad \hat{a}(M^{(k)}) \longrightarrow \ee^{-\ii \alpha} \hat{a}(M^{(k)}),
\]
哈密顿量中没有产生湮灭项意味着哈密顿量中的每一项都含有相同数目的产生算符和湮灭算符,所以诸$\ee^{\ii \alpha}$因子和$\ee^{-\ii \alpha}$因子相互抵消了,于是,哈密顿量在$U(1)$变换下保持不变。
我们可以看到粒子数正是$U(1)$对称性对应的守恒荷。

当然,哈氏量中还可以出现导数,但是通过对相应的算符做傅里叶变换总是可以把所有导数都变成系数,从而得到\eqref{eq:one-kind-many-body-hamitonian}型的哈密顿量。

在哈密顿量仅仅含有\eqref{eq:one-kind-many-body-hamitonian}中第一项的时候,我们有
\[
    \dv{t} \hat{a}(M^{(i)}) = - \ii c_i \hat{a} (M^{(i)}),
\]
这不论是在$F_S$还是$F_A$上都是成立的。因此就有
\[
    \hat{a} (M^{(i)}) = \ee^{- \ii t} \hat{a}_0 (M^{(i)}).
\]
因此这样的哈密顿量描述了一个没有相互作用的体系。我们将会看到,这样的体系实际上对应着一个自由场。
在这样的体系中,也就只有在这样的体系中,场做简谐运动是被允许的。
当然,自由场中粒子数是守恒的。
% TODO:费米场没有弄出来

含有多种粒子的哈密顿量的形式则为
\begin{equation}
    \hat{H} = \sum \hat{H}_\text{single} + \hat{H}_\text{interaction},
\end{equation}
其中每一个$\hat{H}_\text{single}$均取\eqref{eq:one-kind-many-body-hamitonian}的形式,而相互作用部分则取关于不同的粒子的\eqref{eq:general-form-of-fock-observable}的乘积。
显然,要出现粒子数的变化,相互作用部分同样应该出现产生算符的数目和湮灭算符不一样的项,如
\[
    \hat{a} \hat{b} + \hat{b}^\dagger \hat{a}^\dagger
\]

在有相互作用时,实际上并不存在真正的单粒子态:粒子之间会发生相互作用,会出现粒子的产生和湮灭,因此并无稳定存在、可以追踪轨迹的单个粒子。
有时我们会发现,对已有的产生湮灭算符做一个线性组合可以得到另一种产生湮灭算符,它们能够大大简化哈密顿量的形式,这种凭空构造出来的产生湮灭算符描述的多粒子态称为\textbf{准粒子}或者\textbf{元激发},它们是粒子的运动模式。
类似的,准粒子之间也会有相互作用,会有准粒子的产生湮灭,等等。
通常的习惯是,将玻色型的准粒子称为元激发,而将费米型的准粒子称为准粒子。

准粒子可以分成\textbf{个别激发}和\textbf{集体激发},前者的准粒子产生湮灭算符是单个实际粒子的产生湮灭算符做线性变换之后的结果,后者的准粒子产生湮灭算符则是多个实际粒子的产生湮灭算符的线性组合。
个别激发的典型例子包括\textbf{空穴},即做变换
\[
    \hat{b}^\dagger(M^{(i)}) = \hat{a}(M^{(i)}),
\]
以实际粒子的不存在为准粒子。
另一个个别激发的例子是\textbf{裹粒子},即某个实际粒子(相应地称为\textbf{裸粒子})射入已有的一个体系,从而该实际粒子一边发生运动,体系内原有的粒子跟着它运动而形成的模式。
集体激发的例子包括系统的CSCO的一部分可以看成一个场(无论坐标是连续的还是离散的),从而可以通过一个线性变换得到这个场对应的产生湮灭算符;如果系统的一部分能量本征态可以使用一套产生算符从一个“基态”(未必是定义系统时取的真空态)构造出来,那么我们也获得了一种集体激发。

还有一些激发只涉及少数几个粒子。例如如果一种粒子总是成对出现或者消失,那么可以认为$\hat{a}^\dagger_1 \hat{a}^\dagger_2$是一种准粒子。
可以递归地证明,这种复合粒子的自旋(就是组成它的各个粒子的自旋之和)如果是半整数,则它是费米子,否则是玻色子。
由于轨道角动量的$j$一定是整数(空间旋转群是角动量代数的矢量表示,不是旋量表示,因此$j$不可能是半整数),我们也可以讨论复合粒子的角动量代数,$j$为偶数时为玻色子,为奇数时是费米子。

对准粒子而言的真空态和实际粒子的真空态通常是不一样的。实际上,虽然我们使用了准粒子和实际粒子的区分,它们仅有的区别是真空态不相同,我们并不能够保证现在所认为的实际粒子——比如,电子、夸克等——不是某些更为基本的粒子形成的准粒子。

以上讨论了哈密顿量。关于二次量子化系统的绘景还需要说一句:在哈密顿量不显含时间时,海森堡绘景中的产生湮灭算符产生出的二次量子化基矢量以薛定谔绘景的方式演化。

\subsection{从量子场论到多粒子体系}\label{sec:from-qft-to-many-body}

\subsubsection{从自由场算符导出产生湮灭算符}\label{sec:c-a-operator-from-field}

以上我们讨论了怎样从单粒子态得到全同多粒子态,以及全同多粒子态可以使用一个真空态和一组产生湮灭算符产生。
现在讨论一个反向的问题:我们需要的这种产生湮灭算符从何处来?
马上可以看到,我们需要的这种产生湮灭算符实际上就是算符场的一个线性组合。
我们并不在意场算符$\hat{\phi}$是矢量、旋量还是就是标量,因为不管怎么样,$\hat{\phi}^\dagger \hat{\phi}$肯定是标量,因此粒子数算符也是标量,从而不会产生任何矛盾。
% TODO 但是下面的讨论明显把场算符当成了标量

以下我们讨论自由场。这无损一般性,因为相互作用绘景意味着我们能够将带相互作用的问题的态空间和自由场的态空间使用某个幺正变换(也就是相互作用绘景的时间演化算符)相互联系,而只要从自由场场算符构造出一套产生湮灭算符,与场的真空态——也就是所有场都为零的一个唯一的态——我们就完全刻画了自由场的态空间。
\autoref{sec:canonical-quantization}中提到了两种量子化方案:其一是对易方案\eqref{eq:symmetry-commutator},其二是反对易方案\eqref{eq:antisymmetry-commutator}。
对前者,在某个固定的时间点$t$我们设
%
\footnote{具体系数是多少无关紧要,系数只是为了让产生湮灭算符的对易子给出正确的系数。}
\[
    \hat{\phi}^i(\vb*{x}, t) = \frac{1}{\sqrt{2}} \left( \hat{a}_i^\dagger (\vb*{x}, t) + \hat{b}_i(\vb*{x}, t) \right), \quad 
    \hat{\pi}_i(\vb*{x}, t) = \frac{\ii}{\sqrt{2}} \left( \hat{a}_i^\dagger (\vb*{x}, t) - \hat{b}_i (\vb*{x}, t) \right)
\]
考虑到$\hat{\phi}$和$\hat{\pi}$的厄米性,有
\[
    \hat{a}_i^\dagger = \hat{b}^i,
\]
于是
\begin{equation}
    \hat{\phi}^i(\vb*{x}, t) = \frac{1}{\sqrt{2}} \left( \hat{a}_i^\dagger (\vb*{x}, t) + \hat{a}_i(\vb*{x}, t) \right), \quad 
    \hat{\pi}_i(\vb*{x}, t) = \frac{\ii}{\sqrt{2}} \left( \hat{a}_i^\dagger (\vb*{x}, t) - \hat{a}_i (\vb*{x}, t) \right).
\end{equation}
切换到自然单位制下,取$\hbar=1$,则通过\eqref{eq:symmetry-commutator}计算得到
\begin{equation}
    \comm*{\hat{a}_i(\vb*{x}, t)}{\hat{a}_j^\dagger (\vb*{y}, t)} = \delta_{ij} \delta (\vb*{x} - \vb*{y}), \quad \comm*{\hat{a}_i(\vb*{x}, t)}{\hat{a}_j(\vb*{y}, t)} = \comm*{\hat{a}_i^\dagger (\vb*{x}, t)}{\hat{a}_j^\dagger (\vb*{y}, t)} = 0.
    \label{eq:ca-op-from-field}
\end{equation}
因此对一个固定的时间$t$,$\hat{a}_i^\dagger(\vb*{x}, t)$和$\hat{a}_i (\vb*{x}, t)$构成了一组产生湮灭算符,由它们产生的单粒子态可以完全被$\vb*{x}$和$i$标记。
从而,由它们产生的单粒子态可以分解成两个空间$\{\ket{\vb*{x}}\}_{\vb*{x}}$和$\{\ket{i}\}_i$的直积;
单粒子态的CSCO为位置算符$\hat{\vb*{x}}$(当然也可以是与之能够相互导出的另一些算符,比如动量),还有与$\{\ket{i}\}_i$有关的算符。
我们称前者为\textbf{空间自由度},而称后者为\textbf{内禀自由度},因为后者来自于场算符的内部结构(有几个分量、在$\Lambda$作用下如何变换,等等)。%
\footnote{如果场算符的取值有某些限制(例如,电磁波一定是横波,等等),那么不能够保证$i$的可能取值的数目一定是场算符的分量数。}
总之,量子化方案\eqref{eq:symmetry-commutator}给出的场算符导致一个对称的福克空间$F_S$。我们称这样的福克空间描述的粒子为\textbf{玻色子}。
如果场满足特殊的时空对称性——比如我们之后会讨论的洛伦兹对称性——那么内禀自由度实际上由这种时空对称性中不改变单粒子态的$\vb*{x}$(或者也可以是不改变$\vb*{p}$)的群(称为\textbf{小群})确定,这个群的李代数在单粒子态上的表示的Cartan子代数就是内禀自由度空间的CSCO。
当然,也可以通过讨论场的内禀自由度来分析单粒子态的内禀自由度。

% TODO:实际上,内禀自由度还可以用来表示复合粒子。我们将两个场$\psi, \phi$写成一个列向量$\Psi = \pmqty{\psi & \phi}^\top$,它是洛伦兹群的一个可约表示;$\Psi(x)$可以分解成产生算符和湮灭算符之和,它就描述了由$\psi$场对应的粒子和$\phi$场对应的粒子复合而成的粒子。如果哈密顿量可以近似地用$\Psi$表示且不涉及其内部自由度(这又涉及到重整化的概念:$\Psi$的内部自由度仅和较高能的过程耦合,在低能下可以被略去),那么就不必考虑复合粒子的内部结构。

对于量子化方案\eqref{eq:antisymmetry-commutator},我们不能够使用\eqref{eq:ca-op-from-field},因为这不能导出正确的对易关系。
实际上由于使用这种量子化方案的场一定是复的,相当于两个厄米场,我们需要引入两对产生湮灭算符。

费米子和玻色子的概念的导出完全没有用到任何关于洛伦兹对称性的知识——它们只是两种量子化方案\eqref{eq:symmetry-commutator}和\eqref{eq:antisymmetry-commutator}的必然推论罢了。
因此,即使是在非相对论的语境下(例如我们把晶体中的位移看成某种场),只要能够有\eqref{eq:symmetry-commutator}或\eqref{eq:antisymmetry-commutator}这样的关系,就总是能够得到对应的产生湮灭算符,从而得到多粒子态。
值得注意的是,在讨论场的动力学或者粒子的动力学时,我们都是首先获得了一组能够完全描述态空间的可观察量,然后用它们标记不同的态;而在二次量子化中,我们是首先通过带标记$\vb*{x}$或$i$等的产生湮灭算符得到了一组多粒子态,并自然地使用这些量标记多粒子态,然后再讨论关于单粒子、二粒子……的可观察量。

之前我们提到过,场表示自然诱导出时空对称性在这个算符场作用的希尔伯特空间上的表示。
场算符诱导出的产生湮灭算符刻画了一个由单粒子态空间、二粒子态空间……直和而成的福克空间,这表明,算符场作用的希尔伯特空间实际上是一个可约表示。
它取某个粒子数的子空间则是一个不可约表示。由于$n$粒子态空间就是$n$个单粒子态空间的直积的对称化或者反对称化,我们只需要讨论单粒子态即可。于是现在我们把注意力转移到单粒子态上。
% TODO:场表示的标签,比如$P_\mu P^\mu$给出的$m^2$,无论是在场表示还是其$n$粒子态表示中,都是一样的。天晓得这是不是真的。。。
完全刻画空间自由度只需要一个位置算符就可以了,或者按照\autoref{sec:single-particle-quantity}中的做法,使用动量算符。

% TODO:哈密顿量的全体本征态就是单粒子态、二粒子态、三粒子态……

\subsubsection{场算符作为波函数}

\[
    \ket{\vb*{x}} = \hat{\phi}(\vb*{x}, t) \ket{0}
\]
然后我们会发现关于$\hat{\phi}$的海森堡绘景方程实际上就是关于$\ket{\vb*{x}}$的薛定谔绘景方程

\section{对称群和它们的表示}\label{sec:symmetry}

对称性可以大致分为两类:一类来自时空坐标变换,可能是平移,也可能是旋转或者推动,
此时$\var{x}$不为零,且常常$\bar{\var}{\phi}$也不为零(注意即使平移时$\bar{\var}{\phi}$也不为零,这是为了保持$\var{\phi}$始终为零);
另一类为\textbf{内禀对称性},它指的是$\var{x}$为零而$\bar{\var}{\phi}$不为零的变换。

\subsection{平移}\label{sec:translation}

平移可能是我们能想到的最简单的变换,但实际上它相当特殊——在实际的物理问题中平移群通常不使用矩阵群表示,因为它无疑不是线性的。%
\footnote{可以在仿射空间中使用矩阵表示平移,但是这对本文没有太大意义。}

平移群
% TODO:有必要分析单粒子希尔伯特空间上的平移群的作用和场的态空间上的平移群的作用。两者乍一看完全是风马牛不相及啊

生成元$P^\mu$

\begin{equation}
    \comm*{P_\mu}{P_\nu} = 0.
    \label{eq:comm-of-trans}
\end{equation}

接下来讨论平移群的线性表示。非Cartan元素的缺乏意味着平移群的有限维表示中不能够构造其Cartan元素(也就是全体平移生成元)的升降算符,因此平移群的有限维表示一定是平凡的。
因此转而观察其无限维表示。按照\eqref{eq:infinite-dim-rep},我们考虑
\[
    \phi'(x) = \phi(x - a) = \phi(x) - a^\mu \partial_\mu \phi,
\]
其中$a$是一个小量。于是,平移群的李代数的无穷维表示形如
\[
    P_\mu \propto - \partial_\mu.
\]
不失一般性地我们取幺正表示,并适当选取群参数,那么就有
\begin{equation}
    P_0 = \ii \partial_0, \quad P_i = - \ii \partial_i.
    \label{eq:transition-inf-rep}
\end{equation}
我们特意让时间平移变换的方向和空间平移的方向反过来了。
这是为了和物理中通常的时间演化方程\eqref{eq:quantum-evolution}的形式匹配。

\subsection{旋转}\label{sec:rotation}

\subsubsection{三维欧氏空间的旋转矩阵}

本节讨论$\reals^3$中的旋转。所谓旋转,指的是一个可微的(从而可以通过一个物理上的微分方程实现的)等距同构变换。
显然,$\reals^3$中的等距同构变换的全体就是$O(3)$,其中任何一个矩阵的行列式都是$\pm 1$。
另一方面,变换可微意味着,可以找到形如下式的无穷小变换:
\[
    \vb*{r} \longrightarrow \vb*{r}' = \vb*{r} + \dd{\vb*{r}}, \quad \dd{\vb*{r}} = \dd{t} \vb*{A} \cdot \vb*{r},
\]
从而
\[
    \dv{\vb*{r}'}{\vb*{r}} = \vb*{I} + \dd{t} \vb*{A},
\]
\[
    \det \left( \dv{\vb*{r}'}{\vb*{r}} \right) = 1 + \dd{t} \trace \vb*{A} \approx 1.
\]
因此$\reals^3$中的旋转变换的全体就是$SO(3)$,也就是行列式为1的全体$3\times 3$矩阵。

我们来分析$SO(3)$的结构。矩阵$A$在$SO(3)$中,当且仅当
\begin{equation}
    A A^\top = I, \quad \det A = 1.
    \label{eq:def-so3}
\end{equation}
矩阵$A$含有9个分量;$A A^\top = I$是对称的,因此它等价于6个独立的纯数量方程;
$\det A = \pm 1$可以直接从$A A^\top=I$推出,则$\det A = 1$的要求仅仅是去掉了其中的一支,因此对维数没有影响。
这样,$SO(3)$就是$9-6=3$维的,因此它有三个彼此独立的生成元。
注意到绕$x$轴旋转、绕$y$轴旋转、绕$z$轴旋转都是$SO(3)$的子群,这三者的表达式分别为
\begin{equation}
    R_x = \pmqty{1 & 0 & 0 \\ 0 & \cos \theta & -\sin \theta \\ 0 & \sin \theta & \cos \theta}, 
    R_y = \pmqty{\cos \theta & 0 & \sin \theta \\ 0 & 1 & 0 \\ - \sin \theta & 0 & \cos \theta}, 
    R_z = \pmqty{\cos \theta & - \sin \theta & 0 \\ \sin \theta & \cos \theta & 0 \\ 0 & 0 & 1}.
    \label{eq:rotation-with-axis}
\end{equation}
\eqref{eq:rotation-with-axis}自然导出三个生成元:
\begin{equation}
    J_1 = \ii \pmqty{0 & 0 & 0 \\ 0 & 0 & -1 \\ 0 & 1 & 0}, 
    J_2 = \ii \pmqty{0 & 0 & 1 \\ 0 & 0 & 0 \\ -1 & 0 & 0}, \\
    J_3 = \ii \pmqty{0 & -1 & 0 \\ 1 & 0 & 0 \\ 0 & 0 & 0}.
    \label{eq:generators-of-so3}
\end{equation}
注意到各个生成元都是厄米的,因为$SO(3)$在此处的矩阵表示是幺正的。
\eqref{eq:generators-of-so3}也可以直接通过分析$SO(3)$的抽象性质得到。
对无穷小变换$A = I + \theta J$($\theta$是小量),\eqref{eq:def-so3}中的两个方程分别代表
\[
    (I+\theta J) (I + \theta J^\top) = I, \quad \det (1 + \theta J) = 1 + \theta \trace J = 1,
\]
也就是
\[
    J + J^\top = 0, \quad \trace J = 0.
\]
上式描述了一个三阶方阵构成的向量空间,容易看出\eqref{eq:generators-of-so3}正是它的一组基。
通过显式表达式\eqref{eq:generators-of-so3}可以导出
\begin{equation}
    [J_i, J_j] = \ii \epsilon_{ijk} J_k.
    \label{eq:lie-algebra-so3}
\end{equation}
% TODO:这里好像正负号有问题?
这正是\eqref{eq:structure-of-lie-algebra}形式的公式,从而我们已经刻画了$SO(3)$的李代数$\mathfrak{so}(3)$的结构。

\subsubsection{$SU(2)$群}

然而,能够实施空间旋转的并不只有$SO(3)$。实际上,$SU(2)$也能做到这件事。
下面我们引入$SU(2)$群。$SU(2)$是由二阶复方阵组成的矩阵群,由
\begin{equation}
    A^\dagger A = I, \quad \det A = 1
    \label{eq:def-su2}
\end{equation}
定义。
二阶复数方阵一共有$4 \times 2 = 8$个自由度。矩阵方程$A^\dagger A = I$含有4个复数方程,
但是因为它是厄米的,因此它只含有2个独立的复数方程,从而它含有4个独立的实数方程,也即它将总自由度降到了4。
从$A^\dagger A = I$只能够推出$\abs{\det A} = 1$,$\det A$仍然可以连续变化;
$\det A = 1$则将$\det A$确定到一个点上面,因此它将总自由度降到了3。
因此$SU(2)$的维度为3。
可以证明,$SU(2)$中的每一个元素均形如
\begin{equation}
    A = a \mathbf{1} + b \mathbf{i} + c \mathbf{j} + d \mathbf{k}, \quad a^2 + b^2 + c^2 + d^2 = 1,
    \label{eq:su2-expression}
\end{equation}
其中
\begin{equation}
    \mathbf{1} = \pmqty{1 & 0 \\ 0 & 1}, \; \mathbf{i} = \pmqty{0 & -1 \\ 1 & 0}, \; 
    \mathbf{j} = \pmqty{0 & \ii \\ \ii & 0}, \; \mathbf{k} = \pmqty{\ii & 0 \\ 0 & -\ii},
    \label{eq:quad-basis}
\end{equation}
它们实际上就是四元数。
可以直接使用以上两式计算出$SU(2)$的生成元,但是这样比较繁琐。
使用$SU(2)$的一般定义\eqref{eq:def-su2},套用到无穷小变换
\[
    A = I + \ii \epsilon \sigma
\]
上,可以发现$\sigma$是幺正、无迹的矩阵;幺正、无迹的矩阵的一组基(通常称为\textbf{泡利矩阵})为
\begin{equation}
    \sigma_1 = \pmqty{0 & 1 \\ 1 & 0}, \; \sigma_2 = \pmqty{0 & -\ii \\ \ii & 0}, \; \sigma_3 = \pmqty{1 & 0 \\ 0 & -1}.
    \label{eq:sigma-matrix}
\end{equation}
它们也是厄米的。这正好是三个线性独立的矩阵,从而它们就是$SU(2)$的一组彼此独立的生成元。
我们有
\begin{equation}
    \comm{\frac{\sigma_i}{2}}{\frac{\sigma_j}{2}} = \ii \epsilon_{ijk} \frac{\sigma_k}{2},
\end{equation}
这表明$SU(2)$和$SO(3)$的李代数是一致的。

然而,$SU(2)$和$SO(3)$是两个不同的群。%
\footnote{当我们说两个群相同时我们是指它们在群论中同构,也就是说,我们所谓的“群”指的是抽象的群结构而不是具体的矩阵集合。
如果$SU(2)$和$SO(3)$中的元素可以一一对应,且在这个对应下相应的乘法关系不变,
那么这两个群——虽然使用了不同阶数的矩阵来表述——还是相同的。
然而正如我们马上要看到的那样,$SO(3)$中的一个元素可以和$SU(2)$中的两个元素相对应,且这种对应保持相应的乘法关系不变。
因此这两个群不同。}%
要看出这是为什么,考虑$SU(2)$在由\eqref{eq:quad-basis}的所有线性组合形成的空间上的表示。
记
\[
    q = a \mathbf{1} + b \mathbf{i} + c \mathbf{j} + d \mathbf{k}, \quad a^2 + b^2 + c^2 + d^2 = 1,
\]
显然它是$SU(2)$在二维复数空间上的矩阵表示。
可以验证
\[
    q^{-1} = a \mathbf{1} - b \mathbf{i} - c \mathbf{j} - d \mathbf{k},
\]
则$SU(2)$在\eqref{eq:quad-basis}的所有线性组合形成的空间上的作用可写成
\begin{equation}
    x \longrightarrow q x q^{-1}.
    \label{eq:su2-rotation}
\end{equation}
注意我们把$x$当成了算符来看待。
$\reals^3$中的向量$\vb*{v}$与
\begin{equation}
    x = v_1 \mathbf{i} + v_2 \mathbf{j} + v_3 \mathbf{k}
    \label{eq:trans-vec-quad}
\end{equation}
一一对应,且容易证明
\[
    \det x = \abs{\vb*{v}}.
\]
我们注意到
\[
    \det (q x q^{-1}) = (\det q) (\det x) (\det q^{-1}) = \det x,
\]
因此\eqref{eq:su2-rotation}是等距同构。由于$SU(2)$是李群,其表示\eqref{eq:su2-rotation}也必然是可微的。
这表明变换(这是将\eqref{eq:su2-rotation}和\eqref{eq:trans-vec-quad}写在一起的结果)
\[
    \vb*{v} \longrightarrow x \longrightarrow x' = q x q^{-1} \longrightarrow \vb*{v}'
\]
给出了$SO(3)$中全部的成员,而且也仅仅给出这么多成员。
然而,同一个变换\eqref{eq:su2-rotation}实际上对应着两个$q$。
% TODO:证明,不过多半鸽了
这表明$SU(2)$实际上是$SO(3)$的双覆盖。
\eqref{eq:su2-expression}意味着$SU(2)$实际上就是四维球面$S^4$,因此它是单连通群,
因此它是李代数$\mathfrak{so}(3)$——也就是$\mathfrak{su}(2)$——的覆盖群。
所有以$\mathfrak{so}(3)$为李代数的李群中$SU(2)$是最大的。

顺带提一句:以上的推导也表明,一个群在一个特定空间上的表示有时并不能完整地展现这个群的结构。
$\mathfrak{su}(2)$在$\reals^3$上的表示,也就是$\mathfrak{so}(3)$在$\reals^3$上的表示,就是\eqref{eq:generators-of-so3},
把它放进指数映射\eqref{eq:lie-group-element}中得到的就是$SO(3)$的三阶方阵形式。
因此$SU(2)$和$SO(3)$在$\reals^3$上的表示完全一样。
换而言之,$SU(2)$在$\reals^3$上的表示不是忠实的。
我们需要\eqref{eq:quad-basis}这样更大的空间才能完全展示$SU(2)$的结构。%
\footnote{严格来说本节中我们使用了两种$SU(2)$的表示。
其一是$q$,也就是$\complexes^2$上的表示,其二是\eqref{eq:su2-rotation},也就是$GL(\complexes^2)$上的表示。
后者和$SO(3)$在$GL(\complexes^2)$上的表示完全一样,但后者中的每一个都对应两个$q$,
两者间的对应就是\eqref{eq:group-action-on-operators},它并非一一对应。}

\subsubsection{有限维不可约表示}

本节分析$SU(2)$的有限维不可约表示。注意到,$J_1, J_2, J_3$互不对易。因此$J_3$就是$\mathfrak{su}(2)$的一个Cartan子代数。
使用本征值标记这些本征矢为$\ket*{J_3^{(1)}}$, $\ket*{J_3^{(2)}}$, \dots。由于我们仅仅讨论有限维不可约表示,
本征值序列一定有上下限。记最大的本征值为$j$。
现在寻找升降算符。求解\eqref{eq:possible-c}得到$c = \pm 1$,$c=1$对应着$\lambda^2 = \ii \lambda^1$,$\lambda^3 = 0$;$c=-1$对应着$\lambda^2 = - \ii \lambda^1$。
这表明有限维表示中仅有的升降算符为
\[
    J_+ \propto \lambda^1 J_1 + \ii \lambda^1 J_2, \quad J- \propto \lambda^1 J_1 - \ii \lambda^1 J_2. 
\]
习惯上我们取
\begin{equation}
    J_+ = \frac{1}{\sqrt{2}} (J_1 + \ii J_2), \quad J_- = \frac{1}{\sqrt{2}} (J_1 - \ii J_2).
\end{equation}
$J_+$让本征值加一,$J_-$让本征值减一。两个算符采取同样的系数是为了让$J_1,J_2,J_3$厄米时,$J_+$和$J_-$互为共轭转置。
容易得到下面的对易关系:
\begin{equation}
    \comm*{J_3}{J_{\pm}} = \pm J_3, \quad \comm*{J_+}{J_-} = J_3.
\end{equation}

从升降算符的定义可以得到
\[
    J_+ \ket{k} = \alpha_k \ket{k+1}, \quad J_- \ket{k+1} = \alpha_k^* \ket{k},
\]
由于$\ket{j}$是本征值最大的本征态,
\[
    J_+ \ket{j} = 0,
\]
从而
\[
    \bra{j} J_- = 0.
\]
我们有
\[
    J_- \ket{j} = \alpha_{j-1}^* \ket{j-1},
\]
两边求模长,
\[
    \begin{aligned}
        \abs{\alpha_{j-1}}^2 &= \abs{J_- \ket{j}}^2 = \mel{j}{J_+ J_-}{j} \\
        &= \mel{j}{J_+ J_-}{j} - \mel{j}{J_- J_+}{j} \\
        &= \mel{j}{[J_+, J_-]}{j} \\
        &= \mel{j}{J_3}{j} = \mel{j}{j}{j} = j.
    \end{aligned}
\]
没有其它条件能够确定$\alpha_{j-1}$具体应该取什么值。这是因为仅仅靠对易关系并不能够确定$J_1$和$J_2$(从而$J_+$和$J_-$)作用在$\ket{k}$上面会得到什么样的结果。
然而,注意到只要是同维度的非奇异有限维表示之间可以通过相似变换相互转换,%TODO真的吗
不失一般性地我们可以认为所有的$\alpha$都是正实数。于是
\[
    \alpha_{j-1} = \sqrt{j}.
\]
另一方面,
\[
    \begin{aligned}
        \abs{\alpha_k}^2 &= \abs{J_- \ket{k+1}}^2 = \mel{k+1}{J_+ J_-}{k+1} \\
        &= \mel{k+1}{[J_+, J_-]}{k+1} + \mel{k+1}{J_- J_+}{k+1} \\
        &= \mel{k+1}{J_3}{k+1} + \abs{J_+ \ket{k+1}} \\
        &= k+1 + \abs{\alpha_{k+1}}^2,
    \end{aligned} 
\]
于是可以递推得到
\[
    \abs{a_k} = \frac{(j+k+1)(j-k)}{2},
\]
得到显式表达式
\begin{equation}
    J_+ \ket{k} = \sqrt{\frac{(j+k+1)(j-k)}{2}} \ket{k+1}, \quad J_- \ket{k} = \sqrt{\frac{(j+k)(j-k+1)}{2}} \ket{k-1}.
    \label{eq:ladder-operators-of-su2}
\end{equation}
由于是有限维表示,反复作用$J_-$在$\ket{j}$上最后一定会得到$0$。由\eqref{eq:ladder-operators-of-su2},得到零的唯一一种可能就是$k=-j$。这表明从$\ket{j}$出发不断作用$J_-$能够得到的全部非零本征向量为
\[
    \ket{j}, \; \ket{j-1}, \; , \ldots, \; \ket{-j+1}, \; \ket{-j},
\]
而由于$J_+$和$J_-$是仅有的升降算符,它们就是$J_3$仅有的本征向量。
这等价于$j$是半整数,且整个向量空间的维度为$2j+1$。
于是我们找到了$SU(2)$的所有不可约有限维表示。

作为最简单的两个例子:$j=0$时向量空间维数为1,所有李代数中的生成元都是0,而李群中的元素的表示为恒等运算;而在$j=1/2$时向量空间维度为2,相应的表示,使用$\ket{\frac{1}{2}}$和$\ket{-\frac{1}{2}}$为基底,就是$\sigma_1 / 2$,$\sigma_2 / 2$和$\sigma_3 / 2$。

舒尔引理说明,在这种有限维不可约表示中卡西米尔算符的表示一定是单位矩阵的某个倍数,因此可以使用这个倍数来标记有限维不可约表示。
就旋转群而言,
\begin{equation}
    J^2 = (J_1)^2 + (J_2)^2 + (J_3)^2
\end{equation}
足够起到这个作用了。容易验证这的确是一个卡西米尔算符,且
\begin{equation}
    J^2 \ket{k} = j (j+1) \ket{k}.
\end{equation}

$\mathfrak{su}(2)$的任何一个可约表示是若干不可约表示的直和。
我们使用$j$来标记各不可约表示,则$\mathfrak{su}(2)$的任何一个可约表示的基矢量均可以写成$\{\ket{jm}\}$,其中对于标记同一个矢量的$j$和$m$,有
\[
    m = -j, -j+1, \ldots, 0, \ldots, j,
\]
且$j$是半整数。

\subsubsection{无限维表示}

考虑\eqref{eq:fin-and-inf-rep},旋转群的李代数的无穷维表示为
\[
    (J_i)_\text{inf} = - ((J_i)_\text{fin} x) \cdot \grad.
\]
由于旋转群和时间维无关,梯度算符也可以看成是仅仅作用在空间维上的。
代入具体的值就得到
\begin{equation}
    J_1 = \ii (x^3 \partial_2 - x^2 \partial_3), \quad J_2 = \ii (x^1 \partial_3 - x^3 \partial_1), \quad
    J_3 = \ii (x^2 \partial_1 - x^1 \partial_2).
    \label{eq:rotation-inf-rep}
\end{equation}
很容易验证,以上三个算符确实满足李代数\eqref{eq:lie-algebra-so3}。

\subsection{洛伦兹群}

\subsubsection{四维闵可夫斯基时空中的洛伦兹矩阵}

首先讨论洛伦兹群在四维闵可夫斯基空间$\reals^{3, 1}$上的表示,也就是保持闵可夫斯基度规%
\footnote{当然,正如我们在欧氏空间的张量分析当中能够看到的那样,如果基矢量彼此不正交,那么度规就不能够写成对角形式。
这里我们实际上已经做了一个不失一般性的设定,要求度规一定是对角的。
这总是可以做到,因为不对角的度规可以通过一个合同变换(这个变换对应着一个坐标变换)变成对角的。}%
\begin{equation}
    \eta \equiv [\eta_{\mu \nu}]_{\mu \nu} = \diag (1, -1, -1, -1)
\end{equation}
不变的全体坐标变换矩阵$O(1,3)$。记这些矩阵中的一个为$\Lambda$,容易看出保持度规不变等价于
\begin{equation}
    \Lambda_\sigma^\mu \Lambda_\rho^\nu \eta_{\mu \nu} = \eta_{\sigma \rho},
\end{equation}
或者如果将$\Lambda$的矩阵形式看成是$[\Lambda^\mu_\nu]_{\mu \nu}$,%
\footnote{更加规范的写法是$\Lambda^\mu_{\ \nu}$,利用“第一个指标是行号、第二个指标是列号”的习惯。相应的也有$\Lambda_{\nu}^{\ \mu}$。
使用这种记号,
\[
    \Lambda_\sigma^\mu \Lambda_\rho^\nu \eta_{\mu \nu} = \Lambda_{\ \sigma}^\mu \eta_{\mu \nu} \Lambda_{\ \rho}^\nu = (\Lambda^T)_\sigma^{\ \mu} \eta_{\mu \nu} \Lambda_{\ \rho}^\nu = \Lambda^T \eta \Lambda.
\]
}%
那么就是
\begin{equation}
    \Lambda^\top \eta \Lambda = \eta.
    \label{eq:lorentz-matrix}
\end{equation}
从\eqref{eq:lorentz-matrix}可以看出
\begin{equation}
    \det \Lambda = \pm 1. 
    \label{eq:det-lorentz}
\end{equation}

\eqref{eq:lorentz-matrix}意味着
\[
    (\Lambda_0^0)^2 - (\Lambda_0^1)^2 - (\Lambda_0^2)^2 - (\Lambda_0^3)^2 = 1,
\]
从而
\begin{equation}
    \Lambda_0^0 = \pm \sqrt{1 + (\Lambda_0^1)^2 + (\Lambda_0^2)^2 + (\Lambda_0^3)^2}.
\end{equation}

现在我们将满足$\det \Lambda = 1 > 0$——也就是说,坐标系的手性不改变——以及$\Lambda_0^0 > 0$——也就是说,变换前的坐标时增加则变换后的坐标时也增加——的变换称为\textbf{正规洛伦兹群}。
容易验证这确实是一个群。记之为$SO(1,3)^\uparrow$。
正规洛伦兹群由于不改变时间维的指向,设$a^\mu$是一个四维矢量,则$\theta(a^0)$在正规洛伦兹群下不变,其中$\theta$为阶跃函数。
正规洛伦兹群中的成员称为\textbf{洛伦兹变换}。我们不认为接下来要谈到的含有宇称算符和时间反演算符的洛伦兹群成员为洛伦兹变换,因为它们无法通过可微的物理过程实现。

按照$\det \Lambda$和$\Lambda_0^0$的正负,可以将洛伦兹群分成四支。
其中两者皆为正的那一支就是$SO(1,3)^\uparrow$。
考虑矩阵
\begin{equation}
    \Lambda_P = \diag (1, -1, -1, -1), 
\end{equation}
容易看出,$\Lambda_P \Lambda$还是洛伦兹群的成员,并且
\[
    \det \Lambda = - \det (\Lambda_P \Lambda).
\]
它就是\textbf{宇称算符}。同样,\textbf{时间反演算符}
\begin{equation}
    \Lambda_T = \diag(-1, 1, 1, 1)
\end{equation}
也将一个洛伦兹变换转化为另一个洛伦兹变换,且
\[
    \Lambda_0^0 = - (\Lambda_T \Lambda)_0^0.
\]
由于这两个算符都是可逆的,且可以作用在任何洛伦兹群的成员上,实际上洛伦兹群的四支之间只相差一个宇称算符和/或一个时间反演算符,也就是
\begin{equation}
    O(1, 3) = \{ SO(1,3)^\uparrow, \Lambda_P SO(1,3)^\uparrow, \Lambda_T SO(1,3)^\uparrow, \Lambda_T \Lambda_P SO(1,3)^\uparrow \}.
    \label{eq:4-parts-of-o13}
\end{equation}
需注意除了$SO(1,3)^\uparrow$以外的部分只是陪集,并不能构成群,因为它们没有单位元。

\subsubsection{洛伦兹群的李代数}

洛伦兹群的四支之间不能通过可微的变换相互联系。因此,所谓洛伦兹群的李代数就是$SO(1,3)^\uparrow$的李代数。
洛伦兹群的定义\eqref{eq:lorentz-matrix}含有10个彼此独立的实数方程,因此留下6个自由度。
因此只需要寻找6个生成元就可以得到洛伦兹群的李代数。
由\eqref{eq:lorentz-matrix}可以得到无穷小生成元需要满足的关系为
\begin{equation}
    \eta K + K^\top \eta = 0.
    \label{eq:def-generators-of-lorentz}
\end{equation}
容易验证,设\eqref{eq:generators-of-so3}中的$J$为$J^\text{3dim}$,并定义
\begin{equation}
    J_i = \pmqty{\dmat{0 ,J^\text{3dim}_i}},
\end{equation}
则$J_i$,$i=1, 2, 3$满足\eqref{eq:def-generators-of-lorentz},这样我们就得到了洛伦兹群的三个生成元。
当然,$SO(3)$实际上是$SO(1,3)^\uparrow$的一部分,因此这是合理的。
通过考虑涉及$x^0$的矩阵,我们可以得到另外三个生成元:
\begin{equation}
    K_1 = \ii \pmqty{0 & 1 & 0 & 0 \\ 1 & 0 & 0 & 0 \\ 0 & 0 & 0 & 0 \\ 0 & 0 & 0 & 0}, \quad 
    K_2 = \ii \pmqty{0 & 0 & 1 & 0 \\ 0 & 0 & 0 & 0 \\ 1 & 0 & 0 & 0 \\ 0 & 0 & 0 & 0}, \quad
    K_3 = \ii \pmqty{0 & 0 & 0 & 1 \\ 0 & 0 & 0 & 0 \\ 0 & 0 & 0 & 0 \\ 1 & 0 & 0 & 0}.
    \label{eq:boost-generators}
\end{equation}
这些生成元对应的李群元素称为\textbf{推动},物理上它们涉及到时间,因此实际上是参考系变换。
每个矩阵前面都加上了$\ii$是因为使用了\eqref{eq:lie-group-element}的习惯,而我们现在讨论的洛伦兹群中的矩阵都是实数矩阵,因此$\ii \theta \sigma$必须是实数,而由于参数$\theta$是实数,生成元前面就应该多一个纯虚数。
请注意它们不是厄米的。这可以预期,因为洛伦兹群涉及推动的部分是无界的——这是闵可夫斯基时空的性质决定的。
容易验证,有下面的对易关系:
\begin{equation}
    \comm*{J_i}{J_j} = \ii \epsilon_{ijk} J_k, \quad \comm*{J_i}{K_j} = \ii \epsilon_{ijk} K_k, \quad \comm*{K_i}{K_j} = - \ii \epsilon_{ijk} J_k.
    \label{eq:lie-algebra-lorentz}
\end{equation}
这就得到了洛伦兹群的李代数。
\eqref{eq:lie-algebra-lorentz}中$J$之间的运算是封闭的,而$K$之间的运算不是封闭的,而且$J$和$K$之间不对易。
但如果定义
\begin{equation}
    N_i^\pm = \frac{1}{2} (J_i \pm \ii K_i),
    \label{eq:def-n-pm}
\end{equation}
就有
\begin{equation}
    \comm{N_i^+}{N_j^+} = \ii \epsilon_{ijk} N_k^+, \quad \comm{N_i^-}{N_j^-} = \ii \epsilon_{ijk} N_l^-, \quad \comm{N^+_i}{N^-_j} = 0.
    \label{eq:two-su2-algebra}
\end{equation}
这表明$SO(1,3)^\uparrow$的李代数是两个$\mathfrak{su}(2)$的直和,或者说是$\mathfrak{sl}(2, \complexes)$。
% 这个关系其实是需要说明的,因为从“两个SU2”的直和推不出$SL(2, \complexes)$
但可以证明,$SO(1,3)^\uparrow$并不是$SL(2, \complexes)$,事实上后者是前者的覆盖群,而且是双覆盖。

现在我们考虑洛伦兹群中的其它三支。事实上,由于\eqref{eq:4-parts-of-o13},讨论洛伦兹群的作用等价于讨论正规洛伦兹群在四维向量空间以及作用了宇称变换和/或时间反演变换的四维向量空间上的作用。%
\footnote{数学上说,完整的洛伦兹群$O(1,3)$是正规洛伦兹群$SO(1,3)^\uparrow$和$\{I, \Lambda_P, \Lambda_T\}$的半直积。$SO(1,3)^\uparrow$是一个正规子群,因此只需要知道$\Lambda_P$和$\Lambda_T$下$SO(1,3)^\uparrow$的变动即可完全刻画$O(1,3)$。}%
因此只需要讨论$SO(1,3)^\uparrow$在宇称变换和时间反演变换之下会怎么变化,从而只需要讨论生成元$J_i$和$K_i$在宇称变换和时间反演变换之下的变化。
宇称变换,写成分量矩阵的形式,是(取$\mu$为行指标)
\[
    (\Lambda_P)^\mu_\nu = \Lambda_P = \diag(1, -1, -1, -1).
\]
我们没有区分行指标和列指标,因为宇称变换无非是洛伦兹变换的一种,而洛伦兹变换的应用场景中不会出现需要区分行指标和列指标的情况。%
\footnote{洛伦兹变换可以看成一种坐标变换,因此可以把$\Lambda$看成指标变换符号的另一种写法:
\[
    \Lambda_\mu^\nu = \beta_{\mu}^{\nu'}.
\]
由于本文是从洛伦兹群的表示出发讨论问题而淡化“带坐标的流形”这一几何概念,有时对不同坐标系下同一矢量或旋量的各分量我们不使用$x^\mu, x^{\mu'}$这样明确区分坐标系的写法;相反,洛伦兹群的作用被认为是作用在\textbf{同一个}向量空间上的算符。
}%
若对一个四维矢量做宇称变换,那么作用在其上的算符——也就是$4\times 4$矩阵——会发生如下所示的变换:%
\footnote{你可能会疑惑为什么这个变换的形式看起来不是\eqref{eq:group-action-on-operators}。
实际上,如果我们使用带撇的记号,那么此变换形如
\[
    A^\mu_{\ \nu} \longrightarrow A^{\mu'}_{\ \nu'} = (\Lambda_P)^{\mu'}_\rho (\Lambda_P)^\sigma_{\nu'} \Lambda^\rho_\sigma.
\]
由于$\Lambda_P$实际上是坐标变换符号,分量矩阵$(\Lambda_P)^{\mu'}_\rho$和$(\Lambda_P)^\sigma_{\nu'}$的的确确互为逆矩阵。
因此\eqref{eq:group-action-on-operators}确实是成立的。
由于我们使用的度规是正规型,因此无需在意$\Lambda_P$、它的转置或者它的逆的区别——三者实际上是一样的。
}%
\[
    A^\mu_{\ \nu} \longrightarrow (A')^\mu_{\ \nu} = (\Lambda_P)^\mu_\rho (\Lambda_P)^\sigma_\nu A^\rho_{\ \sigma}.
\]
然后简单的计算就表明:
\begin{equation}
    J_i \stackrel{\Lambda_P}{\longrightarrow} J_i, \quad J_i \stackrel{\Lambda_T}{\longrightarrow} J_i, \quad K_i \stackrel{\Lambda_P}{\longrightarrow} - K_i, \quad K_i \stackrel{\Lambda_T}{\longrightarrow} - K_i.
    \label{eq:parity-and-time-reversion-transform}
\end{equation}

\subsubsection{$SO(1,3)^\uparrow$的魏尔旋量表示}\label{sec:weyl-spinor-representation}

为了方便起见,我们把洛伦兹群的表示以及它的双覆盖的表示统称为洛伦兹群的表示。
% TODO:这有没有考虑到$SO(1,3)^\uparrow$以外的部分?
我们先只讨论$SO(1,3)^\uparrow$的表示。完整的洛伦兹群的表示——也就是说考虑了宇称变换和时间反演变换——只需要在$SO(1,3)^\uparrow$的表示上额外增加宇称变换和时间反演变换的规则即可。
由于\eqref{eq:two-su2-algebra},$SO(1,3)^\uparrow$的有限维表示全部可以看成是两个$SU(2)$的有限维不可约表示的直积,或者若干个这样的直积的直和。
% TODO:直积是怎么来的
我们记$SO(1,3)^\uparrow$的有限维表示为$(j_1, j_2)$,$j_1$和$j_2$分别代表其中一个$SU(2)$的有限维表示的$j$(见\autoref{sec:rotation})。
通常用$j_1$表示$N^+_i$那部分李代数的$j$,$j_2$代表$N^-_i$那一部分李代数的$j$。
% TODO:标量、旋量、矢量

首先是$(0,0)$表示。这个表示作用在一个$1\times1 = 1$维向量空间上。由之前的讨论,$N^+_i$和$N^-_i$都是零,也就是说,$(0,0)$维表示是一个平凡的、只有恒等运算的表示。
这个向量空间当然就是\textbf{标量},这个表示称为\textbf{标量表示},
其中的对象是单分量的,在洛伦兹变换下不变。
可以证明只有在这种表示下任何对象都不变。也就是说,只有标量在洛伦兹变换下完全不变;多分量对象不可能规洛伦兹变换下完全不变。

接着是$(\frac{1}{2}, 0)$表示。这是一个二维表示,这个表示中,
\begin{equation}
    N^+_i = \frac{\sigma_i}{2}, \quad N^-_i = 0.
\end{equation}
使用\eqref{eq:def-n-pm}可以推导出
\begin{equation}
    J_i = \frac{1}{2} \sigma_i, \quad K_i = - \frac{\ii}{2} \sigma_i.
\end{equation}
于是使用指数映射就能够得到洛伦兹变换下这种二分量对象的变化方式,也就是
\begin{equation}
    R_\theta = \exp \left( \frac{1}{2} \ii \theta_i \sigma_i \right), \quad B_\phi = \exp \left( \frac{1}{2} \phi_i \sigma_i \right).
\end{equation}
值得注意的是,洛伦兹变换下各个分量混合起来的系数一般来说含有虚部。
类似的,$(0, \frac{1}{2})$表示也是一个二维表示,其中
\begin{equation}
    N^+_i = 0, \quad N^-_i = \frac{\sigma_i}{2},
\end{equation}
这又等价于
\begin{equation}
    J_i = \frac{1}{2} \sigma_i, \quad K_i = \frac{\ii}{2} \sigma_i.
\end{equation}
其变换方式为
\begin{equation}
    R_\theta = \exp \left( \frac{1}{2} \ii \theta_i \sigma_i \right), \quad B_\phi = \exp \left( - \frac{1}{2} \phi_i \sigma_i \right).
\end{equation}
我们称$(\frac{1}{2}, 0)$表示为\textbf{左手旋量},$(0, \frac{1}{2})$表示为\textbf{右手旋量}。
两者统称为\textbf{魏尔旋量}。容易看出,两种旋量在旋转下的变化相同,在推动下的变化差一个负号。

以下,我们仿照向量的指标升降、坐标变换等规则,定义一套旋量的指标升降、坐标变换规则,称为\textbf{范德瓦尔登符号}。
% TODO:是否任何一个洛伦兹变换都可以写成$R_\theta B_\phi$的形式?
首先定义\textbf{旋量度规}
\begin{equation}
    \epsilon = \pmqty{0 & 1 \\ -1 & 0},
\end{equation}
为什么叫做这个名字马上可以看到。很容易看出,
\begin{equation}
    (-\epsilon) \epsilon = I, \quad \epsilon \sigma_i^* (-\epsilon) = - \sigma_i. 
    \label{eq:attributes-of-epsilon}
\end{equation}
使用这两个关系式可以证明:若$\chi_L$是一个左手旋量,那么
\begin{equation}
    \chi_L^C = \epsilon\chi_L^*
    \label{eq:left-spinor-c}
\end{equation}
就是一个右手旋量;若$\chi_R$是一个右手旋量,则
\begin{equation}
    \chi_R^C = -\epsilon \chi_R^*
    \label{eq:right-spinor-c}
\end{equation}
就是一个左手旋量。
(方法是,将$R_\theta$或者$B_\phi$作用到$\chi_L$上得到$\chi_L'$,从而可以计算出$(\chi_L^C)'$,然后使用\eqref{eq:attributes-of-epsilon}凑出$(\chi_L^C)'$和$\chi_L$之间的关系)
在\eqref{eq:right-spinor-c}中我们特意加了一个负号,这样
\[
    (\chi_L^C)^C = \chi_L, \quad (\chi_R^C)^C = \chi_R.
\]
请注意$\epsilon$是可逆的,因此,\eqref{eq:left-spinor-c}和\eqref{eq:right-spinor-c}表明有一样多的左手旋量和右手旋量,它们通过\eqref{eq:left-spinor-c}和\eqref{eq:right-spinor-c}一一对应。
一对通过\eqref{eq:left-spinor-c}和\eqref{eq:right-spinor-c}相对应的左手旋量和右手旋量就可以看成一个抽象的魏尔旋量$\chi$分别在$(\frac{1}{2},0)$和$(0, \frac{1}{2})$中的表示。
我们使用$\chi_a$表示$\chi_L$的第$a$个分量,$\chi^{\dot{a}}$表示$\chi_R$的第$a$个分量,那么由于
\[
    \chi_R = \chi_L^C, \quad \chi_L = \chi_R^C,
\]
有
\[
    \chi^{\dot{a}} = \sum_b (\text{the $(a,b)$-element of $\epsilon$}) \cdot \chi_b^*, \quad \chi_a = \sum_b ( - \text{the $(a,b)$-element of $\epsilon$}) \cdot (\chi^{\dot{b}})^*.
\]
于是定义%
\footnote{到目前为止我们还没有赋予上下指标任何意义,所以我们可以任意地规定涉及它们的表达式。此处上下指标不表示逆变-协变,虽然最后的结果看起来和逆变-协变关系很像。}
\begin{equation}
    \epsilon^{ab} = \epsilon^{\dot{a} \dot{b}} = \pmqty{0 & 1 \\ -1 & 0}, \quad \epsilon_{ab} = \epsilon_{\dot{a} \dot{b}} = \pmqty{0 & -1 \\ 1 & 0},
\end{equation}
以及
\begin{equation}
    \chi^{\dot{a}} = (\chi^a)^*, \quad \chi_{\dot{a}} = (\chi_a)^*,
\end{equation}
我们得到了$\chi$在左右手旋量空间中的表示相互切换的公式
\begin{equation}
    \chi^a = \epsilon^{ab} \chi_b, \quad \chi^{\dot{a}} = \epsilon^{\dot{a} \dot{b}} \chi_{\dot{b}}, \quad \chi_a = \epsilon_{ab} \chi^b, \quad \chi_{\dot{a}} = \epsilon_{\dot{a} \dot{b}} \chi^{\dot{b}}.
\end{equation}
其中一上一下两个相同指标要求和。我们看到了$\epsilon$的地位正是矢量分析中度规的地位,因此称它为旋量度规。

使用指标升降的一般理论,我们发现,若
\begin{equation}
    \psi_b = A_b^{\ a} \chi_a, \quad \psi_{\dot{b}} = A_{\dot{b}}^{\ \dot{a}} \chi_{\dot{a}}, \quad \psi^b = A^b_{\ a} \chi^a, \quad \psi^{\dot{b}} = A^{\dot{b}}_{\ \dot{a}} \chi^{\dot{a}},
    \label{eq:linear-operator-on-spinor}
\end{equation}
则
\begin{equation}
    A^c_{\ d} = \epsilon^{cb} A_{b}^{\ a}\epsilon_{ad}, \quad A^{\dot{c}}_{\ \dot{d}} = (A^c_{\ d})^*, \quad A^{\dot{c}}_{\ \dot{d}} = \epsilon^{\dot{c} \dot{b}} A_{\dot{b}}^{\ \dot{a}} \epsilon_{\dot{a}\dot{d}}.
    \label{eq:left-right-matrix-transform}
\end{equation}
通常对作用在左手旋量上的矩阵$A$,规定$A_a^{\ b}$就是$A$,这样\eqref{eq:linear-operator-on-spinor}中的四个式子全部等价于
\[
    \psi_L = A \chi_L.
\]

得到了指标升降关系,再来看坐标变换关系。按照前述规定,
\[
    (\sigma_i)_b^{\ a} = \sigma_i,
\]
对左手旋量我们有
\[
    \chi'_a = \Lambda_a^{\ b} \chi_b = \exp \left( \frac{1}{2} \ii \theta_i \sigma_i + \frac{1}{2} \phi_i \sigma_i \right)_a^{\; b} \chi_b.
\]
与之对应的右手旋量会怎样变换?很容易想到,应该使用\eqref{eq:left-right-matrix-transform}来得到对应的作用在右手旋量上的变换矩阵。
但实际上右手旋量的变换方式在定义时就已经确定了(因为$(0, \frac{1}{2})$表示本来就是$SO(1,3)^\uparrow$的某种表示)。
我们要验证这两种变换方式是不是一致。
从右手旋量的定义出发我们有
\[
    {\chi'}^{\dot{a}} = \sum_b \text{the $(a,b)$-element of } \exp \left( \frac{1}{2} \ii \theta_i \sigma_i - \frac{1}{2} \phi_i \sigma_i \right) \cdot \chi^{\dot{b}} .
\]
而如果右手旋量的洛伦兹变换服从\eqref{eq:left-right-matrix-transform},那么就有
\[
    {\chi'}^{\dot{a}} = \Lambda^{\dot{a}}_{\ \dot{b}} \chi^{\dot{b}}.
\]
这两种变换方式是一致的,当且仅当
\[
    \text{the $(a,b)$-element of } \exp \left( \frac{1}{2} \ii \theta_i \sigma_i - \frac{1}{2} \phi_i \sigma_i \right) = \sum_{c,d} \epsilon^{\dot{a} \dot{c}} \left(\exp \left( \frac{1}{2} \ii \theta_i \sigma_i + \frac{1}{2} \phi_i \sigma_i \right)_c^{\; d}\right)^* \epsilon_{\dot{d} \dot{b}}
\]
使用$\eqref{eq:attributes-of-epsilon}$很容易证明这确实是对的。
类似的,可以表明$\chi_{\dot{a}}$和$\chi^a$的变换矩阵正是$\Lambda_a^{\ b}$通过\eqref{eq:left-right-matrix-transform}变换得到的。
因此我们就得到了旋量的洛伦兹变换:
\begin{equation}
    \begin{bigcase}
        \Lambda_a^{\ b} = \exp \left( \frac{1}{2} \ii \theta_i \sigma_i + \frac{1}{2} \phi_i \sigma_i \right)_a^{\ b}, \quad \Lambda^{\dot{a}}_{\ \dot{b}} = \exp \left( \frac{1}{2} \ii \theta_i \sigma_i - \frac{1}{2} \phi_i \sigma_i \right)_a^{\ b}, \\
        \Lambda_{\dot{a}}^{\ \dot{b}} = \exp \left( - \frac{1}{2} \ii \theta_i \sigma_i^* + \frac{1}{2} \phi_i \sigma_i^* \right)_a^{\ b}, \quad \Lambda^{a}_{\ b} = \exp \left( - \frac{1}{2} \ii \theta_i \sigma_i^* - \frac{1}{2} \phi_i \sigma_i^* \right)_a^{\ b}.
        \label{eq:lorentz-transform-on-spinors}
    \end{bigcase}
\end{equation}
且它们满足\eqref{eq:left-right-matrix-transform}。
通过$\sigma$矩阵的定义%
\footnote{这里有一个可能引起困惑的细节。在张量代数中,我们有
\[
    (T^\top)_a^{\ b} = T^b_{\ a},
\]
但是在此处我们却似乎写出了这样的表达式:
\[
    (\sigma_a^{\ b})^\top = \sigma_b^{\ a},
\]
两者相差一个指标升降。产生这样的现象的原因在于,当我们通过\eqref{eq:sigma-matrix}定义
\[
    \sigma_a^{\ b} = \sigma
\]
时,左右两边的$\sigma$实际上有微妙的差异——左边的$\sigma$是某种旋量张量,右边的$\sigma$只是一个矩阵。
这就意味着,左边的$\sigma$的转置运算并不是简单的“把矩阵翻转过来”(因为转置之后的结果必须是协变的),$(\sigma^\top)_a^{\ b} = \sigma^b_{\ a}$关于左边的$\sigma$(不是带指标的$\sigma_a^{\ b}$!)成立。
然而,\eqref{eq:sigma-matrix}——从而由它导出的$\sigma$矩阵的厄米性——是关于右边的$\sigma$的,也就是说它仅仅关于左边的$\sigma_a^{\ b}$。因此,通过\eqref{eq:sigma-matrix}导出的$\sigma^*=\sigma^\top$中的转置就是简单的将矩阵翻转过来。
\label{note:confusion-by-transpose}
}%
,会发现$\sigma_i^* = \sigma_i^\top$,于是我们还可以得到
\begin{equation}
    \Lambda_{\dot{a}}^{\ \dot{b}} = \exp \left( - \frac{1}{2} \ii \theta_i \sigma_i + \frac{1}{2} \phi_i \sigma_i \right)_b^{\ a}, \quad \Lambda^a_{\ b} = \exp \left( - \frac{1}{2} \ii \theta_i \sigma_i - \frac{1}{2} \phi_i \sigma_i \right)_b^{\ a}.
\end{equation}
% TODO:向量的情况?
需要注意的是$\Lambda_a^{\ b}$和$\Lambda^b_{\ a}$一般来说是不同的。其原因在于,$\Lambda_a^{\ b}$中下标$a$对应一个洛伦兹变换之后的旋量,上标$b$对应洛伦兹变换之前的旋量;而$\Lambda_{\ a}^b$则正好相反,$b$对应变换前的旋量而$a$对应变换后的旋量,因此$\Lambda_a^{\ b}$和$\Lambda^b_{\ a}$不同。
事实上,我们可以将$\Lambda_a^{\ b}$记为$\Lambda_{a'}^b$,而将$\Lambda^b_{\ a}$记为$\Lambda_a^{b'}$,这样无需通过行指标、列指标区分两者。
与之相匹配地,我们可以使用$\chi^{a'}$这样的记号代替${\chi'}^{a}$,也就是说我们把$\chi$看成某种抽象的实体。%
\footnote{常见的介绍张量分析的文献会在某个流形——通常就是$\reals^3$——中将矢量、张量等处理为一个抽象的、内蕴的几何实体,定义矢量和张量的分量,然后导出指标升降规则和坐标变换规则;我们这里的步骤则正好相反,我们是首先通过洛伦兹群的表示得到指标升降规则和坐标变换规则,然后发现这两个规则允许我们把$\chi$看成某种抽象的实体。}
这样
\begin{equation}
    \Lambda_{a'}^b \Lambda^{a'}_c = \Lambda_a^{\ b} \Lambda^a_{\ c} = \delta_b^a, \quad \Lambda_{\dot{a}'}^{\dot{b}} \Lambda^{\dot{a}'}_{\dot{c}} = \Lambda_{\dot{a}}^{\ \dot{b}} \Lambda^{\dot{a}}_{\ \dot{c}} = \delta_{\dot{c}}^{\dot{b}}.
\end{equation}

下面我们要讨论使用旋量以及它们的一阶导数能够构造出怎样的标量。这是很重要的,因为如果需要使用旋量场来描述某种物理过程,那么对应的拉氏量应该是标量,或者至少是协变的,也就是说在洛伦兹变换下的变化量能够写成一个散度项。
注意到
\[
    \chi_a \xi^a \longrightarrow \chi'_a {\xi'}^a = \Lambda_a^{\ b} \chi_b \Lambda^a_{\ c} \xi^c = \delta_c^b \chi_b \xi^c = \chi_a \xi^a,
\]
同样的有
\[
    \chi_{\dot{a}} \xi^{\dot{a}} \longrightarrow \chi'_{\dot{a}} {\xi'}^{\dot{a}} = \chi_{\dot{a}} \xi^{\dot{a}}.
\]
但$\chi_a \xi^{\dot{a}}$或者$\chi^{\dot{a}} \xi_a$这种量却没有不变性。
这表明,只有“同类”——也就是都带点或者都不带点——的指标才能够一上一下地求和而得到一个标量。
% TODO:含导数的项

\subsubsection{四维矢量表示}\label{sec:4-vector-representation}

% TODO:是不是洛伦兹群的所有不可约表示都可以使用魏尔旋量直积出来?

我们接着讨论$(\frac{1}{2}, \frac{1}{2})$表示。由于$N_i^-$和$N_i^+$对易,这个表示实际上就是$(\frac{1}{2}, 0) \otimes (0, \frac{1}{2})$,于是这个表示可以使用一个左手旋量和一个右手旋量的张量积表示,记作$v_a^{\dot{b}}$。
% 向量空间中元素的张量积和向量空间上的矩阵或者说算符是不同的——前者实际上不需要考虑什么指标是行、什么指标是列!
% 不过,两者却按照同样的方式变换。
其中每个指标都独立地以\eqref{eq:lorentz-transform-on-spinors}变换。
当然,可以使用指标升降关系把$v_a^{\dot{b}}$转化为$v_{a\dot{b}}$。
这么做的好处在于,我们可以发现$(\frac{1}{2}, \frac{1}{2})$表示实际上可以约化为一个厄米的不可约表示和一个反厄米的不可约表示的直和。%
\footnote{同样,这里所谓的厄米和反厄米也是就分量矩阵$v_{a\dot{b}}$而论的,并不涉及旋量张量$v$的(协变的)转置。见\autoref{note:confusion-by-transpose}。}
注意到
\[
    v'_{c\dot{d}} = \Lambda_c^{\ a} \Lambda_{\dot{d}}^{\ \dot{b}} v_{a\dot{b}},
\]
可以得到
\[
    v'_{d\dot{c}} = \Lambda_d^{\ a} \Lambda_{\dot{c}}^{\ \dot{b}} v_{a\dot{b}} = \Lambda_d^{\ b} \Lambda_{\dot{c}}^{\ \dot{a}} v_{b\dot{a}},
\]
从而
\[
    (v'_{d\dot{c}})^* = \Lambda_{\dot{d}}^{\ \dot{b}} \Lambda_{c}^{\ a} (v_{b\dot{a}})^*.
\]
显然,如果分量矩阵$v_{a\dot{b}}$是厄米的,那么它经过洛伦兹变换之后还是厄米的;如果它是反厄米的,那么经过洛伦兹变换之后它还是反厄米的。
因此$(\frac{1}{2}, \frac{1}{2})$可以分解成其厄米子代数和反厄米子代数的直和。

我们来仔细分析其厄米子代数。二阶厄米方阵组成的向量空间有$2^2 \times 2 / 2 = 4$维。因此只需要找到四个独立的厄米方阵即可。
容易看出,三个泡利矩阵连同单位矩阵构成了这样的一组基。
为便于书写下标,使用$\sigma^i, i=1, 2, 3$表示三个泡利矩阵,又使用$\sigma^0$表示单位矩阵。
我们记$\sigma^\mu_{a\dot{b}}$为$\sigma^\mu$的第$a$行$b$列。(这个记号和\autoref{sec:weyl-spinor-representation}中的$\sigma_a^{\ b}$是不一样的!)
于是就可以把$(\frac{1}{2}, \frac{1}{2})$表示的厄米子代数的成员统一地写成
\begin{equation}
    v_{a\dot{b}} = v_\nu \sigma^\nu_{a\dot{b}}, \quad v_a^{\dot{b}} = \epsilon^{\dot{b} \dot{c}} v_\nu \sigma^\nu_{a \dot{c}}.
    \label{eq:vector-is-spin-tensor}
\end{equation}

% TODO: 严格证明$v$的变换正是洛伦兹变换

于是就可以使用$v_\nu$来代替整个$v_{a\dot{b}}$,并且$v_\nu$的变换方式正是洛伦兹变换下的矢量。
总之,$(\frac{1}{2}, \frac{1}{2})$表示的厄米子代数就是四维矢量。
因此,正如矢量可以看成二阶张量的平方根那样,魏尔旋量也可以看成矢量的平方根。
可以预期,四维矢量不见得能够描述所有的物理系统,因为它们不够基本。
% TODO:反厄米子代数呢?

同样我们考虑四维矢量能够构造出来的二阶的不变量。通过坐标变换关系容易看出,这样的不变量一定具有形式$A^\mu B_\mu$。

\subsubsection{狄拉克旋量}

在\autoref{sec:weyl-spinor-representation}和\autoref{sec:4-vector-representation}中我们只讨论规洛伦兹变换的表示。
当然,完整的洛伦兹群的表示肯定还是旋量,只不过我们还需要指定时间反演变换和宇称变换的表示。
% TODO:相当奇怪好像大家都不关注宇称变换作用在旋量上实际上会是怎样一个变换矩阵,当然似乎这也不重要。
注意到\eqref{eq:parity-and-time-reversion-transform},并考虑$N_i^\pm$的定义\eqref{eq:def-n-pm},在宇称变换下
\begin{equation}
    N_i^+ \stackrel{\Lambda_P}{\longrightarrow} N_i^-, \quad N_i^- \stackrel{\Lambda_P}{\longrightarrow} N_i^+.
\end{equation}
这意味着经过宇称变换,原本是$(\frac{1}{2}, 0)$表示的变换矩阵现在变成了$(0, \frac{1}{2})$表示的变换矩阵,原本是$(0, \frac{1}{2})$表示的变换矩阵则变成了$(\frac{1}{2}, 0)$表示的变换矩阵。
也就是说宇称变换下左手旋量按照右手旋量的方式变换,右手旋量按照左手旋量的方式变换。%
\footnote{我们并没有给出宇称变换的变换矩阵到底是什么。实际上因为宇称变换不在正规洛伦兹群中,只靠旋量的定义是不可能完全确定下宇称变换的矩阵表示是什么的。}
这就是“左手”、“右手”名称的来历:它们之间的变换和左右手坐标系之间的变换是完全一样的。
需要注意的是,$\chi_L$经过宇称变换之后未必变成$\chi_R$,同理$\chi_R$经过宇称变换之后也未必变成$\chi_L$。

时间反演变换造成的结果和宇称变换完全一样,除了具体的变换矩阵差一个负号。

% TODO:矢量在宇称变换之下的变换

如果一个体系在宇称变换之下不变,那么不可能仅仅使用一个左手旋量或者一个右手旋量描述它,因为宇称变换会改变旋量的手征。
因此,完整描述一个体系的物理量一定能够写成一对一对相互匹配的左手旋量和右手旋量。
我们称像这样由一个左手旋量和一个右手旋量组合而成的场
\begin{equation}
    \psi = \pmqty{\chi_L \\ \xi_R}
\end{equation}
为\textbf{狄拉克旋量}。它是洛伦兹群的一个可约表示,因为其变换矩阵
\begin{equation}
    \Lambda_\text{Dirac} = \pmqty{\dmat{\Lambda_{(\frac{1}{2}, 0)}, \Lambda_{(0, \frac{1}{2})}}}
\end{equation}
是左手旋量和右手旋量的变换矩阵直和起来得到的结果。
现在我们做一个宇称变换,那么变换矩阵就成为
\[
    \Lambda'_\text{Dirac} = \pmqty{\dmat{\Lambda_{(0, \frac{1}{2})}, \Lambda_{(\frac{1}{2}, 0)}}}.
\]
因此,考虑宇称变换在狄拉克旋量上的表示$(\Lambda_P)_\text{Dirac}$,我们有
\[
    (\Lambda_P)_\text{Dirac} \Lambda_\text{Dirac} (\Lambda_P)_\text{Dirac}^{-1} = \Lambda'_\text{Dirac},
\]
于是
\[
    (\Lambda_P)_\text{Dirac} = a \pmqty{0 & I_{2\times 2} \\ I_{2 \times 2} & 0}.
\]
由于我们考虑的是幺正表示(见\autoref{sec:rep-th}),$a$的模长为1。
请注意宇称变换、时间反演变换和恒等变换自成一个群,因此从有关正规洛伦兹群的表示的任何知识——它们是通过考虑正规洛伦兹群的李代数得到的,和宇称变换没有任何关系——都不可能把$a$的值确定下来。
不失一般性地%TODO:怎么就不失一般性了??
通常取$a=1$,于是就有
\begin{equation}
    (\Lambda_P)_\text{Dirac} = \pmqty{0 & I_{2\times 2} \\ I_{2 \times 2} & 0}.
\end{equation}
那么狄拉克旋量在宇称变换下的变化就是
\begin{equation}
    \psi \stackrel{\Lambda_P}{\longrightarrow} \psi^P = \pmqty{0 & I_{2\times 2} \\ I_{2 \times 2} & 0} \pmqty{\chi_L \\ \xi_R} = \pmqty{\xi_R \\ \chi_L}.
\end{equation}
也就是说宇称变换把狄拉克旋量变成了另一个同样具有$\chi_L$和$\xi_R$的对象,但是上下位置发生了变化。

狄拉克旋量上可以定义一种重要的变换。它不是洛伦兹变换的一种,其形式为
\begin{equation}
    \psi = \pmqty{\chi_L \\ \xi_R} \longrightarrow \psi^C = \pmqty{\xi_L \\ \chi_R}.
\end{equation}
容易看出变换之后得到的结果还是狄拉克旋量。

\subsubsection{无限维表示}

洛伦兹群的无限维表示完全由旋转生成元的无限维表示和推动生成元的无限维表示确定。
前者已经由\eqref{eq:rotation-inf-rep}给出了。虽然\eqref{eq:rotation-inf-rep}是在三维空间中推导出来的,但是因为推导它时各个坐标的标号和本节一致,都是$x^1, x^2, x^3$,它也适用于四维闵可夫斯基时空。
使用基本上一样的方法,从\eqref{eq:fin-and-inf-rep}和\eqref{eq:boost-generators}可以导出
\begin{equation}
    K_1 = - \ii (x^1 \partial_0 + x^0 \partial_1), \quad K_2 = - \ii (x^2 \partial_0 + x^0 \partial_2), \quad K_3 = - \ii (x^3 \partial_0 + x^0 \partial_3).
    \label{eq:boost-inf-rep}
\end{equation}

\subsection{庞加莱群}

% TODO:将下标转为上标
正规子群平移群半直积上洛伦兹群就得到了庞加莱群。

现在我们分析庞加莱群的李代数。独立的生成元总共有10个,4个是平移群生成元,3个是旋转生成元,3个是推动生成元。
平移群的李代数为\eqref{eq:comm-of-trans},而洛伦兹群的李代数为\eqref{eq:lie-algebra-lorentz},因此只需要$P_\mu$和$J_i,K_i$的对易关系就能够完全确定庞加莱群的李代数。
首先由于旋转操作不涉及时间维,显然我们有
\begin{equation}
    \comm*{J_i}{P_0} = 0.
\end{equation}
使用\eqref{eq:transition-inf-rep}、\eqref{eq:rotation-inf-rep}和\eqref{eq:boost-inf-rep}三式,可以推导出以下关系:
\begin{equation}
    \comm*{J_i}{P_j} = \ii \epsilon_{ijk} P_k, \quad \comm*{K_i}{P_j} = \ii \delta_{ij} P_0, \quad \comm{K_i}{P_0} = - \ii P_i.
    \label{eq:comm-k-j-p}
\end{equation}
虽然使用的是特殊的表示,但由于推导出来的都是对易关系,因此它们普遍成立。
\eqref{eq:comm-k-j-p}和\eqref{eq:lie-algebra-lorentz}给出了庞加莱群的李代数。

为了简化记号,我们设$M_{\mu \nu}$是一个反对称的矩阵,且
\begin{equation}
    J_i = \frac{1}{2} \epsilon_{ijk} M_{jk}, \quad K_i = M_{0i},
\end{equation}
那么\eqref{eq:comm-k-j-p}完全等价于
\begin{equation}
    \comm*{M_{\mu \nu}}{P_\rho} = \ii (\eta_{\mu \rho} P_\nu - \eta_{\nu \rho} P_\mu),
    \label{eq:comm-m-p}
\end{equation}
而\eqref{eq:lie-algebra-lorentz}等价于
\begin{equation}
    \comm{M_{\mu \nu}}{M_{\rho \sigma}}  = \ii (\eta_{\mu \rho} M_{\nu \sigma} + \eta_{\nu \sigma} M_{\mu \rho} - \eta_{\mu \sigma} M_{\nu \rho} - \eta_{\nu \rho} M_{\mu \sigma}).
    \label{eq:comm-m}
\end{equation}
\eqref{eq:comm-of-trans},\eqref{eq:comm-m-p}和\eqref{eq:comm-m}共同描述了庞加莱群的李代数。
% TODO:上面的几个公式有没有指标升降?
% TODO:建立群元的升降指标关系

庞加莱群的群参数一共有10个,在定义了$M$之后我们可以把群参数写成$a^\mu$和$\omega^{\mu \nu}$,其中$\omega$是反对称矩阵,从而庞加莱群就可以写成
\[
    \Lambda = \exp \left( \ii a^\mu P_\mu + \ii \omega^{\mu \nu} M_{\mu \nu} \right).
\]
实际上,即使$\omega$不是反对称的,上式照样给出庞加莱群,因为$\omega$的对称部分和$M$相乘得到的一定是零。反对称的要求仅仅是让每个变换对应唯一一个群参数。

庞加莱群的卡西米尔元有两个,它们分别是
\begin{equation}
    P_\mu P^\mu = m^2,
    \label{eq:momentum-and-mass}
\end{equation}
和
\begin{equation}
    W_\mu W^\mu = j_1 + j_2 = j,
\end{equation}
其中
\begin{equation}
    W^\mu = \frac{1}{2} \epsilon^{\mu \nu \rho \sigma} P_\nu M_{\rho \sigma}.
\end{equation}

我们使用$P_\mu P^\mu$给出的$m^2$实际上正是该表示下场的质量的平方。详情见\autoref{sec:k-g-eq}。

\section{单粒子量子力学}\label{sec:single-particle}

本节讨论单粒子量子力学。实际上,本节提供的大部分物理基本上已经在\autoref{sec:quantization-of-free-fields}中讨论了。本节的目的在于给出在完全不知道相对论性量子场论时怎么使用对称性工具重新得到这些物理。
所谓单粒子态指的是这样一种态:其上的一个CSCO是位置算符$\vb*{x}$连同一些别的量,从而其态空间可以写成一个坐标空间(或者说轨道空间)和一个内禀自由度空间的直积。

% TODO:有时候会把波函数当成一个场来定义类似于S矩阵这一类的东西,现在看来这本质上是在使用薛定谔场??

\subsection{重要的物理量}\label{sec:single-particle-quantity}

所谓单粒子态应该能够使用粒子位置和一些附加的自由度完全描述。

\subsubsection{位置算符与动量算符}\label{sec:position-and-momentum}

先考虑一维的位置算符$\hat{x}$。由于我们认为$\hat{x}$对应\textbf{位置},其谱为连续谱,本征值没有上下界,而是跑遍整个实数轴%
\footnote{注意这是\textbf{定义}:我们单纯构造了一个李代数,仅此而已。这一步实际上并没有用到任何物理概念。}%
。
于是,我们考虑一个可以完全由$\hat{x}$描述的希尔伯特空间$\mathcal{H}_{1\text{d}}$,在其上我们可以写出
\begin{equation}
    \hat{x} = \int \dd{x} x \dyad{x},
\end{equation}
有
\begin{equation}
    \hat{x} \ket{x'} = x' \ket{x'},
\end{equation}
其中$\ket{x'}$代表位置在$x'$的本征态。

很自然地,我们考虑空间平移群导致的物理量。空间平移群是李群,它在$\mathcal{H}_{1\text{d}}$上有幺正表示,则其生成元是厄米的,从而是一个可观察量。设%
\footnote{容易看出这代表
\[
    \hat{p} = \ii \pdv{\hat{Q}}{x},
\]
刚好和一般的定义差了一个负号。这是因为平移群对“场算符”位置$\hat{x}$的作用是
\[
    \hat{x} \stackrel{Q_\text{operator}}{\longrightarrow} \hat{x} + a,
\]
使用同一个群参数$a$,上式按照\eqref{eq:field-rep-and-state-rep-gen}诱导出的在态矢量上的作用就是
\[
    \ket{x} \stackrel{Q_\text{operator}}{\longrightarrow} \ket{x - a}.
\]
然而,我们通常希望空间平移群在态矢量上的作用是
\[
    \ket{x} \stackrel{Q_\text{state}}{\longrightarrow} \ket{x + a},
\]
因此为了让$Q_\text{operator}$按\eqref{eq:field-rep-and-state-rep-gen}诱导出的在态矢量上的作用的生成元和$Q_\text{state}$的生成元一致,在定义$Q_\text{state}$的生成元时我们特意加了一个负号。正文中的$\hat{Q}$指的都是$Q_\text{state}$。
\label{note:state-and-operator-minus-symbol}}
\begin{equation}
    \hat{Q}(\dd{x}) = \hat{I} + \frac{1}{\ii} \dd{x} \hat{p},
\end{equation}
其中$\hat{p}$是一个不显含任何参量的厄米算符。容易看出它具有动量量纲。
注意到
\[
    \begin{split}
        \hat{x} \hat{Q}(\dd{x'}) \ket{x'} = \hat{x} \ket{x' + \dd{x'}} = (x' + \dd{x'}) \ket{x' + \dd{x'}}, \\
        \hat{Q}(\dd{x'}) \hat{x} \ket{x'} = \hat{Q}(\dd{x'}) x' \ket{x'} = x' \hat{Q}(\dd{x'}) \ket{x'} = x' \ket{x' + \dd{x'}},
    \end{split}
\]
就有
\[
    [\hat{x}, \hat{Q}(\dd{x'})] \ket{x'} = \dd{x'} \ket{x' + \dd{x'}} \approx \dd{x'} \ket{x'}.
\]
考虑到$\ket{x'}$的任意性,我们得到
\[
    [\hat{x}, \hat{Q}(\dd{x'})] = \left[\hat{x}, \hat{I} + \frac{1}{\ii} \dd{x} \hat{p}\right] = \hat{I},
\]
从而
\begin{equation}
    [\hat{x}, \hat{p}] = \ii . 
    \label{eq:x-p-commutator-1d}   
\end{equation}
对易关系\eqref{eq:x-p-commutator-1d}完全确定了$\hat{x}$和$\hat{p}$的李代数结构。
实际上,完全可以使用更加简单的方法获得\eqref{eq:x-p-commutator-1d}。设一维空间平移群在$\mathcal{H}_\text{1d}$上的表示为$\hat{p}$。一维空间平移群对算符$\hat{x}$的作用的微分为
\[
    \frac{1}{\ii} \dv{a} {((x + a) - x)} = - \ii,
\]
从而由\eqref{eq:field-rep-and-state-rep-gen},我们有
\[
    \comm{\hat{p}}{\hat{x}} = - \ii,
\]
就得到了\eqref{eq:x-p-commutator-1d}。

我们来分析动量算符在坐标表象下的表示。
我们有
\[
    \begin{aligned}
        (1 - \ii \hat{p} \dd{x}) \ket{\psi} &= \hat{Q}(\dd{x'}) \ket{\psi} \\
        &= \int \dd{x'} \hat{Q}(\dd{x}) \ket{x'} \braket{x'}{\psi} \\
        &= \int \dd{x'} \ket{x' + \dd{x}} \braket{x'}{\psi} \\
        &= \int \dd{x'} \ket{x'} \braket{x' - \dd{x}}{\psi} \\
        &= \int \dd{x'} \ket{x'} \left(\braket{x'}{\psi} - \dd{x} \pdv{x'} \braket{x'}{\psi} \right) \\
        &= \ket{\psi} - \dd{x} \int \dd{x'} \ket{x'} \pdv{x'} \braket{x'}{\psi}, 
    \end{aligned}
\]
从而
\begin{equation}
    \hat{p} \ket{\psi} = - \ii \int \dd{x'} \ket{x'} \pdv{x'} \braket{x'}{\psi},
\end{equation}
或者等价的,
\begin{equation}
    \mel{x}{\hat{p}}{\psi} = - \ii \pdv{x} \braket{x}{\psi}.
    \label{eq:p-in-x-representation-1d}
\end{equation}
当然,完全可以使用空间平移群在无限维空间上的表示得到这个结果;由\autoref{note:state-and-operator-minus-symbol},虽然使用同一个群参数的空间平移群在算符$\hat{x}$和态$\ket{x}$上的作用正好是相反的,但由于定义$\hat{p}$时已经考虑了这一点,从\eqref{eq:transition-inf-rep}就能够得到\eqref{eq:p-in-x-representation-1d}。

从\eqref{eq:p-in-x-representation-1d}可以导出位置的本征态和动量本征态之间的切换关系。取$\ket{\psi}$为$\ket{p}$,我们有
\[
    \mel{x}{\hat{p}}{p} = - \ii \pdv{x} \braket{x}{p} = \mel{x}{p}{p} = p \braket{x}{p},
\]
即微分方程
\[
    \pdv{x} \braket{x}{p} = \ii p \braket{x}{p},
\]
从而
\[
    \braket{x}{p} = C \ee^{\ii p x}.
\]
由于总是可以在所有$\ket{p}$上一起乘上一个模长为1的常数而不产生任何影响,不失一般性地我们取$C$为实数。
归一化条件为
\[
    \delta(p - p') = \braket{p}{p'} = \int \dd{x} \braket{p}{x} \braket{x}{p'},
\]
从而计算得到
\begin{equation}
    \braket{x}{p} = \frac{1}{\sqrt{2\pi}} \ee^{\ii p x}.
\end{equation}
这表明坐标表象和动量表象之间的变换是傅里叶变换。

下面转而讨论三维的情况。三维位置算符$\hat{\vb*{x}}$指的是
\begin{equation}
    \hat{\vb*{x}} = \hat{x}^1 \vb*{e}_1 + \hat{x}^2 \vb*{e}_2 + \hat{x}^3 \vb*{e}_3,
\end{equation}
其中为方便起见选取$\vb*{e}_1, \vb*{e}_2, \vb*{e}_3$为一组规范正交基。
$\hat{x}^1, \hat{x}^2, \hat{x}^3$的本征值均跑遍整条实数轴,因此$\hat{\vb*{x}}$的本征值是$\reals^3$中全体矢量。
我们还需要加入另一个假设:$\hat{x}^1, \hat{x}^2, \hat{x}^3$彼此对易。这个假设要求这三个算符满足某种“独立性”。
在一个完全能够由$\hat{\vb*{x}}$描述的希尔伯特空间$\mathcal{H}_{3\text{d}}$中我们写出其形式
\begin{equation}
    \hat{\vb*{x}} = \int \dd[3]{\vb*{x}} \vb*{x} \dyad{\vb*{x}} 
    = \int \dd[3]{\vb*{x}} (\hat{x}^1 \vb*{e}_1 + \hat{x}^2 \vb*{e}_2 + \hat{x}^3 \vb*{e}_3) \dyad{x^1, x^2, x^3}.
\end{equation}

沿着$\vb*{e}_1, \vb*{e}_2, \vb*{e}_3$的平移操作是对易的,这就意味着
\begin{equation}
    [\hat{p}_1, \hat{p}_2] = [\hat{p}_2, \hat{p}_3] = [\hat{p}_3, \hat{p}_1] = 0.
\end{equation}
所有生成元彼此对易。这样三维空间平移群的李代数就被拆分成了三个一维空间平移群的李代数的直和,三维空间平移群就被拆分成了三个一维空间平移群的直积。
这三个一维空间平移群分别是$x^1$方向上的平移群$\hat{Q}_1(a)$,$x^2$方向上的平移群$\hat{Q}_2(a)$,以及$x^3$方向上的平移群$\hat{Q}_3(a)$。
另一方面,注意到三维位置算符对应的本征态张成的空间$\{ \ket{x^1, x^2, x^3} \}_{x_1,x_2,x_3}$实际上是三个一维位置算符对应的本征态张成的空间的直积,
这是因为$\hat{x}^1, \hat{x}^2, \hat{x}^3$彼此对易。于是我们做拆分
\[
    \mathcal{H}_\text{3d} = \mathcal{H}_\text{1d1} \otimes \mathcal{H}_\text{1d2} \otimes \mathcal{H}_\text{1d3},
\]
并且用$\hat{x}^1$完全描述$\mathcal{H}_\text{1d1}$,用$\hat{x}_2$完全描述$\mathcal{H}_\text{1d2}$,用$\hat{x}_3$完全描述$\mathcal{H}_\text{1d3}$。
由于$\hat{Q}_1(a)$不改变$x^2, x^3$,它在$\mathcal{H}_\text{1d2},\mathcal{H}_\text{1d3}$上没有作用。同样也可以对$\hat{Q}_2(a),\hat{Q}_3(a)$做同样的论证。从而,下标不一样的$\hat{x}^i$和$\hat{p}_i$彼此对易。
这样关于诸$\hat{x}$和诸$\hat{p}$的李代数就可以拆分成$\{\hat{x}^1, \hat{p}_1\}$、$\{\hat{x}^2, \hat{p}_2\}$和$\{\hat{x}^3, \hat{p}_3\}$三对量的李代数的直和。
最后注意到在每个空间$\mathcal{H}_\text{1d$i$}$中,
\[
    \hat{Q}_i(a) \ket{x^i} = \ket{x^i + a},
\]
于是我们可以原封不动地套用对一维动量和位置的论证,得到
\[
    [\hat{x}^i, \hat{p}_i] = \ii \delta_{j}^i.
\]
于是,在没有做任何计算,而只是观念性地拆分了态空间之后,我们得到三维情况下的动量-位置对易关系:
\begin{equation}
    [\hat{x}^i, \hat{x}^j] = 0, \quad [\hat{p}_i, \hat{p}_j] = 0, \quad [\hat{x}^i, \hat{p}_j] = \ii .
\end{equation}
同样,也可以通过\eqref{eq:field-rep-and-state-rep-gen}得到上式。
同样,套用一维动量和位置的论证,我们得到动量算符在坐标表象下的形式
\begin{equation}
    \mel{\vb*{x}}{\hat{\vb*{p}}}{\psi} = - \ii \grad{\braket{\vb*{x}}{\psi}},
    \label{eq:p-in-x-representation}
\end{equation}
以及相应的表象变换矩阵
\begin{equation}
    \braket{\vb*{x}}{\vb*{p}} = \frac{1}{(2\pi)^{3/2}} \ee^{\ii \vb*{p} \cdot \vb*{x}}.
    \label{eq:x-p-trans}
\end{equation}

以上我们讨论了三维动量表象。但实际上,空间平移群是在闵可夫斯基时空中的,因此我们还可以讨论四维动量表象。
所谓四维动量就是四维空间平移群的场表示按照\eqref{eq:field-rep-and-state-rep-gen}产生的、作用在态矢量上的生成元。
它的空间部分就是我们已知的$\hat{p}$,它的时间部分$\hat{E}$满足
\[
    - \ii \pdv{\psi}{t} = [\hat{E}, \psi],
\]
其中$\psi$是某个场。因此$\hat{E}$就是能量。我们不说这是哈密顿量是因为它仅仅关于被讨论的单个粒子,而真正的哈密顿量应该关于整个体系。
按照\autoref{sec:translation}中的描述,四维动量的四个分量彼此对易。因此,四维矢量算符$\hat{p}^\mu$是态空间
\[
    \{\ket{p}\}_p = \{\ket{\vb*{p}} \ket{E}\}
\]
的CSCO。然而,完整地描述单粒子的坐标空间实际上并不需要全部的$\{\ket{\vb*{p}} \ket{E}\}$——这是当然的,既然使用$\hat{p}$已经能够完整描述态空间了。
事实上,考虑到\eqref{eq:momentum-and-mass},我们有
\begin{equation}
    \hat{E}^2 - \hat{\vb*{p}}^2 = m^2,
\end{equation}
如果态$\ket{p}$描述的是坐标空间中的某个态,就可以把上式作用在$\ket{p}$上:
\[
    \begin{aligned}
        m^2 \ket{p} &= \hat{E}^2 \ket{p} - \hat{\vb*{p}}^2 \ket{p} \\
        &= (E^2 - \vb*{p}^2) \ket{p},
    \end{aligned}
\]
% TODO:怎么证明这里的$m$和场表示中的$m$一样?
从而
\[
    m^2 = E^2 - \vb*{p}^2 = p_\mu p^\mu.
\]
为了保证$E$的单值性我们要求$E>0$,于是
\begin{equation}
    m^2 = E^2 - \vb*{p}^2 = p_\mu p^\mu, \quad E > 0.
    \label{eq:mass-shell}
\end{equation}
我们看到,\eqref{eq:mass-shell}在四维闵可夫斯基时空中选出了一个三维的子流形。我们称这个子流形为\textbf{质壳}。
态$\ket{p}$描述了一个实际的单粒子的坐标空间,当且仅当,$p$在质壳\eqref{eq:mass-shell}上,或者说$p$\textbf{在壳}。
从而,给定一个$\vb*{p}$,我们把使用质壳方程写出的对应能量记作
\begin{equation}
    E_{\vb*{p}} = \sqrt{ m^2 + \vb*{p}^2 }.
\end{equation}

由于$\eqref{eq:mass-shell}$让我们能够从$\vb*{p}$直接导出$p_0$也就是$E$,实际上完全可以使用$\vb*{p}$来标记在质壳上的$\ket{p}$。但我们并没有这么做,原因马上可以看到。
现在我们来归一化$\ket{p}$。因为我们在四维闵可夫斯基时空中工作,不再能够使用$\int \dd[3]{\vb*{p}}$来做归一化了,因为它不满足洛伦兹协变性:做一个洛伦兹变换,就有可能把一部分$\vb*{p}$弄到$p_0$中。
因此我们需要在四维矢量空间上定义一个积分测度,它只能在质壳上给出非零值,且这个积分测度必须是洛伦兹标量。
容易看出,
\[
    \int \dd[4]{p} \delta(p^2 - m^2) \theta(p_0)
\]
正是一个满足这种条件的积分测度。因此归一化条件为
\[
    1 = \int \dd[4]{p} \delta(p^2 - m^2) \theta(p_0) \dyad{p}.
\]
由于
\[
    \begin{aligned}
        \int \dd[4]{p} \delta(p^2 - m^2) \theta(p_0) &= \int \dd[3]{\vb*{p}} \int \dd{p_0} \delta (p_0^2 - \vb*{p}^2 - m^2) \theta(p_0) \\
        &= \int \dd[3]{\vb*{p}} \int \dd{p_0} \left( \frac{\delta(p_0 - E_{\vb*{p}})}{2 E_{\vb*{p}}} + \frac{\delta(p_0 + E_{\vb*{p}})}{- 2 E_{\vb*{p}}} \right) \theta (p_0) \\
        &= \int \frac{\dd[3]{\vb*{p}}}{2 E_{\vb*{p}}},
    \end{aligned}
\]
我们得到
\[
    \int \frac{\dd[3]{\vb*{p}}}{2 E_{\vb*{p}}} \dyad{p} = 1.
\]
对比
\[
    \int \dd[3]{\vb*{p}} \dyad{\vb*{p}} = 1,
\]
我们不失一般性地将在壳的$\ket{p}$放在坐标空间中,且由于$\ket{p}$与以它的空间部分$\vb*{p}$为标记的态$\ket{\vb*{p}}$一一对应,不失一般性地要求$\ket{p}$和$\ket{\vb*{p}}$差一个实数,那么就有
\begin{equation}
    \ket{p} = \sqrt{2 E_{\text{p}}} \ket{\vb*{p}}.
    \label{eq:relativity-p}
\end{equation}
在三维动量空间中,$\ket{p}$是没有被归一化的;但在质壳中,$\ket{\vb*{p}}$反而是没有被归一化的那个。
为了保证洛伦兹协变性,在讨论相对论性量子场论时多用$\ket{p}$而不是$\ket{\vb*{p}}$,虽然它们表示的物理状态完全是一样的。
在坐标表象和$p$表象之间切换只需要使用
\begin{equation}
    \braket{\vb*{x}}{p} = \sqrt{\frac{2 E_{\vb*{p}}}{(2\pi)^3}} \ee^{\ii \vb*{p} \cdot \vb*{x}}.
    \label{eq:relativity-x-p-trans}
\end{equation}
但实际上这个公式很少真的被使用,因为一来相对论情况下很少需要讨论粒子位置,二来从相对论动量表象也就是$p$表象切换回$\vb*{x}$表象时需要使用积分测度$\int \dd[3]{\vb*{x}}$,而这不是洛伦兹协变的。由于我们在实际计算时有时需要将时间单独拿出来当成一个“演化参数”来看待而不是把它当成坐标,通常不讨论$\int \dd[3]{\vb*{x}}$的洛伦兹协变版本,从而从$p$表象切换回$\vb*{x}$表象会有困难,我们也不去理会它。

% TODO:通过这种方式导出的动量和能量和使用诺特定理导出的动量和能量有什么关系吗

\subsubsection{角动量}

% TODO:把指标升降放到最后做

角动量算符就是旋转生成元在态矢量上的表示。有限维的旋转生成元是一些矩阵,它们实际上就是三个方向上的角动量算符在$\hat{J}_z$的本征态上的分量矩阵(为什么是$\hat{J}_z$是因为在导出具体的矩阵表达式时是使用$J_x$和$J_y$构造$J_z$的本征态的产生湮灭算符的)。
与\autoref{note:state-and-operator-minus-symbol}中提到的类似,如果旋转群$\hat{R}(\vb*{\phi})$是直接作用在态上的,那么定义
\begin{equation}
    \hat{R}(\vb*{\phi}) = \hat{I} + \frac{1}{\ii} (\phi_1 \hat{J}^1 + \phi_2 \hat{J}^2 + \phi_3 \hat{J}^3), 
\end{equation}
而如果它是作用在算符上,而只是通过\eqref{eq:field-rep-and-state-rep-lie-group}诱导出了一个在态上的表示,那么定义
\[
    \hat{R}(\vb*{\phi}) = \hat{I} + \ii (\phi_1 \hat{J}^1 + \phi_2 \hat{J}^2 + \phi_3 \hat{J}^3).
\]
两种定义给出的$\hat{J}^i$是一样的。

$J^i$可以分解成两部分:一部分改变$\vb*{x}$的取值,一部分不改变。改变$\vb*{x}$取值的那一部分记作$L^i$,称为\textbf{轨道角动量},它与粒子的位置自由度有关,因此有这个名称;不改变$\vb*{x}$取值的那一部分记作$S^i$,称为\textbf{自旋角动量},因为它和粒子的内禀自由度有关。
$J^i$服从\eqref{eq:lie-algebra-so3},因此我们也称\eqref{eq:lie-algebra-so3}为\textbf{角动量代数}。
通常设
\begin{equation}
    \vb*{J} = J_1 \vb*{e}_1 + J_2 \vb*{e}_2 + J_3 \vb*{e}_3,
\end{equation}
则角动量代数就是
\begin{equation}
    \vb*{J} \times \vb*{J} = \ii \vb*{J}.
\end{equation}
不过,实际计算时很少用到这个矢量,因为由于$\hat{J}^i$是三维旋转群的李代数的无穷维厄米表示,它们遵循角动量代数,因此彼此间不对易。
$\hat{\vb*{J}}$没有本征态。但由角动量代数,$\hat{\vb*{J}}^2$却和三个$\hat{J}^i$都对易。

需要注意的是,以上定义不能够一般地将轨道角动量和自旋角动量完全分开,因为可以任意地将自旋角动量划归一部分给轨道角动量。因此我们人为要求轨道角动量仅仅作用在坐标空间上,而自旋角动量仅仅作用在内禀自由度上。

按照\eqref{eq:rotation-inf-rep}和\eqref{eq:p-in-x-representation},我们有
\[
    \begin{aligned}
        \mel{\vb*{x}}{\hat{L}^1}{\psi} &= x^2 (- \ii \partial_3) \braket{\vb*{x}}{\psi} - x^3 (- \ii \partial_2) \braket{\vb*{x}}{\psi} \\
        &= \mel{\vb*{x}}{(\hat{x}^2 \hat{p}_3 - \hat{x}^3 \hat{p}_2)}{\psi}, \\
        \hat{L}^1 &= \hat{x}^2 \hat{p}_3 - \hat{x}^3 \hat{p}_2.
    \end{aligned}
\]
同样也可以这样计算出$\hat{L}^2$和$\hat{L}^3$。从而我们得到
\begin{equation}
    \hat{\vb*{L}} = \hat{\vb*{x}} \times \hat{\vb*{p}},
\end{equation}
其中$\hat{\vb*{L}}$就是
\begin{equation}
    \hat{\vb*{L}} = \hat{L}^i \vb*{e}_i.
\end{equation}

\begin{equation}
    \hat{\vb*{L}}^2 = \hat{\vb*{x}}^2 \hat{\vb*{p}}^2 - (\hat{\vb*{x}} \cdot \hat{\vb*{p}})^2 + \ii \hat{\vb*{x}} \cdot \hat{\vb*{p}} = \hat{\vb*{x}}^2 \hat{\vb*{p}}^2 - (\hat{\vb*{x}} \cdot \hat{\vb*{p}}) (\hat{\vb*{p}} \cdot \hat{\vb*{x}}).
\end{equation}

总角动量$\hat{J}^i$和轨道角动量$\hat{L}$都遵循角动量代数,也即
\[
    \begin{aligned}
        \comm*{\hat{L}^i+\hat{S}^i}{\hat{J}^j+\hat{S}^j} &= \ii \epsilon_{ijk} (J^k+S^k), \\
        \comm*{\hat{L}^i}{\hat{L}^j} &= \ii \epsilon_{ijk} L^k,
    \end{aligned}
\]
而由于轨道角动量和自旋角动量作用在态空间的不同自由度上,它们是彼此对易的,从而我们推导出
\[
    \comm*{\hat{S}^i}{\hat{S}^j} = \ii \epsilon_{ijk} \hat{S}^k,
\]
因此自旋角动量确确实实是一种角动量:它也满足角动量代数。

我们也可以定义
\begin{equation}
    H = \hat{\vb*{S}} \cdot \frac{\hat{\vb*{P}}}{\abs{\vb*{P}}}
\end{equation}

如前所述,$J^2$和$\hat{J}_i$中的任何一个都对易。习惯上考虑$\hat{J}_3$,我们会发现$\hat{J}^2$和$J_3$实际上构成了角动量代数中的一组CSCO,因为$\hat{J}^2$是角动量代数的卡西米尔元,它确定了$\hat{J}_3$的取值的上下限,再加上$\hat{J}_3$的取值,系统的角动量状况就完全确定了。
通常称$\hat{L}^2$的本征值$\sqrt{j(j+1)}$中的$j$为\textbf{角量子数},$\hat{L}_3$的本征值为\textbf{磁量子数},而$\hat{S}_3$称为\textbf{自旋量子数}。
$\hat{S}^2$没有专门的称呼,因为对种类确定的粒子它是固定的。

容易验证,若$\vb*{J}_1$和$\vb*{J}_2$都是角动量算符(也即,它们构成角动量代数),则它们之和$\vb*{J}_1 + \vb*{J}_2$也是角动量算符。
我们会看到将$\vb*{J}_1$和$\vb*{J}_2$叠加会得到一个可约表示,$\vb*{J}_1 + \vb*{J}_2$作用在$\vb*{J}_1$和$\vb*{J}_2$各自作用的空间的直积上,而这个直积实际上是一系列空间的直和。
我们设
\[
    \hat{J}^2 \ket{jm} = j(j+1) \ket{jm}, \quad \hat{J} \ket{jm} = m \ket{jm},
\]
特意加上一个$j$的标记是因为这个表示可能是可约的。到现在为止$j$可以取哪些值都是完全不清楚的,不过无论如何,上式构成了$\hat{J}$的表示空间的一组基。另一方面,容易看出
\[
    \ket{j_1 j_2 m_1 m_2} = \ket{j_1 m_1} \otimes \ket{j_2 m_2}
\]
实际上也是$\hat{J}$的表示空间的一组基。
容易验证$\ket{j_1 j_2 m_1 m_2}$是$J_z$的一组本征态,且
\begin{equation}
    m = m_1 + m_2,
    \label{eq:m-m1-m2}
\end{equation}
但是并不是$\vb*{J}^2$的一组本征态。
两组基之间的变换系数$\braket{j_1 j_2 m_1 m_2}{j m}$称为\textbf{CG系数}。
$\{\ket{jm}\}$实际上是由若干不可约表示直和而成的基矢量(由不同的$j$对应的表示直和而成),而$\{\ket{j_1 j_2 m_1 m_2}\}$则是两个不可约表示直积而成的基矢量,CG系数用于在这两组基矢量来回切换。
可以预期,CG系数应该能够反映\eqref{eq:m-m1-m2}这一事实。的确如此:将用CG系数表示的$\ket{jm}$代入本征方程得到
\[
    \sum_{m_1, m_2} (m - m_1 - m_2) \braket{j_1 j_2 m_1 m_2}{j m} \ket{j_1 j_2 m_1 m_2} = 0,
\]
于是
\[
    (m - m_1 - m_2) \braket{j_1 j_2 m_1 m_2}{j m} = 0.
\]
这就意味着
\[
    \ket{jm} = \sum_{m_1} \ket{j_1 j_2 m_1 (m-m_1)} \braket{j_1 j_2 m_1 (m-m_1)}{jm}.
\]
由于$m$是$m_1$和$m_2$之和,$m$的最大值是$j_1+j_2$,最小值是$-j_1-j_2$,在此之间的所有值都可以取到。
% TODO:证明
$j$的取值为
\[
    j = j_1 + j_2, j_1 + j_2 - 1, \ldots, \abs{j_1 - j_2}.
\]
在半经典的图景中,这意味着叠加两个态的角动量时,它们的夹角是量子化的。
计算每个$j$对应的不可约表示的维度,把它们加起来,得到的结果正是$(2j_1 - 1) (2j_2 - 1)$。

CG系数的具体计算需要用到$\vb*{J}$的升降算符,注意到
\begin{equation}
    J_{\pm} \ket{jm} = \sum_{m_1 + m_2 = m} (J_{1 \pm} \ket{j_1 m_1} \ket{j_2 m_2} + \ket{j_1 m_1} J_{2 \pm} \ket{j_2 m_2}) \braket{j_1 j_2 m_1 m_2}{jm},
    \label{eq:addition-of-ca-opearator-su2}
\end{equation}
首先可以用这个关系式计算出$j=m$的CG系数,注意到
\[
    J_+ \ket{jm} = 0,
\]
就有
% TODO
可以递推得到

类似的方法还可以计算出三个角动量算符叠加而得到的结果,乃至更多。

需要注意的是按照以上步骤将角动量算符叠加之后得到的角动量基矢量并没有对称化/反对称化。
反对称化可能会让一些状态实际上取不到。

% TODO:螺旋度

\subsubsection{宇称}

% TODO

\subsection{动力学}

实际上我们并不知道量子情况下$\hat{H}$应该取什么样的形式。
仅有的线索是:$\hbar \to 0$时$\hat{H}$退化为经典哈密顿量,$\hat{p}$退化为经典动量,算符退化为经典的实数。
从经典哈密顿量写出$\hat{H}$算符的过程称为\textbf{量子化}。由于量子化方案不唯一,还是需要实证数据才能得到真正的$\hat{H}$。
一个经验证行之有效的方案是:首先写出经典拉氏量,然后得到经典运动方程。合理选择$\hat{H}$使通过
\begin{equation}
    \dv{\hat{A}}{t} = \frac{1}{\ii \hbar} [\hat{A}, \hat{H}] + \pdv{\hat{A}}{t}
\end{equation}
以及之前通过对称性导出对易关系得到的微分方程在$\hbar \to 0$时能够退化为经典运动方程,这样就完成了量子化。
% TODO:使用路径积分量子化做这件事要更加清晰一些。需要表明路径积分和正则量子化的对应关系。

我们首先尝试构造自由粒子的拉氏量。所谓自由指的是:动力学规律满足最高的时空对称性,并且运动方程是线性的%
\footnote{
    需要注意的是这一点并不是先验成立的。实际上正确的思路是,我们在实验上注意到,直觉上是自由的、不受相互作用粒子都可以使用高度对称的线性方程描述,
    于是理论中把“自由”规定为“对称且线性”,看从这个假设能够导出什么结果。这个假设当然有可能是错的。
}%
。
本节仅仅讨论非相对论情况下的拉氏量,也就是说取时空对称性为伽利略对称性。这是因为相对论性单粒子量子力学会导致一些疑难问题,因此并没有必要去分析它,而只需要从一开始就分析相对论性3+1维场论就可以了。

拉氏量是一个数,因此是伽利略群的平凡表示,则自由粒子的拉氏量在伽利略群下的变化量必须要能够写成某个量的时间全导数,因此伽利略变换前后的拉氏量描述同样的物理过程。
首先考虑什么样的场量是被允许的。
旋转对称性意味着有意义的场量只能是标量、矢量、旋量以及它们的直积。
因此$\vb*{x}$确实是可以出现在拉氏量中的场量。于是我们有
\[
    L = L(\vb*{x}, \dot{\vb*{x}}, t).
\]
时间和空间平移对称性要求$L$不显含$\vb*{x}$和$t$%
\footnote{需要注意的是要配合使用这两个不变性。如果只有时间平移不变性,那么完全可以巧妙地构造$L$使得
\[
    \pdv{L}{\vb*{x}} \cdot \dot{\vb*{x}} + \pdv{L}{\dot{\vb*{x}}} \cdot \ddot{\vb*{x}} + \pdv{L}{t} = 0
\]
来让$L$随时间平移不变;同样如果只有空间平移不变性也可以做同样的构造。
这两种情况都不能从$L$的表达式中排除掉$\vb*{x}, t$。
然而,如果同时有时间平移不变性和空间平移不变性,
我们就可以同时让$t$做一个平移而与此同时让$\vb*{x}$做另一个平移来抵消掉$t$的平移对$\vb*{x}$带来的影响,让$\vb*{x}'(t')=\vb*{x}(t)$,
此时就有
\[
    \dv{L}{t} = \pdv{L}{t} = 0.
\]
空间同理。
% TODO:为什么这里全部把时间全导数看成0了?
}%
,因此我们有
\[
    L = L(\dot{\vb*{x}}).
\]
$L$是标量,而$\dot{\vb*{x}}$是矢量,因此$\dot{\vb*{x}}$必须做缩并。它或者和自己缩并,或者和别的什么矢量或者张量缩并。
然而$L$不能含有别的矢量,不然会违反空间旋转对称性。%
\footnote{
    这里有一个比较迷惑性的细节。如果使用含有$\vb*{a} \cdot \hat{\vb*{x}}$项的拉氏量,算出来的矢量形式欧拉-拉格朗日方程在坐标轴旋转下确实是不变的。
    然而,其中的三个分量方程的形式会发生变化,因为$\vb*{a}$在三个方向的分量会发生变化。
    我们要求的“物理规律不变”指的是每一个分量满足的方程都不能改变。
    实际上,如果只要求矢量方程不变,那么旋转对称性提供不了任何信息,因为矢量方程在空间旋转下肯定不变。
    \label{note:rotation}
}%
因此$L$必定是$\dot{\vb*{x}}^2$的函数,也就是
\[
    L = L(\dot{\vb*{x}}^2) = a_1 \dot{\vb*{x}}^2 + a_2 \dot{\vb*{x}}^4 + \cdots
\]
以上我们已经使用了空间平移对称性、时间平移对称性以及空间旋转对称性。现在我们要使用惯性变换来完全确定$L$的形式。做变换
\[
    \dot{\vb*{x}} \longrightarrow \dot{\vb*{x}} + \vb*{\epsilon},
\]
则有
\[
    \var{L} = 2 \dot{\vb*{x}} \cdot \vb*{\epsilon} \pdv{L}{\dot{\vb*{x}}^2}
\]
由于$L$在惯性变换之下的变化量必须是某个函数的时间全导数,而$\dot{\vb*{x}} \cdot \vb*{\epsilon}$已经是一个时间全导数了,
因此$\partial L / \partial \dot{\vb*{x}}^2$不显含$\dot{\vb*{x}}^2$,也就是说
\[
    L = a \dot{\vb*{x}}^2,
\]
我们不妨重新定义有关常数,使得
\begin{equation}
    L = \frac{1}{2} m \dot{\vb*{x}}^2 = \frac{1}{2} m (\dot{x}_1^2 + \dot{x}_2^2 + \dot{x}_3^2).
    \label{eq:free-particle-lagrangian}
\end{equation}
这就是自由粒子的拉氏量。从\eqref{eq:free-particle-lagrangian}得出哈密顿量为
\[
    H = \frac{p^2}{2m},
\]
而前面已经证明通过平移群得到的$\hat{p}$正是经典动量的推广,因此哈密顿算符就是
\begin{equation}
    \hat{H} = \frac{\hat{p}^2}{2m}.
\end{equation}
于是我们就得到了量子情况下的自由粒子动力学。

现在讨论有相互作用的情况。
此时$\hat{\vb*{x}}$与其它的一些物理量耦合,在仅考虑$\vb*{x}$的动力学时,耦合体现为拉氏量多出来一个耦合项。%
\footnote{这是一个很好地能够体现出“重要的是算符而不是态”的例子。
不妨设与$\hat{\vb*{x}}$耦合的系统能够使用一个算符$\hat{y}$描述,那么整个系统的态矢量就应该写成以$\ket{\vb*{x}, y}$为基的形式。
但是我们只关注$\vb*{x}$,因此将$\hat{y}$有关的机制全部看成是给定的外力项,那么系统的态矢量就可以写成$\ket{\vb*{x}}$张成的。
代数上的定理保证前后两个态空间有同态关系,因此我们这么做完全没有问题。
}
此时拉氏量为
\begin{equation}
    L = \frac{1}{2} m (\dot{x}_1^2 + \dot{x}_2^2 + \dot{x}_3^2) - V(\vb*{x}, \dot{\vb*{x}}).
\end{equation}
相应的写出哈密顿量为
\[
    H = \frac{1}{2m} \left(\vb*{p} + \pdv{L}{\dot{\vb*{x}}}\right)^2 + V - \pdv{V}{\dot{\vb*{x}}} \cdot \dot{\vb*{x}},
\]
同样将上式中所有的$\vb*{x}$都使用$\vb*{p}$表示,然后使用$\hat{\vb*{p}}$代替$\vb*{p}$就得到了哈密顿算符。
它导致的运动方程就是著名的\textbf{薛定谔方程},即
\begin{equation}
    \ii \hbar \dv{t} \ket{\psi} = \left( \frac{1}{2m} \left(\vb*{p} + \pdv{L}{\dot{\vb*{x}}}\right)^2 + V - \pdv{V}{\dot{\vb*{x}}} \cdot \dot{\vb*{x}} \right) \ket{\psi}.
\end{equation}
在外加势和速度无关时它就是
\begin{equation}
    \ii \hbar \dv{t} \ket{\psi} = \frac{\hat{\vb*{p}}^2}{2m} \ket{\psi} + \hat{V} \ket{\psi}.
\end{equation}

\section{相对论性量子场论}

\subsection{自由场动力学}\label{sec:qft-free-dynamics}

在相对论性量子场论中我们仍然要求自由粒子的拉氏量具有最高的对称性,也就是说,在庞加莱群作用下没有变化,
且拉氏量只含有二次项(从而给出线性的运动方程)。%
\footnote{虽然本文主要分析正则量子化,但写出运动方程还是用的是拉氏量。这是更加方便的做法,因为正则表述在理论框架上将时间和空间分开对待了,因此不容易观察哈密顿量在洛伦兹变化之下是不是给出恒定不变的动力学。}
空间平移不变性意味着拉氏量不能显含坐标;空间各向同性意味着拉氏量中的参数必须是标量,不能出现多分量的参数。

本节还将计算三种场的哈密顿量。由于我们通常在欧氏空间中写出哈密顿量而在闵可夫斯基时空中讨论拉氏量,需要格外注意一点:闵可夫斯基时空的度规为$(+, -, -, -)$而欧几里得空间的度规为$(+, +, +)$,因此
\[
    A_\mu A^\mu = (\dot{A}^0)^2 - \vb*{A}^2,
\]
上式左边为闵可夫斯基时空中的表达式,右边为欧几里得空间中的表达式。
换而言之,闵可夫斯基时空中的$A_i A^i$和欧几里得空间中的$A_i A^i$差一个负号。

\subsubsection{实标量场的克莱因-高登方程}\label{sec:k-g-eq}

% 自由标量场$\phi$的不超过二阶项的拉氏量可以写成
% \[
%     \mathcal{L} = A + B \phi + C \phi^2 + D^\mu \partial_\mu \phi + E^{\mu \nu} \partial_\mu \phi \partial_\nu \phi + F^\mu \phi \partial_\mu \phi.
%\]
%空间平移不变性意味着$A,B,C,D,E,F$全部是常数。
%拉氏量中的常数项不提供任何物理,可略去,于是略去$A$项。
%由于空间各向同性(参见\autoref{note:rotation}),所有的奇数次$\partial_\mu \phi$都不应该出现。
%于是拉氏量为
%\[
%    \mathcal{L} = B \phi + C \phi^2 + E^{\mu \nu} \partial_\mu \phi \partial_\nu \phi.
%\]
%还是由于空间旋转不变性,$E^{\mu \nu}$必须是度规张量,否则运动方程中将出现一个旋转时分量会改变的张量,
%% TODO:运动方程本身不会发生改变但是空间中还是会出现一个特殊方向
%因此拉氏量为
%\[
%    \mathcal{L} = B \phi + C \phi^2 + E \partial_\mu \phi \partial^\mu \phi.
%\]
%拉氏量中的$\phi$项实际上无关紧要,因为完全可以通过重新定义一个$\phi' = \phi + \const$来把一次项弄掉,所以不失一般性地
%\[
%    \mathcal{L} = C \phi^2 + E \partial_\mu \phi \partial^\mu \phi.
%\]
标量场的拉氏量中只能够出现$\phi$和$\partial_\mu \phi$构成的一次或二次不变量。
$\phi$构造出的一次不变量有$\phi$,二次不变量有$\phi^2$,$\partial_\mu \phi$是矢量,$\partial_\mu \phi$不可能和$\phi$缩并,而由于拉氏量中的参数都是标量它也不可能和参数缩并,因此它只能和自己缩并,得到$\partial_\mu \phi \partial^\mu \phi$。
这样我们得到
\[
    \mathcal{L} = A \phi + B \phi^2 + C \partial_\mu \phi \partial^\mu \phi.
\]
拉氏量中的$\phi$项实际上无关紧要,因为完全可以通过重新定义一个$\phi' = \phi + \const$来把一次项消除掉,于是我们略去这一项。
由于拉氏量可以任意地乘上非零常数,我们通过重新定义常数可以得到
\begin{equation}
    \mathcal{L} = \frac{1}{2} (\partial_\mu \phi \partial^\mu \phi - m^2 \phi^2).
    \label{eq:klein-gordon-lagrangian}
\end{equation}
这个拉氏量导致下面的运动方程:
\begin{equation}
    (\partial_\mu \partial^\mu + m^2) \phi = 0.
    \label{eq:klein-gordon-eq}
\end{equation}
这就是\textbf{克莱因-高登方程},标量场或者说自旋0场的基本运动方程。
可以证明,为了让\eqref{eq:klein-gordon-eq}给出有物理意义的预言(例如不出现无限下降的能量,等等),应当取$m \geq 0$。

实际上,所有场的运动方程均满足克莱因-高登方程。我们将在推导其它场的运动方程之后证明这一点。
这就导致了一个重要的结果。平移生成元在场表示中为\eqref{eq:transition-inf-rep},从而
\[
    P_\mu P^\mu = - \partial_\mu \partial^\mu,
\]
于是代入\eqref{eq:klein-gordon-eq},得到
\[
    P_\mu P^\mu \phi = - \partial_\mu \partial^\mu = m^2 \phi,
\]
其中$m$为克莱因-高等方程中出现的那个$m$。也即,通过$P_\mu P^\mu$的表示的本征值(实际上就是它和恒等变换之间差的倍数,因为$P_\mu P^\mu$是卡西米尔元)得到的$m$和场的运动方程得到的$m$是一样的。

现在导出标量场的哈密顿表述。计算得到
\begin{equation}
    \pi = \partial_0 \phi = \dot{\phi},
    \label{eq:klein-gordon-pi}
\end{equation}
相应的
\begin{equation}
    \mathcal{H} = \frac{1}{2} \dot{\phi}^2 + \frac{1}{2} (\grad{\phi})^2 + \frac{1}{2} m^2 \phi^2.
\end{equation}
如我们希望的那样,哈氏量密度是正定的。这当然是因为我们适当地选择了$\mathcal{L}$的正负号。

另外,注意到 % TODO:好像我们还从来没有严格定义过下式?
\[
    P_0 = E, \quad P_i \vb*{e}^i = \vb*{p},
\]
在场表示中我们可以写出
\[
    E^2 - \vb*{p}^2 = m^2
\]
或者说
\[
    E^2 = m^2 + \vb*{p}^2.
\]
这正是质壳关系\eqref{eq:mass-shell}。这就提示我们,还有另一种量子化方案:做替换
\[
    E \longrightarrow \hat{E} = \ii \partial_0, \quad \vb*{p} \longrightarrow \hat{\vb*{p}} = - \ii \grad,
\]
则从能量-动量关系就可以得到克莱因-高登方程。
我们不采用这种方案,因为它隐含地引入了太多的假设:算符$E, \vb*{p}$是作用在一个算符场上而不是态矢量上;$E,\vb*{p}$是厄米算符,也即,平移群在此算符场上取幺正表示(注意这一点并不一般成立!例如,场表示\eqref{eq:fin-and-inf-rep}中的有限维表示就常常不是幺正的),等等。

\subsubsection{旋量场的狄拉克方程}

旋量场实际上几乎从来不会在经典情况下遇到,因为它们的场值是复数,因此不具有直接的物理意义。

本节讨论旋量的运动方程。使用凑拉氏量的方法处理旋量会比较困难,因为旋量的指标分带点的和不带点的,因此会频繁地涉及求共轭等运算,在拉格朗日动力学中讨论这些问题并不方便。
因此接下来我们尝试直接构造旋量的运动方程,从这些运动方程反推对应的拉氏量。
我们将尝试构造一阶运动方程。如果对魏尔旋量和狄拉克旋量都能够构造出一阶运动方程,那就没有必要考虑更高阶的运动方程。
% TODO:为什么?

首先讨论魏尔旋量的运动方程。满足平移不变性的方程形如
\[
    \partial_0 \psi = b^i \partial_i \psi + C \psi,
\]
其中$b^i$和$C$是常数,$C$可以是一个旋量矩阵。
由于我们同时还要求旋转不变性,$C$只能是一个标量。显然,这个方程中所有含有导数的项加在一起必然得到一个旋量,即
\[
    \partial_0 \psi - b^i \partial_i \psi = \text{a covariant term} = C \psi.
\]
梯度算符是矢量,按照\eqref{eq:vector-is-spin-tensor},我们可以写出作用在魏尔旋量上的导数算符
\begin{equation}
    \partial_{a \dot{b}} = \partial_\nu \sigma^\nu_{a \dot{b}}.
\end{equation}
从而魏尔旋量的梯度就是
\[
    \partial_{a \dot{b}} \psi^{\dot{b}} = \partial_\nu \sigma^\nu_{a \dot{b}} \psi^{\dot{b}}
\]
和
\[
    \partial^{\dot{a} b} \psi_b = \partial_\mu (\sigma^\mu)^{\dot{a} b} \psi_b.
\]
两个表达式中,$\sigma$的指标都一个带点一个不带点,这是为了保证梯度算符的协变性,因为矢量是一个左手旋量和一个右手旋量直积的结果。
$\partial_0 \psi - b^i \partial_i \psi$应该能够写成以上两种旋量梯度的函数。由于以上两种旋量梯度带一个指标,而$\partial_0 \psi - b^i \partial_i \psi$也是单指标对象,显然两者只应该差一个倍数。(当然,也可以将一个旋量张量参数和旋量梯度做缩并,但这样就没有旋转不变性了)这个倍数可以被吸收到$C$中。
从而运动方程形如
\[
    \mathrm{grad} \psi = C \psi.
\]
然而,注意到左手旋量的梯度是一个右手旋量,右手旋量的梯度是一个左手旋量,因此以上的方程会让一个右手旋量的各个分量等于一个左手旋量的各个分量,从而破坏了洛伦兹协变性。
消除这个矛盾的唯一可能就是让$C=0$,于是左手旋量的运动方程为
\[
    \partial_\mu (\sigma^\mu)^{\dot{a} b} \psi_b = 0,
\]
右手旋量的运动方程为
\[
    \partial_\nu \sigma^\nu_{a \dot{b}} \psi^{\dot{b}} = 0.
\]
由定义,$(\sigma^\mu)_{a \dot{b}}$的分量矩阵就是$\sigma$矩阵,而$(\sigma^\mu)^{\dot{a} b}$的分量矩阵则需要通过指标升降关系
\[
    (\sigma^\mu)^{\dot{a} b} = (\epsilon^{ac} (\sigma^\mu)_{c \dot{d}} \epsilon^{\dot{b} \dot{d}})^*
\]
得到。定义$(\sigma^\mu)^{\dot{a} b}$的分量矩阵为$\bar{\sigma}^\mu$,通过计算可以发现
\begin{equation}
    \bar{\sigma}^0 = \sigma^0 = I, \quad \bar{\sigma}^i = - \sigma^i,
\end{equation}
于是可以使用矩阵形式写出运动方程:
\begin{equation}
    \partial_0 \psi \pm \sigma^i \partial_i \psi = 0 .
    \label{eq:weyl-eq}
\end{equation}
负号为左手旋量,正号为右手旋量。

接下来讨论狄拉克旋量的运动方程。由于它同时含有一个左手旋量和一个右手旋量,可以让这两个旋量之间有线性的相互作用(从而,关于狄拉克旋量的方程仍然是线性的)。
洛伦兹不变性的要求意味着,唯一可能的方程形式如下:
\begin{equation}
    \begin{aligned}
        (\partial_0 + \sigma^i \partial_i) \psi_R = - \ii m \psi_L, \\
        (\partial_0 - \sigma^i \partial_i) \psi_L = - \ii m \psi_R.
    \end{aligned}
    \label{eq:interacting-weyl-eq}
\end{equation}
其中我们为了节省符号,使用$\psi_L$和$\psi_R$分别代表狄拉克旋量$\psi$的左手部分和右手部分。
当然,$m=0$时就狄拉克旋量的运动方程就退化为了一对完全无关的左手旋量和右手旋量。
这也就是实际计算时没有必要单独讨论魏尔旋量的原因。
容易证明
\[
    \partial_\mu \partial^\mu \psi_L = - m^2 \psi_L, \quad \partial_\mu \partial^\mu \psi_R = - m^2 \psi_R,
\]
也就是说狄拉克旋量也满足克莱因-高登方程。从而为了得到物理解,我们要求$m \geq 0$。
为了将\eqref{eq:interacting-weyl-eq}写成更加紧凑的形式,引入$\gamma$矩阵%
\footnote{这里给出的$\gamma$矩阵的形式实际上只是一种可能性。我们称这种将狄拉克旋量的左手部分和右手部分分开处理(或者等价地说,狄拉克旋量的基或者只含有左手旋量,或者只含有右手旋量),并且按照\eqref{eq:gamma-matrix}引入$\gamma$矩阵的方式为\textbf{魏尔表象}。也可以取其它的旋量基,从而获得其它表象。}
\begin{equation}
    \gamma^\mu = \pmqty{0 & \sigma^\mu \\ \bar{\sigma}^\mu & 0}, \gamma^5 = \pmqty{I & 0 \\ 0 & -I},
    \label{eq:gamma-matrix}
\end{equation}
从而
\begin{equation}
    \gamma_\mu = \eta_{\mu \nu} \gamma^\nu = \pmqty{ 0 & \bar{\sigma}^\mu \\ \sigma^\mu & 0 },
\end{equation}
则得到
\begin{equation}
    (\ii \gamma^\mu \partial_\mu - m) \psi = 0.
    \label{eq:dirac-eq}
\end{equation}
这就是\textbf{狄拉克方程}。如前所述,它能够推导出克莱因-高登方程,并且在$m$取零时退化为一个左手旋量场和一个右手旋量场的简单组合。

现在我们尝试拼凑一个拉氏量出来。由于狄拉克场的运动方程是一阶的而它又是一个复场,需要通过$\psi$的复共轭拼凑出一个在\eqref{eq:el-eq}意义下“独立”的场,然后构造一个同时包含$\psi$及其复共轭的拉氏量,由这个拉氏量给出关于$\psi$和它的复共轭的两个方程,并且这两个方程必须等价。
现在我们尝试寻找和\eqref{eq:dirac-eq}等价,但是仅仅包含其复共轭的方程。
由\eqref{eq:dirac-eq}取共轭转置%
\footnote{这里的共轭转置是指场的共轭转置,不需要对作用在场上的算符$\partial_\mu$取共轭转置。}
,得到
\[
    (-\ii) \partial_\mu \psi^\dagger (\gamma^\mu)^\dagger - m \gamma^\dagger = 0.
\]
容易验证$\gamma$矩阵具有下面的性质:
\[
    (\gamma^0)^\dagger = \gamma^0, \quad (\gamma^i)^\dagger = - \gamma^i, 
\]
以及
\[
    \gamma^i \gamma^0 = - \gamma^0 \gamma^i,
\]
我们发现
\[
    \ii \partial_0 \psi^\dagger \gamma^0 \gamma^0 + \ii \partial_i \psi^\dagger \gamma^0 \gamma^i + m \psi^\dagger \gamma^0 = 0.
\]
定义
\begin{equation}
    \bar{\psi} = \psi^\dagger \gamma^0,
\end{equation}
则其运动方程为
\begin{equation}
    \ii \partial_\mu \bar{\psi} \gamma^\mu + m \bar{\psi} = 0.
    \label{eq:cog-dirac-eq}
\end{equation}
这正是我们需要的另一个运动方程。
我们会发现,拉氏量
\begin{equation}
    \mathcal{L} = \bar{\psi} (\ii \gamma^\mu \partial_\mu - m) \psi
    \label{eq:dirac-lagrangian}
\end{equation}
分别对$\psi$和$\bar{\psi}$应用\eqref{eq:el-eq},就得到\eqref{eq:dirac-eq}和\eqref{eq:cog-dirac-eq}。同时容易验证这是一个洛伦兹标量。这表明\eqref{eq:dirac-lagrangian}确实就是狄拉克场的拉氏量。

从\eqref{eq:dirac-lagrangian}可以推导出对应的哈氏量。计算共轭动量可以得到
\begin{equation}
    \pi = \ii \psi^\dagger,
\end{equation}
从而能够得到哈氏量密度
\begin{equation}
    \mathcal{H} = - \ii \bar{\psi} \gamma^i \partial_i \psi  + m \bar{\psi} \psi = - \pi \gamma^0 \gamma^i \partial_i \psi - \ii m \pi \gamma^0 \psi.
\end{equation}

以上我们都设$\psi$的各分量由一个左手旋量和一个右手旋量拼凑而成。这称为\textbf{手性基}或者\textbf{魏尔表象}。
在手性基当中,拉氏量的质量项为
\[
    - \bar{\psi} m \psi = - m (\chi_L^\dagger \xi_R + \xi_R^\dagger \chi_L),
\]
这不是一个对角化的二次型。若做分量变换
\begin{equation}
    \psi' = \frac{1}{\sqrt{2}} \pmqty{-1 & 1 \\ 1 & 1} \psi,
\end{equation}
质量项就被对角化了。我们称这种分量选取为\textbf{质量基}或者\textbf{狄拉克表象}。
容易计算出质量基下
\begin{equation}
    \gamma^0 = 
\end{equation}

\subsubsection{矢量场的布洛卡方程}

% TODO:$(\partial_\mu A^\mu)^2$
由于自由场导数阶数的限制,出现在拉氏量中的只能是$A^\mu$和$\partial^\nu A^\mu$构成的一次或二次不变量。当然,实际上也可以出现$\partial_\mu A^\nu$或者$\partial_\mu A_\mu$这种,但因为它们都可以使用$\partial^\nu A^\mu$表示出来,故没有必要考虑它们。
只含有$A^\mu$二次不变量为$A^\mu A_\mu$,没有一次不变量;只含有$\partial^\mu A^\nu$的一次不变量是它自我缩并得到的$\partial^\mu A_\mu$,二次的不变量是两个$\partial^\mu A^\nu$缩并得到的$\partial^\mu A^\nu \partial_\mu A_\nu$和$\partial^\mu A^\nu \partial_\nu A_\mu$。
由于参数都是标量,$\partial^\mu A^\nu$不能和参数缩并,也不能和$A^\mu$缩并($C^\nu A^\mu \partial_\nu A_\mu$要求参数是矢量,$A^\mu A^\nu \partial_\mu A_\nu$是三次项),因此我们得到了所有可能的不变量。
从而拉氏量形如
\[
    \mathcal{L} = C_1 A^\mu A_\mu + C_2 \partial^\mu A_\mu + C_3 \partial^\mu A^\nu \partial_\mu A_\nu + C_4 \partial^\mu A^\nu \partial_\nu A_\mu.
\]
代入\eqref{eq:el-eq}可以看出,$C_2$项在运动方程中不会引入任何项,故略去。
于是
\[
    \mathcal{L} = C_1 A^\mu A_\mu + C_3 \partial^\mu A^\nu \partial_\mu A_\nu + C_4 \partial^\mu A^\nu \partial_\nu A_\mu.
\]
代入\eqref{eq:el-eq},得到
\begin{equation}
    \partial_\mu (C_3 \partial^\mu A^\nu +  C_4 \partial^\nu A^\mu) = C_1 A^\nu.
    \label{eq:vector-motion-eq}
\end{equation}
我们首先考虑$C_3 = - C_4$时的特殊情况。重新定义各系数,使得
\begin{equation}
    \mathcal{L} = - \frac{1}{2} \partial^\mu A^\nu \partial_\mu A_\nu + \frac{1}{2} \partial^\mu A^\nu \partial_\nu A_\mu + \frac{m^2}{2} A_\mu A^\mu,
\end{equation}
对应的,
\begin{equation}
    \partial_\mu (\partial^\mu A^\nu - \partial^\nu A^\mu) + m^2 A^\nu = 0.
    \label{eq:proca-eq}
\end{equation}
常定义
\begin{equation}
    F^{\mu \nu} = \partial^\mu A^\nu - \partial^\nu A^\mu,
\end{equation}
于是就有
\begin{equation}
    \mathcal{L} = - \frac{1}{4} F_{\mu \nu} F^{\mu \nu} + \frac{1}{2} m^2 A_\mu A^\mu.
    \label{eq:proca-lagrangian}
\end{equation}
现在回到一般情况。我们指出这样一个结论:无论$C_3,C_4$取什么值,对应的场$A^\mu$都可以和$C_3 = - C_4$时的某个场${A'}^\mu$建立一一对应。
% TODO:证明
因此布洛卡方程\eqref{eq:proca-eq}就不失一般性地描写了所有的矢量场的运动方程。

\eqref{eq:proca-eq}在$m \neq 0$时可以推导出克莱因-高登方程。注意到
\[
    m^2 \partial_\nu A^\nu = \partial_\nu \partial_\mu \partial^\nu A^\mu - \partial_\mu \partial^\mu \partial_\nu A^\nu = 0,
\]
于是
\begin{equation}
    \partial_\mu A^\mu = 0.
    \label{eq:lorentz-gauge}
\end{equation}
回代入\eqref{eq:proca-eq},发现其左边第二项为零,于是
\[
    \partial_\mu \partial^\mu A^\nu + m^2 A^\nu = 0.
\]
于是\eqref{eq:proca-eq}就约化成了\eqref{eq:lorentz-gauge}和四个克莱因-高登方程。
而当$m=0$时,运动方程在规范变换
\begin{equation}
    A^\mu \longrightarrow {A'}^\mu = A^\mu + \partial^\mu \varphi
\end{equation}
下不变。这意味着矢量场$A^\mu$的四个自由度实际上是多余的。%
\footnote{显然,只要选定了一个$\varphi$,同一个时间点上的$A^\mu$和${A'}^\mu$之间必定可以建立起一一对应关系。形象地说,不同$\varphi$对应的$A'$的运行轨迹相互平行,因此只需要其中一条轨迹就能够确定所有轨迹。选取特定的一条轨迹就是选取一个规范。
规范自由度——也就是决定“实际的轨道是哪一条”的自由度——是一个隐藏的额外自由度。
这里的情况和对称性自发破缺有点类似,在后者中,隐藏的自由度是序参量。不同的隐藏的额外自由度取值将系统的态空间分成了互不相交的分支。
可以认为规范自由度不是物理的自由度,也就是说它仅仅出现在拉氏量中,而规范自由度取值不同的状态在希尔伯特空间中应该被认为是同样的状态。
选取一个规范意味着先假定规范自由度取值不同的状态真的是不一样的,然后取状态空间中的一个分支。}%
无论$\partial_\mu A^\mu$是什么,总可以找到一个$\varphi$使得
\[
    \partial_\mu \partial^\mu \varphi = - \partial_\mu A^\mu,
\]
从而对应的有
\[
    \partial_\mu {A'}^\mu = 0.
\]
于是我们不失一般性地强行要求\eqref{eq:lorentz-gauge}对$m=0$时的矢量场成立。这称为选取了\textbf{洛伦兹规范}。选取了洛伦兹规范意味着,实际的场自由度只有三个。知道了$A$的三个分量就可以计算出第四个。
当然,这不是唯一的规范选取方式。例如可以直接要求$A^0 = 0$,称为\textbf{辐射规范}。
选取洛伦兹规范的好处在于,方程\eqref{eq:lorentz-gauge}是洛伦兹协变的,因此在做量子化时能够直接套用正则量子化关系\eqref{eq:symmetry-commutator}而不必担心场方程不是洛伦兹协变而产生的修正。

矢量场的共轭动量为
\[
    \pi_\mu = \partial_\mu A^0 - \partial^0 A_\mu,
\]
或者写成
\begin{equation}
    \pi^\mu = \partial^\mu A^0 - \partial^0 A^\mu.
\end{equation}
注意到$\pi^0 = 0$,因此可以只讨论其空间部分$\vb*{\pi}$。
在质量$m$不为零时场没有规范不变性,可以直接做计算得到
\begin{equation}
    A^0 = - \frac{1}{m^2} \div{\vb*{\pi}},
\end{equation}
以及
\begin{equation}
    \partial_0 A^0 = - \partial_i A^i = - \div{\vb*{A}},
\end{equation}
哈氏量为
\begin{equation}
    \mathcal{H} = \frac{1}{2} \vb*{\pi}^2 + \frac{1}{2m^2} (\div{\vb*{\pi}})^2 + \frac{1}{2} (\curl{\vb*{A}})^2 + \frac{1}{2} m^2 \vb*{A}^2.
\end{equation}
$m$出现在了分母中,这意味着无质量的场需要额外处理。

现在来处理无质量的场。其运动方程为
\[
    \partial_\mu (\partial^\mu A^\nu - \partial^\nu A^\mu) = 0.
\]
我们施加洛伦兹规范。当然也可以选取别的规范,但这可能会破坏洛伦兹协变性,从而导致我们得到的哈密顿动力学实际上是带有约束的,从而给之后做量子化带来麻烦。
此时运动方程为
\begin{equation}
    \partial_\mu \partial^\mu A^\nu = 0.
    \label{eq:massless-vector-eq}
\end{equation}
拉氏量\eqref{eq:proca-lagrangian}直接导出的不是这个方程,于是我们使用能够直接导出\eqref{eq:massless-vector-eq}的拉氏量
\begin{equation}
    \mathcal{L} = - \frac{1}{4} F_{\mu \nu} F^{\mu \nu} - \frac{1}{2} (\partial_\mu A^\mu)^2.
\end{equation}
在给定了洛伦兹规范的前提下,这个拉氏量实际上就是$m=0$的\eqref{eq:proca-lagrangian}。
此时
\begin{equation}
    \pi^0 = -\partial_\mu A^\mu, \quad \pi^i = \partial^i A^0 - \partial^0 A^i.
\end{equation}
当然,由洛伦兹规范,$\pi^0$就是零,不过我们完全可以算出哈密顿量之后再施加洛伦兹规范。
哈密顿量为 % TODO:这一部分似乎不需要太多笔墨,反正量子化的时候都是重新算的 关键之处在于哈密顿量和规范是有关的
\begin{equation}
    \mathcal{H} = 
\end{equation}

\subsubsection{退化到非相对论情况}

我们将讨论克莱因-高登方程的退化形式。旋量场和标量场由于也服从克莱因-高登方程,没有必要单独考虑——它们多出来的自由度可以使用其它方式,如自旋等,引入。
实际上我们讨论的应该是复的克莱因-高登方程,因为旋量场是复的,但本节的讨论并不会用到场是不是复的这个信息。

首先我们注意到一个事实:时谐波
\begin{equation}
    \phi = \ee^{- \ii m t}
    \label{eq:lowest-energy}
\end{equation}
是\eqref{eq:klein-gordon-eq}的解,并且它的能量最低,就是零。(代入哈氏量可得)因此,能量不高的场只是微微偏离\eqref{eq:lowest-energy},我们设其为
\begin{equation}
    \phi(\vb*{x}, t) = \psi(\vb*{x}, t) \ee^{- \ii m t},
    \label{eq:low-energy-ansatz}
\end{equation}
则
\[
    (\partial_\mu \partial^\mu + m^2) \phi = \ee^{- \ii m t} (-2 \ii m \partial_t \psi + \partial_t^2 \psi - \laplacian{\psi}).
\]
由于$\phi$只是略微偏离\eqref{eq:lowest-energy}%
\footnote{需要注意的是这个说法字面上实际上是不严谨的。$\hat{\phi}$是一个算符,它包含了所有可能的$\phi$的取值,不应该“只是略微偏离\eqref{eq:lowest-energy}”。
然而,$\hat{\phi}$的本征态中非常偏离\eqref{eq:lowest-energy}的那部分模式在我们的低能有效理论中并不会被涉及到。
换而言之,我们关心的那部分$\ket{\phi}$只是略微偏离\eqref{eq:lowest-energy},因此认为$\hat{\phi}$只是略微偏离\eqref{eq:lowest-energy}并不会显著改变我们的理论的行为。
}%
,可以预期$\psi$的时间部分振荡不会特别明显,于是取近似
\[
    \partial_t^2 \psi \ll m \partial_t \psi,
\]
就得到
\begin{equation}
    \ii \partial_t \psi + \frac{1}{2m} \laplacian{\psi} = 0.
    \label{eq:schodinger-eq}
\end{equation}
\eqref{eq:schodinger-eq}称为\textbf{薛定谔场}的运动方程。容易看出它不是洛伦兹协变的,这是理所当然的,因为它描述的现象发生在低能近似下,此时伽利略对称性就足够了。
薛定谔场是复的,无论$\phi$是不是复场,因为拟设\eqref{eq:low-energy-ansatz}引入了一个复数因子。

方程\eqref{eq:schodinger-eq}是以下拉氏量%
\footnote{$\grad{\psi}^\dagger \cdot \grad{\psi}$代表将两个梯度算符做缩并,行向量$\psi^\dagger$和列向量$\psi$相乘,即$\partial_i \psi^\dagger \partial^i \psi$。混合使用不变量记号和矩阵记号是因为我们并不知道$\psi$的内部结构,只知道$\psi^\dagger \psi$是标量,因此把$\psi$当成一个整体,好像一个标量一样,来做计算。}
\begin{equation}
    \mathcal{L} = \frac{\ii}{2} \left( \psi^\dagger \dot{\psi} - \psi \dot{\psi}^\dagger \right) - \frac{1}{2m} \grad{\psi^\dagger} \cdot \grad{\psi}
    \label{eq:schodinger-lagrangian}
\end{equation}
的运动方程。把$\psi$和$\psi^\dagger$看成两个独立的场,分别应用\eqref{eq:el-eq},就能够得到\eqref{eq:schodinger-eq}和其共轭转置。

容易看出,
\[
    \pi(\psi) = \pdv{\mathcal{L}}{\dot{\psi}} = \frac{\ii}{2} \psi^\dagger, \quad \pi(\psi^\dagger) = \pdv{\mathcal{L}}{\dot{\psi}^\dagger} = - \frac{\ii}{2} \psi^\top,
\]
从而可以计算出
\begin{equation}
    \mathcal{H} = \frac{1}{2m} \grad{\psi^\dagger} \cdot \grad{\psi}.
\end{equation}
这个哈氏量中出现了$\pi$的导数,处理起来会比较麻烦。为了规避这些麻烦,我们将不再讨论经典的哈密顿动力学,而直接开始做量子化。

电荷密度为
\begin{equation}
    \rho(\vb*{r}) = q \psi^\dagger(\vb*{r}) \psi(\vb*{r}),
\end{equation}
而且电流密度为
\begin{equation}
    \vb*{j}(\vb*{r}) = \frac{1}{2m\ii} (\psi^\dagger(\vb*{r}) \grad{\psi}(\vb*{r}) - \psi(\vb*{r}) \grad{\psi}^\dagger(\vb*{r}))
\end{equation}

\subsection{自由场的量子化}\label{sec:quantization-of-free-fields}

\subsubsection{基本步骤}

% TODO:本节只写了实场的量子化,没有写复场。标量场和矢量场可以是实的也可以是复的,旋量场一定是复的。复场实际上含有两个实场,因此它可以含有两种产生湮灭算符,这就是反物质的来源。
% 一种可能的想法是,认为“场的振幅”等经典概念在量子情况下是写在态空间中的而不是写在算符场上的。

本节由于需要讨论算符场的演化,取海森堡绘景。

本节将讨论怎样将量子化关系\eqref{eq:symmetry-commutator}或\eqref{eq:antisymmetry-commutator}施加到场上,以及场算符怎样创造出多粒子态。
实际上,另一种导出量子场论的方法就是通过将洛伦兹群作用在单粒子态空间上,分析其不可约表示,然后得到不同种类的粒子,接着使用和本文正好相反的顺序,使用单粒子态拼凑场算符,最后获得场的运动方程。
我们的方案则正好相反:先获得场,然后从场得到多粒子态。

原则上,我们可以使用\autoref{sec:c-a-operator-from-field}中提到的方法,将自由场算符看成关于位置的产生湮灭算符的叠加,并施加正则对易关系\eqref{eq:symmetry-commutator}或反对易关系\eqref{eq:antisymmetry-commutator},从而将自由场量子化,并且得到它对应的多粒子态。
不过,由于各种场的哈密顿量中都含有空间导数,直接这么做得到的哈密顿量中将会出现产生湮灭算符的空间导数,不方便使用。
注意到坐标表象和动量表象之间的切换公式\eqref{eq:x-p-trans}实际上就是傅里叶变换,可以考虑使用关于动量的产生湮灭算符。

在壳的$\ket{p}$可以使用$\vb*{p}$标记,因此还是使用$\vb*{p}$而不是$p^\mu$标记对应的产生湮灭算符,并使用
\[
    \int \dd[3]{\vb*{p}} / (2 E_{\vb*{p}})
\]
作为积分测度。
首先讨论实场。
场算符在某处的值$\hat{\phi}(\vb*{x}, t)$本身可以看成关于位置单粒子态$\ket{\vb*{x}, \sigma}$(其中$\sigma$表示内禀自由度)的产生湮灭算符的线性叠加。
设算符$\hat{\alpha}_{\vb*{p}, \sigma}$和$\hat{\alpha}^\dagger_{\vb*{p}, \sigma}$是单粒子态$\ket{p, \sigma}$相关的湮灭算符和产生算符,则场算符可以展开为
\begin{equation}
    \hat{\phi} (x) \propto \sum_\sigma \int \frac{\dd[3]{\vb*{p}}}{2 E_{\vb*{p}} (2\pi)^{3/2} } \left( \hat{\alpha}^\dagger_{\vb*{p}, \sigma} \ee^{\ii p_\mu x^\mu} + \hat{\alpha}_{\vb*{p}, \sigma} \ee^{- \ii p_\mu x^\mu} \right) e_\sigma,
    \label{eq:expanding-field-operator-relativity}
\end{equation}
其中$\exp(\ii p_\mu x^\mu)$来自在四维坐标和四维动量之间切换的变换矩阵,$e_\sigma$指的是基。
% TODO:严格证明,不过多半鸽了

由于$\hat{\alpha}$和$\hat{\alpha}^\dagger$彼此不能通过任何算符相互转换(取共轭转置是一个元算符),实际上也可以使用$\hat{\alpha}$和$\hat{\alpha}^\dagger$表示出$\hat{\pi}$。
按照\eqref{eq:def-pi},并且考虑到$\mathcal{L}$关于$\phi$及其导数只有二次项,$\pi$是$\phi$及其导数的线性叠加,从而$\hat{\pi}$也可以使用$\hat{\alpha}$和$\hat{\alpha}^\dagger$线性表示,且形如
\[
    \hat{\pi} \sim \int \dd[3]{\vb*{p}} f(\vb*{p}) (c_1 \hat{\alpha}^\dagger_{\vb*{p}, \sigma} \ee^{\ii p_\mu x^\mu} + c_2 \hat{\alpha}_{\vb*{p}, \sigma} \ee^{- \ii p_\mu x^\mu})
\]
因此,如果选用了方案\eqref{eq:symmetry-commutator},那么我们就得到了满足\eqref{eq:commutator-of-ca-op}的产生湮灭算符。
我们称使用这样的场算符得到的粒子为\textbf{玻色子};反之如果使用了\eqref{eq:antisymmetry-commutator},那么就得到了满足\eqref{eq:anticommutator-of-ca-op}的产生湮灭算符,我们称这样得到的粒子为\textbf{费米子}。

考虑到\eqref{eq:relativity-p},我们有
\[
    \ket{p, \sigma} = \sqrt{2 E_{\vb*{p}}} \ket{\vb*{p}, \sigma},
\]
于是设$\hat{a}_{\vb*{p}, \sigma}$和$\hat{a}^\dagger_{\vb*{p}, \sigma}$为单粒子态$\ket{\vb*{p}, \sigma}$对应的产生湮灭算符,则有
\[
    (\hat{\alpha}_{\vb*{p}, \sigma})^\dagger = \sqrt{2E_{\vb*{p}}} \hat{a}_{\vb*{p}, \sigma}^\dagger,
\]
这样场算符的展开式就是
\[
    \hat{\phi}(x) \propto \sum_\sigma \int \frac{\dd[3]{\vb*{p}}}{\sqrt{(2\pi)^3 2 E_{\vb*{p}}}} \left( \hat{a}^\dagger_{\vb*{p}, \sigma} \ee^{\ii p_\mu x^\mu} + \hat{a}_{\vb*{p}, \sigma} \ee^{- \ii p_\mu x^\mu} \right) e_\sigma,
\]
或者既然我们已经不再要求洛伦兹协变性了,可以使用三维矢量更加清晰地写出
\begin{equation}
    \hat{\phi}(\vb*{x}, t) \propto \sum_\sigma \int \frac{\dd[3]{\vb*{p}}}{\sqrt{(2\pi)^3 2 E_{\vb*{p}}}} \left( \hat{a}^\dagger_{\vb*{p}, \sigma} \ee^{- \ii \vb*{p} \cdot \vb*{x} + \ii E_{\vb*{p}} t} + \hat{a}_{\vb*{p}, \sigma} \ee^{\ii \vb*{p} \cdot \vb*{x} - \ii E_{\vb*{p}} t} \right) e_\sigma. 
    \label{eq:expanding-field-operator}
\end{equation}
由于三种场都服从克莱因-高登方程,将\eqref{eq:expanding-field-operator}代入\eqref{eq:klein-gordon-eq}会发现$E_{\vb*{p}}$和$\vb*{p}$正好服从质壳关系\eqref{eq:mass-shell}。

展开式\eqref{eq:expanding-field-operator-relativity}和\eqref{eq:expanding-field-operator}有各自的好处。
\eqref{eq:expanding-field-operator-relativity}给出的产生湮灭算符以及它们产生的单粒子态是洛伦兹协变的,但是在处理对易关系的时候会略有复杂,因为此时产生湮灭算符的对易关系必定也是协变的,因此必须指定
\[
    \comm*{\hat{\alpha}_{\vb*{p}}}{\hat{\alpha}^\dagger_{\vb*{p}'}} \sim E_{\vb*{p}} \delta^3 (\vb*{p} - \vb*{p}')
\]
这样的对易关系,或者类似的反对易关系;当然,这并没有违背条件\eqref{eq:commutator-of-ca-op},因为此时使用的积分测度是$\int \dd[3]{\vb*{p}} / (2 E_{\vb*{p}})$。
\eqref{eq:expanding-field-operator}给出的产生湮灭算符以及它们产生的单粒子态不是洛伦兹协变的,但是可以简化对易关系以及归一化时使用的积分测度。
例如通常我们选取积分测度为$\int \dd[3]{\vb*{p}}$,那么就需要指定
\[
    \comm*{\hat{a}_{\vb*{p}}}{\hat{a}^\dagger_{\vb*{p}'}} = \delta^3 (\vb*{p} - \vb*{p}'),
\]
或者类似的反对易关系。而如果指定
\[
    \comm*{\hat{a}_{\vb*{p}}}{\hat{a}^\dagger_{\vb*{p}'}} = (2\pi)^3 \delta^3 (\vb*{p} - \vb*{p}'),
\]
此时只需要始终使用积分测度
\[
    \int \frac{\dd[3]{\vb*{p}}}{(2\pi)^3}
\]
即可。
两种展开式都是傅里叶变换,因此都能够消除哈密顿量中的导数。
此外由傅里叶变换的性质,并注意到$E_{\vb*{p}}$和$E_{-\vb*{p}}$是一回事,我们有
\begin{equation}
    \hat{a}_{-\vb*{p}, \sigma} = \hat{a}_{\vb*{p}, \sigma}^\dagger.
\end{equation}

本文为了追求等式简洁,通常使用方案\eqref{eq:expanding-field-operator},又为了避免反复使用根号,通常使用含有$(2\pi)^3$的写法,也就是设
\begin{equation}
    \hat{\phi}(\vb*{x}, t) = \sum_\sigma \int \frac{\dd[3]{\vb*{p}}}{(2\pi)^3} \frac{1}{\sqrt{2 E_{\vb*{p}}}} \left( \hat{a}^\dagger_{\vb*{p}, \sigma} \ee^{- \ii \vb*{p} \cdot \vb*{x} + \ii E_{\vb*{p}} t} + \hat{a}_{\vb*{p}, \sigma} \ee^{\ii \vb*{p} \cdot \vb*{x} - \ii E_{\vb*{p}} t} \right) e_\sigma. 
    \label{eq:field-operator-fourier}
\end{equation}

我们有
\[
    \ket{p, \sigma} = \sqrt{2 E_{\vb*{p}}} \hat{a}^\dagger_{\vb*{p}, \sigma} \ket{0}.
\]
在某一个给定的时间,将$\hat{\phi}(\vb*{x}, t)$作用在真空态上得到
\[
    \begin{aligned}
        \hat{\phi}(\vb*{x}, t) \ket{0} &= \sum_\sigma \int \frac{\dd[3]{\vb*{p}}}{(2\pi)^3} \frac{1}{\sqrt{2 E_{\vb*{p}}}} \hat{a}^\dagger_{\vb*{p}, \sigma} \ee^{- \ii \vb*{p} \cdot \vb*{x} + \ii E_{\vb*{p}} t} e_{\sigma} \ket{0} \\
        &= \sum_\sigma \int \frac{\dd[3]{\vb*{p}}}{(2\pi)^3} \frac{1}{2 E_{\vb*{p}}} \ee^{- \ii \vb*{p} \cdot \vb*{x} + \ii E_{\vb*{p}} t} e_{\sigma} \ket{p, \sigma}.
    \end{aligned}
\]
在非相对论情况下,也就是$\vb*{p}^2$相对$m^2$来说很小的情况下,$E_{\vb*{p}}$几乎就是$m$,此时考虑到\eqref{eq:x-p-trans},我们会发现
\[
    \hat{\phi}^\sigma (\vb*{x}, t) \ket{0} \propto \ket{\vb*{x}(t), \sigma},
\]
也就是说一个海森堡绘景中的场算符导致了一个薛定谔绘景中的用位置标记的单粒子态。
或者说
\[
    \mel{\vb*{x}}{\hat{\phi}^\sigma (\vb*{x}, t)}{0} \propto \text{single particle wavefunction}.
\]
这也就是场算符有时候会被粗略地当成“相对论量子力学中的波函数”的原因。

判断应该使用对易关系来量子化场还是应该使用反对易关系来量子化场应当遵守几个条件:
\begin{itemize}
    \item 非平凡性。哈密顿量不应该给出平凡的结果。
    \item 因果性。在某一个时空点施加相互作用只应该产生局域的影响。特别的,在一个时空点做测量不应该对与之间隔(指的是闵可夫斯基时空中的“距离”)为正的时空点产生影响。
    \item 能量正定性。哈密顿量应该可以写成产生湮灭算符的正定二次型,以避免能量无限下降。
\end{itemize}

以上给出的步骤完全描述了场算符的量子化过程。这种使用傅里叶变换得到对角化的哈密顿量的方式有时也称为\textbf{正则量子化},因为它是算符的正则量子化(即施加对易或反对易关系)之后立刻可以完成的。

关于与场算符配套的真空态要说一句:在自由场下,无论采取哪种绘景,真空态$\ket{0}$或者没有时间演化,或者时间演化只是乘上一个复数因子。这是因为真空态一定是哈密顿算符的本征态。
此外,本节采用的量子化方案也体现出了一个重要的物理图像:具有确定能量$E$的粒子在经典极限下就对应着以圆频率$E$振荡的场。

最后,以上将场做傅里叶变换以消去运动方程中的导数的做法在经典情况下当然也适用。容易看出,动量为$\vb*{p}$的粒子模式的经典极限就是波矢为$\vb*{p}$的平面波;相应的,位置为$\vb*{r}$的粒子模式的经典极限就是
\[
    \phi(\vb*{r}') = \delta(\vb*{r}' - \vb*{r}).
\]
即使在经典场中,也存在动量和位置不能同时确定的现象。场的量子化带来的不是动量和位置不能同时确定,而是场的振幅是离散化的——经典情况下,平面波的振幅可以任意变化,而量子情况下,$\hat{\phi}$(或者别的场算符)的本征值是离散的,此时才能够良定义“粒子”。
% 振幅能够连续变动的平面波实际上对应着动量表象下的相干态,因为振幅能够连续变动的平面波应该是湮灭算符的本征态,也就是相干态
% 经典的波实际上对应着量子的粒子的一个热系综

\subsubsection{实标量场}

无需额外考虑标量场的基,于是直接取
\begin{equation}
    \hat{\phi}(\vb*{x}, t) = \int \frac{\dd[3]{\vb*{p}}}{(2\pi)^3} \frac{1}{\sqrt{2 E_{\vb*{p}}}} \left( \hat{a}^\dagger_{\vb*{p}} \ee^{ - \ii \vb*{p} \cdot \vb*{x} + \ii E_{\vb*{p}} t} + \hat{a}_{\vb*{p}} \ee^{ \ii \vb*{p} \cdot \vb*{x} - \ii E_{\vb*{p}} t} \right),
    \label{eq:expanding-klein-gordon-field}
\end{equation}
显然它是\eqref{eq:klein-gordon-eq}的一个解。相应的使用\eqref{eq:klein-gordon-pi},有
\begin{equation}
    \hat{\pi}(\vb*{x}, t) = \int \frac{\dd[3]{\vb*{p}}}{(2\pi)^3} \  \ii \sqrt{\frac{E_{\vb*{p}}}{2}} \left( \hat{a}^\dagger_{\vb*{p}} \ee^{ - \ii \vb*{p} \cdot \vb*{x} + \ii E_{\vb*{p}} t} - \hat{a}_{\vb*{p}} \ee^{ \ii \vb*{p} \cdot \vb*{x} - \ii E_{\vb*{p}} t} \right)
\end{equation}
共轭动量不是洛伦兹协变的。这并不让人意外,因为其定义和时间维的选取有关。
计算得到
\[
    \hat{H} = \int \frac{\dd[3]{\vb*{p}}}{(2\pi)^3} \frac{1}{2} E_{\vb*{p}} (\hat{a}_{\vb*{p}}^\dagger \hat{a}_{\vb*{p}} + \hat{a}_{\vb*{p}} \hat{a}^\dagger_{\vb*{p}}).
\]

正则对易关系\eqref{eq:symmetry-commutator}施加到标量场$\hat{\phi}$上。
通过计算可以得知,这等价于
\begin{equation}
    \comm*{\hat{a}_{\vb*{p}}}{\hat{a}^\dagger_{\vb*{p}'}} = (2\pi)^3 \delta^3 (\vb*{p} - \vb*{p}'), \quad \comm*{\hat{a}_{\vb*{p}}}{\hat{a}_{\vb*{p}'}} = 0.
    \label{eq:quantization-scalar}
\end{equation}
相应的,反对易关系等价于
\[
    \acomm*{\hat{a}_{\vb*{p}}}{\hat{a}^\dagger_{\vb*{p}'}} = (2\pi)^3 \delta^3 (\vb*{p} - \vb*{p}'), \quad \acomm*{\hat{a}_{\vb*{p}}}{\hat{a}_{\vb*{p}'}} = 0.
\]
将反对易关系代入哈密顿量表达式会导致哈密顿量变成常数,因此这是平凡解,舍去。
将对易关系代入哈密顿量的表达式,得到
\begin{equation}
    \hat{H} = \int \frac{\dd[3]{\vb*{p}}}{(2\pi)^3} E_{\vb*{p}} \left(\hat{a}_{\vb*{p}}^\dagger \hat{a}_{\vb*{p}}  + \frac{1}{2} \comm*{\hat{a}_{\vb*{p}}}{\hat{a}^\dagger_{\vb*{p}}} \right).
    \label{eq:hamiltonian-of-klein-gordon}
\end{equation}
容易看出第二项实际上是发散的。
产生这种发散的原因在于,相对论性量子场论不会被用于处理动量特别高的问题(在那里需要新的物理,通常称为“紫外端的物理”),因此所谓的对整个动量空间的积分实际上只是对动量空间中一块很大的区域的积分。
在这种意义下,\eqref{eq:hamiltonian-of-klein-gordon}中的第二项是一个很大的常数,称为\textbf{真空零点能}。因此在讨论全空间内的问题时,可以丢弃它得到等效的哈密顿量(注意此时哈密顿量的正定性实际上被破坏了)%?真的吗?
\begin{equation}
    \hat{H} = \int \frac{\dd[3]{\vb*{p}}}{(2\pi)^3} E_{\vb*{p}} \hat{a}_{\vb*{p}}^\dagger \hat{a}_{\vb*{p}}.
\end{equation}
这是一个福克空间上的$1$粒子算符。它表明自由场情况下单粒子携带能量为$E_{\vb*{p}}$。
通过反复使用对易关系\eqref{eq:quantization-scalar}以及真空态被湮灭算符作用后得到$0$这一事实,可以计算出
\begin{equation}
    \hat{H} \hat{a}_{\vb*{p}}^\dagger \ket{0} = E_{\vb*{p}} \hat{a}_{\vb*{p}}^\dagger \ket{0}.
\end{equation}
因此正如我们预期的那样,单粒子态$\ket{\vb*{p}}$是哈密顿量的本征态。

真空零点能的出现实际上意味着原来的哈密顿量中的各个项是不对易的,因此真空态的能量不能是零,如果它是零,那么由哈密顿量的正定性,哈密顿量中的每一项作用在真空态上都会得到零,于是真空态是哈密顿量的每一项的本征态,这就产生了矛盾。
% TODO:实际上不对易的算符还是可以有共同本征态的,以上说法不正确,需要进一步说明
不对易性是纯粹的量子概念,因此真空零点能只有在量子场论中才能够得到良好的定义。
如果哈密顿量中所有的项都是彼此对易的,就不会有真空零点能。有时真空零点能的存在也称为量子涨落,因为即使在真空态,也不是所有的物理量都有完全确定的值。

需要注意的是如果我们讨论的问题不是定义在全空间上的,可能不能直接把真空零点能丢弃。例如,设有两块无穷大的金属板,它们施加的边界条件会让\eqref{eq:expanding-klein-gordon-field}中的一些模式为零,通过计算可以发现板间的真空零点能小于板外,从而产生一个板之间的吸引力。

场的动量为
\[
    P_i = \int \dd[3]{\vb*{x}} \pi \partial_i \phi,
\]
从而
\begin{equation}
    \hat{\vb*{P}} = - \int \dd[3]{\vb*{x}} \hat{\pi} \grad{\hat{\phi}} = \int \dd[3]{\vb*{x}} \vb*{p} \hat{a}^\dagger_{\vb*{p}} \hat{a}_{\vb*{p}} + \text{vaccum zero-point item}
\end{equation}
因此场的动量也是单粒子算符。

标量场没有内禀自由度,因此也不携带自旋角动量。

总之,标量场需要使用\eqref{eq:symmetry-commutator}来量子化。因此标量场描述$0$自旋玻色子。

\subsubsection{旋量场}

% 反粒子

\subsubsection{无质量矢量场}

使用\eqref{eq:field-operator-fourier}展开一个无质量矢量场为
\begin{equation}
    A_\mu (\vb*{x}, t) = \int \frac{\dd[3]{\vb*{p}}}{(2\pi)^3} \frac{1}{\sqrt{2 E_{\vb*{p}}}} \sum_{r=0}^3 \epsilon_\mu^r(\vb*{p}) \left(\hat{a}_{\vb*{p}, r}^\dagger \ee^{ - \ii \vb*{p} \cdot \vb*{x} + \ii E_{\vb*{p}} t} + \hat{a}_{\vb*{p}, r} \ee^{ \ii \vb*{p} \cdot \vb*{x} - \ii E_{\vb*{p}} t} \right), 
    \label{eq:expanding-massless-vector-field}
\end{equation}
由于没有质量,
\begin{equation}
    E_{\vb*{p}} = \abs{\vb*{p}}.
\end{equation}
$\epsilon^r$为一组闵可夫斯基时空的基矢量,称它们为\textbf{偏振矢量},也即,
\begin{equation}
    (\epsilon^r)_\mu (\epsilon^{r'})^\mu = \eta^{r r'}.
\end{equation}
为了确定偏振矢量,通常要求
\begin{equation}
    \epsilon^1 \cdot p = \epsilon^2 \cdot p = 0,
\end{equation}
并认为$\epsilon^0$是类时的,而$\epsilon^{1,2,3}$是类空的。这样,当$p^\mu \propto (1, 0, 0, 1)$,即$\vb*{p}$指向$z$轴时,我们有
\begin{equation}
    \epsilon^0 = \pmqty{1 \\ 0 \\ 0 \\ 0}, \quad \epsilon^1 = \pmqty{0 \\ 1 \\ 0 \\ 0}, \quad \epsilon^2 = \pmqty{0 \\ 0 \\ 1 \\ 0}, \quad \epsilon^3 = \pmqty{0 \\ 0 \\ 0 \\ 1}.
    \label{eq:z-axis-p-epsilon}
\end{equation}
% TODO:这是$\epsilon^\mu$还是$\epsilon_\mu$???
$p$取其它值时只需要对\eqref{eq:z-axis-p-epsilon}做洛伦兹变换即可,因为$\epsilon$的定义完全是洛伦兹协变的。

% TODO:为什么?这一片我都没有动手算过,
可以计算出
\begin{equation}
    \pi^\mu (\vb*{x}, t) = \int \frac{\dd[3]{\vb*{p}}}{(2\pi)^3} \sqrt{\frac{E_{\vb*{p}}}{2}} \ii \sum_{r=0}^3 (\epsilon^r)^\mu (\vb*{p}) \left( \hat{a}_{\vb*{p}, r} \ee^{\ii \vb*{p} \cdot \vb*{x} - \ii E_{\vb*{p}} t} - \hat{a}_{\vb*{p}, r}^\dagger \ee^{ - \ii \vb*{p} \cdot \vb*{x} + \ii E_{\vb*{p}} t} \right),
\end{equation}
施加对易关系\eqref{eq:symmetry-commutator},通过计算得到
% TODO:真的可以**等价**地得到下式吗??
\begin{equation}
    \comm*{\hat{a}_{\vb*{p}, \lambda}}{\hat{a}^\dagger_{\vb*{p}', \lambda'}} = - \eta_{\lambda \lambda'} (2\pi)^3 \delta^3(\vb*{p} - \vb*{p}'), \quad \comm*{\hat{a}^\dagger_{\vb*{p}, \lambda}}{\hat{a}^\dagger_{\vb*{p}', \lambda'}} = \comm*{\hat{a}_{\vb*{p}, \lambda}}{\hat{a}_{\vb*{p}', \lambda'}} = 0.
\end{equation}
$\lambda=1, 2, 3$时对易关系是正确的,但是$\lambda=0$给出了一个不正常的对易关系
\[
    \comm*{\hat{a}_{\vb*{p}, 0}}{\hat{a}^\dagger_{\vb*{p}', 0}} = - (2\pi)^3 \delta^3 (\vb*{p} - \vb*{p}').
\]
例如,它产生的同样的单粒子态的内积将会是一个负数,这和我们对单粒子态的通常认识不符。
此外,哈密顿量成为
\begin{equation}
    \hat{H} = \int \frac{\dd[3]{\vb*{p}}}{(2\pi)^3} E_{\vb*{p}} \left( - \hat{a}_{\vb*{p},0}^\dagger \hat{a}_{\vb*{p}, 0} + \sum_{i=1}^3 \hat{a}_{\vb*{p},i}^\dagger \hat{a}_{\vb*{p}, i} \right),
\end{equation}
因此能量非正定。
显然这些问题都和$\hat{a}^\dagger_{\vb*{p},0}$有关,也就是说来自一个非物理的自由度。
会有非物理的自由度是显然的,因为我们在处理一个有规范不变性的场却从来没有选取过一个规范。
现在我们处理的是量子场,因此既可以直接对场做约束,也可以缩小态空间的范围。

经过检验,Gupia-Blenler量子化条件
\begin{equation}
    \partial^\mu \hat{A}_\mu^{(+)} \ket{\psi} = 0
\end{equation}
是一个可行的方案。% TODO:有没有别的选项?
它实际上约束了态空间的范围。
代入\eqref{eq:expanding-massless-vector-field},并注意到$\epsilon^1$与$\epsilon^2$和四维动量做内积得到零,我们发现
\begin{equation}
    (\hat{a}_{\vb*{p}, 0} - \hat{a}_{\vb*{p}, 3}) \ket{\psi} = 0.
\end{equation}
这意味着在无质量矢量场的态空间中哈密顿量实际上是
\begin{equation}
    \hat{H} = \int \frac{\dd[3]{\vb*{p}}}{(2\pi)^3} E_{\vb*{p}} (\hat{a}_{\vb*{p},1}^\dagger \hat{a}_{\vb*{p}, 1} + \hat{a}_{\vb*{p},2}^\dagger \hat{a}_{\vb*{p}, 2}).
\end{equation}
于是负能量问题也就解决了。哈密顿量中没有出现的量可以直接被略去,因为它们对系统的动力学不产生任何影响。
% TODO:严格说明
于是取
\begin{equation}
    A_\mu (\vb*{x}, t) = \int \frac{\dd[3]{\vb*{x}}}{(2\pi)^3} \frac{1}{\sqrt{2 E_{\vb*{p}}}} \sum_{r=1}^2 \epsilon_\mu^r(\vb*{p}) \left(\hat{a}_{\vb*{p}, r}^\dagger \ee^{ - \ii \vb*{p} \cdot \vb*{x} + \ii E_{\vb*{p}} t} + \hat{a}_{\vb*{p}, r} \ee^{ \ii \vb*{p} \cdot \vb*{x} - \ii E_{\vb*{p}} t} \right),
\end{equation}
以及
\begin{equation}
    \pi^\mu (\vb*{x}, t) = \int \frac{\dd[3]{\vb*{p}}}{(2\pi)^3} \sqrt{\frac{E_{\vb*{p}}}{2}} \ii \sum_{r=1}^2 (\epsilon^r)^\mu (\vb*{p}) \left( \hat{a}_{\vb*{p}, r} \ee^{\ii \vb*{p} \cdot \vb*{x} - \ii E_{\vb*{p}} t} - \hat{a}_{\vb*{p}, r}^\dagger \ee^{ - \ii \vb*{p} \cdot \vb*{x} + \ii E_{\vb*{p}} t} \right),
\end{equation}
重新计算对易关系得到
\begin{equation}
    \comm*{\hat{a}_{\vb*{p}, \lambda}}{\hat{a}^\dagger_{\vb*{p}', \lambda'}} = \delta_{\lambda \lambda'} (2\pi)^3 \delta^3(\vb*{p} - \vb*{p}'), \quad \comm*{\hat{a}^\dagger_{\vb*{p}, \lambda}}{\hat{a}^\dagger_{\vb*{p}', \lambda'}} = \comm*{\hat{a}_{\vb*{p}, \lambda}}{\hat{a}_{\vb*{p}', \lambda'}} = 0, \quad \lambda = 1, 2.
\end{equation}

下面我们推导动量和自旋角动量的公式。轨道角动量的由于是动量衍生出来的量,我们暂不考虑。
首先假设$p^\mu \propto (1, 0, 0, 1)$。
按照\eqref{eq:spin-angular-momentum}可以计算得到
\[
    \hat{S}_3 = \int \dd[3]{\vb*{x}} \mathcal{S}_3 = \ii \int \frac{\dd[3]{\vb*{p}}}{(2\pi)^3} (- \hat{a}_{\vb*{p},1} \hat{a}^\dagger_{\vb*{p}, 2} + \hat{a}^\dagger_{\vb*{p}, 1} \hat{a}_{\vb*{p}, 2} + \hat{a}_{\vb*{p}, 2} \hat{a}_{\vb*{p}, 1}^\dagger - \hat{a}_{\vb*{p}, 2}^\dagger \hat{a}_{\vb*{p}, 1} ) ,
\]
另外两个方向上的自旋角动量都是零。

我们原本预期矢量场会有三个自由度(因为\eqref{eq:lorentz-gauge}消除掉了一个自由度),但是实际上无质量矢量场只有两个自由度。
导致这一切的原因当然是无质量这个事实——它使得四维动量$p$不再能够写成$(1, 0, 0, 0)$这样的形式,而只能够写成$(1,0,0,1)$这样,从而让$A^0$和$A^3$相互抵消了。
从洛伦兹群在态空间上的表示出发可以更好地看待这个问题:$m=0$时洛伦兹群保持动量不变的小群不再是旋转群。
以一种更加物理的视角,无质量矢量场对应的粒子一直在以光速运动,不能找到一个相对它静止的参考系,因此对一个这样的粒子,实际上总是有一个特定的空间方向即它的运动方向,为了保持协变性,其自旋只能够沿着这个方向。换而言之,此时有意义的实际上是螺旋度而不是三维的角动量$\hat{\vb*{S}}$,即其内禀自由度是平面旋转群(以运动方向为轴旋转)的表示而不是三维旋转群的表示。
而对有质量的粒子,总是可以找到一个相对它静止的参考系,在这个参考系中空间是各向同性的,因此可以应用$SO(3)$的表示。
这和经典电磁场的偏振只有两个方向是对应的。
% TODO:经典场的傅里叶分量就是量子的产生湮灭算符

% TODO:场实际上只有两个自由度,因此粒子也只有两个内禀自由度,因此螺旋度是粒子的内禀自由度空间的CSCO。

% TODO:所以总之就是,无质量矢量场的自旋只在动量的方向上有非零分量,因此描述无质量矢量场的粒子的内禀自由度需要的实际上是螺度

\subsubsection{重矢量场}

\subsubsection{薛定谔场和单粒子量子力学}

% TODO:但凡有一组正交归一化单粒子态,就可以定义对应的产生湮灭算符,从而二次量子化

% TODO:可以看到,$j=0$的标量场给出的粒子的自旋角动量为0,$j=\pm \frac{1}{2}$的旋量场给出的粒子的自旋角动量为$\pm 1/2$,$j=1$的矢量场给出的粒子的自旋角动量为$\pm 1$。这并不让人意外,因为$j$决定了粒子的内禀自由度的维度($2j+1$)。无质量的情况比较特殊
现在我们来量子化薛定谔场。薛定谔场实际上是标量场、旋量场、矢量场退化而来的场,因此它也有内禀自由度。使用自旋(或者螺旋度)标记这些内禀自由度。
由于薛定谔场不是实场,考虑对易关系
\[
    \comm*{\hat{\psi}^i(\vb*{x}, t)}{\hat{\pi}_j(\vb*{y}, t)} = \ii \delta^i_j \delta^3 (\vb*{x} - \vb*{y}),
\]
即
\[
    \comm*{\hat{\psi}^i(\vb*{x}, t)}{(\hat{\psi}^j)^\dagger (\vb*{y}, t)} = 2 \delta^3 (\vb*{x} - \vb*{y}),
\]
这表明
% TODO:单粒子量子力学足以覆盖薛定谔场的情况,即“非相对论量子场论”就是量子力学。初等量子力学中可以直接定义S算符、单粒子费曼图(“原子吸收一个光子、释放一个光子”,等等),你可能会问为什么这些本来用于场论的概念也可以被用在单粒子量子力学上,毕竟前者是3+1维理论而后者是0+1维理论。但实际上,这些用在量子场论上的概念完全可以被应用在薛定谔场上,而由于薛定谔场不涉及粒子数变化,这些概念就可以被套用在单粒子态量子力学上。
% 量子场论和量子力学的对应实际上有两方面:首先,量子场论和量子力学都可以写成哈密顿动力学的形式,当然前者各个物理量可以使用空间位置作为标签而后者物理量的标签都是离散的;其次,量子场论(通过二次量子化)和量子力学都能够良定义多粒子态。

\subsection{相互作用}

% TODO:多场耦合的态空间

\subsubsection{引入相互作用项}

下面分析场之间的相互作用以及它引入的物理。带有相互作用的拉氏量一般形如
\begin{equation}
    \mathcal{L} = \sum \mathcal{L}_\text{free} + \mathcal{L}_\text{int},
\end{equation}
我们认为相互作用项中不应该出现导数(不然就总是可以重新定义一个场来把导数消除掉)。

相互作用项的出现意味着一个重要的事实。在分析自由场时,我们经常使用真空态$\ket{0}$,它同时具有两个性质:一切湮灭算符作用于其上都会得到零向量(我们称湮灭算符\textbf{摧毁}这个态);它是能量最低的能量本征态(自由场的哈密顿量是产生湮灭算符的二次型,因此有这个结论)。
在有相互作用项时,数学上可以证明,一般来说不再有这样同时满足两个条件的真空态。的确,我们可以有一个能量最低的能量本征态(或者说\textbf{基态})$\ket{\Omega}$,但是不能够保证湮灭算符一定摧毁$\ket{\Omega}$;同样,如果将一个产生算符作用于其上,过一段时间我们得到的可能就不再是最开始的那个粒子了。

由于这种内在的复杂性,像自由场的情况那样明确地写出场的时间演化是不可能的。但实际上也没有这种必要:说到底,我们关注的无非是:在某个时间给定一个系统,过一段时间之后这个系统会变成什么样。
既然要讨论“系统的演化”,我们现在切换到相互作用绘景下,这样就可以有随时间演化的态矢量。
本文基本上只讨论不显含时间的问题,所以海森堡绘景的哈密顿量和薛定谔绘景的哈密顿量是完全一样的,记之为
\begin{equation}
    \hat{H} = \hat{H}_0 + \hat{H}_i,
\end{equation}
其中$\hat{H}_0$为自由哈密顿量(它可以写成场算符的二次型),$\hat{H}_i$为相互作用哈密顿量。
驱动相互作用绘景下态矢量演化的是相互作用绘景下的相互作用哈密顿量,它是
\begin{equation}
    \hat{H}_i^I(t) = \ee^{\ii \hat{H}_0 t} \hat{H}_i \ee^{- \ii \hat{H}_0 t}.
\end{equation}
虽然$\hat{H}_i$是不含时的,绘景变换却引入了对时间的显式依赖。通常$\hat{H}_i$可以写成一系列算符的多项式,则$\hat{H}_i^I(t)$就是将$\hat{H}_i$中的每一个算符都切换到相互作用绘景下之后得到的结果。
在得到了相互作用哈密顿量之后,态矢量的时间演化算符就是
\begin{equation}
    \hat{U}(t,t_0) = T\exp\left(-\ii \int_{t_0}^t \dd{t'} \hat{H}_i^I(t') \right).
\end{equation}
形式地写出时间演化算符之后,就可以计算各种概率振幅,从而得到我们需要的一切物理量。

关于散射振幅需要一个注记。通常的问题是,设我们有一个时间$t_1$处的初态$\ket{\Psi_i}$,让它经过一定的时间演化,得到$\hat{U}(t_2, t_1) \ket{\Psi_i}$,然后使用一系列可能的末态去测量它,得到概率振幅
\[
    \mel{\Psi_f}{\hat{U}(t_2, t_1)}{\Psi_i}.
\]
原则上,末态是完全随意的,但实际上几乎从来都只使用乘积态作为末态。
原因在于,在相互作用绘景下,场算符按照自由哈密顿量不断地发生演化,因此实际上,使用产生湮灭算符(它们是场算符的线性组合)构造出来的二次量子化多粒子态(定域在某点的单粒子,两个动量确定的粒子,等等)也会有一个遵循自由哈密顿量的时间演化。
然而,乘积态是自由哈密顿量的本征态,因此乘积态的时间演化只是乘上一个复数因子。这样,概率
\[
    P(\ket{\Psi_f}) = \abs{\mel{\Psi_f}{\hat{U}(t_2, t_1)}{\Psi_i} }^2
\]
就不会出现任何时间演化。反之,如果末态不是乘积态,$P(\ket{\Psi_f})$会出现时间演化,因此它不是直觉意义上的末态。


% TODO:可重整性、量纲分析
\subsubsection{重整化、可重整性和量纲}

直觉上很合理的量子理论在实际计算时经常会出现莫名其妙的发散,例如类似于
\[
    \int \dd[4]{k} \frac{1}{k^2+m^2}
\]
这种并无良定义的积分。会出现这种问题的原因在于,不是所有的理论都能够保证自己在所有的范围内都是自洽的。例如,假定我们使用关于动量的产生湮灭算符(通常都是这样,因为动量空间内的哈密顿量形式比较简单)写下了一个理论,那么有可能在动量很大时这个理论会给出直觉上不合理的结果,
而当我们写下
\[
    \int \dd[4]{p} \frac{1}{p^2+m^2}
\]
这种积分时,我们假定了这个理论对所有的动量都成立。因此计算结果当然是发散的:这是理论自己在提示我们它具有一定局限性。
但无论如何,我们写下的直觉上合理的量子理论总应该是某个自洽的万有理论在特定范围内(如果使用动量表象,就是在某个动量区间内)的近似。
由于哈密顿量给定,这个特定范围对应着特定的能量范围,这个能量范围称为\textbf{能标}。
% TODO:实际上完全没有必要立刻引入能标的概念。。原则上,也可以对位置做重整化,把边界积掉什么的
例如,发生化学反应要求比较低的粒子动量,发生电子对撞等要求比较高的动量,由于$E \sim p^2$,我们说前者发生在比较低的能标上,后者发生在比较高的能标上。
如果万有理论可以写成一个低能标项加上一个高能标项,那么由于高能标项在低能标过程中的贡献很小,可以略去这一项;但实际的理论通常都存在低能标过程和高能标过程的耦合。这有可能根本上改变低能过程的行为,例如,设低能项中有两个彼此解耦的自由度,然而它们都和一个高能项中的自由度耦合,其结果是这两个自由度实际上仍然是耦合的。(通常称出现在低能项中的自由度构成\textbf{低能子空间},只出现在高能项中的自由度张成\textbf{高能子空间})
因此对一个会产生发散的理论,我们知道几件事:首先,它一定是一个自洽的万有理论在某个能标下的近似(不然不会发散);这个万有理论是未知的,我们唯一确定的是它一定存在(如果它已知,那也不必花力气去分析会产生发散的那个理论了);最后,它所在的能标一定和其它的一些能标下的过程%
\footnote{一个能标下的过程被一个哈密顿量描述,两个过程的耦合就是哈密顿量之间的耦合项。}%
有耦合(否则该理论在它所在的能标下是非常精确的理论,不应该有发散)。
我们的任务是,从以上事实出发,根据该会产生发散的理论计算出有意义的结果。

有时还会面临另一种问题。这里,我们已知一个适用范围很广的理论,而要计算它在某个能标下的近似。由于不同能标下的过程会有耦合,不能简单地将其它能标下的过程直接丢弃。

以上两种任务是\textbf{重整化群}的例子。它指的是从一个理论中移除一部分能标,留下有效哈密顿量的过程。如我们所见,能标实际上是标记该理论的场算符的那些CSCO的特定取值范围,因此移除一部分能标实际上就是要移除一部分自由度。
设原来的理论中的自由度为$\{s_i\}$,和这些自由度有关的参数为$\{C_i\}$,则移除一部分自由度就是做代换
\[
    \{s_i\} \longrightarrow \{s_i'\},
\]
然后用新的理论加上另一组参数$\{C_i'\}$去拟合原理论在$\{s_i'\}$这些自由度上的行为。(如果被移除的自由度远离我们关注的过程,那么移除不同的自由度只会让$\{C_i'\}$发生数值上的微调,而不会显著改变理论的结构,这称为\textbf{参数跑动})如果需要比原来的参数还要多很多的$\{C_i'\}$,那么移除自由度并不能显著简化模型,此时我们说原理论\textbf{不可重整},否则它是\textbf{可重整}的。
这样,重整化就是在做下面的变换:
\[
    \{s_i\} \longrightarrow \{s_i'\}, \quad \{C_i\} \longrightarrow \{C_i'\}.
\]
这个变换是不可逆的,因为它损失了那些被丢弃的自由度的信息。所有这些变换组成一个半群,这就是重整化群。
理论(在本文中是哈密顿量,当然也可以是等价的对象,比如拉格朗日量)在重整化群的作用下发生连续的变化。重整化会导致一个理论在所有可能的理论组成的空间中发生移动,这样产生的流动称为\textbf{重整化流}。如果一族类似的理论在重整化群作用下会汇聚于一点,这就是一个\textbf{重整化群不动点}。这个不动点处的理论是某个能标下满足一些宽泛条件(也就是这族类似的理论共同满足的一些条件,比如对称性要求,等等)的理论必然采取的形式。
很多我们熟悉的理论都具有这样的不动点,因此它们使用重整化群计算出的结果特别精确。
需注意不动点上的理论本身可能会导致发散。通过它是重整化群的不动点这一事实通过一些数学处理(例如,做对动量的积分时虚设一个截断,引入一些无物理意义的参数抵消无穷大,等等)消除这些发散的方式就是\textbf{正规化}。
不可重整的系统当然不能够通过正规化得到任何有意义的结果。

通常有效哈密顿量的形式不发生大的变化,但是一些参数会有变化。这就是所谓的\textbf{参数跑动}。

实际上,在实际的场论计算中我们做的事情正好是反过来的:如果直接使用实验测得的“物理”参数作为粒子质量、耦合常数等,那么往往会得到发散的结果——这当然是因为将所有费曼图求和时会涉及任意高的能标,而实验测得的参数只有在我们关心的能标下才有效。
换句话说,如果要使用实验中测得的参数代入计算,那么必须设置一个能量截断。
但我们怎么能够知道合适的能量截断在哪里呢?为了解决这个问题,我们将拉氏量中的参数设置为值不确定的“裸参数”,用这些裸参数,连同一个值不确定的能量截断,代入计算,并且用这些可调参数吸收掉所有发散。
这样我们就自动地确定了能量截断的位置:在物理参数下能够完全吸收发散的能量截断就是合理的。
能标是物理参数隐式地提供的。
能够这样做的前提是,导致发散的费曼图成分可以约化到有限多个图中(从而我们可以用有限多个可变的裸参数吸收这些发散),满足这个条件的理论是\textbf{可重整的},否则是\textbf{不可重整的}。
的确有这样的可能,就是我们在处理的理论实际上是多个更高能标的理论的低能有效理论,也即,不同的重整化方案会给出不同的结果。

将一些自由度去掉通常称为\textbf{积掉}这些自由度,这实际上是一个路径积分的概念,就是从配分函数中把关于我们不想要的自由度的积分直接计算出来,通常会使用微扰论等工具来得到一个近似结果。
在正则量子化表述中我们要使用哈密顿量的微扰论(实际上有两种微扰论:一种是关于哈密顿量和它的本征态的,另一种是关于散射振幅的)

在不动点附近引入小的形变(等价于向理论引入新的项或者改变已有的项的系数——两者实际上是一回事),或者会远离不动点——此时这种改变的方向(这个方向对应于理论中特定的一种项)称为\textbf{有关的},也即,向理论中加入这一项会显著改变理论在不动点处的行为;或者会回到不动点,此时这种改变的方向称为\textbf{无关的},因为它们对理论在不动点处的行为毫无影响。
也有少见的情况是,一系列不动点排成一条线或者一个面,以至于在某个方向上引入变化只会得到另一个不动点上的理论。

回顾时间演化算符,我们发现
\[
    \left[ \int \dd{t} H \right] = 1,
\]
也就是说
\[
    \left[ \int \dd[4]{x} \mathcal{H} \right] = 1,
\]
从而
\[
    [\mathcal{H}] = [\mathcal{L}] = \text{length}^{-4}.
\]
由于三种场在非相对论情况下都退化为薛定谔场,它们的质量都退化为薛定谔场描述的粒子的质量,可以确信,这三种场的质量具有相同的量纲。
从标量场的拉氏量,我们发现
\[
    [m^2] = [\partial_\mu \partial^\mu] = \text{length}^{-2},
\]
从而
\[
    \text{mass} = \text{length}^{-1}.
\]
由于质量是唯一的参数,我们使用质量的量纲作为比较的标准。我们有
\[
    \text{length} = \text{mass}^{-1},
\]
从而
\[
    [\mathcal{H}] = [\mathcal{L}] = \text{mass}^4.
\]
动量的量纲和质量的量纲一致。
然后就可以分别根据三种场的拉氏量评定场的量纲了。分别考虑

\subsubsection{散射}

本节讨论散射理论。我们使用相互作用绘景。因此,场算符和自由场的情况一模一样,按照\eqref{eq:field-operator-fourier}展示的方式演化,而态矢量则根据相互作用哈密顿量$H_i^I$演化。

假定系统处于散射态,那就是说,在$t=-\infty$和$t=+\infty$处没有相互作用——等价地说,就是系统状态在$t \to -\infty$和$t \to +\infty$时有极限,且均为自由场态。
设$t = -\infty$时系统处于其初态$\ket{\Psi_i}$,这是一个自由场的态。
记$\hat{U}$为相互作用绘景下的时间演化算符,则系统末态为$U(\infty, -\infty) \ket{\Psi_i}$。
所以只需要计算出
\[
    \mel{\Psi_f}{U(\infty, -\infty)}{\Psi_i}
\]
并平方就得到了系统经过观察,终结于态$\ket{\Psi_f}$的概率。
这就是要计算\textbf{S矩阵}
\begin{equation}
    \hat{S} = U(\infty, -\infty).
\end{equation}
按照\eqref{eq:time-evolution-in-interation-picture},我们有
\[
    \hat{U}(t, t_0) = T \exp \left( - \ii \int \dd{t} \hat{H}_i^I(t) \right),
\]
从而
\begin{equation}
    \hat{S} = T \exp \left( - \ii \int_{-\infty}^\infty \dd{t} \hat{H}_i^I(t) \right)
\end{equation}

% Wick定理实际上适用于所有乘积态,也就是形如某个二次量子化基矢量的态。需要清楚地说明什么是乘积态。

散射过程的实验测定通常是所谓的\textbf{散射截面}。

\subsubsection{束缚态}

\section{微扰论}

\subsection{Dyson级数和Wick定理}

% TODO:Wick定理(实际上是Isserlis' theorem)实际上是一个一般性的定理,只要参与期望值计算的每一个函数的期望都是零;真空态和乘积态都满足这个条件。
% 有相互作用的情况下这个条件不满足,因此Wick定理对海森堡绘景下的相互作用场无效;这也就是相互作用绘景如此重要的原因
%只有自由系统的基态才是乘积态。实际上乘积态可以通过重新定义产生湮灭算符而转化为一个等效的真空态。
% 湮灭算符消灭的是相互作用真空态还是自由真空态??
基态指的是能量最低的状态。
加入相互作用之后系统的基态和自由场情况下的基态是不一样的。通常使用$\ket{\Omega}$来表示这个基态。

我们假定相互作用实际上并不强,因此可以合理地将S矩阵做展开:
\begin{equation}
    \begin{aligned}
        \hat{S} &= T \exp \left( - \ii \int_{-\infty}^\infty \dd{t} \hat{H}_i^I(t) \right) \\
        &= \sum_{n=0}^\infty \frac{(-\ii)^n}{n!} \int_{-\infty}^\infty \int_{-\infty}^\infty \cdots \int_{-\infty}^\infty \dd{t_1} \dd{t_2} \cdots \dd{t_n} T [\hat{H}_i^I(t_1) \hat{H}_i^I(t_2) \cdots \hat{H}_i^I(t_n)].
    \end{aligned} 
    \label{eq:dyson-series}
\end{equation}
这就是所谓的Dyson级数。

计算编时算符的作用相对困难,这里介绍一个将编时算符转化为正规序算符的方法。
记两个算符的\textbf{收缩}为
\begin{equation}
    {\contraction{}{\hat{A}}{}{\hat{B}} \hat{A}\hat{B}} = T[\hat{A} \hat{B}] - N[\hat{A} \hat{B}].
\end{equation}
产生湮灭算符和时间没有联系。通常我们人为规定产生算符的时间为$t=-\infty$,湮灭算符的时间为$t=\infty$。
通过对易关系可以计算出,无论是费米子还是玻色子,其场算符或产生湮灭算符的收缩都是一个复数。
进一步,定义任意的算符序列的收缩为
\begin{equation}
    {\contraction{}{\hat{A}}{\cdots}{\hat{B}} \hat{A} \cdots \hat{B}} = (\pm 1)^s {\contraction{}{\hat{A}}{}{\hat{B}} \hat{A}\hat{B}} \cdots,
\end{equation}
其中玻色子取$+1$,费米子取$-1$,$s$指的是将序列$\hat{A} \cdots \hat{B}$重排为$\hat{A}\hat{B} \cdots$需要的置换数目。
一个算符序列的收缩可以是从中取两个算符做收缩,也可以是从里面先取两个算符做收缩,再取两个算符做收缩,也就是说取两对算符做收缩,又可以是取三对算符做收缩,等等。
在自由场情况下我们有\textbf{Wick定理}:
\begin{equation}
    T[\hat{A} \hat{B} \hat{C} \cdots] = N[\hat{A} \hat{B} \hat{C} \cdots + \text{all possible contractions of } \hat{A} \hat{B} \hat{C} \cdots].
\end{equation}

这样我们就把\eqref{eq:dyson-series}转化为了计算$N[\hat{A}\hat{B} \cdots]$这样的表达式。
这样的好处在,
\[
    \mel{0}{T[\hat{A} \hat{B} \hat{C} \cdots]}{0} = \mel{0}{N[\hat{A} \hat{B} \hat{C} \cdots + \text{all possible contractions of } \hat{A} \hat{B} \hat{C} \cdots]}{0},
\]
其中的每一项或者是纯数或者是一个纯数乘以一些算符的正规序。但\autoref{sec:algebra-ca-op}中已经证明,任何算符的正规序的真空态期望值都是零,因此最后只有纯数项能够剩下,而所有给出纯数的项都是所有算符都被成对地收缩的那些项,因此
\[
    \mel{0}{T[\hat{A} \hat{B} \hat{C} \cdots]}{0} = \sum_\text{pairs without intersection} (\pm 1)^s {\contraction{}{\hat{O}_1}{}{\hat{O}_2} \hat{O}_1 \hat{O}_2} {\contraction{}{\hat{O}_3}{}{\hat{O}_4} \hat{O}_3 \hat{O}_4} \cdots.
\]
$s$是每一对收缩的$s$,也就是将参与收缩的两个算符移动到相邻位置需要的交换数,之和。一个规则是:将$\hat{O}_1, \hat{O}_2, \ldots$在原序列中的位置用弧线两两连接起来,如果所有弧线可以均不相交,则$s$取偶数,否则取奇数。
事实上并不需要实际去计算这些收缩,因为
\[
    \begin{aligned}
        {\contraction{}{\hat{O}_1}{}{\hat{O}_2} \hat{O}_1 \hat{O}_2} &= {\contraction{}{\hat{O}_1}{}{\hat{O}_2} \hat{O}_1 \hat{O}_2} \braket{0}{0} \\
        &= \mel{0}{T[\hat{O}_1 \hat{O}_2] - N[\hat{O}_1 \hat{O}_2]}{0} \\
        &= \mel{0}{T[\hat{O}_1 \hat{O}_2]}{0},
    \end{aligned}
\]
我们有
\begin{equation}
    \mel{0}{T[\hat{A} \hat{B} \hat{C} \cdots]}{0} = \sum_\text{pairs without intersection} \mel{0}{T[\hat{O}_1 \hat{O}_2]}{0} \mel{0}{T[\hat{O}_3 \hat{O}_4]}{0} \cdots.
\end{equation}
其中$\hat{O}_1, \hat{O}_2$等是从$\hat{A},\hat{B}, \ldots$中选取出的不重合的一对一对的算符。
% 设\eqref{eq:field-operator-fourier}中的正频部分为$\hat{\phi}^+$,负频部分为$\hat{\phi}^-$,显然前者只含有产生算符,后者只含有湮灭算符。

由于

% TODO: 费曼图
费曼关联函数,又称两点格林函数
\begin{equation}
    G(x, y) = \mel{\Omega}{T \hat{\phi} (x) \hat{\phi}^\dagger(y) }{\Omega},
\end{equation}

四点格林函数
\begin{equation}
    G^{(4)} (x_1, x_2, x_3, x_4) = \mel{\Omega}{T \hat{\phi} (x_1) \hat{\phi}(x_2) \hat{\phi}^\dagger(x_3) \hat{\phi}^\dagger (x_4) }{\Omega}
\end{equation}

自由场
\begin{equation}
    \Delta (x, y) = \mel{0}{T \hat{\phi} (x) \hat{\phi}^\dagger(y) }{0}
\end{equation}
自由场的传播子确实是场方程的格林函数。
% TODO:为什么会有这个巧合?

% TODO
费曼图中的元素可以分成几种:
\begin{itemize}
    \item 节点,即几条柄的交点,实际上就是相互作用哈密顿量中的一项
    \item 柄,表示一种场,连接到同一个节点的柄关于同一个时空坐标,连接到不同的节点的柄关于不同的时空坐标
    \item 外腿,有一段不连接任何东西的线,表示产生湮灭算符(就是$\mel{0}{\hat{a}_{\vb*{p'}}\hat{S}\hat{a}^\dagger_{\vb*{p}}}{0}$中左右那两个),可以看成一种特殊的柄
    \item 线,连接两个柄或者一个柄一个外腿或者两个外腿。连接两个柄的时候它表示传播子$\mel{0}{T\hat{\phi}(x)\hat{\psi}(y)}{0}$,连接一个外腿一个柄的时候表示$\mel{0}{T\hat{\phi}(x)\hat{a}^\dagger_{\vb*{p}}}{0}$,连接两个外腿的时候表示$\mel{0}{\hat{a}_{\vb*{p}}\hat{a}^\dagger_{\vb*{p'}}}{0}$。只有同类的柄才能用同样的线条连接。
    \item 源,表示一个给定的外场的贡献。
\end{itemize}
节点出现几个,就说明这是几阶微扰。

% 费曼图中来自同一节点的线实际上代表一个多粒子态,可以使用一个把这些线覆盖住的圈表示一个态。

绘制费曼图的方法:先画出外腿和节点,彼此间不相连,然后尝试使用线把它们连起来;同一张图能够使用多少种方法连接出来,它在最后的展开式中就会出现多少次,费曼图越对称它重复出现的次数就越多。

从费曼图写出$\hat{S}$算符的矩阵元的方法称为\textbf{费曼规则}。
获得费曼规则的方法:使用不同的线条表示不同的场,然后计算每条线可能表示的因子,一张图中每一条线表示的因子全部乘起来然后对所有涉及的时空坐标(每个节点涉及一个时空坐标)求积分,最后乘上对称性因子和$\alpha^n/n!$($\alpha$指的是相互作用哈密顿量前面的系数,也即是“关联常数”)之类的量就得到了这一阶微扰。$n!$除以费曼图重复的次数就是对称性因子。

由于传播子实际上是自由场的格林函数,它通常有
\[
    \int \frac{\dd[3]{\vb*{p}}}{(2\pi)^3} \ee^{\ii p \cdot x} f(p)
\]
的形式,$f(p)$为动量空间中的传播子。
将对时空坐标的积分应用到这些指数函数上就得到了动量空间中的费曼图:可以使用一个值不确定的动量标记每条线,并且对这些未确定的动量积分:
\[
    \int \frac{\dd[4]p}{(2\pi)^4}
\]
因此费曼图中的线可以被认为是行进中的粒子。
容易看出所有可能的费曼图就是粒子所有可能的碰撞方式。有一些线并不来自任何一条外腿,但是由于相互作用的特征(例如在$\phi^4$理论中任何一个节点都连着四条线)必须有,它们就是所谓的虚粒子。它们的动量是不定的,也就是说它们不必遵循动量守恒。

由于费曼图中的节点实际上是相互作用哈密顿量的一项,由费曼规则,我们可以发现相互作用哈密顿量中的每一项都可以被赋予“允许某个过程发生”的意义,例如$\lambda \hat{a}_1^\dagger \hat{a}_2$可以理解为一个类型2的粒子转化为类型1的粒子的过程,设$\hat{\phi}$是实场,那么产生和湮灭的方向是随意的,例如$\phi^4$可以表示一个粒子转化为三个粒子,两个粒子转化为两个粒子,三个粒子转化为一个粒子。% TODO:说明$\hat{\phi}(\vb*{x})$虽然不是产生湮灭算符,在计算格林函数和费曼图时却可以当成产生湮灭算符用

费曼图也可以用于计算有效哈密顿量,因为从有效哈密顿量计算出的低能空间的散射振幅必须和从原哈密顿量计算出的低能空间的散射振幅一致,因此初末态均为低能过程但中间涉及高能过程的图也必须被考虑进从有效哈密顿量计算出的低能空间的散射振幅中,也即,我们把每一张中间涉及高能过程的部分遮盖掉,使用一个常数代替它,这就是重整化群作用下参数跑动的原因。
% 自能修正实际上也是这样的过程:我们把所有的相互作用都使用一个节点(自能)代替,做无穷求和,得到
\[
    G = G^0 + G^0 \Sigma G^0 + G^0 \Sigma G^0 \Sigma G^0 + \cdots,
\]
% 然后使用等比数列推导出Dyson方程
\[
    G = G^0 + G^0 \Sigma G.
\]
% 换而言之,我们把所有相互作用打包成了$V$,写出有效哈密顿量$H = H_0 + \Sigma$

如果一个二次量子化哈密顿量中只含有单体算符,那么无论入射粒子有几个,其任意阶的费曼图中都不可能出现粒子间相互作用(正如我们所预期的),因此这种问题可以直接化归为单粒子问题。

% 总之一切都可以通过自由场格林函数算出来,而自由场格林函数又遵循Wick定理
% 另外Peskin上面说上述方法实际上只是严格使用$\ket{\Omega}$推出来的方法的一个简化版(见111页)
% 使用$\ket{\Omega}$的格林函数是真空态格林函数除以一个归一化因子,但是在计算散射截面时所有的归一化因子都抵消了

由于傅里叶变换同时也是坐标表象和动量表象之间的变换,对关于坐标的场算符给出的格林函数做傅里叶变换,就得到了对关于动量的场算符给出的格林函数。

% 这里的$\ket{\Omega}$可以不是相互作用的基态
格林函数$\mel{\Omega}{\hat{a}(M^{(i)})\hat{a}^\dagger(M^{(j)})}{\Omega}$实际上给出了“在某个时间点放入一个粒子,在随后的时间点找到参数略有不同的粒子,且其余条件一切不变”的概率振幅。
% 单粒子格林函数$\braket{x'}{x}$就是薛定谔方程的格林函数;在无相互作用时这就是$\mel{0}{\hat{a}(\vb*{x'})\hat{a}^\dagger(\vb*{x})}{0}$,因此我们也将这里的格林函数称为格林函数。单粒子量子力学就是无粒子间相互作用的薛定谔场的量子场论。
如果被放入的粒子能够稳定存在,那么这个粒子的单粒子态就是哈密顿量的本征态,因此该单粒子态的时间演化一定形如
\[
    \ket{M^{(i)}(t)} = \ee^{\ii E t} \ket{M^{(i)}(0)}.
\]
因此格林函数的时间演化就是乘上了因子$\ee^{\ii E (t_2 - t_1)}$,因此格林函数对时间做傅里叶变换之后的结果(称为\textbf{频率格林函数})的奇点就是$\ket{M^{(i)}}$对应的能量。
奇点绘制出的$\omega-M$关系(实际计算时通常使用动量格林函数,从而就是$\omega - \vb*{p}$关系)就给出了粒子的\textbf{能谱},即具有不同参数的粒子分别具有怎样的能量。

有时被放入的粒子长期来看不能够稳定存在,但是在一小段时间内还是能够稳定存在,因此在短时间内近似有
\[
    \ket{M^{(i)}(t)} = \ee^{\ii E t} \ket{M^{(i)}(0)},
\]
但由于长期看它不能够稳定存在,也即,随着$t_2-t_1$拉长,格林函数趋于零,格林函数的时间演化就是乘上了因子$\ee^{\ii E (t_2 - t_1) - \beta (t_2) - t_1}$,$\beta$是衰减因子,则此时频率格林函数的奇点为$E + \ii \beta$,因此奇点的实部为粒子的能量,虚部为粒子寿命的倒数。(粒子寿命无明确定义,但是由于衰减因子中唯一描述时间尺度的就是$1/\beta$,无论采取何种定义它都一定正比于频率格林函数虚部的倒数)
由于对一定范围内的实数$E$都近似有
\[
    \ket{M^{(i)}(t)} = \ee^{\ii E t} \ket{M^{(i)}(0)}
\]
成立,不稳定的粒子具有的能量实际上是不确定的,在格林函数奇点的实部附近有一个展宽,展宽的大小正比于$\beta$。
更一般的,设系统哈密顿量可以写成一个能谱已知的自由哈密顿量加上一个相互作用哈密顿量,在相互作用哈密顿量不是非常强时,自由哈密顿量的能级也会有一个有限的寿命,伴随而来的就是一个能级展宽。只需要将能级看成粒子就能够看出这一点。

在$\ket{\Omega}$就是真空态时,裸粒子就是裹粒子;而当它不是真空态时,通常无法严格计算出裹粒子具体是什么样的模式——我们通常只能够计算出裸粒子的格林函数。
这样一来,$\hat{M}$虽然是裸粒子的CSCO,却未必是裹粒子的CSCO,这回导致两件事:
首先,裸粒子频率格林函数除了在粒子能量附近的峰以外,可能还具有一些其它的结构,因为裸粒子进入系统后会和系统发生相互作用而“裹上外皮”(所谓外皮就是跟着它运动的那些系统中原有的粒子),也即,一开始裸粒子的时域格林函数的时间演化不是严格的指数型振荡和衰减,如果略去这些结构,那么裸粒子格林函数和裹粒子格林函数就是一样的;
其次,频率格林函数可能不止有一个奇点,也即,有不止一支能谱,如果出现了这种情况,就意味着需要$\hat{M}$以外的其它物理量来标记裹粒子。

集体激发的产生湮灭算符是实际粒子的产生湮灭算符的函数。集体激发的产生湮灭算符可以是可观察量(如粒子密度,对应的集体激发就是粒子密度整体的变化,可以看成是声波的量子化),也可以不是(这就意味着产生相应的集体激发的场是复场,从而有正集体激发和反集体激发之分)。
因此,集体激发的格林函数就是实际粒子的四粒子格林函数、八粒子格林函数等等,其奇点反映了集体激发——而不是实际粒子——的能谱。

有时可以将系统哈密顿量的一部分使用某些集体激发代替。记系统中有两个(可以是多分量的)场$\phi$和$\psi$,设集体激发的表达式为$\mathcal{O}(\phi)$,系统哈密顿量可以写成
\[
    \hat{H} = \hat{H}_\phi[\phi] + \hat{H}_{I}[\phi, \psi] + \hat{H}_\psi[\psi],
\]
可以设法将$\hat{H}_{I}[\phi, \psi]$写成关于$\psi$和$\mathcal{O}[\phi]$的形式(不是总能做到这一点,因此需要仔细选取$\mathcal{O}[\phi]$的形式),然后把$\hat{H}_\phi[\phi]$写成$\mathcal{O}[\phi]$或者别的某些集体激发的形式(同样也不是总能做到这一点)。
这么做的成功例子包括一维电子气的玻色化(集体激发是电子数密度的波动)、关于自旋波的有效模型(集体激发是相邻两个电子自旋一上一下)等。

最后,格林函数的解析性质决定了它的奇点或者是极点,或者是支点(即格林函数的表达式意味着它实际上是某个多值函数的一支;支点通常来自无穷多个极点排列在同一个线段上,从而导致一条割线)。

关于傅里叶变换:若空间均匀,则
\[
    \text{FT}_{\vb*{r}, \vb*{r}'}(G(\vb*{r}, \vb*{r}'))(\vb*{p}, \vb*{p'}) = \text{FT}_{\vb*{r}-\vb*{r}'} (G(\vb*{r}-\vb*{r}')) (\vb*{p}) \delta(\vb*{p}-\vb*{p}') ,
\]
即
\[
    G(\vb*{p}, \vb*{p}') = G(\vb*{p}) \delta(\vb*{p}) \delta(\vb*{p} - \vb*{p}'),
\]
前面一个格林函数是分别对两个空间坐标做傅里叶变换的结果,后面一个格林函数是对两个空间坐标之差做傅里叶变换的结果。

格林函数之所以称为格林函数:粒子间无相互作用时它就是单粒子方程受外界激发之后的解。

\subsection{生成泛函和路径积分}

% TODO:是否单粒子量子力学的传播子实际上就是薛定谔场的传播子?
任何一种场的构型都可以通过某种外源激发出来;既然粒子无非是场的激发,我们可以想象,一切场论能够提供的信息(如前所述,无非是各种格林函数)都可以通过观察场在外源激发下的行为得到。
本节将要讨论这种方法。实际上,它构成了量子场论的一种独立的表述。

% TODO:格林函数$G$的自由场的拉格朗日量体现为$\phi G^{-1} \phi$的形式

设想我们给系统一个激发,那就是说,做替换
\begin{equation}
    \mathcal{L}(\phi(x)) \longrightarrow \mathcal{L}(\phi(x)) + \phi(x) J(x),
\end{equation}
并设
\begin{equation}
    Z(J) = \mel{\Omega}{\hat{S}}{\Omega}
\end{equation}

% 原则上,场的路径积分也是通过计算始末概率振幅而得到的。但实际做计算时很少需要使用场的表象,都是用多粒子态表象来计算的,因此路径积分更多的用来做生成泛函

% TODO:格林函数的奇点给出了系统的束缚态,以及格林函数的费曼图;如果粒子的CSCO中的某个量$M$是守恒量,那么两点格林函数只有在两个粒子的$M$相等时才不为零。

\subsubsection{有效哈密顿量与微扰展开方法}

\eqref{eq:effective-hamiltonian-original}通常难以直接求出,可以使用微扰展开来处理。

% TODO:有效作用量
\[
    \int \dd[3]{\vb*{r}} \phi(\vb*{r}) G(\vb*{r} - \vb*{r}') \phi(\vb*{r}') = \int \dd[3]{\vb*{k}} \phi(-\vb*{k}) G(\vb*{k}) \phi(\vb*{k}),
\]
时间同理,因此一个推迟相互作用的频域版本一定显含$\omega$,否则它是$\delta$函数,就没有推迟了。
(单个场变量对时间做傅里叶变换,所得频谱的峰主要集中在能谱附近,两个同类场变量,一个是产生算符一个是湮灭算符,乘起来,对时间做傅里叶变换,所得频谱的峰集中在能量差附近。如果做一个截断,认为所得频谱只有一个,就在单个场变量的能量或者两个场变量的能量差上,那么就把推迟相互作用近似为了一个瞬时相互作用。因此相互作用系数中含有粒子能量、能量差,就意味着它可能是一个推迟相互作用的近似)

\end{document}