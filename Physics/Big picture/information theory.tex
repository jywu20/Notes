\documentclass[UTF8, a4paper]{ctexart}

\usepackage{geometry}
\usepackage{titling}
\usepackage{titlesec}
\usepackage{paralist}
\usepackage{footnote}
\usepackage{enumerate}
\usepackage{amsmath, amssymb, amsthm}
\usepackage{cite}
\usepackage{graphicx}
\usepackage{subfigure}
\usepackage{physics}
\usepackage[colorlinks, linkcolor=black, anchorcolor=black, citecolor=black]{hyperref}

\geometry{left=2.5cm,right=2.5cm,top=2.5cm,bottom=2.5cm}
\titlespacing{\paragraph}{0pt}{1pt}{10pt}[20pt]
\setlength{\droptitle}{-5em}
\preauthor{\vspace{-10pt}\begin{center}}
\postauthor{\par\end{center}}

\newcommand*{\natnums}{\mathbb{N}}
\newcommand*{\integers}{\mathbb{Z}}
\newcommand*{\reals}{\mathbb{R}}

\newcommand*{\diff}{\mathop{}\!\mathrm{d}}
\newcommand*{\st}{\quad \text{s.t.} \quad}
\newcommand*{\const}{\mathrm{const}}
\DeclareMathOperator{\exception}{\mathbb{E}}

% \renewenvironment{itemize}{\begin{compactitem}}{\end{compactitem}}
% \renewenvironment{enumerate}{\begin{compactenum}}{\end{compactenum}}

\newenvironment{bigcase}{\left\{\quad\begin{aligned}}{\end{aligned}\right.}

\title{计算和信息论}
\author{吴何友}

\begin{document}

\maketitle

符号规定:以下如无特殊说明,$\log$以$2$为底。这是不重要的,实际上完全可以使用别的底。

\section{经典信息论}

\subsection{信息熵及其衍生概念}

\subsubsection{香农信息熵}

我们知道,密度算符$\hat{\rho}$的冯诺依曼熵为
\begin{equation}
    S = - \trace (\hat{\rho} \ln \hat{\rho}),
\end{equation}
在经典情况下——也就是密度算符在某一组基底下始终是对角的,这通常是因为环境扰动导致了一组偏好基——密度算符就可以完全使用随机变量表示,而做一次观测就是从随机变量中做一个抽样。
设有离散型随机变量$X \sim p(x)$是某个系统的密度算符在经典情况下对应的随机变量,我们称它为\textbf{信源}。我们需要衡量这个随机变量携带的信息量,或者说,这个随机变量的值如果能够确定,我们可以收到多少信息。
所谓信息,就是不确定性的消除,因此我们只需要衡量这个随机变量有多混乱(也即,这个随机变量的值确定之后能够消除多少混乱),就衡量了它能够提供的信息。
这种混乱程度称为\textbf{信息熵},它是随机变量$X$的不同取值的概率的函数,即
\[
    H = S(p_1, p_2, \ldots).
\]
具体每个$p_i$对应的$X_i$的值实际上不那么重要,因为最关键的显然是这些值能够彼此区分开来;两个$m$元集合之间可以建立一个一一对应,而直觉上它们的信息量肯定是一样的。
我们下面使用
\[
    H_m (p_1, p_2, \ldots, p_m)
\]
表示一个有$m$个不同的可能取值,第$i$个取值的概率为$p_i$的随机变量的信息熵。

当然,可以直接从玻尔兹曼熵推导出(经典的)信息熵,但是也可以通过公理化方法。
我们要求信息熵满足以下条件:
\begin{enumerate}
    \item 连续性,即$S$对各个$p_i$都是连续的。
    \item 在等概率时,即如果$X$有$N$个取值,每个取值的概率是$1/N$时,$N$越大$S$越大(这是很合理的,因为涉及越长的编码的随机变量携带的信息显然越多)。
    \item 可加性,这个性质叙述起来比较麻烦。我们把$m$个可能的结果分到$n$个不交集合中,用$I_s$表示第$s$个集合。
    显然,“观察结果落在$I_s$中”的概率为
    \[
        P(I_s) = \sum_{i \in I_s} p_i,
    \]
    且我们将集合$I_s$中的可能结果放在一起,做成一个新的随机变量,则其概率分布为
    \[
        p_{s, i} = \frac{p_i}{P(I_s)} = \frac{p_i}{\sum_{i \in I_s} p_i}.
    \]
    我们当然可以固定$s$,使用$p_{s, i}$计算信息熵,并且非常合理地,我们会要求信息熵是线性可加的,即
    \begin{equation}
        H_m(p_1, \ldots, p_m) = \sum_s P(I_s) H_{\left\| I_s \right\|} (\{p_{s, i} | i \in I_s \}).
    \end{equation}
\end{enumerate}
满足以上条件的信息熵定义,差一个常数因子,可以唯一确定为
\begin{equation}
    H(X) = - \sum_{\text{possible } x} p(x) \log p(x).
\end{equation}
很容易看出这就是玻尔兹曼熵的特例。$H(X)$不小于零,且当且仅当毫无随机性,即可以找到一个$i$使得$p_i=1$时,$H(X)=0$。

\subsubsection{联合熵、条件熵、互信息}

实际问题中涉及的信源肯定不止一个,从信息熵出发可以定义很多衡量两个信源之间的关系。
首先可以定义\textbf{联合熵}$H(X, Y)$,它是两个信源的笛卡尔积$(X, Y)$的信息熵,计算公式为
\begin{equation}
    H(X, Y) = - \sum_{x, y} p(x, y) \log p(x, y),
\end{equation}
且非常合理地,可以验证$H(X, Y) = H(Y, X)$。
假设我们首先从$X$那里获得了一些信息,然后又从$Y$那里获得了一些信息%
\footnote{在这里以及之后,如无特殊说明都假定$X, Y$等信源在两次观察之间没有发生时间演化。}%
,则从$X$那里获取的信息量为$H(X)$而先从$X$再从$Y$那里获得信息之后获得的信息量为$H(X, Y)$,则“从$X$那里获得信息后,再从$Y$那里获得的额外的信息量”——也即\textbf{条件熵}——为
\begin{equation}
    H(Y|X) = H(X, Y) - H(X),
\end{equation}
根据条件概率的定义马上可以得到
\begin{equation}
    H(Y|X) = - \sum_{x, y} p(x, y) \log p(y|x).
\end{equation}
此外,很容易看出无论条件熵还是联合熵都大于等于零,且
\begin{equation}
    0 \leq H(X|Y) \leq H(X),
\end{equation}
当且仅当$X$和$Y$分布独立时取右边等号(即知道$Y$后再测量$X$获得的纯粹就是$X$的信息),当且仅当存在双射函数$g$使得$X=g(Y)$时取左边等号(即$X$提供的信息就是$Y$提供的信息,知道$Y$后再测量$X$获得不了任何新的信息)。

条件熵和联合熵都是在分析“总的信息量”。如果我们只想知道一个信源——比如说$X$——提供的信息,那么在尚未从$Y$获得信息时,单独观测$X$能够获得信息$H(X)$,而从$Y$获得信息之后,再观测$X$获得的信息就是$H(X|Y)$,从而$Y$提供的关于$X$的信息就是
\begin{equation}
    I(X;Y) = H(X) - H(X|Y) = H(X, Y) - H(X|Y) - H(Y|X),
\end{equation}
称为\textbf{互信息}。显然
\begin{equation}
    I(X;Y) = I(Y;X),
\end{equation}
且由于
\[
    0 \leq H(X|Y) \leq H(x),
\]
我们有
\begin{equation}
    0 \leq I(X;Y) \leq \min(H(X), H(Y)),
\end{equation}
当且仅当$X$和$Y$独立时取左边等号(即$X$和$Y$对确定对方的信息毫无作用),当且仅达存在双射函数$g$使得$X=g(Y)$时取右边等号,此时
\[
    H(X) = H(Y).
\]

\subsubsection{编码和比特}

如果信源$X$可以被一个单射映射到信源$Y$上,我们就说$Y$\textbf{编码}了$X$。编码的概念是直观上非常合理的。当然,如果$X$和$Y$之间能够建立一一对应,那么它们的信息量就是完全一样的;反之,$Y$中就含有\textbf{冗余信息},此时$Y$的信息熵大于$X$的信息熵。
总之,如果$H(Y) \geq H(X)$,那么$Y$就可以编码$X$,否则不行。
% TODO:这里缺乏一个反向的论证,即是否任意两个集合之间都可以建立某种映射;可能涉及一些集合论比较tricky的地方

我们下面寻找一种非常简单,具有非常良好的性质,然而又足够普适的编码方案。设$X$是$N$个取值的随机变量,那么就有
\[
    H(X) = - \sum_{i=1}^N p_i \log p_i, \quad \sum_i p_i = 1,
\]
容易验证让$H(X)$最大的选择是让所有事件都是等可能的,此时
\begin{equation}
    H(X) = \log N.
\end{equation}
进一步,我们假定
\[
    N = 2^M,
\]
那么$X$又可以编码为一系列彼此独立的二值随机变量的积,每一个事件都可以写成$100110\cdots$的形式,且
\[
    H(X) = M.
\]
如果不以$2$为底数上式需要乘上一个因子。现在我们看到了信息熵在编码上的含义:如果$X$可以使用$M$个彼此独立的二值随机变量(称为$n$个\textbf{比特})的积编码,那么它的信息量就是$M$。
一般情况下,$X$的信息量不会是一个整数,设
\[
    M < H(X) < M+1,
\]
则$M$个比特不足以编码$X$,$M+1$个比特足以编码$X$但是会有冗余。

\end{document}