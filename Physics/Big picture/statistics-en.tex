\documentclass[a4paper]{article}

\usepackage{geometry}
\usepackage{titling}
\usepackage{titlesec}
\usepackage{paralist}
\usepackage{footnote}
\usepackage{enumerate}
\usepackage{amsmath, amssymb, amsthm}
\usepackage{cite}
\usepackage{graphicx}
\usepackage{subfigure}
\usepackage{physics}
\usepackage{tikz}
\usepackage[colorlinks, linkcolor=black, anchorcolor=black, citecolor=black]{hyperref}

\geometry{left=3.18cm,right=3.18cm,top=2.54cm,bottom=2.54cm}
\titlespacing{\paragraph}{0pt}{1pt}{10pt}[20pt]
\setlength{\droptitle}{-5em}
\preauthor{\vspace{-10pt}\begin{center}}
\postauthor{\par\end{center}}

\DeclareMathOperator{\timeorder}{T}
\newcommand*{\ii}{\mathrm{i}}
\newcommand*{\ee}{\mathrm{e}}
\newcommand*{\diff}{\mathop{}\!\mathrm{d}}
\newcommand*{\st}{\quad \text{s.t.} \quad}
\newcommand*{\const}{\mathrm{const}}
\newcommand*{\comment}{\paragraph{Comment}}
\newcommand*{\scheq}{Schr\"odinger's Equation}

\newenvironment{bigcase}{\left\{\quad\begin{aligned}}{\end{aligned}\right.}

\title{Statistics}
\author{Wu Jinyuan}

\begin{document}

\maketitle

\section{Ensemble}

Consider a quantum system described by the time independent \scheq 
\[
    \hat{H} \ket{\psi} = E \ket{\psi}
\]
Energy eigenstates $\ket{\psi}$ are called \emph{microstates}.

Suppose $\ket{n}$s are eigenstates of an observable operator $\hat{O}$. 
The expectation value is
\[
    \expval{\hat{O}} = \sum_n p(n) \mel{n}{\hat{O}}{n}, \quad p(n) = \braket{\psi}{n} \braket{n}{\psi}
\] 

Question: does the so called $p(n)$ appropriate for large systems?

System in \emph{equilibrium}: 

\emph{For an isolated system in equilibrium, all accessible microstates are equally likely.}

$\Sigma(E)$ to be the number of states with energy $E$.

\begin{equation}
    p(n) = \frac{1}{\Sigma(E)}
\end{equation}

The probability that the system is in a state with some different energy $E' \neq E$ is zero.

\begin{equation}
    S(E) = k_B \log \Sigma(E)
\end{equation}

Consider two non-interacting systems with energies $E_1$ and $E_2$ respectively. 

\begin{equation}
    \Sigma(E_1, E_2) = \Sigma_1 (E_1) \Sigma_2 (E_2), \quad 
    S(E_1, E_2) = S_1(E_1) + S_2(E_2)
\end{equation}

\begin{equation}
    p(n) = \frac{\ee^{-E_n / k_B T}}{\sum_m \ee^{-E_m / k_B T}}
\end{equation}

\begin{equation}
    \hat{\rho} = \frac{\ee^{-beta \hat{H}}}{Z}
\end{equation}

\begin{equation}
    p(\phi) = \expval{\hat{\rho}}{\phi}
\end{equation}

\[
    S = -k_B \tr \hat{\rho} \log \hat{\rho}
\]

\end{document}