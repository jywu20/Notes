\documentclass[hyperref, UTF8, a4paper]{ctexart}

\usepackage{geometry}
\usepackage{titling}
\usepackage{titlesec}
\usepackage{paralist}
\usepackage{footnote}
\usepackage{enumerate}
\usepackage{amsmath, amssymb, amsthm}
\usepackage{autobreak}
\usepackage{cite}
\usepackage{graphicx}
\usepackage{subfigure}
\usepackage{physics}
\usepackage{tikz}
\usepackage[colorlinks, linkcolor=black, anchorcolor=black, citecolor=black]{hyperref}
\usepackage{prettyref}

\geometry{left=3.18cm,right=3.18cm,top=2.54cm,bottom=2.54cm}
\titlespacing{\paragraph}{0pt}{1pt}{10pt}[20pt]
\setlength{\droptitle}{-5em}
\preauthor{\vspace{-10pt}\begin{center}}
\postauthor{\par\end{center}}

\DeclareMathOperator{\timeorder}{T}
\DeclareMathOperator{\diag}{diag}
\DeclareMathOperator{\legpoly}{P}
\newcommand*{\ii}{\mathrm{i}}
\newcommand*{\ee}{\mathrm{e}}
\newcommand*{\const}{\mathrm{const}}
\newcommand*{\comment}{\paragraph{注记}}

\newrefformat{sec}{第\ref{#1}节}
\newrefformat{note}{注\ref{#1}}
\renewcommand{\autoref}{\prettyref}

\newcommand{\concept}[1]{\underline{\textbf{#1}}}
\renewcommand{\emph}{\textbf}

\newenvironment{bigcase}{\left\{\quad\begin{aligned}}{\end{aligned}\right.}

\title{连续介质力学}
\author{吴何友}

\begin{document}

\maketitle

\section{运动学}

设连续介质在完全不受外力作用时占据空间$\Omega_\text{s}$,经过形变之后占据空间$\Omega_\text{d}$。用$\vb*{r}^\text{s}$表示$\Omega_\text{s}$中的点,用$\vb*{r}^\text{d}$表示$\Omega_\text{d}$中的点,则映射
\[
    f: \Omega_\text{s} \longrightarrow \Omega_\text{d}, \; \vb*{r}^\text{s} \mapsto \vb*{r}^\text{d}
\]
就是\concept{形变映射}。我们通常认为$f$是处处连续的。矢量
\begin{equation}
    \vb*{u} = \vb*{r}^\text{d} - \vb*{r}^\text{s}
\end{equation}
就是\concept{形变}。形变可以定义成$\vb*{r}^\text{d}$的函数,也可以定义成$\vb*{r}^\text{s}$的函数,当然也可以定义成背景坐标系$\vb*{r}$的函数。
$\vb*{r}^\text{d}$是一个不随时间变化的构型;它的作用仅仅是用于扣除一个“背景”。

连续介质在某一点的运动速度$\vb*{v}$的形式略微复杂,因为“某一点”本身是一个含糊不清的概念。
我们可以将$\vb*{v}$写成$\vb*{r}^\text{s}$和$t$的函数,即所谓\concept{拉格朗日法},那么显然有
\begin{equation}
    \vb*{v}(\vb*{r}^\text{s}, t) = \dv{\vb*{r}^\text{s}}{t} = \left( \pdv{\vb*{r}^\text{s}}{t} \right)_{\vb*{r}}.
\end{equation}
第二个等号是因为$\vb*{r}^\text{s}$可以写成$\vb*{r}$和$t$的函数,而考虑到$\vb*{r}$是始终静止的,时间导数就是对$t$的偏导数。
我们也可以将$\vb*{v}$写成$\vb*{r}$和$t$的函数,即所谓\concept{欧拉法}。
时刻$t$,$\vb*{r}$点的介质运动速度当然就是时刻$t$,运动到了$\vb*{r}$点的介质的运动速度,也即是时刻$t$,$\vb*{r}^\text{s}=\vb*{r}$的$\vb*{v}(\vb*{r}^\text{s}, t)$。
等式左边的形式比较有趣,
我们定义场变量$\varphi$的\concept{物质导数}是$\varphi(\vb*{r}=\vb*{r}^\text{d}, t)$的时间全导数,即
\begin{equation}
    \dv{\varphi}{t} = \pdv{\varphi}{t} + \vb*{v} \cdot \grad \varphi,
\end{equation}
或者也可以写成
\begin{equation}
    \dv{t} = \pdv{t} + \vb*{v} \cdot \grad.
\end{equation}

\section{准热力学平衡}

\end{document}