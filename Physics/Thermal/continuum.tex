\documentclass[hyperref, UTF8, a4paper]{ctexart}

\usepackage{geometry}
\usepackage{titling}
\usepackage{titlesec}
\usepackage{paralist}
\usepackage{footnote}
\usepackage{enumerate}
\usepackage{amsmath, amssymb, amsthm}
\usepackage{autobreak}
\usepackage{cite}
\usepackage{graphicx}
\usepackage{subfigure}
\usepackage{physics}
\usepackage{tikz}
\usepackage[colorlinks, linkcolor=black, anchorcolor=black, citecolor=black]{hyperref}
\usepackage{prettyref}

\geometry{left=3.18cm,right=3.18cm,top=2.54cm,bottom=2.54cm}
\titlespacing{\paragraph}{0pt}{1pt}{10pt}[20pt]
\setlength{\droptitle}{-5em}
\preauthor{\vspace{-10pt}\begin{center}}
\postauthor{\par\end{center}}

\DeclareMathOperator{\timeorder}{T}
\DeclareMathOperator{\diag}{diag}
\DeclareMathOperator{\legpoly}{P}
\newcommand*{\ii}{\mathrm{i}}
\newcommand*{\ee}{\mathrm{e}}
\newcommand*{\const}{\mathrm{const}}
\newcommand*{\comment}{\paragraph{注记}}

\newrefformat{sec}{第\ref{#1}节}
\newrefformat{note}{注\ref{#1}}
\renewcommand{\autoref}{\prettyref}

\newcommand{\concept}[1]{\underline{\textbf{#1}}}
\renewcommand{\emph}{\textbf}

\newenvironment{bigcase}{\left\{\quad\begin{aligned}}{\end{aligned}\right.}

\title{连续介质力学}
\author{吴何友}

\begin{document}

\maketitle

以下在没有明确指定空间维数时,记维数为$d$。

\section{运动学}

\subsection{形变和物质导数}

设连续介质在完全不受外力作用时占据空间$\Omega_\text{s}$,经过形变之后占据空间$\Omega_\text{d}$。用$\vb*{r}_\text{s}$表示$\Omega_\text{s}$中的点,用$\vb*{r}_\text{d}$表示$\Omega_\text{d}$中的点,则映射
\[
    f: \Omega_\text{s} \longrightarrow \Omega_\text{d}, \; \vb*{r}_\text{s} \mapsto \vb*{r}_\text{d}
\]
就是\concept{形变映射}。我们通常认为$f$是处处连续的。矢量
\begin{equation}
    \vb*{u} = \vb*{r}_\text{d} - \vb*{r}_\text{s}
\end{equation}
就是\concept{形变}。形变可以定义成$\vb*{r}_\text{d}$的函数,也可以定义成$\vb*{r}_\text{s}$的函数。
$\vb*{r}_\text{s}$是一个不随时间变化的构型;它的作用仅仅是用于扣除一个“背景”。

连续介质在某一点的运动速度$\vb*{v}$的形式略微复杂,因为“某一点”本身是一个含糊不清的概念。
我们可以将$\vb*{v}$写成$\vb*{r}_\text{s}$和$t$的函数,即追踪形变开始前的每一个质点在形变之后流动到了哪里(这是所谓的\concept{拉格朗日法}),那么显然有
\begin{equation}
    \vb*{v}(\vb*{r}_\text{d}, t) = \dv{\vb*{r}_\text{d}}{t} = \left( \pdv{\vb*{r}_\text{d}}{t} \right)_{\vb*{r}_\text{s}} = \left( \pdv{\vb*{u}}{t} \right)_{\vb*{r}_\text{s}}.
\end{equation}
第三个等号是因为$\vb*{r}_\text{d}$可以写成$\vb*{r}_\text{s}$和$t$的函数,而考虑到$\vb*{r}_\text{s}$是始终静止的,时间导数就是对$t$的偏导数。
我们也可以将$\vb*{v}$写成$\vb*{r}_\text{d}$和$t$的函数,即所谓\concept{欧拉法}。
欧拉法中速度的表达式为
\begin{equation}
    \begin{aligned}
        \vb*{v} &= \left( \pdv{\vb*{u}}{t} \right)_{\vb*{r}_\text{s}} = \left( \pdv{\vb*{u}}{t} \right)_{\vb*{r}_\text{d}} + \left( \pdv{\vb*{r}_\text{d}}{t} \right)_{\vb*{r}_\text{d}} \cdot \pdv{\vb*{u}}{\vb*{r}_\text{d}} \\
        &= \left( \pdv{\vb*{u}}{t} \right)_{\vb*{r}_\text{d}} + \vb*{v} \cdot \pdv{\vb*{u}}{\vb*{r}_\text{d}}.
    \end{aligned}
\end{equation}
上式中多出来了一个非线性项,这是因为某一个空间点上这一时刻和下一时刻的微团是不一样的,从而会有一个漂移项。

上面的定义可以推广到一般的场变量上。场变量$\varphi$的\concept{物质导数}是$\varphi(\vb*{r}_\text{d}, t)$的时间全导数,于是
\begin{equation}
    \begin{aligned}
        \dv{\varphi}{t} &= \left( \pdv{\varphi}{t} \right)_{\vb*{r}_\text{d}} + \left( \pdv{\vb*{r}_\text{d}}{t} \right)_{\vb*{r}_\text{d}} \cdot \pdv{\varphi}{\vb*{r}_\text{d}} \\
        &= \left( \pdv{\varphi}{t} \right)_{\vb*{r}_\text{d}} + \vb*{v} \cdot \pdv{\varphi}{\vb*{r}_\text{d}}.
    \end{aligned}
\end{equation}

欧拉法通常比较“现实”,因为实验中只能确定形变场中在某个实验室坐标系下某一点的物理量(流速、密度等),而确定不了各个微团一开始在哪里,即实际可以测量的物理量都是$\vb*{r}_\text{d}$的函数;同样,对连续介质施加外力也是施加在一个实验室坐标系中固定不动的一点上。
于是,我们将所有公式中的$\vb*{r}_\text{d}$替换为$\vb*{r}$,从而物质导数就是
\begin{equation}
    \dv{t} = \pdv{t} + \vb*{v} \cdot \grad.
\end{equation}
这样,加速度就是
\begin{equation}
    \dv{\vb*{v}}{t} = \pdv{\vb*{v}}{t} + \vb*{v} \cdot \grad{\vb*{v}}.
\end{equation}

\subsection{输运}

设$\rho$是某个随流量的密度,也即,它是微观下某个流体粒子携带的荷的密度,设$f$是(欧拉法下的——以下如无特殊说明均取欧拉法)此随流量的源或者汇的分布函数,则守恒性就是
\[
    \rho(\vb*{r}_\text{d}) \dd[d]{\vb*{r}_\text{d}} + f(\vb*{r}_\text{d}) \dd{t} \dd[d]{\vb*{r}_\text{d}} = \rho'(\vb*{r}_\text{d}') \dd[d]{\vb*{r}'_\text{d}},
\]
考虑到
\[
    \frac{\dd[d]{\vb*{r}_\text{d}'}}{\dd[d]{\vb*{r}_\text{d}}} = \det( 1 + \pdv{(\vb*{r}_\text{d} + \vb*{v} \dd{t})}{\vb*{r}_\text{d}} ) = 1 + \dd{t} \div{\vb*{v}},
\]
以及
\[
    \rho'(\vb*{r}_\text{d}') - \rho(\vb*{r}_\text{d}) = \dv{\rho}{t} \dd{t},
\]
我们就得到
\begin{equation}
    \dv{\rho}{t} + \rho \div{\vb*{v}} = \pdv{\rho}{t} + \div{(\rho \vb*{v})} = f.
    \label{eq:transportation-eq}
\end{equation}
这就是\concept{连续性方程}。

\section{准热力学平衡}

我们假定连续介质运动过程中几乎总是保持准热力学平衡——这就是说,虽然介质整体显然没有达到热力学平衡,但是介质中每一个宏观小微观大的微团都可以近似认为达到了热力学平衡。
这样就可以使用诸如“单位体积的熵变”、“微团的热力学”等概念推导其行为。

\section{流体动力学}

对流体,我们有\concept{纳维-斯托克斯方程}
\begin{equation}
    \rho \left( \pdv{\vb*{v}}{t} + \vb*{v} \cdot \grad{\vb*{v}} \right) = - \grad{P} + \vb*{f}.
\end{equation}

\subsection{声波}

当速度的时间变化相比于空间输运非常大时,即
\[
    \pdv{t} \gg \vb*{v} \cdot \grad
\]
时,近似有
\begin{equation}
    \rho \pdv{\vb*{v}}{t} = - \grad{p},
    \label{eq:ns-eq-small-v}
\end{equation}
两边计算散度,并利用输运方程\eqref{eq:transportation-eq},得到
\[
    \laplacian{p} = \pdv[2]{\rho}{t},
\]
再假定压强变化不大,有
\[
    \rho = \eval{\pdv{\rho}{P}}_{P_0} (P - P_0) = \eval{\pdv{\rho}{P}}_{P_0} p,
\]
于是就得到波动方程
\begin{equation}
    \frac{1}{c^2} \pdv[2]{p}{t} = \laplacian{p},
    \label{eq:sound-wave-fluid}
\end{equation}
其中
\begin{equation}
    \frac{1}{c^2} = \eval{\pdv{\rho}{P}}_{P_0}.
\end{equation}
这就是说,快速振动而振幅不大的流体中会有线性机械波,这就是\concept{声波}。

声波一定是横波,因为\eqref{eq:ns-eq-small-v}两边同取旋度,就有
\[
    \pdv{t} \curl{\vb*{v}} = 0, 
\]
即$\curl{\vb*{v}}$不随时间变化。由于\eqref{eq:sound-wave-fluid}是线性的,我们可以只取其小幅振动的解,即从一个一般的解中剥离整体平移运动的部分。这种小幅振动的解的基底不妨取为平面波,而对每个平面波,都有$\curl{\vb*{v}}=0$,于是我们就得出结论:声波是无旋的。
实际上可以直接从压强的性质出发得到这个结论,因为横波要求剪力而压强不能提供剪力。

\end{document}