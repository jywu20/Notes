\documentclass[hyperref, UTF8, a4paper]{ctexart}

\usepackage{geometry}
\usepackage{titling}
\usepackage{titlesec}
\usepackage{paralist}
\usepackage{footnote}
\usepackage{enumerate}
\usepackage{amsmath, amssymb, amsthm}
\usepackage{bbm}
\usepackage{cite}
\usepackage{graphicx}
\usepackage{subfigure}
\usepackage{physics}
\usepackage{tikz}
\usepackage{autobreak}
\usepackage[ruled, vlined, linesnumbered, noend]{algorithm2e}
\usepackage[colorlinks, linkcolor=black, anchorcolor=black, citecolor=black]{hyperref}
\usepackage{prettyref}

% Page style
\geometry{left=3.18cm,right=3.18cm,top=2.54cm,bottom=2.54cm}
\titlespacing{\paragraph}{0pt}{1pt}{10pt}[20pt]
\setlength{\droptitle}{-5em}
\preauthor{\vspace{-10pt}\begin{center}}
\postauthor{\par\end{center}}

% Math operators
\DeclareMathOperator{\timeorder}{T}
\DeclareMathOperator{\diag}{diag}
\DeclareMathOperator{\legpoly}{P}
\DeclareMathOperator{\primevalue}{P}
\DeclareMathOperator{\sgn}{sgn}
\newcommand*{\ii}{\mathrm{i}}
\newcommand*{\ee}{\mathrm{e}}
\newcommand*{\const}{\mathrm{const}}
\newcommand*{\comment}{\paragraph{注记}}
\newcommand*{\suchthat}{\quad \text{s.t.} \quad}
\newcommand*{\argmin}{\arg\min}
\newcommand*{\argmax}{\arg\max}
\newcommand*{\normalorder}[1]{: #1 :}
\newcommand*{\pair}[1]{\langle #1 \rangle}
\newcommand*{\fd}[1]{\mathcal{D} #1}
\DeclareMathOperator{\bigO}{\mathcal{O}}

% prettyref setting
\newrefformat{sec}{第\ref{#1}节}
\newrefformat{note}{注\ref{#1}}
\newrefformat{fig}{图\ref{#1}}
\newrefformat{alg}{算法\ref{#1}}
\renewcommand{\autoref}{\prettyref}

% TikZ setting
\usetikzlibrary{arrows,shapes,positioning}
\usetikzlibrary{arrows.meta}
\usetikzlibrary{decorations.markings}
\tikzstyle arrowstyle=[scale=1]
\tikzstyle directed=[postaction={decorate,decoration={markings,
    mark=at position .5 with {\arrow[arrowstyle]{stealth}}}}]
\tikzstyle ray=[directed, thick]
\tikzstyle dot=[anchor=base,fill,circle,inner sep=1pt]

% Algorithm setting
\renewcommand{\algorithmcfname}{算法}
% Python-style code
\SetKwIF{If}{ElseIf}{Else}{if}{:}{elif:}{else:}{}
\SetKwFor{For}{for}{:}{}
\SetKwFor{While}{while}{:}{}
\SetKwInput{KwData}{输入}
\SetKwInput{KwResult}{输出}
\SetArgSty{textnormal}

\renewcommand{\emph}[1]{\textbf{#1}}
\newcommand*{\concept}[1]{\underline{\textbf{#1}}}
\newcommand*{\Ztwo}{$\mathbb{Z}_2$}

\title{阻尼}
\author{吴何友}

\begin{document}

\maketitle

\section{谐振子热浴}

所谓阻尼无非是我们考虑的系统出于某些原因和某个系统(“热浴”)发生了耦合,导致能量逃逸到了外界。
本节讨论最为简单的谐振子热浴。谐振子的实时间作用量为
\begin{equation}
    S = \int \dd{t} \left( \frac{1}{2} m \dot{x}^2 - \frac{1}{2} m \omega_0^2 x^2 \right) = - \int \dd{t} x \left( \frac{1}{2} m \dv[2]{t} + \frac{1}{2} m \omega_0^2 \right) x,
\end{equation}
于是实时间格林函数为(这里涉及一些比较麻烦的符号问题,比如说格林函数要不要加上一个负号或是虚数单位;由于我们不需要计算散射截面之类,暂且采取以下比较简洁的写法)
\begin{equation}
    G(t-t') = \left( m \dv[2]{t} + m \omega_0^2 \right)^{-1},
\end{equation}
在时域计算此格林函数并不方便,我们切换到频域:
\begin{equation}
    G(\omega) = \frac{1}{- m \omega^2 + m \omega_0^2 - \ii 0^+},
\end{equation}
此处已经为了保证因果性加入了$\ii 0^+$。现在假定某个变量$y(t)$和$x(t)$发生线性耦合,即
\begin{equation}
    S = S_y + S_x + S_\text{int}, \quad S_\text{int} = g \int \dd{t} xy,
\end{equation}
考虑到$S_x$是自由的,我们可以非常容易地严格积掉$x$自由度,只保留$y$,得到
\begin{equation}
    \begin{aligned}
        S_\text{eff} &= S_y + \frac{g}{2} \int \dd{t} \dd{t'} y(t) G(t-t') y(t') \\
        &= S_y + \frac{g}{2} \int \dd{\omega} y(-\omega) G(\omega) y(\omega).
    \end{aligned}
\end{equation}
我们得到一个推迟相互作用,这是不足为怪的,因为$x$变量本身需要时间去响应$y$引入的冲击,然后再传递给$y$,那么必然导致推迟相互作用。

以上步骤可以很容易地推广。我们更换记号,用$x(t)$表示我们关心的场而使用$q_i(t)$表示一群谐振子(因为热库通常非常大,肯定是一群数量很大的谐振子),那么就有
\begin{equation}
    S = \underbrace{\int \dd{t} \left( \frac{1}{2} m \dot{x}^2 + F(t) x(t) - V(x) \right)}_{S_x} + \int \dd{t} \sum_i g_i q_i(t) x(t) + \underbrace{\int \dd{t} \sum_i \left( \frac{1}{2} \dot{q}_i^2 - \frac{1}{2} \omega_i^2 q_i \right)}_{S_q}.
\end{equation}

\end{document}