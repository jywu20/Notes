\documentclass{article}
\usepackage{amsmath}
\usepackage[scale=0.8]{geometry}
\usepackage{graphicx}
\usepackage{subfigure}
\usepackage{latexsym}
\usepackage{amsfonts}
\usepackage{amssymb}
\usepackage{dsfont}
%\usepackage{bblod}
\usepackage{bbm}
\usepackage{eucal}
\usepackage{extarrows}
\usepackage{hyperref}
\usepackage{tikz}
\usepackage{esint}
\usepackage{upgreek}
\usepackage{inputenc}
\usepackage{mathrsfs}

\usepackage[all]{hypcap}

\usepackage[T1]{fontenc}
\usepackage{utopia}
\usepackage{physics}
\usepackage{tikz}
\usepackage{mathdots}
\usepackage{yhmath}
\usepackage{cancel}
\usepackage{color}
\usepackage{siunitx}
\usepackage{array}
\usepackage{multirow}
\usepackage{amssymb}
\usepackage{gensymb}
\usepackage{tabularx}
\usepackage{extarrows}
\usepackage{booktabs}
\usetikzlibrary{fadings}
\usetikzlibrary{patterns}
\usetikzlibrary{shadows.blur}
\usetikzlibrary{shapes}

\usepackage{fontspec}
\setmainfont{Asana-Math}
\usepackage{upgreek}

\title{ The angular momenta theory and Representation of SO(3)}
\date{2021.3.29}
\author{Photon gjq}


\begin{document}
	\maketitle
	\tableofcontents
	
	\newtheorem{definition}{Definition}
	


	\section{Introduction}
	In this article, we will use the representation theory to approach the angular momentum therory of quantum mechanics in $3$ dimensions. We will see Clebsch-Gordon coefficients and Wigner-Eckart theorem, but first and at last, we will figure out the representation of $SO( 3)$ and $SU( 2)$. 
	
	
	
	Notation: Let $SO( 3)$ be the $3$ dimensional rotational group(orthogonal group) and $\mathfrak{so}( 3)$ be its Lie algebra. Call the generator of $SO( 3)$ be $\mathcal{J}_{1} ,\mathcal{J}_{2} ,\mathcal{J}_{3}$:
	\begin{equation*}
		\mathcal{J}_{1} =\begin{pmatrix}
			0 & 0 & 0\\
			0 & 0 & 1\\
			0 & -1 & 0
		\end{pmatrix} ,\mathcal{J}_{2} =\begin{pmatrix}
			0 & 0 & -1\\
			0 & 0 & 0\\
			1 & 0 & 0
		\end{pmatrix} ,\mathcal{J}_{3} =\begin{pmatrix}
			0 & 1 & 0\\
			-1 & 0 & 0\\
			0 & 0 & 0
		\end{pmatrix}
	\end{equation*}
	which is the basis of $\mathfrak{so}( 3)$, namely an $\mathfrak{so}( 3)$ element can be written as:
	\begin{equation*}
		A=\theta _{1}\mathcal{J}_{1} +\theta _{2}\mathcal{J}_{2} +\theta _{3}\mathcal{J}_{3} ,\theta _{i} \in \mathbb{R}
	\end{equation*}
	which generate an $SO( 3)$ element by:
	\begin{equation*}
		D( R) =\mathrm{e}^{\theta _{1}\mathcal{J}_{1} +\theta _{2}\mathcal{J}_{2} +\theta _{3}\mathcal{J}_{3}}
	\end{equation*}
	Here $D( R)$ is in fact the \textbf{defining representation} of $SO( 3)$.	In QM, we often define the generators to be
	\begin{equation*}
		J_{i} =-\mathrm{i} J_{i} ,i=1,2,3
	\end{equation*}
	Then the exponential map becomes $D( R) =\exp(\mathrm{i}\vec{\theta } \cdot \vec{J})$. And we have the commutation relation:
	\begin{equation*}
		\boxed{[ J_{a} ,J_{b}] =\mathrm{i} \epsilon ^{abc} J_{c}}
	\end{equation*}
	\section{Adjoint representation}
	
	Like the regular representation of the Lie group, the \textbf{adjoint represent} of Lie algebra view the generator of the algebra as the basis of the representation space. The generator acts on the basis as:
	\begin{gather*}
		J_{a}\ket{J_{b}} =D^{\mathrm{ad}}( J_{a})\ket{[ J_{a} ,J_{b}]} =\mathrm{i} \epsilon ^{abc}\ket{J_{c}}\\
		\Rightarrow D^{\mathrm{ad}}( J_{a})_{db} =\mel**{J_{d}}{J_{a}}{J_{b}} =\mathrm{i} \epsilon ^{abd}
	\end{gather*}
	For example, if we write $D^{\mathrm{ad}}( J_{a})$ directly:
	\begin{gather*}
		D^{\mathrm{ad}}( J_{1})_{11} =\mathrm{i} \epsilon ^{111} =0,D^{\mathrm{ad}}( J_{1})_{12} =\mathrm{i} \epsilon ^{112} =0\cdots \\
		\Rightarrow D^{\mathrm{ad}}( J_{1}) =\begin{pmatrix}
			0 & 0 & 0\\
			0 & 0 & -\mathrm{i}\\
			0 & \mathrm{i} & 0
		\end{pmatrix} =J_{1}
	\end{gather*}
	We can easily generalize it to $SO( N)$. In quantum mechanics, the angular momentum is given by:
	\begin{equation*}
		\mathbf{L} =\mathbf{r} \times \mathbf{p} =\begin{pmatrix}
			yp_{z} -zp_{y}\\
			zp_{x} -xp_{z}\\
			yp_{x} -xp_{y}
		\end{pmatrix}
	\end{equation*}
	In position space, $\mathbf{p} =-\mathrm{i}\mathbf{\nabla }$, which means:
	\begin{equation*}
		\mathbf{L} =-\mathrm{i}\begin{pmatrix}
			y\partial _{z} -z\partial _{y}\\
			z\partial _{x} -x\partial _{z}\\
			y\partial _{x} -x\partial _{y}
		\end{pmatrix} =\begin{pmatrix}
			J_{1}\\
			J_{2}\\
			J_{3}
		\end{pmatrix}
	\end{equation*}
	because there is a natrual homomorphism:
	\begin{equation*}
		\begin{pmatrix}
			0 & 0 & 0\\
			0 & 0 & 1\\
			0 & -1 & 0
		\end{pmatrix} \mapsto y\frac{\partial }{\partial z} -z\frac{\partial }{\partial y} ,\begin{pmatrix}
			0 & 0 & -1\\
			0 & 0 & 0\\
			1 & 0 & 0
		\end{pmatrix} \mapsto z\frac{\partial }{\partial x} -x\frac{\partial }{\partial z} ,\begin{pmatrix}
			0 & 1 & 0\\
			-1 & 0 & 0\\
			0 & 0 & 0
		\end{pmatrix} \mapsto y\frac{\partial }{\partial x} -x\frac{\partial }{\partial y}
	\end{equation*}
	
	\section{The representation of SO(3)}
	
	If you cannot prove that $SO( 3)$ have infinite different irreducible representations, read this. Route: First separate the $\mathfrak{so}( 3)$ generators into a \textbf{Cartan subalgebra} and the ladder operators, then act the ladder operators on the eigenvectors of the Cartan subalgebra generators, the corresponding representation basis states are labeled by the eigenvalues of Cartan subalgebra generators. 
	
	
	\begin{definition}
		\textbf{Cartan subalgebra}
		
		For a Lie algebra $\mathfrak{g}$, a cartan subalgebra is an nilpotent subalgebra $\mathfrak{h}$ s.t. if $[ x,y] \in \mathfrak{h}$ for all $x\in \mathfrak{h}$, then $y\in \mathfrak{h}$. \ 
	\end{definition}
	We can see the Cartan subalgebra of $\mathfrak{so}( 3)$ only contains one element $J_{3}$, because all generators of $\mathfrak{so}( 3)$ do not commute. Suppose the eigenvector of $J_{3}$ is written as $\ket{m}$, which means
	\begin{equation*}
		J_{3}\ket{m} =m\ket{m}
	\end{equation*}
	We will see $m\in \mathbb{Z}$. As usual, we define the ladder operator to be:
	\begin{gather*}
		J_{\pm } =J_{1} \pm \mathrm{i} J_{2}\\
		\Rightarrow J_{\pm } =( J_{\mp })^{\dagger } ,[ J_{3} ,J_{\pm }] =\pm J_{\pm } ,[ J_{+} ,J_{-}] =2J_{z}
	\end{gather*}
	Next we use the ladder operator to construct the finite-dimensional unitary representation. Because the rep is finite-dimensional, we must have $J_{+}\ket{m} =0$ for some maximal value of $m$ and $J_{-}\ket{m} =0$ for some minimal value of $m$, call it $j$, then as usual derive in elementary QM, we have:
	\begin{equation*}
		J_{\pm }\ket{j,m} =\sqrt{( j\mp m)( j+1\pm m)}\ket{j,m\pm 1}
	\end{equation*}
	Here we have used the convention such that the coefficients are positive. So we can label an irreducible representation of $\mathfrak{so}( 3)$ by $j$ and denoted a basis states by $\ket{j,m}$, with $m=−j,−j+1,\cdots ,j−1,j$, and this means $j$ can only be integers or half integers. The half-integer $j$-representations are projective representations of $SO( 3)$, instead of linear
	
	representations. These representations are christened the \textbf{spinor representations} of $\mathfrak{so}( 3)$ and $SO( 3)$, and are linear representation of $SU( 2)$. 
	
	
	\section{Tensor product decomposition of two different $j$-rep}
	
	All the $j \rangle 2$ representations are irreducible \textbf{tensor representations} with rank higher than two and cannot be described by matrices. For example, $j=2$ irr-rep is the symmetric and traceless matrices. Next, we will decompose a tensor-product representation from two different $j$-rep. Consider $j=1\otimes j=1$, i.e. $1\otimes 1$. The tensor $T$ is given by: $T^{ij} =V^{i} \otimes W^{j}$, where $V^{i}$ and $W^{j}$ are the basis of 3-dimensional representation space of $SO( 3)$, and thus $T^{ij}$ is the basis of the $9$-dim representation space of $SO( 3)$. Like the basis of normal vector representation space transforms under the representation, the tensor transforms as:
	\begin{equation*}
		T^{kl}\rightarrow T^{\prime mn} =R^{mk} R^{nl} T^{kl}
	\end{equation*}
	So if we can show some component of $T^{kl}$ transforms under $R$ also like a tensor, then we can decompose this representation in to smaller representation. So with our eyeballs, we can see:
	\begin{equation*}
		T^{kl} =T^{[ kl]} +\tilde{T}^{\{kl\}} +\frac{1}{3} \delta ^{kl}\operatorname{tr} T
	\end{equation*}
	Where $T^{[ kl]} =\left( T^{kl} -T^{lk}\right) /2$, $T^{\{kl\}} =\left( T^{kl} +T^{lk}\right) /2$, and $\tilde{T}^{kl}$ is defined by
	\begin{equation*}
		T^{\{kl\}} =\tilde{T}^{\{kl\}} +\frac{1}{3} \delta ^{kl}\operatorname{tr} T
	\end{equation*}
	The three components transforms like a tensor under $R$, which are $3$, $5$ and $1$ dimensional representation because they have such numbers of independent of degrees of freedom(which equals the dimension of vector space they have spanned). So we conclude:
	\begin{equation*}
		1\otimes 1=0\oplus 1\oplus 2
	\end{equation*}
	In genreal, we conclude that:
	\begin{equation*}
		j\otimes j'=| j-j'| \oplus | j-j'| +1\oplus | j-j'| +2\oplus \cdots \oplus j+j'
	\end{equation*}
	Note here again, this decomposition means you can find a matrix to block diagonalize all $R^{ij} R^{lm}$ at the same time, after that the $9\times 9$ tensor becomes $5\times 5\oplus 3\times 3\oplus 1\times 1$:
	\begin{equation*}
		\begin{pmatrix}
			\begin{bmatrix}
				\vdots  & \vdots  & \cdots  & \vdots  & \vdots \\
				\vdots  & \vdots  & j=2 & \vdots  & \vdots \\
				\vdots  & \vdots  & D^{2}( R) & \vdots  & \vdots \\
				\vdots  & \vdots  & 5\times 5 & \vdots  & \vdots \\
				\vdots  & \vdots  & \cdots  & \vdots  & \vdots 
			\end{bmatrix} & \begin{matrix}
				&  &  & \\
				&  &  & \\
				&  &  & \\
				&  &  & \\
				&  &  & \\
				&  &  & 
			\end{matrix}\\
			\begin{matrix}
				&  &  &  & \\
				&  &  &  & \\
				&  &  &  & \\
				&  &  &  & 
			\end{matrix} & \begin{matrix}
				\begin{bmatrix}
					\vdots  & j=1 & \vdots \\
					\vdots  & D^{1}( R) & \vdots \\
					\vdots  & 3\times 3 & \vdots 
				\end{bmatrix} & \begin{matrix}
					\\\\
					\\\\
					\\
				\end{matrix}\\
				\begin{matrix}
					&  & 
				\end{matrix} & \begin{bmatrix}
					D^{0}( R)
				\end{bmatrix}
			\end{matrix}
		\end{pmatrix}\begin{pmatrix}
			\begin{bmatrix}
				\tilde{T}^{12}\\
				\tilde{T}^{13}\\
				\tilde{T}^{23}\\
				\tilde{T}^{11}\\
				\tilde{T}^{22} \ \ 
			\end{bmatrix}\\
			\begin{bmatrix}
				T^{[ 12]}\\
				T^{[ 23]}\\
				T^{[ 31]}
			\end{bmatrix}\\
			\operatorname{tr} T
		\end{pmatrix}
	\end{equation*}
	$ $ 
	\section{C-G coefficients of $\mathfrak{so}( 3)$}
	
	According to
	\begin{equation*}
		j_{1} \otimes j_{2} =\underset{J=| j_{1} -J_{2}| }{\overset{j_{1} +j_{2}}{\bigoplus }} J
	\end{equation*}
	We set the basis of the representation space of $j_{1} \otimes j_{2}$ as:
	\begin{equation*}
		\ket{j_{1} ,m_{1} ;j_{2} ,m_{2}} \equiv \ket{j_{1} ,m_{1}} \otimes \ket{j_{2} ,m_{2}}\left( \equiv \ket{m_{1} ;m_{2}}\right)
	\end{equation*}
	Our goal is to write the basis vectors $|J,M\rangle  $, for all $J$ in the decomposition and all $M=-J,-J+1,\dotsc ,J,$ as linear combinations of the basis vectors $\ket{m;m'}$. That is,
	\begin{equation}
		|J,M\rangle  =\sum _{m_{1} =-j_{1}}^{j_{1}}\sum _{m_{2} =-j_{2}}^{j_{2}} C_{j_{1} j_{2} m_{1} m_{2}}^{JM}| j_{1} ,m_{1} ;j_{2} ,m_{2}\rangle 
	\end{equation}
	For example, in decomposition $1\otimes 1=2\oplus 1\oplus 0$, we want to get
	\begin{equation*}
		\ket{2,1} =\frac{1}{\sqrt{2}}\ket{1,0} \otimes \ket{1,1} +\frac{1}{\sqrt{2}}\ket{1,1} \otimes \ket{1,0} =\frac{1}{\sqrt{2}}\left(\ket{1,0;1,1} +\ket{1,1;1,0}\right)
	\end{equation*}
	The naive approach it to use the ladder operator to the the c-g coefficients, see any QM book. 
	
	Since all the basis vectors are assumed orthonormal, we have
	\begin{equation*}
		C_{j_{1} j_{2} m_{1} m_{2}}^{JM} =\bra{j_{1} ,m_{1} ;j_{2} ,m_{2}}\ket{J,M}
	\end{equation*}
	Because $| j_{1} ,m_{1} ;j_{2} ,m_{2}\rangle \rightarrow \ket{J,M}$ is a linear transformation, the inverse can be given by:
	\begin{equation*}
		\ket{j_{1} ,m_{1} ;j_{2} ,m_{2}} =\sum _{J=| j_{1} -j_{2}| }^{j_{1} +j_{2}}\sum _{M=-J}^{J} C_{JM}^{j_{1} j_{2} m_{1} m_{2}} |J,M\rangle  
	\end{equation*}
	where
	\begin{equation*}
		C_{JM}^{j_{1} j_{2} m_{1} m_{2}} =\bra{J,M}\ket{j_{1} ,m_{1} ;j_{2} ,m_{2}} =\left(\bra{j_{1} ,m_{1} ;j_{2} ,m_{2}}\ket{J,M}\right)^{*} =C{_{j_{1} j_{2} m_{1} m_{2}}^{JM}}^{*}
	\end{equation*}
	If we regard $C^{j_{1} j_{2}}$ as a matrix with row index $(J,M)$ and column index $( m,m') ,$ it is a unitary matrix, i.e., $C^{j_{1} j_{2}} =C_{j_{1} j_{2}}^{\dagger } =C_{j_{1} j_{2}}^{-1}$, we have $\bra{J,M}\ket{j_{1} ,m_{1} ;j_{2} ,m_{2}} =\bra{j_{1} ,m_{1} ;j_{2} ,m_{2}}\ket{J,M}$. A real unitary matrix is orthogonal, so we have the orthogonality condition
	\begin{equation*}
		\sum _{j}\sum _{m}\bra{j_{1} ,m_{1} ;j_{2} ,m_{2}}\ket{J,M}\bra{J,M}\ket{j_{1} ,m'_{1} ;j_{2} ,m'_{2}} =\delta _{m_{1} m'_{1}} \delta _{m_{2} m'_{2}}
	\end{equation*}
	which is obvious from the orthonormality of $\left\{\ket{j_{1} ,m_{1} ;j_{2} ,m_{2}}\right\}$ together with the reality of the Clebsch-Gordan coefficients. Likewise, we also have
	\begin{equation*}
		\sum _{m_{1}}\sum _{m_{2}}\bra{j_{1} ,m_{1} ;j_{2} ,m_{2}}\ket{J,M}\bra{J',M'}\ket{j_{1} ,m_{1} ;j_{2} ,m_{2}} =\delta _{JJ'} \delta _{MM'}
	\end{equation*}
	Next we consider the $SO( 3)$ and $\mathfrak{so}( 3)$ element acts on the basis vector, note these two actions are different. Suppose an arbitrary $SO( 3)$ element $R$ acts on $\ket{j_{1} ,m_{1} ;j_{2} ,m_{2}}$:
	\begin{equation*}
		\begin{aligned}
			\hat{R}| j_{1} ,m_{1} ;j_{2} ,m_{2}\rangle  & =D^{j_{1} \otimes j_{2}} (R)| j_{1} ,m_{1}\rangle  \otimes | j_{1} ,m_{2}\rangle  \\
			& =\left( D^{j_{1}} (R)\otimes D^{j_{2}} (R)\right)| j_{1} ,m_{1}\rangle  \otimes | j_{1} ,m_{2}\rangle  \\
			& =D^{j_{1}} (R)| j_{1} ,m_{1}\rangle  \otimes D^{j_{2}} (R)| j_{2} ,m_{2}\rangle  
		\end{aligned}
	\end{equation*}
	Next we look at the action of $\mathfrak{so}( 3)$ generator. Wlog, we can assume that $R$ is generated by a particular $\mathfrak{so} (3)$ generator, say, $J_{a} ,$ i.e., $R=$ $\exp(\mathrm{i} \theta J_{a})$. For an infinitesimal $\theta $, we have approximately $R=1+\mathrm{i} \theta J_{a}$. Then $D^{j} (R)=\mathds{1}^{j_{1}} +\mathrm{i} \theta D^{j}( J_{a})$, where the $\mathds{1}^{j_{1}}$ is the $(2j+1)\times (2j+1)$ identity matrix. Hence,
	\begin{equation*}
		\begin{aligned}
			& \left(\mathds{1}^{j_{1} \otimes j_{2}} +\mathrm{i} \theta D^{j_{1} \otimes j_{2}}( J_{a})\right)| j_{1} ,m_{1} ;j_{2} ,m_{2}\rangle \\
			= & \left(\mathds{1}^{j_{1}} +\mathrm{i} \theta D^{j_{1}}( J_{a})\right)| j_{1} ,m_{1}\rangle  \otimes \left(\mathds{1}^{j_{2}} +\mathrm{i} \theta D^{j_{2}}( J_{a})\right)| j_{2} ,m_{2}\rangle  \\
			= & \left(\mathds{1}^{j_{1}} \otimes \mathds{1}^{j_{2}} +\mathrm{i} \theta D^{j_{1}}( J_{a}) \otimes \mathds{1}^{j_{2}} +\mathrm{i} \theta \mathds{1}^{j_{1}} \otimes D^{j_{2}}( J_{a})\right)| j_{1} ,m_{1} ;j_{2} ,m_{2}\rangle 
		\end{aligned}
	\end{equation*}
	Therefore:
	\begin{equation}
		D^{j_{1} \otimes j_{2}}( J_{a})| j_{1} ,m_{1} ;j_{2} ,m_{2}\rangle =D^{j_{1}}( J_{a}) \otimes \mathds{1}^{j_{2}}| j_{1} ,m_{1} ;j_{2} ,m_{2}\rangle +\mathds{1}^{j_{1}} \otimes D^{j_{2}}( J_{a})| j_{1} ,m_{1} ;j_{2} ,m_{2}\rangle 
	\end{equation}
	In particular, if $a=3$, we have
	\begin{equation*}
		J_{3}| j_{1} ,m_{1} ;j_{2} ,m_{2}\rangle =D^{j_{1} \otimes j_{2}}( J_{3})| j_{1} ,m_{1} ;j_{2} ,m_{2}\rangle =( m_{1} +m_{2})| j_{1} ,m_{1} ;j_{2} ,m_{2}\rangle 
	\end{equation*}
	Now we act ladder operator on eq(1), we have:
	\begin{equation*}
		\begin{array}{ c }
			D^{J}( J_{\pm }) |J,M\rangle  ={\displaystyle \sum\limits _{m_{1}^{\prime } =-j_{1}}^{j_{1}}\sum\limits _{m_{2}^{\prime } =-j_{2}}^{j_{2}}} C_{j_{1} j_{2} m_{1}^{\prime } m_{2}^{\prime }}^{JM} D^{j_{1} \otimes j_{2}}( J_{\pm })| j_{1} ,m'_{1} ;j_{2} ,m'_{2}\rangle  \\
			\Longrightarrow \sqrt{(J\mp M)(J+1\pm M)} |J,M\pm 1\rangle  ={\displaystyle \sum\limits _{m_{1} ,m_{2}}} C_{j_{1} j_{2} m'_{1} m'_{2}}^{JM}\left[ D^{j_{1}}( J_{\pm })| m'_{1} ;m'_{2}\rangle  +D^{j_{2}}( J_{\pm })| m'_{1} ;m'_{2}\rangle  \right]
		\end{array}
	\end{equation*}
	which by multiplying with $\langle  m_{1} ;m_{2}| $ from the left on both sides, leads to
	\begin{equation*}
		\begin{aligned}
			\sqrt{(J\mp M)(J+1\pm M)} C_{j_{1} j_{2} m_{1} m_{2}}^{J(M\pm 1)} = & \sqrt{( j_{1} \mp m_{1})( j_{1} +1\pm m_{1})} C_{j_{1} j_{2}( m_{1} \mp 1) m_{2}}^{JM} +\sqrt{( j_{2} \mp m_{2})( j_{2} +1\pm m_{2})} C_{j_{1} j_{2} m_{1}( m_{2} \mp 1)}^{JM}
		\end{aligned}
	\end{equation*}
	This the \textbf{recursion relation} of C-G coefficients of $\mathfrak{so}( 3)$. If we set $M=J$ on both sides of the recursion relation and consider the upper sign, then the LHS of the resultant equation will be zero, and we obtain the initial recursion relation:
	\begin{equation*}
		0=\sqrt{( j_{1} -m_{1})( j_{1} +1+m_{1})} C_{j_{1} j_{2}( m_{1} -1) m_{2}}^{JJ} +\sqrt{( j_{2} -m_{2})( j_{2} +1+m_{2})} C_{j_{1} j_{2} m_{1}( m_{2} -1)}^{JJ}
	\end{equation*}
	By acting $J_{3}$ on both sides of Eq.(1), we can see
	\begin{equation*}
		M|J,M\rangle  =\sum _{m_{1} ,m_{2}} C_{j_{1} j_{2} m_{1} m_{2}}^{JM}( m_{1} +m_{2})| j_{1} ,m_{1} ;j_{2} ,m_{2}\rangle 
	\end{equation*}
	Since both sides are the same eigenstate of $J_{3}$, they must give rise to the same eigenvalue of $J_{3}$. Hence, the equation above can be written as a selection rule
	\begin{equation*}
		|J,M\rangle  =\sum _{m_{1} ,m_{2}} C_{j_{1} j_{2} m_{1} m_{2}}^{J,M} \delta _{m_{1} +m_{2} ,M}| j_{1} ,m_{1} ;j_{2} ,m_{2}\rangle 
	\end{equation*}
	For example, in $\frac{1}{2} \otimes \frac{1}{2} =1\otimes 0$ case, we have:
	\begin{equation*}
		\begin{aligned}
			\ket{1,1} & =\ket{\frac{1}{2} ;\frac{1}{2}}\\
			\Rightarrow J_{-}\ket{1,1} & =J_{-}\ket{\frac{1}{2} ;\frac{1}{2}}\\
			\Rightarrow \sqrt{2}\ket{1,0} & =\ket{-\frac{1}{2} ;\frac{1}{2}} +\ket{\frac{1}{2} ;-\frac{1}{2}}\\
			\Rightarrow \ket{1,0} & =\frac{1}{\sqrt{2}}\left(\ket{-\frac{1}{2} ;\frac{1}{2}} +\ket{\frac{1}{2} ;-\frac{1}{2}}\right)
		\end{aligned}
	\end{equation*}
	Using these equations, you can find out all C-G coefficients in $SO( 3)$. 
	\subsection{Wigner’s 3j-symbols}
	
	We first define Wigner-s $3j$-symbols by the C-G coefficients:
	\begin{equation*}
		\begin{pmatrix}
			j_{1} & j_{2} & J\\
			m_{1} & m_{2} & -M
		\end{pmatrix} \equiv \frac{( -1)^{j_{1} -j_{2} +M}}{\sqrt{2J+1}} C_{j_{1} j_{2} m_{1} m_{2}}^{JM}
	\end{equation*}
	Then(without proof), the closed-form formula of the Wigner's $3j$ -symbols reads
	\begin{equation*}
		\begin{aligned}
			& \left(\begin{matrix}
				j_{1} & j_{2} & j_{3}\\
				m_{1} & m_{2} & m_{3}
			\end{matrix}\right)\\
			= & (-1)^{j_{1} -j_{2} -m_{3}}\sqrt{\Updelta ( j_{1} ,j_{2} ,j_{3})( j_{1} +m_{1}) !( j_{1} -m_{1}) !( j_{2} +m_{2}) !( j_{2} -m_{2}) !( j_{3} +m_{3}) !( j_{3} -m_{3}) !}\\
			& \times \sum _{q}\frac{(-1)^{q}}{q!( j_{1} -q-m_{1}) !( j_{2} -q+m_{2}) !( j_{1} +j_{2} -j_{3} -q) !( j_{3} -j_{2} +q+m_{1}) !( j_{3} -j_{1} +q-m_{2}) !}
		\end{aligned}
	\end{equation*}
	where
	\begin{equation*}
		\Updelta ( j_{1} ,j_{2} ,j_{3}) \equiv \frac{( j_{1} +j_{2} -j_{3}) !( j_{2} +j_{3} -j_{1}) !( j_{3} +j_{1} -j_{2}) !}{( j_{1} +j_{2} +j_{3} +1) !}
	\end{equation*}
	The Wigner's $3j$ -symbols are in fact the coefficients in the expansion of the singlet state in terms of the basis states $| j_{1} ,m_{1} ;j_{2} ,m_{2} ;J,-M\rangle  $ of the tensor product space $j_{1} \otimes j_{2} \otimes J$. These coefficients are related to the C-G coefficients by scaling because $J$ appears in the tensor product decomposition of $j_{1} \otimes j_{2}$ if and only if $j_{1} \otimes j_{2} \otimes J$ contains a singlet, i.e. $j=0$. Since $J$ is multiplicity-free in $j_{1} \otimes j_{2}$, $j=0$ is also multiplicity-free in $j_{1} \otimes j_{2} \otimes J$. The usual symmetries possessed by the $3j$ -symbols are
	\begin{itemize}
		\item Cyclic symmetry:
	\end{itemize}
	\begin{equation*}
		\left(\begin{matrix}
			j_{1} & j_{2} & j_{3}\\
			m_{1} & m_{2} & m_{3}
		\end{matrix}\right) =\left(\begin{matrix}
			j_{2} & j_{3} & j_{1}\\
			m_{2} & m_{3} & m_{1}
		\end{matrix}\right) =\left(\begin{matrix}
			j_{3} & j_{1} & j_{2}\\
			m_{3} & m_{1} & m_{2}
		\end{matrix}\right)
	\end{equation*}
	\begin{itemize}
		\item Exchange symmetry:
	\end{itemize}
	\begin{equation*}
		\left(\begin{matrix}
			j_{1} & j_{2} & j_{3}\\
			m_{1} & m_{2} & m_{3}
		\end{matrix}\right) =s\left(\begin{matrix}
			j_{2} & j_{1} & j_{3}\\
			m_{2} & m_{1} & m_{3}
		\end{matrix}\right) =s\left(\begin{matrix}
			j_{1} & j_{3} & j_{2}\\
			m_{1} & m_{3} & m_{2}
		\end{matrix}\right) =s\left(\begin{matrix}
			j_{3} & j_{2} & j_{1}\\
			m_{3} & m_{2} & m_{1}
		\end{matrix}\right) ,
	\end{equation*}
	where $s=(-1)^{j_{1} +j_{2} +j_{3}}$ is a sign factor.
	\begin{itemize}
		\item Sign (time reversal) symmetry:
	\end{itemize}
	\begin{equation*}
		\left(\begin{matrix}
			j_{1} & j_{2} & j_{3}\\
			m_{1} & m_{2} & m_{3}
		\end{matrix}\right) =s\left(\begin{matrix}
			j_{1} & j_{2} & j_{3}\\
			-m_{1} & -m_{2} & -m_{3}
		\end{matrix}\right) .
	\end{equation*}
	The symmetries of the $3j$ -symbols induce the symmetries of the C-G coefficients. As an example, we have
	\begin{equation*}
		C_{j_{1} j_{2} m_{1} m_{2}}^{JM} =( -1)^{j_{1} +j_{2} +J-2M} C_{j_{1} j_{2}( -m_{1})( -m_{2})}^{J(-M)}
	\end{equation*}
	
	\section{Wigner-Eckart theorm}
	
	We first do the relabeling:
	\begin{equation*}
		O_{0}^{1} \equiv J_{3} ,O_{1}^{1} \equiv J_{+} ,O_{-1}^{1} \equiv J_{-}
	\end{equation*}
	Then we can verify:
	\begin{equation*}
		\left[ J_{a} ,O_{m}^{1}\right] =O_{m'}^{1} D^{1}( J_{a})_{m'm}
	\end{equation*}
	Where $m'$ is summed. In particular, 
	\begin{equation*}
		\left[ J_{3} ,O_{m}^{1}\right] =mO_{m}^{1}
	\end{equation*}
	Any set of three operators that transform under $\mathfrak{so}( 3)$ in the same way as $\{J_{-} ,J_{3} ,J_{+}\}$ is a set of \textbf{vector}
	
	\textbf{operators}. For example, the set of position operators $\{X_{-} ,X_{3} ,X_{+}\}$ are vector operators, where
	\begin{equation*}
		X_{\pm } =\frac{1}{\sqrt{2}}( X_{1} \pm X_{2})
	\end{equation*}
	More generally, a physical system may possess certain set of $2j+1$ operators $\{O_{m}^{j} | m=-j,-j+1,\dotsc ,j\}$ that transform under $\mathfrak{so} (3)$ as
	\begin{equation}
		\left[ J_{a} ,O_{m}^{j}\right] =O_{m'}^{j} D^{j}( J_{a})_{m' m}
	\end{equation}
	Such operators are called \textbf{tensor operators} because they form a \textbf{basis of the representation space of the tensor representation }$j$. A nontrivial example is the quadrupole moment $Q$ of a charge distribution $\rho (\vec{x} )$ in $\mathbb{R}^{3}$. Since $Q$ as a $3\times 3$ matrix is symmetric traceless, the matrix elements $Q_{kl}$ are tensor operators with $j=2$. Where
	\begin{equation*}
		Q_{kl} =\int _{\mathbb{R}^{3}} \rho (\vec{x})\left( 3x_{k} x_{l} -| \vec{x}| ^{2} \delta _{kl}\right)\mathrm{d}^{3}\vec{x}
	\end{equation*}
	In fact, the transformation rule (3) is a result of the $SO( 3)$ transformation rule of tensor operators:
	\begin{equation*}
		\hat{R} O_{m}^{j}\hat{R}^{-1} =O_{m'}^{j} D^{j} (R)_{m'm}
	\end{equation*}
	where $R$ is an abstract rotation, and the LHS is the definition of the action of $R$ on an operator. Moreover, we can find:
	\begin{equation}
		\begin{array}{ l }
			\left[ J^{2} ,O_{m}^{j}\right] =j(j+1)O_{m}^{j}\\
			\left[ J_{3} ,O_{m}^{j}\right] =mO_{m}^{j}\\
			\left[ J_{\pm } ,O_{m}^{j}\right] =\sqrt{(j\mp m)(j+1\pm m)} O_{m\pm 1}^{j} .
		\end{array}
	\end{equation}
	Now that the set of $2j_{1} +1$ operators $O_{m_{1}}^{j_{1}}$ is regarded a basis of the $j_{1}$-representation, their action on the basis states $| j_{2} ,m_{2}\rangle  $ of the $j_{2}$-representation can be taken as the basis of the tensor product representation $j_{1} \otimes j_{2}$, because we have:
	\begin{equation*}
		\begin{aligned}
			J_{a} O_{m_{1}}^{j_{1}}| j_{2} ,m_{2}\rangle   & =\left[ J_{a} ,O_{m_{1}}^{j_{1}}\right]| j_{2} ,m_{2}\rangle  +O_{m_{1}}^{j_{1}} J_{a}| j_{2} ,m_{2}\rangle  \\
			& =D^{j_{1}}( J_{a})_{m'_{1} m_{1}} O_{m_{1}}^{j_{1}}| j_{2} ,m_{2}\rangle  +D^{j_{2}}( J_{a})_{m'_{2} m_{2}} O_{m_{1}}^{j_{1}}| j_{2} ,m'_{2}\rangle  
		\end{aligned}
	\end{equation*}
	which is precisely the transformation rule (2) of tensor-product states. If we set $a=3$, we have
	\begin{equation*}
		J_{3} O_{m_{1}}^{j_{1}}| j_{2} ,m_{2}\rangle  =( m_{1} +m_{2}) O_{m_{1}}^{j_{1}}| j_{2} ,m_{2}\rangle  
	\end{equation*}
	showing that the states $O_{m_{1}}^{j_{1}}| j_{2} ,m_{2}\rangle  $ are indeed the eigenstates of $J_{3}$ with eigenvalues $m_{1} +m_{2}$ and thus form a basis of the representation $j_{1} \otimes j_{2}$.
	
	Then we can consider the inner products $\mel**{J,M}{O_{m_{1}}^{j_{1}}}{j_{2} ,m_{2}}$, where $J$ runs from $| j_{1} -j_{2}| $ to $j_{1} +j_{2}$. These inner products resemble the C-G coefficients $\langle  j_{1} ,m_{1} ;j_{2} ,m_{2} \mid J,M\rangle  $ but can also be interpreted as the matrix entries of $O_{m_{1}}^{j_{1}}$ in the mixed bases $|J,M\rangle  $ and $| j_{2} ,m_{2}\rangle  $. Physically, the Hilbert space on which the operators $O_{m_{1}}^{j_{1}}$ depends on other physical parameters too, besides the angular momentum (or spin). Let us denote collectively the other physical parameters by $\alpha $. The physical Hilbert space will be spanned by the basis vectors $|j,m;\alpha \rangle  :=\ket{j,m} \otimes \ket{j,\alpha }$ where the $j$ in $|j,\alpha \rangle  $ simply bookmarks the physical parameters $\alpha $ relevant to the $j$ -representation. The actual inner products we are seeking for is
	\begin{equation*}
		\mel**{J,M;\beta }{O_{m_{1}}^{j_{1}}}{j_{2} ,m_{2} ;\alpha }
	\end{equation*}
	By eq(4), we can find:
	\begin{equation*}
		\begin{aligned}
			m_{1}\mel**{J,M;\beta }{O_{m_{1}}^{j_{1}}}{j_{2} ,m_{2} ;\alpha } & =\mel**{J,M;\beta }{\left[ J_{3} ,O_{m_{1}}^{j_{1}}\right]}{j_{2} ,m_{2} ;\alpha }\\
			& =( M-m_{2})\mel**{J,M;\beta }{O_{m_{1}}^{j_{1}}}{j_{2} ,m_{2} ;\alpha }
		\end{aligned}
	\end{equation*}
	hence, the matrix entry is nonzero only if the selection rule
	\begin{equation*}
		m_{1} =M-m_{2}
	\end{equation*}
	holds. 
	
	Because we regard If we regard $O_{m_{1}}^{j_{1}}$ as a basis of the $j_{1}$-representation, we natually have
	\begin{equation*}
		\left\langle  J,M;\beta \left| O_{m_{1}}^{j_{1}}\right| j_{2} ,m_{2} ;\alpha \right\rangle  \varpropto C_{j_{1} j_{2} m_{1} m_{2}}^{JM}
	\end{equation*}
	Since $C_{j_{1} j_{2} m_{1} m_{2}}^{JM}$ is all that is determined by the group and representation theory, what remains in the matrix element $\left\langle  J,M;\beta \left| O_{m_{1}}^{j_{1}}\right| j_{2} ,m_{2} ;\alpha \right\rangle  $ that cannot be determined by the group and representation theory should depend on the physical parameters $\alpha $ only. Thus, we conclude that
	\begin{equation*}
		\boxed{\left\langle  J,M;\beta \left| O_{m_{1}}^{j_{1}}\right| j_{2} ,m_{2} ;\alpha \right\rangle  =C_{j_{1} j_{2} m_{1} m_{2}}^{JM}\left\langle  J,\beta \left| O^{j_{1}}\right| j_{2} ,\alpha \right\rangle  }
	\end{equation*}
	where $\left\langle  J,\beta \left| O^{j_{1}}\right| j_{2} ,\alpha \right\rangle  $ is the factor that is determined by the physics. This result is the\textbf{ Wigner-Eckart} theorem of $\mathfrak{so}( 3)$ tensor operators.
	\subsection{Examples}
	
	For a scalar operator $S$, i.e. a rank-$0$ tensor: Matrix elements of a scalar operator $S$ are zero unless $m=m'$ and $j=j'$
	\begin{equation*}
		\mel**{j,m;\beta }{S}{j',m'\alpha } =\delta _{jj'} \delta _{mm'}\mel**{j,\beta }{S}{j',\alpha }
	\end{equation*}
	
	
	Consider the position expectation value $\langle njm|x|njm\rangle  $. Since vectors are rank-1 spherical tensor operators, it follows that $x$ must be some linear combination of a rank-$1$ spherical tensor $T_{q}^{(1)}$ with $q\in \{-1,0,1\}$. In fact, it can be shown that
	\begin{equation*}
		x=\frac{T_{-1}^{(1)} -T_{1}^{(1)}}{\sqrt{2}}
	\end{equation*}
	where we define the spherical tensors as
	\begin{equation*}
		T_{q}^{(1)} =\sqrt{\frac{4\pi }{3}} rY_{1}^{q}
	\end{equation*}
	and $Y_{l}^{m}$ are spherical harmonics, which themselves are also spherical tensors of rank $l$. Additionally, $T_{0}^{(1)} =z$, and
	\begin{equation*}
		T_{\pm 1}^{(1)} =\mp \frac{x\pm \mathrm{i} y}{\sqrt{2}}
	\end{equation*}
	Therefore,
	\begin{equation*}
		\begin{aligned}
			\langle  njm|x|n'j'm'\rangle   & =\mel**{njm}{\frac{T_{-1}^{(1)} -T_{1}^{(1)}}{\sqrt{2}}}{n'j'm'}\\
			& =\frac{1}{\sqrt{2}}\mel**{nj}{T^{( 1)}}{n'j'}\left(\bra{j'm'1( -1)}\ket{jm} -\bra{j'm'11}\ket{jm}\right)
		\end{aligned}
	\end{equation*}
	The above expression gives us the matrix element for $x$ in the $|njm\rangle  $ basis. To find the expectation value, we set $n'=n,j'=j,$ and $m'=m.$ The selection rule for $m'$ and $m$ is $m\pm 1=m'$ for the $T^{(1)}{}_{\pm 1}$ spherical tensors. As we have $m'=m,$ this makes the Clebsch-Gordan Coefficients zero, leading to the expectation value to be equal to zero.
	
	
	
	I bet you still don't know how to use the W-E theoerm. I will give you a more powerful example. 
	
	Suppose that we need to calculate this integral:
	\begin{equation*}
		\int \mathrm{d} \si{\ohm}\left( Y_{3}^{m_{1}}\right)^{*} Y_{2}^{m_{2}} Y_{1}^{m_{3}}
	\end{equation*}
	where $\mathrm{d} \si{\ohm}=\sin \theta \mathrm{d} \theta \mathrm{d} \phi $. This kind of integrals appear over and over in \textbf{spectroscopy} problems. Let us calculate it for $m_{1} =m_{2} =m_{3} =0$:
	\begin{equation*}
		\int \mathrm{d} \si{\ohm}\left( Y_{3}^{0}\right)^{*} Y_{2}^{0} Y_{1}^{0} =\frac{\sqrt{105}}{32\sqrt{\pi ^{3}}} \cdot 2\pi \int \mathrm{d} \theta \left( 3\cos^{2} \theta \sin \theta -14\cos^{4} \theta \sin \theta +15\cos^{6} \theta \sin \theta \right) =\frac{3}{2}\sqrt{\frac{3}{35\pi }}
	\end{equation*}
	This is really disgusting for non-mma people. The problem is that we usually need to evaluate this for all values of $m_{i} .$ That is $7\times 5\times 3=105$ integrals. But if we use WE theorem:
	\begin{equation*}
		\int \mathrm{d} \si{\ohm}\left( Y_{3}^{m_{1}}\right)^{*} Y_{2}^{m_{2}} Y_{1}^{m_{3}} =\mel**{j=3,m_{1}}{Y_{2}^{m_{2}}}{j=1,m_{3}} =C_{2m_{2} 1m_{3}}^{3m_{1}}\mel**{3}{Y_{2}}{1}
	\end{equation*}
	Where $\mel**{3}{Y_{2}}{1}$ is the reduced matrix element which we can derive from our expression for $m_{1} =m_{2} =m_{3} =0:$
	\begin{equation*}
		\frac{3}{2}\sqrt{\frac{3}{35\pi }} =C_{2010}^{30}\mel**{3}{Y_{2}}{1} \Rightarrow \mel**{3}{Y_{2}}{1} =\frac{1}{2}\sqrt{\frac{3}{\pi }}
	\end{equation*}
	So we have:
	\begin{equation*}
		\int \mathrm{d} \si{\ohm}\left( Y_{3}^{m_{1}}\right)^{*} Y_{2}^{m_{2}} Y_{1}^{m_{3}} =\sqrt{\frac{3}{4\pi }} C_{2m_{2} 1m_{3}}^{3m_{1}}
	\end{equation*}
	This is freaking useful for those who study spectroscopy and computational QM. You may wonder what if it doesn't have to be $3=1+2$? For example, if we want to evaluate $\int \mathrm{d} \si{\ohm}\left( Y_{3}^{m_{1}}\right)^{*} Y_{1}^{m_{2}} Y_{1}^{m_{3}}$, what happens? Of course:
	\begin{equation*}
		\int \mathrm{d} \si{\ohm}\left( Y_{3}^{m_{1}}\right)^{*} Y_{1}^{m_{2}} Y_{1}^{m_{3}} =C_{1m_{2} 1m_{3}}^{3m_{1}}\mel**{3}{Y_{1}}{1} =\underbrace{\bra{3,m_{1}}\ket{1,m_{2} ;1,m_{3}}}_{0}\mel**{3}{Y_{1}}{1} =0
	\end{equation*}
	We can see if $\int \mathrm{d} \si{\ohm}\left( Y_{J}^{m_{1}}\right)^{*} Y_{j_{1}}^{m_{2}} Y_{j_{2}}^{m_{3}} \neq 0$, suppose $J \rangle j_{1} ,j_{2}$ then $J=j_{1} +j_{2}$ immediately from the Wigner-Eckart theorem! Isn't it shocking? 
	
	
	
	
	
	
	
	
	
	This note isn't over, I will write about the details about the representation of $SO( 3)$ and $SU( 2)$. 
	
\end{document}
