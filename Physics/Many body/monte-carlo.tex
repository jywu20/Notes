\documentclass[hyperref, UTF8, a4paper]{ctexart}

\usepackage{geometry}
\usepackage{titling}
\usepackage{titlesec}
\usepackage{paralist}
\usepackage{footnote}
\usepackage{enumerate}
\usepackage{amsmath, amssymb, amsthm}
\usepackage{bbm}
\usepackage{cite}
\usepackage{graphicx}
\usepackage{subfigure}
\usepackage{physics}
\usepackage{tikz}
\usepackage[ruled, vlined, linesnumbered, noend]{algorithm2e}
\usepackage[colorlinks, linkcolor=black, anchorcolor=black, citecolor=black]{hyperref}
\usepackage{prettyref}

% Page style
\geometry{left=3.18cm,right=3.18cm,top=2.54cm,bottom=2.54cm}
\titlespacing{\paragraph}{0pt}{1pt}{10pt}[20pt]
\setlength{\droptitle}{-5em}
\preauthor{\vspace{-10pt}\begin{center}}
\postauthor{\par\end{center}}

% Math operators
\DeclareMathOperator{\timeorder}{T}
\DeclareMathOperator{\diag}{diag}
\DeclareMathOperator{\legpoly}{P}
\DeclareMathOperator{\primevalue}{P}
\DeclareMathOperator{\sgn}{sgn}
\newcommand*{\ii}{\mathrm{i}}
\newcommand*{\ee}{\mathrm{e}}
\newcommand*{\const}{\mathrm{const}}
\newcommand*{\comment}{\paragraph{注记}}
\newcommand*{\suchthat}{\quad \text{s.t.} \quad}
\newcommand*{\argmin}{\arg\min}
\newcommand*{\argmax}{\arg\max}
\newcommand*{\normalorder}[1]{: #1 :}
\newcommand*{\pair}[1]{\langle #1 \rangle}
\newcommand*{\fd}[1]{\mathcal{D} #1}

% prettyref setting
\newrefformat{sec}{第\ref{#1}节}
\newrefformat{note}{注\ref{#1}}
\newrefformat{fig}{图\ref{#1}}
\newrefformat{alg}{算法\ref{#1}}
\renewcommand{\autoref}{\prettyref}

% TikZ setting
\usetikzlibrary{arrows,shapes,positioning}
\usetikzlibrary{arrows.meta}
\usetikzlibrary{decorations.markings}
\tikzstyle arrowstyle=[scale=1]
\tikzstyle directed=[postaction={decorate,decoration={markings,
    mark=at position .5 with {\arrow[arrowstyle]{stealth}}}}]
\tikzstyle ray=[directed, thick]
\tikzstyle dot=[anchor=base,fill,circle,inner sep=1pt]

% Algorithm setting
\renewcommand{\algorithmcfname}{算法}
% Python-style code
\SetKwIF{If}{ElseIf}{Else}{if}{:}{elif:}{else:}{}
\SetKwFor{For}{for}{:}{}
\SetKwFor{While}{while}{:}{}
\SetKwInput{KwData}{输入}
\SetKwInput{KwResult}{输出}

\title{蒙特卡罗方法}
\author{吴何友}

\begin{document}

\maketitle

\section{经典蒙特卡洛方法}

\subsection{马尔科夫链蒙特卡洛方法}\label{sec:mcmc-method}

平衡态统计物理的核心问题就是计算配分函数
\begin{equation}
    Z = \sum_\mathcal{C} \ee^{-\beta H[\mathcal{C}]} = \sum_\mathcal{C} W(\mathcal{C}).
    \label{eq:partition-function}
\end{equation}
这里我们用$\mathcal{C}$表示一个任意的系统能量本征态,经典情况下这就是一个系统构型,量子情况下还需要对哈密顿量做一个对角化。
本节将主要讨论经典系统,因为它不涉及通常难以计算的算符对角化,并且实际上很多时候量子系统可以化归为经典系统。
如果能够将每个$\mathcal{C}$对应的$W[\mathcal{C}]$算出来那还可以大大简化期望值的计算。

显然,涉及系统构型的路径积分的\eqref{eq:partition-function}是非常难以计算的。
一种近似方法是\textbf{马尔科夫链蒙特卡洛法},即构造一个各态遍历、不可约的马尔可夫链
\[
    \cdots \longrightarrow \mathcal{C}_{i-1} \longrightarrow \mathcal{C}_i \longrightarrow \mathcal{C}_{i+1} \longrightarrow \cdots,
\]
使得
\begin{equation}
    \frac{p(\mathcal{C} \rightarrow \mathcal{D})}{p(\mathcal{D} \rightarrow \mathcal{C})} = \frac{W(\mathcal{D})}{W(\mathcal{C})} = \ee^{-\beta(H[\mathcal{D}]-H[\mathcal{C}])},
    \label{eq:markov-mcmc}
\end{equation}
则达到平衡时必定有细致平衡条件
\[
    p(\mathcal{C}) p(\mathcal{C} \rightarrow \mathcal{D}) = p(\mathcal{D}) p(\mathcal{D} \rightarrow \mathcal{C}),
\]
就有
\[
    \frac{W(\mathcal{D})}{W(\mathcal{C})} = \frac{p(\mathcal{D})}{p(\mathcal{C})}.
\]
专门使用$\pi(\mathcal{C})$来表示平衡态概率,即有
\begin{equation}
    \frac{W(\mathcal{D})}{W(\mathcal{C})} = \frac{\pi(\mathcal{D})}{\pi(\mathcal{C})}.
\end{equation}
这样我们只需要让这个马尔可夫链计算到收敛(有限、时不变、不可约、非循环的马尔科夫链肯定可以收敛,而由于可以找到一个$p(\mathcal{C})$的安排让细致平衡条件成立,总是可以收敛到$W(\mathcal{C})/Z$),
此时按照系综平均等于时间平均的原理,任何一个物理量的期望值就是
\begin{equation}
    \expval{O} = \sum_{\mathcal{C}} \frac{W(\mathcal{C})}{Z} O[\mathcal{C}] = \frac{1}{N} \sum_i O[\mathcal{C}_i].
\end{equation}
也即过程收敛之后,只需要在时间序列$\{\mathcal{C}_i\}$上分别计算$O$的值,做时间平均,就得到了$O$的期望值。
需要注意的是实际上$N$不能取无穷大,因此我们希望平衡之后$\{\mathcal{C}_i\}$尽可能随机,即自相关要足够小。

因此问题的核心就是如何设计一个满足\eqref{eq:markov-mcmc}的不可约各态历经马尔可夫链。
这个过程未必要和实际的动力学过程完全一样,只要满足\eqref{eq:markov-mcmc}当然都可以。
不可约性相对来说是容易做到的,因此需要巧妙地设计$p(\mathcal{C} \rightarrow \mathcal{D})$,并且确认平衡后的$\{\mathcal{C}_i\}$在长时间上没有自相关。

\subsection{Metropolis-Hastings算法}

本节讨论一个能够达到\autoref{sec:mcmc-method}中要求的算法:Metropolis-Hastings算法,即\autoref{alg:metro-hast}。

\begin{algorithm}[H]
    \SetAlgoLined
    \KwData{this text}
    \KwResult{how to write algorithm with \LaTeX2e }
    initialization\;
    \While{not at end of this document}{
        read current\;
        \eIf{understand}{
            go to next section\;
            current section becomes this one\;
        }{
            go back to the beginning of current section\;
        }
    }
    \caption{Metropolis-Hastings算法}
    \label{alg:metro-hast}
\end{algorithm}
    

\end{document}