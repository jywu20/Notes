\documentclass[hyperref, UTF8, a4paper]{ctexart}

\usepackage{geometry}
\usepackage{titling}
\usepackage{titlesec}
\usepackage{paralist}
\usepackage{footnote}
\usepackage{enumerate}
\usepackage{amsmath, amssymb, amsthm}
\usepackage{bbm}
\usepackage{cite}
\usepackage{graphicx}
\usepackage{subfigure}
\usepackage{physics}
\usepackage{tikz}
\usepackage{autobreak}
\usepackage[ruled, vlined, linesnumbered, noend]{algorithm2e}
\usepackage[colorlinks, linkcolor=black, anchorcolor=black, citecolor=black]{hyperref}
\usepackage{prettyref}

% Page style
\geometry{left=3.18cm,right=3.18cm,top=2.54cm,bottom=2.54cm}
\titlespacing{\paragraph}{0pt}{1pt}{10pt}[20pt]
\setlength{\droptitle}{-5em}
\preauthor{\vspace{-10pt}\begin{center}}
\postauthor{\par\end{center}}

% Math operators
\DeclareMathOperator{\timeorder}{T}
\DeclareMathOperator{\diag}{diag}
\DeclareMathOperator{\legpoly}{P}
\DeclareMathOperator{\primevalue}{P}
\DeclareMathOperator{\sgn}{sgn}
\newcommand*{\ii}{\mathrm{i}}
\newcommand*{\ee}{\mathrm{e}}
\newcommand*{\const}{\mathrm{const}}
\newcommand*{\comment}{\paragraph{注记}}
\newcommand*{\suchthat}{\quad \text{s.t.} \quad}
\newcommand*{\argmin}{\arg\min}
\newcommand*{\argmax}{\arg\max}
\newcommand*{\normalorder}[1]{: #1 :}
\newcommand*{\pair}[1]{\langle #1 \rangle}
\newcommand*{\fd}[1]{\mathcal{D} #1}
\DeclareMathOperator{\bigO}{\mathcal{O}}

% prettyref setting
\newrefformat{sec}{第\ref{#1}节}
\newrefformat{note}{注\ref{#1}}
\newrefformat{fig}{图\ref{#1}}
\newrefformat{alg}{算法\ref{#1}}
\renewcommand{\autoref}{\prettyref}

% TikZ setting
\usetikzlibrary{arrows,shapes,positioning}
\usetikzlibrary{arrows.meta}
\usetikzlibrary{decorations.markings}
\tikzstyle arrowstyle=[scale=1]
\tikzstyle directed=[postaction={decorate,decoration={markings,
    mark=at position .5 with {\arrow[arrowstyle]{stealth}}}}]
\tikzstyle ray=[directed, thick]
\tikzstyle dot=[anchor=base,fill,circle,inner sep=1pt]

% Algorithm setting
\renewcommand{\algorithmcfname}{算法}
% Python-style code
\SetKwIF{If}{ElseIf}{Else}{if}{:}{elif:}{else:}{}
\SetKwFor{For}{for}{:}{}
\SetKwFor{While}{while}{:}{}
\SetKwInput{KwData}{输入}
\SetKwInput{KwResult}{输出}
\SetArgSty{textnormal}

\renewcommand{\emph}[1]{\textbf{#1}}
\newcommand*{\concept}[1]{\underline{\textbf{#1}}}
\newcommand*{\Ztwo}{$\mathbb{Z}_2$}

\title{\Ztwo规范场论}
\author{吴何友}

\begin{document}

\maketitle

\section{格点上的\Ztwo规范理论}\label{sec:z2-on-lattice}

最广为人知的规范场论可能是电动力学,这是一个$U(1)$规范理论,其中电子场可以发生任意的局域相位转动,而与之配套的规范场——电磁场矢势——发生一个局域平移。
本文中我们不要$U(1)$这么大的对称性,而是只希望电子场或者不发生相位转动,或者相位就转动$\pi$,在这样的规范对称性——也就是\concept{\Ztwo规范对称性}下系统的动力学保持不变。
如果我们还是在通常的四维时空中工作那么局域\Ztwo变换就是不连续的:因为$0$和$\pi$不能连续过渡。
因此我们将在格点上工作,即研究格点规范场论。

格点上的电子的动能项无非是从一个点跃迁到另外一个点,即
\begin{equation}
    \hat{H}_0 = - \sum_{i, j, \alpha} t_{ij} \hat{c}_{i \alpha}^\dagger \hat{c}_{j \alpha}.
    \label{eq:hopping-hamiltonian}
\end{equation}
这个哈密顿量在局域\Ztwo变换下不是不变的。
现在假定出于某些原因,系统具有了\Ztwo规范对称性,那么为了加入局域\Ztwo对称性我们只能修改$t_{ij}$系数,使得它在\Ztwo变换下能够吸收掉电子场带来的变化。
容易看到,只需要指定
\[
    \hat{c}_{i \alpha} \longrightarrow \eta_{i} \hat{c}_{i \alpha}, \quad t_{ij} \longrightarrow \eta_i \eta_j t_{ij},
\]
就能够让哈密顿量具有局域\Ztwo对称性。由于$t_{ij}$只是在正负两种状态之间切换,可以引入一个规范联络$\sigma_{ij} = \pm 1$,于是用哈密顿量
\[
    \hat{H} = - \sum_{i, j, \alpha} t_{ij} \sigma_{ij} \hat{c}_{i \alpha}^\dagger \hat{c}_{j \alpha}
\]
做路径积分,分别以$\hat{c}, \hat{c}^\dagger$和$\sigma_{ij}$为积分变量即可得到一个\Ztwo规范理论。

现在我们回到正则量子化框架中,$\sigma_{ij}$在每一个格点引入了$\pm 1$两个状态,从而我们可以把它当成一个自旋$1/2$的自旋算符%
\footnote{实际上,这个“自旋算符”未必来自某个体系的内禀旋转不变性。
更加数学的说法是,由于每个格点都有两个状态,我们可以在每个格点引入一个$2\times 2$的厄米矩阵
\[
    \hat{\sigma} = \pmqty{1 & 0 \\ 0 & -1}
\]
作为规范场对应的算符,而这正是泡利矩阵中的$\hat{\sigma}^z$。后面引入$\hat{\sigma}^x$等算符的目的也只是用于翻转规范场的状态。
}%
,从而哈密顿量为
\begin{equation}
    \hat{H} = - \sum_{i, j, \alpha} t_{ij} \hat{\sigma}^z_{ij} \hat{c}_{i \alpha}^\dagger \hat{c}_{j \alpha}
    \label{eq:minimal-z2-couple}
\end{equation}
希尔伯特空间为电子的态空间直积上每一点的自旋$1/2$空间。\Ztwo规范变换为
\begin{equation}
    \hat{c}_{i \alpha} \longrightarrow \eta_{i} \hat{c}_{i \alpha}, \quad \hat{\sigma}_{ij}^z \longrightarrow \eta_i \eta_j \hat{\sigma}_{ij}^z.
\end{equation}

特别的,如果\eqref{eq:hopping-hamiltonian}实际上是一个紧束缚模型,\eqref{eq:minimal-z2-couple}就成为
\begin{equation}
    \hat{H} = - t \sum_{\pair{i, j}} \hat{\sigma}^z_{ij} \hat{c}_{i \alpha}^\dagger \hat{c}_{j \alpha} + \text{h.c.}.
    \label{eq:tight-binding-z2}
\end{equation}
此时$\sigma_{ij}^z$实际上仅仅定义在格子的边上,即不需要对不相邻的点对$\pair{i, j}$也对应对应的$\sigma_{ij}^z$。

\section{二维格子上的无物质场\Ztwo规范场理论}

\subsection{积掉电子}

现在我们讨论二维格子的情况。\eqref{eq:tight-binding-z2}中的电子由于相互作用是有能隙的,于是将\eqref{eq:tight-binding-z2}中的电子自由度积掉%
\footnote{我们在正则量子化框架中工作,因此积掉电子自由度实际上意味着原本是纯态的系统将成为混合态。
但是实际上这无关紧要,因为凝聚态理论向来分析有限温度情况,一开始系统就是处于混合态的。
我们只需要假装不知道电子存在,分析\Ztwo规范场的态空间,最后计算配分函数即可,并不需要真的处理混合态。}%
,得到仅仅关于\Ztwo规范场(而没有任何物质场)的一个低能有效理论。
严格做有关的计算是非常不现实的,但是无论如何,积掉电子自由度之后的哈密顿量本身肯定是\Ztwo规范不变的。我们首先先分析\Ztwo规范变换如何写成算符形式,然后分析积掉电子自由度之后的哈密顿量会是什么形式的。

电子自由度积掉之后规范变换就变成了
\begin{equation}
    \hat{\sigma}_{ij}^z \longrightarrow \eta_i \eta_j \hat{\sigma}_{ij}^z,
    \label{eq:pure-sigma-ztwo}
\end{equation}
也就是说对每一条边上的$\hat{\sigma}^z$本征态,规范变换或是不改变它,或是加一个负号。我们希望将\Ztwo规范变换写成算符的形式,为此注意到在自旋$1/2$中,算符$\hat{\sigma}^x$可以翻转$\hat{\sigma}^z$的本征态,且$\hat{\sigma}^x$是厄米算符,于是一条边上的规范场翻转就是
\[
    \hat{\sigma}_{ij}^z \longrightarrow \hat{\sigma}^x_{ij} \hat{\sigma}_{ij}^z \hat{\sigma}^x_{ij}.
\]
任何一个\Ztwo规范变换都可以拆解成一系列作用在格点上的规范变换相乘,而作用在格点$i$上的规范变换翻转和这个格点连接的四条边上的规范场,于是作用在格点$i$上的规范变换为
\begin{equation}
    \hat{Q}_i = \prod_{\pair{i, j}} \hat{\sigma}^x_{ij} = \prod_{j \in +_i} \hat{\sigma}^x_{ij},
    \label{eq:z2-charge}
\end{equation}
于是规范不变量就是和所有$\hat{Q}_i$对易的算符。由于是低能有效理论,我们考虑最低阶的两个\Ztwo规范不变量,得到
\begin{equation}
    \hat{H} = - K \sum_{\pair{i, j}} \hat{\sigma}^x_{ij} - J \sum_{\Box} \prod_{l \in \Box} \hat{\sigma}^z_{l}.
    \label{eq:z2-2d-hamiltonian}
\end{equation}
这就是只含有\Ztwo规范场的\emph{一个}有效理论(当然,实际上还有很多其它的\Ztwo规范理论,是取其它\Ztwo规范不变量得到的)。
关于为什么我们考虑了最低阶的两个\Ztwo规范不变量而不是别的(特别是,$\hat{\sigma}^x$是怎么被牵扯进来的),可以从以下角度考虑:
在零温情况下我们此处给出的\Ztwo理论对应一个三维经典统计理论,这个三维经典统计理论当然也应该具有\Ztwo规范不变性。
我们知道这个三维经典统计理论的自由度实际上就是将\eqref{eq:z2-2d-hamiltonian}的自由度加上一个虚时间指标之后得到的结果,即$\{\sigma^z_{ij}(\tau)\}$。
由于没有$z$方向上的边上定义了$\sigma^z$自由度,我们可以直接沿用\eqref{eq:pure-sigma-ztwo}作为三维经典统计理论的\Ztwo规范变换。
三维经典统计理论的形式应该是
\[
    Z = \sum_{\sigma^z} \exp(\sum_{\tau} (J_{xy} \sum_{\Box} \prod_{l \in \Box} \sigma_l^z(\tau)  ) )
\]
% TODO

\subsection{去除规范自由度}

无论是\eqref{eq:tight-binding-z2}还是\eqref{eq:z2-2d-hamiltonian}都具有\Ztwo规范不变性,如果我们认为规范自由度不具有物理含义(它实际上有没有物理含义取决于我们关心的物理量是不是只涉及规范不变量),那么这两个哈密顿量就含有额外的自由度。
我们要设法把规范等价的构型全部映射到同一个构型上,而把规范不等价的构型映射到不同的构型上。
为此,我们将每个格子赋予一个格点坐标$I$,从而诸$\{i\}$和诸$\{I\}$形成对偶格点坐标。
设$\Box_I$为$I$号格子($I$标记了以所有的格子的中心为格点形成的新格子的格点坐标,称为\concept{对偶格子}),我们定义
\begin{equation}
    \hat{\tau}^x_I = \prod_{l \in \Box_I} \hat{\sigma}^z_l,
    \label{eq:def-tau}
\end{equation}
上标$x$看起来很奇怪,不过我们很快会发现其作用。这样\eqref{eq:z2-2d-hamiltonian}中的第二项就可以很容易地写出了。
至于第一项,如果将$\hat{\sigma}_{ij}^x$作用在某个$\hat{\sigma}^z$表象下的态上面,那么边$ij$上的$\sigma^z$反号,其余什么都不变,这就是说,设边$ij$由方格$I$和$J$共享,则由定义\eqref{eq:def-tau},$I$和$J$对应的$\tau^x$也反号,其余不变;
另一方面,将$\hat{\tau}^x$看成某个表象下的$x$方向泡利矩阵,并将$\hat{\tau}^z_I \hat{\tau}^z_J$作用在一个态上,则$I$和$J$对应的$\tau^x$均反号(同样依据泡利矩阵的性质,即$z$方向泡利矩阵可以翻转$x$方向泡利矩阵的本征态)。
两个算符的作用效果完全一样,所以实际上
\[
    \hat{\tau}^z_I \hat{\tau}^z_J = \hat{\sigma}^x_{ij},
\]
从而我们得到
\begin{equation}
    \hat{H} = - K \sum_{\pair{I, J}} \hat{\tau}^z_I \hat{\tau}^z_J - J \sum_{I} \hat{\tau}^x_I.
    \label{eq:z2-2d-tau-hamiltonian}
\end{equation}

现在没有规范冗余了——$\hat{\tau}^x_{I}$和$\hat{\tau}^z_I$都是规范不变量。
要看出自由度减少了多少,注意到二维格子中一个格子有四条边,每条边由两个格子分享,因此如果有$N$个格子(从而有$N$个格点),那么有$2N$条边。另一方面,只有$N$个方格。
因此如果只以$\hat{\tau}^z_I$为动力学自由度,则我们将希尔伯特空间的维数从$2^{2N}$降到了$2^N$。
丢自由度是正常的,因为在以上过程中我们抛弃了规范自由度,但是需要验证只以$\hat{\tau}^z_I$为动力学自由度是不是把一些并非规范自由度的自由度(它们没有出现在哈密顿量中)也抛弃了。
换句话说,我们需要验证,规范不等价的态是否给出不同的$\hat{\tau}^z_I$取值。
% TODO

\eqref{eq:z2-2d-tau-hamiltonian}正是\concept{横场伊辛模型},它是一个二维量子模型,其零温配分函数的精确形式对应一个三维经典统计模型,实际上这个三维经典统计模型就是一个各向异性的伊辛模型(在虚时间上的最近邻相互作用和空间方向上的最近邻相互作用不同)。
我们知道三维伊辛模型一定会出现相变,有一个顺磁相和一个铁磁相,这来自其普适类%
\footnote{
    虽然\eqref{eq:z2-2d-tau-hamiltonian}对应的经典统计模型是各向异性的,这并不改变其普适类,因为总是可以适当调节$\beta$的尺度让该经典统计模型变成各向同性的。
}%
,因此结论是,零温下横场伊辛模型——从而\Ztwo规范场——也会有一个相变,随着参数$K / J$的变化,从一个相切换到另一个相。

\subsection{规范荷和磁通量}

在$U(1)$规范场论中,设通过一个方格的磁通量为$\Phi$,则
\[
    \ee^{\ii \Phi} = \prod_{l \in \Box} t_{ij},
\]
于是在本文涉及的\Ztwo规范场中可以如法炮制地定义
\begin{equation}
    \ee^{\ii \Phi_I} = \prod_{l \in \Box_I} \hat{\sigma}^z_{ij} = \hat{\tau}^x_I,
\end{equation}
也即,我们用电子在格子上转一圈发生的相位改变来定义磁通量。与$U(1)$的情况不同,\Ztwo规范场中磁通量只有$0$和$\pi$两种,因为四个$\sigma^z$相乘要么是$1$要么是$-1$。
现在我们看到了$\hat{\tau}^x$的另一重意义:它标记了一个格子上的磁通量。
电磁场中的磁通量如果量子化的话需要一定条件,但是\Ztwo规范场中的磁通量就是量子化的,而且只有两个状态。
注意到$\tau_I^x$取$1$时能量较低而取$-1$时能量较高,我们可以将某个格子的磁通量取$\pi$当成一种激发态,称为\concept{m激发},以体现它和磁通量的相似之处。

另一个可以模仿电磁场引入的概念是\Ztwo规范荷,我们已经看到,\Ztwo规范变换对应的规范荷为\eqref{eq:z2-charge},这个量的取值只有$\pm 1$(因为是四个$\sigma^x$的乘积)。
由于\eqref{eq:z2-charge}守恒,我们有如下\Ztwo规范场的高斯定律:
\begin{equation}
    \prod_{j \in +} \sigma^x_{ij} = \text{\Ztwo -charge at $i$} = \const,
    \label{eq:gauss-z2}
\end{equation}
这个常数可以取$1$也可以取$-1$,但是不能一会是$1$,一会是$-1$。
这个额外的条件将希尔伯特空间划分成没有重叠的很多支,不同分支的\Ztwo规范荷分布不同。
虽然规范荷通常是通过规范场和物质场的耦合项引入的,在积掉物质场之后还是可以构造出规范荷的表达式,正如在电磁场中,即使我们积掉了物质场,麦克斯韦方程中还是会有一个电荷守恒方程
\[
    \pdv{\rho}{t} + \div{\vb*{j}} = 0
\]
一样——即使我们不知道电磁场实际上和一个物质场发生了耦合,我们还是可以将电荷当成电场线的某种特殊分布(源和汇),而以它们为某种激发。%
\footnote{
    \eqref{eq:gauss-z2}和麦克斯韦方程导出的电荷守恒方程有一个重要的区别,就是前者要求规范荷在每一点都守恒,而后者允许规范荷的流动。但这实际上并没有什么物理意义。
    麦克斯韦方程本身并不规定电荷应该如何流动(这是本构关系应该做的事情),因此,每一点的电荷密度算符和纯电磁场的哈密顿量也是对易的。
    在本节中尚未引入任何真的携带\Ztwo规范荷的场,如果引入了,本节中的$\hat{H}$就只是\Ztwo规范场的哈密顿量而不是完整的哈密顿量了,此时\eqref{eq:gauss-z2}的第一个等号当然仍然成立,但是每个点上的\Ztwo规范荷就未必总是不变的了,虽然规范对称性要求规范荷总量保持不变。 % TODO:总量是乘起来的还是加起来的??
    我们将这种没有动力学的规范荷称为\concept{测试规范荷},这个名称的意味是显然的;它相比有动力学的规范荷更容易处理,后者的哈密顿量在坐标表象下基本上不会是对角化的(例如\prettyref{sec:z2-on-lattice}中那样)。
}%
同理,在\Ztwo规范场中,$Q_i$取$-1$意味着更高的能量(计算一下能量期望值就知道),那么我们可以认为某个点$i$处$Q_i=-1$意味着这里出现了某个激发,从而一个\Ztwo规范荷被放置在了这里,无论其背后的机制是什么,无论是不是真的有一个物质场和\Ztwo规范场发生了耦合。%
\footnote{
    我们在\prettyref{sec:z2-on-lattice}中是通过对电子的紧束缚模型引入局域\Ztwo规范对称性而得到一个\Ztwo规范理论的,但不难看出,我们这里引入的\Ztwo规范荷自由度肯定不是坐标表象下的电子激发,因为坐标表象下的电子激发的哈密顿量不是对角化的,或者直观地说电子会“四处乱跑”,而这里的\Ztwo规范荷只是测试规范荷。
}%
我们称这种激发为\concept{e激发},以体现它和电荷的相似性。

\subsection{自由相和禁闭相}

现在的问题是,\Ztwo规范场在零温下的两个相都是什么?三维伊辛模型具有一个顺磁相和一个铁磁相,由于铁磁序的形成需要更多相互作用,$J/K$超过某个点时会出现顺磁相,否则出现铁磁相。
映射回二维横场伊辛模型,顺磁相意味着对偶格子上的各个$\tau$基本指向$x$方向,即有确定的磁通量;铁磁相意味着对偶格子上的各个$\tau$基本指向$z$方向,且要么几乎都为$1$要么几乎都为$-1$,没有确定的磁通量。
当$J$相对$K$很大时,系统处于顺磁相,此时的基态几乎就是每个$\sigma^x$都取$1$的纯态,此时直觉上看,电子可以畅行无阻;而当$J$相对$K$很小时,系统处于铁磁相,此时的基态不是$\sigma^x$的本征态,投影在$\sigma^x$上有正有负,那么电子似乎会被“迷惑”住,不知道怎么走。
那么,我们可以合乎情理地猜测,三维伊辛模型的顺磁相对应着\Ztwo规范场模型的一个解禁闭相,而三维伊辛模型的铁磁相对应着\Ztwo规范场模型的一个禁闭相。

这样的论证当然是不够的,所以下面做一些半定量的论证。考虑\eqref{eq:z2-2d-hamiltonian}的格点路径积分(即虚时间轴也是离散化的),也即它对应的三维模型,定义
\begin{equation}
    W(C) = \expval{\prod_C \sigma^z_l(\tau)},
\end{equation}
其中期望值内部的算符乘积称为\concept{Wilson环算符}(如果它不闭合,那么就是\concept{弦算符}),$C$是一个在虚时间方向上有延展的环路。
容易看出$W(C)$给出了从某个虚时间点开始产生一对e激发,按照$C$指示的轨迹扩散,经过一段时间之后又湮灭的概率(在经典理论中这是良定义的,因为没有任何不确定关系;具体有没有手段可以用量子测量的标准方法测出这个概率则是另外一回事),随着$C$扩大它理所当然会衰减,如果随着$C$扩大它衰减得很快那么就说明e激发总是成对地被束缚在一起。
我们本来可以使用两点格林函数来表征e激发被束缚的强度的,但是由于两点格林函数不是规范不变的,任何两点格林函数都是零。
$W(C)$的衰减方式有两种典型的形式:一种是\concept{面积定律},即
\begin{equation}
    W(C) \sim \ee^{-A},
\end{equation}
其中$A$是$C$围成的面积,另一种是\concept{周长定律},即
\begin{equation}
    W(C) \sim \ee^{-L}.
\end{equation}
设$C$持续时间为$\tau$,则按照虚时间演化,应该有
\[
    W(C) \sim \ee^{-\tau \Delta E},
\]
其中$\Delta E$是两个e激发同时出现造成的能量上升。我们马上可以看出,由于
\[
    A \sim \tau R,
\]
其中$R$是两个e激发的距离,如果面积定律成立,必然有
\[
    \Delta E \sim R,
\]
这是典型的禁闭效应:两个e激发离得越远,能量越高,当能量高到一定程度时涨落会导致新的e激发产生,从而产生两对离得非常近的e激发。
另一方面,如果周长定律成立,则
\[
    \Delta E \sim 1 + \frac{R}{\tau},
\]
而由于粒子通常不会跑太远,可以认为$\Delta E$基本上是一个常数,因此没有禁闭效应。

那么,前述横场伊辛模型给出的两个相是不是分别对应面积定律和周长定律?
我们尝试将$W(C)$映射到横场伊辛模型中,因为$C$围成的区域内部的$\sigma^z$会被乘两次,于是就给出$1$,于是
\[
    \prod_{l \in C} \sigma^z_l = \prod_{l \in D} \sigma^z_l,
\]
依照$\tau^x$的定义即得到
\begin{equation}
    W(C) = \expval{\prod_{I \in D} \tau^x_I(\tau)},
\end{equation}
其中$D$是$C$围成的区域。
由于只有两个相,可以在$J/K$很小或很大时分别做微扰论,由此得到的关于相的结构的信息在整个相内部都是成立的。
$J \gg K$的情况对应顺磁相,系统基态形如
\[
    \ket*{\text{ground}} = \ket*{\tau^x = \rightarrow \rightarrow \cdots \rightarrow} + \frac{K}{J} \sum_{i, j} \ket*{\tau^x = \rightarrow \cdots \leftarrow_i \cdots \leftarrow_j \cdots \rightarrow} + \cdots.
\]
上式看起来很奇怪,不过真的去算期望会发现翻转两个$\tau^x$比翻转一个能量更低。(翻转两个的话,$\hat{\tau}^z_I \hat{\tau}^z_J$项的期望不为零)
如果格点$i$和$j$在$C$内部,那么它不会对期望值有什么贡献,因为$-1$平方还是$1$;格点$i$和$j$在$C$外部那肯定也不会有什么贡献。
唯一会让期望值偏离$1$的激发是$i, j$中一个在$C$内部一个在$C$外部,于是我们有
\[
    \expval{\prod_{I \in D} \hat{\tau}^x_I} \sim 1 - \frac{K}{J} L(C).
\]
那么,合理的猜测是,在顺磁相中应该有周长定律成立,于是没有禁闭。
铁磁相的讨论是类似的。 % TODO
我们仅仅讨论了两个极限情况,没有得到完整的$\ee$指数,但是一般的情况计算起来是非常困难的,此处略去。

总之,铁磁相对应禁闭的\Ztwo模型的状态,其中e激发被禁闭;顺磁相则对应非禁闭的\Ztwo模型的状态。禁闭相中,e激发不再是有意义的低能自由度。
由于禁闭相的基态非常接近$\hat{\tau}^z$的本征态,在其上讨论$\hat{\tau}^x$的排列模式——也就是m激发——并没有意义。
因此禁闭相中的“\Ztwo激发”——e激发和m激发——都没有意义,即在这里\Ztwo规范场的行为并不十分有趣。

\subsection{拓扑激发}



\end{document}