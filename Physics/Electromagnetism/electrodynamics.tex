\documentclass[UTF8, a4paper]{ctexart}

\usepackage{geometry}
\usepackage{titling}
\usepackage{titlesec}
\usepackage{paralist}
\usepackage{footnote}
\usepackage{enumerate}
\usepackage{amsmath, amssymb, amsthm}
\usepackage{cite}
\usepackage{graphicx}
\usepackage{subfigure}
\usepackage{physics}
\usepackage{slashed}
\usepackage[colorlinks, linkcolor=black, anchorcolor=black, citecolor=black]{hyperref}
\usepackage{prettyref}

\geometry{left=3.28cm,right=3.28cm,top=2.54cm,bottom=2.54cm}
\titlespacing{\paragraph}{0pt}{1pt}{10pt}[20pt]
\setlength{\droptitle}{-5em}
\preauthor{\vspace{-10pt}\begin{center}}
\postauthor{\par\end{center}}

\newcommand*{\ee}{\mathrm{e}}
\newcommand*{\ii}{\mathrm{i}}
\newcommand*{\st}{\quad \text{s.t.} \quad}
\newcommand*{\const}{\mathrm{const}}
\newcommand*{\natnums}{\mathbb{N}}
\newcommand*{\reals}{\mathbb{R}}
\newcommand*{\complexes}{\mathbb{C}}
\DeclareMathOperator{\timeorder}{T}
\newcommand*{\ogroup}[1]{\mathrm{O}(#1)}
\newcommand*{\sogroup}[1]{\mathrm{SO}(#1)}
\DeclareMathOperator{\legpoly}{P}
\DeclareMathOperator{\diag}{diag}

\renewcommand{\emph}[1]{\textbf{#1}}
\newcommand*{\concept}[1]{\underline{\textbf{#1}}}

\newrefformat{sec}{第\ref{#1}节}
\newrefformat{note}{注\ref{#1}}
\newrefformat{fig}{图\ref{#1}}
\renewcommand{\autoref}{\prettyref}

\title{电动力学基本原理}
\author{吴何友}

\begin{document}

\maketitle

\section{电动力学是局域$U(1)$理论}

\subsection{规范场和狄拉克旋量场的最小耦合}

在相对论性量子场论中——也即,在以闵可夫斯基时空(具体细节见\prettyref{sec:minkowsky})
为底流形的量子场论中,我们尝试将一个自旋$1/2$的狄拉克旋量场和一个无质量矢量场耦合起来。
旋量场的拉氏量为
\begin{equation}
    \mathcal{L}_\text{spin} = - m \bar{\psi} \psi + \ii \bar{\psi} \gamma_\mu \partial^\mu \psi,
    \label{eq:spin-lagrangian}
\end{equation}
而矢量场的拉氏量为
\begin{equation}
    \mathcal{L}_\text{vec} = - \frac{1}{2} (\partial^\mu A^\nu \partial_\mu A_\nu - \partial^\mu A^\nu \partial_\nu A_\mu).
    \label{eq:vec-lagrangian}
\end{equation}
很容易看出\eqref{eq:spin-lagrangian}具有全局$U(1)$对称性:它在变换
\[
    \psi \longrightarrow \psi' = \psi \ee^{\ii \alpha}
\]
下保持不变。同样,\eqref{eq:vec-lagrangian}具有场的全局平移不变性,它在变换
\[
    A^\mu \longrightarrow A'^\mu = A^\mu + a^\mu
\]
下保持不变。这两个对称性都是全局的:如果$\alpha$或$a^\mu$依赖于坐标,由于导数的链式法则,会多出来一些项。
具体来说,我们有
\[
    \mathcal{L}_\text{spin} \longrightarrow \mathcal{L}_\text{spin}' = \mathcal{L}_\text{spin} - \bar{\psi} \psi \gamma_\mu \partial^\mu \alpha.
\]
对于矢量场,在$a^\mu$的形式任意的情况下,$\mathcal{L}_\text{vec}$的变换无规律可循,但是如果我们用某个标量的梯度$\partial^\mu a$代替$a^\mu$,那么有
\[
    A^\mu \longrightarrow A'^\mu = A^\mu + \partial^\mu a, \quad
    \mathcal{L}_\text{vec} \longrightarrow \mathcal{L}_\text{vec}' = \mathcal{L}_\text{vec}.
\]
也就是说,矢量场的场的平移对称性实际上可以稍加推广而仍然成立。

$\psi$的变换的相位因子是一个标量;$A^\mu$的场的平移量也是一个标量的梯度。很容易想到的尝试是,我们是否可以将两个场耦合起来,并要求整个系统在变换($e$是常数而$a(\vb*{x})$依赖于坐标)
\begin{equation}
    \psi \longrightarrow \psi' = \psi \ee^{\ii e a}, \quad A^\mu \longrightarrow A'^\mu = A^\mu + \partial^\mu a
    \label{eq:gauge-transformation}
\end{equation}
下保持不变?自由矢量场部分肯定是不变的,那么就要适当设计相互作用项的形式,把$\psi$做局域$U(1)$变换之后拉氏量多出来的一项吸收掉。
当然,如果相互作用项是$- e A^\mu \bar{\psi} \gamma_\mu \psi$,那就正好,因为
\[
    - e A^\mu \bar{\psi} \gamma_\mu \psi - \bar{\psi} \psi \gamma_\mu \partial^\mu (e a) = - e A'^\mu \bar{\psi'} \gamma_\mu \psi'.
\]
于是我们得出结论:拉氏量
\begin{equation}
    \mathcal{L} = 
    \underbrace{- m \bar{\psi} \psi + \ii \bar{\psi} \gamma_\mu \partial^\mu \psi }_{\mathcal{L}_\text{spin}}
    \underbrace{- \frac{1}{2} (\partial^\mu A^\nu \partial_\mu A_\nu - \partial^\mu A^\nu \partial_\nu A_\mu)}_{\mathcal{L}_\text{vec}}
    \underbrace{- e A^\mu \bar{\psi} \gamma_\mu \psi}_\text{interaction}
    \label{eq:qed-lagrangian}
\end{equation}
具有局域$U(1)$不变性。推导出\eqref{eq:qed-lagrangian}的方法就是\concept{最小耦合}。

于是,我们得出结论:一个自旋$1/2$狄拉克旋量场和一个无质量矢量场耦合,并要求理论具有\emph{局域}$U(1)$对称性,那么就会得到
我们将会看到,这个理论实际上就是电动力学。

还可以引入一些记号来简化\eqref{eq:qed-lagrangian}。首先引入反对称张量\concept{电磁张量}
\begin{equation}
    F^{\mu \nu} = \partial^\mu A^\nu - \partial^\nu A^\mu,
\end{equation}
则自由矢量场拉氏量为
\[
    \mathcal{L}_\text{vec} = - \frac{1}{4} F_{\mu \nu} F^{\mu \nu}.
\]
另一方面,相互作用项和含有旋量场的导数的项形式非常接近,因此可以定义\concept{协变导数}
\begin{equation}
    \ii D^\mu = \ii \partial^\mu - e A^\mu,
\end{equation}
最后将\eqref{eq:qed-lagrangian}写成
\begin{equation}
    \begin{aligned}
        \mathcal{L} &= \bar{\psi} (\ii \gamma^\mu D_\mu - m) \psi - \frac{1}{4} F_{\mu \nu} F^{\mu \nu} \\
        &= \bar{\psi} (\ii \slashed{D} - m) \psi - \frac{1}{4} F_{\mu \nu} F^{\mu \nu}. 
    \end{aligned}
    \label{eq:short-qed-lagrangian}
\end{equation}
这里我们用斜杠记号表示$\gamma_\mu A^\mu$。

\subsection{运动方程和守恒量}

从\eqref{eq:qed-lagrangian}马上可以使用欧拉-拉格朗日方程写出运动方程。对$\psi$我们有
\[
    \ii \partial_\mu \bar{\psi} \gamma^\mu + e \bar{\psi} \gamma_\mu A^\mu + m \bar{\psi} = 0,
\]
对其取共轭,或者对$\bar{\psi}$应用欧拉-拉格朗日方程,就得到
\begin{equation}
    \ii \gamma^\mu \partial_\mu \psi - m \psi = e \gamma_\mu A^\mu \psi.
    \label{eq:movement-eq-1}
\end{equation}
对$A^\mu$应用欧拉-拉格朗日方程,则有
\begin{equation}
    \partial_\mu F^{\mu \nu} = \partial_\mu (\partial^\mu A^\nu - \partial^\nu A^\mu) = e \bar{\psi} \gamma^\nu \psi.
    \label{eq:movement-eq-2}
\end{equation}
以上两个方程给出了\eqref{eq:qed-lagrangian}的运动方程。(如前所述,$\psi$和$\bar{\psi}$虽然是独立的场,但它们的运动方程并不独立,因为运动方程是一阶的)

现在我们分析局域$U(1)$对称性带来的守恒量。在局域$U(1)$变换下,我们有
\[
    \var{\psi} = \ii e \psi \var{a}, \quad \var{A^\mu} = \partial^\mu \var{a},
\]
则守恒流为
\[
    \begin{aligned}
        J^\mu \var{a} &= - e \bar{\psi} \gamma^\mu \psi \var{a} + (-\partial^\mu A^\nu + \partial^\nu A^\mu) \partial_\nu \var{a} \\
        &= - e \bar{\psi} \gamma^\mu \psi \var{a} - \partial_\nu (\partial^\nu A^\mu - \partial^\mu A^\nu) \var{a},
    \end{aligned}
\]
第二个等号实际上是忽略了一个边界项后得到的结果。%
\footnote{考虑到$\var{a}$在每一点都可以独立地变化,$\int A\var{a} = \int B \var{a}$意味着$A=B$。}%
无论如何,这个守恒流的第二项是平凡的,因为它就是电磁张量的一个指标求散度之后的结果,它的散度当然是零。
那么,我们就有以下守恒荷:
\begin{equation}
    J^\mu = e \bar{\psi} \gamma^\mu \psi , \quad \partial_\mu J^\mu = 0.
    \label{eq:four-current}
\end{equation}
回过头看,实际上这是\emph{全局$U(1)$对称性}的守恒荷——全局$U(1)$对称性中$a$在时空上是均匀的,那么$\partial_\mu \var{a}$就是零,正好让含有$A$的那个平凡的项消失。
实际上从\eqref{eq:movement-eq-2}中我们也可以得到这个守恒流。由于电磁张量是反对称的,我们有:
\[
    0 = \partial_\mu \partial_\nu F^{\mu \nu} = \partial_\mu (e \bar{\psi} \gamma^\mu \psi).
\]
这就导出了\eqref{eq:four-current}。我们将会看到实际上这个是\concept{电荷}。或者说,电荷是全局$U(1)$对称性对应的守恒荷。

使用\eqref{eq:four-current}可以将\eqref{eq:movement-eq-2}写成
\begin{equation}
    \partial_\mu F^{\mu \nu} = J^\nu.
    \label{eq:four-maxwell}
\end{equation}
我们将会看到,这实际上就是麦克斯韦方程。

\subsection{规范}

电动力学在局域$U(1)$变换下的对称性实际上是一个\emph{规范对称性},也就是说,做任意的局域$U(1)$变换,不会有任何可以观察到的变化,也就是说\eqref{eq:qed-lagrangian}中实际上有多余的自由度。
我们需要对$A$和$\psi$施加适当的约束,以确保满足这个约束的$A$和$\psi$取值可以覆盖所有物理上可能产生的状态,同时不含有任何冗余的自由度,也即要\emph{选取一个规范}。

形式最漂亮的应该是\concept{洛伦兹规范},也就是
\begin{equation}
    \partial_\mu A^\mu = 0,
\end{equation}
在这个规范下\eqref{eq:four-maxwell}转化为
\begin{equation}
    \Box^2 A = J, \quad \Box^2 = \partial_\mu \partial^\mu.
\end{equation}
我们得到了一个四维波动方程,当然,这就是\concept{电磁波}。

\section{麦克斯韦方程}

如前所述,\eqref{eq:four-maxwell}实际上是麦克斯韦方程。本节将要展示它如何能够化成在实际计算中更加常用的形式。

\subsection{基于电场和磁场的表述}

\subsubsection{三维形式的电场和磁场}

我们可以直接选择将电磁张量的分量写下来。我们首先观察电磁张量在\emph{空间}坐标变换(不涉及时间)下的变换。
首先电磁张量的对角元都是零,非对角元一共有$6$个独立变量,我们不妨设它们为
\[
    F^{\mu \nu} = \pmqty{
        0 & E_x & E_y & E_z \\
        -E_x & 0 & -B_z & B_y \\
        -E_y & B_z & 0 & -B_x \\
        -E_z & -B_y & B_x & 0
    }.
\]
一个不涉及时间的坐标变换一定形如
\[
    \pmqty{\dmat{1, \xmat*{a}{3}{3} }},
\]
我们将它作用在$F^{\mu \nu}$上(注意二阶张量的两个指标都要作用一遍),会发现$\pmqty{E_x & E_y & E_z}$在坐标变换矩阵$\{a_{ij}\}$下发生了坐标变换,而
\[
    \pmqty{
        0 & -B_z & B_y \\
        B_z & 0 & -B_x \\
        -B_y & B_x & 0
    }
\]
在坐标变换矩阵$\{a_{ij}\}$下作为张量发生了坐标变换。
这就意味着$(E_x, E_y, E_z)$构成一个三维矢量的分量,而含有$B_i$的那部分矩阵构成一个三维二阶张量的分量。我们称前者为\concept{电场},它是一个极矢量。
至于后者,它是某个三元组和一个三指标反对称张量缩并而成的,反对称张量在连续的坐标变换下确实按照张量的方式变换,但是在反射下会变号,因此三元组$(B_x, B_y, B_z)$构成一个轴矢量的分量,称为\textbf{磁场}。

\subsubsection{单位制}\label{sec:unit-system}

\subsubsection{边界条件}

\subsubsection{能量}

\subsection{反推出电势和磁矢势}

\section{闵可夫斯基时空和狭义相对论}\label{sec:minkowsky}

\subsection{闵可夫斯基时空中的几何对象}

\subsubsection{四维矢量的分量}\label{sec:components-of-four-vector}

本节暂时不区分时间和空间的单位,具体的讨论见\prettyref{sec:unit-system}。
为了避免不必要的麻烦,我们在直角坐标系加时间维下讨论问题。设时间维为第$0$维,闵可夫斯基时空的度规为
\begin{equation}
    g_{\mu \nu} = g^{\mu \nu} = \diag(1, -1, -1, -1).
\end{equation}
对一个四维矢量$A^\mu$,我们设
\begin{equation}
    A^\mu = (A^0, \vb*{A}),
\end{equation}
则由指标升降关系自然得到
\begin{equation}
    A_\mu = (A_0, -\vb*{A}), \quad A_0 = A^0.
\end{equation}
梯度算符也是矢量,也即应该有
\begin{equation}
    \partial^\mu = \pdv{x_\mu} = (\partial^0, \partial^i) = (\partial_t, \grad), \quad \partial_t = \partial^t,
\end{equation}
对应的
\begin{equation}
    \partial_\mu = \pdv{x^\mu} = (\partial_t, -\grad).
\end{equation}
这样就有
\begin{equation}
    \Box^2 = \partial_t^2 - \laplacian.
\end{equation}

\subsection{从时空结构推导出狭义相对论}

\section{量子化}

\end{document}