\documentclass[UTF8, a4paper]{ctexart}

\usepackage{geometry}
\usepackage{titling}
\usepackage{titlesec}
\usepackage{paralist}
\usepackage{footnote}
\usepackage{enumerate}
\usepackage{amsmath, amssymb, amsthm}
\usepackage{cite}
\usepackage{graphicx}
\usepackage{subfigure}
\usepackage{physics}
\usepackage{slashed}
\usepackage[colorlinks, linkcolor=black, anchorcolor=black, citecolor=black]{hyperref}

\geometry{left=3.28cm,right=3.28cm,top=2.54cm,bottom=2.54cm}
\titlespacing{\paragraph}{0pt}{1pt}{10pt}[20pt]
\setlength{\droptitle}{-5em}
\preauthor{\vspace{-10pt}\begin{center}}
\postauthor{\par\end{center}}

\newcommand*{\ee}{\mathrm{e}}
\newcommand*{\ii}{\mathrm{i}}
\newcommand*{\st}{\quad \text{s.t.} \quad}
\newcommand*{\const}{\mathrm{const}}
\newcommand*{\natnums}{\mathbb{N}}
\newcommand*{\reals}{\mathbb{R}}
\newcommand*{\complexes}{\mathbb{C}}
\DeclareMathOperator{\timeorder}{T}
\newcommand*{\ogroup}[1]{\mathrm{O}(#1)}
\newcommand*{\sogroup}[1]{\mathrm{SO}(#1)}
\DeclareMathOperator{\legpoly}{P}

\renewcommand{\emph}[1]{\textbf{#1}}
\newcommand*{\concept}[1]{\underline{\textbf{#1}}}

\title{电动力学基本原理}
\author{吴何友}

\begin{document}

\maketitle

\section{电动力学是局域$U(1)$理论}

\subsection{规范场和狄拉克旋量场的最小耦合}

在相对论性量子场论中——也即,在以闵可夫斯基时空为底流形的量子场论中,我们尝试将一个自旋$1/2$的狄拉克旋量场和一个无质量矢量场耦合起来。
旋量场的拉氏量为
\begin{equation}
    \mathcal{L}_\text{spin} = - m \bar{\psi} \psi + \ii \bar{\psi} \gamma_\mu \partial^\mu \psi,
    \label{eq:spin-lagrangian}
\end{equation}
而矢量场的拉氏量为
\begin{equation}
    \mathcal{L}_\text{vec} = \partial^\mu A^\nu \partial_\mu A_\nu - \partial^\mu A^\nu \partial_\nu A_\mu.
    \label{eq:vec-lagrangian}
\end{equation}
很容易看出\eqref{eq:spin-lagrangian}具有全局$U(1)$对称性:它在变换
\[
    \psi \longrightarrow \psi' = \psi \ee^{\ii \alpha}
\]
下保持不变。同样,\eqref{eq:vec-lagrangian}具有场的全局平移不变性,它在变换
\[
    A^\mu \longrightarrow A'^\mu = A^\mu + a^\mu
\]
下保持不变。这两个对称性都是全局的:如果$\alpha$或$a^\mu$依赖于坐标,由于导数的链式法则,会多出来一些项。
具体来说,我们有
\[
    \mathcal{L}_\text{spin} \longrightarrow \mathcal{L}_\text{spin}' = \mathcal{L}_\text{spin} - \bar{\psi} \psi \gamma_\mu \partial^\mu \alpha.
\]
对于矢量场,在$a^\mu$的形式任意的情况下,$\mathcal{L}_\text{vec}$的变换无规律可循,但是如果我们用某个标量的梯度$\partial^\mu a$代替$a^\mu$,那么有
\[
    A^\mu \longrightarrow A'^\mu = A^\mu + \partial^\mu a, \quad
    \mathcal{L}_\text{vec} \longrightarrow \mathcal{L}_\text{vec}' = \mathcal{L}_\text{vec}.
\]
也就是说,矢量场的场的平移对称性实际上可以稍加推广而仍然成立。

$\psi$的变换的相位因子是一个标量;$A^\mu$的场的平移量也是一个标量的梯度。很容易想到的尝试是,我们是否可以将两个场耦合起来,并要求整个系统在变换($e$是常数而$a(\vb*{x})$依赖于坐标)
\begin{equation}
    \psi \longrightarrow \psi' = \psi \ee^{\ii e a}, \quad A^\mu \longrightarrow A'^\mu = A^\mu + \partial^\mu a
    \label{eq:gauge-transformation}
\end{equation}
下保持不变?自由矢量场部分肯定是不变的,那么就要适当设计相互作用项的形式,把$\psi$做局域$U(1)$变换之后拉氏量多出来的一项吸收掉。
当然,如果相互作用项是$- e A^\mu \bar{\psi} \gamma_\mu \psi$,那就正好,因为
\[
    - e A^\mu \bar{\psi} \gamma_\mu \psi - \bar{\psi} \psi \gamma_\mu \partial^\mu (e a) = - e A'^\mu \bar{\psi'} \gamma_\mu \psi'.
\]
于是我们得出结论:拉氏量
\begin{equation}
    \mathcal{L} = 
    \underbrace{- m \bar{\psi} \psi + \ii \bar{\psi} \gamma_\mu \partial^\mu \psi }_{\mathcal{L}_\text{spin}}
    \underbrace{+ \partial^\mu A^\nu \partial_\mu A_\nu - \partial^\mu A^\nu \partial_\nu A_\mu}_{\mathcal{L}_\text{vec}}
    \underbrace{- e A^\mu \bar{\psi} \gamma_\mu \psi}_\text{interaction}
    \label{eq:qed-lagrangian}
\end{equation}
具有局域$U(1)$不变性。推导出\eqref{eq:qed-lagrangian}的方法就是\concept{最小耦合}。

于是,我们得出结论:一个自旋$1/2$狄拉克旋量场和一个无质量矢量场耦合,并要求理论具有\emph{局域}$U(1)$对称性,那么就会得到
我们将会看到,这个理论实际上就是电动力学。

还可以引入一些记号来简化\eqref{eq:qed-lagrangian}。首先引入反对称张量\concept{电磁张量}
\begin{equation}
    F^{\mu \nu} = \partial^\mu A^\nu - \partial^\nu A^\mu,
\end{equation}
则自由矢量场拉氏量为
\[
    \mathcal{L}_\text{vec} = - \frac{1}{4} F_{\mu \nu} F^{\mu \nu}.
\]
另一方面,相互作用项和含有旋量场的导数的项形式非常接近,因此可以定义\concept{协变导数}
\begin{equation}
    \ii D^\mu = \ii \partial^\mu - e A^\mu,
\end{equation}
最后将\eqref{eq:qed-lagrangian}写成
\begin{equation}
    \begin{aligned}
        \mathcal{L} &= \bar{\psi} (\ii \gamma^\mu D_\mu - m) \psi - \frac{1}{4} F_{\mu \nu} F^{\mu \nu} \\
        &= \bar{\psi} (\ii \slashed{D} - m) \psi - \frac{1}{4} F_{\mu \nu} F^{\mu \nu}. 
    \end{aligned}
\end{equation}
这里我们用斜杠记号表示$\gamma_\mu A^\mu$。

\subsection{电荷}

考虑

\section{麦克斯韦方程}

\subsection{麦克斯韦方程组的几种形式}

\subsection{单位制}

\subsection{边界条件}

\subsection{能量}

\section{狭义相对论与电动力学}

\end{document}