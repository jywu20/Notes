\documentclass[UTF8, a4paper]{ctexart}

\usepackage{geometry}
\usepackage{titling}
\usepackage{titlesec}
\usepackage{paralist}
\usepackage{footnote}
\usepackage{enumerate}
\usepackage{amsmath, amssymb, amsthm}
\usepackage{cite}
\usepackage{graphicx}
\usepackage{subfigure}
\usepackage{physics}
\usepackage{slashed}
\usepackage[colorlinks, linkcolor=black, anchorcolor=black, citecolor=black]{hyperref}
\usepackage{prettyref}

\geometry{left=3.28cm,right=3.28cm,top=2.54cm,bottom=2.54cm}
\titlespacing{\paragraph}{0pt}{1pt}{10pt}[20pt]
\setlength{\droptitle}{-5em}
\preauthor{\vspace{-10pt}\begin{center}}
\postauthor{\par\end{center}}

\newcommand*{\ee}{\mathrm{e}}
\newcommand*{\ii}{\mathrm{i}}
\newcommand*{\st}{\quad \text{s.t.} \quad}
\newcommand*{\const}{\mathrm{const}}
\newcommand*{\natnums}{\mathbb{N}}
\newcommand*{\reals}{\mathbb{R}}
\newcommand*{\complexes}{\mathbb{C}}
\DeclareMathOperator{\timeorder}{T}
\newcommand*{\ogroup}[1]{\mathrm{O}(#1)}
\newcommand*{\sogroup}[1]{\mathrm{SO}(#1)}
\DeclareMathOperator{\legpoly}{P}
\DeclareMathOperator{\diag}{diag}

\renewcommand{\emph}[1]{\textbf{#1}}
\newcommand*{\concept}[1]{\underline{\textbf{#1}}}

\newrefformat{sec}{第\ref{#1}节}
\newrefformat{note}{注\ref{#1}}
\newrefformat{fig}{图\ref{#1}}
\renewcommand{\autoref}{\prettyref}

\title{电动力学基本原理}
\author{吴晋渊}

\begin{document}

\maketitle

\section{闵可夫斯基时空}\label{sec:minkowsky}

基本上整个高能物理中的场论都以四维闵可夫斯基时空为底流形。
四维闵可夫斯基时空有三个空间维,一个时间维。

\subsection{闵可夫斯基时空的导出}

\subsubsection{光速不变原理的推论}

我们假定存在一些参考系,在其中不受外力的粒子做匀速直线运动,并将这样的参考系称为\concept{惯性参考系}。惯性参考系的存在性是直觉上非常合理的。
考虑两个惯性参考系$S$和$S'$,分别用$(t, x, y, z)$和$(t', x', y', z')$标记其中的事件。
我们假定有某种运动的速度在不同的惯性参考系中都是一样的,这个速度称为\concept{光速},这个假定称为\concept{光速不变原理}。
“光速”一词来自波动方程中的光速在参考系变换下不变这一事实,不过在这里我们暂时将它作为一个无意义的名词使用;实际上,狭义相对论中光速的作用在于在时间和空间之间建立关系。

设一个信号%
\footnote{什么是“信号”其实是需要手动指定的。可以想象非常远的距离的两端依次发生了两个事件,但这并不代表有什么东西从这段距离的一端传到了另一端。
一个粒子显然是一个信号,空间中的场的可以连续传播的构型的“位置”或许也可以看成一个信号。
}%
以光速运动,它某时刻在某地出现,记作事件$P_1$,另一时刻在另一地点出现,记作事件$P_2$。
于是我们有
\[
    (x_1 - x_2)^2 + (y_1 - y_2)^2 + (z_1 - z_2)^2 = c^2 (t_1 - t_2)^2, 
\]
以及
\[
    (x_1' - x_2')^2 + (y_1' - y_2')^2 + (z_1' - z_2')^2 = c^2 (t_1' - t_2')^2.
\]
在这里,光速不变性的不同寻常之处已经可以体现出来了:设$S'$系相对$S$以速度$u$沿着$x$轴运动,则
\[
    x_1' = x_1 + u t_1, \quad x_2' = x_2 + u t_2,
\]
于是
\[
    (x_1 - x_2 + u t_1 - u t_2)^2 - (x_1 - x_2)^2 = c^2 (t_1' - t_2')^2 - c^2 (t_1 - t_2)^2,
\]
因此
\[
    (t_1' - t_2')^2 \neq (t_1 - t_2)^2.
\]
这意味着两个参考系中时间流动的速率不一样。因此满足光速不变性的参考系变换肯定不是伽利略变换。

简而言之,如果一个信号以光速运动,那么在任何参考系中都有
\[
    c^2 (t_1 - t_2)^2 - (x_1 - x_2)^2 - (y_1 - y_2)^2 - (z_1 - z_2)^2 = 0.    
\]
反之,如果某个参考系中有上式成立,那么信号一定以光速运动。于是我们定义两个事件的\concept{间隔}为
\begin{equation}
    s^2 = c^2 (t_1 - t_2)^2 - (x_1 - x_2)^2 - (y_1 - y_2)^2 - (z_1 - z_2)^2,
\end{equation}
并且如果一个参考系中的间隔为零,那么别的参考系中的间隔也为零。

现在我们考虑靠得很近的两个事件,即考虑$\dd{s^2}$。坐标变换是连续的,即$\dd{s^2}$和$\dd{s'^2}$是同阶小量,并且$\dd{s^2}=0$时$\dd{s'^2}=0$。
因此我们设
\[
    \dd{s^2} = a \dd{s'^2}.
\]
现在再引入一个惯性参考系$S''$。由空间的均匀性和各向同性,$a$不应该显含任何坐标,无论是时间还是空间,因此它只应该依赖于两个惯性参考系之间的相对速度(我们不能预期$S$在$S'$中的运动速度和$S'$在$S$中的运动速度只差一个负号,但是它们之间显然是有关系的),而且不能依赖于相对速度的方向。
设$S'$在$S$中的运动速度大小为$V_1$,$S''$在$S$中的运动速度大小为$V_2$,$S''$在$S'$中的运动速度为$V_3$,则显然
\[
    \dd{s'^2} = a(V_1) \dd{s^2}, \quad \dd{s''^2} = a(V_2) \dd{s^2}, \quad \dd{s''^2} = a(V_3) \dd{s'^2},
\]
于是
\[
    a(V_3) = \frac{a(V_1)}{a(V_2)}.
\]
但是$V_3$不仅依赖于$V_1$和$V_2$,肯定还依赖于它们的相对角度,而上式右边却没有出现任何相对角度,因此仅有的可能是$a$根本就是一个常数,代入上式就有$a = 1$,于是
\begin{equation}
    \dd{s^2} = \dd{s'^2},
\end{equation}
即时空间隔在不同的惯性参考系中完全一样。在重新定义时间的单位,从而用$t$代替$ct$之后,不同惯性参考系实际上成了一个\concept{四维闵可夫斯基时空}的坐标系,惯性参考系之间的坐标变换就是参考系变换,需要满足的条件就是间隔微元相等。

不同惯性系之间相对做匀速直线运动,即$(x', y', z')$可以写成$(x, y, z)$加上某个三维矢量乘以$t$,或许加上一个常数,即从$(t, x, y, z)$到$(x', y', z')$的变换是仿射的。
代入间隔微元相等的条件中可以发现到$t'$的变换也是仿射的。
反之,如果两个参考系之间的坐标变换是仿射的,且其中一个是惯性系,那么另一个当然也是惯性系。
因此,惯性系之间的坐标变换总是仿射的,并且保持时空间隔不变,即为等度规变换。
惯性系之间的坐标变换在忽略了平移之后就是一个矩阵群,称为\concept{洛伦兹变换}。

洛伦兹变换

\subsection{闵可夫斯基时空中的几何对象}

\subsubsection{四维矢量的分量}\label{sec:components-of-four-vector}

本节暂时不区分时间和空间的单位,具体的讨论见\prettyref{sec:unit-system}。
为了避免不必要的麻烦,我们在直角坐标系加时间维下讨论问题。设时间维为第$0$维,闵可夫斯基时空的度规为
\begin{equation}
    g_{\mu \nu} = g^{\mu \nu} = \diag(1, -1, -1, -1).
\end{equation}
对一个四维矢量$A^\mu$,我们设
\begin{equation}
    A^\mu = (A^0, \vb*{A}),
\end{equation}
则由指标升降关系自然得到
\begin{equation}
    A_\mu = (A_0, -\vb*{A}), \quad A_0 = A^0.
\end{equation}
虽然梯度算符的行为和矢量非常相似,由于按照定义
\[
    \partial_\mu = \pdv{x^\mu}, \quad \partial^\mu = \pdv{x_\mu},
\]
而坐标显然是货真价实的矢量,我们有
\begin{equation}
    \partial^\mu = \pdv{x_\mu} = (\partial^0, \partial^i) = (\partial_t, - \grad), \quad \partial_t = \partial^t,
\end{equation}
对应的
\begin{equation}
    \partial_\mu = \pdv{x^\mu} = (\partial_t, \grad).
\end{equation}
这里有一个看起来比较奇怪的地方,就是
\[
    \partial^i = - \grad,
\]
但是上式中的$\partial^i$算符定义在闵可夫斯基时空中,而闵可夫斯基时空中空间维的度规为$-1$。
如果$\partial^i$是指三维欧氏空间中的那个梯度,那么就有
\[
    \partial^i = \grad,
\]
也就是说闵可夫斯基时空和欧氏空间中的空间梯度算符差一个负号。为了避免混乱,之后我们将$\partial^i$局限为欧氏空间中的梯度算符。

闵可夫斯基时空版本的拉普拉斯算符——也就是达朗贝尔算符——定义为
\begin{equation}
    \Box^2 = \partial_\mu \partial^\mu = \partial_t^2 - \laplacian.
\end{equation}

\section{电动力学是$U(1)$规范理论}

\subsection{规范场和狄拉克旋量场的最小耦合}

在相对论性量子场论中——也即,在以闵可夫斯基时空(具体细节见\prettyref{sec:minkowsky})
为底流形的量子场论中,我们尝试将一个自旋$1/2$的狄拉克旋量场和一个无质量矢量场耦合起来。
旋量场的拉氏量为
\begin{equation}
    \mathcal{L}_\text{spin} = - m \bar{\psi} \psi + \ii \bar{\psi} \gamma_\mu \partial^\mu \psi,
    \label{eq:spin-lagrangian}
\end{equation}
而矢量场的拉氏量为
\begin{equation}
    \mathcal{L}_\text{vec} = - \frac{1}{2} (\partial^\mu A^\nu \partial_\mu A_\nu - \partial^\mu A^\nu \partial_\nu A_\mu).
    \label{eq:vec-lagrangian}
\end{equation}
很容易看出\eqref{eq:spin-lagrangian}具有全局$U(1)$对称性:它在变换
\[
    \psi \longrightarrow \psi' = \psi \ee^{\ii \alpha}
\]
下保持不变。同样,\eqref{eq:vec-lagrangian}具有场的全局平移不变性(自由无质量矢量场的规范对称性),它在变换
\[
    A^\mu \longrightarrow A'^\mu = A^\mu + a^\mu
\]
下保持不变。这两个对称性都是全局的:如果$\alpha$或$a^\mu$依赖于坐标,由于导数的链式法则,会多出来一些项。
具体来说,我们有
\begin{equation}
    \mathcal{L}_\text{spin} \longrightarrow \mathcal{L}_\text{spin}' = \mathcal{L}_\text{spin} - \bar{\psi} \gamma_\mu \psi \partial^\mu \alpha.
\end{equation}
对于矢量场,在$a^\mu$的形式任意的情况下,$\mathcal{L}_\text{vec}$的变换无规律可循,但是如果我们用某个标量的梯度$\partial^\mu a$代替$a^\mu$,那么有
\[
    A^\mu \longrightarrow A'^\mu = A^\mu + \partial^\mu a, \quad
    \mathcal{L}_\text{vec} \longrightarrow \mathcal{L}_\text{vec}' = \mathcal{L}_\text{vec}.
\]
也就是说,矢量场的场的平移对称性实际上可以稍加推广而仍然成立。

$\psi$的变换的相位因子是一个标量;$A^\mu$的场的平移量也是一个标量的梯度。很容易想到的尝试是,我们是否可以将两个场耦合起来,并要求整个系统在局域变换($e$是常数而$a(\vb*{x})$依赖于坐标)
\begin{equation}
    \psi \longrightarrow \psi' = \psi \ee^{\ii e a}, \quad A^\mu \longrightarrow A'^\mu = A^\mu + \partial^\mu a
    \label{eq:gauge-transformation}
\end{equation}
下保持不变?%
\footnote{早期的物理学家会认为,一个变换应该是物理上可行的,因此它不应该是全局的,而\emph{只能}是局域的(例如我们可以让$a$在很小的范围内才不为零)。
但是实际上这种观点是错误的——使得我们想要从头写下一个拉氏量的原因实际上是我们希望从一个非常简洁的源头推导出麦克斯韦方程,但是后面会看到,在经典情况下描述了一切电磁现象的麦克斯韦方程本身在$U(1)$规范变换下不变,这意味着\eqref{eq:gauge-transformation}展示的对称性实际上是一种冗余,即系统中存在非物理、可以略去的自由度。
这些自由度不参与和实际观测值有关的任何相互作用,作用在它们上面的变换完全没有必要是局域的。
我们要求系统的动力学在局域$U(1)$变换下不变,归根到底还是满足实验观测结论的需要。
}%
自由矢量场部分肯定是不变的,那么就要适当设计相互作用项的形式,把$\psi$做局域$U(1)$变换之后拉氏量多出来的一项吸收掉。
当然,如果相互作用项是$- e A^\mu \bar{\psi} \gamma_\mu \psi$,那就正好,因为
\[
    - e A^\mu \bar{\psi} \gamma_\mu \psi - \bar{\psi} \psi \gamma_\mu \partial^\mu (e a) = - e A'^\mu \bar{\psi'} \gamma_\mu \psi'.
\]
于是我们得出结论:拉氏量
\begin{equation}
    \mathcal{L} = 
    \underbrace{- m \bar{\psi} \psi + \ii \bar{\psi} \gamma_\mu \partial^\mu \psi }_{\mathcal{L}_\text{spin}}
    \underbrace{- \frac{1}{2} (\partial^\mu A^\nu \partial_\mu A_\nu - \partial^\mu A^\nu \partial_\nu A_\mu)}_{\mathcal{L}_\text{vec}}
    \underbrace{- e A^\mu \bar{\psi} \gamma_\mu \psi}_\text{interaction}
    \label{eq:qed-lagrangian}
\end{equation}
具有局域$U(1)$不变性。推导出\eqref{eq:qed-lagrangian}的方法就是\concept{最小耦合}。%
\footnote{需要注意的是最小耦合实际上并不是唯一的能够让理论满足局域$U(1)$对称性的方案。理论中有哪些规范场、相互作用的形式如何,归根到底都需要实验上的提示。
例如,电磁理论中最小耦合适用是因为它能够导出麦克斯韦方程,而麦克斯韦方程是已经验证了的在经典情况下正确的电磁定律。}%

于是,我们得出结论:一个自旋$1/2$狄拉克旋量场和一个无质量矢量场耦合,并要求理论具有\emph{局域}$U(1)$对称性,那么就会得到
我们将会看到,这个理论实际上就是电动力学。

还可以引入一些记号来简化\eqref{eq:qed-lagrangian}。首先引入反对称张量\concept{电磁张量}
\begin{equation}
    F^{\mu \nu} = \partial^\mu A^\nu - \partial^\nu A^\mu,
\end{equation}
则自由矢量场拉氏量为
\[
    \mathcal{L}_\text{vec} = - \frac{1}{4} F_{\mu \nu} F^{\mu \nu}.
\]
另一方面,相互作用项和含有旋量场的导数的项形式非常接近,因此可以定义\concept{协变导数}
\begin{equation}
    \ii D^\mu = \ii \partial^\mu - e A^\mu,
\end{equation}
最后将\eqref{eq:qed-lagrangian}写成
\begin{equation}
    \begin{aligned}
        \mathcal{L} &= \bar{\psi} (\ii \gamma^\mu D_\mu - m) \psi - \frac{1}{4} F_{\mu \nu} F^{\mu \nu} \\
        &= \bar{\psi} (\ii \slashed{D} - m) \psi - \frac{1}{4} F_{\mu \nu} F^{\mu \nu}. 
    \end{aligned}
    \label{eq:short-qed-lagrangian}
\end{equation}
这里我们用斜杠记号表示$\gamma_\mu A^\mu$。

\subsection{运动方程和守恒量}\label{sec:four-eqs}

从\eqref{eq:qed-lagrangian}马上可以使用欧拉-拉格朗日方程写出运动方程。对$\psi$我们有
\[
    \ii \partial_\mu \bar{\psi} \gamma^\mu + e \bar{\psi} \gamma_\mu A^\mu + m \bar{\psi} = 0,
\]
对其取共轭,或者对$\bar{\psi}$应用欧拉-拉格朗日方程,就得到
\begin{equation}
    \ii \gamma^\mu \partial_\mu \psi - m \psi = e \gamma_\mu A^\mu \psi.
    \label{eq:movement-eq-1}
\end{equation}
对$A^\mu$应用欧拉-拉格朗日方程,则有
\begin{equation}
    \partial_\mu F^{\mu \nu} = \partial_\mu (\partial^\mu A^\nu - \partial^\nu A^\mu) = e \bar{\psi} \gamma^\nu \psi.
    \label{eq:movement-eq-2}
\end{equation}
以上两个方程给出了\eqref{eq:qed-lagrangian}的运动方程。(如前所述,$\psi$和$\bar{\psi}$虽然是独立的场,但它们的运动方程并不独立,因为运动方程是一阶的)

现在我们分析局域$U(1)$对称性带来的守恒量。在局域$U(1)$变换下,我们有
\[
    \var{\psi} = \ii e \psi \var{a}, \quad \var{A^\mu} = \partial^\mu \var{a},
\]
则守恒流为
\[
    \begin{aligned}
        J^\mu \var{a} &= - e \bar{\psi} \gamma^\mu \psi \var{a} + (-\partial^\mu A^\nu + \partial^\nu A^\mu) \partial_\nu \var{a} \\
        &= - e \bar{\psi} \gamma^\mu \psi \var{a} - \partial_\nu (\partial^\nu A^\mu - \partial^\mu A^\nu) \var{a},
    \end{aligned}
\]
第二个等号实际上是忽略了一个边界项后得到的结果。%
\footnote{考虑到$\var{a}$在每一点都可以独立地变化,$\int A\var{a} = \int B \var{a}$意味着$A=B$。}%
无论如何,这个守恒流的第二项是平凡的,因为它就是电磁张量的一个指标求散度之后的结果,它的散度当然是零。
那么,我们就有以下守恒荷:
\begin{equation}
    J^\mu = e \bar{\psi} \gamma^\mu \psi , \quad \partial_\mu J^\mu = 0.
    \label{eq:four-current}
\end{equation}
回过头看,实际上这是\emph{全局$U(1)$对称性}的守恒荷——全局$U(1)$对称性中$a$在时空上是均匀的,那么$\partial_\mu \var{a}$就是零,正好让含有$A$的那个平凡的项消失。
实际上从\eqref{eq:movement-eq-2}中我们也可以得到这个守恒流。由于电磁张量是反对称的,我们有:
\[
    0 = \partial_\mu \partial_\nu F^{\mu \nu} = \partial_\mu (e \bar{\psi} \gamma^\mu \psi).
\]
这就导出了\eqref{eq:four-current}。我们将会看到实际上这个是\concept{电荷}。或者说,电荷是全局$U(1)$对称性对应的守恒荷。

使用\eqref{eq:four-current}可以将\eqref{eq:movement-eq-2}写成
\begin{equation}
    \partial_\mu F^{\mu \nu} = J^\nu.
    \label{eq:four-maxwell}
\end{equation}
我们将会看到,这实际上就是麦克斯韦方程的一部分。构成麦克斯韦方程另外一部分的是以下恒等式
\begin{equation}
    \partial_\mu F_{\nu \rho} + \partial_\nu F_{\rho \mu} + \partial_\rho F_{\mu \nu} = 0,
    \label{eq:bianchi-identity}
\end{equation}
它是$F_{\mu \nu}$定义为$A^\mu$的梯度的反对称化导致的结果。

\subsection{规范}\label{sec:gauge-def}

电动力学在局域$U(1)$变换下的对称性实际上是一个\emph{规范对称性},也就是说,做任意的局域$U(1)$变换,不会有任何可以观察到的变化,也就是说\eqref{eq:qed-lagrangian}中实际上有多余的自由度。
我们需要对$A$和$\psi$施加适当的约束,以确保满足这个约束的$A$和$\psi$取值可以覆盖所有物理上可能产生的状态,同时不含有任何冗余的自由度,也即要\emph{选取一个规范}。
既然局域$U(1)$对称性是规范对称性,任何反映系统实际状态的物理量都应该在$U(1)$规范变换下不变。

形式最漂亮的应该是\concept{洛伦兹规范},也就是
\begin{equation}
    \partial_\mu A^\mu = 0,
\end{equation}
在这个规范下\eqref{eq:four-maxwell}转化为
\begin{equation}
    \Box^2 A = J, \quad \Box^2 = \partial_\mu \partial^\mu.
\end{equation}
我们得到了一个四维波动方程,当然,这就是\concept{电磁波}。
一个很自然的问题是,洛伦兹规范是否不失一般性?是否存在一组$A^\mu$不能够通过一个规范变换变换为一组满足洛伦兹规范的$A'^\mu$?
实际上洛伦兹规范确实是不失一般性的,因为波动方程的性质很良好,对一个给定的标量场$C$,总是可以找到一个标量场$a$,使得
\[
    \partial_\mu \partial^\mu a = C,
\]
这样不论原本$A^\mu$取什么值,只需要解出一个$a^\mu$使得
\[
    \partial_\mu \partial^\mu a = \partial_\mu A^\mu,
\]
然后做规范变换
\[
    A'^\mu = A^\mu - \partial^\mu a, \quad \psi' = \psi \ee^{-\ii e a},
\]
得到的$A'^\mu$就是服从洛伦兹规范的——并且表示和$A^\mu$完全一样的物理状态。
因此洛伦兹规范确实是不失一般性的。

容易看出$F^{\mu \nu}$是一个规范不变量。实际上,在选定了规范之后,可以从它恢复出$A^\mu$。
不失一般性地选择洛伦兹规范,则我们有
\[
    \partial_\mu F^{\mu \nu} = \partial_\mu \partial^\mu A^\nu - \partial^\nu \partial_\mu A^\mu = \partial_\mu \partial^\mu A^\nu,
\]
由于波动方程的良好性质,我们就从上式反解出$A^\mu$了。
如果是别的规范,就按照它转换到洛伦兹规范的方式,从洛伦兹规范转换到原有规范即可。
总之,原则上任何规范不变量都可以通过$F^{\mu \nu}$求导、积分等得到。

\section{麦克斯韦方程}

如前所述,\eqref{eq:four-maxwell}实际上是麦克斯韦方程。本节将要展示它如何能够化成在实际计算中更加常用的形式。
本节的讨论全部在经典电动力学下进行,先不考虑量子化。

\subsection{基于电场和磁场的电动力学}

\subsubsection{电场和磁场以及麦克斯韦方程}

我们可以直接选择将电磁张量的分量写下来。我们首先观察电磁张量在\emph{空间}坐标变换(不涉及时间)下的变换。
首先电磁张量的对角元都是零,非对角元一共有$6$个独立变量,我们不妨设它们为
\[
    F^{\mu \nu} = \pmqty{
        0 & -E_x & -E_y & -E_z \\
        E_x & 0 & -B_z & B_y \\
        E_y & B_z & 0 & -B_x \\
        E_z & -B_y & B_x & 0
    }.
\]
一个不涉及时间的坐标变换一定形如
\[
    \pmqty{\dmat{1, \xmat*{a}{3}{3} }},
\]
我们将它作用在$F^{\mu \nu}$上(注意二阶张量的两个指标都要作用一遍),会发现$(E_x, E_y, E_z)$在坐标变换矩阵$\{a_{ij}\}$下发生了坐标变换,而
\[
    \pmqty{
        0 & -B_z & B_y \\
        B_z & 0 & -B_x \\
        -B_y & B_x & 0
    }
\]
在坐标变换矩阵$\{a_{ij}\}$下作为张量发生了坐标变换。
这就意味着$(E_x, E_y, E_z)$构成一个三维矢量的分量,而含有$B_i$的那部分矩阵构成一个三维二阶张量的分量。我们称前者为\concept{电场},它是一个极矢量。
至于后者,它是某个三元组和一个三指标反对称张量缩并而成的,反对称张量在连续的坐标变换下确实按照张量的方式变换,但是在反射下会变号,因此三元组$(B_x, B_y, B_z)$构成一个轴矢量的分量,称为\textbf{磁场}。

电场和磁场满足什么样的动力学方程?首先我们考虑\eqref{eq:four-maxwell},它给出两个方程(注意每一项的正负号,特别是梯度算符;$\partial_i$和$\partial^i$都是\emph{欧氏空间下的},简单地表示对$x_i$(也就是$x^i$)求偏导):
\[
    \div{\vb*{E}} = J^0, \quad - \partial_t E^i - \partial_j \epsilon_{jik} B^k = J^i.
\]
设$J^\mu$对应的守恒荷密度为$\rho$,对应的输运流为$\vb*{j}$,即
\begin{equation}
    J^\mu = (\rho, \vb*{j}), \quad \partial_\mu J^\mu = \pdv{\rho}{t} + \div{\vb*{j}} = 0,
\end{equation}
则这两个方程就是
\begin{equation}
    \div{\vb*{E}} = \rho, \quad \curl{\vb*{B}} = \pdv{\vb*{E}}{t} + \vb*{j}.
    \label{eq:maxwell-first-pair}
\end{equation}

光靠\eqref{eq:maxwell-first-pair}显然不能定解,其原因在于$\vb*{E}$和$\vb*{B}$依照定义是$F^{\mu \nu}$的独立分量,但$F^{\mu \nu}$并不是一个任意的反对称张量,它是矢量场$A^\mu$的梯度反对称化之后的产物。
例如,$F^{\mu \nu}$还需要满足\eqref{eq:bianchi-identity}。
分别考虑\eqref{eq:bianchi-identity}中$\mu, \nu, \rho$完全取空间维度的情况以及三个指标有一个取时间维度另外两个取空间维度的情况,得到
\begin{equation}
    \div{\vb*{B}} = 0, \quad \pdv{\vb*{B}}{t} + \curl{\vb*{E}} = 0.
    \label{eq:maxwell-second-pair}
\end{equation}
\eqref{eq:maxwell-first-pair}和\eqref{eq:maxwell-second-pair}放在一起就给出了著名的\concept{麦克斯韦方程组}:有电场、有磁场,有\concept{电荷密度}$\rho$,有\concept{电流密度}$\vb*{j}$。
整个麦克斯韦方程写出来就是
\begin{equation}
    \left\{
        \begin{aligned}
            \div{\vb*{E}} &= \rho, \\
            \curl{\vb*{E}} &= - \pdv{\vb*{B}}{t}, \\
            \div{\vb*{B}} &= 0, \\
            \curl{\vb*{B}} &= \pdv{\vb*{E}}{t} + \vb*{j}.
        \end{aligned}
    \right.
    \label{eq:maxwell-eq}
\end{equation}
可以看到,\eqref{eq:maxwell-eq}在时间上是一阶的,且电场和磁场的一阶导数都已经确定了,则给定适当的初始条件和边界条件应当能够定解。

在推导\eqref{eq:maxwell-eq}时我们还用到了一个条件,就是$J^\mu$是一个四维守恒流。但实际上这个条件也可以从\eqref{eq:maxwell-eq}中推导出来,只需要分别对第一式做对时间的偏导数,对第四式做散度即可得到连续性条件
\begin{equation}
    \pdv{\rho}{t} + \div{\vb*{j}} = 0.
\end{equation}
这当然完全是预期之中的,因为正如我们在\prettyref{sec:four-eqs}中看到的那样,电荷守恒可以从$A^\mu$的运动方程推导出来,而既然麦克斯韦方程完全描述了电磁场,自然也可以推导出电荷连续性方程。%
\footnote{一个可能的疑难是,\eqref{eq:four-current}要求$(\rho, \vb*{j})$构成四维矢量的分量,而以上推导并未展示出这一点。
但注意到$(\partial_t, \grad)$是四维梯度算符,而$0$是标量,那么坐标变化时必须保证$(\rho, \vb*{j})$是四维矢量才能够让连续性方程恒成立,因此$(\rho, \vb*{j})$确确实实构成四维矢量的分量。
\prettyref{note:vector-component}也提到了类似的论证。}%

求解\eqref{eq:maxwell-eq}得到的只是$\vb*{E}$和$\vb*{B}$,或者说电磁张量,但正如\prettyref{sec:gauge-def}中所说的那样,知道了电磁张量,就可以确定所有规范不变量,那么求解出$\vb*{E}$和$\vb*{B}$也就够了。
需要注意的是这\emph{不代表}电场和磁场就是本质上更基本的自由度,例如在一些情况下(如A-B效应等)直接使用$\vb*{A}$构造规范不变量反而更方便,而且要使用$\vb*{A}$构造一些非局域的量。

总之,\eqref{eq:maxwell-eq}是一切经典电动力学现象关于电磁场的部分背后的机制%
\footnote{但是还需要关于电磁场以外的物质的定律。见\prettyref{sec:completeness}。}%
;\eqref{eq:qed-lagrangian}能够推导出\eqref{eq:maxwell-eq},说明这个拉氏量描写的确实是电动力学。
基于电场和磁场的表述和基于电磁张量的表述之间的转换关系就是
\begin{equation}
    F^{\mu \nu} = \pmqty{
        0 & -E_x & -E_y & -E_z \\
        E_x & 0 & -B_z & B_y \\
        E_y & B_z & 0 & -B_x \\
        E_z & -B_y & B_x & 0
    }, \quad F_{\mu \nu} = \pmqty{
        0 & E_x & E_y & E_z \\
        -E_x & 0 & -B_z & B_y \\
        -E_y & B_z & 0 & -B_x \\
        -E_z & -B_y & B_x & 0
    }.
\end{equation}

\subsubsection{反推出电势和磁矢势}

实际上,我们也可以从\eqref{eq:maxwell-eq}出发,重构出一个关于$A^\mu$的理论。
\eqref{eq:maxwell-eq}的第三式告诉我们,存在一个矢量场$\vb*{A}$使得
\[
    \vb*{B} = \curl{\vb*{A}}.
\]
上式代入\eqref{eq:maxwell-eq}第二式,得到
\[
    \curl{\vb*{E} + \pdv{\vb*{A}}{t}} = 0,
\]
于是存在标量场$\varphi$使得
\[
    \vb*{E} = - \pdv{\vb*{A}}{t} - \grad{\varphi}.
\]
于是我们可以用$(\varphi, \vb*{A})$完全将电场和磁场表示出来,具体说就是
\begin{equation}
    \vb*{B} = \curl{\vb*{A}}, \quad \vb*{E} = - \pdv{\vb*{A}}{t} - \grad{\varphi}.
    \label{eq:e-b-from-a-phi}
\end{equation}
当然,$\varphi$就是我们熟悉的\concept{电势}而$\vb*{A}$就是\concept{磁矢势}。
这样一来\eqref{eq:maxwell-eq}的第一、四式就是
\[
    - \pdv{t} \div{\vb*{A}} - \laplacian \varphi = \rho, \quad \pdv[2]{\vb*{A}}{t} - \laplacian \vb*{A} + \grad(\div{\vb*{A}}) + \pdv{t} \grad{\varphi} = \vb*{j}.
\]
这两个式子看起来毫无规律,但是如果我们假定$(\rho, \vb*{A})$是某个四维矢量$A^\mu$的分量%
\footnote{这里可能会有一个问题:为什么我们确定$(\rho, \vb*{A})$能够构成四维矢量的分量?实际上,正确的思路是先设$(\rho, \vb*{A})$能够构成某个四分量对象的分量,推导出\eqref{eq:movement-eq-2},然后注意到\eqref{eq:movement-eq-2}如果总是成立,那么$A^\mu$一定要是矢量分量。
\label{note:vector-component}}%
,那么立刻可以发现这两个式子就是\eqref{eq:movement-eq-2}。
因此实际上麦克斯韦方程和\eqref{eq:movement-eq-2}是等价的。
如果使用洛伦兹规范,还可以将以上两个方程写得形式漂亮一些,也就是
\begin{equation}
    \pdv[2]{\varphi}{t} - \laplacian \varphi = \rho, \quad \pdv[2]{\vb*{A}}{t} - \laplacian \vb*{A} = \vb*{j}.
    \label{eq:wave-eq}
\end{equation}
回顾从拉氏量推导\eqref{eq:movement-eq-2}的过程,我们会发现它只用到了\eqref{eq:vec-lagrangian}以及$A^\mu$和$\psi$的耦合项中$A^\mu$是线性的这一事实,因此我们得出结论:\eqref{eq:maxwell-eq}描述了四维闵可夫斯基时空中一个单一无质量矢量场受到线性策动后的动力学。

\subsubsection{单位制}\label{sec:unit-system}

到现在为止我们的方程都是\concept{自然单位制}的。实际计算时会使用一些不同的单位制,会让方程变得复杂一些。


\subsection{定解条件}

\subsubsection{完备性}\label{sec:completeness}

前面提到麦克斯韦方程描述了“电磁场”部分的动力学,言下之意是还有别的部分的动力学——的确还有,因为麦克斯韦方程仅仅覆盖了\eqref{eq:movement-eq-2}而还有\eqref{eq:movement-eq-1}。
$\psi$场导致了四维电流密度$J^\mu$,它的物理意义就是携带电荷的粒子,因此我们称$\psi$为\concept{实物场}。
当然电磁辐射也是一种物质,不过我们还是不要玩弄概念了。

直接从\eqref{eq:short-qed-lagrangian}出发做计算,也就是真的去使用实物场做计算是非常复杂的,我们把有关的讨论留到\prettyref{sec:qed},而在使用麦克斯韦方程时仅仅唯象地引入一个“实物在电磁力作用下的运动方程”。
这是必要的,因为麦克斯韦方程\eqref{eq:maxwell-eq}并不封闭,我们并不知道$\rho$和$\vb*{j}$在电磁场作用下会怎么变动。
在本节中我们先假定$\rho$和$\vb*{j}$已经给定,那么问题就是需要哪些方程可以定解。
当然肯定需要初始条件和边界条件,我们将在\prettyref{sec:boundary}中讨论边界条件,而初始条件则是视具体情况而定的。
那么问题就是需要多少泛定方程。

麦克斯韦方程肯定是够用的,而\eqref{eq:movement-eq-2}显然也是够用的。因此我们有以下求解方案:

\begin{enumerate}
    \item 给定边界条件和初始条件;
    \item 暂时将$\rho$和$\vb*{j}$看成已经给定的量;
    \item 依照以下三种方案中的其中一种前进,这三种方案都是等价的,提供了同样多的信息,在一个方案中增加别的方案的方程不会提供新的信息:
    \begin{itemize}
        \item 将$(\rho, \vb*{j})$代入麦克斯韦方程\eqref{eq:maxwell-eq}求解,解出$\vb*{E}, \vb*{B}$;
        \item 选定一个规范\footnote{可以看到,规范实际上就是让我们能够从电场、磁场唯一反推出电势和磁矢势的约束条件。},求解出$(\rho, \vb*{A})$,如在洛伦兹规范下求解\eqref{eq:wave-eq},然后根据\eqref{eq:e-b-from-a-phi}计算出$\vb*{E}, \vb*{B}$;
        \item 直接求解\eqref{eq:four-maxwell},或是使用最小作用量原理等,然后根据\eqref{eq:e-b-from-a-phi}计算出$\vb*{E}, \vb*{B}$;
    \end{itemize}
    \item 如果必要的话,根据\prettyref{sec:gauge-def}中提到的办法计算$A^\mu$,即$(\varphi, \vb*{A})$;
    \item 如果实际上$(\rho, \vb*{j})$没有给定,则将用$\rho, \vb*{j}$表示的$\vb*{E}, \vb*{B}$和实物在电磁力下的运动方程联立求解。
\end{enumerate}

\subsubsection{边界条件}\label{sec:boundary}

\subsubsection{规范}

规范变换
\[
    A'^\mu = A^\mu + a^\mu
\]
使用$(\varphi, \vb*{A})$,可以写成
\begin{equation}
    \varphi' = \varphi + \pdv{\chi}{t}, \quad \vb*{A}' = \vb*{A} - \grad{\chi}.    
    \label{eq:vector-gauge}
\end{equation}
这里我们使用风格和麦克斯韦方程更加一致的$\chi$来代替$a$。
实际上,仅仅知道麦克斯韦方程也可以推导出\eqref{eq:vector-gauge}。我们从麦克斯韦方程知道,电场和磁场可以写成\eqref{eq:e-b-from-a-phi}的形式,那么要让$\vb*{A}$和$\vb*{A}'$对应同一个$\vb*{B}$就是要
\[
    \curl{\vb*{A}} = \curl{\vb*{A}'},
\]
从而
\[
    \vb*{A}' = \vb*{A} - \grad{\chi},
\]
而要让$(\varphi, \vb*{A})$和$(\varphi', \vb*{A}')$对应同一个$\vb*{E}$就是要
\[
    \grad(\varphi' - \pdv{\chi}{t} - \varphi) = 0,
\]
从而
\[
    \varphi' - \pdv{\chi}{t} - \varphi = \const.
\]
重新调整$\varphi'$的值就可以得到\eqref{eq:vector-gauge}。

如前所述,一种形式优美的规范是洛伦兹规范,写成$\varphi$和$\vb*{A}$的形式就是
\begin{equation}
    \pdv{\varphi}{t} + \div{\vb*{A}} = 0.
\end{equation}
当然还可以取别的一些规范。一种常见的规范是\concept{库伦规范},即
\begin{equation}
    \div{\vb*{A}} = 0.
\end{equation}
这相当于把大部分工作都转移给了$\varphi$,从而让待求解的问题看起来很像一个静电学问题(这就是“库伦”这个名字的来历)。
很容易检验任何一组$(\varphi, \vb*{A})$都可以经过一个规范变换而满足库伦规范,只需要让
\[
    \laplacian \chi = \div{\vb*{A}}
\]
即可,而这个方程总是有解。
另一方面,库伦规范也足够强,从\eqref{eq:e-b-from-a-phi}可以得知
\begin{equation}
    \laplacian \varphi = - \div{\vb*{E}} = - \rho, \quad \laplacian \vb*{A} = - \curl{\vb*{B}},
\end{equation}
即可以唯一确定电势和磁矢势。因此库伦规范不太强也不太弱,像洛伦兹规范一样正好。
另一种规范是\concept{辐射规范},与库伦规范正好相反,是尽可能除去理论中的“静电学”味道,即取
\begin{equation}
    \varphi = 0.
\end{equation}
同样,对任意的$(\varphi, \vb*{A})$,取
\[
    \pdv{\chi}{t} = - \varphi
\]
即可让规范变换后的$(\varphi', \vb*{A}')$服从辐射规范。

\subsection{能量和动量}

\section{量子电动力学}\label{sec:qed}

\subsection{非相对论极限}

光子本身没有质量,因此无所谓“电磁场的非相对论极限”,但是实物粒子可以以非常慢的动量运动。
因此有必要讨论电磁场和非相对论性实物场耦合的理论。

\end{document}