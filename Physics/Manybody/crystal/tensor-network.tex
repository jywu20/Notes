\begin{back}{格点模型的一种数值计算方法——张量网络}{tensor-network}
    文献\cite{Orus_tensor}给出了对常用的张量网络的一些介绍。
    所谓\concept{张量网络}是指通过缩并一些张量得到的格点系统波函数(称为\concept{张量网络态})的试探形式。
    使用张量网络方法能够自然地引入对称性,能够很容易地从最终计算结果中获得诸如纠缠熵等信息,能够用于分析各种边界条件的系统,处理无穷系统比较方便,图形化语言意义清晰等等(例如对局域的哈密顿量,相邻格点的纠缠总是比较大,通过分析张量网络态中的纠缠我们能够确定一个模型的基态的“有效格点结构”)。
    相较于量子蒙卡方法,张量网络方法的不足之处则在于难以估计计算质量好坏。

    需要保证使用的张量网络态真的是足够好的拟设——选择一个张量网络波函数拟设本质上是在将系统的希尔伯特空间按照纠缠程度做分类,选择一个张量网络拟设等价于选择纠缠程度适当的一个子空间,如对有能隙系统,低能能量本征态的一个空间部分的纠缠熵正比于该空间部分的表面积(所谓\concept{area law}),因此我们应该选择一个服从area law的张量网络拟设。
    此外,高效地实施张量网络计算需要适当安排缩并顺序。设张量网络图中每条线代表的指标的取值范围大致在1到$d$(这个$d$称为这条边的维数,但是这个维数和晶格的维数、每个格点上的标签的取值数目(所谓“本地维数”)这三者之间没有必然联系),则计算一个有$n_1$条外线,$n_2$条内线的张量网络缩并的时间复杂度大致在$\bigO{d^{n_1+n_2}}$。
    例如,普通的矩阵乘法$[A_{ij} B_{jk}]_{ij}$涉及三条线——$i, j, k$——因此其时间复杂度为$\bigO(d^3)$。
    适当地安排缩并顺序可以大幅降低计算整体的时间复杂度。

    可能最有名的张量网络方法就是\concept{密度矩阵重整化群(DMRG)}了。这是一种主要用于分析一维格点系统的张量网络方法,其基态波函数拟设为\concept{矩阵乘积态(MPS)},绘制为\autoref{fig:mps-state}。
    要施加开放边界条件只需移除左右两条外线,要施加周期性边界条件只需要将左右两个外线连接起来。
    在\cite{Orus_tensor}中介绍了MPS的一些性质:可以做到平移不变(只需让每个蓝点指向同一个张量即可),稠密(只要$d$足够大原则上可以构造出任意的一维格子上的波函数),纠缠熵服从area law,关联长度受控制等等。
\end{back}

\begin{figure}
    \centering
    \input{math/tn/mps.tex}
    \caption{MPS拟设,根据不同的边界条件可以调整最左边和最右边两个外线;向下的线连接每个格点上的基矢量}
    \label{fig:mps-state}
\end{figure}