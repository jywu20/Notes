\documentclass[hyperref, UTF8, a4paper]{ctexart}

\usepackage{geometry}
\usepackage{titling}
\usepackage{titlesec}
\usepackage{paralist}
\usepackage{footnote}
\usepackage{enumerate}
\usepackage{amsmath, amssymb, amsthm}
\usepackage{bbm}
\usepackage{cite}
\usepackage{graphicx}
\usepackage{subfigure}
\usepackage{physics}
\usepackage{siunitx}
\usepackage{tikz}
\usepackage{tikz-feynhand}
\usepackage[colorlinks, linkcolor=black, anchorcolor=black, citecolor=black]{hyperref}
\usepackage{prettyref}

\geometry{left=3.18cm,right=3.18cm,top=2.54cm,bottom=2.54cm}
\titlespacing{\paragraph}{0pt}{1pt}{10pt}[20pt]
\setlength{\droptitle}{-5em}
\preauthor{\vspace{-10pt}\begin{center}}
\postauthor{\par\end{center}}

\DeclareMathOperator{\timeorder}{T}
\DeclareMathOperator{\diag}{diag}
\DeclareMathOperator{\legpoly}{P}
\DeclareMathOperator{\primevalue}{P}
\DeclareMathOperator{\sgn}{sgn}
\newcommand*{\ii}{\mathrm{i}}
\newcommand*{\ee}{\mathrm{e}}
\newcommand*{\const}{\mathrm{const}}
\newcommand*{\comment}{\paragraph{注记}}
\newcommand*{\suchthat}{\quad \text{s.t.} \quad}
\newcommand*{\argmin}{\arg\min}
\newcommand*{\argmax}{\arg\max}
\newcommand*{\normalorder}[1]{: #1 :}
\newcommand*{\pair}[1]{\langle #1 \rangle}
\newcommand*{\fd}[1]{\mathcal{D} #1}

\newrefformat{sec}{第\ref{#1}节}
\newrefformat{note}{注\ref{#1}}
\newrefformat{fig}{图\ref{#1}}
\renewcommand{\autoref}{\prettyref}

\usetikzlibrary{arrows,shapes,positioning}
\usetikzlibrary{arrows.meta}
\usetikzlibrary{decorations.markings}
\tikzstyle arrowstyle=[scale=1]
\tikzstyle directed=[postaction={decorate,decoration={markings,
    mark=at position .5 with {\arrow[arrowstyle]{stealth}}}}]
\tikzstyle ray=[directed, thick]
\tikzstyle dot=[anchor=base,fill,circle,inner sep=1pt]

\renewcommand{\emph}[1]{\textbf{#1}}
\newcommand*{\concept}[1]{\underline{\textbf{#1}}}

\title{固体}
\author{吴晋渊}

\begin{document}

\maketitle

记号约定:费米子的产生湮灭算符为${c}^\dagger$和${c}$,而如果是关于位置的产生湮灭算符,则为${\psi}^\dagger$和${\psi}$。
本文提到的费米子主要是电子,使用一个三维坐标$\vb*{r}$(或者三位动量$\vb*{p}$),以及只有向上和向下两种选择的自旋就可以描述一个电子。
单电子自旋算符为
\[
    {\vb*{S}} = \sum_{\alpha, \beta} \ket{\alpha} \vb*{\sigma}_{\alpha \beta} \bra{\beta},
\]
其中$\alpha$和$\beta$取遍$\uparrow$和$\downarrow$,$\sigma$为泡利矩阵。

由于本文不涉及相对论性过程,设$\vb*{a}$为一个矢量,则使用$a$表示其模长。

本文取普朗克单位制,即认为$\hbar=c=1$,且$4\pi\epsilon_0=1$,$k_B=1$。

$\text{h.c.}$表示厄米共轭,$\text{c.c.}$表示复共轭。

对离散格点系统,使用$\pair{i, j}$表示最接近的一对格点。(只求和一次,即认为$\pair{i, j}$和$\pair{j, i}$相同)

在不涉及自旋-轨道耦合的场合,在书写哈密顿量时我们直接略去自旋的下标,这是合理的,因为只需要把不考虑自旋的哈密顿量中的各个产生湮灭算符根据自旋守恒的性质机械地加上自旋下标再求和就能够得到完整的哈密顿量。
在需要实际计算粒子数时就不能这么做了;需要计算总能量时当然也不能这么做。

若无特殊说明,$f(z)$定义为近独立费米子的分布函数,即
\[
    f(z) = \frac{1}{\ee^{\beta z} + 1}.
\]

\chapter{凝聚态系统的构成}

普通的固体、液体、气体由一系列原子组成。通过实验和计算可以发现,原子的最外层电子在各种过程中容易发生重新排列,称为\concept{价电子};内层电子和原子核(合称为\concept{离子实})则通常保持为一个整体,也即,其内部状态发生变化的物理过程的描述需要使用QCD,其涉及的能标远高于价电子发生变化涉及的能标。

本文基本上只分析涉及价电子低能运动的物理过程,即只讨论非相对论极限下的电荷-电磁场耦合系统,而忽略强相互作用、弱相互作用和引力。
此时,带电粒子由薛定谔场完全描述,系统具有$U(1)$规范对称性且无粒子数生灭,有确定的粒子数,于是可以原则上实物粒子部分可以直接用单粒子量子力学描述。
进一步,我们假定系统中没有变化特别快的电磁场,这意味着实际上我们可以积掉电磁场并且得到一个延迟不明显的相互作用,那就是说,我们可以把所有电磁相互作用都用静电学和静磁学处理。由于无论是电子还是离子实都是非相对论性的,电子-电子相互作用、电子-离子实相互作用、离子实-离子实相互作用几乎完全是库伦相互作用。

\section{电子}

\subsection{电子动能和库伦排斥能}

设有$N_e$个价电子,$N_i$个离子实(i表示离子),固体的不考虑外界扰动的一次量子化哈密顿量为
\begin{equation}
    {H} = {H}_\text{e} + {H}_\text{i} + {H}_\text{ei},
    \label{eq:many-body-hamiltonian}
\end{equation}
其中${H}_\text{e}$表示仅涉及价电子的哈密顿量,${H}_\text{i}$表示仅涉及离子实的哈密顿量,最后一项则是两者的相互作用,所有相互作用是库仑相互作用。
诸价电子组成的系统就好像由电子组成的气体,称为\concept{相互作用电子气}。单体哈密顿量为电子的动能项加上单体势能项。在物质不受外界作用时当然不应该有单体势能项,于是
\[
    {H}_\text{e1} = \frac{{\vb*{p}}^2}{2m},
\]
在坐标表象下它就是
\[
    {H}_\text{e1} = - \frac{\laplacian}{2m}.
\]
二体哈密顿量为电子两两作用而产生的库伦势能是
\[
    {H}_\text{e2} = \frac{e^2}{\abs{\vb*{r}_1 - \vb*{r}_2}},
\]
从而价电子本身的能量以及它们之间发生库伦相互作用的能量就是
\begin{equation}
    {H}_\text{e} = \sum_{i=1}^{N_\text{e}} \frac{{p}_i^2}{2m_\text{e}} + \frac{1}{2} \sum_{i\neq j} \frac{e^2}{\abs{\vb*{r}_i - \vb*{r}_j}}.
\end{equation}
使用类似的方法,离子实的组成的系统(如果是晶体那就是晶格)的哈密顿量为
\begin{equation}
    {H}_\text{i} = \sum_{\alpha=1}^{N_\text{i}} \frac{{p}_\alpha^2}{2m_i} + \frac{1}{2} \sum_{\alpha\neq\beta} V(\vb*{R}_\alpha-\vb*{R}_\beta).
\end{equation}
由于离子实中的内层电子结构复杂,离子实之间的相互作用能写不出特别简单的表达式(我们相当于把内层电子的自由度也积掉了)。请注意这个相互作用能是平移不变的,这是当然的,因为QED是平移不变的;但是实际的固体在短距离上并不是平移不变的,因为在低能下有对称性自发破缺。
离子实和价电子的相互作用则是
\begin{equation}
    {H}_\text{ei} = \sum_{\alpha, i} V_\text{ei}(\vb*{r}_i-\vb*{R}_\alpha). 
\end{equation}
分别使用$i$表示价电子,用$\alpha$表示离子实;由于价电子和离子实不全同,不需要加上$1/2$系数。
同样我们还是假定了相互作用本身的平移不变性。
本文仅仅讨论固体(实际上主要是晶体)的物理,因此我们将假定离子实的位移始终局限在非常小的范围内。

在大部分过程中,由于原子核的质量比电子的质量大至少三个数量级,涉及价电子的过程通常比涉及离子实的过程发生得快很多,从而在价电子的时间尺度上,诸离子实的位置可以看成是给定的。
从而,在分析价电子时我们可以将${H}_i$项直接略去,并忽略离子实位置的量子涨落,而将${H}_\text{ei}(\vb*{r}_i-\vb*{R}_\alpha)$项对$\vb*{R}_\alpha$求和得到$V_\text{ext}(\vb*{r}_i)$(使用ext为下标是因为离子实相当于是给相互作用电子气施加了一个外场)。这个近似称为\concept{玻恩–奥本海默近似}。这样一来相互作用电子气的一次量子化哈密顿量在坐标表象下就是
\begin{equation}
    {H} = \sum_{i=1}^{N_\text{e}} \left( - \frac{\laplacian}{2m_\text{e}} + V_\text{ext}(\vb*{r}_i)\right) + \frac{1}{2} \sum_{i\neq j} \frac{e^2}{\abs{\vb*{r}_i - \vb*{r}_j}},
    \label{eq:electron-gas-hamiltonian}
\end{equation}
从而二次量子化哈密顿量为
\begin{equation}
    \begin{aligned}
        {H} = &\sum_{\sigma} \int \dd[3]{\vb*{r}} {\psi}_\sigma^\dagger(\vb*{r}) \left( - \frac{\laplacian}{2m} + V_\text{ext}(\vb*{r}) \right) {\psi}_\sigma(\vb*{r}) \\
        &+ \frac{1}{2} \sum_{\alpha, \beta} \int \dd[3]{\vb*{r}_1} \int \dd[3]{\vb*{r}_2} 
        {\psi}_\alpha^\dagger (\vb*{r}_1) {\psi}_\beta^\dagger (\vb*{r}_2) \frac{e^2}{\abs{\vb*{r}_1 - \vb*{r}_2}} {\psi}_\beta (\vb*{r}_2) {\psi}_\alpha (\vb*{r}_1). 
    \end{aligned}
    \label{eq:electron-gas-hamiltonian-sq}
\end{equation}
其中${\psi}^\dagger(\vb*{r})$是薛定谔场的场算符,它也是在位置为$\vb*{r}$的位置产生一个电子的产生算符。这个哈密顿量当然也可以通过QED的低能近似得到,但并没有必要这么做。请注意电子是费米子。
\eqref{eq:electron-gas-hamiltonian-sq}实际上不是对角的,因为它的单粒子项涉及一个梯度算符。
另一方面,晶格中离子实的相互作用可以认为遵从由电子总能量确定的等效势。%
\footnote{
    注意,先计算电子-电子库伦散射导致的能量然后用它计算晶格中离子实的相互作用最后计算离子实振动对电子的影响,这个过程中\emph{没有}双重计数,虽然我们表面上似乎“积掉了电子而计算等效晶格相互作用”。
}%
由于离子实通常比较大,我们很多时候会彻底忽略离子实位置的量子涨落而使用分子动力学方法处理它。

最后我们指出,由于固体始终可以和外界交换电子——外界的电子可以进入固体,固体中的电子可以溢出——固体中电子气本身的哈密顿量\eqref{eq:electron-gas-hamiltonian-sq}不足以充分描述系统。
本文将只研究近平衡系统,因此这种与外界的相互作用可以使用化学势描述,即我们需要在\eqref{eq:electron-gas-hamiltonian-sq}中加入一项$-\mu {\psi}^\dagger(\vb*{r}) {\psi}(\vb*{r})$,这样得到的哈密顿量才是完整的。
换而言之,电子气完整的哈密顿量形如
\begin{equation}
    \begin{aligned}
        {H} &= \sum_{\vb*{k}, \sigma} \Big( \underbrace{\frac{\vb*{k}^2}{2m}}_{\epsilon_{\vb*{k}}} - \mu \Big) {c}^\dagger_{\vb*{k} \sigma} {c}_{\vb*{k} \sigma} 
        + \int \dd[3]{\vb*{r}} V_\text{ext}(\vb*{r}) {\psi}^\dagger(\vb*{r}) {\psi}(\vb*{r}) \\ 
        &+ \frac{1}{2} \sum_{\alpha, \beta} \int \dd[3]{\vb*{r}_1} \int \dd[3]{\vb*{r}_2} 
        {\psi}_\alpha^\dagger (\vb*{r}_1) {\psi}_\beta^\dagger (\vb*{r}_2) \frac{e^2}{\abs{\vb*{r}_1 - \vb*{r}_2}} {\psi}_\beta (\vb*{r}_2) {\psi}_\alpha (\vb*{r}_1).
    \end{aligned}
    \label{eq:full-electron-gas-hamiltonian}
\end{equation}
在有些模型中有时也将外势场并入$\epsilon_{\vb*{k}}$项。
为了简便起见,通常用$\xi$表示扣除了化学势的单电子能量,即
\begin{equation}
    \xi_{\vb*{k}} = \epsilon_{\vb*{k}} - \mu.
\end{equation}

\subsection{哈密顿量的一般形式}

以上我们都是将库伦相互作用加入薛定谔场中,可以说是给出了第一性原理计算需要的哈密顿量(虽然实际上从高能物理的角度这远非第一性原理,但对凝聚态理论来说通常已经够用了)。
但实际上还有以下机制没有考虑:
\begin{itemize}
    \item 作用在单体上的外场的束缚,离子实的束缚已经被计入考虑了,但是还有其它外场,比如说或许会有一个磁场,然后哈密顿量中将会有一项$-\vb*{\mu} \cdot \vb*{B}$;无论如何这会是一个二体算符。
    \item 电子-声子相互作用会引入一个二电子和一个声子发生相互作用的顶角,积掉声子自由度之后会留下一个等效的电子-电子相互作用,这会让\eqref{eq:full-electron-gas-hamiltonian}中电子-电子相互作用的系数发生变化,不再是严格的库伦排斥。
\end{itemize}
于是我们写出一般形式的相互作用电子气的二次量子化哈密顿量:
\begin{equation}
    {H} = \sum_{\vb*{k}, \sigma} (T_{\vb*{k}} - \mu) {c}^\dagger_{\vb*{k} \sigma} {c}_{\vb*{k} \sigma} 
    + \sum_{\vb*{k}_1, \vb*{k}_2, \sigma} V^\text{ext}_{\vb*{k}_1 \vb*{k}_2 \sigma} {c}^\dagger_{\vb*{k}_1 \sigma} {c}_{\vb*{k}_2 \sigma}
    + \sum_{\vb*{k}_1, \vb*{k}_2, \vb*{q}, \alpha, \beta} {c}^\dagger_{\vb*{k}_1+\vb*{q}, \alpha} {c}^\dagger_{\vb*{k}_2-\vb*{q}, \beta} V_{\vb*{q}} {c}_{\vb*{k}_2 \beta} {c}_{\vb*{k}_1 \alpha}. 
\end{equation}
相应的,热力学作用量为
\begin{equation}
    S = \sum_n \left( 
        \sum_{\vb*{k}, \sigma} (-\ii \omega_n + T_{\vb*{k}} - \mu) \bar{c}_{\vb*{k} \sigma} c_{\vb*{k} \sigma} 
        + \sum_{\vb*{k}_1, \vb*{k}_2, \sigma} V^\text{ext}_{\vb*{k}_1 \vb*{k}_2 \sigma} \bar{c}_{\vb*{k}_1 \sigma} c_{\vb*{k}_2 \sigma} 
        + \sum_{\vb*{k}_1, \vb*{k}_2, \vb*{q}, \sigma} \bar{c}_{\vb*{k}_1+\vb*{q}, \sigma} \bar{c}_{\vb*{k}_2-\vb*{q}, \sigma} V_{\vb*{q}} c_{\vb*{k}_2 \sigma} c_{\vb*{k}_1 \sigma} \right). 
\end{equation}
上式中的动能项和相互作用项的形式由对称性确定,具体系数可以暂时不设置具体值(因为直接从\eqref{eq:full-electron-gas-hamiltonian}出发做重整化群计算显然是非常困难的)。
处理这类模型的常用方法包括:
\begin{itemize}
    \item 直接做费曼图计算(需要注意由于库伦相互作用是瞬时的,通常将库仑相互作用顶角写成用虚线连接的两个顶角,每个顶角有一个电子入射和一个电子出射,虚线可以携带动量),由于相互作用是二体的,在相互作用较弱时计算一到二圈图就可以得到很好的效果,不过实际上相互作用并不总是那么弱,此时需要一些其它近似手段;
    \item 由于相互作用项是四阶的,可以使用Hubbard-Stratonovich变换引入一个辅助场,通过适当选取辅助场(通常要和某个有趣的序参量具有同样的对称性)并积掉电子自由度,则接近临界点时,辅助场满足的场论就给出了长程自由度;
    \item 做平均场计算并与实验或数值计算作比较(可以对原来的模型做平均场也可以对Hubbard-Stratonovich变换之后的辅助场做平均场);
    \item 在定性论证出来一些现象后考虑其上的涨落,获得一个关于平均值附近涨落的理论(可以和平均场一起使用);
    \item 近似处理,如忽略一部分动能(即忽略一部分电子跃迁项),或者简化相互作用形式。
\end{itemize}
无论如何,凝聚态模型通常都足够复杂,且可能有很强的相互作用,以至于分析方法多种多样,但是没有哪一种能够占有支配地位。

\subsection{外加电磁场}

现在讨论外加电磁场导致的哈密顿量变化,或者说“辐射和物质相互作用”导致的哈密顿量变化。
一般的,设系统被放置在电磁场$(\varphi, \vb*{A})$中,则一次量子化哈密顿量(使用一次量子化哈密顿量是为了和经典的“一群电子定向移动”的物理图像对应上)为
\begin{equation}
    {H} = \frac{1}{2m} \sum_i ({\vb*{p}_i} - q_i \vb*{A}({\vb*{r}_i}))^2 + \sum_i q_i \varphi({\vb*{r}}_i) + {H}_\text{int},
    \label{eq:hamiltonian-with-eb-original}
\end{equation}
其中$\vb*{p}$是正则动量,${H}_\text{int}$表示粒子间相互作用。
本文仅考虑外加电磁场产生的线性响应,于是考虑辐射场不很强以至于$\vb*{A}^2$可以忽略的情况,也即,仅保留单光子过程,忽略一切$\vb*{A}^2$项。
% TODO:真的吗?实际上,当辐射场强到非线性效应开始产生时,是不是应该单独列出来一个${H}_\text{int}$然后将物质和辐射做$U(1)$极小耦合已经很成问题了:
在\eqref{eq:hamiltonian-with-eb-original}中我们将电子间的库伦排斥能和辐射场引入的能量简单相加,因为这的确是两种不同的过程:库伦排斥涉及的光子实际上是不满足横场条件的虚光子,而辐射场中的光子都是可以出现在实际的物理态中的光子。

对电子系统,$q=-e$,那么就有
\begin{equation}
    \begin{aligned}
        {H} &= \frac{1}{2m} \sum_i ({\vb*{p}_i} + e \vb*{A}({\vb*{r}_i}))^2 - e \sum_i \varphi({\vb*{r}}_i) + {H}_\text{int} \\ 
        &= - \frac{1}{2m} \sum_i (\grad + \ii e \vb*{A}({\vb*{r}_i}))^2 - e \sum_i \varphi({\vb*{r}}_i) + {H}_\text{int}.
    \end{aligned}
\end{equation}
设$\Omega$是某个空间区域,电流密度为$\vb*{J}$,则
\begin{equation}
    \int_\Omega \dd[3]{\vb*{r}} {\vb*{J}} = - \sum_i e {\vb*{v}}_i,
\end{equation}
其中$\vb*{v}_i$是电子移动的速度,满足
\begin{equation}
    {\vb*{v}}_i = \pdv{H}{\vb*{r}_i} = \frac{{\vb*{p}}_i + e \vb*{A}({\vb*{r}}_i)}{m}.
\end{equation}
考虑到$\Omega$的任意性,我们就有以下近似表达式:
\begin{equation}
    {\vb*{J}} = \underbrace{- \frac{e}{m} \sum_i ( \delta(\vb*{r} - {\vb*{r}}_i) {\vb*{p}} + {\vb*{p}} \delta(\vb*{r} - {\vb*{r}}_i) )}_{{\vb*{j}}} \underbrace{- \frac{e^2}{m} {n}_\text{e}}_{{\vb*{j}}_\text{D}} \vb*{A}.
\end{equation}
${\vb*{j}}$项特意被写成了厄米的形式;${\vb*{j}}_\text{D}$已经做了一遍粗粒化了,将诸$\vb*{A}_i$平均了一遍。
通常这是合理的,因为电磁波的波长通常不会特别小,从而不会有很大的空间起伏。(而如果有很大的空间起伏,我们就会使用cQED而不是经典电动力学讨论问题了)

现在写出略去高阶项的哈密顿量的形式。选取库伦规范,并认为$\varphi=0$,此时会发现,实际上我们有
\[
    {H} = \frac{1}{2m} \sum_i {\vb*{p}}_i^2 + {H}_\text{int} + \frac{e}{m} \sum_i {\vb*{p}}_i \cdot \vb*{A}({\vb*{r}}_i) - e \sum_i \varphi({\vb*{r}_i}),
\]
$\vb*{A}({\vb*{r}}_i)$和${\vb*{p}}_i$本来是不对易的,但是库伦规范下它们对易。
代入${\vb*{J}}$的表达式并再次略去高阶项,就有
\[
    {H} = \frac{1}{2m} \sum_i {\vb*{p}}_i^2 + {H}_\text{int} - \int \dd[3]{\vb*{r}} {\vb*{J}} \cdot \vb*{A} - e \sum_i \varphi({\vb*{r}_i}).
\]
至于含有电势的那一项,注意到电荷密度为
\[
    {\rho} = - e \sum_{i} \delta({\vb*{r}}_i - \vb*{r}),
\]
于是最后得到
\begin{equation}
    {H} = \frac{1}{2m} \sum_i {\vb*{p}}_i^2 + {H}_\text{int} + \int \dd[3]{\vb*{r}} \varphi {\rho} - \int \dd[3]{\vb*{r}} {\vb*{J}} \cdot \vb*{A}.
\end{equation}
虽然物质本身的哈密顿量不是洛伦兹协变的(因为取了非相对论近似),但是物质和辐射的相互作用项却是洛伦兹协变的——对电磁场的描述一般都是如此。
以上哈密顿量实际上仅仅讨论了轨道部分,电子还有自旋磁矩
\[
    {H}_{\text{spin}} = \sum_i {\vb*{\mu}}_i \cdot \vb*{B}({\vb*{r}}_i),
\]
我们可以如法炮制地将它写成
\[
    {H}_{\text{spin}} = \int \dd[3]{\vb*{r}} {\vb*{\mu}} \cdot \vb*{B}.
\]
因此完整的哈密顿量实际上是
\begin{equation}
    {H} = - \frac{1}{2m} \sum_i {\vb*{p}}_i^2 + {H}_\text{int} + \int \dd[3]{\vb*{r}} \varphi {\rho} - \int \dd[3]{\vb*{r}} {\vb*{J}} \cdot \vb*{A} + \int \dd[3]{\vb*{r}} {\vb*{\mu}} \cdot \vb*{B}.
    \label{eq:hamiltonian-with-eb}
\end{equation}
所有和单粒子携带的电荷数量有关的量全部被藏在电荷密度和电流密度中了,上式在电荷正反变换下不变。

\eqref{eq:hamiltonian-with-eb}当然也可以非常容易地写成二次量子化的形式。
薛定谔场满足$U(1)$对称性,因此通过诺特定理可以得到守恒荷(当然就是电荷)
\begin{equation}
    \rho = - e \sum_\sigma {\psi}^\dagger_\sigma {\psi}_\sigma = - e {n}_\text{e},
\end{equation}
守恒流(也即电流密度)
\begin{equation}
    {\vb*{j}} = - \frac{\ii e}{2m} \sum_\sigma ({\psi}_\sigma^\dagger(\vb*{r}) \grad{{\psi}_\sigma}(\vb*{r}) - (\grad{{\psi}_\sigma^\dagger}(\vb*{r})) {\psi}_\sigma(\vb*{r})) - \frac{e^2}{m} \vb*{A}(\vb*{r}) \sum_\sigma {\psi}^\dagger_\sigma(\vb*{r}) {\psi}_\sigma(\vb*{r}),
\end{equation}
而另一方面自旋磁矩为
\begin{equation}
    {\vb*{\mu}} = {\psi}^\dagger_\alpha \vb*{\sigma}_{\alpha \beta} {\psi}_\beta.
\end{equation}
这样就把三个对外加电磁场的线性响应全部写成二次量子化的形式了;我们可以用推迟格林函数计算出有关的响应大小。

由线性响应理论,我们有
\[
    \begin{aligned}
        \expval*{{J}_i}_A (t) &= \expval*{{J}_i}_0 + \ii \int \dd{t} \int \dd[3]{\vb*{r}'} \theta(t-t') \expval*{\comm*{{J}_i(\vb*{r}, t)}{{J}_j(\vb*{r}', t')}} A_j(\vb*{r}', t') \\
        &= - \frac{e^2}{m} \expval*{{n}_\text{e}} A_i + \ii \int \dd{t} \int \dd[3]{\vb*{r}'} \theta(t-t') \expval*{\comm*{{j}_i(\vb*{r}, t)}{{j}_j(\vb*{r}', t')}} A_j(\vb*{r}', t').
    \end{aligned}
\]
这里的$i, j$为维度脚标,并不表示粒子编号,且使用爱因斯坦求和约定;下标$A$表示有外场$\vb*{A}$时的期望值;${j}_i$的无外场期望是零,这是对称性的结果。

实际上,电阻率定义为%
\footnote{$\vb*{A}(\vb*{r}, t)$完全可以不是时间、空间平移不变的,但是既然我们将$\vb*{A}$当成扰动,只需要无扰动的系统的动力学时间、空间平移不变即可。}%
\begin{equation}
    J_i(\vb*{r}, t) = \int \dd[3]{\vb*{r}'} \int_{-\infty}^t \sigma_{ij}(\vb*{r}-\vb*{r}', t - t') E_j(\vb*{r}', t'),
\end{equation}
于是为了避免繁琐的时间上的微积分我们切换到频域上,有
% TODO:确认记号问题

现在我们回到二次量子化的框架下,考虑怎么计算有关的推迟格林函数。

在虚时间路径积分中不能简单地将\eqref{eq:hamiltonian-with-eb-original}(从而,\eqref{eq:hamiltonian-with-eb})做勒让德变换。
在这两个哈密顿量中有一个$U(1)$规范场$(\varphi, \vb*{A})$,由于规范对称性的存在,实际上只有$\vb*{A}$是独立的自由度,因此不能保证在Wick转动中含有$\varphi$的项是不变的。
最简单的从\eqref{eq:hamiltonian-with-eb-original}推导出虚时间路径积分的方法是最小耦合。我们知道$\varphi \rho$项会出现本质上是因为加入$U(1)$规范场之后需要将导数替换为协变导数,而可以验证以下替换
\begin{equation}
    \partial_\tau \longrightarrow \partial_\tau - \ii e \varphi, \quad - \ii \grad \longrightarrow - \ii \grad + e \vb*{A} 
\end{equation}
给出了虚时间场论中的协变导数,于是在自由理论中做这个替换就得到了与电磁场发生相互作用的电子场的虚时间配分函数。

这样会带来一个疑难,就是Wick转动后电势前面多出来了一个负号;但是凝聚态理论中电子可以被放在一个势场中,显然Wick转动后势场前面不需要多出来一个负号。
这个疑难的解答是,含有$\ii e \varphi \rho$项的理论描述了一个电子场和一个电磁场的耦合,而“电子置于势场中”的模型中电磁场已经被积掉了。
如果将电磁场积掉,按照高斯积分的原理,$e \varphi \rho$项前面会多一个负号,并且要乘以系数$\ii$的平方,于是我们发现Wick转动后前面不需要多出来一个负号的势场出现了。

在以上的推导中,我们均取$e>0$为正的元电荷;在分析电子系统时,还有一种记号是取$e<0$,令$\abs*{e}$为元电荷。
这样可以让电子系统的配分函数看起来更像是直接从$U(1)$规范不变性得到的(即$\vb*{p}$替换成$\vb*{p} - e \vb*{A}$),从而看起来更加接近高能物理中的记号。
此处我们采用第一种记号,认为$e > 0$。

\section{晶格}

所谓晶体指的是一种在三个独立的空间方向上具有离散的平移不变性且并没有连续平移不变性的物体。\eqref{eq:many-body-hamiltonian}显然具有连续的平移不变性,因此晶体的形成必然经历了对称性自发破缺,且在较高的能量下原本的晶体一定会相变成某种更加均匀的东西。
本章将展示组成晶体的离子实是具体如何组成晶体的。原则上可以有哪些晶体见\autoref{chap:lattice-structure}。

将固体分解为组成它的组件所需的能量称为\concept{内聚能}。造成内聚的相互作用方式包括:
\begin{itemize}
    \item 离子键,离子之间的库伦引力。
    \item 共价键,相对局域的电子云的重叠导致的等效原子间吸引力。
    \item 金属键,离域电子组成大范围的电子气,导致等效的原子间吸引力。此时静电屏蔽非常强,从而正电荷可以看成均匀的“凝胶”。很直观地,晶格中原子排列的具体形式没有特别的要求。
    \item 范德华力,一种分子之间较弱的相互作用力,键能很低,大体上可以分为取向力、诱导力、色散力。
    \item 氢键,几乎裸露的氢原子和其它原子的吸引力,比范德华力强但比真正的化学键弱。
\end{itemize}
这些吸引相互作用不能让原子无限制地靠近,因为当电子距离足够近时将会出现明显的排斥,即所谓\concept{泡利排斥能}。%
\footnote{
    % TODO:我现在有点糊涂,计算能量时用的是直积态还是已经反对称化之后的态??
    % 如果是前者那么的确会有明显的“排斥”
    距离很近的电子之间的排斥和距离适中的自旋不同的电子之间的(由于交换相互作用而产生的)吸引实际上都是库伦相互作用的结果。
    费米子算符的反对易性意味着库伦相互作用可以分成密度-密度排斥和 % TODO
}%

可以使用\concept{分子轨道理论}近似地分析共价键:设有两个(可以不一样的)原子A和B。我们在两个原子周围各自取一个单原子电子波函数(即所谓原子轨道),然后将来自另一个原子的库伦吸引力当成微扰。
如果作为出发点的两个原子轨道能量差别不大,那么考虑了来自另一个原子的相互作用之后,将会出现显著的能级劈裂。
较低的那个能级就是\concept{成键轨道},而较高的那个能级就是\concept{反键轨道},它们是彼此正交的、原来两个原子轨道的线性组合。
最后,引入电子-电子相互作用,但是仍然假定多电子波函数近似为两个单电子波函数的直积(这就是\concept{分子轨道}一说的来历:我们假定系统中仍然有定义良好的单电子轨道,尽管此时这个轨道遍布整个分子)。
此时电子-电子相互作用导致的一阶能量修正包括一项密度-密度排斥力和一项交换相互作用。前者会让成键轨道上的电子相互排斥,然而无论怎么放置这两个电子,都会有互相排斥;后一项在成键轨道上有两个自旋相反的电子时却会让两个电子相互吸引。
如果总的电子-电子相互作用导致的修正没有大过成键轨道能量降低的量,那么一个共价键就形成了:因为电子放在成键轨道上能够降低总能量。
直观地看,此时成键轨道上的两个电子的电子云互相重叠,由于没有特别紧密地重叠,库伦排斥并不是特别大,而两个原子中间的这团电子云同时受到两边的原子的库伦吸引,因此就在两个原子之间建立了明显的等效吸引力。
反之,如果一开始的两个原子轨道能量相差很大,那么成键轨道和反键轨道和原来的两个原子轨道没有什么差别,电子放在成键轨道上不能降低多少能量,此时也不会有等效的强烈的原子-原子相互作用,也就没有共价键。

金属键弥散到整个金属晶体中,原子-原子间的等效吸引力没有特别明显的方向性之类,原子排列得越密集,库伦能越低。
这就是金属中密排结构特别常见的原因。
但是另外一方面,原子密排又会导致电子动量增大,以及泡利不相容原理导致的排斥。

\section{晶体中的基本自由度}

本节简单介绍晶体中的基本自由度。基本上,凝聚态系统的任何行为都可以使用准粒子的观点讨论,将自旋自由度称为准粒子可能有些牵强,但是它们无疑是产生准粒子的场。
很多人用准粒子一词表示电子、空穴等和电子结构有关的集体模式而用元激发表示声子等经典极限为波动的集体模式,虽然这个区分并不是很重要——例如,Luttinger液体中的玻色子算是什么呢?
还有一种区分方式是将元激发限定为和系统的激发态相关的准粒子。有些准粒子在基态中也会出现,不将它们称为元激发。

\subsection{能带电子}

固体中固有的成分包括电子和晶格。电子之间存在库伦相互作用,同时受到晶格提供的周期性势场作用。
在静止晶格的周期性势场作用下,原本能量简并的电子状态的能量会发生偏移,从而,一系列能量完全相同的电子态经过周期势场的扰动,其能量将成为一系列相差非常小的能级。
当简并电子态的数目足够多——正如晶体中通常是的那样——就会形成在能谱中连续的\concept{能带}。

周期势的全部影响就是裸电子将会有能量修正而成为能带电子,无相互作用的能带电子组成费米气体,很容易处理。
库伦排斥和声子介导的吸引相互作用合并为电子-电子相互作用,非常难以处理——电子-电子相互作用可以导致各种奇特的现象,并且其量级通常与费米能的量级一致。
电子之间的库伦相互作用的作用是不确定的,如果它不会产生全新的准粒子,那么它主要提供电子自能修正。
对电子-电子两点松原格林函数,我们有如下Dyson方程
\begin{equation}
    G  = G^{0} + G^0 \Sigma G,
\end{equation}
其中$-\Sigma$是1PI自能图之和,或者说是正规自能图之和。假定我们已经计算出了$\Sigma$,即有
\[
    G(\ii \omega_n) = \frac{1}{\ii \omega_n - \Sigma(\ii \omega_n) - H_0},
\]
其中$H_0$表示单粒子哈密顿量,同时包括电子动能和周期势场。
这个格林函数的极点的位置显然可以通过以下方程(这里我们为了和单电子非相对论性量子力学的记号保持一致,用$E$代替了$\ii \omega_n$,但是没有对格林函数乘上$\ii$)
\[
    (E - \Sigma(E) - H_0) \ket{\psi} = 0 
\]
得到,或者写成坐标空间的形式,通过求解
\begin{equation}
    \left( - \frac{\laplacian}{2m} + V_\text{ext}(\vb*{r}) \right) \psi(\vb*{r}) + \int \dd[3]{\vb*{r}'} \Sigma(\vb*{r}', \vb*{r}, E) \psi(\vb*{r}) = E \psi(\vb*{r})
    \label{eq:dyson-wave-eq}
\end{equation}
得到。这个改头换面的Dyson方程看起来和普通的薛定谔方程非常类似,不同之处在于$\Sigma(\vb*{r}', \vb*{r}, E)$可以是非常复杂的,并且由于它显含$E$,且和诸如系统中的电子总数等量有关,以上方程求解起来可能非常困难。$E$给出了重整化后的电子能谱。
% TODO:tm到底什么是单电子近似??
解出\eqref{eq:dyson-wave-eq}之后就得到
\begin{equation}
    G(\vb*{r}, \vb*{r}', E) = \sum_m \frac{1}{E - E_m} \psi_m(\vb*{r})^* \psi_m(\vb*{r}'),
\end{equation}
从而把$\psi(\vb*{r})$当成单电子波函数计算单体算符的期望值是不会出错的。
同样我们也可以在选定一个自能之后通过格林函数的时间演化方程求解格林函数,或者将格林函数约化到一些近似的动理学方程上。

总之,在很大一类情况下,经过重整化的电子可以近似看成\concept{能带电子},其能谱为一条条分离的能带,并且有时可以近似认为不存在相互作用,有时可以认为仅存在较弱的密度-密度相互作用(如在费米液体理论中;此时还可以得到等效的“自由电子”,虽然无论是这种“自由电子”还是紧束缚模型中的自由电子实际上都是经过重整化的)。

在一维情况下这样的能带结构是不稳定的,此时可以做\concept{玻色化}而得到一类非常不同的准粒子。

\subsection{声子}

固体中,晶格可以起到两种作用:一种是提供一个周期性势场,一个是晶格的畸变;前者属于电子的单体哈密顿量,可以通过求解自由模型被考虑在内;后者则导致声子。
这样,解出电子的近独立理论(如紧束缚模型)、声子的自由理论之后再加入电子、声子间的相互作用,我们就可以用电子和声子组成的相互作用气体完整描述晶体的行为,而不需要再显式考虑晶格的存在。
例如,在计算热容等变量时,完全可以认为热容是声子和电子提供的,而不是(裸的)电子和晶格提供的。

离子实的运动让我们得到一个定义在正格子上的场,这个场的激发就是\concept{声子}。晶体中一定会出现这样的激发,且这是一个Goldstone模式,因为晶体的形成实际上破缺了连续平移对称性,那么必然有一种零质量激发产生。
离子实的哈密顿量大致如下:(使用大写字母以和电子区分)
\begin{equation}
    {H} = \sum_n \frac{{P}_n^2}{2M_n} + \frac{1}{2} \sum_{\pair{m, n}} \omega_{mn}^2 ({X}_{m} - {X}_{n})^2 + \cdots, 
\end{equation}
其中$n$指的是晶格坐标,也就是三元组$(n_1, n_2, n_3)$,
\[
    \vb*{X} = \vb*{R} - \vb*{R}_0
\]
为某个离子实的位移。
正则量子化之后,哈密顿量中的前两项给出一个自由场,而离子实之间的非线性相互作用则给出声子-声子散射过程。
由于坐标-动量关系是对易关系而不是反对易关系,声子是玻色子。

离子实和电子的相互作用通常取这样的形式:
\[
    H_\text{ei} = \sum_{n, i} V_\text{ei}(\vb*{r}_i - \vb*{R}_n) = \sum_{n, i} (\vb*{R}_n - \vb*{R}_{n}^0) \cdot \grad{V_\text{ei}}(\vb*{R}_n^0 - \vb*{r}_i) + \cdots,
\]
其中$R^0_n$指的是离子实$n$未发生移动时的位置,即$n$号正格子的位置。
对位移场$\vb*{X}$做正则量子化,并将关于电子的部分做二次量子化,得到
\[
    {H}_{\text{ei}} = \sum_{n, i} V_\text{ei}(\vb*{r}_i - \vb*{R}_n) = \sum_{n} \int \dd[3]{\vb*{r}} {\vb*{X}} \cdot {\psi}^\dagger(\vb*{r}) \grad{V_\text{ei}}(\vb*{R}_n^0 - \vb*{r}) {\psi}(\vb*{r}) + \cdots,
\]
这就是电子-声子相互作用。可以看出以上哈密顿量对${\vb*{X}}$不具有$U(1)$对称性,因此声子一般来说不是守恒的。

声子和电子的量子化过程很不一样。对电子,我们首先写出一个多体一次量子化哈密顿量,然后做二次量子化;而对声子,我们实际上把使用格点坐标标记的离子实位移和动量当成了场算符(离散的场),然后直接对这两个场算符做正则量子化。
换而言之,没有声子的一次量子化:定义声子时我们的理论就是二次量子化的。

\subsection{自旋}

自旋是电子的内禀性质。然而,在一些情况下,电子几乎总是定域在某些格点附近而不发生移动,此时系统中不存在电子位置的变化,主要的自由度是各个格点上的自旋。
这样的模型即为所谓\concept{自旋模型}。自旋模型是非常常见的,如Hubbard模型在$U$非常大时,以动能项为微扰就能够得到一个海森堡模型。
对称性说明自旋-自旋相互作用通常可以取$\vb*{S}_i \cdot \vb*{S}_j$的形式,或者也许是各向异性的$\vb*{S}_i \cdot \vb*{T} \cdot \vb*{S}_j$。

\section{固体系统的表征}

表征一个固体系统通常可以使用的方法,或者说固体系统常见的可测量量(这里不是指可观察量算符,而是指真的做量子测量能够测出来的物理量,比如说期望,等等),包括导电性、磁性、对电磁波的响应、力学和热学性质。

\subsection{热学和固体力学}

热学性质——从而固体力学性质——经常可以使用自由粒子的模型来估计。
一种无相互作用的粒子的哈密顿量是完全对角化的:
\[
    H = \sum_{\vb*{k}, \sigma} \omega_{\vb*{k}\sigma} \left(n_{\vb*{k} \sigma} + \frac{1}{2}\right),
\]
于是对配分函数的贡献为
\begin{equation}
    Z = \sum_{E_i} \ee^{- \beta E_i} = \ee^{-\beta E_{\text{eq}}} \prod_{\vb*{k}, \sigma} \ee^{- \beta \frac{\omega_{\vb*{k}\sigma}}{2}} \sum_{n_{\vb*{k}, \sigma}} \ee^{-\beta \omega_{\vb*{k}\sigma} n_{\vb*{k} \sigma}},
\end{equation}
因此这种粒子对内能的贡献为
\begin{equation}
    U = E_{\text{eq}} + \sum_{\vb*{k}, \sigma} \left( \frac{1}{2} + \frac{1}{\ee^{\beta \omega_{\vb*{k}\sigma}} - 1} \right) \omega_{\vb*{k}\sigma}.
\end{equation}
这样,等容热容为
\begin{equation}
    C_V = \left(\pdv{U}{V}\right)_V = \sum_{\vb*{k}, \sigma} \left(\frac{\omega_{\vb*{k}\sigma}}{T}\right)^2 \frac{\ee^{\omega_{\vb*{k}\sigma} / T}}{(\ee^{\omega_{\vb*{k}\sigma} / T} - 1)^2}.
\end{equation}
每种粒子都会对热容有一个贡献,相互作用会修正这个贡献。

\subsection{电磁响应和输运}

如果一种粒子携带某种荷,那么还可以观察到\concept{输运}。大体上输运过程可以分成两种,一种是粒子平均自由程远小于系统尺寸的过程,此时系统内部足以出现荷的梯度,输运流量和这种梯度有关,可以称为\concept{扩散};另一种是粒子平均自由程大于系统尺寸的过程,此时系统内部的粒子基本上不受到任何阻碍,输运流量由系统边界上的性质确定,即所谓\concept{弹道输运}。

输运经常是依靠外加电磁场而产生的。
由于$\vb*{A}$耦合在$\vb*{j}$上,而能够决定极化电场的也无非是电子数密度,讨论电子气的电磁响应基本上就是要计算密度-密度关联函数。
如果不考虑电子之间的库仑相互作用,那么这是平凡的:诸如“谐振子受到电场策动,然后出现极化”的图像就能工作得很好。
在绝缘体中这大体上是正确的,在金属中则不见得是这样。
因此,比较精确地计算电子气的电磁响应,实际上就是需要将电子-电子库伦相互作用纳入计算,而这就很自然地要求我们使用一种完整的凝聚态场论。

\section{理想晶体的晶格结构}\label{sec:lattice-structure}

本节讨论\concept{理想晶体},即非常大,以至于其表面的情况几乎不会影响内部电子运动的晶体。

\subsection{晶体的几何形状}

\subsubsection{晶格、晶胞和格点坐标}

我们采取玻恩–奥本海默近似,将离子实看成一个背景,而忽略其中的自由度(既然这些自由度在价电子的物理过程的时间尺度上基本上不参加相互作用)。这样一来离散的平移不变性只应该来自$V(\vb*{r})$。
我们于是看到了形成晶体的对称性自发破缺的来源:低能下离子实自发地排成了比较规则的序列,从而虽然晶体服从的物理规律实际上确实是连续平移不变的,近似定律\eqref{eq:electron-gas-hamiltonian}却由于离子实排列成了空间重复的序列而只有离散平移不变性而没有连续平移不变性。
这种离子实周期性排列形成的结构可以写成两部分,一部分是若干个离子的平衡位置,或者称为\concept{基元},一部分是基元的周期性排列方式或者说点阵(lattice),称为\concept{晶格}。
基元——或者说全同的结构单元——占据的空间称为\concept{原胞}(primitive cell)——是晶体中体积最小的重复性成分。

在晶格中只要知道了某个格点的位置,就可以计算出其它任何格点的位置。或者,如果知道了某个原胞中某个原子的位置,就可以知道其它任何一个原胞中同种原子的位置。
任意两个格点或是同种原子之间的位置矢量形如
\begin{equation}
    \vb*{R}_n = n_1 \vb*{a}_1 + n_2 \vb*{a}_2 + n_3 \vb*{a}_3, \quad n = (n_1, n_2, n_3) \in \mathbb{N}^3.
\end{equation}
这些位置矢量构造了一架三维网格,这个网格称为\concept{布拉伐格子},这些矢量称为\concept{布拉伐格矢},$\{\vb*{a}_i\}$称为\concept{晶格常数},$n$称为\concept{格点坐标}。
(也有很多人将$\vb*{a}_i$称为格矢,在我们的术语中$\vb*{a}_i$是“基元格矢”)
布拉伐格子中的所有格点周围环境相同,无重叠无遗漏地覆盖整个空间。在每个格子上放置一个基元,即可构造出整个晶体。
设基元中有$n$种原子,我们任取其中一种原子而忽略其它原子,则被选中的这种原子本身也组成一个晶格,且这个晶格和整体的晶格是完全一样的。
$n=1$的情况称为\concept{简单晶格},而$n>1$的情况称为\concept{复式晶格},因为它实际上是$n$个简单晶格套在一起而得到的。
相当神奇的是,即使晶体中只有一种原子,可能仍然无法剖分出一个只含有一个原子的原胞,换而言之,此时晶体中的物理上完全一样的原子在几何上仍然可以进一步分成若干类,从而晶格是复式晶格。
要看出为什么会有这种情况出现,只需要想象取一个普通的复式晶格晶体,然后将其中所有的原子都替换成同种原子即可。
一些重要的晶体如石墨烯就具有这种性质,虽然物理上只有一种原子,却仍然是复式晶格。

布拉伐格子的原胞有许多划分方法。可以以$\vb*{a}_1, \vb*{a}_2, \vb*{a}_3$张成的平行六面体为一个原胞,称为\concept{初基原胞}。
另一种原胞是\concept{维格纳-赛兹原胞},它是空间中与某个特定格点的距离小于与任何其它格点的距离的点的轨迹,或者等价地说,它是某个特定格点与相邻格点的连线的垂直平分面包围出的立体。
任意一个空间矢量都可以写成某种原胞中的一个矢量加上一个布拉伐格矢,这可以使用非常直观的方式证明。

原胞有时很难直观地展示晶体的特征。\concept{晶体学原胞}或者说\concept{单胞}指的是最大限度反映晶格对称性的最小单元。
它是重复性的单元,因此它应该包含整数倍的原胞。原胞中原子位置可以使用原胞基矢,当然也可以使用单胞基矢。
例如,面心立方格子的原胞乍一看就是一个形状奇怪的平行六面体,而其单胞则是非常直观的“面心立方格子”。
确定一个单胞里面有几个原胞,可以通过计算一个单胞中有几个格点(实际上就是有几个基元)来完成。

由于以下用到晶体离散对称性的地方几乎从来不会用到“原胞是最小的”这一事实,很多时候把“原胞”一词替换成“单胞”是完全可以的。于是我们就模糊地说\concept{晶胞},即晶体中不需要最小的重复周期。
晶体中各个晶胞的坐标同样组成一架周期性格子,和原胞组成的格子的对称性相同。

\subsubsection{晶体中的坐标系和方向}

% TODO:坐标系,晶向,晶面

在获得了一个平行六面体晶胞之后,可以用确定这个平行六面体的三个矢量作为基矢量,建立一个坐标系。
我们通常用$\vb*{a}, \vb*{b}, \vb*{c}$或$\vb*{a}_1, \vb*{a}_2, \vb*{a}_3$标记这三个矢量,用$\alpha, \beta, \gamma$标记它们的夹角,从而确定这个平行六面体的几何形状。

\subsubsection{倒格子}

各个原胞(我们通常使用原胞而不是单胞或者别的类型的晶胞构造正格子;无论如何正格子的原胞大小总是和用于构造正格子的晶胞大小一样的;以下我们将“晶胞”一词局限在“实际晶体的一个周期”这个意义上;在讨论格点——无论是正格子还是倒格子——时我们只用“原胞”)所在的$\vb*{R}_n$构成的空间网格(以下称为\concept{正格子})上的所有可观察物理量均具有和布拉伐格矢一样的对称性,也即,它们在三个方向上以$\vb*{a}_1,\vb*{a}_2,\vb*{a}_3$为周期。
回顾傅里叶级数的公式,我们有
\[
    f(x) = \frac{1}{T} \sum_{m=-\infty}^\infty \ee^{\ii \frac{2\pi m x}{T}} \left(\int \dd{t} f(t) \ee^{-\ii \frac{2\pi m t}{T}}\right) ,
\]
其三维形式就是
\[
    f(\vb*{r}) = \frac{1}{V} \sum_{m=-\infty}^\infty \ee^{\ii \vb*{G}_m \cdot \vb*{r}} \int_V \dd[3]{\vb*{r}'} f(\vb*{r}') \ee^{- \ii \vb*{G}_m \cdot \vb*{r}}, \quad \vb*{G}_m \cdot \vb*{a}_i = 2\pi N_i, \quad N_i \in \mathbb{N},
\]
其中$V$指的是正格子原胞的大小。
$\vb*{G}_m$满足的条件等价于,对任意的布拉伐格矢都有
\begin{equation}
    \vb*{G}_m \cdot \vb*{R}_n = 2\pi N, \quad N \in \mathbb{N},
\end{equation}
这又等价于,
\begin{equation}
    \vb*{G}_m = G_1 \vb*{b}_1 + G_2 \vb*{b}_2 + G_3 \vb*{b}_3, \quad \vb*{a}_i \cdot \vb*{b}_j = 2 \pi \delta_{ij}.
\end{equation}
因此诸$\vb*{G}_m$也构成一个布拉伐格子,我们称它为\concept{倒格子},与正格子相区分,同样,称$\vb*{r}$所在的三维空间为\concept{实空间},$\vb*{G}$所在的空间为\concept{倒空间}。倒格子的基矢量和正格子的基矢量互为共轭基矢量。
两种格子的基矢量可以通过下式
\begin{equation}
    \frac{1}{2\pi} \vb*{b}_1 = \frac{\vb*{a}_2 \times \vb*{a}_3}{\vb*{a}_1 \cdot (\vb*{a}_2 \times \vb*{a}_3)}
\end{equation}
及其轮换相互换算。
在写出倒格子的显式表达式之后,晶体中具有正格子的周期性的物理量的傅里叶变换就是
\begin{equation}
    F(\vb*{r}) = \sum_{\vb*{g}} \tilde{F}(\vb*{g}) \ee^{\ii \vb*{g} \cdot \vb*{r}},
\end{equation}
其中$\vb*{g}$是布拉伐格矢,且
\begin{equation}
    \tilde{F}(\vb*{g}) = \frac{1}{V} \int_V \dd[3]{\vb*{r}} F(\vb*{r}) \ee^{-\ii \vb*{g} \cdot \vb*{r}}.
\end{equation}

倒格子的布拉伐格子可以和正格子同一类型,但是也可以不一样。
但是,倒格子和正格子的最高点群对称性(也即,不考虑基元,仅考虑格子本身,或者只讨论球对称的基元)一定是一样的。
这是因为设$R$是一个点群对称性操作,则
\[
    R \vb*{R}_i = \vb*{R}_i,
\]
% TODO:就是矩阵可以作用在左边也可以作用在右边

倒格子当然也有原胞的概念。倒格子的维格纳-赛兹原胞称为\concept{第一布里渊区},相应的,某格点和它所有次近邻格点的垂直平分面包围成的区域称为\concept{第二布里渊区},等等。
只要给定一种倒格子,其第一布里渊区就是完全确定的,而和怎样选择倒格子的初基格矢没有关系。

倒格子的原胞实际上是定义在正格子上的函数的傅里叶变换的动量取值范围。
使用$i=(i_1, i_2, i_3)$表示格点坐标,则
\begin{equation}
    \frac{1}{N} \sum_{\vb*{a}_i} \ee^{\ii (\vb*{k} - \vb*{k}') \cdot \vb*{a}_i} = \sum_{\vb*{g}} \delta_{\vb*{k} - \vb*{k}' + \vb*{g}},
\end{equation}
其中$\vb*{a}_i$指的是$i$对应的位矢,$N$是晶格中总离子数,$\vb*{g}$扫过整个倒格子。
可以看到方程右边是周期性的,如果限制$\vb*{k}$在一个倒空间原胞中,那么就有非常简单的形式:
\begin{equation}
    \frac{1}{N} \sum_{\vb*{a}_i} \ee^{\ii (\vb*{k} - \vb*{k}') \cdot \vb*{a}_i} = \delta_{\vb*{k} \vb*{k}'},
\end{equation}
从而得到与之对偶的
\[
    \frac{1}{N} \sum_{\vb*{k}} \ee^{\ii (\vb*{a}_i - \vb*{a}_j) \cdot \vb*{k}} = \delta_{ij},
\]
其中$\vb*{k}$扫过
这又意味着倒格子的一个原胞内的动量取值数目可以认为是$N$个,当然这是正确的,因为实空间中的$N$点离散信号做离散傅里叶变换之后会得到倒空间中的周期性离散信号,其周期正好是$N$,一个原胞正好是一个周期。

并非所有晶体中的物理量具有正格子的周期性(有的周期性强于正格子,有的也许弱于正格子),它们的傅里叶变换中的动量不局限在倒格子上。

\subsubsection{有限大小的晶体}

晶格对电子的吸引比较明显,因此电子自发溢出晶格的概率并不大,从而可以将晶格表面看成一个势阱。
晶格表面的形状以及势阱的高度无疑会影响电子气的行为,但由于晶体非常大,这种影响对稍微远离表面的电子都是非常微弱的。(接近表面的电子可能参与表面态,此时关于晶体表面的信息就非常重要了)
因此我们认为晶体是长宽高各为$L$的大正方体,$L$相对电子、原子的尺度都是非常大的;同时我们简单地施加一个周期性边界条件来表示势阱的存在,即认为
\begin{equation}
    \psi(\vb*{r}) = \psi(\vb*{r} + L \vb*{e}_i), \quad i = 1, 2, 3.
    \label{eq:periodic-boundary}
\end{equation}
同时我们暂时忽略在晶体外找到电子的概率,因为它相对于在晶体内部找到电子的概率是非常小的。

由于晶体是有限大小的,$\vb*{k}$的取值是离散化的,因为波函数必须满足\eqref{eq:periodic-boundary},为了尽可能让$u$容纳较多信息,我们用$\vb*{k}$来满足这个要求,即
\[
    \ee^{\ii \vb*{k} \cdot \vb*{r}} = \ee^{\ii \vb*{k} \cdot (\vb*{r} + L \vb*{e}_i)}, \quad i = 1, 2, 3,
\]
这样$\vb*{k}$的取值范围就是一个晶格常数为$2\pi / L$的三维点阵。%
\footnote{这个点阵不是倒格子:倒格子的晶格常数和实际的物理结构——也就是晶格的结构——有关,而此处的点阵的晶格常数完全是我们强加的,且总是趋于零,使得格点动量看起来几乎是连续的,因此可以被划分成连续的布里渊区,等等。
}%
这个三维点阵正是局限在晶体内部的任何函数做空间傅里叶变换得到的波矢的取值范围,且有如下归一化条件:
\begin{equation}
    \frac{1}{V} \int \dd[3]{\vb*{r}} \ee^{\ii (\vb*{k} - \vb*{k}') \cdot \vb*{r}} = \delta_{\vb*{k}\vb*{k}'}.
\end{equation}

\subsubsection{三种动量空间,它们中的动量本征态和对应的傅里叶变换}\label{sec:momentum-space-inner-product}

现在我们有三种动量空间,这三种动量空间、它们适用于什么样的函数以及它们对应的傅里叶变换列举如下。
在这里唯一确定的是坐标空间的场算符$\psi(\vb*{r})$,它满足
\[
    \comm*{\psi_\alpha(\vb*{r})}{\psi^\dagger_\beta(\vb*{r}')} = \delta_{\alpha \beta} \delta(\vb*{r} - \vb*{r}').
\]
实际上由于我们局限在一个大小为$V$的盒子中,这里的$\delta(\vb*{r} - \vb*{r}')$函数并非全空间中的$\delta$函数,而是实空间区域$V$中的$\delta$函数,它能够让任何限制在$V$中的函数$f(\vb*{r})$满足
\[
    \int_V \dd[3]{\vb*{r}'} f(\vb*{r}') \delta(\vb*{r} - \vb*{r}') = f(\vb*{r}).
\]
但显然$V \to \infty$时这种“软化”的$\delta(\vb*{r} - \vb*{r}')$就趋于无限大空间中的$\delta(\vb*{r} - \vb*{r}')$,从而以下我们可以混用两种$\delta(\vb*{r} - \vb*{r}')$。
动量表象的归一化系数是不确定的。
我们需要定义电子波函数的内积,由于$\braket*{\psi}{\psi} = 1$,每种内积定义都会在波函数前面引入一个不同的归一化因子。
归一化是很重要的,因为诸如微扰论给出的能级修正等在归一化不对时是不正确的;此外一些物理量真的会正比于$V$或者说$N$,从而归一化常数中的$V$或是$N$是很重要的。

\begin{enumerate}
    \item 晶格常数为$2\pi / L$的三维点阵是被限制在边长为$L$的实空间方盒中的函数的傅里叶变换的动量取值范围,由于$L$通常很大,这个点阵要比倒格子密得多,在倒格子的尺度上工作时没有必要考虑晶格常数为$2\pi / L$的三维点阵是离散的这一事实。
    该三维点阵动量空间虽然是离散的,却是无限大的,因为它实际上给出的是一个箱子中所有可能的平面波的波矢,而这当然可以取到无限大。
    与这个动量空间相关的傅里叶变换的归一化表达式为
    \begin{equation}
        \frac{1}{V} \int_V \dd[3]{\vb*{r}} \ee^{\ii (\vb*{k} - \vb*{k}') \cdot \vb*{r}} = \delta_{\vb*{k} \vb*{k}'}, \quad \frac{1}{V} \sum_{\vb*{k}} \ee^{\ii \vb*{k} \cdot (\vb*{r} - \vb*{r}')} = \delta(\vb*{r} - \vb*{r}'),
    \end{equation}
    第二个表达式中的$\delta$函数就是动量空间无限大的一个结果。于是有以下几种方案:一种是将动量本征态取为
    \begin{equation}
        \psi_{\vb*{k}}(\vb*{r}) = \frac{1}{\sqrt{V}} \ee^{\ii \vb*{k} \cdot \vb*{r}},
    \end{equation}
    则容易验证我们应该定义内积为
    \begin{equation}
        \braket*{\alpha}{\beta} = \int \dd[3]{\vb*{r}} \braket*{\alpha}{\vb*{r}} \braket*{\vb*{r}}{\beta} = \sum_{\vb*{k}} \braket*{\alpha}{\vb*{k}} \braket*{\vb*{k}}{\beta},
        \label{eq:integrate-ouver-whole-space}
    \end{equation}
    这样可以满足归一化条件。此时表象变换为
    \[
        \braket*{\vb*{k}}{\alpha} = \int \dd[3]{\vb*{r}} \frac{1}{\sqrt{V}} \ee^{- \ii \vb*{k} \cdot \vb*{r}} \braket*{\vb*{r}}{\alpha}, \quad \braket*{\vb*{r}}{\alpha} = \sum_{\vb*{k}} \frac{1}{\sqrt{V}} \ee^{ \ii \vb*{k} \cdot \vb*{r}} \braket*{\vb*{k}}{\alpha}
    \]
    于是通常如此定义傅里叶变换:
    \begin{equation}
        f(\vb*{k}) = \frac{1}{\sqrt{V}} \int \dd[3]{\vb*{r}} \ee^{- \ii \vb*{k} \cdot \vb*{r}} f(\vb*{r}), \quad f(\vb*{r}) = \frac{1}{\sqrt{V}} \sum_{\vb*{k}} \ee^{\ii \vb*{k} \cdot \vb*{r}} f(\vb*{k}),
        \label{eq:sqrt-v-rep}
    \end{equation}
    这样定义的好处在于,我们知道${\psi}^\dagger(\vb*{r})$和${c}^\dagger_{\vb*{k}}$彼此为傅里叶变换,而如果采用上式的傅里叶变换定义,那么从
    \[
        \acomm*{{\psi}_\alpha(\vb*{r})}{{\psi}^\dagger_\beta(\vb*{r}')} = \delta(\vb*{r}-\vb*{r}') \delta_{\alpha \beta}
    \]
    可以得到
    \[
        \acomm*{{c}_{\vb*{k}\alpha}}{{c}^\dagger_{\vb*{k}' \beta}} = \delta_{\vb*{k} \vb*{k}'} \delta_{\alpha \beta},
    \]
    于是动量表象下的产生湮灭算符关系也是非常简单的。
    
    另一种方案是取动量本征态为
    \begin{equation}
        \psi_{\vb*{k}}(\vb*{r}) = \ee^{\ii \vb*{k} \cdot \vb*{r}},
    \end{equation}
    此时内积应该定义为
    \begin{equation}
        \braket*{\alpha}{\beta} = \frac{1}{V} \int \dd[3]{\vb*{r}} \braket*{\alpha}{\vb*{r}} \braket*{\vb*{r}}{\beta} = \frac{1}{V} \sum_{\vb*{k}} \braket*{\alpha}{\vb*{k}} \braket*{\vb*{k}}{\beta}, 
        \label{eq:integrate-ouver-whole-space-divided}
    \end{equation}
    这样的好处在于,在$V$很大时,考虑到点阵的晶格常数为$2\pi / L$,可以做替换
    \begin{equation}
        \frac{1}{V} \sum_{\vb*{k}} \longrightarrow \int \frac{\dd[3]{\vb*{k}}}{(2\pi)^3},
    \end{equation}
    从而$\vb*{k}$基底下实际上没有$V$的依赖,正如动量本征态的定义所展示的那样。此时表象变换为
    \[
        \braket*{\vb*{r}}{\beta} = \frac{1}{V} \sum_{\vb*{k}} \ee^{\ii \vb*{k} \cdot \vb*{r}} \braket*{\vb*{k}}{\beta} = \int \frac{\dd[3]{\vb*{k}}}{(2\pi)^3} \ee^{\ii \vb*{k} \cdot \vb*{r}} \braket*{\vb*{k}}{\beta}, \quad 
    \]
    从而傅里叶变换应该取为
    \begin{equation}
        f(\vb*{r}) = \frac{1}{V} \sum_{\vb*{k}} \ee^{\ii \vb*{k} \cdot \vb*{r}} f(\vb*{k}) = \int \frac{\dd[3]{\vb*{k}}}{(2\pi)^3} \ee^{\ii \vb*{k} \cdot \vb*{r}} f(\vb*{k}).
        \label{eq:v-rep}
    \end{equation}
    此时
    \begin{equation}
        \comm*{c_{\vb*{k} \alpha}}{c_{\vb*{k}' \beta}^\dagger} = (2\pi)^3 \delta_{\alpha \beta} \delta_{\vb*{k} \vb*{k}'}.
    \end{equation}
    在这种记号下对$2n$个坐标空间下的场算符的乘积积分,所得结果用动量空间中的产生湮灭算符表示则一定正比于$V$,因为到最后那个对$\vb*{r}$的积分给出
    \[
        \int \dd[3]{\vb*{r}} \ee^{\ii \sum \vb*{k} \cdot \vb*{r}} \propto V.
    \]
    相反,在\eqref{eq:sqrt-v-rep}的记号下,对$2n$个坐标空间下的场算符的乘积积分,所得结果用动量空间中的产生湮灭算符表示则正比于$V^{1 - n}$。
    \eqref{eq:sqrt-v-rep}相当于将体积依赖放到了与动量空间有关的物理量(比如说$\psi_{\vb*{k}}$)中,而\eqref{eq:v-rep}相当于将体积依赖放到了与坐标空间有关的物理量(主要是对$\dd[3]{\vb*{r}}$积分时总是需要乘以$V$)中。
    我们通常采用第一种记号,也就是\eqref{eq:sqrt-v-rep},因为在固体物理中实空间的物理定律——如库仑定律——是保证和自由情况下完全一样的。例如我们时常需要从库伦相互作用获得某种等效相互作用通道,那么使用记号\eqref{eq:sqrt-v-rep}的话就不需要担心归一化出错。

    有时候,我们不使用平面波基底展开波函数,但是这并不影响$\vb*{r}$,即内积定义\eqref{eq:integrate-ouver-whole-space}和\eqref{eq:integrate-ouver-whole-space-divided}中涉及坐标的部分在任何时候都是可以用的。
    我们通常还是使用\eqref{eq:integrate-ouver-whole-space}。

    \item 倒格子是具有和晶格一样的周期性的连续函数的傅里叶变换的动量取值范围。与之相关的归一化表达式为
    \begin{equation}
        \frac{1}{V_\text{u.c.}} \int_{V_\text{u.c.}} \dd[3]{\vb*{r}} \ee^{\ii \vb*{r} \cdot (\vb*{k} - \vb*{k}')} = \delta_{\vb*{k} \vb*{k}'}, \quad \frac{1}{V_\text{u.c.}} \sum_{\vb*{k}} \ee^{\ii \vb*{r} \cdot (\vb*{k} - \vb*{k}')} = \delta(\vb*{r} - \vb*{r}'), 
        \label{eq:normalization-periodic}
    \end{equation}
    其中$\vb*{k}$取遍整个倒格子而$\vb*{r}$取遍一个正格子晶胞内部的所有点(因为仅讨论周期函数)。这里我们使用$V_\text{u.c.}$表示一个正格子晶胞的大小,来和整块晶体的大小区分开。相应的,傅里叶变换为
    \begin{equation}
        f(\vb*{k}) = \frac{1}{\sqrt{V_\text{u.c.}}} \int \dd[3]{\vb*{r}} \ee^{- \ii \vb*{k} \cdot \vb*{r}} f(\vb*{r}), \quad f(\vb*{r}) = \frac{1}{\sqrt{V_\text{u.c.}}} \sum_{\vb*{k}} \ee^{\ii \vb*{k} \cdot \vb*{r}} f(\vb*{r}).
    \end{equation}
    倒格子是无限大的,因此以上展示的傅里叶变换的动量空间同样是无限大的。
    \item 倒空间的原胞,如第一布里渊区,是定义在正格子上的函数的傅里叶变换的动量取值范围。归一化表达式为
    \begin{equation}
        \frac{1}{N} \sum_{\vb*{a}_i} \ee^{\ii (\vb*{k} - \vb*{k}') \cdot \vb*{a}_i} = \delta_{\vb*{k} \vb*{k}'}, \quad \frac{1}{N} \sum_{\vb*{k}} \ee^{\ii (\vb*{a}_i - \vb*{a}_j)} = \delta_{ij}.
    \end{equation}
    第二个表达式表明对定义在正格子上的函数,它在动量空间中在一个倒空间原胞中的动量取值有$N$个。傅里叶变换为
    \begin{equation}
        f(\vb*{k}) = \frac{1}{\sqrt{N}} \sum_{i} \ee^{- \ii \vb*{k} \cdot \vb*{a}_i} f(i), \quad f(i) = \frac{1}{\sqrt{N}} \sum_{\vb*{k}} \ee^{\ii \vb*{k} \cdot \vb*{a}_i} f(\vb*{k}),
        \label{eq:lattice-fourier}
    \end{equation}
    即我们将动量本征态取为
    \begin{equation}
        \psi_{\vb*{k}}(i) = \frac{1}{\sqrt{N}} \ee^{\ii \vb*{k} \cdot \vb*{a}_i},
    \end{equation}
    从而内积应该定义为
    \begin{equation}
        \braket*{\alpha}{\beta} = \sum_i \braket*{\alpha}{i} \braket*{i}{\beta} = \sum_{\vb*{k}} \braket*{\alpha}{\vb*{k}} \braket*{\vb*{k}}{\beta}
        \label{eq:sum-over-lattice}
    \end{equation}
    以满足归一化条件。
    使用这种记号和\eqref{eq:sqrt-v-rep}类似,一个对$2n$个坐标空间中的场算符乘积求和的项如果改用$c_{\vb*{k}}$表示,将正比于$N^{1-n}$。
    与前两种情况不同,上式中的动量空间中只有有限个(共$N$个)动量。实际上容易验证,这$N$个动量恰恰是一个倒空间原胞中容纳的全部来自晶格常数为$2\pi / L$的点阵的格点。
    这样,我们就有
    \[
        N = \frac{2\pi / V_\text{u.c.}}{2\pi / V},
    \]
    即
    \begin{equation}
        V = N V_\text{u.c.},
    \end{equation}
    正好是预期中的结果。在热力学极限下$N \to 0$,此时倒空间原胞中容纳的动量取值趋于连续。这就是前述“在倒格子的尺度上工作时无需考虑晶体有限大(即$L$有限大)导致的动量离散化”的一个直观理解。
\end{enumerate}

还有一个应该注意的细节:以上推导中我们都使用了$\delta_{\vb*{k} \vb*{k}'}$,一些时候也会把它写成$\delta(\vb*{k}- \vb*{k}')$,从而和坐标空间保持一致,但是要注意,如果我们希望$\delta(\vb*{k} - \vb*{k}')$满足
\[
    \int \dd[3]{\vb*{k}} \delta(\vb*{k} - \vb*{k}') f(\vb*{k}) = f(\vb*{k}'),
\]
那么其实应该取
\begin{equation}
    \delta(\vb*{k} - \vb*{k}') = \frac{V}{(2\pi)^3} \delta_{\vb*{k} \vb*{k}'} , \quad \text{as $V \to \infty$}.
\end{equation}

\subsection{晶体的对称性和分类}

\subsubsection{晶体结构分类概述}\label{sec:crystal-structure-intro}

从本节开始我们尝试用对称性分类晶体。晶体的对称性是欧氏空间的等距变换群作用在离散格子上的一个子群,其中的任何一个元素可以标记为$(R|\vb*{a})$,其中$R$表示一个不包括平移的等距变换(有限大小的对象的对称群只包括这种变换,因为有限大小的对象不可能具有平移对称性),$\vb*{a}$表示平移的距离。

晶体的对称性是非常有用的,例如,它可以直接用于决定晶体的一些参数的形式。例如,对立方晶体,介电张量是一个标量,而对六角晶体介电张量具有
\[
    \pmqty{\dmat{\epsilon_\parallel, \epsilon_\bot, \epsilon_\bot}}
\]
的形式。晶体分为晶格和基元,而晶格的对称性是具有这种晶格的晶体能够具有的最高对称性(即基元具有球对称性时晶体具有的对称性)——如果基元在某种不改变晶格的操作下会发生改变,那么整个晶体的对称性实际上低于晶格的对称性。
例如,金刚石和闪锌矿具有完全一样的晶格,但是金刚石基元具有沿着化学键中心的反射对称性,而闪锌矿则没有。 %TODO
因此,我们可以首先分析晶格的对称性,然后考虑基元的对称性而得到不考虑平移的晶体的对称性,最后加入平移(它是空间群的不变子群)而得到完整的对称性。

我们汇总一下所有的分类:
\begin{enumerate}
    \item \concept{晶系},这是最为粗疏的分类方法,其依据是晶格的对称轴和对称面的数目,共有7种晶系。晶系给定之后晶格的点群(见下文)可以不同,而可以验证,如果两种晶体分别属于不同的晶系,这两种晶体的点群肯定是不一样的。
    晶系提供了晶格的点群,是具有这种晶格的晶体可以具有的最大点群(如果基元对称性较低,实际的晶体点群会小于晶格的点群)。
    \item \concept{布拉伐晶格},在每种晶系内,按照一个晶体学单胞(默认8个角上都有格点)中各个格点的位置——简单,面心,体心,底心(共三种)——可以细分出总计14种布拉伐格子。%
    \footnote{
        角上的格点以外的格点不能位于简单、面心、体心、底心以外的地方,否则我们总是可以通过重新定义基元来把晶体划归到简单、面心、底心中的一类中去。
    }%
    虽然表面上应该有$7 \cdot 6 = 42$种格子,但其中很大一部分是重复的。
    在晶系给定之后,布拉伐格子额外提供了关于平移对称性的信息。
    因此,布拉伐格子概括了晶格的对称性:给定一个布拉伐格子类型,对应的晶格的空间对称群就可以写下来了。
    \item \concept{晶体点群}或者简称\concept{点群},虽然后者也常被用于指代一个更大的概念,指保持空间中一个点不动的群,实际上就是描述有限大小的物体的对称性的群。
    一个晶体群(即晶体的完整对称性,见下文)中除了平移以外的部分一定是$O(3)$的子群,可以写成某个坐标系下的一个矩阵群,矩阵变换下坐标原点始终是不动的,因此点群一定会保持一个点不动,这个点就是如前所述的原点;除此以外的其它点在某个群操作下都会发生变动。
    晶体点群一定是广义上的点群。然而,并非所有广义上的点群都是晶体点群,因为它们可能会和空间平移不兼容。
    一个晶体的点群一定是它的晶系的点群的子群。晶体点群和它的晶系的点群之间的差异表达了基元有多不对称。
    在晶系给定之后,晶体点群和布拉伐格子类型(简单、体心、面心、底心)可以任意组合,彼此无关。
    \item \concept{晶体群}或者说\concept{晶体空间群},最为细致的晶体对称性分类,它是欧氏空间$\mathbb{R}^n$的离散等距变换子群,并且$\mathbb{R} / G$是紧致的(即整个空间中任何一个点都可以等效到某个晶胞中)。
    数学上可以证明这样的群一定包含$n$种彼此独立的平移变换(本该如此,因为我们需要这些平移变换来产生完整的布拉伐格子),具体这些平移变换是什么由布拉伐格子的类型决定。
    在晶体点群和布拉伐格子均给定之后仍然不能完全确定晶体空间群,因为一些复式晶格的点群操作和平移操作一起使用时,部分点群操作并不在基元的点群中,然而,先做它们然后附带小的平移操作却能够保持晶体不变。
    实际上,这意味着“点群”的概念宏观和微观来看会有微妙的不同:典型的例子是金刚石和ZnS晶体,两者都是复式晶格,但是金刚石的几何上不等价的两种格点含有一样的原子,因此,虽然两种晶体的微观点群都是$T_d$,对金刚石,$O_h$点群中不是$T_d$的那些操作如果与一段很小的平移结合,仍然可以保持晶体不变,因此宏观地看,金刚石似乎在$O_h$点群下也不变。
\end{enumerate}

\subsubsection{点群中的变换}

我们来分析点群中可能出现的变换。
首先肯定有旋转,并且由于我们讨论的是离散对象上的对称群,旋转一定是离散旋转,即绕着某个轴,一次旋转$2\pi / n$,$n$为某个整数。
这样的旋转操作构成一个点群的子群,记作$C_{n}$,相应的转轴称为一条\concept{$C_{n}$轴};一个转角为$2\pi k / n$的离散旋转操作记作$c_n^k$。
其次肯定还有\concept{空间反演}或者说\concept{反射},这包括镜面反射和点反演。我们用$\sigma$标记一个反演平面,用$i$标记一个反演点。
在给定一条已知的旋转轴之后,将垂直于它的反演平面记作$\sigma_h$,将包含它的反演平面记作$\sigma_\nu$;如果给定两条对称轴而有一个反演平面正好平分这个角,那么这个反演平面就是$\sigma_d$。
这两个符号也用于指代对应的空间反演操作。
简单考虑一下会发现这些操作和它们的乘积足够给出所有可能的点群变换。

由于讨论范围局限于一个点群,总是存在一个点,在所有变换下都不变。
因此,一个点群内的旋转操作的旋转轴一定有共同的交点,这个交点就是反演中心,反演平面一定包含这个点。

空间反演和反射操作并非简单地被直积在一起——它们之间会有一些特殊的代数关系。
例如,很明显在给定一条旋转轴之后,$c_n^k$,$\sigma_h$和$\sigma_\nu$之间会有某些特殊的代数关系。
我们将$z$轴取在旋转轴上,则
\begin{equation}
    c_n^k = \pmqty{\dmat{ \cos(\frac{2\pi k}{n}) & -\sin(\frac{2\pi k}{n}) \\ \sin(\frac{2\pi k}{n}) & \cos(\frac{2\pi k}{n}), 0 }},
\end{equation}
而
\begin{equation}
    \sigma_h = \pmqty{\dmat{1, 1, -1}}, \quad i = \pmqty{\dmat{-1, -1, -1}},
\end{equation}
于是立刻可以看出
\begin{equation}
    c_n^k \sigma_h = \sigma_h c^n_k.
\end{equation}
在$n=2$时$C_2=\{e, c^1_2\}$,记$c^1_2$为$c_2$,则
\begin{equation}
    c_2 = \pmqty{\dmat{-1, -1, 1}},
\end{equation}
计算可得
\begin{equation}
    c_2 \sigma_h = \sigma_h c_2 = i, \quad c_2 i = i c_2 = \sigma_h, \quad \sigma_h i = i \sigma_h = c_2.
    \label{eq:sigma-c-i}
\end{equation}

假定有两个二阶轴$C_{2A}$和$C_{2B}$交于$O$点,它们之间的夹角为$\varphi$,则
\begin{equation}
    c_{2A} c_{2B} = c(2\varphi),
\end{equation}
其中$c(2\varphi)$的旋转轴垂直于$C_{2A}$和$C_{2B}$确定的平面。
这件事只需要通过
\begin{equation}
    c_{2B} = c(\varphi) c_{2A} c(-\varphi)
    \label{eq:ab-axis-phi-rotation}
\end{equation}
即可得到证明。类似地可以证明
\begin{equation}
    c_{2B} c_{2A} = c(-2\varphi).
    \label{eq:cb-ca-phi-axis}
\end{equation}

也可以对反射面证明类似的公式。设有两个反射面$\sigma_{\nu}$和$\sigma'_\nu$相交于$z$轴,以$z$轴为旋转轴,则有
\begin{equation}
    \sigma'_\nu \sigma_\nu = c(2\varphi), \quad \sigma_\nu \sigma'_\nu = c(-2\varphi).
\end{equation}
证明方法是,取和$\sigma_{\nu}$垂直的$C_2$轴$c_2$,和$\sigma'_\nu$垂直的$C_2$轴$c'_2$,则有
\[
    \sigma_\nu = c_2 i = i c_2, \quad \sigma'_\nu = c'_2 i = i c'_2,
\]
于是
\[
    \sigma'_\nu \sigma_\nu = c_2' c_2 = c(2\varphi).
\]

\eqref{eq:ab-axis-phi-rotation}其实说明一件事:一个旋转轴在特定的对称操作下可以被转换成另一个旋转轴。
我们先讨论这“特定的对称操作”是另一个旋转的情况。
设一个$C_n$轴和一个$C_m$轴相交于$O$点,两者夹角为$\varphi$,则可以获得这样的新的旋转:
\[
    c_n^j c_m^i (c_n^j)^{-1},
\]
因此通过绕着$C_n$轴的旋转可以产生$n$个$C_m$轴,记作$C_{m j}$,其中的操作为
\begin{equation}
    c_{m j}^i = c_n^j c_m^i (c_n^j)^{-1}.
    \label{eq:from-one-axis-to-another}
\end{equation}
特别的,如果有一个$C_2$轴与$C_m$轴垂直,有
\begin{equation}
    c_n^{-k} = c_2 c_n^k c_2.
    \label{eq:inverse-k-c2}
\end{equation}
可以验证,将旋转轴和$\sigma$面做共轭变换也能够得到新的旋转,即
\[
    \sigma c_m^i \sigma
\]
也是一个$C_m$操作,记为$C_m'$,并且可以证明$C_m$的正转被映射为了绕着$C_m'$的反转,即
\begin{equation}
    {c'_{m}}^{-k} = \sigma c_m^k \sigma.
\end{equation}
特别的,
\begin{equation}
    \sigma_\nu c_m^k \sigma_\nu = c_2 i c_m^k i c_2 = c_2 c_m^k c_2 = c_m^{-k},
\end{equation}
而对$\sigma_h$有
\begin{equation}
    c_n^k = \sigma_h c_n^k \sigma_h.
    \label{eq:perpendicular-sigma-c}
\end{equation}
看起来这里似乎有一个矛盾,因为等式左边的$k$应该是$-k$,但是在$\sigma$是$\sigma_h$时变换前后的旋转轴其实是同一根旋转轴,只不过指向相反,那么如果将它们看成同一根轴,就应该将$-k$换成$k$。
旋转操作在点反演的共轭变换下不变,即
\begin{equation}
    i c_m^k i = c_m^k.
\end{equation}
以上方程给出了$c_m^k$在各种共轭变换下的变换,从而给出了一个旋转操作的等价类。
给定一个旋转轴和它的旋转群,做这些共轭变换,会得到另一个旋转群,从而以上方程实际上指出了一个点群中的\concept{等价旋转轴}:对一条旋转轴$C_m$,如果有和它相交的旋转轴$C_n$,那么$C_m$中的等价旋转轴包括$C_{mj}$,$j$取遍$0$到$n-1$;如果有反演面,那么就有等价旋转轴$C_m'$;如果有与之垂直的$C_2$轴,那在$C_m$的旋转群内部就有群元彼此等价。
没有更多方式产生等价旋转轴。需注意两个轴等价不代表以它们为轴的所有旋转操作都在一个等价类中;这只能说明,某个绕一个轴的旋转一定和某个绕另一个轴的旋转等价。

也可以依照类似的方法分析反射变换的共轭等价类。对反演面,做旋转的共轭变换有
\begin{equation}
    \sigma' = c_n^k \sigma c_n^{-k}.
\end{equation}

\subsubsection{点群的分类}

本节给出所有可能的晶体点群,以及如何分类它们。

\paragraph{第一类点群} \concept{第一类点群}包括那些只有纯粹的转动,即没有反演的点群。它们也可以称为离散旋转群。
第一类点群中最简单的是\concept{单轴点群},即只有一条对称轴的点群,它们就是$C_n$。
$C_n$是$n$阶阿贝尔群,从而有$n$个不可约表示,每个不可约表示都是一维的。
晶体点群必须和离散空间平移对称性兼容,换而言之,做完晶体点群变换之后的晶格必须仍然是原来的晶格,格矢的指向不变。
正是这个条件导致$C_n$中的$n$不能任意取值。分子点群就不受到这个限制。
设$\vb*{t}$是一个长度最短的格矢(我们正是在这一步用到了“存在一个晶格”的假设,否则没有一个长度最短但是有限的格矢),我们以它的起点为中心施加一个点群中的转动$c_n^1$,则在转动$c_n^1$之后它必须变成另一个格矢,这等价于说$\vb*{t}' - \vb*{t}$是一个格矢。
我们有
\[
    \abs*{\vb*{t}' - \vb*{t}} = 2 \abs*{\vb*{t}} \sin\left(\frac{\pi}{n}\right),
\]
由于$\vb*{t}$长度最短,有
\[
    2 \abs*{\vb*{t}} \sin\left(\frac{\pi}{n}\right) \geq \abs*{\vb*{t}},
\]
于是$n$只能从下面的值中选取:
\[
    n = 1, 2, 3, 4, 6.
\]
我们会发现每个值都是可以的:$n=1$对应一个平庸的情况,即晶体没有什么特别的点群对称性;$C_2$和$C_4$可以和正方形晶格兼容,$C_3$可以和三角晶格兼容,$C_6$可以和蜂窝晶格兼容。
总之,晶体点群中,单轴点群只有$C_1, C_2, C_3, C_4, C_6$,这就是所谓的\concept{晶体局限定理}。
实际上,点群同样会选择能够与它搭配的晶格,晶格的形状不能完全任意搭配,一些点群只能和特定的一些晶格一起出现。

另一种第一类点群不包含或者只包含一个高阶轴,除此以外的旋转轴都是$C_2$轴。
如果有两个$C_2$轴,那么它们必须垂直,否则按照\eqref{eq:cb-ca-phi-axis}就会产生高阶轴(证明\eqref{eq:cb-ca-phi-axis}时我们是假定存在垂直两根$C_2$轴的轴,但是这没有关系,因为群乘法的封闭性保证了$c_{2A}$和$c_{2B}$相乘的结果总是在原来的点群中)。
在两根$C_2$轴垂直时,围绕它们旋转$\pi$的操作相乘给出绕着垂直于这两根轴确定的平面旋转$\pi$的操作。
因此我们得到一个点群$D_2$。无法再塞入更多的$C_2$轴了——三维空间中彼此相互正交的方向最多能有三个。因此,$D_2$就是只含有$C_2$轴但是和$C_n$不同的唯一一个第一类点群。

现在考虑一个只有一个$C_n$高阶轴的第一类点群,剩下的旋转轴只有$C_2$轴。这些$C_2$轴必须和$C_n$轴垂直,否则根据\eqref{eq:from-one-axis-to-another}将会有两条$C_n$轴。
反过来,通过$c_n^k$共轭变换,从一条$C_2$轴可以产生$n$个$C_2$轴。不能有更多的$C_2$轴了,否则将会出现另一条和$C_n$轴重合的高阶轴。
于是我们又得到了一个点群,它包含一条$C_n$轴和$n$条与之垂直,相邻夹角为$\pi / n$的$C_2$轴(注意这$n$条轴都是直线而不是射线,从而,相邻的$C_2$轴的夹角是$2\pi/2n$)。记这样的群为$D_n$,其阶为$2n$,其中有$C_n$轴给出的$n$个(包含了单位元的)操作和$n$个$C_2$轴给出的旋转$\pi$的操作。
$C_2$轴旋转和$C_n$轴旋转的复合给出的也是$C_2$轴旋转。
可以看出$D_n$实际上就是正$n$边形的对称群。
由于$C_n$中的$n$有限制,因此只有$D_1, D_2, D_4, D_6$能成为晶体点群。

这些$D_n$群的等价类划分可以如下做出。容易验证在$D_n$中,我们有
\[
    c_2 c_2' c_2 = c_2', \quad c_n^k c_2 c_n^{-k} = c_2', \quad c_2 c_n^k c_2 = c_n^{-k}, \quad c_n^k c_n^{k'} c_n^{-k} = c_n^{k'},
\]
因此$D_n$的一个等价类中要么都是$C_n$旋转,要么都是$C_2$旋转。上式直接给出$C_n$的分类:$c_n^{k}$和$c_n^{-k}$被分为一类。
我们发现$c_n^1 c_2 c_n^{-1}$和$c_2$的夹角为$2\pi/n$,即通过$c_n^k$对$c_2$做共轭变换,一次最少导致$2\pi/n$的方向变化。
这意味着,如果$n$是偶数,那么通过$c_n^k$对$c_2$做共轭变换只能够覆盖一半的$C_2$轴,因此全体$C_2$轴被分成两个等价类;而对奇数$n$通过$c_n^k$对$c_2$做共轭变换可以覆盖一半的$C_2$轴。
总之,对奇数$n$,等价类包括:单位元,全体$C_2$轴作为一个等价类,以及
\[
    (c_n^1, c_n^{n-1}), \ \  \ldots, \ \  (c_n^{(n-1)/2}, c_n^{(n+1)/2})
\]
共计$(n+3)/2$个;而对偶数$n$,等价类包括:单位元,两个包含了$C_2$轴的等价类,以及
\[
    (c_n^1, c_n^{n-1}), \ \  \ldots, \ \  (c_n^{n/2-1}, c_n^{n/2+1}), \ \ c_n^{n/2},
\]
共计$3+n/2$个。这些等价类的数目就是不可约表示的数目,而全体不可约表示的维数的平方和就是群的阶,于是不可约表示的维数确定如下:
\begin{itemize}
    \item $D_2$有$4$个$1$维表示;
    \item $D_3$有$2$个$1$维表示,$1$个$2$维表示;
    \item $D_4$有$4$个$1$维表示,$1$个$2$维表示;
    \item $D_6$有$4$个$1$维表示,$2$个$2$维表示。
\end{itemize}

最后我们考虑具有多个高阶轴的第一类点群。设有两根高阶轴相交于$O$点,分别为$C_m$轴和$C_n$轴,并且它们之间的夹角最小。
使用之前的套路,可以获得环绕在$C_n$轴周围的$n$根$C_m$轴,以及环绕在$C_m$轴周围的$m$根$C_n$轴。接着我们从中再次挑选出一对夹角最小的$C_m$轴和$C_n$轴,重复这个步骤。
晶体点群的有限性和封闭性意味着这种迭代最终会停止。一点几何上的直觉就告诉我们,这样产生的全体对称轴和一个以$O$为球心的球面的交点连接起来将会产生一个正多面体。
因此,具有多个高阶轴的第一类点群必须包括某个正多面体的全部旋转对称性。
由于正多面体的高度对称性,实际上无法向正多面体的旋转对称群中再塞入更多的旋转对称性而仍然得到一个离散旋转群。因此具有多个高阶轴的第一类点群一定是某个正多面体的旋转对称群。

于是我们首先分析正多面体的对称性。一共只有五种正多面体,其中正四面体的旋转对称群是$T$,正八面体和立方体的旋转对称群是$O$,正二十面体和正十二面体的旋转对称群是$I$,但是$I$含有一个$C_5$轴,因此不能作为晶体点群。
因此,具有多个高阶轴的第一类点群必须包括$T$或$O$。
$T$群是12阶群,具有四个$C_3$轴和三个$C_2$轴。可以验证$C_3$轴彼此等价,$C_2$轴彼此等价,等价类为$e$,全体$c_2$,全体$c_3^1$和全体$c_3^2$,共4个等价类,从而有3个1维不可约表示和1个3维不可约表示。
$O$群是24阶群,有三个$C_4$轴,四个$C_3$轴和六个$C_2$轴,容易验证所有$C_4$轴都是等价的,所有$C_3$轴和$C_2$轴也是等价的,并且每个高阶轴都有与之垂直的$C_2$轴,从而正转和反转相同角度的旋转操作也是等价的。
这样,等价类包括:单位元,全体$c_2$,全体$c_3^1$和全体$c_3^2$构成的单独一个类,全体$c_4^1$和全体$c_4^3$构成的一个类,全体$c_4^2$,共计5个等价类。

总之,我们得到了11种第一类点群:
\[
    C_1, C_2, C_3, C_4, C_6, D_2, D_3, D_4, D_6, T, O.
\]

\paragraph{第二类点群} 现在将反演操作加入。点群中的旋转部分组成一个正规子群,从而,第二类点群可以通过向第一类点群加入反演操作来得到。
实际上容易发现如果点群中包括$i$,那么$\{e, i\}$也是一个正规子群。
能够加入的反演操作包括面反演操作和中心反演操作。
向非平凡的旋转群加入面反演操作$\sigma$时,如果反射面和一个对称轴既不垂直也没有包含关系,则显然会产生一条新的对称轴,新的对称轴根据\eqref{eq:from-one-axis-to-another}又可以产生新的对称轴。
新的对称轴或是可以没完没了地产生,这样得到的就不是有限大小的点群了,或是在产生一定数目的新对称轴之后停下,此时$\sigma$或者必须能够让某个对称轴不变,或者让两个对称轴相互转化(几何上的考虑表明,不可能让多个对称轴做置换)。
如果反演操作总是将一个对称轴转化为另一个对称轴(并且与此同时让后者也变成前者),则考虑一族围绕在一个$C_m$轴周围的$m$个对称轴,它们均交于一点$O$,则$\sigma$反射面也必须经过$O$,则几何上的考虑意味着$\sigma$反射面或者包含$C_m$轴或者垂直于$C_m$轴。
而相对于那些围绕在$C_m$轴周围的那些对称轴而论,容易看出$\sigma$反射面可以是$\sigma_h$轴,可以是$\sigma_\nu$轴也可以是$\sigma_d$轴。
几何上的考虑表明,奇数$C_m$和$i$一起出现会引发矛盾,从而$i$只能出现在所有旋转轴都是偶数的点群中。
然而,此时,必定存在一个$c_2$操作,从而根据\eqref{eq:sigma-c-i},点群中一定存在一个镜面反演操作,且$i$可以用这个镜面反演和旋转复合给出。

总之,含有$i$的点群一定含有镜面反射,并且所有的镜面反射——无论对哪些轴而言——都或者是$\sigma_h$轴,或者是$\sigma_\nu$轴,或者是$\sigma_d$轴。
可以存在这样的情况:点群中有某个旋转反射的操作,但是这个旋转反射操作中的镜面操作不在点群中,但是
因此,我们接下来可以采用这样的方法获得所有第二类点群:给定一个第一类点群,可以向其中插入镜面反演操作和所谓\concept{旋转反射轴}(即将一个旋转轴的所有旋转操作都乘上一个镜面反演操作而得到的一系列操作,但是该旋转轴的纯旋转操作并不都在群内),并通过群乘法获得所有其它群元。
旋转反射覆盖了所有可能的“旋转和反演相乘”的操作,因为一个形如$c_m^k i$的操作一定可以写成一个形如$c_m^{k'} \sigma$的操作。

最简单的,向$C_1$群加入反演操作,将得到
\begin{equation}
    C_s = \{e, \sigma \}, \quad C_i = \{e, i \}
\end{equation}
两个群,各有两个一维不可约表示。我们无法加入更多的反演操作,因为两个反演操作的复合是一个旋转操作,从而产生了旋转轴,不再是简单的$C_1$群的扩充了。

然后尝试向$C_n$群加入反演操作。如果反演平面垂直于旋转轴,那么我们将获得
\begin{equation}
    C_{nh} = C_n \otimes \{\sigma_h, e\} \simeq C_n \otimes C_s.
\end{equation}
其结构是很简单的,包含$2n$个元素,且为阿贝尔群。注意此处我们讨论的点群都是具体的矩阵群而不是抽象的群结构,群乘法表同构的群我们不认为是同一个。
特别的,
\begin{equation}
    C_{1h} = C_s, \quad C_{2h} = \{e, c_2, \sigma_h, c_2 \sigma_h = i\} = C_2 \otimes C_i.
\end{equation}
如果反演平面包含旋转轴,那么我们将获得$C_{n \nu}$。在$C_{n \nu}$中由于$C_n$轴的作用,我们实际上可以获得$n$个$\sigma_\nu$平面,它们均包含$C_n$轴。
一个有趣的地方是$C_{n \nu}$实际上和$D_n$是同构的,因为$\sigma_\nu$和$C_2$地位等价。
实际上,注意到对多边形,$\sigma_\nu$和$C_2$是同一类操作,就能够发现这一点。
$C_{n \nu}$的不可约表示和等价类和$D_n$均一致。

此外尚可考虑同时向$C_n$加入(一圈共计$n$个)$\sigma_\nu$操作和$\sigma_h$操作,不过此时实际上有$n$个垂直于$C_n$轴的$C_2$轴,因为通过
\[
    c_2 = c_n^2 \sigma_\nu \sigma_h
\]
可以得到一根$C_2$轴,从而得到一根因此我们将这个情况留到$D_n$群的扩充中讨论。

以上得到的几种第二类点群都不包含旋转反射。我们现在考虑包含旋转反射的群。
旋转反射中的镜面反演操作必须是$\sigma_h$。如果是$\sigma_\nu$,那么
\[
    (c_n^{k'} \sigma_\nu) (c_n^k \sigma_\nu) = c_n^k (\sigma_\nu c_n^k \sigma_\nu) = c_n^{k'} c_n^k, 
\]
因此这根轴其实不是旋转反射轴:它的所有旋转操作都可以不乘上$\sigma$就出现在群中。类似的$\sigma$也不能是歪斜的。
因此旋转反射中的镜面反演操作必须是$\sigma_h$。
这又意味着
\begin{equation}
    c_n^k \sigma_h = \sigma_h c_n^k.
\end{equation}
旋转反射轴的$n$不能是奇数,因为对奇数阶的旋转操作,每一个操作的阶都是$n$,我们知道
\[
    (c_n^k \sigma_h) (c_n^{k'} \sigma_h) = c_n^{k+k'},
\]
而使用$c_n^{k+k'}$连续自乘可以得到所有的$C_n$操作,从而这根$S_n$轴并不是旋转反射轴——全部纯旋转操作都可以出现在点群中。
因此在晶体点群中的旋转反射轴只能是$n=4$或$n=6$的。$n$为偶数并不意味着所有$c_n^k \sigma_h$都应该放进点群中,否则,通过
\[
    (c_n^{k+1} \sigma_h) (c_n^k \sigma_h) = c_n^1
\]
就能够得到所有纯旋转操作,从而这根轴不是旋转反射轴。
因此我们需要决定哪些$c_n^k \sigma_h$操作需要放入点群中。肯定必须放入$k$为奇数的$c_n^k \sigma_h$,否则我们得到的其实是退化的$n/2$阶旋转反射操作。
另一方面,注意到,如果点群中出现了一个$k$为奇数的$c_n^k \sigma_h$操作,那么它的某个奇数次自乘将会给出$c_n^1 \sigma_h$,而使用$c_n^1 \sigma_h$的$k$次自乘就能够得到$c_n^k \sigma_h$。
因此我们得出结论:点群中的旋转反射操作包括有且只包括$k$为奇数的$c_n^k \sigma_h$操作,并且它们都可以用$c_n^1 \sigma_h$的自乘给出。
于是我们设
\begin{equation}
    s_n^k = (c_n^1 \sigma_h)^k,
\end{equation}
在$k$为奇数时它是$c_n^k \sigma_h$,在$k$为偶数时,注意到
\[
    (c_n^1 \sigma_h)^2 = c_n^2,
\]
我们有
\begin{equation}
    s_n^k = c_{n}^{k}.
\end{equation}
因此与$n$阶旋转反射轴伴随着还有一系列$C_{n/2}$操作。这应当不出乎意料,因为旋转反射操作不能自己构成一个群——它们的行列式都是$-1$,因此相乘之后会得到行列式为$1$的操作,从而必定是旋转。
由于$\sigma_h$和$c_n^k$是交换的,$c_n^k$彼此是交换的,我们发现$s_n^k$实际上是彼此交换的。

我们首先考虑一个生成元由一根$S_n$旋转反射轴提供的的第二类点群,将它记作$S_n$,它显然就是$s_n^1$连续自乘给出的,或者说是生成元为$s_n^1$的循环群。
我们只需要考虑$S_4$和$S_6$。简单的计算给出
\begin{equation}
    S_4 = \{e, s_4^1, c_2, s_4^3\},
\end{equation}
以及
\begin{equation}
    S_6 = \{e, s_6^1, c_3^1, i, c_3^2, s_6^5\}.
\end{equation}
这里的$i$来自
\begin{equation}
    s_6^3 = c_6^3 \sigma_h = c_2 \sigma_h = i.
\end{equation}
这些群都是阿贝尔群。可以看到旋转反射轴和旋转轴的代数性质是非常相似的。

很自然的问题是,有多根旋转反射轴的点群是什么。由于只有$S_4$和$S_6$两种旋转反射轴,而它们分别伴随$C_2$和$C_3$轴出现,如果一个第二类点群具有多根旋转反射轴,那么它也有多根普通旋转轴。
如果一个第二类点群具有两根$S_4$旋转反射轴,那么它具有两根位置相同的$C_2$旋转反射轴,即它含有$D_2$为子群。
类似的会发现,所有含有多根旋转轴的第二类点群都含有$D_{n}$群或是多面体群。
因此,一个含有旋转反射轴的第二类点群$G$一定包含一个第一类点群$K$,只需要将其中的一些旋转轴替换成旋转反射轴就可以了;任何一个旋转反射轴的空间位置都和$K$中的某个旋转轴的位置是重合的。
我们下面还是会使用“往第一类点群中塞入反演变换”来构造第二类点群,但是对每一种情况,都需要检查是否存在这样的旋转反射轴,它的$\sigma_h$操作本身不在群里(实际上是没有这样的情况的)。

对$D_n$群,由于只有一根$C_n$轴,也可以有一样的构造,让我们得到$D_{nh}$。
此时$n$根$C_2$轴可以和$\sigma_h$组合产生$n$个$\sigma_\nu$反射面。
使用和推导$D_n$群类似的论证,会发现$n$为偶数时这些$\sigma_\nu$反射面被分成两个等价类,而$n$为奇数时它们同属一个等价类。
除了$C_n$旋转,$n$个$C_2$旋转,$n$个$\sigma_\nu$反演面和一个$\sigma_h$反演面以外,$D_n$群中还有$C_n$旋转和$\sigma_h$组合形成的旋转反射操作$S_n$;请注意此时即使$k$为偶数,我们还是可以定义
\begin{equation}
    s_n^k = \sigma_h c_n^k,
\end{equation}
虽然在$S_n$群中只有奇数$k$才有意义;此外$n$也可以取$2$或是$3$。

我们现在可以给出$D_{nh}$群的等价类和不可约表示数目。如果$n$是奇数,那么$D_{nh}$的等价类包括:单位元,全体$C_2$轴作为一个等价类,如下共计$(n-1)/2$个的等价类:
\[
    (c_n^1, c_n^{n-1}), \ \  \ldots, \ \  (c_n^{(n-1)/2}, c_n^{(n+1)/2}),
\]
以及将以上所有等价类都乘上$\sigma_h$得到的新的等价类,这其中,$C_n$操作乘上$\sigma_h$就得到了那些$S_n$操作,而那些垂直于$C_n$轴的$C_2$轴乘上$\sigma_h$后就得到了(相对于$C_n$轴而言的)$\sigma_\nu$操作,它们相对于那些$C_2$轴一半是$\sigma_h$轴,一半是$\sigma_\nu$轴。
总之,合计共有$n+3$个等价类。$n$为奇数时可以验证$\sigma_h$和其它群元均对易,从而
\begin{equation}
    D_{nh} \simeq D_n \otimes C_h.
\end{equation}
偶数$n$的$D_{nh}$群的等价类可以使用类似的套路写出,它包括$D_n$的$3 + n/2$个等价类,和这些等价类乘上$\sigma_h$得到的新等价类,共有$n+6$个。
这里面$\sigma_h$和$C_2$轴相乘得到的轴相对于$C_2$轴来说都是$\sigma_h$轴。%
$n$为偶数时有群元$c_n^{n/2}$,它实际上是一个$c_2$转动,于是$i = c_n^{n/2} \sigma_h$是在群中的,它和其它所有群元对易,即我们有
\begin{equation}
    D_{nh} \simeq D_n \otimes C_i.
\end{equation}
这两种情况下,$D_{nh}$的结构都是一个第一类点群直积上$\{1, -1\}$,并且它们中的旋转反射轴都是那条$C_n$轴,没有更多旋转反射轴。

在$D_{nh}$中,垂直于$C_2$轴的那些$\sigma_h$面(相对于$C_n$轴是$\sigma_\nu$面)由$C_2$操作和相对于$C_n$轴的$\sigma_h$操作相乘得到,言下之意是,如果在一个包含$D_n$的第二类点群$G$中加入垂直于$C_2$的镜面反射面,那么立刻有一个垂直于$C_n$的反射面出现,从而接下来就会发现,$G$一定包含$D_{nh}$。
我们现在尝试构造一个不包含$D_{nh}$,但是还是包含$D_{n}$的第二类点群,那么插入$\sigma$反射面的选择就只有插入相对于$C_2$轴而言的$\sigma_d$反射面了(插入相对于$C_2$轴而言的$\sigma_\nu$在$n$为偶数时会产生$D_{nh}$群,在$n$为奇数时会产生新的旋转轴)。称这样的第二类点群为$D_{nd}$。
我们会发现此时$C_n$轴实际上变成了$S_{2n}$旋转反射轴:我们有
\begin{equation}
    s_{2n}^1 = \sigma_d c_2. 
    \label{eq:from-sigma-d-c-2-to-s-2n}
\end{equation}
$\sigma_d$轴的存在让所有$C_2$轴都在同一个等价类内;而反过来,$\sigma_d c_2 \sigma_d$会给出所有其它的$c_2$操作,即所有$C_2$轴都是在同一个等价类里面的。
$D_{nd}$群中不包含$\sigma_h$群,从而$S_{2n}$轴的存在本身就意味着$2n=4, 6$,即$n=2, 3$。
$D_{2d}$中的等价类包括单位元,$(s_4^1, s_4^3)$,转轴方向和$S_4$轴一致的那个$c_2$旋转,一个包含与$S_4$轴垂直的两个$c_2$旋转的等价类,还有两个$\sigma_d$组成的等价类,共计$5$个等价类,共有$8$个群元,从而有$4$个一维不可约表示和一个二维不可约表示。
$D_{3d}$与之类似,所有旋转反射操作同属一类,所有转轴和旋转反射轴一致的纯旋转同属一类,但是还有一个$i$操作,于是共有$6$个等价类。
计算可得有$12$个群元,不可约表示中有$4$个一维不可约表示和$2$个二维不可约表示。

我们现在看到,向$D_n$中塞入$\sigma_h$操作必然导致$C_n$轴变成旋转反射轴。
剩余的扩充$D_n$的方式有:向$D_{nh}$或$D_{nd}$塞入更多$\sigma$操作,实际上,这就是要构造一个同时具有$\sigma_h$和$\sigma_d$操作,而包含$D_{n}$为子群的群;又或者,可以尝试让$D_n$中的$C_2$轴能够提供旋转反射操作。
让$C_2$轴变成$S_n$这样的旋转反射轴肯定是不可能的,因为如前所述,只有$S_4$和$S_6$;而如果我们加入某个垂直于$C_2$轴的$\sigma_h$操作(它相对于$C_n$轴是$\sigma_\nu$),那么它和$C_2$旋转复合给出的将是$i$,如果$n$是奇数,矛盾就产生了,而如果$n$是偶数,那么实际上我们得到的将是一个$D_{nh}$群。
要求$\sigma_d$和$\sigma_h$同时存在同样会产生一个$D_{nh}$群,因为此时容易验证
\begin{equation}
    c_2 \sigma_d = i,
\end{equation}
因此$n$不能是奇数,否则会引发矛盾;但是在$n$是偶数时,所谓的$\sigma_d$实际上就是$D_{nh}$已有的那些$\sigma$面。
总之,除了$D_{nh}$和$D_{nd}$以外,不存在更多扩充$D_n$群的方法。

最后考虑多面体群。我们首先尝试向$T$群加入空间反演操作。为了让空间反演相对所有反射轴都或者是$\sigma_h$,或者是$\sigma_d$,或者是$\sigma_\nu$,我们只有两种可能:一种是
向$O$群加入空间反演操作只有一种可能:反演面垂直于$C_4$轴。
% TODO

因此现在我们得到了$21$种第二类点群:
\[
    C_i, C_s, C_{2h}, C_{3h}, C_{4h}, C_{6h}, C_{2v}, C_{3v}, C_{4v}, C_{6v}, S_4, S_6, D_{2h}, D_{3h}, D_{4h}, D_{6h}, D_{2d}, D_{3d}, T_d, T_h, O_h.
\]

至此,我们得到了全部32种点群,这包括10种第一类点群和22种第二类点群。
关于这些群的具体的群表、不可约表示、特征标表等可以查阅晶体学用表得到,我们在此处给出了从头计算它们需要的全部信息。

\subsubsection{7种晶系和14种布拉伐格子}\label{sec:lattice-group}

旋转操作和平移操作兼容的要求给出了晶体局限定理,反过来,点群也会挑选能够和它兼容的晶格。
任何一个晶格都必然在某个点群的作用下保持不变。如前所述,晶体的点群对称性总是小于它的晶格,或者说晶体的点群一定是它的晶格的点群的一个子群,因此,分析出晶格有哪些可能的对称性——或者说,\emph{晶格的}点群有哪些——就能够为晶体分类提供依据。

设$P$是某个晶格$T$(这里$T$就是格矢组成的集合,但是很显然,这个集合立刻就确定了晶体的离散平移对称性;之后我们也会用$T$表示晶格的离散平移群)的不需要和任何平移操作联合使用的点群,满足
\begin{equation}
    P = \{ \alpha | \alpha \vb*{r} = \vb*{r}' \in T \},
\end{equation}
显然$P$是32种点群中的一种。下面我们将推导出$P$需要满足的一些约束条件。

首先空间反演操作$i$肯定在$P$当中,理由是显然的。其次,如果$C_n \subset P, n \geq 2$,则一定有$\sigma_h \in P$。
这件事的证明如下:设$S$是一个垂直于$C_n$轴的晶面,设其上最短的格矢为$\vb*{a}_1$,设$\vb*{a}_2 = c_n^1 \vb*{a}_1$。
显然,$\vb*{a}_1$和$\vb*{a}_2$都是格矢。设$\vb*{a}_3$是一个和这两个格矢都线性无关的格矢,并做分解
\[
    \vb*{a}_3 = \vb*{u} + \vb*{v},
\]
其中$\vb*{u}$是平行于$C_n$的而$\vb*{v}$是垂直于$C_n$的。这样我们就有
\[
    c_n^1 \vb*{a}_3 - \vb*{a}_3 = (c_n^1 \vb*{u} - \vb*{u}) + (c_n^1 \vb*{v} - \vb*{v}) = c_n^1 \vb*{v} - \vb*{v} .
\]
可以确定$c_n^1 \vb*{a}_3 - \vb*{a}_3$一定在$T$中,而$c_n^1 \vb*{v} - \vb*{v}$一定在$S$中,从而$c_n^1 \vb*{v} - \vb*{v}$一定是位于$S$上的某个格点,从而能够找到整数$n_1$和$n_2$使得
\[
    c_n^1 \vb*{v} - \vb*{v} = n_1 \vb*{a}_1 + n_2 \vb*{a}_2.
\]
我们在两边作用$c_n^{-1}$,得到
\[
    \vb*{v} - c_n^{-1} \vb*{v} = n_1 c_n^{-1} \vb*{a}_1 + n_2 \vb*{a}_1.
\]
两式相减,得到
\[
    c_n^1 \vb*{v} + c_n^{-1} \vb*{v} - 2 \vb*{v} = n_2 \vb*{a}_2 - n_1 c_n^{-1} \vb*{a}_1 + (n_1 - n_2) \vb*{a}_1.
\]
注意到
\[
    c_n^1 \vb*{v} + c_n^{-1} \vb*{v} = 2 \cos\frac{2\pi}{n} \vb*{v}, \quad c_n^1 \vb*{a}_1 + c_n^{-1} \vb*{a}_1 = \vb*{a}_2 + c_n^{-1} \vb*{a}_1 = 2 \cos\frac{2\pi}{n} \vb*{a}_1,
\]
从而
\begin{equation}
    2\left( \cos\frac{2\pi}{n} - 1 \right) \vb*{v} = \left( n_1 - n_2 - 2 n_1 \cos \frac{2\pi}{n} \right) \vb*{a}_1 + (n_1 + n_2) \vb*{a}_2.
\end{equation}
现在我们尝试构造一个$\sigma_\nu$。如果选取$\sigma_\nu$与$\vb*{a}_1$垂直,那么,只需要证明$\vb*{a}_2$和$\vb*{a}_3$在$\sigma_h$作用下也是一个格矢,就足够证明晶格的点群中有$\sigma_\nu$。
对$n=3, 4, 6$验证,均可发现以上说法正确。因此的确$\sigma_h \in P$。

满足这两个条件的点群共有7个,并且我们马上会看到它们是什么晶格的点群。
我们说,这7个点群标记了\concept{7个晶系},因为如\autoref{sec:crystal-structure-intro}所说那样,晶系是按照晶体中的对称轴和对称面数目划分的,实际上就是按照晶格的点群划分的。
我们下面列出它们和以它们为点群的晶格(以下提到“三个基元格矢”时指的都是六面体晶体学单胞的基元格矢):
\begin{itemize}
    \item $C_i$给出\concept{三斜晶系},三个基元格矢之间没有什么特别明确的关系:它们的长度不一致,并且彼此不垂直,仅有的对称性就是空间点反演对称性;
    \item $C_{2h}$给出\concept{单斜晶系},三个基元格矢长度不同,但是它们的三个夹角中有两个是\SI{90}{\degree},从而晶体学单胞是一个长方体朝着某条棱的方向受剪切力而形成的形状,或者说是一个底面为边长不相等的直棱柱,$C_{2h}$中的$c_2$操作的转轴平行于棱,$\sigma$操作的平面就是这个直棱柱的底面;
    \item $D_{2h}$给出\concept{正交晶系}或是\concept{斜方晶系},三个基元格矢的长度不同,但是全部彼此正交,晶体学单胞是一个三条棱各不相同的长方体,$D_{2h}$群的作用方式是显然的;
    \item $D_{4h}$给出\concept{四方晶系},其晶体学单胞是一个正四棱柱,$D_{4h}$的作用方式也是显然的;
    \item $D_{3d}$给出\concept{三角晶系},这个晶系又可以分成两种类型,其中一种的晶体学单胞是一个棱长全部相同的,三个基元格矢之间夹角相同的平行六面体,$D_{3d}$的$C_3$轴就是它的对角线;另一种的晶体学单胞也是一个平行六面体,其中一个轴垂直于另外两个,它就是$D_{3d}$的$C_3$轴,那另外两个轴的夹角是\SI{120}{\degree};
    \item $D_{6h}$给出\concept{六角晶系},为了更好地展现其对称性,它的晶体学单胞是一个正六棱柱的$1/3$,横截面为边长相等,其中一个角为$\SI{60}{\degree}$的平行四边形,$D_{6h}$的作用方式也是显然的,六角晶系的格子和三角晶系的第二种格子完全一样,但是前者有$C_6$轴而后者只有$C_3$轴;
    \item $O_h$给出的\concept{立方晶系},顾名思义,其晶体学单胞是一个立方体,并且存在空间镜面反演操作,因此其点群自然就是$O_h$。
\end{itemize}
这个列表给出了$P$的所有可能取值。

从这个列表也可以看出“晶系”一词的定义其实有些微妙之处。
前面使用的根据点群给出的晶系定义会将三个基元格矢长度相同,彼此之间夹角相同的平行六面体组成的晶格和正六棱柱组成的晶格都和$D_{3d}$匹配,而这两种晶格的几何形状是不同的,且三角晶系和六角晶系共用一种格子。
如果根据格子的形状划分晶系,那么应该将“三角晶系”一词用于描述那种三个基元格矢长度相同,彼此之间夹角相同的平行六面体组成的晶格,而“六角晶系”则用于描述那种正六棱柱组成的晶格,此时六角晶系可能具有两种点群,并且“不同晶系的晶体的点群不同”不再成立。
无论使用哪种定义都有7种晶系。为了避免混乱,本文将三个基元格矢长度相同,彼此之间夹角相同的平行六面体组成的晶格称为\concept{三角格子},而将正六棱柱组成的晶格称为\concept{六方格子}。

需要注意的是,晶系给定后晶格形状并不能够唯一确定,因为一个晶体学单胞中可以有多个原胞,从而,除了角上的格点以外,晶体学单胞的面、体内也可以有格点。
这些格点在晶格的点群的作用下必须被变换到其它格点上。
能够验证,或者没有这些面、体上的格点(称为简单格子),或者只有体心、面心、底心三种情况。
一些体心、面心、底心的情况在晶格的点群的作用下无法被变换到其它格点上,一些情况则可以归并入另一些情况,具体来说:
\begin{itemize}
    \item 三斜晶系只有简单晶格,其它情况均为简单晶格。
    \item 底心单斜格子的“底”要是长方形面,否则就是简单单斜格子。体心和面心单斜格子满足$C_{2h}$对称性,但是实际上就是底心单斜格子。
    \item 底心四方格子的“底”如果是正方形,拿显然就是简单四方格子,如果是长方形,则和$D_{4h}$群矛盾,因为所有长方形面都是等价的。面心四方格子就是体心四方格子。
    \item 底心三角格子不满足$D_{3d}$对称性,因为$D_{3d}$作用在三角格子上会让各个面都等价。面心和体心三角格子% TODO
    \item 底心六方格子不满足$D_{6h}$对称性,因为% TODO
    \item 底心立方格子不满足$O_h$对称性,因为$O_h$要求各个面都要等价,实际上它是简单四方晶格。
\end{itemize}
推导结果是,总共有14种可能的晶格,即布拉伐格子一共只有14种。

\subsubsection{空间群的结构和分类}

将一个空间群$G$中出现的所有的操作的点群部分都提取出来显然构成了一个群$K$,称为\concept{$G$的点群};空间群的点群\emph{未必}是空间群的子群,因为其中的一些点群操作也许并不能单独出现在空间群中。然而,$K$一定是$P$的子群,从而$K$中的旋转轴方向、反射面位置等都受到很强的限制,因为它们必须和晶格匹配。
$K$对应宏观的晶体的点群对称性,因为点群操作加上一个小的平移即可成为空间群的元素,而小的平移宏观上是看不出来的。
我们称这个小的平移操作为\concept{分数平移},因为它不在离散平移群中,不能将它写成基元格矢的整数倍;分数平移可以加上一个任意的格矢,以下不失一般性,我们要求分数平移的长度取最小值。
这个空间群中单独出现的点群操作(对应微观的基元的对称性)构成的那个点群是空间群的点群$K$的子群,空间群的点群$K$又是其晶系的点群$P$的子群。

\paragraph{空间群的陪集分解} 我们首先分析空间群中会出现什么元素,以及其群结构。空间群$G$可以根据其不变子群$T$做陪集分解,其中的$\vb*{\tau}_i$约束为分数平移:
\begin{equation}
    G = \sum_{i=1}^n \{ \alpha_i | \vb*{\tau}_i \} T, \quad \alpha_i \in K.
    \label{eq:coset-space-group}
\end{equation}
显然其中的某个$\{\alpha_i | \vb*{\tau}_i \}$就是单位元,不妨指定它为$i=1$的情况:
\begin{equation}
    \{\alpha_1 | \vb*{\tau}_1 \} = \{ e | 0 \}.
\end{equation}
对每个$\alpha_i$,我们可以找到一个唯一的$\vb*{\tau}_i$,理由如下:设
\[
    \{\alpha_i | \vb*{\tau}_i \} T = \{ \alpha_i | \vb*{\tau}'_i \} T,
\]
则
\[
    \{\alpha_i | \vb*{\tau}_i \} \{ e | \vb*{r}_n \} = \{\alpha_i | \vb*{\tau}'_i \} \{ e | \vb*{r}_{n'} \},
\]
即
\[
    \vb*{\tau}_i + \alpha_i \vb*{r}_n = \vb*{\tau}_i' + \alpha_i \vb*{r}_n'.
\]
这就意味着
\[
    \vb*{\tau}_i - \vb*{\tau}_i' = \alpha_i (\vb*{r}_n - \vb*{r}_n') \in T,
\]
因为$\alpha_i$肯定在晶格点群$P$中,因为它必定将格矢$\vb*{r}_n - \vb*{r}_n'$变换为另一个格矢。
然而,如前所述,$\vb*{\tau}_i$和$\vb*{\tau}_i'$都是分数格矢,并且已经取为最短,它们的差不可能是有限大小的格矢,因此只能有
\[
    \vb*{\tau}_i - \vb*{\tau}_i' = 0.
\]
这就证明了$\vb*{\tau}_i$的唯一性。请注意前述推导并没有用到$\alpha_i$的编号$i$——实际上,对不同的$i$和$j$,$\alpha_i$和$\alpha_j$也不会重复,因为如果有两个不同的$\{\alpha_i | \vb*{\tau}_i \}$和$\{\alpha_j | \vb*{\tau}_j \}$,而$\alpha_i = \alpha_j$,那么$\vb*{\tau}_i - \vb*{\tau}_j$给出了空间群中的一个平移操作,然而,$\vb*{\tau}_i$和$\vb*{\tau}_j$都是分数平移,它们之差也是,而分数平移是不会出现在空间群中的。这就导致一个矛盾。

不过,这并不是说\emph{整个空间群中}都不存在$\alpha$相同而$\vb*{\tau}$不同的多个$\{ \alpha | \vb*{\tau} \}$操作,因为显然可以通过空间平移获得无数多个点群部分为$\alpha$的操作:我们知道
\[
    \{e | \vb*{R}\} \{\alpha | \vb*{\tau}\} \in G, \quad \vb*{R} \in T,
\]
这就是说
\[
    \{\alpha | \vb*{\tau} + \vb*{R} \} \in G.
\]
因此我们有无数多个$\{ \alpha_i | \vb*{\tau} \}$形式的操作。

空间群$G$的点群$K$中的操作不会出现在$T$中。由于$\alpha_i$是点群操作,它们$\alpha_i$都来自$K$。而由于$\alpha_i$如前所述不会重复,$K$中的每一个操作都给出一个(且只有一个)$\alpha_i$。
因此我们可以用这样的方式推导出所有的空间群:我们首先指定晶体的$K$点群,然后,给这个$K$点群中的每一个操作$\alpha_i$都指派一个唯一的$\vb*{\tau}_i$,并保证群封闭性、有限性、与某种晶格匹配等条件成立,这样我们就得到了所有的空间群。
实际上,注意到
\[
    G / T = \{ \{\alpha_i | \vb*{\tau}_i \} T | \alpha_i \in K \},
\]
其中,群乘法
\[
    (\{\alpha_i | \vb*{\tau}_i \} T) (\{\alpha_j | \vb*{\tau}_j \} T) = \{\alpha_k | \vb*{\tau}_k \} T
\]
能够推出
\[
    \alpha_i \alpha_j = \alpha_k, \quad \vb*{\tau}_k = \alpha_i \vb*{\tau}_j + \vb*{\tau}_i + \vb*{R}, \quad \vb*{R} \in T,
\] 
于是实际上我们可以写出
\begin{equation}
    G / T \simeq K,
\end{equation}
从而空间群$G$和它的$K$点群同态,以及群乘法规则
\begin{equation}
    (\{\alpha_i | \vb*{\tau}_i \} T) (\{\alpha_j | \vb*{\tau}_j \} T) = \{\alpha_i \alpha_j | \vb*{\tau}_i + \alpha_i \vb*{\tau}_j \mod{T} \} T,
\end{equation}
或者我们其实可以略去$T$以简化书写。

\paragraph{空间群中的元素} 表面上看$\vb*{\tau}_i$的指派是很任意的,但是实际上不然,因为有很多限制。
首先,如此构造时需要注意去除一些重复,因为我们可以随意从欧几里得群中找一个行列式为正(这点很重要,因为螺旋轴什么的是有手性的)的元素出来,对空间群做共轭变换,变换前后的空间群是等价的,应该看成一个。
例如,我们可以用$\{e | \vb*{d}\}$做共轭变换,这会导致
\[
    \{e | \vb*{d}\} \{ \alpha | \vb*{\tau} \} \{ e | \vb*{d} \}^{-1} = \{ \alpha | \vb*{\tau} + (1 - \alpha) \vb*{d} \}.
\]
实际上这就是把坐标平移了一下:设$\vb*{r}' = \vb*{r} + \vb*{d}$,则
\[
    \vb*{r} \longrightarrow \alpha \vb*{r} + \vb*{\tau}
\]
被转化为了
\begin{equation}
    \vb*{r}' \longrightarrow \alpha \vb*{r}' + \vb*{\tau} + (1 - \alpha) \vb*{d}.
    \label{eq:conju-alpha}
\end{equation}
我们也可以不局限在平移上,比如说可以用$\{\alpha | \vb*{d}\}$做共轭变换。
因此,对空间群中的纯点群操作,可以通过适当的平移让它具有一个非零的$\vb*{\tau}$,或者当然也可以反过来让一个非零$\vb*{\tau}$变成零;不过,无论怎么做共轭变换,$\{e | 0\}$一直是不变的。

其次,$\vb*{\tau}$的取值也受到限制。对$\{c_n^1 | \vb*{\tau}\}$,用$\vb*{\tau}_\parallel$表示$\vb*{\tau}$平行于$c_n^1$转轴的分量,用$\vb*{\tau}_\bot$表示$\vb*{\tau}$垂直于$c_n^1$的转轴的分量,我们可以验证
\[
    \{c_n^1 | \vb*{\tau}\}^n = \{ c_n^n | n \vb*{\tau}_\parallel + (c_n^1 + c_n^2 + \cdots + c_n^n) \vb*{\tau}_\bot \} \in G,
\]
由对称性
\[
    (c_n^1 + c_n^2 + \cdots + c_n^n) \vb*{\tau}_\bot = 0,
\]
于是$n \vb*{\tau}_\parallel \in T$。这就是说,如果$\vb*{\tau}$不是零,那么$c_n^1$的轴指向必须和某个正格矢一样(这件事本身就对旋转轴的选择施加了很大的限制,因为正如\autoref{sec:lattice-group}中展示的那样,晶格就只有那么几种,而在每个晶格上旋转轴的方向都是确定的;反过来,后面$\vb*{R}$的选择就是根据\autoref{sec:lattice-group}中给出的旋转轴方向自动确定的),且设这个方向上最短的正格矢为$\vb*{R}$,则存在整数$m, n$使得
\begin{equation}
    \vb*{\tau}_\parallel = \frac{m}{n} \vb*{R}, \quad m = 0, 1, \ldots, n-1.
    \label{eq:tau-parallel-and-r}
\end{equation}
请注意$\vb*{\tau}_\parallel$是分数平移这件事对$\vb*{R}$也施加了一个限制:在$\vb*{\tau}$非零时,如果$\vb*{R}$很大,那么即使$m=1$,也不能得到合理的$\vb*{\tau}_\parallel$。

$\vb*{\tau}_\bot$的意义还没有分析。如果$\alpha_i$是$c_n^1$操作(只需要考虑$c_n^1$,因为只需要考虑给生成元指派的$\vb*{\tau}$),那么\eqref{eq:conju-alpha}中的$1 - \alpha$是
\begin{equation}
    1 - \alpha = \pmqty{\dmat{1 - \cos \theta & - \sin \theta \\ \sin \theta & 1 - \cos \theta, 0}}, \quad \theta = \frac{\pi}{6}, \frac{\pi}{4}, \frac{\pi}{3}, \frac{\pi}{2}, \pi.
\end{equation}
这个矩阵的秩是$2$,因此$(1 - \alpha) \vb*{d}$的取值范围不是三维的,而是二维的,并且容易看出其取值范围是$xy$平面,或者说是垂直于$C_n$转轴的那个平面。
因此可以通过适当的平移共轭变换让与$c_n^1$匹配的$\vb*{\tau}$的垂直于转轴的部分取任何的值;而$\vb*{\tau}_\parallel$则由\eqref{eq:tau-parallel-and-r}确定,\emph{不能}通过平移共轭变换改变。
因此,无论怎么平移,$\vb*{\tau}_\parallel$——或者说$m$——都是固定不变的。
没有必要对$G$中的所有操作都做对应的平移共轭变换,否则其它的$\vb*{\tau}_i$会变化,但以上说法意味着我们可以用一种几何意义更加明显的方法确定$\{c_n^1 | \vb*{\tau}_i\}$:给定$n$,在垂直于$C_n$转轴的平面上的一个点(这个点确定了$C_n$转轴的空间位置),以及\eqref{eq:tau-parallel-and-r}中的$m$,就唯一确定了一个$\{c_n^1 | \vb*{\tau}_i\}$,反过来,给定$\{c_n^1 | \vb*{\tau}_i\}$,用$c_n^1$和$\vb*{\tau}_\parallel$可以确定$n$,$m$,而$\vb*{\tau}_\parallel$则给出转轴位置。
实际上,$- c_n^1 \vb*{\tau}_\bot$一定是转轴上的一个点,据此和$c_n^1$的位置就确定了转轴。
容易看出$c_n^i$的转轴位置和$c_n^1$是一样的。
关于转轴位置还值得多说几句。我们知道可以在\eqref{eq:conju-alpha}中让$\vb*{\tau}$取任意一个格矢而得到无穷多个$\{c_n^1 | \vb*{\tau}\}$操作,而$- c_n^1 \vb*{\tau}_\bot$是转轴上的一个点,则取几个简单的情况(如$n=2$)就会发现实际上一个晶胞内可以有\emph{多个}平移操作增生的旋转轴。
在点群中不会有多条不相交而彼此平行的旋转轴,但是在空间群中,由于空间平移对称性,就可以有。

由于$\vb*{\tau}_\parallel$在平移下不变,我们称$\vb*{\tau}_\parallel$不为零的那些旋转轴为\concept{螺旋轴},因为它们给出螺旋线的对称性:做完旋转之后还需要沿着旋转轴做一小段平移。

类似的,对$\{\sigma | \vb*{\tau} \}$,设$\vb*{\tau}_\parallel \parallel \sigma$而$\vb*{\tau}_\bot \bot \sigma$,则能够证明,$\sigma$平行于某个正格矢$\vb*{R}$,并且
\begin{equation}
    \vb*{\tau}_\parallel = 0, \quad \frac{1}{2} \vb*{R}.
\end{equation}
$\{i | \vb*{\tau}\}$的$\vb*{\tau}$没有特殊的限制,然而由于$i$操作肯定可以用其它两个群元乘出来,它的$\vb*{\tau}$也是可以被别的操作确定的。

如果$\alpha_i$是$\sigma$操作,\eqref{eq:conju-alpha}中的$1- \alpha$就是$1 - \sigma$,从而
\[
    (1 - \alpha) \vb*{d} = 2 \vb*{d}_\bot,
\]
从而可以通过适当的平移去掉$\vb*{\tau}$中垂直于$\sigma$的分量,但是不能改变$\vb*{\tau}_\parallel$。与旋转轴类似,我们不去真的做平移,而是使用$\sigma$的法向量的指向,$\sigma$面上的一个点(它和$\sigma$的法向量的指向一起确定了$\sigma$面的位置)和$\vb*{\tau}_\parallel = 0, \vb*{R}/2$确定\eqref{eq:coset-space-group}中的一个$\{ \alpha_i | \vb*{\tau}_i \}$,反过来,给定$\{ \alpha_i | \vb*{\tau}_i \}$,$\vb*{\tau}_\bot$给出了反射面的位置。
由于$\vb*{\tau}_\parallel$在平移下是不变的,我们称$\vb*{\tau}_\parallel = \vb*{R} / 2$的情况为\concept{滑移面}。
根据\autoref{sec:lattice-group}中的分析,可以发现,在所有情况下,$\sigma$面通常或者是原胞的底面,或者是对角面等,总结起来可以分成这么几种:
\begin{itemize}
    \item \concept{轴向滑移面},垂直于$\vb*{a}, \vb*{b}, \vb*{c}$中的任意一个,下面不失一般性地设为$\vb*{c}$。由于这个滑移面对应的反射操作必定是晶格的对称性,$\vb*{c}$必须要垂直于$\vb*{a}, \vb*{b}$,从而$\vb*{a}$和$\vb*{b}$均在滑移面内。
    \item % TODO
\end{itemize}

对旋转反射轴指定非零的$\vb*{\tau}_i$没有意义,因为此时\eqref{eq:conju-alpha}中的$1 - \alpha$是
\[
    1 - \alpha = \pmqty{ \dmat{ 1 - \cos \theta & \sin \theta \\ - \sin \theta & 1 - \cos \theta, 2 } },
\]
这里我们把$z$轴取成$s_n^1$轴。$1 - \alpha$的行列式在$S_4$和$S_6$轴下都不是零,因此实际上可以通过做平移让$\vb*{\tau} = 0$,从而不需要像旋转轴一样,同时指定转轴位置和$\vb*{\tau}_\parallel$,对旋转反射轴只需要指定其位置就完全确定了它。

实际上,我们只需要给$K$点群的每个生成元——旋转,镜面反射,旋转反射操作——指派一个$\vb*{\tau}$,如果能够保证这样会产生一个合理的空间群,那就确定了一个空间群。

\paragraph{空间群的分类和推导} 如果在适当的坐标系下,陪集分解\eqref{eq:coset-space-group}中没有出现分数平移,则此时的空间群称为\concept{简单空间群}或者说\concept{点式空间群}。
因此,在原胞中至少可以找到一个点(即坐标原点),使得简单空间群$G$对它的作用和$G$的点群$K$完全一样。
点式空间群就是它的$K$点群和平移群$T$的半直积;请注意此处$K$和$T$必须要兼容,即$K$要是$T$的点群的子群。
需要注意点式空间群中也是可以有螺旋轴和滑移面的,例如,最简单的,如果某个旋转轴正好沿着某个格矢,那么把这个旋转操作和格矢的平移操作乘起来总能够得到一个螺旋操作。稍微复杂一些,设有一个和$C_n$轴不平行的格矢$\vb*{R}$,则
\[
    \{c_n^1 | 0\} \{ e | \vb*{R} \} = \{ c_n^1 | c_n^1 \vb*{R} \} = \{ c_n^1 | \vb*{R}_\parallel + c_n^1 \vb*{R}_\bot \}
\]
也在空间群中,由于$c_n^1 \vb*{R}_\bot$不为零,实际上我们获得了一根\emph{不通过}$K$点群的不动点的螺旋轴!
但是无论如何,对简单空间群,总是可以不使用它们做陪集分解\eqref{eq:coset-space-group}。

点式空间群的推导是比较平凡的。

\concept{非点式空间群}或者说\concept{非简单空间群}顾名思义是那些在陪集分解\eqref{eq:coset-space-group}中必须包含一些非零$\vb*{\tau}_i$的空间群。
然而应当注意到,任何一个点群$K$与和它兼容的晶格放在一起都能够产生简单空间群(只需要机械地做半直积即可),设我们有一个非简单空间群,我们把它的所有$\vb*{\tau}_i$都重新设置为零,总能够得到一个合法的简单空间群。
因此可以把这个过程反过来,对每个简单空间群,我们尝试向它的一些旋转或是反射操作中按照前面说过的要求——保证旋转轴和某个格矢方向相同,$\vb*{\tau}_\parallel$取特定的值等等——引入非零$\vb*{\tau}$,去掉会导致自相矛盾的情况,去掉重复的情况,最终就能够得到所有的非简单空间群。
实际上,这个过程还可以简化,因为$\vb*{\tau}_\bot$实际上同样是高度受限的。
从之前的推导可以看出,$\vb*{\tau}_\bot$可以确定各个旋转轴和反射平面的位置,而点群的乘法关系实际上已经将一个点群的空间群的旋转轴和反射平面的位置高度限制了,因为使用。% TODO:为什么单单依靠点群就能够确定各个旋转轴的相对位置??
因此到最后,调整$\vb*{\tau}_\bot$实际上就是对空间群中的操作做了一个整体的平移,从而和完全没有做平移彼此等价。
因此,我们只需要尝试向点式空间群导入$\vb*{\tau}_\parallel$就够了。这正是熊夫利最初推导空间群的方法。

\subsubsection{二维材料}

% TODO:17种墙纸群
二维材料的晶体结构分析起来要简单很多。可以采用和分析三维晶体的点群和空间群类似的方式做分析,也可以考虑三维晶体的点群和空间群的退化。

\subsubsection{记号和常用术语}

至此我们已经完全推导并分类了晶体的空间对称性。本节介绍一些非常常见的晶体结构的“俗名”,以及一种能够让有经验的人一眼看出点群和空间群中包含哪些对称性操作的记号。

\paragraph{常用结构的俗名} \concept{最密堆积}指的是一系列占据体积的球堆在一起,最为节省空间的排列方式。最密堆积给出一些晶格,可以分成ABAB结构和ABCABC结构。

\paragraph{点群的国际符号}

\paragraph{空间群的国际符号}

TODO:给定一个晶格,如何判断其类型;一种可能的方法:先尝试得到一个重复性单元,然后分析其对称性。如金刚石可以分解出一个立方体单元,然后发现它没法画成简单立方

fcc, bcc, hcp, NaCl, CsCl, diamond, zinc blend, wurtzite, perovskite


\subsection{静态晶格的物理效应}

\subsubsection{衍射}

设一束电磁波以波矢$\vb*{k}$照射在晶体上。本节分析晶格中的原子造成的衍射。
通常,只有X射线衍射才能够观察到这一类的现象,因为可见光波长远远大于晶格常数。

单个电荷的散射为
\begin{equation}
    \vb*{E} = -\frac{1}{4\pi \epsilon_0} \frac{e^2 E_0}{r} \vb*{e} \times (\vb*{e} \times \vb*{e}_0) \ee^{\ii \vb*{k} \cdot \vb*{R}}, \quad \vb*{R} = \vb*{r} - \vb*{r}'.
\end{equation}
晶体通常不会致密到散射电磁波被多次散射,于是我们假定电磁波只被散射一次,散射波也不会对晶体内部的入射波有非常显著的修正。
这样,散射场就是
\begin{equation}
    E_\text{scatter} = -\frac{1}{4\pi \epsilon_0} \frac{e^2 E_0}{r} \vb*{e} \times (\vb*{e} \times \vb*{e}_0) \int \dd[3]{\vb*{R}} n(\vb*{r}') \ee^{\ii \vb*{k} \cdot \vb*{R}}.
\end{equation}
在无穷远处散射波肯定是球面波,$\vb*{r}$和$\vb*{R}$相差不会太大,且波矢$\vb*{k}$和$\vb*{r}$方向一致。
在固定$r$大小、电场强度等不动,而仅仅改变$\vb*{k}$指向时,我们可以用波矢$\vb*{k}$来标记散射振幅:
\begin{equation}
    \begin{aligned}
        A(\vb*{k}) &\propto \ee^{\ii k R} \int \dd[3]{\vb*{r}'} n(\vb*{r}') \ee^{- \ii \vb*{R} \cdot \vb*{k}_0} \ee^{- \ii \vb*{r}' \cdot \vb*{k}} \\
        &\propto \int \dd[3]{\vb*{r}'} n(\vb*{r}') \ee^{- \ii \vb*{r}' \cdot (\vb*{k} - \vb*{k}_0)}.
    \end{aligned}
    \label{eq:scatter-amplitude-x-ray-original}
\end{equation}
因子$\ee^{- \ii \vb*{k}_0 \cdot \vb*{R}}$是因为入射光打到不同的位置的相位不同。
在晶体中$n(\vb*{r}')$由周期性的原胞给出,即
\begin{equation}
    n(\vb*{r}') = \sum_{\vb*{R}_i} n_\text{u.c.}(\vb*{r}' - \vb*{R}_i),
\end{equation}
代入\eqref{eq:scatter-amplitude-x-ray-original},就得到
\[
    \begin{aligned}
        A(\vb*{k}) &\propto \sum_{\vb*{R}_i} \int \dd[3]{\vb*{r}'} n_\text{u.c.}(\vb*{r}' - \vb*{R}_i) \ee^{-\ii \vb*{r}' \cdot (\vb*{k} - \vb*{k}_0)} \\
        &= \sum_{\vb*{R}_i} \int \dd[3]{\vb*{r}'} n_\text{u.c.}(\vb*{r}') \ee^{-\ii (\vb*{k} - \vb*{k}_0) \cdot (\vb*{r}' + \vb*{R}_i)} \\
        &= \sum_{\vb*{R}_i} \ee^{-\ii (\vb*{k} - \vb*{k}_0) \cdot \vb*{R}_i} \int_{V_{\text{u.c.}}} \dd[3]{\vb*{r}'} n_\text{u.c.}(\vb*{r}') \ee^{-\ii (\vb*{k} - \vb*{k}_0) \cdot \vb*{r}'} \\
        &= N \sum_{\vb*{G}_m} \delta_{\vb*{k} \vb*{G}_m} \int_{V_{\text{u.c.}}} \dd[3]{\vb*{r}'} n_\text{u.c.}(\vb*{r}') \ee^{-\ii (\vb*{k} - \vb*{k}_0) \cdot \vb*{r}'}.
    \end{aligned}
\]
定义
\begin{equation}
    S(\vb*{p}) = \int_{V_{\text{u.c.}}} \dd[3]{\vb*{r}'} n_\text{u.c.}(\vb*{r}') \ee^{-\ii \vb*{p} \cdot \vb*{r}'}
\end{equation}
为\concept{几何结构因子},则散射振幅形如
\begin{equation}
    A(\vb*{k}) \propto N \sum_{\vb*{G}_m} \delta_{\vb*{k} - \vb*{k}_0, \vb*{G}_m} S(\vb*{G}_m).
\end{equation}
当晶胞中只有一个原子时这就是\concept{原子形状因子},设$\rho(\vb*{r})$给出了原子中的电子数密度,则
\begin{equation}
    S(\vb*{p}) = F(\vb*{p}) = \int \dd[3]{\vb*{r}} \ee^{-\ii \vb*{p} \cdot \vb*{r}} \rho(\vb*{r}).
\end{equation}
如果晶格中有多个原子,用$j$标记每个原子,用$\vb*{r}_j$表示晶胞内各个原子的位置矢量,则
\begin{equation}
    \begin{aligned}
        S(\vb*{p}) &= \int_{V_{\text{u.c.}}} \dd[3]{\vb*{r}'} \sum_j \rho_j(\vb*{r}' - \vb*{r}_j) \ee^{-\ii \vb*{p} \cdot \vb*{r}'} \\
        &= \sum_j \ee^{-\ii \vb*{p} \cdot \vb*{r}_j} F_j(\vb*{p}).
    \end{aligned}
\end{equation}

因此,只有当散射波波矢的变化和某个倒格矢一致时,才有非零的散射振幅;但即使散射波波矢为某个倒格矢,由于特殊的对称性可能让结构因子变成零,散射振幅仍然可以是零。
这就是说,我们可以将一束非常准直单色的X光打到晶体样品(最好大一些,因为散射振幅正比于$N$)上,然后在不同角度测量散射光光强,这样能够确定晶体中可能的各种倒格矢的方向,从而可以帮助确定晶体晶格的类型。

\subsection{空间群的幺正表示和准粒子}

“准粒子”或者说“元激发”,顾名思义,可以用坐标或动量,以及一些离散标签来标记,并且有“单粒子波函数”$\mel{\Omega}{\psi(\vb*{r})}{\Psi}$。
既然准粒子给出了系统的能量本征态,准粒子的多粒子态应该携带空间群的一组表示。
特别的,单个准粒子的状态——也就是“单粒子波函数”——携带了空间群的一组场表示。
因此,简单地通过对称性分析,就可以获得系统中的准粒子的能谱和单粒子波函数的一些性质,而不必关心这些准粒子是怎样的底层机制演生出来的。
如果我们有信心认为准粒子之间的相互作用可以忽略,那么实际上仅仅依靠对称性我们就完全确定了一个多体系统的行为。

\subsubsection{平移群的表示}\label{sec:transition-group-rep}

设系统的空间群为$G$,由于哈密顿量和$G$中的操作均对易,系统的希尔伯特空间携带$G$的一个表示,且能量本征态同时也是$G$中对称性操作的本征态,而单准粒子波函数也是$G$中对称性操作的本征态。

设$D$是$G$的一个不可约表示。首先考虑$G$的离散平移群部分$T$,由于平移群是阿贝尔群,$D(\{ e | \vb*{r}\})$一定是某个复数乘上单位矩阵$I$。显然,我们有
\begin{equation}
    T = \{ \{e | \vb*{a}_1\}^{n_1} \{e | \vb*{a}_2\}^{n_2} \{e | \vb*{a}_3\}^{n_3} | n_1, n_2, n_3 \in \mathbb{Z} \},
\end{equation}
由于实际的晶体都是有限大小的,为了便于处理,我们假定
\begin{equation}
    \{e | \vb*{a}_i\}^{N_i} = \{e | 0\}, \quad i = 1, 2, 3,
\end{equation}
很显然这是给晶体设置了周期性边界条件。这样$T$的结构大体上和无限大的平移群一样,但是同时又是有限的。由Schur引理,
\[
    D(\{e | \vb*{a}_i \})^{N_i} = D(\{e | \vb*{a}_i \}^{N_i}) = D(\{e | 0\}) = \lambda I,
\]
不失一般性地取$\lambda = 1$,就得到
\begin{equation}
    D(\{e | \vb*{a}_i\}) = \exp(- \frac{\ii 2 \pi}{N_i} p_i), \quad p_i = 0, 1, \ldots, N_i - 1.
\end{equation}
这意味着$T$的不可约表示的标签一定包括$(p_1, p_2, p_3)$,而$(p_1, p_2, p_3)$确定下来之后,$T$的不可约表示就完全确定了:我们有
\begin{equation}
    D(\{e | n_1 \vb*{a}_1 + n_2 \vb*{a}_2 + n_3 \vb*{a}_3\}) = \exp(- \ii \sum_{i=1}^3 \frac{2\pi}{N_i} p_i n_i ).
\end{equation}
我们马上发现,设
\begin{equation}
    \vb*{k} = \frac{p_1}{N_1} \vb*{b}_1 + \frac{p_2}{N_2} \vb*{b}_2 + \frac{p_3}{N_3} \vb*{b}_3,
\end{equation}
则可以使用$\vb*{k}$标记$D$:设$\vb*{r}_n$是一个正格矢,则
\begin{equation}
    D^{\vb*{k}}(\{e | \vb*{r}_n \}) = \ee^{- \ii \vb*{k} \cdot \vb*{r}_n}.
\end{equation}
由于指数函数的周期性,设$\vb*{G}$是一个任意的倒格矢,那么就有
\[
    \ee^{- \ii \vb*{k} \cdot \vb*{r}} = \ee^{- \ii (\vb*{k} + \vb*{G}) \cdot \vb*{r}},
\]
因此,没有必要让$\vb*{k}$取遍整个空间。
我们可以要求$\vb*{k}$在任何一个倒空间的原胞内部,因为对任何一个$\vb*{k}$,都存在一个局限在某个原胞内部的$\vb*{k}'$,使得
\[
    \ee^{- \ii \vb*{k} \cdot \vb*{r}} = \ee^{- \ii \vb*{k}' \cdot \vb*{r}},
\]
而另一方面,原胞内部的矢量彼此的差值不可能达到一个倒格子格矢的大小,因此原胞内部的$\vb*{k}$不会导致重复的$D$。

以上分析都没有指出表示空间是什么。现在我们设$D^{\vb*{k}}$作用在波函数(无论是哪种粒子的波函数)组成的希尔伯特空间上。
显然此时只可能有所谓场表示,因为有平移,即由
\[
    D(g) \phi(\vb*{r}) = \phi(g^{-1} \vb*{r})
\]
给出的表示,在这里就是
\begin{equation}
    D(\{e | \vb*{r}' \}) \phi(\vb*{r}) = \phi(\vb*{r} - \vb*{r}'). 
    \label{eq:field-representation-transition}
\end{equation}
设有一组$\psi(\vb*{r})$是按照\eqref{eq:field-representation-transition}定义的$D$的本征函数,且它们张成的空间携带了一个不可约表示,则我们可以用$\vb*{k}$标记它,有
\begin{equation}
    \psi(\vb*{r} - \vb*{R}_n) = D^{\vb*{k}}(\{e | \vb*{R}_n \}) \psi(\vb*{r}) = \ee^{- \ii \vb*{k} \cdot \vb*{R}_n} \psi(\vb*{r}),
\end{equation}
或者说
\begin{equation}
    \psi(\vb*{r}+\vb*{R}_n) = \ee^{\ii \vb*{k} \cdot \vb*{R}_n} \psi(\vb*{r}).
    \label{eq:periodic-wavefunction}
\end{equation}
考虑到不同的$\vb*{k}$都能够让\eqref{eq:periodic-wavefunction}成立,且指定了$\vb*{k}$之后$\psi$还是不止一种,我们发现离散空间平移对称群在波函数上的场表示是可约的,可以使用$\vb*{k}$和一个额外的标记$n$区分。同一个$\vb*{k}$,不同$n$的$\psi_{n \vb*{k}}$组成的波函数组携带$T$的一个不可约表示,$n$则区分这个表示空间的不同基矢量。

总之,\eqref{eq:periodic-wavefunction}构成晶体的态空间的单准粒子子空间的一组基,且可以用好量子数$\vb*{k}$和另一个好量子数$n$标记它。
$\vb*{k}$称为\concept{晶格动量}或者\concept{准动量}。它的性质类似动量,但并不是自由空间中的动量。
从以上各式(比如\eqref{eq:periodic-wavefunction})中可以看出,这里的$\vb*{k}$和给定一个单粒子哈密顿量,通过傅里叶变换对角化时引入的$\vb*{k}$应该是一样的(至多因为记号不同差一个$-1$),而没有相差某个尺度变换。
$\vb*{k}$的取值范围如前所述被约束在一个倒空间原胞中,且由于周期性边界条件,其取值是离散的。
虽然定义$\vb*{k}$时用到了分量,但由于晶体中倒格子基矢量本来就是三个特殊方向,$\vb*{k}$是坐标系无关的真正的矢量。
满足\eqref{eq:periodic-wavefunction}的波函数在空间平移下多出来了一个因子,这是合理的,因为波函数的对称性可以略微低于哈密顿量,只要由它计算出的物理量的对称性和哈密顿量一致就可以。
\eqref{eq:periodic-wavefunction}又说明,我们可以设
\begin{equation}
    \psi_{\vb*{k}}(\vb*{r}) = \ee^{\ii \vb*{k} \cdot \vb*{r}} u_{\vb*{k}}(\vb*{r}), \quad u_{\vb*{k}}(\vb*{r}+\vb*{a}_i) = u_{\vb*{k}}(\vb*{r}), \quad i = 1, 2, \ldots.
    \label{eq:bloch-wavefunction}
\end{equation}
因此,晶体中的单粒子波函数是一个受到一个周期为$\vb*{a}_1,\vb*{a}_2, \vb*{a}_3$的振幅调制的平面波。这个结论称为\concept{Bloch定理},\eqref{eq:bloch-wavefunction}称为\concept{Bloch波函数}。

\subsubsection{准粒子能谱}\label{sec:quasi-particle-spectrum}

准粒子的能谱具有一些特殊性质。晶体具有时间反演对称性,在时间反演变换下哈密顿量不变,而$\psi_{n \vb*{k}}$被转化为$\psi^*_{n, -\vb*{k}}$,能量本征值则不变。
另一方面,如果$\psi^*_{n, -\vb*{k}}$是一个能量本征态,那么$\psi_{n, -\vb*{k}}$当然也是一个能量本征态,并且两者的能量本征值相同。
因此$\psi_{n, -\vb*{k}}$的能量本征值和$\psi_{n \vb*{k}}$的能量本征值一致,即
\begin{equation}
    \epsilon_{n \vb*{k}} = \epsilon_{n, -\vb*{k}}.
\end{equation}
这个结论的成立\emph{不依赖}晶体是否具有空间反演对称性。

如果空间群操作$\{ \alpha | \vb*{r} \}$是晶体的对称操作,则它和哈密顿量对易,从而
\begin{equation}
    \epsilon_{n \vb*{k}} = \epsilon_{n, \alpha \vb*{k}}.
\end{equation}

这两个性质反过来让我们发现,将倒空间的原胞设置为第一布里渊区实际上是最好的选择。
设晶体中的准粒子哈密顿量可以看成某个对称性高一些的、更加简单的哈密顿量$h_0$加上对称性就是$G$的微扰。加入微扰后,倒空间原胞缩小,根据$h_0$计算出的某一个模式的能谱将会因此出现自我交叉,而微扰导致的能级修正会让由此产生的简并消失一部分。
对称性为$G$的微扰是周期性的,从而,如果两个模式$\vb*{k}$和$\vb*{k}'$之间存在因为微扰导致的跃迁,那么必然
\[
    \vb*{k} = \vb*{k}' + \vb*{G}_m,
\]
而由于它们之间要存在简并,需要有$\omega_{\vb*{k}} = \omega_{\vb*{k}'}$。
由于能带具有和第一布里渊区一样的对称性以及空间反演不变性,我们会发现(通过穷举各种晶格,或者一些几何上的论证TODO),如果取第一布里渊区作为倒空间原胞,则定义在原布里渊区与新布里渊区的差集中的那部分能谱通过用新的倒格矢做平移变换放进缩小了的第一布里渊区之后,缩小了的第一布里渊区边界上仍然处处有能量简并(这是很不平凡的,因为用新的倒格矢做平移后,原本能量简并、$\vb*{k}$很接近的两个态会被移开)。
此外,由于能级简并主要是在第一布里渊区边界上,微扰导致的简并能级解除简并不会在第一布里渊区内部产生不连续的能谱,而是会让缩小了的第一布里渊区中原本两端相连的两条能谱完全分开。
因此以下无特殊说明,我们总是将倒空间原胞取为第一布里渊区,并且将布里渊区缩小称为\concept{布里渊区折叠}。

设晶体点群为$P$,则第一布里渊区实际上可以被分成$\abs*{P}$份,从其中一份就可以重构出其它$\abs*{P} - 1$份。
我们将这$\abs*{P}$份第一布里渊区中的任意一份称为\concept{不可约体积}。

\subsubsection{空间群的表示}



\section{电子能带结构}

本节展示几种能够产生能带的电子模型。它们是传统凝聚态物理中分析导电性的常用理论。
我们还将讨论它们产生的导电和导热特性。

\subsection{近独立电子气的一般理论}

\subsubsection{哈密顿量、分布和格林函数}

很难一上手就处理带有复杂相互作用的电子气,因此我们首先处理\concept{近独立电子气},也就是电子之间近似没有相互作用的电子气。此时我们可以单独考虑每个电子的哈密顿量
\begin{equation}
    {H} = \frac{{\vb*{p}}^2}{2m_\text{e}} + V(\vb*{r}).
    \label{eq:single-electron-hamiltonian}
\end{equation}
整团电子气的哈密顿量是关于各个电子的哈密顿量之和。
上式中的$V(\vb*{r})$未必就是物理意义明确(比如由原子核施加的库伦能)的势能,而有可能是相互作用电子气在一定情况下产生的等效势能。
实际上,对任何一个相互作用系统,都可以找到赝势$V(\vb*{r})$使得其能量近似可以写成\eqref{eq:single-electron-hamiltonian}的形式,但如果相互作用很强,即使往系统中放入一个电子,赝势$V(\vb*{r})$的形式也会发生很大改变,因此将强关联系统看成近独立电子气并无意义。
本节主要讨论$V(\vb*{r})$的形式基本上固定的情况,即电子间相互作用比较弱,与此同时电子数涨落不大的情况。

近独立电子气的基态是什么?使用巨正则系综%
\footnote{当然,我们认为系统能够达到统计平衡,就意味着电子之间不可能真的完全没有相互作用,否则能量无法传递。}%
,对很大的近独立费米子系统,处在能量本征态$\ket{n}$上的粒子数的平均值为%
\footnote{以下使用$\epsilon$表示单个电子的能量而使用$E$表示系统总能量。}%
\begin{equation}
    \expval*{{n}_n} = f(\epsilon_n) = \frac{1}{\ee^{\beta (\epsilon_n-\mu)} + 1}.
    \label{eq:fermi-dirac-distribution}
\end{equation}
我们让能量尽可能低,那就是要让$T\to 0$,也就是让$\beta\to \infty$,此时就有
\begin{equation}
    \expval*{{n}_n} = \begin{cases}
        1, \quad \epsilon_i \leq \mu, \\
        0, \quad \epsilon_i > \mu.
    \end{cases}
\end{equation}
这意味着,$T=0$时电子占据的所有状态就是
\begin{equation}
    \epsilon_i = \mu
\end{equation}
以内的所有能量本征态(这部分能量本征态称为\concept{费米海})。在动量空间中这就是一个曲面,称为\concept{费米面}。位于费米面上的所有能量本征态共同组成了一个能量正好是零温化学势的能级,称为\concept{费米能级},其能量称为\concept{费米能量}。与费米能级对应的动量称为\concept{费米动量}。
基态的表达式就是一个乘积态,为
\begin{equation}
    \ket{\Psi} = \prod_{\abs{\vb*{k}} < k_\text{F}} {c}^\dagger_{\vb*{k}} \ket{0}.
    \label{eq:ground-state}
\end{equation}

统计物理的论证只能把我们带到这里。具体化学势是多少需要根据
\begin{equation}
    \mu_i = \pdv{U}{n_i}
\end{equation}
计算。当然,化学势和粒子数、温度等因素都有关系。在$T=0$且电子数$N$给定时,常用的做法是显式地写出所有能量本征态,从小到大排列$N$个电子,从而计算出费米能量,然后我们就知道了$T=0$时的化学势。

不同粒子数对应的费米能量是不同的;并且,在分析有限温问题时,化学势不再是费米能量。
温度不很高、粒子数很大时,不同粒子数对应的费米能量相差不大,并且化学势和费米能量(也就是$T=0$时的化学势)相差不大,因此有时会使用费米能量近似作为化学势。%
\footnote{关于本节的论述要着重指出一点:虽然我们采用了统计物理的论证来表明必然存在着一个费米面,从而有对应的费米能量,但统计物理的论证仅仅为我们提供了系统基态的性质,而无论系统是不是需要使用平衡态系综描述,它一定有一个基态。因此,费米面、费米能级等概念在任何情况下——无论是平衡态还是非平衡态、纯态还是混合态——全部是适用的。这些概念并不依赖统计物理的框架!}%
但诸如电子比热之类的强烈依赖费米面性质的物理量(因为只有费米面附近的电子会被激发),一点对费米能量的偏离就会产生很大的影响。

化学势的大小由系统中的电子数决定,或者也可以反过来说,化学势的高度决定了系统基态中哪些位置被电子填充。这个对应关系形象地展示如\autoref{fig:chemical-potential},化学势越高,被描黑的态——也就是基态有电子填充的态——就越多。
再次强调,这个“电子能量形式固定,改变化学势往系统中填充电子”的物理图像不适用于强关联系统,因为在那里等效的电子能量形式\eqref{eq:single-electron-hamiltonian}随着电子填入会发生剧烈变化;相应的,从这个物理图像衍生出来的理论,如能带理论也不再适用。

\begin{figure}
    \centering
    \subfigure[$\xi_{\vb*{k}}$和$\vb*{k}$的关系]{
        \begin{tikzpicture}
        
            % 动量横轴
            \draw[->] (-3,0) -- (3,0) node[right] {$\vb*{k}$};
            % 动能纵轴
            \draw[->] (0,-1.5) -- (0,2.5) node[above] {$\xi_{\vb*{k}}$};
            
            % 画出$\xi_{\vb*{k}}$曲线
            \draw[samples=50, smooth, domain=-3:3] plot(\x,{0.25*(\x*\x)-1});
            % 描黑被占据的部分
            \draw[samples=50, smooth, thick, domain=-2:2] plot(\x,{0.25*(\x*\x)-1});
    
            % 标出带底
            \draw[dash pattern=on5pt off3pt] (-2.5, -1) -- (2.5,-1) node[right] {$-\mu$};
    
            % 标出费米点
            \node[dot, label=above:$\vb*{k}_\text{F}$] at (2, 0) {};
            \node[dot, label=above:$-\vb*{k}_\text{F}$] at (-2, 0) {};
    
        \end{tikzpicture}
    }
    \subfigure[$\epsilon_{\vb*{k}}$和$\vb*{k}$的关系]{
        \begin{tikzpicture}
        
            % 动量横轴
            \draw[->] (-3,0) -- (3,0) node[right] {$\vb*{k}$};
            % 动能纵轴
            \draw[->] (0,-1.5) -- (0,2.5) node[above] {$\epsilon_{\vb*{k}}$};
            
            % 画出$\epsilon_{\vb*{k}}$曲线
            \draw[samples=50, smooth, domain=-3:3] plot(\x,{0.25*(\x*\x)});
            % 描黑被占据的部分
            \draw[samples=50, smooth, thick, domain=-2:2] plot(\x,{0.25*(\x*\x)});
    
            % 标出费米面
            \draw[dash pattern=on5pt off3pt] (-2.5, 1) -- (2.5,1) node[right] {$\mu$};

            % 标出满带
            \draw[dash pattern=on5pt off3pt] (2, 0) -- (2,1);
            \draw[dash pattern=on5pt off3pt] (-2, 0) -- (-2,1);
    
            % 标出费米点
            \node[dot, label=below:$\vb*{k}_\text{F}$] at (2, 0) {};
            \node[dot, label=below:$-\vb*{k}_\text{F}$] at (-2, 0) {};
    
        \end{tikzpicture}
    }
    \caption{化学势和电子填充,描黑的态有电子填充}
    \label{fig:chemical-potential}
\end{figure}

近独立电子气的格林函数和它的谱函数如下。谱函数为
\begin{equation}
    A(\vb*{k}, \omega) = \delta(\omega - \epsilon_{\vb*{k}}),
\end{equation}
推迟格林函数为
\begin{equation}
    G^\text{ret}(\vb*{k}, \omega) = \frac{1}{\omega - \epsilon_{\vb*{k}} + \ii 0^+}.
\end{equation}
这些都和单电子计算出来的结果完全一样,当然也毫不意外。

\subsubsection{空穴}

虽然多电子态是将场算符作用在真空态上得到的,但由于本文讨论的电子都由薛定谔场描述,在任何一个物理过程中电子数都是守恒的。
换而言之,设有一个$N$电子的系统,这个系统实际可能具有的态矢量只是态空间的一小部分,即保持电子数为$N$的部分。
这个$N$电子空间$\mathcal{H}_N$当然可以使用一次量子化理论来描述,但能否使用二次量子化的语言描述?
费米子的特性让这一点成为可能。请注意由于泡利不相容原理,
\[
    {c}^\dagger_{\vb*{k}} \ket{\Psi} = 0, \quad k < k_\text{F},
\]
而由产生湮灭算符的性质显然有
\[
    {c}_{\vb*{k}} \ket{\Psi} = 0, \quad k > k_\text{F},
\]
因此如果定义%
\footnote{费米面上的态相对来说非常少,因此忽略。}%
\begin{equation}
    {b}^\dagger_{-\vb*{k}} = \begin{cases}
        {c}_{\vb*{k}}, \quad k < k_\text{F}, \\
        {c}_{\vb*{k}}^\dagger, \quad k > k_\text{F},
    \end{cases}
\end{equation}
那么基态$\ket{\Psi}$实际上是${b}^\dagger$产生的准粒子的真空态。${b}^\dagger$产生的是什么?当$k>k_\text{F}$时它产生的就是费米面以上的电子,而$k<k_\text{F}$时它产生的是费米海之内的空穴。
空穴的动量就是它占据的态如果有电子的话,这个电子的动量的相反数。
这样,我们可以将空穴看成电子的反粒子,虽然这种“反粒子”并不像高能物理中那样,来自洛伦兹群的表示。
使用${b}^\dagger$写出的哈密顿量在省略一个无关紧要的常数之后为
\[
    {H} = \sum_{k > k_\text{F}} \xi_{\vb*{k}} {b}^\dagger_{\vb*{k}} {b}_{\vb*{k}} - \sum_{k < k_\text{F}} \xi_{\vb*{k}} {b}^\dagger_{\vb*{k}} {b}_{\vb*{k}},
\]
因此一个空穴的能量为$-\epsilon_{\vb*{k}}$。以上的哈密顿量不是正定的,但这不会导致负能量疑难,因为费米海虽然很大,但大小有限,因此不会出现能量无限下降的问题。

使用${b}^\dagger$的结果是,保持电子数不变的相互作用需要被看成粒子数可变的,例如一个费米海中的电子被激发到费米海之上就意味着产生了一个电子-空穴对。

\subsection{自由电子气}

现在我们讨论最为简单的近独立电子气,也就是$V(\vb*{r})$在物体内部为常数(可以看成零)的情况。

\subsubsection{能谱}

我们在坐标表象下处理问题。计算单个电子的波函数:
\[
    - \frac{\laplacian}{2m_\text{e}} \psi(\vb*{r}) = \epsilon \psi(\vb*{r}),
\]
这种方程的解当然是平面波解的线性组合。一个这样的平面波解形如
\[
    \psi(\vb*{r}) \propto \ee^{\ii \vb*{k} \cdot \vb*{r}}.
\]
只能保证这个式子在物体内部成立,因为物体边界处$V(\vb*{r})$不可能是常数。
然后我们归一化这些平面波。电子可以自发地溢出物体,但是这样的概率并不大,所以我们可以简单地认为电子只会出现在物体内部(也即,物体被放置在一个无限深势陷当中)。设物体体积为$V$,就有
\[
    \int \dd[3]{\vb*{r}} \abs{\psi(\vb*{r})}^2 = 1,
\]
于是
\[
    \psi (\vb*{r}) = \frac{1}{\sqrt{V}} \ee^{\ii \vb*{k} \cdot \vb*{r}}, \quad \epsilon = \frac{k^2}{2m_\text{e}}.
\]
很容易看出这些波函数实际上是动量算符的本征态,$\vb*{k}$实际上就是动量。另一方面,这些波函数定义在坐标空间中,坐标空间中的一切都和自旋算符对易,因此这些波函数也是自旋本征态。于是动量和自旋的一组共同正交本征函数为
\begin{equation}
    \psi_{\vb*{k},\sigma} (\vb*{r}) = \frac{1}{\sqrt{V}} \ee^{\ii \vb*{k} \cdot \vb*{r}}, \quad \epsilon_{\vb*{k},\sigma} = \frac{k^2}{2m_\text{e}}.
\end{equation}
% 真的是这个名字吗?这些波函数称为\concept{布洛赫波函数}。
$\vb*{k}$能够取什么值取决于边界条件。由于物体通常比较大,具体取什么样的边界条件对物体内部的过程毫无影响。

\subsubsection{零温状态}

自由电子气的费米面是球状的(即\concept{费米球})。对三维系统,动量空间中大小为$\dd[3]{\vb*{k}}$的区域内的状态数为
\begin{equation}
    \dd{N} = 2 \frac{V}{(2\pi)^3} \dd[3]{\vb*{k}},
\end{equation}
因子$2$是因为电子有两个自旋。积掉无用的动量分布角自由度就得到
\begin{equation}
    \dd{N} = \frac{V k^2 \dd{k}}{\pi^2}.
\end{equation}
在本问题中,使用“相空间仍然由动量和坐标确定,只不过被划分为相格”的假设也可以推导出正确的$\dd{N}$:我们有
\[
    \dd{N} = 2 \frac{\dd[3]{\vb*{r}} \dd[3]{\vb*{k}}}{(2\pi)^3},
\]
而积掉无用的坐标自由度和动量角自由度之后得到
\[
    \dd{N} = \frac{V k^2 \dd{k}}{\pi^2}.
\]

我们根据能谱计算态密度。由于
\[
    \epsilon = \frac{k^2}{2m},
\]
可以推导出
\[
    \dd{N} = \frac{V (2m)^{3/2} \sqrt{\epsilon} \dd{\epsilon}}{2 \pi^2},
\]
即单位能量间隔中的态有
\begin{equation}
    D(\epsilon) = \frac{V (2m)^{3/2} \sqrt{\epsilon} }{2 \pi^2}
\end{equation}
这么多。这样就可以计算出总粒子数和费米能(即零温化学势)之间的关系:
\begin{equation}
    N = \int_{\epsilon=0}^{\epsilon_{\text{F}}} \dd{N} = \frac{V (2m)^{3/2}}{3 \pi^2} \epsilon_\text{F}^{3/2},
\end{equation}
以及总能量
\begin{equation}
    E = \int_{\epsilon=0}^{\epsilon_{\text{F}}} \epsilon \dd{N} = \frac{V (2m)^{3/2}}{5 \pi^2} \epsilon_\text{F}^{5/2} = \frac{3}{5} N \epsilon_{\text{F}}.
\end{equation}
对二维或者一维的自由电子气也可以使用类似的方法得到系数不同的结论。

\subsubsection{热力学性质}

我们现在计算有限温下自由电子气的一些最基本的性质。
设$Q(\epsilon)$为$\epsilon$下方的所有状态数,即
\begin{equation}
    Q(\epsilon) = \int_0^\epsilon \dd{\epsilon'} D(\epsilon'),
\end{equation}
于是总电子数为
\[
    N = \int_0^\infty \dd{\epsilon} D(\epsilon) f(\epsilon) = \int_0^\infty \dd{\epsilon} Q(\epsilon) \left( - \pdv{f}{\epsilon} \right).
\]
我们有
\[
    - \pdv{f}{\epsilon} = \frac{\beta}{(\ee^{\beta (\epsilon - \mu)} + 1) (\ee^{- \beta (\epsilon - \mu)} + 1)},
\]
它在温度不高时是$\epsilon = \mu$附近的一个尖峰,并且相对于$\epsilon - \mu$是偶函数,从而可以当成一个经过某些修正的$\delta$函数。
这样,我们可以将$N$中的积分下限拓展到$-\infty$而对结果没有太大影响。这样我们有
\[
    \begin{aligned}
        N &= \int_{-\infty}^\infty \dd{\epsilon} Q(\epsilon) \left(-\pdv{f}{\epsilon}\right) \\
        &= \int_{-\infty}^\infty \dd{\epsilon} \left( Q(\mu) + Q'(\mu) (\epsilon - \mu) + \frac{1}{2} Q''(\mu) (\epsilon - \mu)^2 + \cdots \right) \left(-\pdv{f}{\epsilon}\right) .
    \end{aligned}
\]
这其中,第二项一定是零,因为它关于$\epsilon - \mu$是奇函数。
第一项就是
\[
    - \int_{-\infty}^\infty \dd{\epsilon} Q(\mu) \pdv{f}{\epsilon} = Q(\mu) (f(-\infty) - f(\infty)) = Q(\mu), 
\]
而第三项是
\[
    \begin{aligned}
        &\quad \frac{1}{2} Q''(\mu) \int_{-\infty}^\infty \dd{\epsilon} (\epsilon - \mu)^2 \frac{\beta}{(\ee^{\beta (\epsilon - \mu)} + 1) (\ee^{- \beta (\epsilon - \mu)} + 1)} \\
        &= \frac{1}{2 \beta^2} Q''(\mu) \int_{-\infty}^\infty \dd{\xi} \frac{\xi^2}{(\ee^{-\xi} + 1) (\ee^{\xi} + 1)} \\
        &= \frac{\pi^2}{6} Q''(\mu) T^2.
    \end{aligned}
\]
因此在$T^2$精度下我们有
\begin{equation}
    N = Q(\mu) + \frac{\pi^6}{6} Q''(\mu) T^2.
\end{equation}
我们现在尝试写出$\mu$的一个显式表达式。首先$Q(\mu)$和$N$其实是很接近的,可以做展开
\[
    \begin{aligned}
        Q(\mu) - N &= Q(\epsilon_\text{F}) + Q'(\epsilon_\text{F}) (\mu - \epsilon_\text{F}) - N \\
        &= D(\epsilon_\text{F}) (\mu - \epsilon_\text{F}),
    \end{aligned}
\]
因此我们有
\[
    \mu - \epsilon_\text{F} \sim \frac{Q''(\mu)}{D(\epsilon_\text{F})} T^2.
\]
这又意味着展开式
\[
    \frac{\pi^6}{6} Q''(\mu) T^2 = \frac{\pi^6}{6} \left( Q''(\epsilon_\text{F}) + Q'''(\epsilon_\text{F}) (\epsilon_\text{F} - \mu) + \cdots \right) T^2
\]
中的第一项就达到了$T^2$的精度,从而如果我们只需要计算到$T^2$精度,只需要求解
\[
    D(\epsilon_\text{F}) (\mu - \epsilon_\text{F}) + \frac{\pi^2}{6} Q''(\epsilon_\text{F}) T^2 = 0
\]
即可,最终计算得到
\begin{equation}
    \mu = \epsilon_\text{F} \left( 1 - \frac{\pi^2}{12} \left( \frac{T}{\epsilon_\text{F}} \right)^2 + \cdots \right).
\end{equation}

类似地计算热容,电子贡献的内能为
\[
    U = \int_0^\infty \dd{\epsilon} \epsilon f(\epsilon) D(\epsilon),
\]
类似地引入
\[
    R(\epsilon) = \int_0^\infty \dd{\epsilon} \epsilon D(\epsilon),
\]
做分部积分,泰勒展开,
\begin{equation}
    C_V = \frac{n \pi^2}{2} \frac{T}{T_\text{F}},
\end{equation}
与经典Drude模型非常不一样。直观地看,这是因为只有费米面附近的电子才会被激发。

在极低温下晶体的比热主要由电子贡献,因为此时$T$相比$T^3$要明显。

金属热导率

\subsection{周期势场和能带形成}

本节给出自由电子放在周期势场中而形成能带这一事实的具体计算。

\subsubsection{布洛赫电子和能带}\label{sec:energy-band}

由于晶格具有离散平移不变性,\eqref{eq:electron-gas-hamiltonian}也具有(而且只有)离散平移不变性,从而波函数也具有这样的不变性。我们有
\[
    \psi(\vb*{r}+\vb*{R}_n) = \ee^{\ii \alpha} \psi(\vb*{r}).
\]
请注意波函数的对称性可以略微低于哈密顿量,只要由它计算出的物理量的对称性和哈密顿量一致就可以,因此我们加上了复数因子。由于平移运算构成群,且$n$是群参数,有
\[
    R_{n_1+n_2} = R(n_1) R(n_2),
\]
波函数的形式只能是
\begin{equation}
    \psi(\vb*{r}+\vb*{R}_n) = \ee^{\ii \vb*{k} \cdot \vb*{R}_n} \psi(\vb*{r})
    \label{eq:periodic-wavefunction}
\end{equation}
及其线性组合。因此\eqref{eq:periodic-wavefunction}构成晶体中电子气中单个电子的态空间的一组基,且$\vb*{k}$是一个好量子数,称为\concept{格点动量}或者\concept{准动量}。它的性质类似动量,但并不是动量。
$\vb*{k}$是坐标系无关的真正的矢量。\eqref{eq:periodic-wavefunction}又说明,我们可以设
\begin{equation}
    \psi_{\vb*{k}}(\vb*{r}) = \ee^{\ii \vb*{k} \cdot \vb*{r}} u_{\vb*{k}}(\vb*{r}), \quad u_{\vb*{k}}(\vb*{r}+\vb*{a}_i) = u_{\vb*{k}}(\vb*{r}), \quad i = 1, 2, 3.
    \label{eq:bloch-wavefunction}
\end{equation}
因此,晶体中的波函数是一个受到一个周期为$\vb*{a}_1,\vb*{a}_2, \vb*{a}_3$的振幅调制的平面波。
\eqref{eq:bloch-wavefunction}称为\concept{布洛赫波函数},处于这种状态的电子称为\concept{布洛赫电子}。

由于指数函数的周期性,设$\vb*{G}$是一个任意的倒格子格矢,那么就有
\[
    \psi_{\vb*{k}}(\vb*{r}) = \psi_{\vb*{k}+\vb*{G}}(\vb*{r}).
\]
因此,没有必要让$\vb*{k}$取遍整个空间。由维格纳-赛兹原胞的性质,对任何一个$\vb*{k}$,都存在一个第一布里渊区内部的$\vb*{k}'$,使得
\[
    \psi_{\vb*{k}}(\vb*{r}) = \psi_{\vb*{k}'}(\vb*{r}).
\]
另一方面,第一布里渊区内部的矢量彼此的差值不可能达到一个倒格子格矢的大小,因此第一布里渊区内部的$\vb*{k}$不会导致重复的波函数。于是不失一般性地我们可以要求$\vb*{k}$在第一布里渊区内部。

波函数表达式\eqref{eq:bloch-wavefunction}没有时间演化,所以现在我们讨论布洛赫电子的动力学。即使使用了波恩-奥本海姆近似,\eqref{eq:electron-gas-hamiltonian}仍然要求把系统内所有的电子均考虑进去。
我们假定系统可以使用近独立电子气描述——也许是通过一个平均场理论,也许是更加完整的费米液体计算——于是只需要考虑单个电子的薛定谔方程即可。
于是布洛赫电子的动力学等价于求解形如\eqref{eq:bloch-wavefunction}的波函数在某个势场下的束缚态解。将\eqref{eq:bloch-wavefunction}代入
\[
    -\frac{\laplacian}{2m_\text{e}} \psi(\vb*{r}) + V(\vb*{r}) \psi(\vb*{r}) = \epsilon \psi(\vb*{r}),
\]
得到
\begin{equation}
    \left( - \frac{(\grad+\vb*{k})^2}{2m_\text{e}} + V(\vb*{r}) \right) u_{\vb*{k}}(\vb*{r}) = \epsilon u_{\vb*{k}}(\vb*{r}).
    \label{eq:block-energy-problem}
\end{equation}
在$\vb*{k}$已知的情况下求解该本征值问题,可以得到一组$u$以及对应的$E$。由于先前要求$u(\vb*{r})$是周期函数,该本征值问题必定给出离散谱,也即,我们会获得一组$(\psi_{n \vb*{k}}, \epsilon_{n \vb*{k}})$,使得
\begin{equation}
    \left( - \frac{(\grad+\vb*{k})^2}{2m_\text{e}} + V(\vb*{r}) \right) u_{n \vb*{k}}(\vb*{r}) = \epsilon_{n \vb*{k}} u_{n \vb*{k}}(\vb*{r}).
\end{equation}
显然,$n$是另一个(离散的)好量子数。一旦$\vb*{k}$和$n$给定,布洛赫波函数及其时间演化就完全求解出来了,从而$\vb*{k}$和$n$是坐标空间中的布洛赫波函数的全部好量子数。我们称取不同$n$值的电子处于不同的\concept{能带}上,$n$为能带标记。%
\footnote{注意到,能带的导出实际上并未用到太多晶体的性质(空间周期性等),因此非晶体很多时候也有能带。}%
能带中的电子在空间上非常不定域,它们实际上是一系列格点上的定域态叠加而成的结果。也可以这么理解这一点:写出一个格点上的哈密顿量之后为了计算能带能量肯定要做对角化,对角化得到的本征向量肯定是把一系列格点上的定域态都线性组合起来了。
总之,能带中的布洛赫电子常常称为\concept{巡游电子}。

由于晶体近似在一个无限深势阱中,波函数在晶体边界处快速衰减为零。这就意味着$\vb*{k}$实际上是离散的。然而,由于晶体的尺度通常远大于原子的尺度,$\vb*{k}$近似可认为是连续的。于是可以写出函数
\[
    \epsilon = \epsilon_n(\vb*{k}),
\]
由于$\vb*{k}$加上任何一个倒格矢之后给出同样的布洛赫波函数,$E$相对于$\vb*{k}$应该具有周期性,那么它必定是有界的。
这就是“能带”这个名称的来源:穷举第一布里渊区内部的所有$\vb*{k}$,得到的所有能量组成一条有限宽的条带。

还可以从另一个角度理解为什么会出现多条能带。自由电子的能谱是抛物线,既然是在晶体内,动量加上一个倒格矢之后描述的物理状态没变,于是可以通过将任意一个动量平移一个倒格矢,使所有动量约束在第一布里渊区内部。
这个过程等价于将单电子能谱平移任意一个倒格矢,然后将第一布里渊区以外的部分全部抹除掉。
单电子能谱$\epsilon_{\vb*{k}}$穿过第一布里渊区边界而折返回来,这样在布里渊区边界反复和自我交叉。
现在加上周期势,使用微扰论计算其影响,会发现周期势会抹去奇异性,交叉点退开,自我交叉的单一能带分离成一系列不相交的波浪线,这就得到了一系列不同的能带。
此时我们说,能带之间\concept{打开了能隙}或者说\concept{回避交叉},让原本交叉的能带之间出现了间隙。

\begin{figure}
    \centering
    \subfigure[自由空间中的单电子能谱]{
        \begin{tikzpicture}
        
            % 动量横轴
            \draw[->] (-3,0) -- (3,0) node[right] {$\vb*{k}$};
            % 动能纵轴
            \draw[->] (0,-0.25) -- (0,6) node[above] {$\epsilon_{\vb*{k}}$};
            
            % 画出$\epsilon_{\vb*{k}}$曲线
            \draw[samples=50, smooth, domain=-3:3] plot(\x,{0.5*(\x*\x)});
    
        \end{tikzpicture}
    }
    \subfigure[由于晶格的周期性,出现第一布里渊区折叠,能谱在第一布里渊区中折返]{
        \begin{tikzpicture}
        
            % 动量横轴
            \draw[->] (-3,0) -- (3,0) node[right] {$\vb*{k}$};
            % 动能纵轴
            \draw[->] (0,-0.25) -- (0,6) node[above] {$\epsilon_{\vb*{k}}$};

            % 布里渊区边界
            \draw[dash pattern=on5pt off3pt, thick] (-1, 0) -- (-1, 6);
            \draw[dash pattern=on5pt off3pt, thick] (1, 0) -- (1, 6);
            
            % 画出布里渊区以外的能谱
            
            \draw[dash pattern=on5pt off3pt,samples=50, smooth, domain=-3:1.46] plot(\x,{0.5*((\x+2)*(\x+2))});
            \draw[dash pattern=on5pt off3pt,samples=50, smooth, domain=-3:3] plot(\x,{0.5*((\x)*(\x))});
            \draw[dash pattern=on5pt off3pt,samples=50, smooth, domain=-1.46:3] plot(\x,{0.5*((\x-2)*(\x-2))});
            \draw[dash pattern=on5pt off3pt,samples=50, smooth, domain=0.52:3] plot(\x,{0.5*((\x-4)*(\x-4))});
            \draw[dash pattern=on5pt off3pt,samples=50, smooth, domain=-3:-0.52] plot(\x,{0.5*((\x+4)*(\x+4))});

            % 画出布里渊区内部的$\epsilon_{\vb*{k}}$曲线,以及由于晶格周期性而导致的能谱平移
            \draw[samples=50, smooth, domain=-1:1] plot(\x,{0.5*(\x*\x)});
            \draw[samples=50, smooth, domain=-1:1] plot(\x,{0.5*((\x-2)*(\x-2))});
            \draw[samples=50, smooth, domain=-1:1] plot(\x,{0.5*((\x+2)*(\x+2))});
            \draw[samples=50, smooth, domain=0.52:1] plot(\x,{0.5*((\x-4)*(\x-4))});
            \draw[samples=50, smooth, domain=-1:-0.52] plot(\x,{0.5*((\x+4)*(\x+4))});
    
        \end{tikzpicture}
        
    }
    \subfigure[相互作用打开能隙,形成分离的能带]{
        \begin{tikzpicture}
        
            % 动量横轴
            \draw[->] (-3,0) -- (1.5,0) node[right] {$\vb*{k}$};
            % 动能纵轴
            \draw[->] (0,-0.25) -- (0,6) node[above] {$\epsilon_{\vb*{k}}$};

            % 布里渊区边界
            \draw[dash pattern=on5pt off3pt, thick] (-1, 0) -- (-1, 6);
            \draw[dash pattern=on5pt off3pt, thick] (1, 0) -- (1, 6);
            
            % 画出布里渊区内部的$\epsilon_{\vb*{k}}$曲线,以及由于晶格周期性而导致的能谱平移
            \draw[samples=50, smooth, domain=-1:1] plot(\x,{0.25-0.25*cos(3.14*\x r)});
            \draw[samples=50, smooth, domain=-1:1] plot(\x,{2.5+1.6*sin(1.57*\x r)});
            \draw[samples=50, smooth, domain=-1:1] plot(\x,{2.5-1.6*sin(1.57*\x r)});
            \draw[samples=50, smooth, domain=0.39:1] plot(\x,{4.5+4*(\x-1)*(\x-1)});
            \draw[samples=50, smooth, domain=-1:-0.39] plot(\x,{4.5+4*(\x+1)*(\x+1)});

            % 标记能带和带隙
            \foreach \hei in {0.5, 0.9, 4.1, 4.5}
                \draw[dash pattern=on5pt off3pt] (-1,\hei) -- (-3, \hei);
    
        \end{tikzpicture}
        
    }
    \caption{能带结构}
    \label{fig:bloch-energy-band}
\end{figure}

两条不同的能带之间的间隙提供了一个自然的能量截断,因此在我们已经知道了系统的能带之后,如果需要一个低能有效理论,可以只考虑能量较低的能带,而将粒子跃迁到能量较高的能带再跃迁回来作为微扰,进行能量修正,即可以很自然地将高能能带积掉。

最后我们指定波函数的归一化方式。可以在积分号前面加上一个系数,即
\begin{equation}
    \frac{1}{V} \int \dd[3]{\vb*{r}} \psi_{n\vb*{k}}^*(\vb*{r}) \psi_{m\vb*{k}'}(\vb*{r}) = \delta_{mn} \delta(\vb*{k}-\vb*{k}'),
    \label{eq:bloch-is-basis}
\end{equation}
从而让简单的平面波$\exp(\ii \vb*{k} \cdot \vb*{r})$不需要乘上归一化因子就能够成为归一化本征态。
设$V_\text{u.c.}$是单个晶胞的大小,则
\[
    V = N V_\text{u.c.},
\]
从而可以得到
\begin{equation}
    \frac{1}{V_\text{u.c.}} \int_\text{u.c.} \dd[3]{\vb*{r}} u_{m\vb*{k}}^*(\vb*{r}) u_{n\vb*{k}}(\vb*{r}) = \delta_{mn}.
\end{equation}
只需要求解出一组满足以上条件的$\{u_{n\vb*{k}}\}$,就得到了一组正交归一化波函数$\{\psi_{n\vb*{k}}\}$。
归一化系数使用$V_\text{u.c.}$是非常合理的,因为如\eqref{eq:normalization-periodic}所示这正是具有正格子的周期性的函数通常使用的归一化系数。

\eqref{eq:bloch-is-basis}意味着布洛赫波函数是正交归一化波函数且对应的积分测度为
\[
    \frac{1}{\sqrt{V}} \int \dd[3]{\vb*{r}},
\]
记${c}_{n\vb*{k}}^\dagger$为位于能带$n$、格点动量为$\vb*{k}$的布洛赫电子的产生算符,那么%
\footnote{其中的$1/\sqrt{V}$的因子是因为二次量子化场算符通常使用全空间的积分为内积的定义,在此定义下,归一化的波函数是$\psi_{n\vb*{k}} / \sqrt{V}$而不是$\psi_{n\vb*{k}}$。}%
\begin{equation}
    {c}_{n \vb*{k}}^\dagger = \frac{1}{\sqrt{V}} \int \dd[3]{\vb*{r}} \psi_{n \vb*{k}}(\vb*{r}) {\psi}^\dagger(\vb*{r}),
\end{equation}
哈密顿量在这一组基下是对角化的,于是写出二次量子化哈密顿量(已经加入化学势项)
\begin{equation}
    {H} = \sum_{n, \vb*{k}} (\epsilon_{n\vb*{k}} - \mu) {c}^\dagger_{n\vb*{k}} {c}_{n\vb*{k}}.
    \label{eq:bloch-band-hamiltonian}
\end{equation}
$\vb*{k}$的取值局限在第一布里渊区内部,作为对比,不考虑周期势的边长为$L$的正方体势阱中的电子的$\vb*{k}$可以取遍所有位于那个边长为$2\pi / L$的格点。
但实际上两者的自由度是一样的,因为我们还有$n$标记各个能带,也即,我们相当于把所有能带中的动量都移动到了第一布里渊区内部。

\subsubsection{一维周期势}

具体“抹去奇异性”的方式可以通过一阶简并微扰论看出来。先考虑一个简单的一维的例子,设晶格常数为$a$,则周期性势场为
\[
    V(x) = \sum_n \ee^{\ii \frac{2 \pi}{a} n x}.
\]
在没有周期性势场时电子波函数就是一维无限深势阱中的形式:
\[
    \psi_n(x) = \frac{1}{\sqrt{L}} \ee^{\ii \frac{2 \pi n x}{L}},
\]
简并出现在$n$和$-n$之间。我们可以尝试做非简并微扰论,
% TODO

\subsubsection{导电性分类}

晶体的导电性可以分类如下:
\begin{itemize}
    \item 绝缘体就是有电子的能带被完全填满的系统。
    \item 金属就是有一些有电子的能带只是部分被填满的系统。
    \item \concept{半金属}是一些有电子的能带只有少量电子的系统,其导电性不好,但是仍然呈现一些金属的性质。
    \item \concept{半导体}是掺杂了的绝缘体,如果杂质能够形成能量大体上是在满带顶附近的局域空穴态,以及能量大体上是在空带底附近的局域电子态,那么热涨落就会让一些电子填充进空带中,而让满带中出现空穴,这就同时形成了两种载流子。
    不能依靠热涨落让电子从满带跳往空带,因为这要求费米温度量级的温度;我们需要掺杂才能够形成半导体。
\end{itemize}

能带论能够解释为什么绝缘体不导电。自由电子气模型无法解释为什么绝缘体不导电——绝缘体中同样有大量的电子,似乎本应该导电。

\subsubsection{赝势}

离子实中的核心电子轨道和价电子轨道同时都是单电子哈密顿量的本征态:
\begin{equation}
    H \ket{\psi_c} = E_c \ket{\psi_c}, \quad H \ket{\psi_\text{v}} = E_\text{v} \ket{\psi_\text{v}},
\end{equation}
这里$c$标记了各个内层电子轨道。定义
\begin{equation}
    \ket{\psi_\text{v}^\text{ps}} = \ket{\psi_\text{v}} + \sum_c \ket{\psi_c} \braket*{\psi_c}{\psi_\text{v}^\text{ps}},
\end{equation}
则计算得到
\[
    (H - E_\text{v}) \ket{\psi_\text{v}^\text{ps}} = \sum_c (E_c - E_\text{v}) \ket{\psi_c} \braket*{\psi_c}{\psi_\text{v}^\text{ps}},
\]
从而
\begin{equation}
    \left(H - \sum_c (E_c - E_\text{v}) \dyad*{\psi_c}\right) \ket{\psi_\text{v}^\text{ps}} = E_\text{v} \ket{\psi_\text{v}^\text{ps}}.
\end{equation}
这意味着可以定义一个相较于晶体中实际的周期势要平滑得多的“势”
\begin{equation}
    V^\text{ps} = V - \sum_c (E_c - E_\text{v}) \dyad*{\psi_c},
\end{equation}
从这个势出发做计算能够得到正确的能谱。

\subsection{金属和朗道费米液体}

金属中存在可以四处移动的巡游电子,这些电子之间存在库伦排斥,而又有晶格对电子施加周期势。
先考虑周期势的影响,裸电子将会有能量修正而成为能带电子,自由能带电子组成费米气体,很容易处理。
库伦排斥和声子介导的吸引相互作用合并为电子-电子相互作用,非常难以处理——电子-电子相互作用可以导致各种奇特的现象,并且其量级通常与费米能的量级一致。
然而,金属的行为在很多方面非常像费米气体,这意味着实际上描述金属的低能有效理论基本上就是一个费米气体加上一定的相互作用。
\concept{朗道费米液体}是一个这样的低能有效理论,其基本假设为:
\begin{enumerate}
    \item 费米液体的状态可以和费米气体一样,使用某种费米子的占据数标记,并且系统能量可以写成占据数的函数。
    这个假设对哈密顿量的形式做出了很强的限制,因为此时的哈密顿量已经在粒子数表象下被严格对角化了。(但是这个哈密顿量仍然可以不是自由哈密顿量,因为可以有密度-密度相互作用)不过我们将会看到,这样的限制是合理的。
    \item 当相互作用趋于零(“被关闭”)时,费米液体态回退到实际的费米气体态,我们假设此时费米液体态的占据数和实际的费米气体态的占据数相同。
    这个假设实际上是说,费米液体中的占据数所描述的准粒子实际上是经过某种重整化的电子,和电子可以一一对应。
\end{enumerate}

可以通过这样的方式想象费米液体可以怎样产生:对无相互作用的能带电子%
\footnote{
    应当注意能带电子这个名称本身有些模糊不清。只计算周期性势场的影响可以得到能带,只计算经过平滑处理的赝势也可以得到能带,将一部分相互作用作为自能修正而留下最为显著的相互作用通道作为碰撞也可以得到能带(虽然还需要额外计算碰撞的影响,如费米液体中那样),对很多系统,其实相互作用全都考虑进去了还是可以得到能带——如同用DFT计算出来的那样。
    例如,在费米液体中,如果粒子数涨落不大,完全可以用粒子数平均值代替相互作用项中的准粒子占据数。
    即使相互作用大一些,也可以理解成“能带被拉歪了”。
}%
,我们可以浸染地将相互作用加入,如果没有出现诸如费米子配对之类的情况,那么第二个假设就已经满足了。
相互作用会导致电子出现自能修正,由于库仑相互作用的顶角是保持粒子数守恒的,不存在一个电子衰变成多个其它粒子的过程,因此自能修正是实数,即单电子态的确是本征态。%
\footnote{
    虽然如此,如果我们将高于费米面的电子称为电子而将低于费米面的电子称为空穴,那么的确可以出现一个电子和费米球中的空穴发生相互作用,而衰变成电子和空穴的过程。
    见\autoref{sec:fermi-liquid-ground}。
}%
在很多粒子数守恒的理论——如$\phi^4$理论或是QED——中,可以有稳定的单粒子、二粒子、三粒子……本征态,虽然$n$粒子态的能量不是$n$个粒子的能量简单相加,但是无论如何,一个本征态的能量可以写成各个动量上的粒子的占据数的函数,因此量子态可以用占据数标记,能量也可以用粒子数标记,即第一个假设也是成立的。

在什么情况下费米液体理论失效?如果相互作用是吸引的,那么低温下可能出现费米子配对,此时总是会出现相变。
我们将在\autoref{sec:low-and-super}中讨论这种情况。
如果相互作用较强,能带的图像可能就完全不适用了,此时系统的元激发可能都不是经过修正的电子,也无所谓费米液体。
如果有一些不能写成粒子数算符乘积的相互作用通道,费米液体理论同样不成立
令人震惊的是,虽然大部分实际体系中库仑相互作用的确很强,费米液体图像仍然是适用的。

由于费米液体中的准粒子是电子的单体激发,可以从费米液体中准粒子的行为中看到很明显的普通电子的影子,由于准粒子场和电子场的对称性相同,准粒子也是费米子,准粒子能谱和电子能谱形式类似,自旋均为$1/2$,带电量相同,等等。
唯象的讨论中可以直接将费米液体中的准粒子当成电子。

\subsubsection{基态附近的激发}\label{sec:fermi-liquid-ground}

与高能物理中不同,实际的凝聚态体系的基态中都有大量准粒子。由于准粒子是费米子,系统基态中有一个费米球。
由于粒子数守恒,系统基态附近的激发态就是一些准粒子从费米球内部被打到费米球外部之后形成的,并且距离费米面均不远。
这样,我们可以忽略费米海而认为系统中实际上有两种元激发:“电子”和“空穴”。
实际上,在费米液体理论中通常也只分析费米面附近的物理,部分原因在于费米海的结构可以非常复杂,因此只考虑费米面附近的物理是比较容易的,也是比较有实际意义的(因为是低温近似),部分原因在于只有这里才确定有稳定的准粒子——通常准粒子的寿命在接近费米面时比较长,因此看起来像是“真正的”粒子(否则会有非常明显的能级展宽)。

这件事的原因如下。首先应当注意,虽然准粒子的自能修正总是实数,从而,在没有其它任何准粒子时,单个准粒子可以永远稳定存在而不会衰变,但与高能物理中的情况不同,费米球的存在意味着如果费米面上方出现了一个准粒子,它会激发费米海内部的准粒子,从而产生一个空穴。
因此,在基态(不是真空态,而是带有费米球的态),费米面上方的准粒子的确会发生衰变。
将费米面上方的准粒子和费米球内部的空穴看成两种元激发可以更加清楚地看到这一点:此时费米面上方的准粒子的个数根本不是守恒的(从而,积掉费米海会产生一个非幺正的理论),从而其自能修正不可能是实数。

设准粒子寿命为$\tau$,则$\tau$反比于散射速率,而散射速率正比于库仑相互作用的强度。完整地做这个计算是很困难的,因为涉及到静电屏蔽等复杂的效应。
由于我们只做数量级估计,暂时将寿命对整个费米面做平均,从而使用一个常数$M$表示相互作用强度。
散射的过程可以概括为:一个动量为$\vb*{p}$的费米面外部的粒子(实际上是费米液体中的准粒子,下同)的能量降低,变成了动量为$\vb*{p}_1$的粒子,同时激发了一个费米面内的动量为$\vb*{p}_2$的粒子。
结果是,动量为$\vb*{p}$的费米面以外的粒子衰变成了两个粒子,动量分别为$\vb*{p}_1$和$\vb*{p} - \vb*{p}_1 + \vb*{p}_2$,还有一个动量为$\vb*{p}_2$的空穴。
设动量分别为$\vb*{p}_1$和$\vb*{p} - \vb*{p}_1 + \vb*{p}_2$的两个粒子和动量为$\vb*{p}_2$的空穴的总态密度在当前温度下的期望值为$n$,由费米黄金法则有
\[
    \frac{1}{\tau} \propto \text{transition rate} \sim \abs{M}^2 n.
\]
由于系统中的粒子非常多,不同能级上粒子数的涨落可以略去,即认为不同能级上不多不少正好就有费米-狄拉克分布给出的粒子个数,%
\footnote{这是热力学系统的一般性质:系统规模大时涨落可略去。由于本文涉及的系统都是多体系统,总是可以做这样的近似。}%
那么就有
\[
    n = \int \dd[3]{\vb*{p}_1} \int \dd[3]{\vb*{p}_2} (1 - f(\epsilon_{\vb*{p}_1})) f(\epsilon_{\vb*{p}_2}) (1 - f(\epsilon_{\vb*{p} - \vb*{p}_1 + \vb*{p}_2})) \delta(\epsilon_{\vb*{p}} - \epsilon_{\vb*{p}_1} + \epsilon_{\vb*{p}_2} - \epsilon_{\vb*{p} - \vb*{p}_1 + \vb*{p}_2}).
\]
因子$(1-\epsilon_{\vb*{p}_1})$表示动量为$\vb*{p}_1$的粒子应该占据一个空态(或者说在接近零温时应该在费米面以外),因子$f(\epsilon_{\vb*{p}_2})$表示空穴一定来自一个已有的粒子,最后的$\delta$函数强制要求能量守恒。
我们不严格计算这个积分,而是做一些数量级估计。
由于$\vb*{p}_2$在费米面以下而$\vb*{p}- \vb*{p}_1 + \vb*{p}_2$在费米面以上,容易写出以下不等式
\[
    0 < \xi_{\vb*{p}_1} < \xi_{\vb*{p}}, \quad 0 < - \xi_{\vb*{p}_2} < \xi_{\vb*{p}} - \xi_{\vb*{p}_1} < \xi_{\vb*{p}},
\]
对$n$有贡献的$\vb*{p}_1$和$\vb*{p}_2$均满足这些不等式,这些不等式给出了两个宽度为$\xi_{\vb*{p}}$的球壳,因此
\[
    n \leq (4 \pi k_\text{F}^2 \xi_{\vb*{p}})^2,
\]
于是
\begin{equation}
    \frac{1}{\tau} \lesssim \xi_{\vb*{p}}^2.
\end{equation}
因此,如果准粒子非常接近费米液体的费米面,那么它是非常稳定的,因为此时$\xi_{\vb*{p}}$很小。从物理图像上看,此时的准粒子虽然会和费米海中的准粒子发生相互作用,但其能量不足以产生粒子-空穴对,因此也不会衰变。

粒子“稳定”的数值判据是什么?按照费米-狄拉克分布,$\expval*{{n}}$在$\epsilon_\text{F}$附近一个大约长为$T$的区域内明显偏离阶跃函数;另一方面,由于相互作用能量本身会有一个弥散,为
\[
    \Delta E \sim \frac{1}{\Delta t} \sim \frac{1}{\tau},
\]
其中$\tau$为准粒子平均自由时间。粒子稳定意味着
\[
    \Delta E \ll \frac{1}{T},
\]
也即
\[
    \frac{1}{\tau} \ll T.
\]
准粒子平均自由时间本身和温度有关,它大约是单位时间发生碰撞的概率的倒数,而只有费米子附近准粒子数明显偏离阶跃函数的那一部分准粒子比较有可能发生碰撞(费米球内部的准粒子不怎么会被激发,费米球外面根本没有准粒子),因此
\[
    \frac{1}{\tau} \sim T^2,
\]
最后就发现我们有
\[
    T \ll 1.
\]
因此费米液体图像只在低温下适用。

% TODO
然而,设$\Delta x$为准粒子坐标的不确定度,数量级上有
\[
    \frac{1}{\Delta x} \sim \frac{1}{\Delta t} \sim \Gamma,
\]
其中$\Gamma$是单位时间的散射几率,而我们又有
\[
    \Delta p \Delta x \sim \hbar,
\]
那么就有
\[
    \Delta p \sim \Gamma.
\]
由于准粒子散射是二体过程,在低温下$\Gamma$正比于$\expval*{{n}}$在$\epsilon_\text{F}$附近明显偏离阶跃函数的区域的厚度的平方。
可以证明低温下这确实是正确的。

总之,准粒子之间的碰撞不可忽略意味着费米面以上的准粒子会衰变。
然而,在费米面附近,这种衰变是非常缓慢的,以至于我们可以认为任何一种动量确定的准粒子分布都可以长期稳定存在,即哈密顿量在以动量标记的粒子数表象下是对角化的。
因此,费米液体的能级结构和费米气体完全一致。

\subsubsection{哈密顿量和相互作用形式}

然而,费米液体的能级结构和费米气体完全一致并不意味着电子之间的相互作用完全不产生任何影响,因为每个能级具体的能量大小可以经历修正。
此外,即使电子之间的相互作用可以忽略,向系统加入或取出电子也会让费米球发生变化,从而让系统总能量平摊到每个准粒子上的份额发生变化。
这意味着准粒子的哈密顿量并不是简单的
\[
    H = \sum_{\vb*{k}, \sigma} \epsilon_{\vb*{k}} n_{\vb*{k} \sigma},
\]
而有一些高阶项,它们表示密度-密度相互作用。

考虑一个能量本征态,其中准粒子在费米面之上的数量为$\var{n}$($\delta$表示相对基态的偏离,正表示有粒子,负表示有空穴)。
考虑到费米液体理论的第一条假设,把能量本征值相对于零温平衡态(由于费米海的结构可以非常复杂,零温能量反而是算不出来的,我们也不需要计算它)的变化写成以下级数展开($\vb*{k}$在费米面附近):
\begin{equation}
    \var{E} = \underbrace{\sum_{\vb*{k}, \sigma} \epsilon^0_{\vb*{k}} \var{n_{\vb*{k} \sigma}}}_{\var{E_1}} + \underbrace{\frac{1}{2V} \sum_{\vb*{k}, \vb*{k}', \sigma, \sigma'} f_{\sigma \sigma' \vb*{k} \vb*{k}'} \var{n_{\vb*{k} \sigma}} \var{n_{\vb*{k}' \sigma'}}}_{\var{E_2}},
    \label{eq:fermi-liquid-energy}
\end{equation}
其中$\var{n}$表示准粒子数目相对基态的变化。把能量写成粒子数的函数假定了自旋守恒。对动量做求和化积分,就得到
\begin{equation}
    \frac{\var{E}}{V} = \underbrace{\sum_{\sigma} \int \frac{\dd[3]{\vb*{k}}}{(2\pi)^3} \epsilon^0_{\vb*{k}} \var{n_{\vb*{k} \sigma}}}_{\var{E_1} / V} + \underbrace{\frac{1}{2} \sum_{\sigma, \sigma'} \int \frac{\dd[3]{\vb*{k}}}{(2\pi)^3} \int \frac{\dd[3]{\vb*{k}'}}{(2\pi)^3} f_{\sigma \sigma' \vb*{k} \vb*{k}'} \var{n_{\vb*{k} \sigma}} \var{n_{\vb*{k}' \sigma'}}}_{\var{E_2} / V}
\end{equation}
我们保留到二阶项,一阶项给出准粒子的自由理论,二阶项给出准粒子的相互作用。
这种相互作用并不会让粒子数发生涨落或是让单个准粒子的动量发生涨落,而只会对能级做修正,是所谓的“前向散射”。

重整化群可能给出更高阶项,而也许二阶项实际上可以忽略,而绕了一大圈之后发现$\abs{M}$很小,从而实际上在很大的区域内费米液体图像均适用。
为了表明我们保留到二阶项是正确的,我们给出一个数量级估计,说明一阶项和二阶项的量级是同阶的,而低于更高阶项。
我们知道由于相互作用的存在,总能量$E$肯定不是单粒子哈密顿量(比如说$k^2/2m$这种形式)的期望值简单加起来的结果,但是显然能量具有可加性,设想我们改变了准粒子数分布,这样应该有
\[
    \var{E} = \sum_{\vb*{k}, \sigma} \epsilon_{\vb*{k} \sigma} \var{n_{\vb*{k} \sigma}},
\]
其中$\epsilon_{\vb*{k}}$是在有限温度下的近平衡态激发一个准粒子的能量,它的一部分是单准粒子能量,一部分是其它准粒子给它的相互作用能之和。
因此,$\epsilon_{\vb*{k}}$会依赖准粒子数分布。由于我们只研究二体相互作用,我们有
\[
    \epsilon \sim \sum_{\vb*{k}'} \text{something} \cdot n_{\vb*{k}'},
\]
于是设
\[
    \var{\epsilon_{\vb*{k} \sigma}} = \frac{1}{V} \sum_{\vb*{k}', \sigma'} f_{\sigma \sigma' \vb*{k} \vb*{k}'} \var{n_{\vb*{k}' \sigma'}},
\]
记$\epsilon_{\vb*{k}}^0$为$n_{\vb*{k}}$一概为零的$\epsilon_{\vb*{k}}$,代入$\var{E}$中就得到\eqref{eq:fermi-liquid-energy};第二项的$1/2$因子是因为一对粒子会被计数两次,所以要除以$2$;由于我们假定准粒子分布相对于零温只有微小的偏离,$\epsilon_{\vb*{k}}$被取为零温的值。
虽然$\epsilon^0 \var{n}$看起来比$f\var{n} \var{n}$大,但需要注意到我们在巨正则系综中工作,则真的有意义的应该是$E-\mu N$(且由于是近平衡,应有$\var{E} = \mu \var{N}$),而
\[
    \sum_{\vb*{k}} (\epsilon^0_{\vb*{k}} - \mu) \var{n_{\vb*{k}}} \sim \var{n}^2,
\]
于是$\epsilon^0 \var{n}$项和$f\var{n} \var{n}$项的贡献是同阶的,都需要考虑。
更高阶相互作用则涉及$\var{n}$的更高阶项,于是不考虑。因此,\eqref{eq:fermi-liquid-energy}的确是成立的。

在已经知道了$E$的表达式之后(比如说微扰计算出了体系能量),可以用变分计算出各个物理量:
\begin{equation}
    \epsilon_{\vb*{k}} = \fdv{E}{n_{\vb*{k} \sigma}} , \quad f_{\sigma \sigma' \vb*{k} \vb*{k}'} = V \frac{\var[2]{E}}{\var{n_{\vb*{k} \sigma}}\var{n_{\vb*{k}' \sigma'}}}, \quad \mu = \pdv{E}{N}.
\end{equation}

\eqref{eq:fermi-liquid-energy}中的一阶项可以看成一个等效的单粒子能量。由于只讨论费米面附近的理论,我们让能量从费米面量起,即使用$\xi$代替$\epsilon$,$k=k_\text{F}$时$\xi^0_{\vb*{k}}$就是零,在假定费米面具有旋转对称性的情况下可以做展开
\[
    \xi^0_{\vb*{k}} = \frac{k_\text{F}}{m^*} (k - k_\text{F}).
\]
我们仿照自由粒子的能量
\[
    \xi_{\vb*{k}} = \frac{k^2}{2m} - \frac{k_\text{F}^2}{2m} \approx \frac{k_\text{F}}{m} (k - k_\text{F})
\]
得到了一个等效质量$m^*$。能够像上面这样做的前提是准粒子能谱要足够光滑,如果像声子那样,就没法定义任何等效质量。%
\footnote{应注意此处的等效质量和“激发有能隙,是有质量的”中的“质量”是不同的;前者并不代表有一个能隙,而只是$\epsilon_{\vb*{k}}$的$k^2$项的系数而已。}%
如果温度很高,以至于不能保证有趣的行为仅仅发生在费米面附近,那有效质量的概念也没有什么用;实际上此时费米液体的理论就没有什么用。
请注意\eqref{eq:fermi-liquid-energy}完全是形式上的:诸如晶格势能造成的单粒子能量修正已经被纳入了$\var{E_1}$中,而只要费米面保持旋转对称性,就可以引入等效质量的概念。
并且,在只有费米面附近才有明显的激发的情况下,可以不失一般性地设
\[
    \epsilon_{\vb*{k}}^0 = \frac{k^2}{2m^*},
\]
因为真正有意义的是$\epsilon_{\vb*{k}} - \mu$,只需要同时调节$\epsilon_{\vb*{k}}$和$\mu$就可以让准粒子的$\epsilon_{\vb*{k}}$取自由粒子的形式。
再次强调,调节$\epsilon_{\vb*{k}}$和$\mu$之类的操作只适用于费米面附近;因此对一个费米液体我们通常避免讨论费米球深处有什么——这些东西对费米面附近的行为并不重要。

对二阶项,假定系统具有自旋旋转不变性,则$f$的取值完全由$f_{\uparrow \uparrow \vb*{k} \vb*{k}'}$和$f_{\uparrow \downarrow \vb*{k} \vb*{k}'}$决定。
实际上,由于$\vb*{k}$局限在费米面附近,我们有
\[
    f_{\alpha \beta \vb*{k} \vb*{k}'} = f_{\alpha \beta}(\vartheta),
\]
$\vartheta$是$\vb*{k}$和$\vb*{k}'$的夹角。这样,设
\begin{equation}
    \begin{aligned}
        f_{\uparrow \uparrow}(\vartheta) &= f^\text{S}(\vartheta) + f^\text{A}(\vartheta), \\
        f_{\uparrow \downarrow}(\vartheta) &= f^\text{S}(\vartheta) - f^\text{A}(\vartheta),
    \end{aligned}
\end{equation}
或者等价地可以设
\begin{equation}
    {f}(\vartheta) = f^\text{S}(\vartheta) + {\sigma} {\sigma}' f^\text{A}(\vartheta)
\end{equation}
从而将自旋守恒这一事实一并表示出来(${\sigma}^z$就是$z$方向的泡利矩阵),并将$f^\text{S}(\vartheta)$和$f^\text{A}(\vartheta)$展开成无量纲常数:
\begin{equation}
    \frac{k_\text{F} m^*}{\pi^2} f^\text{S,A}(\vartheta) = \sum_{l=0}^\infty F_l^\text{S,A} \legpoly (\cos \vartheta).
\end{equation}
于是,给定参数$m^*$,$k_\text{F}$以及$\{F_l^\text{S,A}\}$,费米液体服从的物理规律就给定了。
在这里,我们实际上又把准粒子当成了可以彼此散射、有相互作用的粒子,“准粒子动能”$k^2/2m^*$和“准粒子势能”$f_{\alpha \beta}(\vartheta)$是“单个准粒子能量”$\epsilon_{\vb*{k}}$的两部分;单单一个$k^2/2m^*$肯定和$\epsilon_{\vb*{k}}$是不一样的。

实际上,如果一个费米液体系统可以确定是一个实际的费米气体加入相互作用的结果,并且如前所述,能够保证准粒子个数和实际费米子的个数一样,自旋相同,等等,并且保证自旋旋转不变性、空间平移不变性、空间各向同性,那么费米液体中准粒子的等效质量和实际费米子的质量有一个简单的,使用$\{F_l^\text{S,A}\}$写出的关系。
由于
\[
    E - E_0 = \sum_{\vb*{k}} \var{n_{\vb*{k}}} \epsilon_{\vb*{k}},
\]
由动量为$\vb*{k}$的一个准粒子的运动速度为
\[
    \vb*{v}_i = \pdv{E}{\vb*{p}_i} = \pdv{\epsilon_{\vb*{k}}}{\vb*{k}},
\]
上式的量子版本就是
\[
    {\vb*{v}} = \pdv{{\epsilon}_{\vb*{k}}}{\vb*{k}}.
\]
我们于是可以将${\vb*{v}}$当成费米液体中准粒子的流速。设$\rho$是某种守恒荷的密度,则任意一个流算符的期望可以写成
\[
    \begin{aligned}
        \expval*{\rho \vb*{v}} = V \int \frac{\dd[3]{\vb*{k}}}{(2\pi)^3} \expval{\rho \pdv{{\epsilon}}{\vb*{k}}} , 
    \end{aligned}
\]
准粒子和实际的费米子数量相同,准粒子的粒子数流密度算符就是实际费米子的粒子数流密度算符,且由于动量守恒,一个准粒子的总动量就是与它对应的实际费米子的总动量%
\footnote{
    可以这样论证这件事:在关闭相互作用时费米液体“无缝地”退化到实际的费米气体上,因此在无相互作用点处费米液体中准粒子的动量就是实际粒子的动量。
    现在缓慢地加上相互作用,则费米液体准粒子的动量可以发生连续变化,但有限体系中动量实际上是分立的,从而动量只能不变。
}%
% TODO:这里的推导细节有问题
,而实际费米子的总动量就是总质量流(因为$\vb*{p}=m\vb*{v}$);由于动量和自旋守恒,我们将费米子占据数算符用$\vb*{k}$和$\sigma$标记,于是我们有
\[
    \sum_{\vb*{k}, \sigma} \int \frac{\dd[3]{\vb*{k}}}{(2\pi)^3} \vb*{k} {n}_{\vb*{k} \sigma} = \trace \int \frac{\dd[3]{\vb*{k}}}{(2\pi)^3} m \pdv{{\epsilon}}{\vb*{k}} {n}.
\]
同样${\epsilon}$也适用一样的推导,于是就有
\[
    \int \frac{\dd[3]{\vb*{k}}}{(2\pi)^3} m \pdv{\epsilon_{\vb*{k} \sigma}}{\vb*{k}} n_{\vb*{k} \sigma} = \int \frac{\dd[3]{\vb*{k}}}{(2\pi)^3} \vb*{k} n_{\vb*{k} \sigma},
\]
对上式求变分,就有
\[
    \begin{aligned}
        \int \frac{\dd[3]{\vb*{k}}}{(2\pi)^3} \vb*{k} \var{n_{\vb*{k} \sigma}} &= \var \int \frac{\dd[3]{\vb*{k}}}{(2\pi)^3} m \pdv{\epsilon_{\vb*{k} \sigma}}{\vb*{k}} n_{\vb*{k} \sigma} \\
        &= m \int \frac{\dd[3]{\vb*{k}}}{(2\pi)^3} \pdv{\epsilon_{\vb*{k} \sigma}}{\vb*{k}} \var{n_{\vb*{k} \sigma}} + m \int \frac{\dd[3]{\vb*{k}}}{(2\pi)^3} \int \frac{\dd[3]{\vb*{k}'}}{(2\pi)^3} n_{\vb*{k} \sigma} \var{n_{\vb*{k}' \sigma}} \pdv{f_{\sigma}(\vartheta)}{\vb*{p}} \\
        &= m \int \frac{\dd[3]{\vb*{k}}}{(2\pi)^3} \pdv{\epsilon_{\vb*{k} \sigma}}{\vb*{k}} \var{n_{\vb*{k} \sigma}} - m \int \frac{\dd[3]{\vb*{k}}}{(2\pi)^3} \int \frac{\dd[3]{\vb*{k}'}}{(2\pi)^3} \var{n_{\vb*{k} \sigma}} \pdv{n_{\vb*{k}' \sigma}}{\vb*{k}'} f_{\sigma}(\vartheta) .
    \end{aligned}
\]
第三个等号交换了$\vb*{k}$和$\vb*{k}'$,但这是合理的,因为$f$只和这两者的夹角有关。
考虑到$\var{n}$的任意性,就有
\[
    \frac{\vb*{k}}{m} = \pdv{\epsilon_{\vb*{k} \sigma}}{\vb*{k}} - \int \frac{\dd[3]{\vb*{k}'}}{(2\pi)^3} \pdv{n_{\vb*{k}' \sigma}}{\vb*{k}'} f_{\sigma}(\vartheta).
\]
在上式两边点乘$\vb*{k}$,代入$n$是阶跃函数这一事实,并且注意到动量几乎总是在费米面上,从而$\vb*{k} = k_\text{F} \vb*{n}$,就得到
\begin{equation}
    \frac{1}{m} = \frac{1}{m^*} + \frac{k_\text{F}}{(2\pi)^3} \int \dd{\Omega} \cos \vartheta f_\sigma(\vartheta).
    \label{eq:m-and-m-star}
\end{equation}
上式的形式其实有些容易让人误解,毕竟,$f$和$m^*$都是加入相互作用之后重整化得到的有效参数,而上式看起来似乎是“$f$和$m$决定了$m^*$”。
这里还有一个疑难,就是在使用电子动量计算总动量时我们直接用了$p=mv$,而以费米液体的观点计算总动量时我们却没有这么做。
这是因为,只是根据\eqref{eq:fermi-liquid-energy}计算出的动量并不是真正的总动量,庞大的费米海的动量被忽略了:当一个费米液体中的准粒子被激发起来时,实际的系统中的费米海会受到扰动,从而会有额外的动量。
$f$通常和费米能级有关,因此可以提供一些关于“费米海有多重”的信息,这就是\eqref{eq:m-and-m-star}中会出现$f$的原因:\eqref{eq:m-and-m-star}来自动量守恒关系,动量守恒方程的一边含有$m$,另一边含有$m^*$和通过$f$反解出的费米海的动量,经过化简就得到\eqref{eq:m-and-m-star}。
当然,如果实际系统中根本没有电子间排斥,那么$f$肯定一直是零。

费米液体的思想是传统凝聚态物理的基础:在费米液体提出之后的很长一段时间,分析凝聚态系统的标准方式是将系统的基本激发当成某种经过重整化的电子(如费米液体理论中的准粒子),写出形式各异的无相互作用的能带,然后加入占主导地位的相互作用通道。
一些新兴的系统,如Luttinger液体,则无法归入这个框架:一点相互作用就足以破坏能带,从而让系统的低能自由度看起来完全不像电子。

\subsubsection{费米液体的宏观性质}

使用以上参数:$m^*$,$k_\text{F}$以及$\{F_l^\text{S,A}\}$,可以计算费米液体的各种宏观性质。

首先考虑零温附近的比热。费米气体的比热在低温极限下正比于温度,费米液体实际上也一样。
能量由\eqref{eq:fermi-liquid-energy}给出,随着$T$增大,一些粒子从费米海溢出,从而能量增大,产生一个热容。
实际上,在零温极限附近,\eqref{eq:fermi-liquid-energy}中的$E_2$部分没有贡献。
这是因为
\[
    E_2 = \sum_{\sigma, \sigma'} \underbrace{\frac{1}{2V} \sum_{\vb*{k}} f_{\sigma \sigma'}(\theta) \var{n}_{\vb*{k} \sigma}}_{\text{constant}} \var{n}_{\vb*{k}' \sigma'},
\]
被大括号括起来的部分和$\vb*{k}$无关,而显然
\[
    \sum_{\vb*{k}} \var{n}_{\vb*{k} \sigma} = 0,
\]
因此$E_2$对总能量没有贡献。这样费米液体的热容和费米气体的热容就是完全一致的,为
\begin{equation}
    C_V = \frac{1}{3} m^* k_\text{F} T.
\end{equation}
这个公式在实验上非常重要,如果确定一个体系是费米液体(如发现低温下热容正比于温度),那么就可以据此测出粒子的有效质量。

也可以计算费米液体的磁化率。考虑弱场近似,则磁化率
\[
    \chi = \pdv{M}{H}
\]
近似为
\[
    \chi = \frac{M}{H},
\]
其中$M$表示磁化强度,$H$表示磁场强度(不是哈密顿量),而磁化强度为
\[
    M = \pdv{E}{H},
\]
于是得到
\[
    \frac{1}{\chi} = \pdv[2]{E}{M}.
\]
这样只需要使用$M$表示出$E$就可以了。
记自旋向上(以磁场方向为$z$轴)和向下的粒子数为$N_\uparrow$和$N_\downarrow$,则
\[
    M = \mu_\text{B} (N_\uparrow - N_\downarrow),
\]
其中$\mu_\text{B}$为玻尔磁子。磁场导致自旋向上和向下的粒子数发生变化的原因是,自旋和磁场一致的粒子的费米面会扩大,自旋和磁场相反的粒子的费米面会缩小,从而让$N_\uparrow$变大,$N_\downarrow$缩小。
由于粒子数不变,有
\[
    \var{N_\uparrow} = - \var{N_\downarrow},
\]
而没有磁场时向上和向下的粒子数一样,于是
\[
    M = 2 \mu_\text{B} \var{N}_\uparrow.
\]
$\var{N_\uparrow}$和费米动量的变化之间的关系是
\[
    \var{N_\uparrow} = \int_{k_\text{F} < k < k_\text{F} + \var{k_\text{F}}} \frac{V}{(2\pi)^3} \dd[3]{\vb*{k}} = \frac{V k_\text{F}^2 \var{k_\text{F}}}{2\pi^2}.
\]
现在可以将$M$用$\var{k_\text{F}}$表示出来了。接下来将能量写成$\var{k_\text{F}}$的函数。
对动能部分$E_1$,我们有
\[
    \var{E_1} = \sum_{\sigma, \vb*{k}} \frac{k_\text{F}}{m^*} (k - k_\text{F}) \var{n}_{\vb*{k} \sigma},
\]
$n_{\vb*{k} \uparrow}$仅有的变化是在$k_\text{F} < k < k_\text{F} + \var{k_\text{F}}$的区域内从$0$变成$1$,$n_{\vb*{k} \downarrow}$仅有的变化是在$k_\text{F} - \var{k_\text{F}} < k < k_\text{F}$的区域内从$1$变成$0$。
这样就有
\[
    \begin{aligned}
        \var{E_1} &= \int_{k_\text{F} < k < k_\text{F} + \var{k_\text{F}}} \frac{V}{(2\pi)^3} \dd[3]{\vb*{k}} \frac{k_\text{F}}{m^*} (k - k_\text{F}) + \int_{k_\text{F} - \var{k_\text{F}} < k < k_\text{F}} \frac{V}{(2\pi)^3} \dd[3]{\vb*{k}} \frac{k_\text{F}}{m^*} (k - k_\text{F}) (-1) \\
        &= \frac{V k_\text{F}^3}{2 \pi^2 m^*} (\var{k_\text{F}})^2.
    \end{aligned}
\]
最后,得到$\var{E_1}$和$M$的关系:
\[
    \var{E_1} = \frac{\pi^2}{2 m^* \mu_\text{B}^2 V k_\text{F}} M^2.
\]
同理,可以计算得到(计算的关键点在于意识到对全空间计算积分,则只有零阶勒让德多项式能够给出非零结果)
\[
    \var{E_2} = \frac{\pi^2}{2 m^* \mu_\text{B}^2 V k_\text{F}} F_0^\text{A} M^2.
\]
这样就得到了$\var{E}$关于$M$的表达式,从而最终得到
\begin{equation}
    \chi = \frac{1}{1 + F_0^\text{A}} \frac{m^* \mu_\text{B}^2 V k_\text{F}}{\pi^2}.
\end{equation}

\subsubsection{费米液体理论的合理性}

确定一个系统是否适用费米液体理论首先需要我们确定系统的低能激发是否真的与电子的行为类似,因为原则上完全可以出现诸如电子配对等现象。
在确定系统的低能激发和电子行为类似之后,费米球的存在性,以及低能激发主要是费米面附近的准粒子和空穴就可以确定。
此时需要讨论的主要是相互作用的类型,即是否前向散射占据主要地位。

在重整化群计算中,一个二体散射项的强弱主要由这个散射项中允许的的粒子动量的取值多少确定,可能的动量占据的空间体积越大,对应的相互作用通道就越强。

懒得写了,总之就是BCS相互作用和密度-密度相互作用是特别强的两种。

% TODO:那Hubbard模型?

\subsection{紧束缚模型}

\subsubsection{单能带系统}

对单带模型,在Wannier表象下,库仑相互作用可以表示成如下矩阵元形式:
\begin{equation}
    \begin{aligned}
        H_{i i' j' j} &= \frac{1}{2} \int \dd[3]{\vb*{r}} \int \dd[3]{\vb*{r}'} \braket{i}{\vb*{r}} \braket{i'}{\vb*{r}'} V(\vb*{r} - \vb*{r}') \braket{\vb*{r}'}{j'} \braket{\vb*{r}}{j} \\
        &= \frac{1}{2} \int \dd[3]{\vb*{r}} \int \dd[3]{\vb*{r}'} w_i^*(\vb*{r}) w_{i'}^*(\vb*{r}') V(\vb*{r} - \vb*{r}') w_{j'}^*(\vb*{r}') w_j^*(\vb*{r}),
    \end{aligned}
    \label{eq:wannier-basis-interaction}
\end{equation}
于是总的哈密顿量为
\begin{equation}
    H = \sum_{i, i', \sigma} c^\dagger_{i \sigma} t_{i i'} c_{j \sigma} + \sum_{i, i', j, j', \sigma, \sigma'} V_{i i' j' j} c^\dagger_{i \sigma} c^\dagger_{i' \sigma'} c_{j' \sigma'} c_{j \sigma}.
\end{equation}

上面的哈密顿量是最为一般的,原则上可以描述一切晶体中的电子系统。
例如,原则上,自由电子气可以通过引入非常远的跃迁矩阵元来实现。
在我们假定电子真的只存在最近邻跃迁时,我们有
\begin{equation}
    \sum_{i, i', \sigma} c^\dagger_{i \sigma} t_{i i'} c_{j \sigma} = \sum_{\pair{i, j}, \sigma} t_{ij} c^\dagger_{i \sigma} c_{j \sigma}.
\end{equation}
我们称满足这个条件的自由电子模型为\concept{紧束缚模型}。我们当然还可以引入次近邻跃迁等来修正紧束缚模型。

如果只有最近邻或是次近邻的跃迁,那么我们就有一个非常直观的物理图像:似乎可以用“某个电子正在某个格点附近”来标记一个电子,即电子似乎被紧紧地束缚在晶格上,无法长距离移动,这就是“紧束缚”一词的来历。
或者,也有可能电子并不是特别局域,但是原子间距很大,以至于在系统中电子通常具有的动能下,非常远距离的跃迁似乎无法一次完成。
这就是说,紧束缚模型中,电子的动能不应该能够到达特别大,从而不应该有太大的能带带宽。

与自由电子气模型或凝胶模型不同,紧束缚模型对相互作用是非常敏感的,不能保证加入(即使比较弱的)相互作用后,系统能够用费米液体理论描述,也不能保证能带论仍然有效。
这是出于一个非常直观的原因:紧束缚模型的能带通常更窄,从而态密度更大,从而库伦散射更明显,从而也更加容易在相互作用加入之后变成强关联系统。
或者,我们也可以如此理解紧束缚模型对相互作用的敏感性:由于电子动能不能取特别大的值,相比之下相互作用能量应该占据主导,从而系统具有强关联效应。

总之,以下几个说法基本上是等价的:
\begin{enumerate}
    \item 系统可以用紧束缚模型或者考虑了次近邻、再次近邻的紧束缚模型加上一个相互作用项描述;
    \item 系统中的电子跃迁能力不大;
    \item 系统能带不宽;
    \item 系统中电子动能有较低的上限。
\end{enumerate}
满足这些条件的模型通常对相互作用更加敏感,也更加容易出现强关联效应。
这就导致了一个非常有趣的现象:一些数值计算方法,如DMRG,需要将所有自由度都定义在晶格上,从而,它们可以毫无困难地用于模拟很大一类强关联系统,但是却无法用于有效地模拟普通的费米液体系统。

需要注意,并非所有强关联系统都出现在紧束缚模型中。费米液体系统加入了适当的相互作用之后也可以出现强关联效应,此时使用基于格点的计算方法就不能获得很好的效果了。

原则上$U_{i i' j' j}$对很多不同的$i, i', j', j$组合都有非零值。物理上这是很好理解的,因为库伦相互作用可以将任意原子附近的两个电子散射到任意的其它地方,由于高度定域的电子动量不确定,这里无所谓动量守恒的约束。
然而,由于只有原子间距很大时,紧束缚模型才成立,实际上,足够大的矩阵元的$i, i', j', j$中,$j$和$j'$不是最近邻,就是完全一样(即所谓on-site repulsion)。
由于Wannier波函数高度定域,从\eqref{eq:wannier-basis-interaction}中可以看出,仅有的可能是$i=i'=j=j'$,或是$i=j \neq i'=j'$,或是$i=j' \neq i' = j$。
于是就有
\[
    \begin{aligned}
        &\quad \sum_{i, i', j, j', \sigma, \sigma'} V_{i i' j' j} c^\dagger_{i \sigma} c^\dagger_{i' \sigma'} c_{j' \sigma'} c_{j \sigma} \\
        &= \sum_{i, \sigma, \sigma'} V_{iiii} c^\dagger_{i \sigma} c^\dagger_{i \sigma} c_{i \sigma} c_{i \sigma} 
        + \sum_{i, j, \sigma, \sigma'} V_{i j j i} c^\dagger_{i \sigma} c^\dagger_{j \sigma'} c_{j \sigma'} c_{i \sigma}
        + \sum_{i, j, \sigma, \sigma'} V_{i j i j} c^\dagger_{i \sigma} c^\dagger_{j \sigma'} c_{i \sigma'} c_{j \sigma},
    \end{aligned}
\]
第一项为
\[
    \begin{aligned}
        \sum_{i, \sigma, \sigma'} V_{iiii} c^\dagger_{i \sigma} c^\dagger_{i \sigma} c_{i \sigma} c_{i \sigma} &= 2 \sum_{i} V_{iiii} n_{i \uparrow} n_{i \downarrow} + \sum_{i} U_{iiii} n_i \\
        &= \sum_{i} U_i n_{i \uparrow} n_{i \downarrow} + \frac{1}{2} \sum_{i} U_{i} n_i,
    \end{aligned}
\]
这里我们重新定义
\begin{equation}
    U_i = 2 V_{iiii}.
\end{equation}
第二项实际上是密度-密度相互作用,即为
\[
    \sum_{i, j, \sigma, \sigma'} V_{i j j i} c^\dagger_{i \sigma} c^\dagger_{j \sigma'} c_{j \sigma'} c_{i \sigma} = \sum_{\pair{i, j}} V_{ij} n_i n_j,
\]
其中我们重新定义了$V_{ij} = V_{ijji}$。使用公式
\[
    \vb*{\sigma}_{\alpha \beta} \cdot \vb*{\sigma}_{\alpha' \beta'} = 2 \delta_{\alpha \beta'} \delta_{\beta \alpha'} - \delta_{\alpha \beta} \delta_{\alpha' \beta'},
\]
可以验证
\[
    \sum_{i, j, \sigma, \sigma'} V_{i j i j} c^\dagger_{i \sigma} c^\dagger_{j \sigma'} c_{i \sigma'} c_{j \sigma} = - 2 \sum_{\pair{i, j}} V_{i j i j} \left( \vb*{S}_{i} \cdot \vb*{S}_j + \frac{1}{4} n_i n_j \right).
\]
我们把上式右边的第二项吸收进$V_{ij}$中,并且重新定义$2V_{ijij}=J_{ij}$,于是总的相互作用哈密顿量就是
\[
    \sum_{i} U_i n_{i \uparrow} n_{i \downarrow} + \sum_{\pair{i, j}} V_{ij} n_i n_j - \sum_{\pair{i, j}} J_{i j} \vb*{S}_{i} \cdot \vb*{S}_j + \sum_{i} U_i n_i.
\]
由空间平移对称性,on-site repulsion肯定是均一的,这样我们可以将$\sum_i U_i n_i$丢进化学势中;而如果它不是均一的,就意味着系统实际上不具有完美的离散平移对称结构,即系统中存在无序,直观地看,就是散布的杂质对电子产生散射,这基本上是一个单体算符,于是我们就得到了紧束缚模型最一般的哈密顿量:
\begin{equation}
    H = - \sum_{\pair{i, j}, \sigma} t_{ij} c^\dagger_{i \sigma} c_{j \sigma} 
    + \sum_{i} U_i n_{i \uparrow} n_{i \downarrow} 
    + \sum_{\pair{i, j}} V_{ij} n_i n_j - \sum_{\pair{i, j}} J_{i j} \vb*{S}_{i} \cdot \vb*{S}_j 
    - \mu \sum_i n_i + \sum_{i, j, \sigma, \sigma'} \epsilon_{ij\alpha \beta} c^\dagger_{i \alpha} c_{j \beta}.  
\end{equation}
这里的每一项都是有意义的,从左到右分别是:
\begin{enumerate}
    \item 动能项,衡量电子在格点之间跳跃的可能性;
    \item on-site repulsion,仅考虑这个相互作用得到的模型是所谓\concept{Hubbard模型};
    \item 密度-密度相互作用项,可能导致系统中出现持续的、不能平息的电子密度涨落,即所谓电荷密度波;
    \item 自旋-自旋相互作用项,可能导致出现自旋密度波,这一项意味着在有多个轨道的情况下,每个轨道上放置一个电子,且所有电子的自旋都平行时,系统能量最低,这就是所谓洪特规则;
    \item 化学势,调节系统中电子个数;
    \item 无序,可能来自杂质或是晶格的缺陷。
\end{enumerate}

\subsubsection{多能带的情况}

以上的讨论集中在单能带模型中;多个能带的情况允许一些新的能带之间的相互作用通道。

实际的原子内部的洪特规则应该就是在这里能够拿到

\subsubsection{能带和原子电子轨道}


总之,在原子轨道波函数之间真的没有什么交叠的情况下,原子轨道波函数就是Wannier波函数。
然而,在大多数情况下,原子轨道波函数还是比较不局域的,我们需要先从原子轨道波函数出发计算得到Bloch波函数,然后做傅里叶变换得到比原子轨道波函数更加局域的Wannier波函数。

\subsection{能带理论何时失效?}

简而言之:态密度反比于群速度,从而能带越窄,态密度越大,
平带是最容易出强关联系统的。


\subsection{能带和对称性}

% TODO:能带和群表示

\documentclass[hyperref, a4paper]{article}

\usepackage{geometry}
\usepackage{titling}
\usepackage{titlesec}
% No longer needed, since we will use enumitem package
% \usepackage{paralist}
\usepackage{enumitem}
\usepackage{footnote}
\usepackage{amsmath, amssymb, amsthm}
\usepackage{mathtools}
\usepackage{bbm}
\usepackage{graphicx}
\usepackage{subfigure}
\usepackage{physics}
\usepackage{tensor}
\usepackage{siunitx}
\usepackage[version=4]{mhchem}
\usepackage{tikz}
\usepackage{xcolor}
\usepackage{listings}
\usepackage{underscore}
\usepackage{autobreak}
\usepackage[ruled, vlined, linesnumbered]{algorithm2e}
\usepackage{nameref,zref-xr}
\zxrsetup{toltxlabel}
\usepackage[sorting=none]{biblatex}
\addbibresource{phonon.bib}
\usepackage[colorlinks,unicode]{hyperref} % , linkcolor=black, anchorcolor=black, citecolor=black, urlcolor=black, filecolor=black
\usepackage[most]{tcolorbox}
\usepackage{prettyref}

% Page style
\geometry{left=3.18cm,right=3.18cm,top=2.54cm,bottom=2.54cm}
\titlespacing{\paragraph}{0pt}{1pt}{10pt}[20pt]
\setlength{\droptitle}{-5em}

% More compact lists 
\setlist[itemize]{
    %itemindent=17pt, 
    %leftmargin=1pt,
    listparindent=\parindent,
    parsep=0pt,
}

\setlist[enumerate]{
    %itemindent=17pt, 
    %leftmargin=1pt,
    listparindent=\parindent,
    parsep=0pt,
}

% Math operators
\DeclareMathOperator{\timeorder}{\mathcal{T}}
\DeclareMathOperator{\diag}{diag}
\DeclareMathOperator{\legpoly}{P}
\DeclareMathOperator{\primevalue}{P}
\DeclareMathOperator{\sgn}{sgn}
\DeclareMathOperator{\res}{Res}
\newcommand*{\ii}{\mathrm{i}}
\newcommand*{\ee}{\mathrm{e}}
\newcommand*{\const}{\mathrm{const}}
\newcommand*{\suchthat}{\quad \text{s.t.} \quad}
\newcommand*{\argmin}{\arg\min}
\newcommand*{\argmax}{\arg\max}
\newcommand*{\normalorder}[1]{: #1 :}
\newcommand*{\pair}[1]{\langle #1 \rangle}
\newcommand*{\fd}[1]{\mathcal{D} #1}
\DeclareMathOperator{\bigO}{\mathcal{O}}

% TikZ setting
\usetikzlibrary{arrows,shapes,positioning}
\usetikzlibrary{arrows.meta}
\usetikzlibrary{decorations.markings}
\usetikzlibrary{calc}
\tikzstyle arrowstyle=[scale=1]
\tikzstyle directed=[postaction={decorate,decoration={markings,
    mark=at position .5 with {\arrow[arrowstyle]{stealth}}}}]
\tikzstyle ray=[directed, thick]
\tikzstyle dot=[anchor=base,fill,circle,inner sep=1pt]

% Algorithm setting
% Julia-style code
\SetKwIF{If}{ElseIf}{Else}{if}{}{elseif}{else}{end}
\SetKwFor{For}{for}{}{end}
\SetKwFor{While}{while}{}{end}
\SetKwProg{Function}{function}{}{end}
\SetArgSty{textnormal}

\newcommand*{\concept}[1]{{\textbf{#1}}}

% Embedded codes
\lstset{basicstyle=\ttfamily,
  showstringspaces=false,
  commentstyle=\color{gray},
  keywordstyle=\color{blue}
}

\lstdefinestyle{console}{
    basicstyle=\footnotesize\ttfamily,
    breaklines=true,
    postbreak=\mbox{\textcolor{red}{$\hookrightarrow$}\space}
}

% Reference formatting
\newrefformat{fig}{Figure~\ref{#1}}

% Color boxes
\tcbuselibrary{skins, breakable, theorems}

\newtcbtheorem{infobox}{Box}{
    enhanced,
    boxrule=0pt,
    colback=blue!5,
    colframe=blue!5,
    coltitle=blue!50,
    borderline west={4pt}{0pt}{blue!65},
    sharp corners,
    fonttitle=\bfseries, 
    breakable,
    before upper={\parindent15pt\noindent}}{box}
\newtcbtheorem[use counter from=infobox]{theorybox}{Box}{
    enhanced,
    boxrule=0pt,
    colback=orange!5, 
    colframe=orange!5, 
    coltitle=orange!50,
    borderline west={4pt}{0pt}{orange!65},
    sharp corners,
    fonttitle=\bfseries, 
    breakable,
    before upper={\parindent15pt\noindent}}{box}
\newtcbtheorem[use counter from=infobox]{learnbox}{Box}{
    enhanced,
    boxrule=0pt,
    colback=green!5,
    colframe=green!5,
    coltitle=green!50,
    borderline west={4pt}{0pt}{green!65},
    sharp corners,
    fonttitle=\bfseries, 
    breakable,
    before upper={\parindent15pt\noindent}}{box}

% Displaying texts in bookmarkers

\pdfstringdefDisableCommands{%
  \def\\{}%
  \def\ce#1{<#1>}%
}

\pdfstringdefDisableCommands{%
  \def\texttt#1{<#1>}%
  \def\mathbb#1{#1}%
}
\pdfstringdefDisableCommands{\def\eqref#1{(\ref{#1})}}

\makeatletter
\pdfstringdefDisableCommands{\let\HyPsd@CatcodeWarning\@gobble}
\makeatother

\newenvironment{shelldisplay}{\begin{lstlisting}}{\end{lstlisting}}

\newcommand{\shortcode}[1]{\texttt{#1}}

\lstset{style = console}

% Make subsubsection labeled
\setcounter{secnumdepth}{4}
\setcounter{tocdepth}{4}

\newcommand*{\abinitio}{\textit{ab initio}}

\title{First-principle phonon calculation}
\author{Jinyuan Wu}

\begin{document}

\maketitle

\section{Feynman diagrams for phonons}

The structure of the phonon propagator:
the minus sign between $1 / (\omega \pm \omega_{\vb*{q}})$: 


\section{The molecular dynamics approach}

See \cite{zhang2022finite}. 
The idea is rather simple: 
the phonon spectrum just illustrates 
the vibration modes of atoms,
so if we just let atoms move 
(with an average kinetic energy that agrees 
with the temperature $T$ in question),
then the Fourier transform 
of the orbitals of the atoms 
gains high intensity 
on the dispersion relations.
This seems to be the most \abinitio method I've ever learned about.

A more efficient way is the so-called temperature-dependent effective potential,
which is obtained by curve-fitting of the atomic force fields in MD.

\printbibliography

\end{document}

\section{电子气的电磁响应和库伦相互作用}

本节讨论电子气的电磁响应。由于$\vb*{A}$耦合在$\vb*{j}$上,而能够决定极化电场的也无非是电子数密度,讨论电子气的电磁响应基本上就是要计算密度-密度关联函数。
如果不考虑电子之间的库仑相互作用,那么这是平凡的:诸如“谐振子受到电场策动,然后出现极化”的图像就能工作得很好。
在绝缘体中这大体上是正确的,在金属中则不见得是这样。
因此,比较精确地计算电子气的电磁响应,实际上就是需要将电子-电子库伦相互作用纳入计算,而这就很自然地要求我们使用一种完整的凝聚态场论:以某种电子气——自由电子气,或者紧束缚模型等——为自由理论,以库伦相互作用为微扰。

本节讨论那些把库仑相互作用直接当成微扰尚可取得较好结果的系统。对强关联系统,这个做法是很可能失效的。

\subsection{能带电子电磁响应的波包动力学处理}

\subsubsection{Bloch电子波包}

考虑一个可以被看成晶格动量为$\vb*{k}_0$的Bloch电子波包。显然这个波包由同一个能带中的不同晶格动量成分组成(不同能带的电子具有非常不同的色散关系,它们各自运行的速度非常不同,用它们是无法组成一个比较稳定的波包的;以下略去能带标记$n$),且它们的晶格动量$\vb*{k}'$均很接近$\vb*{k}_0$,假定大体上
\begin{equation}
    - \frac{\Delta}{2} \leq k_i = k_{0i} - k'_i \leq \frac{\Delta}{2},
\end{equation}
从而可以做展开
\begin{equation}
    \epsilon_{\vb*{k}'} = \epsilon_{\vb*{k}_0} + \vb*{k} \cdot (\grad_{\vb*{k}} \epsilon_{\vb*{k}})|_{\vb*{k}_0}.
\end{equation}
波函数的形式形如
\[
    \psi(\vb*{r}, t) = \sum_{\vb*{k}'} a_{\vb*{k}} \psi_{n \vb*{k}}(\vb*{r}, t) = \sum_{\vb*{k}'} a_{\vb*{k}} \ee^{\ii (\vb*{k}' \cdot \vb*{r} - \epsilon_{\vb*{k}'})} u_{\vb*{k}}(\vb*{r}, t),
\]
其中$a_{\vb*{k}}$满足归一化条件。代入$\epsilon_{\vb*{k}'}$的展开式,能够得到
\[
    \psi(\vb*{r}, t) = \ee^{\ii (\vb*{k}_0 \cdot \vb*{r} - \epsilon_{\vb*{k}_0} t)} \sum_{\vb*{k}} a_{\vb*{k}} \ee^{\ii }
\]
\begin{equation}
    \psi(\vb*{r}, t)
\end{equation}

\subsection{自由电子气加入库伦相互作用后对外加电场的响应}\label{sec:ext-e}

\subsubsection{托马斯-费米静态屏蔽势}

本节讨论一个足够稀疏,从而可以忽略交换作用的电子气系统。
由于可以忽略交换作用,可以完全照搬经典静电学,即认为一个电子的全部能量是其动能加上电势能,而不考虑任何交换能或者关联能。
想象我们在一个满足托马斯-费米近似的近独立电子气当中放入一个电荷。显然,异号电荷会移动到前者附近而产生一个屏蔽效应。
在电子气规模很大时(我们所研究的固体总是在热力学极限之下,因此这是成立的),屏蔽效应很强,这让外加电荷对电子气的扰动实际上局限在外加电荷的一个小邻域内。
不管怎么说求解外加电荷的影响就是求解以下自洽问题:
\begin{enumerate}
    \item 外加电荷导致外加电势能
    \begin{equation}
        \phi_\text{ext} = \frac{Q}{r},
    \end{equation}
    \item 外加电势与屏蔽电荷形成的电势$\var{\phi}$叠加,导致总的电势变化为
    \begin{equation}
        \phi_\text{eff} = \phi_\text{ext} + \var{\phi},
    \end{equation}
    \item 总电势变化$\phi_\text{eff}$导致了电荷密度变化$\var{n}$,电荷密度变化给出屏蔽电势$\var{\phi}$。
\end{enumerate}
自洽求解以上问题就可以确定所有物理量。

以上自洽求解步骤还需要一个方程:$V_\text{eff}$是怎么影响电荷密度的。
$V_\text{eff}$改变粒子排布的方式是这样的:单电子的能量$\epsilon_{\vb*{k}}$会由于$V_\text{eff}$的引入发生一个小的变化,且这个小的变化在不同位置通常是不一样的。
由于托马斯-费米近似成立,我们认为每个$\vb*{r}$位置附近的诸电子都可以认为是组成了一个正则系综%
\footnote{这么做总是可以的,因为我们讨论近平衡态理论,确实允许电荷密度在空间上出现分布,虽然完全的平衡态电子气应该是均匀的。}%
,且$\vb*{k}$和$\vb*{r}$可以同时确定,于是
\[
    n(\vb*{r}) = \sum_{\vb*{k}, \sigma} \frac{1}{1 + \ee^{\beta (\epsilon_{\vb*{k}}(\vb*{r}) - \mu)}} = 2 \sum_{\vb*{k}} \frac{1}{1 + \ee^{\beta (\epsilon_{\vb*{k}}(\vb*{r}) - \mu)}}, \quad \epsilon_{\vb*{k}}(\vb*{r}) = \epsilon_{\vb*{k}}^0 + V_\text{eff}(\vb*{r}),
\]
由于体系足够大,屏蔽作用让外加电荷的作用几乎是局域的,则$\var{n}(\vb*{r})$可以写成$V_\text{eff}(\vb*{r})$的函数而不涉及长程相互作用。
而由于外加电荷很小,可以采取线性近似,于是(第二个等号是因为费米-狄拉克分布的形式)
\[
    \var{n} = \dv{n}{V_\text{eff}} V_\text{eff} = \sum_{\vb*{k}} \pdv{n}{\epsilon_{\vb*{k}}} V_\text{eff} = - \pdv{n}{\mu} V_\text{eff},
\]
关于能量的态密度(即单位$\dd{\epsilon}$、单位体积中有多少可能的态)为
\begin{equation}
    N(\mu' - \mu) = \pdv{n\big|_{\mu \to \mu'}}{\mu'},
\end{equation}
其中能量从费米面量起,且我们将$n$的表达式中的$\mu$换成了变量$\mu'$,而使用$\mu$表示实际系统的化学势。使用该记号,则
\begin{equation}
    \var{n}(\vb*{r}) = - N(0) V_\text{eff}(\vb*{r}).
    \label{eq:from-veff-to-varn}
\end{equation}
这就是从$V_\text{eff}$计算$\var{n}$的方法。通过$\var{n}$计算$\var{\phi}$的方程为泊松方程
\[
    - \laplacian \phi_\text{eff} = 4 \pi \left( Q \delta(\vb*{r}) - e \var{n}(\vb*{r}) \right),
\]
其中我们不失一般性地将外加电荷放在原点上。使用傅里叶变换并且使用$\var{n}$和$V_\text{eff}$之间的线性关系\eqref{eq:from-veff-to-varn},可以得到
\[
    k^2 V_\text{eff}(\vb*{k}) = - 4\pi e Q - 4 \pi e^2 N(0) V_\text{eff}(\vb*{k}),
\]
解出$V_\text{eff}(\vb*{k})$为
\[
    V_\text{eff}(\vb*{k}) = - \frac{4 \pi e Q}{k^2 + \kappa
    _\text{TF}^2},
\]
做傅里叶逆变换得到
\begin{equation}
    V_\text{eff}(\vb*{r}) = - \frac{eQ}{r} \ee^{- \kappa_\text{TF} r},
    \label{eq:thomas-fermi-potential}
\end{equation}
其中
\begin{equation}
    \kappa_\text{TF}^2 = 4 \pi e^2 N(0).
\end{equation}
这个结果就是\concept{托马斯-费米屏蔽}。

\subsubsection{RPA近似下的动态屏蔽响应}

托马斯-费米屏蔽是静态的。现在考虑一般的近独立电子气的线性响应。
外加电势导致如下的相互作用哈密顿量:
\[
    {H}_\text{ext} = \int \dd[3]{\vb*{r}} {n}(\vb*{r}, t) V_\text{ext} (\vb*{r}, t),
\]
则它对电子数密度的影响可以使用(各向同性不含时体系的)推迟格林函数
\[
    G^\text{ret}_{nn}(\vb*{r}-\vb*{r}', t-t') = - \ii \theta(t-t') \expval*{\comm{\var{{n}}(\vb*{r}, t)}{\var{{n}}(\vb*{r}', t')}}
\]
描述,而响应就是%
\footnote{我们没有将$V_\text{ext}$写进${H}$中而是把它看成了外界扰动,这样电子气本身的哈密顿量仍然是空间平移不变的。}%
\begin{equation}
    \expval*{\var{{n}}(\vb*{r}, t)} = \int \dd{t'} G^\text{ret}_{nn}(\vb*{r}-\vb*{r}', t-t') V_\text{ext}(\vb*{r}', t'),
\end{equation}
在频域中就是
\begin{equation}
    \expval*{\var{{n}}(\vb*{k}, \omega)} = G^\text{ret}_{nn}(\vb*{k}, \omega) V_\text{ext}(\vb*{k}, \omega).
    \label{eq:green-function-electro-shielding}
\end{equation}
当然,在系统的构成已知的情况下,可以直接用费曼图计算出格林函数$G^\text{ret}_{nn}$,但是本节不这么处理问题。
但如果真的这么做很快就会遇到数学上的困难:费曼图求和需要经过一些特殊的重求和步骤才能够给出有意义的结果。
因此本节只是做一个近似,将最为奇异的那部分费曼图做一个求和,然后发现它确实能够给出有意义的结果。

为简化问题我们做\concept{RPA(Random Phase Approximation)近似}%
\footnote{这个名称最早来自核物理,在那里它是通过对某个随机相位的求和得到的。}%
,这个近似要求以下两点:
\begin{itemize}
    \item $\var{{n}}$和$V_\text{eff}$之间的响应函数近似是自由电子的$\var{{n}}$-$\var{{n}}$格林函数,即我们认为$V_\text{eff}$中来自电子屏蔽的部分在其它电子上的作用和一个外加势场完全没有区别;
    \item 只考虑经典静电能,即将电子数密度期望当成经典电子数密度,忽略所有交换能或关联能。
\end{itemize}
RPA近似看起来似乎没有什么明确的意义,但它实际上把最为奇异剧烈效应考虑在内,并且给出了定性上正确的屏蔽库伦势(从一个长程的相互作用变成了一个短程的相互作用)。如果RPA近似不准,只需要多算几个微扰就可以。
这样,自洽方程可以通过下面的方法给出。
\begin{enumerate}
    \item 外加电荷引入外加电势能,
    \item 外加电势能加上屏蔽电荷产生的电势能等于总的电势能变化$V_\text{ext}$,
    \item 总的电势能变化可以反推出屏蔽电荷。
\end{enumerate}

粒子数的平均值近似为一个经典的数密度,从而$\var{V}$可以使用泊松方程写成
\[
    - \laplacian \var{V}(\vb*{r}, t) = - 4\pi e^2 \expval*{\var{{n}}(\vb*{r}, t)},
\]
也即
\begin{equation}
    k^2 \var{V}(\vb*{k}, \omega) = 4\pi e^2 \expval*{\var{{n}}(\vb*{k}, \omega)}.
\end{equation}
于是第二步完成了。

第三步通过RPA近似的基本假设得到。这里我们用$G^\text{0, ret}_{nn} (\vb*{k}, \omega)$作为$G^\text{0, ret}_{\var{n}\var{n}} (\vb*{k}, \omega)$的简写:
\begin{equation}
    \expval*{\var{{n}}(\vb*{k}, \omega)} = G^\text{0, ret}_{nn} (\vb*{k}, \omega) V_\text{eff} (\vb*{k}, \omega).
    \label{eq:rpa-approximation}
\end{equation}
这样就得到自洽方程
\[
    V_\text{ext}(\vb*{k}, \omega) + \frac{4\pi e^2}{k^2} \expval*{\var{{n}}(\vb*{k}, \omega)} = (G^\text{0, ret}_{nn} (\vb*{k}, \omega))^{-1} \expval*{\var{{n}}(\vb*{k}, \omega)},
\]
从而
\begin{equation}
    G^\text{ret}_{nn}(\vb*{k}, \omega) = \frac{G^\text{0, ret}_{nn}(\vb*{k}, \omega)}{1 - \frac{4\pi e^2}{k^2} G^\text{0, ret}_{nn}(\vb*{k}, \omega)} = \frac{G^\text{0, ret}_{nn}(\vb*{k}, \omega)}{\epsilon(\vb*{k}, \omega)},
    \label{eq:ext-electron-ret-nn}
\end{equation}
其中$\epsilon(\vb*{k}, \omega)$是介电常数。
这样,计算出自由电子的密度-密度格林函数之后问题就完全解决了。

实际上从\eqref{eq:ext-electron-ret-nn}就可以看出RPA近似的实质。将\eqref{eq:ext-electron-ret-nn}做级数展开,我们有
\[
    G^\text{ret}_{nn}(\vb*{k}, \omega) = G^\text{0, ret}_{nn}(\vb*{k}, \omega) + G^\text{0, ret}_{nn}(\vb*{k}, \omega) \frac{4\pi e^2}{k^2} G^\text{0, ret}_{nn}(\vb*{k}, \omega) + \cdots,
\]
上式两边加上电子数密度平方,并注意到$G^\text{0, ret}_{nn}(\vb*{k}, \omega)$就是电子-空穴组成的环的贡献——依照定义,自由电子的密度-密度格林函数为
\[
    \begin{aligned}
        G_{nn}^\text{0}(\vb*{r}-\vb*{r}', \tau-\tau') &= - T_\tau \expval*{\var{{n}}(\vb*{r}, \tau) \var{{n}}(\vb*{r}', \tau')} \\
        &= - T_\tau \expval*{{{n}}(\vb*{r}, \tau) {{n}}(\vb*{r}', \tau')} + \expval*{{n}(\vb*{r}, \tau)} \expval*{{n}(\vb*{r}', \tau')} \\
        &= \sum_{\sigma, \sigma'} T_\tau \expval*{{\psi}_{\sigma'}(\vb*{r}', \tau') {\psi}^\dagger_\sigma(\vb*{r}, \tau)} T_\tau \expval*{ {\psi}_\sigma(\vb*{r}, \tau) {\psi}^\dagger_{\sigma'}(\vb*{r}', \tau') } ,
    \end{aligned}
\]
其中最后一个等号用到了Wick定理——我们就发现\eqref{eq:ext-electron-ret-nn}实际上就是环形图近似。
\eqref{eq:ext-electron-ret-nn}实际上给出了$G^\text{ret}_{nn}$中最为奇异的部分,因为环形图会产生次数非常高的$1/k^n$型的项,从而发散得最厉害;但是将所有这些发散的图求和起来却能够得到一个收敛的结果。
环形图近似的成立条件是粒子数密度很高,从而主要的相互作用形式是电子通过库仑相互作用,激发出一些电子-空穴对(即环形图中的环),即主要的相互作用是环形图。
也可以从另一个角度理解这件事:粒子数密度较高意味着屏蔽作用很强,所以$1/k^n$发散得不那么厉害的那些项没有什么贡献。

由于自旋守恒性,只有$\sigma = \sigma'$的项才非零,我们得到
\[
    G_{nn}^\text{0}(\vb*{r}-\vb*{r}', \tau-\tau') = \sum_\sigma G_\sigma(\vb*{r}' - \vb*{r}, \tau' - \tau) G_\sigma(\vb*{r} - \vb*{r}', \tau - \tau').
\]
上式具有卷积的形式,切换到频谱,就是
\[
    G_{nn}^\text{0}(\vb*{q}, \ii \omega_n) = \sum_{\sigma} \sum_{\vb*{k}} \frac{1}{\beta} \sum_{\nu_n} G_\sigma(\vb*{q}, \ii \nu_n) G_\sigma(\vb*{k} + \vb*{q}, \ii (\omega_n + \nu_n)),
\]
其中
\[
    \nu_n = \frac{\pi(2n+1)}{\beta}, \quad \omega_n = \frac{2\pi n}{\beta},
\]
前者是费米子的松原频率,后者则是玻色子的,因为电子配对形成玻色子。自由费米子的谱函数为
\[
    A(\vb*{k}, \omega) = \delta(\omega - \xi_{\vb*{k}}), \quad \xi_{\vb*{k}} = \frac{\vb*{k}^2}{2m} - \mu,
\]
格林函数为
\[
    G_\sigma(\vb*{r}, \ii \omega_n) = \frac{1}{\ii \omega_n - \xi_{\vb*{k}}}.
\]
于是,有
\[
    G_{nn}^\text{0}(\vb*{q}, \ii \omega_n) = 2 \sum_{\vb*{k}} \frac{1}{\beta} \sum_{\nu_n} \frac{1}{\ii \nu_n - \xi_{\vb*{k}}} \frac{1}{\ii \omega_n + \ii \nu_n - \xi_{\vb*{k}+\vb*{q}}}.
\]
要计算上式,常用的一个技巧是将求和转化为围道积分。注意到$\{\ii \nu_n\}$正是$f(z)$的全部极点,而且这些极点都是一阶极点。
容易看出这些极点的留数都是$-1/\beta$,于是对一个在虚轴上解析的函数$F(z)$,有
\[
    \frac{1}{\beta} \sum_{\nu_n} F(\ii \nu_n) = - \oint \frac{\dd{z}}{2\pi \ii} f(z) F(z),
\]
于是设围绕虚轴的环路为$C$,它实际上就是从$-\ii \infty + 0^+$到$\ii \infty + 0^+$,再从$\ii \infty - 0^+$到$-\ii \infty - 0^+$,
有
\[
    \begin{aligned}
        G_{nn}^\text{0}(\vb*{q}, \ii \omega_n) &= - 2 \sum_{\vb*{k}} \oint_C \frac{\dd{z}}{2\pi \ii} f(z) \frac{1}{z - \xi_{\vb*{k}}} \frac{1}{z +  \ii \omega_n - \xi_{\vb*{k}+\vb*{q}}} \\
        &= 2 \sum_{\vb*{k}} \left( \frac{f(\xi_{\vb*{k}})}{\xi_{\vb*{k}} + \ii \omega_n - \xi_{\vb*{k} + \vb*{q}}} + \frac{f(\xi_{\vb*{k} + \vb*{q}} - \ii \omega_n)}{\xi_{\vb*{k} + \vb*{q}} - \ii \omega_n - \xi_{\vb*{k}}} \right).
    \end{aligned}
\]
负号消失了是因为绕极点$z=\xi_{\vb*{k}}$和$z = \xi_{\vb*{k} + \vb*{q}} - \ii \omega_n$是顺时针而不是留数定理通常是的逆时针。
由$\{\omega_n\}$的定义,$f(\xi_{\vb*{k} + \vb*{q}} - \ii \omega_n)$就是$f(\xi_{\vb*{k} + \vb*{q}})$,于是得到松原格林函数
\begin{equation}
    G_{nn}^\text{0}(\vb*{q}, \ii \omega_n) = 2 \sum_{\vb*{k}} \frac{f(\xi_{\vb*{k}}) - f(\xi_{\vb*{k} + \vb*{q}})}{\ii \omega_n + \xi_{\vb*{k}} - \xi_{\vb*{k} + \vb*{q}}}.
\end{equation}
做解析延拓就得到推迟格林函数
\begin{equation}
    G_{nn}^\text{ret, 0}(\vb*{q}, \ii \omega_n) = 2 \sum_{\vb*{k}} \frac{f(\xi_{\vb*{k}}) - f(\xi_{\vb*{k} + \vb*{q}})}{\omega + \xi_{\vb*{k}} - \xi_{\vb*{k} + \vb*{q}} + \ii 0^+}.
    \label{eq:ext-electron-retarded-green-function}
\end{equation}
介电常数为
\begin{equation}
    \epsilon(\vb*{k}, \omega) = \frac{V_\text{ext}}{V_\text{eff}} = 1 - \frac{4 \pi e^2}{k^2} G^\text{ret, 0}_{nn}(\vb*{k}, \omega)
\end{equation}

\eqref{eq:ext-electron-retarded-green-function}可以推导出托马斯-费米屏蔽。
在$V_\text{eff}$在时间和空间上都变化非常缓慢时,只需要考虑$\vb*{q}$接近零的那部分傅里叶分量。
让$\omega$和$\vb*{q}$趋于零,我们有
\[
    G_{nn}^\text{ret, 0}(\vb*{q}, \ii \omega_n) = 2 \sum_{\vb*{k}} \dv{f(\xi_{\vb*{k}})}{\xi_{\vb*{k}}} = - 2 \sum_{\vb*{k}} \dv{f(\xi_{\vb*{k}})}{\mu} = - \pdv{n}{\mu} = - N(0),
\]
这正是我们在托马斯-费米屏蔽中做的近似。这也表明RPA近似至少在低频下是合理的。

\subsubsection{等离子体和等离激元}

考虑$\omega$非常高的极限,此时做展开
\[
    \frac{1}{\omega + \xi_{\vb*{k}} - \xi_{\vb*{k} + \vb*{q}} + \ii 0^+} = \frac{1}{\omega} - \frac{\xi_{\vb*{k}} - \xi_{\vb*{k} + \vb*{q}}}{\omega^2} + \cdots,
\]
展开到前两阶,代入\eqref{eq:ext-electron-retarded-green-function},得到
\[
    \begin{aligned}
        G_{nn}^\text{ret, 0}(\vb*{q}, \ii \omega_n) &= - \frac{2}{\omega^2} \sum_{\vb*{k}} (f(\xi_{\vb*{k}}) - f(\xi_{\vb*{k} + \vb*{q}})) (\xi_{\vb*{k}} - \xi_{\vb*{k} + \vb*{q}}) \\
        &= \frac{2 q^2}{m \omega^2} \sum_{\vb*{k}} f(\xi_{\vb*{k}}), 
    \end{aligned}
\]
于是就得到
\begin{equation}
    G_{nn}^\text{ret, 0}(\vb*{q}, \ii \omega_n) = \frac{q^2 n_\text{e}}{m \omega^2},
\end{equation}
相应的,
\[
    \epsilon(\vb*{k}, \omega \to 0) = 1 - \frac{4 \pi e^2 n_\text{e}}{m \omega^2},
\]
定义
\begin{equation}
    \omega_\text{p}^2 = \frac{4 \pi e^2 n_\text{e}}{m},
\end{equation}
就得到
\begin{equation}
    \epsilon(\vb*{k}, \omega \to 0) = 1 - \frac{\omega_\text{p}^2}{\omega^2}.
\end{equation}
考虑到\eqref{eq:ext-electron-ret-nn},当$\omega = \omega_\text{p}$时格林函数出现奇点,即此处有集体激发。实际上,这正是\concept{等离激元}。
$G^\text{ret}_{nn}(\vb*{k}, \omega)$是一个密度-密度格林函数,对任意的动量,在$\omega = \omega_\text{p}$处都有一个密度波的集体激发模式。
这个集体激发的频谱是平的,这当然是因为我们假定光速无穷大,完全使用库伦相互作用。如果考虑到光速有限,使用麦克斯韦方程可以推出
\begin{equation}
    \omega_{\vb*{k}}^2 = \omega_\text{p}^2 + c^2 k^2,
\end{equation}
即光子的一个模式和固体发生耦合,打开了一个能隙。在假定光速无限时,丢弃不可见、恒定的无穷大项(相当于将元激发的产生湮灭算符的时间演化加上一个因子$\ee^{\ii c k t}$)就得到
\[
    \omega_{\vb*{k}} = \omega_\text{p},
\]
即我们算出的结果。

TODO:自能修正会影响吗?注意$-\Sigma$才是全体自能正规图的和。

\input{impurity.tex}

\chapter{铁磁性和反铁磁性的自旋模型}\label{chap:magnetic}

自旋是电子的内禀性质。然而,在一些情况下,电子几乎总是定域在某些格点附近而不发生移动,此时系统中不存在电子位置的变化,主要的自由度是各个格点上的自旋。
这样的模型即为所谓\concept{自旋模型}。自旋模型是非常常见的,如Hubbard模型在$U$非常大时,以动能项为微扰就能够得到一个海森堡模型。
对称性说明自旋-自旋相互作用通常可以取$\vb*{S}_{\vb*{i}} \cdot \vb*{S}_{\vb*{j}}$的形式,或者也许是各向异性的$\vb*{S}_{\vb*{i}} \cdot \vb*{T} \cdot \vb*{S}_{\vb*{j}}$。

自旋模型是解释铁磁性和反铁磁性的重要模型。

\section{海森堡模型}

\subsection{海森堡模型的哈密顿量}

\subsubsection{局域电子相互作用给出的海森堡模型}

考虑一种晶体中的局域化电子,在一个晶胞上只有一个,一方面未配对,一方面不能远距离移动。
在没有外加电磁场激励时它稳定地呆在自己的轨道上,不发生跃迁,因此作为一个低能有效理论,我们暂时只需要考虑它的位置。
切换到Wannier表象下,由于这种电子是高度局域的,它自己的轨道波函数几乎就是Wannier波函数,跃迁能力很差,能带很窄。
极端情况下动能项可以忽略,只需要考虑相互作用项。
我们正在讨论一个单带模型,且电子跃迁能力差,则哈密顿量形式为\eqref{eq:tight-binding-single-band-interaction},且无序项不存在。
由于假定每个晶胞上只有一个电子,化学势项、on-site repulsion项和密度-密度相互作用均可以直接略去,因为他们基本上是常数。
因此唯一剩下的就是自旋-自旋相互作用,因此哈密顿量为
\begin{equation}
    H = - J \sum_{\pair{\vb*{i}, \vb*{j}}} \vb*{S}_{\vb*{i}} \cdot \vb*{S}_{\vb*{j}},
    \label{eq:heisenberg-nearest}
\end{equation}
这里的$\vb*{S}_i$和$\vb*{S}_j$仍然是电子产生湮灭算符相乘得到的;然而,由于电子基本上没有跃迁,我们可以认为每个电子的$\vb*{i}$其实也是不怎么会变化的,因此我们可以\emph{彻底}忽略电子的轨道自由度,只保留自旋自由度。
这样就得到了自旋$1/2$的\concept{海森堡模型},如前所述,它是描述晶体中单占据、高度局域、跃迁能力差、绝缘的电子的模型。

回顾\eqref{eq:tight-binding-single-band-interaction}的导出,我们发现导致海森堡模型中的自旋-自旋耦合的主要是电子之间的交换相互作用。
我们后面会看到铁磁序和反铁磁序在海森堡模型中能够观察到,因此,这些磁性序的相当一部分是纯粹的量子效应的产物。

\subsubsection{更加一般的海森堡模型}

\eqref{eq:heisenberg-nearest}在两个方面可以推广:首先,可以有非最近邻的自旋-自旋相互作用;其次,实际上有相当一类体系不是自旋$1/2$的系统,但是仍然能够使用\eqref{eq:heisenberg-nearest}形式的哈密顿量描述。
\begin{equation}
    H = 
\end{equation}

\subsubsection{一般自旋的海森堡模型}



\subsection{磁性离子的铁磁序和自旋波}

\begin{back}{自旋自由度的一些性质}{spin-degree-of-freedom}
    用$S^j_{\vb*{i}}$表示格点$\vb*{i}$上的自旋,$j=1, 2, 3$或是$x, y, z$。通常习惯用$S^z$标记一个状态。
    我们有
    \begin{equation}
        \comm*{S^{j}_{\vb*{i}}}{{S}^{j'}_{\vb*{i}'}} = \ii \epsilon_{j j' j''} S_{\vb{i}}^{j''} \delta_{\vb*{i} \vb*{i}'},
    \end{equation}
    根据此关系可以对易所谓自旋升降算符,定义
    \begin{equation}
        {S}_{\vb*{i}}^+ = \frac{1}{\sqrt{2}} ({S}^x_{\vb*{i}} + \ii {S}^y_{\vb*{i}}) , \quad {S}_{\vb*{i}}^- = \frac{1}{\sqrt{2}} ({S}^x_{\vb*{i}} - \ii {S}^y_{\vb*{i}}) ,
    \end{equation}
    则有
    \begin{equation}
        \comm*{S^z_{\vb*{i}}}{S_{\vb*{i}}^+} = S^+_{\vb*{i}}, \quad \comm*{S^z_{\vb*{i}}}{S_{\vb*{i}}^-} = - S^-_{\vb*{i}},
    \end{equation}
    从而
    \begin{equation}
        \begin{aligned}
            S_{\vb*{i}}^+ \ket{\cdots, m, \cdots} &= \sqrt{\frac{(s + m + 1) (s - m)}{2}} \ket{m + 1}, \\
            S_{\vb*{i}}^- \ket{\cdots, m, \cdots} &= \sqrt{\frac{(s + m) (s - m + 1)}{2}} \ket{m - 1}.
        \end{aligned}
    \end{equation}
    特别的,对自旋$1/2$的情况,以$\ket{\uparrow}$和$\ket{\downarrow}$为基底,有
    \begin{equation}
        \vb*{S}_{\vb*{i}} = \frac{1}{2} \vb*{\sigma}_{\vb*{i}}, \quad {S}_{\vb*{i}}^+ = \frac{1}{\sqrt{2}} \pmqty{0 & 1 \\ 0 & 0}, \quad S_{\vb*{i}}^- = \frac{1}{\sqrt{2}} \pmqty{0 & 0 \\ 1 & 0}.
    \end{equation}
\end{back}

在$J > 0$时相邻自旋的相互作用会让自旋倾向于形成铁磁态。我们将海森堡哈密顿量用自旋升降算符写出,为
\begin{equation}
    H = 
\end{equation}

我们在这里只讨论铁磁序确确实实已经形成了的情况,不失一般性设其方向为$z$方向。

\subsection{反铁磁序}

海森堡模型中同时有$S^z$和$S^x$,$S^y$算符,因此存在很明显的量子涨落。
一些人猜测,这种量子涨落甚至可能在特定的晶格上完全破坏铁磁序!有关的情况见\autoref{chap:spin-liquid}。

\subsubsection{Hubbard模型和海森堡模型}

Goodenough规则:如果电子跃迁是在两个半满轨道之间,那么就是反铁磁序,如果电子跃迁是从一个半满轨道到一个空轨道,或是从一个全满轨道到一个半满轨道,那么就是铁磁序。

\section{各向异性海森堡模型}

% TODO:XXZ模型等

\section{伊辛模型}

\begin{equation}
    H = - J \sum_{\pair{\vb*{i}, \vb*{j}}} S_{\vb*{i}}^z S_{\vb*{j}}^z ,
\end{equation}
在自旋$1/2$的情况下,很多时候我们会重新定义$J$而使用泡利矩阵给出哈密顿量:
\begin{equation}
    H = - J \sum_{\pair{\vb*{i}, \vb*{j}}} \sigma^z_{\vb*{i}} \sigma^z_{\vb*{j}}.
\end{equation}

横场

\section{t-J模型}

\begin{equation}
    H = - t \sum_{\vb*{i}, \sigma} (c_{\vb*{i}})
\end{equation}

\section{XXZ模型}

\chapter{自旋玻璃}


\section{低温下的电子配对}\label{sec:low-and-super}

很多集体行为和相变均可以通过经典的金斯堡-朗道理论加以描述:相变的出现是因为某个对称性(由于一些特殊的相互作用)被破缺了,然后我们可以使用某个序参量来描述相变,而序参量的涨落就给出了相变之后产生的元激发。
我们可以使用Hubbard-Stratonovich变换将二体相互作用解耦,适当选取Hubbard-Stratonovich参量使之和序参量对应,然后积掉电子自由度和声子自由度,这样就得到了元激发的有效理论。
序参量应该是通常是一些电子场算符的乘积或其线性组合,元激发是序参量的涨落,即元激发实际上是多个电子(或者也许还有声子或其他粒子)的集体行为,或者说\concept{粒子凝聚}。
此时的系统不再是一个普通的费米液体了,我们也可以说,\emph{特殊的相互作用通道让费米液体不稳定}。

费米子体系中可以有各种各样的这种电子凝聚:自旋密度波%
\footnote{
    为了避免引起混淆我们要区分自旋密度波和自旋波。后者是一个自旋系统中的现象,前者是一个费米子系统中的现象。
    或者,更加形象地说,后者是固定在格点上的自旋的空间涨落,前者是可以运动的、携带自旋的费米子密度的空间涨落。
}%
、电荷密度波、超导配对等。电子配对顾名思义,可能导致可以不受阻碍地随意移动的“超导电子”出现;所谓密度波实际上就是一种长程序,它在空间中有一种周期性振荡,可以是相位的振荡也可以是大小的振荡,
也即,这样的长程序对应的序参量形如%
\footnote{请注意这个序参量实际上是一种二电子集体运动模式,即相邻的两个电子的自旋总是保持一上一下的这种运动模式。
这暗示我们,如果需要超越平均场近似的理论,只需要把这种模式定义成一个新的(玻色)场就可以。
这个技巧和一维电子系统的玻色化很相似。}%
\begin{equation}
    \Delta(\vb*{r}) \sim \Delta_0 \ee^{\ii \vb*{Q} \cdot \vb*{r}}.
\end{equation}
这些现象对应的序参量对应于Hubbard-Stratonovich参量的不同选择;换句话说,不同的相互作用通道相互竞争,最终哪一种相互作用通道占据优势——从而,单单考虑它得到的相变出现——取决于很多条件。

应当指出的是这个图像并不能覆盖所有相变。例如在

\subsection{BCS超导}\label{sec:bcs-theory}

本节介绍一种常见的超导机制,即交换声子导致电子出现有效吸引相互作用而产生的BCS超导。

\subsubsection{交换声子导致的有效电子吸引相互作用}\label{sec:phonon-caused-interaction}

电子-声子相互作用的顶角为一个电子入射,一个电子出射,产生/消灭一个声子。
现在尝试积掉声子。我们先将电子场当成给定的,则可以从\eqref{eq:simple-phonon-electron-int}写出声子场加上电子-声子相互作用的虚时间作用量。本节仅考虑晶格是简单正方晶格的情况,于是没有各向异性,我们有
\[
    S_\text{ph} + S_\text{int} = \sum_{\omega_n, \vb*{q}, \lambda} \Big(
        \bar{\phi}_{\vb*{q} \lambda} (- \ii \omega_n + \omega_q) \phi_{\vb*{q} \lambda}
        + \gamma \frac{\ii q_\lambda}{\sqrt{2 M \omega_q}} (\phi_{\vb*{q} \lambda} + \bar{\phi}_{-\vb*{q} \lambda}) \underbrace{\sum_{\vb*{k}, \sigma} \bar{\psi}_{(\vb*{k} + \vb*{q}) \sigma} \psi_{\vb*{k} \sigma}}_{\rho_{\vb*{q}}}
    \Big),
\]
其中$\phi$表示声子,$\psi$表示电子。使用处理自由场受到线性激励的方法,配平方并积掉二次方项,忽略产生的因子,得到
\begin{equation}
    \begin{aligned}
        S_\text{eff} &= - \sum_{\omega_n, \vb*{q}} \frac{\gamma^2 q^2 \rho_{\vb*{q}} \rho_{-\vb*{q}}}{2 M \omega_q (\omega_q - \ii \omega_n)} 
        = - \sum_{\omega_n, \vb*{q}} \frac{\gamma^2 q^2 \rho_{\vb*{q}} \rho_{-\vb*{q}}}{2 M (\omega_q^2 + \omega_n^2)} \\ 
        &= - \frac{\gamma^2}{2 M} \sum_{\omega_n} \sum_{\vb*{q}, \vb*{k}, \vb*{k}', \alpha, \beta} \frac{q^2}{\omega_n^2 + \omega_q^2} \bar{\psi}_{(\vb*{k}+\vb*{q}) \alpha} \bar{\psi}_{(\vb*{k}' - \vb*{q}) \beta} \psi_{\vb*{k}' \beta} \psi_{\vb*{k} \alpha} .
    \end{aligned}
    \label{eq:retarded-two-electron}
\end{equation}
第二个等号是考虑到每个$\omega_n$有一个对应的$-\omega_n$而得到的。总之,电子之间可以通过交换声子来产生一个四电子相互作用。
相互作用\eqref{eq:retarded-two-electron}是推迟相互作用,因此原则上不能够仅仅使用一个哈密顿量描述。
不过,在推迟不明显时我们还是可以近似写出一个哈密顿量。这里有一个微妙的地方:\eqref{eq:retarded-two-electron}是定义在虚时间下的,即算符的时间演化为$\ee^{\omega t}$,而我们需要一个实时间下的哈密顿量,所以需要首先做频率上的Wick转动$\omega = \ii \omega_n$,得到
\[
    S_\text{eff}^\text{real} = - \frac{\gamma^2}{2 M} \int \dd{\omega} \sum_{\vb*{q}, \vb*{k}, \vb*{k}', \alpha, \beta} \frac{q^2}{- \omega^2 + \omega_q^2} \bar{\psi}_{(\vb*{k}+\vb*{q}) \alpha} \bar{\psi}_{(\vb*{k}' - \vb*{q}) \beta} \psi_{\vb*{k}' \beta} \psi_{\vb*{k} \alpha}.
\]
上式中的所有$\psi$都是$\omega$的函数,可以将它们转换到时域;我们也知道,上式的无推迟近似是它的无时间变化的傅里叶分量。
因此实际上我们只需要简单地取
\begin{equation}
    \omega = \epsilon_{\vb*{k}} - \epsilon_{\vb*{k} + \vb*{q}},
    \label{eq:phonon-introduced-omega}
\end{equation}
就得到了无推迟的近似
\begin{equation}
    \begin{aligned}
        {H} &= - \frac{\gamma^2}{2 M} \sum_{\vb*{q}, \vb*{k}, \vb*{k}', \alpha, \beta} \frac{q^2}{- \omega^2 + \omega_q^2} {c}^\dagger_{(\vb*{k}+\vb*{q}) \alpha} {c}^\dagger_{(\vb*{k}' - \vb*{q}) \beta} {c}_{\vb*{k}' \beta} {c}_{\vb*{k} \alpha} \\
        &= \frac{\gamma^2}{2 M} \sum_{\vb*{q}, \vb*{k}, \vb*{k}', \alpha, \beta} \frac{q^2}{(\epsilon_{\vb*{k}} - \epsilon_{\vb*{k}+\vb*{q}})^2 - \omega_q^2} {c}^\dagger_{(\vb*{k}+\vb*{q}) \alpha} {c}^\dagger_{(\vb*{k}' - \vb*{q}) \beta} {c}_{\vb*{k}' \beta} {c}_{\vb*{k} \alpha}.
    \end{aligned}
\end{equation}

更加一般地,设声子频率为$\omega_{\vb*{q}}$,电子-声子相互作用为
\begin{equation}
    H = \frac{1}{\sqrt{N}} \sum_{\vb*{k}, \vb*{q}} M_{\vb*{q}} c^\dagger_{\vb*{k} + \vb*{q}} c_{\vb*{k}} (b_{\vb*{q}} + b_{-\vb*{q}}^\dagger),
\end{equation}
重复以上步骤将得到% TODO
\begin{equation}
    {H}_{4\text{e}} = \frac{1}{2N} \sum_{\vb*{k}, \vb*{k}', \vb*{q}'} \sum_{\alpha, \beta} \abs{M_{\vb*{q}}}^2 \frac{\omega_{\vb*{q}}}{(\epsilon_{\vb*{k}} - \epsilon_{\vb*{k}+\vb*{q}})^2 - \omega_{\vb*{q}}^2} {c}^\dagger_{(\vb*{k}+\vb*{q}) \alpha} {c}^\dagger_{(\vb*{k}'-\vb*{q}) \beta} {c}_{\vb*{k}' \beta} {c}_{\vb*{k} \alpha},
    \label{eq:4-electron-interaction-by-phonon}
\end{equation}
其中$\omega_{\vb*{q}}$表示声子能量。任何更复杂的含有声子的费曼图都可以化归为二电子和声子的相互作用和一些电子相互作用的组合,因此\eqref{eq:4-electron-interaction-by-phonon}就给出了完整的没有声子的有效理论。
\eqref{eq:4-electron-interaction-by-phonon}本身是瞬时的,但即使我们不知道它来自一个推迟相互作用,\eqref{eq:4-electron-interaction-by-phonon}含有电子能量差而\eqref{eq:phonon-introduced-omega}也有这一事实也暗示它有可能实际上代表一个推迟相互作用。
另一方面,库伦势则没有推迟(当然这实际上做了近似,我们只不过不考虑相对论效应而已,但是电磁相互作用的推迟远小于声子传递导致的推迟)。

现在考虑一个低能有效理论%
\footnote{这就是超导往往发生在低温下的原因:否则没法形成电子的有效吸引,四散排斥的电子会造成很大耗散。}%
,其中我们需要考虑的过程全部发生在费米面附近,入射电子和出射电子能量都接近于$\epsilon_\text{F}$。
费米面是球形的,通过几何关系可以看出$\vb*{k} + \vb*{k}' = 0$的过程是最重要的,此时发生相互作用前两个电子始终在费米面的两端,相互作用之后两个电子还是位于费米面的两端。
这样\eqref{eq:4-electron-interaction-by-phonon}就简化为
\begin{equation}
    {H}_{4\text{e}} = \frac{1}{2N} \sum_{\vb*{k}, \vb*{q}'} \underbrace{\sum_{\alpha, \beta} \abs{M_{\vb*{q}}}^2 \frac{\omega_{\vb*{q}}}{(\epsilon_{\vb*{k}} - \epsilon_{\vb*{k}+\vb*{q}})^2 - \omega_{\vb*{q}}^2}}_{V(\vb*{q}, \epsilon_{\vb*{k}} - \epsilon_{\vb*{k}+\vb*{q}})} {c}^\dagger_{(\vb*{k}+\vb*{q}) \alpha} {c}^\dagger_{(-\vb*{k}-\vb*{q}) \beta} {c}_{-\vb*{k} \beta} {c}_{\vb*{k} \alpha}.
    \label{eq:low-energy-4-electron}
\end{equation}
由于电子能量差非常小,显然
\[
    V(\vb*{q}, \epsilon_{\vb*{k}} - \epsilon_{\vb*{k}+\vb*{q}}) = \abs{M_{\vb*{q}}}^2 \frac{\omega_{\vb*{q}}}{(\epsilon_{\vb*{k}} - \epsilon_{\vb*{k}+\vb*{q}})^2 - \omega_{\vb*{q}}^2} < 0,
\]
因此这个声子中介的相互作用是吸引相互作用。

库伦相互作用是排斥的,电-声子相互作用是吸引的;绝对强度显然是前者强。
然而,实际上库伦相互作用很容易被屏蔽,因为把低能电子看成外加电荷,那么高能电子就会来屏蔽它(见\autoref{sec:ext-e}),因此把高能电子积掉之后得到的屏蔽库仑相互作用并不强,因此在重整化下,库仑相互作用实际上只是对电子能带的修正(这和费米液体的想法很相似:渐染地加入一个不大的相互作用仅仅会导致电子自能修正而已)。
另一方面,电-声子相互作用是不容易屏蔽的,实际上推迟相互作用就是不容易屏蔽的。
因此,\eqref{eq:low-energy-4-electron}在重整化之后是主要的电子间相互作用。
还有一种看待这个问题的方式是,电-声子相互作用会导致有效的吸引,在重整化下一个吸引相互作用通常会导致粒子配对取代单粒子自由度成为主要的自由度,因此吸引相互作用不可能只是对能带的修正;库仑相互作用在这里是排斥的,不会造成粒子配对,因此正如Hartree-Fock近似中的那样,仅仅对电子能级产生了一个修正。

声子的频率只出现在0和德拜频率$\omega_\text{D}$之间,换而言之吸引相互作用只出现在
\[
    \omega = \abs{\epsilon_{\vb*{k}} - \epsilon_{\vb*{k}+\vb*{q}}} < \omega_\text{D}
\]
时。数值计算可以表明对超导现象而言最重要的是建立起电子之间的吸引相互作用,其具体形式并不重要,这是因为会参与超导的实际上只有费米面附近的非常小的一个能量范围内的电子,因此$V$的具体形式根本不重要,实际发挥作用的只有费米面上的$V$值。
这样我们设
\begin{equation}
    V(\vb*{q}, \omega) = \begin{cases}
        - V_0, \quad \omega < \omega_\text{D}, \\
        0, \quad \omega > \omega_\text{D}
    \end{cases}.
    \label{eq:superconductive-interaction-simplified}
\end{equation}
其中$\omega_\text{D}$是一个硬截断。这是一个非常粗糙的截断,但后面会发现它是合理的。
这样整个系统的哈密顿量就是
\begin{equation}
    {H} = \sum_{\vb*{k}, \alpha} (\epsilon_{\vb*{k}} - \mu) {c}_{\vb*{k} \alpha}^\dagger {c}_{\vb*{k} \alpha} - \frac{V_0}{2} \sum_{\vb*{k}, \vb*{q}} \sum_{\alpha, \beta} {c}^\dagger_{(\vb*{k} + \vb*{q}) \alpha} {c}^\dagger_{( - \vb*{k} - \vb*{q}) \beta} {c}_{-\vb*{k} \beta} {c}_{\vb*{k} \alpha}.
    \label{eq:simple-super-conductive-hamiltonian}
\end{equation}
$\vb*{k}$和$\vb*{k}+\vb*{q}$都在费米面附近。

\subsubsection{库伯对}

数值计算或者手算二体问题会发现可能出现\concept{库伯对},即一对位于费米面上的电子配对。
这种配对写成算符形式就是${c}^\dagger {c}^\dagger$的形式,或者等价的${c} {c}$形式。
如果其期望值不为零,那么$U(1)$对称性就破缺了,即电荷守恒对称性被破缺了(物理图像是,一部分电荷被封存到了库伯对中,不再以独立电子的形式流动)。我们会看到电荷守恒的破缺实际上是低温超导中最重要的物理。
于是,库伯对序参量的一般形式显然是$\expval*{{c}_{\vb*{k}\alpha} {c}_{\vb*{k}' \beta}}$。

现在我们做对称性分析。注意到时间反演对称性要求
\[
    \epsilon_{\vb*{k} \uparrow} = \epsilon_{-\vb*{k} \downarrow},
\]
而\eqref{eq:simple-super-conductive-hamiltonian}中单电子能量和自旋无关,于是
\[
    \epsilon_{\vb*{k}} = \epsilon_{-\vb*{k}}.
\]
这样就容易验证\eqref{eq:simple-super-conductive-hamiltonian}在变换$\vb*{k} \longrightarrow -\vb*{k}$下不变,这个对称性没有被破缺掉,于是
\[
    \expval*{{c}_{\vb*{k}\alpha} {c}_{\vb*{k}' \beta}} \propto \delta(\vb*{k} + \vb*{k}').
\]
换而言之,我们只考虑总动量为零的配对$\expval*{{c}_{\vb*{k} \alpha} {c}_{- \vb*{k} \beta}}$。
此外还可以发现\eqref{eq:simple-super-conductive-hamiltonian}具有自旋旋转不变性,因此相应的序参量$\expval*{{c}_{\vb*{k} \alpha} {c}_{- \vb*{k} \beta}}$应该是一个二分量的自旋协变的对象。二粒子配对对应的$SU(2)$的表示为
\[
    \frac{1}{2} \otimes \frac{1}{2} = 0 \oplus 1,
\]
即可能有自旋单态也可能有自旋三重态。如果该库伯对是自旋单态,那么应该有
\[
    \expval*{{c}_{\vb*{k} \alpha} {c}_{- \vb*{k} \beta}} \propto \epsilon_{\alpha \beta} \propto \delta_{\alpha+\beta,0},
\]
其中$\epsilon$为所谓的旋量度规;而如果该库珀对为自旋三重态,那么就应该有
\[
    \expval*{{c}_{\vb*{k} \alpha} {c}_{- \vb*{k} \beta}} \propto \vb*{d} \cdot \vb*{\sigma}_{\alpha \beta}.
\]
% TODO:为什么??
最后,由于没有自旋-轨道耦合,$\expval*{{c}_{\vb*{k} \alpha} {c}_{- \vb*{k} \beta}}$可以写成“动量部分乘以自旋部分”的形式。
这样我们设
\[
    \expval*{{c}_{\vb*{k} \alpha} {c}_{- \vb*{k} \beta}} \propto \Delta(\vb*{k}).
\]
如果该库伯对是自旋单态的,那么其自旋部分是反对称的,则其轨道部分就是对称的,也即
\[
    \Delta(\vb*{k}) = \Delta(-\vb*{k}),
\]
从而可以将$\Delta(\vb*{k})$展开成一系列s波、d波等对称球谐函数的线性组合;
而如果该库伯对是自旋三重态的,那么其自旋部分就是对称的,于是轨道部分满足
\[
    \Delta(\vb*{k}) = -\Delta(-\vb*{k}),
\]
此时$\Delta(\vb*{k})$是一系列反对称球谐函数的线性组合。
容易看出单态或者三重态的出现和费米面对称性的关系,比如如果费米面不对称,那么s波配对就不能产生。

\subsubsection{平均场近似和Bogoliubov变换}

对\eqref{eq:simple-super-conductive-hamiltonian}做平均场近似,有
\[
    \begin{aligned}
        & \quad {c}^\dagger_{(\vb*{k} + \vb*{q}) \alpha} {c}^\dagger_{( - \vb*{k} - \vb*{q}) \beta} {c}_{-\vb*{k} \beta} {c}_{\vb*{k} \alpha} \\
        &\approx \expval*{{c}^\dagger_{(\vb*{k} + \vb*{q}) \alpha} {c}^\dagger_{( - \vb*{k} - \vb*{q}) \beta}} {c}_{-\vb*{k} \beta} {c}_{\vb*{k} \alpha} + {c}^\dagger_{(\vb*{k} + \vb*{q}) \alpha} {c}^\dagger_{( - \vb*{k} - \vb*{q}) \beta} \expval*{{c}_{-\vb*{k} \beta} {c}_{\vb*{k} \alpha}} - \expval*{{c}^\dagger_{(\vb*{k} + \vb*{q}) \alpha} {c}^\dagger_{( - \vb*{k} - \vb*{q}) \beta} {c}_{-\vb*{k} \beta} {c}_{\vb*{k} \alpha}},
    \end{aligned}
\]
重新选定求和哑指标,并略去对体系动力学没有影响的常数项,得到
\begin{equation}
    {H}_\text{MF} = \sum_{\vb*{k}, \alpha} (\epsilon_{\vb*{k}} - \mu) {c}_{\vb*{k} \alpha}^\dagger {c}_{\vb*{k} \alpha} - \frac{V_0}{2} \sum_{\vb*{k}, \vb*{k}', \alpha, \beta} \left(
        \expval*{{c}_{-\vb*{k} \beta} {c}_{\vb*{k} \alpha}} {c}^\dagger_{\vb*{k}' \alpha} {c}^\dagger_{- \vb*{k}' \beta} + \text{h.c.} 
    \right).
\end{equation}
的确,电荷守恒对称性被破缺了,原因是电子形成库伯对之后看起来就像被“冻结”了一样,因此不再被计入${c}^\dagger$自由度中,而是被计入序参量$\expval*{{c}_{-\vb*{k} \beta} {c}_{\vb*{k} \alpha}}$中。

现在我们考虑单态、s波的库伯对,这样发生配对的就是一个$\vb*{k}, \uparrow$态的电子和一个$-\vb*{k}, \downarrow$态的电子,或者做一个自旋旋转。总之,序参量可以选取为
\begin{equation}
    \Delta = - \frac{V_0}{2} \sum_{\vb*{k}} (
        \expval*{{c}_{-\vb*{k}\uparrow} {c}_{\vb*{k} \downarrow}} - \expval*{{c}_{-\vb*{k} \downarrow} {c}_{\vb*{k} \uparrow}}
    ) = -V_0 \sum_{\vb*{k}} \expval*{{c}_{-\vb*{k}\uparrow} {c}_{\vb*{k} \downarrow}} ,
    \label{eq:superconductive-order-parameter}
\end{equation}
则可以证明
\begin{equation}
    {H}_\text{MF} = \sum_{\vb*{k}, \alpha} \xi_{\vb*{k}} {c}_{\vb*{k} \alpha}^\dagger {c}_{\vb*{k} \alpha} 
    + \Delta \sum_{\vb*{k}} {c}_{-\vb*{k} \downarrow}^\dagger {c}^\dagger_{\vb*{k} \uparrow}
    + \Delta^* \sum_{\vb*{k}} {c}_{\vb*{k} \uparrow} {c}_{-\vb*{k} \downarrow}.
    \label{eq:s-wave-superconductive-hamiltonian}
\end{equation}

在\eqref{eq:s-wave-superconductive-hamiltonian}中总是可以对${c}$和${c}^\dagger$做一个幺正变换,让$\Delta$为实数,因此以下假定$\Delta$为实数。
用一个幺正变换重新定义一组准粒子(这就称为\concept{Bogoliubov变换}),使这组准粒子本身是费米子,并且能够让\eqref{eq:s-wave-superconductive-hamiltonian}对角化(从而这组准粒子的能谱就是\eqref{eq:s-wave-energy-band})。
首先我们有费米子的对易关系
\[
    \acomm*{{\gamma}_{\vb*{k}_1 \alpha}}{{\gamma}^\dagger_{\vb*{k}_2 \beta}} = \delta_{\vb*{k}_1 \vb*{k}_2} \delta_{\alpha \beta}, \quad \acomm*{{\gamma}_{\vb*{k}_1 \alpha}}{{\gamma}_{\vb*{k}_2 \beta}} = 0,
\]
并且可以看到,以下正交变换
\[
    \pmqty{{\gamma}_{\vb*{k} \uparrow} \\ {\gamma}^\dagger_{-\vb*{k} \downarrow}} = \pmqty{u_{\vb*{k}} & -v_{\vb*{k}} \\ v_{\vb*{k}} & u_{\vb*{k}}} \pmqty{{c}_{\vb*{k} \uparrow} \\ {c}^\dagger_{-\vb*{k} \downarrow}},
    \quad u_{\vb*{k}}^2 + v_{\vb*{k}}^2 = 1
\]
能够给出正确的对易关系,则可以解出
\begin{equation}
    u_{\vb*{k}} = \sqrt{\frac{E_{\vb*{k}} + \xi_{\vb*{k}}}{2 E_{\vb*{k}}}}, \quad v_{\vb*{k}} = \sqrt{\frac{E_{\vb*{k}} - \xi_{\vb*{k}}}{2 E_{\vb*{k}}}}.
\end{equation}
这就得到了Bogoliubov变换的显式形式:
\begin{equation}
    \begin{cases}
        {\gamma}_{\vb*{k} \uparrow} = u_{\vb*{k}} {c}_{\vb*{k} \uparrow} - v_{\vb*{k}} {c}_{-\vb*{k} \downarrow}^\dagger, \\
        {\gamma}_{\vb*{k} \downarrow} = u_{\vb*{k}} {c}_{\vb*{k} \downarrow} + v_{\vb*{k}} {c}_{-\vb*{k} \uparrow}^\dagger,
    \end{cases}
    \label{eq:bogoliubov-transform}
\end{equation}
还有它的逆变换
\begin{equation}
    \begin{cases}
        {c}_{\vb*{k} \uparrow} = u_{\vb*{k}} {\gamma}_{\vb*{k} \uparrow} + v_{\vb*{k}} {\gamma}_{-\vb*{k} \downarrow}^\dagger, \\
        {c}_{\vb*{k} \downarrow} = u_{\vb*{k}} {\gamma}_{\vb*{k} \downarrow} - v_{\vb*{k}} {\gamma}_{-\vb*{k} \uparrow}^\dagger.
    \end{cases}
    \label{eq:inverse-bogoliubov-transform}
\end{equation}
由于从${c}$到${\gamma}$的线性变换是一个正交变换%
\footnote{这件事不是一般成立的,有时候真的要用一个非正交变换;关键在于对易关系必须正确,不是所有的线性变换形式都能够给出正确的对易关系。}%
,\eqref{eq:s-wave-superconductive-hamiltonian}的能谱可以通过对角化矩阵的方式求出。
求解之后会发现能带为
\begin{equation}
    E_{\vb*{k}} = \pm \sqrt{ \xi_{\vb*{k}}^2 + \abs{\Delta}^2 }.
    \label{eq:s-wave-energy-band}
\end{equation}
这个结果具有粒子-空穴对称性,但是这并不具有太多物理意义,因为它实际上是对角化时交换了一对产生湮灭算符,从而把一部分粒子自由度写成了空穴而已。
\eqref{eq:s-wave-energy-band}中,序参量$\Delta$把原本发生交叉的两条能带$E=\pm \xi_{\vb*{k}}$的交叉点分开了,即打开了一个能隙。

% TODO:系统基态

\subsubsection{平均场自洽计算}

得到平均场理论后我们来做自洽计算。将\eqref{eq:inverse-bogoliubov-transform}代入\eqref{eq:superconductive-order-parameter},并利用近独立电子气的数目
\[
    \expval*{{\gamma}_{\vb*{k} \alpha}^\dagger {\gamma}_{\vb*{k} \alpha}} = n_\text{F}(E_{\vb*{k}}) =  \frac{1}{\ee^{\beta E_{\vb*{k}}} + 1},
\]
得到
\[
    \Delta = - V_0 \sum_{\vb*{k}} u_{\vb*{k}} v_{\vb*{k}} (2 n_\text{F}(E_{\vb*{k}}) - 1).
\]
可以验证,
\[
    u_{\vb*{k}} v_{\vb*{k}} = \frac{\Delta}{2 E_{\vb*{k}}},
\]
于是最终得到
\begin{equation}
    \Delta V_0 \sum_{\vb*{k}} \frac{1}{2 E_{\vb*{k}}} \frac{\ee^{\beta E_{\vb*{k}}} - 1}{\ee^{\beta E_{\vb*{k}}} + 1} = \Delta.
    \label{eq:superconductive-self-consistency}
\end{equation}
如果$\Delta$为零,即出现超导现象,就可以把$\Delta$消掉。$E_{\vb*{k}}$依赖于$\Delta$,于是给定一个温度就可以把$\Delta$解出。

例如,在$T=0$时,我们有
\[
    1 = V_0 \sum_{\vb*{k}} \frac{1}{2 E_{\vb*{k}}} = V_0 \int \dd{\epsilon} N(\epsilon) \frac{1}{2 \sqrt{\epsilon^2 + \Delta^2}},
\]
这里我们在对单个电子的能量做积分。当然由于电子能量过高时\autoref{sec:phonon-caused-interaction}中的机制不再适用,积分能量肯定有一个截断。
\eqref{eq:superconductive-interaction-simplified}给出了截断$\omega_\text{D}$。我们假定$\omega_\text{D}$相对费米能非常小,也即发生库伯配对的电子只是费米面附近非常小的一个能量范围内的,则近似有
\[
    1 = V_0 N(0) \int_{-\omega_\text{D}}^{\omega_\text{D}} \dd{\epsilon} \frac{1}{\sqrt{\epsilon^2 + \Delta^2}} = N(0) V_0 \sinh^{-1} \left( \frac{\omega_\text{D}}{\Delta} \right) \approx N(0) V_0 \ln \frac{2 \omega_\text{D}}{\Delta},
\]
于是
\begin{equation}
    \Delta = 2 \omega_\text{D} \exp(- \frac{1}{N(0) V_0}).
\end{equation}
显然,$\Delta$对$V_0$的依赖比对$\omega_\text{D}$的依赖要强得多。这是合理的,因为$\omega_\text{D}$实际上是一个非常粗糙的硬截断。
现在我们看到,硬截断\eqref{eq:superconductive-interaction-simplified}是合理的,因为截断只是给$\Delta$提供了一个能量尺度而已,不会影响更为复杂的行为。
我们也可以看出,只要有相互作用,不管多强,都会产生一个非零的$\Delta$,因此只要有相互作用就会出现超导转变,并打开能隙。
这意味着\eqref{eq:4-electron-interaction-by-phonon}具有非微扰行为。

\eqref{eq:superconductive-self-consistency}中总是有一个平庸解$\Delta = 0$,消掉因子$\Delta$之后得到的方程就不总是有解。
消掉因子$\Delta$之后得到的方程有没有解就区分了超导相和非超导相。
现在我们计算临界温度,即$\Delta$从非零的一侧趋于零时的温度:
\[
    \begin{aligned}
        1 &= \lim_{\Delta \to 0} V_0 \sum_{\vb*{k}} \frac{1}{2 E_{\vb*{k}}} \frac{\ee^{\beta E_{\vb*{k}}} - 1}{\ee^{\beta E_{\vb*{k}}} + 1} \\
        &= V_0 \sum_{\vb*{k}} \frac{1}{2 \xi_{\vb*{k}}} \frac{\ee^{\beta \xi_{\vb*{k}}} - 1}{\ee^{\beta \xi_{\vb*{k}}} + 1} \\
        &= V_0 \int_{-\omega_\text{D}}^{\omega_\text{D}} \dd{\epsilon} N(\epsilon) \frac{1}{2 \epsilon} \frac{\ee^{\beta \epsilon} - 1}{\ee^{\beta \epsilon} + 1} \\
        &\approx V_0 N(0) \int_{0}^{\omega_\text{D}} \dd{\epsilon} \frac{1}{\epsilon} \frac{\ee^{\beta \epsilon} - 1}{\ee^{\beta \epsilon} + 1}.
    \end{aligned}
\]
最后一个积分中的因子$(\ee^{\beta \epsilon} - 1) / (\ee^{\beta \epsilon} + 1)$的作用在于在$\epsilon$接近$0$时压低$1/\epsilon$的值从而避免发散,它是一个特征尺度为$\beta$的红外截断,于是
\[
    1 \sim V_0 N(0) \int_{\beta}^{\omega_\text{D}} \dd{\epsilon} \frac{1}{\epsilon},
\]
最后得到
\begin{equation}
    T_\text{c} \sim \frac{\omega_\text{D}}{k_\text{B}} \exp \left( - \frac{1}{N(0) V_0} \right).
\end{equation}
更为精确的计算会给出
\begin{equation}
    T_\text{c} = 1.14 \frac{\omega_\text{D}}{k_\text{B}} \exp \left( - \frac{1}{N(0) V_0} \right),
\end{equation}
不过实际上以上公式自身不大,因为首先很难精确计算$V_0$,其次$V_0$的微小变化会带来很大误差。
真正会在实验上验证的通常是
\begin{equation}
    \frac{2 \Delta(T=0)}{k_\text{B} T_\text{c}} = 3.5.
\end{equation}
如果实验测量出来的比值远离3.5,那就可以确定这个超导现象不来自BCS机制。

\subsubsection{朗道-金斯堡理论}

现在我们采取另一条路,尝试直接从$U(1)$对称性被破缺这件事来获得关于超导的一些解释。
设$\Phi$为库伯对序参量,我们知道它是一个复标量场。有时称它为超导波函数,虽然它并不是任何粒子的波函数,但它服从的方程和薛定谔方程形式一致。
我们假定超导相中仅有的重要的自由度是这个序参量,其他量(比如单个电子)全部不重要。%
\footnote{这个假设是整个理论中最需要物理直觉的部分,因为并非对全部体系都有这个结论,例如做一维电子的玻色化时就不能只用一个标量场,在分析二维正方晶格的自旋波时也需要同时考虑序参量和电子。
在讨论BCS系统时能用这个假设是因为如前所述,破缺的是$U(1)$对称性,而$U(1)$对称性被破缺是因为电子在低温下配对为库伯对。
这个机制才是最关键的,因为它表明没有必要考虑单个电子的行为,电子的全部行为都被库伯对反映了,因此只需要考虑库伯对序参量即可。}%
设外加一个大小为$\vb*{A}$的磁矢势,按照规范不变性我们写下一个理论
\begin{equation}
    F = \int \dd[3]{\vb*{r}} \left( \abs*{(\grad - \ii 2 e \vb*{A}) \Phi}^2 + r \abs{\Phi}^2 + u\abs{\Phi}^4 \right).
    \label{eq:gl-theory}
\end{equation}
这里我们已经使用协变导数代替了原有的导数;系数为$2e$是因为一个库伯对带有两个负电荷,或者更加数学地说,一个库伯对包含两个湮灭算符,因此做$U(1)$变换时会有两个复数因子而不只是一个。
\eqref{eq:gl-theory}在相变点附近保证成立,一方面,相变点附近库伯对序参量不大,可以级数展开取前几项,另一方面,通过量纲分析可以发现
\[
    [\Phi] \sim [L]^{-1/2},
\]
容易验证没有提到的项在重整化不动点附近全部是不相关的,在重整化群作用下会被压低。
我们遵从做朗道-金斯堡理论时的惯例,适当调节单位制来让梯度平方项前面的系数为1。

计算\eqref{eq:gl-theory}的极小值点,可以得到
\begin{equation}
    - \frac{(\grad - \ii e \vb*{A})^2}{4 m^*} \Phi + \alpha \Phi + \beta \abs{\Phi}^2 \Phi = 0.
\end{equation}
这个方程的形式和薛定谔方程非常接近,当然这更多是对称性带来的结果,即薛定谔场也具有$U(1)$对称性。

$U(1)$对称性给出的守恒流为
\begin{equation}
    \vb*{j} \propto \Phi^* \grad{\Phi} - \Phi \grad{\Phi^*},
\end{equation}
当然这就是电流。当然,由于我们只是在做对称性分析,并不能明确地得出式子右边的系数。
上式说明超导的电流来自序参量的梯度,这和通常导电的机制(电场下费米面发生移动)不同。由于序参量是复的,即使不存在振幅输运,仅仅靠不同点相位不同就足够产生持续电流。这是一个稳态解,所以不存在能量消耗。这也就是超导体唯象地看起来似乎有某种运动全然不会受到阻碍的“反常电子”的原因。

\subsection{二维正方晶格的反铁磁长程序}

\subsubsection{二维正方晶格上的自旋}

现在考虑一个二维正方晶格,假定它可以有一个反铁磁相,也即,相邻格点的自旋倾向于变得反平行,或者说形成一个自旋密度波。
设系统的哈密顿量为
\begin{equation}
    {H} = \sum_{\vb*{k}, \alpha} \xi_{\vb*{k}} {c}_{\vb*{k} \alpha}^\dagger {c}_{\vb*{k} \alpha} + J \sum_{\pair{i, j}} {\vb*{S}}_i \cdot {\vb*{S}}_j,
    \label{eq:2dim-square-spin}
\end{equation}
其中
\begin{equation}
    {\vb*{S}}_i = \sum_{\alpha, \beta} {c}^\dagger_{i \alpha} \vb*{\sigma}_{\alpha \beta} {c}_{i \beta}
    \label{eq:spin-wave-order-parameter}
\end{equation}
为格点$i$的自旋矢量。这个模型本身其实并不非常现实,因为自旋相互作用通常来自交换能,但是交换能通常在绝缘系统中比较重要,那么就不应该有一个动能项。
设反铁磁序的序参量为$\vb*{\phi}$,一个不错的选择是
\begin{equation}
    \expval*{{\vb*{S}}_i} = (-1)^i \vb*{\phi},
\end{equation}
这里的$(-1)^i$实际上是一种滥用记号:它实际上是
\[
    (-1)^i = (-1)^{i_x + i_y}
\]
的简写。为了方便起见,我们把$(i_x + i_y)$为奇数的格点的全体记为$A$,将$(i_x + i_y)$为偶数的格点的全体记为$B$,于是$A$中任何一个格点的近邻格点都在$B$中,反之亦然。
如果$\vb*{\phi}$非零,那么显然$\expval*{{\vb*{S}}_i}$在$i \in A$时和$\expval*{{\vb*{S}}_i}$在$i \in B$时差一个负号,即相邻的自旋一定是反向的,正好意味着形成了反铁磁序。

不失一般性地令晶格常数为1,则第一布里渊区为$[-\pi, \pi)^2$。在形成了一个完整的反铁磁序之后序参量$\Delta$的周期应该是两个格点,于是应有$\vb*{Q}=(\pi, \pi)$。

\subsubsection{平均场近似}

在\eqref{eq:2dim-square-spin}中做平均场近似
\begin{equation}
    \vb*{{S}}_i \cdot \vb*{{S}}_j = \expval*{\vb*{{S}}_i} \cdot \vb*{{S}}_j + \vb*{{S}}_i \cdot \expval*{\vb*{{S}}_j} - \expval*{\vb*{{S}}_i} \cdot \expval*{\vb*{{S}}_j},
\end{equation}
忽略平均场近似引入的常数项(我们这里不做自洽计算而只是分析相变点具有的性质),自旋相互作用项为
\[
    \begin{aligned}
        {H}_\text{int} = J \sum_{\pair{i, j}} {\vb*{S}}_i \cdot {\vb*{S}}_j &\sim J \sum_{\pair{i, j}} (\expval*{\vb*{{S}}_i} \cdot \vb*{{S}}_j + \vb*{{S}}_i \cdot \expval*{\vb*{{S}}_j}) \\
        &= 2 J \sum_{\pair{i, j}} \expval*{\vb*{{S}}_j} \cdot \vb*{{S}}_i \\
        &= J \sum_i \sum_{j \in \text{nn of } i} \expval*{\vb*{{S}}_j} \cdot \vb*{{S}}_i,
    \end{aligned}
\]
其中nn表示“最近邻”,第二个等号中因子2消失了是因为$\pair{i, j}$对一对近邻只求和一次,而第三行中一对近邻实际上被求和了两次。
代入\eqref{eq:spin-wave-order-parameter},注意到两个相邻格点的$(-1)^i$差一个负号,有
\[
    \begin{aligned}
        {H}_\text{int} &= J \sum_i \sum_{j \in \text{nn of } i} (-1)^j \vb*{\phi} \cdot \vb*{{S}}_i \\
        &= - J \sum_i (-1)^{i} \sum_{j \in \text{nn of } i} \vb*{\phi} \cdot \vb*{{S}}_i,
    \end{aligned}
\]
由于系统具有自旋旋转不变性,不失一般性地要求$\vb*{\phi}$指向$z$轴,并设
\begin{equation}
    \Delta = 4 J \phi,
\end{equation}
则
\begin{equation}
    {H}_\text{int} = - \sum_i (-1)^i {S}_i^z \Delta = - \sum_i (-1)^i \Delta ({c}_{i\uparrow}^\dagger {c}_{i \uparrow} - {c}_{i \downarrow}^\dagger {c}_{i \downarrow}).
\end{equation}
设$\vb*{Q}=(\pi, \pi)$,则相互作用哈密顿量可以写成以下傅里叶变换的形式: 
\begin{equation}
    {H}_\text{int} = - \sum_i \ee^{\ii \vb*{Q} \cdot \vb*{r}_i} \Delta ({c}_{i\uparrow}^\dagger {c}_{i \uparrow} - {c}_{i \downarrow}^\dagger {c}_{i \downarrow}) 
    = - \Delta \sum_{\vb*{k}} ({c}_{(\vb*{k} + \vb*{Q})\uparrow}^\dagger {c}_{\vb*{k} \uparrow} - {c}_{(\vb*{k}+\vb*{Q}) \downarrow}^\dagger {c}_{\vb*{k} \downarrow}).
\end{equation}
于是最后平均场哈密顿量就是
\begin{equation}
    {H}_\text{MF} = \sum_{\vb*{k}, \alpha} (\xi_{\vb*{k}} {c}_{\vb*{k} \alpha}^\dagger {c}_{\vb*{k} \alpha} - \alpha \Delta {c}_{(\vb*{k} + \vb*{Q}) \alpha}^\dagger {c}_{\vb*{k} \alpha}).
    \label{eq:2dim-square-spin-mf}
\end{equation}
这里指定$\uparrow$对应$1$,$\downarrow$对应$-1$,虽然电子的自旋为$\pm 1/2$而不是$\pm 1$。
\eqref{eq:2dim-square-spin-mf}两边乘上2,并注意对$\vb*{k}$求和等价于对$\vb*{k}+Q$求和,且$\vb*{k}$等价于$\vb*{k}+2\vb*{Q}$,\eqref{eq:2dim-square-spin-mf}给出的能谱等价于以下矩阵
\[
    \pmqty{\xi_{\vb*{k}} & - \alpha \Delta \\ - \alpha \Delta & \xi_{\vb*{k} + \vb*{Q}}}
\]
的本征值,也就是
\begin{equation}
    E_{\vb*{k}} = \frac{\xi_{\vb*{k}} + \xi_{\vb*{k} + \vb*{Q}}}{2} \pm \sqrt{\frac{(\xi_{\vb*{k}} - \xi_{\vb*{k}+\vb*{Q}})^2}{4} + \alpha^2 \Delta^2}.
\end{equation}
这样可以得到两条能带,在$\Delta$为零时它们可以交叉,在$\Delta$不为零时能带交叉的部位就打开了一个能隙。

\subsubsection{“热点”和它附近的低能有效理论}

\eqref{eq:2dim-square-spin-mf}是\eqref{eq:2dim-square-spin}在反铁磁序相的一个有效理论。反铁磁序破缺了晶格上的平移不变性,因为反铁磁序状态中每个晶格和临近的晶格的自旋差一个负号。
反铁磁序还保留了一部分平移不变性:子格点$A$和$B$上仍有平移不变性,这两个子格点对应的倒格矢都是$\vb*{Q}$。
这样,\eqref{eq:2dim-square-spin-mf}的第一布里渊区是
\[
    \abs{k_x} + \abs{k_y} \leq \pi,
\]
这是\eqref{eq:2dim-square-spin}的第一布里渊区折叠之后得到的。
布里渊区变小当然是因为原胞变大了——从单个格点变成了一个$A$格点加上一个$B$格点。(见\autoref{sec:quasi-particle-spectrum})

如果费米面和折叠后的第一布里渊区相交,费米面交叉处打开能隙,形成一系列小费米pocket。热点附近的$\vb*{k}$和$\vb*{k} + \vb*{Q}$都在热点附近。

接下来讨论在热点附近的低能有效理论,换而言之,我们开始讨论超越平均场的物理。
需要注意的是考虑超越平均场的物理意味着将原本被忽略的涨落重新加入,这可能改变热点的位置或者甚至让热点消失(在本问题中不可能因为涨落不大,但确实有这样的体系,涨落会让平均场理论中出现的现象——如相变——消失,比如一维伊辛模型)。
因此,下面讨论的热点附近的低能有效理论建立在三个基础上:
\begin{enumerate}
    \item 费米面上存在热点,这是通过平均场理论算出来的,我们假定这个现象确实存在,而不是像一维伊辛模型一样只是幻象;
    \item 热点附近的低能有效理论是\eqref{eq:2dim-square-spin}的低能有效理论,即相互作用项一定是自旋-自旋相互作用(这并非假设,而是必然的事实);
    \item 低能自由度是低能电子和自旋波模式这两种场(实际上这是最重要的一个假设,因为低能自由度是什么通常难以直接计算得到)。
\end{enumerate}

\eqref{eq:2dim-square-spin}中的相互作用项是一个电子和一个自旋波模式(这是玻色子)发生散射,得到另一个电子,或者也可以说是自由电子的自旋和自旋波模式的相互作用。%
\footnote{自旋波模式是一部分形成了自旋波的电子被积掉之后得到的,和未形成自旋波长程序的电子是两群不同的电子。}%
因此低能有效理论的散射项形如%
\footnote{一个可能的问题是,为什么一定保留低能电子自由度和自旋波自由度?为什么不能是自旋波自由度和密度波自由度?不考虑自旋波-密度波自由度是因为没有明确的物理机制让这两个模式发生耦合,保留低能电子自由度是因为自旋算符和电子数算符对易,但和单个费米子算符并不对易,因此可能出现费米子自由度积不掉的情况。
这和BCS理论是不一样的,在BCS理论中大量电子参与配对,以至于哪怕电子自由度积不掉,它也不会产生太大作用。}%
\[
    {\phi}_{\vb*{q}} {c}^\dagger_{\vb*{k}+\vb*{q}\sigma} {c}_{\vb*{k}\sigma'},
\]
由于是低能有效理论,我们认为形成了一个基本上稳定的反铁磁序,于是$\vb*{q}$接近$\vb*{Q}$,而电子能量$\epsilon_{\vb*{k}}$和$\epsilon_{\vb*{k}+\vb*{q}}$都在费米面附近。
这些条件只有在热点附近才能达到。
当然由于热点是费米面交叠之后打开能隙的位置,在这附近有低能有效理论是非常合理的。
这样,一个散射过程涉及两个热点。分别用1和2标记两个热点,记它们的动量为$\vb*{K}_1$和$\vb*{K}_2$,则有
\[
    \vb*{K}_2 = \vb*{K}_1 + \vb*{Q},
\]
并重新定义$\vb*{q}$和$\vb*{k}$为它们偏离$\vb*{Q}$和$\vb*{K}_1$的大小。这样,相互作用哈密顿量就是
\[
    {\vb*{\phi}}_{\vb*{q}} \cdot (\sum_{\alpha, \beta} {c}^\dagger_{1(\vb*{k} + \vb*{q})\alpha} \vb*{\sigma}_{\alpha \beta} {c}_{2\vb*{k} \beta} + \text{h.c.}),
\]
这里任何一个散射过程中入射电子为热点1附加,出射电子在热点2附加或相反的原因是要保证所有电子的动量都在热点附加。于是完整的有效理论的哈密顿量为
\begin{equation}
    {H}_\text{eff} = \sum_{\vb*{k}, \sigma} (\xi_{1\vb*{k}} {c}^\dagger_{1\vb*{k} \sigma} {c}_{1 \vb*{k} \sigma} + \xi_{2\vb*{k}} {c}^\dagger_{2\vb*{k} \sigma} {c}_{2 \vb*{k} \sigma}) + {H}[\vb*{\phi}] + \lambda \sum_{\vb*{q}} {\vb*{\phi}}_{\vb*{q}} \cdot (\sum_{\alpha, \beta} {c}^\dagger_{1(\vb*{k} + \vb*{q})\alpha} \vb*{\sigma}_{\alpha \beta} {c}_{2\vb*{k} \beta} + \text{h.c.}).
    \label{eq:2dim-square-spin-eff}
\end{equation}
上式中没有明确提玻色场——也就是自旋波场——的自由哈密顿量。这个自由哈密顿量通常是通过对称性分析得出的。
进一步,由于是低能理论,将$\xi_{\vb*{k}}$在费米面附近做展开,仅保留一阶项,得到
\begin{equation}
    \xi_{1\vb*{k}} = \epsilon_{\vb*{k}} - \mu = \vb*{v}_1 \cdot \vb*{k}, \quad \xi_{2\vb*{k}} = \epsilon_{\vb*{k}} - \mu = \vb*{v}_2 \cdot \vb*{k}.
\end{equation}
使用$\vb*{v}_1$和$\vb*{v}_2$来标记$\grad_{\vb*{k}}{\xi_{\vb*{k}}}$是因为如果是自由电子气,那么它们就是费米速度。

\subsubsection{朗道阻尼}

现在我们尝试把电子完全积掉,只留下玻色型的自旋波模式。需要指出的是这个操作并不总是可行的:由于自旋波模式和电子都是低能自由度,简单地积掉其中一个自由度可能不能得到一个良定义的有效理论。
实际上,对二维体系的确会有这种棘手的细节。
下面的操作都是在默认确实可以积掉电子的前提下进行的。

从\eqref{eq:2dim-square-spin-eff}可以看出,一个自旋波模式可以衰变成一对电子空穴对,或者说一个自旋波模式可以将动量转移给一个电子而得到另一个电子,而自身湮灭。
因此自旋波模式是有有限的寿命的。由费米黄金法则,一个动量为$\vb*{q}$,能量为大于零的$\omega$的自旋波模式的寿命倒数为
\[
    \begin{aligned}
        \frac{1}{\tau} &\sim 2 \lambda^2 \int \frac{\dd[2]{\vb*{k}}}{(2\pi)^2} \delta(\omega + \epsilon_{1 \vb*{k}} - \epsilon_{2 (\vb*{k} + \vb*{q})}) \theta(- \epsilon_{1 \vb*{k}}) \theta(\epsilon_{2 (\vb*{k} + \vb*{q})}) \\
        &\sim \lambda^2 \int \frac{\dd{p_1} \dd{p_2}}{(2\pi)^2 \abs*{\vb*{v}_1 \times \vb*{v}_2}} \delta(\omega + p_1 - p_2) \theta(- p_1) \theta(p_2),
    \end{aligned}
\]
其中我们设
\[
    p_1 = \vb*{v}_1 \cdot \vb*{k}, \quad p_2 = \vb*{v}_2 \cdot (\vb*{k} + \vb*{q}),
\]
我们仅考虑低能理论,因此假定入射电子在费米面以下,出射电子在费米面以上。由几何关系,
\[
    \int \frac{\dd{p_1} \dd{p_2}}{\abs*{\vb*{v}_1 \times \vb*{v}_2}} \delta(\omega + p_1 - p_2) \theta(- p_1) \theta(p_2) = \sqrt{2} \omega,
\]
$\omega < 0$的情况也是一样的。总之最后自旋波模式的寿命为
\begin{equation}
    \frac{1}{\tau} \sim \frac{\lambda^2}{\abs*{\vb*{v}_1 \times \vb*{v}_2}} \abs*{\omega} = \gamma \abs*{\omega}.
\end{equation}
在格林函数中,衰变几率对应着自能修正,会直接反映在$\vb*{\phi}$的格林函数——从而有效作用量——中,也即,自旋波模式的推迟格林函数形如
\[
    G_{\vb*{\phi}}^{-1} = \ii \gamma \abs{\omega} + \cdots = \ii  \gamma \sgn(\omega) \omega + \cdots,
\]
相应的,松原格林函数形式为%
\footnote{这里的步骤是:先做Wick转动,即$\omega = \ii \omega_n$,}%
\[
    G_{\vb*{\phi}}^{-1} = \gamma \abs{\omega_n} + \cdots.
\]
这就意味着自旋波模式的有效热力学作用量为
\begin{equation}
    \begin{aligned}
        S_\text{eff} &= \sum_{\vb*{q}, \omega_n} \vb*{\phi}(-\vb*{q}, -\omega_n) \cdot (\gamma \abs{\omega_n} + \omega_n^2 + c^2 \vb*{q}^2 ) \vb*{\phi}(\vb*{q}, \omega_n) \\
        &= \sum_{\vb*{q}, \omega_n} \vb*{\phi}^*(\vb*{q}, \omega_n) \cdot (\gamma \abs{\omega_n} + \omega_n^2 + c^2 \vb*{q}^2) \vb*{\phi}(\vb*{q}, \omega_n).
    \end{aligned}
    \label{eq:effective-spin-action}
\end{equation}
第二个等号是因为$\vb*{\phi}$是实场,因为负的动量/频率等价于取复共轭。
$\omega_n^2$项和$\vb*{q}^2$项都是对称性分析加入的项,除此以外的项在低能有效理论中并不重要。%
$\omega_n^2$项和$\vb*{k}^2$项可以容易地切换到实空间,它们分别对应着时间导数平方项$(\partial_\tau \phi)^2$和梯度平方项$(\grad{\phi})^2$,但$\abs{\omega_n}$在实空间中没有简单的形式。
不过,一个正比于$\omega_n$的项意味着实空间中有某种阻尼,这也是正确的,因为自旋波模式如前所述会衰变。这种阻尼或者衰变称为\concept{朗道阻尼}。朗道阻尼指的是没有粒子之间的相互碰撞,仅仅粒子和波强烈耦合也能够产生阻尼。

\subsubsection{RPA近似计算朗道阻尼}

还可以通过直接计算格林函数的方式来得到$\phi$的有效理论。
这个有效理论当然取
\[
    Z_\text{eff} = \int \mathcal{D}\vb*{\phi} \ee^{- \sum_{q, \omega_n} \vb*{\phi}^*(\vb*{q}, \omega_n) G^{-1}_{\vb*{\phi}}(\vb*{q}, \omega_n) \vb*{\phi}(\vb*{q}, \omega_n)}
\]
的形式。

\subsubsection{Hertz理论}

最后我们讨论上一节得到的低能有效理论的

\subsubsection{RPA近似}

我们尝试对平均场近似做一些修正。为此我们将不再直接处理自旋算符,而是把一切都转化到费米子算符上。
做傅里叶变换
\[
    {S}_{\vb*{q}} = \frac{1}{\sqrt{N_\text{site}}} \sum_i {S}_i \ee^{\ii \vb*{r}_i \cdot \vb*{q}} = \frac{1}{\sqrt{N_\text{site}}} \sum_{\vb*{k}} {c}^\dagger_{\vb*{k}+\vb*{q}, \alpha} \vb*{\sigma}_{\alpha \beta} {c}_{\vb*{k} \beta},
\]
这样相互作用项就是(请注意$\vb*{\sigma}$)

对相互作用项做正规序不会影响定性的结果,我们不需要动手算就知道正规序和原本的相互作用只会差一个单体项,而这个单体项是自选旋转不变的,那么它只会对$\epsilon_{\vb*{k}} - \mu$做一个修正。
于是我们将要处理以下相互作用哈密顿量:% TODO:自旋
\begin{equation}
    {H} = \frac{1}{2 N_\text{site}} \sum_{\vb*{k}, \vb*{k}', \vb*{q}, \alpha, \beta} {c}^\dagger_{\vb*{k}-\vb*{q}, \alpha} {c}^\dagger_{\vb*{k}'+\vb*{q}, \beta} V(\vb*{q}) {c}_{\vb*{k}'} {c}_{\vb*{k}}, \quad V(\vb*{q}) = 2 J (\cos(q_x) + \cos(q_y)).
\end{equation}

\begin{equation}
    Z = \int \fd{\vb*{\phi}} \exp(- \int_0^\beta \dd{\tau} )
\end{equation}

$\vb*{\phi}$就像驱动自旋的一个外场一样,在只取鞍点近似时它就是平均场序参量。

\input{one-dim-electron.tex}

\section{二维系统和拓扑序}

所谓“拓扑物态”指的是由拓扑性质而不是对称性描述的特殊物态。

\subsection{量子霍尔效应}

\subsubsection{经典霍尔效应}

首先考虑经典电动力学预言的霍尔效应。在有自由载流子的固体材料中电导率是一个张量。设我们在$y$方向施加一个电场$E_y$,并施加一个在$-z$方向的磁场$B$。
设载流子为电子,则稳态时电子应当朝左运动,且
\[
    \frac{e v B}{c} = e E_y,
\]
而电流方向朝右,大小为
\[
    j_x = - n_\text{e} e v = \frac{e c n_\text{e}}{B} E_y.
\]
我们将$\sigma_{xy}$这种横向的电导称为\concept{霍尔电导},记为
\begin{equation}
    \sigma_\text{H} = \frac{ec}{B} n_\text{e}.
\end{equation}

定义\concept{磁通量子}(它是超导中磁通量量子化的单位乘以2)
\begin{equation}
    \Phi_0 = \frac{h c}{e},
\end{equation}
一个电子绕着磁通量子整数倍的一个磁通量转一圈,不会产生可观测的效果(因为Berry相的变化是$2\pi$的整数倍)。
磁通量子是体系磁通量的一个自然的尺度。相应地,定义归一化磁通密度为
\begin{equation}
    n_{\Phi} = \frac{B}{\Phi_0},
\end{equation}
这给出了一个自然的电子数尺度,于是定义
\begin{equation}
    \nu = \frac{n_\text{e}}{n_\Phi} = \frac{n_\text{e} h c}{B e}.
\end{equation}
使用以上归一化之后的物理量,就可以写出霍尔电导为
\begin{equation}
    \sigma_\text{H} = \nu \frac{e^2}{h}.
\end{equation}
当然,到这里,$\nu$其实还是可以连续取值的。

\subsubsection{二维电子气的朗道能级}



\subsubsection{整数电导平台}

实际的体系中有杂质,因此朗道能级会出现展宽,并且两个朗道能级之间的态基本上是局域化态(朗道能级中的态则比较舒展)。
换而言之,朗道能级附近的电子可以自由移动,贡献电导,而两个朗道能级之间的电子高度定域,并不贡献电导。
因此,随着电子数上升,在填充延展态时电导线性上升,而在填充定域态时电导没有变化。这就形成了\concept{电导平台}。
相应的,平台处的霍尔电导为
\[
    \sigma_\text{H} = n \frac{e^2}{h}, \quad n = 0, 1, 2, \ldots,
\]
称为\concept{整数霍尔效应}。
总之,要形成量子霍尔效应,对电导有贡献的能级必须有能隙。

这就造成了一个非常矛盾的情况:要观察到电导平台,体系必须比较“脏”,这样才能够有明确的定域态,从而形成平台;但实际上,在体系很脏时朗道能级附近的延展态也被破坏了,因此太脏的体系展现不出太多平台。
因此明显的量子霍尔效应需要体系有些脏但又不太脏。

“对电导有贡献的能级必须有能隙”还产生了一个问题:既然有能隙,那么在化学势位于能隙之中时如何形成显著的电流?
唯一的可能是,体态有能隙,但边界态没有能隙,从而边界态导电。(这还意味着一件事:体态有能隙说明体态内部的关联长度有限,因此边界态和体态相对独立)
可以从一个经典图像看到这一点:在体态中电子可以不停做圆周运动,而在边界附近电子做完半个圆周运动后被反弹,而又往前做半个圆周运动。
因此,可以形成绕着整个边界运行的(手性的)电流。边界态上如果有杂质,电子可以潜到体态中,绕过杂质,形成一个新的边界态。因此边界态上的电流是无损耗的。

关于为什么霍尔电导取非常简洁的形式(通常的电导涉及关于材料的复杂性质),\concept{Laughlin论证}给出了一个解释。
设有一个半径为$L$的圆柱体,其体态有能隙而边界态导电。
沿着它的轴向加入一个磁场,总磁通量为$\Phi$。让$\Phi$缓慢地发生变化,从零变化到$\frac{h c}{e}$,则$\Phi$变化前后圆柱体均处于同样的状态,因为绕着大小为$\frac{h c}{e}$的磁通量转一圈什么也不会发生。
磁通量的变化会导致一个电场,在边界上,我们有
\[
    2 \pi L E = - \frac{1}{c} \dv{\Phi}{t},
\]
从而
\[
    E(t) = \frac{1}{2\pi L c} \dv{\Phi}{t}.
\]
边界上的电场方向垂直轴向,从而产生一个平行于轴向的霍尔电流
\[
    I = 2 \pi L j = 2 \pi L \sigma_\text{H} E = \frac{\sigma_\text{H}}{c} \dv{\Phi}{t}.
\]
这个电流造成的电荷量变化为
\[
    \Delta Q = \int \dd{t} I = \frac{\sigma_\text{H}}{c} \Delta \Phi = \sigma_\text{H} \frac{h}{e},
\]
而由于电荷由电子携带,有
\[
    \Delta Q = me, \quad m = 0, 1, 2, \ldots,
\]
于是
\[
    \sigma_\text{H} = m \frac{e^2}{h}, \quad m = 0, 1, 2, \ldots.
\]
这就是整数阶量子霍尔效应的来源:它和体系的结构完全无关,只要体系体态有能隙,就能够得出存在整数霍尔效应的结论。

\subsubsection{$1/m$型分数量子霍尔效应}

既然Laughlin论证只能够得到整数量子霍尔效应,我们要问,分数量子霍尔效应是怎么产生的。
显然,唯一的可能是,电子之间的相互作用产生了分数阶能级。
本节讨论
\[
    \nu = \frac{1}{m}, \quad m = 1, 3, 5, \ldots
\]
型的分数量子霍尔效应,这是最简单的情况。

Laughlin通过其天才的创造,一步到位地给出了能产生分数霍尔效应的波函数:
\begin{equation}
    \Phi(z_1, \ldots, z_N) = \prod_{i < j} (z_i - z_j)^{m} \ee^{- \sum_i \abs*{z_i}^2 / 4 l_0^2}, \quad m = 1, 3, 5, \ldots.
\end{equation}
容易验证以上波函数满足交换反对称性;当$z_i$趋于$z_j$时波函数趋于零,这是库伦排斥的结果。

\subsection{Toric-code模型}

\subsubsection{Toric-code哈密顿量与解析解}

Kitaev最早提出了一种模型,作为一种可能的量子计算纠错编码,他发现这个模型放在一个环面上可以有非常有趣的结果。
然而,事后发现这个模型实际上展现出了一个拓扑序。

考虑一个正方格点,在每条边(不是每个格点!)上放有一个自旋$1/2$的粒子——比如说电子。
哈密顿量为
\begin{equation}
    {H} = - \sum_s {A}_s - \sum_p {B}_p,
    \label{eq:toric-code-hamiltonian}
\end{equation}
其中下标$s$表示格点,${A}_s$指的是格点$s$周围的四条边上的$x$方向上的自旋算符的乘积,即
\begin{equation}
    {A}_s = \prod_{i\text{ near } s} {\sigma}_i^x,
\end{equation}
而$p$表示格点中的一个最小正方形,${B}_p$指的是正方形$p$的四条边上的$z$方向上的自旋算符的乘积,即
\begin{equation}
    {B}_p = \prod_\text{$i$ of $p$} {\sigma}_i^z.
\end{equation}

\eqref{eq:toric-code-hamiltonian}是严格可解的。
首先可以验证$\{{A}_s\}$和$\{{B}_p\}$构成一组对易稳定子(即平方为1的一组彼此对易的厄米算符),这样就有
\begin{equation}
    \comm*{{A}_s}{{H}} = \comm*{{B}_p}{{H}} = 0.
\end{equation}
另一方面,平方为1的厄米算符的本征值是$\pm 1$,于是我们就可以用它们的本征值$A_s = \pm 1$和$B_p = \pm 1$标记体系的能量本征态。
实际上,在热力学极限下只需要$\{A_s\}$和$\{B_p\}$就可以唯一地标记体系的能量本征态。
这是因为设体系有$N$个格点,那么有$4N/2=2N$条边,于是体系的希尔伯特空间的维数为$2^{2N}$。
$s$和$p$均有$N$个,于是所有可能的$\{A_s\}$和$\{B_p\}$的组合总数为$2^N \cdot 2^N=2^{2N}$。
这样如果不考虑边界引入的微妙之处,只需要$\{A_s\}$和$\{B_p\}$就可以唯一地标记体系的能量本征态。
很容易看出体系的基态为所有$A_s$和$B_p$均为$1$的状态,于是我们可以把$A_s$和$B_p$为$-1$的情况看成激发态。这样我们就得到了\eqref{eq:toric-code-hamiltonian}的全部能量本征态,从而完全求解出了它。

\subsubsection{环面上的情况}

为了解析求解,我们施加一个周期性边界条件,这相当于把体系放在了一个二维环面上。
此时诸$\{A_s\}$和$\{B_p\}$实际上不是彼此独立的,因为此时显然有
\[
    \prod_s {A}_s = 1,
\]
因为所有的$\{A_s\}$乘起来,每一条边被乘了两边,所以一定会得到$1$。类似的有
\[
    \prod_p {B}_p = 1.
\]
这两个方程要求
\begin{equation}
    \prod_{s} A_s = \prod_{p} B_p = 1.
    \label{eq:toric-code-pair-condition}
\end{equation}
这就意味着$A_s$激发和$B_p$激发必须成对出现,否则乘积将会是$-1$。我们将$A_s$激发称为e粒子,而将$B_p$激发称为m粒子,因为在某种意义上可以将$A_s$激发类比为电荷而将$B_p$激发理解为磁通量子。
这两种粒子的性质和空间的拓扑结构显然关系很大,因此称它们为拓扑激发。

两种激发成对出现的事实意味着可以使用弦算符描述它们的产生和消灭。首先考虑由一条边连接的两个格点,这条边上的${\sigma}^z_i$算符可以将这条边上的$x$方向的自旋翻转,因此它可以做到以下三件事:
\begin{itemize}
    \item 如果两个格点上原本没有e粒子,那么在两个格点上同时产生e粒子;
    \item 如果两个格点上原本都有e粒子,那么在两个格点上同时消灭e粒子; 
    \item 如果两个格点一个有e粒子一个没有,那么该e粒子将被转移到原本没有e粒子的格点上。
\end{itemize}
这样设一系列首尾相连的边$\{i\}$连接了两个格点,则弦算符
\begin{equation}
    {O}_\text{e} = \prod_{i} {\sigma}_i^z
\end{equation}
同样可以做到以上三件事。
同样,将以上论述中的${\sigma}^z$换成${\sigma}^x$,“格点”换成“正方形格子”,“连接两个格点的边”换成“正方形格子共享的边”(我们可以在每个正方形格子中间放置一个点,从而m粒子也定义在一个格点上),同样可以定义弦算符
\begin{equation}
    {O}_\text{m} = \prod_{i} {\sigma}_i^x.
\end{equation}
以上讨论的都是开放的弦,闭合的弦的行为需要具体分析,且对闭弦有
\begin{equation}
    {O}_\text{e} \ket{0} = {O}_\text{m} \ket{0} = \ket{0}.
\end{equation}

通过弦算符可以检查e粒子和m粒子绕对方转一圈(实际上就是使用一个闭合的弦算符作用在一个有e粒子或者m粒子的格点上),都会多出来一个$\pi$的相位,这是因为如果一个m粒子闭弦和一个e粒子开弦有单个交点,那么它们反对易(因为同一个边上的${\sigma}^x$和${\sigma}^z$反对易)。
换而言之,e粒子和m粒子均为任意子激发:这是二维的特殊现象,因为二维的环路在二维平面上它围绕的区域被挖掉一个点之后就不可缩了,因此一个粒子转一圈之后可以有一个非零相位变化。
本节涉及的激发尚为阿贝尔统计,即转一圈之后得到的量子态和转之前只差一个$U(1)$变换;还有非阿贝尔统计,即转一圈可以转移到别的量子态上。

\subsubsection{任意子表}

现在的问题是,环面上的Toric-code模型中最多能够弄出来多少任意子?显然e粒子和m粒子都是任意子,虽然两者自己满足玻色统计,但它们之间有一个非平凡的相位。
我们下面将以拓扑性质分类激发,即,拓扑性质相同的激发算作一种。
可以用两个量来标记一种激发的拓扑性质:设$M_{ab}$为$b$绕着$a$转一圈导致的复数因子,$\theta_a$指的是交换两个$a$导致的复数因子(或者说一个$a$绕着另一个$a$转半圈导致的复数因子)。这样,有
\begin{equation}
    \theta_\mathrm{e} = \theta_\mathrm{m} = 1, \quad M_\mathrm{em} = - 1.
\end{equation}
除了e粒子和m粒子以外肯定还有一种$\mathrm{\epsilon}$粒子,它是一个e粒子和m粒子聚合%
\footnote{所谓聚合指的是将两个激发放得尽可能近,从而得到的复合激发。e粒子和m粒子定义在不同的格点上,因此一个e粒子和一个m粒子的聚合就是在一个正方格子中央放置一个m粒子,在它的某个角上放置一个e粒子之后得到的激发,从远处看这近似于一个粒子。}%
而成的粒子,即
\begin{equation}
    \mathrm{\epsilon} = \mathrm{e} \otimes \mathrm{m}.
\end{equation}
可以容易地验证
\begin{equation}
    M_\mathrm{e\epsilon} = M_\mathrm{m\epsilon} = -1, \quad \theta_\mathrm{\epsilon} = -1.
\end{equation}
e粒子、m粒子和$\mathrm{\epsilon}$粒子这三种拓扑激发都只能成对出现。
除了这三种激发以外还有一些平凡的激发,比如声子之类,将它们全部记为$\mathbbm{1}$。

实际上,e粒子、m粒子和$\epsilon$粒子和$\mathbbm{1}$就是全部拓扑激发。
由于$\mathbbm{1}$无论如何绕圈都不会产生附加的相位,就有
\[
    \mathbbm{1} \otimes a = a.
\]
两个e粒子放在一起,得到的就是某个边上的$\sigma^x$发生了翻转,这是一个普通的激发;m粒子和$\mathrm{\epsilon}$粒子也是如此,于是
\[
    \mathrm{e} \otimes \mathrm{e} = \mathrm{m} \otimes \mathrm{m} = \mathrm{\epsilon} \otimes \mathrm{\epsilon} = \mathbbm{1}.
\]
上式实际上说明了一个非常重要的事实:封闭流形上无论有多少拓扑激发,这个态都可以通过对基态作用一些产生算符得到,或者等价地说改变基态上某些格点的值得到,那么如果将这些拓扑激发聚合到一起,得到的只是基态上局域的一些点被改变了,也即得到了一个平凡的激发。
总之,封闭流形上所有的拓扑激发聚合在一起,只会得到平凡的激发。这就从另一个角度解释了为什么非平凡的拓扑激发一定成对出现。
$\mathrm{\epsilon}$和e粒子聚合,就相当于两个e粒子先聚合得到一个平凡的激发,剩下一个m粒子,$\mathrm{\epsilon}$粒子和m粒子聚合则会留下一个e粒子和一个平凡的激发,于是
\[
    \mathrm{\epsilon} \otimes \mathrm{e} = \mathrm{m}, \quad \mathrm{\epsilon} \otimes \mathrm{m} = \mathrm{e}.
\]
因此,e粒子、m粒子和$\epsilon$粒子和$\mathbbm{1}$在聚合运算$\otimes$下是封闭的。

\subsubsection{四重简并和Berry相}

回忆一下,体系的希尔伯特空间维数为$2^{2N}$。当$2N-2$个边的自旋已经确定之后,系统的状态实际上已经确定了,因为约束条件\eqref{eq:toric-code-pair-condition}会确定剩下两条边的自旋。
换而言之,实际物理的希尔伯特空间维数只有$2^{2N-2}$。
这就意味着总希尔伯特空间$2^{2N}$分裂成了4支,或者说每个状态都有四重简并。
这个事实——环面上的Toric-code模型会出现基态四重简并——是\eqref{eq:toric-code-pair-condition}决定的,而\eqref{eq:toric-code-pair-condition}本身又来自环面的拓扑性质。
如果我们在哈密顿量中引入一个局部的扰动,基态能量和基态波函数显然会发生扰动,但是由于${A}$和${B}$的定义没有变化,系统拓扑没有变化,\eqref{eq:toric-code-pair-condition}也是始终成立的。
换句话说,环面上基态的四重简并是\concept{受到拓扑保护}的,局域的扰动不能让它消失。

用什么标记这四重简并?容易想到,完全可以定义一种全局性的闭弦算符,它贯穿整个环面,而由周期性边界条件它是闭弦算符。(这些算符的定义本身和拓扑紧密相关,显然如果系统被放在一个平面上那么根本没法定义全局性的闭弦算符)
分别沿着$x$轴和$y$轴定义
\begin{equation}
    {L}^x_\text{e} = \prod_{x} {\sigma}^z_i, \quad {L}^x_\text{m} = \prod_{x} {\sigma}^x_i,
\end{equation}
并可以验证它们和哈密顿量是对易的,且它们构成一对对易稳定子。
这就意味着它们的本征值均为$\pm 1$,这就唯一地标记了四重简并。
以上两个弦算符显然形成了某种序,但是并不是金斯堡-朗道理论中的那种局域的序,而是一种\concept{拓扑序};相应的激发为\concept{拓扑激发}。

${L}^x_\text{e}$和${L}^x_\text{m}$将e粒子绕着$x$轴转动一圈,因此它们的本征值实际上给出了x方向类似于磁通量的一个通量,这个通量导致了一个Berry相位。

类似地还可以定义${L}^y_\text{e}$和${L}^y_\text{m}$,并且
\begin{equation}
    \acomm*{{L}^x_\text{e}}{{L}^y_\text{m}} = 0.
\end{equation}
我们知道
\begin{equation}
    \ket{0} = \ket{L_\text{e}^y=1, L_\text{m}^y=1},
\end{equation}
而使用这些关系可以证明,
\begin{equation}
    \begin{aligned}
        {L}^x_\text{e} \ket{0} &= \ket{L_\text{e}^y=1, L_\text{m}^y=-1}, \\
        {L}^x_\text{m} \ket{0} &= \ket{L_\text{e}^y=-1, L_\text{m}^y=1}, \\
        {L}^x_\text{m} {L}^x_\text{e} \ket{0} &= \ket{L_\text{e}^y=-1, L_\text{m}^y=-1}.
    \end{aligned}
\end{equation}
我们发现四重简并和四种基本的任意子正好能够对应上。这是拓扑序的一般特征:基态简并和任意子有对应,基态简并数目就是任意子数目的亏格次方。
我们这里是在亏格(洞的数目)为1的环面上工作,因此基态简并的数目为$4^1=4$种。
如果在亏格为0的球面上,基态简并的数目就是$4^0=1$种。
还有另一种方法也可以推导出这个结果。设亏格为$g$,由欧拉公式
\[
    V - E + F = 2 - 2g,
\]
于是
\[
    E - (V + F - 2) = 2g.
\]
而$V$是$A_s$格点的数目,$F$是$B_p$格点的数目,再减去\eqref{eq:toric-code-pair-condition}造成的两个约束,则$V+F-2$是一个二维表面Toric-code态的自由度个数。
Toric-code模型总的自由度个数为$E$,因此有$2g$个自由度用于标记简并态,由于每个自由度有两个取值,简并度为
\[
    2^{2g} = 4^g.
\]

我们看到,拓扑性质让基态简并出现,而基态简并意味着基态中可以有持续存在的弦——基态可以不是空无一物的!
这个看起来非常神奇——但是完全在预料之中——的性质让Toric-code模型成为一类允许出现弦网凝聚的模型中比较简单的一个。

\section{短程拓扑物态}

上一节所谓的拓扑序指的是长程的。本节是短程的,如SPT等。

\subsection{一维拓扑超导体}

\subsubsection{Kitaev链及其解析解}

以下一维模型称为\concept{Kitaev链}:
\begin{equation}
    {H} = - t \sum_i ({c}_i^\dagger {c}_{i+1} + \text{h.c.}) - \mu \sum_i {c}_i^\dagger {c}_i + \sum_i (\Delta {c}_i {c}_{i+1} + \text{h.c.} ).
    \label{eq:kitaev-chain-hamiltonian}
\end{equation}
\eqref{eq:kitaev-chain-hamiltonian}是一个p波超导模型,这个模型通常是这么来的:一个一维纳米线被放置在一个超导体上,两者的相互作用诱导前者也发生$U(1)$对称性破缺,然后我们使用平均场理论分析问题而引入一个$\Delta$参量。
\eqref{eq:kitaev-chain-hamiltonian}是一个紧束缚模型,
对\eqref{eq:kitaev-chain-hamiltonian}做傅里叶变换,可以得到
\begin{equation}
    {H} = \frac{1}{2} \sum_{\vb*{k}} \underbrace{\pmqty{{c}^\dagger_{\vb*{k}} & {c}_{-\vb*{k}}}}_{{\Psi}^\dagger} \pmqty{\epsilon_{\vb*{k}} - \mu & -2 \ii \Delta^* \sin k \\ 2 \ii \Delta \sin k & - \epsilon_{\vb*{k}} + \mu} \underbrace{\pmqty{{c}_{\vb*{k}} \\ {c}^\dagger_{-\vb*{k}}}}_{{\Psi}},
\end{equation}
然后再做Bogoliubov变换,计算出以下能谱:
\begin{equation}
    E_{\vb*{k}} = \pm \sqrt{(2t \cos k - \mu)^2 + 4 \abs{\Delta}^2 \sin^2 k}.
\end{equation}

\eqref{eq:kitaev-chain-hamiltonian}具有粒子-空穴对称性。% TODO
总之就有一个约束就是设$P$为粒子空穴变换,我们有
\[
    P {\Psi}_{\vb*{k}} P^{-1} = \tau^* {\Psi}_{-\vb*{k}}^*
\]

Kitaev链不存在对称性自发破缺,但能隙可开可闭。当
\begin{equation}
    \mu = \pm 2t
    \label{eq:kitaev-gap-point}
\end{equation}
时,能隙会关闭。除此之外任何参数的变动都只会引起连续的变化。
因此,如果体系发生相变,那么只能是在\eqref{eq:kitaev-gap-point}处发生一个和对称性无关的相变。在化学势很低时,即$\mu$趋于负无穷时,根本就没有电子,因此从$-\infty$到$-2t$的部分肯定是平庸的。
化学势非常高时(大于$2t$时)电子全满,同样是平庸的。
因此有趣的行为集中在$-2t$到$2t$之间。下面会看到,当$\mu$越过\eqref{eq:kitaev-gap-point}这两个点时,会发生一个拓扑相变。

$W = \pm 1$,这是一个定义在立体中的量?

\subsubsection{Kitaev链中的拓扑不变量}

下面定义一个能带的拓扑不变量。

总之,当$\mu$扫过$\mu=-2t$时,我们有
\[
    {c}_i = \frac{1}{2} ({\gamma}_{i\text{A}} + \ii {\gamma}_{i \text{B}}),
\]
容易验证均为
\begin{equation}
    \acomm*{{\gamma}_\alpha}{{\gamma}_\beta^\dagger} = 2 \delta_{\alpha \beta},
\end{equation}

\subsubsection{时间反演对称性保护的拓扑超导}

刚才描述的拓扑超导和对称性没有特别明确的对称性。当然可以说它有粒子-空穴对称性,但是这完全是一个数学上的处理。


\section*{索引}

\subsection{电子结构}

固体中的电子,经过电子-电子库伦相互作用和晶格作用,可能形成以下电子结构:
\begin{itemize}
    \item 产生看起来像是电子的激发:
    \begin{itemize}
        \item 布洛赫电子和Wannier电子,紧束缚模型,各向同性电子气模型(特别的,凝胶模型)
        \item 这些可以归结为费米液体中的准粒子
    \end{itemize}
    \item 低温下的配对:
    \begin{itemize}
        \item 超导
        \item SDW,CDW
    \end{itemize}
    \item 分数化激发
\end{itemize}

\subsection{电动力学相关}

\subsubsection{导电性}

能带,超导

\subsubsection{光学性质}

\begin{itemize}
    \item 晶格衍射,由静态晶格导致
    \item 光子与固体中的非局域玻色子模式耦合:
    \begin{itemize}
        \item 光子和光学声子耦合,见\autoref{sec:huang-eq}。
    \end{itemize}
    \item 等离子激元,由电子导致
\end{itemize}

\subsection{力学和热学性质}

热容:
\begin{itemize}
    \item 晶格振动;
    \item 电子的贡献,在低温下通常没那么多。
\end{itemize}

金属的热传导主要来自电子气。声子气同样会导致传导。

\end{document}