\part{基本原理}

\chapter{连续介质力学的宏观理论框架}

以下在没有明确指定空间维数时,记维数为$d$。记号$\phi(x)$表示“在讨论$\phi$时用坐标系$x$标记空间中各点”,从而$\phi(x')$\emph{不是}简单地将$x'$代入$x \mapsto \phi(x)$得到的场。

\section{运动学}

本节以一种几何的方式,辅以对“连续介质”的概念的一些直觉性的假设,来讨论连续介质的运动学,或者说怎样描述“形变”。
在某种意义上,连续介质的动力学和广义相对论有些相似,因为形变场——一个动力学自由度——本身也可以被用作坐标。这就容易造成一些概念的混淆。

\subsection{形变和物质导数}

连续介质的基本自由度是平滑化的粒子位置。一个连续介质可以用一个空间体积$\Omega$表示。
设连续介质在某个基准时间点(比如说$t=0$)占据空间$\Omega_\text{s}$,经过形变之后占据空间$\Omega_\text{d}$。用$\vb*{r}_\text{s}$表示$\Omega_\text{s}$中的点,用$\vb*{r}_\text{d}$表示$\Omega_\text{d}$中的点,则映射
\[
    f: \Omega_\text{s} \longrightarrow \Omega_\text{d}, \; \vb*{r}_\text{s} \mapsto \vb*{r}_\text{d}
\]
就是\concept{形变映射}。我们通常认为$f$是处处连续的。矢量
\begin{equation}
    \vb*{u} = \vb*{r}_\text{d} - \vb*{r}_\text{s}
\end{equation}
就是\concept{形变}。形变可以定义成$\vb*{r}_\text{d}$的函数,也可以定义成$\vb*{r}_\text{s}$的函数。
$\vb*{r}_\text{s}$是一个不随时间变化的构型;它的作用仅仅是用于扣除一个“背景”。
这两个矢量场都可以构成连续介质中的点的坐标,即它们可以是动力学自由度,也可以是坐标。
为了避免混乱,我们通常用$\vb*{u}$做动力学自由度,而用$\vb*{r}_\text{d}$做坐标,虽然这并没有一定之规。
换句话说,我们通常让$\vb*{u}$出现在被求导的地方,而让$\vb*{r}_\text{d}$出现在求导时的自变量的位置上。

直观地看,$\vb*{r}_\text{s}$相当于微观理论中的粒子编号的一个连续化(因为可以认为在某个特定的时间点,不同编号的粒子非常不可能正好位于完全同样的位置),从而从基于粒子的微观理论做粗粒化时,用$\vb*{r}_\text{s}$是非常方便的。
然而,实验中只能确定形变场中在某个实验室坐标系下某一点的物理量(流速、密度等),而确定不了各个微团一开始在哪里,即实际可以测量的物理量都是$\vb*{r}_\text{d}$的函数;同样,对连续介质施加外力也是施加在一个实验室坐标系中固定不动的一点上。
因此做实际计算时通常应该使用$\vb*{r}_\text{d}$做坐标。

连续介质在某一点的运动速度$\vb*{v}$的形式略微复杂,因为“某一点”本身是一个含糊不清的概念。
我们可以将$\vb*{v}$写成$\vb*{r}_\text{s}$和$t$的函数,即追踪形变开始前的每一个质点在形变之后流动到了哪里,那么显然有
\begin{equation}
    \vb*{v}(\vb*{r}_\text{s}, t) = \dv{\vb*{u}}{t} = \left( \pdv{\vb*{u}}{t} \right)_{\vb*{r}_\text{s}} = \dv{\vb*{r}_\text{d}}{t} = \left( \pdv{\vb*{r}_\text{d}}{t} \right)_{\vb*{r}_\text{s}}.
\end{equation}
第三个等号是因为$\vb*{r}_\text{d}$可以写成$\vb*{r}_\text{s}$和$t$的函数,而考虑到$\vb*{r}_\text{s}$是始终静止的,时间导数就是对$t$的偏导数。
我们也可以将$\vb*{v}$写成$\vb*{r}_\text{d}$和$t$的函数,速度的表达式为
\begin{equation}
    \begin{aligned}
        \vb*{v}(\vb*{r}_\text{d}, t) &= \left( \pdv{\vb*{u}}{t} \right)_{\vb*{r}_\text{s}} = \left( \pdv{\vb*{u}}{t} \right)_{\vb*{r}_\text{d}} + \left( \pdv{\vb*{r}_\text{d}}{t} \right)_{\vb*{r}_\text{s}} \cdot \left( \pdv{\vb*{u}}{\vb*{r}_\text{d}} \right)_t \\
        &= \left( \pdv{\vb*{u}}{t} \right)_{\vb*{r}_\text{d}} + \vb*{v} \cdot \left(\pdv{\vb*{u}}{\vb*{r}_\text{d}}\right)_t.
    \end{aligned}
\end{equation}
上式中多出来了一个非线性项,这是因为某一个空间点上这一时刻和下一时刻的是不一样的,从而会有一个漂移项。

上面的定义可以推广到一般的场变量上。场变量$\varphi$的\concept{物质导数}是$\varphi(\vb*{r}_\text{d}, t)$的时间全导数,于是
\begin{equation}
    \begin{aligned}
        \dv{\varphi}{t} &= \left( \pdv{\varphi}{t} \right)_{\vb*{r}_\text{d}} + \left( \pdv{\vb*{r}_\text{d}}{t} \right)_{\vb*{r}_\text{d}} \cdot \pdv{\varphi}{\vb*{r}_\text{d}} \\
        &= \left( \pdv{\varphi}{t} \right)_{\vb*{r}_\text{d}} + \vb*{v} \cdot \pdv{\varphi}{\vb*{r}_\text{d}}.
    \end{aligned}
\end{equation}

我们通常将用$\vb*{r}_\text{s}$标记所有场量的方法称为\concept{拉格朗日法}而将用$\vb*{r}_\text{d}$标记所有场量的方法称为\concept{欧拉法}。
前者可以看成是追踪每个粒子的位置变化,后者可以看成固定观察实验室坐标系中的固定点。
一旦确定了使用哪种方法,就可以去掉下标d或是s。
例如,在欧拉法中,我们将所有公式中的$\vb*{r}_\text{d}$替换为$\vb*{r}$,从而物质导数就是
\begin{equation}
    \dv{t} = \pdv{t} + \vb*{v} \cdot \grad.
\end{equation}
这样,加速度就是
\begin{equation}
    \dv{\vb*{v}}{t} = \pdv{\vb*{v}}{t} + \vb*{v} \cdot \grad{\vb*{v}}.
\end{equation}

\subsection{几何体的形变}

\begin{equation}
    \dv{t} \dd{V} = \vb*{v} \cdot \dd{\vb*{S}}
\end{equation}

\begin{equation}
    \dv{t} \dd{\vb*{S}} = \dd{\vb*{S}} \div{\vb*{v}} - \dd{\vb*{S}} \cdot \vb*{v} \grad
\end{equation}

\subsection{密度和输运}

设$\rho$是某个随流量的密度,也即,它是微观下某个流体粒子携带的荷的密度的粗粒化。
应当注意虽然$\rho$看起来像是一个标量,在不同坐标系中定义的密度是不能通过简单的坐标变换相转化的,因为我们有
\[
    \rho_1(\vb*{r}_1) \dd[d]{\vb*{r}_1} = \rho_2(\vb*{r}_2) \dd[d]{\vb*{r}_2}.
\]
在坐标系变换涉及时间时,事情稍微复杂一些,因为此时可能存在随流量的产生和消灭。

我们以$\vb*{r}_\text{d}$为坐标,即使用欧拉法。设$f(\vb*{r}_\text{d}, t)$是流量的源或者汇的分布函数,则守恒性就是
\[
    \rho_\text{d}(\vb*{r}_\text{d}) \dd[d]{\vb*{r}_\text{d}} + f_\text{d} (\vb*{r}_\text{d}) \dd{t} \dd[d]{\vb*{r}_\text{d}} = \rho'_\text{d} (\vb*{r}_\text{d}') \dd[d]{\vb*{r}'_\text{d}},
\]
考虑到
\[
    \frac{\dd[d]{\vb*{r}_\text{d}'}}{\dd[d]{\vb*{r}_\text{d}}} = \det( 1 + \pdv{(\vb*{r}_\text{d} + \vb*{v} \dd{t})}{\vb*{r}_\text{d}} ) = 1 + \dd{t} \div{\vb*{v}},
\]
以及
\[
    \rho'(\vb*{r}_\text{d}') - \rho(\vb*{r}_\text{d}) = \dv{\rho}{t} \dd{t},
\]
我们就得到
\begin{equation}
    \dv{\rho}{t} + \rho \div{\vb*{v}} = \pdv{\rho}{t} + \div{(\rho \vb*{v})} = f.
    \label{eq:transportation-eq}
\end{equation}
这就是\concept{连续性方程}。

\section{连续介质的动力学}

\subsection{动能}

本节我们开始讨论连续介质的动力学。在拉格朗日法中动能项对应的作用量就是简单地将多粒子的情况连续化一下,即
\begin{equation}
    S_T = \int \dd{t} \int \dd[d]{\vb*{r}_\text{s}} \frac{1}{2} \rho_\text{s} \left( \pdv{\vb*{u}}{t} \right)_{\vb*{r}_\text{s}}^2.
    \label{eq:kinetic-action}
\end{equation}
这里我们始终有质量守恒成立,即
\begin{equation}
    \rho_\text{d} \dd[d]{\vb*{r}_\text{d}} = \rho_\text{s} \dd[d]{\vb*{r}_\text{s}},
    \label{eq:density-eq}
\end{equation}
或者在欧拉法中就是
\begin{equation}
    \pdv{\rho}{t} + \div{(\rho \vb*{v})} = 0.
    \label{eq:mass-conservation}
\end{equation}

对\eqref{eq:kinetic-action}做变分,由于动力学自由度是$\vb*{u}$,而$\vb*{r}_\text{s}$本身从来不会发生变化,我们可以非常轻松地得到
\begin{equation}
    \begin{aligned}
        \var{S_T} &= \int \dd{t} \int \dd[d]{\vb*{r}_\text{s}}  \rho_\text{s} \left( \pdv{\vb*{u}}{t} \right)_{\vb*{r}_\text{s}} \cdot \var{\left( \pdv{\vb*{u}}{t} \right)_{\vb*{r}_\text{s}}} \\
        &= - \int \dd{t} \int \dd[d]{\vb*{r}_\text{s}}  \rho_\text{s} \left( \pdv[2]{\vb*{u}}{t} \right)_{\vb*{r}_\text{s}} \cdot \var{\vb*{u}}.
    \end{aligned}
\end{equation}
我们没有对$\rho_\text{s}$做任何操作,因为它是给定的、不变的。当然,上式给出的就是牛顿第二定律$F=ma$中的$ma$。

在欧拉法中事情要稍微复杂一些。由动量守恒条件和\eqref{eq:kinetic-action}我们有
\begin{equation}
    S_T = \int \dd{t} \int \dd[d]{\vb*{r}_\text{d}} \frac{1}{2} \rho_\text{d} \vb*{v}^2.
\end{equation}
首先要注意这里的$\rho_\text{d}$不是独立的自由度,因为它可以通过$\rho_\text{s}$和$\vb*{r}_\text{s}$,按照\eqref{eq:density-eq}直接计算出来。
其次,这里最基本的变分是$\var{\vb*{r}_\text{d}}$,$\var{\vb*{v}}$中会同时包含$\var{\vb*{r}_\text{d}}$及其一阶导数。
由于\eqref{eq:density-eq}我们有
\[
    \var{S_T} = \int \dd{t} \int \dd[d]{\vb*{r}_\text{d}} \rho_\text{d} \vb*{v} \cdot \var{\vb*{v}}.
\]
我们注意到
\[
    \var{\vb*{v}} = \left( \pdv{\var{\vb*{r}_\text{d}}}{t} \right)_{\vb*{r}_\text{s}} = \left( \pdv{\var{\vb*{r}_\text{d}}}{t} \right)_{\vb*{r}_\text{d}} + \vb*{v} \cdot \left(\pdv{\var{\vb*{r}_\text{d}}}{\vb*{r}_\text{d}}\right)_t,
\]
做分部积分,就得到
\[
    \var{S_T} = - \int \dd{t} \int \dd[d]{\vb*{r}_\text{d}} \left( \left(\pdv{(\rho_\text{d} \vb*{v})}{t}\right)_{\vb*{r}_\text{d}} \cdot \var{\vb*{r}_\text{d}} + \grad_{\vb*{r}_\text{d}} \cdot (\rho_\text{d} \vb*{v} \vb*{v}) \cdot \var{\vb*{r}_\text{d}} \right),
\]
展开并代入\eqref{eq:mass-conservation},就得到最终的
\begin{equation}
    S_T = - \int \dd{t} \int \dd[d]{\vb*{r}_\text{d}}  \rho_\text{d} \left( \left(\pdv{\vb*{v}}{t}\right)_{\vb*{r}_\text{d}} + \vb*{v} \cdot \grad_{\vb*{r}_\text{d}} \vb*{v} \right) \cdot \var{\vb*{r}_\text{d}}.
\end{equation}
上式中出现了熟悉的输运项。

\subsection{耗散}

\subsection{作为低能有效理论的连续介质力学}

只要一个系统的最低能自由度可以被赋予位移场的意义,并且这些位移场给出的能量的形式和牛顿力学中的动能和势能一样,这个系统就可以被连续介质力学描述。
这种情况下输运项$\vb*{v} \cdot (\div{\vb*{v}})$在一定条件下会产生比较明显的效应。

一个量子液体系统——如费米液体——一般来说是不能用连续介质力学描述的,因为无法定义位移场:为多粒子系统定义位移场需要用到“某个粒子一开始在某个位置,然后位移到了某个位置”这样的说法,实际上隐含地为粒子编号了(“这个粒子”)。但是由全同粒子假设,这是办不到的。
粒子运动范围高度定域的系统可以定义位移场,因为此时各个粒子的运动范围足够作为近似的区分不同粒子的标签或者说编号,但是此时,输运项$\vb*{v} \cdot (\div{\vb*{v}})$基本上毫无用处。
得到粒子运动范围高度定域的系统的位移场之后我们立刻可以求解其波动,得到声子,然后后面所有的工作都可以用声子完成。
另一种情况是温度较高的系统,这种系统中的粒子,无论是费米子还是玻色子,看起来似乎都服从玻尔兹曼分布,和经典的、没有交换对称性或反对称性的粒子的行为完全一致,因此给粒子编号后做计算得到的结果是可靠的。
如果一个多电子系统能够用这种模型描述,那基本上就是等离子体,因为普通的凝聚态系统中电子在不同能级上的分布情况基本上可以用费米球描述,远远没有到服从经典玻尔兹曼分布的程度。

因此对一个电子运动范围非常不局域的凝聚态系统,我们有两个极端:在温度较低时它是“量子液体”,基本的自由度不确定,可以就是重整化之后的电子,即费米液体中的准粒子,也可以是Luttinger液体等;在温度很高时它是等离子体,可以使用完全经典的理论描述,基本的自由度是位移场——实际上是流速场,此时它是通常的、直观意义上的流体。
介于这两者之间的情况一般来说是比较难以分析的。从上述说法可以看到“液体”一词被用于形容粒子间有微弱相互作用的系统的原因:一个这样的系统如果被放在较高的温度下,它的行为和真正的液体是一样的。

严格来说流体力学需要在一个非平衡态统计场论中做。但是如果这个formalism里面能够用拉氏量的概念,下面的所有讨论都是正确的。
实际上,\cite{eft-fluid-rel}中非常狡猾地没有提我们如何量子化流体力学。

大概套路:基本自由度可以是位移场,即comoving coordinates相对于实际的坐标和时间的分布。
先根据不可压缩流体的性质,将“自由场拉氏量”写成一个保度规坐标变换下的不变量。
然后可以求出能量动量张量
\begin{equation}
    T_{\mu \nu} = (\rho + p) u_\mu u_\nu + p \eta_{\mu \nu}
\end{equation}
正好是正确的能动张量。

计算偏离基态的小的振动,可以得到声波和涡旋,后者是退化的,这意味着流体中不存在横波。

那么耗散来自哪儿呢?我们对位移场做一个小的偏移(由于位移场的物理意义,这等价于坐标变换),位移场部分的作用量不变。此外,可以确信,整个理论的作用量——同时包括位移场部分的作用量、不考虑的微观自由度的作用量和耦合项——也是不变的。
理论中那些微观自由度也是连通着流体一起动的,我们据此确定,耦合项随着坐标变换发生的变分和仅仅关于微观自由度的作用量随着坐标变换发生的变分正好抵消。
这意味着耦合项相对于位移场的一阶展开的形式就是微观自由度自己的能动张量乘上位移场的导数。

后面使用文小刚书上那种“积掉热浴”的方法就会发现位移场的格林函数存在有限大的一个虚部,因此存在耗散,并且通过计算格林函数的运动方程可以看出这个耗散就是$-k v$。

这就得到了不可压缩流体的纳维-斯托克斯方程。

但是这种场论语言并没有什么特殊的用处。耗散项是$-kv$这件事本身可以通过与前述分析等价的,基于微分方程形式的对称性分析得到。
微扰论提供不了太多信息因为常见的流体问题都涉及很大的位移场,实际上是非微扰的。

其实即使能够做微扰论,场论语言也没太大意义,因为基于微分方程的计算实际上就能够画费曼图。

\section{材料性能和准热力学平衡}

我们假定连续介质运动过程中几乎总是保持准热力学平衡——这就是说,虽然介质整体显然没有达到热力学平衡,但是介质中每一个宏观小微观大的微团都可以近似认为达到了热力学平衡。
这样就可以使用诸如“单位体积的熵变”、“微团的热力学”等概念推导其行为。

\chapter{微观机制}

