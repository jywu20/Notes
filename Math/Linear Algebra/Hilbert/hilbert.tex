\documentclass[UTF8, a4paper]{ctexart}

\usepackage{geometry}
\usepackage{titling}
\usepackage{titlesec}
\usepackage{paralist}
\usepackage{footnote}
\usepackage{enumerate}
\usepackage{amsmath, amssymb, amsthm}
\usepackage{cite}
\usepackage{graphicx}
\usepackage{subfigure}
\usepackage{physics}
\usepackage{tikz}
\usepackage[colorlinks, linkcolor=black, anchorcolor=black, citecolor=black]{hyperref}

\geometry{left=2.5cm,right=2.5cm,top=2.5cm,bottom=2.5cm}
\titlespacing{\paragraph}{0pt}{1pt}{10pt}[20pt]
\setlength{\droptitle}{-5em}
\preauthor{\vspace{-10pt}\begin{center}}
\postauthor{\par\end{center}}

\newcommand*{\diff}{\mathop{}\!\mathrm{d}}
\newcommand*{\st}{\quad \text{s.t.} \quad}
\newcommand*{\const}{\mathrm{const}}
\newcommand*{\comment}{\paragraph{注记}}
\newcommand*{\warning}{\paragraph{注意}}
\newcommand*{\ii}{\mathrm{i}}
\newcommand*{\ee}{\mathrm{e}}
\newcommand*{\reals}{\mathbb{R}}

%\DeclareMathOperator{\null}{null}
\DeclareMathOperator{\range}{range}
\DeclareMathOperator{\diag}{diag}

\usetikzlibrary{arrows,shapes,positioning}
\usetikzlibrary{arrows.meta}
\usetikzlibrary{decorations.markings}
\tikzstyle arrowstyle=[scale=1]
\tikzstyle directed=[postaction={decorate,decoration={markings,
    mark=at position .5 with {\arrow[arrowstyle]{stealth}}}}]
\tikzstyle ray=[directed, thick]

\newenvironment{bigcase}{\left\{\quad\begin{aligned}}{\end{aligned}\right.}

\begin{document}

\section{希尔伯特空间}

内积、完备性

\section{算子}

一个算子$P$是\textbf{投影算子},当且仅当$P^2 = P$。在直观上这很好理解,因为第二次投影和第一次投影应该是一样的。

投影算子$P$是正交投影算子,当且仅当,$P^\dagger=P$。

设$P_1$和$P_2$是正交投影算子。
若$P_1 P_2 = P_2 P_1$,则$P_1 P_2$是正交投影算子。
若$P_1 P_2 = P_2 P_1 = O$,则$P_1 + P_2$是正交投影算子。


\section{算子的谱}

TODO:连续谱和剩余谱

\[
    A = \sum_i a_i 
\]

广义函数:希尔伯特空间并不同构于它的对偶空间。例如,将一个函数在某一点的值返回的泛函是一个很正常的线性泛函,但是找不到一个正常的函数使
\[
    \int g(x) f(x) \dd x = f(0) 
\]
但是很多时候我们却希望将这样的$g$当成函数来操作(显然,$g$对应着一系列越来越尖锐的$x=0$处的峰的极限),于是引入广义函数的概念。
具体来说,如果能够找到一个线性的对应规则使对某个实体$f$可以定义关于$g$的线性泛函$\langle f, g \rangle$,那么称$f$为广义函数。

把尖括号解释为内积,那么就会发现普通的函数也是广义函数。

两个几乎处处相等的函数对应的广义函数之差为0,也就是说,不同的广义函数对应着不同的普通函数,如果能够找到普通函数对应的话。

广义函数导数的定义
\[
    \langle f', \phi \rangle = \langle f, -\phi' \rangle
\]

\[
    a(x) \delta(x - \xi) = a(\xi) \delta(x - \xi)
\]
从而方程$xf(x)=0$的解为$f(x)=c\delta(x)$,$c$为任意常数。
方程$a(x)f(x)=c$的解为
\[
    f(x) = \frac{c}{a(x)} + C \delta(a(x))
\]

因此广义函数中“除法”运算不能唯一确定。规定不带有$\delta(x)$这一类奇异广义函数的那个除法结果为柯西主值(它的积分就是柯西主值积分,和通常的定义结果一致)

\[
    \lim_{y \to 0+} \frac{1}{x + y \ii} = \text{v.p. } \frac{1}{x} - \ii \pi \delta(x)
\]

\[
    \lim_{y \to 0-} \frac{1}{x + y \ii} = \text{v.p. } \frac{1}{x} + \ii \pi \delta(x)
\]

\end{document}