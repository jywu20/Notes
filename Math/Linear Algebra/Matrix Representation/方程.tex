\hypertarget{ux7ebfux6027ux65b9ux7a0bux7ec4}{%
\subsection{线性方程组}\label{ux7ebfux6027ux65b9ux7a0bux7ec4}}

任何一个n元线性方程组都可以写成矩阵形式: \[
\boldsymbol{A}\boldsymbol{x} = \boldsymbol{b},
\]
其中\(\boldsymbol{A} \in \reals^{m\times n}\)是已知的矩阵,\(\boldsymbol{b} \in \reals^m\)是已知的向量,而\(\boldsymbol{x} \in \reals^n\)是需要计算的一个向量。容易看出\(\boldsymbol{A}\)实际上是\(\mathcal{L}(\mathbb{R}^n, \mathbb{R}^m)\)中某个映射的一个矩阵表示。

显然,这个方程或是没有解,或是有唯一解,或是有无穷多个解。这是因为,只要方程有两个不同的解,就可以通过线性组合构造出更多的解。

\hypertarget{ux89e3ux4e0eux7ebfux6027ux7a7aux95f4ux7684ux5173ux7cfb}{%
\subsection{解与线性空间的关系}\label{ux89e3ux4e0eux7ebfux6027ux7a7aux95f4ux7684ux5173ux7cfb}}

\hypertarget{ux5217ux7a7aux95f4}{%
\subsubsection{列空间}\label{ux5217ux7a7aux95f4}}

调节不同的\(\boldsymbol{x}\),就可以得到不同的\(\boldsymbol{A} \boldsymbol{x}\)。最后得到的全体\(\boldsymbol{A} \boldsymbol{x}\)实际上形成了一个向量空间。注意到
\[
\boldsymbol{A} \boldsymbol{x} = \sum_i x_i \boldsymbol{A} \text{的第$i$列}
\]
因此,我们可以把\(\boldsymbol{A}\)的列向量看成从原点出发的不同方向,将\(x_i\)看成沿着第\(i\)个列向量走出了多远。因此,全体\(\boldsymbol{A} \boldsymbol{x}\)就是\(\boldsymbol{A}\)的\textbf{列空间},实际上也就是\(\boldsymbol{A}\)对应的线性映射的值域。\(\boldsymbol{A}\boldsymbol{x} = \boldsymbol{b}\)
有解意味着\(\boldsymbol{b}\)在\(\boldsymbol{A}\)的各个列向量张成的\textbf{生成子空间}当中,也就是说,\(\boldsymbol{b}\)在\(\boldsymbol{A}\)的\textbf{列空间}当中。

从线性相关有关的结论可以看出:给定\(\boldsymbol{b}\), -
\(\boldsymbol{A} \boldsymbol{x} = \boldsymbol{0}\)只有零解,当且仅当\(\boldsymbol{A}\)的列向量线性无关,当且仅当\(\boldsymbol{A} \boldsymbol{x} = \boldsymbol{b}\)只有唯一解
-
\(\boldsymbol{A} \boldsymbol{x} = \boldsymbol{b}\)有解,当且仅当,\(\boldsymbol{b}\)可以使用\(\boldsymbol{A}\)的列向量线性表示

推论:\(\boldsymbol{B}\)的列向量都可以被\(\boldsymbol{A}\)的列向量线性表示(即:前者的列向量张成的空间是后者的列向量张成的空间的子空间),当且仅当,存在\(\boldsymbol{D}\)使\(\boldsymbol{B} = \boldsymbol{A} \boldsymbol{D}\)。

一个矩阵的列向量线性无关当且仅当这个矩阵有非零子式。

\hypertarget{ux65b9ux9635}{%
\subsubsection{方阵}\label{ux65b9ux9635}}

设\(\boldsymbol{A}\)为方阵,\(\boldsymbol{b}\)与\(\boldsymbol{A}\)同阶,则下面的说法等价:
- \(\boldsymbol{A}\)可逆 -
\(\boldsymbol{A} \boldsymbol{x} = \boldsymbol{b}\)有唯一解 -
\(\boldsymbol{A} \boldsymbol{x} = \boldsymbol{0}\)只有零解 -
\(\boldsymbol{A}\)的列向量线性无关 - \(\det \boldsymbol{A} \neq 0\)

没有这样的情况:对有些\(\boldsymbol{b}\),\(\boldsymbol{A} \boldsymbol{x} = \boldsymbol{b}\)有唯一解,对另外一些,没有解或者有无穷多个解。

反之,如果\(\boldsymbol{A}\)不可逆,那么一定有一些\(\boldsymbol{b}\)使\(\boldsymbol{A} \boldsymbol{x} = \boldsymbol{b}\)无解,另有一些\(\boldsymbol{b}\)使\(\boldsymbol{A} \boldsymbol{x} = \boldsymbol{b}\)有无穷多个解。
