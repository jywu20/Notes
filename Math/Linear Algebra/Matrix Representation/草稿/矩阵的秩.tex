\hypertarget{ux77e9ux9635ux7684ux79e9}{%
\section{矩阵的秩}\label{ux77e9ux9635ux7684ux79e9}}

定义矩阵的\textbf{行秩}为矩阵的各行组成的向量组的秩,矩阵的\textbf{列秩}为矩阵的各列组成的向量组的秩。

矩阵的列秩为$$r$$当且仅当它有一个$$r$$阶子式不为零,而大于$$r$$阶的子式全部都是零。

矩阵的行秩等于列秩。则定义矩阵的\textbf{秩}为$$\mathrm{rank} \boldsymbol{A}$$为它的行秩,即列秩。秩等于行数的矩阵为\textbf{行满秩矩阵},秩等于列数的矩阵为\textbf{列满秩矩阵}。秩等于阶数的方阵为\textbf{满秩矩阵}。

有以下结论: 
-矩阵的秩必定同时小于等于行数和列数(从而,行比列多的矩阵不可能行满秩,列比行多的矩阵不可能列满秩)
- 转置不改变矩阵的秩 -
$$\mathrm{rank} (\boldsymbol{A} + \boldsymbol{B}) \leq \mathrm{rank} \boldsymbol{A} + \mathrm{rank} \boldsymbol{B}$$
-
$$\mathrm{rank} (\boldsymbol{A} \boldsymbol{B}) \leq \min (\mathrm{rank} \boldsymbol{A}, \mathrm{rank} \boldsymbol{B})$$
-
左乘或者右乘一个可逆矩阵不改变矩阵的秩。从而,对矩阵做初等变换不改变矩阵的秩。

如果一个矩阵没有一行全部是零,并且每一行第一个非零元素所在的列数随着行数递增,那么这个矩阵一定是行满秩的。因此,我们可以将一个矩阵初等变换成行阶梯矩阵(\textbf{行阶梯矩阵}指的是不全为零的行全部在全部为零的行上方,并且每一行第一个非零元素所在的列数随着行数递增的矩阵),然后非零的行的个数就是矩阵的秩。(问题:这样一定是可行的吗?是不是有不能够转化成行阶梯矩阵的情况?)

最后,$$\boldsymbol{A}\in \mathbb{R}^{m\times n}$$的秩为$$r$$,当且仅当存在$$m$$阶可逆矩阵$$\boldsymbol{P}$$和$$n$$阶可逆矩阵$$\boldsymbol{Q}$$使
\[
\boldsymbol{P} \boldsymbol{A} \boldsymbol{Q} = \left[
    \begin{matrix}
        \boldsymbol{I}_r & \boldsymbol{O} \\
        \boldsymbol{O} & \boldsymbol{O}
    \end{matrix}
\right]
\]

$$\boldsymbol{A} \in \mathbb{R}^{m\times n}, \boldsymbol{B} \in \mathbb{R}^{n \times p}$$,则
\[
\mathrm{rank} (\boldsymbol{A} \boldsymbol{B}) \geq \mathrm{rank}\boldsymbol{A} + \mathrm{rank}\boldsymbol{B} - n
\]

\hypertarget{ux4f7fux7528ux79e9ux6765ux5206ux6790ux7ebfux6027ux65b9ux7a0bux89e3ux7684ux6570ux76ee}{%
\subsection{使用秩来分析线性方程解的数目}\label{ux4f7fux7528ux79e9ux6765ux5206ux6790ux7ebfux6027ux65b9ux7a0bux89e3ux7684ux6570ux76ee}}

设$$\boldsymbol{A} \in \mathbb{R}^{m \times n}$$。

\hypertarget{ux9f50ux6b21ux65b9ux7a0bux89e3ux7684ux7ed3ux6784}{%
\subsubsection{齐次方程解的结构}\label{ux9f50ux6b21ux65b9ux7a0bux89e3ux7684ux7ed3ux6784}}

先前已经分析过,方程$$\boldsymbol{A} \boldsymbol{x} = \boldsymbol{0}$$仅有零解,当且仅当,$$\boldsymbol{A}$$的列向量线性无关。我们还可以加入一个充要条件:$$\boldsymbol{A}$$是列满秩的。

因此,在$$\boldsymbol{A}$$的行数少于列数的时候,$$\boldsymbol{A} \boldsymbol{x} = \boldsymbol{0}$$必定有非零解。

下面我们考虑有非零解的情况。这些解连同零向量构成了$$\boldsymbol{A}$$的零空间。我们称一组向量是$$\boldsymbol{A}$$的\textbf{基础解系},当且仅当,任何一个$$\boldsymbol{A} \boldsymbol{x} = \boldsymbol{0}$$的解都可以使用它们线性表示。使用线性映射零空间和值域维数的关系可以得到:基础解系中共有$$n-\mathrm{rank}\boldsymbol{A}$$个向量。

\hypertarget{ux6c42ux89e3ux9f50ux6b21ux65b9ux7a0b}{%
\subsubsection{求解齐次方程}\label{ux6c42ux89e3ux9f50ux6b21ux65b9ux7a0b}}

下面的问题是怎样把这些向量(注意它们不是唯一的)计算出来。$$\boldsymbol{A}$$的行空间的维数是$$\mathrm{rank}\boldsymbol{A}$$,于是我们取$$\boldsymbol{A}$$的行空间的一组基(比如说一个极大无关组,但也可以是极大无关组做了某个线性变换的结果),以它们为行得到一个新矩阵$$\boldsymbol{A}_1$$,则方程$$\boldsymbol{A} \boldsymbol{x} = \boldsymbol{0}$$等价于$$\boldsymbol{A}_1 \boldsymbol{x} = \boldsymbol{0}$$。

同样,$$\mathrm{rank}\boldsymbol{A}_1=\mathrm{rank}\boldsymbol{A}$$。这就意味着,$$\boldsymbol{A}_1$$有一个$$\mathrm{rank}\boldsymbol{A}$$阶子式非零。不失一般性地我们设它的前$$\mathrm{rank}\boldsymbol{A}$$列组成的矩阵的行列式非零(否则,只需要交换有限行就可以达到这个结果)。设$$\boldsymbol{A}_{11}$$是$$\boldsymbol{A}_1$$前$$\mathrm{rank}\boldsymbol{A}$$列组成的矩阵(一个方阵),$$\boldsymbol{A}_{12}$$是剩下来的部分,则方程$$\boldsymbol{A}_1 \boldsymbol{x} = \boldsymbol{0}$$等价于
\[
\boldsymbol{A}_{11} \boldsymbol{x}_1 + \boldsymbol{A}_{12} \boldsymbol{x}_2 = \boldsymbol{0}, \boldsymbol{x} = \boldsymbol{x}_1 \oplus \boldsymbol{x}_2.
\]

也就是 \[
\boldsymbol{x} = \left[
    \begin{matrix}
        - \boldsymbol{A}_{11}^{-1} \boldsymbol{A}_{12} \\
        \boldsymbol{I}_{n-\mathrm{rank}\boldsymbol{A}}
    \end{matrix}    
\right] \boldsymbol{x}_2
\]
注意到$$\boldsymbol{x}_2$$的选取完全没有受到限制。于是我们可以随便取$$n-\mathrm{rank}\boldsymbol{A}$$维向量代入上式。比较好的选择是取某个分量为1其它分量全是0的向量,一共得到$$n-\mathrm{rank}\boldsymbol{A}$$独立解,正好组成一个基础解系。

也可以使用初等行变换的方法做到这一点。容易看出,对于任意一个列不满秩的矩阵$$\boldsymbol{A}$$,我们都可以交换它的有限列之后,将得到的矩阵初等变换为下面的形式
\[
\left[
    \begin{matrix}
        \boldsymbol{I}_{\mathrm{rank}\boldsymbol{A}} & \boldsymbol{B} \\
        \boldsymbol{O} & \boldsymbol{O}
    \end{matrix}    
\right]
\]
并且,$$-\boldsymbol{B}$$的第$$i$$列和总长为$$n-\mathrm{rank}\boldsymbol{A}$$,第$$i$$个分量为1其它分量为零的向量拼接后再按照先前交换列的方式反过来交换分量后得到的向量是方程$$\boldsymbol{A} \boldsymbol{x} = \boldsymbol{0}$$的解。使用这个方式可以求出一整组基础解系。

\hypertarget{ux975eux9f50ux6b21ux65b9ux7a0bux7684ux89e3ux7684ux7ed3ux6784}{%
\subsubsection{非齐次方程的解的结构}\label{ux975eux9f50ux6b21ux65b9ux7a0bux7684ux89e3ux7684ux7ed3ux6784}}

称$$(\boldsymbol{A} \; \boldsymbol{b})$$为非齐次方程$$\boldsymbol{A} \boldsymbol{x} = \boldsymbol{b}$$的\textbf{增广矩阵}。

非齐次方程$$\boldsymbol{A} \boldsymbol{x} = \boldsymbol{b}$$有解,当且仅当,$$\mathrm{rank}\boldsymbol{A} = \mathrm{rank}(\boldsymbol{A} \; \boldsymbol{b})$$。

如果$$\mathrm{rank}\boldsymbol{A} = n$$,那么非齐次方程就有唯一解;否则非齐次方程有无穷多个解,彼此之间相差$$\boldsymbol{A} \boldsymbol{x} = \boldsymbol{0}$$的一个解。

\hypertarget{ux6c42ux89e3ux975eux9f50ux6b21ux65b9ux7a0b}{%
\subsubsection{求解非齐次方程}\label{ux6c42ux89e3ux975eux9f50ux6b21ux65b9ux7a0b}}

因此只要求解出了非齐次方程的一个特解就求出了所有通解。容易看出,按照之前的定义,设$$\boldsymbol{b}_r$$代表$$\boldsymbol{b}$$的前$$\mathrm{rank}\boldsymbol{A}$$个元素组成的向量,则$$\boldsymbol{A}_{11}^{-1} \boldsymbol{b}_r$$再拼接上一个零向量就构成了一个特解。

同样也可以使用初等变换求解。增广矩阵在交换有效次列之后(不要动最后一列,也就是$$\boldsymbol{b}$$),经过初等行变换得到
\[
\left[
    \begin{matrix}
        \boldsymbol{I}_{\mathrm{rank}\boldsymbol{A}} & \boldsymbol{B} & \boldsymbol{b}' \\
        \boldsymbol{O} & \boldsymbol{O} & *
    \end{matrix}    
\right]
\]

如果$$*$$处不全零,方程无解。否则,$$-\boldsymbol{B}$$的第$$i$$列和总长为$$n-\mathrm{rank}\boldsymbol{A}$$,第$$i$$个分量为1其它分量为零的向量拼接后再按照先前交换列的方式反过来交换分量后得到的向量是$$\boldsymbol{A} \boldsymbol{x} = \boldsymbol{0}$$的基础解,而最右边一列则是方程$$\boldsymbol{A} \boldsymbol{x} = \boldsymbol{b}$$的一个特解。
