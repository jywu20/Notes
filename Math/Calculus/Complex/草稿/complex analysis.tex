\hypertarget{ux590dux53d8ux51fdux6570ux7684ux79efux5206}{%
\section{复变函数的积分}\label{ux590dux53d8ux51fdux6570ux7684ux79efux5206}}

\hypertarget{ux67efux897fux5b9aux7406ux4e0eux67efux897fux516cux5f0fux7b49}{%
\subsection{柯西定理与柯西公式等}\label{ux67efux897fux5b9aux7406ux4e0eux67efux897fux516cux5f0fux7b49}}

\textbf{柯西定理}
设\(f\)在闭单连通区域\(D\)上解析,则\(f\)围\(D\)的环路积分为零。从而,设\(f\)在闭复连通区域的所有境界线上取正方向的环路积分为零。

推论: -
\(f\)沿闭复连通区域的外境界线逆时针的积分等于\(f\)沿各内境界线逆时针方向的积分之和;
- 闭区域内的积分,固定开头结尾连续变形积分路径不会改变积分值。

微积分基本定理在复变函数的情况下仍然成立。不过要注意有些函数的不定积分有可能是多值的,这个时候必须要取一个单值分支。

在计算具体的积分的时候有下面的定理:

\textbf{有界闭区域的柯西公式}
设\(f\)在闭区域\(D\)上解析,且\(l\)为\(D\)取正方向的全部境界线(在\(D\)复连通的时候包含好多条路径),\(n=0, 1, 2, \ldots\),那么
\[
f^{(n)}(z) = \frac{n!}{2\pi \mathrm{i}} \oint_l \frac{f(\zeta)}{(\zeta - z)^{n+1}} \mathrm{d}\zeta, \; z \in D
\]

推论: -
区域上的复变函数的一阶导数的存在性就保证了任何高阶导数的存在性,并且对解析函数而言,边界上的值确定了内部的值
-
若函数在闭区域上解析,那么它的模在边界上取最大值(在区域内部取最大值的充要条件是函数为常函数)
- \textbf{刘维尔定理}
如果\(f\)在全平面上解析且\(z \rightarrow \infty\)时有界,那么\(f\)是常数。因此,在闭平面上解析的复变函数只有常函数。例如,在全平面上解析的函数在无穷远处一定发散。
- 设\(|f|\)在圆周\(|z-\zeta|=R\)上的上界为\(M\),则 \[
  |f^{(n)}(z)| \leq \frac{n! M}{R^n}
  \]

在无界区域上也有类似的结论。

\textbf{无界区域的柯西公式}
设\(f\)在闭曲线\(l\)上以及其外的无界区域上解析,\(z\)是\(l\)外一点,那么
\[
f(z) = \frac{1}{2\pi \mathrm{i}} \oint_{l^-} \frac{f(\zeta)}{\zeta - z} \mathrm{d}\zeta + f(\infty  )
\] 且若\(n=1, 2, \ldots\),有 \[
f^{(n)}(z) = \frac{n!}{2 \pi \mathrm{i}} \oint_{l^-} \frac{f(\zeta)}{(\zeta - z)^{n+1}}\mathrm{d}\zeta
\] 注意积分路径都是顺时针的。

\hypertarget{ux7559ux6570}{%
\subsection{留数}\label{ux7559ux6570}}

设\(b\)是\(f\)的一个孤立奇点,则\(f\)在\(b\)处的留数\(\mathrm{res}f (b)\)定义为\(f\)在\(b\)处的洛朗级数的-1次幂项的系数。则我们有

\textbf{留数定理}
设\(f\)在闭曲线\(l\)包围的区域内除了有限个奇点\(b_k, k=1, 2, \ldots m\)以外是单值解析的,且在\(l\)上也是解析的,那么
\[
\oint_l f(z) \mathrm{d}z = 2 \pi \mathrm{i} \sum_{k=1}^m \mathrm{res} f(b_k).
\]

那么下面就需要讨论计算留数的方法。可以证明

\begin{itemize}
\tightlist
\item
  可去奇点的留数为零。
\item
  若\(b\)为\(m\)阶极点,那么 \[
  \mathrm{res} f(b) = \frac{1}{(m-1)!} \lim_{z \to b} \frac{\mathrm{d}^{m-1}}{\mathrm{d} z^{m-1}} \left( (z-b)^m f(z) \right)
  \]
\item
  本性奇点的留数一般需要直接计算洛朗级数
\end{itemize}

今后,常使用 \[
\sum_D \mathrm{res} f(z)
\] 表示区域\(D\)内\(f\)的所有奇点的留数之和。

\hypertarget{ux4e0eux5b9eux51fdux6570ux5b9aux79efux5206ux7684ux8054ux7cfb}{%
\subsection{与实函数定积分的联系}\label{ux4e0eux5b9eux51fdux6570ux5b9aux79efux5206ux7684ux8054ux7cfb}}

\hypertarget{piux4e0aux7684ux4e09ux89d2ux51fdux6570ux79efux5206}{%
\subsubsection{\texorpdfstring{\((0, 2\pi)\)上的三角函数积分}{(0, 2\textbackslash{}pi)上的三角函数积分}}\label{piux4e0aux7684ux4e09ux89d2ux51fdux6570ux79efux5206}}

考虑积分 \[
\int_0^{2\pi} R(\cos x, \sin x) \mathrm{d}x
\] 设\(z=e^{\mathrm{i}x}\),则 \[
R(\cos x, \sin x) \mathrm{d}x = R \left(\frac{z + z^{-1}}{2}, \frac{z - z^{-1}}{2 \mathrm{i}}\right) \frac{\mathrm{d} z}{\mathrm{i}z} \equiv F(z) \mathrm{d}z,
\] 于是 \[
\int_0^{2\pi} R(\cos x, \sin x) \mathrm{d}x = 2 \pi \mathrm{i} \sum_{|z|=1} \mathrm{res} F(z)
\] 其中 \[
F(z) = \frac{1}{\mathrm{i}z} R \left(\frac{z + z^{-1}}{2}, \frac{z - z^{-1}}{2 \mathrm{i}}\right).
\]

\hypertarget{ux6cbfux6574ux6761ux5b9eux6570ux8f74ux7684ux5b9aux79efux5206}{%
\subsubsection{沿整条实数轴的定积分}\label{ux6cbfux6574ux6761ux5b9eux6570ux8f74ux7684ux5b9aux79efux5206}}

\[
\mathrm{P} \int_{-\infty}^{+\infty} f(x) \mathrm{d}x = 2 \pi  \mathrm{i} \sum_{\text{upper half-plane}} \mathrm{res} f(z) + \pi \mathrm{i} \sum_{\text{real axis}} \mathrm{res} f(z)
\]
使用这个公式,对在上半平面处处解析且在\(z \to \infty\)时一致趋于0的函数\(f\),以及\(\alpha \in \reals\),有\textbf{希尔伯特变换}
\[
\begin{aligned}
    \mathrm{Re} f(\alpha) &= \frac{1}{\pi} \mathrm{P} \int_{-\infty}^{+\infty} \frac{\mathrm{Im} f(x)}{x - \alpha} \mathrm{d} x \\
    \mathrm{Im} f(\alpha) &= - \frac{1}{\pi} \mathrm{P} \int_{-\infty}^{+\infty} \frac{\mathrm{Re} f(x)}{x - \alpha} \mathrm{d} x 
\end{aligned}
\]

\hypertarget{ux65e0ux7a77ux5c0fux5206ux6bcdux865aux90e8}{%
\subsubsection{无穷小分母虚部}\label{ux65e0ux7a77ux5c0fux5206ux6bcdux865aux90e8}}

设\(f\)在实轴上没有奇点。考虑积分 \[
\int_a^b \frac{f(x)}{x-x_0 - \mathrm{i}\epsilon} \mathrm{d}x
\]
其中\(\epsilon\)是一个很小的正数。需要计算的是这个积分在\(\epsilon\to 0\)时的极限。我们可以让积分路径在\(x_0 - \mathrm{i}\epsilon\)附近沿着一个小半圆转一下,然后让小半圆的半径趋于零。接着我们交换两个极限的顺序,这样我们要计算的极限就是:极点位于实轴上,而积分路径转过一个半径趋于零的小半圆。从而可得
\[
\int_a^b \frac{f(x)}{x-x_0 - \mathrm{i}\epsilon} \mathrm{d}x = \mathrm{P} \int_a^b \frac{f(x)}{x-x_0} \mathrm{d}x + \mathrm{i} \pi f(x_0)
\] 这个结论可以推广为 \[
\int_a^b \frac{f(x)}{x-x_0 \mp \mathrm{i}\epsilon} \mathrm{d}x = \mathrm{P} \int_a^b \frac{f(x)}{x-x_0} \mathrm{d}x \pm \mathrm{i} \pi f(x_0)
\] 考虑到\(f\)的任意性,上式又可以写成 \[
\frac{1}{x - x_0 \mp \mathrm{i}\epsilon} = \mathrm{P} \frac{1}{x - x_0} \pm \mathrm{i} \pi \delta(x - x_0)
\]

\hypertarget{ux6d89ux53caux591aux503cux51fdux6570ux7684ux5b9aux79efux5206}{%
\subsubsection{涉及多值函数的定积分}\label{ux6d89ux53caux591aux503cux51fdux6570ux7684ux5b9aux79efux5206}}

设\(Q\)在全平面上除了有限个奇点以外都是解析的,且当\(z \to 0\)或\(z \to \infty\)时\(z^\alpha Q(x)\)一致趋于零,则
\[
\int_0^{+\infty} x^{\alpha - 1} Q(x)
\] TODO

\hypertarget{ux65e0ux7a77ux7ea7ux6570}{%
\section{无穷级数}\label{ux65e0ux7a77ux7ea7ux6570}}

在复变函数的情况下,柯西收敛准则也是成立的。

无穷级数基本性质: - 绝对收敛的级数收敛,并且其和与各项排列顺序无关 -
绝对收敛的级数的乘积也绝对收敛 -
一个函数级数的各项在一个区域内连续,而级数在这个区域内一致收敛,则级数的和在该区域内连续
- 各项均连续的一致收敛级数可以逐项积分 -
若\(|u_k(z)| \leq a_k\)而\(\sum_k a_k\)在\(D\)上收敛,那么\(\sum_k u_k(z)\)在\(D\)上一致收敛且绝对收敛

\hypertarget{ux6cf0ux52d2ux7ea7ux6570ux6536ux655bux5706}{%
\subsection{泰勒级数、收敛圆}\label{ux6cf0ux52d2ux7ea7ux6570ux6536ux655bux5706}}

\hypertarget{ux6d1bux6717ux7ea7ux6570}{%
\subsection{洛朗级数}\label{ux6d1bux6717ux7ea7ux6570}}

\textbf{洛朗定理}
设\(f\)在环形区域\(R_2 < |z-b| < R_1\)内单值解析,则\(f\)可以在这个区域内展开为绝对收敛且一致收敛的级数
\[
f(z) = \sum_{k=-\infty}^\infty a_k (z-b)^k,
\] 其中 \[
a_k = \frac{1}{2\pi \mathrm{i}} \oint \frac{f(\zeta)}{(\zeta - b)^{k+1}}\mathrm{d}\zeta
\] 因此这个展开是唯一的。

如果洛朗级数中有负幂,那么\(f\)在\(|z-b|\leq R_2\)上一定有奇点,否则洛朗级数就是泰勒级数了,也就不应该出现负幂。但是这个奇点却不一定就是\(b\),这是因为\(f\)并不一定在\(|z-b|\leq R_2\)内解析,所以也不能够将洛朗级数延拓到\(b\)附近。当然,如果我们能够将\(R_2\)取为零,那么\(b\)就是\(f\)的一个奇点。

\hypertarget{ux5b64ux7acbux5947ux70b9ux4e0eux6d1bux6717ux7ea7ux6570ux7684ux8054ux7cfb}{%
\subsection{孤立奇点与洛朗级数的联系}\label{ux5b64ux7acbux5947ux70b9ux4e0eux6d1bux6717ux7ea7ux6570ux7684ux8054ux7cfb}}

设\(b\)是\(f\)的一个非无穷远的孤立奇点,我们在这个奇点附近对它做洛朗展开,
\[
f(z) = \sum_{k=-\infty}^\infty a_k (z - b)^k
\]
称这个级数的正幂部分(\(k\geq 0\)的部分)为解析部分,复幂部分(\(k<0\)的部分)为主要部分。(两个级数在\(b\)以外都是解析的!)那么,奇点\(b\)的性质、\(f\)在\(z\rightarrow b\)时的性质和洛朗级数的性质有着很重要的联系。
\#\#\# 可去奇点
\(b\)是一个可去奇点,当且仅当,级数的主要部分不存在,即\(k \geq 0\)的时候才可能有\(a_k \neq 0\),当且仅当,\(\lim_{z\rightarrow b} f(z)\)为有限复数。

我们规定 \[
F(z) =
\begin{cases}
    f(z) & \text{when $z \neq b$} \\
    \lim_{z\rightarrow b} f(z) & \text{when $z=b$}
\end{cases}
\] 则\(F\)在\(b\)处可导,此时 \[
\sum_{k=-\infty}^\infty a_k (z - b)^k = \sum_{k=0}^\infty a_k (z - b)^k
\] 就是\(F\)在\(b\)处的泰勒级数。

\hypertarget{ux6781ux70b9}{%
\subsubsection{极点}\label{ux6781ux70b9}}

\(b\)是一个\(m\)阶极点,当且仅当,只有\(k \geq -m\)时才可能有\(a_k \neq 0\),当且仅当,\(\lim_{z\rightarrow b}(z-b)^m f(z)\)为有限复数。

事实上,\(b\)是一个\(m\)阶极点,当且仅当 \[
\lim_{z\rightarrow b} (z-b)^n f(z) = 
\begin{cases}
    \infty & \text{when $n<m$} \\
    a_{-m} \neq 0 & \text{when $n=m$} \\
    0 & \text{when $n>m$}
\end{cases}
\] 因此\(b\)是一个极点当且仅当\(\lim_{z\rightarrow b}f(z) = \infty\)。

\hypertarget{ux672cux6027ux5947ux70b9}{%
\subsubsection{本性奇点}\label{ux672cux6027ux5947ux70b9}}

\(b\)是一个本性奇点,当且仅当级数有无穷个负幂项(也即,使\(a_k\)非零的\(k\)无下界),当且仅当\(\lim_{z\rightarrow b} f(z)\)不存在。

注意:在考虑了无穷远点的情况下,\(\lim_{z\rightarrow b} f(z)\)不存在不包括\(f(z)\)趋于\(\infty\)的情况。\(\lim_{z\rightarrow b} f(z)\)不存在通常意味着从不同方向接近\(b\)会得到不同的极限值,或者得到正无穷和负无穷等。

\hypertarget{ux89e3ux6790ux5ef6ux62d3}{%
\section{解析延拓}\label{ux89e3ux6790ux5ef6ux62d3}}

\textbf{解析函数的唯一性定理}
设\(f_1\)和\(f_2\)是区域\(D\)内的两个解析函数,\(D\)内的点列\(\{z_k\}\)有至少一个极限点,且在整个\(\{z_k\}\)上\(f_1(z)=f_2(z)\),则在整个\(D\)上都有\(f_1(z)=f_2(z)\)。

推论: -
若两个函数在某一区域中某一点的邻域或者某一曲线段上相等,那么它们在整个区域上都相等;
-
若两个函数在某一区域上解析,并且在某一点上的函数值和各阶导数都相同,那么它们在整个区域上都相等

设函数\(f\)在闭平面上有一系列奇点\(z_i\),我们在\(z_0\)处对\(f\)做泰勒展开,得到的级数的收敛半径就是\(z_0\)到离它最近的奇点的距离。考虑到任何一个不是常函数的复变函数都必定有不可去的奇点,我们得出结论:不可能使用单独一个泰勒级数覆盖整个闭平面上的非常函数的复变函数。(覆盖全平面还是可以做到的,比如指数函数和三角函数)

\hypertarget{ux79efux5206ux53d8ux6362}{%
\section{积分变换}\label{ux79efux5206ux53d8ux6362}}

\hypertarget{ux5085ux91ccux53f6ux53d8ux6362}{%
\subsection{傅里叶变换}\label{ux5085ux91ccux53f6ux53d8ux6362}}
