\section{复变函数的基本概念}

设$f$是定义在复平面上的一个区域上的函数,其实部为$u$,虚部为$v$,自变量$z = x + y \mathrm{i}$,其中$x,y$为实变量。又使用极坐标表示法,记$z = \rho e^{\mathrm{i}\phi}$。取$z_0$为闭复平面上一个点。

可以按照一般的方式定义$f$是不是连续,定义$f$的导数、极限并且判断它的可导条件。可以证明$f$在某一点的极限正是$u$的极限加上$v\mathrm{i}$的极限,$f$连续当且仅当$u,v$连续。

$f$在无穷远点的定义可以使用链式法则: \[
\frac{\mathrm{d} f(z)}{\mathrm{d} z} \bigg |_{\infty} = \lim_{\zeta \rightarrow 0} \left( - \zeta^2 \frac{\mathrm{d}}{\mathrm{d} \zeta} f \left( \frac{1}{\zeta} \right) \right)
\]

\hypertarget{c-rux6761ux4ef6ux4e0eux53efux5bfcux6027}{%
\subsection{C-R条件与可导性}\label{c-rux6761ux4ef6ux4e0eux53efux5bfcux6027}}

我们说$f$在$z_0$满足C-R条件,当且仅当在这一点有 \[
\frac{\partial u}{\partial x} = \frac{\partial v}{\partial y}, \; \frac{\partial u}{\partial y} + \frac{\partial v}{\partial x} = 0.
\] 也即, \[
\frac{\partial u}{\partial \rho} = \frac{1}{\rho} \frac{\partial v}{\partial \phi}, \; \frac{\partial v}{\partial \rho} = - \frac{1}{\rho} \frac{\partial u}{\partial \phi}.
\]

如果$f$在$z_0$可导,那么它必定在这一点满足C-R条件,但是\textbf{反之不然}。

不过,如果在$z_0$处$\partial_x u, \partial_y u, \partial_x v, \partial_y v$存在且连续,且C-R条件成立,则$f$在这一点可导。

因此,$f$在$z_0$可导,当且仅当,在$z_0$处$\partial_x u, \partial_y u, \partial_x v, \partial_y v$存在且连续,且C-R条件成立。

\hypertarget{ux89e3ux6790ux51fdux6570}{%
\subsection{解析函数}\label{ux89e3ux6790ux51fdux6570}}

$f$在集合$D$内解析,当且仅当$f$在$D$内处处可导。$f$的定义域内不可导的点称为奇点。

如果一个函数在某个区域内是解析的,那么它一定在这个区域内连续。并且,如果一个函数在一个区域内是连续的,那么它在这个区域内解析的充要条件是C-R条件处处成立。因此,一个函数在一个区域内解析的充要条件是,它在这个区域内连续且C-R条件处处成立。

若$f$在$D$内解析,则$D$内$u,v$的等值线正交,并且$u, v$都是调和函数。事实上,$u, v$是共轭调和函数,当且仅当
\[
\frac{\partial u}{\partial x} = \frac{\partial v}{\partial y}, \; \frac{\partial u}{\partial y} + \frac{\partial v}{\partial x} = 0,
\] 因此,$u, v$是共轭调和函数,当且仅当$f$是解析函数。

\emph{在得出上面的结论的时候我们用到了$\partial_x, \partial_y$可以交换的性质。这要求$u, v$具有比较好的性质。}

因此,如果$f$是解析函数,那么我们有: \[
u = \int_L (\frac{\partial v}{\partial y} \mathrm{d}x - \frac{\partial v}{\partial x} \mathrm{d}y) + C_1, \; 
v = \int_L (-\frac{\partial u}{\partial y}\mathrm{d}x + \frac{\partial u}{\partial x} \mathrm{d}y) + C_2,
\]
其中积分路径要让积分有定义,$C_1, C_2$在不同的子区域内保持不变,但未必在整个区域上取固定的值。(例如,它们也许在上半空间取0,在下半空间取$2\pi$)在$f$保持解析的区域是单连通的时候,$C_1, C_2$一定是真正的常数。

\hypertarget{ux591aux503cux51fdux6570}{%
\subsection{多值函数}\label{ux591aux503cux51fdux6570}}

\hypertarget{ux5b9aux4e49}{%
\subsubsection{定义}\label{ux5b9aux4e49}}

设$f, g$是某个区域$D$上的解析函数。对于$z$,我们考虑满足下面条件的$w$:
\[
f(w) = g(z)
\]
不失一般性地要求$f(0)=0$。形式地,我们可以使用反函数解出$w$,但显然$f$有可能将不同的复数映射到同一个结果上。因此,对同一个$z$,我们会有不同的$w$。我们称从$z$到所有可能的$w$有一个\textbf{多值函数}。多值函数不是一个从$\mathbb{C}$到$\mathbb{C}$的函数。

\hypertarget{ux652fux70b9}{%
\subsubsection{支点}\label{ux652fux70b9}}

我们首先考虑一个简单的情况:$g$是一个多项式函数。在这种情况下我们总是可以把$g$做因式分解,得到
\[
g(z) = a (z - z_1)^{i_1} \ldots (z - z_n)^{i_n},
\] 当然不失一般性地可以将$a$移到左边并重新定义$f$,从而 \[
f(w) = g(z) = (z - z_1)^{i_1} \ldots (z - z_n)^{i_n}.
\]
我们现在讨论某个$z_k$。在它的一个很小的邻域内取一点$z=z_k + \rho e^{\mathrm{i}\phi}$。我们有
\[
(z_k + \rho e^{\mathrm{i}\phi} - z_1)^{i_1} \cdots \rho^{i_k} e^{\mathrm{i}i_k \phi} \cdots (z_k + \rho e^{\mathrm{i}\phi} - z_n)^{i_n} = f(w)
\] 使用小量近似,令$z \rightarrow z_k$,有 \[
(z_k - z)^{i_1} \cdots (z_k - z_n)^{i_n} \rho^{i_k} e^{\mathrm{i} i_k \phi} \approx f(w)
\]

现在我们让$z$绕某个自己不交叉的闭合路径$L$(直观来看就是一个不规则的圆圈)运动,即取$z=z(t)$,$t$为实数,$z(t)$是一个连续的周期函数;我们同时要求$w$沿着某条轨道\textbf{连续地}运动,即$w=w(t)$,且要让$f(w)=g(z)$始终成立。显然,对同一个$t$,$w(t)$给出了$w$在$z=z(t)$时的\textbf{一个}(未必是全部)取值。使用这样的方式我们实际上使用了一个局域上的$g$的反函数,它总是存在的。

在$z$的运动轨迹的内部没有包含$z_k$的时候,$z$绕$L$一周,运动开始和结束时$f(w)$的取值并不会有变化,因为$\phi$随着$t$或是先变大后减小,或是先减小后变大。但是当$L$的内部有$z_k$时,在运动开始和结束的时候$\phi$相差了$2\pi$,因此在运动开始、停止的点$z_0 = r_0 e^{\mathrm{i}\phi_0}=z(t_0)=z(t_1)$处同时有
\[
g(w(t_0)) = g(w(t_1)) = f(r_0 e^{\mathrm{i}\phi_0})
\]
如果不管怎么指定$w(t)$都有$w(t_0)=w(t_1)$,那么$g$就是可逆的,此时$w$就可以写成$z$的一个单值函数。但如果$w(t_1) \neq w(t_0)$,那么$g$是不可逆的,此时$w$要写成$z$的一个多值函数。

上面是以$w(t_0)$为一开始$w$的取值,转一圈以后的结果。我们当然可以一直这样转下去,不断增大$t$,然后得到一连串$w(t_n)$。我们看到,如果把$w(t)$在$z(t)=z_0$时的值视为$z_0$处$w$的取值,那么$w$是有可能随着转圈数目增大而变化的。

这种``使转圈以后$z_0$处的$w$会变化''的$z_k$就称为\textbf{支点}。

还有一种情况是,取$z \rightarrow \infty$,此时 \[
f(w) \approx \rho^{i_1 + \cdots + i_n} e^{\mathrm{i}(i_1 + \cdots + i_n)\phi}
\]
我们同样,让$z$绕着无穷远点转圈,也就是让$\rho$取很大,而让$\phi$转一整圈。设$z_0$为运动开始和结束的点,同样可以得到
\[
g(w(t_0)) = g(w(t_1)) = f(r_0 e^{\mathrm{i}\phi_0})
\]
如果转圈之后$z_0$处的$w$会变化,那么我们说$\infty$也是一个支点。

容易看出,如果一个点是支点,那么任意一条非常接近它的轨道上任意取一个$z_k$,$z_k$都对应多个$w$。事实上我们可以说,支点是满足若$w$存在则必定唯一的所有点。

复平面上的同伦曲线理论(又或者别的不知道什么东西)保证了 -
只要我们选择的轨道$L$(也就是$z(t)$)和$w(t)$性质足够良好,就能够保证不同轨道给出的$w(t_0), w(t_1), \ldots$最终都能够覆盖对应$z_0=z(t_0)=z(t_1)=\ldots$的全部$w$,也就是说,多值函数在这个点的所有取值都可以通过``转圈''获得(要同时考虑顺时针转和逆时针转);而显然,通过``转圈''产生的所有$w(t_i)$都是多值函数在这个点的取值;
-
如果某一组$z(t), w(t)$下有$z(t_n)=z(t_0)$,也就是转了$n$圈以后$w$的值恢复,那么所有$z(t), w(t)$下$w$都是转$n$圈以后恢复的;如果某一组$z(t), w(t)$下$w$的值永远不会恢复,那么所有$z(t), w(t)$下都是如此。

因此,若在$z_0$处绕$n$圈之后$w$的值恢复,则称$z_0$是一个$n-1$\textbf{阶支点}。所有整数阶支点统称\textbf{代数支点};永远恢复不了的支点称为\textbf{超越支点}。

在实际判断支点的时候,比较方便的一种做法是取$t=\phi$,并作出一个可行的$w(\phi)$(只需要一个就够了!)使$f(w(\phi))=g(z_0 + \rho e^{\mathrm{i}\phi})$。有可能是支点的点只有诸$z_k$和$\infty$。在讨论诸$z_k$的时候我们取一个充分小并且保持不变的$\rho$,尝试构造一个$w(\phi)$使
\[
(z_k - z)^{i_1} \cdots (z_k - z)^{i_{k-1}} (z_k - z)^{i_{k+1}} \cdots (z_k - z)^{i_k} \rho^{i_k} e^{\mathrm{i}i_k \phi} \approx f(w(\phi))
\]
然后看满足$w(\phi+2\pi n) = w(\phi)$的最小$n$,$n-1$就是$z_k$的阶数($n=1$那这就不是支点)。
在讨论$\infty$的时候则构造一个$w(\phi)$使 \[
\rho^{i_1 + \cdots + i_n} e^{\mathrm{i}(i_1 + \cdots + i_n)\phi} \approx f(w(\phi))
\]
然后看满足$w(\phi+2\pi n) = w(\phi)$的最小$n$,$n-1$就是$z_k$的阶数($n=1$那这就不是支点)。

上面的这一套说法应该也能够推广到$g$是无穷级数的情况。(大概吧\ldots{}\ldots{})

\hypertarget{ux5173ux4e8eux8bb0ux53f7ux7684ux6ce8ux8bb0}{%
\subsubsection{关于记号的注记}\label{ux5173ux4e8eux8bb0ux53f7ux7684ux6ce8ux8bb0}}

为了避免冗长,我们形式地引入记号$f^{-1}$(虽然并不能求反函数)并规定
\[
P(f^{-1}(z_1), \ldots, f^{-1}(z_k)) \Leftrightarrow \exists w_1, \ldots w_k (z_1 = f(w_1) \land \ldots z_k = f(w_k) \land P(w_1, \ldots, w_k))
\] 因此,从复平面上也成立的 \[
e^{a+b} = e^a e^b
\] 就可以直接写出 \[
\ln z \equiv \exp^{-1} z, \; \ln (z_1 z_2) = \ln z_1 + \ln z_2,
\] 虽然对数函数并不是单值的。

这种记号是非常好用的。比如说在判断支点的时候就要处理下面形式的方程: \[
f(w(\phi);a) = \ldots
\] 然后我们就可以形式上写出 \[
w(\phi) = f^{-1}(\ldots;a) 
\] 使用一些像$\ln(z_1z_2)$展开的公式就可以得到 \[
w(\phi) = f^{-1}(\ldots;a) = F(h^{-1}(\phi), f^{-1}(a))
\]
之类的公式。$h^{-1}$也许有定义,也许没有,但是因为我们只需要\textbf{找到一个}符合条件的$w(\phi)$,我们只需要\textbf{随便找一个能称为$h^{-1}$的函数}就已经把$w(\phi)$构造出来了。举例:
\[
w(\phi) = \ln (a\rho e^{\mathrm{i}\phi}) = \ln \rho + \ln a + \phi \mathrm{i}
\]
当然这没有考虑到多值性,但是因为我们只需要\textbf{一个}$w(\phi)$,我们实际上已经构造出我们需要的东西了。上式的严格形式大概是
\[
\exist z \in \mathrm{C} (a = e^z \land w(\phi) = \ln \rho + z + \phi \mathrm{i})
\]

\hypertarget{ux5272ux7ebf}{%
\subsubsection{割线}\label{ux5272ux7ebf}}

TODO:多个支点的情况 \#\#\# 单值分支的选取
为了方便起见将所有的割线都选为直线。设$z_0$是一个支点,则某个单值分支内,$z_0$的一个邻域内的自变量的相角将受到一定的限制:
\[
z = z_0 + \rho e^{\mathrm{i} \phi}, \quad \phi \in [\phi_0 + 2\pi k, \phi_0 + 2 \pi (k+1)) \backslash \{ \phi_1, \ldots, \phi_n \}
\]
被去掉的$\phi_1, \ldots, \phi_n$是为了避免$z$撞上某一条割线。只允许$\phi$转动$2\pi$是因为如果它转动超过了$2\pi$那就一定会撞上一条割线。

我们总是可以求出一个 \[
h_0: (\rho, \phi) \longrightarrow w, 
\]
使$f(h_0(\rho, \phi)) = g(z_0 + \rho e^{\mathrm{i}\phi})$。只要有\textbf{一个}这样的$h_0$,就可以获得\textbf{所有的}单值分支,因为只需要求出$z$满足
\[
z = z_0 + \rho e^{\mathrm{i} \phi}, \quad \phi \in [\phi_0 + 2\pi k, \phi_0 + 2 \pi (k+1)) \backslash \{ \phi_1, \ldots, \phi_n \}
\]
的形式就可以将$\rho, \phi$代入$h_0$中来得到对应的$w$,从而求出$z_0$的邻域内的单值分支值,然后延拓到整个复平面。(但是为什么是所有的单值分支呢?)

在$h_0$已经给定了的情况下,只需要给定一个点$z_1$的值就可以确定一个单值分支。首先合理安排$z_0, \phi_0, \phi_1, \ldots, \phi_n$使$z_1$不在任何一条割线上并且在$z_0$的某个合适的邻域内。然后从下面的方程求解出$k$即可
\[
f(h_0(|z_1|, \arg z_1 + 2\pi k)) = g(z_1)
\]

\hypertarget{ux6d89ux53caux591aux503cux51fdux6570ux7684ux79efux5206}{%
\subsubsection{涉及多值函数的积分}\label{ux6d89ux53caux591aux503cux51fdux6570ux7684ux79efux5206}}

要计算积分 \[
\int f(z) \mathrm{d}z
\]
如果有一个多值函数在它的某一支上面的导数就是$f$,那么$f$的一切原函数都是多值的。此时如果积分路径环绕支点就有可能产生比较复杂的情况。比较好的一种做法还是选取一个$z=z(t)$使得随着实数参量$t$的\textbf{连续变化}$z$可以正好扫过整个积分路径,同时选取一个在整条积分路径上都有定义的$f$的原函数的单值化$w(t)$,也就是要求
\[
\mathrm{d} w(t) = f(z(t)) \mathrm{d} z(t)
\] 那么我们就有 \[
\int_{z(t_1)}^{z(t_2)} f(z) \mathrm{d}z = w(t) \big |_{t_1}^{t_2}
\]
为了获得$w$,只需要取$\int f(z) \mathrm{d}z$的一个单值分支就可以。

举例: \[
\int \frac{1}{z} \mathrm{d}z = \ln z + C
\] 但是对数函数是多值的,那么实际做积分的时候就取 \[
w(t) = \ln \rho (t) + \phi (t) \mathrm{i}, \\
\int_{z_1}^{z_2} \frac{1}{z} \mathrm{d}z = (\ln \rho (t) + \phi (t) \mathrm{i}) \big |_{t_1}^{t_2}
\]

在大多数情况下,考虑到积分路径可以固定首尾而连续变形这一事实,我们总是可以将积分路径化成一个圆形,这时直接取$t=\phi$最为方便。

例如,计算一个环绕原点的$1/z$的环路积分: \[
\begin{aligned}
    \oint \frac{1}{z} \mathrm{d}z &= \oint_{|z|=1} \frac{1}{z} \mathrm{d}z \\
    &= (\ln 1 + \phi \mathrm{i}) \big |_0^{2\pi} = 2 \pi \mathrm{i}.
\end{aligned}
\]
我们的计算给出了正确的结果。注意,我们要求参数$t$------这里是$\phi$------做连续的变动,因此积分上下限是0和$2\pi$,虽然它们代表同一个点。正是这个连续性的要求让我们得以获得正确的结果。
