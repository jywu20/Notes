\documentclass[hyperref, a4paper]{article}

\usepackage{geometry}
\usepackage{titling}
\usepackage{titlesec}
% No longer needed, since we will use enumitem package
% \usepackage{paralist}
\usepackage{enumitem}
\usepackage{footnote}
\usepackage[colorinlistoftodos]{todonotes}
\usepackage{amsmath, amssymb, amsthm}
\usepackage{mathtools}
\usepackage{bbm}
\usepackage{graphicx}
\usepackage{subcaption}
\usepackage{soulutf8}
\usepackage{physics}
\usepackage{tensor}
\usepackage{siunitx}
\usepackage[version=4]{mhchem}
\usepackage{tikz}
\usepackage{xcolor}
\usepackage{listings}
\usepackage{autobreak}
\usepackage[ruled, vlined, linesnumbered]{algorithm2e}
\usepackage{nameref,zref-xr}
\zxrsetup{toltxlabel}
\usepackage[backend=bibtex]{biblatex}
\addbibresource{nonequilibrium.bib}
\usepackage[colorlinks,unicode]{hyperref} % , linkcolor=black, anchorcolor=black, citecolor=black, urlcolor=black, filecolor=black
\usepackage[most]{tcolorbox}
\usepackage{prettyref}

% Page style
\geometry{left=3.18cm,right=3.18cm,top=2.54cm,bottom=2.54cm}
\titlespacing{\paragraph}{0pt}{1pt}{10pt}[20pt]
\setlength{\droptitle}{-5em}

% More compact lists 
\setlist[itemize]{
    itemindent=17pt, 
    leftmargin=1pt,
    listparindent=\parindent,
    parsep=0pt,
}

% Math operators
\DeclareMathOperator{\timeorder}{\mathcal{T}}
\DeclareMathOperator{\diag}{diag}
\DeclareMathOperator{\legpoly}{P}
\DeclareMathOperator{\primevalue}{P}
\DeclareMathOperator{\sgn}{sgn}
\DeclareMathOperator{\res}{Res}
\newcommand*{\ii}{\mathrm{i}}
\newcommand*{\ee}{\mathrm{e}}
\newcommand*{\const}{\mathrm{const}}
\newcommand*{\suchthat}{\quad \text{s.t.} \quad}
\newcommand*{\argmin}{\arg\min}
\newcommand*{\argmax}{\arg\max}
\newcommand*{\normalorder}[1]{: #1 :}
\newcommand*{\pair}[1]{\langle #1 \rangle}
\newcommand*{\fd}[1]{\mathcal{D} #1}
\DeclareMathOperator{\bigO}{\mathcal{O}}

% TikZ setting
\usetikzlibrary{arrows,shapes,positioning}
\usetikzlibrary{arrows.meta}
\usetikzlibrary{decorations.markings}
\tikzstyle arrowstyle=[scale=1]
\tikzstyle directed=[postaction={decorate,decoration={markings,
    mark=at position .5 with {\arrow[arrowstyle]{stealth}}}}]
\tikzstyle ray=[directed, thick]
\tikzstyle dot=[anchor=base,fill,circle,inner sep=1pt]

% Algorithm setting
% Julia-style code
\SetKwIF{If}{ElseIf}{Else}{if}{}{elseif}{else}{end}
\SetKwFor{For}{for}{}{end}
\SetKwFor{While}{while}{}{end}
\SetKwProg{Function}{function}{}{end}
\SetArgSty{textnormal}

\newcommand*{\concept}[1]{{\textbf{#1}}}

% Embedded codes
\lstset{basicstyle=\ttfamily,
  showstringspaces=false,
  commentstyle=\color{gray},
  keywordstyle=\color{blue}
}

% Reference formatting
\newcommand*{\citesec}[1]{\S~{#1}}
\newcommand*{\citechap}[1]{chap.~{#1}}
\newcommand*{\citefig}[1]{Fig.~{#1}}
\newcommand*{\citetable}[1]{Table~{#1}}
\newcommand*{\citepage}[1]{pp.~{#1}}
\newrefformat{fig}{Fig.~\ref{#1}}
\newcommand*{\term}[1]{\textit{#1}}

% Color boxes
\tcbuselibrary{skins, breakable, theorems}

\newtcbtheorem{infobox}{Box}{
    enhanced,
    boxrule=0pt,
    colback=blue!5,
    colframe=blue!5,
    coltitle=blue!50,
    borderline west={4pt}{0pt}{blue!65},
    sharp corners,
    fonttitle=\bfseries, 
    breakable,
    before upper={\parindent15pt\noindent}}{box}
\newtcbtheorem[use counter from=infobox]{theorybox}{Box}{
    enhanced,
    boxrule=0pt,
    colback=orange!5, 
    colframe=orange!5, 
    coltitle=orange!50,
    borderline west={4pt}{0pt}{orange!65},
    sharp corners,
    fonttitle=\bfseries, 
    breakable,
    before upper={\parindent15pt\noindent}}{box}
\newtcbtheorem[use counter from=infobox]{learnbox}{Box}{
    enhanced,
    boxrule=0pt,
    colback=green!5,
    colframe=green!5,
    coltitle=green!50,
    borderline west={4pt}{0pt}{green!65},
    sharp corners,
    fonttitle=\bfseries, 
    breakable,
    before upper={\parindent15pt\noindent}}{box}


\newenvironment{shelldisplay}{\begin{lstlisting}}{\end{lstlisting}}

\newcommand*{\kB}{k_{\text{B}}}
\newcommand*{\muB}{\mu_{\text{B}}}
\newcommand*{\efermi}{E_{\text{F}}}
\newcommand*{\pfermi}{p_{\text{F}}}
\newcommand*{\vfermi}{v_{\text{F}}}
\newcommand*{\sA}{\text{A}}
\newcommand*{\sB}{\text{B}}
\newcommand*{\Tc}{T_{\text{c}}}
\newcommand*{\hethree}{$^3$He}
\newcommand*{\hefour}{$^4$He}
\newcommand{\epsr}{\epsilon_{\text{r}}}
\newcommand{\chie}{\chi_{\text{e}}}
\newcommand{\cf}{c_{\text{F}}}
\newcommand{\fn}{F_{\text{N}}}
\newcommand{\ff}{F_{\text{f}}}

\title{Brownian motions}
\author{Jinyuan Wu}

\begin{document}

\maketitle

\section{The framework}

A stochastic process can be described by a stochastic differential equation
and hence we have something like (below we mostly use the \emph{Ito scheme})
\begin{equation}
    X(t) = \int_{0}^{t} \dd{X}, \quad 
    \dd{X} = \beta(X, t) \dd{t} + \gamma(X, t) \dd{W},
\end{equation}
where $W$ is a ``noise'' term.

It's quite common for $W$ to be \concept{Brownian},
and thus $\dd{W}$ at different times are independent to each other,
making to process Markovian,
and 
\begin{equation}
    \dd{W}(t) \sim N(0, \dd{t}).
\end{equation}
This means 
\begin{equation}
    \expval{\dd{W}(t) \cdot \dd{W}(t')} = \begin{cases}
        \dd{t}, &\quad t = t',\\
        0,      &\quad t \neq t'.
    \end{cases}
    \label{eq:w-sq}
\end{equation}
The first line leads to the famous distinction between Ito calculus and Stratonovich calculus,
as it brings the second order derivative into $\dd{f}$.
For the same reason, the \concept{Ito's lemma} 
\begin{equation}
    \dd{f} = \underbrace{\left(
        \pdv{f}{t} + \beta \pdv{f}{x} + \frac{1}{2} \gamma^2 \pdv[2]{f}{x} 
    \right)}_{\text{drift rate}} \dd{t} 
    + \gamma \pdv{f}{x} \dd{W}.
\end{equation}
It can be proven by Taylor expansion of $f$ and keep the $\dd{X}^2$ term as well,
which, because of \eqref{eq:w-sq}, has a term proportional to $\dd{t}$.

\begin{theorybox}{Types of noises}{noise-classification}
    More generally, we can have correlation between the noise at different times (making the stochastic process no longer Markovian),
    and we can have higher order correlations.
    All the correlation functions can be captured within a path integral formalism,
    using the standard technique in physics.
    We're only concerned of the simplest type of noise here.
\end{theorybox}

\section{Martingales}

A \concept{martingale} is a stochastic process in which 
\begin{equation}
    \mathbb{E}(X_{t} | X_0, X_1, \ldots, X_{t-\dd{t}}) = X_{t - \dd{t}}.
\end{equation}
This is equivalent to the drift term $\beta(X, t)$ term being zero.
$\dd{W}$ itself is a martingale.

We can prove that 
\begin{equation}
    Y(t) = W(t)^2 - t 
\end{equation}
is martingale as well. In the same way the Ito's lemma is proven:
\begin{equation}
    \dd{Y} = 2 W \dd{W} - \dd{t} + \underbrace{\dd{W}^2}_{\dd{t}},
\end{equation}
meaning that $\dd{Y}$ clearly misses the drift term,
meaning that $Y$ is a martingale.

\section{Black-Scholes equation}



\end{document}
