\documentclass[hyperref, a4paper]{article}

\usepackage{geometry}
\usepackage{titling}
\usepackage{titlesec}
% No longer needed, since we will use enumitem package
% \usepackage{paralist}
\usepackage{enumitem}
\usepackage{footnote}
\usepackage[colorinlistoftodos]{todonotes}
\usepackage{amsmath, amssymb, amsthm}
\usepackage{mathtools}
\usepackage{bbm}
\usepackage{graphicx}
\usepackage{subcaption}
\usepackage{soulutf8}
\usepackage{physics}
\usepackage{tensor}
\usepackage{siunitx}
\usepackage[version=4]{mhchem}
\usepackage{tikz}
\usepackage{xcolor}
\usepackage{listings}
\usepackage{autobreak}
\usepackage[ruled, vlined, linesnumbered]{algorithm2e}
\usepackage{nameref,zref-xr}
\zxrsetup{toltxlabel}
\usepackage[backend=bibtex]{biblatex}
\addbibresource{nonequilibrium.bib}
\usepackage[colorlinks,unicode]{hyperref} % , linkcolor=black, anchorcolor=black, citecolor=black, urlcolor=black, filecolor=black
\usepackage[most]{tcolorbox}
\usepackage{prettyref}

% Page style
\geometry{left=3.18cm,right=3.18cm,top=2.54cm,bottom=2.54cm}
\titlespacing{\paragraph}{0pt}{1pt}{10pt}[20pt]
\setlength{\droptitle}{-5em}

% More compact lists 
\setlist[itemize]{
    itemindent=17pt, 
    leftmargin=1pt,
    listparindent=\parindent,
    parsep=0pt,
}

% Math operators
\DeclareMathOperator{\timeorder}{\mathcal{T}}
\DeclareMathOperator{\diag}{diag}
\DeclareMathOperator{\legpoly}{P}
\DeclareMathOperator{\primevalue}{P}
\DeclareMathOperator{\sgn}{sgn}
\DeclareMathOperator{\res}{Res}
\newcommand*{\ii}{\mathrm{i}}
\newcommand*{\ee}{\mathrm{e}}
\newcommand*{\const}{\mathrm{const}}
\newcommand*{\suchthat}{\quad \text{s.t.} \quad}
\newcommand*{\argmin}{\arg\min}
\newcommand*{\argmax}{\arg\max}
\newcommand*{\normalorder}[1]{: #1 :}
\newcommand*{\pair}[1]{\langle #1 \rangle}
\newcommand*{\fd}[1]{\mathcal{D} #1}
\DeclareMathOperator{\bigO}{\mathcal{O}}

% TikZ setting
\usetikzlibrary{arrows,shapes,positioning}
\usetikzlibrary{arrows.meta}
\usetikzlibrary{decorations.markings}
\tikzstyle arrowstyle=[scale=1]
\tikzstyle directed=[postaction={decorate,decoration={markings,
    mark=at position .5 with {\arrow[arrowstyle]{stealth}}}}]
\tikzstyle ray=[directed, thick]
\tikzstyle dot=[anchor=base,fill,circle,inner sep=1pt]

% Algorithm setting
% Julia-style code
\SetKwIF{If}{ElseIf}{Else}{if}{}{elseif}{else}{end}
\SetKwFor{For}{for}{}{end}
\SetKwFor{While}{while}{}{end}
\SetKwProg{Function}{function}{}{end}
\SetArgSty{textnormal}

\newcommand*{\concept}[1]{{\textbf{#1}}}

% Embedded codes
\lstset{basicstyle=\ttfamily,
  showstringspaces=false,
  commentstyle=\color{gray},
  keywordstyle=\color{blue}
}

% Reference formatting
\newcommand*{\citesec}[1]{\S~{#1}}
\newcommand*{\citechap}[1]{chap.~{#1}}
\newcommand*{\citefig}[1]{Fig.~{#1}}
\newcommand*{\citetable}[1]{Table~{#1}}
\newcommand*{\citepage}[1]{pp.~{#1}}
\newrefformat{fig}{Fig.~\ref{#1}}
\newcommand*{\term}[1]{\textit{#1}}

% Color boxes
\tcbuselibrary{skins, breakable, theorems}

\newtcbtheorem{infobox}{Box}{
    enhanced,
    boxrule=0pt,
    colback=blue!5,
    colframe=blue!5,
    coltitle=blue!50,
    borderline west={4pt}{0pt}{blue!65},
    sharp corners,
    fonttitle=\bfseries, 
    breakable,
    before upper={\parindent15pt\noindent}}{box}
\newtcbtheorem[use counter from=infobox]{theorybox}{Box}{
    enhanced,
    boxrule=0pt,
    colback=orange!5, 
    colframe=orange!5, 
    coltitle=orange!50,
    borderline west={4pt}{0pt}{orange!65},
    sharp corners,
    fonttitle=\bfseries, 
    breakable,
    before upper={\parindent15pt\noindent}}{box}
\newtcbtheorem[use counter from=infobox]{learnbox}{Box}{
    enhanced,
    boxrule=0pt,
    colback=green!5,
    colframe=green!5,
    coltitle=green!50,
    borderline west={4pt}{0pt}{green!65},
    sharp corners,
    fonttitle=\bfseries, 
    breakable,
    before upper={\parindent15pt\noindent}}{box}


\newenvironment{shelldisplay}{\begin{lstlisting}}{\end{lstlisting}}

\newcommand*{\kB}{k_{\text{B}}}
\newcommand*{\muB}{\mu_{\text{B}}}
\newcommand*{\efermi}{E_{\text{F}}}
\newcommand*{\pfermi}{p_{\text{F}}}
\newcommand*{\vfermi}{v_{\text{F}}}
\newcommand*{\sA}{\text{A}}
\newcommand*{\sB}{\text{B}}
\newcommand*{\Tc}{T_{\text{c}}}
\newcommand*{\hethree}{$^3$He}
\newcommand*{\hefour}{$^4$He}
\newcommand{\epsr}{\epsilon_{\text{r}}}
\newcommand{\chie}{\chi_{\text{e}}}
\newcommand{\cf}{c_{\text{F}}}
\newcommand{\fn}{F_{\text{N}}}
\newcommand{\ff}{F_{\text{f}}}

\title{Discrete Markovian processes}
\author{Jinyuan Wu}

\begin{document}

\maketitle

\section{Notations}

Suppose $X_t$ is the state of the system at time $t$.
The transition probability 
\begin{equation}
    p(X_{t+1} = j | X_{t} = i) \eqqcolon p(i \to j) = p_{ij}.
\end{equation}
Note that the order of $i$ and $j$ is the opposite of the order in physics.

An \concept{absorbing state} is a state which you can't leave.

\section{Absorption probability}

Given an absorbing state $s$,
we denote the probability for the system to start from $i$ and end at absorbing state $s$ as $a_i$.
By definition,
\begin{equation}
    a_i = \lim_{t \to \infty} p(X_t = s | X_0 = i).
\end{equation}
There is actually no need for the initial time to be $t = 0$.
That's to say, we also have 
\begin{equation}
    a_i = \lim_{t \to \infty} p(X_t = s | X_1 = i).
\end{equation}

Obviously, by taking the $t \to \infty$ limit of the 
\[
    \begin{aligned}
        p(X_{t} = s | X_0 = i) &=  \sum_i p(X_t = s | X_1 = j) p(X_1 = j | X_0 = i) \\
        &= \sum_i p(X_t = s | X_1 = j)  p_{ij},
    \end{aligned}
\]
we get 
\begin{equation}
    a_i = \sum_{j} p_{ij} a_j.
\end{equation}
This, together with the conditions that for all absorbing states,
\begin{equation}
    a_i = \begin{cases}
        1, & i = s, \\
        0, & i \neq s,
    \end{cases}
\end{equation}
gives us a system of equations that can be solved to find all $a_i$'s.
Note that $p(X_t = s | X_0 = i)$ is \emph{not} the probability for the system to go to $s$ using exactly $t$ steps:
instead, it's the sum of the probabilities for the system to arrive at $s$ in 0, 1, 2, \dots, $s$ steps.

\section{Expected absorption time}

Now suppose $\mu_i$ is the expected total time it takes to reach absorption.
Following the same method in the last section, we have 
\[
    \begin{aligned}
        \mu_i &= \sum_{n \geq 1} n p(\text{it takes $n$ steps from $i$ to $s$}) \\
        &= \sum_j \sum_{n \geq 1} (n - 1 + 1) p(\text{it takes $n-1$ steps from $j$ to $s$}) p_{ij} \\
        &= \sum_{j} p_{ij} \sum_{n \geq 0} (n + 1) p(\text{it takes $n$ steps from $j$ to $s$}) \\
        &= \sum_j p_{ij} \mu_j + \sum_j p_{ij} \sum_{n \geq 0}  p(\text{it takes $n$ steps from $j$ to $s$}) ,
    \end{aligned}
\]
and we note that 
\[
    \sum_{j} p_{ij} \sum_{n \geq 0}  p(\text{it takes $n$ steps from $j$ to $s$}) a_j = \sum_{j} p_{ij} a_j = a_i,
\]
and thus 
\begin{equation}
    \mu_i = \sum_j p_{ij} \mu_j + a_i.
\end{equation}

We can also sum over all final states, which leads to 
\begin{equation}
    \mu_i = \sum_{j} p_{ij} \mu_j + 1.
\end{equation}

\end{document}