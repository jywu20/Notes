\documentclass[hyperref, UTF8, a4paper]{ctexart}

\usepackage{geometry}
\usepackage{titling}
\usepackage{titlesec}
\usepackage{paralist}
\usepackage{footnote}
\usepackage{enumerate}
\usepackage{amsmath, amssymb, amsthm}
\usepackage{bbm}
\usepackage{cite}
\usepackage{graphicx}
\usepackage{subfigure}
\usepackage{physics}
\usepackage{tikz}
\usepackage{autobreak}
\usepackage[ruled, vlined, linesnumbered, noend]{algorithm2e}
\usepackage[colorlinks, linkcolor=black, anchorcolor=black, citecolor=black]{hyperref}
\usepackage{prettyref}

% Page style
\geometry{left=3.18cm,right=3.18cm,top=2.54cm,bottom=2.54cm}
\titlespacing{\paragraph}{0pt}{1pt}{10pt}[20pt]
\setlength{\droptitle}{-5em}
\preauthor{\vspace{-10pt}\begin{center}}
\postauthor{\par\end{center}}

% Math operators
\DeclareMathOperator{\timeorder}{T}
\DeclareMathOperator{\diag}{diag}
\DeclareMathOperator{\legpoly}{P}
\DeclareMathOperator{\primevalue}{P}
\DeclareMathOperator{\sgn}{sgn}
\newcommand*{\ii}{\mathrm{i}}
\newcommand*{\ee}{\mathrm{e}}
\newcommand*{\const}{\mathrm{const}}
\newcommand*{\comment}{\paragraph{注记}}
\newcommand*{\suchthat}{\quad \text{s.t.} \quad}
\newcommand*{\argmin}{\arg\min}
\newcommand*{\argmax}{\arg\max}
\newcommand*{\normalorder}[1]{: #1 :}
\newcommand*{\pair}[1]{\langle #1 \rangle}
\newcommand*{\fd}[1]{\mathcal{D} #1}
\DeclareMathOperator{\bigO}{\mathcal{O}}

% prettyref setting
\newrefformat{sec}{第\ref{#1}节}
\newrefformat{note}{注\ref{#1}}
\newrefformat{fig}{图\ref{#1}}
\newrefformat{alg}{算法\ref{#1}}
\renewcommand{\autoref}{\prettyref}

% TikZ setting
\usetikzlibrary{arrows,shapes,positioning}
\usetikzlibrary{arrows.meta}
\usetikzlibrary{decorations.markings}
\tikzstyle arrowstyle=[scale=1]
\tikzstyle directed=[postaction={decorate,decoration={markings,
    mark=at position .5 with {\arrow[arrowstyle]{stealth}}}}]
\tikzstyle ray=[directed, thick]
\tikzstyle dot=[anchor=base,fill,circle,inner sep=1pt]

% Algorithm setting
\renewcommand{\algorithmcfname}{算法}
% Python-style code
\SetKwIF{If}{ElseIf}{Else}{if}{:}{elif:}{else:}{}
\SetKwFor{For}{for}{:}{}
\SetKwFor{While}{while}{:}{}
\SetKwInput{KwData}{输入}
\SetKwInput{KwResult}{输出}
\SetArgSty{textnormal}

\renewcommand{\emph}[1]{\textbf{#1}}
\newcommand*{\concept}[1]{\underline{\textbf{#1}}}

\title{统计推断}
\author{吴何友}

\begin{document}

\maketitle

\section{统计推断的基本概念}

我们常常需要面对这样的问题:设有一个在我们关心的范围内不随时间变化的随机变量(称为\concept{总体})$\mathcal{X}$,我们并不完全知道这个随机变量的形式。
现在我们获知了和$\mathcal{X}$有关的一系列事件$X$(称为\concept{证据}),要通过它们来推知关于$\mathcal{X}$的一些信息,这个过程就是\concept{统计推断}。
例如,证据可以是反复运行$\mathcal{X}$而获得的一组取值
\begin{equation}
    \mathcal{D} = \{X_i\},
\end{equation}
这称为\concept{样本},获得取值的方式称为\concept{采样})。

\subsection{常见的统计推断任务}

常见的统计推断任务包括
\begin{itemize}
    \item \concept{假设检验},即从一系列证据出发,判断是否应该认为关于$\mathcal{X}$的某个命题成立;
    \item \concept{概率估计},即判断某个命题为真的概率有多大,这和假设检验有非常显然的联系;
    \item \concept{参数估计},即判断与总体相关的一些参数的取值(或者取值的概率分布);
    \item \concept{相关性分析},这是在$\mathcal{X}$有多个分量时使用的,即分析这些变量之间的关系;
\end{itemize}
还有许许多多其它的任务。

从样本出发做统计推断的一个特殊的例子是
\begin{equation}
    \mathcal{X} = (\vb{x}, y),
\end{equation}
$\vb{x}$是一组数据(通常称为\concept{特征}),$y$是和这组数据有关的另一个数据(通常称为\concept{标签}),此时的统计推断就是\concept{机器学习}。
如果直接从样本分析$p(\vb{x}, y)$的特征,就是\concept{生成模型},而如果从样本分析$p(y|\vb{x})$的特征,那就是\concept{判别模型}。两者之间有着这样的关系:
\begin{equation}
    p(y | \vb{x}) = \frac{p(\vb{x}, y)}{p(\vb{x})} = \frac{p(\vb{x}, y)}{\sum_{y'} p(\vb{x}, y')}, 
\end{equation}
这里使用的记号需要特殊说明一下:$p(y | \vb{x})$实际上是“在某次运行$\mathcal{X}$,得到的特征是$\vb{x}$的条件下,发现其标签为$y$”的概率的简写,$p(\vb{x}, y)$同理。
为了简洁我们接下来在不至于混淆时不强调“随机变量”和“随机变量的可能取值”的区别。

\subsection{模型}

当然,既然$\mathcal{X}$是自然界中一系列随机性过程给出的输出,我们总是可以找到一系列数目往往非常大的随机变量$\alpha_1, \alpha_2, \ldots, \alpha_N$,它们决定了$\mathcal{X}$,即
\[
    \mathcal{X} = F(\alpha_1, \alpha_2, \ldots, \alpha_N).
\]
我们可以把物理定律当成某种“计算过程”,向它输入一些事件,就可以输出一个$\mathcal{X}$的值。
我们当然不可能使用如此大量的随机变量来描述$\mathcal{X}$,即使技术上可行,这也没有提供给我们任何有价值的信息,因为这相当于做了一次虚拟仿真实验。
但,如果我们能够找到某个事件$H$和一个随机变量$\theta$(它可能也含有大量的分量,但无论如何比$\{\alpha\}$简洁),使得
\[
    p(\mathcal{X} = C | H, \theta) = f(C ; \theta), 
\]
且$p(H)$非常大,那就可以大大简化我们要处理的问题,并且提供给我们一个“因为所以”式的\emph{解释}——因为$\theta$取了某些值,所以$\mathcal{X}$取了这个特定的值。
我们称$H$为\concept{模型假设},$f(\mathcal{X};\theta)$为\concept{模型},$\theta$为\concept{参数}。
在$p(H)$非常大(以至于可以认为$H$是确凿无疑的一个命题)时,我们就有
\begin{equation}
    p(\mathcal{X} = C | \theta) = f(C ; \theta).
\end{equation}
模型假设是否正确可以通过假设检验来判断,也可以通过专家知识给出。

\section{频率学派}

\end{document}