\documentclass[hyperref, a4paper]{article}

\usepackage{geometry}
\usepackage{titling}
\usepackage{titlesec}
% No longer needed, since we will use enumitem package
% \usepackage{paralist}
\usepackage{enumitem}
\usepackage{footnote}
\usepackage[colorinlistoftodos]{todonotes}
\usepackage{amsmath, amssymb, amsthm}
\usepackage{mathtools}
\usepackage{bbm}
\usepackage{graphicx}
\usepackage{subcaption}
\usepackage{soulutf8}
\usepackage{physics}
\usepackage{tensor}
\usepackage{siunitx}
\usepackage[version=4]{mhchem}
\usepackage{tikz}
\usepackage{xcolor}
\usepackage{listings}
\usepackage{autobreak}
\usepackage[ruled, vlined, linesnumbered]{algorithm2e}
\usepackage{nameref,zref-xr}
\zxrsetup{toltxlabel}
\usepackage[backend=bibtex]{biblatex}
\addbibresource{nonequilibrium.bib}
\usepackage[colorlinks,unicode]{hyperref} % , linkcolor=black, anchorcolor=black, citecolor=black, urlcolor=black, filecolor=black
\usepackage[most]{tcolorbox}
\usepackage{prettyref}

% Page style
\geometry{left=3.18cm,right=3.18cm,top=2.54cm,bottom=2.54cm}
\titlespacing{\paragraph}{0pt}{1pt}{10pt}[20pt]
\setlength{\droptitle}{-5em}

% More compact lists 
\setlist[itemize]{
    itemindent=17pt, 
    leftmargin=1pt,
    listparindent=\parindent,
    parsep=0pt,
}

% Math operators
\DeclareMathOperator{\timeorder}{\mathcal{T}}
\DeclareMathOperator{\diag}{diag}
\DeclareMathOperator{\legpoly}{P}
\DeclareMathOperator{\primevalue}{P}
\DeclareMathOperator{\sgn}{sgn}
\DeclareMathOperator{\res}{Res}
\DeclareMathOperator{\expect}{\mathbb{E}}
\newcommand*{\ii}{\mathrm{i}}
\newcommand*{\ee}{\mathrm{e}}
\newcommand*{\const}{\mathrm{const}}
\newcommand*{\suchthat}{\quad \text{s.t.} \quad}
\newcommand*{\argmin}{\arg\min}
\newcommand*{\argmax}{\arg\max}
\newcommand*{\normalorder}[1]{: #1 :}
\newcommand*{\pair}[1]{\langle #1 \rangle}
\newcommand*{\fd}[1]{\mathcal{D} #1}
\DeclareMathOperator{\bigO}{\mathcal{O}}

% TikZ setting
\usetikzlibrary{arrows,shapes,positioning}
\usetikzlibrary{arrows.meta}
\usetikzlibrary{decorations.markings}
\tikzstyle arrowstyle=[scale=1]
\tikzstyle directed=[postaction={decorate,decoration={markings,
    mark=at position .5 with {\arrow[arrowstyle]{stealth}}}}]
\tikzstyle ray=[directed, thick]
\tikzstyle dot=[anchor=base,fill,circle,inner sep=1pt]

% Algorithm setting
% Julia-style code
\SetKwIF{If}{ElseIf}{Else}{if}{}{elseif}{else}{end}
\SetKwFor{For}{for}{}{end}
\SetKwFor{While}{while}{}{end}
\SetKwProg{Function}{function}{}{end}
\SetArgSty{textnormal}

\newcommand*{\concept}[1]{{\textbf{#1}}}

% Embedded codes
\lstset{basicstyle=\ttfamily,
  showstringspaces=false,
  commentstyle=\color{gray},
  keywordstyle=\color{blue}
}

% Reference formatting
\newcommand*{\citesec}[1]{\S~{#1}}
\newcommand*{\citechap}[1]{chap.~{#1}}
\newcommand*{\citefig}[1]{Fig.~{#1}}
\newcommand*{\citetable}[1]{Table~{#1}}
\newcommand*{\citepage}[1]{pp.~{#1}}
\newrefformat{fig}{Fig.~\ref{#1}}
\newcommand*{\term}[1]{\textit{#1}}

% Color boxes
\tcbuselibrary{skins, breakable, theorems}

\newtcbtheorem{infobox}{Box}{
    enhanced,
    boxrule=0pt,
    colback=blue!5,
    colframe=blue!5,
    coltitle=blue!50,
    borderline west={4pt}{0pt}{blue!65},
    sharp corners,
    fonttitle=\bfseries, 
    breakable,
    before upper={\parindent15pt\noindent}}{box}
\newtcbtheorem[use counter from=infobox]{theorybox}{Box}{
    enhanced,
    boxrule=0pt,
    colback=orange!5, 
    colframe=orange!5, 
    coltitle=orange!50,
    borderline west={4pt}{0pt}{orange!65},
    sharp corners,
    fonttitle=\bfseries, 
    breakable,
    before upper={\parindent15pt\noindent}}{box}
\newtcbtheorem[use counter from=infobox]{learnbox}{Box}{
    enhanced,
    boxrule=0pt,
    colback=green!5,
    colframe=green!5,
    coltitle=green!50,
    borderline west={4pt}{0pt}{green!65},
    sharp corners,
    fonttitle=\bfseries, 
    breakable,
    before upper={\parindent15pt\noindent}}{box}


\newenvironment{shelldisplay}{\begin{lstlisting}}{\end{lstlisting}}

\newcommand*{\kB}{k_{\text{B}}}
\newcommand*{\muB}{\mu_{\text{B}}}
\newcommand*{\efermi}{E_{\text{F}}}
\newcommand*{\pfermi}{p_{\text{F}}}
\newcommand*{\vfermi}{v_{\text{F}}}
\newcommand*{\sA}{\text{A}}
\newcommand*{\sB}{\text{B}}
\newcommand*{\Tc}{T_{\text{c}}}
\newcommand*{\hethree}{$^3$He}
\newcommand*{\hefour}{$^4$He}
\newcommand{\epsr}{\epsilon_{\text{r}}}
\newcommand{\chie}{\chi_{\text{e}}}
\newcommand{\cf}{c_{\text{F}}}
\newcommand{\fn}{F_{\text{N}}}
\newcommand{\ff}{F_{\text{f}}}

\title{Betting}
\author{Jinyuan Wu}

\begin{document}

\maketitle

Probabilistic theory involved in betting is simpler than it is in many other situations,
making closed form solutions often possible,
as correlation between events is mostly absent.

\section{Value betting}

When a game has an odds of $a : 1$,
it means that if you win, your profit $a$ times your money you've betted,
that's to say, the bookmaker pays you $(a + 1)$ times your stake.
So when the actual probability of winning is $p$, 
the expected value (EV) of your profit per dollar you bet is 
\begin{equation}
    \expect[\text{profit}] = p \cdot a - (1 - p).
\end{equation}
If the bookmaker is neither being silly nor being dishonest,
the expected value should vanish, meaning that 
\begin{equation}
    p = \frac{1}{1 + a}.
\end{equation}
The RHS is known as the \concept{implied probability}.
The variance is 
\begin{equation}
     \expect[\text{profit}^2] - \expect[\text{profit}]^2 = p(1-p) (a+1)^2.
\end{equation}

A gambler should only participate in the game when the EV is positive.
But this alone does not guarantee that they will not go broke.
The square root of the variance of the profit represents
the ``error range'', and it's not a good sign when it exceeds the EV.

\concept{Kelly's criteria} is one way to decide how much you should pay 

\section{Arbitrage}

When there are two games in which the winning criteria are opposite to each other,
and the odds are set in a rather irrational way,
there is a chance of always winning.
Suppose the probability of something happening is $p$,
and in one game, the odds of it happening is $a : 1$,
while in another, the odds of it \emph{not} happening is $b : 1$.
This means if you bet a percentage of $f$ of your total bankroll on the first game
and $1 - f$ on the second game,
the probability of you profit being $a f - (1 - f)$ is $p$,
and the probability of your profit being $b (1 - f) - f$ is $1 - p$.
The expectation value then reads 
\begin{equation}
    \expect[\text{profit}] = p ((a + 1) f - 1) + (1 - p) ((b + 1) (1 - f) - 1),
\end{equation}
and to make the RHS independent to $p$, we need 
\begin{equation}
    (a + 1) f - (b + 1) (1 - f) = 0 \Rightarrow
    f = \frac{b + 1}{a + b + 2},
    \label{eq:arbitrage-f}
\end{equation}
and when this is the case,
\begin{equation}
    \expect[\text{profit}] = \frac{(a + 1) (b + 1)}{a + b + 2} - 1.
\end{equation}
To maintain a positive EV, we need 
\begin{equation}
    \frac{1}{a + 1} + \frac{1}{b + 1} < 1,
    \label{eq:arbitrage}
\end{equation}
which is equivalent to saying that the sum of the implied possibilities is smaller than one.

\eqref{eq:arbitrage} is known as the \concept{arbitrage condition},
which, when satisfied, guarantees zero-risk profit,
because it can be verified that when \eqref{eq:arbitrage-f} is satisfied,
the profit stays the same in any circumstance and equals the EV.

\end{document}