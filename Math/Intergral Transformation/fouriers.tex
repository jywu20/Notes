\hypertarget{sec:fourier}{%
\section{各种带着``傅里叶''的积分变换}\label{sec:fourier}}

\hypertarget{sec:fourier-series}{%
\subsection{傅里叶级数}\label{sec:fourier-series}}

\hypertarget{sec:fourier-series-real}{%
\subsubsection{实数形式的傅里叶级数定义}\label{sec:fourier-series-real}}

设$f$是定义在$\mathbb{R}$上的一个以$2\pi$为周期的函数。我们有
\[
f(x) = \frac{a_0}{2} + \sum_{n=0}^{\infty} a_n \cos (nx) + b_n \sin(nx), 
\] 
以及
\[
a_n = \frac{1}{\pi} \int_{-\pi}^{\pi} f(x) \cos (nx) \mathrm{d}x, b_n =  \frac{1}{\pi} \int_{-\pi}^{\pi} f(x) \sin (nx) \mathrm{d}x.
\]

\hypertarget{sec:fourier-series-attributes}{%
\subsubsection{傅里叶级数的一些性质}\label{sec:fourier-series-attributes}}

首先当然有线性性质。其次,对于一些函数变换,我们有

设$g(x) = f(-x)$,$a_n, b_n$为$f$的傅里叶系数,$a_n', b_n'$为$g$的傅里叶系数,则
\[
a_n' = a_n, b_n' = -b_n.
\]
设$h(x) = f(x+C)$,C为常数,$a_n, b_n$为$f$的傅里叶系数,$a_n', b_n'$为$h$的傅里叶系数,则
\[
a_n' = a_n \cos (nC) + b_n \sin (nC), b_n' = -a_n \sin (nC) + b_n \cos (nC).
\]

设
\[
F(x) = \frac{1}{\pi} \int_{-\pi}^{\pi} f(t) f(x-t) \mathrm{d}t,
\]
$a_n, b_n$为$f$的傅里叶系数,$a_n', b_n'$为$F$的傅里叶系数,则
\[
a_0' = a_0, a_n' = a_n^2 - b_n^2 (n = 1, 2, 3, \ldots); \qquad b_n' = 2a_n b_n.
\]

\hypertarget{ux6700ux4f73ux5e73ux65b9ux903cux8fd1}{%
\paragraph{最佳平方逼近}\label{ux6700ux4f73ux5e73ux65b9ux903cux8fd1}}

设$f$在$[-\pi, \pi]$上黎曼可积,则$f$在$n$阶三角多项式中的最佳平方逼近元素就是$f$的傅里叶级数的部分和。

\hypertarget{ux5468ux671fux4efbux610fux7684sec:fourier-series}{%
\subsubsection{周期任意的傅里叶级数}\label{ux5468ux671fux4efbux610fux7684sec:fourier-series}}

在$f$的周期为任意值$2T$的时候还是可以把$f$写成级数形式,只是这个时候需要做一个变量代换。设
\[
F(t) = f(x), \quad t = \frac{\pi}{T} x.
\] 
我们有 
\[
\begin{aligned}
    f(x) &= F(t) = \frac{a_0}{2} + \sum_{n=0}^{\infty} a_n \cos (nt) + b_n \sin(nt) \\
    &= \frac{a_0}{2} + \sum_{n=0}^{\infty} a_n \cos \left(\frac{\pi n x}{T}\right) + b_n \sin \left(\frac{\pi n x}{T}\right),
\end{aligned}
\]
级数系数为
\[
\begin{aligned}
    a_n &= \frac{1}{\pi} \int_{-\pi}^\pi F(t) \cos(nt) \mathrm{d}t = \frac{1}{\pi} \int_{-\pi}^\pi f(x) \cos \left(\frac{\pi n x}{T} \right) \frac{\pi}{T} \mathrm{d}x \\
    &= \frac{1}{T} \int_{-T}^T f(x) \cos \left(\frac{\pi n x}{T} \right) \mathrm{d}x,
\end{aligned}
\]
同理
\[
b_n = \frac{1}{T} \int_{-T}^T f(x) \sin \left(\frac{\pi n x}{T} \right) \mathrm{d}x.
\]
因此就得到任意周期的函数的傅里叶级数:
\[
\begin{aligned}
    f(x) &= \frac{a_0}{2} + \sum_{n=0}^{\infty} a_n \cos \left(\frac{\pi n x}{T}\right) + b_n \sin \left(\frac{\pi n x}{T}\right), \\
    a_n &= \frac{1}{T} \int_{-T}^T f(x) \cos \left(\frac{\pi n x}{T} \right) \mathrm{d}x, \\
    b_n &= \frac{1}{T} \int_{-T}^T f(x) \sin \left(\frac{\pi n x}{T} \right) \mathrm{d}x.
\end{aligned}
\]

类似的可以得到复数形式的傅里叶级数:
\[
f(x) = \frac{1}{2} \sum_{n = -\infty}^{n = \infty} c_n e^{\frac{\mathrm{i}\pi n x}{T}}, 
\]
\[
c_n = \frac{1}{T}\int_{-T}^{T} f(t) e^{-\frac{\mathrm{i}\pi n t}{T}} \mathrm{d}t, \quad c_n = \overline{c}_{-n}.
\]

如果$T$是$f$的一个周期,那么

\hypertarget{ux6b63ux5f26ux7ea7ux6570ux4e0eux4f59ux5f26ux7ea7ux6570}{%
\subsubsection{正弦级数与余弦级数}\label{ux6b63ux5f26ux7ea7ux6570ux4e0eux4f59ux5f26ux7ea7ux6570}}

\hypertarget{ux5173ux4e8eux6536ux655bux6027ux7684ux6ce8ux8bb0}{%
\subsubsection{关于收敛性的注记}\label{ux5173ux4e8eux6536ux655bux6027ux7684ux6ce8ux8bb0}}

\hypertarget{ux590dux6570ux5f62ux5f0fux7684sec:fourier-series}{%
\subsubsection{复数形式的傅里叶级数}\label{ux590dux6570ux5f62ux5f0fux7684sec:fourier-series}}

将实数形式的傅里叶级数中的正弦项和余弦项替换成指数函数的形式,即利用
\[
\cos \theta = \frac{1}{2} (e^{\mathrm{i} \theta} + e^{ - \mathrm{i} \theta}),
\sin \theta = \frac{1}{2 \mathrm{i}} (e^{\mathrm{i} \theta} - e^{ - \mathrm{i} \theta}),
\]
可以从实数形式的傅里叶级数得到
\[
f(x) = \frac{1}{2} \sum_{n = -\infty}^{n = \infty} c_n e^{\mathrm{i}n x}, \\
c_n = \frac{1}{\pi}\int_{-\pi}^{\pi} f(t) e^{-\mathrm{i}nt} \mathrm{d}t, \quad c_n = \overline{c}_{-n}.
\]

\hypertarget{ux5b9eux6570ux5f62ux5f0fux7684ux5085ux91ccux53f6ux79efux5206ux4e0eux590dux6570ux5f62ux5f0fux7684ux5085ux91ccux53f6ux79efux5206}{%
\subsection{实数形式的傅里叶积分与复数形式的傅里叶积分}\label{ux5b9eux6570ux5f62ux5f0fux7684ux5085ux91ccux53f6ux79efux5206ux4e0eux590dux6570ux5f62ux5f0fux7684ux5085ux91ccux53f6ux79efux5206}}

\hypertarget{ux5b9eux6570ux5f62ux5f0fux7684ux5085ux91ccux53f6ux79efux5206}{%
\subsubsection{实数形式的傅里叶积分}\label{ux5b9eux6570ux5f62ux5f0fux7684ux5085ux91ccux53f6ux79efux5206}}

对于周期为$2T$的周期函数我们有
\[
f(x) = \frac{1}{2T} \int_{-T}^{T} f(t) \mathrm{d}t + 
\sum_{n \in \natnums} \frac{1}{T} \cos \left( \frac{\pi n x}{T} \right) \int_{-T}^{T} f(t) \cos \left( \frac{\pi n t}{T} \right) \mathrm{d}t +
\frac{1}{T} \sin \left( \frac{\pi n x}{T} \right) \int_{-T}^{T} f(t) \sin \left( \frac{\pi n t}{T} \right) \mathrm{d}t.
\]
如果我们设
\[
\omega = \frac{\pi n}{T}, \quad \Delta \omega = \frac{\pi}{T},
\]
上式就可以写成
\[
f(x) = \frac{1}{2T} \int_{-T}^{T} f(t) \mathrm{d}t +
\frac{1}{\pi} \sum_{n \in \natnums} \left[ \cos(\omega x) \int_{-T}^{T} f(t) \cos(\omega t) \mathrm{d}t + \sin(\omega x) \int_{-T}^{T} f(t) \sin(\omega t) \mathrm{d}t \right] \Delta \omega
\]
现在想象$T$变得很大,以至于上式中的第一项趋于0(这其实是需要进一步说明的,因为完全有可能积分发散使得第一项发散或者第一项趋于某个有限的非零值),而第二项则看起来像个积分,那么我们就得到了实数形式的傅里叶积分。

令$T$趋向于无限大,我们得到
\[
f(x) = \frac{1}{\pi}\int_0^{+\infty} \left(
    \cos(\omega x) \int_{-\infty}^{+\infty} f(t) \cos(\omega t) \mathrm{d}t
    +  \sin(\omega x) \int_{-\infty}^{+\infty} f(t) \sin(\omega t) \mathrm{d}t
\right) \mathrm{d}\omega,
\]

\hypertarget{ux590dux6570ux5f62ux5f0fux7684ux5085ux91ccux53f6ux79efux5206}{%
\subsubsection{复数形式的傅里叶积分}\label{ux590dux6570ux5f62ux5f0fux7684ux5085ux91ccux53f6ux79efux5206}}

\[
f(x) = \frac{1}{2\pi} \int_{-\infty}^{+\infty} \left(\int_{-\infty}^{+\infty} f(t) e^{-\mathrm{i} \omega t} \mathrm{d}t \right) e^{\mathrm{i} \omega x} \mathrm{d} \omega,
\]

\hypertarget{ux5085ux91ccux53f6ux53d8ux6362}{%
\subsection{傅里叶变换}\label{ux5085ux91ccux53f6ux53d8ux6362}}

现在定义定义
\[
\begin{aligned}
    \mathrm{FT}[f](\omega) &= \int_{-\infty}^\infty f(t) e^{-\mathrm{i}\omega t} \mathrm{d}t, \\
    \mathrm{FT}^{-1}[\hat{f}] &= \frac{1}{2\pi} \int_{-\infty}^\infty \hat{f}(\omega) e^{\mathrm{i}\omega t} \mathrm{d}\omega.
\end{aligned}
\]
称$\mathrm{FT}[f](\omega)$为$f$的傅里叶变换,通常使用$\hat{f}$表示它;称$\mathrm{FT}^{-1}[\hat{f}](t)$为$\hat{f}$的傅里叶逆变换。

\hypertarget{ux5e7fux4e49ux51fdux6570}{%
\subsubsection{广义函数}\label{ux5e7fux4e49ux51fdux6570}}

在$f$在整条数轴上的积分不收敛等情况下(例如,如果$f$是正弦函数,那么积分就不收敛),仍然可以有傅里叶变换,但此时会得到一些广义函数。

一个典型的例子是狄拉克Delta函数。我们知道
\[
\int_{-\infty}^\infty f(x) \delta(x-x_0) \mathrm{d}x = f(x_0), \quad \delta(-x) = \delta(x).
\]
可以证明上面的公式\textbf{唯一确定了}Delta函数。所以现在就需要凑出一个广义函数,它经常出现在傅里叶变换式当中,但确实就是Delta函数:
\[
\begin{aligned}
    f(x) &= \frac{1}{2\pi} \int_{-\infty}^{+\infty} \left(\int_{-\infty}^{+\infty} f(t) e^{-\mathrm{i} \omega t} \mathrm{d}t \right) e^{\mathrm{i} \omega x} \mathrm{d} \omega \\
    &= \frac{1}{2\pi} \int_{-\infty}^{+\infty} \left(\int_{-\infty}^{+\infty} f(t) e^{\mathrm{i} \omega x -\mathrm{i} \omega t} \mathrm{d}t \right) \mathrm{d} \omega \\
    &= \int_{-\infty}^\infty f(t) \left( \int_{-\infty}^\infty \frac{1}{2\pi} e^{\mathrm{i} \omega (x-t)} \mathrm{d}\omega \right) \mathrm{d}t.
\end{aligned}
\]
所以我们发现
\[
\delta(t-x) = \frac{1}{2\pi} \int_{-\infty}^\infty e^{\mathrm{i} \omega (x-t)} \mathrm{d}\omega, 
\]
或者说
\[
\delta(t'-t) = \frac{1}{2\pi} \int_{-\infty}^\infty e^{\mathrm{i} \omega (t'-t)} \mathrm{d}\omega.
\]

\hypertarget{ux6b63ux5f26ux4f59ux5f26}{%
\paragraph{正弦、余弦}\label{ux6b63ux5f26ux4f59ux5f26}}

例如,正弦函数和余弦函数的傅里叶变换就可以使用Delta函数表示。设
\[
f(t) = e^{\mathrm{i}\omega_0t},
\]
则
\[
\mathrm{FT}[f](\omega) = 2\pi \delta(\omega - \omega_0).
\]

\hypertarget{ux5468ux671fux51fdux6570ux7684ux5085ux91ccux53f6ux53d8ux6362}{%
\subsubsection{周期函数的傅里叶变换}\label{ux5468ux671fux51fdux6570ux7684ux5085ux91ccux53f6ux53d8ux6362}}

设$f$是周期为$2T$的周期函数。我们知道在这种情况下$f$可以被展开为傅里叶级数:
\[
f(x) = \frac{1}{2} \sum_{n = -\infty}^{n = \infty} c_n e^{\frac{\mathrm{i}\pi n x}{T}}, 
\]
根据上式我们计算$f$的傅里叶变换:
\[
\mathrm{DFT}[f] = \pi \sum_{n=-\infty}^\infty c_n \delta(\omega - \frac{\pi n}{T}).
\]

\hypertarget{ux79bbux6563ux5085ux91ccux53f6ux53d8ux6362}{%
\subsection{离散傅里叶变换}\label{ux79bbux6563ux5085ux91ccux53f6ux53d8ux6362}}

在离散的情况下有一个类似于傅里叶积分的公式:
\[
f(x) = \frac{1}{N} \sum_{u=0}^{N-1} e^{\mathrm{i} \frac{2\pi}{N} xu} \sum_{t=0}^{N-1} f(t) e^{-\mathrm{i} \frac{2\pi}{N}tu}.
\]

按照连续傅里叶变换类似的套路,可以定义
\[
\begin{aligned}
    \mathrm{DFT}[f](u) &= \sum_{t=0}^{N-1} f(t) e^{-\mathrm{i} \frac{2\pi}{N}tu}, \\
    \mathrm{DFT}^{-1}[\hat{f}](t) &= \frac{1}{N} \sum_{u=0}^{N-1} \hat{f}(u) e^{\mathrm{i} \frac{2\pi}{N} tu}.
\end{aligned}
\]
称$\mathrm{DFT}$为离散傅里叶变换,而$\mathrm{DFT}^{-1}$为离散傅里叶逆变换。

\hypertarget{ux79bbux6563ux7684deltaux51fdux6570-ux514bux7f57ux5185ux514bdeltaux51fdux6570}{%
\subsubsection{离散的Delta函数:
克罗内克Delta函数}\label{ux79bbux6563ux7684deltaux51fdux6570-ux514bux7f57ux5185ux514bdeltaux51fdux6570}}

用于定义离散傅里叶变换的式子本身立刻可以导出一个结果:
\[
f(x) = \sum_{t=0}^{N-1} \left( \frac{1}{N} \sum_{u=0}^{N-1}e^{\mathrm{i} \frac{2\pi}{N} (x-t) u} \right) f(t),
\]
因此
\[
\delta_{t,t'} = \frac{1}{N} \sum_{u=0}^{N-1}e^{\mathrm{i} \frac{2\pi}{N} (t'-t) u} = \frac{1}{N} \sum_{u=0}^{N-1}e^{\mathrm{i} \frac{2\pi}{N} (t-t') u}, \quad t,t' = 0,1, \ldots, N-1.
\]
$t,t'$超过$N$之后方程的右半边会发生周期性循环。在方程右边的指数当中变换正负号不会影响结果是因为方程左边是实数,所以右边取复共轭之后结果不变。

\hypertarget{ux91c7ux6837}{%
\subsubsection{采样}\label{ux91c7ux6837}}

为什么称它们为\textbf{离散}傅里叶变换呢?它们和连续的傅里叶变换有什么关系呢?

设$f$是一个连续的周期函数,且它有周期$2T$。$f_s$是对$f$在$[0, 2T)$上做$N$点采样之后的结果。我们可以分别计算前者的傅里叶变换和后者的离散傅里叶变换。

设样本点总数为$N$,则
\[
f_s (n) = f \left( \frac{2T}{N}n \right),
\]
此时,
\[
\begin{aligned}
    \mathrm{DFT}[f_s](u) &= \sum_{n=0}^{N-1} f\left( \frac{2T}{N}n \right) e^{-\mathrm{i} \frac{2\pi n}{N}u} \\
    &= \sum_{n=0}^{N-1} \frac{1}{2} \sum_{m = -\infty}^{\infty} c_m e^{\frac{\mathrm{i}\pi m}{T} \frac{2T}{N}n } e^{-\mathrm{i} \frac{2\pi n}{N}u} \\
    &= \frac{1}{2} \sum_{m=-\infty}^\infty \sum_{n=0}^{N-1} c_m e^{\mathrm{i}\frac{2\pi}{N} (m-u)n} \\
    &= \frac{1}{2} \sum_{m=-\infty}^\infty c_m N \delta_{m,u} = \frac{1}{2} N c_u.
\end{aligned}
\]
而另一方面,我们有$f$的频谱
\[
\mathrm{FT}[f](\omega) = \pi \sum_{n=-\infty}^\infty c_n \delta(\omega - \frac{\pi n}{T}).
\]
我们发现$f_s$做离散傅里叶变换和$f$做连续傅里叶变换之后得到的结果是相对应的:(注:在取不同的$T$的时候,计算出来的$c_n$会有所不同,但是,无论取什么样的$T$,最后得到的傅里叶级数的形式都是一样的,从而傅里叶变换的形式也是一样的。这里使用$\mathrm{FT}$来代替$c_n$就是为了消除这个影响)
\[
\mathrm{FT}[f](\omega) = \frac{2\pi}{N} \sum_{n=-\infty}^\infty \mathrm{DFT}[f_s](n) \delta(\omega - \frac{\pi n}{T}), \\
\mathrm{DFT}[f_s](n) = \frac{N}{2\pi \delta(0)} \mathrm{FT}[f](\frac{\pi n}{T}).
\]
在$\omega$可以写成$\pi/T$的整数倍的时候,有
\[
\mathrm{FT}[f](\omega) = \frac{2\pi \delta(0)}{N} \mathrm{DFT}[f_s](\frac{\omega T}{\pi}).
\]

这就意味着,\textbf{在采样长度为$2T$,总采样数为$N$的情况下,样本信号的频谱中每一格对应$\pi/T$的圆频率间隔}。

我们还可以注意到$\mathrm{DFT}$运算得出的结果以$N$为周期做循环。因此,通常限定$u=0, 1, \ldots, N-1,$也即,离散傅里叶变换输入一个长度为$N$的序列,也输出一个长度为$N$的序列。

假如采样长度并不精确为$f$的某个倍数,那么$\mathrm{DFT}[f_s]$和$\mathrm{FT}[f]$不会有严格的线性关系。可以证明,此时$\mathrm{DFT}[f_s]$中的峰会发生展宽,即所谓的能量溢出。然而,在采样长度$L$足够大的情况下,虽然$L$并非$f$的周期,由于采样结果中大部分的区段都是周期性的,最后画出来的$\mathrm{DFT}[f_s]$不会有太大的畸变,因此,\textbf{在采样长度为$L$,总采样数为$N$的情况下,样本信号的频谱中每一格对应$2\pi/L$的频率间隔}。相应的,输出的$\mathrm{DFT}[f_s]$的长度还是$N$。这就意味着,如果使用$\mathrm{DFT}[f_s]$作为$\mathrm{FT}[f]$的近似值,那么$\mathrm{DFT}[f_s]$只能覆盖零到$2\pi N/L$的圆频率范围。

另一方面,$f_s$的离散傅里叶变换频谱上峰的高度正比于$N$。如果需要比较不同采样点数的样本的频谱,应该记得这一点。
