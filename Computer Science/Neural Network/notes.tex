\documentclass[hyperref, a4paper, 12pt]{report}

\usepackage{geometry}
\usepackage{titling}
\usepackage{titlesec}
% No longer needed, since we will use enumitem package
% \usepackage{paralist}
\usepackage{enumitem}
\usepackage{footnote}
\usepackage{amsmath, amssymb, amsthm}
\usepackage{mathtools}
\usepackage{bbm}
\usepackage{graphicx}
\usepackage{subfigure}
\usepackage{simpler-wick}
\usepackage{physics}
\usepackage{tensor}
\usepackage{siunitx}
\usepackage[version=4]{mhchem}
\usepackage{tikz}
\usepackage{xcolor}
\usepackage{listings}
\usepackage{underscore}
\usepackage{autobreak}
\usepackage[ruled, vlined, linesnumbered]{algorithm2e}
\usepackage{nameref,zref-xr}
\zxrsetup{toltxlabel}
\usepackage[colorlinks,unicode, bookmarksnumbered]{hyperref} % , linkcolor=black, anchorcolor=black, citecolor=black, urlcolor=black, filecolor=black
\usepackage[most]{tcolorbox}
\usepackage[backend=bibtex,doi=false,isbn=false,url=false]{biblatex}
\addbibresource{probability.bib}
\usepackage{prettyref}

% Page style
\geometry{left=3.18cm,right=3.18cm,top=2.54cm,bottom=2.54cm}
\titlespacing{\paragraph}{0pt}{1pt}{10pt}[20pt]
\setlength{\droptitle}{-5em}

% More compact lists 
\setlist[itemize]{
    %itemindent=17pt, 
    %leftmargin=1pt,
    listparindent=\parindent,
    parsep=0pt,
}

\setlist[enumerate]{
    %itemindent=17pt, 
    %leftmargin=1pt,
    listparindent=\parindent,
    parsep=0pt,
}

% Math operators
\DeclareMathOperator{\timeorder}{\mathcal{T}}
\DeclareMathOperator{\diag}{diag}
\DeclareMathOperator{\legpoly}{P}
\DeclareMathOperator{\primevalue}{P}
\DeclareMathOperator{\sgn}{sgn}
\DeclareMathOperator{\res}{Res}
\newcommand*{\ii}{\mathrm{i}}
\newcommand*{\ee}{\mathrm{e}}
\newcommand*{\const}{\mathrm{const}}
\newcommand*{\suchthat}{\quad \text{s.t.} \quad}
\newcommand*{\argmin}{\arg\min}
\newcommand*{\argmax}{\arg\max}
\newcommand*{\normalorder}[1]{: #1 :}
\newcommand*{\pair}[1]{\langle #1 \rangle}
\newcommand*{\fd}[1]{\mathcal{D} #1}
\DeclareMathOperator{\bigO}{\mathcal{O}}

% TikZ setting
\usetikzlibrary{arrows,shapes,positioning}
\usetikzlibrary{arrows.meta}
\usetikzlibrary{decorations.markings}
\usetikzlibrary{calc}
\tikzstyle arrowstyle=[scale=1]
\tikzstyle directed=[postaction={decorate,decoration={markings,
    mark=at position .5 with {\arrow[arrowstyle]{stealth}}}}]
\tikzstyle ray=[directed, thick]
\tikzstyle dot=[anchor=base,fill,circle,inner sep=1pt]

% Algorithm setting
% Julia-style code
\SetKwIF{If}{ElseIf}{Else}{if}{}{elseif}{else}{end}
\SetKwFor{For}{for}{}{end}
\SetKwFor{While}{while}{}{end}
\SetKwProg{Function}{function}{}{end}
\SetArgSty{textnormal}

\newcommand*{\concept}[1]{{\textbf{#1}}}
\newcommand*{\term}[1]{\textit{#1}}

% Embedded codes
\lstset{basicstyle=\ttfamily,
  showstringspaces=false,
  commentstyle=\color{gray},
  keywordstyle=\color{blue}
}

\lstdefinestyle{console}{
    basicstyle=\footnotesize\ttfamily,
    breaklines=true,
    postbreak=\mbox{\textcolor{red}{$\hookrightarrow$}\space}
}
\definecolor{mygreen}{rgb}{0,0.6,0}
\definecolor{mygray}{rgb}{0.5,0.5,0.5}
\definecolor{mymauve}{rgb}{0.58,0,0.82}
\lstdefinestyle{python}{
        language=Python,
        backgroundcolor=\color{white},
        basicstyle=\footnotesize\ttfamily,
        keywordstyle=\color{blue},
        commentstyle=\color{mygreen},
        stringstyle=\color{mymauve},
        numberstyle=\tiny\color{mygray},
        numbers=left,
        breaklines=true,
        showstringspaces=false
}

% Reference formatting
\newrefformat{fig}{Figure~\ref{#1}}

% Color boxes
\tcbuselibrary{skins, breakable, theorems}

\newtcbtheorem[number within=chapter]{infobox}{Box}{
    enhanced,
    boxrule=0pt,
    %colback=blue!5,
    %colframe=blue!5,
    colback=white,
    colframe=white,
    coltitle=blue!60,
    borderline west={4pt}{0pt}{blue!65},
    sharp corners,
    fonttitle=\bfseries, 
    breakable,
    before upper={\parindent15pt\noindent}}{box}
\definecolor{my-orange}{HTML}{F58123}
\newtcbtheorem[number within=chapter, use counter from=infobox]{theorybox}{Box}{
    enhanced,
    boxrule=0pt,
    %colback=orange!5, 
    %colframe=orange!5, 
    colback=white,
    colframe=white,
    coltitle=my-orange!65,
    borderline west={4pt}{0pt}{my-orange!65},
    sharp corners,
    fonttitle=\bfseries, 
    breakable,
    before upper={\parindent15pt\noindent}}{box}
\newtcbtheorem[number within=chapter, use counter from=infobox]{todobox}{Box}{
    enhanced,
    boxrule=0pt,
    colback=red!5,
    colframe=red!5,
    coltitle=red!50,
    borderline west={4pt}{0pt}{red!65},
    sharp corners,
    fonttitle=\bfseries, 
    breakable,
    before upper={\parindent15pt\noindent}}{box}
\newtcbtheorem[number within=chapter, use counter from=infobox]{perspectivebox}{Box}{
    enhanced,
    boxrule=0pt,
    %colback=red!5,
    %colframe=red!5,
    colback=white,
    colframe=white,
    coltitle=red!50,
    borderline west={4pt}{0pt}{red!65},
    sharp corners,
    fonttitle=\bfseries, 
    breakable,
    before upper={\parindent15pt\noindent}}{box}


% Displaying texts in bookmarkers

\pdfstringdefDisableCommands{%
  \def\\{}%
  \def\ce#1{<#1>}%
}

\pdfstringdefDisableCommands{%
  \def\texttt#1{<#1>}%
  \def\mathbb#1{#1}%
}
\pdfstringdefDisableCommands{\def\eqref#1{(\ref{#1})}}

\makeatletter
\pdfstringdefDisableCommands{\let\HyPsd@CatcodeWarning\@gobble}
\makeatother

\newenvironment{shelldisplay}{\begin{lstlisting}}{\end{lstlisting}}

\newcommand*{\citesec}[1]{\S~{#1}}
\newcommand*{\citechap}[1]{Ch.~{#1}}
\newcommand*{\citefig}[1]{Fig.~{#1}}
\newcommand*{\citetable}[1]{Table~{#1}}
\newcommand*{\citepage}[1]{p.~{#1}}
\newcommand*{\citepages}[1]{pp.~{#1}}
\newcommand*{\citefootnote}[1]{fn.~{#1}}
\newcommand*{\citechapsec}[2]{\citechap{#1}.\citesec{#2}}
\newcommand{\literature}[1]{\textit{#1}}

\newrefformat{sec}{\citesec{\ref{#1}}}
\newrefformat{fig}{\citefig{\ref{#1}}}
\newrefformat{tbl}{\citetable{\ref{#1}}}
\newrefformat{chap}{\citechap{\ref{#1}}}
\newrefformat{fn}{\citefootnote{\ref{#1}}}
\newrefformat{box}{Box~\ref{#1}}
\newrefformat{ex}{\ref{#1}}

\newcommand{\shortcode}[1]{\texttt{#1}}

\lstset{style = python}

% Make subsubsection labeled
\setcounter{secnumdepth}{4}
\setcounter{tocdepth}{4}

\newcommand*{\abinitio}{\textit{ab initio}}
\newcommand*{\kB}{k_{\text{B}}}
\newcommand*{\epsr}{\epsilon_{\text{r}}}

\title{Modern neural networks}
\author{Jinyuan Wu}

\begin{document}

\maketitle

\chapter{Framework}

\section{Modern neural networks used in deep learning, compared with other algorithms}
\label{sec:modern-network}

Historically, incredibly diverse architectures have been proposed within the family of neural network algorithms.
What proliferates in modern machine learning belongs to a small subset of these algorithms.
What set them apart from ``traditional'' neural network algorithms
(i.e. algorithms before 2010s) include several aspects.

\paragraph*{How they make predications}
All modern neural networks are feedforward. That's to say -- 
\begin{itemize}
    \item The output of a neuron is a real number%
    \footnote{
        In actual implementations, often a reduced precision float number.
    }, not a time sequence or a spark or something else.
    There are no loops in the structure of the neural network:
    prediction therefore is \concept{forward propagation},
    because data propagates from the input neurons to the output neurons.
    
    \item What happens in a neuron is as simple as possible:
    usually the output is calculated by 
    passing the inputs weighted by weight factors plus a bias
    to the activation function.
    The weights and the biases can be adjusted;
    the activation function contains no learnable parameters.

    \item Since a neuron is very simple, a network contains \emph{layers} of neuron:
    a feedforward network consisting of several layers -- a \concept{multi-layer percepton (MLP)} can therefore be represented as 
    \begin{equation}
        \mathrm{MLP}(\vb{x}) = f_N(\vb{W}_N \cdot (\cdots (\vb{W}_2 \cdot f_1(\vb{W}_1 \cdot \vb{x} + \vb{b}_1) + \vb{b}_2) \cdots) + \vb{b}_N),
        \label{eq:mlp}
    \end{equation}
    where at layer $i$, $\vb{W}_i$ is a matrix containing weights of all neurons at that layer, $\vb{b}_i$ contains the biases of neurons at that layer, and $f_i (\cdot)$ is the activation function that acts element-wise on a vector.
    In actual calculations, the concept of \emph{neuron} does \emph{not} explicitly appear:
    we work solely on vectors and matrices.

    \item Several simple non-linear operations, like element-wise multiplication,
    are often used in modern neural networks.
    By the universal approximation theorem, though, they can be reasonably well approximated by a ``pristine'' feedforward neural network within a given range of inputs.
    
    \item Modern neural networks have mechanisms to accept inputs with variable sizes,
    by a sliding window (as in CNNs), by letting the network to have a latent state,
    which is a part of both the output and the input, or other methods.
    Basically this means we are allowed to use bounded \shortcode{for} loops,
    although in practice this is usually done by \emph{tensor} operations (see \prettyref{sec:tensor}).
    The existence of these mechanisms essentially enables us to dynamically increase or decrease the number of neurons to match the input size:
    this further render the neuron-based representation of modern neural networks somehow unnatural.
\end{itemize}

\paragraph*{Universal approximation}
A modern neural network can approximate any function that is ``well-behaved''.
This conclusion -- actually a family of conclusions of this type --
is known as the \concept{Universal Approxmation Theorem(s)}.
It should be noted that universal approximation is not directly related to Turing completeness.
A theoretical neural network, which works on real numbers,
is \emph{beyond} the capacity of all Turing machines,
because almost all real numbers are not computable.
On the other hand, existing studies on universal approximation generally focus on
neural networks with a fixed number of input variables,
which goes against the usual notion of Turing completeness,
which in theory allows an infinite input.

Therefore, the function classes determined by neural networks and by Turing machines
are both elegant, natural definitions of what computation means,
but they are not the same, and nor is there a subset relation between the two.

In reality, the memory of a computer is never infinite,
and neural networks always have some ``noises'' (either from float number errors or, in certain alternative computing platforms, from thermal noises).
The latter makes neural networks strictly within the class of Turing computable functions,
while the former is less substantial in the sense that
if we can increase the memory allocated to a program in an unbounded manner 
without drastically change the programs 
(for instance, we can write C programs without utilizing the length of the pointer type),
then the concept of Turing completeness is still meaningful.
In this practical sense, universal approximation is a concept that is strictly weaker than Turing completeness.

Whether certain neural network architectures, accompanied by the aforementioned sliding window/latent state/whatever mechanism, can achieve Turing completeness,
is an interesting and not completely explored question.
For instance we may want to know the grammatical complexity class a Transformer is able to recognize and, if it clearly does not cover the attested grammatical complexity of natural languages, then why it is so successfully.

We finally note that universal approximation is about \emph{existence}.
How to train a model is another problem (also see \prettyref{sec:theoretical-aspects}). Below we outline the current training paradigm.

\paragraph*{Training}

Modern neural networks are all trained by \concept{backpropagation}.
Therefore they are also known as \concept{BP neural networks}.
This means we choose a \concept{loss function} (which can be derived by statistical estimations,
like the maximum a posteriori estimation; see \prettyref{sec:framework.statistics}),
and use it to measure the difference between the training dataset and the prediction.
Then we take the \emph{gradient} of the loss function with respect to parameters in the model,
and update the parameters according to the gradient.
This is known as backpropagation because when taking the gradient of the loss function,
we need to take the gradient of e.g. \eqref{eq:mlp},
and this means we first take the gradient of the last layer,
then the second last layer, etc.,
and the propagation of the derivative operation is backwards.

Modern neural networks are trained in \concept{batches},
which enables parallelism and avoids overfitting,
because if samples within a batch are randomly chosen,
then all kinds of noises are seen by the model.

\paragraph*{Deep learning}
Modern neural networks are usually \emph{deep}:
they are much larger than traditional models and usually contain multiple layers,
and are data thirsty.
Modern neural networks therefore are almost identical to what we call \concept{deep learning}.

It should be noted that in theory, ``deep learning'' by definition means a machine learning paradigm that is large, complex, and scales up well.
It is in theory possible to have non-neural network deep learning methods.
An instance is deep forest, or in other words, a decision tree ensemble;
or consider a large, structured probabilistic graphic model.
Currently, these approaches are less popular than the aforementioned deep neural networks
-- but maybe this will change one day.
In this note, the term \emph{modern neural network} and \emph{deep learning model} will be used interchangeably.

\section{``Tensors'' and PyTorch}\label{sec:tensor}

\paragraph*{Motivation: ``variables'' in algebra and calculus}
In doing algebra and calculus, it is handy to not explicitly specify the input and output of a function, but do the follows:
we define some \emph{independent variables} like $x, y, z$,
and then write $w = f(x, y, z)$, and say that ``$w$ is a function of $x, y, z$''.
We expect $w$ to \emph{remember} that it is defined in terms of $x, y, z$:
this means we can then talk about the gradient of $z$ with respect to $x$ or $y$,
or talk about the value of $w$ when $x, y, z$ are assigned certain values,
and then evaluate the gradient of $w$ ``at the point $(x, y, z)$''.

For someone who does more formalized math,
what is done here is rather weird, because variables, in a logical sense,
are either bound by quantifiers ($\forall, \exists$),
or are place holders when doing $\forall$- or $\exists$-introduction and elimination:
in this way if $x, y, z$ are declared to be, say, real numbers,
when we write $z = f(x, y, z)$,
$z$ is also just a real number:
we're not supposed to take its derivative or do other things like that!
In a formalized context, we should only talk about the gradient of $f$ with respect to its first or second or third variable,
or the value of $f$ at a certain point:
$x, y ,z$ are merely placeholders used in expressions like $f(x, y, z) = \dots$.

One way to make sense of the concept of variables in informal math is 
to collect the definition of these independent variables at one place,
and say that we have a big ``configuration space'',
on which $x, y$ and $z$ are mappings from points in that space to a number.
This is how we define \concept{random variables}:
in formalized math, a random variable essentially is a function that maps an ``event'' (i.e. an element of a sample space) to a subset of a measurable space (specifying the internal makeup of the event),
which is then measured by a metric (i.e. how probability is define).
This is what is actually going on when we talk about the probability $p(X=x)$ that a random variable $X$ takes the value $x$.
Still this formalized definition doesn't stop us from imagining that a random variable is just a ``variable'' -- just something out there -- that changes itself in a crazy way.

The term \term{variable} may refer to what is to be solved in an equation.
This is similar to the definition of variable in formal logic,
with one important difference:
in everyday math we often treat an equation as an \emph{object}.
Thus an equation can't be a sentence in logic,
because no one defines predicates of sentences.
Instead we can imagine that we formalize the semantics of Mathematica in first-order logic,
and a \emph{variable} in an equation, together with the equation in question,
are objects in that formalized Mathematica,
i.e. an embedded language defined in first order logic
just to make discussion everyday math easier,
without the necessity of being allowed to only talk about zero points of functions.

And finally, a \term{variable} is a concept in procedural programming.
This meaning of variable is completely different from all the other meanings of variable above:
when we write \shortcode{w = f(x, y, z)} in procedural programming,
changing \shortcode{x} doesn't alternate \shortcode{w}.

It is clear that variables in formal logic are a subset of variables in equations in everyday algebra.
On the other hand, with operation overloading,
we can make \shortcode{w = f(x, y, z)} to mean $w = f(x, y, z)$ in the algebraic sense.
A library doing this is essentially making \concept{computational graphs}
-- which is the topic discussed below.

\paragraph*{Introducing tensors}
A \concept{tensor}, in the context of deep learning,
is just a multidimensional \emph{variable} in function definitions in everyday algebra and calculus:
when we write \shortcode{a = b + c},
where \shortcode{a}, \shortcode{b} and \shortcode{c} are tensors,
\shortcode{a} \emph{remembers} that it comes from \shortcode{b} and \shortcode{c}.
This means we can take the derivative of \shortcode{a} with respect to \shortcode{b} and \shortcode{c}.

Various tensor libraries have been available for quite a while.
In this note we are going to use PyTorch.
In PyTorch, for instance, we can do the follows:

\begin{lstlisting}
import torch

x = torch.tensor(2.0, requires_grad=True)
y = torch.tensor(3.0, requires_grad=True)

u = 3*x + y**2
v = x**3

u.backward()
print(x.grad, y.grad) # Output: tensor(3.) tensor(6.)

# By default the gradient recorded on a PyTorch tensor accumulates
# to enable certain optimization tricks.
# we need to call .zero_() to zero out the gradient. 
x.grad.zero_()
y.grad.zero_()

v.backward()
print(x.grad, y.grad) # tensor(12.) tensor(0.)

\end{lstlisting}

What happens here is that we define 
\[
    u = 3x + y^2, \quad v = x^3,
\]
where $x = 2$, $y = 3$, and therefore 
\[
    \pdv{u}{x} = 3, \quad \pdv{u}{y} = 2y = 6, \quad \pdv{v}{x} = 3x^2 = 12, \quad \pdv{v}{y} = 0.
\]
Both \shortcode{u} and \shortcode{v} remember how they were defined:
this is known as \concept{computational graphs}.
When backpropagation starts at a tensor, say \shortcode{u},
the \shortcode{grad} attribute of all tensors that contribute to the calculation of \shortcode{u}
will be updated.
Sometimes we don't want backpropagation to go to certain tensors,
like the tensors containing training data:
this can be done by removing the \shortcode{requires_grad} parameter in the definition of the tensor.
PyTorch also enables us to define multidimensional arrays,
which then can be used to construct deep learning models that can be trained by backpropagation.

In theory, functions in programming languages can replace by computational graphs.
Probably to make the coding style more uniform,
PyTorch does not provide a straightforward way to reassign values to tensors
and then automatically trigger a forward propagation:
if we want to reassign values to \shortcode{x} or \shortcode{y},
the most idiomatic way to do so is the follows:

\begin{lstlisting}
import torch 

def calculate_u(x, y):
    return 3*x + y**2

x = torch.tensor(2.0, requires_grad=True)
y = torch.tensor(3.0, requires_grad=True)
u = calculate_u(x, y)
u.backward()
print(x.grad, y.grad) # Output: tensor(3.) tensor(6.)

x = torch.tensor(3.0, requires_grad=True)
y = torch.tensor(4.0, requires_grad=True)
u = calculate_u(x, y)
u.backward()
print(x.grad, y.grad) # Output: tensor(3.) tensor(8.)
\end{lstlisting}

As the listing demonstrates, if the value of \shortcode{x} or \shortcode{y} needs to change,
it is probably a good idea to seal the computational graph into a regular Python function,
and create a new tensor whenever we want a reassignment.
This makes sure that doing deep learning with PyTorch is still like writing regular Python code,
not working with an embedded DSL.
Computational graphs in PyTorch also do not support taking second-order derivatives:
therefore optimization can only rely on the first-order derivative of the model with respect to the model parameters.

\paragraph*{Defining a module}
A model, as is said above, is a function that takes in a tensor (by convention, the first dimension being the batch dimension) and outputs a tensor,
and it contains many learnable parameters.
A model can then be put into another model,
and when doing backpropagation,
\emph{all} parameters of the latter -- including those of the first model -- 
should be exposed to the optimizer.
The \shortcode{torch.nn.Module} is a base class that makes registering parameters easy:
it also provides a uniform coding style for developers to define models.
Below is an example of how to define a PyTorch model:

\begin{lstlisting}
from torch import nn 

class FashionMINSTClassifier(nn.Module):
def __init__(self):
    super().__init__()
    
    self.flatten = nn.Flatten()
    self.linear_relu_stack = nn.Sequential(
        nn.Linear(28*28, 512),
        nn.ReLU(),
        nn.Linear(512, 512),
        nn.ReLU(),
        nn.Linear(512, 10)
    )
    
def forward(self, x):
    x = self.flatten(x)
    logits = self.linear_relu_stack(x)
    return logits
\end{lstlisting}

A subclass of \shortcode{nn.Module} is callable:
predication therefore takes the form of \shortcode{model(X)}.
Assigning a \shortcode{nn.Module} object -- here it is \shortcode{linear_relu_stack} -- to \shortcode{self} triggers the \shortcode{__setattr__} method,
which, according to the definition in the \shortcode{nn.Module} class,
registers all parameters of the \shortcode{nn.Module} object in question.
Note that PyTorch has a \shortcode{nn.Flatten} network component,
which flattens all dimensions of the input besides the first dimension,
intended as the batch dimension.
\shortcode{nn.Sequential} is a subclass of \shortcode{nn.Module},
which assembles several models into a chain.
We can also define a reusable sequential model by writing

\begin{lstlisting}
class LinearReluStack(nn.Sequential):
    def __init__(self):
        super().__init__(
            nn.Linear(28*28, 512),
            nn.ReLU(),
            nn.Linear(512, 512),
            nn.ReLU(),
            nn.Linear(512, 10)
        )
\end{lstlisting}

\paragraph*{Hardware acceleration}
Since as is shown in \eqref{eq:mlp}, a neural network model involves a lot of linear algebra operations,
both prediction and training can be drastically accelerated by GPUs.
PyTorch tensors and \shortcode{nn.Module} objects can be sent to GPUs
by the \shortcode{.to("cuda")} method and sent back to the host by \shortcode{.to("cpu")}.
Most users will not need to write CUDA kernels themselves.

\section{What is approximated by neural networks}\label{sec:framework.statistics}

As is implied in \prettyref{sec:modern-network},
modern neural networks are not conventional statistical models:
their inductive biases are not explicitly endowed to them by making assumptions about the statistical structure of the problem.
Still by treating the whole neural network model as a whole as a statistical model,
we can have some insights on the training target. 

Machine learning, from the perspective of statistics, is parameter estimation.
There are two main school of thoughts in parameter estimation.
\concept{Frequentists} consider the parameters in the model as unknown constants,
and the training dataset they consider a sampling of a ``real'', physical distribution.
Parameter estimation in the frequentist approach is a \emph{deterministic} algorithm
from a sampling to a deterministic estimation of the parameters.
How fast the estimation of the parameters given by the algorithm
converges to the true parameters as the number of samples increase
and the distribution of the error (which is also a random variable,
because the probabilistic distribution of the sample is a random variable)
decide how good a parameter estimation algorithm is.
Frequentist parameter estimation often employs the method of \concept{maximum likelihood estimation (MLE)}%
\footnote{
    Here we are using a rather sloppy notation.
    As $x$ is essentially a random variable,
    we should have written something like $p(x = x_0)$, where $x_0$ is a deterministic quantity.
    Or we can write $p_{x}(x_0)$.
    The standard in statistics is actually $p(X = x)$,
    where $X$ is the random variable and $x$ is one of its possible values.
    In machine learning, though, we rarely need to make such a distinction.
}
\begin{equation}
    \hat{\theta} = \argmax_\theta p_\theta(x) = \argmax_\theta \log p_\theta(x),
    \label{eq:mle}
\end{equation}
where $x$ is the observation, and we have modeled the hidden distribution of the observation as a parameterized function $p_\theta(x)$.

In \concept{Bayesian statistics}, the parameters are \emph{also} considered as random variables.
This is because in Bayesian statistics, probability is \emph{subjective}.
On how reasoning on uncertain facts naturally results in probability theory, see e.g. Ref.~\cite{jaynes2003probability};
For some reason, the same mathematical discipline seems to be the underlying framework for both reasoning about uncertainty and for modeling a physical system interacting with a ``thermal bath''.
In Bayesian statistics, parameter estimation results in a likely distribution of $\theta$,
and not just a single estimation $\hat{\theta}$.
What is often done is the follows:
\begin{equation}
    p(\theta | x) = \frac{p(\theta)}{p(x)} p(x | \theta) = \frac{p(\theta) p(x | \theta)}{\sum_{\theta'} p(\theta') p(x | \theta')},
    \label{eq:bayesian}
\end{equation}
where we first assume a \concept{prior distribution} of $p(\theta)$,
and build a model by assuming a form of $p(x | \theta)$,
and then calculate $p(\theta | x)$ according the equation above.
When $\theta$ is continuous, $p(\theta)$ and $p(\theta | x)$ are actually probability densities,
and $p(\theta) \dd{\theta}$ is the probability for the random variable $\theta$ to fall between the value $\theta$ and $\theta + \dd{\theta}$.
Since the $\dd{\theta}$ factors on both sides of \eqref{eq:bayesian}, we get 
\begin{equation}
    p(\theta | x) = \frac{p(\theta) p(x | \theta)}{\int p(\theta') p(x | \theta') \dd{\theta'}}.
\end{equation}
The prior can be viewed as the $p(\theta | x)$ from last inference,
provided that the sample used in the last inference and the sample used in this inference
are independent to each other.
Indeed, we can prove that 
\begin{equation}
    p(\theta | x_2, x_1) = \frac{p(\theta | x_1)}{p(x_2)} p(x_2 | \theta, x_1),
\end{equation}
if and only if $p(x_1) p(x_2) = p(x_1 x_2)$,
and therefore if the prior if determined by past experiences (i.e. $p(\theta | x_1)$),
then the LHS of \eqref{eq:bayesian} is determined by both past experiences and this sample.

The sum or integral in \eqref{eq:bayesian} usually cannot be evaluated analytically,
and techniques like Monte Carlo sampling have to be used.
As a compromise -- which is what is frequently done when $p(x | \theta)$ is modeled by a deep learning model
-- we may assume that $p(\theta | x)$ has a sharp peak,
and this results in the \concept{maximum a posteriori estimation (MAP)}
\begin{equation}
    \hat{\theta} = \argmax_\theta p(\theta | x) = \argmax_\theta (\log p(x | \theta) + \log p(\theta)).
    \label{eq:map}
\end{equation}

When the prior is trivial, which means it's the same for all $\theta$,
MAP reduces to MLE \eqref{eq:mle}.
Some thing similar to \eqref{eq:map} can also be derived from \eqref{eq:mle}:
from the perspective of the frequentists,
if we stipulate that certain values of $\theta$ are not allowed in our model in \eqref{eq:mle}
(probably because otherwise the model is to complex),
then we can add a Lagrangian multiplier term (called \concept{regularization}) into the minimization problem \eqref{eq:mle}:
\begin{equation}
    \hat{\theta} = \argmax_{\|\theta\| \leq C} p_\theta(x) = \argmax_{\theta, \lambda} (p_\theta(x) + \underbrace{\lambda (\norm*{\theta} - C)}_{\text{regularization}}).
    \label{eq:mle-lagrangian}
\end{equation}
Here $\|\theta\|$ is a way to measure the complexity of the model.
The equivalence is a consequence of Karush-Kuhn-Tucker conditions.
After maximization finishes, $\lambda$ in \eqref{eq:mle-lagrangian} is zero if $\norm*{\theta} < C$,
and is non-zero when $\norm*{\theta} = C$.
Still, usually we do not know what $C$ actually is,
and we often need to try different values of $C$ to see how well the model works.
This is equivalent to fixing $\lambda$ to a given value in \eqref{eq:mle-lagrangian},
and do maximization with respect to $\theta$ only.
What is done above is equivalent to setting
\begin{equation}
    p(\theta) \propto \ee^{\lambda \norm*{\theta}}
\end{equation}
in \eqref{eq:map}.

Therefore, it seems that MAP in Bayesian statistics (with or without a non-trivial prior)
is equivalent to MLE in frequentist statistics (with or without regularization).
Since MAP is just a specific case of Bayesian statistics,
it seems to follow that frequentist parameter estimation can be seen as a subset of Bayesian parameter estimation.
There are actually more correspondences between the two.
The \emph{estimated} parameter is a random variable,
from the perspective of frequentists,
and it gets its value \emph{after} a sampling.
The frequentist distribution of the parameter calculated on a randomly chosen sample
looks quite like the a posteriori distribution of $\theta$ in Bayesian parameter estimation.
This similarity is formalized by the Bernstein–von Mises theorem and similar ``bridging'' theorems.

So how is this related to neural networks?
Note that \eqref{eq:mle-lagrangian} is a maximization problem -- or a minimization problem,
if we work with $- p_\theta(x)$.
So when the forms of $p_\theta(x)$ and $\norm*{\theta}$ are given 
-- in the language of Bayesian parameter estimation, when the forms of $p(x | \theta)$ and $p(\theta)$ are given --
we have already gotten an optimization target function.
Basically this is how loss functions are derived in \prettyref{chap:loss}.

\section{Training}

\paragraph*{The PyTorch training API}
As is mentioned in \prettyref{sec:modern-network}, deep learning models are to be trained by gradient-based optimization,
the simplest being the \concept{gradient descent}
\begin{equation}
    \theta \leftarrow \theta - \eta \grad_\theta \mathcal{R}.
    \label{eq:sgd-example}
\end{equation}
A training procedure usually contains a number of \concept{epoches},
in each of which, we use up all data available in the training dataset.
As is said above in \prettyref{sec:modern-network}, 
we train neural networks in batches, so an epoch consists of several batches.
When the loss function is chosen, 
an epoch usually look like this:

\begin{lstlisting}
optimizer = torch.optim.SGD(model.parameters(), lr=1e-3)
# For each epoch:
for (X, y) in dataloader: # a torch.utils.data.DataLoader object
    model.train() # Entering training mode 
    X = X.to(device)
    y = y.to(device)
    loss_Xy = loss(model(X), y)
    loss_Xy.backward()     # backpropagation
    optimizer.step()       # Apply optimizer once 
    optimizer.zero_grad()  # Avoid gradient accumulation
\end{lstlisting}

Again PyTorch provides us with a very unified API:
we calculate the loss, and the result is a tensor which remembers what parameters are used in the calculation,
and then we do backpropagation.
The optimizer knows all parameters we want to update,
and it changes their values according to the gradients recorded in their \shortcode{grad} attributes
and then zeroes out the recorded gradients.

It is not uncommon to have a \concept{validation dataset} besides the training dataset
as an unbiased evaluation of how well the model performs:
if, as training goes, the loss calculated on the training dataset drops,
but the loss calculated on the validation dataset doesn't,
then usually we're overfitting the data.

As for how to prepare the data,
PyTorch has two classes: \shortcode{Dataset} and \shortcode{DataLoader}.
The first should be the parent class of all user defined dataset classes:
a dataset object is expected to be iterable,
and at each iteration, it should provide a tuple
containing the input and the expected output,
\emph{without} a batch dimension.
A \shortcode{DataLoader} reads a \shortcode{Dataset} object and is also iterable:
at each iteration, it is expected to give a tuple 
containing a \emph{batch} of inputs and a \emph{batch} of expected outputs,
both of which \emph{have} an additional dimension at the first, i.e. the batch dimension.
It is encouraged to define subclasses of \shortcode{Dataset}
for different types of training datasets.

\paragraph*{Batching in the perspective of parameter estimation}
It is clear that when we have multiple batches -- and not a single, giant batch -- and use \emph{different} batches in different iterations of \eqref{eq:sgd-example},
we are \emph{not} doing any of the parameter estimations in \prettyref{sec:framework.statistics},
the latter requires the training dataset to be the same in the training process.
Still, we can treat each batch as the ``ideal'' sample with some variations,
and since usually the training dataset is shuffled before training (when creating \shortcode{dataloader}),
we are essentially doing parameter estimation (as is outlined in \prettyref{sec:framework.statistics})
with a gradient-based optimizer (e.g. the one described in \eqref{eq:sgd-example}),
and at each iteration, we have \emph{random noises} corresponding to the makeup of the batch being used.
Batching therefore makes training more robust as it avoids the model falling into a local minimum.
How to choose the batch size however is not clear from a statistical perspective, and usually needs empirical testing to decide.

\section{Theoretical aspects}\label{sec:theoretical-aspects}

The \concept{no free lunch theorem} states it is not possible to have a one-size-fit-all machine learning paradigm.
The wild success of deep learning seems to go against it -- or if we sit down and take a deep breath, it seems to indicate that most machine learning problems we are interested in have something deep in common.
This observation has important consequences.

Large language models (LLM) do not have any explicit biases towards possible natural languages.
Does this mean that Chomsky is wrong,
and natural languages are completely results of domain-general cognitive abilities,
and there is no dedicated language faculty in human brain?%
\footnote{
    Note that the success of large language models is a challenge to \emph{all} schools of linguistics,
    and also old-styled NLP (yeah, including not-so-large neural models).
    Proponents of usage-based approaches to linguistics often claim that the success of LLM 
    demonstrate that they are right in emphasizing statistical learning in language acquisition.
    Yet LLMs do not have explicit structural biases for statistical learning as we know it either.
    LLMs know no X-bar syntactic trees, and neither do they know anything about HPSG unification, lambda expressions, constructions or embodied cognition.
    And statistical learning models designed by usage-based linguists suck
    just like minimalist grammars do.
}
Currently nothing substantial can be said on this question:
``human linguisticality'' (to avoid the term Universal Grammar) is likely beyond context-free grammars (see e.g. Swiss German's cross-serial dependencies),
and yet we do not even know whether typical LLMs can learn all context-free languages.
It is not impossible that the relation between neural networks and Turing machines
is also the relation between LLMs and human linguisticality:
there is evidence for LLMs' ability to learn impossible languages better than humans do.%
\footnote{
    Designing the correct metric and interpreting the results is extremely hard here.
    arXiv 2401.06416 observes that LLMs also seem to have difficulties learning languages impossible for humans.
    The impossible language used in the paper has a count-based grammar,
    which shifts the English verbal inflection rightwards by four words.
    The problem with this test, however, is that it is comparing adjacency (i.e. inflectional morphemes being attached to the verb stem) and count-based non-adjacency,
    while Chomskyan generative syntax focuses on more subtle differences,
    like \emph{constituency-based non-adjacency} vs. \emph{count-based non-adjacency}
    (see arXiv 2410.12271).
    arXiv 2502.12317 shows that typologically implausible languages can be learned by LLMs at a slower pace.
    They take this fact to indicate that typological preferences can be explained by domain-general factors.
    Yet if LLMs indeed \emph{learn} the impossible language at the end,
    while certain typologically implausible languages turn out to be \emph{impossible} and \emph{cannot} be learned by humans,
    then the conclusion of the paper should be drastically revised:
    it means the \emph{competence} of LLMs exceed that of humans in certain aspects,
    although their performances show certain similar patterns.
    Similarly, arXiv 2502.18795 demonstrates that human learners show a stronger dislike towards unattested word orders than LLMs,
    although the latter also show weak inductive biases against them.
    (The paper does prove one thing: LLMs like constituency.
    A rather ironic situation is that certain radical usage-based schools actually do not think constituencies exist.)
    Large Language Models and the Argument From the Poverty of the Stimulus by Nur Lan, Emmanuel Chemla, and Roni Katzir also shows that 
    how an initial success story of a LLM learning a subtle syntactic phenomenon
    (and thus possibly refuting the Poverty of Stimulus argument)
    can be turned into a story \emph{supporting} PoS by pointing out that the LLM
    fails at some cleverly constructed (but still simple) examples,
    and how can more data (\emph{ample} stimulus!) partially cure the problem.
    LLMs learn certain impossible languages better than us do,
    and they learn possible languages worse than we do:
    this is a demonstrated fact,
    and whether this means anything to psycholinguistics,
    no one really knows.
}
And the fact that LLMs do have certain behaviors comparable to human psycholinguistics
is quite hard to interpret:
it has been proposed, for instance, that the human brain does store common syntactic structures, and the full combinatorics of grammar is not always on,
and the whole system is related to the working memory or other domain general systems (see https://facultyoflanguage.blogspot.com/2016/09/brains-and-syntax-part-2.html).
This picture of human linguisticality is consistent with LLMs' success
and also consistent with the divergence of LLMs from human psycholinguistics:
this theory is no more complicated than the ``LLM=human psycholinguistics'' theory (which struggles to explain certain fMRI phenomena) or the constructionist theory.
So does the success of LLMs tell us anything about human languages?
Probably, but no one really knows what it is.
And finally, note that LLMs are probably not that domain general, after all,
considering that devils hide in details of fine tuning and minor modifications of the model architecture.

\chapter{Derivation of loss functions}\label{chap:loss}

\section{Regression}

\subsection{Overall theory}

Let $\theta$ be the collection of all parameters in the model.
The relation between the input $x$ and the output $y$ generally takes the following form
\begin{equation}
    y = f(x; \theta) + \epsilon(x),
\end{equation}
where $\epsilon$ is a random variable whose mean vanishes.
In physics, for example, $\epsilon$ may come from Langevin noise.

We now work in the Bayesian approach and assume that $\theta$ also has a distribution.
Suppose $(\vb{x}, \vb{y})$ is a sampling of $x$ and $y$.
It is straightforward to verify that
\begin{equation}
    p(\theta|\vb{x}, \vb{y}) = \frac{p(\theta)}{p(\vb{x}, \vb{y})} p(\vb{x}, \vb{y} | \theta).
\end{equation}
By MAP, we have 
\begin{equation}
    \begin{aligned}
        \hat{\theta} &= \argmax_\theta p(\theta | \vb{x}, \vb{y}) \\
        &= \argmax_\theta (\log p(\theta) - \underbrace{\log p(\vb{x}, \vb{y})}_{\text{doesn't  depend on $\theta$}} + \log p(\vb{x}, \vb{y} | \theta)) \\
        &= \argmax_\theta (\log p(\vb{x},  \vb{y} | \theta) + \underbrace{\log p(\theta)}_{\text{regularization}}) .
    \end{aligned}
\end{equation}
Here regularization naturally appears as a consequence of the prior distribution of $\theta$.
The form of $p(\vb{x}, \vb{y} | \theta)$ is determined by $\epsilon$,
under the assumption that different samples are independent to each other: TODO: why?
\begin{equation}
    p(\vb{x}, \vb{y} | \theta) = \prod_i p(\epsilon = y_i - f(x_i; \theta)).
\end{equation}
Therefore 
\begin{equation}
    \hat{\theta} = \argmax_\theta \left( \log \sum_i p(\epsilon = y_i- f(x_i; \theta)) + \log p(\theta) \right).
\end{equation}

People with a strong background in physics may tend to model the relation between $\vb{x}$ and $\vb{y}$ as 
\begin{equation}
    p(\theta | \vb{x}, \vb{y}) = \frac{p(\theta)}{p(\vb{x}, \vb{y})} p(\vb{x} | \theta) p(\vb{y} | \vb{x}, \theta).
\end{equation}
The problem is that if $\theta$ are parameters in a \emph{physical} process,
then $\vb{x}$ \emph{of course} depends on $\theta$:
some scenarios are simply easier for experimentalists to handle.
Now if we still want to do MAP, we immediately realize we are faced with difficulties: we get 
\begin{equation}
    \begin{aligned}
        \hat{\theta} &= \argmax_\theta p(\theta | \vb{x}, \vb{y}) \\
        &= \argmax_\theta (\log p(\theta) + \underbrace{\log p(\vb{x} | \theta)}_{\text{???}} + \log p(\vb{y} | \vb{x}, \theta)) .
    \end{aligned}
\end{equation}

\section{KL divergence}

\section{Variational Autoencoder}

Suppose we want to compress a bunch of data into a much smaller \concept{latent} space.
This involves an encoder from $x$ to $z$, and a decoder from $z$ to $x$.
We do not know what exactly $z$.
Still, if we have an approximate form of $p(x)$,
we can do a maximal likelihood estimation.

\subsection{Derivation of the ELBO}

A \concept{variational autoencoder (VAE)} is a compressor which maximizes the \emph{lower bound}
(hence the term ``variational'') of $p(x)$ derived from the non-negativity of the KL divergence.
Suppose we approximate $p(z|x)$ by an \concept{encoder} $q_\theta(z | x)$,
and we approximate $p(x|z)$ by an \concept{decoder} $p_\phi(x | z)$.
Here we adopt a frequentist perspective and treat the parameters $\theta, \phi$ as unknown constants.
Of course, $p(z | x)$ can also be approximated using 
\begin{equation}
    p_\phi(z | x) \coloneqq \frac{p(z)}{p(x)} p_\phi(x | z).
\end{equation}
The two estimations of the real $p(z | x)$ should be consistent.
We calculate their KL divergence, and note that because the KL divergence is always non-negative,
\begin{equation}
    \begin{aligned}
        D_{\text{KL}}(q_\theta(z|x) || p_\phi(z|x)) &= - \int q_\theta(z|x) \log \frac{p_\phi(z|x)}{q_\theta(z|x)} \dd{z} \\
        &= - \int q_\theta(z|x) \log \frac{p_\phi(x|z) p(z)}{q_\theta(z|x) p(x)} \dd{z} \geq 0.
    \end{aligned}
\end{equation} 
Equality occurs when the two distributions are indeed the same.
Manipulating this inequality we get 
\[
    \underbrace{- \int q_\theta(z|x) \log \frac{p(z)}{q_\theta(z|x)} \dd{z}}_{D_{\text{KL}}(q_\theta(z|x) ||p(z))} - \underbrace{\int q_\theta(z|x)  \log p_\phi(x|z) \dd{z}}_{\mathbb{E}_{z\sim q_\theta(z|x)}[\log p_\phi(x|z)]} + \log p(x) \geq 0,
\]
and therefore 
\begin{equation}
    \log p(x) \geq \underbrace{- D_{\text{KL}}(q_\theta(z|x) ||p(z)) + \mathbb{E}_{z\sim q_\theta(z|x)}[\log p_\phi(x|z)]}_{\text{ELBO}}.
    \label{eq:loss.vae.gaussian.elbo}
\end{equation}
The equality occurs when the encoder and the decoder describe the same $p(x|z)$ distribution.
The right hand side is known as the \concept{evidence lower bound (ELBO)}.

When the neural network architecture used in $p_\phi(x|z)$ and $q_\theta(z|x)$ are flexible enough,
there exists a subset of possible $(\theta, \phi)$ where the inequality is almost an equality,
and in that region, maximizing the ELBO is equivalent to the maximal likelihood estimation.
Praying that our optimization method can bring us there,
we use the ELBO function as the optimization target.

\subsection{Closed form VAE loss: Gaussian latents}

We assume that 
\begin{equation}
    p(z) = \frac{1}{\sqrt{2\pi \sigma_p^2}} \exp(- \frac{(z - \mu_p)^2}{2 \sigma_p^2}),
    \label{eq:loss.vae.gaussian.z-prior}
\end{equation} 
and 
\begin{equation}
    q_\theta(z | x) = \frac{1}{\sqrt{2\pi \sigma_q^2}} \exp(- \frac{(z - \mu_q)^2}{2 \sigma_q^2}).
    \label{eq:loss.vae.gaussian.q-ansatz}
\end{equation}
The first assumption is a prior, as we assume that $\mu_p$ and $\sigma_p$ are all constants.
Whether it is a good prior can only be checked after we have the training dataset.
The second assumption is of particular importance,
because when making it, we have drastically simplify the structure of the encoder.
The encoder no longer takes both $x$ and $z$ as inputs;
rather, it takes in $x$ and returns $(\sigma_q^2, \mu_q)$
(which means $\mu_q$ and $\sigma_q^2$ are functions of $x$ -- and of course also $\theta$),
which is then given to \eqref{eq:loss.vae.gaussian.q-ansatz} to model the distribution of $z$.
This step is known as the \concept{reparameterization trick} (see arXiv 1312.6114).

The reparameterization trick distills the truly random part of $z$ (as predicted by $q_\theta(z|x)$),
and makes the rest of the model as deterministic as possible.
We note that the reparameterization trick is actually quite generic:
most ``reasonable'' probabilistic distributions in which $z$ is a continuous variable
cna be represented by reparameterization.

Besides making the model structure easier to write within a model neural network framework,
reparameterization solves one problem in training:
the calculation of the gradient of the ELBO.
If $p(z)$ and $q_\theta(z|x)$ were arbitrary probabilistic distributions,
the ELBO would have to be calculated using e.g. Monte-Carlo estimation,
and it is notoriously hard to calculate the gradient of a quantity estimated by Monte Carlo.
After reparameterization, we can write down a closed-form expression of the ELBO,
making training much easier.

The first term in the ELBO in \eqref{eq:loss.vae.gaussian.elbo} can be evaluated by definition
in a straightforward manner, although the calculation details are tedious (see arXiv 1907.08956).
We end up having
\begin{equation}
    - D_{\text{KL}}(q_\theta(z|x) || p(z)) = \log \frac{\sigma_q}{\sigma_p} - \frac{\sigma_q^2 + (\mu_q - \mu_p)^2}{2 \sigma_p^2} + \frac{1}{2}.
\end{equation}
We note that applying a linear transform to $z$ changes $\mu_p$ and $\sigma_p$ however we want,
and preserves the structure of \eqref{eq:loss.vae.gaussian.q-ansatz}.
Thus, without loss of generality, we set $\mu_p = 0$, $\sigma_p = 1$,
and 
\begin{equation}
    \begin{aligned}
        - D_{\text{KL}}(q_\theta(z|x) || p(z)) &= \log \sigma_q - \frac{\sigma_q^2 + \mu_q^2}{2} + \frac{1}{2} \\
        &= \frac{1}{2} \left(
            1 + \log \sigma_q^2 - \sigma_q^2 - \mu_q^2
        \right).
    \end{aligned}
\end{equation}

The second term in \eqref{eq:loss.vae.gaussian.elbo} depends on the architecture of the decoder,
which, given the assumptions \eqref{eq:loss.vae.gaussian.z-prior} and \eqref{eq:loss.vae.gaussian.q-ansatz},
cannot be chosen arbitrarily.
A common choice is to assume $p(x|z)$ should be gaussian, which is mentioned in the appendix C of arXiv 1312.6114 without much discussions.
In this way the second term in \eqref{eq:loss.vae.gaussian.elbo} takes a form similar to that of \eqref{eq:loss.vae.gaussian.q-ansatz},
and 
\begin{equation}
    \mathbb{E}_{z \sim q_\theta(z|x)} [\log p_\phi (x|z)] \sim \sum_z
    \text{normalization factor w.r.t. $z$} \cdot \|x - \mu_p(z)\|^2.
\end{equation}
It is not not unusual to ignore the variance of the normalization factor with respect to $z$.
We can interpret $\mu_p(z)$ (now meaning the $\mu$ of $p_\phi(x|z)$, not $p(z)$) as the output of the decoder with a given $z$,
which is $\mu_q(x)$ plus a random noise (of which $\sigma^2 = 1$, $\mu = 0$).

The ELBO now has a closed-form expression, which leads to the following loss function:
\begin{equation}
    \mathcal{L}^{\text{ELBO}}(\theta, \phi) \sim \frac{1}{2} \sum_x 
    \left(
        (1 + \log \sigma_q^2 - \sigma_q^2 - \mu_q^2) + \sum_\epsilon N(\epsilon) \| x - \mu_p(\mu_q(x) + \epsilon) \|^2
    \right),
    \label{eq:loss.vae.gaussian.loss}
\end{equation}
where we may call $\mu_q(x)$ the encoder in the inference mode and $\mu_p(z)$ the decoder.
The neural network architecture has three parts: $\sigma_q^2(x; \theta)$, $\mu_q(x; \theta)$,
and $\mu_p(z; \phi)$.
$N(\epsilon)$ is typically reduced to a single constant which is adjusted manually.
Batching of course applies to \eqref{eq:loss.vae.gaussian.loss},
and as the same $x$ will be fed to the model for many times during training,
the summation over $\epsilon$ can be merged into batching:
for a single batch, the loss function is 
\begin{equation}
    \mathcal{L}^{\text{ELBO}}_{\text{single batch}}(\theta, \phi) 
    = \frac{\beta}{2} \sum_{x}^{\text{batch}} (1 + \log \sigma_q^2 - \sigma_q^2 - \mu_q^2) 
    + \sum_x^{\text{batch}} \| x - \mu_p(\mu_q(x) + \epsilon) \|^2,
\end{equation}
where we have reduced $N(\epsilon)$ into a single hyperparameter $\beta$. 

\chapter{Array manipulation}

\printbibliography[title=References]

\end{document}