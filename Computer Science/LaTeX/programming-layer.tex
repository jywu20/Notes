\documentclass[hyperref, a4paper, 12pt]{article}

\usepackage{expl3}
\usepackage{geometry}
\usepackage{titling}
\usepackage{titlesec}
% No longer needed, since we will use enumitem package
% \usepackage{paralist}
\usepackage{enumitem}
\usepackage{footnote}
\usepackage{amsmath, amssymb, amsthm}
\usepackage{mathtools}
\usepackage{bbm}
\usepackage{graphicx}
\usepackage{subfigure}
\usepackage{simpler-wick}
\usepackage{physics}
\usepackage{tensor}
\usepackage{siunitx}
\usepackage[version=4]{mhchem}
\usepackage{tikz}
\usepackage{xcolor}
\usepackage{listings}
\usepackage{showexpl}
\usepackage{underscore}
\usepackage{autobreak}
\usepackage[ruled, vlined, linesnumbered]{algorithm2e}
\usepackage{nameref,zref-xr}
\zxrsetup{toltxlabel}
\usepackage[colorlinks,unicode, bookmarksnumbered]{hyperref} % , linkcolor=black, anchorcolor=black, citecolor=black, urlcolor=black, filecolor=black
\usepackage[most]{tcolorbox}
\usepackage[backend=bibtex,doi=false,isbn=false,url=false]{biblatex}
\addbibresource{probability.bib}
\usepackage{prettyref}

% Page style
\geometry{left=3.18cm,right=3.18cm,top=2.54cm,bottom=2.54cm}
\titlespacing{\paragraph}{0pt}{1pt}{10pt}[20pt]
\setlength{\droptitle}{-5em}

% More compact lists 
\setlist[itemize]{
    %itemindent=17pt, 
    %leftmargin=1pt,
    listparindent=\parindent,
    parsep=0pt,
}

\setlist[enumerate]{
    %itemindent=17pt, 
    %leftmargin=1pt,
    listparindent=\parindent,
    parsep=0pt,
}

% Math operators
\DeclareMathOperator{\timeorder}{\mathcal{T}}
\DeclareMathOperator{\diag}{diag}
\DeclareMathOperator{\legpoly}{P}
\DeclareMathOperator{\primevalue}{P}
\DeclareMathOperator{\sgn}{sgn}
\DeclareMathOperator{\res}{Res}
\newcommand*{\ii}{\mathrm{i}}
\newcommand*{\ee}{\mathrm{e}}
\newcommand*{\const}{\mathrm{const}}
\newcommand*{\suchthat}{\quad \text{s.t.} \quad}
\newcommand*{\argmin}{\arg\min}
\newcommand*{\argmax}{\arg\max}
\newcommand*{\normalorder}[1]{: #1 :}
\newcommand*{\pair}[1]{\langle #1 \rangle}
\newcommand*{\fd}[1]{\mathcal{D} #1}
\DeclareMathOperator{\bigO}{\mathcal{O}}

% TikZ setting
\usetikzlibrary{arrows,shapes,positioning}
\usetikzlibrary{arrows.meta}
\usetikzlibrary{decorations.markings}
\usetikzlibrary{calc}
\tikzstyle arrowstyle=[scale=1]
\tikzstyle directed=[postaction={decorate,decoration={markings,
    mark=at position .5 with {\arrow[arrowstyle]{stealth}}}}]
\tikzstyle ray=[directed, thick]
\tikzstyle dot=[anchor=base,fill,circle,inner sep=1pt]

% Algorithm setting
% Julia-style code
\SetKwIF{If}{ElseIf}{Else}{if}{}{elseif}{else}{end}
\SetKwFor{For}{for}{}{end}
\SetKwFor{While}{while}{}{end}
\SetKwProg{Function}{function}{}{end}
\SetArgSty{textnormal}

\newcommand*{\concept}[1]{{\textbf{#1}}}
\newcommand*{\term}[1]{\textit{#1}}

% Embedded codes
\lstset{basicstyle=\ttfamily,
  showstringspaces=false,
  commentstyle=\color{gray},
  keywordstyle=\color{blue}
}

\lstdefinestyle{console}{
    basicstyle=\footnotesize\ttfamily,
    breaklines=true,
    postbreak=\mbox{\textcolor{red}{$\hookrightarrow$}\space}
}
\definecolor{mygreen}{rgb}{0,0.6,0}
\definecolor{mygray}{rgb}{0.5,0.5,0.5}
\definecolor{mymauve}{rgb}{0.58,0,0.82}
\lstdefinestyle{python}{
        language=Python,
        backgroundcolor=\color{white},
        basicstyle=\footnotesize\ttfamily,
        keywordstyle=\color{blue},
        commentstyle=\color{mygreen},
        stringstyle=\color{mymauve},
        numberstyle=\tiny\color{mygray},
        numbers=left,
        breaklines=true,
        showstringspaces=false
}

% Reference formatting
\newrefformat{fig}{Figure~\ref{#1}}

% Color boxes
\tcbuselibrary{skins, breakable, theorems}

\newtcbtheorem[number within=chapter]{infobox}{Box}{
    enhanced,
    boxrule=0pt,
    %colback=blue!5,
    %colframe=blue!5,
    colback=white,
    colframe=white,
    coltitle=blue!60,
    borderline west={4pt}{0pt}{blue!65},
    sharp corners,
    fonttitle=\bfseries, 
    breakable,
    before upper={\parindent15pt\noindent}}{box}
\definecolor{my-orange}{HTML}{F58123}
\newtcbtheorem[number within=chapter, use counter from=infobox]{theorybox}{Box}{
    enhanced,
    boxrule=0pt,
    %colback=orange!5, 
    %colframe=orange!5, 
    colback=white,
    colframe=white,
    coltitle=my-orange!65,
    borderline west={4pt}{0pt}{my-orange!65},
    sharp corners,
    fonttitle=\bfseries, 
    breakable,
    before upper={\parindent15pt\noindent}}{box}
\newtcbtheorem[number within=chapter, use counter from=infobox]{todobox}{Box}{
    enhanced,
    boxrule=0pt,
    colback=red!5,
    colframe=red!5,
    coltitle=red!50,
    borderline west={4pt}{0pt}{red!65},
    sharp corners,
    fonttitle=\bfseries, 
    breakable,
    before upper={\parindent15pt\noindent}}{box}
\newtcbtheorem[number within=chapter, use counter from=infobox]{perspectivebox}{Box}{
    enhanced,
    boxrule=0pt,
    %colback=red!5,
    %colframe=red!5,
    colback=white,
    colframe=white,
    coltitle=red!50,
    borderline west={4pt}{0pt}{red!65},
    sharp corners,
    fonttitle=\bfseries, 
    breakable,
    before upper={\parindent15pt\noindent}}{box}


% Displaying texts in bookmarkers

\pdfstringdefDisableCommands{%
  \def\\{}%
  \def\ce#1{<#1>}%
}

\pdfstringdefDisableCommands{%
  \def\texttt#1{<#1>}%
  \def\mathbb#1{#1}%
}
\pdfstringdefDisableCommands{\def\eqref#1{(\ref{#1})}}

\makeatletter
\pdfstringdefDisableCommands{\let\HyPsd@CatcodeWarning\@gobble}
\makeatother

\newenvironment{shelldisplay}{\begin{lstlisting}}{\end{lstlisting}}

\newcommand*{\citesec}[1]{\S~{#1}}
\newcommand*{\citechap}[1]{Ch.~{#1}}
\newcommand*{\citefig}[1]{Fig.~{#1}}
\newcommand*{\citetable}[1]{Table~{#1}}
\newcommand*{\citepage}[1]{p.~{#1}}
\newcommand*{\citepages}[1]{pp.~{#1}}
\newcommand*{\citefootnote}[1]{fn.~{#1}}
\newcommand*{\citechapsec}[2]{\citechap{#1}.\citesec{#2}}
\newcommand{\literature}[1]{\textit{#1}}

\newrefformat{sec}{\citesec{\ref{#1}}}
\newrefformat{fig}{\citefig{\ref{#1}}}
\newrefformat{tbl}{\citetable{\ref{#1}}}
\newrefformat{chap}{\citechap{\ref{#1}}}
\newrefformat{fn}{\citefootnote{\ref{#1}}}
\newrefformat{box}{Box~\ref{#1}}
\newrefformat{ex}{\ref{#1}}

\newcommand{\shortcode}[1]{\texttt{#1}}

% Make subsubsection labeled
\setcounter{secnumdepth}{4}
\setcounter{tocdepth}{4}

\newcommand*{\abinitio}{\textit{ab initio}}
\newcommand*{\kB}{k_{\text{B}}}
\newcommand*{\epsr}{\epsilon_{\text{r}}}


\lstset{basicstyle=\ttfamily\footnotesize, keywordstyle=\color{blue}, pos=b}

\title{Programming in LaTeX}
\author{Jinyuan Wu}

\begin{document}

\maketitle

\section{Overall ideas}

The way \TeX{} works is like this:
it reads a list of tokens, i.e. characters, numerical digits, spaces, symbols, and macro callings,
and expands the macros on and on until nothing can be expanded,
and then starts typesetting according to the Knuth-Plass algorithm.
\TeX{} is therefore high-level in the sense that it doesn't require the author
to place each word manually.

What we want is to have more abstractions above the \TeX{} layer.
\TeX{} is actually Turing complete, and can perform a quite generic transform from
what an author writes to what the author intends to give to the bare \TeX{} engine.%
\footnote{
    Note that Turing completeness itself doesn't mean such a transform is possible:
    it is possible that a programming model can produce sequences that contain the same information
    with the intended sequences,
    but theoretically cannot produce the latter.
    For instance, it is possible that a programming model can only produce sequences like 
    \shortcode{\dots \textbackslash \textbackslash macroname \dots},
    but cannot produce \shortcode{\dots \textbackslash macroname \dots}.
    This is comparable to how C is Turing-complete but without knowing the address mapping conventions,
    we can't let a CPU C program to talk to GPUs.
    Fortunately the expressivity of \TeX{} programming is enough for everyday typesetting works.
}
This kind of \TeX{} programming, using the primitives in \TeX{} or even \LaTeXe,
is however painful for ``ordinary'' programmers,
especially when it comes to controlling when to and not to expand certain macros.

\LaTeX 3 was originally imagined to be a new \LaTeX{} format but ends up being a programming layer
that can be used together with the existing \LaTeXe{} ecosystem.

The programming model of \LaTeX 3 can be divided into two parts.
The first part is a procedural programming model,
with the familiar branches, loops, function definitions and invocations.
What makes this procedural programming model special 
is that (a) the data types include integers, booleans, and also \concept{lists of tokens},
and (b) function arguments are endowed with additional mechanisms to 
play with sequences of tokens containing macro calls (i.e. \concept{command sequences}).

The second part is the macro-and-expansion-based way to do things mentioned above.
Note that a macro defined in \LaTeX 3 can call a \LaTeX 3 function as well.

One additional peculiarity of \LaTeX 3 is the way \LaTeX 3 programs give their outputs.
A statement that looks like ``just writing something on paper'',
commonly understood as an \emph{expression},
directly prints things to the document,
or the ``console'' if we conceive \LaTeX 3 programming as command line programming.
Thus, we have the follows:

\begin{LTXexample}
\ExplSyntaxOn

\seq_new:N \l_data_seq % Declare a new variable; l means local variable
% Assignment to a variable of sequence type
\seq_set_from_clist:Nn \l_data_seq { apple, banana, cherry }  

\seq_map_inline:Nn \l_data_seq {
    % This is a printing statement!
    % #1 refers to the argument passed to the closure function 
    #1 \ 
}

% This is a printing statement!
\par

\seq_map_inline:Nn \l_data_seq {
    % This is a printing statement!
    Item~#1 \par
}

\ExplSyntaxOff
\end{LTXexample}

In the listing above, we have three printing statements.
They look like expressions
-- which they are, from the perspective of the macro language primitives in \TeX{}  --
but if we are to regard the \LaTeX 3 layer as a procedural language,
then what they do is just to print things to the console, i.e. the final PDF generated.
We especially need to note that the output of the second statement does \emph{not}
override the output of the first statement,
and the output of the third statement does \emph{not} override the output of the second statement.

On the other hand, \LaTeX 3 lacks what we usually call return values in function calls.
We instead need to write things in the Fortran style, like this:

\begin{lstlisting}
    \function:NNN \l_c_tl \l_a_tl \l_b_tl
\end{lstlisting}

However, a much easier way to do this is to redirect what would be printed to the PDF 
into a variable.
That's to say, as long as we can write a function carefully enough 
so that it prints once and only once,
we can write 

\begin{lstlisting}
    \tl_set:Nn \l_c_tl { \function_name:NN \l_a_tl \l_b_tl }
\end{lstlisting}

In this way we can program as if using an ordinary procedural programming language.



\end{document}