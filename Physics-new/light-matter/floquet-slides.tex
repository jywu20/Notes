\documentclass{beamer}
\usepackage{physics}
\usepackage{amsmath}
\usepackage{tikz}
\usepackage{mathdots}
\usepackage{yhmath}
\usepackage{cancel}
\usepackage{color}
\usepackage{siunitx}
\usepackage{array}
\usepackage{multirow}
\usepackage[version=4]{mhchem}
\usepackage{amssymb}
\usepackage{textcomp, gensymb}
\usepackage{pifont}
\newcommand{\cmark}{\ding{51}}%
\newcommand{\xmark}{\ding{55}}%
\usepackage{tabularx}
\usepackage{extarrows}
\usepackage{booktabs}
\usetikzlibrary{fadings}
\usetikzlibrary{patterns}
\usetikzlibrary{shadows.blur}
\usetikzlibrary{shapes}
\usepackage[style=verbose,backend=bibtex]{biblatex}
\addbibresource{wannier.bib}
\addbibresource{hubbard.bib}
\addbibresource{pt.bib}
\renewcommand{\footnotesize}{\scriptsize}
\usepackage{listings}
\usepackage{hyperref}

\newcommand{\pair}[1]{\langle #1 \rangle}
\DeclareMathOperator{\ee}{e}
\DeclareMathOperator{\ii}{i}

\newcommand{\concept}[1]{\textbf{#1}}
\newcommand*{\abinitio}{\textit{ab initio}}
\newcommand{\shortcode}[1]{\texttt{#1}}

%Information to be included in the title page:
\title{Floquet physics}
\subtitle{Periodic driving, formalism, and spetroscopy}
\author{Jinyuan Wu}

\usetheme{metropolis}   

\begin{document}

\maketitle

\begin{frame}
\frametitle{Introduction}



\end{frame}

\begingroup

\title{The formalisms}
\subtitle{``full'' Floquet, perturbation theory, and RWA}
\author{}
\date{}

\begin{frame}
    \titlepage
\end{frame}

\begin{frame}
\frametitle{The complete Floquet formalism}

 

\end{frame}

\endgroup

\begingroup

\title{Angular-resolved photonemission spetroscopy}
\subtitle{Are Floquet states ``real''?}
\author{}
\date{}

\begin{frame}
    \titlepage
\end{frame}

\endgroup

\end{document}