\documentclass[hyperref, a4paper]{article}

\usepackage{geometry}
\usepackage{titling}
\usepackage{titlesec}
% No longer needed, since we will use enumitem package
% \usepackage{paralist}
\usepackage{enumitem}
\usepackage{footnote}
\usepackage[colorinlistoftodos]{todonotes}
\usepackage{amsmath, amssymb, amsthm}
\usepackage{mathtools}
\usepackage{bbm}
\usepackage{graphicx}
\usepackage{subcaption}
\usepackage{soulutf8}
\usepackage{physics}
\usepackage{tensor}
\usepackage{siunitx}
\usepackage[version=4]{mhchem}
\usepackage{tikz}
\usepackage{xcolor}
\usepackage{listings}
\usepackage{autobreak}
\usepackage[ruled, vlined, linesnumbered]{algorithm2e}
\usepackage{nameref,zref-xr}
\zxrsetup{toltxlabel}
\usepackage[backend=bibtex,sorting=none]{biblatex}
\addbibresource{floquet.bib}
\usepackage[colorlinks,unicode]{hyperref} % , linkcolor=black, anchorcolor=black, citecolor=black, urlcolor=black, filecolor=black
\usepackage[most]{tcolorbox}
\usepackage{prettyref}

% Page style
\geometry{left=3.18cm,right=3.18cm,top=2.54cm,bottom=2.54cm}
\titlespacing{\paragraph}{0pt}{1pt}{10pt}[20pt]
\setlength{\droptitle}{-5em}

% More compact lists 
\setlist[itemize]{
    itemindent=17pt, 
    leftmargin=1pt,
    listparindent=\parindent,
    parsep=0pt,
}

% Math operators
\DeclareMathOperator{\timeorder}{\mathcal{T}}
\DeclareMathOperator{\diag}{diag}
\DeclareMathOperator{\legpoly}{P}
\DeclareMathOperator{\primevalue}{P}
\DeclareMathOperator{\sgn}{sgn}
\DeclareMathOperator{\res}{Res}
\newcommand*{\ii}{\mathrm{i}}
\newcommand*{\ee}{\mathrm{e}}
\newcommand*{\const}{\mathrm{const}}
\newcommand*{\suchthat}{\quad \text{s.t.} \quad}
\newcommand*{\argmin}{\arg\min}
\newcommand*{\argmax}{\arg\max}
\newcommand*{\normalorder}[1]{: #1 :}
\newcommand*{\pair}[1]{\langle #1 \rangle}
\newcommand*{\fd}[1]{\mathcal{D} #1}
\DeclareMathOperator{\bigO}{\mathcal{O}}

% TikZ setting
\usetikzlibrary{arrows,shapes,positioning}
\usetikzlibrary{arrows.meta}
\usetikzlibrary{decorations.markings}
\tikzstyle arrowstyle=[scale=1]
\tikzstyle directed=[postaction={decorate,decoration={markings,
    mark=at position .5 with {\arrow[arrowstyle]{stealth}}}}]
\tikzstyle ray=[directed, thick]
\tikzstyle dot=[anchor=base,fill,circle,inner sep=1pt]

% Algorithm setting
% Julia-style code
\SetKwIF{If}{ElseIf}{Else}{if}{}{elseif}{else}{end}
\SetKwFor{For}{for}{}{end}
\SetKwFor{While}{while}{}{end}
\SetKwProg{Function}{function}{}{end}
\SetArgSty{textnormal}

\newcommand*{\concept}[1]{{\textbf{#1}}}

% Embedded codes
\lstset{basicstyle=\ttfamily,
  showstringspaces=false,
  commentstyle=\color{gray},
  keywordstyle=\color{blue}
}

% Reference formatting
\newcommand*{\citesec}[1]{\S~{#1}}
\newcommand*{\citechap}[1]{chap.~{#1}}
\newcommand*{\citefig}[1]{Fig.~{#1}}
\newcommand*{\citetable}[1]{Table~{#1}}
\newcommand*{\citepage}[1]{pp.~{#1}}
\newrefformat{fig}{Fig.~\ref{#1}}
\newcommand*{\term}[1]{\textit{#1}}

% Color boxes
\tcbuselibrary{skins, breakable, theorems}

\newtcbtheorem{infobox}{Box}{
    enhanced,
    boxrule=0pt,
    colback=blue!5,
    colframe=blue!5,
    coltitle=blue!50,
    borderline west={4pt}{0pt}{blue!65},
    sharp corners,
    fonttitle=\bfseries, 
    breakable,
    before upper={\parindent15pt\noindent}}{box}
\newtcbtheorem[use counter from=infobox]{theorybox}{Box}{
    enhanced,
    boxrule=0pt,
    colback=orange!5, 
    colframe=orange!5, 
    coltitle=orange!50,
    borderline west={4pt}{0pt}{orange!65},
    sharp corners,
    fonttitle=\bfseries, 
    breakable,
    before upper={\parindent15pt\noindent}}{box}
\newtcbtheorem[use counter from=infobox]{learnbox}{Box}{
    enhanced,
    boxrule=0pt,
    colback=green!5,
    colframe=green!5,
    coltitle=green!50,
    borderline west={4pt}{0pt}{green!65},
    sharp corners,
    fonttitle=\bfseries, 
    breakable,
    before upper={\parindent15pt\noindent}}{box}


\newenvironment{shelldisplay}{\begin{lstlisting}}{\end{lstlisting}}

\newcommand*{\kB}{k_{\text{B}}}
\newcommand*{\muB}{\mu_{\text{B}}}
\newcommand*{\efermi}{E_{\text{F}}}
\newcommand*{\pfermi}{p_{\text{F}}}
\newcommand*{\vfermi}{v_{\text{F}}}
\newcommand*{\sA}{\text{A}}
\newcommand*{\sB}{\text{B}}
\newcommand*{\Tc}{T_{\text{c}}}
\newcommand*{\hethree}{$^3$He}
\newcommand*{\hefour}{$^4$He}
\newcommand{\epsr}{\epsilon_{\text{r}}}
\newcommand{\chie}{\chi_{\text{e}}}
\newcommand{\Efreq}{\tilde{\vb*{E}}}
\newcommand{\Dfreq}{\tilde{\vb*{D}}}
\newcommand{\Pfreq}{\tilde{\vb*{P}}}

\title{Floquet theory}
\author{Jinyuan Wu}

\begin{document}

\maketitle


\section{The Floquet formalism: quasienergies and quasi-stationary states}

In this section we outline the basic formalism of Floquet physics,
following the notation in \cite{rudner2020floquet}.
As is mentioned in the introduction,
Floquet effects happen with a time-periodic Hamiltonian;
below we let $T = 2 \pi / \omega$ be the period.
Such a Hamiltonian is usually an effective Hamiltonian
when the system (hereafter ``matter'')
is coupled with another degree of freedom
which does not change much in the time evolution;
the latter is hereafter called ``light'',
since in condensed matter systems, 
periodic driving is usually achieved by 
shedding a beam of light to the matter.

From the Floquet theory of differential equation, 
we know it is possible to expand an arbitrary state that 
evolves according to $H$ into 
a linear combination (the coefficients are constants) of 
$\{\ket*{\psi_n(t)}\}$ where 
\begin{equation}
    \ket*{\psi_n (t + T)} = \ee^{- \ii \varepsilon_n t / \hbar} \ket*{\Phi_n(t)},
    \quad \ket*{\Phi_n(t + T)} = \ket*{\Phi_n(t)}.
    \label{eq:floquet-wave-function}
\end{equation}
By discrete periodicity of $\ket*{\Phi_n(t)}$ we make Fourier expansion 
\begin{equation}
    \ket*{\Phi_n(t)} = \sum_m \ee^{- \ii m \omega t} \ket*{\phi_n^{(m)}},
    \label{eq:big-phi-to-extended-hilbert}
\end{equation}
where $m$ goes over all integers.
Note that here $\ket*{\phi_n^{(m)}}$
are \emph{Fourier coefficients} and are not eigenstates of anything; 
there is no normalization or orthogonality condition.
Using $i$ to label the eigenstates of the matter, 
we have 
\begin{equation}
    \ket*{\Phi_n(t)} = \sum_{i} \sum_{m}
    \ee^{- \ii m \omega t} \braket*{i}{\phi_n(m)} \ket*{i}.
\end{equation}
The coefficients before $\ket*{i}$, 
not coefficients before $\ket*{\phi_n^{(m)}}$ in \eqref{eq:big-phi-to-extended-hilbert}, 
give the expansion of $\ket*{\Phi}$ in a 
complete, orthogonal basis.
The significance of $\ket*{\phi}$ vectors can be seen immediately below.

The Schrodinger equation 
\begin{equation}
    \dv{t} \ket*{\psi_n(t)} = H \ket*{\psi_n(t)}
\end{equation}
now reads 
\begin{equation}
    (\varepsilon_n + m \hbar \omega) \ket*{\phi_n^{(m)}} 
    = \sum_{m'} H^{(m - m')} \ket*{\phi_n^{(m')}},
\end{equation}
where 
\begin{equation}
    H(t) = \sum_{m} \ee^{- \ii m \omega t} H^{(m)}.
\end{equation}
Thus we find that if we use $i$ to label the eigenstates of the matter part, 
we have 
\begin{equation}
    \varepsilon_n \braket*{i}{\phi^{(m)}_n}
    = \sum_{m'} (
        H^{(m - m')} - m \hbar \omega \delta_{m m'}
    ) \braket*{i}{\phi^{(m')}_{n}}.
    \label{eq:floquet-ham}
\end{equation}
Recall that $\braket*{i}{\phi_n^{(m)}}$ is the 
$m \omega$-frequency component of $\ket*{\Phi_n(t)}$
projected on the basis vector $\ket*{i}$.
$\varepsilon_n$ is known as the \emph{Floquet quasienergy} of 
the \emph{Floquet quasi-stationary state (or quasi-eigenstate)} $\ket*{\Psi_n(t)}$.

Floquet formalisms can be understood in a more generic framework of non-equilibrium physics:
Floquet Green function can be calculated within the Keldysh formalism,
and the RHS of \eqref{eq:floquet-ham} can be understood as the 
non-equilibrium self-energy \cite{lubatsch2019evolution,aoki2014nonequilibrium}.
As a simple demonstration in the zero-temperature situation, 
consider the general form of light-matter interaction Hamiltonian 
with only one active photon mode
\begin{equation}
    H_{\text{full}} = 
    H \otimes 1_{\text{light}} + 1_{\text{matter}} \otimes \hbar \left(
        b^\dagger b + \frac{1}{2} 
    \right) 
    + \underbrace{
        b V + b^\dagger V^\dagger  
    }_{H_{\text{light-matter coupling}}} ,
\end{equation}
and we assume that the state of the electromagnetic part 
is close to a coherent state $\ket*{\alpha \ee^{- \ii \omega t}}$ 
with strong intensity that almost has zero time evolution. 

We work under the basis $\ket*{i} \otimes \ket*{m}$,
where $m$ refers to the photon number. 

Under this assumption, we can project out the electromagnetic degree of freedom by 
\begin{equation}
    P = \sum_{i} \ket*{i} \bra*{i} \otimes \bra*{\alpha \ee^{- \ii \omega t}}, 
    \label{eq:from-extended-to-matter}
\end{equation} 
where $i$ labels eigenstates of the matter degrees of freedom,
and this means the Hamiltonian for the matter part is 
\begin{equation}
    H_{\text{eff}} = P H_{\text{full}} P = 
    H + \underbrace{\hbar \abs*{\alpha}^2 }_{\const.} + \alpha V \ee^{- \ii \omega t} + \alpha^* V^\dagger \ee^{\ii \omega t},
\end{equation}
and the relation between the full and projected wave function is 
\begin{equation}
    \bra*{i, m} (\ket*{\psi_n} \otimes \ket*{m}) \mapsto \braket*{i}{\phi^{(m)}_n}.
\end{equation}
Now we see the true meaning of $\phi^{(m)}_n$:
we are just grouping the components of the complete matter-light wave function 
$\ket*{\phi_n} \otimes \ket*{\text{light}}$
with the same photon number $m$
into a vector $\phi_n^{(m)}$.

As an example, when the light field is approximately always in a coherent state
$\ket*{\alpha \ee^{- \ii \omega t}}$
($\alpha$ should be large enough so that the matter part 
does not significantly change the state of the light part), 
approximately we have 
\begin{equation}
    H_{\text{light-matter coupling}} \approx
    \alpha V \ee^{- \ii \omega t} + \text{h.c.},
\end{equation}
and the effective Hamiltonian for the matter part is then  
\begin{equation}
    H = H_{\text{matter}} + \alpha V \ee^{- \ii \omega t} + \text{h.c.}.
\end{equation}

Based on the above perspective, 
we call the coefficient matrix on the RHS of \eqref{eq:floquet-ham}
the effective Hamiltonian in the \concept{extended Hilbert space}, 
i.e. the space containing both the matter degrees of freedom 
and the light field.
Since the coefficient matrix on the RHS of \eqref{eq:floquet-ham} 
contains components of the Hamiltonian in the extended space, 
while we are actually working in the Hilbert space of the matter part,
\eqref{eq:floquet-ham}'s solutions are overcomplete.
We can actually point out where overcompletion appears:
note that if $\varepsilon_n$ satisfies \eqref{eq:floquet-wave-function},
then so does $\varepsilon_n + m \hbar \omega$.

In conclusion, a Floquet system has a set of quasi-eigenstates $\{ \ket*{\psi_n} \}$,
the number of which is the same as 
the dimension of the Hilbert space;
but for each quasi-eigenstate,
we have countable infinite quasi-energies,
the difference between the nearest two being $\hbar \omega$;
thus all distinct Floquet quasi-eigenstates can be indexed 
by quasi-energies that are within one ``Floquet-Brillouin zone''.
\todo{Orthogonal relation between $\ket{\psi_n}$'s?}



\section{Self-driven Floquet effects}

it is however possible to use light to stimulate 
some long-lived degrees of freedom in a solid 
and let it drive the rest of the system, 
which sometimes is known as ``self-driving''.

\printbibliography

\end{document}