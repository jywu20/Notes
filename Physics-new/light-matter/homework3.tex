\documentclass[hyperref, a4paper]{article}

\usepackage{geometry}
\usepackage{titling}
\usepackage{titlesec}
% No longer needed, since we will use enumitem package
% \usepackage{paralist}
\usepackage{enumitem}
\usepackage{footnote}
\usepackage[colorinlistoftodos]{todonotes}
\usepackage{amsmath, amssymb, amsthm}
\usepackage{autobreak}
\usepackage{mathtools}
\usepackage{bbm}
\usepackage{graphicx}
\usepackage{subcaption}
\usepackage{soulutf8}
\usepackage{physics}
\usepackage{tensor}
\usepackage{siunitx}
\usepackage[version=4]{mhchem}
\usepackage{tikz}
\usepackage{xcolor}
\usepackage{listings}
\usepackage{autobreak}
\usepackage[ruled, vlined, linesnumbered]{algorithm2e}
\usepackage{nameref,zref-xr}
\zxrsetup{toltxlabel}
\usepackage[backend=bibtex]{biblatex}
\addbibresource{electrodynamics.bib}
\usepackage[colorlinks,unicode]{hyperref} % , linkcolor=black, anchorcolor=black, citecolor=black, urlcolor=black, filecolor=black
\usepackage[most]{tcolorbox}
\usepackage{prettyref}

% Page style
\geometry{left=3.18cm,right=3.18cm,top=2.54cm,bottom=2.54cm}
\titlespacing{\paragraph}{0pt}{1pt}{10pt}[20pt]
\setlength{\droptitle}{-5em}

% More compact lists 
\setlist[itemize]{
    itemindent=17pt, 
    leftmargin=1pt,
    listparindent=\parindent,
    parsep=0pt,
}

% Math operators
\DeclareMathOperator{\timeorder}{\mathcal{T}}
\DeclareMathOperator{\diag}{diag}
\DeclareMathOperator{\legpoly}{P}
\DeclareMathOperator{\primevalue}{P}
\DeclareMathOperator{\sgn}{sgn}
\DeclareMathOperator{\res}{Res}
\newcommand*{\ii}{\mathrm{i}}
\newcommand*{\ee}{\mathrm{e}}
\newcommand*{\const}{\mathrm{const}}
\newcommand*{\suchthat}{\quad \text{s.t.} \quad}
\newcommand*{\argmin}{\arg\min}
\newcommand*{\argmax}{\arg\max}
\newcommand*{\normalorder}[1]{: #1 :}
\newcommand*{\pair}[1]{\langle #1 \rangle}
\newcommand*{\fd}[1]{\mathcal{D} #1}
\DeclareMathOperator{\bigO}{\mathcal{O}}

% TikZ setting
\usetikzlibrary{arrows,shapes,positioning}
\usetikzlibrary{arrows.meta}
\usetikzlibrary{decorations.markings}
\tikzstyle arrowstyle=[scale=1]
\tikzstyle directed=[postaction={decorate,decoration={markings,
    mark=at position .5 with {\arrow[arrowstyle]{stealth}}}}]
\tikzstyle ray=[directed, thick]
\tikzstyle dot=[anchor=base,fill,circle,inner sep=1pt]

% Algorithm setting
% Julia-style code
\SetKwIF{If}{ElseIf}{Else}{if}{}{elseif}{else}{end}
\SetKwFor{For}{for}{}{end}
\SetKwFor{While}{while}{}{end}
\SetKwProg{Function}{function}{}{end}
\SetArgSty{textnormal}

\newcommand*{\concept}[1]{{\textbf{#1}}}

% Embedded codes
\lstset{basicstyle=\ttfamily,
  showstringspaces=false,
  commentstyle=\color{gray},
  keywordstyle=\color{blue}
}

% Reference formatting
\newcommand*{\citesec}[1]{\S~{#1}}
\newcommand*{\citechap}[1]{chap.~{#1}}
\newcommand*{\citefig}[1]{Fig.~{#1}}
\newcommand*{\citetable}[1]{Table~{#1}}
\newcommand*{\citepage}[1]{pp.~{#1}}
\newrefformat{fig}{Fig.~\ref{#1}}
\newcommand*{\term}[1]{\textit{#1}}

% Color boxes
\tcbuselibrary{skins, breakable, theorems}

\newtcbtheorem{infobox}{Box}{
    enhanced,
    boxrule=0pt,
    colback=blue!5,
    colframe=blue!5,
    coltitle=blue!50,
    borderline west={4pt}{0pt}{blue!65},
    sharp corners,
    fonttitle=\bfseries, 
    breakable,
    before upper={\parindent15pt\noindent}}{box}
\newtcbtheorem[use counter from=infobox]{theorybox}{Box}{
    enhanced,
    boxrule=0pt,
    colback=orange!5, 
    colframe=orange!5, 
    coltitle=orange!50,
    borderline west={4pt}{0pt}{orange!65},
    sharp corners,
    fonttitle=\bfseries, 
    breakable,
    before upper={\parindent15pt\noindent}}{box}
\newtcbtheorem[use counter from=infobox]{learnbox}{Box}{
    enhanced,
    boxrule=0pt,
    colback=green!5,
    colframe=green!5,
    coltitle=green!50,
    borderline west={4pt}{0pt}{green!65},
    sharp corners,
    fonttitle=\bfseries, 
    breakable,
    before upper={\parindent15pt\noindent}}{box}


\newenvironment{shelldisplay}{\begin{lstlisting}}{\end{lstlisting}}

\newcommand*{\omegap}{\omega_{\text{p}}}    
\newcommand*{\kB}{k_{\text{B}}}
\newcommand*{\muB}{\mu_{\text{B}}}
\newcommand*{\efermi}{E_{\text{F}}}
\newcommand*{\pfermi}{p_{\text{F}}}
\newcommand*{\vfermi}{v_{\text{F}}}
\newcommand*{\sA}{\text{A}}
\newcommand*{\sB}{\text{B}}
\newcommand*{\Tc}{T_{\text{c}}}
\newcommand*{\hethree}{$^3$He}
\newcommand*{\hefour}{$^4$He}
\newcommand{\epsr}{\epsilon_{\text{r}}}
\newcommand*{\mur}{\mu_{\text{r}}}
\newcommand{\chie}{\chi_{\text{e}}}
\newcommand{\Efreq}{\tilde{E}}
\newcommand{\Dfreq}{\tilde{D}}
\newcommand{\Pfreq}{\tilde{\vb*{P}}}

\newcommand*{\Gammae}{\Gamma_{\text{e}}}
\newcommand*{\Gammag}{\Gamma_{\text{g}}}
\newcommand*{\omegae}{\omega_{\text{e}}}
\newcommand*{\omegag}{\omega_{\text{g}}}
\newcommand*{\omegaeg}{\omega_{\text{eg}}}
\newcommand*{\ptwfc}[2]{\psi^{(#2)}_{#1}}
\newcommand*{\mueg}{\mu_{\text{eg}}}
\newcommand*{\muge}{\mu_{\text{ge}}}
\newcommand*{\Ezzero}{E_{z0}}
\newcommand*{\kete}{\ket*{\text{e}}}
\newcommand*{\ketg}{\ket*{\text{g}}}
\newcommand*{\coeffe}{c_{\text{e}}}
\newcommand*{\coeffg}{c_{\text{g}}}
\newcommand*{\pope}{p_{\text{e}}}
\newcommand*{\popg}{p_{\text{g}}}

\allowdisplaybreaks

\title{Homework 3}
\author{Jinyuan Wu}

\begin{document}

\maketitle

\section{Two-photon absorption in a three-level system}

In this problem, we use perturbation theory to investigate two-photon absorption within a three-level atom, with states $|a\rangle,|b\rangle$ and $|c\rangle$ with energy eigenvalues $E_a=\hbar \omega_a, E_b=\hbar \omega_b$, and $E_c=\hbar \omega_c$ such that $E_c>E_b>E_a$; here $|a\rangle$ and $|c\rangle$ are assumed to have even parity and $|b\rangle$ has odd parity. In the problem that follows, we drive this three-level atom with a monochromatic field $E(t)=E_o \cos (\omega t)$ producing an interaction of the form $H_{\text {int }}=-\hat{\mu} \cdot E(t)$, and we use time-dependent perturbation theory to find the evolution of our quantum state of the form $|\psi\rangle=\sum_n \gamma_n e^{-i \omega_n t}|n\rangle$.

\paragraph*{(a)} \textit{
    Assuming that our atom starts in the ground state (i.e. $\gamma_a^{(0)}=1$ ), use second-order perturbation theory to find $\gamma_c^{(2)}(t)$. Through these calculations, we will assume that states $|b\rangle$ and $|c\rangle$ have finite upper state lifetimes. To emulate population decay, be sure to include a phenomenological damping into your susceptibility by making the replacement $\omega_b \rightarrow \omega_b-i \Gamma_b$ and $\omega_c \rightarrow \omega_c-i \Gamma_c$ where $\Gamma_a$ and $\Gamma_b$ are small compared to the transition frequencies.
}

The time dependent perturbation theory is summarized as 
\begin{equation}
    \braket*{k}{{\psi(t)}} = 
    \sum_{i=0}^\infty \gamma_k^{(i)} \ee^{- \ii \omega_k t},  
\end{equation}
\begin{equation}
    \gamma^{(0)}_k = \const., \quad 
    \dv{\gamma^{(i+1)}_k}{t} = \frac{1}{\ii \hbar} \sum_n 
        H_{kn}^{(1)} \gamma_n^{(i-1)} \ee^{\ii (\omega_k - \omega_n) t}.
\end{equation}
In the current case, due to parity conservation in dipole transition, 
only transitions $a \to b$ and $b \to c$ are possible.
Therefore in the first order perturbation theory, 
only $\gamma_b^{(1)}$ is non-zero.
We have 
\begin{equation}
    \dv{\gamma_b^{(1)}}{t} = \frac{1}{\ii\hbar} H^{\text{dipole}}_{ba} \gamma_a^{(0)} \ee^{\ii (\omega_b - \omega_a) t},
\end{equation}
\begin{equation}
    \begin{aligned}
        \gamma_b^{(1)} &= - \frac{\vb*{\mu}_{ba} \cdot \vb*{E}_0}{\ii\hbar} 
        \cdot \frac{1}{2} \int_{-\infty}^{t} \dd{t'} \left(
            \ee^{\ii (\omega + \omega_b - \omega_a) t'}
            + \ee^{\ii (- \omega + \omega_b - \omega_a) t'}
        \right) \\
        &= \frac{\vb*{\mu}_{ba} \cdot \vb*{E}_0}{\hbar} \cdot 
        \frac{1}{2} \left(
            \frac{
                \ee^{\ii (\omega + \omega_b - \omega_a) t} 
            }{
                \omega + \omega_b - \omega_a
            } +
            \frac{
                \ee^{\ii (- \omega + \omega_b - \omega_a) t} 
            }{
                - \omega + \omega_b - \omega_a
            } 
        \right).
    \end{aligned}
\end{equation}
Here since we are interested in the 
response properties of the three-level system
under plane waves, 
not, say, how fast the atom goes away from its initial state, 
we have pushed the lower bound to $-\infty$, 
and since each excited state has a finite lifetime, 
the $- 1 / (\pm \omega + \omega_b - \omega_a)$ term is simply thrown away,
following the practice in the last homework 
and in the lectures.

Similarly, from 
\begin{equation}
    \dv{\gamma_c^{(2)}}{t} 
    = \frac{1}{\ii\hbar} H^{\text{dipole}}_{cb} \gamma_b^{(1)} \ee^{\ii (\omega_c - \omega_b) t}
    = - \frac{1}{\ii\hbar} \cdot \frac{1}{2} \vb*{\mu}_{cb} \cdot \vb*{E}_0 (
        \ee^{\ii (\omega + \omega_c - \omega_b) t} 
        + \ee^{\ii (- \omega + \omega_c - \omega_b) t}
    ) \cdot \gamma_b^{(1)} ,
\end{equation}
we get 
\begin{align}
    \begin{autobreak}
        \gamma_c^{(2)} = \frac{\vb*{\mu}_{cb} \cdot \vb*{E}_0}{\hbar} 
        \frac{\vb*{\mu}_{ba} \cdot \vb*{E}_0}{\hbar} \cdot \frac{1}{4} \Biggl(
            \frac{
                \ee^{\ii (2 \omega + \omega_c - \omega_a) t}
            }{
                (2\omega + \omega_c - \omega_a) (\omega + \omega_b - \omega_a)
            } + 
            \frac{
                \ee^{\ii (- 2 \omega + \omega_c - \omega_a) t}
            }{
                (- 2\omega + \omega_c - \omega_a) (- \omega + \omega_b - \omega_a)
            } + 
            \frac{
                \ee^{\ii (\omega_c - \omega_a) t}
            }{
                (\omega_c - \omega_a) (\omega + \omega_b - \omega_a)
            } + 
            \frac{
                \ee^{\ii (\omega_c - \omega_a) t}
            }{
                (\omega_c - \omega_a) (- \omega + \omega_b - \omega_a)
            }
        \Biggr).
    \end{autobreak}
\end{align}
Since each excited state has finite lifetime, 
we do the substitution $\omega_{b, c} \to \omega_{b, c} - \ii \Gamma_{b, c}$
and get 
\begin{align}
    \begin{autobreak}
        \gamma_c^{(2)} = \frac{\vb*{\mu}_{cb} \cdot \vb*{E}_0}{\hbar} 
        \frac{\vb*{\mu}_{ba} \cdot \vb*{E}_0}{\hbar} \cdot \frac{1}{4} \Biggl(
            \frac{
                \ee^{\ii (2 \omega + \omega_c - \omega_a ) t}
            }{
                (2\omega + \omega_c - \omega_a - \ii \Gamma_c) 
                (\omega + \omega_b - \omega_a - \ii \Gamma_b)
            } + 
            \frac{
                \ee^{\ii (- 2 \omega + \omega_c - \omega_a ) t}
            }{
                (- 2\omega + \omega_c - \omega_a - \ii \Gamma_c) (- \omega + \omega_b - \omega_a - \ii \Gamma_b)
            } + 
            \frac{
                \ee^{\ii (\omega_c - \omega_a) t}
            }{
                (\omega_c - \omega_a - \ii \Gamma_c) (\omega + \omega_b - \omega_a - \ii \Gamma_b)
            } + 
            \frac{
                \ee^{\ii (\omega_c - \omega_a) t}
            }{
                (\omega_c - \omega_a - \ii \Gamma_c) (- \omega + \omega_b - \omega_a - \ii \Gamma_b)
            }
        \Biggr).
    \end{autobreak}
\end{align}
Here in principle we should also include the imaginary parts of $\omega_{b, c}$
in the oscillating factors $\ee^{\ii (\cdots) t}$; 
if we do so, eventually we should multiply $\ee^{\Gamma_c t}$ 
to the expression above.
But similar to the case in the last homework, 
when evaluating $\bra*{\psi}$, 
we should do the substitution $\omega_{b, c} \to \omega_{b, c} + \ii \Gamma_{b, c}$, 
and in the $c$ component of $\bra{\psi}$ we get a $\ee^{ \ii (\ii \Gamma_c) t}$ factor, 
which cancels the $\ee^{ \Gamma_c t}$ factor in $\ket*{\psi}$;
on the other hand, the imaginary parts in the denominator do not cancel 
and have real physical consequences when we calculate the expectation values. 
So we can ignore the $\ee^{\Gamma_c t}$ factor in $\gamma_c^{(2)}$
and still get everything right. 

\paragraph*{(b)} \textit{
    Assuming that our energy spacing meets the condition for resonant twophoton absorption (i.e., $\omega_c-\omega_a \cong 2 \omega$ ), perform the rotating wave approximation to simplify your result from part (a). [Hint: all of the terms with fast-oscillating phases should vanish] Use this result find the probability, $P_c$, that the atom occupies state $|c\rangle$. Since this probability, $P_c$, is independent of time it can also be viewed as the steady-state population of state $|c\rangle$.
}

Since $\omega_c - \omega_a \simeq 2 \omega$, 
only the $- 2 \omega + \omega_c - \omega_a$ term is the slowly-oscillating term, 
and by rotating wave approximation 
\begin{equation}
    \gamma_c^{(2)} \approx \frac{\vb*{\mu}_{cb} \cdot \vb*{E}_0}{\hbar} 
    \frac{\vb*{\mu}_{ba} \cdot \vb*{E}_0}{\hbar} \cdot \frac{1}{4} 
    \frac{
        \ee^{\ii (- 2 \omega + \omega_c - \omega_a ) t}
    }{
        (- 2\omega + \omega_c - \omega_a - \ii \Gamma_c) (- \omega + \omega_b - \omega_a - \ii \Gamma_b)
    } .
\end{equation}
Thus the steady-state population on state $c$ is 
\begin{equation}
    P_c = \abs*{\psi_c^{(2)}} = \frac{
        \abs*{\vb*{\mu}_{cb} \cdot \vb*{E}_0}^2 
        \abs*{\vb*{\mu}_{ba} \cdot \vb*{E}_0}^2
    }{16 \hbar^4} 
    \frac{1}{
        ((\omega_c - \omega_a - 2 \omega)^2 + \Gamma_c^2 )
        ((\omega_b - \omega_a - \omega)^2 + \Gamma_b^2)
    }.
\end{equation}

\paragraph*{(c)} \textit{
    Notice that the excited state population satisfies the following rate equation $d P_c / d t=-2 \Gamma_c P_c+R_c$, where $R_c$ is the two-photon absorption rate for the atom. Use this expression to find $R_c$ and the two-photon absorption crosssection, $\sigma_{T P A}(I)$, at steady state; here, $I$ is the intensity of light. For the purposes of this calcluation, assume the atom lives in a material with refractive index $n$. [Hint: $R_c=I \sigma_{T P A}(I) / \hbar \omega$]
}

When equilibrium hasn't be achieved, since $\psi_c^{(2)} \sim \ee^{- \Gamma_c t} \Rightarrow 
P_c \sim \ee^{- 2 \Gamma_c t}$, the rate equation for $P_c$ is 
\begin{equation}
    \dv{P_c}{t} = - 2 \Gamma_c P_c + R_c,
\end{equation}
where $R_c$ is the two-photon absorption rate for the atom. 
Now since the system is already in equilibrium, 
we have 
\begin{equation}
    R_c = 2 \Gamma_c P_c 
    = \frac{
        \abs*{\vb*{\mu}_{cb} \cdot \vb*{E}_0}^2 
        \abs*{\vb*{\mu}_{ba} \cdot \vb*{E}_0}^2
    }{8 \hbar^4} 
    \frac{\Gamma_c}{
        ((\omega_c - \omega_a - 2 \omega)^2 + \Gamma_c^2 )
        ((\omega_b - \omega_a - \omega)^2 + \Gamma_b^2)
    }.
\end{equation} 

The two-photon absorption (TPA) cross section can be determined by 
\begin{equation}
    R_c \cdot \hbar \omega = I \sigma_{\text{TPA}} = \text{energy absorbed per second}.
\end{equation}
The intensity is the time average of Poynting vector, 
\begin{equation}
    I = \frac{1}{2} \epsilon_0 c n \abs*{\vb*{E}_0}^2,
\end{equation}
and therefore by replacing $E_0^2$ by $I$, we get 
\begin{equation}
    \sigma_{\text{TPA}} = \frac{\omega I}{2 n^2 \epsilon_0^2 c^2 \hbar^3} 
    \abs*{\vb*{\mu}_{cb} \cdot \vu*{\epsilon}}^2 
    \abs*{\vb*{\mu}_{ba} \cdot \vu*{\epsilon}}^2
    \frac{\Gamma_c}{
        ((\omega_c - \omega_a - 2 \omega)^2 + \Gamma_c^2 )
        ((\omega_b - \omega_a - \omega)^2 + \Gamma_b^2)
    }.
\end{equation}
Here $\vu*{\epsilon}$ is the polarization direction of the incident beam.

\paragraph*{(d)} It takes $\dd{t} = \dd{z} / (c / n)$ for a plane wave beam to propagate $\dd{z}$.
During this period of time, the energy loss due to second order effect is 
\begin{equation}
    \dd{I} = \underbrace{R_c \cdot \hbar \omega}_{\text{energy absorbed per second}} \cdot \dd{t}
    = \frac{n}{c} R_c \hbar \omega \dd{z}.
\end{equation}
The definition of $\beta_2$ is 
\begin{equation}
    \dv{I}{z} = - \alpha I - \beta_2 I^2,
\end{equation}
and we find 
\begin{equation}
    \beta_2 = \frac{1}{I^2} \cdot \frac{n}{c} R_c \hbar \omega
    = \frac{\omega}{2 n \epsilon_0^2 c^3 \hbar^3} 
    \abs*{\vb*{\mu}_{cb} \cdot \vu*{\epsilon}}^2 
    \abs*{\vb*{\mu}_{ba} \cdot \vu*{\epsilon}}^2
    \frac{\Gamma_c}{
        ((\omega_c - \omega_a - 2 \omega)^2 + \Gamma_c^2 )
        ((\omega_b - \omega_a - \omega)^2 + \Gamma_b^2)
    }.
\end{equation}

\section{Quantum treatment of nonlinear susceptibility}

\paragraph*{(a)} In this section we generalize the procedure in the first problem.
The external electric field now is  
\begin{equation}
    \vb*{E}(t) = \sum_p \vb*{E}_p \ee^{- \ii \omega_p t} + \text{c.c.}.
\end{equation}
Here to simply the derivation, 
we follow the convention in Boyd and slightly misuse the notation in the following way:
\begin{itemize}
    \item $\sum_p$ means to sum over all optical modes $p$, 
    \emph{and} the positive- and negative-frequencies.
    \item $\vb*{E}(\omega_p)$, likewise, means 
    $E_p$ when $\omega_p > 0$, 
    and $E_p^*$ when $\omega_p < 0$.
\end{itemize}
In this way $\sum_p F(E(\omega_p)^*, - \omega_p) \ee^{\ii \omega_p t}$
can be replaced by $\sum_p F(E(\omega_p), \omega_p) \times \ee^{- \ii \omega_p t}$.

The zeroth order state is  
\begin{equation}
    \gamma^{(0)}_n = \delta_{n, \text{g}} .
\end{equation}
Therefore the first order perturbation is determined by 
\begin{equation}
    \dv{\gamma_m^{(1)}}{t} = \frac{1}{\ii \hbar} 
    \underbrace{
        (- \vb*{\mu}_{m \text{g}} \cdot \sum_p \vb*{E}(\omega_p) \ee^{- \ii \omega_p t}) 
    }_{H_{\text{int}}}
    \ee^{\ii (\omega_m - \omegag) t} ,
\end{equation}
and therefore after integration we get 
\begin{equation}
    \gamma_m^{(1)}(t) = \frac{1}{\hbar} \sum_p \frac{\vb*{\mu}_{m \text{g}} \cdot \vb*{E}(\omega_p)}{(\omega_m - \omegag - \omega_p)} \ee^{\ii (\omega_m - \omegag - \omega_p) t}.
\end{equation}
The second order perturbation is determined by 
\begin{equation}
    \dv{\gamma_m^{(2)}}{t} = \frac{1}{\ii \hbar} \sum_n
    \underbrace{
        (- \vb*{\mu}_{m n} \cdot \sum_q \vb*{E}(\omega_q) \ee^{- \ii \omega_q t}) 
    }_{H_{\text{int}}}
    \ee^{\ii (\omega_m - \omega_n) t} 
    \cdot \underbrace{\frac{1}{\hbar} \sum_p \frac{\vb*{\mu}_{n \text{g}} \cdot \vb*{E}(\omega_p)}{(\omega_n - \omegag - \omega_p)} \ee^{\ii (\omega_n - \omegag - \omega_p) t}}_{\gamma_n^{(1)}} ,
\end{equation}
and after integration we get 
\begin{equation}
    \gamma_m^{(2)}(t) = \frac{1}{\hbar^2} \sum_{p, q} \sum_n
    \frac{
        (\vb*{\mu}_{mn} \cdot \vb*{E}(\omega_q)) (\vb*{\mu}_{n \text{g}} \cdot \vb*{E}(\omega_p))
    }{
        (\omega_m - \omegag - \omega_p - \omega_q)
        (\omega_n - \omegag - \omega_p)
    } \ee^{\ii (\omega_m - \omegag - \omega_p - \omega_q) t}.
\end{equation}

\paragraph*{(b)} The second order polarization is given by 
\begin{equation}
    \expval{\vb*{\mu}}^{(2)} = \mel*{\psi^{(2)}}{\vb*{\mu}}{\psi^{(0)}} + 
    \mel*{\psi^{(1)}}{\vb*{\mu}}{\psi^{(1)}}
    + \mel*{\psi^{(0)}}{\vb*{\mu}}{\psi^{(2)}},
\end{equation}
and therefore (note that we still need a $\ee^{- \ii \omega_i t}$ factor 
besides $\gamma_i$)
\begin{equation}
    \begin{aligned}
        \expval{\vb*{\mu}}^{(2)} &= 
        \frac{1}{\hbar^2} \sum_{p, q} \sum_{m, n} \ee^{\ii \omega_m t}
        \frac{
            (\vb*{\mu}_{nm} \cdot \vb*{E}(- \omega_q)) 
            (\vb*{\mu}_{\text{g} n} \cdot \vb*{E}(- \omega_p)) 
            \vb*{\mu}_{m \text{g}}
        }{
            (\omega_m^* - \omegag - \omega_p - \omega_q)
            (\omega_n^* - \omegag - \omega_p)
        } \ee^{- \ii (\omega_m - \omegag - \omega_p - \omega_q) t} 
        \ee^{- \ii \omegag t} \\
        &\quad + \frac{1}{\hbar^2} 
        \sum_{m, n} \sum_{p, q} \ee^{\ii \omega_m t} \frac{\vb*{\mu}_{\text{g} m} \cdot \vb*{E}(- \omega_p)}{(\omega_m^* - \omegag - \omega_p)} 
        \ee^{- \ii (\omega_m - \omegag - \omega_p) t} 
        \vb*{\mu}_{m n} 
        \frac{\vb*{\mu}_{n \text{g}} \cdot \vb*{E}(\omega_q)}{(\omega_n - \omegag - \omega_q)} \ee^{\ii (\omega_m - \omegag - \omega_p) t} \ee^{- \ii \omega_n t}
          \\
        &\quad + 
        \frac{1}{\hbar^2} \sum_{p, q} \sum_{m, n} \ee^{\ii \omegag t}
        \frac{
            (\vb*{\mu}_{mn} \cdot \vb*{E}(\omega_q)) 
            (\vb*{\mu}_{n \text{g}} \cdot \vb*{E}(\omega_p))
            \vb*{\mu}_{\text{g} m}
        }{
            (\omega_m - \omegag - \omega_p - \omega_q)
            (\omega_n - \omegag - \omega_p)
        } \ee^{\ii (\omega_m - \omegag - \omega_p - \omega_q) t} \ee^{- \ii \omega_m t} .
    \end{aligned}
\end{equation}
Here to simply the form of the denominator, 
we have attract $\ii \Gamma_{m, n}$ into $\omega_{m, n}$.
After simplification and changing the signs of $\omega_{p, q}$
(see the discussion at the beginning of problem (a)) we have 
\begin{equation}
    \begin{aligned}
        \expval{\vb*{\mu}}^{(2)} = 
        \frac{1}{\hbar^2} \sum_{p, q} \sum_{m, n}
        \ee^{- \ii (\omega_p + \omega_q) t} 
        \Biggl(
            \frac{
                (\vb*{\mu}_{nm} \cdot \vb*{E}(\omega_q)) 
                (\vb*{\mu}_{\text{g} n} \cdot \vb*{E}(\omega_p)) 
                \vb*{\mu}_{m \text{g}}
            }{
                (\omega_m^* - \omegag + \omega_p + \omega_q)
                (\omega_n^* - \omegag + \omega_p)
            } & \\
            \quad +
            \frac{
                (\vb*{\mu}_{\text{g} m} \cdot \vb*{E}(\omega_p)) 
                (\vb*{\mu}_{n \text{g}} \cdot \vb*{E}(\omega_q)) 
                \vb*{\mu}_{m n} 
            }{
                (\omega_m^* - \omegag + \omega_p) 
                (\omega_n - \omegag - \omega_q)
            }& \\
            +  
            \frac{
                (\vb*{\mu}_{mn} \cdot \vb*{E}(\omega_q)) 
                (\vb*{\mu}_{n \text{g}} \cdot \vb*{E}(\omega_p))
                \vb*{\mu}_{\text{g} m}
            }{
                (\omega_m - \omegag - \omega_p - \omega_q)
                (\omega_n - \omegag - \omega_p)
            }& 
        \Biggr).
    \end{aligned}
    \label{eq:mu-second-order}
\end{equation} 
After swapping $p$ and $q$ and $m$ and $n$ in some of the terms, 
we get Boyd (3.2.26).
The expression can also be obtained in a many-body perturbation theoretic way: 
we set the chemical potential to somewhere between $\omegag$ 
and the first excited state, 
and then \eqref{eq:mu-second-order} can be derived from 
the two photon in, one photon out electron loop diagram 
in the same way Lindhard function is derived from the electron-hole loop.

\paragraph*{(c)} $\chi^{(2)}$ is \eqref{eq:mu-second-order} 
divided by $E(\omega_p)$ and $E(\omega_q)$;
so it's proportional to $\mu_{mn} \mu_{n \text{g}} \mu_{\text{g} m}$.
When the system has inversion symmetry 
and therefore eigenstates in the system are all parts of 
representations of the inversion operation, 
and therefore have definite parities,
in order for this product to be non-zero, we need $\ket*{m}, \ket*{n}, \ket*{\text{g}}$
to all have different parities.
But this is not possible: 
if $\ket*{\text{g}}$ and $\ket*{m}$ have different parities 
and $\ket*{m}$ and $\ket*{n}$ have different parities, 
and the parities of $\ket*{\text{g}}$ and $\ket*{n}$
have to be the same, because the parity of a state is either even or odd: 
and then $\mu_{n \text{g}}$ is zero.
So whenever the eigenstates have definite parities 
(in other words, the system has inversion symmetry), 
$\chi^{(2)}$ vanishes.

\section{Three-wave interaction} 

\subsection{Classical stimulated Raman effect}

When the polarizability depends on a phonon mode, we have 
\begin{equation}
    L = \int \dd[3]{\vb*{r}} \left(
        \frac{1}{2} \epsilon(q) \dot{\vb*{A}}^2 
        - \frac{1}{2 \mu_0} (\curl{\vb*{A}})^2
    \right) + L_{\text{non-interactive}}[q]
    - H_{\text{dipole}},
\end{equation}
and from this Lagrangian we can verify that the displacement vector is 
\begin{equation}
    \vb*{D} = \fdv{L}{(- \dot{\vb*{A}})}
    = \epsilon(q) \vb*{E} + \vb*{\mu}_{\text{linear}}.
\end{equation}
If on the other hand, the electric field is seen as a given external field, 
the effective Hamiltonian about the phonon mode then is 
\begin{equation}
    H = \frac{p^2}{2m} + \frac{1}{2} m \Omega_0^2 q^2 + \frac{1}{2} \epsilon(q) \dot{\vb*{A}}^2 
    + H_{\text{dipole}},
\end{equation}
and after Taylor expansion of $\epsilon(q)$ we get 
\begin{equation}
    H = \frac{p^2}{2m} + \frac{1}{2} m \Omega_0^2 q^2 + \frac{1}{2} \eval{\pdv{\alpha}{q}}_0 q E^2 
    - e q E,
\end{equation}
where $e$ is the charge of the dipole induced by that phonon mode. 

\paragraph*{(a)} The optical force on coordinate $q$ is therefore 
\begin{equation}
    F_q = - \pdv{H}{q} 
    = - m \Omega_0^2 q \underbrace{
        - \frac{1}{2} \eval{\pdv{\alpha}{q}}_0 E^2 + e E
    }_{\text{produced by light}}.
\end{equation} 

\paragraph*{(b)} \textit{
    Next, we consider the presence of two driving fields such that $E(t)=E_1(t)+$ $E_2(t)$, where $E_1(t)=E_1^o \cdot \cos \left(\omega_1 t\right)$ and $E_2(t)=E_2^o \cdot \cos \left(\omega_2 t\right)$. The driving frequencies are related by $\omega_1=\omega_2+\Omega$, permitting us to view $\omega_1$ as the (blue) pump-field and $\omega_2$ as the (red) Stokes-field. (Here, we assume that $\Omega \approx \Omega_o$ and $\left.\Omega_o \ll \omega_1, \omega_2.\right)$
}

The force on $q$ then is 
\begin{equation}
    F_q = - m \Omega_0^2 q 
    + e (E_1^0 \cos \omega_1 t + E_2^0 \cos \omega_2 t)
    - \frac{1}{2} \eval{\pdv{\alpha}{q}}_0 (E_1^0 \cos \omega_1 t + E_2^0 \cos \omega_2 t)^2.
\end{equation}

\paragraph*{(c)} \textit{
    Find the equation of motion for $q(t)$, keeping only frequency terms close to the Raman resonance. (Hint, only the cross-terms between $E_1$ and $E_2$ should survive.)
}

Using the product-to-sum formula, we find the only term in $F_q$ 
whose frequency is close to the Raman resonance 
is the $\omega_1 - \omega_2$ term in the $q E^2$ term.
So the equation of motion for the $\Omega$ component of $q$ is 
(below we use $q$ to refer to the $\Omega$ component of $q$)
\begin{equation}
    m \ddot{q} = F_q(\Omega)
    = - m \Omega_0^2 q - \frac{1}{2} \eval{\pdv{\alpha}{q}}_0 E_1^0 E_2^0 \cos \Omega t.
    \label{eq:continuous-raman-eom-1}
\end{equation}

\paragraph*{(d)} \textit{
    Next, include damping with an amplitude decay, $1 / \tau$, and find an expression for the steady-state phonon amplitude. [Hint: it makes more sense to work with the envelope equations (i.e. $q=\operatorname{Re}\left[\bar{q} \cdot e^{-i \Omega t}\right]$ ). Remember, at steadystate, $\dot{\bar{q}}=0$.]
}

Adding the damping term to \eqref{eq:continuous-raman-eom-1}, we have 
\begin{equation}
    \ddot{q} + \frac{1}{\tau} \dot{q} + \Omega_0^2 q = 
    - \frac{1}{2 m} \eval{\pdv{\alpha}{q}}_0 E_1^0 E_2^0 \cos \Omega t.
\end{equation}
Using the envelope ansatz 
\begin{equation}
    q = \frac{1}{2} (\tilde{q} \ee^{- \ii \Omega t} + \text{c.c.}) ,
\end{equation}
and ignoring the $\ddot{\tilde{q}}$ term because $\tilde{q}$ 
is assumed to vary slowly,
we have 
\begin{equation}
    - \Omega^2 \tilde{q} - 2 \ii \Omega \dot{\tilde{q}} 
    + \frac{1}{\tau} \dot{\tilde{q}} 
    - \frac{\ii \Omega}{\tau} \tilde{q}
    + \Omega_0^2 \tilde{q} 
    = - \frac{1}{2m} \eval{\pdv{\alpha}{q}}_0 E_1^0 E_2^0 .
\end{equation} 
At steady state, we have $\dot{\tilde{q}} = 0$, and the above equation becomes 
\begin{equation}
    \tilde{q} = \frac{1}{
        \Omega^2 - \Omega_0^2 + \ii \Omega / \tau 
    } \frac{1}{2m} \eval{\pdv{\alpha}{q}}_0 E_1^0 E_2^0.
    \label{eq:amplitude-q-1}
\end{equation}
With the approximation $\Omega \approx \Omega_0$ we have 
\begin{equation}
    \tilde{q} = \frac{\tau}{
        \ii \Omega 
    } \frac{1}{2m} \eval{\pdv{\alpha}{q}}_0 E_1^0 E_2^0.
\end{equation}
This is the steady-state phonon amplitude.

\paragraph*{(e)} \textit{
    Next, we find the nonlinear polarizability produced by the Raman interaction. We do this by taking the steady-state phonon amplitude from part (d) and substituting it into $\mu_R=\left(\frac{\partial \alpha}{\partial q}\right) q E=\left(\frac{\partial \alpha}{\partial q}\right) q\left(E_1(t)+E_2(t)\right)$. Using this expression, find the complex polarizability associated with $\mu_R\left(\omega_2 ; \omega_1,-\omega_1, \omega_2\right)$.
}

The electric dipole induced by the $qE^2$ term, under \eqref{eq:amplitude-q-1}, is 
\begin{equation}
    \mu_\text{R} = \eval{\pdv{\alpha}{q}}_0 \cdot \frac{1}{2} 
    \left(
        \frac{1}{
            \Omega^2 - \Omega_0^2 + \ii \Omega / \tau 
        } \frac{1}{2m} \eval{\pdv{\alpha}{q}}_0 E_1^0 E_2^0 \ee^{-\ii \Omega t}
        + \text{c.c.} 
    \right)
    \cdot \frac{1}{2} (E_1^0 \ee^{-\ii \omega_1 t} + E_2^0 \ee^{-\ii \omega_2 t} + \text{c.c.}),
\end{equation}
and therefore the $\mu_{\text{R}}(\omega_2; \omega_1, - \omega_1, \omega_2)$ component is 
\begin{equation}
    \mu_{\text{R}}(\omega_2; \omega_1, - \omega_1, \omega_2)
    = \frac{1}{2} \eval{\pdv{\alpha}{q}}_0 \cdot 
    \frac{1}{
        \Omega^2 - \Omega_0^2 - \ii \Omega / \tau 
    } \frac{1}{2m} \eval{\pdv{\alpha}{q}}_0 E_1^0 E_2^0
    \cdot \frac{1}{2} E_1^0 \ee^{- \ii \omega_2 t} + \text{c.c.}
\end{equation}
The polarizability therefore is 
\begin{equation}
    \chi_{\text{R}}(\omega_2; \omega_1, - \omega_1, \omega_2)
    = N \alpha_{\text{R}}
    = \left(\eval{\pdv{\alpha}{q}}_0 \right)^2
    \frac{N}{
        m (\Omega^2 - \Omega_0^2 - \ii \Omega / \tau )
    } .
\end{equation} 
Here we treat $E = \Re \tilde{E} \ee^{- \ii \omega t}$ 
as the external field, not just $\tilde{E}$, 
so all the $1/2$ factors disappear.

\subsection{Idealized Raman interaction via Fermi's Golden rule}



\end{document}