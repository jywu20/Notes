\documentclass[hyperref, a4paper]{article}

\usepackage{geometry}
\usepackage{titling}
\usepackage{titlesec}
% No longer needed, since we will use enumitem package
% \usepackage{paralist}
\usepackage{enumitem}
\usepackage{footnote}
\usepackage[colorinlistoftodos]{todonotes}
\usepackage{amsmath, amssymb, amsthm}
\usepackage{mathtools}
\usepackage{bbm}
\usepackage{graphicx}
\usepackage{subcaption}
\usepackage{soulutf8}
\usepackage{physics}
\usepackage{tensor}
\usepackage{siunitx}
\usepackage[version=4]{mhchem}
\usepackage{tikz}
\usepackage{xcolor}
\usepackage{listings}
\usepackage{autobreak}
\usepackage[ruled, vlined, linesnumbered]{algorithm2e}
\usepackage{nameref,zref-xr}
\zxrsetup{toltxlabel}
\usepackage[backend=bibtex]{biblatex}
\addbibresource{elasticity.bib}
\usepackage[colorlinks,unicode]{hyperref} % , linkcolor=black, anchorcolor=black, citecolor=black, urlcolor=black, filecolor=black
\usepackage[most]{tcolorbox}
\usepackage{prettyref}

% Page style
\geometry{left=3.18cm,right=3.18cm,top=2.54cm,bottom=2.54cm}
\titlespacing{\paragraph}{0pt}{1pt}{10pt}[20pt]
\setlength{\droptitle}{-5em}

% More compact lists 
\setlist[itemize]{
    itemindent=17pt, 
    leftmargin=1pt,
    listparindent=\parindent,
    parsep=0pt,
}

% Math operators
\DeclareMathOperator{\timeorder}{\mathcal{T}}
\DeclareMathOperator{\diag}{diag}
\DeclareMathOperator{\legpoly}{P}
\DeclareMathOperator{\primevalue}{P}
\DeclareMathOperator{\sgn}{sgn}
\DeclareMathOperator{\res}{Res}
\newcommand*{\ii}{\mathrm{i}}
\newcommand*{\ee}{\mathrm{e}}
\newcommand*{\const}{\mathrm{const}}
\newcommand*{\suchthat}{\quad \text{s.t.} \quad}
\newcommand*{\argmin}{\arg\min}
\newcommand*{\argmax}{\arg\max}
\newcommand*{\normalorder}[1]{: #1 :}
\newcommand*{\pair}[1]{\langle #1 \rangle}
\newcommand*{\fd}[1]{\mathcal{D} #1}
\DeclareMathOperator{\bigO}{\mathcal{O}}

% TikZ setting
\usetikzlibrary{arrows,shapes,positioning}
\usetikzlibrary{arrows.meta}
\usetikzlibrary{decorations.markings}
\tikzstyle arrowstyle=[scale=1]
\tikzstyle directed=[postaction={decorate,decoration={markings,
    mark=at position .5 with {\arrow[arrowstyle]{stealth}}}}]
\tikzstyle ray=[directed, thick]
\tikzstyle dot=[anchor=base,fill,circle,inner sep=1pt]

% Algorithm setting
% Julia-style code
\SetKwIF{If}{ElseIf}{Else}{if}{}{elseif}{else}{end}
\SetKwFor{For}{for}{}{end}
\SetKwFor{While}{while}{}{end}
\SetKwProg{Function}{function}{}{end}
\SetArgSty{textnormal}

\newcommand*{\concept}[1]{{\textbf{#1}}}

% Embedded codes
\lstset{basicstyle=\ttfamily,
  showstringspaces=false,
  commentstyle=\color{gray},
  keywordstyle=\color{blue}
}

% Reference formatting
\newcommand*{\citesec}[1]{\S~{#1}}
\newcommand*{\citechap}[1]{chap.~{#1}}
\newcommand*{\citefig}[1]{Fig.~{#1}}
\newcommand*{\citetable}[1]{Table~{#1}}
\newcommand*{\citepage}[1]{pp.~{#1}}
\newrefformat{fig}{Fig.~\ref{#1}}
\newcommand*{\term}[1]{\textit{#1}}

% Color boxes
\tcbuselibrary{skins, breakable, theorems}

\newtcbtheorem{infobox}{Box}{
    enhanced,
    boxrule=0pt,
    colback=blue!5,
    colframe=blue!5,
    coltitle=blue!50,
    borderline west={4pt}{0pt}{blue!65},
    sharp corners,
    fonttitle=\bfseries, 
    breakable,
    before upper={\parindent15pt\noindent}}{box}
\newtcbtheorem[use counter from=infobox]{theorybox}{Box}{
    enhanced,
    boxrule=0pt,
    colback=orange!5, 
    colframe=orange!5, 
    coltitle=orange!50,
    borderline west={4pt}{0pt}{orange!65},
    sharp corners,
    fonttitle=\bfseries, 
    breakable,
    before upper={\parindent15pt\noindent}}{box}
\newtcbtheorem[use counter from=infobox]{learnbox}{Box}{
    enhanced,
    boxrule=0pt,
    colback=green!5,
    colframe=green!5,
    coltitle=green!50,
    borderline west={4pt}{0pt}{green!65},
    sharp corners,
    fonttitle=\bfseries, 
    breakable,
    before upper={\parindent15pt\noindent}}{box}


\newenvironment{shelldisplay}{\begin{lstlisting}}{\end{lstlisting}}

\newcommand*{\omegap}{\omega_{\text{p}}}    
\newcommand*{\kB}{k_{\text{B}}}
\newcommand*{\muB}{\mu_{\text{B}}}
\newcommand*{\efermi}{E_{\text{F}}}
\newcommand*{\pfermi}{p_{\text{F}}}
\newcommand*{\vfermi}{v_{\text{F}}}
\newcommand*{\sA}{\text{A}}
\newcommand*{\sB}{\text{B}}
\newcommand*{\Tc}{T_{\text{c}}}
\newcommand*{\hethree}{$^3$He}
\newcommand*{\hefour}{$^4$He}
\newcommand{\epsr}{\epsilon_{\text{r}}}
\newcommand{\chie}{\chi_{\text{e}}}
\newcommand{\Efreq}{\tilde{\vb*{E}}}
\newcommand{\Dfreq}{\tilde{\vb*{D}}}
\newcommand{\Pfreq}{\tilde{\vb*{P}}}

\title{Homework 1}
\author{Jinyuan Wu}

\begin{document}

\maketitle

\section{Maxwell's equations in dielectrics, Lorentz oscillators, and complex notation}

\subsection{Time-Average Quantities in Complex Notation}

It is often important to be able to compute time-averaged quantities, such as the potential energy of a harmonic oscillator $U_{p e}=\frac{k}{2}\left\langle x^2\right\rangle$ or the electric field energy density $U_{\mathrm{el}}=\frac{\varepsilon_0}{2}\left\langle\mathbf{E}^2\right\rangle$. Here, the time-average of a function, $f(t)$, is defined as, $\langle f(t)\rangle=(1 / T) \int_{t-T / 2}^{t+T / 2} d t^{\prime} f\left(t^{\prime}\right)$, where $T$ is defined as either the characteristic period of the oscillating system (i.e., $T=2 \pi / \omega)$ or infinity. Such time averaging is drastically simplified by using complex notation.

To see this, suppose that we have any two functions $A(t)$ and $B(t)$, both of which take on a time harmonic form. Without loss of generality, we assume that $A(t)=A_0 \cos (\omega t+\phi)$, and $B(t)=B_0 \cos (\omega t+\theta)$, where $\phi$ and $\theta$ are arbitrary phase factors.

\subsubsection{}

We have 
\begin{equation}
    \begin{aligned}
        \expval{A(t) B(t)} &= 
        \frac{1}{T} \int_{t - T/2}^{t + T/2} \dd{t'} 
            A_0 \cos(\omega t' + \phi) 
            B_0 \cos (\omega t' + \theta) \\
        &= A_0 B_0 \frac{1}{T} \int_{t - T/2}^{t + T/2} \dd{t'} 
            \frac{1}{2} (
                \cos(\omega t' + \phi + \omega t' + \theta) 
                + \cos (\omega t' + \phi - \omega t' - \theta)
            ) \\
        &= \frac{1}{2} A_0 B_0 \cos (\phi - \theta).
    \end{aligned}
\end{equation}
Here we have used the condition that $T = 2 \pi / \omega$ 
so that the first term vanishes.

\subsubsection{}

We have 
\begin{equation}
    A(t) = \tilde{A}_0 \ee^{- \ii \omega t} , \quad 
    B(t) = \tilde{B}_0 \ee^{- \ii \omega t} , \quad 
    \tilde{A}_0 = A_0 \ee^{- \ii \phi}, \quad 
    \tilde{B}_0 = B_0 \ee^{- \ii \theta},
\end{equation}
and therefore 
\begin{equation}
    \Re \tilde{A}_0 B_0 = \Re A_0 \tilde{B}_0 = 
    \Re A_0 B_0 \ee^{\ii (\phi - \theta)}
    = A_0 B_0 \cos(\phi - \theta),
\end{equation}
and hence 
\begin{equation}
    \expval{A(t) B(t)} = \frac{1}{2} \Re \tilde{A}_0 B_0 
    = \frac{1}{2} \Re A_0 \tilde{B}_0 .
\end{equation}

We can also straightforwardly do the follows. 
We have 
\begin{equation}
    \begin{aligned}
        \expval{A(t) B(t)} &= \expval{
            \frac{1}{2} (\tilde{A}(t) + \tilde{A}^*(t))
            \cdot \frac{1}{2} (\tilde{B}(t) + \tilde{B}^*(t))
        } \\
        &= \frac{1}{4} \expval{
            \tilde{A}_0 \tilde{B}_0 \ee^{- 2 \ii \omega t} + 
            \tilde{A}_0 \tilde{B}_0^* + 
            \tilde{A_0}^* \tilde{B}_0^* \ee^{2 \ii \omega t} 
            + \tilde{A}_0^* \tilde{B}_0
        } \\
        &= \frac{1}{4} \expval{A_0^* B_0 + \text{c.c.}} \\
        &= \frac{1}{2} A_0^* B_0 = \frac{1}{2} A_0 B_0^*.
    \end{aligned}
\end{equation}

\subsubsection{}

When 
\begin{equation}
    \vb*{E} = \vu*{x} \Re \tilde{E}_0 \ee^{- \ii (\omega t - k z)},
\end{equation}
from 
\begin{equation}
    \curl{\vb*{E}} = - \pdv*{\vb*{B}}{t}
\end{equation}
we obtain 
\begin{equation}
    \begin{aligned}
        &\ii \vb*{k} \times \vb*{E} = - (- \ii \omega) \vb*{B}  \\
        \Rightarrow& \vb*{B} = \frac{1}{\omega} k \vu*{z} \times \vb*{E}
        = \frac{k}{\omega} \vu*{y} \Re \tilde{E}_0 \ee^{- \ii (\omega t - k z)},
    \end{aligned}
\end{equation}
and therefore 
\begin{equation}
    \expval{\vb*{S}} = \frac{1}{\mu_0} \expval{\vb*{E} \times \vb*{B}}
    = \frac{1}{\mu_0}  \cdot \frac{1}{2} \Re 
    \underbrace{\vu*{x} \tilde{E}_0 \ee^{\ii k z}}_{\Efreq_0} \times 
    \underbrace{\frac{k}{\omega} \vu*{y} \tilde{E}_0^* \ee^{- \ii k z}}_{\tilde{\vb*{B}}_0}
    = \frac{k}{2 \mu_0 \omega} \abs*{\tilde{E}_0}^2 \vu*{z},
\end{equation}
and since the refraction index is $n$, 
we eventually get 
\begin{equation}
    \omega = k \cdot \frac{c}{n}
\end{equation}
and therefore 
\begin{equation}
    \expval{\vb*{S}} = \frac{n}{2} \sqrt{\frac{\epsilon_0}{\mu_0}} \abs*{\tilde{E}_0}^2 \vu*{z}.
\end{equation}
The direction of the energy flow is parallel to the $z$ axis.

\subsubsection{}

The expected value of the electric energy density is 
\begin{equation}
    \expval{u_e} = \frac{1}{2} \epsilon_0 \epsr \expval{\vb*{E}^2}
    = \frac{1}{2} \epsilon_0 n^2 \cdot \frac{1}{2} \abs*{\tilde{E}_0}^2 
    = \frac{1}{4} \epsilon_0 n^2 \abs*{\tilde{E}}^2,
\end{equation}
and the expected value of the magnetic energy density is 
\begin{equation}
    \expval{u_m} = \frac{1}{2 \mu_0} \expval{\vb*{B}^2} 
    = \frac{1}{2 \mu_0} \cdot \frac{1}{2} \frac{k^2}{\omega^2} \abs*{\tilde{E}_0}^2 
    = \frac{1}{4} \frac{n^2}{c^2 \mu_0} \abs{\tilde{E}_0}^2 
    = \frac{1}{4} \epsilon_0 n^2 \abs{\tilde{E}_0}^2.
\end{equation}
So we find 
\begin{equation}
    \frac{\expval{u_e}}{\expval{u_m}} = 1.
\end{equation}

\section{Lorentz oscillator in an AC field and optical forces}

\subsection{Optical response of an ensemble of Lorentz oscillators}

Consider a dilute ensemble of Lorentz oscillators, uniformly distributed over space with number density $N$, in an $\mathrm{AC}$ electric field given by $\mathbf{E}=\operatorname{Re}\left[\tilde{\mathbf{E}}_0 e^{-i \omega t}\right]$. Each oscillator is driven by the local electric field according to the equation of motion given by
$$
\ddot{\mathbf{p}}+\gamma \dot{\mathbf{p}}+\Omega^2 \mathbf{p}=\frac{q^2}{m} \mathbf{E}(\mathbf{r}),
$$
where $\mathbf{r}, m$, and $q$ are the respective oscillator position, reduced mass, and charge.

\subsubsection{}

The polarization density is 
\begin{equation}
    \vb*{P} = N \vb*{p}.
\end{equation}
The EOM for $\vb*{P}$ is 
\begin{equation}
    \ddot{\vb*{P}} + \gamma \dot{\vb*{P}} + \Omega^2 \vb*{P} = \frac{N q^2}{m} \vb*{E}.
\end{equation}
We can switch to the Fourier representation.
Thus we have 
\begin{equation}
    ((-\ii \omega)^2 + \gamma (- \ii \omega) + \Omega^2) \tilde{\vb*{P}} = \frac{N q^2}{m} \Efreq,
\end{equation}
and from 
\begin{equation}
    \vb*{D} = \epsilon_0 \vb*{E} + \vb*{P}
\end{equation}
we get 
\begin{equation}
    \Dfreq = \epsilon_0 \epsr \Efreq, \quad 
    \epsr(\vb*{k}, \omega) = 1 + \frac{\omegap^2}{- \omega^2 - \ii \gamma \omega + \Omega^2}, \quad 
    \omegap^2 = \frac{N q^2}{m \epsilon_0}.
\end{equation}
So we already get $\epsr$; it has explicit dependence on $\omega$,
but not $\vb*{k}$.

\subsubsection{}

The form of a propagating plane wave in the free space is 
\begin{equation}
    \vb*{E} = \vb*{E}_0 \ee^{\ii (\vb*{k} \cdot \vb*{r} - \omega t)} + \text{c.c.}, \quad 
    \vb*{k} = k \vu*{k}, \quad k = \frac{\sqrt{\epsr} \omega}{c} = \frac{n \omega}{c}.
\end{equation}
Note that it's possible that $k$ has an imaginary part 
and $n$ is the complex refraction index.
The direction of $\Re \vb*{k}$ and $\Im \vb*{k}$ 
is assumed to be the same.
The phase velocity is given by 
\begin{equation}
    v = \frac{\omega}{\Re k} = \frac{c}{\Re n} = \frac{c}{\Re \sqrt{\epsr}}
    = \frac{c}{
        \Re \sqrt{
            1 + \frac{\omegap^2}{- \omega^2 - \ii \gamma \omega + \Omega^2}
        }
    }.
\end{equation}
The group velocity is 
\begin{equation}
    v_{\text{g}} = \frac{\dd{\omega}}{\dd{\Re k}}
    = c \left(
        \dv{
            \Re \sqrt{
                1 + \frac{\omegap^2}{- \omega^2 - \ii \gamma \omega + \Omega^2}
            }
        }{\omega}
    \right)^{-1}.
\end{equation}

\subsubsection{}

Since $\epsr$ has frequency dependence,
the relation between $\vb*{E}(t)$ and $\vb*{D}(t)$ 
is not localized in the time domain,
and therefore although we still know that 
the energy would be a quadratic form of $\vb*{E}$ or $\vb*{D}$,
since 
\begin{equation}
    \pdv{\vb*{E}}{t} \cdot \vb*{D} \neq \pdv{\vb*{D}}{t} \cdot \vb*{E},
\end{equation}
the simple relation 
\[
    u_e = \frac{1}{2} \vb*{D} \cdot \vb*{E} 
\]
no longer holds.
Instead, we should start from the most generic theory 
and utilize 
\begin{equation}
    \pdv{u_e}{t} = \vb*{E} \cdot \pdv{\vb*{D}}{t}.
    \label{eq:change-of-ue}
\end{equation} 
To use this equation to get an expression of $u_e$,
we should no longer work with plane waves, 
or otherwise $u_e$ is a constant and 
we don't see any change of $u_e$ at all.
Below we work with a wave packet centered at $\pm \omega_0$.
For the wave packet, the electric field is 
\begin{equation}
    \vb*{E}(t) = \ee^{- \ii \omega_0 t} \cdot 
        \underbrace{
            \int \frac{\dd{\omega}}{2\pi} \ee^{- \ii (\omega - \omega_0) t}
            \Efreq(\omega)
        }_{\eqqcolon \vb*{E}_0(t)},
\end{equation}
\begin{equation}
    \vb*{D}(t) = \ee^{- \ii \omega_0 t} \cdot 
        \int \frac{\dd{\omega}}{2\pi} \ee^{- \ii (\omega - \omega_0) t}
            \varepsilon(\omega) \Efreq(\omega) .
\end{equation}
By Taylor expansion of $\varepsilon$ we have 
\begin{equation}
    \begin{aligned}
        \pdv*{\vb*{D}}{t} &= 
        \ee^{- \ii \omega_0 t} \cdot 
        \int \frac{\dd{\omega}}{2\pi} \ee^{- \ii (\omega - \omega_0) t} \Efreq(\omega) 
        (-\ii \omega) \left(
            \varepsilon(\omega_0) + 
            (\omega - \omega_0) \eval{\dv{\varepsilon}{\omega}}_{\omega = \omega_0}
            + \cdots 
        \right) \\
        &\approx \ee^{- \ii \omega_0 t} \cdot 
        \int \frac{\dd{\omega}}{2\pi} \ee^{- \ii (\omega - \omega_0) t} \Efreq(\omega) 
        \left(
            - \ii \omega_0 \varepsilon(\omega_0) 
            - \ii (\omega - \omega_0) \varepsilon(\omega)_0 
            - \ii (\omega - \omega_0) \omega 
                \eval{\dv{\varepsilon}{\omega}}_{\omega = \omega_0}
        \right) \\
        &\approx \ee^{- \ii \omega_0 t} \cdot 
        \int \frac{\dd{\omega}}{2\pi} \ee^{- \ii (\omega - \omega_0) t} \Efreq(\omega) 
        \Bigg(
            - \ii \omega_0 \varepsilon(\omega_0) 
            \underbrace{
                - \ii (\omega - \omega_0) \varepsilon(\omega)_0 
                - \ii (\omega - \omega_0) \omega_0  
                    \eval{\dv{\varepsilon}{\omega}}_{\omega = \omega_0}
            }_{
                = - \ii (\omega - \omega_0) 
                \eval{\dv{(\omega \varepsilon)}{\omega}}_{\omega = \omega_0} 
            }
        \Bigg) \\
        &= \ee^{- \ii \omega_0 t} \underbrace{
            \left(
                - \ii \omega_0 \varepsilon(\omega_0) \vb*{E}_0(t)
                + \dv{(\omega \varepsilon)}{\omega} 
                \pdv{\vb*{E}_0}{t}  
            \right)
        }_{\eqqcolon \vb*{D}_0(t)} .
    \end{aligned}
    \label{eq:pdv-d-t-derivation}
\end{equation}
In the second line we throw away the higher order Taylor terms;
in the third line we only keep terms linear to $(\omega - \omega_0)$.
These approximations require the wave packet 
to be focused enough.
We use $\expval{\cdots}$ to refer to 
averaging over the fast oscillations;
thus, $\vb*{E}_0(t)$ and $\vb*{D}_0(t)$ above 
can be regarded as constants when applying $\expval{\cdots}$,
and hence we find 
\begin{equation}
    \begin{aligned}
        \expval{\pdv{u_e}{t}} &= \expval{\vb*{E} \cdot \pdv{\vb*{D}}{t}}
        = \frac{1}{2} \cdot \frac{1}{4} \Re (
            \vb*{D}_0^*(t) \cdot \vb*{E}_0(t)
            + \vb*{D}_0(t) \cdot \vb*{E}_0^*(t)
        ) \\
        &\approx \frac{1}{4} \Re \left(
            \left(
                \omega_0 \varepsilon_2(\omega_0) 
                + \dv{(\omega \varepsilon_1)}{\omega} 
            \right)
                \pdv{\vb*{E}_0^*}{t} \cdot \vb*{E}_0 
            + \left(
                \omega_0 \varepsilon_2(\omega_0)
                + \dv{(\omega \varepsilon_1)}{\omega} 
            \right)
            \vb*{E}_0^* \cdot \pdv{\vb*{E}_0}{t} 
        \right) \\
        &= \frac{1}{4} \left(
            \omega_0 \varepsilon_2(\omega_0) +
            \dv{(\omega \varepsilon_1)}{\omega}
        \right) \pdv{\abs*{\vb*{E}_0}^2}{t} .
    \end{aligned}
\end{equation}
In the second line we have considered 
both the real and imaginary parts of $\epsilon$.%
\footnote{
    Note that $\epsilon(\omega) = \epsilon(- \omega)^*$
    comes from the fact that $\epsilon$ is real in the time domain;
    it says nothing about whether the system is Hermitian;
    the Hermitian condition is $\epsilon(\omega) = \epsilon(\omega)^*$.
} 
Since $u_e$ contains no fast oscillation, we have 
\begin{equation}
    \pdv{\expval{u_e}}{t} = \expval{\pdv{u_e}{t}} 
    = \frac{1}{4} \left(
        \omega_0 \varepsilon_2(\omega_0) +
        \dv{(\omega \varepsilon_1)}{\omega}
    \right)  
    \pdv{\abs*{\vb*{E}_0}^2}{t} .
\end{equation}
Similarly we have 
\begin{equation}
    \pdv{\expval{u_m}}{t} = \expval{\pdv{u_m}{t}} 
    = \frac{1}{4} \left(
        \omega_0 \mu_2(\omega_0) +
        \dv{(\omega \mu_1)}{\omega}
    \right) 
    \pdv{\abs*{\vb*{H}_0}^2}{t} .
\end{equation}
When the matter is modeled by harmonic oscillators, 
$\mu$ doesn't undergo any correction, 
but let's work with a slightly generalized case.
Eventually we have 
\begin{equation}
    \expval{u} = \expval{u_e + u_m} 
    = \frac{1}{4} \left(
        \omega_0 \varepsilon_2(\omega_0) +
        \dv{(\omega \varepsilon_1)}{\omega}
    \right)  
    \abs*{\vb*{E}_0}^2 + 
    \frac{1}{4} \left(
        \omega_0 \mu_2(\omega_0) +
        \dv{(\omega \mu_1)}{\omega}
    \right) 
    \abs*{\vb*{H}_0}^2.
\end{equation}
In the case of the Lorentz oscillator,
$\mu$ is real, and we have 
\begin{equation}
    \expval{u_m} = \frac{1}{4} \mu_0 \abs*{\vb*{H}_0}^2 
    = \frac{1}{4} \mu_0 \cdot \frac{\abs*{\varepsilon}}{\mu_0} \abs*{\vb*{E}_0}^2
    = \frac{1}{4} \abs{\varepsilon} \abs*{\vb*{E}_0}^2.
\end{equation}

The evaluation of the time averaged Poynting vector is more straightforward:
since 
\begin{equation}
    \curl{\vb*{E}} = - \pdv{\vb*{B}}{t} \Rightarrow
    \ii \vb*{k} \times \vb*{E} = - (- \ii \omega) \vb*{B},
\end{equation}
we just have 
\begin{equation}
    \begin{aligned}
        \expval{\vb*{S}} &= \frac{1}{\mu} \expval{\vb*{E} \times \vb*{B}} \\
        &= \frac{1}{\mu} \cdot \frac{1}{4} \Re \left(
            \vb*{E}^*_0 \times \vb*{B}_0 + \vb*{E}_0 \times \vb*{B}^*_0 
        \right) \\
        &= \frac{1}{2 \mu} \frac{\Re \vb*{k}}{\omega} \abs*{\vb*{E}_0}^2
        = \frac{1}{2} \Re \sqrt{\frac{\varepsilon}{\mu}} \abs*{\vb*{E}_0}^2 \vu*{k} ,
    \end{aligned}
\end{equation}
where we have used the condition $\vb*{k} \cdot \vb*{E} = 0$,
and in the third line we have used the condition that 
the directions of the real and imaginary parts of $\vb*{k}$
are the same, and therefore 
from $\vb*{k} \cdot \vb*{E}_0 = 0$ we also have 
$\vb*{k}^* \cdot \vb*{E} = 0$.
The energy velocity is therefore 
\begin{equation}
    v_{E} = \frac{\abs*{\expval{\vb*{S}}}}{\expval{u_e + u_m}}
    = 
\end{equation}

\subsection{Optical Tweezers}

\subsubsection{}

\subsection{A simple derivation of electrostrictive pressure}

\paragraph{(a)}
The interaction energy between electromagnetic field and the matter is 
\begin{equation}
    U_{\text{int}} = - \vb*{P} \cdot \vb*{E}
    = - \epsilon_0 (\epsr - 1) \vb*{E} \cdot \vb*{E},
\end{equation}
and therefore 
\begin{equation}
    \expval{U_{\text{int}}} = - \epsilon_0 
\end{equation}

\subsection{Fourier-Domain Treatment of the Wave Equation}

\paragraph{(a)} Separation of variables fails whenever 
formally carrying out the separation of variable procedure 
\emph{doesn't} lead to good Sturm-Liouville eigenvalue problems.
This happens in several cases.
Maybe the structure of the equation is not good, e.g. 
when the problem is to find how the density change 
with a given but complicated velocity distribution:
\begin{equation}
    \pdv{\rho}{t} + \div{\rho \vb*{v}} = 0,
\end{equation}
from which separation of variable simply can't proceed. 
Another case is when the boundary condition is very complicated 
or simply has a weird geometrical shape.

\paragraph{(b)}
From the 1D wave equation 
\begin{equation}
    \partial^2_x E(x, t) - \frac{\varepsilon}{c^2} \partial_t^2 E(x, t) = 0,
\end{equation}
and the ansatz 
\begin{equation}
    E(x, t) = \bar{E} (x) f(t), \quad 
    \bar{E}(x) = \int \dd{k} \tilde{E}(k) \ee^{\ii k x} , \quad 
    f(t) = \int \dd{\omega} \tilde{f}(\omega) \ee^{- \ii \omega t}
\end{equation}
we get 
\begin{equation}
    \int \dd{\omega} \int \dd{k}
        \left(
            (-\ii k)^2 - \frac{\varepsilon}{c^2} (- \ii \omega)^2
        \right) 
        \ee^{\ii (kx - \omega t)} \tilde{f}(\omega) \tilde{E}(k) = 0,
\end{equation}
and therefore the equation in frequency domain is 
\begin{equation}
    \left(
        k^2 - \frac{\varepsilon \omega^2}{c^2}
    \right) \tilde{f}(\omega) \tilde{E}(k) = 0. 
\end{equation}

\paragraph{(c)}
The dispersion relation is 
\begin{equation}
    k = \pm \sqrt{\varepsilon} \frac{\omega}{c}.
\end{equation}
Since $\varepsilon$ is a constant, 
this means the equation has no dispersion.
The solution of the equation therefore can be 
rewritten as (we here redefine $\tilde{E}$) 
\begin{equation}
    E(x, t) = \int \frac{\dd{\omega}}{2\pi} \left(
        \tilde{E}_1(\omega) \ee^{\ii \left(
            \sqrt{\varepsilon} \frac{\omega}{c} x - \omega t
        \right)} + 
        \tilde{E}_2(\omega) \ee^{\ii \left(
            - \sqrt{\varepsilon} \frac{\omega}{c} x - \omega t
        \right)}
    \right)  . 
\end{equation}
So there are two solutions corresponding to each $\omega$;
by linear recombination of the solutions, 
the two are left-going and right-going, correspondingly.

\paragraph{(d)} We define the complex refraction index as 
\begin{equation}
    \tilde{n} = \sqrt{\varepsilon} = n + \ii \kappa,
\end{equation}
and 
\begin{equation}
    k = \frac{\omega}{c} (n + \ii \kappa).
\end{equation}
If we let $k$ be complex, we have 
\begin{equation}
    \begin{aligned}
        E(x, t) &= E_0 \ee^{\ii (kx - \omega t)} + \text{c.c.} \\
        &= E_0 \ee^{- \kappa \frac{\omega}{c} x} \ee^{\ii \left(
            n \frac{\omega}{c} x - \omega t
        \right)} + \text{c.c.},
    \end{aligned}
    \label{eq:damping-case-1}
\end{equation} 
and the decay length is 
\begin{equation}
    l = \frac{1}{\kappa \frac{\omega}{c}}
    = \frac{c}{\omega \kappa}  .
    \label{eq:damping-length-1}
\end{equation}

On the other hand, if we let $\omega$ be complex, 
we have 
\begin{equation}
    \omega = \frac{ck}{n} = \frac{ck}{n^2 + \kappa^2} (n - \ii \kappa),
\end{equation}
and the wave looks like 
\begin{equation}
    E(x, t) = E_0 \ee^{
        - k c \frac{\kappa}{n^2 + \kappa^2} t 
    } \ee^{
        \ii \left(
            kx - k c \frac{n}{n^2 + \kappa^2} t
        \right)
    } + \text{c.c.}
\end{equation}
This solution can be achieved by injecting a plane wave into the medium 
and removing the pumping source, 
letting the wave ``cooling down'' within the medium,
while the first solution \eqref{eq:damping-case-1}, where $k$ is complex, 
describes the field configuration where at the boundary of the medium 
the field strength is fixed, 
probably because of a strong and stable source outside.
Now the time scale of damping is 
\begin{equation}
    \tau = \frac{n^2 + \kappa^2}{\kappa} \frac{1}{kc},
\end{equation}
and in this ``homogeneously cooling down'' case, 
the speed of the light becomes 
\begin{equation}
    v = \frac{n}{n^2 + \kappa^2} c,
\end{equation}
not just the $c / n$ in \eqref{eq:damping-case-1}.
Now the decay length can be found from $\tau$ as 
\begin{equation}
    l = v \tau = \frac{n}{\kappa k}.
    \label{eq:damping-length-2}
\end{equation}
If we make identification $c / n = \omega / k$,
we find \eqref{eq:damping-length-1} and \eqref{eq:damping-length-2} are the same.
\todo{But I have no idea why we don't make identification $\omega = kc n / (n^2 + \kappa^2)$}

\subsubsection{Non-instantaneous material response}

The most general relation between $D$ and $E$ is 
\begin{equation}
    D(x, t) = \varepsilon_0 \int_{-\infty}^{\infty} \dd{t'} 
    \int_{-\infty}^{\infty} \dd{x'} 
    \epsr(x - x', t - t') E(x', t').
\end{equation}

\paragraph{(a)} When we have locality and causality, 
$\epsr(x - x', t - t')$ should be proportional to 
$\delta(x - x')$ and $\theta(t - t')$,
and thus 
\begin{equation}
    D(x, t) = \varepsilon_0 \int_{-\infty}^{t} \dd{t'} 
    \epsr(t - t') E(x, t'), 
\end{equation}
\begin{equation}
    \epsr(t - t') = \epsr(t - t') \theta(t - t').
\end{equation}
Here $\epsr(x - x', t - t')$ shouldn't have any space dependence 
other than $\delta(x - x')$ or it's not a function of $x - x'$.

\paragraph{(b)} The 1D electro-magnetic wave function then
is the consequence of the following equation systems
in the 1D case:
\begin{equation}
    \curl{\vb*{E}} = - \pdv{\vb*{B}}{t} , \quad 
    \curl{\vb*{H}} = \pdv{\vb*{D}}{t}, \quad 
    \vb*{B} = \mu_0 \vb*{H}, \quad 
    \vb*{D}(x, t) = \varepsilon_0 \int_{-\infty}^{\infty} \dd{t'} 
    \epsr(t - t') \vb*{E}(x, t').
\end{equation}
Here we have assumed that $\epsr$ doesn't map $E_x$ to $D_y$, etc.,
or otherwise the problem can't be restricted to the 1D case,
and thus from $\div{\vb*{D}} = 0$ we get $\div{\vb*{E}} = 0$.
Hence 
\begin{equation}
    \begin{aligned}
        - \laplacian \vb*{E} = \curl{(\curl{\vb*{E}})} = 
        - \mu_0 \pdv{t} \curl{\vb*{H}}
        = - \mu_0 \pdv[2]{t} \vb*{D},
    \end{aligned}
\end{equation}
and therefore the final wave equation is 
\begin{equation}
    \partial_x^2 E(x, t) = \frac{1}{c^2} \pdv[2]{t} 
    \int_{-\infty}^{\infty} \dd{t'} \epsr(t - t') E(x, t').
    \label{eq:wave-eq-retarded}
\end{equation}

\paragraph{(c)} We do Fourier transform 
\begin{equation}
    E(x, t) = \int \frac{\dd{\omega} \dd{k}}{(2\pi)^2}
    \ee^{\ii (kx - \omega t)} \tilde{E}(k, \omega), \quad 
    \varepsilon(t - t') = \int \frac{\dd{\omega}}{2\pi} \ee^{- \ii \omega t} 
    \tilde{\varepsilon}_{\text{r}}(\omega),
\end{equation}
and LHS of \eqref{eq:wave-eq-retarded} becomes 
\begin{equation}
    \int \frac{\dd{\omega} \dd{k}}{(2\pi)^2} (\ii k)^2
    \ee^{\ii (kx - \omega t)} \tilde{E}(k, \omega),
\end{equation}
and the RHS becomes 
\begin{equation}
    \begin{aligned}
        &\frac{1}{c^2} \pdv[2]{t} \int \dd{t'} 
        \int \frac{\dd{\omega'}}{2\pi} \int \frac{\dd{\omega} \dd{k}}{(2\pi)^2} 
        \tilde{\varepsilon}_{\text{r}}(\omega') \ee^{-\ii \omega' (t - t')} 
        \tilde{E}(k, \omega) \ee^{\ii (kx - \omega t')} \\
        =& \frac{1}{c^2} \int \frac{\dd{\omega'}}{2\pi} \frac{\dd{\omega}}{2\pi} \frac{\dd{k}}{2\pi}
        (- \ii \omega)^2 \tilde{\varepsilon}_{\text{r}}(\omega') \tilde{E}(k, \omega) 
        \ee^{\ii k x} \int \dd{t'} \ee^{\ii (\omega' - \omega) t'} \\
        =&  \frac{1}{c^2} \int \frac{\dd{\omega'}}{2\pi} \frac{\dd{\omega}}{2\pi} \frac{\dd{k}}{2\pi}
        (- \ii \omega)^2 \tilde{\varepsilon}_{\text{r}}(\omega') \tilde{E}(k, \omega) 
        \ee^{\ii k x - \ii \omega' t} \cdot 2\pi \delta(\omega - \omega') \\
        =& \frac{1}{c^2} \int \frac{\dd{\omega} \dd{k}}{(2\pi)^2}
        \ee^{\ii k x - \ii \omega t} (- \ii \omega)^2 \tilde{\varepsilon}_{\text{r}}(\omega) \tilde{E}(k, \omega) .
    \end{aligned}
\end{equation}
The final equation therefore becomes 
\begin{equation}
    \left(
        k^2 - \tilde{\varepsilon}_{\text{r}}(\omega) \frac{\omega^2}{c^2}
    \right) \tilde{E}(k, \omega) = 0.
\end{equation}

\paragraph{(d)} The dispersion relation now is
\begin{equation}
    \omega = \frac{c \abs*{k}}{\sqrt{\tilde{\varepsilon}_{\text{r}}(\omega)}} .
\end{equation}

\subsubsection{Energy absorption}

\paragraph{(a)} In the specific case of 
\begin{equation}
    E(x, t) = \bar{E}(x) \ee^{- \ii \omega t} + \bar{E}^*(x) \ee^{\ii \omega t},
\end{equation}
we have 
\begin{equation}
    \begin{aligned}
        D(x, t) &= \int \dd{t'} \varepsilon(t - t') (
            \bar{E}(x) \ee^{- \ii \omega t'} + \bar{E}^*(x) \ee^{\ii \omega t'}
        ) \\
        &= \bar{E}(x) \ee^{- \ii \omega t} \int \dd{t'} 
        \ee^{\ii \omega (t - t')} \varepsilon(t - t') 
        + \bar{E}^*(x) \ee^{\ii \omega t} \int \dd{t'} 
        \ee^{- \ii \omega (t - t')} \varepsilon(t - t' ) \\
        &= \bar{E}(x) \ee^{- \ii \omega t} \tilde{\varepsilon}(\omega)
        + \bar{E}^*(x) \ee^{\ii \omega t} \tilde{\varepsilon}(- \omega).
    \end{aligned}
\end{equation}
Since $\varepsilon(t)$ is real, we can rewrite 
\begin{equation}
    \tilde{\varepsilon}(- \omega) = \tilde{\varepsilon}(\omega)^*.
\end{equation}

\paragraph{(b)} Following the practice in \eqref{eq:change-of-ue}, we have 
\begin{equation}
    \begin{aligned}
        \pdv{u_e}{t} &= E \cdot \pdv{D}{t} \\
        &= (
            \bar{E}(x) \ee^{- \ii \omega t} + \bar{E}^*(x) \ee^{\ii \omega t}
        ) \cdot (
            (- \ii \omega) \bar{E}(x) \ee^{- \ii \omega t} \tilde{\varepsilon}(\omega)
            + (\ii \omega) \bar{E}^*(x) \ee^{\ii \omega t} \tilde{\varepsilon}(- \omega)
        ) \\
        &= - 2 \omega \abs*{\bar{E}(\omega)}^2 \varepsilon_2(\omega)
        + \bar{E}(x)^2 (- \ii \omega) \ee^{- 2 \ii \omega t} \tilde{\varepsilon}(\omega)
        + \text{c.c.}
    \end{aligned}
\end{equation}
Here $\varepsilon_1$ and $\varepsilon_2$ 
are real and imaginary parts of $\varepsilon$, respectively.
The sum of the second and third terms is a sine function; 
it oscillates and has zero average value.
The first term gives the damping.
Thus, the energy absorption rate is 
\begin{equation}
    \partial_t \expval{u_e} = - 2 \omega \abs*{\bar{E}(\omega)}^2 \varepsilon_2(\omega).
\end{equation}

\paragraph{(c)} The energy absorbed is proportional to the imaginary part 
of the Fourier transform of the dielectric constant, 
which is expected since it's related to the imaginary part of the energy spectrum.

\end{document}