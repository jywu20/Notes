\documentclass[hyperref, a4paper]{article}

\usepackage{geometry}
\usepackage{titling}
\usepackage{titlesec}
% No longer needed, since we will use enumitem package
% \usepackage{paralist}
\usepackage{enumitem}
\usepackage{footnote}
\usepackage[colorinlistoftodos]{todonotes}
\usepackage{amsmath, amssymb, amsthm}
\usepackage{mathtools}
\usepackage{bbm}
\usepackage{graphicx}
\usepackage{subcaption}
\usepackage{soulutf8}
\usepackage{physics}
\usepackage{tensor}
\usepackage{siunitx}
\usepackage[version=4]{mhchem}
\usepackage{tikz}
\usepackage{xcolor}
\usepackage{listings}
\usepackage{autobreak}
\usepackage[ruled, vlined, linesnumbered]{algorithm2e}
\usepackage{nameref,zref-xr}
\zxrsetup{toltxlabel}
\usepackage[backend=bibtex]{biblatex}
\addbibresource{elasticity.bib}
\usepackage[colorlinks,unicode]{hyperref} % , linkcolor=black, anchorcolor=black, citecolor=black, urlcolor=black, filecolor=black
\usepackage[most]{tcolorbox}
\usepackage{prettyref}

% Page style
\geometry{left=3.18cm,right=3.18cm,top=2.54cm,bottom=2.54cm}
\titlespacing{\paragraph}{0pt}{1pt}{10pt}[20pt]
\setlength{\droptitle}{-5em}

% More compact lists 
\setlist[itemize]{
    itemindent=17pt, 
    leftmargin=1pt,
    listparindent=\parindent,
    parsep=0pt,
}

% Math operators
\DeclareMathOperator{\timeorder}{\mathcal{T}}
\DeclareMathOperator{\diag}{diag}
\DeclareMathOperator{\legpoly}{P}
\DeclareMathOperator{\primevalue}{P}
\DeclareMathOperator{\sgn}{sgn}
\DeclareMathOperator{\res}{Res}
\newcommand*{\ii}{\mathrm{i}}
\newcommand*{\ee}{\mathrm{e}}
\newcommand*{\const}{\mathrm{const}}
\newcommand*{\suchthat}{\quad \text{s.t.} \quad}
\newcommand*{\argmin}{\arg\min}
\newcommand*{\argmax}{\arg\max}
\newcommand*{\normalorder}[1]{: #1 :}
\newcommand*{\pair}[1]{\langle #1 \rangle}
\newcommand*{\fd}[1]{\mathcal{D} #1}
\DeclareMathOperator{\bigO}{\mathcal{O}}

% TikZ setting
\usetikzlibrary{arrows,shapes,positioning}
\usetikzlibrary{arrows.meta}
\usetikzlibrary{decorations.markings}
\tikzstyle arrowstyle=[scale=1]
\tikzstyle directed=[postaction={decorate,decoration={markings,
    mark=at position .5 with {\arrow[arrowstyle]{stealth}}}}]
\tikzstyle ray=[directed, thick]
\tikzstyle dot=[anchor=base,fill,circle,inner sep=1pt]

% Algorithm setting
% Julia-style code
\SetKwIF{If}{ElseIf}{Else}{if}{}{elseif}{else}{end}
\SetKwFor{For}{for}{}{end}
\SetKwFor{While}{while}{}{end}
\SetKwProg{Function}{function}{}{end}
\SetArgSty{textnormal}

\newcommand*{\concept}[1]{{\textbf{#1}}}

% Embedded codes
\lstset{basicstyle=\ttfamily,
  showstringspaces=false,
  commentstyle=\color{gray},
  keywordstyle=\color{blue}
}

% Reference formatting
\newcommand*{\citesec}[1]{\S~{#1}}
\newcommand*{\citechap}[1]{chap.~{#1}}
\newcommand*{\citefig}[1]{Fig.~{#1}}
\newcommand*{\citetable}[1]{Table~{#1}}
\newcommand*{\citepage}[1]{pp.~{#1}}
\newrefformat{fig}{Fig.~\ref{#1}}
\newcommand*{\term}[1]{\textit{#1}}

% Color boxes
\tcbuselibrary{skins, breakable, theorems}

\newtcbtheorem{infobox}{Box}{
    enhanced,
    boxrule=0pt,
    colback=blue!5,
    colframe=blue!5,
    coltitle=blue!50,
    borderline west={4pt}{0pt}{blue!65},
    sharp corners,
    fonttitle=\bfseries, 
    breakable,
    before upper={\parindent15pt\noindent}}{box}
\newtcbtheorem[use counter from=infobox]{theorybox}{Box}{
    enhanced,
    boxrule=0pt,
    colback=orange!5, 
    colframe=orange!5, 
    coltitle=orange!50,
    borderline west={4pt}{0pt}{orange!65},
    sharp corners,
    fonttitle=\bfseries, 
    breakable,
    before upper={\parindent15pt\noindent}}{box}
\newtcbtheorem[use counter from=infobox]{learnbox}{Box}{
    enhanced,
    boxrule=0pt,
    colback=green!5,
    colframe=green!5,
    coltitle=green!50,
    borderline west={4pt}{0pt}{green!65},
    sharp corners,
    fonttitle=\bfseries, 
    breakable,
    before upper={\parindent15pt\noindent}}{box}


\newenvironment{shelldisplay}{\begin{lstlisting}}{\end{lstlisting}}

\newcommand*{\kB}{k_{\text{B}}}
\newcommand*{\muB}{\mu_{\text{B}}}
\newcommand*{\efermi}{E_{\text{F}}}
\newcommand*{\pfermi}{p_{\text{F}}}
\newcommand*{\vfermi}{v_{\text{F}}}
\newcommand*{\sA}{\text{A}}
\newcommand*{\sB}{\text{B}}
\newcommand*{\Tc}{T_{\text{c}}}
\newcommand*{\hethree}{$^3$He}
\newcommand*{\hefour}{$^4$He}
\newcommand{\epsr}{\epsilon_{\text{r}}}
\newcommand{\chie}{\chi_{\text{e}}}
\newcommand{\Efreq}{\tilde{\vb*{E}}}
\newcommand{\Dfreq}{\tilde{\vb*{D}}}
\newcommand{\Pfreq}{\tilde{\vb*{P}}}

\title{Homework 1}
\author{Jinyuan Wu}

\begin{document}

\maketitle

\section{Maxwell's equations in dielectrics, Lorentz oscillators, and complex notation}

\subsection{Time-Average Quantities in Complex Notation}

It is often important to be able to compute time-averaged quantities, such as the potential energy of a harmonic oscillator $U_{p e}=\frac{k}{2}\left\langle x^2\right\rangle$ or the electric field energy density $U_{\mathrm{el}}=\frac{\varepsilon_0}{2}\left\langle\mathbf{E}^2\right\rangle$. Here, the time-average of a function, $f(t)$, is defined as, $\langle f(t)\rangle=(1 / T) \int_{t-T / 2}^{t+T / 2} d t^{\prime} f\left(t^{\prime}\right)$, where $T$ is defined as either the characteristic period of the oscillating system (i.e., $T=2 \pi / \omega)$ or infinity. Such time averaging is drastically simplified by using complex notation.

To see this, suppose that we have any two functions $A(t)$ and $B(t)$, both of which take on a time harmonic form. Without loss of generality, we assume that $A(t)=A_0 \cos (\omega t+\phi)$, and $B(t)=B_0 \cos (\omega t+\theta)$, where $\phi$ and $\theta$ are arbitrary phase factors.

\subsubsection{}

We have 
\begin{equation}
    \begin{aligned}
        \expval{A(t) B(t)} &= 
        \frac{1}{T} \int_{t - T/2}^{t + T/2} \dd{t'} 
            A_0 \cos(\omega t' + \phi) 
            B_0 \cos (\omega t' + \theta) \\
        &= A_0 B_0 \frac{1}{T} \int_{t - T/2}^{t + T/2} \dd{t'} 
            \frac{1}{2} (
                \cos(\omega t' + \phi + \omega t' + \theta) 
                + \cos (\omega t' + \phi - \omega t' - \theta)
            ) \\
        &= \frac{1}{2} A_0 B_0 \cos (\phi - \theta).
    \end{aligned}
\end{equation}
Here we have used the condition that $T = 2 \pi / \omega$ 
so that the first term vanishes.

\subsubsection{}

We have 
\begin{equation}
    A(t) = \tilde{A}_0 \ee^{- \ii \omega t} , \quad 
    B(t) = \tilde{B}_0 \ee^{- \ii \omega t} , \quad 
    \tilde{A}_0 = A_0 \ee^{- \ii \phi}, \quad 
    \tilde{B}_0 = B_0 \ee^{- \ii \theta},
\end{equation}
and therefore 
\begin{equation}
    \Re \tilde{A}_0 B_0 = \Re A_0 \tilde{B}_0 = 
    \Re A_0 B_0 \ee^{\ii (\phi - \theta)}
    = A_0 B_0 \cos(\phi - \theta),
\end{equation}
and hence 
\begin{equation}
    \expval{A(t) B(t)} = \frac{1}{2} \Re \tilde{A}_0 B_0 
    = \frac{1}{2} \Re A_0 \tilde{B}_0 .
\end{equation}

We can also straightforwardly do the follows. 
We have 
\begin{equation}
    \begin{aligned}
        \expval{A(t) B(t)} &= \expval{
            \frac{1}{2} (\tilde{A}(t) + \tilde{A}^*(t))
            \cdot \frac{1}{2} (\tilde{B}(t) + \tilde{B}^*(t))
        } \\
        &= \frac{1}{4} \expval{
            \tilde{A}_0 \tilde{B}_0 \ee^{- 2 \ii \omega t} + 
            \tilde{A}_0 \tilde{B}_0^* + 
            \tilde{A_0}^* \tilde{B}_0^* \ee^{2 \ii \omega t} 
            + \tilde{A}_0^* \tilde{B}_0
        } \\
        &= \frac{1}{4} \expval{A_0^* B_0 + \text{c.c.}} \\
        &= \frac{1}{2} A_0^* B_0 = \frac{1}{2} A_0 B_0^*.
    \end{aligned}
\end{equation}

\subsubsection{}

When 
\begin{equation}
    \vb*{E} = \vu*{x} \Re \tilde{E}_0 \ee^{- \ii (\omega t - k z)},
\end{equation}
from 
\begin{equation}
    \curl{\vb*{E}} = - \pdv*{\vb*{B}}{t}
\end{equation}
we obtain 
\begin{equation}
    \begin{aligned}
        &\ii \vb*{k} \times \vb*{E} = - (- \ii \omega) \vb*{B}  \\
        \Rightarrow& \vb*{B} = \frac{1}{\omega} k \vu*{z} \times \vb*{E}
        = \frac{k}{\omega} \vu*{y} \Re \tilde{E}_0 \ee^{- \ii (\omega t - k z)},
    \end{aligned}
\end{equation}
and therefore 
\begin{equation}
    \expval{\vb*{S}} = \frac{1}{\mu_0} \expval{\vb*{E} \times \vb*{B}}
    = \frac{1}{\mu_0}  \cdot \frac{1}{2} \Re 
    \underbrace{\vu*{x} \tilde{E}_0 \ee^{\ii k z}}_{\Efreq_0} \times 
    \underbrace{\frac{k}{\omega} \vu*{y} \tilde{E}_0^* \ee^{- \ii k z}}_{\tilde{\vb*{B}}_0}
    = \frac{k}{2 \mu_0 \omega} \abs*{\tilde{E}_0}^2 \vu*{z},
\end{equation}
and since the refraction index is $n$, 
we eventually get 
\begin{equation}
    \omega = k \cdot \frac{c}{n}
\end{equation}
and therefore 
\begin{equation}
    \expval{\vb*{S}} = \frac{n}{2} \sqrt{\frac{\epsilon_0}{\mu_0}} \abs*{\tilde{E}_0}^2 \vu*{z}.
\end{equation}
The direction of the energy flow is parallel to the $z$ axis.

\subsubsection{}

The expected value of the electric energy density is 
\begin{equation}
    \expval{u_e} = \frac{1}{2} \epsilon_0 \epsr \expval{\vb*{E}^2}
    = \frac{1}{2} \epsilon_0 n^2 \cdot \frac{1}{2} \abs*{\tilde{E}_0}^2 
    = \frac{1}{4} \epsilon_0 n^2 \abs*{\tilde{E}}^2,
\end{equation}
and the expected value of the magnetic energy density is 
\begin{equation}
    \expval{u_m} = \frac{1}{2 \mu_0} \expval{\vb*{B}^2} 
    = \frac{1}{2 \mu_0} \cdot \frac{1}{2} \frac{k^2}{\omega^2} \abs*{\tilde{E}_0}^2 
    = \frac{1}{4} \frac{n^2}{c^2 \mu_0} \abs{\tilde{E}_0}^2 
    = \frac{1}{4} \epsilon_0 n^2 \abs{\tilde{E}_0}^2.
\end{equation}
So we find 
\begin{equation}
    \frac{\expval{u_e}}{\expval{u_m}} = 1.
\end{equation}

\section{Lorentz oscillator in an AC field and optical forces}

\subsection{Optical response of an ensemble of Lorentz oscillators}

Consider a dilute ensemble of Lorentz oscillators, uniformly distributed over space with number density $N$, in an $\mathrm{AC}$ electric field given by $\mathbf{E}=\operatorname{Re}\left[\tilde{\mathbf{E}}_0 e^{-i \omega t}\right]$. Each oscillator is driven by the local electric field according to the equation of motion given by
$$
\ddot{\mathbf{p}}+\gamma \dot{\mathbf{p}}+\Omega^2 \mathbf{p}=\frac{q^2}{m} \mathbf{E}(\mathbf{r}),
$$
where $\mathbf{r}, m$, and $q$ are the respective oscillator position, reduced mass, and charge.

\subsubsection{}

The polarization density is 
\begin{equation}
    \vb*{P} = N \vb*{p}.
\end{equation}
The EOM for $\vb*{P}$ is 
\begin{equation}
    \ddot{\vb*{P}} + \gamma \dot{\vb*{P}} + \Omega^2 \vb*{P} = \frac{N q^2}{m} \vb*{E}.
\end{equation}
We can switch to the Fourier representation.
Thus we have 
\begin{equation}
    ((-\ii \omega)^2 + \gamma (- \ii \omega) + \Omega^2) \tilde{\vb*{P}} = \frac{N q^2}{m} \Efreq,
\end{equation}
and from 
\begin{equation}
    \vb*{D} = \epsilon_0 \vb*{E} + \vb*{P}
\end{equation}
we get 
\begin{equation}
    \Dfreq = \epsilon_0 \underbrace{\left(
        1 + \frac{N q^2}{m \epsilon_0} \frac{1}{- \omega^2 - \ii \gamma \omega + \Omega^2}
    \right)}_{\epsr} \Efreq.
\end{equation}
So we already get $\epsr$; it has explicit dependence on $\omega$,
but not $\vb*{k}$.

\subsubsection{}

The phase velocity is given by 
\begin{equation}
    v = \frac{c}{\sqrt{\epsr}} = \frac{c}{
        \sqrt{
            1 + \frac{N q^2}{m \epsilon_0} \frac{1}{- \omega^2 - \ii \gamma \omega + \Omega^2}
        }
    }.
\end{equation}
As for the group velocity, we have 
\begin{equation}
    \begin{aligned}
        &\quad \omega^2 = \frac{c^2 k^2}{\epsr} \\
        &\Rightarrow 2 \omega \dd{\omega} = 
        \frac{2 c^2 k \dd{k}}{\epsr} 
        - c^2 k^2 \frac{\dd{\epsr}}{\epsr^2} \\
        &\Rightarrow v_{\text{g}} = 
        \frac{2 c^2 k}{\epsr} \frac{1}{
            2 \omega + \frac{c^2 k^2}{\epsr^2} \dv{\epsr}{\omega}
        },
    \end{aligned}
\end{equation}
where 
\begin{equation}
    \dv{\epsr}{\omega} = \frac{N q^2}{m \epsilon_0}
    \frac{2 \omega + \ii \gamma}{(- \omega^2 - \ii \gamma \omega + \Omega^2)^2}.
\end{equation}

\subsubsection{}

\todo{General relation between energy velocity and group velocity and phase velocity }

\subsection{Optical Tweezers}

\todo{}


\end{document}