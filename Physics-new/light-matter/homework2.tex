\documentclass[hyperref, a4paper]{article}

\usepackage{geometry}
\usepackage{titling}
\usepackage{titlesec}
% No longer needed, since we will use enumitem package
% \usepackage{paralist}
\usepackage{enumitem}
\usepackage{footnote}
\usepackage[colorinlistoftodos]{todonotes}
\usepackage{amsmath, amssymb, amsthm}
\usepackage{mathtools}
\usepackage{bbm}
\usepackage{graphicx}
\usepackage{subcaption}
\usepackage{soulutf8}
\usepackage{physics}
\usepackage{tensor}
\usepackage{siunitx}
\usepackage[version=4]{mhchem}
\usepackage{tikz}
\usepackage{xcolor}
\usepackage{listings}
\usepackage{autobreak}
\usepackage[ruled, vlined, linesnumbered]{algorithm2e}
\usepackage{nameref,zref-xr}
\zxrsetup{toltxlabel}
\usepackage[backend=bibtex]{biblatex}
\addbibresource{electrodynamics.bib}
\usepackage[colorlinks,unicode]{hyperref} % , linkcolor=black, anchorcolor=black, citecolor=black, urlcolor=black, filecolor=black
\usepackage[most]{tcolorbox}
\usepackage{prettyref}

% Page style
\geometry{left=3.18cm,right=3.18cm,top=2.54cm,bottom=2.54cm}
\titlespacing{\paragraph}{0pt}{1pt}{10pt}[20pt]
\setlength{\droptitle}{-5em}

% More compact lists 
\setlist[itemize]{
    itemindent=17pt, 
    leftmargin=1pt,
    listparindent=\parindent,
    parsep=0pt,
}

% Math operators
\DeclareMathOperator{\timeorder}{\mathcal{T}}
\DeclareMathOperator{\diag}{diag}
\DeclareMathOperator{\legpoly}{P}
\DeclareMathOperator{\primevalue}{P}
\DeclareMathOperator{\sgn}{sgn}
\DeclareMathOperator{\res}{Res}
\newcommand*{\ii}{\mathrm{i}}
\newcommand*{\ee}{\mathrm{e}}
\newcommand*{\const}{\mathrm{const}}
\newcommand*{\suchthat}{\quad \text{s.t.} \quad}
\newcommand*{\argmin}{\arg\min}
\newcommand*{\argmax}{\arg\max}
\newcommand*{\normalorder}[1]{: #1 :}
\newcommand*{\pair}[1]{\langle #1 \rangle}
\newcommand*{\fd}[1]{\mathcal{D} #1}
\DeclareMathOperator{\bigO}{\mathcal{O}}

% TikZ setting
\usetikzlibrary{arrows,shapes,positioning}
\usetikzlibrary{arrows.meta}
\usetikzlibrary{decorations.markings}
\tikzstyle arrowstyle=[scale=1]
\tikzstyle directed=[postaction={decorate,decoration={markings,
    mark=at position .5 with {\arrow[arrowstyle]{stealth}}}}]
\tikzstyle ray=[directed, thick]
\tikzstyle dot=[anchor=base,fill,circle,inner sep=1pt]

% Algorithm setting
% Julia-style code
\SetKwIF{If}{ElseIf}{Else}{if}{}{elseif}{else}{end}
\SetKwFor{For}{for}{}{end}
\SetKwFor{While}{while}{}{end}
\SetKwProg{Function}{function}{}{end}
\SetArgSty{textnormal}

\newcommand*{\concept}[1]{{\textbf{#1}}}

% Embedded codes
\lstset{basicstyle=\ttfamily,
  showstringspaces=false,
  commentstyle=\color{gray},
  keywordstyle=\color{blue}
}

% Reference formatting
\newcommand*{\citesec}[1]{\S~{#1}}
\newcommand*{\citechap}[1]{chap.~{#1}}
\newcommand*{\citefig}[1]{Fig.~{#1}}
\newcommand*{\citetable}[1]{Table~{#1}}
\newcommand*{\citepage}[1]{pp.~{#1}}
\newrefformat{fig}{Fig.~\ref{#1}}
\newcommand*{\term}[1]{\textit{#1}}

% Color boxes
\tcbuselibrary{skins, breakable, theorems}

\newtcbtheorem{infobox}{Box}{
    enhanced,
    boxrule=0pt,
    colback=blue!5,
    colframe=blue!5,
    coltitle=blue!50,
    borderline west={4pt}{0pt}{blue!65},
    sharp corners,
    fonttitle=\bfseries, 
    breakable,
    before upper={\parindent15pt\noindent}}{box}
\newtcbtheorem[use counter from=infobox]{theorybox}{Box}{
    enhanced,
    boxrule=0pt,
    colback=orange!5, 
    colframe=orange!5, 
    coltitle=orange!50,
    borderline west={4pt}{0pt}{orange!65},
    sharp corners,
    fonttitle=\bfseries, 
    breakable,
    before upper={\parindent15pt\noindent}}{box}
\newtcbtheorem[use counter from=infobox]{learnbox}{Box}{
    enhanced,
    boxrule=0pt,
    colback=green!5,
    colframe=green!5,
    coltitle=green!50,
    borderline west={4pt}{0pt}{green!65},
    sharp corners,
    fonttitle=\bfseries, 
    breakable,
    before upper={\parindent15pt\noindent}}{box}


\newenvironment{shelldisplay}{\begin{lstlisting}}{\end{lstlisting}}

\newcommand*{\omegap}{\omega_{\text{p}}}    
\newcommand*{\kB}{k_{\text{B}}}
\newcommand*{\muB}{\mu_{\text{B}}}
\newcommand*{\efermi}{E_{\text{F}}}
\newcommand*{\pfermi}{p_{\text{F}}}
\newcommand*{\vfermi}{v_{\text{F}}}
\newcommand*{\sA}{\text{A}}
\newcommand*{\sB}{\text{B}}
\newcommand*{\Tc}{T_{\text{c}}}
\newcommand*{\hethree}{$^3$He}
\newcommand*{\hefour}{$^4$He}
\newcommand{\epsr}{\epsilon_{\text{r}}}
\newcommand{\chie}{\chi_{\text{e}}}
\newcommand{\Efreq}{\tilde{\vb*{E}}}
\newcommand{\Dfreq}{\tilde{\vb*{D}}}
\newcommand{\Pfreq}{\tilde{\vb*{P}}}

\newcommand*{\Gammae}{\Gamma_{\text{e}}}
\newcommand*{\Gammag}{\Gamma_{\text{g}}}
\newcommand*{\omegae}{\omega_{\text{e}}}
\newcommand*{\omegag}{\omega_{\text{g}}}
\newcommand*{\omegaeg}{\omega_{\text{eg}}}
\newcommand*{\ptwfc}[2]{\psi^{(#2)}_{#1}}
\newcommand*{\mueg}{\mu_{\text{eg}}}
\newcommand*{\muge}{\mu_{\text{ge}}}
\newcommand*{\Ezzero}{E_{z0}}

\title{Homework 2}
\author{Jinyuan Wu}

\begin{document}

\maketitle

\section{Polarizability, absorption cross-section, stimulated emission, and optical amplification}

\paragraph{(a)} The EOM of a classical atom described by the Drude-Lorentz model is 
\begin{equation}
    \ddot{x} + \gamma \dot{x} + \omega_0^2 x = \frac{q}{m} E.
\end{equation}
For a plane wave external electric field whose phasor is 
\begin{equation}
    \tilde{E} = \tilde{E}_0 \ee^{- \ii (\omega t - k z)},
\end{equation}
the response of $x$, in the form of phasor, is 
\begin{equation}
    \tilde{x} = \frac{q}{m} \frac{\tilde{E}}{- \omega^2 + \omega_0^2 - \ii \gamma \omega},
\end{equation}
and therefore 
\begin{equation}
    \tilde{\dot{x}} = - \ii \omega \frac{q}{m} \frac{\tilde{E}}{- \omega^2 + \omega_0^2 - \ii \gamma \omega},
\end{equation}
and the time average power of dissipation is 
\begin{equation}
    \expval{P_\text{abs}} = \expval{F_\text{dissipation} \cdot v} 
    = - \gamma \expval{v \cdot v} = - \gamma \cdot \frac{1}{2} \Re \tilde{\dot{x}}^* \cdot \tilde{\dot{x}}
    = - \frac{\gamma q^2}{2 m^2} \frac{\omega^2 \abs*{\tilde{E}_0}^2}{
        (\omega^2 - \omega_0^2)^2 + \gamma^2 \omega^2
    }.
\end{equation}

\paragraph{(b)} The absorption cross section can be calculated by 
\begin{equation}
    \sigma_\text{abs} = \frac{\abs*{\expval{P_\text{abs}}}}{\text{input intensity}}
    = \frac{\abs*{\expval{P_\text{abs}}}}{
        \frac{1}{\mu_0} \frac{k}{\omega} \underbrace{
            E^2
        }_{\frac{1}{2} \abs*{\tilde{E}_0}^2}   
    }
    = \frac{1}{c \epsilon_0} \frac{\omega^2}{
        (\omega^2 - \omega_0^2)^2 + \gamma^2 \omega^2
    } \gamma.
\end{equation}

The absorption coefficient can then be found 
by multiplying the number density to $\sigma_{\text{abs}}$.

\paragraph{(c)} Now consider a quantum two-level system described by 
\begin{equation}
    H = H_0 + H_{\text{dipole}}, \quad 
    H_0 = \hbar \omegag \dyad{\text{g}}
    + \hbar \omegae \dyad{\text{e}} , \quad 
\end{equation}
where
\begin{equation}
    H_{\text{dipole}} = - \vb*{\mu} \cdot \vb*{E} 
    = - \mu_{\text{eg}} \dyad{\text{e}}{\text{g}} E_{z0} \cos \omega t - \text{h.c.},
\end{equation}
and the $z$ direction is set to the direction of $\vb*{\mu}_{\text{eg}}$.
We start from the ground state in the $t \to - \infty$ limit, 
and the first order perturbation is 
\begin{equation}
    \ket*{\psi(t)} = \ket*{\text{g}} \ee^{- \ii \omegag t} 
    + \gamma^{(1)}_{\text{g}} \ket*{\text{g}} \ee^{- \ii \omegag t} 
    + \gamma^{(1)}_{\text{e}} \ket*{\text{e}} \ee^{- \ii \omegae t} ,
    \label{eq:first-pt-wave-function}
\end{equation}
where 
\begin{equation}
    \dv{\gamma^{(1)}_{\text{g}}}{t} = \frac{1}{\ii \hbar} \gamma^{(0)}_{\text{g}} \cdot \expval*{H_{\text{dipole}}}{\text{g}} = 0
\end{equation}
because dipole transition is only allowed between states with different parities, and 
\begin{equation}
    \begin{aligned}
        &\dv{\gamma_\text{e}^{(1)}}{t} = \frac{1}{\ii \hbar} 
        \gamma_{\text{g}}^{(0)} \ee^{\ii (\omegae - \omegag) t} 
        \cdot (- \mu_{\text{eg}} E_{z0} \cos \omega t ) \\
        \Rightarrow & 
        \gamma^{(1)}_{\text{e}}(t) - \underbrace{
            \Gammae^{(1)}(t=0)
        }_{0} = \ii \frac{\mu_{\text{eg}} E_{z0}}{\hbar} 
        \int_{- \infty}^{t} \dd{t'} \ee^{\ii (\omegae - \omegag) t}  \cos \omega t  .
    \end{aligned}
\end{equation}
Adding a small imaginary part to $\omegae$ because of spontaneous emission:
\begin{equation}
    \omegae \to \omegae - \ii \Gammae,
\end{equation}
and by completing this integral 
and not throwing away the high-frequency term, 
we get 
\begin{equation}
    \gamma^{(1)}_{\text{e}} = \ii \frac{\mu_{\text{eg}} E_{z0}}{\hbar} 
    \cdot \frac{1}{2} \left(
        \frac{\ee^{\ii (\omegaeg - \ii \Gammae + \omega) t}}{
            \ii (\omegaeg + \omega - \ii \Gammae)
        } + 
        \frac{\ee^{\ii (\omegaeg - \ii \Gammae - \omega) t}}{
            \ii (\omegaeg - \omega - \ii \Gammae)
        }
    \right), \quad 
    \omegaeg = \omegae - \omegag.
\end{equation}
Here we have thrown away the terms at $t = - \infty$ 
because when $t \to -\infty$, $\ee^{\Gammae t}$ vanishes.
$\Gammae^{(1)}$ now explodes when $t \to \infty$
but this doesn't matter because we still need to multiply 
$\ee^{- \ii (\omegae - \ii \Gammae) t}$ to it
because $\omegae$ in \eqref{eq:first-pt-wave-function}
also needs to be modified by the spontaneous emission rate.
The expectation value of the dipole is 
\begin{equation}
    \begin{aligned}
        \expval*{\mu}(t) &= \expval*{\mu}{\psi(t)}
        = \Gammae^{(1)}
        \ee^{- \ii (\omegaeg - \ii \Gammae) t} \mu_{\text{ge}} + \text{h.c.} \\ 
        &= \frac{\abs*{\mu_{\text{eg}}}^2 E_{0z}}{\hbar} \cdot \frac{1}{2} \left(
            \frac{\ee^{\ii \omega t}}{\omegaeg + \omega - \ii \Gammae}
            + \frac{\ee^{- \ii \omega t}}{\omegaeg - \omega - \ii \Gammae}
            + \frac{\ee^{- \ii \omega t}}{\omegaeg + \omega + \ii \Gammae}
            + \frac{\ee^{\ii \omega t}}{\omegaeg - \omega + \ii \Gammae}
        \right).
    \end{aligned}
\end{equation}
Here is a subtlety: in the above calculation, $\bra*{\psi(t)}$ is calculated using $H_0^\dagger$,
because strictly speaking, here we are working with a non-Hermitian system 
because of the $- \ii \Gammae$ correction.
Since 
\begin{equation}
    E = \frac{1}{2} E_{z0} (\ee^{\ii \omega t} + \ee^{- \ii \omega t}),
\end{equation}
the frequency-domain response function from $E$ to $\mu$ is 
\begin{equation}
    \begin{aligned}
        \alpha(\omega) &= \frac{\abs*{\mu_{\text{eg}}}^2}{\hbar} \left(
            \frac{1}{\omegaeg - \omega - \ii \Gammae}
            + \frac{1}{\omegaeg + \omega + \ii \Gammae}
        \right) \\
        &\approx \frac{\abs*{\mu_{\text{eg}}}^2}{\hbar} \cdot 
        \frac{2\omegaeg}{
            \omegaeg^2 - \omega^2 - 2 \ii \Gammae \omega
        }.
    \end{aligned} 
    \label{eq:alpha-two-level}
\end{equation}
In the last line we have used the $\Gammae \ll \omegae - \omegag$ condition.

\paragraph{(d)} The prefactor in \eqref{eq:alpha-two-level} differs from that 
in the susceptibility of a harmonic oscillator, 
and hence the necessity of the $f_i$ factor.
But the structure of the denominator is the same:
we have 
\begin{equation}
    \omega_0 = \omegaeg , \quad 
    \gamma = 2 \Gammae.
\end{equation}

\paragraph{(e)} We need to calculate energy loss due to spontaneous emission.
Since 
\begin{equation}
    \abs*{\gamma_{\text{e}}^{(1)}}^2 \propto \ee^{- 2 \ii \Gammae t},
\end{equation}
we find that in time period $\dd{t}$ the probability of spontaneous emission is $2 \Gammae \dd{t}$,
and therefore power of energy loss is 
\begin{equation}
    \begin{aligned}
        P_{\text{abs}} &=  2 \Gammae \cdot \hbar (\omegae - \omegag) 
        \braket{\ptwfc{\text{e}}{1}}{\ptwfc{\text{e}}{1}}  \\
        &= 2 \Gammae \cdot \hbar \omegaeg \cdot 
        \frac{\abs*{\mu_{\text{eg}}}^2 E_{z0}^2}{\hbar^2} 
        \cdot \frac{1}{2} \left(
            \frac{\ee^{\ii (\omegaeg - \ii \Gammae + \omega) t}}{
                \omegaeg + \omega - \ii \Gammae
            } + 
            \frac{\ee^{\ii (\omegaeg - \ii \Gammae - \omega) t}}{
                \omegaeg - \omega - \ii \Gammae
            }
        \right) \cdot 
        \frac{1}{2} \left(
            \frac{\ee^{- \ii (\omegaeg - \ii \Gammae + \omega) t}}{
                \omegaeg + \omega + \ii \Gammae
            } + 
            \frac{- \ee^{\ii (\omegaeg - \ii \Gammae - \omega) t}}{
                \omegaeg - \omega + \ii \Gammae
            }
        \right).
    \end{aligned}
\end{equation}
We need to ignore the high oscillation terms, 
just as what we did for the harmonic oscillator,
so 
\begin{equation}
    \expval{P_{\text{abs}}} = \frac{1}{2} \Gammae \omegaeg \frac{\abs*{\mueg}^2 E_{0z}^2}{\hbar}
    \left(
        \frac{1}{(\omegaeg + \omega)^2 + \Gammae^2} + 
        \frac{1}{(\omegaeg - \omega)^2 + \Gammae^2}
    \right).
\end{equation}
The absorption cross section then is 
\begin{equation}
    \sigma_{\text{abs}} = \frac{\expval{P_{\text{abs}}}}{
        \underbrace{\expval{S}}_{
            \frac{1}{2} \epsilon_0 c E_{0z}^2
        }
    }
    =  \frac{\abs*{\mueg}^2 \Gammae \omegaeg}{\hbar \epsilon_0 c}
    \left(
        \frac{1}{(\omegaeg + \omega)^2 + \Gammae^2} + 
        \frac{1}{(\omegaeg - \omega)^2 + \Gammae^2}
    \right).
\end{equation}

\paragraph{(f)} Now we consider the same system but the initial state is the excited state.
Correspondingly, the imaginary part of the energy now comes to the ground state, 
which denotes the probability for the atom to be brought to the excited state due to external pumping.
Repeating the above procedure but this time 
replacing $\omegag$ with $\omegag - \ii \Gammag$,
we have 
\begin{equation}
    \begin{aligned}
        \gamma_{\text{g}}^{(1)}(t) &= 
        \ii \frac{\muge \Ezzero}{\hbar} \cdot \frac{1}{2} \left(
            \frac{
                \ee^{- \ii (\omegaeg + \ii \Gammag + \omega) t}
            }{
                - \omegaeg - \omega - \ii \Gammag
            } + 
            \frac{
                \ee^{\ii (- \omegaeg - \ii \Gammag + \omega) t}
            }{
                - \omegaeg + \omega - \ii \Gammag
            }   
        \right),
    \end{aligned}
\end{equation}
and 
\begin{equation}
    \begin{aligned}
        \alpha(\omega) &= \frac{\abs*{\mueg}^2}{\hbar^2} \left(
            \frac{1}{- \omegaeg - \omega - \ii \Gammag}
            + \frac{1}{- \omegaeg + \omega + \ii \Gammag}
        \right) \\
        &\approx \frac{\abs*{\mueg}^2}{\hbar^2} \frac{- 2 \omegaeg}{
            \omegaeg^2 - \omega^2 - 2 \ii \Gammag \omega
        } .
    \end{aligned}
\end{equation}
This can be obtained by swapping $\ket*{\text{e}}$ and $\ket*{\text{g}}$ in \eqref{eq:alpha-two-level}.

\paragraph{(g)} Again by swapping the two energy levels, we have 
\begin{equation}
    \expval{P_{\text{abs}}} = 
    - \frac{1}{2} \Gammag \omegaeg \frac{\abs*{\mueg}^2 E_{0z}^2}{\hbar}
    \left(
        \frac{1}{(\omegaeg + \omega)^2 + \Gammag^2} + 
        \frac{1}{(\omegaeg - \omega)^2 + \Gammag^2}
    \right).
\end{equation}
The absorption power is negative, 
because the system is gaining energy, not losing it.
The stimulated emission cross section then is 
\begin{equation}
    \sigma_{\text{st}} = - \frac{\abs*{\mueg}^2 \Gammag \omegaeg}{\hbar \epsilon_0 c}
    \left(
        \frac{1}{(\omegaeg + \omega)^2 + \Gammae^2} + 
        \frac{1}{(\omegaeg - \omega)^2 + \Gammae^2}
    \right).
\end{equation}

\paragraph{(h)} Suppose we have a dilute gas containing $N$ atoms per unit volume,
and the excited and ground state populations are 
$N_{\text{e}} = N p_{\text{e}}$ and 
$N_{\text{g}} = N p_{\text{g}}$.
Since the gas is dilute, we don't need to analyze the interferences of 
the $\Gammag$ and $\Gammae$ processes, 
and we have 
\begin{equation}
    \begin{aligned}
        \chi(\omega) &= N_{\text{g}} \alpha_{\text{abs}}(\omega)
        + N_{\text{e}} \alpha_{\text{st}}(\omega) \\
        &= N \frac{\abs*{\mu_{\text{eg}}}^2}{\hbar} \cdot 
        \left(
            p_{\text{g}} \frac{2\omegaeg}{
                \omegaeg^2 - \omega^2 - 2 \ii \Gammae \omega
            } 
            - p_{\text{e}} \frac{ 2 \omegaeg}{
                \omegaeg^2 - \omega^2 - 2 \ii \Gammag \omega
            }
        \right) .
    \end{aligned}
\end{equation}
We want $\Im \chi < 0$ 
to have a negative imaginary part in $\vb*{k}$,
and therefore $\ee^{\ii \vb*{k} \cdot \vb*{r}}$ increases as the light goes forward.
Assuming $\Gammag = \Gammae$, we find the condition needed is 
\begin{equation}
    p_{\text{e}} > p_{\text{g}}.
\end{equation}

\section{Spontaneous decay rate computed from Fermi's golden rule and Poynting's Theorem}

\paragraph{(a)} For a two-level atom, 
the interaction Hamiltonian between it and the electromagnetic field in vacuum is 
\begin{equation}
    H_{\text{dipole}} = - \vb*{\mu} \cdot \vb*{E}.
\end{equation}
When the direction of $\vb*{\mu}$ is known to be $\vu*{z}$,
we have 

\end{document}