\documentclass[hyperref, a4paper]{article}

\usepackage{geometry}
\usepackage{titling}
\usepackage{titlesec}
% No longer needed, since we will use enumitem package
% \usepackage{paralist}
\usepackage{enumitem}
\usepackage{footnote}
\usepackage[colorinlistoftodos]{todonotes}
\usepackage{amsmath, amssymb, amsthm}
\usepackage{mathtools}
\usepackage{bbm}
\usepackage{graphicx}
\usepackage{subcaption}
\usepackage{soulutf8}
\usepackage{physics}
\usepackage{tensor}
\usepackage{siunitx}
\usepackage[version=4]{mhchem}
\usepackage{tikz}
\usepackage{xcolor}
\usepackage{listings}
\usepackage{autobreak}
\usepackage[ruled, vlined, linesnumbered]{algorithm2e}
\usepackage{nameref,zref-xr}
\zxrsetup{toltxlabel}
\usepackage[backend=bibtex]{biblatex}
\addbibresource{electrodynamics.bib}
\usepackage[colorlinks,unicode]{hyperref} % , linkcolor=black, anchorcolor=black, citecolor=black, urlcolor=black, filecolor=black
\usepackage[most]{tcolorbox}
\usepackage{prettyref}

% Page style
\geometry{left=3.18cm,right=3.18cm,top=2.54cm,bottom=2.54cm}
\titlespacing{\paragraph}{0pt}{1pt}{10pt}[20pt]
\setlength{\droptitle}{-5em}

% More compact lists 
\setlist[itemize]{
    itemindent=17pt, 
    leftmargin=1pt,
    listparindent=\parindent,
    parsep=0pt,
}

% Math operators
\DeclareMathOperator{\timeorder}{\mathcal{T}}
\DeclareMathOperator{\diag}{diag}
\DeclareMathOperator{\legpoly}{P}
\DeclareMathOperator{\primevalue}{P}
\DeclareMathOperator{\sgn}{sgn}
\DeclareMathOperator{\res}{Res}
\newcommand*{\ii}{\mathrm{i}}
\newcommand*{\ee}{\mathrm{e}}
\newcommand*{\const}{\mathrm{const}}
\newcommand*{\suchthat}{\quad \text{s.t.} \quad}
\newcommand*{\argmin}{\arg\min}
\newcommand*{\argmax}{\arg\max}
\newcommand*{\normalorder}[1]{: #1 :}
\newcommand*{\pair}[1]{\langle #1 \rangle}
\newcommand*{\fd}[1]{\mathcal{D} #1}
\DeclareMathOperator{\bigO}{\mathcal{O}}

% TikZ setting
\usetikzlibrary{arrows,shapes,positioning}
\usetikzlibrary{arrows.meta}
\usetikzlibrary{decorations.markings}
\tikzstyle arrowstyle=[scale=1]
\tikzstyle directed=[postaction={decorate,decoration={markings,
    mark=at position .5 with {\arrow[arrowstyle]{stealth}}}}]
\tikzstyle ray=[directed, thick]
\tikzstyle dot=[anchor=base,fill,circle,inner sep=1pt]

% Algorithm setting
% Julia-style code
\SetKwIF{If}{ElseIf}{Else}{if}{}{elseif}{else}{end}
\SetKwFor{For}{for}{}{end}
\SetKwFor{While}{while}{}{end}
\SetKwProg{Function}{function}{}{end}
\SetArgSty{textnormal}

\newcommand*{\concept}[1]{{\textbf{#1}}}

% Embedded codes
\lstset{basicstyle=\ttfamily,
  showstringspaces=false,
  commentstyle=\color{gray},
  keywordstyle=\color{blue}
}

% Reference formatting
\newcommand*{\citesec}[1]{\S~{#1}}
\newcommand*{\citechap}[1]{chap.~{#1}}
\newcommand*{\citefig}[1]{Fig.~{#1}}
\newcommand*{\citetable}[1]{Table~{#1}}
\newcommand*{\citepage}[1]{pp.~{#1}}
\newrefformat{fig}{Fig.~\ref{#1}}
\newcommand*{\term}[1]{\textit{#1}}

% Color boxes
\tcbuselibrary{skins, breakable, theorems}

\newtcbtheorem{infobox}{Box}{
    enhanced,
    boxrule=0pt,
    colback=blue!5,
    colframe=blue!5,
    coltitle=blue!50,
    borderline west={4pt}{0pt}{blue!65},
    sharp corners,
    fonttitle=\bfseries, 
    breakable,
    before upper={\parindent15pt\noindent}}{box}
\newtcbtheorem[use counter from=infobox]{theorybox}{Box}{
    enhanced,
    boxrule=0pt,
    colback=orange!5, 
    colframe=orange!5, 
    coltitle=orange!50,
    borderline west={4pt}{0pt}{orange!65},
    sharp corners,
    fonttitle=\bfseries, 
    breakable,
    before upper={\parindent15pt\noindent}}{box}
\newtcbtheorem[use counter from=infobox]{learnbox}{Box}{
    enhanced,
    boxrule=0pt,
    colback=green!5,
    colframe=green!5,
    coltitle=green!50,
    borderline west={4pt}{0pt}{green!65},
    sharp corners,
    fonttitle=\bfseries, 
    breakable,
    before upper={\parindent15pt\noindent}}{box}


\newenvironment{shelldisplay}{\begin{lstlisting}}{\end{lstlisting}}

\newcommand*{\omegap}{\omega_{\text{p}}}    
\newcommand*{\kB}{k_{\text{B}}}
\newcommand*{\muB}{\mu_{\text{B}}}
\newcommand*{\efermi}{E_{\text{F}}}
\newcommand*{\pfermi}{p_{\text{F}}}
\newcommand*{\vfermi}{v_{\text{F}}}
\newcommand*{\sA}{\text{A}}
\newcommand*{\sB}{\text{B}}
\newcommand*{\Tc}{T_{\text{c}}}
\newcommand*{\hethree}{$^3$He}
\newcommand*{\hefour}{$^4$He}
\newcommand{\epsr}{\epsilon_{\text{r}}}
\newcommand*{\mur}{\mu_{\text{r}}}
\newcommand{\chie}{\chi_{\text{e}}}
\newcommand{\Efreq}{\tilde{E}}
\newcommand{\Dfreq}{\tilde{D}}
\newcommand{\Pfreq}{\tilde{\vb*{P}}}

\newcommand*{\Gammae}{\Gamma_{\text{e}}}
\newcommand*{\Gammag}{\Gamma_{\text{g}}}
\newcommand*{\omegae}{\omega_{\text{e}}}
\newcommand*{\omegag}{\omega_{\text{g}}}
\newcommand*{\omegaeg}{\omega_{\text{eg}}}
\newcommand*{\ptwfc}[2]{\psi^{(#2)}_{#1}}
\newcommand*{\mueg}{\mu_{\text{eg}}}
\newcommand*{\muge}{\mu_{\text{ge}}}
\newcommand*{\Ezzero}{E_{z0}}
\newcommand*{\kete}{\ket*{\text{e}}}
\newcommand*{\ketg}{\ket*{\text{g}}}
\newcommand*{\coeffe}{c_{\text{e}}}
\newcommand*{\coeffg}{c_{\text{g}}}
\newcommand*{\pope}{p_{\text{e}}}
\newcommand*{\popg}{p_{\text{g}}}

\title{Homework 2}
\author{Jinyuan Wu}

\begin{document}

\maketitle

\section{Group velocity in dissipative media}

\subsection{Analysing a wave packet}

\paragraph{(a)} The so-called situation of \concept{normal dispersion} 
is when $\dv*{n}{\omega} > 0$, 
i.e. when a beam of light that is more blue gets refracted more severely 
than a beam of reddish light does.
On the other hand, \concept{anomalous dispersion} happens when $\dv*{n}{\omega} < 0$.

Since $v_{\text{p}} = c / n$,
in normal dispersion, the red part of the spectrum moves faster, 
while in anomalous dispersion, the blue part moves faster.

\paragraph{(b)} Consider a wave packet which can be written in the following form:
\begin{equation}
    E(t) = \Re \bar{E}(t) \ee^{- \ii \omega t},
\end{equation}
where $\bar{E}(t)$ varies much more slowly than $\ee^{- \ii \omega t}$.
The relation between the Fourier components and $\bar{E}(t)$ is given below:
\begin{equation}
    E(\omega') = \int \dd{t} \ee^{\ii \omega' t} E(t)
    = \frac{1}{2} \int \dd{t} \ee^{\ii (\omega' - \omega) t} \bar{E}(t)
    + \frac{1}{2} \int \dd{t} \ee^{\ii (\omega' + \omega) t} \bar{E}^*(t).
\end{equation}
The response of $D(t)$ to the electric field is therefore 
\begin{equation}
    \begin{aligned}
        D(t) &= \int \frac{\dd{\omega'}}{2\pi} \ee^{- \ii \omega' t} \epsilon(\omega') E(\omega') \\
        &= \frac{1}{2} \int \frac{\dd{\omega'}}{2\pi} \ee^{- \ii \omega' t} \epsilon(\omega') 
        \int \dd{t'} \ee^{\ii (\omega' - \omega) t'} \bar{E}(t') + 
        \frac{1}{2} \int \frac{\dd{\omega'}}{2\pi} \ee^{- \ii \omega' t} \epsilon(\omega') 
        \int \dd{t'} \ee^{\ii (\omega' + \omega) t'} \bar{E}^*(t').
    \end{aligned}
    \label{eq:e-to-d-1}
\end{equation}

We further assume that the dielectric constant is real and positive, 
with a frequency dependence that varies slowly 
compared with spectral band-width of the electric field,
we can do the following Taylor expansion 
in the two terms of \eqref{eq:e-to-d-1}:
\begin{equation}
    \begin{aligned}
        &\quad \int \frac{\dd{\omega'}}{2\pi} \ee^{- \ii \omega' t} \epsilon(\omega') 
        \int \dd{t'} \ee^{\ii (\omega' - \omega) t'} \bar{E}(t') \\
        &= \int \frac{\dd{\omega'}}{2\pi} \ee^{- \ii \omega' t} 
        \int \dd{t'} \ee^{\ii (\omega' - \omega) t'} \bar{E}(t')  \left(
            \epsilon(\omega) + 
            (\omega' - \omega) \eval{\dv{\epsilon}{\omega'}}_{\omega = \omega}
            + \cdots 
        \right) \\
        &= \epsilon(\omega) \ee^{- \ii \omega t} \bar{E}(t) 
        + \frac{1}{\ii} \int \frac{\dd{\omega'}}{2\pi} \ee^{- \ii \omega' t} \int \dd{t'} 
        \eval{\dv{\epsilon}{\omega'}}_{\omega = \omega} \bar{E}(t') 
        \dv{t'} \ee^{\ii (\omega' - \omega) t'} \\
        &= \epsilon(\omega) \ee^{- \ii \omega t} \bar{E}(t) 
        + \ii \int \frac{\dd{\omega'}}{2\pi} \ee^{- \ii \omega' t} \int \dd{t'} 
        \eval{\dv{\epsilon}{\omega'}}_{\omega = \omega} 
        \ee^{\ii (\omega' - \omega) t'}
        \dv{t'} \bar{E}(t')  \\
        &= \epsilon(\omega) \ee^{- \ii \omega t} \bar{E}(t) 
        + \ii \pdv{\epsilon}{\omega} \ee^{- \ii \omega t} \pdv{\bar{E}}{t},
    \end{aligned}
    \label{eq:pdv-d-t-derivation}
\end{equation}
and by repeating the same procedure for the second term in \eqref{eq:e-to-d-1},
we eventually find 
\begin{equation}
    D(t) = \Re \left(
        \ee^{- \ii \omega t} \epsilon(\omega) \bar{E}(t) 
        + \ii \ee^{- \ii \omega t} \pdv{\epsilon}{\omega}  \pdv{\bar{E}}{t}
    \right).
    \label{eq:d-final}
\end{equation}

\paragraph{(c)} From \eqref{eq:d-final} we have 
\begin{equation}
    \pdv{D}{t} = \Re \left(
        - \ii \omega \epsilon(\omega) \ee^{- \ii \omega t} \bar{E}(t)
        + \ee^{- \ii \omega t} \epsilon(\omega) \pdv{\bar{E}}{t} 
        + \omega \pdv{\epsilon}{\omega} \pdv{\bar{E}}{t} \ee^{- \ii \omega t} 
    \right) ,
\end{equation}
where we have used the condition that $\bar{E}(t)$ changes slowly enough
and therefore ignored the $\pdv*[2]{\bar{E}}{t}$ term.
By doing time averaging in a region that's large enough compared with $1 / \omega$ 
but is still small compared with the characteristic time scale of $\bar{E}(t)$,
we find 
\begin{equation}
    \begin{aligned}
        \pdv{u_E}{t} = E \cdot \pdv{D}{t} \Rightarrow
        \expval{\pdv{u_E}{t}} &= \frac{1}{2} \Re \tilde{E}^* \cdot \widetilde{\pdv{D}{t}} \\
        &= \frac{1}{2} \Re \bar{E}^*(t) \cdot 
        \left(
            \epsilon(\omega) \pdv{\bar{E}}{t} 
            + \omega \pdv{\epsilon}{\omega} \pdv{\bar{E}}{t} 
        \right) \\
        &= \frac{1}{2} \pdv{(\omega \epsilon)}{\omega}
        \cdot \frac{1}{2} \left(
            \bar{E}^*(t) \cdot \pdv{E}{t} 
            + \bar{E} \cdot \pdv{E^*}{t}
        \right)  \\
        &= \frac{1}{4} \pdv{(\omega \epsilon)}{\omega} \pdv{t} \abs{\bar{E}}^2.
    \end{aligned}
\end{equation}
So we have found 
\begin{equation}
    \expval{\pdv{u_E}{t}} = \frac{1}{4} \pdv{(\omega \epsilon)}{\omega} \pdv{t} \abs{\bar{E}}^2
    = \frac{1}{4} \left(
        \epsilon(\omega) + \omega \pdv{\epsilon}{\omega}
    \right) \pdv{t} \abs{\bar{E}}^2.
\end{equation}

\paragraph{(d)} Considering the mathematical equivalences between 
$\vb*{D}$ and $\vb*{B}$ and between $\vb*{E}$ and $\vb*{H}$ in the equations, 
we immediately find 
\begin{equation}
    \expval{\pdv{u_M}{t}} = \expval{H \cdot \pdv{B}{t}} 
    = \frac{1}{4} \left(
        \mu(\omega) + \omega \pdv{\mu}{\omega}
    \right) \pdv{t} \abs{\bar{H}}^2.
\end{equation}
Similar to \eqref{eq:d-final}, we also have 
\begin{equation}
    B(t) = \Re \left(
        \ee^{- \ii \omega t} \mu(\omega) \bar{H}(t) 
        + \ii \ee^{- \ii \omega t} \pdv{\mu}{\omega} \pdv{\bar{H}}{t}
    \right).
\end{equation}
Now we find that due to the imaginary unit in the second term, 
the contribution of $\pdv*{\mu}{\omega}$ to $\abs*{\bar{B}}^2$
is $\sim (\pdv*{\mu}{\omega})^2$,
and thus from the condition that the permeability changes very slowly
in the band width of the wave packet, approximately we have 
\begin{equation}
    \abs*{\bar{H}(t)}^2 = \frac{1}{\mu(\omega)} \abs*{\bar{B}(t)}^2.
\end{equation} 
Similarly, from 
\begin{equation}
    \curl{\vb*{E}} = - \pdv{\vb*{B}}{t} 
    = - \Re \left(
        \pdv{\bar{\vb*{B}}}{t} \ee^{- \ii \omega t} 
        - \ii \omega \bar{\vb*{B}}(t) \ee^{- \ii \omega t}
    \right)
\end{equation}
and the fact that $\bar{B}(t)$ changes very slowly, approximately we have 
\begin{equation}
    \abs*{\bar{E}(t)}^2 = \frac{\omega^2}{k^2} \abs*{\bar{B}}^2
    = \frac{1}{\epsilon \mu}  \abs*{\bar{B}}^2.
\end{equation}
So we find 
\begin{equation}
    \begin{aligned}
        \expval{\pdv{u_M}{t}} 
        &= \frac{1}{4} \left(
            \mu(\omega) + \omega \pdv{\mu}{\omega}
        \right) \pdv{t} \abs{\bar{H}}^2 \\
        &= \frac{1}{4} \frac{\epsilon}{\mu} \left(
            \mu(\omega) + \omega \pdv{\mu}{\omega}
        \right) \pdv{t} \abs*{\bar{E}}^2 \\
        &= \frac{1}{4} \left(
            \epsilon(\omega) + \frac{\omega \epsilon}{\mu} \pdv{\mu}{\omega}
        \right) \pdv{t} \abs*{\bar{E}}^2 \\
        &= \frac{1}{4} \epsilon_0 \left(
            \epsr(\omega) + \frac{\omega \epsr}{\mur} \pdv{\mur}{\omega}
        \right) \pdv{t} \abs*{\bar{E}}^2 
    \end{aligned}
\end{equation}

\paragraph{(e)} Assuming no magnetic susceptibility,
we have 
\begin{equation}
    \expval{\pdv{u}{t}} = \expval{\pdv{u_E}{t}} + \expval{\pdv{u_M}{t}}
    = \frac{1}{4} \left(
        2 \epsilon(\omega) + \omega \pdv{\epsilon}{\omega}
    \right) \pdv{t} \abs{\bar{E}}^2.
\end{equation}
Thus we have 
\begin{equation}
    \expval{u} = \frac{1}{4} \left(
        2 \epsilon(\omega) + \omega \pdv{\epsilon}{\omega}
    \right) \abs{\bar{E}}^2.
\end{equation}
There are high oscillation terms in $u$ 
(which can be seen by considering the simplest case of $\bar{E} = \const$),
and they have large time derivations, 
but their all average to zero,
and thus we still have 
\begin{equation}
    \pdv{t}\expval{u} \approx \expval{\pdv{u}{t}}.
\end{equation}
Deep inside the wave packet (i.e. far from the boundaries), 
we shouldn't see huge change of $\bar{E}$,
and thus the expectation value of the Poynting vector 
can be evaluated as if the wave packet is a plane wave:
\begin{equation}
    \abs*{\expval{\vb*{S}}}  = \frac{1}{2} \sqrt{\frac{\epsilon}{\mu_0}} \abs*{\bar{E}(t)}^2.
\end{equation}
So the group velocity is 
\begin{equation}
    v_{\text{g}} = \frac{\abs*{\expval{\vb*{S}}}}{\expval{u}}
    = \frac{
        \frac{1}{2} \sqrt{\frac{\epsilon}{\mu_0}}
    }{
        \frac{1}{4} \left(
        2 \epsilon(\omega) + \omega \pdv{\epsilon}{\omega}
        \right)
    }
    = \frac{1}{\sqrt{\mu_0}} \frac{1}{
        \sqrt{\epsilon} + \frac{1}{2} \omega \sqrt{\epsilon} \pdv{\epsilon}{\omega}
    }.
\end{equation}
On the other hand, we have 
\begin{equation}
    k = \frac{\omega}{c / n} = \omega \sqrt{\epsilon \mu_0} \Rightarrow
    \pdv{k}{\omega} = \sqrt{\epsilon \mu_0} + \omega \sqrt{\mu_0} \frac{1}{2 \sqrt{\epsilon}} \pdv{\epsilon}{\omega}, 
\end{equation}
and therefore 
\begin{equation}
    v_{\text{g}} = \frac{1}{\pdv{k}{\omega}} = \pdv{\omega}{k}.
\end{equation}

\subsection{Energy storage and energy loss}

\paragraph{(a)} When the ``wave packet'' is 

\begin{equation}
    H = \sum_i \Bigl(
        \underbrace{
            \frac{p_i^2}{2m}
        }_{T_i} 
        + \underbrace{
            \frac{1}{2} m \omega_0^2 x_i^2
        }_{V_i} 
        - q x_i E
    \Bigr) + H_{\text{EM}}.
\end{equation}
The response of any $x_i$ is 
\begin{equation}
    \tilde{x}_i = \frac{q \tilde{E}}{m(- \omega^2 + \omega_0^2)},
\end{equation}
and thus 
\begin{equation}
    T_i = \expval{\frac{p_i^2}{2m}}
    = \frac{1}{2m} \cdot \frac{1}{2} \cdot  \abs{- \ii \omega m \tilde{x}}^2 
    = \frac{1}{2} \frac{\omega^2 q^2 \abs*{\tilde{E}}^2}{
        2 m (\omega^2 - \omega_0^2)^2
    },
\end{equation}
and 
\begin{equation}
    V_i = \expval{\frac{1}{2} m \omega_0^2 x^2}
    = \frac{1}{2} m \omega_0^2 \cdot \frac{1}{2} \cdot \abs*{\tilde{x}}^2 
    = \frac{1}{2} \frac{\omega_0^2 q^2 \abs*{\tilde{E}}^2}{
        2 m (\omega^2 - \omega_0^2)^2
    },
\end{equation}
and 
\begin{equation}
    \expval{- q x_i E} = - \frac{q^2}{m (- \omega^2 + \omega_0^2)} \cdot \frac{1}{2} \abs*{\tilde{E}}^2.
\end{equation}

On the other hand, we have 
\begin{equation}
    \begin{aligned}
        \dv{(\omega \epsr)}{\omega} &= 
        1 + \frac{N q^2}{\epsilon_0 m} \frac{1}{- \omega^2 + \omega_0^2}
        + \frac{N q^2}{\epsilon_0 m} \cdot \omega \cdot \frac{2 \omega}{(\omega^2 - \omega_0^2)^2} \\ 
        &= 1 + \frac{Nq^2}{\epsilon_0 m} \frac{
            \omega^2 + \omega_0^2
        }{(\omega^2 - \omega_0^2)^2}. 
    \end{aligned}
\end{equation}
Thus 
\begin{equation}
    \underbrace{
        \frac{1}{4} \epsilon_0 \dv{(\omega \epsr)}{\omega} \abs*{\tilde{E}}^2
    }_{\expval{u_E}} 
    = \sum_i (\expval{T_i} + \expval{V_i}) + \underbrace{
        \frac{1}{4} \epsilon_0 \abs*{\tilde{E}}^2
    }_{\expval{u_E^{\text{vacuum}}}} ,
\end{equation}
and the expectation value of $u_{E}$ is the sum 
of the energy of the oscillators and the vacuum electric energy.
\todo{
    Why is $- \vb*{P} \cdot \vb*{E}$ not included into $\int \dd{t} \vb*{E} \cdot \pdv*{\vb*{D}}{t}$?
}

\paragraph{(b)} Suppose $N$ is the number of oscillators per unit volume in the medium.
For a damped oscillator, we have 
\begin{equation}
    \tilde{x} = \frac{q \tilde{E}}{m (- \omega^2 - \ii \gamma \omega + \omega_0^2)},
\end{equation}
and therefore 
\begin{equation}
    \epsr(\omega) = 1 + \underbrace{N q \alpha_{
         E \to x
    } (\omega)}_{E \to P} / \epsilon_0
    = 1 + \frac{N q^2}{m \epsilon_0} \frac{1}{
        - \omega^2 - \ii \gamma \omega + \omega_0^2
    },
\end{equation}
and therefore
\begin{equation}
    \epsilon_2(\omega) \coloneqq \Im \epsr(\omega)
    = \frac{N q^2}{m \epsilon_0} \frac{\gamma \omega}{
        (\omega^2 - \omega_0^2)^2 + \gamma^2 \omega^2
    }.
\end{equation}
The time average of the rate with which the fields do work 
on charges in a unit volume on the other hand is 
\begin{equation}
    \begin{aligned}
        Q &= N \expval{\dot{x} \cdot q E}
        = q N \cdot \frac{1}{2} \Re \tilde{x} \cdot \tilde{E}^* \\
        &= q N \cdot \frac{1}{2} \Re \expval{
            \frac{q \tilde{E} (- \ii \omega)}{
                m (- \omega^2 + \omega_0^2 - \ii \gamma \omega)
            } \cdot \tilde{E}^*
        } \\
        &= \frac{1}{2} \omega \epsilon_0 \abs*{\tilde{E}}^2 \cdot 
        \Im \frac{1}{
            - \omega^2 + \omega_0^2 - \ii \gamma \omega
        } \\
        &= \frac{1}{2} \omega \epsilon_0 \epsilon_2 \abs*{\tilde{E}}^2.
    \end{aligned}
\end{equation}

\section{Polarizability, absorption cross-section, stimulated emission, and optical amplification}

\paragraph{(a)} The EOM of a classical atom described by the Drude-Lorentz model is 
\begin{equation}
    \ddot{x} + \gamma \dot{x} + \omega_0^2 x = \frac{q}{m} E.
\end{equation}
For a plane wave external electric field whose phasor is 
\begin{equation}
    \tilde{E} = \tilde{E}_0 \ee^{- \ii (\omega t - k z)},
\end{equation}
the response of $x$, in the form of phasor, is 
\begin{equation}
    \tilde{x} = \frac{q}{m} \frac{\tilde{E}}{- \omega^2 + \omega_0^2 - \ii \gamma \omega},
    \label{eq:x-e-relation}
\end{equation}
and therefore 
\begin{equation}
    \tilde{\dot{x}} = - \ii \omega \frac{q}{m} \frac{\tilde{E}}{- \omega^2 + \omega_0^2 - \ii \gamma \omega},
\end{equation}
and the time average power of dissipation is 
\begin{equation}
    \expval{P_\text{abs}} = \expval{F_\text{dissipation} \cdot v} 
    = - \gamma \expval{v \cdot v} = - \gamma \cdot \frac{1}{2} \Re \tilde{\dot{x}}^* \cdot \tilde{\dot{x}}
    = - \frac{\gamma q^2}{2 m^2} \frac{\omega^2 \abs*{\tilde{E}_0}^2}{
        (\omega^2 - \omega_0^2)^2 + \gamma^2 \omega^2
    }.
\end{equation}

\paragraph{(b)} The absorption cross section can be calculated by 
\begin{equation}
    \sigma_\text{abs} = \frac{\abs*{\expval{P_\text{abs}}}}{\text{input intensity}}
    = \frac{\abs*{\expval{P_\text{abs}}}}{
        \frac{1}{\mu_0} \frac{k}{\omega} \underbrace{
            E^2
        }_{\frac{1}{2} \abs*{\tilde{E}_0}^2}   
    }
    = \frac{1}{c \epsilon_0} \frac{\omega^2}{
        (\omega^2 - \omega_0^2)^2 + \gamma^2 \omega^2
    } \gamma.
\end{equation}

The absorption coefficient can then be found 
by multiplying the number density to $\sigma_{\text{abs}}$.

\paragraph{(c)} Now consider a quantum two-level system described by 
\begin{equation}
    H = H_0 + H_{\text{dipole}}, \quad 
    H_0 = \hbar \omegag \dyad{\text{g}}
    + \hbar \omegae \dyad{\text{e}} , \quad 
\end{equation}
where
\begin{equation}
    H_{\text{dipole}} = - \vb*{\mu} \cdot \vb*{E} 
    = - \mu_{\text{eg}} \dyad{\text{e}}{\text{g}} E_{z0} \cos \omega t - \text{h.c.},
\end{equation}
and the $z$ direction is set to the direction of $\vb*{\mu}_{\text{eg}}$.
We start from the ground state in the $t \to - \infty$ limit, 
and the first order perturbation is 
\begin{equation}
    \ket*{\psi(t)} = \ket*{\text{g}} \ee^{- \ii \omegag t} 
    + \gamma^{(1)}_{\text{g}} \ket*{\text{g}} \ee^{- \ii \omegag t} 
    + \gamma^{(1)}_{\text{e}} \ket*{\text{e}} \ee^{- \ii \omegae t} ,
    \label{eq:first-pt-wave-function}
\end{equation}
where 
\begin{equation}
    \dv{\gamma^{(1)}_{\text{g}}}{t} = \frac{1}{\ii \hbar} \gamma^{(0)}_{\text{g}} \cdot \expval*{H_{\text{dipole}}}{\text{g}} = 0
\end{equation}
because dipole transition is only allowed between states with different parities, and 
\begin{equation}
    \begin{aligned}
        &\dv{\gamma_\text{e}^{(1)}}{t} = \frac{1}{\ii \hbar} 
        \gamma_{\text{g}}^{(0)} \ee^{\ii (\omegae - \omegag) t} 
        \cdot (- \mu_{\text{eg}} E_{z0} \cos \omega t ) \\
        \Rightarrow & 
        \gamma^{(1)}_{\text{e}}(t) - \underbrace{
            \Gammae^{(1)}(t=0)
        }_{0} = \ii \frac{\mu_{\text{eg}} E_{z0}}{\hbar} 
        \int_{- \infty}^{t} \dd{t'} \ee^{\ii (\omegae - \omegag) t}  \cos \omega t  .
    \end{aligned}
\end{equation}
Adding a small imaginary part to $\omegae$ because of spontaneous emission:
\begin{equation}
    \omegae \to \omegae - \ii \Gammae,
\end{equation}
and by completing this integral 
and not throwing away the high-frequency term, 
we get 
\begin{equation}
    \gamma^{(1)}_{\text{e}} = \ii \frac{\mu_{\text{eg}} E_{z0}}{\hbar} 
    \cdot \frac{1}{2} \left(
        \frac{\ee^{\ii (\omegaeg - \ii \Gammae + \omega) t}}{
            \ii (\omegaeg + \omega - \ii \Gammae)
        } + 
        \frac{\ee^{\ii (\omegaeg - \ii \Gammae - \omega) t}}{
            \ii (\omegaeg - \omega - \ii \Gammae)
        }
    \right), \quad 
    \omegaeg = \omegae - \omegag.
\end{equation}
Here we have thrown away the terms at $t = - \infty$ 
because when $t \to -\infty$, $\ee^{\Gammae t}$ vanishes.
$\Gammae^{(1)}$ now explodes when $t \to \infty$
but this doesn't matter because we still need to multiply 
$\ee^{- \ii (\omegae - \ii \Gammae) t}$ to it
because $\omegae$ in \eqref{eq:first-pt-wave-function}
also needs to be modified by the spontaneous emission rate.
The expectation value of the dipole is 
\begin{equation}
    \begin{aligned}
        \expval*{\mu}(t) &= \expval*{\mu}{\psi(t)}
        = \Gammae^{(1)}
        \ee^{- \ii (\omegaeg - \ii \Gammae) t} \mu_{\text{ge}} + \text{h.c.} \\ 
        &= \frac{\abs*{\mu_{\text{eg}}}^2 E_{0z}}{\hbar} \cdot \frac{1}{2} \left(
            \frac{\ee^{\ii \omega t}}{\omegaeg + \omega - \ii \Gammae}
            + \frac{\ee^{- \ii \omega t}}{\omegaeg - \omega - \ii \Gammae}
            + \frac{\ee^{- \ii \omega t}}{\omegaeg + \omega + \ii \Gammae}
            + \frac{\ee^{\ii \omega t}}{\omegaeg - \omega + \ii \Gammae}
        \right).
    \end{aligned}
\end{equation}
Here is a subtlety: in the above calculation, $\bra*{\psi(t)}$ is calculated using $H_0^\dagger$,
because strictly speaking, here we are working with a non-Hermitian system 
because of the $- \ii \Gammae$ correction.
Since 
\begin{equation}
    E = \frac{1}{2} E_{z0} (\ee^{\ii \omega t} + \ee^{- \ii \omega t}),
\end{equation}
by looking at $\ee^{- \ii \omega t}$ components,
we find the frequency-domain response function from $E$ to $\mu$ is 
\begin{equation}
    \begin{aligned}
        \alpha(\omega) &= \frac{\abs*{\mu_{\text{eg}}}^2}{\hbar} \left(
            \frac{1}{\omegaeg - \omega - \ii \Gammae}
            + \frac{1}{\omegaeg + \omega + \ii \Gammae}
        \right) \\
        &\approx \frac{\abs*{\mu_{\text{eg}}}^2}{\hbar} \cdot 
        \frac{2\omegaeg}{
            \omegaeg^2 - \omega^2 - 2 \ii \Gammae \omega
        }.
    \end{aligned} 
    \label{eq:alpha-two-level}
\end{equation}
In the last line we have used the $\Gammae \ll \omegae - \omegag$ condition.

\paragraph{(d)} The prefactor in \eqref{eq:alpha-two-level} differs from that 
in the susceptibility of a harmonic oscillator, 
and hence the necessity of the $f_i$ factor.
But the structure of the denominator is the same:
we have 
\begin{equation}
    \omega_0 = \omegaeg , \quad 
    \gamma = 2 \Gammae.
\end{equation}

\paragraph{(e)} We need to calculate energy loss due to spontaneous emission.
Since 
\begin{equation}
    \abs*{\gamma_{\text{e}}^{(1)}}^2 \propto \ee^{- 2 \ii \Gammae t},
\end{equation}
we find that in time period $\dd{t}$ the probability of spontaneous emission is $2 \Gammae \dd{t}$,
and therefore power of energy loss is 
\begin{equation}
    \begin{aligned}
        P_{\text{abs}} &=  2 \Gammae \cdot \hbar (\omegae - \omegag) 
        \braket{\ptwfc{\text{e}}{1}}{\ptwfc{\text{e}}{1}}  \\
        &= 2 \Gammae \cdot \hbar \omegaeg \cdot 
        \frac{\abs*{\mu_{\text{eg}}}^2 E_{z0}^2}{\hbar^2} 
        \cdot \frac{1}{2} \left(
            \frac{\ee^{\ii (\omegaeg - \ii \Gammae + \omega) t}}{
                \omegaeg + \omega - \ii \Gammae
            } + 
            \frac{\ee^{\ii (\omegaeg - \ii \Gammae - \omega) t}}{
                \omegaeg - \omega - \ii \Gammae
            }
        \right) \cdot 
        \frac{1}{2} \left(
            \frac{\ee^{- \ii (\omegaeg - \ii \Gammae + \omega) t}}{
                \omegaeg + \omega + \ii \Gammae
            } + 
            \frac{- \ee^{\ii (\omegaeg - \ii \Gammae - \omega) t}}{
                \omegaeg - \omega + \ii \Gammae
            }
        \right).
    \end{aligned}
\end{equation}
We need to ignore the high oscillation terms, 
because they are sine waves and have zero expectation;
this is the same as what we did for the harmonic oscillator.
So 
\begin{equation}
    \expval{P_{\text{abs}}} = \frac{1}{2} \Gammae \omegaeg \frac{\abs*{\mueg}^2 E_{0z}^2}{\hbar}
    \left(
        \frac{1}{(\omegaeg + \omega)^2 + \Gammae^2} + 
        \frac{1}{(\omegaeg - \omega)^2 + \Gammae^2}
    \right).
\end{equation}
The absorption cross section then is 
\begin{equation}
    \sigma_{\text{abs}} = \frac{\expval{P_{\text{abs}}}}{
        \underbrace{\expval{S}}_{
            \frac{1}{2} \epsilon_0 c E_{0z}^2
        }
    }
    =  \frac{\abs*{\mueg}^2 \Gammae \omegaeg}{\hbar \epsilon_0 c}
    \left(
        \frac{1}{(\omegaeg + \omega)^2 + \Gammae^2} + 
        \frac{1}{(\omegaeg - \omega)^2 + \Gammae^2}
    \right).
\end{equation}

\paragraph{(f)} Now we consider the same system but the initial state is the excited state.
Correspondingly, the imaginary part of the energy now comes to the ground state, 
which denotes the probability for the atom to be brought to the excited state due to external pumping.
Repeating the above procedure but this time 
replacing $\omegag$ with $\omegag - \ii \Gammag$,
we have 
\begin{equation}
    \begin{aligned}
        \gamma_{\text{g}}^{(1)}(t) &= 
        \ii \frac{\muge \Ezzero}{\hbar} \cdot \frac{1}{2} \left(
            \frac{
                \ee^{- \ii (\omegaeg + \ii \Gammag + \omega) t}
            }{
                - \omegaeg - \omega - \ii \Gammag
            } + 
            \frac{
                \ee^{\ii (- \omegaeg - \ii \Gammag + \omega) t}
            }{
                - \omegaeg + \omega - \ii \Gammag
            }   
        \right),
    \end{aligned}
\end{equation}
and 
\begin{equation}
    \begin{aligned}
        \alpha(\omega) &= \frac{\abs*{\mueg}^2}{\hbar^2} \left(
            \frac{1}{- \omegaeg - \omega - \ii \Gammag}
            + \frac{1}{- \omegaeg + \omega + \ii \Gammag}
        \right) \\
        &\approx \frac{\abs*{\mueg}^2}{\hbar^2} \frac{- 2 \omegaeg}{
            \omegaeg^2 - \omega^2 - 2 \ii \Gammag \omega
        } .
    \end{aligned}
\end{equation}
This can be obtained by swapping $\ket*{\text{e}}$ and $\ket*{\text{g}}$ in \eqref{eq:alpha-two-level}.

\paragraph{(g)} Again by swapping the two energy levels, we have 
\begin{equation}
    \expval{P_{\text{abs}}} = 
    - \frac{1}{2} \Gammag \omegaeg \frac{\abs*{\mueg}^2 E_{0z}^2}{\hbar}
    \left(
        \frac{1}{(\omegaeg + \omega)^2 + \Gammag^2} + 
        \frac{1}{(\omegaeg - \omega)^2 + \Gammag^2}
    \right).
\end{equation}
The absorption power is negative, 
because the system is gaining energy, not losing it.
The stimulated emission cross section then is 
\begin{equation}
    \sigma_{\text{st}} = - \frac{\abs*{\mueg}^2 \Gammag \omegaeg}{\hbar \epsilon_0 c}
    \left(
        \frac{1}{(\omegaeg + \omega)^2 + \Gammae^2} + 
        \frac{1}{(\omegaeg - \omega)^2 + \Gammae^2}
    \right).
\end{equation}

\paragraph{(h)} Suppose we have a dilute gas containing $N$ atoms per unit volume,
and the excited and ground state populations are 
$N_{\text{e}} = N p_{\text{e}}$ and 
$N_{\text{g}} = N p_{\text{g}}$.
Since the gas is dilute, we don't need to analyze the interferences of 
the $\Gammag$ and $\Gammae$ processes, 
and we have 
\begin{equation}
    \begin{aligned}
        \chi(\omega) &= N_{\text{g}} \alpha_{\text{abs}}(\omega)
        + N_{\text{e}} \alpha_{\text{st}}(\omega) \\
        &= N \frac{\abs*{\mu_{\text{eg}}}^2}{\hbar} \cdot 
        \left(
            p_{\text{g}} \frac{2\omegaeg}{
                \omegaeg^2 - \omega^2 - 2 \ii \Gammae \omega
            } 
            - p_{\text{e}} \frac{ 2 \omegaeg}{
                \omegaeg^2 - \omega^2 - 2 \ii \Gammag \omega
            }
        \right) .
    \end{aligned}
\end{equation}
We want $\Im \chi < 0$ 
to have a negative imaginary part in $\vb*{k}$,
and therefore $\ee^{\ii \vb*{k} \cdot \vb*{r}}$ increases as the light goes forward.
Assuming $\Gammag = \Gammae$, we find the condition needed is 
\begin{equation}
    p_{\text{e}} > p_{\text{g}}.
\end{equation}

\section{Spontaneous decay rate computed from Fermi's golden rule and Poynting's Theorem}

\paragraph{(a)} For a two-level atom, 
the interaction Hamiltonian between it and the electromagnetic field in vacuum is 
\begin{equation}
    H_{\text{dipole}} = - \vb*{\mu} \cdot \vb*{E}.
\end{equation}
The electric field, as a quantum operator, is 
\begin{equation}
    \vb*{E}(\vb*{r}) = \sum_{\vb*{k}, \sigma}
    \sqrt{\frac{\hbar \omega_{\vb*{k}}}{2 \epsilon_0 V}}
    \vu*{\vb*{k} \epsilon}_\sigma \ee^{\ii \vb*{k} \cdot \vb*{r}} a_{\vb*{k} \sigma} 
    + \text{h.c.}
\end{equation}
The term in $H_{\text{dipole}}$ that are relevant to 
the excited state-to-ground state transition is  
\begin{equation}
    \mel**{\text{g}, n_{\vb*{k} \sigma} = 1}{H_{\text{dipole}}}{\text{e}, 0}
    = - \sqrt{\frac{\hbar \omega_{\vb*{k}}}{2 \epsilon_0 V}} 
    \muge \ee^{\ii \vb*{k} \cdot \vb*{r}} \cos \theta_{\vb*{k} \sigma}.
\end{equation}
By Fermi's golden rule, the spontaneous decay rate is 
\begin{equation}
    \begin{aligned}
        \Gamma_{\text{e} \to \text{g}} &= 
        \frac{2\pi}{\hbar} \sum_{\vb*{k}, \sigma} 
        \abs*{
            \mel**{\text{g}, n_{\vb*{k} \sigma} = 1}{H_{\text{dipole}}}{\text{e}, 0}
        }^2
        \delta(\hbar \omegae - \hbar \omegag - \hbar \omega_{\vb*{k} \sigma}) \\
        &= \frac{2\pi}{\hbar^2} \frac{\hbar}{2 \epsilon_0 V}  \abs*{\mueg}^2 
        \sum_{\vb*{k}, \sigma} \omega_{\vb*{k}}
        \cos^2 \theta_{\vb*{k} \sigma} 
        \delta(\omegaeg - \omega_{\vb*{k} \sigma}) \\
        &= \frac{2\pi}{\hbar^2} \frac{\hbar}{2 \epsilon_0 V} \abs*{\mueg}^2 
        \cdot V \int \dd{\omega} \omega D(\omega) \int \frac{\dd{\Omega}}{4\pi} 
        \cos^2 \theta
        \delta(\omegaeg - \omega)  \\
        &= \frac{2\pi}{\hbar^2} \frac{\hbar}{2 \epsilon_0 V} \abs*{\mueg}^2 
        \cdot V \cdot \omegaeg D(\omegaeg) \cdot \frac{1}{3} ,
    \end{aligned}
\end{equation}
where we have decomposed the $\vb*{k}$ dependence 
of the transition matrix element 
into dependence on $\omega$ and on $\vu*{k}$,
and by averaging over the latter, 
the $\cos^2 \theta_{\vb*{k} \sigma}$  factor evaluates to $1/3$.
The density of state $D(\omega)$ contains a factor of $2$ 
because of polarization degeneracy;
for a single polarization the DOS in three-dimensional space is 
\begin{equation}
    D_{\text{single polarization}} = \frac{\omega^2}{2 \pi^2 c^3},
\end{equation}
so finally we have 
\begin{equation}
    \Gamma_{\text{e} \to \text{g}} = \frac{
        \omegaeg^3 \abs*{\mueg}^2
    }{
        3 \pi \hbar \epsilon_0 c^3
    }.
    \label{eq:fermi-golden-spontaneous-emission}
\end{equation}

\paragraph{(b)} When the state of the atom is 
\begin{equation}
    \ket*{\psi} = \coeffg \ketg + \coeffe \kete
\end{equation}
at $t = 0$, the time-dependent wave function is 
\begin{equation}
    \ket*{\psi(t)} = \coeffe \ee^{- \ii \omegae t} \kete 
    + \coeffg \ee^{- \ii \omegag t} \ketg.
\end{equation}
Since the dipole operator only has non-zero matrix element 
between $\ketg$ and $\kete$,
we have 
\begin{equation}
    \expval{\vb*{\mu}}(t) = (\coeffe^* \coeffg \mueg \ee^{\ii \omegaeg t}
    + \coeffg^* \coeffe \muge \ee^{- \ii \omegaeg t}) \vu*{z}.
    \label{eq:dipole-expectation-superposition}
\end{equation}

\paragraph{(c)} From classical electrodynamics 
(calculating $\vb*{E}$ and $\vb*{B}$ from 
the retarded potential radiated by a constantly oscillating dipole),
we have 
\begin{equation}
    P_{\text{rad}} = \frac{\mu_0 \omega^4 \abs*{\mu}^2}{12 \pi c},
\end{equation}
where the dipole, in the phasor form, is 
\begin{equation}
    \tilde{\vb*{\mu}} = \mu \ee^{- \ii \omega t} \vu*{z}.
\end{equation}
Applying this to \eqref{eq:dipole-expectation-superposition},
whose phasor form is 
\begin{equation}
    \tilde{\expval{\vb*{\mu}}}(t) =
    2 \coeffg^* \coeffe \muge \ee^{- \ii \omegaeg t}  \vu*{z},
\end{equation}
we find the radiation power of the atom -- 
if we assume its radiation somehow can be captured by classical electrodynamics --
is 
\begin{equation}
    P_{\text{rad}} = \frac{\mu_0 \omegaeg^4 }{3 \pi c} 
    \abs*{\coeffg}^2 \abs*{\coeffe}^2 \abs*{\mueg}^2
    = \frac{\mu_0 \omegaeg^4 }{3 \pi c} 
    \pope \popg \abs*{\mueg}^2
    = \frac{\omegaeg^4 }{3 \pi \epsilon_0 c^3} 
    \pope \popg \abs*{\mueg}^2 .
    \label{eq:superposition-state-classical-power-loss}
\end{equation} 

\paragraph{(d)} \eqref{eq:superposition-state-classical-power-loss} 
naturally leads to the following rate equation:
\begin{equation}
    \dv{E}{t} = - P_{\text{rad}}, 
\end{equation}
where 
\begin{equation}
    E = \hbar \omegae \pope + \hbar \omegag \popg, \quad 
    \pope + \popg = 1.
\end{equation}
The rate equation concerning $\popg$ is therefore 
\begin{equation}
    \begin{aligned}
        \hbar \omegae (- \dot{p}_{\text{g}}) + \hbar \omegag \dot{p}_{\text{g}}
        &= - \frac{\omegaeg^4 }{3 \pi \epsilon_0 c^3} 
        \underbrace{(1 - \popg)}_{\pope} \popg \abs*{\mueg}^2 \\
        \Rightarrow 
        \dv{\popg}{t} &= \frac{\omegaeg^3}{3 \pi \epsilon_0 c^3} \abs*{\mueg}^2 \popg \underbrace{
            (1 - \popg)
        }_{\pope}.
    \end{aligned}
\end{equation}


\paragraph{(e)} When $\pope \simeq 1$ and $\popg \ll 1$, 
and we work in a period of time that's much smaller than the decay time
(so that the condition that the atom is most likely 
to be in the excited state is always true),
from the rate equation, the time evolution of $\popg$ is given by 
\begin{equation}
    \begin{aligned}
        \dv{\popg}{t} &\approx \frac{\omegaeg^3}{3 \pi \epsilon_0 c^3} \abs*{\mueg}^2 \popg  \\
        \Rightarrow \Gamma_{\text{e} \to \text{g}}
        &= \frac{\omegaeg^3}{3 \pi \epsilon_0 c^3} \abs*{\mueg}^2.
    \end{aligned}
\end{equation}
If we work with the evolution of $\pope$ instead, 
we have no transition rate at all, but this is due to 
our fixing $\pope$ to $1$.
The rate given here is the same as \eqref{eq:fermi-golden-spontaneous-emission}.

\section{Oscillator sum rules}

From \eqref{eq:x-e-relation} we have 
\begin{equation}
    \alpha_{\text{H.O.}} = \frac{q^2}{m} \frac{1}{- \omega^2 + \omega_0^2 - \ii \gamma \omega}.
\end{equation}
Comparing this with \eqref{eq:alpha-two-level}
(and noting that $\omega_0$ in the former 
is $\omegaeg$ in the latter 
and not the ground state frequency in the discussion below), 
we find the ``oscillator strength'' of the two-level system is 
\begin{equation}
    f_{\text{two-level}} =  \alpha_{\text{two-level}} / \alpha_{\text{H.O.}}
    = \frac{\abs*{\mueg}^2}{\hbar} \cdot 2 \omegaeg \cdot \frac{m}{q^2}
    = 2m \frac{\omegae - \omegag}{\hbar} \abs*{\mel**{\text{e}}{x}{\text{g}}}^2.
\end{equation}
Now consider a multi-level system,
the oscillator strength of the ground state $\leftrightarrow$ excited state $\ket*{n}$ pair 
is 
\begin{equation}
    f_{0, n} = 2m \frac{\omega_n - \omega_0}{\hbar} \abs*{\mel**{n}{x}{0}}^2.
\end{equation}
What we want to prove is 
\begin{equation}
    \sum_n f_{0, n} = \sum_n \frac{2m}{\hbar} (\omega_n - \omega_0) \abs*{\mel**{n}{x}{0}}^2 = 1.
\end{equation}

We start from 
\begin{equation}
    H = \frac{p^2}{2m} + V(x) \Rightarrow \comm*{x}{H} = \frac{\ii \hbar}{m} p,
\end{equation}
and therefore 
\begin{equation}
    \begin{aligned}
        \ii \hbar &= \expval*{\comm*{x}{p}}{0} \\
        &= \frac{m}{\ii \hbar} \expval*{\comm*{x}{\comm*{H}{x}}}{0} \\
        &= \frac{m}{\ii \hbar} \expval*{
            x^2 H - 2 xHx + H x^2
        }{0} \\
        &= \frac{m}{\ii \hbar} (
            2 \hbar \omega_0 \expval*{x^2}{0} 
            - 2 \expval*{x H x}{0}
        ) ,
    \end{aligned}
\end{equation}
where 
\begin{equation}
    \begin{aligned}
        \expval*{x^2}{0} = \sum_{n} \mel**{0}{x}{n} \mel**{n}{x}{0} 
        = \sum_n \abs*{\mel**{n}{x}{0}}^2,
    \end{aligned}
\end{equation}
and 
\begin{equation}
    \begin{aligned}
        \expval*{x H x}{0} &= \sum_{m, n} \mel**{0}{x}{m} \hbar \omega_n \delta_{mn} \mel**{n}{x}{0} \\
        &= \sum_n \hbar \omega_n \abs*{\mel**{n}{x}{0}}^2.
    \end{aligned}
\end{equation}
Thus 
\begin{equation}
    \begin{aligned}
        &\quad \ii \hbar = \frac{2m}{\ii} \sum_n (\omega_0 - \omega_n) \abs*{\mel**{n}{x}{0}}^2 \\
        &\Rightarrow \sum_n \frac{2m}{\hbar} (\omega_n - \omega_0) \abs*{\mel**{n}{x}{0}}^2 = 1.
    \end{aligned}
\end{equation}

\section{Kramers-Kronig relations}

The key point here is we can't swap integration and differentiation 
when dealing with principle integral. 
Actually we have 
\begin{equation}
    \begin{aligned}
        &\quad \dv{\omega} \primevalue \int_{-\infty}^{\infty} \frac{
            A(\omega')
        }{
            \omega' - \omega
        } \dd{\omega'} \\
        &= 
        \dv{\omega} \left(
            \int_{-\infty}^{\omega - \epsilon} + 
            \int_{\omega + \epsilon}^{+ \infty} 
        \right)
        \frac{
            A(\omega')
        }{
            \omega' - \omega
        } \dd{\omega'} \\
        &= \frac{A(\omega - \epsilon)}{
            \omega - \epsilon - \omega
        }
        - \frac{
            A(\omega + \epsilon)
        }{
            \omega + \epsilon - \omega
        }
        + \left(
            \int_{-\infty}^{\omega - \epsilon} + 
            \int_{\omega + \epsilon}^{+ \infty} 
        \right)
        \dv{\omega}  \frac{
            A(\omega')
        }{
            \omega' - \omega
        } \dd{\omega'} \\
        &= - \frac{2 A(\omega)}{\epsilon} 
        - \left(
            \int_{-\infty}^{\omega - \epsilon} + 
            \int_{\omega + \epsilon}^{+ \infty} 
        \right)
        A(\omega') \dv{\omega'}  \frac{
            1
        }{
            \omega' - \omega
        } \dd{\omega'} \\
        &= - \frac{2 A(\omega)}{\epsilon} 
        + \left(
            \int_{-\infty}^{\omega - \epsilon} + 
            \int_{\omega + \epsilon}^{+ \infty} 
        \right)
        \frac{
            1
        }{
            \omega' - \omega
        } \dv{A(\omega')}{\omega'}   \dd{\omega'}
        - \frac{A(\omega - \epsilon)}{
            \omega - \epsilon - \omega
        }
        + \frac{A(\omega + \epsilon)}{
            \omega + \epsilon - \omega
        } \\
        &= \left(
            \int_{-\infty}^{\omega - \epsilon} + 
            \int_{\omega + \epsilon}^{+ \infty} 
        \right)
        \frac{
            1
        }{
            \omega' - \omega
        } \dv{A(\omega')}{\omega'}   \dd{\omega'} \\
        &= 
        \primevalue \int_{-\infty}^{\infty} 
        \frac{
            1
        }{
            \omega' - \omega
        } \dv{A(\omega')}{\omega'}   \dd{\omega'}.
    \end{aligned}
\end{equation}
Thus from 
\begin{equation}
    \epsilon_1(\omega) = \frac{1}{\pi} \primevalue \int_{-\infty}^{\infty} \frac{
            \epsilon_2(\omega')
        }{
            \omega' - \omega
        } \dd{\omega'} 
\end{equation}
we have 
\begin{equation}
    \dv{\epsilon_1}{\omega} = \frac{1}{\pi} \primevalue \int_{-\infty}^{\infty} \frac{
            1
        }{
            \omega' - \omega
        } \dv{\epsilon_2}{\omega'} \dd{\omega'} ,
\end{equation}
and similarly 
\begin{equation}
    \dv{\epsilon_2}{\omega} = - \frac{1}{\pi} \primevalue \int_{-\infty}^{\infty} \frac{
            1
        }{
            \omega' - \omega
        } \dv{\epsilon_1}{\omega'} \dd{\omega'} .
\end{equation}

\section{Waves and envelopes}

\paragraph{(a)} We employ the following ansatz 
\begin{equation}
    a(x, t) = A(\epsilon x, \epsilon t) \alpha(x, t)
\end{equation}
to the wave equation 
\begin{equation}
    c^2 \partial_x^2 a - \partial_t^2 a = 0.
\end{equation}
We have 
\begin{equation}
    c^2 \left(
        A \partial_x^2 \alpha 
        + 2 \epsilon \partial_{\epsilon x} A \partial_x \alpha
        + \epsilon^2 \alpha \partial_{\epsilon x}^2 A
    \right) = 
    \left(
        A \partial_t^2 \alpha 
        + 2 \epsilon \partial_{\epsilon t} A \partial_t \alpha
        + \epsilon^2 \alpha \partial_{\epsilon t}^2 A
    \right),
\end{equation}
and therefore the $\epsilon^0$ order equation is 
\begin{equation}
    c^2 \partial_x^2 \alpha - \partial_t^2 \alpha = 0,
    \label{eq:eps-0}
\end{equation}
and the $\epsilon^1$ order equation is 
\begin{equation}
    c^2 \partial_x \alpha \partial_{\epsilon x} A - \partial_t \alpha \partial_{\epsilon t} A = 0.
    \label{eq:eps-1}
\end{equation}

\paragraph{(b)} A general wave solution for \eqref{eq:eps-0} is 
\begin{equation}
    \alpha(x, t) = \sum_{k} \alpha_k \ee^{\ii (k x - \omega t)},
    \label{eq:fourier-alpha}
\end{equation}
where 
\begin{equation}
    c^2 k^2 - \omega^2 = 0 \Rightarrow
    \omega = c \abs*{k}.
\end{equation}

\paragraph{(c)} Choosing only one component in \eqref{eq:fourier-alpha} 
(without loss of generality, since the equation is linear),
\eqref{eq:eps-1} becomes 
\begin{equation}
    c^2 (\ii k) \partial_{\epsilon x} A - (- \ii \omega) \partial_{\epsilon t} A = 0 
    \Rightarrow \sgn(k) c \partial_x A + \partial_t A = 0,
\end{equation}
where we have set $\epsilon$ back to 1.
This equation is equivalent to say that $A$ remains the same 
on the line 
\begin{equation}
    x - c \cdot \sgn(k) t = \const,
\end{equation}
and thus 
\begin{equation}
    A = A (x - c t \sgn(k)),
\end{equation}
and 
\begin{equation}
    a(x, t) = \sum_k A_k(x - c t \sgn(k)) \ee^{\ii (k x - \omega t)}.
\end{equation}

The conditions required for the above decomposition 
is that the band with of $a(x,  t)$ should be very narrow, 
so a ``major'' frequency can still be recognized.
Usually we also assume that the medium is not very dispersive, 
or otherwise although $a(x, t)$ still looks like 
$A(x - ct \sgn(k)) \ee^{\ii (kx - \omega t)}$,
the width of the wave packet in the real space rapidly grows 
and in the end it can't be recognized anymore.

\end{document}