\documentclass[hyperref, a4paper]{article}

\usepackage{geometry}
\usepackage{titling}
\usepackage{titlesec}
% No longer needed, since we will use enumitem package
% \usepackage{paralist}
\usepackage{enumitem}
\usepackage{footnote}
\usepackage[colorinlistoftodos]{todonotes}
\usepackage{amsmath, amssymb, amsthm}
\usepackage{mathtools}
\usepackage{bbm}
\usepackage{graphicx}
\usepackage{subcaption}
\usepackage{soulutf8}
\usepackage{physics}
\usepackage{tensor}
\usepackage{siunitx}
\usepackage[version=4]{mhchem}
\usepackage{tikz}
\usepackage{xcolor}
\usepackage{listings}
\usepackage{autobreak}
\usepackage[ruled, vlined, linesnumbered]{algorithm2e}
\usepackage{nameref,zref-xr}
\zxrsetup{toltxlabel}
\usepackage[backend=bibtex]{biblatex}
\addbibresource{elasticity.bib}
\usepackage[colorlinks,unicode]{hyperref} % , linkcolor=black, anchorcolor=black, citecolor=black, urlcolor=black, filecolor=black
\usepackage[most]{tcolorbox}
\usepackage{prettyref}

% Page style
\geometry{left=3.18cm,right=3.18cm,top=2.54cm,bottom=2.54cm}
\titlespacing{\paragraph}{0pt}{1pt}{10pt}[20pt]
\setlength{\droptitle}{-5em}

% More compact lists 
\setlist[itemize]{
    itemindent=17pt, 
    leftmargin=1pt,
    listparindent=\parindent,
    parsep=0pt,
}

% Math operators
\DeclareMathOperator{\timeorder}{\mathcal{T}}
\DeclareMathOperator{\diag}{diag}
\DeclareMathOperator{\legpoly}{P}
\DeclareMathOperator{\primevalue}{P}
\DeclareMathOperator{\sgn}{sgn}
\DeclareMathOperator{\res}{Res}
\newcommand*{\ii}{\mathrm{i}}
\newcommand*{\ee}{\mathrm{e}}
\newcommand*{\const}{\mathrm{const}}
\newcommand*{\suchthat}{\quad \text{s.t.} \quad}
\newcommand*{\argmin}{\arg\min}
\newcommand*{\argmax}{\arg\max}
\newcommand*{\normalorder}[1]{: #1 :}
\newcommand*{\pair}[1]{\langle #1 \rangle}
\newcommand*{\fd}[1]{\mathcal{D} #1}
\DeclareMathOperator{\bigO}{\mathcal{O}}

% TikZ setting
\usetikzlibrary{arrows,shapes,positioning}
\usetikzlibrary{arrows.meta}
\usetikzlibrary{decorations.markings}
\tikzstyle arrowstyle=[scale=1]
\tikzstyle directed=[postaction={decorate,decoration={markings,
    mark=at position .5 with {\arrow[arrowstyle]{stealth}}}}]
\tikzstyle ray=[directed, thick]
\tikzstyle dot=[anchor=base,fill,circle,inner sep=1pt]

% Algorithm setting
% Julia-style code
\SetKwIF{If}{ElseIf}{Else}{if}{}{elseif}{else}{end}
\SetKwFor{For}{for}{}{end}
\SetKwFor{While}{while}{}{end}
\SetKwProg{Function}{function}{}{end}
\SetArgSty{textnormal}

\newcommand*{\concept}[1]{{\textbf{#1}}}

% Embedded codes
\lstset{basicstyle=\ttfamily,
  showstringspaces=false,
  commentstyle=\color{gray},
  keywordstyle=\color{blue}
}

% Reference formatting
\newcommand*{\citesec}[1]{\S~{#1}}
\newcommand*{\citechap}[1]{chap.~{#1}}
\newcommand*{\citefig}[1]{Fig.~{#1}}
\newcommand*{\citetable}[1]{Table~{#1}}
\newcommand*{\citepage}[1]{pp.~{#1}}
\newrefformat{fig}{Fig.~\ref{#1}}
\newcommand*{\term}[1]{\textit{#1}}

% Color boxes
\tcbuselibrary{skins, breakable, theorems}

\newtcbtheorem{infobox}{Box}{
    enhanced,
    boxrule=0pt,
    colback=blue!5,
    colframe=blue!5,
    coltitle=blue!50,
    borderline west={4pt}{0pt}{blue!65},
    sharp corners,
    fonttitle=\bfseries, 
    breakable,
    before upper={\parindent15pt\noindent}}{box}
\newtcbtheorem[use counter from=infobox]{theorybox}{Box}{
    enhanced,
    boxrule=0pt,
    colback=orange!5, 
    colframe=orange!5, 
    coltitle=orange!50,
    borderline west={4pt}{0pt}{orange!65},
    sharp corners,
    fonttitle=\bfseries, 
    breakable,
    before upper={\parindent15pt\noindent}}{box}
\newtcbtheorem[use counter from=infobox]{learnbox}{Box}{
    enhanced,
    boxrule=0pt,
    colback=green!5,
    colframe=green!5,
    coltitle=green!50,
    borderline west={4pt}{0pt}{green!65},
    sharp corners,
    fonttitle=\bfseries, 
    breakable,
    before upper={\parindent15pt\noindent}}{box}


\newenvironment{shelldisplay}{\begin{lstlisting}}{\end{lstlisting}}

\newcommand*{\omegap}{\omega_{\text{p}}}    
\newcommand*{\kB}{k_{\text{B}}}
\newcommand*{\muB}{\mu_{\text{B}}}
\newcommand*{\efermi}{E_{\text{F}}}
\newcommand*{\pfermi}{p_{\text{F}}}
\newcommand*{\vfermi}{v_{\text{F}}}
\newcommand*{\sA}{\text{A}}
\newcommand*{\sB}{\text{B}}
\newcommand*{\Tc}{T_{\text{c}}}
\newcommand*{\hethree}{$^3$He}
\newcommand*{\hefour}{$^4$He}
\newcommand{\epsr}{\epsilon_{\text{r}}}
\newcommand{\chie}{\chi_{\text{e}}}
\newcommand{\Efreq}{\tilde{\vb*{E}}}
\newcommand{\Dfreq}{\tilde{\vb*{D}}}
\newcommand{\Pfreq}{\tilde{\vb*{P}}}

\title{Homework 1}
\author{Jinyuan Wu}

\begin{document}

\maketitle

\section{The energy velocity}

Since $\epsr$ has frequency dependence,
the relation between $\vb*{E}(t)$ and $\vb*{D}(t)$ 
is not localized in the time domain,
and therefore although we still know that 
the energy would be a quadratic form of $\vb*{E}$ or $\vb*{D}$,
since 
\begin{equation}
    \pdv{\vb*{E}}{t} \cdot \vb*{D} \neq \pdv{\vb*{D}}{t} \cdot \vb*{E},
\end{equation}
the simple relation 
\[
    u_e = \frac{1}{2} \vb*{D} \cdot \vb*{E} 
\]
no longer holds.
Instead, we should start from the most generic theory 
and utilize 
\begin{equation}
    \pdv{u_e}{t} = \vb*{E} \cdot \pdv{\vb*{D}}{t}.
    \label{eq:change-of-ue}
\end{equation} 
To use this equation to get an expression of $u_e$,
we should no longer work with plane waves, 
or otherwise $u_e$ is a constant and 
we don't see any change of $u_e$ at all.
Below we work with a wave packet centered at $\pm \omega_0$.
For the wave packet, the electric field is 
\begin{equation}
    \vb*{E}(t) = \ee^{- \ii \omega_0 t} \cdot 
        \underbrace{
            \int \frac{\dd{\omega}}{2\pi} \ee^{- \ii (\omega - \omega_0) t}
            \Efreq(\omega)
        }_{\eqqcolon \vb*{E}_0(t)},
\end{equation}
\begin{equation}
    \vb*{D}(t) = \ee^{- \ii \omega_0 t} \cdot 
        \int \frac{\dd{\omega}}{2\pi} \ee^{- \ii (\omega - \omega_0) t}
            \varepsilon(\omega) \Efreq(\omega) .
\end{equation}
By Taylor expansion of $\varepsilon$ we have 
\begin{equation}
    \begin{aligned}
        \pdv*{\vb*{D}}{t} &= 
        \ee^{- \ii \omega_0 t} \cdot 
        \int \frac{\dd{\omega}}{2\pi} \ee^{- \ii (\omega - \omega_0) t} \Efreq(\omega) 
        (-\ii \omega) \left(
            \varepsilon(\omega_0) + 
            (\omega - \omega_0) \eval{\dv{\varepsilon}{\omega}}_{\omega = \omega_0}
            + \cdots 
        \right) \\
        &\approx \ee^{- \ii \omega_0 t} \cdot 
        \int \frac{\dd{\omega}}{2\pi} \ee^{- \ii (\omega - \omega_0) t} \Efreq(\omega) 
        \left(
            - \ii \omega_0 \varepsilon(\omega_0) 
            - \ii (\omega - \omega_0) \varepsilon(\omega)_0 
            - \ii (\omega - \omega_0) \omega 
                \eval{\dv{\varepsilon}{\omega}}_{\omega = \omega_0}
        \right) \\
        &\approx \ee^{- \ii \omega_0 t} \cdot 
        \int \frac{\dd{\omega}}{2\pi} \ee^{- \ii (\omega - \omega_0) t} \Efreq(\omega) 
        \Bigg(
            - \ii \omega_0 \varepsilon(\omega_0) 
            \underbrace{
                - \ii (\omega - \omega_0) \varepsilon(\omega)_0 
                - \ii (\omega - \omega_0) \omega_0  
                    \eval{\dv{\varepsilon}{\omega}}_{\omega = \omega_0}
            }_{
                = - \ii (\omega - \omega_0) 
                \eval{\dv{(\omega \varepsilon)}{\omega}}_{\omega = \omega_0} 
            }
        \Bigg) \\
        &= \ee^{- \ii \omega_0 t} \underbrace{
            \left(
                - \ii \omega_0 \varepsilon(\omega_0) \vb*{E}_0(t)
                + \dv{(\omega \varepsilon)}{\omega} 
                \pdv{\vb*{E}_0}{t}  
            \right)
        }_{\eqqcolon \vb*{D}_0(t)} .
    \end{aligned}
    \label{eq:pdv-d-t-derivation}
\end{equation}
In the second line we throw away the higher order Taylor terms;
in the third line we only keep terms linear to $(\omega - \omega_0)$.
These approximations require the wave packet 
to be focused enough.
We use $\expval{\cdots}$ to refer to 
averaging over the fast oscillations;
thus, $\vb*{E}_0(t)$ and $\vb*{D}_0(t)$ above 
can be regarded as constants when applying $\expval{\cdots}$,
and hence we find 
\begin{equation}
    \begin{aligned}
        \expval{\pdv{u_e}{t}} &= \expval{\vb*{E} \cdot \pdv{\vb*{D}}{t}}
        = \frac{1}{2} \cdot \frac{1}{4} \Re (
            \vb*{D}_0^*(t) \cdot \vb*{E}_0(t)
            + \vb*{D}_0(t) \cdot \vb*{E}_0^*(t)
        ) \\
        &\approx \frac{1}{4} \Re \left(
            \left(
                \omega_0 \varepsilon_2(\omega_0) 
                + \dv{(\omega \varepsilon_1)}{\omega} 
            \right)
                \pdv{\vb*{E}_0^*}{t} \cdot \vb*{E}_0 
            + \left(
                \omega_0 \varepsilon_2(\omega_0)
                + \dv{(\omega \varepsilon_1)}{\omega} 
            \right)
            \vb*{E}_0^* \cdot \pdv{\vb*{E}_0}{t} 
        \right) \\
        &= \frac{1}{4} \left(
            \omega_0 \varepsilon_2(\omega_0) +
            \dv{(\omega \varepsilon_1)}{\omega}
        \right) \pdv{\abs*{\vb*{E}_0}^2}{t} .
    \end{aligned}
\end{equation}
In the second line we have considered 
both the real and imaginary parts of $\epsilon$.%
\footnote{
    Note that $\epsilon(\omega) = \epsilon(- \omega)^*$
    comes from the fact that $\epsilon$ is real in the time domain;
    it says nothing about whether the system is Hermitian;
    the Hermitian condition is $\epsilon(\omega) = \epsilon(\omega)^*$.
} 
Since $u_e$ contains no fast oscillation, we have 
\begin{equation}
    \pdv{\expval{u_e}}{t} = \expval{\pdv{u_e}{t}} 
    = \frac{1}{4} \left(
        \omega_0 \varepsilon_2(\omega_0) +
        \dv{(\omega \varepsilon_1)}{\omega}
    \right)  
    \pdv{\abs*{\vb*{E}_0}^2}{t} .
\end{equation}
Similarly we have 
\begin{equation}
    \pdv{\expval{u_m}}{t} = \expval{\pdv{u_m}{t}} 
    = \frac{1}{4} \left(
        \omega_0 \mu_2(\omega_0) +
        \dv{(\omega \mu_1)}{\omega}
    \right) 
    \pdv{\abs*{\vb*{H}_0}^2}{t} .
\end{equation}
When the matter is modeled by harmonic oscillators, 
$\mu$ doesn't undergo any correction, 
but let's work with a slightly generalized case.
Eventually we have 
\begin{equation}
    \expval{u} = \expval{u_e + u_m} 
    = \frac{1}{4} \left(
        \omega_0 \varepsilon_2(\omega_0) +
        \dv{(\omega \varepsilon_1)}{\omega}
    \right)  
    \abs*{\vb*{E}_0}^2 + 
    \frac{1}{4} \left(
        \omega_0 \mu_2(\omega_0) +
        \dv{(\omega \mu_1)}{\omega}
    \right) 
    \abs*{\vb*{H}_0}^2.
\end{equation}
In the case of the Lorentz oscillator,
$\mu$ is real, and we have 
\begin{equation}
    \expval{u_m} = \frac{1}{4} \mu_0 \abs*{\vb*{H}_0}^2 
    = \frac{1}{4} \mu_0 \cdot \frac{\abs*{\varepsilon}}{\mu_0} \abs*{\vb*{E}_0}^2
    = \frac{1}{4} \abs{\varepsilon} \abs*{\vb*{E}_0}^2.
\end{equation}

The evaluation of the time averaged Poynting vector is more straightforward:
since 
\begin{equation}
    \curl{\vb*{E}} = - \pdv{\vb*{B}}{t} \Rightarrow
    \ii \vb*{k} \times \vb*{E} = - (- \ii \omega) \vb*{B},
\end{equation}
we just have 
\begin{equation}
    \begin{aligned}
        \expval{\vb*{S}} &= \frac{1}{\mu} \expval{\vb*{E} \times \vb*{B}} \\
        &= \frac{1}{\mu} \cdot \frac{1}{4} \Re \left(
            \vb*{E}^*_0 \times \vb*{B}_0 + \vb*{E}_0 \times \vb*{B}^*_0 
        \right) \\
        &= \frac{1}{2 \mu} \frac{\Re \vb*{k}}{\omega} \abs*{\vb*{E}_0}^2
        = \frac{1}{2} \Re \sqrt{\frac{\varepsilon}{\mu}} \abs*{\vb*{E}_0}^2 \vu*{k} ,
    \end{aligned}
\end{equation}
where we have used the condition $\vb*{k} \cdot \vb*{E} = 0$,
and in the third line we have used the condition that 
the directions of the real and imaginary parts of $\vb*{k}$
are the same, and therefore 
from $\vb*{k} \cdot \vb*{E}_0 = 0$ we also have 
$\vb*{k}^* \cdot \vb*{E} = 0$.
The energy velocity is therefore 
\begin{equation}
    v_{E} = \frac{\abs*{\expval{\vb*{S}}}}{\expval{u_e + u_m}}
    = 
\end{equation}

\end{document}