\documentclass[hyperref, a4paper]{article}

\usepackage{geometry}
\usepackage{titling}
\usepackage{titlesec}
% No longer needed, since we will use enumitem package
% \usepackage{paralist}
\usepackage{enumitem}
\usepackage{footnote}
\usepackage[colorinlistoftodos]{todonotes}
\usepackage{amsmath, amssymb, amsthm}
\usepackage{autobreak}
\usepackage{mathtools}
\usepackage{bbm}
\usepackage{graphicx}
\usepackage{subcaption}
\usepackage{soulutf8}
\usepackage{physics}
\usepackage{tensor}
\usepackage{siunitx}
\usepackage[version=4]{mhchem}
\usepackage{tikz}
\usepackage{xcolor}
\usepackage{listings}
\usepackage{autobreak}
\usepackage[ruled, vlined, linesnumbered]{algorithm2e}
\usepackage{nameref,zref-xr}
\zxrsetup{toltxlabel}
\usepackage[backend=bibtex]{biblatex}
\addbibresource{electrodynamics.bib}
\usepackage[colorlinks,unicode]{hyperref} % , linkcolor=black, anchorcolor=black, citecolor=black, urlcolor=black, filecolor=black
\usepackage[most]{tcolorbox}
\usepackage{prettyref}

% Page style
\geometry{left=3.18cm,right=3.18cm,top=2.54cm,bottom=2.54cm}
\titlespacing{\paragraph}{0pt}{1pt}{10pt}[20pt]
\setlength{\droptitle}{-5em}

% More compact lists 
\setlist[itemize]{
    itemindent=17pt, 
    leftmargin=1pt,
    listparindent=\parindent,
    parsep=0pt,
}

% Math operators
\DeclareMathOperator{\timeorder}{\mathcal{T}}
\DeclareMathOperator{\diag}{diag}
\DeclareMathOperator{\legpoly}{P}
\DeclareMathOperator{\primevalue}{P}
\DeclareMathOperator{\sgn}{sgn}
\DeclareMathOperator{\res}{Res}
\newcommand*{\ii}{\mathrm{i}}
\newcommand*{\ee}{\mathrm{e}}
\newcommand*{\const}{\mathrm{const}}
\newcommand*{\suchthat}{\quad \text{s.t.} \quad}
\newcommand*{\argmin}{\arg\min}
\newcommand*{\argmax}{\arg\max}
\newcommand*{\normalorder}[1]{: #1 :}
\newcommand*{\pair}[1]{\langle #1 \rangle}
\newcommand*{\fd}[1]{\mathcal{D} #1}
\DeclareMathOperator{\bigO}{\mathcal{O}}

% TikZ setting
\usetikzlibrary{arrows,shapes,positioning}
\usetikzlibrary{arrows.meta}
\usetikzlibrary{decorations.markings}
\tikzstyle arrowstyle=[scale=1]
\tikzstyle directed=[postaction={decorate,decoration={markings,
    mark=at position .5 with {\arrow[arrowstyle]{stealth}}}}]
\tikzstyle ray=[directed, thick]
\tikzstyle dot=[anchor=base,fill,circle,inner sep=1pt]

% Algorithm setting
% Julia-style code
\SetKwIF{If}{ElseIf}{Else}{if}{}{elseif}{else}{end}
\SetKwFor{For}{for}{}{end}
\SetKwFor{While}{while}{}{end}
\SetKwProg{Function}{function}{}{end}
\SetArgSty{textnormal}

\newcommand*{\concept}[1]{{\textbf{#1}}}

% Embedded codes
\lstset{basicstyle=\ttfamily,
  showstringspaces=false,
  commentstyle=\color{gray},
  keywordstyle=\color{blue}
}

% Reference formatting
\newcommand*{\citesec}[1]{\S~{#1}}
\newcommand*{\citechap}[1]{chap.~{#1}}
\newcommand*{\citefig}[1]{Fig.~{#1}}
\newcommand*{\citetable}[1]{Table~{#1}}
\newcommand*{\citepage}[1]{pp.~{#1}}
\newrefformat{fig}{Fig.~\ref{#1}}
\newcommand*{\term}[1]{\textit{#1}}

% Color boxes
\tcbuselibrary{skins, breakable, theorems}

\newtcbtheorem{infobox}{Box}{
    enhanced,
    boxrule=0pt,
    colback=blue!5,
    colframe=blue!5,
    coltitle=blue!50,
    borderline west={4pt}{0pt}{blue!65},
    sharp corners,
    fonttitle=\bfseries, 
    breakable,
    before upper={\parindent15pt\noindent}}{box}
\newtcbtheorem[use counter from=infobox]{theorybox}{Box}{
    enhanced,
    boxrule=0pt,
    colback=orange!5, 
    colframe=orange!5, 
    coltitle=orange!50,
    borderline west={4pt}{0pt}{orange!65},
    sharp corners,
    fonttitle=\bfseries, 
    breakable,
    before upper={\parindent15pt\noindent}}{box}
\newtcbtheorem[use counter from=infobox]{learnbox}{Box}{
    enhanced,
    boxrule=0pt,
    colback=green!5,
    colframe=green!5,
    coltitle=green!50,
    borderline west={4pt}{0pt}{green!65},
    sharp corners,
    fonttitle=\bfseries, 
    breakable,
    before upper={\parindent15pt\noindent}}{box}


\newenvironment{shelldisplay}{\begin{lstlisting}}{\end{lstlisting}}

\newcommand*{\omegap}{\omega_{\text{p}}}    
\newcommand*{\kB}{k_{\text{B}}}
\newcommand*{\muB}{\mu_{\text{B}}}
\newcommand*{\efermi}{E_{\text{F}}}
\newcommand*{\pfermi}{p_{\text{F}}}
\newcommand*{\vfermi}{v_{\text{F}}}
\newcommand*{\sA}{\text{A}}
\newcommand*{\sB}{\text{B}}
\newcommand*{\Tc}{T_{\text{c}}}
\newcommand*{\hethree}{$^3$He}
\newcommand*{\hefour}{$^4$He}
\newcommand{\epsr}{\epsilon_{\text{r}}}
\newcommand*{\mur}{\mu_{\text{r}}}
\newcommand{\chie}{\chi_{\text{e}}}
\newcommand{\Efreq}{\tilde{E}}
\newcommand{\Dfreq}{\tilde{D}}
\newcommand{\Pfreq}{\tilde{\vb*{P}}}

\newcommand*{\Gammae}{\Gamma_{\text{e}}}
\newcommand*{\Gammag}{\Gamma_{\text{g}}}
\newcommand*{\omegae}{\omega_{\text{e}}}
\newcommand*{\omegag}{\omega_{\text{g}}}
\newcommand*{\omegaeg}{\omega_{\text{eg}}}
\newcommand*{\ptwfc}[2]{\psi^{(#2)}_{#1}}
\newcommand*{\mueg}{\mu_{\text{eg}}}
\newcommand*{\muge}{\mu_{\text{ge}}}
\newcommand*{\Ezzero}{E_{z0}}
\newcommand*{\kete}{\ket*{\text{e}}}
\newcommand*{\ketg}{\ket*{\text{g}}}
\newcommand*{\coeffe}{c_{\text{e}}}
\newcommand*{\coeffg}{c_{\text{g}}}
\newcommand*{\pope}{p_{\text{e}}}
\newcommand*{\popg}{p_{\text{g}}}

\allowdisplaybreaks

\title{Homework 4}
\author{Jinyuan Wu}

\begin{document}

\maketitle

\section{Phonons and IR Active Modes}

\subsection{}

Phononic dispersion of a linear diatomic chain and discrete Floquet modes: Consider a 1D linear chain of atoms that are constrained to move along the $x$-axis, as seen in Fig. 1. The system has two atoms per unit cell with inter-atomic spacing $b$ and $c$ such that $a=(b+c)$. To keep things more general, we assume that each atom can have a distinct mass $(M, m)$ and spring constant $\left(k_1, k_2\right)$ and we use $u_n$ $\left(v_n\right)$ to represents the displacement of the blue (red) atom of the $n^{\text {th }}$ unit cell from equilibrium.

\paragraph*{(a)} Find the coupled finite-difference equations that describe the motion of the diatomic system.

The EOM of $v_n$ is 
\begin{equation}
    M \ddot{v}_n = k_2 (u_{n+1} - v_n) - k_1 (v_n - u_n), \quad 
\end{equation}
and the EOM of $u_n$ is 
\begin{equation}
    m \ddot{u}_n = k_1 (v_n - u_n) - k_2 (u_n - v_{n-1}).
\end{equation}

Discrete Floquet modes: One can show that the time-harmonic wave solutions of such discrete periodic systems take the form
$$
\left[\begin{array}{l}
u_n \\
v_n
\end{array}\right]=\left[\begin{array}{l}
g_n \\
h_n
\end{array}\right] e^{i q x} \text {, where }\left[\begin{array}{l}
g_{n+1} \\
h_{n+1}
\end{array}\right]=\left[\begin{array}{l}
g_n \\
h_n
\end{array}\right] \text { such that }\left[\begin{array}{l}
u_{n+1} \\
v_{n+1}
\end{array}\right]=\left[\begin{array}{l}
u_n \\
v_n
\end{array}\right] e^{q a} .
$$

In other words, our wave solution is a discrete version of a Floquet mode; remember that advancing from index $n$ to $n+1$ translates our system by a unit cell, corresponding to a distance $x_{n+1}-x_n=a$. Above we use $q$ to represent the wave-vector.

\paragraph*{(b)} Use a trial solution of this form to find a dispersion relation for this system. [In this case, it is sufficient to find an expression for $\omega_{ \pm}^2(q)$.]

The two EOMs becomes 
\begin{equation}
    - M \omega^2 v_n = k_2 (\ee^{\ii q a} u_n - v_n) - k_1 (v_n - u_n), 
\end{equation}
and 
\begin{equation}
    - m \omega^2 u_n = k_1 (v_n - u_n) - k_2 (u_n - v_n \ee^{- \ii q a}), 
\end{equation}
and therefore 
\begin{equation}
    \pmqty{
        m \omega^2 - k_1 - k_2    & k_1 + k_2 \ee^{- \ii q a} \\
        k_1 + \ee^{\ii q a} k_2   & M \omega^2 - k_1 - k_2 
    } \pmqty{u_n \\ v_n} = 0.
\end{equation}
Taking the determinant to be zero, we get 
\begin{equation}
    (m \omega^2 - k_1 - k_2)(M \omega^2 - k_1 - k_2) - (k_1 + k_2 \ee^{- \ii q a})(k_1 + k_2 \ee^{ \ii q a}) = 0,
\end{equation}
and hence 
\begin{equation}
    \omega_\pm(q) = \frac{
        (M + m)(k_1 + k_2) \pm \sqrt{
            (M+m)^2 (k_1 + k_2)^2 - 16 Mm k_1 k_2 \sin^2 (qa/2)
        }
    }{
        2 Mm
    }.
\end{equation}

\paragraph*{(c)} From this relatively simple expression, we can learn a lot about band gap formation within phononic lattices. Use this result to find size of the band gap in the situations tabulated below.

The band gap is expected to appear when $qa = \pm \pi$;
therefore the band gap is 
\begin{equation}
    \Delta \omega = \omega_+ - \omega_- = 
    \frac{ \sqrt{
        (M+m)^2 (k_1 + k_2)^2 - 16 Mm k_1 k_2 
    }}{Mm}.
\end{equation}
It can be seen that $b$ and $c$ are irrelevant in the band gap, 
so below we only care about $k$ and $m$.

\begin{itemize}
    \item When $M = m$ but $k_1 \neq k_2$ we have 
    \begin{equation}
        \Delta \omega = \frac{
            4m^2 (k_1 + k_2)^2 - 16 m^2 k_1 k_2
        }{m^2}
        = \frac{2 \abs*{k_1 - k_2}}{m} > 0.
    \end{equation}
    \item When $M \neq m$ and $k_1 = k_2 = k$, we have 
    \begin{equation}
        \Delta \omega = \frac{
            (M + m)^2 \cdot 4 k^2 - 16 Mm k^2
        }{Mm} = \frac{2 k \abs*{M - m}}{Mm}.
    \end{equation}
    \item When $M = m$ and $k_1 = k_2 = k$, $\Delta \omega = 0$.
\end{itemize}

\section{Periodic systems}

\subsection{}

Perturbative treatment of Bragg scattering: In Lecture 22, we treated a periodic modulation of dielectric constant as a perturbation on a uniform dielectric background to introduce Bragg scattering and photonic bandgap formation. Here, we fill in some of the steps that we skipped in class (for further context, see Lecture 22.N3). Since we we draw heavily on methods from solid state physics, you may find it helpful to consult Simon's book "The Oxford solid state basics," (2013).

Assuming 1D wave propagation in the z-direction and a dielectric distribution of the form $\varepsilon(z)=\bar{\varepsilon}+\Delta \varepsilon(z)$, we found that the wave equation, $\nabla \times \nabla \times \mathbf{E}=$ $\varepsilon_r(r)(\omega / c)^2 \mathbf{E}$, can be reduced to $\left[-\left(c^2 / \bar{\varepsilon}\right) \partial_z^2-\omega^2 \Delta \varepsilon_r(z) / \bar{\varepsilon}\right] \phi(z)=\omega^2 \phi(z)$ given a
wave solution of the form $\mathbf{E}(r, t)=\hat{y} \phi(z) e^{-i \omega t}$. Here, $\Delta \varepsilon_r(z)$ is a perturbation with very small amplitude.

\paragraph*{(a)} We begin by identify an appropriate first order correction to our eigenvalue equation. Assuming that $\Delta \varepsilon_r(z)$ is first order in smallness [i.e. $\Delta \varepsilon_r(z) \rightarrow$ $\gamma \Delta \varepsilon_r(z)$, where $\gamma$ is a unitless order parameter] expand the eigenvalue $\left(\omega^2\right)$ and eigenfunction $(\phi)$ in orders of $\gamma$ to properly formulate perturbation theory. Collect terms of order $(\gamma)^0$ and $(\gamma)^1$ and describe the general procedure by which you obtain the first order correction to the mode frequency.

[Hint: after subbing $\omega^2=\left(\omega_0^2+\gamma \omega_1^2+\gamma^2 \omega_2^2 \ldots\right)$ and $\phi(x)=\left(\phi_0(x)+\gamma \phi_1(x)+\right.$ $\left.\gamma^2 \phi_2(x)+\ldots\right)$ you should find that your zero- and first-order operators of the form $\hat{O}_0=-\left(c^2 / \bar{\varepsilon}\right) \partial_z^2$ and $\hat{O}_1=-\omega_0^2 \Delta \varepsilon_r(z) / \bar{\varepsilon}$, respectively.]

We have 
\[
    \begin{aligned}
        \text{LHS} &= - \frac{c^2}{\bar{\epsilon}} \partial_z^2 \phi_0 
        - \frac{\gamma c^2}{\bar{\epsilon}} \partial_z^2 \phi_1 - \cdots 
        - \frac{\gamma \omega_0^2 \Delta \epsr(z)}{\bar{\epsilon}} \phi_0 - \cdots ,
    \end{aligned}
\]
and 
\[
    \begin{aligned}
        \text{RHS} &= \omega_0^2 \phi_0 + \gamma \omega_0^2 \phi_1 + \gamma \omega_1^2 \phi_0 + \cdots,
    \end{aligned}
\]
and therefore 
\begin{equation}
    - \frac{c^2}{\bar{\epsilon}} \partial_z^2 \phi_0 = \omega_0^2 \phi_0,
\end{equation}
and 
\begin{equation}
    - \frac{\gamma c^2}{\bar{\epsilon}} \partial_z^2 \phi_1
    - \frac{\gamma \omega_0^2 \Delta \epsr(z)}{\bar{\epsilon}} \phi_0
    = \gamma \omega_0^2 \phi_1 + \gamma \omega_1^2 \phi_0.
\end{equation}
The second equation is not enough to decide $\omega_1$ and $\phi_1$;
kind of arbitrarily, we dictate that (here we set $\gamma$ back to 1)
\begin{equation}
    \omega_1^2 = \int \dd[3]{\vb*{r}} \phi_0^* \frac{- \omega_0^2 \Delta \epsr(z)}{\bar{\epsilon}} \phi_0 ,
\end{equation}
or its equivalence when there is degeneracy; and $\phi_1$ can then be found if it's necessary.

Now we can just first find $\phi_0$ and then $\omega_1$.

\paragraph*{(b)} Next, we assume that our dielectric perturbation takes the form $\Delta \varepsilon_r(z)=$ $\left(\Delta \varepsilon_{p p} / 2\right) \cos (2 \pi z / a)$, where $\Delta \varepsilon_{p p}$ is the peak-to-peak modulation of dielectric constant. Assuming that the unperturbed solutions are plane waves, use two such plane waves $( \pm k)$ to find a first order estimate energy splitting (or photonic band gap) at the zone boundary.
[Hint: We need to use degenerate perturbation theory here! You should find a bandgap of $\Delta \omega_{g a p} \cong\left(\omega_0 / 4\right)\left(\Delta \varepsilon_{p p} / \bar{\varepsilon}\right)$.]

Solving the problem regarding $\phi_0$ and $\omega_0$, we have 
\begin{equation}
    \phi_{0, \vb*{q}}(\vb*{r}) = \frac{1}{\sqrt{V}} \ee^{\ii \vb*{q} \cdot \vb*{r}}, \quad 
    \omega_0(\vb*{q}) = \frac{c}{\sqrt{\bar{\epsilon}}} \abs*{\vb*{q}}.
\end{equation}
Now we consider the perturbation of $O_1$ to $\phi_{0, \pm \vb*{q}}$. We have 
\begin{equation}
    \begin{aligned}
        \int \dd[3]{\vb*{r}} \phi_{0, - \vb*{q}}^* O_1 \phi_{0, \vb*{q}}
        &= \frac{1}{V} \int \dd[3]{\vb*{r}} \ee^{\ii \vb*{q} \cdot \vb*{r}} 
        \frac{- \omega_0^2 \Delta \epsr(z)}{\bar{\epsilon}} 
        \ee^{\ii \vb*{q} \cdot \vb*{r}} \\
        &= - \frac{\omega_0^2}{\bar{\epsilon}} \cdot 
        \frac{1}{L} \int \dd{z} \ee^{2 \ii q z} \cdot \frac{\Delta \epsilon_\text{pp}}{2} \cos(2 \pi z / a) \\
        &= - \frac{\omega_0^2}{\bar{\epsilon}} \cdot \frac{\Delta \epsilon_\text{pp}}{4}
        \cdot \frac{1}{L} \int \dd{z} \ee^{\ii 2 q z} \left(
            \ee^{\ii \frac{2 \pi z}{a}} + \ee^{- \ii \frac{2 \pi z}{a}}
        \right) \\
        &= - \frac{\omega_0^2}{\bar{\epsilon}} \cdot \frac{\Delta \epsilon_\text{pp}}{4}
        (\delta_{2q, 2\pi / a} + \delta_{2q, - 2\pi / a}).
    \end{aligned}
\end{equation}
So the transition matrix element is non-zero when $q = \pm \pi / a$,
and the two wave vector points are connected by a $G$ vector and are equivalent.
What we need to do then is to diagonalize 
\begin{equation}
    \pmqty{
        \omega_0 & - \frac{\omega_0^2}{\bar{\epsilon}} \cdot \frac{\Delta \epsilon_\text{pp}}{4} \\
        - \frac{\omega_0^2}{\bar{\epsilon}} \cdot \frac{\Delta \epsilon_\text{pp}}{4} & \omega_0 
    }
\end{equation}
and the eigenvalues are 
\begin{equation}
    \omega_\pm = \omega_0 \pm \frac{\omega_0^2}{\bar{\epsilon}} \cdot \frac{\Delta \epsilon_\text{pp}}{4},
\end{equation}
where 
\begin{equation}
    \omega_0 = \frac{c}{\sqrt{\bar{\epsilon}}} \frac{\pi}{a}.
\end{equation}


\subsection{The Coherent-State and Classical Correspondence}

In this problem we examine some general properties of a coherent state. The coherent state, $|\alpha\rangle$, is defined as $|\alpha\rangle=e^{-|\alpha|^2 / 2} \sum_{n=0} \alpha^n(n !)^{-1 / 2}|n\rangle$, where $\alpha$ is a complex number that has a similar significance to our classical complex mode amplitude. Remember that $|n\rangle$ are eigenstates of the Hamiltonian, $H=\hbar \omega\left(a^{\dagger} a+\frac{1}{2}\right)$.

\paragraph*{(a)} Show that the lowering operator, $a$, produces $a|\alpha\rangle=\alpha|\alpha\rangle$.

We have 
\begin{equation}
    \begin{aligned}
        a \ket*{\alpha} &= \ee^{- \abs*{\alpha}^2 / 2} \sum_{n=0}^{\infty} \frac{\alpha^n}{\sqrt{n!}} \sqrt{n} \ket*{n-1} \\
        &= \ee^{- \abs*{\alpha}^2 / 2} \sum_{n=0}^{\infty} \frac{\alpha^{n+1}}{\sqrt{(n+1)!}} \sqrt{n+1} \ket*{n} \\
        &= \alpha \ee^{- \abs*{\alpha}^2 / 2} \sum_{n=0}^{\infty} \frac{\alpha^{n}}{\sqrt{n!}}  \ket*{n} \\
        &= \alpha \ket*{\alpha}.
    \end{aligned}
\end{equation}

\paragraph*{(b)} Use your result from (a) to find the time-evolution of a coherent state, $\frac{d}{d t}\langle\alpha(t)|a| \alpha(t)\rangle$, where $|\alpha(t)\rangle \equiv e^{-i \omega t} e^{-|\alpha|^2 / 2} \sum_{n=0} \alpha^n(n !)^{-1 / 2}|n\rangle$.
[Hint: Remember that $\frac{d}{d t}\langle\psi|a| \psi\rangle=\frac{i}{\hbar}\langle\psi|[H, a]| \psi\rangle$.]

Actually we have 
\begin{equation}
    \ket*{\alpha(t)} = \ee^{- \abs*{\alpha}^2 / 2} \sum_{n=0}^{\infty} \frac{(\alpha \ee^{- \ii \omega n t})^n}{\sqrt{n!}} \ket*{n}. 
\end{equation}
We have 
\begin{equation}
    \begin{aligned}
        \dv{t} \expval*{a}{\alpha} &= \frac{\ii}{\hbar} \expval*{\comm*{H}{a}}{\alpha} \\
        &= \frac{\ii}{\hbar} \expval*{-\hbar \omega a}{\alpha} = - \ii \omega \alpha \braket*{\alpha}{\alpha} \\ 
        &= - \ii \omega \alpha .
    \end{aligned}
\end{equation}
So we find the time evolution of $\alpha$ is 
\begin{equation}
    \alpha(t) = \alpha \ee^{- \ii \omega t}.
\end{equation}

\end{document}