\documentclass[hyperref, a4paper]{article}

\usepackage{geometry}
\usepackage{titling}
\usepackage{titlesec}
% No longer needed, since we will use enumitem package
% \usepackage{paralist}
\usepackage{enumitem}
\usepackage{footnote}
\usepackage[colorinlistoftodos]{todonotes}
\usepackage{amsmath, amssymb, amsthm}
\usepackage{mathtools}
\usepackage{bbm}
\usepackage{graphicx}
\usepackage{subcaption}
\usepackage{soulutf8}
\usepackage{physics}
\usepackage{tensor}
\usepackage{siunitx}
\usepackage[version=4]{mhchem}
\usepackage{tikz}
\usepackage{xcolor}
\usepackage{listings}
\usepackage{autobreak}
\usepackage[ruled, vlined, linesnumbered]{algorithm2e}
\usepackage{acro}
\usepackage{nameref,zref-xr}
\zxrsetup{toltxlabel}
\usepackage[backend=bibtex,sorting=none]{biblatex}
\addbibresource{floquet.bib}
\usepackage[colorlinks,unicode]{hyperref} % , linkcolor=black, anchorcolor=black, citecolor=black, urlcolor=black, filecolor=black
\usepackage[most]{tcolorbox}
\usepackage{prettyref}

% Page style
\geometry{left=3.18cm,right=3.18cm,top=2.54cm,bottom=2.54cm}
\titlespacing{\paragraph}{0pt}{1pt}{10pt}[20pt]
\setlength{\droptitle}{-5em}

% More compact lists 
\setlist[itemize]{
    itemindent=17pt, 
    leftmargin=1pt,
    listparindent=\parindent,
    parsep=0pt,
}

% Math operators
\DeclareMathOperator{\timeorder}{\mathcal{T}}
\DeclareMathOperator{\diag}{diag}
\DeclareMathOperator{\legpoly}{P}
\DeclareMathOperator{\primevalue}{P}
\DeclareMathOperator{\sgn}{sgn}
\DeclareMathOperator{\res}{Res}
\newcommand*{\ii}{\mathrm{i}}
\newcommand*{\ee}{\mathrm{e}}
\newcommand*{\const}{\mathrm{const}}
\newcommand*{\suchthat}{\quad \text{s.t.} \quad}
\newcommand*{\argmin}{\arg\min}
\newcommand*{\argmax}{\arg\max}
\newcommand*{\normalorder}[1]{: #1 :}
\newcommand*{\pair}[1]{\langle #1 \rangle}
\newcommand*{\fd}[1]{\mathcal{D} #1}
\DeclareMathOperator{\bigO}{\mathcal{O}}

% TikZ setting
\usetikzlibrary{arrows,shapes,positioning}
\usetikzlibrary{arrows.meta}
\usetikzlibrary{decorations.markings}
\tikzstyle arrowstyle=[scale=1]
\tikzstyle directed=[postaction={decorate,decoration={markings,
    mark=at position .5 with {\arrow[arrowstyle]{stealth}}}}]
\tikzstyle ray=[directed, thick]
\tikzstyle dot=[anchor=base,fill,circle,inner sep=1pt]

% Algorithm setting
% Julia-style code
\SetKwIF{If}{ElseIf}{Else}{if}{}{elseif}{else}{end}
\SetKwFor{For}{for}{}{end}
\SetKwFor{While}{while}{}{end}
\SetKwProg{Function}{function}{}{end}
\SetArgSty{textnormal}

\newcommand*{\concept}[1]{{\textbf{#1}}}

% Embedded codes
\lstset{basicstyle=\ttfamily,
  showstringspaces=false,
  commentstyle=\color{gray},
  keywordstyle=\color{blue}
}

% Reference formatting
\newcommand*{\citesec}[1]{\S~{#1}}
\newcommand*{\citechap}[1]{chap.~{#1}}
\newcommand*{\citefig}[1]{Fig.~{#1}}
\newcommand*{\citetable}[1]{Table~{#1}}
\newcommand*{\citepage}[1]{pp.~{#1}}
\newrefformat{fig}{Fig.~\ref{#1}}
\newcommand*{\term}[1]{\textit{#1}}

% Color boxes
\tcbuselibrary{skins, breakable, theorems}

\newtcbtheorem{infobox}{Box}{
    enhanced,
    boxrule=0pt,
    colback=blue!5,
    colframe=blue!5,
    coltitle=blue!50,
    borderline west={4pt}{0pt}{blue!65},
    sharp corners,
    fonttitle=\bfseries, 
    breakable,
    before upper={\parindent15pt\noindent}}{box}
\newtcbtheorem[use counter from=infobox]{theorybox}{Box}{
    enhanced,
    boxrule=0pt,
    colback=orange!5, 
    colframe=orange!5, 
    coltitle=orange!50,
    borderline west={4pt}{0pt}{orange!65},
    sharp corners,
    fonttitle=\bfseries, 
    breakable,
    before upper={\parindent15pt\noindent}}{box}
\newtcbtheorem[use counter from=infobox]{learnbox}{Box}{
    enhanced,
    boxrule=0pt,
    colback=green!5,
    colframe=green!5,
    coltitle=green!50,
    borderline west={4pt}{0pt}{green!65},
    sharp corners,
    fonttitle=\bfseries, 
    breakable,
    before upper={\parindent15pt\noindent}}{box}


\newenvironment{shelldisplay}{\begin{lstlisting}}{\end{lstlisting}}

% Shorthands
\DeclareAcronym{arpes}{long = angle-resolved photoemission spectroscopy, short = ARPES}

\newcommand*{\kB}{k_{\text{B}}}
\newcommand*{\muB}{\mu_{\text{B}}}
\newcommand*{\efermi}{E_{\text{F}}}
\newcommand*{\pfermi}{p_{\text{F}}}
\newcommand*{\vfermi}{v_{\text{F}}}
\newcommand*{\sA}{\text{A}}
\newcommand*{\sB}{\text{B}}
\newcommand*{\Tc}{T_{\text{c}}}
\newcommand*{\hethree}{$^3$He}
\newcommand*{\hefour}{$^4$He}
\newcommand{\epsr}{\epsilon_{\text{r}}}
\newcommand{\chie}{\chi_{\text{e}}}
\newcommand{\Efreq}{\tilde{\vb{E}}}
\newcommand{\Dfreq}{\tilde{\vb{D}}}
\newcommand{\Pfreq}{\tilde{\vb{P}}}

\title{Floquet theory}
\author{Jinyuan Wu}

\begin{document}

\maketitle


\paragraph*{Floquet theory in the temporal dimension} 
\textit{
    Apply the Floquet theory of differential equation to 
    how a wave function evolves according to a periodic Hamiltonian.
    What are counterparts of $\vb{k}$ and $\vb{G}$ vectors in band theory?
}

we know it is possible to expand an arbitrary state that 
evolves according to a Hamiltonian $H$ with period $T$ into 
a linear combination (the coefficients are constants) of 
$\{\ket*{\psi_n(t)}\}$ where 
\begin{equation}
    \ket*{\psi_n (t)} = \ee^{- \ii \varepsilon_n t / \hbar} \ket*{\Phi_n(t)},
    \quad \ket*{\Phi_n(t + T)} = \ket*{\Phi_n(t)}.
    \label{eq:floquet-wave-function}
\end{equation}
By discrete periodicity of $\ket*{\Phi_n(t)}$ we make Fourier expansion 
\begin{equation}
    \ket*{\Phi_n(t)} = \sum_m \ee^{- \ii m \omega t} \ket*{\phi_n^{(m)}},
    \label{eq:big-phi-to-extended-hilbert}
\end{equation}
where $m$ goes over all integers.
The quasienergies $\varepsilon_n$ are comparable to the crystal momentum,
and $m \omega$ are comparable to $\vb{G}$ vectors.
Note that here $\ket*{\phi_n^{(m)}}$
are \emph{Fourier coefficients} and are not eigenstates of anything; 
there is no normalization or orthogonality condition for them.

\paragraph*{Floquet effective Hamiltonian} 
\textit{
    Rewrite the Schrodinger equation 
    \begin{equation}
        \ii \hbar \dv{t} \ket*{\psi_n(t)} = H \ket*{\psi_n(t)}
    \end{equation}
    in a form that doesn't explicitly contain time. 
    The resulting matrix is known as the Floquet effective Hamiltonian.
}

The Schrodinger equation now reads 
\[
    \ii \hbar \sum_m (- \ii \varepsilon_n / \hbar - \ii m \omega) 
    \ee^{- \ii (\varepsilon_n / \hbar + m \omega) t} \ket*{\phi_n^{(m)}}
    = \sum_{m'} \ee^{- \ii \varepsilon_n / \hbar - \ii m' \omega t} H \ket*{\phi_n^{(m')}},
\] 
and by cancelling the $\ee^{- \ii \varepsilon_n t / \hbar}$ factor 
and doing inverse Fourier transform, we get 
\begin{equation}
    (\varepsilon_n + m \hbar \omega) \ket*{\phi_n^{(m)}} 
    = \sum_{m'} H^{(m - m')} \ket*{\phi_n^{(m')}},
\end{equation}
where 
\begin{equation}
    H(t) = \sum_{m} \ee^{- \ii m \omega t} H^{(m)}, \quad 
    H^{(m)} = \frac{1}{N T} \int_{0}^{NT} \dd{t} \ee^{\ii m \omega t} H(t),
\end{equation}
$N$ is a large positive integer. Thus we find 
\begin{equation}
    \varepsilon_n \ket*{\phi^{(m)}_n}
    = \sum_{m'} (
        H^{(m - m')} - m \hbar \omega \delta_{m m'}
    ) \ket*{\phi^{(m')}_{n}}.
    \label{eq:floquet-ham}
\end{equation}
$\varepsilon_n$ and the structure of $\ket*{\psi_n}$ can then be solved 
by diagonalizing \eqref{eq:floquet-ham}.

\paragraph*{The non-equilibrium nature of Floquet effective Hamiltonian}
\textit{
    Compare the Floquet effective Hamiltonian with an ordinary effective Hamiltonian, 
    say a two band model or a two-level atom model.
    Is there any substantial difference?
}

Although we may want to interpret \eqref{eq:floquet-ham}
as an effective Hamiltonian where photon degrees of freedom 
have been integrated out,
unlike conventional effective Hamiltonians
whose eigenstates can in principle be obtained by 
applying a projection operator on a subset of eigenstates of the full Hamiltonian,
eigenstates of \eqref{eq:floquet-ham} 
\emph{do not} correspond to any eigenstate of 
the full, time-independent Hamiltonian that contains both light and matter degrees of freedom.
If we consider the state of the full Hamiltonian corresponding to 
an eigenstate of \eqref{eq:floquet-ham},
we find that the light part of the former is in a coherent state,
which is far from any eigenstate of the linear electromagnetic Hamiltonian 
$\hbar \omega (n + \frac{1}{2})$.
Instead, Floquet formalisms is to be understood in a more generic framework of non-equilibrium physics:
Floquet Green function can be calculated within the Keldysh formalism,
and \eqref{eq:floquet-ham} can be understood as the 
non-equilibrium self-energy \cite{lubatsch2019evolution,aoki2014nonequilibrium}
and therefore is not necessarily an equilibrium effective Hamiltonian.

\paragraph*{The structure of the eigensystem of the Floquet effective Hamiltonian} 
\textit{
It can be seen that even when the dimension of the Hilbert space of matter is finite, 
the dimension of the Floquet effective Hamiltonian, if written as a matrix, 
is still infinite,
as we have the the index of ``photon number''.
Analyze the structure of the eigensystem of the Floquet effective Hamiltonian; 
compare it with the eigensystem in a spatially periodic system. 
}

If $\varepsilon_n$ is a solution of \eqref{eq:floquet-wave-function},
then so is $\varepsilon_n + m \hbar \omega$.
As for the relation between quasi-stationary eigenstates corresponding to 
$\varepsilon_n$ and $\varepsilon_n + m \hbar \omega$, 
we note that 
\begin{equation}
    \ket*{\psi_n(t)} = \ee^{- \ii \varepsilon_n t / \hbar} \sum_m \ee^{- \ii m \omega t} \ket*{\phi_n^{(m)}}
    = \ee^{- \ii (\varepsilon + \hbar m' \omega) t / \hbar}
    \sum_m \ee^{- \ii m \omega t} \ket*{\phi_n^{(m + m')}}.
\end{equation}
Therefore, if $(\varepsilon, \{\ket*{\phi^{(m)}}\}_m)$ 
is a solution of \eqref{eq:floquet-ham},
then so is $(\varepsilon + \hbar m' \omega, \{\ket*{\phi^{(m + m')}}\}_m)$.
This can also be directly shown from the form of \eqref{eq:floquet-ham}.

Although $(\varepsilon, \{\ket*{\phi^{(m)}}\}_m)$ and 
$(\varepsilon + \hbar m' \omega, \{\ket*{\phi^{(m + m')}}\}_m)$ 
seem to be independent solutions of \eqref{eq:floquet-ham},
they correspond to the the \emph{same} $\ket*{\psi_n(t)}$.
Therefore the fact that a Floquet quasienergy plus an integer times $\omega$
is also a Floquet quasienergy 
is not a symmetry of the system, but merely a redundancy.
Suppose we keep $M_m$ choices of $m$ when diagonalizing \eqref{eq:floquet-ham};
\eqref{eq:floquet-ham} then has $N_m \times N$ solutions, 
where $N$ is the dimension of the Hilbert space of the matter degrees of freedom.
Since one solution of \eqref{eq:floquet-ham} has $N_m - 1$ physically equivalent counterparts, 
\eqref{eq:floquet-ham} eventually has 
\[
    N \times N_m / N_m = N
\]
physically independent solutions.
There however is no redundant information in $\{\ket*{\phi_n^{(m)}}\}$,
because every $\ket*{\phi_n^{(m)}}$ is needed to determine $\ket*{\Phi_n(t)}$
and thus $\ket*{\psi_n(t)}$.

In conclusion, for each $\ket*{\psi_n(t)}$ 
we have countable infinite quasi-energies,
the difference between the nearest two being $\hbar \omega$;
thus all distinct Floquet quasi-eigenstates can be indexed 
by quasi-energies that are within one ``Floquet-Brillouin zone''.
Unlike the case in solid state physics, 
the state with quasienergy $\varepsilon_n$ is equivalent to 
the state with quasienergy $\varepsilon_n + m \hbar \omega$,
while in solid state physics 
of course $\ee^{\ii \vb{k} \cdot \vb{r}}$ is \emph{not} equivalent to $\ee^{\ii (\vb{k} + \vb{G}) \cdot \vb{r}}$;
the number of physically non-equivalent $\ket*{\psi_n(t)}$
is the same as the dimension of the Hilbert space.
(And therefore, in the ARPES spectrum of a Floquet system, 
the signature at $\varepsilon_{n \vb{k}} + \hbar m \omega$
\emph{does not} correspond to a new band other than the $n$th band: 
it's just a satellite in the signature of the Floquet-renormalized $n$th band 
that comes from the $\phi_{n \vb{k}}^{(m)}$ Floquet component of $\ket*{\psi_{n \vb{k}}}(t)$).

\paragraph*{Orthogonal relations} 
\textit{
    Under which condition are the Floquet quasi-stationary states orthogonal to each other?
}

Although there is no generic orthogonal relation pertaining to $\{\ket*{\psi_n(t)}\}$,
after time averaging an orthogonal relation can be obtained.
From the fact that \eqref{eq:floquet-ham} is Hermitian, we have 
\begin{equation}
    \sum_m \braket*{\phi_n^{(m)}}{\psi_{n'}^{(m)}} = \delta_{nn'},
\end{equation}
and therefore the time average of $\braket*{\psi_n(t)}{\psi_{n'}(t)}$ is 
\begin{equation}
    \begin{aligned}
        \overline{\braket*{\psi_n(t)}{\psi_{n'}(t)}} &= 
        \lim_{T \to \infty} \frac{1}{T} \int_{0}^{T} \dd{t} 
        \ee^{\ii (\varepsilon_n - \varepsilon_{n'}) t}
        \sum_{m, m'}  \ee^{\ii (m - m') t}
        \braket*{\phi_n^{(m)}}{\phi_{n'}^{(m')}} \\
        &= \sum_{m, m'} \delta_{\varepsilon_n, \varepsilon_{n'}}
        \delta_{mm'} \braket*{\phi_n^{(m)}}{\phi_{n'}^{(m')}}  \\
        &= \delta_{\varepsilon_n, \varepsilon_{n'}} \sum_{m} \braket*{\phi_n^{(m)}}{\psi_{n'}^{(m)}}
        = \delta_{nn'}.
    \end{aligned}
    \label{eq:orthogonal}
\end{equation}
Therefore, after time averaging, the Floquet quasi-stationary states 
are orthogonal to each other.




\printbibliography

\end{document}