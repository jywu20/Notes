\documentclass[hyperref, a4paper]{article}

\usepackage{geometry}
\usepackage{titling}
\usepackage{titlesec}
% No longer needed, since we will use enumitem package
% \usepackage{paralist}
\usepackage{enumitem}
\usepackage{footnote}
\usepackage[colorinlistoftodos]{todonotes}
\usepackage{amsmath, amssymb, amsthm}
\usepackage{autobreak}
\usepackage{mathtools}
\usepackage{bbm}
\usepackage{graphicx}
\usepackage{subcaption}
\usepackage{soulutf8}
\usepackage{physics}
\usepackage{tensor}
\usepackage{siunitx}
\usepackage[version=4]{mhchem}
\usepackage{tikz}
\usepackage{xcolor}
\usepackage{listings}
\usepackage{autobreak}
\usepackage[ruled, vlined, linesnumbered]{algorithm2e}
\usepackage{nameref,zref-xr}
\zxrsetup{toltxlabel}
\usepackage[backend=bibtex]{biblatex}
\addbibresource{electrodynamics.bib}
\usepackage[colorlinks,unicode]{hyperref} % , linkcolor=black, anchorcolor=black, citecolor=black, urlcolor=black, filecolor=black
\usepackage[most]{tcolorbox}
\usepackage{prettyref}

% Page style
\geometry{left=3.18cm,right=3.18cm,top=2.54cm,bottom=2.54cm}
\titlespacing{\paragraph}{0pt}{1pt}{10pt}[20pt]
\setlength{\droptitle}{-5em}

% More compact lists 
\setlist[itemize]{
    itemindent=17pt, 
    leftmargin=1pt,
    listparindent=\parindent,
    parsep=0pt,
}

% Math operators
\DeclareMathOperator{\timeorder}{\mathcal{T}}
\DeclareMathOperator{\diag}{diag}
\DeclareMathOperator{\legpoly}{P}
\DeclareMathOperator{\primevalue}{P}
\DeclareMathOperator{\sgn}{sgn}
\DeclareMathOperator{\res}{Res}
\newcommand*{\ii}{\mathrm{i}}
\newcommand*{\ee}{\mathrm{e}}
\newcommand*{\const}{\mathrm{const}}
\newcommand*{\suchthat}{\quad \text{s.t.} \quad}
\newcommand*{\argmin}{\arg\min}
\newcommand*{\argmax}{\arg\max}
\newcommand*{\normalorder}[1]{: #1 :}
\newcommand*{\pair}[1]{\langle #1 \rangle}
\newcommand*{\fd}[1]{\mathcal{D} #1}
\DeclareMathOperator{\bigO}{\mathcal{O}}

% TikZ setting
\usetikzlibrary{arrows,shapes,positioning}
\usetikzlibrary{arrows.meta}
\usetikzlibrary{decorations.markings}
\usetikzlibrary{calc}
\tikzstyle arrowstyle=[scale=1]
\tikzstyle directed=[postaction={decorate,decoration={markings,
    mark=at position .5 with {\arrow[arrowstyle]{stealth}}}}]
\tikzstyle ray=[directed, thick]
\tikzstyle dot=[anchor=base,fill,circle,inner sep=1pt]

% Algorithm setting
% Julia-style code
\SetKwIF{If}{ElseIf}{Else}{if}{}{elseif}{else}{end}
\SetKwFor{For}{for}{}{end}
\SetKwFor{While}{while}{}{end}
\SetKwProg{Function}{function}{}{end}
\SetArgSty{textnormal}

\newcommand*{\concept}[1]{{\textbf{#1}}}

% Embedded codes
\lstset{basicstyle=\ttfamily,
  showstringspaces=false,
  commentstyle=\color{gray},
  keywordstyle=\color{blue}
}

% Reference formatting
\newcommand*{\citesec}[1]{\S~{#1}}
\newcommand*{\citechap}[1]{chap.~{#1}}
\newcommand*{\citefig}[1]{Fig.~{#1}}
\newcommand*{\citetable}[1]{Table~{#1}}
\newcommand*{\citepage}[1]{pp.~{#1}}
\newrefformat{fig}{Fig.~\ref{#1}}
\newcommand*{\term}[1]{\textit{#1}}

% Color boxes
\tcbuselibrary{skins, breakable, theorems}

\newtcbtheorem{infobox}{Box}{
    enhanced,
    boxrule=0pt,
    colback=blue!5,
    colframe=blue!5,
    coltitle=blue!50,
    borderline west={4pt}{0pt}{blue!65},
    sharp corners,
    fonttitle=\bfseries, 
    breakable,
    before upper={\parindent15pt\noindent}}{box}
\newtcbtheorem[use counter from=infobox]{theorybox}{Box}{
    enhanced,
    boxrule=0pt,
    colback=orange!5, 
    colframe=orange!5, 
    coltitle=orange!50,
    borderline west={4pt}{0pt}{orange!65},
    sharp corners,
    fonttitle=\bfseries, 
    breakable,
    before upper={\parindent15pt\noindent}}{box}
\newtcbtheorem[use counter from=infobox]{learnbox}{Box}{
    enhanced,
    boxrule=0pt,
    colback=green!5,
    colframe=green!5,
    coltitle=green!50,
    borderline west={4pt}{0pt}{green!65},
    sharp corners,
    fonttitle=\bfseries, 
    breakable,
    before upper={\parindent15pt\noindent}}{box}


\newenvironment{shelldisplay}{\begin{lstlisting}}{\end{lstlisting}}

\newcommand*{\omegap}{\omega_{\text{p}}}    
\newcommand*{\kB}{k_{\text{B}}}
\newcommand*{\muB}{\mu_{\text{B}}}
\newcommand*{\efermi}{E_{\text{F}}}
\newcommand*{\pfermi}{p_{\text{F}}}
\newcommand*{\vfermi}{v_{\text{F}}}
\newcommand*{\sA}{\text{A}}
\newcommand*{\sB}{\text{B}}
\newcommand*{\Tc}{T_{\text{c}}}
\newcommand*{\hethree}{$^3$He}
\newcommand*{\hefour}{$^4$He}
\newcommand{\epsr}{\epsilon_{\text{r}}}
\newcommand*{\mur}{\mu_{\text{r}}}
\newcommand{\chie}{\chi_{\text{e}}}
\newcommand{\Efreq}{\tilde{E}}
\newcommand{\Dfreq}{\tilde{D}}
\newcommand{\Pfreq}{\tilde{\vb*{P}}}

\newcommand*{\Gammae}{\Gamma_{\text{e}}}
\newcommand*{\Gammag}{\Gamma_{\text{g}}}
\newcommand*{\omegae}{\omega_{\text{e}}}
\newcommand*{\omegag}{\omega_{\text{g}}}
\newcommand*{\omegaeg}{\omega_{\text{eg}}}
\newcommand*{\ptwfc}[2]{\psi^{(#2)}_{#1}}
\newcommand*{\mueg}{\mu_{\text{eg}}}
\newcommand*{\muge}{\mu_{\text{ge}}}
\newcommand*{\Ezzero}{E_{z0}}
\newcommand*{\kete}{\ket*{\text{e}}}
\newcommand*{\ketg}{\ket*{\text{g}}}
\newcommand*{\coeffe}{c_{\text{e}}}
\newcommand*{\coeffg}{c_{\text{g}}}
\newcommand*{\pope}{p_{\text{e}}}
\newcommand*{\popg}{p_{\text{g}}}

\allowdisplaybreaks

\title{Bloch equation and rate equation in two-level systems}
\author{Jinyuan Wu}

\begin{document}

\maketitle

Consider Eq. (5.125) in Steck's lecture notes.
The optical Bloch equations can be derived in a diagrammatic way. 
Basically, we have the following diagrams:
for the $\rho_{\text{ee}}$ we have (possibly the directions of the lines are wrong; check later)
\begin{equation}
    \tikzset{every picture/.style={line width=0.75pt}} %set default line width to 0.75pt     
    \begin{tikzpicture}[x=0.75pt,y=0.75pt,yscale=-1,xscale=1, baseline=(XXXX.south) ]
        \path (0,64);\path (110.66667175292969,0);\draw    ($(current bounding box.center)+(0,0.3em)$) node [anchor=south] (XXXX) {};
        %Straight Lines [id:da046377729854131466] 
        \draw    (11.33,17.83) -- (55.83,17.83) ;
        \draw [shift={(36.18,17.83)}, rotate = 180] [fill={rgb, 255:red, 0; green, 0; blue, 0 }  ][line width=0.08]  [draw opacity=0] (7.14,-3.43) -- (0,0) -- (7.14,3.43) -- (4.74,0) -- cycle    ;
        %Straight Lines [id:da24542155198649573] 
        \draw [line width=1.5]    (55.83,17.83) -- (100.33,17.83) ;
        \draw [shift={(82.48,17.83)}, rotate = 180] [fill={rgb, 255:red, 0; green, 0; blue, 0 }  ][line width=0.08]  [draw opacity=0] (8.75,-4.2) -- (0,0) -- (8.75,4.2) -- (5.81,0) -- cycle    ;
        %Straight Lines [id:da18856340149280704] 
        \draw    (55.83,17.83) .. controls (57.5,19.5) and (57.5,21.16) .. (55.83,22.83) .. controls (54.16,24.5) and (54.16,26.16) .. (55.83,27.83) .. controls (57.5,29.5) and (57.5,31.16) .. (55.83,32.83) .. controls (54.16,34.5) and (54.16,36.16) .. (55.83,37.83) .. controls (57.5,39.5) and (57.5,41.16) .. (55.83,42.83) .. controls (54.16,44.5) and (54.16,46.16) .. (55.83,47.83) .. controls (57.5,49.5) and (57.5,51.16) .. (55.83,52.83) -- (55.83,55.15) -- (55.83,55.15) ;
        \draw [shift={(55.83,55.15)}, rotate = 135] [color={rgb, 255:red, 0; green, 0; blue, 0 }  ][line width=0.75]    (-5.59,0) -- (5.59,0)(0,5.59) -- (0,-5.59)   ;
        % Text Node
        \draw (11.33,16.33) node [anchor=south] [inner sep=0.75pt]   [align=left] {e};
        % Text Node
        \draw (100.33,16.33) node [anchor=south] [inner sep=0.75pt]   [align=left] {e};
        % Text Node
        \draw (53.83,16.33) node [anchor=south east] [inner sep=0.75pt]   [align=left] {e};
        % Text Node
        \draw (57.83,16.33) node [anchor=south west] [inner sep=0.75pt]   [align=left] {g};
        \end{tikzpicture}
        \simeq \frac{1}{\partial _{t}}\frac{\Omega }{2} \rho _{\text{eg}} ,\quad \tikzset{every picture/.style={line width=0.75pt}} %set default line width to 0.75pt        
        \begin{tikzpicture}[x=0.75pt,y=0.75pt,yscale=-1,xscale=1, baseline=(XXXX.south) ]
        \path (0,64);\path (110.66667175292969,0);\draw    ($(current bounding box.center)+(0,0.3em)$) node [anchor=south] (XXXX) {};
        %Straight Lines [id:da9990993052852792] 
        \draw    (11.33,17.83) -- (55.83,17.83) ;
        \draw [shift={(29.48,17.83)}, rotate = 0] [fill={rgb, 255:red, 0; green, 0; blue, 0 }  ][line width=0.08]  [draw opacity=0] (7.14,-3.43) -- (0,0) -- (7.14,3.43) -- (4.74,0) -- cycle    ;
        %Straight Lines [id:da19159582923833396] 
        \draw [line width=1.5]    (55.83,17.83) -- (100.33,17.83) ;
        \draw [shift={(72.18,17.83)}, rotate = 0] [fill={rgb, 255:red, 0; green, 0; blue, 0 }  ][line width=0.08]  [draw opacity=0] (8.75,-4.2) -- (0,0) -- (8.75,4.2) -- (5.81,0) -- cycle    ;
        %Straight Lines [id:da1694913979940429] 
        \draw    (55.83,17.83) .. controls (57.5,19.5) and (57.5,21.16) .. (55.83,22.83) .. controls (54.16,24.5) and (54.16,26.16) .. (55.83,27.83) .. controls (57.5,29.5) and (57.5,31.16) .. (55.83,32.83) .. controls (54.16,34.5) and (54.16,36.16) .. (55.83,37.83) .. controls (57.5,39.5) and (57.5,41.16) .. (55.83,42.83) .. controls (54.16,44.5) and (54.16,46.16) .. (55.83,47.83) .. controls (57.5,49.5) and (57.5,51.16) .. (55.83,52.83) -- (55.83,55.15) -- (55.83,55.15) ;
        \draw [shift={(55.83,55.15)}, rotate = 135] [color={rgb, 255:red, 0; green, 0; blue, 0 }  ][line width=0.75]    (-5.59,0) -- (5.59,0)(0,5.59) -- (0,-5.59)   ;
        % Text Node
        \draw (11.33,16.33) node [anchor=south] [inner sep=0.75pt]   [align=left] {e};
        % Text Node
        \draw (100.33,16.33) node [anchor=south] [inner sep=0.75pt]   [align=left] {e};
        % Text Node
        \draw (53.83,16.33) node [anchor=south east] [inner sep=0.75pt]   [align=left] {e};
        % Text Node
        \draw (57.83,16.33) node [anchor=south west] [inner sep=0.75pt]   [align=left] {g};
        \end{tikzpicture}
        \simeq -\frac{1}{\partial _{t}}\frac{\Omega }{2} \rho _{\text{ge}} ,
\end{equation}
and for the $\rho_{\text{eg}}$ part we have 
\begin{equation}
    \tikzset{every picture/.style={line width=0.75pt}} %set default line width to 0.75pt     
    \begin{tikzpicture}[x=0.75pt,y=0.75pt,yscale=-1,xscale=1, baseline=(XXXX.south) ]
        \path (0,64);\path (110.66667175292969,0);\draw    ($(current bounding box.center)+(0,0.3em)$) node [anchor=south] (XXXX) {};
        %Straight Lines [id:da11753018701233797] 
        \draw    (11.33,17.83) -- (55.83,17.83) ;
        \draw [shift={(36.18,17.83)}, rotate = 180] [fill={rgb, 255:red, 0; green, 0; blue, 0 }  ][line width=0.08]  [draw opacity=0] (7.14,-3.43) -- (0,0) -- (7.14,3.43) -- (4.74,0) -- cycle    ;
        %Straight Lines [id:da3195946132664591] 
        \draw [line width=1.5]    (55.83,17.83) -- (100.33,17.83) ;
        \draw [shift={(82.48,17.83)}, rotate = 180] [fill={rgb, 255:red, 0; green, 0; blue, 0 }  ][line width=0.08]  [draw opacity=0] (8.75,-4.2) -- (0,0) -- (8.75,4.2) -- (5.81,0) -- cycle    ;
        %Straight Lines [id:da20149642058076367] 
        \draw    (55.83,17.83) .. controls (57.5,19.5) and (57.5,21.16) .. (55.83,22.83) .. controls (54.16,24.5) and (54.16,26.16) .. (55.83,27.83) .. controls (57.5,29.5) and (57.5,31.16) .. (55.83,32.83) .. controls (54.16,34.5) and (54.16,36.16) .. (55.83,37.83) .. controls (57.5,39.5) and (57.5,41.16) .. (55.83,42.83) .. controls (54.16,44.5) and (54.16,46.16) .. (55.83,47.83) .. controls (57.5,49.5) and (57.5,51.16) .. (55.83,52.83) -- (55.83,55.15) -- (55.83,55.15) ;
        \draw [shift={(55.83,55.15)}, rotate = 135] [color={rgb, 255:red, 0; green, 0; blue, 0 }  ][line width=0.75]    (-5.59,0) -- (5.59,0)(0,5.59) -- (0,-5.59)   ;
        % Text Node
        \draw (11.33,16.33) node [anchor=south] [inner sep=0.75pt]   [align=left] {e};
        % Text Node
        \draw (100.33,16.33) node [anchor=south] [inner sep=0.75pt]   [align=left] {g};
        % Text Node
        \draw (53.83,16.33) node [anchor=south east] [inner sep=0.75pt]   [align=left] {e};
        % Text Node
        \draw (57.83,16.33) node [anchor=south west] [inner sep=0.75pt]   [align=left] {g};
        \end{tikzpicture}
        \simeq \frac{1}{\partial _{t}}\frac{\Omega }{2} \rho _{\text{gg}} ,\quad \tikzset{every picture/.style={line width=0.75pt}} %set default line width to 0.75pt        
        \begin{tikzpicture}[x=0.75pt,y=0.75pt,yscale=-1,xscale=1, baseline=(XXXX.south) ]
        \path (0,64);\path (110.66667175292969,0);\draw    ($(current bounding box.center)+(0,0.3em)$) node [anchor=south] (XXXX) {};
        %Straight Lines [id:da5156777717229821] 
        \draw [line width=1.5]    (11.33,17.83) -- (55.83,17.83) ;
        \draw [shift={(27.68,17.83)}, rotate = 0] [fill={rgb, 255:red, 0; green, 0; blue, 0 }  ][line width=0.08]  [draw opacity=0] (8.75,-4.2) -- (0,0) -- (8.75,4.2) -- (5.81,0) -- cycle    ;
        %Straight Lines [id:da3459479993085457] 
        \draw [line width=0.75]    (55.83,17.83) -- (100.33,17.83) ;
        \draw [shift={(73.98,17.83)}, rotate = 0] [fill={rgb, 255:red, 0; green, 0; blue, 0 }  ][line width=0.08]  [draw opacity=0] (7.14,-3.43) -- (0,0) -- (7.14,3.43) -- (4.74,0) -- cycle    ;
        %Straight Lines [id:da8154188451232849] 
        \draw    (55.83,17.83) .. controls (57.5,19.5) and (57.5,21.16) .. (55.83,22.83) .. controls (54.16,24.5) and (54.16,26.16) .. (55.83,27.83) .. controls (57.5,29.5) and (57.5,31.16) .. (55.83,32.83) .. controls (54.16,34.5) and (54.16,36.16) .. (55.83,37.83) .. controls (57.5,39.5) and (57.5,41.16) .. (55.83,42.83) .. controls (54.16,44.5) and (54.16,46.16) .. (55.83,47.83) .. controls (57.5,49.5) and (57.5,51.16) .. (55.83,52.83) -- (55.83,55.15) -- (55.83,55.15) ;
        \draw [shift={(55.83,55.15)}, rotate = 135] [color={rgb, 255:red, 0; green, 0; blue, 0 }  ][line width=0.75]    (-5.59,0) -- (5.59,0)(0,5.59) -- (0,-5.59)   ;
        % Text Node
        \draw (11.33,16.33) node [anchor=south] [inner sep=0.75pt]   [align=left] {e};
        % Text Node
        \draw (100.33,16.33) node [anchor=south] [inner sep=0.75pt]   [align=left] {g};
        % Text Node
        \draw (53.83,16.33) node [anchor=south east] [inner sep=0.75pt]   [align=left] {e};
        % Text Node
        \draw (57.83,16.33) node [anchor=south west] [inner sep=0.75pt]   [align=left] {g};
        \end{tikzpicture}
        \simeq \frac{1}{\partial _{t}}\frac{\Omega }{2} \rho _{\text{ee}} ,
\end{equation}
and also 
\begin{equation}
    \tikzset{every picture/.style={line width=0.75pt}} %set default line width to 0.75pt     
    \begin{tikzpicture}[x=0.75pt,y=0.75pt,yscale=-1,xscale=1, baseline=(XXXX.south) ]
        \path (0,38);\path (67.33333587646484,0);\draw    ($(current bounding box.center)+(0,0.3em)$) node [anchor=south] (XXXX) {};
        %Straight Lines [id:da2856832892874426] 
        \draw    (10.33,22.83) -- (54.83,22.83) ;
        \draw [shift={(35.18,22.83)}, rotate = 180] [fill={rgb, 255:red, 0; green, 0; blue, 0 }  ][line width=0.08]  [draw opacity=0] (7.14,-3.43) -- (0,0) -- (7.14,3.43) -- (4.74,0) -- cycle    ;
        % Text Node
        \draw (10.33,21.33) node [anchor=south] [inner sep=0.75pt]   [align=left] {e};
        % Text Node
        \draw (54.83,21.33) node [anchor=south] [inner sep=0.75pt]   [align=left] {g};
        \end{tikzpicture}
        \simeq \frac{1}{\partial_t} \Delta \rho_{\text{eg}} .
    \label{eq:g-lesser-free-eg}
\end{equation}
Here we need to note that we are dealing with lesser Green functions, 
and therefore the free $G^<_{\text{gg}}$ or $G^<_{\text{ee}}$ is just $1 / \ii \partial_t$;
on the other hand, we have \eqref{eq:g-lesser-free-eg}.
This can be illustrated by the following diagrams: 
\begin{equation}
    \begin{tikzpicture}[x=0.75pt,y=0.75pt,yscale=-1,xscale=1, baseline=(XXXX.south) ]
        \path (0,43);\path (115.66667175292969,0);\draw    ($(current bounding box.center)+(0,0.3em)$) node [anchor=south] (XXXX) {};
        %Straight Lines [id:da5327526507160516] 
        \draw [line width=0.75]    (11.33,17.83) -- (55.83,17.83) ;
        \draw [shift={(36.18,17.83)}, rotate = 180] [fill={rgb, 255:red, 0; green, 0; blue, 0 }  ][line width=0.08]  [draw opacity=0] (7.14,-3.43) -- (0,0) -- (7.14,3.43) -- (4.74,0) -- cycle    ;
        %Straight Lines [id:da8400904796600857] 
        \draw [line width=0.75]    (55.83,17.83) -- (100.33,17.83) ;
        \draw [shift={(80.68,17.83)}, rotate = 180] [fill={rgb, 255:red, 0; green, 0; blue, 0 }  ][line width=0.08]  [draw opacity=0] (7.14,-3.43) -- (0,0) -- (7.14,3.43) -- (4.74,0) -- cycle    ;
        % Text Node
        \draw (11.33,16.33) node [anchor=south] [inner sep=0.75pt]   [align=left] {e};
        % Text Node
        \draw (100.33,16.33) node [anchor=south] [inner sep=0.75pt]   [align=left] {g};
        % Text Node
        \draw (55.83,16.33) node [anchor=south] [inner sep=0.75pt]   [align=left] {g};
        % Text Node
        \draw (11.33,19.33) node [anchor=north] [inner sep=0.75pt]   [align=left] {+};
        % Text Node
        \draw (55.83,19.33) node [anchor=north] [inner sep=0.75pt]   [align=left] {\mbox{-}};
        % Text Node
        \draw (100.33,19.33) node [anchor=north] [inner sep=0.75pt]   [align=left] {\mbox{-}};
        \end{tikzpicture}
        ,\quad \tikzset{every picture/.style={line width=0.75pt}} %set default line width to 0.75pt        
        \begin{tikzpicture}[x=0.75pt,y=0.75pt,yscale=-1,xscale=1, baseline=(XXXX.south) ]
        \path (0,43);\path (115.66667175292969,0);\draw    ($(current bounding box.center)+(0,0.3em)$) node [anchor=south] (XXXX) {};
        %Straight Lines [id:da28216203396670925] 
        \draw [line width=0.75]    (11.33,17.83) -- (55.83,17.83) ;
        \draw [shift={(36.18,17.83)}, rotate = 180] [fill={rgb, 255:red, 0; green, 0; blue, 0 }  ][line width=0.08]  [draw opacity=0] (7.14,-3.43) -- (0,0) -- (7.14,3.43) -- (4.74,0) -- cycle    ;
        %Straight Lines [id:da6223132732301591] 
        \draw [line width=0.75]    (55.83,17.83) -- (100.33,17.83) ;
        \draw [shift={(80.68,17.83)}, rotate = 180] [fill={rgb, 255:red, 0; green, 0; blue, 0 }  ][line width=0.08]  [draw opacity=0] (7.14,-3.43) -- (0,0) -- (7.14,3.43) -- (4.74,0) -- cycle    ;
        % Text Node
        \draw (11.33,16.33) node [anchor=south] [inner sep=0.75pt]   [align=left] {e};
        % Text Node
        \draw (100.33,16.33) node [anchor=south] [inner sep=0.75pt]   [align=left] {g};
        % Text Node
        \draw (55.83,16.33) node [anchor=south] [inner sep=0.75pt]   [align=left] {e};
        % Text Node
        \draw (11.33,19.33) node [anchor=north] [inner sep=0.75pt]   [align=left] {+};
        % Text Node
        \draw (55.83,19.33) node [anchor=north] [inner sep=0.75pt]   [align=left] {+};
        % Text Node
        \draw (100.33,19.33) node [anchor=north] [inner sep=0.75pt]   [align=left] {{-}};
    \end{tikzpicture}
\end{equation}
and we can see that the diagrams for $G^<_{\text{eg}}$, when the external driving field is turned off, 
contains ``ordinary'' time-ordered and anti-time-ordered Green functions, 
and after the summation, the difference between the energies of the e state and the g state 
enters the self-energy of $G^{<}_{\text{eg}}$.
We can do the same thing for $G^<_{\text{ee}}$ or $G^<_{\text{gg}}$ 
but this time we just get 
\begin{equation}
    \tikzset{every picture/.style={line width=0.75pt}} %set default line width to 0.75pt        
    \begin{tikzpicture}[x=0.75pt,y=0.75pt,yscale=-1,xscale=1, baseline=(XXXX.south) ]
    \path (0,43);\path (115.66667175292969,0);\draw    ($(current bounding box.center)+(0,0.3em)$) node [anchor=south] (XXXX) {};
    %Straight Lines [id:da6227673593287701] 
    \draw [line width=0.75]    (11.33,17.83) -- (55.83,17.83) ;
    \draw [shift={(36.18,17.83)}, rotate = 180] [fill={rgb, 255:red, 0; green, 0; blue, 0 }  ][line width=0.08]  [draw opacity=0] (7.14,-3.43) -- (0,0) -- (7.14,3.43) -- (4.74,0) -- cycle    ;
    %Straight Lines [id:da25170960008624355] 
    \draw [line width=0.75]    (55.83,17.83) -- (100.33,17.83) ;
    \draw [shift={(80.68,17.83)}, rotate = 180] [fill={rgb, 255:red, 0; green, 0; blue, 0 }  ][line width=0.08]  [draw opacity=0] (7.14,-3.43) -- (0,0) -- (7.14,3.43) -- (4.74,0) -- cycle    ;
    % Text Node
    \draw (11.33,16.33) node [anchor=south] [inner sep=0.75pt]   [align=left] {e};
    % Text Node
    \draw (100.33,16.33) node [anchor=south] [inner sep=0.75pt]   [align=left] {e};
    % Text Node
    \draw (55.83,16.33) node [anchor=south] [inner sep=0.75pt]   [align=left] {e};
    % Text Node
    \draw (11.33,19.33) node [anchor=north] [inner sep=0.75pt]   [align=left] {+};
    % Text Node
    \draw (55.83,19.33) node [anchor=north] [inner sep=0.75pt]   [align=left] {\mbox{-}};
    % Text Node
    \draw (100.33,19.33) node [anchor=north] [inner sep=0.75pt]   [align=left] {\mbox{-}};
    \end{tikzpicture}
    ,\quad \tikzset{every picture/.style={line width=0.75pt}} %set default line width to 0.75pt        
    \begin{tikzpicture}[x=0.75pt,y=0.75pt,yscale=-1,xscale=1, baseline=(XXXX.south) ]
    \path (0,43);\path (115.66667175292969,0);\draw    ($(current bounding box.center)+(0,0.3em)$) node [anchor=south] (XXXX) {};
    %Straight Lines [id:da13754447014645366] 
    \draw [line width=0.75]    (11.33,17.83) -- (55.83,17.83) ;
    \draw [shift={(36.18,17.83)}, rotate = 180] [fill={rgb, 255:red, 0; green, 0; blue, 0 }  ][line width=0.08]  [draw opacity=0] (7.14,-3.43) -- (0,0) -- (7.14,3.43) -- (4.74,0) -- cycle    ;
    %Straight Lines [id:da10487365181821628] 
    \draw [line width=0.75]    (55.83,17.83) -- (100.33,17.83) ;
    \draw [shift={(80.68,17.83)}, rotate = 180] [fill={rgb, 255:red, 0; green, 0; blue, 0 }  ][line width=0.08]  [draw opacity=0] (7.14,-3.43) -- (0,0) -- (7.14,3.43) -- (4.74,0) -- cycle    ;
    % Text Node
    \draw (11.33,16.33) node [anchor=south] [inner sep=0.75pt]   [align=left] {e};
    % Text Node
    \draw (100.33,16.33) node [anchor=south] [inner sep=0.75pt]   [align=left] {e};
    % Text Node
    \draw (55.83,16.33) node [anchor=south] [inner sep=0.75pt]   [align=left] {e};
    % Text Node
    \draw (11.33,19.33) node [anchor=north] [inner sep=0.75pt]   [align=left] {+};
    % Text Node
    \draw (55.83,19.33) node [anchor=north] [inner sep=0.75pt]   [align=left] {+};
    % Text Node
    \draw (100.33,19.33) node [anchor=north] [inner sep=0.75pt]   [align=left] {\mbox{-}};
    \end{tikzpicture},
\end{equation}
and we find that the contributions of time-ordered and anti-time-ordered Green functions
just cancel each other.
It should also be noted that $\Delta$ itself contains corrections from external fields
and is the energy gap minus the external phonon frequency.

Now we turn to how to obtain rate equations -- 
equations where the transitions between e and g are modeled as ``scattering'' -- 
from the aforementioned optical Bloch equation, 
or more generally, quantum master equations.
The idea is just to first replace $G^<_{\text{eg}}$ by diagrams in terms of $G^<_{\text{ee}, \text{gg}}$,
and then ignore any frequency dependence in the internal components
(similar to COHSEX).
It's kind of similar to how we use generalized plasmon-pole model to capture the structure of the RPA dielectric function.
Note that until now the theory we have is still a pure-state theory:
we just happen to be able to track the dynamics of the system 
by just looking at the diagonal parts of the single-atom reduced density matrix $G^<_{\text{ee,gg}} / \ii$.
It's however often further assumed that that's \emph{all we need},
that the non-diagonal components of $G^<$ are not important at all:
by taking this step further, we have implicitly assumed that there are strong dephasing factors in the system
that keep ``observing'' the atom into either $\ket{\text{g}}$ or $\ket{\text{e}}$,
but forbidding long existence of states like $(\ket{\text{e}} + \ket{\text{g}}) / \sqrt{2}$.
Of course, the dephasing factors shouldn't be too strong,
or otherwise whenever optical pumping drives the system from $\ketg$ to $\kete$, 
$\rho_{\text{eg}}$ is almost immediately driven back to zero
because of the strong dephasing effect.
The consequence is that it's impossible for optical pumping to actually happen;
in the same manner, if the atom is already at $\kete$, it doesn't go down.
This is known as quantum Zeno effect,
where intense observations freeze the system to the state it's currently in.

\end{document}