\documentclass[hyperref, a4paper]{article}

\usepackage{geometry}
\usepackage{titling}
\usepackage{titlesec}
% No longer needed, since we will use enumitem package
% \usepackage{paralist}
\usepackage{enumitem}
\usepackage{footnote}
\usepackage[colorinlistoftodos]{todonotes}
\usepackage{amsmath, amssymb, amsthm}
\usepackage{mathtools}
\usepackage{bbm}
\usepackage{graphicx}
\usepackage{subcaption}
\usepackage{soulutf8}
\usepackage{physics}
\usepackage{tensor}
\usepackage{siunitx}
\usepackage[version=4]{mhchem}
\usepackage{tikz}
\usepackage{xcolor}
\usepackage{listings}
\usepackage{autobreak}
\usepackage[ruled, vlined, linesnumbered]{algorithm2e}
\usepackage{nameref,zref-xr}
\zxrsetup{toltxlabel}
\usepackage[backend=bibtex]{biblatex}
\addbibresource{elasticity.bib}
\usepackage[colorlinks,unicode]{hyperref} % , linkcolor=black, anchorcolor=black, citecolor=black, urlcolor=black, filecolor=black
\usepackage[most]{tcolorbox}
\usepackage{prettyref}

% Page style
\geometry{left=3.18cm,right=3.18cm,top=2.54cm,bottom=2.54cm}
\titlespacing{\paragraph}{0pt}{1pt}{10pt}[20pt]
\setlength{\droptitle}{-5em}

% More compact lists 
\setlist[itemize]{
    itemindent=17pt, 
    leftmargin=1pt,
    listparindent=\parindent,
    parsep=0pt,
}

% Math operators
\DeclareMathOperator{\timeorder}{\mathcal{T}}
\DeclareMathOperator{\diag}{diag}
\DeclareMathOperator{\legpoly}{P}
\DeclareMathOperator{\primevalue}{P}
\DeclareMathOperator{\sgn}{sgn}
\DeclareMathOperator{\res}{Res}
\newcommand*{\ii}{\mathrm{i}}
\newcommand*{\ee}{\mathrm{e}}
\newcommand*{\const}{\mathrm{const}}
\newcommand*{\suchthat}{\quad \text{s.t.} \quad}
\newcommand*{\argmin}{\arg\min}
\newcommand*{\argmax}{\arg\max}
\newcommand*{\normalorder}[1]{: #1 :}
\newcommand*{\pair}[1]{\langle #1 \rangle}
\newcommand*{\fd}[1]{\mathcal{D} #1}
\DeclareMathOperator{\bigO}{\mathcal{O}}

% TikZ setting
\usetikzlibrary{arrows,shapes,positioning}
\usetikzlibrary{arrows.meta}
\usetikzlibrary{decorations.markings}
\tikzstyle arrowstyle=[scale=1]
\tikzstyle directed=[postaction={decorate,decoration={markings,
    mark=at position .5 with {\arrow[arrowstyle]{stealth}}}}]
\tikzstyle ray=[directed, thick]
\tikzstyle dot=[anchor=base,fill,circle,inner sep=1pt]

% Algorithm setting
% Julia-style code
\SetKwIF{If}{ElseIf}{Else}{if}{}{elseif}{else}{end}
\SetKwFor{For}{for}{}{end}
\SetKwFor{While}{while}{}{end}
\SetKwProg{Function}{function}{}{end}
\SetArgSty{textnormal}

\newcommand*{\concept}[1]{{\textbf{#1}}}

% Embedded codes
\lstset{basicstyle=\ttfamily,
  showstringspaces=false,
  commentstyle=\color{gray},
  keywordstyle=\color{blue}
}

% Reference formatting
\newcommand*{\citesec}[1]{\S~{#1}}
\newcommand*{\citechap}[1]{chap.~{#1}}
\newcommand*{\citefig}[1]{Fig.~{#1}}
\newcommand*{\citetable}[1]{Table~{#1}}
\newcommand*{\citepage}[1]{pp.~{#1}}
\newrefformat{fig}{Fig.~\ref{#1}}
\newcommand*{\term}[1]{\textit{#1}}

% Color boxes
\tcbuselibrary{skins, breakable, theorems}

\newtcbtheorem{infobox}{Box}{
    enhanced,
    boxrule=0pt,
    colback=blue!5,
    colframe=blue!5,
    coltitle=blue!50,
    borderline west={4pt}{0pt}{blue!65},
    sharp corners,
    fonttitle=\bfseries, 
    breakable,
    before upper={\parindent15pt\noindent}}{box}
\newtcbtheorem[use counter from=infobox]{theorybox}{Box}{
    enhanced,
    boxrule=0pt,
    colback=orange!5, 
    colframe=orange!5, 
    coltitle=orange!50,
    borderline west={4pt}{0pt}{orange!65},
    sharp corners,
    fonttitle=\bfseries, 
    breakable,
    before upper={\parindent15pt\noindent}}{box}
\newtcbtheorem[use counter from=infobox]{learnbox}{Box}{
    enhanced,
    boxrule=0pt,
    colback=green!5,
    colframe=green!5,
    coltitle=green!50,
    borderline west={4pt}{0pt}{green!65},
    sharp corners,
    fonttitle=\bfseries, 
    breakable,
    before upper={\parindent15pt\noindent}}{box}


\newenvironment{shelldisplay}{\begin{lstlisting}}{\end{lstlisting}}

\newcommand*{\kB}{k_{\text{B}}}
\newcommand*{\muB}{\mu_{\text{B}}}
\newcommand*{\efermi}{E_{\text{F}}}
\newcommand*{\pfermi}{p_{\text{F}}}
\newcommand*{\vfermi}{v_{\text{F}}}
\newcommand*{\sA}{\text{A}}
\newcommand*{\sB}{\text{B}}
\newcommand*{\Tc}{T_{\text{c}}}
\newcommand*{\hethree}{$^3$He}
\newcommand*{\hefour}{$^4$He}
\newcommand{\epsr}{\epsilon_{\text{r}}}
\newcommand{\chie}{\chi_{\text{e}}}

\title{Light-matter interaction}
\author{Jinyuan Wu}

\begin{document}

\maketitle

TODO:
\begin{itemize}
    \item There are rare cases where the energy velocity 
    (i.e. the velocity that connects $\vb*{s}$ and $u$) 
    is not the same as group velocity;
    for example in metal wave guide?
    \item From the current response of an external field to $\epsr$.
    (That's to say, 
    from TD-aGW to dielectric function.)
    \item Specifically, noise exists when there is damping. 
    Could this be modeled by some sort of non-equilibrium field theory?
    (I think I asked this question before \dots)
\end{itemize}

\section{Coarse-grained description of medium}

A linearly polarized medium can be described by 
\begin{equation}
    \vb*{D} = \epsilon_0 \vb*{E} + \vb*{P}
    = \epsilon_0 \vb*{E} + \epsilon_0 \chi_{\text{e}} \vb*{E} 
    = \epsilon_0 \underbrace{(1 + \chi_\text{e})}_{\eqqcolon \epsilon_{\text{r}}} \vb*{E}.
\end{equation}
Similarly we can deal with magnetization, 
although here the notation, due to historical reasons, needs to be altered slightly.
Now we have 
\begin{equation}
    \vb*{B} = \mu_0 (\vb*{H} + \vb*{M})
    = \mu_0 \underbrace{(1 + \chi_{\text{m}})}_{\mu_{\text{r}}} \vb*{H},
\end{equation}
as if $\vb*{B}$ has the same status of $\vb*{D}$,
although it's $\vb*{H}$ that is the auxiliary field.

From the fourth Maxwell equation we directly find
that time oscillation of polarization stimulates a magnetic field, 
which is correct since $\partial_t \vb*{P}$
can be understood as a current: 
charges have to be rearranged 
so that the orientation of a dipole changes,
which then give us a current.

The Poynting's theorem now reads 
\begin{equation}
    \div{\vb*{S}} + \vb*{H} \cdot \pdv{\vb*{B}}{t} + \vb*{E} \cdot \pdv{\vb*{D}}{t} 
    + \vb*{J}_{\text{free}} \cdot \vb*{E} = 0.
\end{equation}
This is the conservation equation of energy 
only when the material is linear, 
where we can redefine the electromagnetic energy as 
\begin{equation}
    u = \frac{1}{2} (\vb*{E} \cdot \vb*{D} + \vb*{B} \cdot \vb*{H}),
\end{equation} 
which includes the energy stored in the medium.
What we \emph{always} have is 
\begin{equation}
    \div{\vb*{S}} + \pdv{t} \left(
        \frac{\epsilon_0}{2} \vb*{E}^2 
        + \frac{\mu_0}{2} \vb*{H}^2
    \right) + 
    \left(
        \vb*{E} \cdot \pdv{\vb*{P}}{t} 
        + \mu_0 \vb*{H} \cdot \pdv{\vb*{M}}{t}
        + \vb*{J}_{\text{free}} \cdot \vb*{E}
    \right) = 0,
\end{equation} 
which is the conservation equation for the ``pure'' electromagnetic energy, 
e.g. the energy stored in the electromagnetic field.
Of course, the internal degrees of freedom of the medium 
may contain microscopic electromagnetic field modes, 
so by ``the energy stored in the electromagnetic field''
we mean ``the energy stored in the electromagnetic modes we care''.

\section{A more generalized description of linear response}

The $\vb*{D} = \epsilon_0 \epsilon_{\text{r}} \vb*{E}$ case is almost trivial: 
they respond instantaneously and they don't absorb energy.
In this section we still assume that the response is linear,
but don't assume that the response is spontaneous.
Thus 
\begin{equation}
    \vb*{P}(t) = \int_{-\infty}^{\infty} \alpha(t - \tau) \vb*{E}(\tau) \dd{\tau}.
\end{equation}
Thus the polarization field is the convolution 
of the response function and the electric field.
In principle things can be even more complicated: 
$\alpha$ can have momentum dependence in the Fourier space, 
and the consequence is that the response is non-local:
this is important in, say, plasmon, 
where long-range Coulomb interaction requires a non-local description of the response.

The wave equation now is 
\begin{equation}
    \curl{\curl{\vb*{E}}} = - \mu_0 \epsilon_0 \pdv[2]{t}
    \epsr \otimes \vb*{E},
\end{equation}
where $\otimes$ means convolution.
This equation is more clearly illustrated by its form in the frequency space: 
\begin{equation}
    \curl{\curl{\vb*{E}}} = \mu_0 \epsilon_0 \omega^2 \epsr(\omega) \vb*{E}(\omega),
    \label{eq:freq-space-time-retardation-wave-eq}
\end{equation}
where 
\begin{equation}
    \epsr(\omega) = \int_{-\infty}^\infty \epsr(t) \ee^{\ii \omega t} \dd{t},
\end{equation}
and from the fact that $\epsr(t)$ is real, 
\begin{equation}
    \epsr(-\omega) = \epsr(\omega)^*.
\end{equation}
For clarity we may want to use $\tilde{\epsilon}_{\text{r}}$ 
to refer to $\epsr(\omega)$.
The damping behavior of $\epsr(t)$ in the time domain 
is represented by the imaginary part of $\epsr(\omega)$ in the frequency domain.

\eqref{eq:freq-space-time-retardation-wave-eq} tells us an important point:
if the input is time harmonic, 
so is the output.
Indeed we can measure $\epsr(\omega)$ in this way.

The plane wave mode with a frequency-dependent $\epsr(\omega)$ 
has the following dispersive relation:
\begin{equation}
    \left(
        \vb*{k}^2 - \frac{\omega^2}{c^2} \epsr(\omega)
    \right) \vb*{E} = 0 \Rightarrow
    \vb*{k} = \pm \frac{\omega \tilde{n}(\omega)}{c} \vu*{k}, 
\end{equation}
where 
\begin{equation}
    \tilde{n}(\omega) = \sqrt{\epsr(\omega)}.
\end{equation}
From the dispersive relation 
we can define phase velocity and group velocity.

It should be noted that $n$ -- and hence $\vb*{k}$ -- 
is allowed to have an imaginary part,
which tells us absorption in the material. 
Defining 
\begin{equation}
    \tilde{n}(\omega) = n(\omega) + \ii \kappa(\omega),
\end{equation}
we find 
\begin{equation}
    \vb*{E}(\vb*{r}, t) = \ee^{
        - \frac{\omega}{c} \kappa(\omega) \vu*{k} \cdot \vb*{r}
    }
    \ee^{
        \ii \frac{\omega}{c} n(\omega) \vu*{k} \cdot \vb*{r}
        - \ii \omega t
    } E \vu*{e},
\end{equation}
and therefore the decaying coefficient is  
\begin{equation}
    \alpha(\omega) = \frac{\omega}{c} \kappa(\omega),  
\end{equation}
from which we find 
\begin{equation}
    \expval{S(z)} = \frac{1}{2} \abs*{
        \Re \vb*{E}^* \times \vb*{H}
    } \propto \ee^{- 2 \alpha z}.
\end{equation}

We can also evaluate the impact of imaginary part of $\epsr$ 
-- equivalently, of $\chi_{\text{e}}$ -- 
from the perspective of the energy of the electromagnetic field.
Consider a region with no energy flow into or out of it.
We have 
\begin{equation}
    \begin{aligned}
        &\div{\vb*{S}} + \pdv{u}{t} = - \vb*{E} \cdot \vb*{J} \\
        \Rightarrow& \dv{t} \int \expval{u} \dd[d]{\vb*{r}} 
        = - \int \dd[d]{\vb*{r}} \frac{1}{2} \Re \vb*{E}^* \cdot \vb*{J}   
        = - \frac{1}{2} \int \dd[d]{\vb*{r}} \omega \epsilon_0 \Im \chi(\omega) \abs*{\vb*{E}}^2.
    \end{aligned}
\end{equation}
Thus, when $\chi_2$ is positive, 
the field loses energy, 
and when $\chi_2$ is negative, 
the field gets energy.

\section{Microscopic model of dispersive media: the harmonic oscillator model}

The harmonic oscillator is often used as a simplistic model of the atom.
The physical picture seems wrong at the first glance, 
but if we regard the displacement in the harmonic oscillator
as the dipole of the atom, 
then things begin to make sense; 
indeed, the ``classical'' model of harmonic oscillator 
can be shockingly accurate in certain limits.

The EOM is  
\begin{equation}
    m \ddot{\vb*{r}} = - m \Omega^2 \vb*{r} - m \gamma \dot{\vb*{r}}
    + q \vb*{E},
\end{equation}
or in other words 
\begin{equation}
    \ddot{\vb*{p}} + \gamma \dot{\vb*{p}} + \Omega^2 \vb*{p}
    = \frac{q^2}{m} \vb*{E}(t),
\end{equation}
where $\vb*{p} = q \vb*{r}$ is the dipole.
\todo{Whether a more realistic atom model reduces to the above EOM 
requires further investigation.}%
The polarizability can be routinely found as 
\begin{equation}
    \vb*{p}(\omega) = \underbrace{
        \frac{q^2 / m}{\Omega^2 - \omega^2 - \ii \omega \gamma}
    }_{\alpha(\omega)} \vb*{E}(\omega).
\end{equation}
This leads to expected limit cases: 
when $\omega \ll \Omega$, 
$\vb*{E}$ is parallel to $\vb*{p}$,
while when $\omega \gg \Omega$,
$\vb*{E}$ is in the opposite direction to $\vb*{p}$.
\todo{How can this be used for light trapping???}

Now consider an assembly of atoms, 
and we find the total polarization field is now 
\begin{equation}
    \vb*{P} = \left(
        \frac{N}{V}
    \right) \cdot \vb*{p},
\end{equation}
and therefore the behavior of the media, 
in terms of quantities in the theory of electromagnetism, is now 
\begin{equation}
    (\ddot{\vb*{P}} + \gamma \dot{\vb*{P}} + \Omega^2 \vb*{P})
    = \frac{N}{V} \frac{q^2}{m} \vb*{E}.
\end{equation}
Recall that from Maxwell's equations we also have 
\begin{equation}
    \curl \curl \vb*{E} = 
    - \mu_0 \epsilon_0 \pdv[2]{t} \vb*{E}
    - \mu_0 \pdv[2]{t} \vb*{P}.
\end{equation}
The coupled EOMs, in the strong coupling regime, 
are a simple model of \concept{polariton}.
\todo{
    Is it possible to use solely $\epsr$ to capture the 
    behaviors of a polariton?
    Note that with $\epsr$ we still only have one mode 
    but here we actually have two modes, 
    EM field and polarization mode.
    
    Also, how could, say, phonon, be modeled as $\vb*{P}$.
}
\todo{the state of the material is changed after radiation happens 
(and the coupling between matter and the optical field 
also means the state of the material is a mixed state??
This question always puzzles me.)
Specifically, what will Fermi golden rule give us in this exactly solvable case?}

Real materials have more than one internal modes, 
and the relation between the polarizability and the electric field becomes 
\todo{TRK sum rule}

\section{Scattering cross sections}

For a single oscillator, the absorption power is  
\begin{equation}
    \expval{P_{\text{abs}}} = \expval{q \vb*{E} \cdot \dot{\vb*{d}}}
    = \gamma \abs*{\vb*{E}}^2 
    = \sigma_{\text{abs}} \underbrace{
        u_{\text{em}} v_{\text{g}}
    }_{\text{incident intensity}} .
\end{equation}
From this equation we naturally find a 
constant measuring how strong the absorption is 
with area dimension,
which is righteously named as the absorption cross section.
From this linear relation between absorption 
and incident intensity we immediate get Beer's law 
\begin{equation}
    I(z) = I(0) \ee^{- \frac{N}{V} \sigma_{\text{abs}} z},
\end{equation} 
where $N$ is the number of absorption centers.
The equation can be derived straightforwardly by 
using the definition to find 
\[
    \Delta P = - I(z) N \sigma_{\text{abs}}
\]
and noticing that 
\[
    P = I \cdot A.
\]

\section{Local field correction}

The good old ``a hole in a material'' argument.

\section{Kramers-Kronig relations}

It's possible that we have dispersion but no loss 
(as in, say, a wave guide, 
caused by the so-called geometric dispersion); 
in this case K-K relation may fail altogether,
due to some quirky properties of the response function.

\section{Semiclassical field-atom coupling}

In this section we consider the coupling between 
a semiclassical field and a degree of freedom 
that is expected to represent an atom,
be it a two-level system or a quantum oscillator or something else.
The coupling Hamiltonian reads 
\begin{equation}
    H_1 = - \vb*{\mu} \cdot \vb*{E} = - q \vb*{r} \cdot \vb*{E},
\end{equation}
which means we ignore the space dependence of $\vb*{E}$
and just apply a (possibly time-dependent) uniform electric field to 
the atom.
For example, the EOM of a harmonic oscillator is now 
\begin{equation}
    \dot{p} = - m \omega_0^2 x + q E, \quad 
    \dot{x} = \frac{p}{m},
\end{equation}
which means the toy model of driven harmonic oscillator 
is in fact quantitatively correct.

One thing that can be immediately noticed is that 
the dipole moment expectation is always constant 
when the system is at a stationary state, 
since the $\ee^{- \ii \omega t}$ factors
of the bra and the ket cancel each other.
This means if we treat the light field as a classical field, 
we don't have spontaneous emission,
since radiation requires a time evolving dipole moment.
If 
\begin{equation}
    \ket*{\psi} = c_a \ket*{a} + c_b \ket*{b},
\end{equation}
we get 
\begin{equation}
    \expval{\vb*{\mu}(t)}_{\text{oscillating part}} = 
    c^*_a c_b \ee^{- \ii (\omega_b - \omega_a) t} \mel**{a}{\vb*{\mu}}{b} + \text{c.c.},
\end{equation}
and radiative coupling is possible when the matrix element is non-zero.
When a realistic atomic model is used,
this means that the parity of $a$ and $b$ should be different 
(and thus it's impossible to have transition between 1s and 2s),
and similarly we have selection rules for $m$ and $l$.
Note that we can still have multiple dipole-allowed transitions 
with the help of some intermediate states 
to connect two states that, say, have the same parity;
or we can make use of other transition channels
when the gradient of the electric field is truly large.\todo{Incoherent nonlinear processes}

The time-dependent perturbation usually can't be solved exactly.
Below we review time-dependent perturbation theory.
The Schrodinger equation is 
\begin{equation}
    \ii \hbar \pdv{t} \ket*{\psi(t)} = 
    (H_0 + \lambda H_1(t)) \ket*{\psi(t)},
\end{equation}
and we do the decomposition 
(by including the $\ee^{- \ii \omega_n t}$ factor 
we implicitly come into the interaction picture)
\begin{equation}
    \ket*{\psi(t)} = \sum_{n} \underbrace{  \left(
        \gamma_n^{(0)} + \lambda \gamma_n^{(1)} + \lambda^2 \gamma_n^{(2)} + \cdots
    \right) }_{\gamma_n} \ket*{n} \ee^{- \ii \omega_n t}, 
\end{equation}
and from the 
\begin{equation}
    \dv{\gamma_k}{t} = \frac{1}{\ii \hbar} \lambda \sum_n \mel**{k}{H_1}{n} \gamma_n (t) 
    \ee^{\ii (\omega_k - \omega_n) t}
\end{equation}
we get 
\begin{equation}
    \dv{t} \gamma^{(0)}_k = 0,
\end{equation}
\begin{equation}
    \dv{t} \gamma^{(1)}_k = \frac{1}{\ii \hbar} \sum_n H_{1, kn} 
    \gamma^{(0)}_n \ee^{\ii (\omega_k - \omega_n) t},
\end{equation}
\begin{equation}
    \dv{t} \gamma^{(2)}_k = \frac{1}{\ii \hbar} \sum_n H_{1, kn} 
    \gamma^{(1)}_n \ee^{\ii (\omega_k - \omega_n) t},
\end{equation}
and so on.  

We can organize the perturbed coefficients in the scattering matrix formalism:
\begin{equation}
    \gamma_k(t) = \gamma_k^{(0)} + \sum_n \underbrace{(S_{kn}^{(0)}(t) + S_{kn}^{(1)}(t) + \cdots)}_{S_{kn}} \gamma_n^{(0)},
\end{equation}
and the scattering matrix $S_{kn(t)}$ can be obtained from the 
aforementioned series of equations.

\end{document}