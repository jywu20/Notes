\documentclass[hyperref, a4paper]{article}

\usepackage{geometry}
\usepackage{titling}
\usepackage{titlesec}
% No longer needed, since we will use enumitem package
% \usepackage{paralist}
\usepackage{enumitem}
\usepackage{footnote}
\usepackage[colorinlistoftodos]{todonotes}
\usepackage{amsmath, amssymb, amsthm}
\usepackage{mathtools}
\usepackage{bbm}
\usepackage{graphicx}
\usepackage{subcaption}
\usepackage{soulutf8}
\usepackage{physics}
\usepackage{tensor}
\usepackage{siunitx}
\usepackage[version=4]{mhchem}
\usepackage{tikz}
\usepackage{xcolor}
\usepackage{listings}
\usepackage{autobreak}
\usepackage[ruled, vlined, linesnumbered]{algorithm2e}
\usepackage{nameref,zref-xr}
\zxrsetup{toltxlabel}
\usepackage[backend=bibtex]{biblatex}
\addbibresource{elasticity.bib}
\usepackage[colorlinks,unicode]{hyperref} % , linkcolor=black, anchorcolor=black, citecolor=black, urlcolor=black, filecolor=black
\usepackage[most]{tcolorbox}
\usepackage{prettyref}

% Page style
\geometry{left=3.18cm,right=3.18cm,top=2.54cm,bottom=2.54cm}
\titlespacing{\paragraph}{0pt}{1pt}{10pt}[20pt]
\setlength{\droptitle}{-5em}

% More compact lists 
\setlist[itemize]{
    itemindent=17pt, 
    leftmargin=1pt,
    listparindent=\parindent,
    parsep=0pt,
}

% Math operators
\DeclareMathOperator{\timeorder}{\mathcal{T}}
\DeclareMathOperator{\diag}{diag}
\DeclareMathOperator{\legpoly}{P}
\DeclareMathOperator{\primevalue}{P}
\DeclareMathOperator{\sgn}{sgn}
\DeclareMathOperator{\res}{Res}
\DeclareMathOperator{\sinc}{sinc}
\newcommand*{\ii}{\mathrm{i}}
\newcommand*{\ee}{\mathrm{e}}
\newcommand*{\const}{\mathrm{const}}
\newcommand*{\suchthat}{\quad \text{s.t.} \quad}
\newcommand*{\argmin}{\arg\min}
\newcommand*{\argmax}{\arg\max}
\newcommand*{\normalorder}[1]{: #1 :}
\newcommand*{\pair}[1]{\langle #1 \rangle}
\newcommand*{\fd}[1]{\mathcal{D} #1}
\DeclareMathOperator{\bigO}{\mathcal{O}}

% TikZ setting
\usetikzlibrary{arrows,shapes,positioning}
\usetikzlibrary{arrows.meta}
\usetikzlibrary{decorations.markings}
\usetikzlibrary{calc}
\tikzstyle arrowstyle=[scale=1]
\tikzstyle directed=[postaction={decorate,decoration={markings,
    mark=at position .5 with {\arrow[arrowstyle]{stealth}}}}]
\tikzstyle ray=[directed, thick]
\tikzstyle dot=[anchor=base,fill,circle,inner sep=1pt]

% Algorithm setting
% Julia-style code
\SetKwIF{If}{ElseIf}{Else}{if}{}{elseif}{else}{end}
\SetKwFor{For}{for}{}{end}
\SetKwFor{While}{while}{}{end}
\SetKwProg{Function}{function}{}{end}
\SetArgSty{textnormal}

\newcommand*{\concept}[1]{{\textbf{#1}}}

% Embedded codes
\lstset{basicstyle=\ttfamily,
  showstringspaces=false,
  commentstyle=\color{gray},
  keywordstyle=\color{blue}
}

% Reference formatting
\newcommand*{\citesec}[1]{\S~{#1}}
\newcommand*{\citechap}[1]{chap.~{#1}}
\newcommand*{\citefig}[1]{Fig.~{#1}}
\newcommand*{\citetable}[1]{Table~{#1}}
\newcommand*{\citepage}[1]{pp.~{#1}}
\newrefformat{fig}{Fig.~\ref{#1}}
\newcommand*{\term}[1]{\textit{#1}}

% Color boxes
\tcbuselibrary{skins, breakable, theorems}

\newtcbtheorem{infobox}{Box}{
    enhanced,
    boxrule=0pt,
    colback=blue!5,
    colframe=blue!5,
    coltitle=blue!50,
    borderline west={4pt}{0pt}{blue!65},
    sharp corners,
    fonttitle=\bfseries, 
    breakable,
    before upper={\parindent15pt\noindent}}{box}
\newtcbtheorem[use counter from=infobox]{theorybox}{Box}{
    enhanced,
    boxrule=0pt,
    colback=orange!5, 
    colframe=orange!5, 
    coltitle=orange!50,
    borderline west={4pt}{0pt}{orange!65},
    sharp corners,
    fonttitle=\bfseries, 
    breakable,
    before upper={\parindent15pt\noindent}}{box}
\newtcbtheorem[use counter from=infobox]{learnbox}{Box}{
    enhanced,
    boxrule=0pt,
    colback=green!5,
    colframe=green!5,
    coltitle=green!50,
    borderline west={4pt}{0pt}{green!65},
    sharp corners,
    fonttitle=\bfseries, 
    breakable,
    before upper={\parindent15pt\noindent}}{box}


\newenvironment{shelldisplay}{\begin{lstlisting}}{\end{lstlisting}}

\newcommand*{\kB}{k_{\text{B}}}
\newcommand*{\muB}{\mu_{\text{B}}}
\newcommand*{\efermi}{E_{\text{F}}}
\newcommand*{\pfermi}{p_{\text{F}}}
\newcommand*{\vfermi}{v_{\text{F}}}
\newcommand*{\sA}{\text{A}}
\newcommand*{\sB}{\text{B}}
\newcommand*{\Tc}{T_{\text{c}}}
\newcommand*{\hethree}{$^3$He}
\newcommand*{\hefour}{$^4$He}
\newcommand{\epsr}{\epsilon_{\text{r}}}
\newcommand{\chie}{\chi_{\text{e}}}
\newcommand*{\Gammae}{\Gamma_{\text{e}}}
\newcommand*{\Gammag}{\Gamma_{\text{g}}}
\newcommand*{\omegae}{\omega_{\text{e}}}
\newcommand*{\omegag}{\omega_{\text{g}}}
\newcommand*{\omegaeg}{\omega_{\text{eg}}}
\newcommand*{\ptwfc}[2]{\psi^{(#2)}_{#1}}
\newcommand*{\mueg}{\mu_{\text{eg}}}
\newcommand*{\muge}{\mu_{\text{ge}}}
\newcommand*{\Ezzero}{E_{z0}}
\newcommand*{\kete}{\ket*{\text{e}}}
\newcommand*{\ketg}{\ket*{\text{g}}}
\newcommand*{\coeffe}{c_{\text{e}}}
\newcommand*{\coeffg}{c_{\text{g}}}
\newcommand*{\pope}{p_{\text{e}}}
\newcommand*{\popg}{p_{\text{g}}}
\newcommand*{\ptwo}{P^{(2)}}
\newcommand*{\vp}{v_{\text{p}}}
\newcommand*{\chitwo}{\chi^{(2)}}
\newcommand*{\chithree}{\chi^{(3)}}
\newcommand*{\omegap}{\omega_{\text{p}}}

\title{Topics in optics and quantum electronics}
\author{Jinyuan Wu}

\begin{document}

\maketitle

\section{Mode description of electromagnetic fields}

In this section we mainly focus on the mode description of classical electrodynamics.
Modes are seen in resonators, waveguides, photonic crystals, and more.
Loss can then be included perturbatively and we get leaky modes.

In the case of electrodynamics in vacuum, in the Coulomb gauge (i.e. $\varphi = 0$), we have 
\begin{equation}
    \curl\curl{\vb*{A}} = \left(\frac{\omega}{c}\right)^2 \vb*{A}, 
\end{equation}
and under the inner product definition 
\begin{equation}
    \braket*{\vb*{A}}{\vb*{B}} \coloneqq \int \dd{\vb*{r}} \vb*{A}^* \cdot \vb*{B},
\end{equation}
the LHS of the equation on $\vb*{A}$ is Hermitian, guaranteeing that $\omega$ is real.

From the well-known relation $\comm*{a}{a^\dagger} = 1$, 
it follows that the classical counterparts of creation and annihilation operators 
are complex variables which satisfies the relation 
\begin{equation}
    \poissonbracket{a}{a^*} = - \ii.
    \label{eq:a-astar-relation}
\end{equation}
This means if the Poisson brackets are to be defined in terms of $a$ and $a^*$, it should be defined as 
\begin{equation}
    \poissonbracket{A}{B} = \frac{1}{\ii} \left(\pdv{A}{a} \pdv{B}{a^*} - \pdv{A}{a^*} \pdv{B}{a}\right),
    \label{eq:poisson-ab}
\end{equation}
where $a$ and $a^*$ are to be regarded as independent variables.
The general relation between $a, a^*$ and $x, p$ is 
\begin{equation}
    a = \alpha q + \ii \beta p, \quad 
    q = \frac{a + a^*}{2 \alpha}, \quad p = \frac{a - a^*}{2 \ii \beta}, \quad 2 \alpha \beta = 1.
    \label{eq:a-astar-x-p-relation}
\end{equation}
If $q$ and $p$ follow the canonical commutation relation,
then $a$ and $a^*$ follow the commutation relation between 
a creation operator and a annihilation operator, and vice versa. 

We can also justify \eqref{eq:a-astar-relation} by Hamiltonian's equations.
From \eqref{eq:a-astar-x-p-relation} and the cian rule, we find 
\begin{equation}
    \begin{aligned}
    \dot{a} &= \pdv{a}{q} \dot{q} + \pdv{a}{p} \dot{p}
    = \pdv{a}{q} \pdv{H}{p} - \pdv{a}{p} \pdv{H}{q} \\
    &= \pdv{a}{q} \left(\pdv{H}{a} \pdv{a}{p} + \pdv{H}{a^*} \pdv{a^*}{p}\right)
    - \pdv{a}{p}  \left(\pdv{H}{a} \pdv{a}{q} + \pdv{H}{a^*} \pdv{a^*}{q}\right) \\
    &= - 2 \ii \alpha \beta \pdv{H}{a^*},
    \end{aligned}
\end{equation}
and similarly 
\begin{equation}
    \dot{a}^* = 2 \ii \alpha \beta \pdv{H}{a}.
\end{equation}
Now that means we have to define the Poisson bracket as  \eqref{eq:poisson-ab}
to keep the form 
\begin{equation}
    \dv{A}{t} = \poissonbracket{A}{H}.
\end{equation}

The $2 \alpha \beta = 1$ constraint can be explained by conservation of area in phase space: 
roughly speaking the area inside the trajectory of $a(t)$ or $(q(t), p(t))$ 
is the ``strength of oscillation'', like the number of photons.

Now going back to the problem of electromagnetic fields.
Consider the one dimensional periodic boundary condition problem.
We have (in phaser form)
\begin{equation}
    \tilde{E}_\pm(x, t) = \tilde{E}_\pm^0 \ee^{\ii k x - \ii \omega t}, 
\end{equation}
and we consider $A$ to be the coordinate and hence $-E = \pdv*{A}{t}$ is the momentum.
To decompose $A$ into modes, we write $A$ as static, spatially varying factors weighted by time-dependent real values,
and thus $A = q(t) A_0 \ee^{\ii k z}$, 

\section{Perturbation theory}

Stationary electrodynamics in terms of $\vb*{E}$ is not a Hermitian problem, 
but a generalized Hermitian one:
the problem is 
\begin{equation}
    \curl{\curl{\vb*{E}}} = \left(\frac{\omega}{c}\right)^2 \epsr(\vb*{r}) \vb*{E}(\vb*{r}),
\end{equation}
and although the operators on both LHS and RHS are Hermitian, 
the LHS multiplied with the inverse of the RHS is not.
The time-independent perturbation theory for this problem is therefore slightly more complicated, 
and the ``energy'' being solved is 
\begin{equation}
    \alpha \coloneqq \frac{\omega^2}{c^2},
\end{equation}
instead of the frequency. 
\begin{equation}
    \Delta \omega^{(1)}_n = - \frac{\omega_n^{(0)}}{2} \mel*{\tilde{\vb*{E}}_n}{\epsilon_0 \Delta \epsr}{\tilde{\vb*{E}}_n}.
    \label{eq:first-order-delta-eps}
\end{equation}
The minus sign can be understood using the following line of argumentation:
if a piece of dielectric is inserted into an isolated system, 
the total energy is conserved but some energy is stored 
within the internal degree of freedom dielectric, 
so $\omega (a^\dagger a + 1/2)$, the vacuum part of the Hamiltonian has to decrease.
One important caveat is that the leading order of energy correction 
\emph{can't} be evaluated by $\mel*{\vb*{E}_n}{\Delta \epsr}{\vb*{E}_n}$:
the absolute value of this term is right, 
but the sign is wrong. TODO: why?

\section{Loss as perturbation}

Now we study the influence of $\epsilon_2$, the imaginary part of $\epsr$, as a perturbation.
We can just inject $\Delta \epsr = \ii \epsilon_2$ into \eqref{eq:first-order-delta-eps}, 
and we get 
\begin{equation}
    \Delta \omega_n^{(1)} = - \ii \frac{1}{\tau_n}, \quad 
    \frac{1}{\tau_n} = \frac{\omega_n}{2} \mel*{\tilde{\vb*{E}}_n}{\epsilon_0 \epsilon_2}{\tilde{\vb*{E}}_n}.
\end{equation}
This means the time evolution of the corresponding mode $a_n$ contains a $\ee^{- t / \tau_n}$ factor, 
and the EOM seems to be 
\begin{equation}
    \dot{a} = - \ii \omega a - \frac{1}{\tau} a.
    \label{eq:eom-a-0}
\end{equation}
We can add a driving field to the RHS of this equation of motion; 
without considering any nonlinear effects,%
\footnote{
    Damping always comes from coupling of the system with a much larger, external bath.
    The first order correction from the bath 
    adds a finite imaginary part to the self energy of the system; 
    the interaction with the bath of course also modifies the 
    coupling between the bath and the system, 
    but this is not first-ordered, 
    and can be ignored when damping is small.
}
the lifetime remains the same regardless of the possible external driving.
The question is whether \eqref{eq:eom-a-0} is truly the correct equation 
for a damped system without external driving.
It is of course possible to have a different kind of damping:
if we choose to not start from the imaginary part of $\epsr$ 
and instead insert a $- k v$ type of damping into the EOM of $p$ and $q$, 
then the damping term for $a$ should assumes the form of 
\begin{equation}
    \dot{a} + \ii \omega_0 a = - \frac{a}{\tau} + \frac{a^*}{\tau}.
    \label{eq:eom-a-0-coupled}
\end{equation}
Interestingly, from an equally intuitive starting point, 
now we get a set of coupled EOMs for $a$ and $a^*$, 
while in \eqref{eq:eom-a-0}, the EOMs for $a, a^*$ are not coupled.

The ``real'' form of the damping term depends on how the system couples to the bath; 
if we are studying a harmonic oscillator then \eqref{eq:eom-a-0-coupled} is exact, 
while \eqref{eq:eom-a-0} is not; 
the most natural way to include loss in Maxwell's equations also leads to 
something similar to \eqref{eq:eom-a-0-coupled}.
So indeed, although \eqref{eq:eom-a-0} seems quite natural, 
it's not exact, and the question is how it's related to the exact equations.

The answer is rotating wave approximation. 
From the usual procedure we will find the coupled terms in the loss are related to fast oscillation terms, 
which can be ignored for the low frequency part of $a, a^*$. 
This is valid as long as there is separation between the time scales of damping and of oscillation.
This is the condition of the validity of RWA of an external driving field, 
which, in the latter case, means the frequency of driving should be close to $\omega_0$
while the driving \emph{amplitude} should be small.
Since the damping term is not oscillating, 
the validity condition of RWA for damping is simpler.

TODO: does rotating wave approximation change the way the system couple to external fields?
(c.f. general relativity and the $f(R)$ theory)

\section{Mode coupling}

Consider a ring resonator. 
We have a right mover mode and a left mover mode, 
which we refer to as 
\begin{equation}
    \tilde{E}_k = a_k E_k^0 \ee^{\ii k z}
\end{equation}
and 
\begin{equation}
    \tilde{E}_{-k} = a_{-k} E_{-k}^0 \ee^{- \ii k z},
\end{equation}
respectively. 

Now suppose we have such a field configuration:
\begin{equation}
    E = a_k \tilde{E}_k + \text{c.c.} + a_{-k} \tilde{E}_{-k},
\end{equation}
and there is a $\Delta \epsr$ perturbation in the system.
The change of the frequency is therefore given by  
\begin{equation}
    \Delta \omega^{(1)} = - \frac{\omega^{(0)}}{2} \expval{\Delta \epsr}
    \propto (a_k^* + a_{-k}) (a_k^* + a_{-k}) + \text{c.c.},
\end{equation}
and therefore the contribution of $\Delta \epsr$ to the energy of the system looks like
\begin{equation}
    \Delta E = - \tilde{\gamma} (a_k^* + a_{-k}) (a_k^* + a_{-k}) + \text{c.c.} 
\end{equation}
The EOM of $a_k$ therefore is (TODO: formalism)
\begin{equation}
    \dot{a}_k = - \ii \omega_k a_k + 2 \ii \gamma (a_k^* + a_{-k}).
\end{equation}
Again we can apply RWA and remove the ``fast'' $a_k^*$ term, 
so eventually the EOMs of the two modes are (TODO: RWA without external driving)
\begin{equation}
    \dot{a}_k = - \ii \omega_0 a_k + \ii \tilde{g} a_{-k}, \quad 
    \dot{a}_{-k} = - \ii \omega_0 a_{-k} + \ii \tilde{g}^* a_{k}, \quad 
    \tilde{g} = 2 \gamma.
\end{equation}
The EOMs are still within the Hamiltonian dynamics framework, 
so we can diagonalize them and find that when 
\begin{equation}
    \Delta \epsr = \Delta \epsilon \cos(2kz),
\end{equation}
the new mode 
$a_{k} + a_{-k}$ has a lower frequency $\omega_0 - \abs*{\tilde{g}}$
while the new mode $a_{k} - a_{-k}$ has a higher frequency $\omega_0 + \abs*{\tilde{g}}$.
But the above formalism is not restricted to this perturbation: 
whatever $\Delta \epsr$ is, the way it affects the two modes we consider here 
is by influencing $\tilde{g}$.
Specifically, $\abs*{\tilde{g}}$ can be used to estimate the roughness of a ring resonator.

Of course, we can also perturb the system by modulation in \emph{time}.
Consider a classic $LC$-circuit, for example:
it cam be verified that an oscillating \(C\) results in 
\begin{equation}
    H = \omega_0 a^* a + \frac{\gamma(t)}{2} (a + a^*)^2,
\end{equation}
because there is no \(L^2\) term that cancels the \(a a^*\) terms.
The resulting EOMs are 
\begin{equation}
    \dot{a} = - \ii \omega_0 a - \ii \gamma(t) (a + a^*), 
\end{equation}
and the perturbation $\gamma(t)$ usually takes the form like 
\begin{equation}
    \gamma(t) = \frac{\tilde{\gamma}_0}{2} \ee^{- \ii \Omega t} + \text{c.c.}
\end{equation}
where $\Omega = 2 \omega_0$.
After RWA the only term that survives is 
\begin{equation}
    \dot{a} = - \ii \omega_0 a - \ii \frac{\tilde{\gamma}_0}{2} \ee^{- \ii \Omega t} a^*
\end{equation}
and hence we find 
\begin{equation}
    \dv{t} \bmqty{a \ee^{- \ii \omega_0 t} \\ a^* \ee^{\ii \omega t}} = \bmqty{
        0 & - \ii \frac{\tilde{\gamma}_0}{2} \\
        \ii \frac{\tilde{\gamma}_0^*}{2} & 0 
    } \bmqty{a \ee^{- \ii \omega_0 t} \\ a^* \ee^{\ii \omega t}}
\end{equation}
We can easily see that the ``source'' of the $\ee^{- \ii \Omega t}$ term is $a^*$, 
which in turn comes from the $\ee^{\ii \Omega t}$ term and $a$.
This eventually gives us a feedback (usually \emph{positive} but not always -- see below) to $a$,
leading to the increase of the total energy.
The derivative of the ``free'' energy is 
\begin{equation}
    \dot{H}_{\text{SHO}} = \Re (- \ii \abs*{\tilde{\gamma}_0} \ee^{\ii \phi} a_0), 
\end{equation}
where $\phi$ is the phase of $\tilde{\gamma}_0$, 
and $a_0$ is the slow oscillating part of $a$.
This expression tells us two things.
First, the phase of the pumping matters: 
it can be chosen in a specific way to \emph{reduce} the total energy.
Second, the growth or damping of the total energy is exponential, 
and therefore if at first you don't have a finite amplitude, 
there is no growth at all.
To start the growth, noise is important; 
classically this comes from thermal fluctuation; 
in the quantum case, the above EOM involves some terms from non-trivial commutation relations 
and we may say the growth is started by the zero-point energy.

\end{document}