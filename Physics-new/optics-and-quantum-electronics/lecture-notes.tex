\documentclass[hyperref, a4paper]{article}

\usepackage{geometry}
\usepackage{titling}
\usepackage{titlesec}
% No longer needed, since we will use enumitem package
% \usepackage{paralist}
\usepackage{enumitem}
\usepackage{footnote}
\usepackage[colorinlistoftodos]{todonotes}
\usepackage{amsmath, amssymb, amsthm}
\usepackage{mathtools}
\usepackage{bbm}
\usepackage{graphicx}
\usepackage{subcaption}
\usepackage{soulutf8}
\usepackage{physics}
\usepackage{tensor}
\usepackage{siunitx}
\usepackage[version=4]{mhchem}
\usepackage{tikz}
\usepackage{xcolor}
\usepackage{listings}
\usepackage{autobreak}
\usepackage[ruled, vlined, linesnumbered]{algorithm2e}
\usepackage{nameref,zref-xr}
\zxrsetup{toltxlabel}
\usepackage[backend=bibtex]{biblatex}
\addbibresource{elasticity.bib}
\usepackage[colorlinks,unicode]{hyperref} % , linkcolor=black, anchorcolor=black, citecolor=black, urlcolor=black, filecolor=black
\usepackage[most]{tcolorbox}
\usepackage{prettyref}

% Page style
\geometry{left=3.18cm,right=3.18cm,top=2.54cm,bottom=2.54cm}
\titlespacing{\paragraph}{0pt}{1pt}{10pt}[20pt]
\setlength{\droptitle}{-5em}

% More compact lists 
\setlist[itemize]{
    itemindent=17pt, 
    leftmargin=1pt,
    listparindent=\parindent,
    parsep=0pt,
}

% Math operators
\DeclareMathOperator{\timeorder}{\mathcal{T}}
\DeclareMathOperator{\diag}{diag}
\DeclareMathOperator{\legpoly}{P}
\DeclareMathOperator{\primevalue}{P}
\DeclareMathOperator{\sgn}{sgn}
\DeclareMathOperator{\res}{Res}
\DeclareMathOperator{\sinc}{sinc}
\newcommand*{\ii}{\mathrm{i}}
\newcommand*{\ee}{\mathrm{e}}
\newcommand*{\const}{\mathrm{const}}
\newcommand*{\suchthat}{\quad \text{s.t.} \quad}
\newcommand*{\argmin}{\arg\min}
\newcommand*{\argmax}{\arg\max}
\newcommand*{\normalorder}[1]{: #1 :}
\newcommand*{\pair}[1]{\langle #1 \rangle}
\newcommand*{\fd}[1]{\mathcal{D} #1}
\DeclareMathOperator{\bigO}{\mathcal{O}}

% TikZ setting
\usetikzlibrary{arrows,shapes,positioning}
\usetikzlibrary{arrows.meta}
\usetikzlibrary{decorations.markings}
\usetikzlibrary{calc}
\tikzstyle arrowstyle=[scale=1]
\tikzstyle directed=[postaction={decorate,decoration={markings,
    mark=at position .5 with {\arrow[arrowstyle]{stealth}}}}]
\tikzstyle ray=[directed, thick]
\tikzstyle dot=[anchor=base,fill,circle,inner sep=1pt]

% Algorithm setting
% Julia-style code
\SetKwIF{If}{ElseIf}{Else}{if}{}{elseif}{else}{end}
\SetKwFor{For}{for}{}{end}
\SetKwFor{While}{while}{}{end}
\SetKwProg{Function}{function}{}{end}
\SetArgSty{textnormal}

\newcommand*{\concept}[1]{{\textbf{#1}}}

% Embedded codes
\lstset{basicstyle=\ttfamily,
  showstringspaces=false,
  commentstyle=\color{gray},
  keywordstyle=\color{blue}
}

% Reference formatting
\newcommand*{\citesec}[1]{\S~{#1}}
\newcommand*{\citechap}[1]{chap.~{#1}}
\newcommand*{\citefig}[1]{Fig.~{#1}}
\newcommand*{\citetable}[1]{Table~{#1}}
\newcommand*{\citepage}[1]{pp.~{#1}}
\newrefformat{fig}{Fig.~\ref{#1}}
\newcommand*{\term}[1]{\textit{#1}}

% Color boxes
\tcbuselibrary{skins, breakable, theorems}

\newtcbtheorem{infobox}{Box}{
    enhanced,
    boxrule=0pt,
    colback=blue!5,
    colframe=blue!5,
    coltitle=blue!50,
    borderline west={4pt}{0pt}{blue!65},
    sharp corners,
    fonttitle=\bfseries, 
    breakable,
    before upper={\parindent15pt\noindent}}{box}
\newtcbtheorem[use counter from=infobox]{theorybox}{Box}{
    enhanced,
    boxrule=0pt,
    colback=orange!5, 
    colframe=orange!5, 
    coltitle=orange!50,
    borderline west={4pt}{0pt}{orange!65},
    sharp corners,
    fonttitle=\bfseries, 
    breakable,
    before upper={\parindent15pt\noindent}}{box}
\newtcbtheorem[use counter from=infobox]{learnbox}{Box}{
    enhanced,
    boxrule=0pt,
    colback=green!5,
    colframe=green!5,
    coltitle=green!50,
    borderline west={4pt}{0pt}{green!65},
    sharp corners,
    fonttitle=\bfseries, 
    breakable,
    before upper={\parindent15pt\noindent}}{box}


\newenvironment{shelldisplay}{\begin{lstlisting}}{\end{lstlisting}}

\newcommand*{\kB}{k_{\text{B}}}
\newcommand*{\muB}{\mu_{\text{B}}}
\newcommand*{\efermi}{E_{\text{F}}}
\newcommand*{\pfermi}{p_{\text{F}}}
\newcommand*{\vfermi}{v_{\text{F}}}
\newcommand*{\sA}{\text{A}}
\newcommand*{\sB}{\text{B}}
\newcommand*{\Tc}{T_{\text{c}}}
\newcommand*{\hethree}{$^3$He}
\newcommand*{\hefour}{$^4$He}
\newcommand{\epsr}{\epsilon_{\text{r}}}
\newcommand{\chie}{\chi_{\text{e}}}
\newcommand*{\Gammae}{\Gamma_{\text{e}}}
\newcommand*{\Gammag}{\Gamma_{\text{g}}}
\newcommand*{\omegae}{\omega_{\text{e}}}
\newcommand*{\omegag}{\omega_{\text{g}}}
\newcommand*{\omegaeg}{\omega_{\text{eg}}}
\newcommand*{\ptwfc}[2]{\psi^{(#2)}_{#1}}
\newcommand*{\mueg}{\mu_{\text{eg}}}
\newcommand*{\muge}{\mu_{\text{ge}}}
\newcommand*{\Ezzero}{E_{z0}}
\newcommand*{\kete}{\ket*{\text{e}}}
\newcommand*{\ketg}{\ket*{\text{g}}}
\newcommand*{\coeffe}{c_{\text{e}}}
\newcommand*{\coeffg}{c_{\text{g}}}
\newcommand*{\pope}{p_{\text{e}}}
\newcommand*{\popg}{p_{\text{g}}}
\newcommand*{\ptwo}{P^{(2)}}
\newcommand*{\vp}{v_{\text{p}}}
\newcommand*{\chitwo}{\chi^{(2)}}
\newcommand*{\chithree}{\chi^{(3)}}
\newcommand*{\omegap}{\omega_{\text{p}}}

\title{Topics in optics and quantum electronics}
\author{Jinyuan Wu}

\begin{document}

\maketitle

\section{Mode description of electromagnetic fields}

In this section we mainly focus on the mode description of classical electrodynamics.
Modes are seen in resonators, waveguides, photonic crystals, and more.
Loss can then be included perturbatively and we get leaky modes.

In the case of electrodynamics in vacuum, in the Coulomb gauge (i.e. $\varphi = 0$), we have 
\begin{equation}
    \curl\curl{\vb*{A}} = \left(\frac{\omega}{c}\right)^2 \vb*{A}, 
\end{equation}
and under the inner product definition 
\begin{equation}
    \braket*{\vb*{A}}{\vb*{B}} \coloneqq \int \dd{\vb*{r}} \vb*{A}^* \cdot \vb*{B},
\end{equation}
the LHS of the equation on $\vb*{A}$ is Hermitian, guaranteeing that $\omega$ is real.

From the well-known relation $\comm*{a}{a^\dagger} = 1$, 
it follows that the classical counterparts of creation and annihilation operators 
are complex variables which satisfies the relation 
\begin{equation}
    \poissonbracket{a}{a^*} = - \ii.
    \label{eq:a-astar-relation}
\end{equation}
This means if the Poisson brackets are to be defined in terms of $a$ and $a^*$, it should be defined as 
\begin{equation}
    \poissonbracket{A}{B} = \frac{1}{\ii} \left(\pdv{A}{a} \pdv{B}{a^*} - \pdv{A}{a^*} \pdv{B}{a}\right),
    \label{eq:poisson-ab}
\end{equation}
where $a$ and $a^*$ are to be regarded as independent variables.
The general relation between $a, a^*$ and $x, p$ is 
\begin{equation}
    a = \alpha q + \ii \beta p, \quad 
    q = \frac{a + a^*}{2 \alpha}, \quad p = \frac{a - a^*}{2 \ii \beta}, \quad 2 \alpha \beta = 1.
    \label{eq:a-astar-x-p-relation}
\end{equation}
If $q$ and $p$ follow the canonical commutation relation,
then $a$ and $a^*$ follow the commutation relation between 
a creation operator and a annihilation operator, and vice versa. 

We can also justify \eqref{eq:a-astar-relation} by Hamiltonian's equations.
From \eqref{eq:a-astar-x-p-relation} and the cian rule, we find 
\begin{equation}
    \begin{aligned}
    \dot{a} &= \pdv{a}{q} \dot{q} + \pdv{a}{p} \dot{p}
    = \pdv{a}{q} \pdv{H}{p} - \pdv{a}{p} \pdv{H}{q} \\
    &= \pdv{a}{q} \left(\pdv{H}{a} \pdv{a}{p} + \pdv{H}{a^*} \pdv{a^*}{p}\right)
    - \pdv{a}{p}  \left(\pdv{H}{a} \pdv{a}{q} + \pdv{H}{a^*} \pdv{a^*}{q}\right) \\
    &= - 2 \ii \alpha \beta \pdv{H}{a^*},
    \end{aligned}
\end{equation}
and similarly 
\begin{equation}
    \dot{a}^* = 2 \ii \alpha \beta \pdv{H}{a}.
\end{equation}
Now that means we have to define the Poisson bracket as  \eqref{eq:poisson-ab}
to keep the form 
\begin{equation}
    \dv{A}{t} = \poissonbracket{A}{H}.
\end{equation}

The $2 \alpha \beta = 1$ constraint can be explained by conservation of area in phase space: 
roughly speaking the area inside the trajectory of $a(t)$ or $(q(t), p(t))$ 
is the ``strength of oscillation'', like the number of photons.

Now going back to the problem of electromagnetic fields.
Consider the one dimensional periodic boundary condition problem.
We have (in phaser form)
\begin{equation}
    \tilde{E}_\pm(x, t) = \tilde{E}_\pm^0 \ee^{\ii k x - \ii \omega t}, 
\end{equation}
and we consider $A$ to be the coordinate and hence $-E = \pdv*{A}{t}$ is the momentum.
To decompose $A$ into modes, we write $A$ as static, spatially varying factors weighted by time-dependent real values,
and thus $A = q(t) A_0 \ee^{\ii k z}$, 



\end{document}