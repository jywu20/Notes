\documentclass[hyperref, a4paper]{article}

\usepackage{geometry}
\usepackage{titling}
\usepackage{titlesec}
% No longer needed, since we will use enumitem package
% \usepackage{paralist}
\usepackage{enumitem}
\usepackage{footnote}
\usepackage[colorinlistoftodos]{todonotes}
\usepackage{amsmath, amssymb, amsthm}
\usepackage{mathtools}
\usepackage{bbm}
\usepackage{graphicx}
\usepackage{subcaption}
\usepackage{soulutf8}
\usepackage{physics}
\usepackage{tensor}
\usepackage{siunitx}
\usepackage[version=4]{mhchem}
\usepackage{tikz}
\usepackage{xcolor}
\usepackage{listings}
\usepackage{autobreak}
\usepackage[ruled, vlined, linesnumbered]{algorithm2e}
\usepackage{nameref,zref-xr}
\zxrsetup{toltxlabel}
\usepackage[backend=bibtex]{biblatex}
\addbibresource{jc.bib}
\usepackage[colorlinks,unicode]{hyperref} % , linkcolor=black, anchorcolor=black, citecolor=black, urlcolor=black, filecolor=black
\usepackage[most]{tcolorbox}
\usepackage{prettyref}

% Page style
\geometry{left=3.18cm,right=3.18cm,top=2.54cm,bottom=2.54cm}
\titlespacing{\paragraph}{0pt}{1pt}{10pt}[20pt]
\setlength{\droptitle}{-5em}

% More compact lists 
\setlist[itemize]{
    itemindent=17pt, 
    leftmargin=1pt,
    listparindent=\parindent,
    parsep=0pt,
}

% Math operators
\DeclareMathOperator{\timeorder}{\mathcal{T}}
\DeclareMathOperator{\diag}{diag}
\DeclareMathOperator{\legpoly}{P}
\DeclareMathOperator{\primevalue}{P}
\DeclareMathOperator{\sgn}{sgn}
\DeclareMathOperator{\res}{Res}
\DeclareMathOperator{\sinc}{sinc}
\newcommand*{\ii}{\mathrm{i}}
\newcommand*{\ee}{\mathrm{e}}
\newcommand*{\const}{\mathrm{const}}
\newcommand*{\suchthat}{\quad \text{s.t.} \quad}
\newcommand*{\argmin}{\arg\min}
\newcommand*{\argmax}{\arg\max}
\newcommand*{\normalorder}[1]{: #1 :}
\newcommand*{\pair}[1]{\langle #1 \rangle}
\newcommand*{\fd}[1]{\mathcal{D} #1}
\DeclareMathOperator{\bigO}{\mathcal{O}}

% TikZ setting
\usetikzlibrary{arrows,shapes,positioning}
\usetikzlibrary{arrows.meta}
\usetikzlibrary{decorations.markings}
\usetikzlibrary{calc}
\tikzstyle arrowstyle=[scale=1]
\tikzstyle directed=[postaction={decorate,decoration={markings,
    mark=at position .5 with {\arrow[arrowstyle]{stealth}}}}]
\tikzstyle ray=[directed, thick]
\tikzstyle dot=[anchor=base,fill,circle,inner sep=1pt]

% Algorithm setting
% Julia-style code
\SetKwIF{If}{ElseIf}{Else}{if}{}{elseif}{else}{end}
\SetKwFor{For}{for}{}{end}
\SetKwFor{While}{while}{}{end}
\SetKwProg{Function}{function}{}{end}
\SetArgSty{textnormal}

\newcommand*{\concept}[1]{{\textbf{#1}}}

% Embedded codes
\lstset{basicstyle=\ttfamily,
  showstringspaces=false,
  commentstyle=\color{gray},
  keywordstyle=\color{blue}
}

% Reference formatting
\newcommand*{\citesec}[1]{\S~{#1}}
\newcommand*{\citechap}[1]{chap.~{#1}}
\newcommand*{\citefig}[1]{Fig.~{#1}}
\newcommand*{\citetable}[1]{Table~{#1}}
\newcommand*{\citepage}[1]{pp.~{#1}}
\newrefformat{fig}{Fig.~\ref{#1}}
\newcommand*{\term}[1]{\textit{#1}}

% Color boxes
\tcbuselibrary{skins, breakable, theorems}

\newtcbtheorem{infobox}{Box}{
    enhanced,
    boxrule=0pt,
    colback=blue!5,
    colframe=blue!5,
    coltitle=blue!50,
    borderline west={4pt}{0pt}{blue!65},
    sharp corners,
    fonttitle=\bfseries, 
    breakable,
    before upper={\parindent15pt\noindent}}{box}
\newtcbtheorem[use counter from=infobox]{theorybox}{Box}{
    enhanced,
    boxrule=0pt,
    colback=orange!5, 
    colframe=orange!5, 
    coltitle=orange!50,
    borderline west={4pt}{0pt}{orange!65},
    sharp corners,
    fonttitle=\bfseries, 
    breakable,
    before upper={\parindent15pt\noindent}}{box}
\newtcbtheorem[use counter from=infobox]{learnbox}{Box}{
    enhanced,
    boxrule=0pt,
    colback=green!5,
    colframe=green!5,
    coltitle=green!50,
    borderline west={4pt}{0pt}{green!65},
    sharp corners,
    fonttitle=\bfseries, 
    breakable,
    before upper={\parindent15pt\noindent}}{box}


\newenvironment{shelldisplay}{\begin{lstlisting}}{\end{lstlisting}}

\newcommand*{\kB}{k_{\text{B}}}
\newcommand*{\muB}{\mu_{\text{B}}}
\newcommand*{\efermi}{E_{\text{F}}}
\newcommand*{\pfermi}{p_{\text{F}}}
\newcommand*{\vfermi}{v_{\text{F}}}
\newcommand*{\sA}{\text{A}}
\newcommand*{\sB}{\text{B}}
\newcommand*{\Tc}{T_{\text{c}}}
\newcommand*{\hethree}{$^3$He}
\newcommand*{\hefour}{$^4$He}
\newcommand{\epsr}{\epsilon_{\text{r}}}
\newcommand*{\mur}{\mu_{\text{r}}}
\newcommand{\chie}{\chi_{\text{e}}}
\newcommand*{\Gammae}{\Gamma_{\text{e}}}
\newcommand*{\Gammag}{\Gamma_{\text{g}}}
\newcommand*{\omegae}{\omega_{\text{e}}}
\newcommand*{\omegag}{\omega_{\text{g}}}
\newcommand*{\omegaeg}{\omega_{\text{eg}}}
\newcommand*{\ptwfc}[2]{\psi^{(#2)}_{#1}}
\newcommand*{\mueg}{\mu_{\text{eg}}}
\newcommand*{\muge}{\mu_{\text{ge}}}
\newcommand*{\Ezzero}{E_{z0}}
\newcommand*{\kete}{\ket*{\text{e}}}
\newcommand*{\ketg}{\ket*{\text{g}}}
\newcommand*{\dyade}{\dyad{\text{e}}}
\newcommand*{\dyadg}{\dyad{\text{g}}}
\newcommand*{\dyadeg}{\dyad{\text{e}}{\text{g}}}
\newcommand*{\dyadge}{\dyad{\text{g}}{\text{e}}}
\newcommand*{\coeffe}{c_{\text{e}}}
\newcommand*{\coeffg}{c_{\text{g}}}
\newcommand*{\pope}{p_{\text{e}}}
\newcommand*{\popg}{p_{\text{g}}}
\newcommand*{\ptwo}{P^{(2)}}
\newcommand*{\vp}{v_{\text{p}}}
\newcommand*{\chitwo}{\chi^{(2)}}
\newcommand*{\chithree}{\chi^{(3)}}
\newcommand*{\omegap}{\omega_{\text{p}}}
\newcommand*{\mvb}[1]{\tilde{\vb*{#1}}}
\newcommand*{\Si}[1]{S_{\text{i#1}}}
\newcommand*{\So}[1]{S_{\text{o#1}}}
\newcommand*{\taug}{\tau_{\text{g}}}
\newcommand*{\dieg}{\vb*{d}_{\text{eg}}} 
\newcommand*{\dige}{\vb*{d}_{\text{ge}}} 


\title{Some problems in cavity QED}
\author{Jinyuan Wu}

\begin{document}

\maketitle

\textit{Consider a two-level atom in a cavity, 
whose coupling with one single photon mode significantly exceeds its coupling with the rest of the photon modes.
Write down the most generic light-matter interaction Hamiltonian.}

The most general Hamiltonian reads 
\begin{equation}
    H = \hbar \omega_0 \dyade + H_{\text{EM}} \underbrace{- \vb*{d} \cdot \vb*{E}}_{H_{\text{dipole}}},
    \label{eq:whole-ham}
\end{equation} 
where we have assumed that at $\kete$ there is no dipole 
or otherwise the excited state of the atom is not optically stable.
Therefore the dipole has the form 
\begin{equation}
    \vb*{d} = \dieg \dyadeg + \text{h.c.}
\end{equation}
The electric field can be expanded into photon modes in the following way:
\begin{equation}
    \vb*{E} = \sum_k \sqrt{\frac{\hbar \omega_k}{2 \epsilon_0 V}} \vb*{f}_k a_k + \text{h.c.}, \quad 
    \frac{1}{V} \int \dd[3]{\vb*{r}} \vb*{f}_k^* \cdot \frac{\vb*{\epsilon}}{\epsilon_0} \cdot \vb*{f}_{k'} = \delta_{kk'}.
\end{equation}
Suppose only one mode $a$ has strong influences to the atom, we have 
\begin{equation}
    H = \hbar \omega_0 \dyade + \hbar \omega \left( a^\dag a + \frac{1}{2} \right) 
    - \sqrt{\frac{\hbar \omega_k}{2 \epsilon_0 V}} (\dieg \dyadeg + \text{h.c.}) \cdot (\vb*{f} a + \text{h.c.}) ,
\end{equation}
where $\omega$ is the frequency of mode $a$ and $\vb*{f}$ is its polarization direction.

\textit{Apply the rotating wave approximation (RWA). 
When does this approximation work?}

Switching to the interaction picture, we have 
\begin{equation}
    H_{\text{dipole}} = - \sqrt{\frac{\hbar \omega_k}{2 \epsilon_0 V}} 
    (\dieg \dyadeg \ee^{\ii \omega_0 t} + \text{h.c.}) \cdot (\vb*{f} a \ee^{- \ii \omega t} + \text{h.c.}) ,
\end{equation}
and assuming that $\omega_0$ is close to $\omega$, RWA eliminates the anti-resonant terms 
$\dyadeg a^\dag$ and its conjugate transpose, and we have 
\begin{equation}
    H_{\text{dipole}} \approx \underbrace{- \sqrt{\frac{\hbar \omega_k}{2 \epsilon_0 V}} \dieg \cdot \vb*{f} }_{\eqqcolon \hbar g}
    \dyadeg a \ee^{\ii (\omega_0 - \omega) t} + \text{h.c.}
\end{equation}
Going back to Schrodinger picture, we have 
\begin{equation}
    H^{\text{RWA}} = \hbar \omega_0 \dyade + \hbar \omega \left( a^\dag a + \frac{1}{2} \right) + \hbar g \dyadeg a + \hbar g^* \dyadge a^\dag. 
\end{equation}

In this Hamiltonian the electric field is no longer an external driving field,
and therefore the RWA applied here can't be understood as a specific case of Floquet theory.
The condition of RWA however is still 
\begin{equation}
    \abs*{\omega - \omega_0} \ll \omega_0 + \omega, \quad 
    g \ll \omega_0.
    \label{eq:rwa-condition}
\end{equation}
We can conceive the RWA applied here within the framework of quantum field theory:
the self-energy correction to, say, $\kete$, always takes a form like this:
\[
    G^{\text{int}}_{\text{e}} = \frac{1}{E - \omegae + \underbrace{\frac{A \abs*{g}^2}{E - \omega - \omegag} + \frac{B \abs*{g}^2}{E + \omega - \omegag}}_\Sigma},
\]
where $\Sigma =\tikzset{every picture/.style={line width=0.75pt}} %set default line width to 0.75pt        
\begin{tikzpicture}[x=0.75pt,y=0.75pt,yscale=-0.4,xscale=0.4, baseline=(XXXX.south) ]
\path (0,45);\path (100,0);\draw    ($(current bounding box.center)+(0,0.3em)$) node [anchor=south] (XXXX) {};
%Straight Lines [id:da011052389066785295] 
\draw    (11.33,36.33) -- (91.45,36.33) ;
%Curve Lines [id:da14053098666969843] 
\draw    (11.33,36.33) .. controls (10.94,33.7) and (11.88,32.14) .. (14.16,31.65) .. controls (16.4,31.32) and (17.3,30.06) .. (16.85,27.86) .. controls (16.74,25.33) and (17.81,24.04) .. (20.05,24.01) .. controls (22.47,23.89) and (23.73,22.64) .. (23.82,20.25) .. controls (23.79,18.07) and (24.97,17.1) .. (27.36,17.34) .. controls (29.63,17.77) and (31.13,16.78) .. (31.86,14.37) .. controls (32.51,12.13) and (33.92,11.42) .. (36.09,12.25) .. controls (38.37,13.14) and (39.96,12.57) .. (40.86,10.54) .. controls (41.96,8.54) and (43.58,8.19) .. (45.71,9.49) .. controls (47.61,10.96) and (49.24,10.83) .. (50.61,9.1) .. controls (52.5,7.48) and (54.13,7.57) .. (55.52,9.37) .. controls (56.98,11.29) and (58.61,11.6) .. (60.41,10.3) .. controls (62.7,9.24) and (64.31,9.77) .. (65.24,11.89) .. controls (66.24,14.15) and (67.81,14.9) .. (69.96,14.13) .. controls (72.25,13.54) and (73.64,14.42) .. (74.14,16.75) .. controls (74.47,19.07) and (75.82,20.12) .. (78.18,19.91) .. controls (80.37,19.64) and (81.53,20.74) .. (81.67,23.22) .. controls (81.67,25.67) and (82.78,26.92) .. (85.01,26.97) .. controls (87.28,27.16) and (88.34,28.56) .. (88.17,31.16) .. controls (87.66,33.4) and (88.54,34.76) .. (90.81,35.25) -- (91.45,36.33) ;
\end{tikzpicture}$ is the self-energy in which the photon propagator contains 
both the resonant and anti-resonant term, 
and $A, B$ are constants with respect to the number of photons in the system.%
\footnote{
    If there is no photon at all when the atom is in the $\kete$ state,
    the anti-resonant term vanishes.
    This can be verified by noticing that the $a$ term in $\dyadge (a + a^\dag) \ket*{\text{e}, n=0}$ vanishes,
    or by noticing that in the contour integral that evaluates $\Sigma = G_{\text{g}} \otimes G_{\text{photon}}$,
    the pole of the anti-resonant branch of $G_{\text{photon}}$ is below the real axis 
    and therefore has no contribution to the contour integral, 
    which can be solved by working with only poles above the real axis.
    The rotating wave approximation only matters for the multi-photon case.
    
    Also, it's more convenient to designate $G_{\text{photon}}$ to the Green function of the \emph{electric field},
    not the vector potential, and in this way $G_{\text{photon}}$ takes the form of $2 E / (E^2 - \omega^2 + \ii 0^+)$, not just $1/(E^2 - \omega^2 + \ii 0^+)$.
}
Clearly, for the anti-resonant term in $\Sigma$ to be unimportant, we need 
$\abs*{E - \omega - \omegag} \gg \abs*{E + \omega - \omegag}$ at the pole of $G^{\text{int}}_{\text{e}}$,
but we know nothing about the value of $E$:
to estimate it with $\omega_0 = \omegae - \omegag$,
we have to then assume that the order of magnitude of $g$ is controlled.
Hence we get \eqref{eq:rwa-condition}.

The fact that only resonant terms in \eqref{eq:whole-ham} explains 
why we can get away with only considering the coupling between the atom and one photon mode:
as long as all other photon modes are much less resonant with the atom,
this is acceptable.
It also explains why usually a cavity is needed for coupling an atom with one specific photon mode:
in the free space, photon modes are continuous, 
and it's impossible to make sure only one photon mode is resonant with the atom.

\textit{The Hamiltonian after RWA is known as the Jaynes-Cummings model. 
Find its eigenstates. Give a physical picture of the eigenstates.}

There is only coupling between $\ket*{\text{g}, n+1}$ and $\ket*{\text{e}, n}$.
In this subspace, the Jaynes-Cummings Hamiltonian reads 
\begin{equation}
    \begin{aligned}
        H^{\text{JC}} &= \hbar \omega \left( n + \frac{1}{2} \right) + \pmqty{
            \hbar \omega  & \hbar g^* \sqrt{n+1} \\
            \hbar g \sqrt{n+1} & \hbar \omega_0
        } \\
        &= \hbar \omega \left( n + \frac{1}{2} \right) + \hbar \omega_0 + \frac{1}{2} \hbar \delta + \hbar 
        \pmqty{
            \delta / 2 & \Omega_n / 2 \\
            \Omega_n / 2 & - \delta / 2
        },
    \end{aligned}
    \label{eq:jc-subspace}
\end{equation}
where we define 
\begin{equation}
    \delta = \omega - \omega_0, \quad \frac{\Omega_n}{2} = g \sqrt{n+1}.
    \label{eq:omega-delta-def}
\end{equation}
Here, without loss of generality, we assume $g$ is real;
if it's not we can always redefine $\kete$ to remove its phase factor.

The eigenvalues of \eqref{eq:jc-subspace} are 
\begin{equation}
    E^{\pm}_n = \hbar \omega \left( n + \frac{1}{2} \right) + \hbar \omega_0 + \frac{1}{2} \hbar \delta
    \pm \frac{\hbar}{2} \underbrace{\sqrt{\Omega_n^2 + \delta^2}}_{\eqqcolon R_n}.
\end{equation}
When $\Omega_n = 0$ or in other words $g=0$,
as we change $\delta$, we have energy level crossing between $\ket*{\text{e},n}$ and $\ket*{\text{g},n+1}$;
after adding the coupling $\Omega_n$ the energy level crossing vanishes;
this is expected since perturbation makes energy levels avoid crossing. 

The eigenstates are given by 
\begin{equation}
    \ket*{n+} = \frac{1}{\sqrt{\Omega_n^2 + (R_n - \delta)^2}} (\Omega_n \ket*{\text{g},n+1} + (R_n - \delta) \ket*{\text{e},n}),
\end{equation}
\begin{equation}
    \ket*{n-} = \frac{1}{\sqrt{\Omega_n^2 + (R_n - \delta)^2}} ((R_n - \delta) \ket*{\text{g},n+1} - \Omega_n \ket*{\text{e},n}) ,
\end{equation}
or, by defining 
\begin{equation}
    \sin \theta_n = \frac{\Omega_n}{\sqrt{\Omega_n^2 + (R_n - \delta)^2}},
\end{equation}
\begin{equation}
    \ket*{n+} = \pmqty{\sin \theta_n \\ \cos \theta_n}, \quad 
    \ket*{n-} = \pmqty{\cos \theta_n \\ - \sin \theta_n}.
\end{equation}

The physical picture these two dressed states is spontaneous emission plus re-absorption:
the fact that optical pumping can bring the atom from $\ketg$ to $\kete$ 
means we have a $\dyadeg a$ term in the interactive Hamiltonian,
and then because of the Hermitian nature of the Hamiltonian,
there has to be a $\dyadge a^\dag$ term in the interactive Hamiltonian,
which is missing in the naive semi-classical approach. 
Therefore if an atom is initial in $\kete$, 
a photon will be spontaneous emitted, and the atom falls to $\ketg$;
but unlike ``prototypical'' spontaneous emission,
the number of modes coupled to the atom is not large,
and the photon can't just ``go away'', 
so eventually it gets absorbed by the atom again and the atom jumps to $\kete$.%
\footnote{
    Spontaneous emission and exponential decay are two independent phenomena:
    the first is about the matrix structure of dipole interaction,
    and the second is about being coupled to infinite modes.
    Emission of a photon from a lossy cavity involves exponential decay 
    because the modes outside the cavity outnumber the modes inside,
    but it's not the same as the spontaneous emission of a photon by an atom 
    because it involves no dipole interaction.
}

\textit{Analyze what will happen to a state in which the atom is in the excited state  
and photons are in a coherent state.
For simplicity, assume the detuning is zero, that's to say,
assume that $\omega$ is the same as $\omega_0$.}

A coherent state takes the form of 
\begin{equation}
    \ket*{\alpha} = \ee^{- \frac{\abs*{\alpha}^2}{2}} \sum_{n=0}^{\infty} \frac{\alpha^n}{\sqrt{n!}} \ket*{n},
\end{equation}
and therefore the initial state takes the following form:
\begin{equation}
    \kete \otimes \ket*{\alpha} = \ee^{- \frac{\abs*{\alpha}^2}{2}} \sum_{n=0}^{\infty} \frac{\alpha^n}{\sqrt{n!}} \ket*{\text{e}, n}.
    \label{eq:collapse-initial}
\end{equation}
The time evolution operator within the $\{ \ket*{\text{g}, n+1}, \ket*{\text{e},n} \}$ subspace, 
when $\delta = \omega - \omega_0 = 0$, is 
\begin{equation}
    \begin{aligned}
        U &= \ee^{- \ii \omega \left( n + \frac{1}{2} \right) t - \ii \omega_0 t} \ee^{- \ii \frac{\Omega_n t}{2} \sigma^x } \\
        &=  \ee^{- \ii \omega \left( n + \frac{1}{2} \right) t - \ii \omega_0 t}
        \left( \cos \frac{\Omega_n t}{2} \sigma^0 - \ii \sin \frac{\Omega_n t}{2} \sigma^x \right),
    \end{aligned}
\end{equation}
and therefore the time evolution of \eqref{eq:collapse-initial} is 
\begin{equation}
    \ket*{\psi(t)} = \ee^{- \frac{\abs*{\alpha}^2}{2}} \sum_{n=0}^{\infty} \frac{\alpha^n}{\sqrt{n!}} 
    \ee^{- \ii \omega \left( n + \frac{1}{2} \right) t - \ii \omega_0 t}
    \left(  \cos \frac{\Omega_n t}{2} \ket*{\text{e}, n} - \ii \sin \frac{\Omega_n t}{2} \ket*{\text{g}, n+1} \right) .
\end{equation}
The expression is complicated, but if we focus only on the probability of the atom staying on $\kete$,
then we get a slightly simpler expression (without the annoying phase factor):
\begin{equation}
    \begin{aligned}
        \pope &= \ee^{- \abs*{\alpha}^2} \sum_{n=0}^{\infty} \frac{\abs*{\alpha}^{2n}}{n!} 
        \cos^2 \frac{\Omega_n t}{2} \\
        &= \ee^{- \abs*{\alpha}^2} \sum_{n=0}^{\infty} \frac{\abs*{\alpha}^{2n}}{n!} 
        \cdot \frac{1}{2} \left( 1 + \cos \Omega_n t \right) \\
        &= \frac{1}{2} + \frac{1}{2} \ee^{- \abs*{\alpha}^2} \sum_{n=0}^{\infty} \frac{\abs*{\alpha}^{2n}}{n!} \cos \Omega_n t.
    \end{aligned}
\end{equation}

\textit{Assuming $\abs*{\alpha} \gg 1$ and $g t \abs*{\alpha} \ll 1$, what happens to the probability of the atom staying at $\kete$? Why does it decay? Can a similar phenomenon happen if the photons are in a thermal state?}

Recall that the definition of $\Omega_n$ is \eqref{eq:omega-delta-def}
and $\Omega_n = 2 g \sqrt{n+1}$.
If we take the $\abs*{\alpha} \to \infty$ limit,
we will find that the weight factor $\abs*{\alpha}^{2n} / n!$ reaches its peak when $n = \abs*{\alpha}^2$.
Therefore we can expand $\Omega_n$ around $n = \abs*{\alpha}^2$.
We have 
\[
    \sqrt{n+1} = \sqrt{\abs*{\alpha}^2 + 1 + n - \abs*{\alpha}^2 }
    \approx \abs*{\alpha} \left( 1 + \frac{n - \abs*{\alpha}^2 }{2 \abs*{\alpha}^2} \right),
\]
where we have used the condition that $\abs{\alpha}$ is very large, 
and the assumption that the peak of  $\abs*{\alpha}^{2n} / n!$ is narrow enough 
to make sure that only first order expansion is needed.
Therefore 
\begin{equation}
    \begin{aligned}
        \pope &= \frac{1}{2} + \frac{1}{2} \ee^{- \abs*{\alpha}^2} \sum_{n=0}^{\infty} \frac{\abs*{\alpha}^{2n}}{n!} 
        \cos(
            2 g \abs*{\alpha} t + g t \frac{n - \abs*{\alpha}^2}{\abs*{\alpha}}
        ) \\
        &= \frac{1}{2} + \frac{\ee^{- \abs*{\alpha}^2}}{2} \sum_{n=0}^{\infty} \frac{\abs*{\alpha}^{2n}}{n!} 
        \left(\cos( 2 g \abs*{\alpha} t) \cos(
            g t \frac{n - \abs*{\alpha}^2 }{\abs*{\alpha}}
        ) - \sin( 2 g \abs*{\alpha} t) \sin(
            g t \frac{n - \abs*{\alpha}^2 }{\abs*{\alpha}}
        )\right).
    \end{aligned}
\end{equation}
Since sin is odd and $n - \abs{\alpha}^2$ can be positive or negative,
the second term largely vanishes after the summation over $n$.
Therefore we are left with 
\begin{equation}
    \begin{aligned}
        \pope = \frac{1}{2} + \frac{\ee^{- \abs*{\alpha}^2}}{2} 
        \cos( 2 g \abs*{\alpha} t)  \sum_{n=0}^{\infty} \frac{\abs*{\alpha}^{2n}}{n!} 
        \cos(
            g t \frac{n - \abs*{\alpha}^2 }{\abs*{\alpha}}
        ).
    \end{aligned}
\end{equation}
We have 
\[
    \begin{aligned}
        &\quad \sum_{n=0}^{\infty} \frac{\abs*{\alpha}^{2n}}{n!} 
        \cos(
            g t \frac{n - \abs*{\alpha}^2 }{\abs*{\alpha}}
        ) \\
        &= \frac{1}{2} \sum_{n=0}^{\infty} \frac{\abs*{\alpha}^{2n}}{n!} 
        (
            \ee^{\ii g t \frac{n - \abs*{\alpha}^2 }{\abs*{\alpha}}}
            + \ee^{-\ii g t \frac{n - \abs*{\alpha}^2 }{\abs*{\alpha}}}
        ) \\
        &= \frac{1}{2} \ee^{- \ii \abs{\alpha} g t} \ee^{\abs{\alpha}^2 \ee^{\ii g t / \abs{\alpha}}} + \text{c.c.}
    \end{aligned}
\]
Now we use the $g \abs{\alpha} t \ll 1$ assumption, and we have 
\begin{equation}
    \begin{aligned}
        \pope &= \frac{1}{2} + \frac{\ee^{- \abs*{\alpha}^2}}{4} 
        \cos( 2 g \abs*{\alpha} t) \ee^{- \ii \abs{\alpha} g t} \ee^{\abs{\alpha}^2 \ee^{\ii g t / \abs{\alpha}}} + \text{c.c.} \\
        &\approx  \frac{1}{2} + \frac{\ee^{- \abs*{\alpha}^2}}{4} 
        \cos( 2 g \abs*{\alpha} t) \ee^{- \ii \abs{\alpha} g t} \ee^{\abs{\alpha}^2 \left(1 + \ii g t / \abs{\alpha} - \frac{1}{2 \abs{\alpha}^2} g^2 t^2  \right) } + \text{c.c.} \\
        &= \frac{1}{2} + \frac{1}{2} \cos (2 g \abs{\alpha} t) \ee^{- \frac{1}{2} g^2 t^2}.
    \end{aligned}
\end{equation}
So it can be seen that when $\abs{\alpha} \gg 1$ and $g t \abs{\alpha} \ll 1$,
the probability of the atom residing at $\kete$ decays exponentially,
and the time scale of the decay is $1/g$.

The root cause of the decay is different $\ket*{\text{e}, n}$ states evolve at different rates,
so their phases have deconstructive interference;
this can also happen when the initial state of photons is thermal:
phase deconstructive interference can also happen there.

\textit{Will the probability of the atom staying at $\kete$ revive? Can this happen in a thermal state?}

Since the problem is deconstructive interference,
after a period $T$ such that 
\begin{equation}
    T (\Omega_{\abs*{\alpha}^2} - \Omega_{\abs*{\alpha}^2-1}) = 2\pi,
\end{equation}
or after $t=nT$, we should observe a revival of $\pope$.
This is not something that can appear in a thermalized electromagnetic field,
because the latter is truly incoherent all the time.
The root cause of the revivals is the quantized nature of photons (note the $-1$ term),
so if they are observed experimentally, 
it's evidence that the electromagnetic field is indeed quantized. 

\textit{}

\end{document}