\documentclass[hyperref, a4paper]{article}

\usepackage{geometry}
\usepackage{titling}
\usepackage{titlesec}
% No longer needed, since we will use enumitem package
% \usepackage{paralist}
\usepackage{enumitem}
\usepackage{footnote}
\usepackage[colorinlistoftodos]{todonotes}
\usepackage{amsmath, amssymb, amsthm}
\usepackage{mathtools}
\usepackage{bbm}
\usepackage{graphicx}
\usepackage{subcaption}
\usepackage{soulutf8}
\usepackage{physics}
\usepackage{tensor}
\usepackage{siunitx}
\usepackage[version=4]{mhchem}
\usepackage{tikz}
\usepackage{xcolor}
\usepackage{listings}
\usepackage{autobreak}
\usepackage[ruled, vlined, linesnumbered]{algorithm2e}
\usepackage{nameref,zref-xr}
\zxrsetup{toltxlabel}
\usepackage[backend=bibtex,sorting=none]{biblatex}
\addbibresource{jc.bib}
\addbibresource{qed.bib}
\addbibresource{material.bib}
\usepackage[colorlinks,unicode]{hyperref} % , linkcolor=black, anchorcolor=black, citecolor=black, urlcolor=black, filecolor=black
\usepackage[most]{tcolorbox}
\usepackage{prettyref}

% Page style
\geometry{left=3.18cm,right=3.18cm,top=2.54cm,bottom=2.54cm}
\titlespacing{\paragraph}{0pt}{1pt}{10pt}[20pt]
\setlength{\droptitle}{-5em}

% More compact lists 
\setlist[itemize]{
    itemindent=17pt, 
    leftmargin=1pt,
    listparindent=\parindent,
    parsep=0pt,
}

% Math operators
\DeclareMathOperator{\timeorder}{\mathcal{T}}
\DeclareMathOperator{\diag}{diag}
\DeclareMathOperator{\legpoly}{P}
\DeclareMathOperator{\primevalue}{P}
\DeclareMathOperator{\sgn}{sgn}
\DeclareMathOperator{\res}{Res}
\DeclareMathOperator{\sinc}{sinc}
\newcommand*{\ii}{\mathrm{i}}
\newcommand*{\ee}{\mathrm{e}}
\newcommand*{\const}{\mathrm{const}}
\newcommand*{\suchthat}{\quad \text{s.t.} \quad}
\newcommand*{\argmin}{\arg\min}
\newcommand*{\argmax}{\arg\max}
\newcommand*{\normalorder}[1]{: #1 :}
\newcommand*{\pair}[1]{\langle #1 \rangle}
\newcommand*{\fd}[1]{\mathcal{D} #1}
\DeclareMathOperator{\bigO}{\mathcal{O}}

% TikZ setting
\usetikzlibrary{arrows,shapes,positioning}
\usetikzlibrary{arrows.meta}
\usetikzlibrary{decorations.markings}
\usetikzlibrary{calc}
\tikzstyle arrowstyle=[scale=1]
\tikzstyle directed=[postaction={decorate,decoration={markings,
    mark=at position .5 with {\arrow[arrowstyle]{stealth}}}}]
\tikzstyle ray=[directed, thick]
\tikzstyle dot=[anchor=base,fill,circle,inner sep=1pt]

% Algorithm setting
% Julia-style code
\SetKwIF{If}{ElseIf}{Else}{if}{}{elseif}{else}{end}
\SetKwFor{For}{for}{}{end}
\SetKwFor{While}{while}{}{end}
\SetKwProg{Function}{function}{}{end}
\SetArgSty{textnormal}

\newcommand*{\concept}[1]{{\textbf{#1}}}

% Embedded codes
\lstset{basicstyle=\ttfamily,
  showstringspaces=false,
  commentstyle=\color{gray},
  keywordstyle=\color{blue}
}

% Reference formatting
\newcommand*{\citesec}[1]{\S~{#1}}
\newcommand*{\citechap}[1]{chap.~{#1}}
\newcommand*{\citefig}[1]{Fig.~{#1}}
\newcommand*{\citetable}[1]{Table~{#1}}
\newcommand*{\citepage}[1]{pp.~{#1}}
\newrefformat{fig}{Fig.~\ref{#1}}
\newcommand*{\term}[1]{\textit{#1}}

% Color boxes
\tcbuselibrary{skins, breakable, theorems}

\newtcbtheorem{infobox}{Box}{
    enhanced,
    boxrule=0pt,
    colback=blue!5,
    colframe=blue!5,
    coltitle=blue!50,
    borderline west={4pt}{0pt}{blue!65},
    sharp corners,
    fonttitle=\bfseries, 
    breakable,
    before upper={\parindent15pt\noindent}}{box}
\newtcbtheorem[use counter from=infobox]{theorybox}{Box}{
    enhanced,
    boxrule=0pt,
    colback=orange!5, 
    colframe=orange!5, 
    coltitle=orange!50,
    borderline west={4pt}{0pt}{orange!65},
    sharp corners,
    fonttitle=\bfseries, 
    breakable,
    before upper={\parindent15pt\noindent}}{box}
\newtcbtheorem[use counter from=infobox]{learnbox}{Box}{
    enhanced,
    boxrule=0pt,
    colback=green!5,
    colframe=green!5,
    coltitle=green!50,
    borderline west={4pt}{0pt}{green!65},
    sharp corners,
    fonttitle=\bfseries, 
    breakable,
    before upper={\parindent15pt\noindent}}{box}


\newenvironment{shelldisplay}{\begin{lstlisting}}{\end{lstlisting}}

\newcommand*{\kB}{k_{\text{B}}}
\newcommand*{\muB}{\mu_{\text{B}}}
\newcommand*{\efermi}{E_{\text{F}}}
\newcommand*{\pfermi}{p_{\text{F}}}
\newcommand*{\vfermi}{v_{\text{F}}}
\newcommand*{\sA}{\text{A}}
\newcommand*{\sB}{\text{B}}
\newcommand*{\Tc}{T_{\text{c}}}
\newcommand*{\hethree}{$^3$He}
\newcommand*{\hefour}{$^4$He}
\newcommand{\epsr}{\epsilon_{\text{r}}}
\newcommand*{\mur}{\mu_{\text{r}}}
\newcommand{\chie}{\chi_{\text{e}}}
\newcommand*{\Gammae}{\Gamma_{\text{e}}}
\newcommand*{\Gammag}{\Gamma_{\text{g}}}
\newcommand*{\omegae}{\omega_{\text{e}}}
\newcommand*{\omegag}{\omega_{\text{g}}}
\newcommand*{\omegaeg}{\omega_{\text{eg}}}
\newcommand*{\ptwfc}[2]{\psi^{(#2)}_{#1}}
\newcommand*{\mueg}{\mu_{\text{eg}}}
\newcommand*{\muge}{\mu_{\text{ge}}}
\newcommand*{\Ezzero}{E_{z0}}
\newcommand*{\kete}{\ket*{\text{e}}}
\newcommand*{\ketg}{\ket*{\text{g}}}
\newcommand*{\dyade}{\dyad{\text{e}}}
\newcommand*{\dyadg}{\dyad{\text{g}}}
\newcommand*{\dyadeg}{\dyad{\text{e}}{\text{g}}}
\newcommand*{\dyadge}{\dyad{\text{g}}{\text{e}}}
\newcommand*{\coeffe}{c_{\text{e}}}
\newcommand*{\coeffg}{c_{\text{g}}}
\newcommand*{\pope}{p_{\text{e}}}
\newcommand*{\popg}{p_{\text{g}}}
\newcommand*{\ptwo}{P^{(2)}}
\newcommand*{\vp}{v_{\text{p}}}
\newcommand*{\chitwo}{\chi^{(2)}}
\newcommand*{\chithree}{\chi^{(3)}}
\newcommand*{\omegap}{\omega_{\text{p}}}
\newcommand*{\mvb}[1]{\tilde{\vb*{#1}}}
\newcommand*{\Si}[1]{S_{\text{i#1}}}
\newcommand*{\So}[1]{S_{\text{o#1}}}
\newcommand*{\taug}{\tau_{\text{g}}}
\newcommand*{\dieg}{\vb*{d}_{\text{eg}}} 
\newcommand*{\dige}{\vb*{d}_{\text{ge}}} 
\newcommand*{\statee}{\text{e}}
\newcommand*{\stateg}{\text{g}}

\title{Cavity QED: light-matter interaction to the extreme}
\author{Jinyuan Wu}

\begin{document}

\maketitle

\begin{abstract}
    In the strong light-matter coupling limit,
    a high-coherence system can no longer be faithfully described by conventional approaches 
    like semi-classical theories, 
    effective dielectric function approaches,
    or master equation approaches assuming thermalization of some of the degrees of freedoms,
    and a fully quantum treatment of both light and matter is needed.
    In this report, we review cavity quantum electrodynamics (cavity QED), 
    the fully quantum theory of strong, coherent light-matter interaction in a cavity.
    We then introduce Jaynes-Cummings model, the simplest cavity QED model,
    and discuss exotic phenomena that can be observed in Jaynes-Cummings model,
    including reversible spontaneous emission and quantum Rabi oscillation and
    the collapse and revival of the probability for the atom to appear in the excited state
    with the initial photon state being a coherent state.
    We also discuss non-conventional experimental platforms for cavity QED, 
    including excitonic materials in a cavity 
    and nanoparticle in a dielectric-metal-dielectric sandwich.
\end{abstract}

% Something worth exploring:
% Strong-field dressing effects \cite{di2016cutting}

\section{Introduction}

Cavity quantum electrodynamics (cavity QED) refers literally to 
quantum electrodynamics in a cavity,
or in other words a fully quantum treatment of light-matter interaction in a cavity.
To see why such a treatment is necessary,
it is instructive to review alternative methodologies to light-matter interaction.

The textbook level treatment of light-matter interaction 
is often the so-called semi-classical approach,
in which the matter part of the system is treated quantum mechanically,
while the electromagnetic field is classical \cite{scully1997quantum}.
This approach by definition fails when the state of the electromagnetic field 
deviates far from a coherent state.
Some cases of this will also be reviewed in this report.
Alternatively, the dynamics of the matter degrees of freedom can be implicitly 
represented as an effective dielectric function,
and by inserting this dielectric function into the (quantum) Maxwell equation,
we have a quantum theory of light.
The problem with this approach is that optically active excitation modes in the medium 
are represented as poles in the dielectric function and 
hence the dielectric function has strong frequency dependence,
and the Kramers-Kronig theorem means the dielectric function has a non-negligible imaginary part.
To keep the theory unitary,
some sort of auxiliary field, like the ``noise'' or ``Langevin'' term, 
has to be introduced for compensation,
whose correlation functions are quantities independent to the dielectric function
\cite{huttner1992quantization,knoll2000qed,philbin2010canonical}. 
Apart from this additional complexity,
obtaining the dielectric function and the aforementioned correlation functions 
from the dynamics of the medium
is challenging when light-matter interaction is strong.
When the dielectric function can indeed be obtained reliably,
and we can assume that the medium is thermalized fast enough,
a Maxwell-Langevin equation can be established
\cite{henry1986theory,henry1986phase,amooghorban2013quantum},
which has been used to estimate laser noises in laser physics
\cite{henry1986theory,henry1986phase}.
Alternatively we can assume that the light part of the system is thermalized
and establish a quantum master equation of the medium part
\cite{scully1997quantum}.
This simplification however is not feasible when 
both the light and matter part of the system is highly coherent.

Therefore, in the high light-matter coupling, high frequency dependence, high coherence limit,
and when the state of the electromagnetic field is of strong quantum nature,
all simplified approaches fail and a fully quantum treatment of light-matter interaction is needed.
These conditions are relatively easily satisfied 
in a good optical cavity.
We first note that by adjusting the shape of the cavity,
it is relatively easy to achieve high field strength of electromagnetic modes in the cavity
at some locations,
and atoms placed at these positions will be strongly coupled to the optical modes.
We note that when the light-matter interaction strength exceeds
incoherent interaction (e.g. spontaneous emission out of the cavity and cavity leaking),
the latter should be ignored for the short-time dynamics of the cavity.
The strong light-matter coupling also implies strong frequency dependence.
Therefore, for a high-coherence cavity with a very small beam waist of its photon modes,
a fully quantum treatment of light-matter interaction, i.e. cavity QED,
is needed, as long as the electromagnetic part of the system is of quantum nature,
which we will find is true even for a coherent state.
The fact that high coupling strength requires a fully quantum treatment of light-matter interaction
of course is not restricted to a cavity.
Actually it is not even restricted to photon-atom interaction:
in this report we will review scenarios where 
the roles of the photon mode and the atom are replaced by other degrees of freedom,
but the overall theoretical framework of cavity QED remains valid.

This report reviews is organized as follows.
In \prettyref{sec:jc},
we introduce the simplest cavity QED model,
the Jaynes-Cummings model,
and briefly discuss its implications.
In \prettyref{sec:collapse-revival} 
we discuss the time evolution of a cavity 
containing an excited atom and a coherent state under the Jaynes-Cummings model
and show that coherent states cannot always be treated in a classical way.
\prettyref{sec:condense} discusses cavity QED with condensed matter systems 
and how the formalisms describing photon-atom interaction can be transplanted to these systems. 
We conclude this report with \prettyref{sec:conclusion}.

\section{The Jaynes-Cummings model}\label{sec:jc}

In this section we consider an extremely simple but still physically relevant scenario. 
When the matter in a cavity is extremely dilute and
atom-atom interaction can be ignored,
we can model the light-matter interaction in a cavity
as the interaction between one single atom and the cavity photon modes.
We also assume that the dipole interaction strength, $g$, between this atom and the cavity modes
is much more important than 
the spontaneous emission rate $\gamma$ or cavity leakage rate $\kappa$.
Further, we assume that the energy difference $\hbar \omega_0$ between two atom energy levels
is very close to the frequency $\omega$ of one cavity photon mode (assuming there is no degeneracy),
while the interaction strength $g$ does not exceed $\omega_0$.
Therefore we can do rotating wave approximation (RWA)
and consider only the coupling between the aforementioned two atom energy levels and the photon mode,
and ignore all anti-resonant terms in the dipole interaction Hamiltonian.
The resulting model then can be represented by the following simple Hamiltonian
containing only a photon mode and a two-level atom,
\begin{equation}
    H^{\text{JC}} = \hbar \omega_0 \dyade + \hbar \omega \left( a^\dag a + \frac{1}{2} \right) + \hbar g \dyadeg a + \hbar g^* \dyadge a^\dag,
\end{equation}
known as Jaynes-Cuming model \cite{bina2012coherent}.
Here we can redefine $\kete$ such that $g$ is a real number.

The Jaynes-Cummings model can be analytically solved
by noticing that it is block-diagonal in the $\{\ket*{\stateg, n+1}, \ket*{\statee, n}\}$ subspace.
In this subspace, the Hamiltonian reads
\begin{equation}
    H^{\text{JC}} = \hbar \omega \left(n + \frac{1}{2}\right) + \hbar \pmqty{
        \omega & g \sqrt{n+1} \\
        g \sqrt{n+1} & \omega_0 
    }.
\end{equation}
Defining 
\begin{equation}
    g \sqrt{n+1} = \frac{\Omega_n}{2}, \quad \delta = \omega - \omega_0,
\end{equation}
the Hamiltonian reads 
\begin{equation}
    H^{\text{JC}} = \hbar \omega \left( n + \frac{1}{2} \right) + \hbar \omega_0 + \frac{1}{2} \hbar \delta + \hbar 
    \pmqty{
        \delta / 2 & \Omega_n / 2 \\
        \Omega_n / 2 & - \delta / 2
    }.
\end{equation}
The eigenvalues of this Hamiltonian are 
\begin{equation}
    E_\pm = \hbar \omega \left( n + \frac{1}{2} \right) + \hbar \omega_0 + \frac{1}{2} \hbar \delta 
    \pm \frac{1}{2} \hbar \sqrt{\delta^2 + \Omega_n^2}.
\end{equation}
An avoided crossing can be observed when the detuning $\delta=0$;
the split is determined by the interaction strength $\Omega_n$.

The hybridization of $\{\ket*{\stateg, n+1}, \ket*{\statee, n}\}$
can also be understood as quantum Rabi oscillation.
For simplicity, we again take the detuning $\delta$ to be zero.
The time evolution starting with $\ket*{\statee, n}$ therefore is 
\begin{equation}
    \ket*{\psi(t)} = - \ii \sin \frac{\Omega_n}{2} t \ket*{\stateg, n+1}  
    + \cos \frac{\Omega_n}{2} t \ket*{\statee, n}.
\end{equation}
This quantum Rabi oscillation is a consequence of reversible spontaneous emission in Jaynes-Cummings model.
The $\hbar g \dyadge a^\dag$ term is by definition a spontaneous emission term.
However, unlike the case where we have a continuum of outgoing photon modes 
and photons spontaneously emitted go away,
in Jaynes-Cummings model, the spontaneously emitted photon can only be re-absorbed by the atom.
This leads to the breakdown of the Markovian approximation
and the probability of the atom to appear in the excited state no longer exponentially decays but oscillates.
Specifically, when the state of the system starts with $\kete$,
an oscillation can still be observed.

\section{Quantum collapse and revival}\label{sec:collapse-revival}

A more intriguing property of Jaynes-Cummings model is that
even when the photons are in a coherent state,
we can still observe inherently quantum phenomena.
Consider an initial state $\ket*{\statee, \alpha}$.
The state, under the Fock basis, is
\begin{equation}
    \ket*{\statee, \alpha} = \ee^{- \frac{\abs*{\alpha}^2}{2}} 
    \sum_{n=0}^{\infty} \frac{\alpha^n}{\sqrt{n!}} \ket*{\text{e}, n}.
\end{equation}
As we have already seen, quantum Rabi oscillation happens between $\ket*{\statee, n}$ and $\ket*{\stateg, n+1}$.
Taking $\delta=0$, the probability for the atom to stay in the excited state is
\begin{equation}
    \begin{aligned}
        \pope &= \bra*{\statee, \alpha} U^\dag(t) \dyad{\statee} U(t) \ket*{\statee, \alpha} \\
        &= \ee^{-\abs{\alpha}^2} \sum_{n=0}^{\infty} \frac{\abs*{\alpha}^{2n}}{n!} \cos^2 \frac{\Omega_n}{2} t \\
        &= \ee^{-\abs{\alpha}^2} \sum_{n=0}^{\infty} \frac{\abs*{\alpha}^{2n}}{n!} \cdot \frac{1}{2} (\cos \Omega_n t + 1) \\
        &= \frac{1}{2} + \frac{1}{2} \ee^{-\abs{\alpha}^2} \sum_{n=0}^{\infty} \frac{\abs*{\alpha}^{2n}}{n!} \cos \Omega_n t .
    \end{aligned}
    \label{eq:pope-fock-expansion}
\end{equation}

We note that the oscillation frequencies $\Omega_n = 2 g \sqrt{n+1}$ of the terms in $\pope$ are different:
this means the relative phases of these terms become messy after a short period of time.
If we take the $\abs*{\alpha} \to \infty$ limit,
we will find that the weight factor $\abs*{\alpha}^{2n} / n!$ reaches its peak when $n = \abs*{\alpha}^2$.
Therefore we can expand $\Omega_n$ around $n = \abs*{\alpha}^2$.
We have 
\[
    \sqrt{n+1} = \sqrt{\abs*{\alpha}^2 + 1 + n - \abs*{\alpha}^2 }
    \approx \abs*{\alpha} \left( 1 + \frac{n - \abs*{\alpha}^2 }{2 \abs*{\alpha}^2} \right),
\]
where we have used the condition that $\abs{\alpha}$ is very large, 
and the assumption that the peak of  $\abs*{\alpha}^{2n} / n!$ is narrow enough 
to make sure that only first order expansion is needed.
Therefore 
\begin{equation}
    \begin{aligned}
        \pope &= \frac{1}{2} + \frac{1}{2} \ee^{- \abs*{\alpha}^2} \sum_{n=0}^{\infty} \frac{\abs*{\alpha}^{2n}}{n!} 
        \cos(
            2 g \abs*{\alpha} t + g t \frac{n - \abs*{\alpha}^2}{\abs*{\alpha}}
        ) \\
        &= \frac{1}{2} + \frac{\ee^{- \abs*{\alpha}^2}}{2} \sum_{n=0}^{\infty} \frac{\abs*{\alpha}^{2n}}{n!} 
        \left(\cos( 2 g \abs*{\alpha} t) \cos(
            g t \frac{n - \abs*{\alpha}^2 }{\abs*{\alpha}}
        ) - \sin( 2 g \abs*{\alpha} t) \sin(
            g t \frac{n - \abs*{\alpha}^2 }{\abs*{\alpha}}
        )\right).
    \end{aligned}
\end{equation}
Since sin is odd and $n - \abs{\alpha}^2$ can be positive or negative,
the second term largely vanishes after the summation over $n$.
Therefore we are left with 
\begin{equation}
    \begin{aligned}
        \pope = \frac{1}{2} + \frac{\ee^{- \abs*{\alpha}^2}}{2} 
        \cos( 2 g \abs*{\alpha} t)  \sum_{n=0}^{\infty} \frac{\abs*{\alpha}^{2n}}{n!} 
        \cos(
            g t \frac{n - \abs*{\alpha}^2 }{\abs*{\alpha}}
        ).
    \end{aligned}
\end{equation}
We have 
\[
    \begin{aligned}
        &\quad \sum_{n=0}^{\infty} \frac{\abs*{\alpha}^{2n}}{n!} 
        \cos(
            g t \frac{n - \abs*{\alpha}^2 }{\abs*{\alpha}}
        ) \\
        &= \frac{1}{2} \sum_{n=0}^{\infty} \frac{\abs*{\alpha}^{2n}}{n!} 
        (
            \ee^{\ii g t \frac{n - \abs*{\alpha}^2 }{\abs*{\alpha}}}
            + \ee^{-\ii g t \frac{n - \abs*{\alpha}^2 }{\abs*{\alpha}}}
        ) \\
        &= \frac{1}{2} \ee^{- \ii \abs{\alpha} g t} \ee^{\abs{\alpha}^2 \ee^{\ii g t / \abs{\alpha}}} + \text{c.c.}
    \end{aligned}
\]
Now we use the $g \abs{\alpha} t \ll 1$ assumption, and we have 
\begin{equation}
    \begin{aligned}
        \pope &= \frac{1}{2} + \frac{\ee^{- \abs*{\alpha}^2}}{4} 
        \cos( 2 g \abs*{\alpha} t) \ee^{- \ii \abs{\alpha} g t} \ee^{\abs{\alpha}^2 \ee^{\ii g t / \abs{\alpha}}} + \text{c.c.} \\
        &\approx  \frac{1}{2} + \frac{\ee^{- \abs*{\alpha}^2}}{4} 
        \cos( 2 g \abs*{\alpha} t) \ee^{- \ii \abs{\alpha} g t} \ee^{\abs{\alpha}^2 \left(1 + \ii g t / \abs{\alpha} - \frac{1}{2 \abs{\alpha}^2} g^2 t^2  \right) } + \text{c.c.} \\
        &= \frac{1}{2} + \frac{1}{2} \cos (2 g \abs{\alpha} t) \ee^{- \frac{1}{2} g^2 t^2}.
    \end{aligned}
\end{equation}
So it can be seen that when $\abs{\alpha} \gg 1$ and $g t \abs{\alpha} \ll 1$,
the probability of the atom residing at $\kete$ decays exponentially,
and the time scale of the decay is $1/g$.
This is known as the quantum collapse of the excited state population of the atom,
which originates from the phase mismatch among terms in \eqref{eq:pope-fock-expansion}.

The quantum collapse phenomenon is exotic and demonstrates that 
sometimes even when the state of the electromagnetic field is a coherent state,
the semiclassical approach of light-matter interaction still fails.
We note however that collapse also appears with the presence 
of a statistical noise in the electromagnetic field \cite{knight1982quantum}.
What reveals the quantum nature best is the phenomenon of quantum revival.
Inspecting \eqref{eq:pope-fock-expansion} again,
we notice that when 
\begin{equation}
    (\Omega_{\abs*{\alpha}^2} - \Omega_{\abs*{\alpha}^2-1}) t = 2\pi n,
    \label{eq:granular-revival}
\end{equation}
$n$ being an integer,
the phases of the predominant terms in \eqref{eq:pope-fock-expansion} realign with each other again,
and therefore $\pope$ revives periodically.
This is a phenomenon that is truly beyond any semiclassical treatment,
deterministic or random,
because it relies on the granular nature of the electromagnetic field:
if the ``strength'' of the field is somehow continuous,
the $\Delta \Omega$ factor in \eqref{eq:granular-revival} can be arbitrarily small,
and the revival period goes to infinity.

\section{Cavity QED with condensed matter}\label{sec:condense}

An interesting question is what will happen if in place of the atom,
we insert a solid.
As is mentioned in the introduction,
the general theory of quantum optics in media is hard,
and usually involves including auxiliary degrees of freedom,
whose correlation functions are to be calculated alongside the dielectric function,
to reproduce the frequency dependence and imaginary part of the dielectric function.
In the high coherence limit,
these auxiliary degrees of freedom can be chosen as
the most optically active excitation modes in the medium,
which take the place of the atom in Jaynes-Cummings model.

The most prototypical example of this approach 
is the theory of coherent coupling between cavity modes and excitons
\cite{latini2019cavity}.
Because an exciton mode is an interactively corrected electron-hole pair,
it is hard to create two excitons in one mode because of depletion of electron-hole pairs,
and therefore an exciton mode can be perceived as a two-level atom.
The theory about coherent exciton-cavity photon coupling
therefore is an extension of the Jaynes-Cummings model
in that it contains multiple exciton modes, i.e. multiple ``atoms''.
The eigenstates of this extended Jaynes-Cummings model
are exciton-polaritons in the cavity.
We note that this extended Jaynes-Cummings model
contains additional ``two-atom'', one-photon terms in the form of
$\dyad{S_1}{S_2} a$ or $\dyad{S_1}{S_2} a^\dag$,
where $S_{1,2}$ are exciton modes,
because there can be a non-zero dipole matrix between the two exciton modes
and therefore the two modes are coupled together by creation or annihilation of a photon.
Since nothing prohibits $\mel*{S_1}{\vb*{d}}{S_2} \neq 0$
even if $\mel*{S_2}{\vb*{d}}{\text{vacuum}} = 0$,
bright exciton modes (i.e. modes for which $\mel*{S_2}{\vb*{d}}{\text{vacuum}} \neq 0$)
are hybridized with dark exciton modes 
(i.e. modes for which $\mel*{S_2}{\vb*{d}}{\text{vacuum}} = 0$).
Without a cavity, dark exciton modes are not visible,
but with the presence of a cavity and hence the hybridization effects,
dark exciton modes now directly appear on the optical spectrum of the medium,
though with a shifted frequency \cite{latini2019cavity}.

Finally, the photon mode(s) in cavity QED can be replaced by other bosonic excitations as well,
provided that the behavior of the latter is closer to photons
(e.g. no depletion effects).
For example, a surface plasmon polariton mode can be formed 
by sandwiching a layer of metal between two layers of dielectric,
forming a ``cavity'', which interacts with exciton modes of a nanoparticle planted near the surface,
and phenomena like quantum Rabi oscillations can again be observed in such a device
\cite{gonzalez2014reversible}.

\section{Conclusion}\label{sec:conclusion}

In this report we have reviewed several settings with strong, coherent light-matter interaction,
namely a near-ideal cavity with a small beam waist,
an excitonic material slice in a near-ideal cavity,
and a nanoparticle in a dielectric-metal-dielectric cavity.
Although the underlying degrees of freedom are different,
they can all be theoretically treated with the formalisms designed to handle the first case, 
namely cavity QED.
Phenomena including reversible spontaneous emission and quantum Rabi oscillation,
quantum collapse and revival can be observed in cavity QED,
which demonstrate the quantum and granular nature of light,
and in the case of exciton-photon interaction in a cavity,
the distinction between dark and bright exciton modes is broken,
which can be utilized as a detection method of dark excitons.

The field of cavity QED is, of course, much broader than what can be contained in this report.
There are for example several possible extensions to Jaynes-Cummings model,
for example situations where RWA is not physical \cite{larson2012absence}
or where atom-atom interaction is not negligible \cite{vaidya2018tunable}.
Alternative experimental platforms of cavity QED exist 
\cite{le2006cavity,hummer2013weak,blais2021circuit},
the most famous one being circuit QED \cite{blais2021circuit},
which itself has multiple varieties.
As a still developing field,
cavity QED has challenged and will continue to challenge
intuitions that we take for granted about light-matter interaction.

\printbibliography

\end{document}