\documentclass[hyperref, a4paper]{article}

\usepackage{geometry}
\usepackage{titling}
\usepackage{titlesec}
% No longer needed, since we will use enumitem package
% \usepackage{paralist}
\usepackage{enumitem}
\usepackage{footnote}
\usepackage[colorinlistoftodos]{todonotes}
\usepackage{amsmath, amssymb, amsthm}
\usepackage{mathtools}
\usepackage{bbm}
\usepackage{graphicx}
\usepackage{subcaption}
\usepackage{soulutf8}
\usepackage{physics}
\usepackage{tensor}
\usepackage{siunitx}
\usepackage[version=4]{mhchem}
\usepackage{tikz}
\usepackage{xcolor}
\usepackage{listings}
\usepackage{autobreak}
\usepackage[ruled, vlined, linesnumbered]{algorithm2e}
\usepackage{nameref,zref-xr}
\zxrsetup{toltxlabel}
\usepackage[backend=bibtex]{biblatex}
\addbibresource{jc.bib}
\addbibresource{qed.bib}
\usepackage[colorlinks,unicode]{hyperref} % , linkcolor=black, anchorcolor=black, citecolor=black, urlcolor=black, filecolor=black
\usepackage[most]{tcolorbox}
\usepackage{prettyref}

% Page style
\geometry{left=3.18cm,right=3.18cm,top=2.54cm,bottom=2.54cm}
\titlespacing{\paragraph}{0pt}{1pt}{10pt}[20pt]
\setlength{\droptitle}{-5em}

% More compact lists 
\setlist[itemize]{
    itemindent=17pt, 
    leftmargin=1pt,
    listparindent=\parindent,
    parsep=0pt,
}

% Math operators
\DeclareMathOperator{\timeorder}{\mathcal{T}}
\DeclareMathOperator{\diag}{diag}
\DeclareMathOperator{\legpoly}{P}
\DeclareMathOperator{\primevalue}{P}
\DeclareMathOperator{\sgn}{sgn}
\DeclareMathOperator{\res}{Res}
\DeclareMathOperator{\sinc}{sinc}
\newcommand*{\ii}{\mathrm{i}}
\newcommand*{\ee}{\mathrm{e}}
\newcommand*{\const}{\mathrm{const}}
\newcommand*{\suchthat}{\quad \text{s.t.} \quad}
\newcommand*{\argmin}{\arg\min}
\newcommand*{\argmax}{\arg\max}
\newcommand*{\normalorder}[1]{: #1 :}
\newcommand*{\pair}[1]{\langle #1 \rangle}
\newcommand*{\fd}[1]{\mathcal{D} #1}
\DeclareMathOperator{\bigO}{\mathcal{O}}

% TikZ setting
\usetikzlibrary{arrows,shapes,positioning}
\usetikzlibrary{arrows.meta}
\usetikzlibrary{decorations.markings}
\usetikzlibrary{calc}
\tikzstyle arrowstyle=[scale=1]
\tikzstyle directed=[postaction={decorate,decoration={markings,
    mark=at position .5 with {\arrow[arrowstyle]{stealth}}}}]
\tikzstyle ray=[directed, thick]
\tikzstyle dot=[anchor=base,fill,circle,inner sep=1pt]

% Algorithm setting
% Julia-style code
\SetKwIF{If}{ElseIf}{Else}{if}{}{elseif}{else}{end}
\SetKwFor{For}{for}{}{end}
\SetKwFor{While}{while}{}{end}
\SetKwProg{Function}{function}{}{end}
\SetArgSty{textnormal}

\newcommand*{\concept}[1]{{\textbf{#1}}}

% Embedded codes
\lstset{basicstyle=\ttfamily,
  showstringspaces=false,
  commentstyle=\color{gray},
  keywordstyle=\color{blue}
}

% Reference formatting
\newcommand*{\citesec}[1]{\S~{#1}}
\newcommand*{\citechap}[1]{chap.~{#1}}
\newcommand*{\citefig}[1]{Fig.~{#1}}
\newcommand*{\citetable}[1]{Table~{#1}}
\newcommand*{\citepage}[1]{pp.~{#1}}
\newrefformat{fig}{Fig.~\ref{#1}}
\newcommand*{\term}[1]{\textit{#1}}

% Color boxes
\tcbuselibrary{skins, breakable, theorems}

\newtcbtheorem{infobox}{Box}{
    enhanced,
    boxrule=0pt,
    colback=blue!5,
    colframe=blue!5,
    coltitle=blue!50,
    borderline west={4pt}{0pt}{blue!65},
    sharp corners,
    fonttitle=\bfseries, 
    breakable,
    before upper={\parindent15pt\noindent}}{box}
\newtcbtheorem[use counter from=infobox]{theorybox}{Box}{
    enhanced,
    boxrule=0pt,
    colback=orange!5, 
    colframe=orange!5, 
    coltitle=orange!50,
    borderline west={4pt}{0pt}{orange!65},
    sharp corners,
    fonttitle=\bfseries, 
    breakable,
    before upper={\parindent15pt\noindent}}{box}
\newtcbtheorem[use counter from=infobox]{learnbox}{Box}{
    enhanced,
    boxrule=0pt,
    colback=green!5,
    colframe=green!5,
    coltitle=green!50,
    borderline west={4pt}{0pt}{green!65},
    sharp corners,
    fonttitle=\bfseries, 
    breakable,
    before upper={\parindent15pt\noindent}}{box}


\newenvironment{shelldisplay}{\begin{lstlisting}}{\end{lstlisting}}

\newcommand*{\kB}{k_{\text{B}}}
\newcommand*{\muB}{\mu_{\text{B}}}
\newcommand*{\efermi}{E_{\text{F}}}
\newcommand*{\pfermi}{p_{\text{F}}}
\newcommand*{\vfermi}{v_{\text{F}}}
\newcommand*{\sA}{\text{A}}
\newcommand*{\sB}{\text{B}}
\newcommand*{\Tc}{T_{\text{c}}}
\newcommand*{\hethree}{$^3$He}
\newcommand*{\hefour}{$^4$He}
\newcommand{\epsr}{\epsilon_{\text{r}}}
\newcommand*{\mur}{\mu_{\text{r}}}
\newcommand{\chie}{\chi_{\text{e}}}
\newcommand*{\Gammae}{\Gamma_{\text{e}}}
\newcommand*{\Gammag}{\Gamma_{\text{g}}}
\newcommand*{\omegae}{\omega_{\text{e}}}
\newcommand*{\omegag}{\omega_{\text{g}}}
\newcommand*{\omegaeg}{\omega_{\text{eg}}}
\newcommand*{\ptwfc}[2]{\psi^{(#2)}_{#1}}
\newcommand*{\mueg}{\mu_{\text{eg}}}
\newcommand*{\muge}{\mu_{\text{ge}}}
\newcommand*{\Ezzero}{E_{z0}}
\newcommand*{\kete}{\ket*{\text{e}}}
\newcommand*{\ketg}{\ket*{\text{g}}}
\newcommand*{\dyade}{\dyad{\text{e}}}
\newcommand*{\dyadg}{\dyad{\text{g}}}
\newcommand*{\dyadeg}{\dyad{\text{e}}{\text{g}}}
\newcommand*{\dyadge}{\dyad{\text{g}}{\text{e}}}
\newcommand*{\coeffe}{c_{\text{e}}}
\newcommand*{\coeffg}{c_{\text{g}}}
\newcommand*{\pope}{p_{\text{e}}}
\newcommand*{\popg}{p_{\text{g}}}
\newcommand*{\ptwo}{P^{(2)}}
\newcommand*{\vp}{v_{\text{p}}}
\newcommand*{\chitwo}{\chi^{(2)}}
\newcommand*{\chithree}{\chi^{(3)}}
\newcommand*{\omegap}{\omega_{\text{p}}}
\newcommand*{\mvb}[1]{\tilde{\vb*{#1}}}
\newcommand*{\Si}[1]{S_{\text{i#1}}}
\newcommand*{\So}[1]{S_{\text{o#1}}}
\newcommand*{\taug}{\tau_{\text{g}}}
\newcommand*{\dieg}{\vb*{d}_{\text{eg}}} 
\newcommand*{\dige}{\vb*{d}_{\text{ge}}} 
\newcommand*{\statee}{\text{e}}
\newcommand*{\stateg}{\text{g}}

\title{Cavity QED: light-matter interaction to the extreme}
\author{Jinyuan Wu}

\begin{document}

\maketitle

\paragraph*{Abstract} Light-matter interaction is often studied by focusing on  
the atomic part or the photon part only, 
with the other part being modeled as 
intermediate states (as in Maxwell's Equations in media), 
external sources (as in scattering theory), 
or a thermalized bath (as in quantum Langevin equation).
For ultracold systems with strong light-matter coupling,
none of the formalisms mentioned above works 
and both the atoms and the photons need to be modeled in a 
fully quantum and coherent way.
In this report, we review cavity quantum electrodynamics (cavity QED or cQED), 
the theory of atoms interacting with photons within a reflective cavity.
We will study how cQED reduces to aforementioned formalisms 
and discuss several phenomena that can only be reliably captured in cQED, 
including tunable and even reversible spontaneous emission, 
strong-field dressing effects, 
and the collapse and revival of the probability for the atom to appear in the excited state
with the initial photon state being a coherent state.
We will also discuss non-conventional experimental platforms, 
like excitonic materials in a cavity,
for which cQED is a good effective theory.

\paragraph*{Outline} 
\begin{itemize}
    \item Basic formalism \cite{bina2012coherent}
    \item Collapse and revival \cite{meystre2021quantum}
    \item Tuning spontaneous emission \cite{meystre2021quantum}
    \item Strong-field dressing effects \cite{di2016cutting}
    \item cQED in excitons \cite{latini2019cavity}
\end{itemize}

\section{Introduction}

Cavity quantum electrodynamics (cavity QED) refers literally to 
quantum electrodynamics in a cavity,
or in other words a fully quantum treatment of light-matter interaction in a cavity.
To see why such a treatment is necessary,
it is instructive to review alternative methodologies to light-matter interaction.

The textbook level treatment of light-matter interaction 
is often the so-called semi-classical approach,
in which the matter part of the system is treated quantum mechanically,
while the electromagnetic field is classical \cite{scully1997quantum}.
This approach by definition fails when the state of the electromagnetic field 
deviates far from a coherent state.
Some cases of this will also be reviewed in this report.
Alternatively, the dynamics of the matter degrees of freedom can be implicitly 
represented as an effective dielectric function,
and by inserting this dielectric function into the (quantum) Maxwell equation,
we have a quantum theory of light.
The problem with this approach is that optically active excitation modes in the medium 
are represented as poles in the dielectric function and 
hence the dielectric function has strong frequency dependence,
and the Kramers-Kronig theorem means the dielectric function has a non-negligible imaginary part.
To keep the theory unitary,
some sort of auxiliary field, like the ``noise'' or ``Langevin'' term, 
has to be introduced for compensation,
whose correlation functions are quantities independent to the dielectric function
\cite{huttner1992quantization,knoll2000qed,philbin2010canonical}. 
Apart from this additional complexity,
obtaining the dielectric function and the aforementioned correlation functions 
from the dynamics of the medium
is challenging when light-matter interaction is strong.
When the dielectric function can indeed be obtained reliably,
and we can assume that the medium is thermalized fast enough,
a Maxwell-Langevin equation can be established
\cite{henry1986theory,henry1986phase,amooghorban2013quantum},
which has been used to estimate laser noises in laser physics
\cite{henry1986theory,henry1986phase}.
Alternatively we can assume that the light part of the system is thermalized
and establish a quantum master equation of the medium part
\cite{scully1997quantum}.
This simplification however is not feasible when 
both the light and matter part of the system is highly coherent.

Therefore, in the high light-matter coupling, high frequency dependence, high coherence limit,
and when the state of the electromagnetic field is of strong quantum nature,
all simplified approaches fail and a fully quantum treatment of light-matter interaction is needed.
These conditions are relatively easily satisfied 
in a good optical cavity.
We first note that by adjusting the shape of the cavity,
it is relatively easy to achieve high field strength of electromagnetic modes in the cavity
at some locations,
and atoms placed at these positions will be strongly coupled to the optical modes.
We note that the high light-matter coupling limit logically
necessitates a high-coherence treatment of the system,
as when the light-matter interaction strength exceeds
incoherent interaction (e.g. spontaneous emission out of the cavity and cavity leaking),
the latter should be ignored for the short-time dynamics of the cavity.
The strong light-matter coupling also implies strong frequency dependence.
Therefore, for a cavity with a very small beam waist,
a fully quantum treatment of light-matter interaction, i.e. cavity QED,
is needed, as long as the electromagnetic part of the system is of quantum nature,
which we will find is true even for a coherent state.

This report reviews is organized as follows.
In \prettyref{sec:jc},
we introduce the simplest cavity QED model,
the Jaynes-Cummings model,
and briefly discuss its implications.
In \prettyref{sec:collapse-revival} 
we discuss the time evolution of a cavity 
containing an excited atom and a coherent state under the Jaynes-Cummings model
and show that coherent states cannot always be treated in a classical way.
\prettyref{sec:condense} discusses cavity QED with condensed matter systems 
and how the formalisms describing photon-atom interaction can be transplanted to these systems. 
We conclude this report with \prettyref{sec:conclusion}.

\section{The Jaynes-Cummings model}\label{sec:jc}

In this section we consider an extremely simple but still physically relevant scenario. 
When the matter in a cavity is extremely dilute and
atom-atom interaction can be ignored,
we can model the light-matter interaction in a cavity
as the interaction between one single atom and the cavity photon modes.
We also assume that the dipole interaction strength, $g$, between this atom and the cavity modes
is much more important than 
the spontaneous emission rate $\gamma$ or cavity leakage rate $\kappa$.
Further, we assume that the energy difference $\hbar \omega_0$ between two atom energy levels
is very close to the frequency $\omega$ of one cavity photon mode (assuming there is no degeneracy),
while the interaction strength $g$ does not exceed $\omega_0$.
Therefore we can do rotating wave approximation (RWA)
and consider only the coupling between the aforementioned two atom energy levels and the photon mode,
and ignore all anti-resonant terms in the dipole interaction Hamiltonian.
The resulting model then can be represented by the following simple Hamiltonian,
\begin{equation}
    H^{\text{JC}} = \hbar \omega_0 \dyade + \hbar \omega \left( a^\dag a + \frac{1}{2} \right) + \hbar g \dyadeg a + \hbar g^* \dyadge a^\dag,
\end{equation}
known as Jaynes-Cuming model.
Here we can redefine $\kete$ such that $g$ is a real number.

The Jaynes-Cummings model can be analytically solved
by noticing that it is block-diagonal in the $\{\ket*{\stateg, n+1}, \ket*{\statee, n}\}$ subspace.
In this subspace, the Hamiltonian reads
\begin{equation}
    H^{\text{JC}} = \hbar \omega \left(n + \frac{1}{2}\right) + \hbar \pmqty{
        \omega & g \sqrt{n+1} \\
        g \sqrt{n+1} & \omega_0 
    }.
\end{equation}
Defining 
\begin{equation}
    g \sqrt{n+1} = \frac{\Omega_n}{2}, \quad \delta = \omega - \omega_0,
\end{equation}
the Hamiltonian reads 
\begin{equation}
    H^{\text{JC}} = \hbar \omega \left( n + \frac{1}{2} \right) + \hbar \omega_0 + \frac{1}{2} \hbar \delta + \hbar 
    \pmqty{
        \delta / 2 & \Omega_n / 2 \\
        \Omega_n / 2 & - \delta / 2
    }.
\end{equation}
The eigenvalues of this Hamiltonian are 
\begin{equation}
    E_\pm = \hbar \omega \left( n + \frac{1}{2} \right) + \hbar \omega_0 + \frac{1}{2} \hbar \delta 
    \pm \frac{1}{2} \hbar \sqrt{\delta^2 + \Omega_n^2}.
\end{equation}
An avoided crossing can be observed when the detuning $\delta=0$;
the split is determined by the interaction strength $\Omega_n$.

The hybridization of $\{\ket*{\stateg, n+1}, \ket*{\statee, n}\}$
can also be understood as quantum Rabi oscillation.
For simplicity, we again take the detuning $\delta$ to be zero.
The time evolution starting with $\ket*{\statee, n}$ therefore is 
\begin{equation}
    \ket*{\psi(t)} = - \ii \sin \frac{\Omega_n}{2} t \ket*{\stateg, n+1}  
    + \cos \frac{\Omega_n}{2} t \ket*{\statee, n}.
\end{equation}
This quantum Rabi oscillation is a consequence of reversible spontaneous emission in Jaynes-Cummings model.
The $\hbar g \dyadge a^\dag$ term is by definition a spontaneous emission term.
However, unlike the case where we have a continuum of outgoing photon modes 
and photons spontaneously emitted go away,
in Jaynes-Cummings model, the spontaneously emitted photon can only be re-absorbed by the atom.
This leads to the breakdown of the Markovian approximation
and the probability of the atom to appear in the excited state no longer exponentially decays but oscillates.
Specifically, when the state of the system starts with $\kete$,
an oscillation can still be observed.

\section{Quantum collapse and revival}\label{sec:collapse-revival}

A more intriguing property of Jaynes-Cummings model is that
even when the photons are in a coherent state,
we can still observe inherently quantum phenomena.
Consider an initial state $\ket*{\statee, \alpha}$.
The state, under the Fock basis, is
\begin{equation}
    \ket*{\statee, \alpha} = \ee^{- \frac{\abs*{\alpha}^2}{2}} 
    \sum_{n=0}^{\infty} \frac{\alpha^n}{\sqrt{n!}} \ket*{\text{e}, n}.
\end{equation}
As we have already seen, quantum Rabi oscillation happens between $\ket*{\statee, n}$ and $\ket*{\stateg, n+1}$.
Taking $\delta=0$, the probability for the atom to stay in the excited state is
\begin{equation}
    \begin{aligned}
        \pope &= \bra*{\statee, \alpha} U^\dag(t) \dyad{\statee} U(t) \ket*{\statee, \alpha} \\
        &= \ee^{-\abs{\alpha}^2} \sum_{n=0}^{\infty} \frac{\abs*{\alpha}^{2n}}{n!} \cos^2 \frac{\Omega_n}{2} t \\
        &= \ee^{-\abs{\alpha}^2} \sum_{n=0}^{\infty} \frac{\abs*{\alpha}^{2n}}{n!} \cdot \frac{1}{2} (\cos \Omega_n t + 1) \\
        &= \frac{1}{2} + \frac{1}{2} \ee^{-\abs{\alpha}^2} \sum_{n=0}^{\infty} \frac{\abs*{\alpha}^{2n}}{n!} \cos \Omega_n t .
    \end{aligned}
\end{equation}

\section{Cavity QED with condensed matter}\label{sec:condense}

\section{Conclusion}\label{sec:conclusion}

\printbibliography

\end{document}