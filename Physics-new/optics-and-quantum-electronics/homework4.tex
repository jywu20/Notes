\documentclass[hyperref, a4paper]{article}

\usepackage{geometry}
\usepackage{titling}
\usepackage{titlesec}
% No longer needed, since we will use enumitem package
% \usepackage{paralist}
\usepackage{enumitem}
\usepackage{footnote}
\usepackage[colorinlistoftodos]{todonotes}
\usepackage{amsmath, amssymb, amsthm}
\usepackage{mathtools}
\usepackage{bbm}
\usepackage{graphicx}
\usepackage{subcaption}
\usepackage{soulutf8}
\usepackage{physics}
\usepackage{tensor}
\usepackage{siunitx}
\usepackage[version=4]{mhchem}
\usepackage{tikz}
\usepackage{xcolor}
\usepackage{listings}
\usepackage{autobreak}
\usepackage[ruled, vlined, linesnumbered]{algorithm2e}
\usepackage{nameref,zref-xr}
\zxrsetup{toltxlabel}
\usepackage[backend=bibtex]{biblatex}
\addbibresource{elasticity.bib}
\usepackage[colorlinks,unicode]{hyperref} % , linkcolor=black, anchorcolor=black, citecolor=black, urlcolor=black, filecolor=black
\usepackage[most]{tcolorbox}
\usepackage{prettyref}

% Page style
\geometry{left=3.18cm,right=3.18cm,top=2.54cm,bottom=2.54cm}
\titlespacing{\paragraph}{0pt}{1pt}{10pt}[20pt]
\setlength{\droptitle}{-5em}

% More compact lists 
\setlist[itemize]{
    itemindent=17pt, 
    leftmargin=1pt,
    listparindent=\parindent,
    parsep=0pt,
}

% Math operators
\DeclareMathOperator{\timeorder}{\mathcal{T}}
\DeclareMathOperator{\diag}{diag}
\DeclareMathOperator{\legpoly}{P}
\DeclareMathOperator{\primevalue}{P}
\DeclareMathOperator{\sgn}{sgn}
\DeclareMathOperator{\res}{Res}
\DeclareMathOperator{\sinc}{sinc}
\newcommand*{\ii}{\mathrm{i}}
\newcommand*{\ee}{\mathrm{e}}
\newcommand*{\const}{\mathrm{const}}
\newcommand*{\suchthat}{\quad \text{s.t.} \quad}
\newcommand*{\argmin}{\arg\min}
\newcommand*{\argmax}{\arg\max}
\newcommand*{\normalorder}[1]{: #1 :}
\newcommand*{\pair}[1]{\langle #1 \rangle}
\newcommand*{\fd}[1]{\mathcal{D} #1}
\DeclareMathOperator{\bigO}{\mathcal{O}}

% TikZ setting
\usetikzlibrary{arrows,shapes,positioning}
\usetikzlibrary{arrows.meta}
\usetikzlibrary{decorations.markings}
\usetikzlibrary{calc}
\tikzstyle arrowstyle=[scale=1]
\tikzstyle directed=[postaction={decorate,decoration={markings,
    mark=at position .5 with {\arrow[arrowstyle]{stealth}}}}]
\tikzstyle ray=[directed, thick]
\tikzstyle dot=[anchor=base,fill,circle,inner sep=1pt]

% Algorithm setting
% Julia-style code
\SetKwIF{If}{ElseIf}{Else}{if}{}{elseif}{else}{end}
\SetKwFor{For}{for}{}{end}
\SetKwFor{While}{while}{}{end}
\SetKwProg{Function}{function}{}{end}
\SetArgSty{textnormal}

\newcommand*{\concept}[1]{{\textbf{#1}}}

% Embedded codes
\lstset{basicstyle=\ttfamily,
  showstringspaces=false,
  commentstyle=\color{gray},
  keywordstyle=\color{blue}
}

% Reference formatting
\newcommand*{\citesec}[1]{\S~{#1}}
\newcommand*{\citechap}[1]{chap.~{#1}}
\newcommand*{\citefig}[1]{Fig.~{#1}}
\newcommand*{\citetable}[1]{Table~{#1}}
\newcommand*{\citepage}[1]{pp.~{#1}}
\newrefformat{fig}{Fig.~\ref{#1}}
\newcommand*{\term}[1]{\textit{#1}}

% Color boxes
\tcbuselibrary{skins, breakable, theorems}

\newtcbtheorem{infobox}{Box}{
    enhanced,
    boxrule=0pt,
    colback=blue!5,
    colframe=blue!5,
    coltitle=blue!50,
    borderline west={4pt}{0pt}{blue!65},
    sharp corners,
    fonttitle=\bfseries, 
    breakable,
    before upper={\parindent15pt\noindent}}{box}
\newtcbtheorem[use counter from=infobox]{theorybox}{Box}{
    enhanced,
    boxrule=0pt,
    colback=orange!5, 
    colframe=orange!5, 
    coltitle=orange!50,
    borderline west={4pt}{0pt}{orange!65},
    sharp corners,
    fonttitle=\bfseries, 
    breakable,
    before upper={\parindent15pt\noindent}}{box}
\newtcbtheorem[use counter from=infobox]{learnbox}{Box}{
    enhanced,
    boxrule=0pt,
    colback=green!5,
    colframe=green!5,
    coltitle=green!50,
    borderline west={4pt}{0pt}{green!65},
    sharp corners,
    fonttitle=\bfseries, 
    breakable,
    before upper={\parindent15pt\noindent}}{box}


\newenvironment{shelldisplay}{\begin{lstlisting}}{\end{lstlisting}}

\newcommand*{\kB}{k_{\text{B}}}
\newcommand*{\muB}{\mu_{\text{B}}}
\newcommand*{\efermi}{E_{\text{F}}}
\newcommand*{\pfermi}{p_{\text{F}}}
\newcommand*{\vfermi}{v_{\text{F}}}
\newcommand*{\sA}{\text{A}}
\newcommand*{\sB}{\text{B}}
\newcommand*{\Tc}{T_{\text{c}}}
\newcommand*{\hethree}{$^3$He}
\newcommand*{\hefour}{$^4$He}
\newcommand{\epsr}{\epsilon_{\text{r}}}
\newcommand*{\mur}{\mu_{\text{r}}}
\newcommand{\chie}{\chi_{\text{e}}}
\newcommand*{\Gammae}{\Gamma_{\text{e}}}
\newcommand*{\Gammag}{\Gamma_{\text{g}}}
\newcommand*{\omegae}{\omega_{\text{e}}}
\newcommand*{\omegag}{\omega_{\text{g}}}
\newcommand*{\omegaeg}{\omega_{\text{eg}}}
\newcommand*{\ptwfc}[2]{\psi^{(#2)}_{#1}}
\newcommand*{\mueg}{\mu_{\text{eg}}}
\newcommand*{\muge}{\mu_{\text{ge}}}
\newcommand*{\Ezzero}{E_{z0}}
\newcommand*{\kete}{\ket*{\text{e}}}
\newcommand*{\ketg}{\ket*{\text{g}}}
\newcommand*{\coeffe}{c_{\text{e}}}
\newcommand*{\coeffg}{c_{\text{g}}}
\newcommand*{\pope}{p_{\text{e}}}
\newcommand*{\popg}{p_{\text{g}}}
\newcommand*{\ptwo}{P^{(2)}}
\newcommand*{\vp}{v_{\text{p}}}
\newcommand*{\chitwo}{\chi^{(2)}}
\newcommand*{\chithree}{\chi^{(3)}}
\newcommand*{\omegap}{\omega_{\text{p}}}
\newcommand*{\mvb}[1]{\tilde{\vb*{#1}}}
\newcommand*{\Si}{{S_{\text{i}}}}
\newcommand*{\So}{S_{\text{o}}}
\newcommand*{\taug}{\tau_{\text{g}}}
\newcommand*{\taue}{\tau_{\text{e}}}
\newcommand*{\bi}{b_{\text{i}}}
\newcommand*{\bo}{b_{\text{o}}}
\newcommand*{\bii}[1]{b_{\text{i#1}}}
\newcommand*{\boi}[1]{b_{\text{o#1}}}
\newcommand*{\fn}{f_{\text{N}}} 

\title{Homework 4}
\author{Jinyuan Wu}

\begin{document}

\maketitle


\section{Quantum treatment of a beam splitter}

\textit{In this question, we will examine an important component in optical experiments, the beam splitter. Practically, beam splitters can be implemented using a partially-reflecting mirror, or a directional coupler, which can be a fiber-optical component, or an integrated photonic device, as illustrated below.}

\textit{We have already analyzed such devices in a classical framework in Question ??. However, when treating the beam splitter within the context of quantum-optics, the classical analysis leads to misleading results when considering non-classical states of light, and a full quantum analysis must be carried out. For a full quantum analysis, we replace the classical waves we have used to describe the modes with annihilation operators $\hat{a}_k$, where $k=\{1,2,3,4\}$ (see illustration).}

\textit{Here, we will consider a beam splitter that equally splits power between the output ports (commonly referred to as a 50:50 splitter). As we have seen in HW3, the matrix relating the output modes $\left(\hat{a}_3, \hat{a}_4\right)$ to the input waves $\left(\hat{a}_1, \hat{a}_2\right)$ is not uniquely defined, and for this question, we will choose the form}
$$
\left(\begin{array}{l}
\hat{a}_3 \\
\hat{a}_4
\end{array}\right)=\frac{1}{\sqrt{2}}\left(\begin{array}{ll}
1 & i \\
i & 1
\end{array}\right)\left(\begin{array}{l}
\hat{a}_1 \\
\hat{a}_2
\end{array}\right)
$$

\paragraph*{(a)} \textit{In the Schrödinger picture, we can calculate the state at the output for a given input state. Find the output state, given that at the beam splitter input there is a single photon in one of the ports $\mid$ in $\rangle=|1\rangle_1|0\rangle_2$. What is the probability of finding the photon at each of the output ports?} 

The equation 
\begin{equation}
    \pmqty{a_3 \\ a_4} = \frac{1}{\sqrt{2}} \pmqty{1 & \ii \\ \ii & 1} \pmqty{a_1 \\ a_2} 
\end{equation}
gives the time evolution of annihilation operators in the Heisenberg picture.
Since we are working in the Schrödinger picture, the matrix 
\begin{equation}
    S = \frac{1}{\sqrt{2}} \pmqty{1 & \ii \\ \ii & 1}
\end{equation}
should be understood as the $S$-matrix of a single photon.
Thus 
\[
    S \pmqty{1 \\ 0} = \frac{1}{\sqrt{2}} \pmqty{1 \\ \ii} \Rightarrow 
    S \ket*{1}_1 = \frac{1}{\sqrt{2}} (\ket*{1}_3 + \ii \ket*{1}_4) .
\]
Since there is no photon on mode 2, the time evolution there should be trivial.
Thus the output state at $t = \infty$ corresponding to $\ket*{\text{in}}$ should be 
\begin{equation}
    \ket*{\text{out}} = \frac{1}{\sqrt{2}} (\ket*{1}_3 + \ii \ket*{1}_4)
    = \frac{1}{\sqrt{2}} (\ket*{1}_3 \ket*{0}_4 + \ii \ket*{0}_3 \ket*{1}_4).
\end{equation}
The probabilities to find the photon at port 3 and port 4 are 
\begin{equation}
    P_3 = \frac{1}{2}, \quad P_4 = \frac{1}{2}.
\end{equation} 

This result can also be found in the Heisenberg picture.
From the relation between $a_{3,4}$ and $a_{1,2}$, we find 
\[
    a_1 = \frac{1}{\sqrt{2}} (a_3 - \ii a_4),
\]
and therefore 
\begin{equation}
    \ket*{\Psi} = a_1^\dag \ket*{0} = \frac{1}{\sqrt{2}} (a^\dag_3 + \ii a^\dag_4) \ket*{0}.
\end{equation}

\paragraph*{(b)} \textit{Next, we consider the case of two photons at one of the input ports $\mid$ in $\rangle=|2\rangle_1|0\rangle_2$. Find the output state, and calculate the probabilities of finding zero, one, and two photons at each output port.} 

It's much more convenient to work in the Heisenberg picture and the Fock representation 
for multi-photon cases.
The many-body wave function, in the output basis, is 
\begin{equation}
    \begin{aligned}
        \ket*{\Psi} &= \frac{(a_1^\dag)^2}{\sqrt{2}} \ket*{0} 
        = \frac{1}{2 \sqrt{2}} ((a_3^\dag)^2 + 2 \ii a_3^\dag a_4^\dag - (a_4^\dag)^2 ) \ket*{0} \\
        &= \frac{1}{2} \ket*{2}_3 \ket*{0}_4 + \frac{\ii}{\sqrt{2}} \ket*{1}_3 \ket*{1}_4 - \frac{1}{2} \ket*{0}_3 \ket*{2}_4.
    \end{aligned}
\end{equation} 
$P(n_i = N)$, probabilities to find $N$ photons at port $i$, 
regardless of what's happening at the other port, are given as (this is a projective measurement)
\begin{equation}
    \begin{aligned}
        P(n_3=2) = \frac{1}{4}, \quad P(n_3=1) = \frac{1}{2}, \quad P(n_3=0) = \frac{1}{4}, \\ 
        P(n_4=2) = \frac{1}{4}, \quad P(n_4=1) = \frac{1}{2}, \quad P(n_4=1) = \frac{1}{4}.
    \end{aligned}
\end{equation}

\paragraph*{(c)} \textit{Now, consider the case of one photon at each of the input ports $\mid$ in $\rangle=|1\rangle_1|1\rangle_2$. Find the output state, and calculate the probabilities of finding zero, one or two photons at each output port.} 

From the relation between $a_{3,4}$ and $a_{1,2}$ we can also find 
\begin{equation}
    a_2 = \frac{1}{\sqrt{2}} (- \ii a_3 + a_4) ,
\end{equation}
and the many-body wave function now is 
\begin{equation}
    \begin{aligned}
        \ket*{\Psi} &= a_1^\dag a_2^\dag \ket*{0}
        = \frac{1}{2} (a_3 + \ii a_4^\dag) (\ii a_3^\dag + a_4) \\
        &= \frac{\ii}{\sqrt{2}} (\ket*{2}_3 \ket*{0}_4 + \ket*{0}_3 \ket*{2}_4).
    \end{aligned}
\end{equation}
Therefore
\begin{equation}
    \begin{aligned}
        P(n_3=2) = \frac{1}{2}, \quad P(n_3=1) = 0, \quad P(n_3=0) = \frac{1}{2}, \\
        P(n_4=2) = \frac{1}{2}, \quad P(n_4=1) = 0, \quad P(n_4=0) = \frac{1}{2}. \\
    \end{aligned}
\end{equation}

\paragraph*{(d)} \textit{How do the results we have obtained compare to the ones expected from a classical picture, where the beam splitter simply splits the input power?} 

The one-photon case is not very different from the classical picture:
indeed the probabilities to find one photon at port 3 and port 4 are both \SI{50}{\percent},
and therefore the input power is split equally between the two ports. 
This is expected because in this example we have no photon-photon interaction 
and the EOMs of single-photon quantities are determined by linear Maxwell equations,
whose form stays the same regardless of whether the system is quantum.

In the multi-photon case, the power is still split equally between the two ports.
The main difference from the classical case 
is the \emph{distribution} of the photon number at each port 
is changed after the photons interacting with the beam splitter.


\section{Noise and correlations}

\subsection{Noise in resonators}

\textit{In class, we used the we used correlation functions to analyze a resonant system, with mode amplitude $a$, coupled to a bath (with input and output noise fields denoted $b_i$ and $b_o$, respectively) and a coupling rate $1 / \tau_e$. Here, we will look at the noise characteristics of the system in both the frequency and time domains, based on our discussion from Lecture 18.}

\paragraph*{(a)} \textit{Use the frequency-domain representation of the temporal coupled-mode equations that describe the amplitude decay and the corresponding noise fields to calculate the total energy contained in mode $a$ at thermal equilibrium. [Hint: You can do all calculations in the frequency domain. The energy can be expressed as integrals over frequency, and you should know the two-frequency correlation function of the input bath modes.]} 

The EOM of mode $a$ is  
\begin{equation}
    \dot{a} = - \ii \omega_0 a - \frac{a}{\taue} + \sqrt{\frac{2}{\taue}} \bi,
\end{equation}
or in other words, in the frequency domain,
\begin{equation}
    a(\omega) = \frac{1}{- \ii \omega + \ii \omega_0 + 1 / \taue} \sqrt{\frac{2}{\taue}} \bi(\omega).
\end{equation}
The normalization conditions of $a(t)$ and $\bi(t)$ that relate $a$ and $\bi$ to energy and power 
and the normalization conditions of $a(\omega)$ and $\bi(\omega)$ in Fourier transforms are 
\begin{equation}
    U_a(t) = \abs*{a(t)}^2, \quad 
    P_{\text{input}} = \expval*{\abs*{\bi(t)}^2} = \int \dd{\omega} C_{\bi}(\omega), \quad 
    C_{\bi}(\omega) = \frac{1}{\delta(0)} \expval*{\bi^*(\omega) \bi(\omega)},
    \label{eq:bi-normalization}
\end{equation}
and for a one-dimensional transmission line when $\kB T \gg \hbar \omega$, we have 
\begin{equation}
    \dv{\expval*{P_{\text{input}}}}{\omega} = \frac{\kB T}{2 \pi}.
\end{equation}
Therefore 
\begin{equation}
    \expval*{\bi^*(\omega) \bi(\omega')} = \frac{\kB T}{2\pi} \delta(\omega - \omega'),
\end{equation} 
and 
\begin{equation}
    \begin{aligned}
        \expval*{a^*(\omega) a(\omega')} &= 
        \frac{1}{- \ii \omega + \ii \omega_0 + 1 / \taue} 
        \frac{1}{  \ii \omega - \ii \omega_0 + 1 / \taue} 
        \frac{2}{\taue} \frac{\kB T}{2\pi} \delta(\omega - \omega') \\
        &= \frac{\kB T}{\taue \pi} \frac{1}{(\omega - \omega_0)^2 + 1 / \taue^2} \delta(\omega - \omega').
    \end{aligned}
    \label{eq:a-correlation}
\end{equation}
Therefore the energy stored in mode $a$ is 
\begin{equation}
    \expval{U_a} = \int \dd{\omega} \frac{\expval*{\abs*{a(\omega)}^2}}{\delta(0)}
    = \kB T \int \dd{\omega} \frac{1}{\taue \pi} \frac{1}{(\omega - \omega_0)^2 + 1 / \taue^2}
    = \kB T.
    \label{eq:single-mode-energy}
\end{equation}

\paragraph*{(b)} \textit{Using the expressions you found in part (a), calculate the power spectrum of the output field $b_o$. [Hint: Your result should be consistent with the fact that the system is in thermal equilibrium.]} 

The output field is
\begin{equation}
    \bo = - \bi + \sqrt{\frac{2}{\taue}} a,
\end{equation}
and in the frequency domain we have 
\begin{equation}
    \begin{aligned}
        \expval*{\bo^*(\omega) \bo(\omega')} &= 
        \expval*{\bi^*(\omega) \bi(\omega')}
        - \sqrt{\frac{2}{\taue}} \expval*{\bi^*(\omega) a(\omega')}
        - \sqrt{\frac{2}{\taue}} \expval*{a^*(\omega) \bi(\omega')}
        + \frac{2}{\tau} \expval*{a^*(\omega) a(\omega')} \\
        &= \kB T \delta(\omega - \omega')
        \left(
            \frac{2}{\taue^2 \pi} \frac{1}{(\omega - \omega_0)^2 + 1 / \taue^2}
            - \frac{1}{\pi \taue} \frac{1}{- \ii \omega_0 + \ii \omega + 1 / \taue} + \text{c.c.}
            + \frac{1}{2\pi} 
        \right) \\
        &= \frac{\kB T}{2\pi} \delta(\omega - \omega').
    \end{aligned}
\end{equation}
So it can be seen that the spectrum of $\bo$ is exactly the same as that of $\bi$,
and that's expected since the system is in thermal equilibrium.

\paragraph*{(c)} \textit{Next, we analyze the same system in the time domain. Using the temporal coupled-mode equations in the time domain, find the particular solution that permits you to express the mode $(a)$ in terms of the incoming noise field $\left(b_i\right)$.} 

Formally we write 
\begin{equation}
    a(t) = \int_{t_0}^{t} \sqrt{\frac{2}{\taue}} \bi(t') 
    \ee^{\left(\ii \omega_0 + \frac{1}{\taue}\right) (t' - t)} \dd{t'}
\end{equation}
The Fourier transform normalization convention chosen here is 
\begin{equation}
    \bi(t) = \int \dd{\omega} \ee^{- \ii \omega t} \bi(\omega),
\end{equation}
which can be verified to be consistent with \eqref{eq:bi-normalization}.

\paragraph*{(d)} \textit{Use your result from part (c) in conjunction with the correlation properties of the bath fields discussed in Lecture 18 to calculate the steady state energy contained in the mode. [Hint: this should be consistent with part (a) of this question.]} 

The temporal correlation function of $\bi$ is therefore
\begin{equation}
    \begin{aligned}
        \expval{\bi^*(t) \bi(t')} &= \int \dd{\omega} \int \dd{\omega'}
        \expval{\bi^*(\omega) \bi(\omega')} \ee^{\ii \omega t - \ii \omega' t'} \\
        &= \int \dd{\omega} \int \dd{\omega'}
        \frac{\kB T}{2\pi} \delta(\omega - \omega') \ee^{\ii \omega t - \ii \omega' t'} \\
        &= \kB T \delta(t - t').
    \end{aligned}
    \label{eq:si-temporal}
\end{equation} 
The energy stored in the system therefore is 
\begin{equation}
    \begin{aligned}
        \expval*{\abs*{a(t)}^2} &= \frac{2}{\taue} \int_{-\infty}^{t} \dd{t'} \int_{-\infty}^{t} \dd{t''}
        \ee^{\left( - \ii \omega_0 + \frac{1}{\taue} \right) (t' - t)} \ee^{\left( \ii \omega_0 + \frac{1}{\taue} \right) (t'' - t)}
        \expval*{\bi^*(t') \bi(t'')}  \\
        &= \frac{2\kB T}{\taue} \int_{-\infty}^{t} \dd{t'} \ee^{\left(- \ii \omega_0 + \frac{1}{\taue}\right) (t' - t)} \ee^{\left( \ii \omega_0 + \frac{1}{\taue} \right) (t' - t)} \\
        &= \kB T,
    \end{aligned}
\end{equation} 
which is exactly \eqref{eq:single-mode-energy}.

\subsection{}

\textit{An atomic force microscope (AFM) uses the vibrational modes of cantilever to image atoms and molecules on material interfaces. As a sharp tip (radius $\sim 10 \mathrm{~nm}$ ) at the end of the cantilever interacts with a material substrate (via Van der Waals or other inter-molecular forces) the response of the cantilever is modified/perturbed, permitting us to form exquisite images of atoms and molecules a surfaces via force-imaging. Since thermo-mechanical noise poses a fundamental limitation the force sensitivity and spatial resolution of such AFM systems, it is typically necessary to operate an AFM at cryogenic temperatures to obtain images of individual atoms. In this problem, we examine the thermo-mechanical noise characteristics of a typical AFM cantilever.}

\textit{Since AFM cantilevers typically have high quality factors (e.g., $Q \sim 10,000$ ), we can use temporal coupled mode theory to cast the equation of motion for our cantilever in the form $\dot{a}=-i \omega_o a-a / \tau+\sqrt{2 / \tau} S_i$. Here, $a$ is the complex mode amplitude, $(1 / \tau)$ is the damping rate, and $S_i(t)$ is the thermal noise driving term (or Langevin term) that arises from coupling to the environment (or the 'thermal bath').}

\paragraph*{(a)} \textit{Assuming that the motion of our cantilever follows the dynamics of a driven simple harmonic oscillator with coordinate, $q$, resonance frequency, $\omega_o$, spring constant, $k$, quality factor, $Q$, and driving force $f_n(t)$, use the rotating wave approximation to cast our equation of motion in the form $\dot{a}=-i \omega_o a-a / \tau+\sqrt{2 / \tau} S_i$. Be sure to relate the parameters of our coupled-mode equation [i.e., $(1 / \tau), a$, and $S_i$ ] to those of the harmonic oscillator.} 
\textit{[Hint: From power decay, we know that $(2 / \tau)=\left(\omega_o / Q\right)$; see Section 7.2 of HAH1 for further details.]}

The EOM of the cantilever is 
\begin{equation}
    \dot{x} = \frac{p}{m}, \quad \dot{p} = - m \omega_0^2 x - \gamma \frac{p}{m} + \fn(t).
\end{equation}
The relation between $x, p$ and the mode amplitude $a, a^*$ is 
\begin{equation}
    a = \sqrt{\frac{m \omega_0}{2}} x + \frac{\ii}{\sqrt{2 m \omega_0}} p, \quad 
    x = \frac{1}{\sqrt{2 m \omega_0}} (a + a^*), \quad 
    p = - \ii \sqrt{\frac{m \omega_0}{2}} (a - a^*),
    \label{eq:x-p-newton}
\end{equation}
and therefore the EOM of $a$ is found to be 
\begin{equation}
    \dot{a} = - \ii \omega_0 a - \frac{\gamma}{2 m} (a - a^*) + \frac{\ii}{\sqrt{2 m \omega_0}} \fn(t).
\end{equation}
Under the RWA, the $a^*$ term contains a $\ee^{2 \ii \omega t}$ factor and can be eliminated after time averaging,
so the approximate EOM after RWA is 
\begin{equation}
    \dot{a} = - \ii \omega_0 a - \frac{\gamma}{2m} a + \frac{\ii}{\sqrt{2 m \omega_0}} \fn(t).
\end{equation}
To make sure we have the energy normalization condition $U_a = \abs*{a}^2$,
we multiply the equation by $\sqrt{\omega_0}$ and redefine $\sqrt{\omega_0} a$ by $a$;
under the new definition, 
\begin{equation}
    a = \sqrt{\omega_0} \cdot \left(
        \sqrt{\frac{m \omega_0}{2}} x + \frac{\ii}{\sqrt{2 m \omega_0}} p
    \right).
\end{equation}
We further define
\begin{equation}
    \frac{1}{\tau} = \frac{\gamma}{2m} = \frac{\omega_0}{2Q}, \quad 
    \sqrt{\frac{2}{\tau}} \Si =  \sqrt{\omega_0} \cdot \frac{\ii}{\sqrt{2 m \omega_0}} \fn(t) 
    \Rightarrow \Si = \frac{\ii}{\sqrt{2 \gamma}} \fn ,
\end{equation}
and eventually we have 
\begin{equation}
    \dot{a} = - \ii \omega_0 a - \frac{a}{\tau} + \Si.
\end{equation}

\paragraph*{(b)} \textit{Next, use the results from Problem 2.1 to find an expression for the energy spectrum of the cantilever.} 

We have 
\begin{equation}
    C_{a}(\omega) = \frac{\kB T}{\tau \pi} \frac{1}{(\omega - \omega_0)^2 + 1 / \tau^2}.
\end{equation}

\paragraph*{(c)} \textit{Find a simple expression for the root mean square (RMS) displacement of the cantilever.} 
\textit{[Hint: All you need is the total energy, $\left\langle E_{t o t}\right\rangle=\langle T\rangle+\langle U\rangle$, stored in your mode. In other words, since the Virial Theorem tells us that $\langle T\rangle=\langle U\rangle$ we have the tidy result $\left\langle E_{t o t}\right\rangle=2\langle T\rangle$, which greatly simplifies the problem.]}

We have already known that the total energy for a mode governed by Langevin equation is $\kB T$.
From Virial's theorem we have 
\begin{equation}
    \kB T = \expval{E_{\text{total}}} = 2 \expval{U} = m \omega_0^2 \expval*{x^2} \Rightarrow
    \sqrt{\expval*{x^2}} = \sqrt{\frac{\kB T}{m \omega_0^2}}.
\end{equation}

\paragraph*{(d)} \textit{Assuming that our cantilever has a resonance frequency of $\omega_o /(2 \pi)=100 \mathrm{kHz}$, a quality factor of $Q=10,000$, and a spring constant of $k=10$ Newtons/meter, find the RMS displacement of the cantilever at room temperature $(300 \mathrm{~K})$.} 
\textit{[Hint: Your result from part (b) should be independent of Q-factor.]}

The RMS displacement is now 
\begin{equation}
    \sqrt{\expval*{x^2}} = \sqrt{\frac{\kB T}{k}} = \SI{2e-11}{m} = \SI{0.2}{\angstrom}.
\end{equation}

\paragraph*{(e)} \textit{Next, find the correlation function for the thermal noise, $f_n(t)$, that drives the motion of the cantilever.} 
\textit{[Hint: Feel free to map all of the results we have derived using coupled mode theory onto the simple harmonic oscillator. In this case, we need only find the relation between $S_i$ and the thermal driving force to find the correlation function. ]}

The correlation function of the Langevin noise in \eqref{eq:x-p-newton} is also $\propto \delta(t-t')$,
and therefore so is $\Si$,
and we can just reuse the relation \eqref{eq:si-temporal}, which reads
\begin{equation}
    \expval{\Si^*(t) \Si(t')} = \kB T \delta(t - t')
\end{equation} 
in our problem. From this we find 
\[
    \frac{1}{2 \gamma} \expval{\fn^*(t) \fn(t')} = \kB T \delta(t - t')
\]  
and therefore
\begin{equation}
    \expval{\fn^*(t) \fn(t')} = 2 \gamma \kB T \delta(t - t'),
    \label{eq:correlation-f}
\end{equation}
which is exactly the usual form of the correlation function of the Langevin force 
in a system with a dissipation force $- \gamma v$.

\paragraph*{(f)} \textit{Estimate the RMS thermal force that drives the cantilever by integrating the power spectral density of the force over its mechanical resonance.} 
\textit{[Hint: From part (d) you should have found that the thermal driving force is delta-function correlated. Hence it's spectrum is white. Here, I'm asking you to estimate the RMS thermal force that drives the system over the bandwidth of the resonator.]}

When $t = t'$, \eqref{eq:correlation-f} diverges, indicating that its frequency-domain form needs a cutoff.
Suppose that the frequency-domain correlation function of $\Si$ 
\begin{equation}
    \expval{\Si^*(\omega) \Si(\omega')} = \frac{\kB T}{2\pi} \delta(\omega - \omega')
\end{equation}
is only valid in a bandwidth $\Delta \omega$, we have  
\begin{equation}
    \expval*{\abs*{\Si}^2} = \int_{\text{bandwidth}} \dd{\omega} \frac{\kB T}{2\pi} 
    = \frac{\kB T}{2\pi} \Delta \omega,
\end{equation}
and therefore 
\begin{equation}
    \frac{1}{2\gamma} \expval*{\abs*{\fn}^2} = \frac{\kB T}{2\pi} \Delta \omega
    \quad \Rightarrow \quad
    \sqrt{\expval*{\abs*{\fn}^2}} = \sqrt{\frac{\kB T \gamma}{\pi} \Delta \omega}.
\end{equation}
This can also be derived from \eqref{eq:correlation-f}, as 
\[
    \delta(0) = \int \frac{\dd{\omega}}{2\pi} = \frac{\Delta \omega}{2\pi}
\]
and therefore 
\begin{equation}
    \expval*{\abs*{\fn}^2} = \frac{\gamma \kB T \Delta \omega}{\pi}.
\end{equation}

\subsection{A mode connected to two baths}

\textit{Next, we consider a single mode that is coupled to different baths, with distinct coupling strengths $\kappa_1=\sqrt{2 / \tau_1}$ and $\kappa_2=\sqrt{2 / \tau_2}$, as seen in Fig. 2. We assume that Bath 1 and Bath 2 are held at fixed temperatures $T_1$ and $T_2$, respectively. Hence, distinct noise power spectral densities impinge on our resonator from each bath (port) of the system.}

The complete Langevin equation is  
\begin{equation}
    \dot{a} = - \ii \omega_0 a - \left( \frac{1}{\tau_1} + \frac{1}{\tau_2} \right) a + \kappa_1 \bii{1} + \kappa_2 \bii{2}, \quad  
    \expval*{\bii{$i$}^*(t) \bii{$j$}(t')} = \kB T_i \delta(t - t').
\end{equation}

\paragraph*{(a)} \textit{First, let's assume that $\kappa_2=0$. In other words, we begin by assuming that Bath 2 is decoupled from the mode. In this case, only Bath 1 injects noise into the system. Find the correlation function for the mode, $C_a(\tau)=\left\langle a^*(t) a(t+\tau)\right\rangle$, and the power spectral density $C_a[\omega]$.} 

When $\kappa_2 = 0$, the Langevin equation of $a$ is just the simplest one mode-one port one,
and by reusing \eqref{eq:a-correlation}, we have 
\begin{equation}
    \expval{a^*(\omega) a(\omega)} = \frac{\kB T_1}{\tau_1 \pi} \frac{1}{(\omega - \omega_0)^2 + 1 / \tau_1^2} \delta(\omega - \omega'),
\end{equation}
and therefore
\begin{equation}
    C_a(\omega) = \frac{\kB T_1}{\tau_1 \pi} \frac{1}{(\omega - \omega_0)^2 + 1 / \tau_1^2}.
\end{equation}
The temporal correlation function is 
\begin{equation}
    \expval{a^*(t) a(t + \tau)} = \int \ee^{- \ii \omega \tau} C_a(\omega) \dd{\omega} = 
    \kB T_1 \ee^{- \ii \omega_0 \tau} \ee^{- \frac{\abs*{\tau}}{\tau_1}}.
\end{equation}

\paragraph*{(b)} \textit{Integrate the power spectral density to find the total energy in the mode.} 

THe total energy in the mode is 
\begin{equation}
    \expval*{\abs*{a(t)}^2} = \kB T_1.
\end{equation}

\paragraph*{(c)} \textit{At steady state, how much power flows into and out of the mode?} 

The $\bi$ and $\bo$ fields are normalized so that $P_{\text{in}} = \abs*{\bi}^2$ 
and $P_{\text{out}} = \abs*{\bo}^2$.
Since the temperature is fixed, the input power is 
\begin{equation}
    P_{\text{in,1}} = \abs*{\bii{1}}^2 = \kB T_1 \delta(0) = \kB T_1 \frac{\Delta \omega_1}{2\pi},
\end{equation}
where the singularity at $t = t'$ needs to be regularized by 
the bandwidth cutoff $\Delta \omega_1$ of port 1.
Since port 2 is essentially detached from the mode, energy can only flow back to port 1, so 
\begin{equation}
    P_{\text{out,1}} = \kB T_1 \frac{\Delta \omega_1}{2\pi}.
\end{equation} 

\paragraph*{(d)} Next, let's assume that $\kappa_2 \neq 0$, meaning that Bath 2 couples to the system. For simplicity, we also assume that $T_2=0$. In this special case, noise will never flow from Bath 2 into the mode. Find the steady-state powers flowing into and out of the mode.

The input powers never change, because they only depend on the temperature.
Therefore we use the results of the last question and write 
\begin{equation}
    P_{\text{in,1}} = \expval*{\abs*{\bii{1}}^2} = \kB T_1 \frac{\Delta \omega_1}{2\pi}, \quad 
    P_{\text{in,2}} = \expval*{\abs*{\bii{2}}^2} = \kB T_2 \frac{\Delta \omega_2}{2\pi}.
\end{equation} 
When $T_2 = 0$,
\begin{equation}
    P_{\text{in,1}} = \kB T_1 \frac{\Delta \omega_1}{2\pi}, \quad 
    P_{\text{in,2}} = 0.
\end{equation}

The output fields 
\begin{equation}
    \boi{$i$} = - \bii{$i$} + \kappa_i a
\end{equation}
depends on 
\begin{equation}
    a(t) = \int_{t_0}^{t} \ee^{\left( \ii \omega_0 + \frac{1}{\tau_1} + \frac{1}{\tau_2} \right) (t' - t)} (\kappa_1 \bii{1}(t') + \kappa_2 \bii{2}(t')) \dd{t'}.
    \label{eq:a-eom-from-b1-b2}
\end{equation}
By repeating the procedure in Problem 2.1(c), we have 
\begin{equation}
    \expval*{\abs*{a}^2} = \frac{\kappa_1^2 \kB T_1 + \kappa_2^2 \kB T_2}{2 \left( \frac{1}{\tau_1} + \frac{1}{\tau_2} \right)},
    \label{eq:a-energy-double-port}
\end{equation}
and 
\begin{equation}
    \expval*{\bii{$i$}^* a} = \kappa_i \kB T_i.
\end{equation}
Assuming $T_2 = 0$, we have 
\begin{equation}
    P_{\text{out,1}} = \frac{\kB T_1}{1 + \tau_1 / \tau_2} \frac{2}{\tau_1} + \kB T_1 \frac{\Delta \omega_1}{2\pi} - \frac{4}{\tau_1} \kB T_1,
\end{equation}
and 
\begin{equation}
    P_{\text{out,2}} = \frac{\kB T_1}{1 + \tau_1 / \tau_2} \frac{2}{\tau_2} .
\end{equation}

\paragraph*{(e)} What is the steady-state mode energy? (i.e., I'm asking you to apply detailed balance.)

The steady-state mode energy is \eqref{eq:a-energy-double-port}.
We can see $P_{\text{in,1}} = P_{\text{out,1}} + P_{\text{out,2}}$,
so detailed balance is achieved.
This is because in \eqref{eq:a-eom-from-b1-b2},
we have ignored any initial state information of $a(t)$,
which implicitly assumes that $a$ is in equilibrium with the bath.

\paragraph*{(f)} Use the equipartition theorem to identify the effective temperature of the mode. [Hint: Remember to apply a sanity check to your answers by taking the limit as $\kappa_1 \rightarrow 0$; in this case, the temperature should approach $T_2$ (and vice versa).]

The effective temperature of the mode can be intuitively defined as 
\begin{equation}
    \kB T_{\text{eff}} = \expval*{\abs*{a}^2} = \frac{\kappa_1^2 \kB T_1 + \kappa_2^2 \kB T_2}{2 \left( \frac{1}{\tau_1} + \frac{1}{\tau_2} \right)}
\end{equation}
and therefore 
\begin{equation}
    T_{\text{eff}} = \frac{\frac{T_1}{\tau_1} + \frac{T_2}{\tau_2}}{\frac{1}{\tau_1} + \frac{1}{\tau_2}}.
\end{equation}
When $\kappa_1 \to 0$ and therefore $1/\tau_1 \to 0$, we find $T_{\text{eff}} \to T_2$,
and similarly when $\kappa_2 \to 0$, $T_{\text{eff}} \to T_1$.
This result holds regardless of the relative magnitude of $T_1$ and $T_2$;
this means even when $T_1 \ll T_2$, as long as $1/\tau_1 \gg 1/\tau_2$,
the temperature of the mode approaches $T_1$.
This is why laser cooling works. 

\end{document}