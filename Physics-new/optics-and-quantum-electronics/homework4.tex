\documentclass[hyperref, a4paper]{article}

\usepackage{geometry}
\usepackage{titling}
\usepackage{titlesec}
% No longer needed, since we will use enumitem package
% \usepackage{paralist}
\usepackage{enumitem}
\usepackage{footnote}
\usepackage[colorinlistoftodos]{todonotes}
\usepackage{amsmath, amssymb, amsthm}
\usepackage{mathtools}
\usepackage{bbm}
\usepackage{graphicx}
\usepackage{subcaption}
\usepackage{soulutf8}
\usepackage{physics}
\usepackage{tensor}
\usepackage{siunitx}
\usepackage[version=4]{mhchem}
\usepackage{tikz}
\usepackage{xcolor}
\usepackage{listings}
\usepackage{autobreak}
\usepackage[ruled, vlined, linesnumbered]{algorithm2e}
\usepackage{nameref,zref-xr}
\zxrsetup{toltxlabel}
\usepackage[backend=bibtex]{biblatex}
\addbibresource{elasticity.bib}
\usepackage[colorlinks,unicode]{hyperref} % , linkcolor=black, anchorcolor=black, citecolor=black, urlcolor=black, filecolor=black
\usepackage[most]{tcolorbox}
\usepackage{prettyref}

% Page style
\geometry{left=3.18cm,right=3.18cm,top=2.54cm,bottom=2.54cm}
\titlespacing{\paragraph}{0pt}{1pt}{10pt}[20pt]
\setlength{\droptitle}{-5em}

% More compact lists 
\setlist[itemize]{
    itemindent=17pt, 
    leftmargin=1pt,
    listparindent=\parindent,
    parsep=0pt,
}

% Math operators
\DeclareMathOperator{\timeorder}{\mathcal{T}}
\DeclareMathOperator{\diag}{diag}
\DeclareMathOperator{\legpoly}{P}
\DeclareMathOperator{\primevalue}{P}
\DeclareMathOperator{\sgn}{sgn}
\DeclareMathOperator{\res}{Res}
\DeclareMathOperator{\sinc}{sinc}
\newcommand*{\ii}{\mathrm{i}}
\newcommand*{\ee}{\mathrm{e}}
\newcommand*{\const}{\mathrm{const}}
\newcommand*{\suchthat}{\quad \text{s.t.} \quad}
\newcommand*{\argmin}{\arg\min}
\newcommand*{\argmax}{\arg\max}
\newcommand*{\normalorder}[1]{: #1 :}
\newcommand*{\pair}[1]{\langle #1 \rangle}
\newcommand*{\fd}[1]{\mathcal{D} #1}
\DeclareMathOperator{\bigO}{\mathcal{O}}

% TikZ setting
\usetikzlibrary{arrows,shapes,positioning}
\usetikzlibrary{arrows.meta}
\usetikzlibrary{decorations.markings}
\usetikzlibrary{calc}
\tikzstyle arrowstyle=[scale=1]
\tikzstyle directed=[postaction={decorate,decoration={markings,
    mark=at position .5 with {\arrow[arrowstyle]{stealth}}}}]
\tikzstyle ray=[directed, thick]
\tikzstyle dot=[anchor=base,fill,circle,inner sep=1pt]

% Algorithm setting
% Julia-style code
\SetKwIF{If}{ElseIf}{Else}{if}{}{elseif}{else}{end}
\SetKwFor{For}{for}{}{end}
\SetKwFor{While}{while}{}{end}
\SetKwProg{Function}{function}{}{end}
\SetArgSty{textnormal}

\newcommand*{\concept}[1]{{\textbf{#1}}}

% Embedded codes
\lstset{basicstyle=\ttfamily,
  showstringspaces=false,
  commentstyle=\color{gray},
  keywordstyle=\color{blue}
}

% Reference formatting
\newcommand*{\citesec}[1]{\S~{#1}}
\newcommand*{\citechap}[1]{chap.~{#1}}
\newcommand*{\citefig}[1]{Fig.~{#1}}
\newcommand*{\citetable}[1]{Table~{#1}}
\newcommand*{\citepage}[1]{pp.~{#1}}
\newrefformat{fig}{Fig.~\ref{#1}}
\newcommand*{\term}[1]{\textit{#1}}

% Color boxes
\tcbuselibrary{skins, breakable, theorems}

\newtcbtheorem{infobox}{Box}{
    enhanced,
    boxrule=0pt,
    colback=blue!5,
    colframe=blue!5,
    coltitle=blue!50,
    borderline west={4pt}{0pt}{blue!65},
    sharp corners,
    fonttitle=\bfseries, 
    breakable,
    before upper={\parindent15pt\noindent}}{box}
\newtcbtheorem[use counter from=infobox]{theorybox}{Box}{
    enhanced,
    boxrule=0pt,
    colback=orange!5, 
    colframe=orange!5, 
    coltitle=orange!50,
    borderline west={4pt}{0pt}{orange!65},
    sharp corners,
    fonttitle=\bfseries, 
    breakable,
    before upper={\parindent15pt\noindent}}{box}
\newtcbtheorem[use counter from=infobox]{learnbox}{Box}{
    enhanced,
    boxrule=0pt,
    colback=green!5,
    colframe=green!5,
    coltitle=green!50,
    borderline west={4pt}{0pt}{green!65},
    sharp corners,
    fonttitle=\bfseries, 
    breakable,
    before upper={\parindent15pt\noindent}}{box}


\newenvironment{shelldisplay}{\begin{lstlisting}}{\end{lstlisting}}

\newcommand*{\kB}{k_{\text{B}}}
\newcommand*{\muB}{\mu_{\text{B}}}
\newcommand*{\efermi}{E_{\text{F}}}
\newcommand*{\pfermi}{p_{\text{F}}}
\newcommand*{\vfermi}{v_{\text{F}}}
\newcommand*{\sA}{\text{A}}
\newcommand*{\sB}{\text{B}}
\newcommand*{\Tc}{T_{\text{c}}}
\newcommand*{\hethree}{$^3$He}
\newcommand*{\hefour}{$^4$He}
\newcommand{\epsr}{\epsilon_{\text{r}}}
\newcommand*{\mur}{\mu_{\text{r}}}
\newcommand{\chie}{\chi_{\text{e}}}
\newcommand*{\Gammae}{\Gamma_{\text{e}}}
\newcommand*{\Gammag}{\Gamma_{\text{g}}}
\newcommand*{\omegae}{\omega_{\text{e}}}
\newcommand*{\omegag}{\omega_{\text{g}}}
\newcommand*{\omegaeg}{\omega_{\text{eg}}}
\newcommand*{\ptwfc}[2]{\psi^{(#2)}_{#1}}
\newcommand*{\mueg}{\mu_{\text{eg}}}
\newcommand*{\muge}{\mu_{\text{ge}}}
\newcommand*{\Ezzero}{E_{z0}}
\newcommand*{\kete}{\ket*{\text{e}}}
\newcommand*{\ketg}{\ket*{\text{g}}}
\newcommand*{\coeffe}{c_{\text{e}}}
\newcommand*{\coeffg}{c_{\text{g}}}
\newcommand*{\pope}{p_{\text{e}}}
\newcommand*{\popg}{p_{\text{g}}}
\newcommand*{\ptwo}{P^{(2)}}
\newcommand*{\vp}{v_{\text{p}}}
\newcommand*{\chitwo}{\chi^{(2)}}
\newcommand*{\chithree}{\chi^{(3)}}
\newcommand*{\omegap}{\omega_{\text{p}}}
\newcommand*{\mvb}[1]{\tilde{\vb*{#1}}}
\newcommand*{\Si}[1]{S_{\text{i#1}}}
\newcommand*{\So}[1]{S_{\text{o#1}}}
\newcommand*{\taug}{\tau_{\text{g}}}

\title{Homework 4}
\author{Jinyuan Wu}

\begin{document}

\maketitle

\section{}

\section{Quantum treatment of a beam splitter}

In this question, we will examine an important component in optical experiments, the beam splitter. Practically, beam splitters can be implemented using a partially-reflecting mirror, or a directional coupler, which can be a fiber-optical component, or an integrated photonic device, as illustrated below.

We have already analyzed such devices in a classical framework in Question ??. However, when treating the beam splitter within the context of quantum-optics, the classical analysis leads to misleading results when considering non-classical states of light, and a full quantum analysis must be carried out. For a full quantum analysis, we replace the classical waves we have used to describe the modes with annihilation operators $\hat{a}_k$, where $k=\{1,2,3,4\}$ (see illustration).

Here, we will consider a beam splitter that equally splits power between the output ports (commonly referred to as a 50:50 splitter). As we have seen in HW3, the matrix relating the output modes $\left(\hat{a}_3, \hat{a}_4\right)$ to the input waves $\left(\hat{a}_1, \hat{a}_2\right)$ is not uniquely defined, and for this question, we will choose the form
$$
\left(\begin{array}{l}
\hat{a}_3 \\
\hat{a}_4
\end{array}\right)=\frac{1}{\sqrt{2}}\left(\begin{array}{ll}
1 & i \\
i & 1
\end{array}\right)\left(\begin{array}{l}
\hat{a}_1 \\
\hat{a}_2
\end{array}\right)
$$

\paragraph*{(a)} In the Schrödinger picture, we can calculate the state at the output for a given input state. Find the output state, given that at the beam splitter input there is a single photon in one of the ports $\mid$ in $\rangle=|1\rangle_1|0\rangle_2$. What is the probability of finding the photon at each of the output ports?

The equation 
\begin{equation}
    \pmqty{a_3 \\ a_4} = \frac{1}{\sqrt{2}} \pmqty{1 & \ii \\ \ii & 1} \pmqty{a_1 \\ a_2} 
\end{equation}
gives the time evolution of annihilation operators in the Heisenberg picture.
Since we are working in the Schrödinger picture, the matrix 
\begin{equation}
    S = \frac{1}{\sqrt{2}} \pmqty{1 & \ii \\ \ii & 1}
\end{equation}
should be understood as the $S$-matrix of a single photon.
Thus 
\[
    S \pmqty{1 \\ 0} = \frac{1}{\sqrt{2}} \pmqty{1 \\ \ii} \Rightarrow 
    S \ket*{1}_1 = \frac{1}{\sqrt{2}} (\ket*{1}_3 + \ii \ket*{1}_4) .
\]
Since there is no photon on mode 2, the time evolution there should be trivial.
Thus the output state at $t = \infty$ corresponding to $\ket*{\text{in}}$ should be 
\begin{equation}
    \ket*{\text{out}} = \frac{1}{\sqrt{2}} (\ket*{1}_3 + \ii \ket*{1}_4)
    = \frac{1}{\sqrt{2}} (\ket*{1}_3 \ket*{0}_4 + \ii \ket*{0}_3 \ket*{1}_4).
\end{equation}
The probabilities to find the photon at port 3 and port 4 are 
\begin{equation}
    P_3 = \frac{1}{2}, \quad P_4 = \frac{1}{2}.
\end{equation} 

This result can also be found in the Heisenberg picture.
From the relation between $a_{3,4}$ and $a_{1,2}$, we find 
\[
    a_1 = \frac{1}{\sqrt{2}} (a_3 - \ii a_4),
\]
and therefore 
\begin{equation}
    \ket*{\Psi} = a_1^\dag \ket*{0} = \frac{1}{\sqrt{2}} (a^\dag_3 + \ii a^\dag_4) \ket*{0}.
\end{equation}

\paragraph*{(b)} Next, we consider the case of two photons at one of the input ports $\mid$ in $\rangle=|2\rangle_1|0\rangle_2$. Find the output state, and calculate the probabilities of finding zero, one, and two photons at each output port.

It's much more convenient to work in the Heisenberg picture and the Fock representation 
for multi-photon cases.
The many-body wave function, in the output basis, is 
\begin{equation}
    \begin{aligned}
        \ket*{\Psi} &= \frac{(a_1^\dag)^2}{\sqrt{2}} \ket*{0} 
        = \frac{1}{2 \sqrt{2}} ((a_3^\dag)^2 + 2 \ii a_3^\dag a_4^\dag - (a_4^\dag)^2 ) \ket*{0} \\
        &= \frac{1}{2} \ket*{2}_3 \ket*{0}_4 + \frac{\ii}{\sqrt{2}} \ket*{1}_3 \ket*{1}_4 - \frac{1}{2} \ket*{0}_3 \ket*{2}_4.
    \end{aligned}
\end{equation} 
$P(n_i = N)$, probabilities to find $N$ photons at port $i$, 
regardless of what's happening at the other port, are given as (this is a projective measurement)
\begin{equation}
    \begin{aligned}
        P(n_3=2) = \frac{1}{4}, \quad P(n_3=1) = \frac{1}{2}, \quad P(n_3=0) = \frac{1}{4}, \\ 
        P(n_4=2) = \frac{1}{4}, \quad P(n_4=1) = \frac{1}{2}, \quad P(n_4=1) = \frac{1}{4}.
    \end{aligned}
\end{equation}

\paragraph*{(c)} Now, consider the case of one photon at each of the input ports $\mid$ in $\rangle=|1\rangle_1|1\rangle_2$. Find the output state, and calculate the probabilities of finding zero, one or two photons at each output port.

From the relation between $a_{3,4}$ and $a_{1,2}$ we can also find 
\begin{equation}
    a_2 = \frac{1}{\sqrt{2}} (- \ii a_3 + a_4) ,
\end{equation}
and the many-body wave function now is 
\begin{equation}
    \begin{aligned}
        \ket*{\Psi} &= a_1^\dag a_2^\dag \ket*{0}
        = \frac{1}{2} (a_3 + \ii a_4^\dag) (\ii a_3^\dag + a_4) \\
        &= \frac{\ii}{\sqrt{2}} (\ket*{2}_3 \ket*{0}_4 + \ket*{0}_3 \ket*{2}_4).
    \end{aligned}
\end{equation}
Therefore
\begin{equation}
    \begin{aligned}
        P(n_3=2) = \frac{1}{2}, \quad P(n_3=1) = 0, \quad P(n_3=0) = \frac{1}{2}, \\
        P(n_4=2) = \frac{1}{2}, \quad P(n_4=1) = 0, \quad P(n_4=0) = \frac{1}{2}. \\
    \end{aligned}
\end{equation}

\paragraph*{(d)} How do the results we have obtained compare to the ones expected from a classical picture, where the beam splitter simply splits the input power?

The one-photon case is not very different from the classical picture:
indeed the probabilities to find one photon at port 3 and port 4 are both \SI{50}{\percent},
and therefore the input power is split equally between the two ports. 
This is expected because in this example we have no photon-photon interaction 
and the EOMs of single-photon quantities are determined by linear Maxwell equations,
whose form stays the same regardless of whether the system is quantum.

In the multi-photon case, the power is still split equally between the two ports.
The main difference from the classical case 
is the \emph{distribution} of the photon number at each port 
is changed after the photons interacting with the beam splitter.

\subsection{}

\section{Noise and correlations}

\subsection{Noise in resonators}



\end{document}