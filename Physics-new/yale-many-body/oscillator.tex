\documentclass[hyperref, a4paper]{article}

\usepackage{geometry}
\usepackage{titling}
\usepackage{titlesec}
% No longer needed, since we will use enumitem package
% \usepackage{paralist}
\usepackage{enumitem}
\usepackage{footnote}
\usepackage{enumerate}
\usepackage{amsmath, amssymb, amsthm}
\usepackage{mathtools}
\usepackage{bbm}
\usepackage{cite}
\usepackage{graphicx}
\usepackage{subfigure}
\usepackage{physics}
\usepackage{tensor}
\usepackage{siunitx}
\usepackage[version=4]{mhchem}
\usepackage{tikz}
\usepackage{xcolor}
\usepackage{listings}
\usepackage{autobreak}
\usepackage[ruled, vlined, linesnumbered]{algorithm2e}
\usepackage{nameref,zref-xr}
\zxrsetup{toltxlabel}
\usepackage[colorlinks,unicode]{hyperref} % , linkcolor=black, anchorcolor=black, citecolor=black, urlcolor=black, filecolor=black
\usepackage[most]{tcolorbox}
\usepackage{prettyref}

% Page style
\geometry{left=3.18cm,right=3.18cm,top=2.54cm,bottom=2.54cm}
\titlespacing{\paragraph}{0pt}{1pt}{10pt}[20pt]
\setlength{\droptitle}{-5em}
%\preauthor{\vspace{-10pt}\begin{center}}
%\postauthor{\par\end{center}}

% More compact lists 
\setlist[itemize]{
    itemindent=17pt, 
    leftmargin=1pt,
    listparindent=\parindent,
    parsep=0pt,
}

% Math operators
\DeclareMathOperator{\timeorder}{\mathcal{T}}
\DeclareMathOperator{\diag}{diag}
\DeclareMathOperator{\legpoly}{P}
\DeclareMathOperator{\primevalue}{P}
\DeclareMathOperator{\sgn}{sgn}
\newcommand*{\ii}{\mathrm{i}}
\newcommand*{\ee}{\mathrm{e}}
\newcommand*{\const}{\mathrm{const}}
\newcommand*{\suchthat}{\quad \text{s.t.} \quad}
\newcommand*{\argmin}{\arg\min}
\newcommand*{\argmax}{\arg\max}
\newcommand*{\normalorder}[1]{: #1 :}
\newcommand*{\pair}[1]{\langle #1 \rangle}
\newcommand*{\fd}[1]{\mathcal{D} #1}
\DeclareMathOperator{\bigO}{\mathcal{O}}

% TikZ setting
\usetikzlibrary{arrows,shapes,positioning}
\usetikzlibrary{arrows.meta}
\usetikzlibrary{decorations.markings}
\tikzstyle arrowstyle=[scale=1]
\tikzstyle directed=[postaction={decorate,decoration={markings,
    mark=at position .5 with {\arrow[arrowstyle]{stealth}}}}]
\tikzstyle ray=[directed, thick]
\tikzstyle dot=[anchor=base,fill,circle,inner sep=1pt]

% Algorithm setting
% Julia-style code
\SetKwIF{If}{ElseIf}{Else}{if}{}{elseif}{else}{end}
\SetKwFor{For}{for}{}{end}
\SetKwFor{While}{while}{}{end}
\SetKwProg{Function}{function}{}{end}
\SetArgSty{textnormal}

\newcommand*{\concept}[1]{{\textbf{#1}}}

% Embedded codes
\lstset{basicstyle=\ttfamily,
  showstringspaces=false,
  commentstyle=\color{gray},
  keywordstyle=\color{blue}
}

% Reference formatting
\newrefformat{fig}{Figure~\ref{#1}}

% Color boxes
\tcbuselibrary{skins, breakable, theorems}
\newtcbtheorem[number within=section]{warning}{Warning}%
  {colback=orange!5,colframe=orange!65,fonttitle=\bfseries, breakable}{warn}
\newtcbtheorem[number within=section]{note}{Note}%
  {colback=green!5,colframe=green!65,fonttitle=\bfseries, breakable}{note}
\newtcbtheorem[number within=section]{info}{Info}%
  {colback=blue!5,colframe=blue!65,fonttitle=\bfseries, breakable}{info}

\newenvironment{shelldisplay}{\begin{lstlisting}}{\end{lstlisting}}

\title{Correlation function and response function}
\author{Jinyuan Wu}

\begin{document}

\maketitle

This document gives several examples of the calculation of 
the correlation function and the response function.

\section{Important formulae}

For disturbance $H' = \lambda O_1$, 
the first order response of the expectation of the operator $O_1$ is 
\begin{equation}
    D(t_2, t_1) = - \ii \theta(t_2 - t_1) \expval{\comm{O_2(t_2)}{O_1(t_1)}}.
\end{equation}
This function, as its definition hints, 
is hard to calculate. 
We usually calculate the time-ordered correlation function
\begin{equation}
    G(t_2, t_1) = - \ii \expval{\timeorder O_2(t_2) O_1(t_1)}.
\end{equation}
For calculation of the time-ordered correlation function, we have 
\begin{equation}
    \expval{\timeorder O_2(t_2) O_1(t_1)} = 
    \frac{
        \int \fd{x} O_2(t_2) O_1(t_1) \ee^{\ii S[x]}
    }{
        \int \fd{x} \ee^{\ii S[x]}
    }.
\end{equation}
Here when calculating the action,
we need to replace $t$ with $t (1 - \ii \epsilon)$ and then let $\epsilon \to 0$.
For free systems, if 
\begin{equation}
    S = x G^{-1} x,
\end{equation}
then the correlation function is exactly $G$, because 
\begin{equation}
    \frac{
        \int \dd[n]{\vb{x}} x_k x_l \ee^{\frac{\ii}{2} \vb{x} \vb{A} \vb{x}}
    }{
        \int \dd[n]{\vb{x}} \ee^{\frac{\ii}{2} \vb{x} \vb{A} \vb{x}}
    } = \ii (\vb{A}^{-1})_{kl}.
\end{equation}

\section{The harmonic oscillator}

The action is 
\begin{equation}
    S = \int \dd{t} \left( \frac{1}{2} m \dot{x}^2 - \frac{1}{2} m \omega_0^2 x^2 \right).
\end{equation}
The kernel is 
\begin{equation}
    K(t_1, t_2) = (- m \partial_{t_1}^2 - m \omega_0^2) \delta(t_1 - t_2).
\end{equation}
Here the derivative \emph{doesn't} act on the $\delta$-function.
So we need to find the inverse of the kernel.
The definition of ``summation'' in the continuous path integral is $\int \dd{t}$,
so we have 
\begin{equation}
    \int \dd{t_2} K(t_1, t_2) G(t_2, t_3) = \delta(t_1 - t_3),
\end{equation}
and thus we have 
\[
    (-m \partial^2_{t_1} - m \omega_0^2) G(t_1, t_3) = \delta(t_1, t_3).
\]
The equation has time translational symmetry,
so we have 
\begin{equation}
    (-m \partial_t^2 - m \omega_0^2) G(t - t') = \delta(t - t').
\end{equation}
By Fourier transformation 
\begin{equation}
    G(t - t') = \int \frac{\dd \omega}{2\pi} \ee^{- \ii \omega (t - t')} G(\omega),
\end{equation}
we have 
\[
    \int \frac{\dd{\omega}}{2\pi} \ee^{- \ii \omega (t - t')} (m \omega^2 - m \omega_0^2) G(\omega) 
    = \int \frac{\dd{\omega}}{2\pi} \ee^{- \ii \omega (t - t')},
\]
so 
\begin{equation}
    G(\omega) = \frac{1}{m (\omega^2 - \omega_0^2)},
\end{equation}
and we get 
\begin{equation}
    \ii G (\omega) = \frac{\ii}{m (\omega^2 - \omega_0^2)} \sim \expval{x x},
\end{equation}
which is the propagator in perturbation calculation.
We naively do the inverse Fourier transformation 
\begin{equation}
    G(t - t') = \int \frac{\dd{\omega}}{2\pi} \ee^{- \ii \omega (t - t')} \frac{1}{m (\omega^2 - \omega_0^2)}
\end{equation}
and find the integral is not well-defined.
The solution is well-known:
to add an infinitesimal imaginary part that fits the definition of the time-ordered correlation function,
and when we have a $\omega^2$ term in the denominator, 
it should be 
\begin{equation}
    G(\omega) = \frac{1}{m (\omega^2 - \omega_0^2 + \ii 0^+)}.
    \label{eq:g-omega-0-plus}
\end{equation}
But here I'll show this choice of imaginary part can be directly derived 
from the infinitesimal imaginary part in the time variable in the path integral.
The action, after adding the imaginary part to $t$, is 
\begin{equation}
    S = \int \dd{t} (1 - \ii \epsilon) \left( \frac{m}{2} \frac{1}{(1 - \ii \epsilon)^2} \dot{x}^2 
    - \frac{1}{2} m \omega_0^2 x^2 \right),
\end{equation}
and the kernel is 
\begin{equation}
    K(t_1, t_2) = - \frac{m}{2} 
    \left(
        \frac{1}{1 - \ii \epsilon} \partial_{t_1}^2 + (1 - \ii \epsilon) \omega_0^2
    \right) \delta(t_1 - t_2).
\end{equation}
So the Green function is now 
(just replace $m$ by $m / (1 - \ii \epsilon)$, 
and replace $\omega_0^2$ by $(1 - \ii \epsilon)^2 \omega_0^2$)
\begin{equation}
    \begin{aligned}
        G(\omega) &= \frac{1 - \ii \epsilon}{m ( (1 - \ii \epsilon)^2 \omega^2 - \omega_0^2 )}.
    \end{aligned}
\end{equation}
Now we let $\epsilon \to 0$,
and the $- \ii \epsilon$ term in the numerator can be thrown away because it doesn't affect the pole structure;
in the denominator,
the $\epsilon^2$ term is relatively small and can also be ignored.
So we get 
\[
    G(\omega) = \frac{1}{m ( \omega^2 - 2 \ii \epsilon \omega^2 - \omega_0^2 )}.
\]
Since the exact value of $\epsilon$ is not important 
(only its sign is important),
we can replace $\epsilon \omega^2$ -- which has the same sign with $\epsilon$ -- by $\epsilon$ or $0^+$,
and thus 
\begin{equation}
    G(\omega) = \frac{1}{m (\omega^2 - \omega_0^2 + \ii \epsilon)},
\end{equation}
which is exactly \eqref{eq:g-omega-0-plus}.

\end{document}