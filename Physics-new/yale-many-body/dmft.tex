\documentclass[hyperref, a4paper]{article}

\usepackage{geometry}
\usepackage{titling}
\usepackage{titlesec}
% No longer needed, since we will use enumitem package
% \usepackage{paralist}
\usepackage{enumitem}
\usepackage{footnote}
%\usepackage{enumerate}
\usepackage{amsmath, amssymb, amsthm}
\usepackage{mathtools}
\usepackage{bbm}
\usepackage{graphicx}
\usepackage[labelformat=simple]{subcaption}
\usepackage{physics}
\usepackage{tensor}
\usepackage{siunitx}
\usepackage[version=4]{mhchem}
\usepackage{tikz}
\usepackage{xcolor}
\usepackage{listings}
\usepackage{autobreak}
\usepackage[ruled, vlined, linesnumbered]{algorithm2e}
\usepackage{nameref,zref-xr}
\zxrsetup{toltxlabel}
\usepackage[sorting=none]{biblatex}
\addbibresource{dmft.bib}
\usepackage[colorlinks,unicode]{hyperref} % , linkcolor=black, anchorcolor=black, citecolor=black, urlcolor=black, filecolor=black
\usepackage[most]{tcolorbox}
\usepackage{prettyref}

% Page style
\geometry{left=3.18cm,right=3.18cm,top=2.54cm,bottom=2.54cm}
\titlespacing{\paragraph}{0pt}{1pt}{10pt}[20pt]
\setlength{\droptitle}{-5em}
%\preauthor{\vspace{-10pt}\begin{center}}
%\postauthor{\par\end{center}}

% More compact lists 
\setlist[itemize]{
    itemindent=17pt, 
    leftmargin=1pt,
    listparindent=\parindent,
    parsep=0pt,
}

% Math operators
\DeclareMathOperator{\polylog}{\mathrm{Li}}
\DeclareMathOperator{\arctanh}{\mathrm{arctanh}}
\DeclareMathOperator{\timeorder}{\mathcal{T}}
\DeclareMathOperator{\diag}{diag}
\DeclareMathOperator{\legpoly}{P}
\DeclareMathOperator{\primevalue}{P}
\DeclareMathOperator{\sgn}{sgn}
\DeclarePairedDelimiter\ceil{\lceil}{\rceil}
\DeclarePairedDelimiter\floor{\lfloor}{\rfloor}
\newcommand*{\ii}{\mathrm{i}}
\newcommand*{\ee}{\mathrm{e}}
\newcommand*{\const}{\mathrm{const}}
\newcommand*{\suchthat}{\quad \text{s.t.} \quad}
\newcommand*{\argmin}{\arg\min}
\newcommand*{\argmax}{\arg\max}
\newcommand*{\normalorder}[1]{: #1 :}
\newcommand*{\pair}[1]{\langle #1 \rangle}
\newcommand*{\fd}[1]{\mathcal{D} #1}
\DeclareMathOperator{\bigO}{\mathcal{O}}

% TikZ setting
\usetikzlibrary{arrows,shapes,positioning}
\usetikzlibrary{arrows.meta}
\usetikzlibrary{decorations.markings}
\tikzstyle arrowstyle=[scale=1]
\tikzstyle directed=[postaction={decorate,decoration={markings,
    mark=at position .5 with {\arrow[arrowstyle]{stealth}}}}]
\tikzstyle ray=[directed, thick]
\tikzstyle dot=[anchor=base,fill,circle,inner sep=1pt]

% Algorithm setting
% Julia-style code
\SetKwIF{If}{ElseIf}{Else}{if}{}{elseif}{else}{end}
\SetKwFor{For}{for}{}{end}
\SetKwFor{While}{while}{}{end}
\SetKwProg{Function}{function}{}{end}
\SetArgSty{textnormal}

\newcommand*{\concept}[1]{{\textbf{#1}}}

% Embedded codes
\lstset{basicstyle=\ttfamily,
  showstringspaces=false,
  commentstyle=\color{gray},
  keywordstyle=\color{blue}
}

% Reference formatting
\renewcommand\thesubfigure{(\alph{subfigure})}
\newrefformat{fig}{Fig.~\ref{#1}}
\newrefformat{subfig}{Fig.~\ref{#1}(\subref{#1})}

% Color boxes
\tcbuselibrary{skins, breakable, theorems}
\newtcbtheorem[number within=section]{warning}{Warning}%
  {colback=orange!5,colframe=orange!65,fonttitle=\bfseries, breakable}{warn}
\newtcbtheorem[number within=section]{note}{Note}%
  {colback=green!5,colframe=green!65,fonttitle=\bfseries, breakable}{note}
\newtcbtheorem[number within=section]{info}{Info}%
  {colback=blue!5,colframe=blue!65,fonttitle=\bfseries, breakable}{info}

\newenvironment{shelldisplay}{\begin{lstlisting}}{\end{lstlisting}}

\title{DMRG overview}
\author{Jinyuan Wu}

\begin{document}
    
\maketitle

\section{Introduction}

Suppose we are to find the single-particle Green function 
of a condensed matter system.
The standard procedure is to find the irreducible self-energy $\Sigma$
by Feynman diagram resummation
(although there are subtleties 
concerning the well-definedness and uniqueness of Feynman diagram resummation techniques 
\cite{PhysRevLett.119.056402}).
In strongly correlated systems,
deciding diagrams with most contributions is generally hard,
and brutal-force resummation proves intractable.

If, however, it is found that the self-energy 
(or two-particle vertex, or similar ``$n$-particle self-energy diagrams'')
is highly local, 
then in principle,
it can be replicated in a \emph{few-body} model:
suppose, for example, that $\Sigma_{\vb*{i} \vb*{j}}$ is only important 
when $\abs{\vb*{i} - \vb*{j}} \leq n$,
then we can just choose a patch of sites satisfying the $\abs{\vb*{i} - \vb*{j}} \leq n$ condition,
and integrate out the rest of electron modes,
and in the resulting few-body model,
$\Sigma_{\vb*{i} \vb*{j}}$ is exactly the same as in the original model.
The main obstacle now is to decide the parameters in this new few-body model,
which can be solved by adopting a self-consistent scheme:
$\Sigma_{\vb*{i} \vb*{j}}$, together with the free part of the original model,
decides the one-particle Green function,
which then can be used to fit the parameters in the few-body model,
and then the few-body model can be solved to update the self-energy.

It can be seen that diagrammatically speaking, 
this idea is also a resummation strategy,
though here we pick up diagrams according to their locality,
and \emph{all} diagrams, as long as they are local enough
and fit in the ansatz of the form of the few-body model,
are included when we solve the few-body model.
This can also be seen as a mean-field approach,
just like many other self-consistent Feynman diagram resummation strategies.
Since the parameters in the few-body model may contain time explicitly,
we may say what we are doing is a \emph{dynamic} mean-field theory.

This report reviews 

\section{DMFT for the Hubbard model}

We map the 

The single-site model can be instantiated by replacing $\mathcal{G}$ by a bath of non-interactive electrons,
and we get an Anderson impurity model.
TODO: whether this is actually done in real calculation

For Hubbard model, the interaction term in the few-body problem is exactly $U n_{\uparrow} n_{\downarrow}$,
because it is strictly local and is not corrected 
when electrons on other sites are integrated out.
For models with nearest-neighbor interaction,
this is no longer exact,
because now integrating out other electron modes 
means screening of the interaction.
An even more important fact about models with repulsion between electrons from different sites 
is the interaction term cannot appear in a single-site model.
To solve the above problems, 
a possible way is to use a Hubbard-Stratonovich transformation 
and use an auxiliary boson field to introduce the interaction.
The resulting single-site model 
is in an itinerant electron bath 
and a bosonic bath.


This approximation is exact in an infinite lattice \cite{PhysRevB.45.6479}.

\section{Going beyond DMFT: the D$\Gamma$A}

Unfortunately, in some scenarios the original DMFT is bound to fail.
This is exemplified in phrase transition,
in which long-wavelength behaviors are highly important,
and a strictly local $\Sigma$ cannot be the case.
A possible direction of improvement is 
to assume that $\Sigma$ is not local,
but the 2-particle vertex $\Gamma$ is local.
The spatial-varying self-energy can then be calculated from $\Gamma$ by TODO

\printbibliography

\end{document}