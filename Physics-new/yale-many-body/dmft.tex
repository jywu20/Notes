\documentclass[hyperref, a4paper]{article}

\usepackage{geometry}
\usepackage{titling}
\usepackage{titlesec}
% No longer needed, since we will use enumitem package
% \usepackage{paralist}
\usepackage{enumitem}
\usepackage{footnote}
%\usepackage{enumerate}
\usepackage{amsmath, amssymb, amsthm}
\usepackage{mathtools}
\usepackage{bbm}
\usepackage{graphicx}
\usepackage[labelformat=simple]{subcaption}
\usepackage{physics}
\usepackage{tensor}
\usepackage{siunitx}
\usepackage[version=4]{mhchem}
\usepackage{tikz}
\usepackage{xcolor}
\usepackage{listings}
\usepackage{autobreak}
\usepackage[ruled, vlined, linesnumbered]{algorithm2e}
\usepackage{nameref,zref-xr}
\zxrsetup{toltxlabel}
\usepackage[sorting=none]{biblatex}
\addbibresource{dmft.bib}
\usepackage[colorlinks,unicode]{hyperref} % , linkcolor=black, anchorcolor=black, citecolor=black, urlcolor=black, filecolor=black
\usepackage[most]{tcolorbox}
\usepackage{prettyref}

% Page style
\geometry{left=3.18cm,right=3.18cm,top=2.54cm,bottom=2.54cm}
\titlespacing{\paragraph}{0pt}{1pt}{10pt}[20pt]
\setlength{\droptitle}{-5em}
%\preauthor{\vspace{-10pt}\begin{center}}
%\postauthor{\par\end{center}}

% More compact lists 
\setlist[itemize]{
    itemindent=17pt, 
    leftmargin=1pt,
    listparindent=\parindent,
    parsep=0pt,
}

% Math operators
\DeclareMathOperator{\polylog}{\mathrm{Li}}
\DeclareMathOperator{\arctanh}{\mathrm{arctanh}}
\DeclareMathOperator{\timeorder}{\mathcal{T}}
\DeclareMathOperator{\diag}{diag}
\DeclareMathOperator{\legpoly}{P}
\DeclareMathOperator{\primevalue}{P}
\DeclareMathOperator{\sgn}{sgn}
\DeclarePairedDelimiter\ceil{\lceil}{\rceil}
\DeclarePairedDelimiter\floor{\lfloor}{\rfloor}
\newcommand*{\ii}{\mathrm{i}}
\newcommand*{\ee}{\mathrm{e}}
\newcommand*{\const}{\mathrm{const}}
\newcommand*{\suchthat}{\quad \text{s.t.} \quad}
\newcommand*{\argmin}{\arg\min}
\newcommand*{\argmax}{\arg\max}
\newcommand*{\normalorder}[1]{: #1 :}
\newcommand*{\pair}[1]{\langle #1 \rangle}
\newcommand*{\fd}[1]{\mathcal{D} #1}
\DeclareMathOperator{\bigO}{\mathcal{O}}

% TikZ setting
\usetikzlibrary{arrows,shapes,positioning}
\usetikzlibrary{arrows.meta}
\usetikzlibrary{decorations.markings}
\tikzstyle arrowstyle=[scale=1]
\tikzstyle directed=[postaction={decorate,decoration={markings,
    mark=at position .5 with {\arrow[arrowstyle]{stealth}}}}]
\tikzstyle ray=[directed, thick]
\tikzstyle dot=[anchor=base,fill,circle,inner sep=1pt]

% Algorithm setting
% Julia-style code
\SetKwIF{If}{ElseIf}{Else}{if}{}{elseif}{else}{end}
\SetKwFor{For}{for}{}{end}
\SetKwFor{While}{while}{}{end}
\SetKwProg{Function}{function}{}{end}
\SetArgSty{textnormal}

\newcommand*{\concept}[1]{{\textbf{#1}}}

% Embedded codes
\lstset{basicstyle=\ttfamily,
  showstringspaces=false,
  commentstyle=\color{gray},
  keywordstyle=\color{blue}
}

% Reference formatting
\renewcommand\thesubfigure{(\alph{subfigure})}
\newrefformat{fig}{Fig.~\ref{#1}}
\newrefformat{subfig}{Fig.~\ref{#1}(\subref{#1})}

% Color boxes
\tcbuselibrary{skins, breakable, theorems}
\newtcbtheorem[number within=section]{warning}{Warning}%
  {colback=orange!5,colframe=orange!65,fonttitle=\bfseries, breakable}{warn}
\newtcbtheorem[number within=section]{note}{Note}%
  {colback=green!5,colframe=green!65,fonttitle=\bfseries, breakable}{note}
\newtcbtheorem[number within=section]{info}{Info}%
  {colback=blue!5,colframe=blue!65,fonttitle=\bfseries, breakable}{info}

\newenvironment{shelldisplay}{\begin{lstlisting}}{\end{lstlisting}}

% Displaying math, etc in bookmarkers

\pdfstringdefDisableCommands{%
  \def\\{}%
  \def\ce#1{<#1>}%
}

\pdfstringdefDisableCommands{%
  \def\texttt#1{<#1>}%
  \def\mathbb#1{#1}%
}
\pdfstringdefDisableCommands{\def\eqref#1{(\ref{#1})}}

\makeatletter
\pdfstringdefDisableCommands{\let\HyPsd@CatcodeWarning\@gobble}
\makeatother

\title{Dynamic mean-field theory and the like}
\author{Jinyuan Wu}

\begin{document}
    
\maketitle

\begin{abstract}
    Mapping a highly localized many-body strongly correlated electronic system 
    to an impurity problem is possible,
    which motivates the development of DMFT
    as a numerical approach to such systems.
    This report reviews the theory foundation of DMFT,
    its improved versions, and DFT+DMFT and $GW$+DMFT \textit{ab initio} methods.
\end{abstract}

\section{Introduction}

Suppose we are to find the single-particle Green function 
of a condensed matter system.
The standard procedure is to find the irreducible self-energy $\Sigma$
by Feynman diagram resummation.
In strongly correlated systems,
deciding diagrams with most contributions is generally hard,
and brutal-force resummation proves intractable.
Besides, there are subtleties 
concerning the well-definedness and uniqueness of Feynman diagram resummation techniques.
When dealing with truly strongly correlated systems,
attention should be paid to avoiding going into an unphysical branch 
\cite{PhysRevLett.119.056402,PhysRevLett.114.156402}.

If, however, it is found that the self-energy 
(or two-particle vertex, or similar ``$n$-particle self-energy diagrams'')
is highly local, 
then in principle,
the self-energy can be replicated in a \emph{few-body} model,
or as we are going to see below,
a \emph{many-body impurity model}
in Hamiltonian formalism.
Suppose, for example, that the real space self-energy $\Sigma_{\vb*{i} \vb*{j}}$ is only important 
when $\abs{\vb*{i} - \vb*{j}} \leq n$.
Then we can just choose a cluster of sites satisfying the $\abs{\vb*{i} - \vb*{j}} \leq n$ condition,
and integrate out the rest of electron modes,
and in the resulting impurity model,
$\Sigma_{\vb*{i} \vb*{j}}$ is exactly the same as in the original model.
The main obstacle now is to decide the parameters in this new impurity model,
which can be solved by adopting a self-consistent scheme:
$\Sigma_{\vb*{i} \vb*{j}}$, together with the free part of the original model,
decides the one-particle Green function,
which then can be used to fit the parameters in the impurity model,
and then the impurity model can be solved to update the self-energy.

It can be seen that diagrammatically speaking, 
this idea is also a resummation strategy,
though here we truncate diagrams according to their \emph{locality} instead of structures,
and the local parts of \emph{all} diagrams, 
as long as they fit in the ansatz of the impurity model,
are included when we solve the impurity model.
This can also be seen as a mean-field approach,
just like other self-consistent Feynman diagram resummation strategies.
Since the parameters in the impurity model may contain time explicitly,
we may say what we are doing is a \emph{dynamic} mean-field theory.

This report reviews the development of dynamic mean-field theory (DMFT).
The term \emph{dynamic mean-field theory} is usually reserved for 
the most coarse approximation with a single-site cluster
and a completely space-independent self-energy
discussed in \prettyref{sec:dmft}.
In \prettyref{sec:more}, we move beyond the simplest DMFT scheme 
and increase the number of sites 
and loose the locality requirement for the self-energy.
\prettyref{sec:ab-initio}
is about combining DMFT and \textit{ab initio} methods together.

\section{DMFT for the Hubbard model}\label{sec:dmft}

\subsection{Duality between the Hubbard model and the Anderson impurity model}

In this section we discuss DMFT for the Hubbard model
\begin{equation}
    H = - t \sum_{\pair{\vb*{i}, \vb*{j}}, \sigma} c^\dagger_{\vb*{i} \sigma} c_{\vb*{j} \sigma} + \text{h.c.}
    + U \sum_{\vb*{i}} n_{\vb*{i} \uparrow} n_{\vb*{j} \downarrow} - \mu N.
\end{equation}
The free part results in the imaginary time Green function
\begin{equation}
    G_{0}(\ii \omega_n, \vb*{k} \sigma) = G_0(\ii \omega_n, \vb*{k}) , 
    \quad G_0(\ii \omega_n, \vb*{k}) = \frac{1}{\ii \omega_n - \varepsilon_{\vb*{k}} + \mu},
\end{equation}
where $\varepsilon_{\vb*{k}}$ is the band structure of the tight-binding model part.
Now we map the Hubbard model to an impurity theory.
We keep only one site in the impurity theory
and integrate out the rest, 
and the final effective theory looks like
\begin{equation}
    S_{\text{single site}} = 
    - \int \dd{\tau} \int \dd{\tau'} \bar{c}(\tau) \mathcal{G}^{-1}(\tau - \tau') c(\tau') 
    + \int \dd{\tau} U n_{\uparrow}(\tau) n_{\downarrow}(\tau).
    \label{eq:single-site}
\end{equation}
The Green function $\mathcal{G}(\tau)$ is 
$G_{0, \vb*{i} \vb*{i}}(\tau)$ corrected by the integrated out electronic degrees of freedom;
the local interaction $U n_{\uparrow} n_{\downarrow}$ is not included in $\mathcal{G}$.

Note that $\mathcal{G}$ can also be seen as a single-particle Green function 
with self-energy correction,
and therefore has the same structure of 
a self-energy-corrected single-particle Green function,
although its value is different from $G_{\vb*{i} \vb*{i}}$,
because its self-energy is different from the self-energy of the whole Hubbard system,
since the former only considers sites beside $\vb*{i}$
while the latter considers all sites.
Assuming the self-energy of $\mathcal{G}$ is well-behaved
for Hilbert transformation \cite{georges1996dynamical},
we write the structure of $\mathcal{G}$ as  
\[
    \mathcal{G}(\ii \omega_n)^{-1} = 
    \ii \omega_n + \text{const.} + \int \dd{\epsilon} g(\epsilon) \frac{1}{\ii \omega_n - \epsilon},
\]
and this means the single-site model can be instantiated 
by replacing $\mathcal{G}$ by a bath of non-interactive electrons,
the dispersion relations of which 
are decided by poles of $\mathcal{G}$,
and we get an Anderson impurity model
\begin{equation}
    H_{\text{Anderson}} = 
    \underbrace{
        \epsilon_0 \sum_\sigma c_{\sigma}^\dagger c_{\sigma}
        + \sum_{\vb*{k}, \sigma} E_{\vb*{k}} d^\dagger_{\vb*{k} \sigma} d_{\vb*{k} \sigma}
        + \sum_{\vb*{k}} (V_{\vb*{k}} d^\dagger_{\vb*{k} \sigma} c_{\sigma} + \text{h.c.})
    }_{\simeq c^\dagger \mathcal{G} c}
    + U n_{\uparrow} n_{\downarrow}.
\end{equation}
The structure of $\mathcal{G}$ is therefore 
\begin{equation}
    \mathcal{G} (\ii \omega_n) = \frac{1}{\ii \omega_n - \epsilon_0 - \Delta(\ii \omega_n)}, \quad 
    \Delta(\ii \omega_n) = \sum_{\vb*{k}} \frac{\abs{V_{\vb*{k}}}^2}{\ii \omega_n - E_{\vb*{k}}}.
\end{equation}
Note that the influence of $U n_{\uparrow} n_{\downarrow}$ is not 
accounted for by $\mathcal{G}$.
Other Hamiltonian realizations of \eqref{eq:single-site} are of course possible
\cite{georges1996dynamical}.
Note that here $\Delta$ contains explicitly $\ii \omega_n$
and therefore has retardation effects 
in the real space.
Therefore \eqref{eq:single-site} contains retardation,
and this is an evidence why it is better to put it into 
the form of a \emph{many-body} impurity model instead of a single-body model.

We use $G_{\text{imp}}$ to refer to the interaction-corrected Green function in the impurity model.
Note that by definition of the mapping, we have 
$G_{\text{imp}}(\ii \omega_n) = G_{\vb*{i} \vb*{i}}(\ii \omega_n)$,
and therefore 
\begin{equation}
    \frac{1}{\mathcal{G}(\ii \omega_n)^{-1} - \Sigma_{\text{imp}}} 
    = {G}_{\text{imp}}(\ii \omega_n) = G_{\vb*{i} \vb*{i}}(\ii \omega_n)
    = \frac{1}{N} \sum_{\vb*{k}} G(\ii \omega_n, \vb*{k})
    = \frac{1}{N} \sum_{\vb*{k}} \frac{1}{\ii \omega_n - \epsilon_{\vb*{k}} - \Sigma(\ii \omega_n, \vb*{k})}.
\end{equation}
where $G$ is the fully corrected Green function in the Hubbard model,
$\Sigma$ is the self-energy of the Hubbard model,
and $\Sigma_{\text{imp}}$ is the self-energy correction to $\mathcal{G}$
caused by $U n_{\uparrow} n_{\downarrow}$.

\subsection{The self-energy locality condition}

The mapping from a Hubbard model to an Anderson impurity model 
involves no physical approximation.
Now we impose the locality condition of the self-energy mentioned in the introduction:
we assume the Hubbard model self-energy $\Sigma(\ii \omega_n, \vb*{k})$ 
is highly localized in the real space,
and therefore involves no $\vb*{k}$-dependence,
and thus we immediately get 
\begin{equation}
    \Sigma_{\text{imp}} = \Sigma(\ii \omega_n \vb*{k}).
    \label{eq:locality}
\end{equation}
This approximation is exact in an infinite lattice \cite{PhysRevB.45.6479},
and therefore expectedly works better for bulk materials than monolayer materials.

\subsection{The self-consistent loop}

The cutoff \eqref{eq:locality} gives us a self-consistent calculation scheme:
first using $G_0$ (known) and $\Sigma_{\text{imp}}$ (from last iteration), we have 
\begin{equation}
    G(\ii \omega_n, \vb*{k}) = \frac{1}{\ii \omega_n - \varepsilon_{\vb*{k}} + \mu - \Sigma_{\text{imp}}(\ii \omega_n)},
\end{equation}
and then we calculate $G_{\vb*{i} \vb*{i}}$ (essentially $G_{\text{imp}}$) by 
\begin{equation}
    G_{\vb*{i} \vb*{i}}(\ii \omega_n) = \frac{1}{N} \sum_{\vb*{k}} G(\ii \omega_n, \vb*{k}),
\end{equation}
and then $\mathcal{G}$ can be found by 
\begin{equation}
    \mathcal{G}(\ii \omega_n)^{-1} = G_{\vb*{i} \vb*{i}}(\ii \omega_n)^{-1} + \Sigma_{\text{imp}}(\ii \omega_n).
\end{equation}
Now the impurity model is fully determined,
and with $U$ and $\mathcal{G}(\ii \omega_n)$, 
we solve the impurity model by calling an existing impurity solver and get $G_{\text{imp}}$.
So now $G_{\text{imp}}$ is updated,
and we can then find the updated $\Sigma_{\text{imp}}$, 
and we go back to the starting point of the loop.

Note that here $\Sigma_{\text{imp}}$ is \emph{not} calculated 
using diagrammatic resummation techniques.
Instead, we find $G_{\text{imp}}$ without explicitly mentioning $\Sigma_{\text{imp}}$ 
in the impurity solver step, 
and $\Sigma_{\text{imp}}$ is found \emph{after} $G_{\text{imp}}$ (and $\mathcal{G}$) is known.
This guarantees enough accuracy of $\Sigma_{\text{imp}}$:
if it is evaluated in the same way with, say, the $GW$ method,
infinite diagrams are thrown away,
but here \emph{all} Feynman diagrams contributing to $\Sigma_{\text{imp}}$ 
-- and therefore $\Sigma(\ii \omega_n, \vb*{k})$ -- are taken into account;
what is ignored is just their spatial independent parts.

For Hubbard model, the interaction term in the impurity problem is exactly $U n_{\uparrow} n_{\downarrow}$,
because it is strictly local and is not corrected 
when electrons on other sites are integrated out.
For models with nearest-neighbor interaction,
this is no longer correct,
because now integrating out other electron modes 
means screening of the interaction.
An even more important fact about models with repulsion between electrons from different sites 
is the interaction term cannot appear in a single-site model.
To solve the above problems, 
a possible way is to use a Hubbard-Stratonovich transformation 
and use an auxiliary boson field to introduce the interaction.
The resulting single-site model 
is in an itinerant electron bath 
and a bosonic bath \cite{sun_extended_2002}.



\section{Going beyond DMFT}\label{sec:more}

Unfortunately, in some scenarios the original single-site DMFT is bound to fail.
This is exemplified in phrase transition,
in which long-wavelength behaviors are highly important,
and a strictly local $\Sigma$ cannot be the case.
A possible direction of improvement is 
to use an impurity dual Hamiltonian involving a larger cluster of states 
\cite{kotliar_cellular_2001,park_cluster_2008}.
The main problem of this approach is the speed:
the impurity model concerning a large impurity cluster has to be solved with high accuracy,
and this is demanding for existing solvers 
\cite{potthoff_cluster_2018,gull_continuous-time_2011}.

\begin{figure}
    \centering
    \begin{tikzpicture}[x=0.75pt,y=0.75pt,yscale=-1,xscale=1]
    %uncomment if require: \path (0,432); %set diagram left start at 0, and has height of 432
    
    %Straight Lines [id:da09445240585149794] 
    \draw [color={rgb, 255:red, 155; green, 155; blue, 155 }  ,draw opacity=1 ]   (51,151) -- (78.5,151) ;
    \draw [shift={(69.75,151)}, rotate = 180] [fill={rgb, 255:red, 155; green, 155; blue, 155 }  ,fill opacity=1 ][line width=0.08]  [draw opacity=0] (10.72,-5.15) -- (0,0) -- (10.72,5.15) -- (7.12,0) -- cycle    ;
    %Straight Lines [id:da2316016117993358] 
    \draw [color={rgb, 255:red, 155; green, 155; blue, 155 }  ,draw opacity=1 ]   (78.5,151) -- (134,151) ;
    \draw [shift={(111.25,151)}, rotate = 180] [fill={rgb, 255:red, 155; green, 155; blue, 155 }  ,fill opacity=1 ][line width=0.08]  [draw opacity=0] (10.72,-5.15) -- (0,0) -- (10.72,5.15) -- (7.12,0) -- cycle    ;
    %Straight Lines [id:da38765479391911994] 
    \draw  [dash pattern={on 0.84pt off 2.51pt}]  (78.5,108.19) -- (78.5,151) ;
    %Straight Lines [id:da5127378632155728] 
    \draw [color={rgb, 255:red, 155; green, 155; blue, 155 }  ,draw opacity=1 ]   (78.5,108.19) -- (126.5,76.19) ;
    \draw [shift={(106.66,89.41)}, rotate = 146.31] [fill={rgb, 255:red, 155; green, 155; blue, 155 }  ,fill opacity=1 ][line width=0.08]  [draw opacity=0] (10.72,-5.15) -- (0,0) -- (10.72,5.15) -- (7.12,0) -- cycle    ;
    %Straight Lines [id:da6835619840569007] 
    \draw [color={rgb, 255:red, 155; green, 155; blue, 155 }  ,draw opacity=1 ]   (132.5,116.19) -- (78.5,108.19) ;
    \draw [shift={(100.55,111.45)}, rotate = 8.43] [fill={rgb, 255:red, 155; green, 155; blue, 155 }  ,fill opacity=1 ][line width=0.08]  [draw opacity=0] (10.72,-5.15) -- (0,0) -- (10.72,5.15) -- (7.12,0) -- cycle    ;
    %Shape: Polygon Curved [id:ds8593919113924382] 
    \draw  [color={rgb, 255:red, 245; green, 166; blue, 35 }  ,draw opacity=1 ][fill={rgb, 255:red, 245; green, 166; blue, 35 }  ,fill opacity=0.2 ] (126.5,76.19) .. controls (126.5,56.19) and (167.92,84.35) .. (172.46,108.77) .. controls (177,133.19) and (134.54,170.73) .. (134,151) .. controls (133.46,131.27) and (134.21,128.27) .. (133.21,116.27) .. controls (132.21,104.27) and (126.5,96.19) .. (126.5,76.19) -- cycle ;
    %Straight Lines [id:da8607891712817617] 
    \draw [color={rgb, 255:red, 155; green, 155; blue, 155 }  ,draw opacity=1 ]   (151,150) -- (187.46,150) ;
    \draw [shift={(174.23,150)}, rotate = 180] [fill={rgb, 255:red, 155; green, 155; blue, 155 }  ,fill opacity=1 ][line width=0.08]  [draw opacity=0] (10.72,-5.15) -- (0,0) -- (10.72,5.15) -- (7.12,0) -- cycle    ;
    %Straight Lines [id:da863066386553091] 
    \draw [color={rgb, 255:red, 155; green, 155; blue, 155 }  ,draw opacity=1 ]   (273.46,150) -- (320.46,150) ;
    \draw [shift={(301.96,150)}, rotate = 180] [fill={rgb, 255:red, 155; green, 155; blue, 155 }  ,fill opacity=1 ][line width=0.08]  [draw opacity=0] (10.72,-5.15) -- (0,0) -- (10.72,5.15) -- (7.12,0) -- cycle    ;
    %Straight Lines [id:da11431768817586696] 
    \draw [color={rgb, 255:red, 155; green, 155; blue, 155 }  ,draw opacity=1 ]   (226.46,150) -- (273.46,150) ;
    \draw [shift={(254.96,150)}, rotate = 180] [fill={rgb, 255:red, 155; green, 155; blue, 155 }  ,fill opacity=1 ][line width=0.08]  [draw opacity=0] (10.72,-5.15) -- (0,0) -- (10.72,5.15) -- (7.12,0) -- cycle    ;
    %Straight Lines [id:da8001760468464807] 
    \draw [color={rgb, 255:red, 155; green, 155; blue, 155 }  ,draw opacity=1 ][line width=0.75]    (293.5,107.19) -- (320.46,107.19) ;
    \draw [shift={(300.48,107.19)}, rotate = 0] [fill={rgb, 255:red, 155; green, 155; blue, 155 }  ,fill opacity=1 ][line width=0.08]  [draw opacity=0] (10.72,-5.15) -- (0,0) -- (10.72,5.15) -- (7.12,0) -- cycle    ;
    %Straight Lines [id:da07129256481006307] 
    \draw [color={rgb, 255:red, 155; green, 155; blue, 155 }  ,draw opacity=1 ][line width=0.75]    (226.46,107.19) -- (255.5,107.19) ;
    \draw [shift={(234.48,107.19)}, rotate = 0] [fill={rgb, 255:red, 155; green, 155; blue, 155 }  ,fill opacity=1 ][line width=0.08]  [draw opacity=0] (10.72,-5.15) -- (0,0) -- (10.72,5.15) -- (7.12,0) -- cycle    ;
    %Straight Lines [id:da04137891039280173] 
    \draw [color={rgb, 255:red, 155; green, 155; blue, 155 }  ,draw opacity=1 ]   (411.5,150) -- (436.46,150) ;
    \draw [shift={(428.98,150)}, rotate = 180] [fill={rgb, 255:red, 155; green, 155; blue, 155 }  ,fill opacity=1 ][line width=0.08]  [draw opacity=0] (10.72,-5.15) -- (0,0) -- (10.72,5.15) -- (7.12,0) -- cycle    ;
    %Straight Lines [id:da06436521106351112] 
    \draw [color={rgb, 255:red, 155; green, 155; blue, 155 }  ,draw opacity=1 ]   (342.46,150) -- (366.5,150) ;
    \draw [shift={(359.48,150)}, rotate = 180] [fill={rgb, 255:red, 155; green, 155; blue, 155 }  ,fill opacity=1 ][line width=0.08]  [draw opacity=0] (10.72,-5.15) -- (0,0) -- (10.72,5.15) -- (7.12,0) -- cycle    ;
    %Straight Lines [id:da05366302622339503] 
    \draw [color={rgb, 255:red, 155; green, 155; blue, 155 }  ,draw opacity=1 ]   (411.5,107.19) -- (436.46,107.19) ;
    \draw [shift={(417.48,107.19)}, rotate = 0] [fill={rgb, 255:red, 155; green, 155; blue, 155 }  ,fill opacity=1 ][line width=0.08]  [draw opacity=0] (10.72,-5.15) -- (0,0) -- (10.72,5.15) -- (7.12,0) -- cycle    ;
    %Straight Lines [id:da7077968907338072] 
    \draw [color={rgb, 255:red, 155; green, 155; blue, 155 }  ,draw opacity=1 ]   (342.46,107.19) -- (368.5,107.19) ;
    \draw [shift={(348.98,107.19)}, rotate = 0] [fill={rgb, 255:red, 155; green, 155; blue, 155 }  ,fill opacity=1 ][line width=0.08]  [draw opacity=0] (10.72,-5.15) -- (0,0) -- (10.72,5.15) -- (7.12,0) -- cycle    ;
    %Shape: Arc [id:dp5172042096139051] 
    \draw  [draw opacity=0] (411.5,150) .. controls (403.17,149.14) and (396.62,139.91) .. (396.62,128.65) .. controls (396.62,117.11) and (403.5,107.7) .. (412.13,107.26) -- (412.77,128.65) -- cycle ; \draw  [color={rgb, 255:red, 155; green, 155; blue, 155 }  ,draw opacity=1 ] (411.5,150) .. controls (403.17,149.14) and (396.62,139.91) .. (396.62,128.65) .. controls (396.62,117.11) and (403.5,107.7) .. (412.13,107.26) ;  
    %Shape: Arc [id:dp9647778099863114] 
    \draw  [draw opacity=0][line width=1.5]  (366.5,150) .. controls (374.83,149.14) and (381.38,139.91) .. (381.38,128.65) .. controls (381.38,117.11) and (374.5,107.7) .. (365.87,107.26) -- (365.23,128.65) -- cycle ; \draw  [color={rgb, 255:red, 0; green, 0; blue, 0 }  ,draw opacity=1 ][line width=1.5]  (366.5,150) .. controls (374.83,149.14) and (381.38,139.91) .. (381.38,128.65) .. controls (381.38,117.11) and (374.5,107.7) .. (365.87,107.26) ;  
    %Straight Lines [id:da642448704976921] 
    \draw [line width=1.5]    (255.5,107.19) -- (293.5,107.19) ;
    %Straight Lines [id:da2714548869106579] 
    \draw [color={rgb, 255:red, 155; green, 155; blue, 155 }  ,draw opacity=1 ]   (50,284) -- (77.5,284) ;
    \draw [shift={(68.75,284)}, rotate = 180] [fill={rgb, 255:red, 155; green, 155; blue, 155 }  ,fill opacity=1 ][line width=0.08]  [draw opacity=0] (10.72,-5.15) -- (0,0) -- (10.72,5.15) -- (7.12,0) -- cycle    ;
    %Straight Lines [id:da35646323785910305] 
    \draw [color={rgb, 255:red, 0; green, 0; blue, 0 }  ,draw opacity=1 ][line width=1.5]    (77.5,284) -- (119.5,284) ;
    \draw [shift={(104.1,284)}, rotate = 180] [fill={rgb, 255:red, 0; green, 0; blue, 0 }  ,fill opacity=1 ][line width=0.08]  [draw opacity=0] (11.07,-5.32) -- (0,0) -- (11.07,5.32) -- (7.35,0) -- cycle    ;
    %Straight Lines [id:da39912372458600487] 
    \draw  [dash pattern={on 0.84pt off 2.51pt}]  (77.5,241.19) -- (77.5,284) ;
    %Straight Lines [id:da6832222978204241] 
    \draw [color={rgb, 255:red, 155; green, 155; blue, 155 }  ,draw opacity=1 ]   (161.98,284) -- (196.5,284) ;
    \draw [shift={(184.24,284)}, rotate = 180] [fill={rgb, 255:red, 155; green, 155; blue, 155 }  ,fill opacity=1 ][line width=0.08]  [draw opacity=0] (10.72,-5.15) -- (0,0) -- (10.72,5.15) -- (7.12,0) -- cycle    ;
    %Shape: Square [id:dp06299260637844539] 
    \draw  [color={rgb, 255:red, 0; green, 0; blue, 0 }  ,draw opacity=1 ][fill={rgb, 255:red, 0; green, 0; blue, 0 }  ,fill opacity=0.2 ] (119.5,241.52) -- (161.98,241.52) -- (161.98,284) -- (119.5,284) -- cycle ;
    %Straight Lines [id:da2654756998286112] 
    \draw [color={rgb, 255:red, 0; green, 0; blue, 0 }  ,draw opacity=1 ][line width=1.5]    (77.5,241.52) -- (119.5,241.52) ;
    \draw [shift={(91.4,241.52)}, rotate = 0] [fill={rgb, 255:red, 0; green, 0; blue, 0 }  ,fill opacity=1 ][line width=0.08]  [draw opacity=0] (11.07,-5.32) -- (0,0) -- (11.07,5.32) -- (7.35,0) -- cycle    ;
    %Curve Lines [id:da19566170891014667] 
    \draw [line width=1.5]    (77.5,241.19) .. controls (76.5,205.52) and (159.5,202.52) .. (161.98,241.52) ;
    \draw [shift={(125.33,213.58)}, rotate = 182.43] [fill={rgb, 255:red, 0; green, 0; blue, 0 }  ][line width=0.08]  [draw opacity=0] (11.07,-5.32) -- (0,0) -- (11.07,5.32) -- (7.35,0) -- cycle    ;
    %Straight Lines [id:da3293601828130672] 
    \draw [color={rgb, 255:red, 155; green, 155; blue, 155 }  ,draw opacity=1 ]   (236.98,203.52) -- (266.5,203.52) ;
    \draw [shift={(245.24,203.52)}, rotate = 0] [fill={rgb, 255:red, 155; green, 155; blue, 155 }  ,fill opacity=1 ][line width=0.08]  [draw opacity=0] (10.72,-5.15) -- (0,0) -- (10.72,5.15) -- (7.12,0) -- cycle    ;
    %Straight Lines [id:da8902297468070699] 
    \draw [color={rgb, 255:red, 155; green, 155; blue, 155 }  ,draw opacity=1 ]   (266.5,203.52) -- (296.02,203.52) ;
    \draw [shift={(274.76,203.52)}, rotate = 0] [fill={rgb, 255:red, 155; green, 155; blue, 155 }  ,fill opacity=1 ][line width=0.08]  [draw opacity=0] (10.72,-5.15) -- (0,0) -- (10.72,5.15) -- (7.12,0) -- cycle    ;
    %Straight Lines [id:da9458205621799789] 
    \draw [color={rgb, 255:red, 155; green, 155; blue, 155 }  ,draw opacity=1 ]   (236.98,285.69) -- (266.5,285.69) ;
    \draw [shift={(256.74,285.69)}, rotate = 180] [fill={rgb, 255:red, 155; green, 155; blue, 155 }  ,fill opacity=1 ][line width=0.08]  [draw opacity=0] (10.72,-5.15) -- (0,0) -- (10.72,5.15) -- (7.12,0) -- cycle    ;
    %Straight Lines [id:da655757729656997] 
    \draw [color={rgb, 255:red, 155; green, 155; blue, 155 }  ,draw opacity=1 ]   (266.5,285.69) -- (296.02,285.69) ;
    \draw [shift={(286.26,285.69)}, rotate = 180] [fill={rgb, 255:red, 155; green, 155; blue, 155 }  ,fill opacity=1 ][line width=0.08]  [draw opacity=0] (10.72,-5.15) -- (0,0) -- (10.72,5.15) -- (7.12,0) -- cycle    ;
    %Shape: Circle [id:dp48897484769696153] 
    \draw   (247.75,244.6) .. controls (247.75,234.25) and (256.14,225.85) .. (266.5,225.85) .. controls (276.86,225.85) and (285.25,234.25) .. (285.25,244.6) .. controls (285.25,254.96) and (276.86,263.35) .. (266.5,263.35) .. controls (256.14,263.35) and (247.75,254.96) .. (247.75,244.6) -- cycle ;
    %Straight Lines [id:da875080331985675] 
    \draw  [dash pattern={on 0.84pt off 2.51pt}]  (266.5,203.52) -- (266.5,225.85) ;
    %Straight Lines [id:da6102466152656685] 
    \draw  [dash pattern={on 0.84pt off 2.51pt}]  (266.5,263.35) -- (266.5,285.69) ;
    %Straight Lines [id:da7004745454938024] 
    \draw    (247.75,244.6) -- (247.86,242.27) ;
    \draw [shift={(248,239.27)}, rotate = 92.68] [fill={rgb, 255:red, 0; green, 0; blue, 0 }  ][line width=0.08]  [draw opacity=0] (10.72,-5.15) -- (0,0) -- (10.72,5.15) -- (7.12,0) -- cycle    ;
    %Straight Lines [id:da934678084430149] 
    \draw    (284.94,242.97) -- (284.91,245.6) ;
    \draw [shift={(284.88,248.6)}, rotate = 270.71] [fill={rgb, 255:red, 0; green, 0; blue, 0 }  ][line width=0.08]  [draw opacity=0] (10.72,-5.15) -- (0,0) -- (10.72,5.15) -- (7.12,0) -- cycle    ;
    %Straight Lines [id:da6329651295020562] 
    \draw [color={rgb, 255:red, 155; green, 155; blue, 155 }  ,draw opacity=1 ]   (423.5,264) -- (448.46,264) ;
    \draw [shift={(440.98,264)}, rotate = 180] [fill={rgb, 255:red, 155; green, 155; blue, 155 }  ,fill opacity=1 ][line width=0.08]  [draw opacity=0] (10.72,-5.15) -- (0,0) -- (10.72,5.15) -- (7.12,0) -- cycle    ;
    %Straight Lines [id:da48841399622903525] 
    \draw [color={rgb, 255:red, 155; green, 155; blue, 155 }  ,draw opacity=1 ]   (342.46,264) -- (369.13,263.93) ;
    \draw [shift={(360.79,263.95)}, rotate = 179.85] [fill={rgb, 255:red, 155; green, 155; blue, 155 }  ,fill opacity=1 ][line width=0.08]  [draw opacity=0] (10.72,-5.15) -- (0,0) -- (10.72,5.15) -- (7.12,0) -- cycle    ;
    %Straight Lines [id:da3240247433742891] 
    \draw [color={rgb, 255:red, 155; green, 155; blue, 155 }  ,draw opacity=1 ]   (423.5,221.19) -- (448.46,221.19) ;
    \draw [shift={(429.48,221.19)}, rotate = 0] [fill={rgb, 255:red, 155; green, 155; blue, 155 }  ,fill opacity=1 ][line width=0.08]  [draw opacity=0] (10.72,-5.15) -- (0,0) -- (10.72,5.15) -- (7.12,0) -- cycle    ;
    %Straight Lines [id:da5792010704706783] 
    \draw [color={rgb, 255:red, 155; green, 155; blue, 155 }  ,draw opacity=1 ]   (342.46,221.19) -- (368.5,221.19) ;
    \draw [shift={(348.98,221.19)}, rotate = 0] [fill={rgb, 255:red, 155; green, 155; blue, 155 }  ,fill opacity=1 ][line width=0.08]  [draw opacity=0] (10.72,-5.15) -- (0,0) -- (10.72,5.15) -- (7.12,0) -- cycle    ;
    %Shape: Arc [id:dp6670803336608755] 
    \draw  [draw opacity=0] (423.5,264) .. controls (415.17,263.14) and (408.62,253.91) .. (408.62,242.65) .. controls (408.62,231.11) and (415.5,221.7) .. (424.13,221.26) -- (424.77,242.65) -- cycle ; \draw  [color={rgb, 255:red, 155; green, 155; blue, 155 }  ,draw opacity=1 ] (423.5,264) .. controls (415.17,263.14) and (408.62,253.91) .. (408.62,242.65) .. controls (408.62,231.11) and (415.5,221.7) .. (424.13,221.26) ;  
    %Straight Lines [id:da9715132077955346] 
    \draw  [dash pattern={on 0.84pt off 2.51pt}]  (408.5,242.85) -- (384.5,242.85) ;
    %Shape: Arc [id:dp446700524707202] 
    \draw  [draw opacity=0] (369.13,263.93) .. controls (377.45,263.07) and (384.01,253.84) .. (384.01,242.58) .. controls (384.01,231.04) and (377.12,221.63) .. (368.5,221.19) -- (367.86,242.58) -- cycle ; \draw  [color={rgb, 255:red, 155; green, 155; blue, 155 }  ,draw opacity=1 ] (369.13,263.93) .. controls (377.45,263.07) and (384.01,253.84) .. (384.01,242.58) .. controls (384.01,231.04) and (377.12,221.63) .. (368.5,221.19) ;  
    
    % Text Node
    \draw (112,168) node [anchor=north west][inner sep=0.75pt]   [align=left] {(a)};
    % Text Node
    \draw (251,76) node [anchor=north west][inner sep=0.75pt]   [align=left] {Hartree};
    % Text Node
    \draw (367,76) node [anchor=north west][inner sep=0.75pt]   [align=left] {Fock};
    % Text Node
    \draw (334,163.52) node [anchor=north west][inner sep=0.75pt]   [align=left] {(b)};
    % Text Node
    \draw (111,306.52) node [anchor=north west][inner sep=0.75pt]   [align=left] {(c)};
    % Text Node
    \draw (258,307.52) node [anchor=north west][inner sep=0.75pt]   [align=left] {(d)};
    % Text Node
    \draw (386,306.52) node [anchor=north west][inner sep=0.75pt]   [align=left] {(e)};
    
    
    \end{tikzpicture}
    
    \caption{
        Finding the self-energy from the reducible vertex function.
        Here grey lines are external lines, 
        which do not contribute propagator factors and merely provides a momentum value 
        when the diagram is evaluated;
        bold lines are lines with interaction correction,
        like self-energy correction for electron lines.
        (a) The schematic illustration of the structure of all self-energy diagrams.
        (b) Two simplest self-energy diagrams.
        (c) The structure of all self-energy diagrams besides the two in (b).
        The grey box is the two-particle irreducible vertex.
        (d, e) Two examples of the content of the two-particle irreducible vertex.
    }
    \label{fig:dga}
\end{figure}

Improvement can also be made by relaxing the strict locality condition for the self-energy,
which leads to the dynamic vertex approximation.
In the dynamic vertex approximation (D$\Gamma$A),
we assume that $\Sigma$ is not local,
but the two-particle vertex $\Gamma$ is local.
The spatial-varying self-energy can then be calculated from $\Gamma$ \cite{bickers1991conserving}.
Consider \prettyref{fig:dga}(a):  
a diagram in the self-energy always starts with a Coulomb interaction vertex,
so all self-energy diagrams can be seen as a Coulomb vertex plus a four-legged diagram.
When the four-legged diagram is disconnected, we only have two diagrams that are logically possible,
and we can find they are just the Hartree diagram and the Fock diagram,
and the propagators inside are modified by self-energy correction (\prettyref{fig:dga}(b)).
When there is at least one Coulomb interaction line connecting the upper lines and the lower lines 
of the four-leg diagram, the four legs have to be linked to four self-energy-corrected electron lines,
or otherwise the diagram is not well-formed.
Thus we find the sum of all diagrams besides diagrams in (b) 
is proportion to the reducible two-particle vertex diagram (c),
and (a) = (b) + (c) \cite{bickers1991conserving}.
Specifically, the reducible two-particle vertex diagram contains two diagrams (d) and (e), 
and they correspond to a term in the screening correction 
(i.e. correction of the interaction line, or ``vacuum polarization'' in terms of particle physics)
and a term in the vertex correction,
both of which are expected self-energy diagrams.
Thus, in D$\Gamma$A, we can self-consistently update the $\Gamma$ function, 
i.e. the two-particle vertex function, 
which already contains all the information of the single-electron self-energy.

\section{DMFT in \textit{ab initio} calculation}\label{sec:ab-initio}

As is mentioned above, 
DMFT (and even its improved versions)
assumes strong locality conditions to the system,
and therefore has inherent difficulties to capture itinerant behaviors.
Common \textit{ab initio} band structure calculation approaches,
including DFT and $GW$, 
on the other hand, 
do not capture localized behaviors.
Mixing DFT or $GW$ with DMFT is therefore needed 
for large-scale \textit{ab initio} simulation 
of materials involving electron localization,
or in other words, magnetism.

All electronic structure calculation approaches that use the quasiparticle picture 
can be summarized as an instance of the minimization problem 
of the functional 
\begin{equation}
    \Omega[\mathbf{G}]= \Phi[\mathbf{G}]+\operatorname{Tr} \ln (-\mathbf{G})-\operatorname{Tr}\left(\left(\mathbf{G}_0^{-1}-\mathbf{G}^{-1}\right) \mathbf{G}\right)
\end{equation}
with respect to the matrix form of the renormalized Green function $G$,
where $\Omega$ is the grand potential
and equals $E - \mu N$ at $T = 0$, 
and the Luttinger-Ward functional $\Phi$ 
is the sum of of all closed, irreducible, and renormalized skeleton diagrams,
and we have 
\begin{equation}
    \mathbf{\Sigma} = \fdv{\Omega}{\mathbf{G}}, 
\end{equation}
so $\Phi$ and $\Omega$ are some kinds of generating functionals
\cite{PhysRev.118.1417,potthoff2003self}.

Specifically, Kohn-Sham density functional theory (DFT) 
can be formulated in the framework of Luttinger-Ward functional: 
the exchange-correlation functional 
is $\Phi$ in terms of the electron number density,
which then is a functional of the single-particle Green function.
In principle, 
this means the exchange-correlation functional used in DFT
can be improved systematically using Feynman diagrammatic techniques
\cite{aryasetiawan2002total,gruning2006density,haule2015free},
although the infamous semiconductor band gap problem 
if we consider Kohn-Sham energies as quasiparticle energies
persists even with an accurate functional
(and actually it becomes worse because now there is no cancellation of errors),
because we still need to consider the contribution of 
energy functional discontinuity
\cite{gruning2006density}.
The idea that the density functional can be obtained from Feynman diagrams 
however is important when we need to develop 
brand-new DFTs for systems like, say, 
superconductors and exciton insulators,
in which interaction channels other than Coulomb interaction exist 
and no past experience is there for reference;
in that case, 
summing up several low-order bubble diagrams 
and using the result as an initial guess for the energy functional 
may be one way to proceed \cite{kurth1999local,chen2022development}.

On the other hand, the DMFT version of $\Phi$ is 
\begin{equation}
    \Phi = \sum_{\vb*{i}} \Phi^U_{\text{single site}}[G_{\text{imp}, \vb*{i}}],
    \label{eq:phi-dmft-sum}
\end{equation}
in which we sum over all atoms in the crystal 
to make $\Phi$ an extensive quantity,
and indeed, we can show that when $d \to \infty$,
\eqref{eq:phi-dmft-sum} is exact, 
in order to justify DMFT in the $d \to \infty$ limit \cite{georges1996dynamical}.
Therefore to mix DMFT with DFT, 
we need to replace the part in $\Phi_{\text{DFT}}$ that could be better described by DMFT,
and the resulting Luttinger-Ward functional is 
\begin{equation}
    \Phi[G] = \Phi_{\text{DFT}}[G] + \Phi_{\text{DMFT}}[G_{\text{loc}}] - \Phi_{\text{DC}}[G],
\end{equation}
where the subscript loc means to replace $G_{\vb*{i} \vb*{j}}$ by $G_{\vb*{i} \vb*{i}}$, 
and $\Phi_{\text{DC}}$ means double-counting and can be evaluated exactly \cite{haule2015exact},
and the Hubbard $U$ can be calculated 
by considering the Coulomb interaction screened by 
electrons not governed by DMFT.
Similarly, we can also design a $GW$+DMFT scheme,
where 
\begin{equation}
    \Phi[G] = \Phi_{GW}[G] + \Phi_{\text{DMFT}}[G_{\text{loc}}] - \Phi_{\text{Hubbard } GW}[G_{\text{loc}}].
\end{equation}
Here the double counting term (the third term) can be evaluated more explicitly,
because DMFT considers the local, Hubbard-like versions of all $GW$ diagrams,
so to avoid double counting, 
we can just skip these diagrams.
The self-consistent equations can be found by taking the variation of these functionals.

\section{Conclusion and discussion}

Diagrammatic approaches in computational condensed matter physics 
usually truncate diagrams 
according to the number of interaction vertices (MP2, MP3) 
or according to the structure of the diagrams (RPA, GW) \cite{ren2012random,bruneval2021gw}.
DMFT, on the other hand, 
utilizes the \emph{locality} condition,
and hence maps an intractable many-body problem into 
a hard yet tractable impurity problem.
This makes it especially important for strongly-correlated magnetic systems,
where itinerant behaviors are absent.

Despite being a method originally designed for effective models,
among other numerical approaches to strongly correlated electronic systems,
DMFT is remarkable for operating well with \textit{ab initio} methods
(DMRG also appears in \textit{ab initio} calculations, 
but only for molecules \cite{baiardi2020density}).
DFT+$U$ is another attempt to mix ideas from effective models to \textit{ab initio} calculation,
but it only works well when the electrons remain itinerant \cite{pavarini2021solving}.
DFT+DMFT, on the other hand, is able to capture electron localization.

The bottleneck of DMFT seems to be the impurity solver,
with new solvers still being reported and discussed
\cite{weber2012augmented,ganahl2015efficient,PhysRevB.104.035160,de2019quantitative}.
The existing approaches include 
exact diagonalization, 
quantum Monte Carlo, 
another diagrammatic resummation scheme (this time for the impurity problem),
and more.
The development of DMFT is by no means finished,
and further work is needed for achieving the balance between accuracy and speed.

\printbibliography

\end{document}