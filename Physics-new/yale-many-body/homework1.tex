\documentclass[hyperref, a4paper]{article}

\usepackage{geometry}
\usepackage{titling}
\usepackage{titlesec}
% No longer needed, since we will use enumitem package
% \usepackage{paralist}
\usepackage{enumitem}
\usepackage{footnote}
%\usepackage{enumerate}
\usepackage{amsmath, amssymb, amsthm}
\usepackage{mathtools}
\usepackage{bbm}
\usepackage{cite}
\usepackage{graphicx}
\usepackage{subfigure}
\usepackage{physics}
\usepackage{tensor}
\usepackage{siunitx}
\usepackage[version=4]{mhchem}
\usepackage{tikz}
\usepackage{xcolor}
\usepackage{listings}
\usepackage{autobreak}
\usepackage[ruled, vlined, linesnumbered]{algorithm2e}
\usepackage{nameref,zref-xr}
\zxrsetup{toltxlabel}
\usepackage[colorlinks,unicode]{hyperref} % , linkcolor=black, anchorcolor=black, citecolor=black, urlcolor=black, filecolor=black
\usepackage[most]{tcolorbox}
\usepackage{prettyref}

% Page style
\geometry{left=3.18cm,right=3.18cm,top=2.54cm,bottom=2.54cm}
\titlespacing{\paragraph}{0pt}{1pt}{10pt}[20pt]
\setlength{\droptitle}{-5em}
%\preauthor{\vspace{-10pt}\begin{center}}
%\postauthor{\par\end{center}}

% More compact lists 
\setlist[itemize]{
    itemindent=17pt, 
    leftmargin=1pt,
    listparindent=\parindent,
    parsep=0pt,
}

% Math operators
\DeclareMathOperator{\timeorder}{\mathcal{T}}
\DeclareMathOperator{\diag}{diag}
\DeclareMathOperator{\legpoly}{P}
\DeclareMathOperator{\primevalue}{P}
\DeclareMathOperator{\sgn}{sgn}
\newcommand*{\ii}{\mathrm{i}}
\newcommand*{\ee}{\mathrm{e}}
\newcommand*{\const}{\mathrm{const}}
\newcommand*{\suchthat}{\quad \text{s.t.} \quad}
\newcommand*{\argmin}{\arg\min}
\newcommand*{\argmax}{\arg\max}
\newcommand*{\normalorder}[1]{: #1 :}
\newcommand*{\pair}[1]{\langle #1 \rangle}
\newcommand*{\fd}[1]{\mathcal{D} #1}
\DeclareMathOperator{\bigO}{\mathcal{O}}

% TikZ setting
\usetikzlibrary{arrows,shapes,positioning}
\usetikzlibrary{arrows.meta}
\usetikzlibrary{decorations.markings}
\tikzstyle arrowstyle=[scale=1]
\tikzstyle directed=[postaction={decorate,decoration={markings,
    mark=at position .5 with {\arrow[arrowstyle]{stealth}}}}]
\tikzstyle ray=[directed, thick]
\tikzstyle dot=[anchor=base,fill,circle,inner sep=1pt]

% Algorithm setting
% Julia-style code
\SetKwIF{If}{ElseIf}{Else}{if}{}{elseif}{else}{end}
\SetKwFor{For}{for}{}{end}
\SetKwFor{While}{while}{}{end}
\SetKwProg{Function}{function}{}{end}
\SetArgSty{textnormal}

\newcommand*{\concept}[1]{{\textbf{#1}}}

% Embedded codes
\lstset{basicstyle=\ttfamily,
  showstringspaces=false,
  commentstyle=\color{gray},
  keywordstyle=\color{blue}
}

% Reference formatting
\newrefformat{fig}{Figure~\ref{#1}}

% Color boxes
\tcbuselibrary{skins, breakable, theorems}
\newtcbtheorem[number within=section]{warning}{Warning}%
  {colback=orange!5,colframe=orange!65,fonttitle=\bfseries, breakable}{warn}
\newtcbtheorem[number within=section]{note}{Note}%
  {colback=green!5,colframe=green!65,fonttitle=\bfseries, breakable}{note}
\newtcbtheorem[number within=section]{info}{Info}%
  {colback=blue!5,colframe=blue!65,fonttitle=\bfseries, breakable}{info}

\newenvironment{shelldisplay}{\begin{lstlisting}}{\end{lstlisting}}

\title{Many-body Physics Homework 1}
\author{Jinyuan Wu}

\begin{document}

\maketitle

\paragraph{Problem 1} Consider the 1D harmonic oscillator Hamiltonian $\hat{H}=\frac{\hat{p}^2}{2 m}+\frac{1}{2} m \omega^2 \hat{x}^2$. Define the ladder operator $\hat{a}=\sqrt{\frac{m \omega}{2}}\left(\hat{x}+i \frac{\hat{p}}{m \omega}\right)$. The Hamiltonian can be written as $\hat{H}=\omega \hat{a}^{\dagger} \hat{a}$. Note that we neglect the zero point energy in this problem. For any complex number $\alpha$, define a coherent state by
$$
|\alpha\rangle=e^{-|\alpha|^2} e^{\alpha \hat{a}^{\dagger}}|0\rangle .
$$
They satisfy
$$
\hat{a}|\alpha\rangle=\alpha|\alpha\rangle .
$$
An explicit expression for $|\alpha\rangle$ is $|\alpha\rangle=e^{-|\alpha|^2 / 2} \sum_{n=0}^{\infty} \frac{\alpha^n}{\sqrt{n !}}|0\rangle$, although it is not needed in this problem. One can further check that coherent states are not orthogonal:
$$
\langle\alpha \mid \beta\rangle=e^{-\frac{1}{2}\left(|\alpha|^2+|\beta|^2\right)+\alpha^* \beta} .
$$
But they still form a complete basis, in the sense that there is a resolution of identity:
$$
\int \frac{\mathrm{d}^2 \alpha}{\pi}|\alpha\rangle\langle\alpha|=1 .
$$
\begin{enumerate}
    \item Consider the propagator in the coherent state basis: $U\left(\alpha_f, t_f ; \alpha_i, t_i\right)=\left\langle\alpha_f\left|e^{-i \hat{H}\left(t_f-t_i\right)}\right| \alpha_i\right\rangle$. Derive an expression of $U$ in terms of a discretized path integral over paths $\alpha(t)$.
    \item Take the continuum limit and show that the Lagrangian is
    $$
    L=i \alpha^* \dot{\alpha}-\omega|\alpha|^2 .
    $$
    \item Show that the Lagrangian (5) is the same as the phase-space Lagrangian $L=p \dot{x}-\frac{p^2}{2 m}-\frac{1}{2} m \omega^2 x^2$ (may be up to a total time derivative term).
\end{enumerate}

\paragraph{Solution} \begin{itemize}
\item[1.] We make the Trotter decomposition:
\begin{equation}
    \begin{aligned}
        \mel{\alpha_f}{\ee^{- \ii H(t_f - t_i)}}{\alpha_i} &= \lim_{N \to \infty}
        \prod_{j=1}^{N-1} \int \frac{\dd{\alpha_j}}{\pi} \prod_{j=1}^N \mel{\alpha_j}{\ee^{- \ii \Delta t H}}{\alpha_{j-1}} \\
        &= \lim_{N\to \infty} \prod_{j=1}^{N-1} \int \frac{\dd{\alpha_j}}{\pi} \prod_{j=1}^N 
        \ee^{- \ii \Delta t \omega \alpha_{j-1}^* \alpha_{j-1}} \braket*{\alpha_j}{\alpha_{j-1}} \\
        &= \lim_{N \to \infty} \prod_{j=1}^{N-1} \int \frac{\dd{\alpha_j}}{\pi} \prod_{j=1}^N 
        \ee^{- \ii \Delta t \omega \alpha_{j-1}^* \alpha_{j-1}}
        \ee^{- \frac{1}{2} (\abs{\alpha_j}^2 + \abs{\alpha_{j-1}}^2) + \alpha_j^* \alpha_{j-1}} ,
    \end{aligned}
\end{equation}
where $\Delta \tau = (t_f - t_i) / N$, $\alpha_N = \alpha_f$,
and $\alpha_0 = \alpha_i$.

\item[2.] To continue, we can use the condition that $\alpha_j$ and $\alpha_{j-1}$ is close to each other 
and make the following derivation: 
\[
    \begin{aligned}
        \mel{\alpha_f}{\ee^{- \ii H(t_f - t_i)}}{\alpha_i} 
        &= \lim_{N \to \infty} \prod_{j=1}^{N-1} \int \frac{\dd{\alpha_j}}{\pi} \prod_{j=1}^N 
        \ee^{- \ii \Delta t \omega \alpha_{j-1}^* \alpha_{j-1}}
        \ee^{- \alpha^*_j \alpha_j + \alpha_j^* \alpha_{j-1}} \\ 
        &= \lim_{N \to \infty} \prod_{j=1}^{N-1} \int \frac{\dd{\alpha_j}}{\pi} \prod_{j=1}^N 
        \ee^{- \ii \Delta t \omega \alpha_{j-1}^* \alpha_{j-1} - \alpha_j^* (\alpha_j - \alpha_{j-1})} \\
        &= \lim_{N \to \infty} \prod_{j=1}^{N-1} \int \frac{\dd{\alpha_j}}{\pi} \prod_{j=1}^N 
        \ee^{\ii \Delta t (\ii \alpha_j^* (\alpha_j - \alpha_{j-1}) / \Delta t - \omega \alpha_{j-1}^* \alpha_{j-1}) } \\
        &= \lim_{N \to \infty} \left( \prod_{j=1}^{N-1} \int \frac{\dd{\alpha_j}}{\pi} \right)
        \ee^{\ii \sum_j \Delta t (\ii \alpha_j^* (\alpha_j - \alpha_{j-1}) / \Delta t - \omega \alpha_{j-1}^* \alpha_{j-1}) },
    \end{aligned}
\]
so after taking the continuous limit, we get 
\begin{equation}
    \mel{\alpha_f}{\ee^{- \ii H (t_f - t_i)}}{\alpha_i} = \int \mathcal{D} \alpha 
    \ee^{\ii \int_{t_i}^{t_f} \dd{t} (\ii \alpha^* \dot{\alpha} - \omega \abs{\alpha}^2 )}.
\end{equation}
So the Lagrangian is 
\begin{equation}
    L = \ii \alpha^* \dot{\alpha} - \omega \abs{\alpha}^2.
    \label{eq:lagrangian-oscillator}
\end{equation}

\item[3.] By putting 
\begin{equation}
    \alpha = \sqrt{\frac{m \omega}{2}} (x + \ii p / m \omega) 
\end{equation}
into \eqref{eq:lagrangian-oscillator}, we get 
\[
    \begin{aligned}
        L &= \ii \frac{m \omega}{2} \left( x - \frac{\ii p}{m \omega} \right) 
        \left( \dot{x} + \frac{\ii \dot{p}}{m \omega} \right) 
        - \omega \frac{m \omega}{2} \left( x - \frac{\ii p}{m \omega} \right) 
        \left( x + \frac{\ii p}{m \omega} \right) \\
        &= \ii \frac{m \omega}{2} \left( x - \frac{\ii p}{m \omega} \right) 
        \left( \dot{x} + \frac{\ii \dot{p}}{m \omega} \right) 
        - \frac{p^2}{2m} - \frac{1}{2} m \omega^2 x^2.
    \end{aligned}
\]
By integration by parts, we have 
\[
    \dot{p} \left( x - \frac{\ii p}{m \omega} \right) 
    = \dv{t} \left( px - \frac{\ii p^2}{m \omega} \right) 
    - p \left( \dot{x} - \frac{\ii \dot{p}}{m \omega} \right),
\]
and thus 
\[
    \begin{aligned}
        &\quad \ii \frac{m \omega}{2} \left( x - \frac{\ii p}{m \omega} \right) 
        \left( \dot{x} + \frac{\ii \dot{p}}{m \omega} \right) \\
        &= \ii \frac{m \omega}{2} \left( x \dot{x} - \frac{\ii}{m \omega} p \dot{x} + \frac{\ii}{m \omega} \left( \dv{t} \left( px - \frac{\ii p^2}{m \omega} \right) 
        - p \left( \dot{x} - \frac{\ii \dot{p}}{m \omega} \right) \right) \right) \\
        &= \frac{\ii m \omega}{2} \left( - \frac{2 \ii}{m \omega} p \dot{x} + 
        \dv{t} \left( x^2 - \frac{p^2}{m^2 \omega^2} \right) \right) \\
        &= p \dot{x} + \text{total time derivative}.
    \end{aligned}
\]
So we have 
\begin{equation}
    L = p \dot{x} - \frac{p^2}{2m} - \frac{1}{2} m \omega^2 x^2 + \text{total time derivative}.
\end{equation}

\end{itemize}

\paragraph{Problem 2} A quantum particle in a magnetic field is described by the quantum Hamiltonian
$$
\hat{H}=\frac{1}{2 m}(\hat{\mathbf{p}}-A(\hat{\mathbf{x}}))^2=\frac{1}{2 m}\left[\hat{\mathbf{p}}^2-\hat{\mathbf{p}} A(\hat{\mathbf{x}})-A(\hat{\mathbf{x}}) \hat{\mathbf{p}}+A(\hat{\mathbf{x}})^2\right] .
$$
We set $q=c=1$ for simplicity.
\begin{enumerate}
    \item Derive a discrete (Lagrangian) path integral for $U\left(\mathbf{x}_f, t_f ; \mathbf{x}_i, t_i\right)$, using the ordering of $\hat{\mathbf{p}}, A(\hat{\mathbf{x}})$ in (6).
    \item The Hamiltonian can be equivalently written as
    $$
    \hat{H}=\frac{1}{2 m}\left[\hat{\mathbf{p}}^2-2 \hat{\mathbf{p}} A(\hat{\mathbf{x}})-i \nabla A(\hat{\mathbf{x}})+A(\hat{\mathbf{x}})^2\right] .
    $$
    Derive a discrete (Lagrangian) path integral for $U$ using this ordering.
    \item Take the continuum limit and show that the first discrete integral leads to a continuum path integral with Lagrangian $L=\frac{1}{2} m \dot{\mathbf{x}}^2+A(\mathbf{x}) \dot{\mathbf{x}}$, and the second leads to $L=\frac{1}{2} m \dot{\mathbf{x}}^2+A(\mathbf{x}) \dot{\mathbf{x}}+\frac{i}{2 m} \nabla A(\mathbf{x})$.
\end{enumerate}

\paragraph{Solution} \begin{itemize}
\item[1.] The discrete path integral is 
\[
    \begin{aligned}
        \mel{\vb*{x}_f}{\ee^{- \ii H t}}{\vb*{x}_i} &= 
        \lim_{N \to \infty} \prod_{j=1}^{N-1} \int \dd[3]{\vb*{x}_j} 
        \prod_{j=1}^N \mel{\vb*{x}_j}{\ee^{- \ii \Delta t H}}{\vb*{x}_{j-1}} \\
        &= \lim_{N \to \infty} \prod_{j=1}^{N-1} \int \dd[3]{\vb*{x}_j} 
        \prod_{j=1}^N \mel{\vb*{x}_j}{\ee^{- \ii \Delta t ( 
            \hat{\vb*{p}}^2  - \hat{\vb*{p}} \cdot \vb*{A}(\vb*{x}_{j-1}) 
            - \vb*{A}(\vb*{x}_j) \cdot \hat{\vb*{p}} + \vb*{A}(\vb*{x}_{j-1})^2
        ) / 2m}}{\vb*{x}_{j-1}}.
    \end{aligned}
\]
Now we introduce a $\vb*{p}$ variable to eliminate the momentum operator:
\[
    \begin{aligned}
        &\quad \mel{\vb*{x}_j}{\ee^{- \ii \Delta t ( 
            \hat{\vb*{p}}^2  - \hat{\vb*{p}} \cdot \vb*{A}(\vb*{x}_{j-1}) 
            - \vb*{A}(\vb*{x}_j) \cdot \hat{\vb*{p}} + \vb*{A}(\vb*{x}_{j-1})^2
        ) / 2m}}{\vb*{x}_{j-1}} \\
        &= \int \dd[3]{\vb*{p}} \braket*{\vb*{x}_j}{\vb*{p}}
        \mel{\vb*{p}}{\ee^{- \ii \Delta t ( 
            \hat{\vb*{p}}^2  - \hat{\vb*{p}} \cdot \vb*{A}(\vb*{x}_{j-1}) 
            - \vb*{A}(\vb*{x}_j) \cdot \hat{\vb*{p}} + \vb*{A}(\vb*{x}_{j-1})^2
        ) / 2m}}{\vb*{x}_{j-1}} \\
        &= \int \frac{\dd[3]{\vb*{p}}}{(2\pi)^3} \ee^{\ii \vb*{p} \cdot \vb*{x}_j}
        \ee^{- \ii \Delta t ( 
            {\vb*{p}}^2  - {\vb*{p}} \cdot \vb*{A}(\vb*{x}_{j-1}) 
            - \vb*{A}(\vb*{x}_j) \cdot {\vb*{p}} + \vb*{A}(\vb*{x}_{j-1})^2
        ) / 2m}
        \ee^{- \ii \vb*{p} \cdot \vb*{x}_{j-1}} \\
        &= \ee^{- \ii \Delta t \vb*{A}(\vb*{x}_{j-1})^2 / 2m} 
        \int \frac{\dd[3]{\vb*{p}}}{(2\pi)^3} \ee^{- \frac{1}{2} \frac{\ii \Delta t}{m} \vb*{p}^2}
        \ee^{\ii \vb*{p} \cdot \left(
            \vb*{x}_j - \vb*{x}_{j-1} +
            \frac{\Delta t}{2m} (\vb*{A} (\vb*{x}_j) + \vb*{A}(\vb*{x}_{j-1}) )
        \right)} \\
        &= \ee^{- \ii \Delta t \vb*{A}(\vb*{x}_{j-1})^2 / 2m} \frac{1}{(2\pi)^3} \sqrt{\frac{(2\pi)^3}{(\ii \Delta t / m)^3}} \ee^{- \frac{1}{2} \frac{m}{\ii \Delta t} \left(
            \vb*{x}_j - \vb*{x}_{j-1} +
            \frac{\Delta t}{2m} (\vb*{A} (\vb*{x}_j) + \vb*{A}(\vb*{x}_{j-1}) )
        \right)^2 } \\
        &\approx \sqrt{\frac{- \ii m^3}{(2\pi)^3 \Delta t^3}} \ee^{\ii \Delta t \frac{m}{2} 
        \left(\frac{(\vb*{x}_j - \vb*{x}_{j-1})^2}{\Delta t^2} 
        + \frac{\vb*{x}_j - \vb*{x}_{j-1}}{\Delta t} \cdot \frac{\vb*{A}(\vb*{x}_j) + \vb*{A}(\vb*{x}_{j-1})}{2}\right)}.
    \end{aligned}
\]
Here in the last line we make the approximation that $\vb*{A}(\vb*{x}_j)$ and $\vb*{A}(\vb*{x}_{j-1})$
are close to each other,
so the two $\vb*{A}^2$ terms cancel with each other.
So the final discrete path integral is 
\begin{equation}
    \begin{aligned}
        \mel{\vb*{x}_f}{\ee^{- \ii H t}}{\vb*{x}_i} &= 
        \lim_{N \to \infty} \prod_{j=1}^{N-1} \left( \frac{- \ii m^3}{(2\pi)^3 \Delta t^3} \right)^{N/2}
        \int \dd[3]{\vb*{x}_j} \\
        &\quad \quad \cdot \ee^{\sum_{j=1}^N \ii \Delta t 
            \left( \frac{m}{2}  \frac{(\vb*{x}_j - \vb*{x}_{j-1})^2}{\Delta t^2} 
            + \frac{\vb*{x}_j - \vb*{x}_{j-1}}{\Delta t} \cdot \frac{\vb*{A}(\vb*{x}_j) + \vb*{A}(\vb*{x}_{j-1})}{2}\right)}.
    \end{aligned}
    \label{eq:magnetic-path-integral-1}
\end{equation}

\item[2.] The derivation is largely the same, but now in each time stp,
the $- 2 \hat{\vb*{p}} \cdot \vb*{A}(\vb*{x})$ term 
results in $- 2 \hat{\vb*{p}} \cdot \vb*{A}(\vb*{x}_{j-1})$,
and the result is 
\begin{equation}
    \begin{aligned}
        \mel{\vb*{x}_f}{\ee^{- \ii H t}}{\vb*{x}_i} &= 
        \lim_{N \to \infty} \prod_{j=1}^{N-1} \left( \frac{- \ii m^3}{(2\pi)^3 \Delta t^3} \right)^{N/2}
        \int \dd[3]{\vb*{x}_j} \\
        &\quad \quad \cdot \ee^{\sum_{j=1}^N \ii \Delta t 
            \left(\frac{m}{2} \frac{(\vb*{x}_j - \vb*{x}_{j-1})^2}{\Delta t^2} 
            + \frac{\vb*{x}_j - \vb*{x}_{j-1}}{\Delta t} \cdot \vb*{A}(\vb*{x}_{j-1}) + \frac{\ii}{2m} \div{\vb*{A} (\vb*{x}_{j-1})} \right)}.
    \end{aligned}
    \label{eq:magnetic-path-integral-2}
\end{equation}

\item[3.] We make the following replacements:
\[
    \frac{(\vb*{x}_j - \vb*{x}_{j-1})^2}{\Delta t^2} \longrightarrow \dot{\vb*{x}}^2, \quad 
    \sum_j \Delta t = \int \dd{t},
\]
and from \eqref{eq:magnetic-path-integral-1} we get 
\begin{equation}
    \mel{\vb*{x}_f}{\ee^{- \ii H t}}{\vb*{x}_i} = \int \fd{\vb*{x}} \ee^{\ii \int_{t_i}^{t_f} \dd{t}
    \left(  \frac{1}{2} m \dot{\vb*{x}}^2 + \dot{\vb*{x}} \cdot \vb*{A}\right) },
\end{equation} 
and from \eqref{eq:magnetic-path-integral-2} we get 
\begin{equation}
    \mel{\vb*{x}_f}{\ee^{- \ii H t}}{\vb*{x}_i} = \int \fd{\vb*{x}} \ee^{\ii \int_{t_i}^{t_f} \dd{t}
    \left(  \frac{1}{2} m \dot{\vb*{x}}^2 + \dot{\vb*{x}} \cdot \vb*{A} + \frac{\ii}{2m} \div{\vb*{A}} \right) }.
\end{equation}
So for the first path integral the Lagrangian is 
\begin{equation}
    L = \frac{1}{2} m \dot{\vb*{x}}^2 + \dot{\vb*{x}} \cdot \vb*{A},
\end{equation}
while for the second, it is 
\begin{equation}
    L = \frac{1}{2} m \dot{\vb*{x}}^2 + \dot{\vb*{x}} \cdot \vb*{A} + \frac{\ii}{2m} \div{\vb*{A}}.
\end{equation}


\end{itemize}

\paragraph{Problem 3} Consider the propagator $U\left(x_f, t_f ; x_i, t_i\right)$ for a harmonic oscillator $\hat{H}=\frac{\hat{p}^2}{2 m}+\frac{1}{2} m \omega^2 \hat{x}^2$.
\begin{enumerate}
    \item Compute $U$ by generalizing the free particle calculation from class.
    \item Write down the imaginary time evolution operator $U\left(x_f, \tau_f ; x_i, \tau_i\right)$ by analytical continuation.
    \item From the decay of $U(0, \beta ; 0,0)$ in the limit $\beta \rightarrow \infty$, determine the ground state energy.
    The following mathematical result may be useful: define $C_N$ as the tridiagonal $N \times N$ matrix
    $$
    C_N=\left(\begin{array}{cccc}
    2 \cos x & -1 & 0 & \cdots \\
    -1 & 2 \cos x & -1 & \cdots \\
    0 & -1 & 2 \cos x & \cdots \\
    \vdots & \vdots & \vdots & \ddots
    \end{array}\right),
    $$
    then we have $\operatorname{det} C_N=\frac{\sin (N+1) x}{\sin x}$.
\end{enumerate}

\paragraph{Solution} \begin{itemize}
\item[1.] The path integral can be derived similar to what has been done above:
\begin{equation}
    \mel*{x_f}{\ee^{- \ii H (t_f - t_i)}}{x_i} = 
    \lim_{N \to \infty} \left( \frac{m}{2\pi \ii \Delta t} \right)^{N/2} 
    \int \prod_{j=1}^{N-1} \dd{x_j}
    \ee^{\ii \Delta t \sum_{j=1}^N \left( \frac{m}{2} \frac{(x_j - x_{j-1})^2}{\Delta t^2} - \frac{1}{2} m \omega^2 x_{j-1}^2  \right)}.
\end{equation}
Again, we do the decomposition 
\begin{equation}
    x = x_{\text{cl}} + y,
\end{equation}
and the path integral becomes 
\[
    \begin{aligned}
        &\quad \mel*{x_f}{\ee^{- \ii H (t_f - t_i)}}{x_i} \\
        &= \ee^{\ii S_{\text{cl}}} 
        \lim_{N \to \infty} \left( \frac{m}{2\pi \ii \Delta t} \right)^{N/2} 
        \int \prod_{j=1}^{N-1} \dd{y_j}
        \ee^{\ii \Delta t \sum_{j=1}^N \left( \frac{m}{2} \frac{(y_j - y_{j-1})^2}{\Delta t^2} - \frac{1}{2} m \omega^2 y_{j-1}^2  \right)},
    \end{aligned}
\]
where $y_0 = y_N = 0$.
Thus the kernel of the Gaussian integral is
\[
    \vb{A} = \frac{m}{\Delta t} \pmqty{
        2 - \omega^2 \Delta t^2 & -1 & & \\
        -1 & 2 - \omega^2 \Delta t^2 & -1 & \\
        & -1 & \ddots \\
        & \ddots \\
        & & -1 & 2 - \omega^2 \Delta t^2
    },
\]
and the path integral is 
\begin{equation}
    \mel*{x_f}{\ee^{- \ii H (t_f - t_i)}}{x_i} = \ee^{\ii S_{\text{cl}}} 
    \lim_{N \to \infty} \left( \frac{m}{2\pi \ii \Delta t} \right)^{N/2} 
    \left( \frac{(2\pi)^{N-1}}{\det (- \ii \vb{A})} \right)^{1/2}.
\end{equation}
We find when $N$ is large,
\[
    \begin{aligned}
        &\quad \det \pmqty{
        2 - \omega^2 \Delta t^2 & -1 & & \\
        -1 & 2 - \omega^2 \Delta t^2 & -1 & \\
        & -1 & \ddots \\
        & \ddots \\
        & & -1 & 2 - \omega^2 \Delta t^2
        } \\
        &= \det \pmqty{
            2 \cos (\omega \Delta t) & -1 & & \\
            -1 & 2 \cos (\omega \Delta t) & -1 & \\
            & -1 & \ddots \\
            & \ddots \\
            & & -1 & 2 \cos (\omega \Delta t) 
        }
        &= \frac{\sin (N+1) \omega \Delta t}{\sin \omega \Delta t} = \frac{\sin \omega (t_f - t_i)}{\omega \Delta t}.
    \end{aligned}
\]
So 
\[
    \begin{aligned}
        \mel*{x_f}{\ee^{- \ii H (t_f - t_i)}}{x_i} &= \ee^{\ii S_{\text{cl}}} 
        \lim_{N \to \infty} \left( \frac{m}{2\pi \ii \Delta t} \right)^{N/2} 
        \left( \frac{(2\pi)^{N-1}}{ 
            \left(\frac{- \ii m }{\Delta t}\right)^{N-1}  
            \frac{\sin \omega (t_f - t_i)}{\omega \Delta t}
        } \right)^{1/2}.
    \end{aligned}
\]
Simplifying this equation, we get 
\begin{equation}
    \mel*{x_f}{\ee^{- \ii H (t_f - t_i)}}{x_i} = 
    \sqrt{\frac{m \omega}{2 \pi \ii \sin \omega (t_f - t_i)}} \ee^{\ii S_{\text{cl}}}.
\end{equation}

\item[2.] The Wick rotation is $\tau = \ii t$. 
So 
\[
    \begin{aligned}
        \sin \omega (t_f - t_i) &= \sin(- \ii \omega (\tau_f - \tau_i)) = - \ii \sinh(\omega (\tau_f - \tau_i)) \\
        &= \frac{ \ee^{ \omega (\tau_f - \tau_i) } - \ee^{- \omega (\tau_f - \tau_i)} }{2\ii}.
    \end{aligned}
\]
Similarly $S_{\text{cl}}$ should be changed into 
\[
    S_{\text{cl, im}} = -\ii \int \dd {\tau} \left( - \frac{1}{2} \left(\dv{x_{\text{cl}}}{\tau}\right)^2 
    - \frac{1}{2} m \omega^2 x_{\text{cl}}^2 \right).
\]
Thus after the Wick rotation, we get 
\begin{equation}
    U(x_f, \tau_f; x_i, \tau_i) = \sqrt{\frac{m \omega}{2 \pi \sinh (\omega (\tau_f - \tau_i))}} 
    \ee^{- \int_{\tau_i}^{\tau_f} \dd{\tau} \left( \frac{1}{2} m \left(\dv{x_{\text{cl}}}{\tau} \right)^2 
    + \frac{1}{2} m \omega^2 x_{\text{cl}}^2 \right)}.
\end{equation}

\item[3.] In this case the classical configuration is $x_{\text{cl}} = 0$:
that's the trajectory with the boundary conditions $x_f = x_i = 0$.
So we have 
\begin{equation}
    U(0, \beta; 0, 0) = \sqrt{\frac{m \omega}{2 \pi \sinh (\omega (\tau_f - \tau_i))}} \sim 
    \text{const} \times \sqrt{\frac{1}{\ee^{\omega (\tau_f - \tau_i)}}} \sim \ee^{- \frac{1}{2} \omega (t_f - t_i)} .
\end{equation}
Therefore the ground state energy 
(which is the coefficient $\alpha$ in the $\ee^{- \alpha t}$ damping) is $\omega / 2$.

\end{itemize}

\paragraph{Problem 4} Consider a single particle in a periodic potential: $\hat{H}_0=\frac{\hat{\mathbf{p}}^2}{2 m}+V(\hat{\mathbf{x}})$, where $V(\mathbf{x}+\mathbf{a})=V(\mathbf{x})$ for Bravais lattice vector a. According to Bloch's theorem, the eigenstates are of the form $\psi_{n, \mathbf{k}}(\mathbf{x})=e^{i \mathbf{k} \cdot \mathbf{x}} u_{n \mathbf{k}}(\mathbf{x})$ where $u_{n k}(\mathbf{x})$ are periodic functions $\left(u_{n \mathbf{k}}(\mathbf{x}+a)=u_{n \mathbf{k}}(\mathbf{x})\right)$. Here $\mathbf{k}$ is the lattice momentum in the Brillouin zone $(\mathrm{BZ})$ and $n$ is the band index. Denote the corresponding energy eigenvalue by $\epsilon_n(\mathbf{k})$. We do not need to know explicitly the Bloch wavefunctions $\psi_{m \mathbf{k}}$ and $\epsilon_n(\mathbf{k})$, so will keep them general.

In this problem we will study the semiclassical dynamics of a wave packet, of the form $\int_{\mathrm{BZ}} c(\mathbf{k}) \psi_{n \mathbf{k}}(\mathbf{x})$. Here the wave packet is composed entirely of states from a single band $n$, and there is a large gap $\Delta$ separating $n$ from neighboring bands, so we can ignore the other bands. From now on we drop the band index $n$, and denote $\left|\psi_{n \mathbf{k}}\right\rangle$ by $|\mathbf{k}\rangle,\left|u_{n \mathbf{k}}\right\rangle$ by $\left|u_{\mathbf{k}}\right\rangle$.
\begin{enumerate}
    \item 1. It is useful to analyze the system in the presence of a weak harmonic potential, and a weak (uniform) electric field:
    $$
    \hat{H}=\hat{H}_0+\frac{1}{2 \alpha} \hat{\mathbf{x}}^2-\mathbf{E} \cdot \hat{\mathbf{x}}
    $$
    Construct a path integral in the $\mathbf{k}$-space for the propagator $\left\langle\mathbf{k}_f\left|e^{-i \hat{H}}\right| \mathbf{k}_i\right\rangle$ for electron in one band, and show that the effective Lagrangian takes the form
    $$
    L_{\mathrm{eff}}=\mathcal{A}(k) \cdot \dot{\mathbf{k}}+\mathcal{F}(\dot{\mathbf{k}}, \mathbf{k}) .
    $$
    where $\mathcal{A}(\mathbf{k})=i\left\langle u_{\mathbf{k}}\left|\nabla_{\mathbf{k}}\right| u_{\mathbf{k}}\right\rangle$ is the "Berry connection" of the band. Find out $\mathcal{F}(\dot{\mathbf{k}}, \mathbf{k})$.
    Hint: To describe electron dynamics in one band, the resolution of identity should only involve states in the band.
    \item Find $\boldsymbol{\pi}$, the momentum canonically conjugate to $\mathbf{k}$, and compute the effective Hamiltonian $H_{\text {eff }}(\mathbf{k}, \boldsymbol{\pi})$.
    \item Find the position $\mathbf{x}$ in terms of $\mathbf{k}, \boldsymbol{\pi}$ by differentiating $H_{\text {eff }}$ with respect to $\mathbf{E}$.
    \item Find the classical equations of motion for $H_{\text {eff }}$ and express them in terms of $\mathbf{x}, \mathbf{k}$. Taking the limit of vanishing harmonic potential $\alpha \rightarrow \infty$, derive the semiclassical equations of motion
    $$
    \begin{aligned}
    &\dot{\mathbf{x}}=\nabla_{\mathbf{k}} \epsilon(\mathbf{k})-\mathbf{E} \times \Omega(\mathbf{k}) \\
    &\dot{\mathbf{k}}=\mathbf{E}
    \end{aligned}
    $$
    Here $\boldsymbol{\Omega}(\mathbf{k})=\nabla_{\mathbf{k}} \times \mathcal{A}$ is the "Berry curvature". Notice that there is an "anomalous velocity" term in $\dot{\mathbf{x}}$ coming from Berry phase effect. This term is neglected in many standard textbooks (e.g. Ashcroft and Mermin)
    \item As an example, consider a $2 \mathrm{D}$ particle in a uniform perpendicular magnetic field $B$. This system can be analyzed in terms of Bloch states if we work in a periodic gauge with a unit cell of area $\frac{2 \pi}{B}$, again setting electric charge unit and speed of light to 1 (we do not need the specific form of this gauge). The resulting band structure consists of perfectly flat bands (Landau levels) with $\varepsilon(\mathbf{k})=$ const., and $\Omega(\mathbf{k})=\Omega_0$ also a constant. Let us consider the dynamics of electrons in one Landau level. Find $\Omega_0$ in terms of $B$ by comparing the semi-classical equations (12) to the behavior of a classical particle in electric and magnetic fields.
    \item The integral of the Berry curvature over a closed surface is always quantized in multiples of $2 \pi$. In particular, this is true for the integral of the Berry curvature over the $2 \mathrm{D}$ Brillouin zone: $\int \mathrm{d}^2 \mathbf{k} \Omega(\mathbf{k})=2 \pi C$, where the integer $C$ is known as the "Chern number" of the band. Find the Chern number of the Landau level.
\end{enumerate}

\paragraph{Solution} \begin{itemize}
\item[1.] We do the Trotter decomposition again:
\[
    \mel{\vb*{k}_f}{\ee^{- \ii H t}}{\vb*{k}_i} = \lim_{N \to \infty}
    \prod_{j=1}^{N-1} \int \frac{V}{(2\pi)^3} \dd[3]{\vb*{k}_j} \cdot \prod_{j=1}^{N} 
    \mel{\vb*{k}_j}{\ee^{- \ii \Delta t H}}{\vb*{k}_{j-1}}, \quad 
    \vb*{k}_{0} = \vb*{k}_i, \quad \vb*{k}_N = \vb*{k}_f.
\]
Each time step is given by 
\[
    \begin{aligned}
        &\quad \mel{\vb*{k}_j}{\ee^{- \ii \Delta t (H_0 + \hat{\vb*{x}}^2 / 2 \alpha - \vb*{E} \cdot \hat{\vb*{x}})}}{\vb*{k}_{j-1}} \\
        &= \mel{\vb*{k}_j}{\ee^{- \ii \Delta t ( \hat{\vb*{x}}^2 / 2 \alpha - \vb*{E} \cdot \hat{\vb*{x}})}}{\vb*{k}_{j-1}} \ee^{- \Delta t \epsilon_{\vb*{k}_{j-1}}} \\
        &= \ee^{- \Delta t \epsilon_{\vb*{k}_{j-1}}} \int \dd[3]{\vb*{r}} u^*_{\vb*{k}_j}(\vb*{r}) \ee^{- \ii \vb*{k}_j \cdot \vb*{r}}
        \ee^{- \ii \Delta t (\vb*{r}^2 / 2\alpha - \vb*{E} \cdot \vb*{r})} u_{\vb*{k}_{j-1}}(\vb*{r}) \ee^{\ii \vb*{k}_{j-1} \cdot \vb*{r}} \\
        &= \ee^{- \Delta t \epsilon_{\vb*{k}_{j-1}}} \int \dd[3]{\vb*{r}} u^*_{\vb*{k}_{j}}(\vb*{r}) u_{\vb*{k}_{j-1}}(\vb*{r})
        \ee^{- \frac{1}{2} \frac{\ii \Delta t}{\alpha} \vb*{r}^2 + \ii \vb*{r} \cdot (\Delta t \vb*{E} + \vb*{k}_{j-1} - \vb*{k}_j) }.
    \end{aligned}
\]
Due to the strong variation of the potential in each unit cell,
$u_{\vb*{k}}(\vb*{r})$ varies much quicker than $\ee^{\ii \vb*{k} \cdot \vb*{r}}$.%
\footnote{
    Another way to derive the semiclassical equations is to consider a wave pocket.
    In that case, the envelope varies \emph{slower} than $\ee^{\ii \vb*{k} \cdot \vb*{r}}$.
}
Thus, we can separate the two scales and have
\[
    \begin{aligned}
        &\quad \int \dd[3]{\vb*{r}} u^*_{\vb*{k}_{j}}(\vb*{r}) u_{\vb*{k}_{j-1}}(\vb*{r})
        \ee^{- \frac{1}{2} \frac{\ii \Delta t}{\alpha} \vb*{r}^2 + \ii \vb*{r} \cdot (\Delta t \vb*{E} + \vb*{k}_{j-1} - \vb*{k}_j) } \\
        &= \frac{1}{V_{\text{u.c.}}} \int_{\text{u.c.}} \dd[3]{\vb*{r}}
        u^*_{\vb*{k}_{j}}(\vb*{r}) u_{\vb*{k}_{j-1}}(\vb*{r})
        \int \dd[3]{\vb*{r}} \ee^{- \frac{1}{2} \frac{\ii \Delta t}{\alpha} \vb*{r}^2 + \ii \vb*{r} \cdot (\Delta t \vb*{E} + \vb*{k}_{j-1} - \vb*{k}_j) } ,
    \end{aligned}
\]
and the Gaussian integral on the RHS can be evaluated as 
\[
    \begin{aligned}
        &\int \dd[3]{\vb*{r}} \ee^{- \frac{1}{2} \frac{\ii \Delta t}{\alpha} \vb*{r}^2 + \ii \vb*{r} \cdot (\Delta t \vb*{E} + \vb*{k}_{j-1} - \vb*{k}_j) }  \\
        &= \sqrt{\frac{(2\pi)^3}{(\ii \Delta t / \alpha)^3}} \ee^{\frac{1}{2} \frac{\alpha}{\ii \Delta t} (\ii (\Delta t \vb*{E} + \vb*{k}_{j-1} - \vb*{k}_j))^2 } \\
        &= \sqrt{\frac{(2\pi)^3}{(\ii \Delta t / \alpha)^3}} \ee^{ \frac{\ii \alpha}{2} (\vb*{E} - \dot{\vb*{k}})^2 \Delta t }.
    \end{aligned}
\]
Thus 
\[
    \begin{aligned}
        \mel{\vb*{k}_f}{\ee^{- \ii H t}}{\vb*{k}_i} &= \lim_{N \to \infty} \mathcal{N} \prod_{j=1}^{N-1} 
        \int \dd[3]{\vb*{k}_j} \ee^{- \ii \Delta t \epsilon_{\vb*{k}_{j-1}}} \ee^{\frac{\ii \Delta t \alpha}{2} (\dot{\vb*{k}} - \vb*{E})^2}
        \braket*{u_{\vb*{k}_j}}{u_{\vb*{k}_{j-1}}} \\
        &= \lim_{N \to \infty} \mathcal{N} \prod_{j=1}^{N-1} 
        \int \dd[3]{\vb*{k}_j} \ee^{- \ii \Delta t \epsilon_{\vb*{k}_{j-1}}} \ee^{\frac{\ii \Delta t \alpha}{2} (\dot{\vb*{k}} - \vb*{E})^2}
        (1 - \Delta t \dot{\vb*{k}} \cdot \mel*{u_{\vb*{k}_{j}}}{\grad_{\vb*{k}_{j}}}{u_{\vb*{k}_{j}}}) \\
        &= \lim_{N \to \infty} \mathcal{N} \prod_{j=1}^{N-1} 
        \int \dd[3]{\vb*{k}_j} \ee^{- \ii \Delta t \epsilon_{\vb*{k}_{j-1}} + \frac{\ii \Delta t \alpha}{2} (\dot{\vb*{k}} - \vb*{E})^2 - \Delta t \dot{\vb*{k}} \cdot \mel*{u_{\vb*{k}_{j}}}{\grad_{\vb*{k}_{j}}}{u_{\vb*{k}_{j}}}} \\
        &= \lim_{N \to \infty}  \left(\mathcal{N} \prod_{j=1}^{N-1} 
        \int \dd[3]{\vb*{k}_j}\right) \ee^{\ii \Delta t (\sum_{j} - \epsilon_{\vb*{k}_{j-1}} + \frac{ \alpha}{2} (\dot{\vb*{k}} - \vb*{E})^2 + \ii \dot{\vb*{k}} \cdot \mel*{u_{\vb*{k}_{j}}}{\grad_{\vb*{k}_{j}}}{u_{\vb*{k}_{j}}})}.
    \end{aligned}
\]
Putting all normalization factors into the measure, we get 
\begin{equation}
    \begin{aligned}
        \mel{\vb*{k}_f}{\ee^{- \ii H t}}{\vb*{k}_i} = \int \mathcal{D} \vb*{k} \ee^{\ii \int_i^f \dd{t} L_{\text{eff}}}, \\
        L_{\text{eff}} = \dot{\vb*{k}} \cdot \vb*{\mathcal{A}} + \frac{\alpha}{2} (\dot{\vb*{k}} - \vb*{E})^2 - \epsilon_{\vb*{k}}.
    \end{aligned}
\end{equation}
So we find 
\begin{equation}
    \mathcal{F}(\vb*{k}, \dot{\vb*{k}}) = \frac{\alpha}{2} (\dot{\vb*{k}} - \vb*{E})^2 - \epsilon_{\vb*{k}}.
\end{equation}

\item[2.] We have 
\begin{equation}
    \vb*{\pi} = \pdv{L_{\text{eff}}}{\dot{\vb*{k}}} = \vb*{\mathcal{A}} + \alpha (\dot{\vb*{k}} - \vb*{E}).
\end{equation}
So 
\[
    \begin{aligned}
        H_{\text{eff}} &= \dot{\vb*{k}} \cdot \vb*{\pi} - L_{\text{eff}} \\
        &= \frac{1}{2} \alpha \dot{\vb*{k}}^2 - \frac{1}{2} \alpha \vb*{E}^2 + \epsilon_{\vb*{k}}.
    \end{aligned}
\]
Replacing $\dot{\vb*{k}}$ by $\vb*{\pi}$, 
we get 
\[
    \begin{aligned}
        H_{\text{eff}} &= \frac{1}{2} \alpha \left( \frac{\vb*{\pi} - \vb*{\mathcal{A}}}{\alpha} + \vb*{E} \right)^2
        - \frac{1}{2} \alpha \vb*{E}^2 + \epsilon_{\vb*{k}} \\
        &= \frac{(\vb*{\pi} - \vb*{\mathcal{A}})^2}{2 \alpha} + (\vb*{\pi} - \vb*{\mathcal{A}}) \cdot \vb*{E} + \epsilon_{\vb*{k}}.
    \end{aligned}
\]
So the answer is 
\begin{equation}
    H_{\text{eff}} = \frac{(\vb*{\pi} - \vb*{\mathcal{A}})^2}{2 \alpha} + (\vb*{\pi} - \vb*{\mathcal{A}}) \cdot \vb*{E} + \epsilon_{\vb*{k}}.
\end{equation}

\item[3.] We can interpret $-\vb*{\pi}$ as some sort of ``position'',
because 
\[
    \comm{\vb*{k}}{\vb*{\pi}} = 1 \Leftrightarrow \comm{- \vb*{\pi}}{\vb*{k}} = 1,
\]
so we replace $\vb*{\pi}$ by $- \vb*{x}$, and thus in the $\alpha \to \infty$ limit, we have
\begin{equation}
    H_{\text{eff}} = - (\vb*{x} + \vb*{\mathcal{A}}) \cdot \vb*{E} + \epsilon_{\vb*{k}}.
\end{equation}

\item[4.] We have 
\item[] \begin{equation}
    \dot{\vb*{k}} = - \pdv{H_{\text{eff}}}{\vb*{x}} = \vb*{E}.
    \label{eq:k-eom}
\end{equation}
Also
\[
    \dot{\vb*{x}} = \pdv{H_{\text{eff}}}{\vb*{k}} = \grad_{\vb*{k}} \epsilon_{\vb*{k}} - \grad_{\vb*{k}} (\vb*{E} \cdot \vb*{\mathcal{A}}).
\]
By vector analysis formula, and by the condition that $\vb*{E}$ is a constant,
we have 
\[
    \grad_{\vb*{k}} (\vb*{E} \cdot \mathcal{\vb*{A}}) = \vb*{E} \times (\grad_{\vb*{k}} \times \vb*{\mathcal{A}}) + (\vb*{E} \cdot \grad_{\vb*{k}}) \vb*{\mathcal{A}},
\]
and since $\vb*{E} = \dot{\vb*{k}}$, we have 
\[
    (\vb*{E} \cdot \grad_{\vb*{k}}) \vb*{\mathcal{A}}
    = \dv{\vb*{k}}{t} \cdot \pdv{\vb*{k}} \vb*{\mathcal{A}} = \dv{\vb*{\mathcal{A}}}{t} = 0,
\]
so finally we get
\begin{equation}
    \dot{\vb*{x}} = \grad_{\vb*{k}} \epsilon_{\vb*{k}} - \vb*{E} \times \vb*{\Omega},
    \label{eq:x-eom}
\end{equation}
where 
\begin{equation}
    \vb*{\Omega} = \grad_{\vb*{k}} \times \vb*{\mathcal{A}}.
\end{equation}


\item[5.] From \eqref{eq:x-eom} and \eqref{eq:k-eom} we have%
\footnote{
    Here we assume there is a very weak electric field $\vb*{E}$,
    so we can put \eqref{eq:k-eom} and \eqref{eq:x-eom} into one equation,
    and then we let $\vb*{E} \to 0$.
}
\[
    \dot{\vb*{x}} = - \vb*{E} \times \vb*{\Omega} = - \dot{\vb*{k}} \times \vb*{\Omega}, 
\]
and therefore 
\begin{equation}
    \vb*{\Omega} \times \dot{\vb*{x}} = - \Omega^2 \dot{\vb*{k}} + (\vb*{\Omega} \cdot \dot{\vb*{k}}) \vb*{\Omega}.
    \label{eq:eom-iqhe}
\end{equation}
On the other hand, the classical EOM is (here $e = 1$)
\begin{equation}
    \dot{\vb*{p}} = - \dot{\vb*{x}} \times \vb*{B}.
\end{equation}
So 
\begin{equation}
    \vb*{\Omega} = \Omega_0 \vu*{z}, \quad \Omega_0 = \frac{1}{B}.
\end{equation}

\item[6.] The size of the first Brillouin zone is 
\[
    \frac{(2\pi)^2}{2\pi / B} = 2\pi B.
\]
So 
\[
    2\pi C = \int \dd[2]{\vb*{k}} \Omega = 2\pi B \cdot \Omega_0 = 2\pi,
\]
and thus the Chern number of the Landau level is 1.

\end{itemize}

\end{document}