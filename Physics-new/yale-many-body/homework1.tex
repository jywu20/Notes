\documentclass[hyperref, a4paper]{article}

\usepackage{geometry}
\usepackage{titling}
\usepackage{titlesec}
% No longer needed, since we will use enumitem package
% \usepackage{paralist}
\usepackage{enumitem}
\usepackage{footnote}
\usepackage{enumerate}
\usepackage{amsmath, amssymb, amsthm}
\usepackage{mathtools}
\usepackage{bbm}
\usepackage{cite}
\usepackage{graphicx}
\usepackage{subfigure}
\usepackage{physics}
\usepackage{tensor}
\usepackage{siunitx}
\usepackage[version=4]{mhchem}
\usepackage{tikz}
\usepackage{xcolor}
\usepackage{listings}
\usepackage{autobreak}
\usepackage[ruled, vlined, linesnumbered]{algorithm2e}
\usepackage{nameref,zref-xr}
\zxrsetup{toltxlabel}
\usepackage[colorlinks,unicode]{hyperref} % , linkcolor=black, anchorcolor=black, citecolor=black, urlcolor=black, filecolor=black
\usepackage[most]{tcolorbox}
\usepackage{prettyref}

% Page style
\geometry{left=3.18cm,right=3.18cm,top=2.54cm,bottom=2.54cm}
\titlespacing{\paragraph}{0pt}{1pt}{10pt}[20pt]
\setlength{\droptitle}{-5em}
\preauthor{\vspace{-10pt}\begin{center}}
\postauthor{\par\end{center}}

% More compact lists 
\setlist[itemize]{
    itemindent=17pt, 
    leftmargin=1pt,
    listparindent=\parindent,
    parsep=0pt,
}

% Math operators
\DeclareMathOperator{\timeorder}{\mathcal{T}}
\DeclareMathOperator{\diag}{diag}
\DeclareMathOperator{\legpoly}{P}
\DeclareMathOperator{\primevalue}{P}
\DeclareMathOperator{\sgn}{sgn}
\newcommand*{\ii}{\mathrm{i}}
\newcommand*{\ee}{\mathrm{e}}
\newcommand*{\const}{\mathrm{const}}
\newcommand*{\suchthat}{\quad \text{s.t.} \quad}
\newcommand*{\argmin}{\arg\min}
\newcommand*{\argmax}{\arg\max}
\newcommand*{\normalorder}[1]{: #1 :}
\newcommand*{\pair}[1]{\langle #1 \rangle}
\newcommand*{\fd}[1]{\mathcal{D} #1}
\DeclareMathOperator{\bigO}{\mathcal{O}}

% TikZ setting
\usetikzlibrary{arrows,shapes,positioning}
\usetikzlibrary{arrows.meta}
\usetikzlibrary{decorations.markings}
\tikzstyle arrowstyle=[scale=1]
\tikzstyle directed=[postaction={decorate,decoration={markings,
    mark=at position .5 with {\arrow[arrowstyle]{stealth}}}}]
\tikzstyle ray=[directed, thick]
\tikzstyle dot=[anchor=base,fill,circle,inner sep=1pt]

% Algorithm setting
% Julia-style code
\SetKwIF{If}{ElseIf}{Else}{if}{}{elseif}{else}{end}
\SetKwFor{For}{for}{}{end}
\SetKwFor{While}{while}{}{end}
\SetKwProg{Function}{function}{}{end}
\SetArgSty{textnormal}

\newcommand*{\concept}[1]{{\textbf{#1}}}

% Embedded codes
\lstset{basicstyle=\ttfamily,
  showstringspaces=false,
  commentstyle=\color{gray},
  keywordstyle=\color{blue}
}

% Reference formatting
\newrefformat{fig}{Figure~\ref{#1}}

% Color boxes
\tcbuselibrary{skins, breakable, theorems}
\newtcbtheorem[number within=section]{warning}{Warning}%
  {colback=orange!5,colframe=orange!65,fonttitle=\bfseries, breakable}{warn}
\newtcbtheorem[number within=section]{note}{Note}%
  {colback=green!5,colframe=green!65,fonttitle=\bfseries, breakable}{note}
\newtcbtheorem[number within=section]{info}{Info}%
  {colback=blue!5,colframe=blue!65,fonttitle=\bfseries, breakable}{info}

\newenvironment{shelldisplay}{\begin{lstlisting}}{\end{lstlisting}}

\title{Many-body Physics Homework 2}
\author{Jinyuan Wu}

\begin{document}

\maketitle

\paragraph{Problem 4} 

\paragraph{Solution} \begin{itemize}
\item[(a)] We do the Trotter decomposition again:
\[
    \mel{\vb*{k}_f}{\ee^{- \ii H t}}{\vb*{k}_i} = \lim_{N \to \infty}
    \prod_{j=1}^{N-1} \int \frac{V}{(2\pi)^3} \dd[3]{\vb*{k}_j} \cdot \prod_{j=1}^{N} 
    \mel{\vb*{k}_j}{\ee^{- \ii \Delta t H}}{\vb*{k}_{j-1}}, \quad 
    \vb*{k}_{0} = \vb*{k}_i, \quad \vb*{k}_N = \vb*{k}_f.
\]
Each time step is given by 
\[
    \begin{aligned}
        &\quad \mel{\vb*{k}_j}{\ee^{- \ii \Delta t (H_0 + \hat{\vb*{x}}^2 / 2 \alpha - \vb*{E} \cdot \hat{\vb*{x}})}}{\vb*{k}_{j-1}} \\
        &= \mel{\vb*{k}_j}{\ee^{- \ii \Delta t ( \hat{\vb*{x}}^2 / 2 \alpha - \vb*{E} \cdot \hat{\vb*{x}})}}{\vb*{k}_{j-1}} \ee^{- \Delta t \epsilon_{\vb*{k}_{j-1}}} \\
        &= \ee^{- \Delta t \epsilon_{\vb*{k}_{j-1}}} \int \dd[3]{\vb*{r}} u^*_{\vb*{k}_j}(\vb*{r}) \ee^{- \ii \vb*{k}_j \cdot \vb*{r}}
        \ee^{- \ii \Delta t (\vb*{r}^2 / 2\alpha - \vb*{E} \cdot \vb*{r})} u_{\vb*{k}_{j-1}}(\vb*{r}) \ee^{\ii \vb*{k}_{j-1} \cdot \vb*{r}} \\
        &= \ee^{- \Delta t \epsilon_{\vb*{k}_{j-1}}} \int \dd[3]{\vb*{r}} u^*_{\vb*{k}_{j}}(\vb*{r}) u_{\vb*{k}_{j-1}}(\vb*{r})
        \ee^{- \frac{1}{2} \frac{\ii \Delta t}{\alpha} \vb*{r}^2 + \ii \vb*{r} \cdot (\Delta t \vb*{E} + \vb*{k}_{j-1} - \vb*{k}_j) }.
    \end{aligned}
\]
The semi-classical dynamics only works when $\psi_{\vb*{k}}(\vb*{r})$ is 
``concentrated'' enough in the reciprocal space,
which means $u_{\vb*{k}}(\vb*{r})$ should be very smooth compared with $\ee^{\ii \vb*{k} \cdot \vb*{r}}$
(or otherwise the picture of an electron with a certain momentum traveling in the material is simply wrong).
Thus, we have 
\[
    \begin{aligned}
        &\quad \int \dd[3]{\vb*{r}} u^*_{\vb*{k}_{j}}(\vb*{r}) u_{\vb*{k}_{j-1}}(\vb*{r})
        \ee^{- \frac{1}{2} \frac{\ii \Delta t}{\alpha} \vb*{r}^2 + \ii \vb*{r} \cdot (\Delta t \vb*{E} + \vb*{k}_{j-1} - \vb*{k}_j) } \\
        &= \frac{1}{V_{\text{u.c.}}} \int_{\text{u.c.}} \dd[3]{\vb*{r}}
        u^*_{\vb*{k}_{j}}(\vb*{r}) u_{\vb*{k}_{j-1}}(\vb*{r})
        \int \dd[3]{\vb*{r}} \ee^{- \frac{1}{2} \frac{\ii \Delta t}{\alpha} \vb*{r}^2 + \ii \vb*{r} \cdot (\Delta t \vb*{E} + \vb*{k}_{j-1} - \vb*{k}_j) } ,
    \end{aligned}
\]
and the Gaussian integral on the RHS can be evaluated as 
\[
    \begin{aligned}
        &\int \dd[3]{\vb*{r}} \ee^{- \frac{1}{2} \frac{\ii \Delta t}{\alpha} \vb*{r}^2 + \ii \vb*{r} \cdot (\Delta t \vb*{E} + \vb*{k}_{j-1} - \vb*{k}_j) }  \\
        &= \sqrt{\frac{(2\pi)^3}{(\ii \Delta t / \alpha)^3}} \ee^{\frac{1}{2} \frac{\alpha}{\ii \Delta t} (\ii (\Delta t \vb*{E} + \vb*{k}_{j-1} - \vb*{k}_j))^2 } \\
        &= \sqrt{\frac{(2\pi)^3}{(\ii \Delta t / \alpha)^3}} \ee^{ \frac{\ii \alpha}{2} (\vb*{E} - \dot{\vb*{k}})^2 \Delta t }.
    \end{aligned}
\]
Thus 
\[
    \mel{\vb*{k}_f}{\ee^{- \ii H t}}{\vb*{k}_i} = \lim_{N \to \infty} \mathcal{N} \prod_{j=1}^{N-1} 
    \int \dd[3]{\vb*{k}_j} \frac{1}{V_{\text{u.c.}}} \int \dd[3]{\vb*{r}} u^*
\]
For 
Putting all normalization factors into the measure, we get 
\begin{equation}
    \mel{\vb*{k}_f}{\ee^{- \ii H t}}{\vb*{k}_i} = \int \mathcal{D} \vb*{k} 
\end{equation}
\item[] 
\end{itemize}

\end{document}