\documentclass[hyperref, a4paper]{article}

\usepackage{geometry}
\usepackage{titling}
\usepackage{titlesec}
% No longer needed, since we will use enumitem package
% \usepackage{paralist}
\usepackage{enumitem}
\usepackage{footnote}
%\usepackage{enumerate}
\usepackage{amsmath, amssymb, amsthm}
\usepackage{mathtools}
\usepackage{bbm}
\usepackage{cite}
\usepackage{graphicx}
\usepackage{subfigure}
\usepackage{physics}
\usepackage{tensor}
\usepackage{siunitx}
\usepackage[version=4]{mhchem}
\usepackage{tikz}
\usepackage{xcolor}
\usepackage{listings}
\usepackage{autobreak}
\usepackage[ruled, vlined, linesnumbered]{algorithm2e}
\usepackage{nameref,zref-xr}
\zxrsetup{toltxlabel}
\usepackage[colorlinks,unicode]{hyperref} % , linkcolor=black, anchorcolor=black, citecolor=black, urlcolor=black, filecolor=black
\usepackage[most]{tcolorbox}
\usepackage{prettyref}

% Page style
\geometry{left=3.18cm,right=3.18cm,top=2.54cm,bottom=2.54cm}
\titlespacing{\paragraph}{0pt}{1pt}{10pt}[20pt]
\setlength{\droptitle}{-5em}
%\preauthor{\vspace{-10pt}\begin{center}}
%\postauthor{\par\end{center}}

% More compact lists 
\setlist[itemize]{
    itemindent=17pt, 
    leftmargin=1pt,
    listparindent=\parindent,
    parsep=0pt,
}

% Math operators
\DeclareMathOperator{\timeorder}{\mathcal{T}}
\DeclareMathOperator{\diag}{diag}
\DeclareMathOperator{\legpoly}{P}
\DeclareMathOperator{\primevalue}{P}
\DeclareMathOperator{\sgn}{sgn}
\newcommand*{\ii}{\mathrm{i}}
\newcommand*{\ee}{\mathrm{e}}
\newcommand*{\const}{\mathrm{const}}
\newcommand*{\suchthat}{\quad \text{s.t.} \quad}
\newcommand*{\argmin}{\arg\min}
\newcommand*{\argmax}{\arg\max}
\newcommand*{\normalorder}[1]{: #1 :}
\newcommand*{\pair}[1]{\langle #1 \rangle}
\newcommand*{\fd}[1]{\mathcal{D} #1}
\DeclareMathOperator{\bigO}{\mathcal{O}}

% TikZ setting
\usetikzlibrary{arrows,shapes,positioning}
\usetikzlibrary{arrows.meta}
\usetikzlibrary{decorations.markings}
\tikzstyle arrowstyle=[scale=1]
\tikzstyle directed=[postaction={decorate,decoration={markings,
    mark=at position .5 with {\arrow[arrowstyle]{stealth}}}}]
\tikzstyle ray=[directed, thick]
\tikzstyle dot=[anchor=base,fill,circle,inner sep=1pt]

% Algorithm setting
% Julia-style code
\SetKwIF{If}{ElseIf}{Else}{if}{}{elseif}{else}{end}
\SetKwFor{For}{for}{}{end}
\SetKwFor{While}{while}{}{end}
\SetKwProg{Function}{function}{}{end}
\SetArgSty{textnormal}

\newcommand*{\concept}[1]{{\textbf{#1}}}

% Embedded codes
\lstset{basicstyle=\ttfamily,
  showstringspaces=false,
  commentstyle=\color{gray},
  keywordstyle=\color{blue}
}

% Reference formatting
\newrefformat{fig}{Figure~\ref{#1}}

% Color boxes
\tcbuselibrary{skins, breakable, theorems}
\newtcbtheorem[number within=section]{warning}{Warning}%
  {colback=orange!5,colframe=orange!65,fonttitle=\bfseries, breakable}{warn}
\newtcbtheorem[number within=section]{note}{Note}%
  {colback=green!5,colframe=green!65,fonttitle=\bfseries, breakable}{note}
\newtcbtheorem[number within=section]{info}{Info}%
  {colback=blue!5,colframe=blue!65,fonttitle=\bfseries, breakable}{info}

\newenvironment{shelldisplay}{\begin{lstlisting}}{\end{lstlisting}}

\title{Homework 2}
\author{Jinyuan Wu}

\begin{document}

\maketitle

\paragraph{Problem 1} Suppose we have a quantum rotor (or particle on a circle) with the following imaginary-time action:
$$
S[\varphi]=\int_0^T \mathrm{~d} \tau\left(\frac{1}{2} \dot{\varphi}^2-i \frac{\theta}{2 \pi} \dot{\varphi}+V(\varphi)\right) .
$$
Here the potential $V(\varphi)=V_0(1-\cos \varphi)$. We will use semi-classical approximation to study the $\theta$ dependence of the ground state energy. For this purpose it is useful to consider the boundary condition $\varphi_i=\varphi_f=0$, so the path integral computes the imaginary-time propagator $\left\langle 0\left|e^{-T \hat{H}}\right| 0\right\rangle$.
1. Show that the equation of motion admits instanton solutions corresponding to the particle going around the circle. Calculate the action of a single instanton $S_0(\theta)$. Hint: there are two distinct single instaton solutions.
2. Calculate the path integral using semi-classical approximation, by summing over all instanton solutions under the "dilute gas" approximation. Following what we did in class for double well potential, you may introduce a parameter $K$ for the contributions of Gaussian fluctuations and assume that $K$ does not depend on $\theta$. You do not need to compute $K$.
3. Determine the $\theta$ dependence of the ground state energy $E_0$.

\paragraph{Solution} \begin{itemize}
\item[1.] The EOM is 
\[
    \dv{t} (\dot{\varphi} - \ii \theta / 2\pi) = \dv{\varphi} V_0 (1 - \cos \varphi),
\]
\begin{equation}
    \ddot{\varphi} = V_0 \sin \varphi.
\end{equation}
Integrating over $\varphi$, we have 
\begin{equation}
    \frac{1}{2} \dot{\varphi}^2 = - V_0 \cos \varphi + C.
\end{equation}
The range of $C$ is between $\pm V_0$,
because it corresponds to the $\varphi$ when $\dot{\varphi} = 0$.
The boundary condition that when $\tau \to \pm \infty$, 
$\varphi$ stays zero,
so we have $\varphi = 0$ and $\dot{\varphi} = 0$ in the two limits,
so $C = V_0$,
and therefore 
\begin{equation}
    \frac{1}{2} \dot{\varphi}^2 = \underbrace{V_0 (1 - \cos \varphi)}_{V(\varphi)} ,
\end{equation}
\[
    \pm \sqrt{2 V_0} \tau = \int \frac{\dd{\varphi}}{\sqrt{1 - \cos \varphi}}
    = \sqrt{2} \ln \tan \frac{\varphi}{4},
\]
\[
    \varphi = 4 \arctan \ee^{\pm \sqrt{V_0} (\tau - \tau_0)}.
\]
By checking continuity and the $\tau \to \pm \infty$ limits,
we find 
\begin{equation}
    \varphi_+ = 4 \arctan \ee^{\sqrt{V_0} (\tau - \tau_0)},
    \quad \varphi_- = 4 \arctan \ee^{- \sqrt{V_0} (\tau - \tau_0)}
\end{equation}
are exactly the two single instanton solutions we need -- 
there is no need ``cut and connect'' branches of solutions.

We have 
\[
    \begin{aligned}
        S[\varphi] &= \int \dd{\tau} \left( \frac{1}{2} \dot{\varphi}^2 
        - \ii \frac{\theta}{2\pi} \dot{\varphi}
        + V(\varphi) \right) \\
        &= - \ii \frac{\theta}{2\pi} (\varphi(\infty) - \varphi(-\infty)) + 
        2 \int \dd{\tau} V(\varphi).
    \end{aligned}
\]
For $\varphi_+$, the first term is $- \ii \theta$, 
while for $\varphi_-$, the first term is $\ii \theta$.
For $\varphi_+$, the second term is 
\[
    \begin{aligned}
        &\quad 2 V_0 \int^\infty_{-\infty} \dd{\tau} \left( 
            1 - \cos(4 \arctan \ee^{\sqrt{V_0} (\tau - \tau_0)})
        \right) \\
        &= 2 \sqrt{V_0} \int_{-\infty}^\infty \dd{x}  \left( 
            1 - \cos(4 \arctan \ee^{x})
        \right) \\
        &= 8 \sqrt{V_0}.
    \end{aligned}
\]
The same is true for $\varphi_{-}$ because of the time reversal symmetry.
So we have 
\begin{equation}
    S_{0, +}(\theta) = - \ii \theta + 8 \sqrt{V_0}, \quad 
    S_{0, -}(\theta) =   \ii \theta + 8 \sqrt{V_0}.
    \label{eq:single-instanton}
\end{equation}

\item[2.] The saddle point approximation, without considering the instantons, gives 
\begin{equation}
    U(0, T; 0, 0) = \sqrt{\frac{m \omega}{2 \pi \sinh \omega T}},
\end{equation}
where the oscillation frequency is just 
\begin{equation}
    \omega = \sqrt{V_0}.
\end{equation}

Now we insert instantons into the paths taken into consideration.
We make the dilute instanton gas approximation,
assuming that the total time $T$ and the distances between instantons 
are largely enough compared with the temporal size of each instanton ($\sim 1 / \sqrt{V_0}$),
and in this case, action has additivity,
and the contribution to the action of each instanton 
is approximately the same as the action of the instanton with
the $-\infty < \tau < \infty$ time span,
which we just evaluated in \eqref{eq:single-instanton}.
So for a configuration with $n_+$ $\varphi_+$ instantons and $n_-$ $\varphi_-$ instantons,
the total saddle-point action is 
\[
    K^{n_1 + n_2} \ee^{- n_+ S_{0, +} - n_- S_{0, -}}.
\]
The number of the possible orders of the instantons is $\binom{n_1 + n_2}{n_1}$,
so the path integral is%
\footnote{
    Note that here both $K$ and $U(0, T; 0, 0)$ have something to do with harmonic oscillators.
    $U(0, T; 0, 0)$ is the ``global'' harmonic fluctuation,
    while $K$ is the fluctuation of an instanton,
    which can also be seen as a normalization coefficient 
    linking 
    \[
        \int \dd{\tau_1} \int \dd{\tau_2} \cdots \int \dd{\tau_n}
    \]
    and $\fd{x}$.
} 
\[
    \begin{aligned}
        &\quad \sum_{n_-, n_+} 
        \int_{0}^T \dd{\tau_1} \int_{\tau_1}^T \dd{\tau_2} \cdots \int_{\tau_{n-1}}^T \dd{\tau_n}
        \binom{n_+ + n_-}{n_+} K^{n_+ + n_-} \ee^{- n_+ S_{0, +} - n_- S_{0, -}}
        U(0, T; 0, 0) \\
        &= U(0, T; 0, 0) \sum_{n_-, n_+} \frac{T^{n_+ + n_-}}{(n_+ + n_-) !} 
        \frac{(n_+ + n_-) !}{n_+! n_-!} 
        K^{n_+ + n_-} \ee^{- n_+ S_{0, +} - n_- S_{0, -}} \\
        &= U(0, T; 0, 0) \sum_{n_+} \frac{(T K \ee^{- S_{0, +}})^{n_+}}{n_+!} 
        \sum_{n-} \frac{(T K \ee^{- S_{0, -}})^{n_-}}{n_-!} \\
        &= U(0, T; 0, 0) \ee^{TK \ee^{- S_{0,+}}} \ee^{TK \ee^{- S_{0,-}}}.
    \end{aligned}
\]
So we get 
\begin{equation}
    \expval{\ee^{- H T}}{0} = U(0, T; 0, 0) \ee^{TK \ee^{- S_{0,+}}} \ee^{TK \ee^{- S_{0,-}}}.
\end{equation}

\item[3.] When $T \to \infty$,
we know (in the last homework) 
\begin{equation}
    U(0, T; 0, 0) \sim \ee^{- \frac{1}{2} \omega T},
\end{equation}
Since in the long run 
\begin{equation}
    \expval{\ee^{- HT}}{0} \sim \ee^{- \frac{1}{2} \omega T} \ee^{TK (\ee^{- S_{0, +}} + \ee^{- S_{0, -}})} 
    \eqqcolon \ee^{- E T},
\end{equation}
we have 
\begin{equation}
    \begin{aligned}
        E &= \frac{1}{2} \omega - K ( \ee^{- S_{0, +}} + \ee^{- S_{0, -}} ) \\
        &= \frac{1}{2} \omega - K \ee^{- 8 \sqrt{V_0}} ( \ee^{\ii \theta} + \ee^{- \ii \theta} ) \\
        &= \frac{1}{2} \sqrt{V_0} - 2 K \ee^{- 8 \sqrt{V_0}} \cos\theta.
    \end{aligned}
\end{equation}
So the ground state energy oscillates with respect to $\theta$.

\end{itemize}



\paragraph{Problem 2} Consider a harmonic oscillator $\hat{H}=\frac{\hat{p}^2}{2 m}+\frac{1}{2} m \omega_0^2 \hat{x}^2$, and let $j=e \dot{x}$ be the current operator.
1. Use the path integral formalism to calculate the (time-ordered) current correlation function $i G_{j j}(t)=$ $\langle 0|\mathcal{T}(j(t) j(0))| 0\rangle$, where $|0\rangle$ is the ground state. Find its form in frequency space $G_{j j}(\omega)=\int d t G_{j j}(t) e^{i \omega t}$.
2. Let $\sigma(\omega)$ be the finite frequency conductance, defined by $j(\omega)=\sigma(\omega) E(\omega)$, where $E(t)$ is a perturbing electric field: $H \rightarrow H-e E(t) x$. Compute $\sigma(\omega)$ for the classical harmonic oscillator, adding in a small amount of friction by hand.
3. Show that for $\omega>0, \sigma(\omega)$ is of the form $\sigma(\omega)=C \frac{G_{j j}(\omega)}{\omega}$, and find the constant $C$.
4. Show that $\sigma(\omega)$ has a nonzero real part only when $\omega=\omega_0$. At that frequency the oscillator can absorb energies by jumping to higher excited states. This example demonstrates that a finite real DC conductance requires gapless excitations.

\paragraph{Solution} \begin{itemize}
\item[1.] We have 
\[
    \timeorder \expval{j(t) j(0)}{0} = \expval{j(t) j(0)} ,
\]
where 
\[
    \expval{\cdots} \coloneqq 
    \frac{\int \fd{x} (\cdots) \ee^{\ii \int \dd{t} L}}{\int \fd{x} \ee^{\ii \int \dd{t} L}}.
\]
So we can do Fourier expansion to $j(t)$ without fears of details of normal ordering.
We have 
\[
    \begin{aligned}
        j(t) &= \dv{t} \int \frac{\dd{\omega}}{2\pi} \ee^{- \ii \omega t} e x(\omega) \\
        &= \int \frac{\dd{\omega}}{2\pi} \ee^{- \ii \omega t}
        (- \ii \omega) e x(\omega),
    \end{aligned}
\]
so 
\[
    \begin{aligned}
        \int \dd{t} \ee^{\ii \omega t} \expval{j(t) j(0)} &= 
        \frac{1}{2\pi \delta(0)} \int \dd{t} \ee^{\ii \omega t} \int \dd{t_2}
        \expval{j(t + t_2) j(t_2)} \\
        &= \frac{1}{2\pi \delta(0)} \int \dd{t} \ee^{\ii \omega t} \int \dd{t_2}
        \int \frac{\dd{\omega_1}}{2\pi} (- e \ii \omega_1) \ee^{- \ii \omega_1 (t + t_2)} \\
        &\quad \quad \cdot \int \frac{\dd{\omega_2}}{2\pi} (- \ii e \omega_2) \ee^{- \ii \omega_2 t_2}
        \expval{j(\omega_1) j(\omega_2)} \\
        &= \frac{1}{2\pi \delta(0)} \int \frac{\dd{\omega_1}}{2\pi} \int \frac{\dd{\omega_2}}{2\pi}
        (- \ii e \omega_1) (- \ii e \omega_2) 
        2\pi \delta (\omega - \omega_1) 2\pi \delta (\omega_1 + \omega_2) 
        \expval{j(\omega_1) j(\omega_2)} \\
        &= e^2 \frac{1}{2\pi} \omega^2 \expval{j(\omega) j(- \omega)}.
    \end{aligned}
\]
On the other hand, we have 
\[
    \int \frac{\dd{\omega}}{2\pi} \expval{x(t) x(0)} 
    = \frac{1}{2\pi} \expval{x(\omega) x(-\omega)},
\]
and thus 
\begin{equation}
    \ii G_{jj}(\omega) = \int \dd{t} \ee^{\ii \omega t} \expval{j(t) j(0)} = 
    e^2 \omega^2 \int \dd{t} \ee^{\ii \omega t} \expval{x(t) x(0)}
    = e^2 \omega^2 \frac{\ii}{m (\omega^2 - \omega_0^2 + \ii \epsilon)},
    \label{eq:g-jj-derivation-final}
\end{equation}
\begin{equation}
    G_{jj}(\omega) = \frac{e^2 \omega^2}{m (\omega^2 - \omega_0^2 + \ii \epsilon)}.
\end{equation}

\item[2.] The EOMs are 
\[
    \dot{x} = \pdv{H}{p} = \frac{p}{m}, \quad \dot{p} = - \pdv{H}{x} = - m \omega_0^2 + eE,
\]
\[
    m\ddot{x} + m \omega_0^2 x = e E.
\]
After adding a small friction we get 
\begin{equation}
    m\ddot{x} + m \epsilon \dot{x} + m \omega_0^2 x = e E.
\end{equation}
Again by Fourier transformation
\[
    x(t) = \int \frac{\dd{\omega}}{2\pi} \ee^{- \ii \omega t} x(\omega), \quad 
    E(t) = \int \frac{\dd{\omega}}{2\pi} \ee^{- \ii \omega t} E(\omega),
\] 
we have 
\[
    (- m \omega^2 + m \omega_0^2 - \ii m \epsilon \omega) x(\omega) = e E(\omega),
\]
\begin{equation}
    \sigma(\omega) = \frac{j(\omega)}{E(\omega)} = - \ii \omega e \frac{x(\omega)}{E(\omega)}
    = \ii \frac{e^2 \omega}{m (\omega^2 - \omega_0^2 + \ii \sgn(\omega) \epsilon)}.
\end{equation}

\item[3.] So when $\omega > 0$, $\sgn(\omega) \epsilon$ is just $0^+$, and we get 
\begin{equation}
    \sigma(\omega) = C \frac{G_{jj} (\omega)}{\omega}, \quad C = \ii.
\end{equation}
This is expected: the correlation function corresponding to $\sigma(\omega)$ is $G_{j, ex}$,
not $G_{jj}$.
The two all contain a $e^2$ factor but they differ with a time derivative,
which is the origin of the $- \ii \omega$ in the denominator.

\item[4.] We have 
\begin{equation}
    \begin{aligned}
        \sigma(\omega) &= \ii \frac{e^2}{m} \omega \left(
            \mathrm{P} \frac{1}{\omega^2 - \omega_0^2} - \pi \ii \sgn(\omega) \delta(\omega^2 - \omega_0^2) 
        \right) \\
        &= \frac{\pi e^2}{m} \omega \delta (\omega^2 - \omega_0^2) 
        + \ii \frac{e^2 \omega}{m} \mathrm{P} \frac{1}{\omega^2 - \omega_0^2}.
    \end{aligned}
\end{equation}
So the real part is non-zero only when $\omega = \omega_0$.

\end{itemize}

\paragraph{Problem 3} Consider a system of two coupled harmonic oscillators:
$$
L[x, X]=\frac{1}{2} m \dot{x}^2-\frac{1}{2} m \omega_0^2 x^2+\frac{1}{2} M \dot{X}^2-\frac{1}{2} M \Omega_0^2 X^2-g x X .
$$
Assume that $\Omega_0 \gg \omega_0$. In this regime, $X$ is a high energy degree of freedom while $x$ is a low energy "soft" degree of freedom. The two are coupled by the $g x X$ term.
1. Find the low energy effective theory $L_{\text {eff }}$ that describes the low energy dynamics of the soft degree of freedom $x$ by writing down the path integral for the system and integrating out $X$ (You should include at least the leading $1 / \Omega_0$ corrections). In particular, find the effective mass $m^*$ and effective spring constant $m^*\left(\omega_0^*\right)^2$. What happens when $g$ becomes large?
2. Express $X$ and $\dot{X}$ in terms of the variables in the effective theory. (Hint: introduce a coupling EX in the theory).
3. At what energy scale is the effective theory a good description of the system?

\paragraph{Solution} \begin{itemize}
\item[1.] The path integral is 
\begin{equation}
    \begin{aligned}
        Z &= \int \fd{X} \fd{x} \ee^{\ii \int \dd{t} \left(
            \frac{1}{2} m \dot{x}^2-\frac{1}{2} m \omega_0^2 x^2
            +\frac{1}{2} M \dot{X}^2-\frac{1}{2} M \Omega_0^2 X^2-g x X
        \right)} \\
        &= \int \fd{x} \ee^{\ii \int \dd{t} \left( \frac{1}{2} m \dot{x}^2-\frac{1}{2} m \omega_0^2 x^2 \right)}
        \int \fd{X} \ee^{\ii \int \dd{t} \left(
            \frac{1}{2} M \dot{X}^2-\frac{1}{2} M \Omega_0^2 X^2-g x X
        \right)} .
    \end{aligned}
    \label{eq:integrate-x-big-x}
\end{equation}
We need to integrate out the $X$ variable to obtain an effective theory for $x$.
We have 
\[
    \begin{aligned}
        &\quad \int \fd{X} \ee^{\ii \int \dd{t} \left(
            \frac{1}{2} M \dot{X}^2-\frac{1}{2} M \Omega_0^2 X^2-g x X
        \right)} \\
        &= \int \fd{X} \ee^{\ii \int \dd{t} \left(
            - \frac{1}{2} M X (\partial_t^2 + \Omega_0^2) X -g x X
        \right)}  \\
        &= \const \cdot \exp(\ii \int \dd{t} \frac{1}{2} g^2 x \frac{1}{M (\partial_t^2 + \Omega_0^2)} x) \\
        &= \const \cdot \exp(\ii \int \dd{t} \frac{1}{2} g^2 x \frac{1}{M \Omega_0^2} 
        \left( 1 - \frac{1}{\Omega_0^2} \dv[2]{t} + \cdots \right) x) \\
        &= \const \cdot \exp(\ii \int \dd{t} \left(
            \frac{g^2}{2 M \Omega_0^2} x^2 
            + \frac{1}{2} \frac{g^2}{M \Omega_0^4} \dot{x}^2 + \cdots
        \right)) .
    \end{aligned}
\]
Only keeping the first-order correction, we have 
\begin{equation}
    m^* = m + \frac{g^2}{M \Omega_0^4},
\end{equation}
\begin{equation}
    m^* (\omega_0^*)^2 = m \omega_0^2 - \frac{g^2}{M \Omega_0^2}.
    \label{eq:corrected-m}
\end{equation}
When $g$ is large, \eqref{eq:corrected-m} becomes negative,
and the theory about $x$ becomes instable.
This is faithful to the original theory,
because when $g$ is large,
the potential 
\[
    \pmqty{ - \frac{1}{2} m \omega_0 & - \frac{g}{2} \\ - \frac{g}{2} & - \frac{1}{2} M \Omega_0^2 }
\]
is also not stable.

\item[2.] We just need to replace $gx$ by $gx - E$ in \eqref{eq:integrate-x-big-x}.
Now after integrating out $X$, we get 
\begin{equation}
    \begin{aligned}
        &\quad \const \cdot \exp(\ii \int \dd{t} \frac{1}{2} 
        (g x - E) \frac{1}{M (\partial_t^2 + \Omega_0^2)} (gx - E)) \\
        &= \const \cdot \exp(\ii \int \dd{t} \frac{1}{2} (gx - E) \frac{1}{M \Omega_0^2} 
        \left( 1 - \frac{1}{\Omega_0^2} \dv[2]{t} + \cdots \right) (gx - E)) \\
        &= \const \cdot \exp(\ii \int \dd{t} \left(
            \frac{1}{2 M \Omega_0^2} (gx - E)^2 
            + \frac{1}{2} \frac{1}{M \Omega_0^4} \left(\dv{(gx - E)}{t}\right)^2 + \cdots
        \right)) \\
        &= \const \cdot \exp(\ii \int \dd{t} \left(
            \frac{1}{2 M \Omega_0^2} (gx - E)^2 
            + \frac{1}{2} \frac{g^2}{M \Omega_0^4} \dot{x}^2 + \cdots
        \right)).
    \end{aligned}
    \label{eq:integrating-out-big-x}
\end{equation}
So now the effective theory is 
\begin{equation}
    L_{\text{eff}} = \frac{1}{2} \left( m + \frac{g^2}{M \Omega_0^4} \right) \dot{x}^2 
    - \frac{1}{2} m \omega_0^2 x^2 + \frac{1}{2 M \Omega_0^2} (gx - E)^2 .
\end{equation}
To find the expression of $X$, we just need to take the derivative of $L_{\text{eff}}$ with respect to $E$,
because to find an $n$-order correlation function of $X$,
we just find the $n$-th derivative of $Z$, 
and if this is done with $L_{\text{eff}}$, 
then what is averaged over is just $\pdv*{L_{\text{eff}}}{E}$ to the $n$.
So 
\begin{equation}
    X = \eval{\pdv{L_{\text{eff}}}{E}}_{E=0} = - \frac{g}{M \Omega_0^2} x,
\end{equation}
and 
\begin{equation}
    \dot{X} = - \frac{g}{M \Omega_0^2} \dot{x}.
\end{equation}

\item[3.] Note that integrating out $X$ in this problem is \emph{exact}:
The place requiring approximation is not integrating out $X$,
but only taking the first-order correction in \eqref{eq:integrating-out-big-x}.
So in order for the effective theory to make sense, 
we just need $\Omega_0 \gg \partial_t$, or in other words 
\begin{equation}
    \Omega_0 \gg \omega_0.
\end{equation}

\end{itemize}

\paragraph{Problem 4} In this problem, we will compute various correlation functions for non-interacting and interacting superfluids. For these calculations, you may find Wick's theorem useful. Let $A_i$ be operators which are linear combinations of $\hat{a}, \hat{a}^{\dagger}$, and let $\langle\cdot\rangle$ denote the expectation value in the zero-boson state. Also let : $\hat{O}$ : denote the normal ordered form of $\hat{O}$ (i.e. with all the $\hat{a}^{\dagger}$ 's to the left of $\hat{a}$ 's). Then Wick's theorem for 4 operators says:
$$
\begin{aligned}
T\left(A_1 A_2 A_3 A_4\right) &=: A_1 A_2 A_3 A_4:+: A_1 A_2:\left\langle T\left(A_3 A_4\right)\right\rangle+: A_1 A_3:\left\langle T\left(A_2 A_4\right)\right\rangle+: A_1 A_4:\left\langle T\left(A_1 A_4\right)\right\rangle \\
&+: A_2 A_3:\left\langle T\left(A_2 A_3\right)\right\rangle+: A_2 A_4:\left\langle T\left(A_2 A_3\right)\right\rangle+: A_3 A_4:\left\langle T\left(A_3 A_4\right)\right\rangle+\\
&+\left\langle T\left(A_1 A_2\right)\right\rangle\left\langle T\left(A_3 A_4\right)\right\rangle+\left\langle T\left(A_1 A_3\right)\right\rangle\left\langle T\left(A_2 A_4\right)\right\rangle+\left\langle T\left(A_1 A_4\right)\right\rangle\left\langle T\left(A_2 A_3\right)\right\rangle .
\end{aligned}
$$
Here $T$ is time ordering.
While the expression is long, there is a clear pattern: the first term is the product $T\left(A_1 A_2 A_3 A_4\right)$ normal ordered, then all possible ways to normal order two of them, while the other two are replaced by their expectation value, etc. The generalization to any even number of $\hat{A}_i$ 's should be straightforward.
1. First consider a system of $N$ non-interacting bosons $\hat{H}=\sum_{\mathbf{k}} \frac{\mathbf{k}^2}{2 m} \hat{a}_{\mathbf{k}}^{\dagger} \hat{a}_{\mathbf{k}}$ in (3D) volume $\mathcal{V}$. Let $\left|\Psi_0\right\rangle$ denote the ground state. Calculate the time-ordered correlation function
$$
i G(x, t)=\left\langle\Psi_0\left|T\left(\hat{a}(\mathbf{x}, t) \hat{a}^{\dagger}(\mathbf{0}, 0)\right)\right| \Psi_0\right\rangle .
$$
For $N=0$ and for finite $N$. Take the thermodynamic limit $N \rightarrow \infty, \mathcal{V} \rightarrow \infty$ but with $N / \mathcal{V}=\rho_0$ a constant. Show that $i G(x, t) \rightarrow$ const as $x \rightarrow \infty$ at a fixed $t$. This shows that the Bose-Einstein condensation has long range order.
2. Calculate the time ordered density-density correlation function $\left\langle\Psi_0\right| T\left(\hat{\rho}(\mathbf{x}, t) \hat{\rho}(\mathbf{0}, 0)\left|\Psi_0\right\rangle\right.$.
3. Now consider the 3D interacting superfluid, with the low-energy effective theory discussed in class:
$$
L=\int \mathrm{d}^3 \mathrm{x}\left[\frac{1}{2 U_0}\left(\partial_t \theta\right)^2-\frac{\rho_0}{2 m}(\nabla \theta)^2\right] .
$$
Denote by $|0\rangle$ the ground state of the interacting superfluid. Calculate $\langle 0|T(\hat{\rho}(\mathbf{x}, t) \hat{\rho}(\mathbf{0}, 0))| 0\rangle$, in $\omega, \mathrm{k}$ space and in real space.
4. Compare the asymptotic behavior of the density-density correlation functions for the (non-interacting) Bose condensate and the (interacting) superfluid state in the limit $x \rightarrow \infty, t$ fixed. Both approach the constant $\langle\hat{\rho}\rangle^2$, but is the approach described by a power law, exponential or some other function?

\paragraph{Solution} \begin{enumerate}
\item We have 
\begin{equation}
    a(\vb*{x}, t) = \frac{1}{\sqrt{V}} \sum_{\vb*{k}} 
    \ee^{\ii \vb*{k} \cdot \vb*{x} - \ii \omega_{\vb*{k}} t} a_{\vb*{k}}.
\end{equation}
So 
\[
    \begin{aligned}
        \ii G(\vb*{x}, t) &= \frac{1}{V} \sum_{\vb*{k}} \sum_{\vb*{k}'}
        \ee^{\ii \vb*{k} \cdot \vb*{x}} 
        \expval{\timeorder \ee^{- \ii \omega_{\vb*{k}} t} a_{\vb*{k}} a^\dagger_{\vb*{k}'} }{0}.
    \end{aligned}
\]
When $t > 0$, we have 
\[
    \timeorder \ee^{- \ii \omega_{\vb*{k}} t} a_{\vb*{k}} a^\dagger_{\vb*{k}'}
    = \ee^{- \ii \omega_{\vb*{k}} t} a_{\vb*{k}} a^\dagger_{\vb*{k}'}
    = \ee^{- \ii \omega_{\vb*{k}} t} (a^\dagger_{\vb*{k}'} a_{\vb*{k}} + \delta_{\vb*{k} \vb*{k}'}),
\]
and when $t < 0$, we have 
\[
    \timeorder \ee^{- \ii \omega_{\vb*{k}} t} a_{\vb*{k}} a^\dagger_{\vb*{k}'}
    = \ee^{- \ii \omega_{\vb*{k}} t} a^\dagger_{\vb*{k}'} a_{\vb*{k}}.
\]
The momentum correlation function is then evaluated as follows:
\[
    \begin{aligned}
        \expval{\timeorder \ee^{- \ii \omega_{\vb*{k}} t} a_{\vb*{k}} a^\dagger_{\vb*{k}'} }{\Psi_0}
        &= \theta(t) \delta_{\vb*{k} \vb*{k}'} \ee^{- \ii \omega_{\vb*{k}} t} 
        + \ee^{- \ii \omega_{\vb*{k}} t} \expval{a_{\vb*{k}'}^\dagger a_{\vb*{k}}}{\Psi_0} \\
        &= \theta(t) \delta_{\vb*{k} \vb*{k}'} \ee^{- \ii \omega_{\vb*{k}} t} 
        + \ee^{- \ii \omega_{\vb*{k}} t} \delta_{\vb*{k}, 0} \delta_{\vb*{k}', 0} 
        \expval{a_0^\dagger a_0}{\Psi_0} \\
        &=  \theta(t) \delta_{\vb*{k} \vb*{k}'} \ee^{- \ii \omega_{\vb*{k}} t} 
        + \ee^{- \ii \omega_{\vb*{k}} t} \delta_{\vb*{k}, 0} \delta_{\vb*{k}', 0} N.
    \end{aligned}
\]
The correlation function is therefore 
\begin{equation}
    \begin{aligned}
        \ii G(\vb*{x}, t) &= \frac{1}{V} \sum_{\vb*{k}, \vb*{k}'} 
        \ee^{\ii \vb*{k} \cdot \vb*{x}} \left(
            \theta(t) \delta_{\vb*{k} \vb*{k}'} \ee^{- \ii \omega_{\vb*{k}} t} 
            + \ee^{- \ii \omega_{\vb*{k}} t} \delta_{\vb*{k}, 0} \delta_{\vb*{k}', 0} N
        \right) \\
        &= \frac{N}{V} + 
        \theta(t) \frac{1}{V} \sum_{\vb*{k}} \ee^{\ii \vb*{k} \cdot \vb*{x}} \ee^{- \ii \frac{\vb*{k}^2}{2m} t} \\
        &= \rho_0 +
        \theta(t) \int \frac{\dd[3]{\vb*{k}}}{(2\pi)^3} \ee^{ \ii \vb*{k} \cdot \vb*{x} - \ii \frac{\vb*{k}^2}{2m} t } .
    \end{aligned}
    \label{eq:sf-correlation-phonon}
\end{equation}
After completing the integral, we get 
\begin{equation}
    \ii G(\vb*{x}, t) = \rho_0 + \theta(t) \sqrt{\frac{m^3}{(2 \pi \ii t)^3}} \ee^{\frac{\ii}{2} \frac{m x^2}{t}}.
    \label{eq:spatial-oscillating}
\end{equation}
When $\abs{\vb*{x}} \to \infty$, 
the second term oscillates fast.
The reason that we have such an oscillation term is 
because the sum over $\vb*{k}$ is actually the sum over $\vb*{k} \neq 0$:
The $\vb*{k}=0$ modes are not phonon modes, 
but Anderson tower of states.
Since the low-energy effective theory of superfluid is free,
we can talk about the theory of each Fourier component,
because there is no interaction energy between the components.
For the spatial uniform component, the effective theory is 
\begin{equation}
    L = \frac{V}{2U_0} (\partial_t \theta)^2 - \rho_0 V \partial_t \theta, \quad \mu = U_0 \rho_0,
\end{equation}
and in canonical quantization we have 
\begin{equation}
    p = \frac{V}{U_0} (\dot{\theta} - \mu), \quad 
    H = p \dot{\theta} - L = \frac{V}{2 U_0} \left(\frac{U_0}{V} p + \mu \right)^2,
\end{equation}
and since the following relation 
\begin{equation}
    \rho = \rho_0 - \frac{1}{U_0} \dot{\theta} 
\end{equation}
connects the effective theory and the original theory,
we have
\begin{equation}
    p = - \rho V = - N,
\end{equation}
and therefore 
\begin{equation}
    H_{\vb*{k} = 0} = \frac{U_0}{2V} N^2 - \mu N + \frac{V}{2U_0} \mu^2. 
    \label{eq:phonon-anderson}
\end{equation}
Indeed, this part of Hamiltonian doesn't contain phonon modes,
so when summing over phonon modes, $\vb*{k} = 0$ should be skipped.
\eqref{eq:phonon-anderson} also poses a larger problem:
It seems there is no ground state degeneracy here,
and the ground state 
(or ground states, 
if we only consider phonon modes as excitations and view the ground state $N$ 
-- the number of superfluid bosons, not phonons -- as a background) 
has full $U(1)$ symmetry.
The point here is the energy gap of \eqref{eq:phonon-anderson} is 
\begin{equation}
    \Delta E_N \sim \frac{U_0}{V} \Delta N^2,
\end{equation} 
while for phonons, the minimal energy of a phonon 
-- there is always an energy gap for phonons in a finite system -- 
is 
\begin{equation}
    \Delta E_{\text{ph}} \sim \min \abs{\vb*{k}} \sim \frac{1}{L} \sim \frac{1}{L^d}.
\end{equation}
So keeping everything else invariant and just changing the length scale of the system, 
we find as $L \to \infty$,
$\Delta E_\text{ph}$ goes to zero much faster than $\Delta E_{\text{ph}}$
as long as $d > 1$.
So in the eyes of phonons, 
in the thermodynamic limit, 
the $\ket{N, \text{no phonon}}$ states -- we call them Anderson tower of states -- are indeed degenerate,
and the now we can use linear combinations of $\ket{N}$ states -- which are degenerate ground states -- 
to construct ground states like $\ket{\theta}$ which breaks the $U(1)$ symmetry,
and everything goes well.
So indeed we find perfect spontaneous symmetry breaking only happens when $L \to \infty$,
and it doesn't happen when $d = 1$.
(Of course, finally phonons will also be considered as gapless.)

Going back to \eqref{eq:spatial-oscillating}, 
to skip the $\vb*{k} = 0$ point in the continuous theory,
we need to introduce an IR length cutoff, 
which can be done by introducing a small imaginary part in $\vb*{x}$
so \eqref{eq:sf-correlation-phonon} should be interpreted as TODO: it's IR cutoff, why we are actually doing a UV cutoff??
\begin{equation}
    \ii G(\vb*{x}, t) = \rho_0 + \theta(t) \int \frac{\dd[3]{\vb*{k}}}{(2\pi)^3} \ee^{\ii \vb*{k} \cdot ()}
\end{equation}
\begin{equation}
    \ii G(\vb*{x}, t) \simeq \rho_0,
\end{equation}
so there is indeed a long-range order in the ground state of BEC.

\item[2.] By Wick's theorem and the fact that $\expval{\timeorder a_1 a_2}{0} = 0$,
as well as $\expval{a^\dagger a}{0}$, 
we have 
\begin{equation}
    \begin{aligned}
        &\quad \timeorder a^\dagger(\vb*{x}, t) a(\vb*{x}, t) a^\dagger(0, 0) a(0, 0) \\
        &= \normalorder{a^\dagger(\vb*{x}, t) a(\vb*{x}, t) a^\dagger(0, 0) a(0, 0)} \\
        &\quad + \normalorder{a^\dagger(\vb*{x}, t) a(0, 0)} 
        \expval{\timeorder a(\vb*{x}, t) a^\dagger(0, 0) }{0} \\
        &\quad + \normalorder{a(\vb*{x}, t) a^\dagger(0, 0)} 
        \expval{\timeorder a^\dagger (\vb*{x}, t) a(0, 0)}{0} \\
        &\quad + \expval{\timeorder a(\vb*{x}, t) a^\dagger(0, 0) }{0}
        \expval{\timeorder a^\dagger (\vb*{x}, t) a(0, 0)}{0}.
    \end{aligned}
    \label{eq:expected-operator-density}
\end{equation}
By space and time translational symmetry, we have 
\[
    \begin{aligned}
        \expval{\timeorder a^\dagger (\vb*{x}, t) a(0, 0)}{0}
        &= \expval{\timeorder a(0, 0) a^\dagger (\vb*{x}, t)}{0} 
        = \expval{\timeorder a(- \vb*{x}, -t) a^\dagger (0, 0)}{0}  \\
        &= \underbrace{\rho_0}_{\text{$=0$ for $\ket{0}$}} +
        \theta(-t) \int \frac{\dd[3]{\vb*{k}}}{(2\pi)^3} \ee^{- \ii \vb*{k} \cdot \vb*{x} + \ii \frac{\vb*{k}^2}{2m} t }.
    \end{aligned}
\]
Thus the last term in \eqref{eq:expected-operator-density} vanishes,
because it contains both $\theta(t)$ and $\theta(-t)$.
So we get 
\begin{equation}
    \begin{aligned}
        &\quad \expval{\timeorder a^\dagger(\vb*{x}, t) a(\vb*{x}, t) a^\dagger(0, 0) a(0, 0)}{\Psi_0} \\
        &= \expval{\normalorder{a^\dagger(\vb*{x}, t) a(\vb*{x}, t) a^\dagger(0, 0) a(0, 0)}}{\Psi_0} \\
        &\quad + \expval{\normalorder{a^\dagger(\vb*{x}, t) a(0, 0)}}{\Psi_0} 
        \expval{\timeorder a(\vb*{x}, t) a^\dagger(0, 0) }{0} \\
        &\quad + \expval{\normalorder{a^\dagger(0, 0) a(\vb*{x}, t)} }{\Psi_0}
        \expval{\timeorder a(-\vb*{x}, -t) a^\dagger(0, 0)}{0} .
    \end{aligned}
\end{equation}
The normal ordered operator factor in the second term is 
\[
    \begin{aligned}
        \expval{\normalorder{a^\dagger(\vb*{x}, t) a(0, 0)}}{\Psi_0} &= 
        \frac{1}{V} \sum_{\vb*{k}, \vb*{k}'} \ee^{- \ii \vb*{k} \cdot \vb*{x}} \ee^{\ii \omega_{\vb*{k}} t}
        \expval{a^\dagger_{\vb*{k}} a_{\vb*{k}'}}{\Psi_0}  \\
        &= \frac{1}{V} \sum_{\vb*{k}, \vb*{k}'} \ee^{- \ii \vb*{k} \cdot \vb*{x}} \ee^{\ii \omega_{\vb*{k}} t}
        N \delta_{\vb*{k}, 0} \delta_{\vb*{k}', 0} = \frac{N}{V},
    \end{aligned}
\]
and similarly the normal ordered operator factor in the third term is $N / V$.
The first term is 
\[
    \begin{aligned}
        &\quad \expval{\normalorder{a^\dagger(\vb*{x}, t) a(\vb*{x}, t) a^\dagger(0, 0) a(0, 0)}}{\Psi_0}  \\
        &= \frac{1}{V^2} \sum_{\vb*{k}_1, \vb*{k}_2, \vb*{k}_3, \vb*{k}_4} 
        \ee^{- \ii \vb*{k}_1 \cdot \vb*{x} + \ii \omega_{\vb*{k}_1} t} 
        \ee^{  \ii \vb*{k}_2 \cdot \vb*{x} - \ii \omega_{\vb*{k}_2} t} 
        \expval{a^\dagger_{\vb*{k}_1} a^\dagger_{\vb*{k}_3} a_{\vb*{k}_2} a_{\vb*{k}_4} }{\Psi_0} \\
        &= \frac{1}{V^2} \sum_{\vb*{k}_1, \vb*{k}_2, \vb*{k}_3, \vb*{k}_4} 
        \ee^{- \ii \vb*{k}_1 \cdot \vb*{x} + \ii \omega_{\vb*{k}_1} t} 
        \ee^{  \ii \vb*{k}_2 \cdot \vb*{x} - \ii \omega_{\vb*{k}_2} t} 
        \delta_{\vb*{k}_1, 0} \delta_{\vb*{k}_2, 0} \delta_{\vb*{k}_3, 0} \delta_{\vb*{k}_4, 0} N(N-1) \\
        &= \frac{N(N-1)}{V^2}.
    \end{aligned}
\]
So the final result is 
\begin{equation}
    \begin{aligned}
        &\quad \expval{\timeorder \rho(\vb*{x}, t) a(\vb*{x}, t) \rho(0, 0)}{\Psi_0} \\
        &= \frac{N(N-1)}{V^2} + 
        \frac{N}{V} \theta(t) \int \frac{\dd[3]{\vb*{k}}}{(2\pi)^3} 
        \ee^{ \ii \vb*{k} \cdot \vb*{x} - \ii \frac{\vb*{k}^2}{2m} t } + 
        \frac{N}{V} \theta(- t) \int \frac{\dd[3]{\vb*{k}}}{(2\pi)^3} 
        \ee^{- \ii \vb*{k} \cdot \vb*{x} + \ii \frac{\vb*{k}^2}{2m} t } .
    \end{aligned}
    \label{eq:original-density-density}
\end{equation}

\item[3.] By the same procedure in \eqref{eq:integrating-out-big-x},
it can be found that (Eq. (3.3.12) in Xiao-gang Wen's textbook) 
\begin{equation}
    \rho = \rho_0 - \frac{1}{U_0} \partial_t \theta.
\end{equation}
Since we are working on $\ket{0}$,
all terms with odd $\theta$'s have vanishing expectation,
so we have 
\begin{equation}
    \expval{\timeorder \rho(\vb*{x}, t) \rho(0, 0)}{0} 
    = \rho_0^2 + \frac{1}{U_0^2} \expval{\timeorder \partial_t \theta(\vb*{x}, t) \partial_t \theta (0, 0)}{0}.
\end{equation}
In the frequency domain we have (following the same procedure in \eqref{eq:g-jj-derivation-final})
\begin{equation}
    \begin{aligned}
        &\quad \int \dd[3]{\vb*{x}} \dd{t} \ee^{\ii (- \vb*{k} \cdot \vb*{x} + \omega t)} 
        \expval{\timeorder \rho(\vb*{x}, t) \rho(0, 0)}{0}  \\
        &= (2\pi)^4 \rho_0^2 \delta^3(\vb*{k}) \delta(\omega)
        + \frac{1}{U_0^2} (-\ii \omega) (\ii \omega) 
        \int \dd[3]{\vb*{x}} \dd{t} \ee^{\ii (- \vb*{k} \cdot \vb*{x} + \omega t)} 
        \expval{\timeorder \partial_t \theta(\vb*{x}, t) \partial_t \theta (0, 0)}{0} \\
        &= (2\pi)^4 \rho_0^2 \delta^3(\vb*{k}) \delta(\omega)
        + \frac{1}{U_0^2} \omega^2 \frac{\ii}{ \frac{1}{U_0} \omega^2 - \frac{\rho_0}{m} \vb*{k}^2 + \ii 0^+ }.
    \end{aligned}
\end{equation}
Here the propagator in the last line can be found in Section~3.3.8.1 in Wen's book
or by directly taking the inverse of the Lagrangian.
The inverse Fourier transformation gives 
\begin{equation}
    \expval{\timeorder \rho(\vb*{x}, t) \rho(0, 0)}{0} = 
    \rho_0^2 + \frac{\ii}{U_0} \int \frac{\dd[3]{\vb*{k}} \dd{\omega}}{(2\pi)^4} 
    \ee^{\ii (- \omega t + \vb*{k} \cdot \vb*{x})}
    \frac{\omega^2}{\omega^2 - \underbrace{\frac{\rho_0 U_0}{m} \vb*{k}^2}_{\eqqcolon \omega_{\vb*{k}}^2} + \ii 0^+}.
\end{equation}

The next step is to calculate the second term explicitly.
When $t > 0$, we should construct a contour surrounding the lower plane 
where $\ee^{- \ii \omega t} \to 0$ at the infinity, so we have 
\[
    \int \dd{\omega} \ee^{- \ii \omega t} \frac{\omega^2}{\omega^2 - \omega_{\vb*{k}}^2 + \ii 0^+} 
    = - 2\pi \ii \lim_{\omega \to \omega_{\vb*{k}}} 
    \frac{\omega^2}{\omega^2 - \omega_{\vb*{k}}^2} \ee^{- \ii \omega t} (\omega - \omega_{\vb*{k}})
    = - \pi \ii \omega_{\vb*{k}} \ee^{- \ii \omega_{\vb*{k}} t},
\]
and (below $x$ is $\abs{\vb*{x}}$)
\[
    \begin{aligned}
        \begin{aligned}
            &\quad \int \frac{\dd[3]{\vb*{k}} \dd{\omega}}{(2\pi)^4} \frac{\omega^2}{\omega^2 - \omega_{\vb*{k}}^2 + \ii 0^+} 
            \ee^{\ii (\vb*{k} \cdot \vb*{x} - \omega t)} \\
            &= \frac{1}{(2\pi)^4} \int \dd[3]{\vb*{k}} 
            (- \pi \ii) \omega_{\vb*{k}} \ee^{\ii (\vb*{k} \cdot \vb*{x} - \omega_{\vb*{k}} t)} \\
            &= - \frac{\pi \ii}{(2\pi)^4} \int k^2 \dd k \int \dd{\varphi} \int \sin \theta \dd{\theta} 
            \sqrt{\frac{\rho_0 U_0}{m}} k 
            \ee^{\ii k x \cos \theta - \ii k t \sqrt{\frac{\rho_0 U_0}{m}}} \\
            &= - \frac{\pi \ii}{(2\pi)^4} \int k^2 \dd k \cdot 2\pi \cdot 
            \sqrt{\frac{\rho_0 U_0}{m}} k \ee^{- \ii k t \sqrt{\frac{\rho_0 U_0}{m}}}
            \frac{1}{\ii k x} (\ee^{\ii k x} - \ee^{- \ii k x}) \\
            &= - \frac{1}{8 \pi^2} \frac{1}{x} v 
            \int_0^\infty k^2 (\ee^{\ii k (x - vt)} - \ee^{- \ii k (x + vt)}) \dd{k} \\
            &= - \frac{1}{8 \pi^2} \frac{1}{x} v \left(
                \frac{- 2\ii }{(x - vt)^3} 
                - \frac{ - 2\ii}{(- x - vt)^3}.
            \right)
        \end{aligned}
    \end{aligned}
\]
Here we define 
\begin{equation}
    v = \sqrt{\frac{\rho_0 U_0}{m}}, \omega_{\vb*{k}} = c \abs{\vb*{k}}.
\end{equation}
Assuming a small damping in the spectrum of $\theta$,
i.e. a small negative imaginary part of $\omega_{\vb*{k}}$ and hence $v$,
we get 
\[
    \int \frac{\dd[3]{\vb*{k}} \dd{\omega}}{(2\pi)^4} \frac{\omega^2}{\omega^2 - \omega_{\vb*{k}}^2 + \ii 0^+} 
    =   \frac{\ii}{4 \pi^2} \frac{v}{x} \left(
        \frac{1}{(x - vt)^3} + \frac{1}{(x + vt)^3}
    \right).
\]
When $t < 0$, the only change is now the pole we integrate around becomes $\omega = - \omega_{\vb*{k}}$,
and 
\[
    \int \dd{\omega} \ee^{- \ii \omega t} \frac{\omega^2}{\omega^2 - \omega_{\vb*{k}}^2 + \ii 0^+} 
    = 2\pi \ii \lim_{\omega \to - \omega_{\vb*{k}}} 
    \frac{\omega^2}{\omega^2 - \omega_{\vb*{k}}^2} \ee^{- \ii \omega t} (\omega + \omega_{\vb*{k}})
    = - \pi \ii \omega_{\vb*{k}} \ee^{ \ii \omega_{\vb*{k}} t},
\]
so the only change is replacing $t$ with $\abs{t}$.
So finally the correlation function is 
\begin{equation}
    \begin{aligned}
        \expval{\timeorder \rho(\vb*{x}, t) \rho(0, 0)}{0} &=
        \rho_0^2 - \frac{1}{4 \pi^2 U_0}  
        \frac{v}{x} \left( \frac{1}{(x - v\abs{t})^3} + \frac{1}{(x + v\abs{t})^3} \right). 
    \end{aligned}
    \label{eq:superfluid}
\end{equation}

\item[4.] The second and third terms of \eqref{eq:original-density-density} 
are all rapidly oscillating in the same way we see in \eqref{eq:spatial-oscillating}.
so in BEC, we have 
\begin{equation}
    \expval{\timeorder \rho(\vb*{x}, t) \rho(0, 0)}{\Psi_0} 
    \stackrel{\abs{\vb*{x}} \to \infty}{\longrightarrow} \rho_0^2.
\end{equation}
This is also true for superfluid \eqref{eq:superfluid},
but the decay is a $x^{-4}$ power law.

\end{enumerate}

\end{document}