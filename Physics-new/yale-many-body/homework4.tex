\documentclass[hyperref, a4paper]{article}

\usepackage{geometry}
\usepackage{titling}
\usepackage{titlesec}
% No longer needed, since we will use enumitem package
% \usepackage{paralist}
\usepackage{enumitem}
\usepackage{footnote}
%\usepackage{enumerate}
\usepackage{amsmath, amssymb, amsthm}
\usepackage{mathtools}
\usepackage{bbm}
\usepackage{cite}
\usepackage{graphicx}
\usepackage{subfigure}
\usepackage{physics}
\usepackage{tensor}
\usepackage{siunitx}
\usepackage[version=4]{mhchem}
\usepackage{tikz}
\usepackage{xcolor}
\usepackage{listings}
\usepackage{autobreak}
\usepackage[ruled, vlined, linesnumbered]{algorithm2e}
\usepackage{nameref,zref-xr}
\zxrsetup{toltxlabel}
\usepackage[colorlinks,unicode]{hyperref} % , linkcolor=black, anchorcolor=black, citecolor=black, urlcolor=black, filecolor=black
\usepackage[most]{tcolorbox}
\usepackage{prettyref}

% Page style
\geometry{left=3.18cm,right=3.18cm,top=2.54cm,bottom=2.54cm}
\titlespacing{\paragraph}{0pt}{1pt}{10pt}[20pt]
\setlength{\droptitle}{-5em}
%\preauthor{\vspace{-10pt}\begin{center}}
%\postauthor{\par\end{center}}

% More compact lists 
\setlist[itemize]{
    itemindent=17pt, 
    leftmargin=1pt,
    listparindent=\parindent,
    parsep=0pt,
}

% Math operators
\DeclareMathOperator{\polylog}{\mathrm{Li}}
\DeclareMathOperator{\arctanh}{\mathrm{arctanh}}
\DeclareMathOperator{\timeorder}{\mathcal{T}}
\DeclareMathOperator{\diag}{diag}
\DeclareMathOperator{\legpoly}{P}
\DeclareMathOperator{\primevalue}{P}
\DeclareMathOperator{\sgn}{sgn}
\DeclarePairedDelimiter\ceil{\lceil}{\rceil}
\DeclarePairedDelimiter\floor{\lfloor}{\rfloor}
\newcommand*{\ii}{\mathrm{i}}
\newcommand*{\ee}{\mathrm{e}}
\newcommand*{\const}{\mathrm{const}}
\newcommand*{\suchthat}{\quad \text{s.t.} \quad}
\newcommand*{\argmin}{\arg\min}
\newcommand*{\argmax}{\arg\max}
\newcommand*{\normalorder}[1]{: #1 :}
\newcommand*{\pair}[1]{\langle #1 \rangle}
\newcommand*{\fd}[1]{\mathcal{D} #1}
\DeclareMathOperator{\bigO}{\mathcal{O}}

% TikZ setting
\usetikzlibrary{arrows,shapes,positioning}
\usetikzlibrary{arrows.meta}
\usetikzlibrary{decorations.markings}
\tikzstyle arrowstyle=[scale=1]
\tikzstyle directed=[postaction={decorate,decoration={markings,
    mark=at position .5 with {\arrow[arrowstyle]{stealth}}}}]
\tikzstyle ray=[directed, thick]
\tikzstyle dot=[anchor=base,fill,circle,inner sep=1pt]

% Algorithm setting
% Julia-style code
\SetKwIF{If}{ElseIf}{Else}{if}{}{elseif}{else}{end}
\SetKwFor{For}{for}{}{end}
\SetKwFor{While}{while}{}{end}
\SetKwProg{Function}{function}{}{end}
\SetArgSty{textnormal}

\newcommand*{\concept}[1]{{\textbf{#1}}}

% Embedded codes
\lstset{basicstyle=\ttfamily,
  showstringspaces=false,
  commentstyle=\color{gray},
  keywordstyle=\color{blue}
}

% Reference formatting
\newrefformat{fig}{Figure~\ref{#1}}

% Color boxes
\tcbuselibrary{skins, breakable, theorems}
\newtcbtheorem[number within=section]{warning}{Warning}%
  {colback=orange!5,colframe=orange!65,fonttitle=\bfseries, breakable}{warn}
\newtcbtheorem[number within=section]{note}{Note}%
  {colback=green!5,colframe=green!65,fonttitle=\bfseries, breakable}{note}
\newtcbtheorem[number within=section]{info}{Info}%
  {colback=blue!5,colframe=blue!65,fonttitle=\bfseries, breakable}{info}

\newenvironment{shelldisplay}{\begin{lstlisting}}{\end{lstlisting}}

\title{Homework 4}
\author{Jinyuan Wu}

\begin{document}

\maketitle

\paragraph{Problem 1} In this problem we will introduce another way to think about Hall conductance. As discussed in class, the edge of a Chern insulator with Chern number $n$ should have $n$ chiral edge modes. Suppose that the $n$ edge modes move one way near the $y=0$ edge, and the opposite way at the $y=L_y$ edge.
1. First consider the $y=L_y$ edge. Each of the $n$ edge modes near $y=L_y$ can be modeled as a non-interacting free fermion model $H=\sum_k\left(\epsilon_{1 k}-\mu_1\right) c_k^{\dagger} c_k$, where $\epsilon_{1 k}$ is a monotonically increasing function of $k, \mu_1$ corresponds to the chemical potential at the $y=L_y$ edge. We assume that the infinitely many states with $\epsilon_{1 k}<\mu_1$ are all filled (in reality, the limit $k \rightarrow-\infty$ corresponds to moving into the bulk, and $k$ is cut off accordingly). Suppose one changes the chemical potential $\mu_1 \rightarrow \mu_1+\Delta \mu_1$. What is the corresponding change $\Delta I_1$ in the $y=L_y$ edge current? (Include all $n$ edge modes).
2. Now consider the $y=0$ edge mode, modeled by $n$ chiral modes of the form $H=\sum_k\left(\epsilon_{2 k}-\mu_2\right) c_k^{\dagger} c_k$, where $\epsilon_{2 k}$ is a monotonically decreasing function of $k$. What is the corresponding change $\Delta I_2$ in the $y=0$ edge current if one changes the chemical potential $\mu_2 \rightarrow \mu_2+\Delta \mu_2$ ?
3. Find the total current $I=I_1+I_2$ induced by a chemical potential difference $\mu_1-\mu_2=\Delta \mu$ between the two edges, assuming that $I=0$ when $\mu_1=\mu_2$. Interpret your result in terms of the Hall conductance $\sigma_H$.
4. Now repeat the calculation at finite temperature $T$. Show that the edge currents and the Hall conductance do not change. (In reality, there is a temperature dependence coming from thermal activation above the bulk gap $\Delta$, but it is exponentially small $\left.\sim e^{-\Delta / T}\right)$.

\begin{figure}
    \centering
    \begin{tikzpicture}[x=0.75pt,y=0.75pt,yscale=-0.8,xscale=0.8]
    %uncomment if require: \path (0,338); %set diagram left start at 0, and has height of 338
    
    %Shape: Rectangle [id:dp40795066059982865] 
    \draw  [draw opacity=0][fill={rgb, 255:red, 248; green, 231; blue, 28 }  ,fill opacity=0.2 ] (111,110) -- (250.5,110) -- (250.5,175) -- (111,175) -- cycle ;
    %Straight Lines [id:da36690056790944925] 
    \draw [color={rgb, 255:red, 155; green, 155; blue, 155 }  ,draw opacity=1 ]   (49,175) -- (310.5,175) ;
    \draw [shift={(312.5,175)}, rotate = 180] [fill={rgb, 255:red, 155; green, 155; blue, 155 }  ,fill opacity=1 ][line width=0.08]  [draw opacity=0] (12,-3) -- (0,0) -- (12,3) -- cycle    ;
    %Straight Lines [id:da6580792285138533] 
    \draw    (111,110) -- (250.5,110) ;
    %Straight Lines [id:da2884770612696377] 
    \draw    (111,175) -- (250.5,175) ;
    %Straight Lines [id:da7254005793637499] 
    \draw [color={rgb, 255:red, 155; green, 155; blue, 155 }  ,draw opacity=1 ]   (180.75,190.74) -- (180.75,60.52) ;
    \draw [shift={(180.75,58.52)}, rotate = 90] [fill={rgb, 255:red, 155; green, 155; blue, 155 }  ,fill opacity=1 ][line width=0.08]  [draw opacity=0] (12,-3) -- (0,0) -- (12,3) -- cycle    ;
    %Straight Lines [id:da5230554548972957] 
    \draw [color={rgb, 255:red, 74; green, 144; blue, 226 }  ,draw opacity=1 ][fill={rgb, 255:red, 74; green, 144; blue, 226 }  ,fill opacity=1 ]   (157,120.63) -- (202.5,120.63) ;
    \draw [shift={(204.5,120.63)}, rotate = 180] [fill={rgb, 255:red, 74; green, 144; blue, 226 }  ,fill opacity=1 ][line width=0.08]  [draw opacity=0] (12,-3) -- (0,0) -- (12,3) -- cycle    ;
    %Straight Lines [id:da26319630873635913] 
    \draw [color={rgb, 255:red, 74; green, 144; blue, 226 }  ,draw opacity=1 ][fill={rgb, 255:red, 74; green, 144; blue, 226 }  ,fill opacity=1 ]   (157,164) -- (202.5,164) ;
    \draw [shift={(204.5,164)}, rotate = 180] [fill={rgb, 255:red, 74; green, 144; blue, 226 }  ,fill opacity=1 ][line width=0.08]  [draw opacity=0] (12,-3) -- (0,0) -- (12,3) -- cycle    ;
    %Straight Lines [id:da6574829504537725] 
    \draw [color={rgb, 255:red, 155; green, 155; blue, 155 }  ,draw opacity=1 ]   (325,99) -- (441.5,99) ;
    \draw [shift={(443.5,99)}, rotate = 180] [fill={rgb, 255:red, 155; green, 155; blue, 155 }  ,fill opacity=1 ][line width=0.08]  [draw opacity=0] (12,-3) -- (0,0) -- (12,3) -- cycle    ;
    %Straight Lines [id:da9847646340319416] 
    \draw [color={rgb, 255:red, 155; green, 155; blue, 155 }  ,draw opacity=1 ]   (354,120) -- (354,20.78) ;
    \draw [shift={(354,18.78)}, rotate = 90] [fill={rgb, 255:red, 155; green, 155; blue, 155 }  ,fill opacity=1 ][line width=0.08]  [draw opacity=0] (12,-3) -- (0,0) -- (12,3) -- cycle    ;
    %Curve Lines [id:da6788189381158845] 
    \draw    (335.5,131.85) .. controls (375.5,101.85) and (363.5,71.85) .. (403.5,41.85) ;
    %Straight Lines [id:da9121250772726961] 
    \draw    (325,60) -- (443.5,60) ;
    %Straight Lines [id:da3059306191704432] 
    \draw  [dash pattern={on 0.84pt off 2.51pt}]  (325,60) -- (325,99) ;
    %Straight Lines [id:da5759142882874566] 
    \draw  [dash pattern={on 0.84pt off 2.51pt}]  (384.25,60) -- (384.25,99) ;
    
    %Straight Lines [id:da40262000917674734] 
    \draw [color={rgb, 255:red, 155; green, 155; blue, 155 }  ,draw opacity=1 ]   (435.5,269) -- (333.5,269) ;
    \draw [shift={(437.5,269)}, rotate = 180] [fill={rgb, 255:red, 155; green, 155; blue, 155 }  ,fill opacity=1 ][line width=0.08]  [draw opacity=0] (12,-3) -- (0,0) -- (12,3) -- cycle    ;
    %Straight Lines [id:da7191872218882651] 
    \draw [color={rgb, 255:red, 155; green, 155; blue, 155 }  ,draw opacity=1 ]   (393.5,290) -- (393.5,190.78) ;
    \draw [shift={(393.5,188.78)}, rotate = 90] [fill={rgb, 255:red, 155; green, 155; blue, 155 }  ,fill opacity=1 ][line width=0.08]  [draw opacity=0] (12,-3) -- (0,0) -- (12,3) -- cycle    ;
    %Curve Lines [id:da8130155872478908] 
    \draw    (412,301.85) .. controls (372,271.85) and (384,241.85) .. (344,211.85) ;
    %Straight Lines [id:da4797731726124772] 
    \draw    (413.5,230) -- (331.5,230) ;
    %Straight Lines [id:da508293409330584] 
    \draw  [dash pattern={on 0.84pt off 2.51pt}]  (363.25,230) -- (363.25,269) ;
    %Straight Lines [id:da8016612780997885] 
    \draw  [dash pattern={on 0.84pt off 2.51pt}]  (413.5,230) -- (413.5,269) ;
    
    
    % Text Node
    \draw (182.75,106.6) node [anchor=south west] [inner sep=0.75pt]    {$L_{y}$};
    % Text Node
    \draw (314.5,175) node [anchor=west] [inner sep=0.75pt]    {$x$};
    % Text Node
    \draw (180.75,55.12) node [anchor=south] [inner sep=0.75pt]    {$y$};
    % Text Node
    \draw (206.5,120.63) node [anchor=west] [inner sep=0.75pt]    {$I_{1}$};
    % Text Node
    \draw (206.5,164) node [anchor=west] [inner sep=0.75pt]    {$I_{2}$};
    % Text Node
    \draw (439.5,269) node [anchor=west] [inner sep=0.75pt]    {$k$};
    % Text Node
    \draw (395.5,188.78) node [anchor=west] [inner sep=0.75pt]    {$\epsilon _{2k}$};
    % Text Node
    \draw (413.5,272.4) node [anchor=north] [inner sep=0.75pt]    {$\Lambda _{2}$};
    % Text Node
    \draw (329.5,230) node [anchor=east] [inner sep=0.75pt]   [align=left] {$\displaystyle \epsilon _{\text{2F}}$};
    % Text Node
    \draw (445.5,99) node [anchor=west] [inner sep=0.75pt]    {$k$};
    % Text Node
    \draw (356,18.78) node [anchor=west] [inner sep=0.75pt]    {$\epsilon _{1k}$};
    % Text Node
    \draw (325,102.4) node [anchor=north] [inner sep=0.75pt]    {$\Lambda _{1}$};
    % Text Node
    \draw (323,60) node [anchor=east] [inner sep=0.75pt]   [align=left] {$\displaystyle \epsilon _{1\text{F}}$};
    % Text Node
    \draw (141,207) node [anchor=north west][inner sep=0.75pt]   [align=left] {(a)};
    % Text Node
    \draw (385,133) node [anchor=north west][inner sep=0.75pt]   [align=left] {(b)};
    % Text Node
    \draw (379,308) node [anchor=north west][inner sep=0.75pt]   [align=left] {(c)};
    
    
    \end{tikzpicture}
    
    \caption{Boundary modes of a 2D Chern insulator 
    (a) The device and sign conventions
    (b) The spectrum of the boundary modes at $y = L_y$, where $\Lambda_1$ is a momentum cutoff 
    (c) The spectrum of the boundary modes at $y = 0$, where $\Lambda_2$ is a momentum cutoff}
    \label{fig:chern-boundary}
\end{figure}

\paragraph{Solution} The problem is illustrated in \prettyref{fig:chern-boundary}.
\begin{itemize}
\item[(a)] The current is 
\begin{equation}
    \begin{aligned}
        I_1 &= - e \cdot \text{density of states} \cdot \text{cross section} \cdot v \\
        &= - e \frac{1}{L} \sum_{\text{ocupied states}} \frac{1}{\hbar} \pdv{\epsilon_{1k}}{k} 
        = - \frac{e}{\hbar} \frac{n}{L} \sum_{\Lambda_1 \leq k \leq k_{\text{1F}}} \pdv{\epsilon_{1k}}{k} 
        = - n \frac{e}{\hbar} \int_{\Lambda_1}^{k_{\text{1F}}} \frac{\dd{k}}{2\pi} \pdv{\epsilon_{1k}}{k} \\
        &= - n \frac{e}{h} (\epsilon_{1 \text{F}} - \epsilon_{1, k = \Lambda_1})
        = - n \frac{e}{h} (\mu_1 - \epsilon_{1, k = \Lambda_1}).
    \end{aligned}
\end{equation}
So we have 
\begin{equation}
    \Delta I_1 = - n \frac{e}{h} \Delta \mu_1.
\end{equation}

\item[(b)] Similarly, we have 
\begin{equation}
    \begin{aligned}
        I_2 &= - e \cdot \text{density of states} \cdot \text{cross section} \cdot v \\
        &= - e \frac{1}{L} \sum_{\text{ocupied states}} \frac{1}{\hbar} \pdv{\epsilon_{1k}}{k} 
         = - \frac{e}{\hbar} \frac{n}{L} \sum_{k_{2\text{F}} \leq k \leq \Lambda_2} \pdv{\epsilon_{2k}}{k} 
         = - n \frac{e}{\hbar} \int^{\Lambda_2}_{k_{2\text{F}}} \frac{\dd{k}}{2\pi} \pdv{\epsilon_{2k}}{k} \\
        &= - n \frac{e}{h} (\epsilon_{2, k = \Lambda_2} - \epsilon_{2 \text{F}}) 
         = - n \frac{e}{h} (\epsilon_{2, k = \Lambda_2} - \mu) , 
    \end{aligned}
\end{equation}
and 
\begin{equation}
    \Delta I_2 = n \frac{e}{\hbar} \Delta \mu_2.
\end{equation}

\item[(c)] We have 
\begin{equation}
    I = - n \frac{e}{h} (\mu_1 - \mu_2 + \epsilon_{2, k = \Lambda_2} - \epsilon_{1, k = \Lambda_1}).
\end{equation}
When $\Delta \mu = 0$, we have $I = 0$, so we find the two cutoff terms cancel each other, and 
\begin{equation}
    I = - n \frac{e}{h} \Delta \mu.
\end{equation}

In a Hall effect setting, the difference of the chemical potentials arises from the external electric field:
conservation of energy tells us 
\begin{equation}
    \mu_1 - e U = \mu_2,
\end{equation}
and therefore 
\begin{equation}
    I = - n \frac{e}{h} \cdot e U = \underbrace{ - n \frac{e^2}{h} }_{1 / R_{\text{H}}} U.
\end{equation}
So we get the expected quantized conductance.

\item[(d)] Ignoring excited bulk modes, we have 
\begin{equation}
    \begin{aligned}
        I_1 &= - n \frac{e}{\hbar} \int_{\Lambda_1}^{\infty} 
        \frac{\dd{k}}{2\pi} \frac{1}{1 + \ee^{(\epsilon_{1k} - \mu_1) / k_{\text{B}} T}} \pdv{\epsilon_{1k}}{k} \\
        &= - n \frac{e}{h} \int_{\epsilon_{1, k = \Lambda_1} - \mu_1}^{\infty} 
        \dd{\xi} \frac{1}{1 + \ee^{\xi / k_{\text{B}} T} }, 
    \end{aligned}
\end{equation}
and similarly 
\begin{equation}
    \begin{aligned}
        I_2 &= - n \frac{e}{\hbar} \int^{\Lambda_2}_{-\infty} 
        \frac{\dd{k}}{2\pi} \frac{1}{1 + \ee^{(\epsilon_{2k} - \mu_2) / k_{\text{B}} T}} \pdv{\epsilon_{2k}}{k} \\
        &= - n \frac{e}{h} \int^{\epsilon_{2, k = \Lambda_2} - \mu_2}_{\infty} 
        \dd{\xi} \frac{1}{1 + \ee^{\xi / k_{\text{B}} T} } 
        = n \frac{e}{h} \int_{\epsilon_{2, k = \Lambda_2} - \mu_2}^{\infty} 
        \dd{\xi} \frac{1}{1 + \ee^{\xi / k_{\text{B}} T} } , 
    \end{aligned}
    \label{eq:i-2-e}
\end{equation}
and therefore 
\begin{equation}
    I = - n \frac{e}{h} \int_{\epsilon_{1, k = \Lambda_1} - \mu_1}^{\epsilon_{2, k = \Lambda_2} - \mu_2}
    \frac{\dd{\xi}}{1 + \ee^{\xi / k_{\text{B}} T}}.
\end{equation}
Since the Fermi-Dirac distribution function is always positive,
the condition that when $\Delta \mu = 0$, $I = 0$
implies $\epsilon_{1, k=\Lambda_1} = \epsilon_{2, k=\Lambda_2} \eqqcolon \epsilon_\Lambda$, so we have 
\begin{equation}
    I = - n \frac{e}{h} \int_{\epsilon_{\Lambda} - \mu_1}^{\epsilon_{\Lambda} - \mu_2}
    \frac{\dd{\xi}}{1 + \ee^{\xi / k_{\text{B}} T}}.
\end{equation}
Since $\Lambda$ must be large, a very good approximation is 
\begin{equation}
    I = - n \frac{e}{h} \Delta \mu = - n \frac{e^2}{h} U,
\end{equation}
so the edge current and the Hall conductance are not affected much by a finite $T$.

\end{itemize}

\paragraph{Problem 2}

\paragraph{Solution} \begin{itemize}
\item[(a)] The only change we need to do is to replace the electric charge $-e$ 
by the ``energy charge'' $\epsilon_{k}$, and following the line of reasoning in Problem 1 (d), 
we have 
\begin{equation}
    \begin{aligned}
        I_{Q1} &= \frac{n}{h} \int^\infty_{\epsilon_{1, k = \Lambda_1}} \dd{\epsilon}  
        \frac{1}{1 + \ee^{(\epsilon - \mu_1) / k_{\text{B}} T_1}}
        \epsilon \\
        &= k_{\text{B}} T_1 \frac{n}{h} 
        \left(
            k_{\text{B}} T_1 \int^\infty_{\frac{\epsilon_{1, k = \Lambda_1} - \mu}{k_{\text{B}} T_1}} \dd{x}
            \frac{x}{1 + \ee^x}
            + \mu \int^\infty_{\frac{\epsilon_{1, k = \Lambda_1} - \mu}{k_{\text{B}} T_1}} \dd{x}
            \frac{1}{1 + \ee^x}
        \right),
    \end{aligned}
\end{equation}
and similarly 
\begin{equation}
    I_{Q2} = - k_{\text{B}} T_2 \frac{n}{h} 
    \left(
        k_{\text{B}} T_2 \int^\infty_{\frac{\epsilon_{2, k = \Lambda_2} - \mu}{k_{\text{B}} T_2}} \dd{x}
        \frac{x}{1 + \ee^x}
        + \mu \int^\infty_{\frac{\epsilon_{2, k = \Lambda_2} - \mu}{k_{\text{B}} T_2}} \dd{x}
        \frac{1}{1 + \ee^x}
    \right).
\end{equation}
The sign difference between $I_{Q1}$ and $I_{Q2}$ comes in the same way of \eqref{eq:i-2-e}.
Again, when $T_1 = T_2$, 
we need $I = I_{Q1} + I_{Q2} = 0$, 
so $\epsilon_{1, k=\Lambda_1} = \epsilon_{2, k=\Lambda_2} \eqqcolon \epsilon_\Lambda$.

\item[(b)] We have 
\begin{equation}
    \int_{t}^\infty \dd{x} \frac{1}{1 + \ee^x} = 2 \arctanh (1 + 2 \ee^t),
\end{equation}
\begin{equation}
    \int_{t}^\infty \dd{x} \frac{x}{1 + \ee^x} = t \ln(1 + \ee^{-t}) - \polylog_2(- \ee^{-t}).
\end{equation}


\end{itemize}

\end{document}