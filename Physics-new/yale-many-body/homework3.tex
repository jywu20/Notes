\documentclass[hyperref, a4paper]{article}

\usepackage{geometry}
\usepackage{titling}
\usepackage{titlesec}
% No longer needed, since we will use enumitem package
% \usepackage{paralist}
\usepackage{enumitem}
\usepackage{footnote}
%\usepackage{enumerate}
\usepackage{amsmath, amssymb, amsthm}
\usepackage{mathtools}
\usepackage{bbm}
\usepackage{cite}
\usepackage{graphicx}
\usepackage{subfigure}
\usepackage{physics}
\usepackage{tensor}
\usepackage{siunitx}
\usepackage[version=4]{mhchem}
\usepackage{tikz}
\usepackage{xcolor}
\usepackage{listings}
\usepackage{autobreak}
\usepackage[ruled, vlined, linesnumbered]{algorithm2e}
\usepackage{nameref,zref-xr}
\zxrsetup{toltxlabel}
\usepackage[colorlinks,unicode]{hyperref} % , linkcolor=black, anchorcolor=black, citecolor=black, urlcolor=black, filecolor=black
\usepackage[most]{tcolorbox}
\usepackage{prettyref}

% Page style
\geometry{left=3.18cm,right=3.18cm,top=2.54cm,bottom=2.54cm}
\titlespacing{\paragraph}{0pt}{1pt}{10pt}[20pt]
\setlength{\droptitle}{-5em}
%\preauthor{\vspace{-10pt}\begin{center}}
%\postauthor{\par\end{center}}

% More compact lists 
\setlist[itemize]{
    itemindent=17pt, 
    leftmargin=1pt,
    listparindent=\parindent,
    parsep=0pt,
}

% Math operators
\DeclareMathOperator{\timeorder}{\mathcal{T}}
\DeclareMathOperator{\diag}{diag}
\DeclareMathOperator{\legpoly}{P}
\DeclareMathOperator{\primevalue}{P}
\DeclareMathOperator{\sgn}{sgn}
\DeclarePairedDelimiter\ceil{\lceil}{\rceil}
\DeclarePairedDelimiter\floor{\lfloor}{\rfloor}
\newcommand*{\ii}{\mathrm{i}}
\newcommand*{\ee}{\mathrm{e}}
\newcommand*{\const}{\mathrm{const}}
\newcommand*{\suchthat}{\quad \text{s.t.} \quad}
\newcommand*{\argmin}{\arg\min}
\newcommand*{\argmax}{\arg\max}
\newcommand*{\normalorder}[1]{: #1 :}
\newcommand*{\pair}[1]{\langle #1 \rangle}
\newcommand*{\fd}[1]{\mathcal{D} #1}
\DeclareMathOperator{\bigO}{\mathcal{O}}

% TikZ setting
\usetikzlibrary{arrows,shapes,positioning}
\usetikzlibrary{arrows.meta}
\usetikzlibrary{decorations.markings}
\tikzstyle arrowstyle=[scale=1]
\tikzstyle directed=[postaction={decorate,decoration={markings,
    mark=at position .5 with {\arrow[arrowstyle]{stealth}}}}]
\tikzstyle ray=[directed, thick]
\tikzstyle dot=[anchor=base,fill,circle,inner sep=1pt]

% Algorithm setting
% Julia-style code
\SetKwIF{If}{ElseIf}{Else}{if}{}{elseif}{else}{end}
\SetKwFor{For}{for}{}{end}
\SetKwFor{While}{while}{}{end}
\SetKwProg{Function}{function}{}{end}
\SetArgSty{textnormal}

\newcommand*{\concept}[1]{{\textbf{#1}}}

% Embedded codes
\lstset{basicstyle=\ttfamily,
  showstringspaces=false,
  commentstyle=\color{gray},
  keywordstyle=\color{blue}
}

% Reference formatting
\newrefformat{fig}{Figure~\ref{#1}}

% Color boxes
\tcbuselibrary{skins, breakable, theorems}
\newtcbtheorem[number within=section]{warning}{Warning}%
  {colback=orange!5,colframe=orange!65,fonttitle=\bfseries, breakable}{warn}
\newtcbtheorem[number within=section]{note}{Note}%
  {colback=green!5,colframe=green!65,fonttitle=\bfseries, breakable}{note}
\newtcbtheorem[number within=section]{info}{Info}%
  {colback=blue!5,colframe=blue!65,fonttitle=\bfseries, breakable}{info}

\newenvironment{shelldisplay}{\begin{lstlisting}}{\end{lstlisting}}

\title{Homework 3}
\author{Jinyuan Wu}

\begin{document}

\maketitle

\paragraph{Problem 1} In class, we show that the dynamics of the (approximate) ground states for a $d$-dimensional superfluid is described by an effective Lagrangian of the form
$$
L=V\left(\frac{1}{2 U_0} \dot{\theta}^2-\frac{\mu}{U_0} \dot{\theta}\right) .
$$
Here $V=L^d$ is the volume of the system.
1. If the system is initialized in a superposition of different $|\theta\rangle$ with $\langle\theta\rangle=0,\left\langle\theta^2\right\rangle=\sigma^2$ with $\sigma \ll 1$, estimate the time it takes for the spread of $\theta$ to become of order of $2 \pi$. This is the time scale at which the phase becomes ill-defined.
2. Estimate this time scale for $1 \mathrm{~cm}^3$ of superfluid ${ }^4 \mathrm{He}$ at zero temperature with an initial spread $\sigma=\frac{\pi}{50}$ (you can estimate $V_0$ from the sound velocity of $\left.{ }^4 \mathrm{He}\right)$.
3. Now imagine adding a small perturbation to the superfluid Hamiltonian:
$$
\hat{H} \rightarrow \hat{H}-g \int \mathrm{d}^d \mathbf{x}\left[\hat{a}^2+\left(\hat{a}^{\dagger}\right)^2\right], g>0 .
$$
Under this perturbation, the particle number is only conserved mod 2, breaking the symmetry of the system from $\mathrm{U}(1)$ to $\mathbb{Z}_2$. In the presence of this perturbation, the ground state is now a $\mathbb{Z}_2$ symmetry-breaking state, similar to an Ising ferromagnet. Instead of an infinite number of degenerate ground states $|\theta\rangle$ with $\langle\theta|\hat{a}| \theta\rangle=\sqrt{\rho_0} e^{i \theta}$, there are only two degenerate ground states corresponding to $\theta=0, \pi$. If the system is initialized in one of these states, estimate the time scale for it to tunnel to the other ground state. This is the time scale on which discrete symmetry breaking disappears. You may restrict yourself to spatially uniform configurations and neglect non-exponential prefactors and numerical constants.

\paragraph{Solution} \begin{itemize}
\item[(a)] The conjugate momentum of $\theta$ is 
\begin{equation}
    p = \pdv{L}{\dot{\theta}}
    = V \left( \frac{\dot{\theta}}{U_0} - \frac{\mu}{U_0} \right),
\end{equation}
and therefore 
\begin{equation}
    \dot{\theta} = \frac{U_0}{V} p + \mu.
\end{equation}
The Hamiltonian is 
\begin{equation}
    \begin{aligned}
        H &= p \dot{\theta} - L \\
        &= p \left( \frac{U_0}{V} p + \mu \right) - 
        V \left( 
            \frac{1}{2U_0} \left( \frac{U_0}{V} p + \mu \right)^2 
            - \frac{\mu}{U_0} \left( \frac{U_0}{V} p + \mu \right) 
        \right) \\ 
        &= \frac{1}{2} \frac{U_0}{V} \left( p + \frac{\mu V}{U_0} \right)^2.
    \end{aligned}
\end{equation}
In Heisenberg's picture, the variance of $\theta$ can be evaluated in the follows.
We know 
\[
    \begin{aligned}
        \dv{\theta^2}{t} &= \frac{1}{\ii} \comm{\theta^2}{H} \\
        &= \frac{U_0}{2 \ii V} \comm{\theta^2}{\left( p + \frac{\mu V}{U_0} \right)^2} \\
        &= \frac{U_0}{V} \left(
            \theta \left( p + \frac{\mu V}{U_0} \right)
            + \left( p + \frac{\mu V}{U_0} \right) \theta
        \right),
    \end{aligned}
\]
and therefore 
\begin{equation}
    \begin{aligned}
        \dv[2]{\theta^2}{t} &= \frac{U_0}{V} \left(
            \dot{\theta} \left( p + \frac{\mu V}{U_0} \right)
            + \theta \dot{p} 
            + \dot{p} \theta + 
            \left(p + \frac{\mu V}{U_0}\right) \dot{\theta} 
        \right) \\
        &= \frac{2 U_0^2}{V^2} \left( p + \frac{\mu V}{U_0} \right)^2.
    \end{aligned}
    \label{eq:eom-sigma-op}
\end{equation}
Here we use the EOMs 
\begin{equation}
    \dot{\theta} = \frac{1}{\ii} \comm{\theta}{H} = \frac{U_0}{V} \left( p + \frac{\mu V}{U_0} \right),
    \quad \dot{p} = 0.
\end{equation}
To calculate $\sigma_\theta^2$, we also need $\expval*{\theta}^2$. Its time evolution is given by 
\begin{equation}
    \begin{aligned}
        \dv[2]{\expval*{\theta}}{t} &= 2 \expval*{\theta} \dv[2]{\expval*{\theta}}{t} 
       + 2 \left( \dv{\expval*{\theta}}{t} \right)^2  \\
       &= 2 \expval*{\theta} \dv{t} \left(  \frac{U_0}{V} \expval*{p} + \mu \right)
       + 2 \left(  \frac{U_0}{V} \expval*{p} + \mu \right)^2 \\
       &= \frac{2 U_0^2}{V^2} \left( \expval*{p} + \frac{\mu V}{U_0} \right)^2 .
    \end{aligned}
    \label{eq:theta-exp-2-eom}
\end{equation}
From \eqref{eq:eom-sigma-op} and \eqref{eq:theta-exp-2-eom}, we have 
\begin{equation}
    \begin{aligned}
        \dv[2]{\sigma_\theta^2}{t} &= 
        \dv[2]{t} (\expval*{\theta^2} - \expval*{\theta}^2) 
        = \frac{2 U_0^2}{V^2} \left( 
            \expval{\left( p + \frac{\mu V}{U_0} \right)^2} 
            - \left( \expval*{p} + \frac{\mu V}{U_0} \right)^2 
        \right)  \\
        &= \frac{2 U_0^2}{V^2} \sigma_p^2.
    \end{aligned}
\end{equation}
From uncertainty relation we have 
\begin{equation}
    \sigma_p^2 \sigma_x^2 \simeq \frac{1}{4},
\end{equation}
and therefore 
\begin{equation}
    \dv[2]{\sigma_\theta^2}{t} \simeq \frac{2 U_0^2}{V^2} \frac{1}{4 \sigma^2},
\end{equation}
and therefore 
\begin{equation}
    \sigma_\theta \simeq \sqrt{ \frac{ U_0^2}{4 V^2 \sigma^2} t^2 + \sigma^2 }.
\end{equation}

\item[(b)] The speed sound is 
\begin{equation}
    v = \sqrt{\frac{\rho_0 U_0}{m}},
\end{equation}
so 
\begin{equation}
    \sigma_\theta  = \sqrt{ \frac{m^2 v^4}{4 V^2 \rho_0^2 \sigma^2} t^2 + \sigma^2 },
\end{equation}
and the time it takes to have $\sigma_\theta = 2 \pi$ is 
\begin{equation}
    t = \sqrt{ \frac{4 V^2 \rho_0^2 \sigma^2}{m^2 v^4} (4\pi^2 - \sigma^2 )}.
\end{equation} 
For $^4$\ce{He}, we have $m = \SI{6.65e-27}{kg}$,
and the sound speed is around $\sim \SI{20}{m/s}$ \cite{PhysRev.71.600},
and the mass density -- the product of $\rho_0$ and $m$ -- is $\sim \SI{125}{g/L}$,
so $t \sim \SI{5.6e45}{s}$.

\item[(c)] Ignoring density fluctuation,
the perturbation is 
\begin{equation}
    - g \int \dd[d]{\vb*{x}} (a^2 + (a^\dagger)^2) = 
    - g \int \dd[d]{\vb*{x}} \rho_0 (\ee^{\ii 2 \theta} + \ee^{- \ii 2 \theta})
    = - 2 g \rho_0 \int \dd[d]{\vb*{x}} \cos(2 \theta).
\end{equation}
In the uniform $\theta$ case, we further have 
\begin{equation}
    H = \frac{1}{2} \frac{U_0}{V} \left( p + \frac{\mu V}{U_0} \right)^2
    - 2 g \rho_0 V \cos(2 \theta).
\end{equation}
The corresponding Lagrangian is 
\begin{equation}
    L=V\left(\frac{1}{2 U_0} \dot{\theta}^2-\frac{\mu}{U_0} \dot{\theta}\right)
    + 2 g \rho_0 V \cos(2 \theta).
\end{equation}
Since we are dealing with a tunneling problem, we have the following imaginary time action:
\begin{equation}
    S = V \int_0^{T} \dd{\tau}
    \left( 
        \frac{1}{2U_0} (\partial_\tau \theta)^2 
        + \frac{\mu}{U_0} \partial_\tau \theta 
        - 2 g \rho_0 \cos(2\theta) \right) .
\end{equation}
The EOM is 
\begin{equation}
    \frac{V}{U_0} \ddot{\theta} - 4 g \rho_0 V \sin(2\theta) = 0,
\end{equation}
and therefore 
\begin{equation}
    \frac{V}{2U_0} \dot{\theta}^2 + 2g \rho_0 V \cos 2 \theta = E.
\end{equation}
Here we need to maximize $E$ to find saddle points,
and therefore 
\begin{equation}
    \partial_\tau \theta = \pm \sqrt{ 4 g \rho_0 U_0 (1 - \cos 2 \theta) }.
\end{equation}
So the instanton actions are 
\begin{equation}
    \begin{aligned}
        S_\pm &= \pm \ii V \frac{\mu}{U_0} \pi + V \int_0^T \dd{\tau} 2 g \rho_0 (1 - 2 \cos 2\theta)  \\
        &= \pm \ii V \frac{\mu}{U_0} \pi 
        \pm V \int \dd{\theta} 
        \frac{2 g \rho_0 (2 - 2 \cos 2\theta)}{\sqrt{ 4 g \rho_0 U_0 (1 - \cos 2 \theta) }} 
        - 2 g \rho_0 V T .
    \end{aligned}
\end{equation}
Here $+$ means the $\theta$ jumps from $0$ to $\pi$ or from $\pi$ to $2\pi$,
and $-$ is the inverse.
We can say there are only two kinds of instantons,
because when $\theta = \pi$,
we only have $\pi \to 2\pi$ and $\pi \to 0$ types of instantons,
and when $\theta = 0$,
we only have $0 \to \pi$ and $0 \to - \pi$,
and the actions of the ``increasing'' type instanton are always the same,
and so are the actions of the ``decreasing'' type instanton.
Thus we find 
\begin{equation}
    S_+ = \frac{\ii V \mu \pi}{U_0} + 4 V \sqrt{\frac{2 g \rho_0}{U_0}} - 2 \rho_0 g V T,
\end{equation}
\begin{equation}
    S_- = - \frac{\ii V \mu \pi}{U_0} + 4 V \sqrt{\frac{2 g \rho_0}{U_0}} - 2 \rho_0 g V T.
\end{equation}
Similar to the procedure in the last homework,
suppose the contribution of Gaussian fluctuations of each instanton around the saddle point solutions is $K$,
the path integral is 
\begin{equation}
    \begin{aligned}
        \mel{\theta = \pi}{\ee^{- H T}}{\theta = 0} &= \sum_{\text{$n_+ + n_-$ odd}}
        \int_0^{T} \dd{\tau_1} \int_{\tau_1}^{T} \dd{\tau_2} \dots \int_{\tau_{n_+ + n_- - 1}}^T 
        \dd{\tau_{n_+ + n_-}} K^{n_+ + n_-}  \\
        &\quad \quad \quad \cdot \binom{n_+ + n_-}{n_+} \ee^{- n_+ S_+ - n_- S_-} U(\theta = \pi, T; \theta = 0, 0) \\
        &= \sum_{\text{$N$ odd}} \sum_{n_+ = 0}^N \frac{T^N}{N!} 
        \binom{N}{n_+} (K \ee^{- S_+})^{n_+} (K \ee^{- S_-})^{N - n_+}.
    \end{aligned}
\end{equation}
How to proceed?

\end{itemize}

\paragraph{Problem 2} Consider a $d$-dimensional boson system with a long-range repulsive interaction of the form
$$
\frac{1}{2} \int \mathrm{d} \mathbf{x} \mathrm{d} \mathbf{y}|\varphi(\mathbf{x})|^2 U(\mathbf{x}-\mathbf{y})|\varphi(\mathbf{y})|^2,
$$
where the potential is given by
$$
U(r)=\frac{U_0}{r^{d-\epsilon}}, 0<\epsilon<d .
$$
1. Assuming that the system is a superfluid, derive the low-energy effective theory.
2. Find the dispersion relation for the low-energy phonons. You should find a gap for a $\frac{1}{r}$ potential in $(3+1) \mathrm{d}$.
Remark: this is the Anderson-Higgs mechanism: introducing Coulomb interactions, or equivalently coupling to a U(1) gauge field, gaps out the low-energy phonons in a superfluid.

\paragraph{Solution} \begin{itemize}
\item[(a)] Repeating the procedure used in ordinary superfluid, we do the decomposition
\begin{equation}
    \varphi = \sqrt{\rho} \ee^{\ii \theta} = \sqrt{\rho_0 + \var{\rho}} \ee^{\ii \theta},
\end{equation}
and therefore 
\begin{equation}
    - \frac{\varphi^* \laplacian \varphi}{2m} = 
    \frac{\rho}{2m} (\grad{\theta})^2 + \frac{(\grad{\rho})^2}{8 \rho m} ,
\end{equation}
\begin{equation}
    \varphi^* \partial_\tau \varphi 
    = \underbrace{\frac{1}{2} \partial_\tau \rho}_{\text{time derivative, ignored}} + 
    \ii \rho \partial_\tau \theta,
\end{equation}
\begin{equation}
    \abs{\varphi(\vb*{x})} U(\vb*{x} - \vb*{y}) \abs{\varphi(\vb*{y})} = 
    \rho(\vb*{x}) U(\vb*{x} - \vb*{y}) \rho(\vb*{y}),
\end{equation}
the theory is now 
\begin{equation}
    S = \int \dd{\tau} \left(
        \int \dd[d]{\vb*{x}} \left(
            \ii \rho \partial_\tau \theta 
            + \frac{\rho}{2m} (\grad{\theta})^2 + \frac{(\grad{\rho})^2}{8 \rho m}
            - \mu \rho
        \right) + 
        \frac{1}{2} \int \dd[d]{\vb*{x}} \int \dd[d]{\vb*{y}}
        \rho(\vb*{x}) U(\vb*{x} - \vb*{y}) \rho(\vb*{y})
    \right).
\end{equation}
Around the ground state, we have 
(note that since we are around a saddle point,
the sum of all terms containing $\var{\rho}$ only is always zero;
the resulting theory has the form of $c_1 \var{\rho} \partial_\tau \theta + c_2 \var{\rho}^2$;
the chemical potential term is therefore missing in the theory around the saddle point)
\[
    \ii \rho \partial_\tau \theta = \underbrace{\ii \rho_0 \partial_\tau \theta}_{\text{time derivative}} 
    + \ii \var{\rho} \partial_\tau \theta,
\]
and since $\grad{\rho} = \grad{\var{\rho}}$,
we have 
\[
    \frac{(\grad{\rho})^2}{8 \rho m} \approx \frac{(\grad{\var{\rho}})^2}{8 \rho_0 m},
\]
ignoring the fluctuation of the $\rho$ in the denominator.
Similarly, since we are working on a low energy theory,
the fluctuation of $\theta$ shouldn't be large,
and we have 
\[
    \frac{\rho}{2m} (\grad{\theta})^2 \approx \frac{\rho_0}{2m} (\grad{\theta})^2.
\]
The theory is then (here we use the condition that $U(\vb*{x}) = U(- \vb*{x})$)
\begin{equation}
    S = \int \dd[d+1]{x} \left(
        \frac{\rho_0}{2m} (\grad{\theta})^2 
        + \ii \var{\rho} \partial_\tau \theta  
        + \frac{(\grad{\var{\rho}})^2}{8 \rho_0 m} 
        + \frac{1}{2} \var{\rho}(\vb*{x}) \int \dd[d]{\vb*{y}} U(\vb*{x} - \vb*{y}) \var{\rho}(\vb*{y})
    \right) + S_{\text{saddle}}.
\end{equation}
Integrating out $\var{\rho}$, we get 
\begin{equation}
    \begin{aligned}
        S_{\text{eff}} &= \int \dd[d+1]{x} \frac{\rho_0}{2m} (\grad{\theta})^2  \\
        &\quad \quad \quad - \frac{1}{2} \int \dd{\tau} \int \dd[d]{\vb*{x}} \dd[d]{\vb*{y}} 
        \ii \partial_\tau \theta(\vb*{x}, \tau)
            \frac{1}{
                \int \dd[d]{\vb*{y}} U(\vb*{x} - \vb*{y}) 
                - \frac{1}{4 \rho_0 m} \delta(\vb*{x} - \vb*{y}) \laplacian_{\vb*{x}}
            } \ii \partial_\tau \theta (\vb*{y}, \tau) \\
        &= \int \dd[d+1]{x} \frac{\rho_0}{2m} (\grad{\theta})^2 + 
        \frac{1}{2} \int \dd{\tau} \int \dd[d]{\vb*{x}} \dd[d]{\vb*{y}}
        \partial_\tau \theta(\vb*{x}, \tau) G(\vb*{x} - \vb*{y}) \partial_\tau \theta(\vb*{y}),
    \end{aligned}
    \label{eq:eft-theta}
\end{equation}
where 
\begin{equation}
    \int \dd[d]{\vb*{y}} U(\vb*{x} - \vb*{y}) G(\vb*{y} - \vb*{z}) 
    - \frac{1}{4 \rho_0 m} \laplacian_{\vb*{x}} G(\vb*{x} - \vb*{z})  = \delta(\vb*{x} - \vb*{z}).
    \label{eq:g-def}
\end{equation}
Similar to the procedure in dealing with ordinary superfluid,
since we are only interested in the long wave length behaviors of $\theta$,
the $\laplacian$ term in $G(\vb*{x} - \vb*{y})$ can be thrown away in the effective theory.
Evaluation of $G(\vb*{x} - \vb*{y})$ is done below.

\item[(b)] The EOM of \eqref{eq:eft-theta} is (again the fact that $U(\vb*{x}) = U(- \vb*{x})$ is used)
\[
    - \frac{\rho_0}{m} \laplacian \theta - \int \dd[d]{\vb*{y}} G(\vb*{x} - \vb*{y}) \partial_\tau^2 \theta(\vb*{y}) = 0,
\]
or in real time, 
\begin{equation}
    \frac{\rho_0}{m} \laplacian \theta = \int \dd[d]{\vb*{y}} G(\vb*{x} - \vb*{y}) \partial_\tau^2 \theta(\vb*{y}).
\end{equation}
Switching to the frequency domain, 
we have 
\begin{equation}
    \frac{\rho_0}{m} (\ii \vb*{k})^2 \theta(\vb*{k}, \omega)
    = \int \dd[d]{\vb*{z}} \ee^{- \ii \vb*{z} \cdot \vb*{k}} G(\vb*{x}) (- \ii \omega)^2 \theta(\vb*{k}, \omega)
    = G(\vb*{k}) (- \ii \omega)^2 \theta(\vb*{k}, \omega),
\end{equation}
and the dispersion relation is 
\begin{equation}
    \omega^2 = \frac{\rho_0}{m} \frac{\vb*{k}^2}{G(\vb*{k})}.
\end{equation}

By definition, 
\[
    \begin{aligned}
        \delta(\vb*{x} - \vb*{z}) 
        = \int \frac{\dd[d]{\vb*{p}}}{(2\pi)^d} \ee^{\ii \vb*{p} \cdot (\vb*{x} - \vb*{z})}
        &= \int \frac{\dd[d]{\vb*{p}}}{(2\pi)^d} \int \dd[d]{\vb*{y}} U(\vb*{x} - \vb*{y}) 
        G(\vb*{p}) \ee^{\ii \vb*{p} \cdot (\vb*{y} - \vb*{z})} \\
        &= \int \frac{\dd[d]{\vb*{p}}}{(2\pi)^d} 
        \int \dd[d]{\vb*{r}} U(\vb*{r}) \ee^{- \ii \vb*{p} \cdot \vb*{r}} 
        G(\vb*{p}) \ee^{\ii \vb*{p} \cdot (\vb*{x} - \vb*{z})} \quad (\vb*{r} = \vb*{x} - \vb*{y}) ,
    \end{aligned}
\]
so 
\begin{equation}
    G(\vb*{r}) = \int \frac{\dd[d]{\vb*{p}}}{(2\pi)^d} \ee^{\ii \vb*{p} \cdot \vb*{r}} G(\vb*{p}), \quad 
    G(\vb*{p}) = \frac{1}{U(\vb*{p})} 
    = \frac{1}{\int \dd[d]{\vb*{r}} U(\vb*{r}) \ee^{- \ii \vb*{p} \cdot \vb*{r}}}.
\end{equation}
So now the dispersion relation is 
\begin{equation}
    \omega_{\vb*{k}} = \sqrt{ \frac{\rho_0}{m} U(\vb*{k}) \vb*{k}^2  }.
\end{equation}
In the $d = 3$, $\epsilon = 2$ case, we already know
\begin{equation}
    U(\vb*{k}) = \frac{4\pi U_0}{\vb*{k}^2},
\end{equation}
and therefore 
\begin{equation}
    \omega_{\vb*{k}} = \sqrt{\frac{4 \pi \rho_0 U_0}{m}}.
    \label{eq:coulomb-gap}
\end{equation}
When $\vb*{k}$ is large enough there will still be dispersion,
because in this case the $\laplacian$ term in \eqref{eq:g-def} is important;
anyway \eqref{eq:coulomb-gap} is gapped,
and this means the Coulomb interaction between superfluid bosons gaps out the phonon modes.

\end{itemize}

\paragraph{Problem 3} 3. The Bose-Hubbard model and superfluidity for lattice bosons
In this problem we study the physics of interacting bosons on a $d$-dimensional cubic lattice. They are described by the following Bose-Hubbard model:
$$
\hat{H}=-t \sum_{\langle i j\rangle}\left(\hat{a}_i^{\dagger} \hat{a}_j+\text { h.c. }\right)+\frac{U}{2} \sum_i \hat{n}_i\left(\hat{n}_i-1\right)-\mu \sum_i \hat{n}_i .
$$
Here $t, \mu, U>0$. $\hat{a}_i$ is the boson annihilation operator on the site $i$ of the lattice, and $\hat{n}_i=\hat{a}_i^{\dagger} \hat{a}_i$ is the number operator on site $i$. The first term describes hopping between adjacent sites $i, j$, the second is an on-site repulsive interaction, and the last term is a chemical potential.
1. When $t \ll U$, the ground state is a "Mott insulator" with an integer number of bosons, $M$, on each site. Find $M$ as a function of $\mu, U$ when $t=0$. Find the phase diagram as a function of $\mu / U$ at $t=0$ and plot the energy gap in units of $U$ as a function of $\mu / U$.
2. When $t \gg U$, the ground state is a superfluid. Using this fact and the results in Part (a), sketch the simplest consistent qualitative phase diagram for the system as a function of $t / U$ and $\mu / U$.
3. It is particularly easy to analyze the model in the regime where the average density is much larger than the density fluctuations: $\left\langle\hat{n}_i\right\rangle \gg \sqrt{\left\langle\left(\hat{n}_i-\left\langle\hat{n}_i\right\rangle\right)^2\right\rangle}$. In this regime, the low-energy physics of the BoseHubbard model can be approximately mapped onto a lattice of quantum rotors. To understand this mapping, write down matrix representations for the bosonic operators $\left\{\hat{a}, \hat{a}^{\dagger}, \hat{n}\right\}$ within the subspace of states $\left\{\left|n_0-k\right\rangle,\left|n_0-k+1\right\rangle, \ldots,\left|n_0+k\right\rangle\right\}, k<n_0$. Similarly write down matrix representations for the quantum rotor operators $\left\{e^{-i \hat{\theta}}, e^{i \hat{\theta}}, \hat{\pi}\right\}$, where $[\hat{\theta}, \hat{\pi}]=i$, within the subspace of states $\{|-k\rangle,|-k+1\rangle, \ldots,|k\rangle\}$, where $\hat{\pi}|k\rangle=k|k\rangle$. Notice that because $\theta$ is $2 \pi$ periodic, the conjugate momentum $\hat{\pi}$ must have integer eigenvalues.
Show that for $k \ll n_0$, the two sets of matrices are approximately related by $\hat{a} \approx \sqrt{n_0} e^{-i \hat{\theta}}, \hat{a}^{\dagger} \approx \sqrt{n_0} e^{i \hat{\theta}}, \hat{n}=$ $n_0+\hat{\pi}$.
4. Assume that we are in the above regime of small density fluctuations, and that $\left\langle\hat{n}_i\right\rangle$ is an integer $n_0$. Show that the Bose-Hubbard Hamiltonian onto a lattice rotor Hamiltonian of the form
\begin{equation}
    \hat{H}=-J \sum_{\langle i j\rangle} \cos \left(\hat{\theta}_i-\hat{\theta}_j\right)+u \sum_i \hat{\pi}_i^2 .
    \label{eq:o-6}
\end{equation}
Why is there no term linear in $\hat{\pi}_i$?
5. When $u \gg J$, the ground state of \eqref{eq:o-6} is a gapped state where $\hat{\pi}_i$ is essentially locked at 0 on each site, an analogue of a Mott insulator. When $J \gg u$, the ground state is a $\mathrm{U}(1)$ symmetry-breaking ferromagnet, roughly given by $\hat{\theta}_i=\theta$ on each site, the analogue of a superfluid. The low-energy excitations in this state corresponds to slowly varying $\theta_i$. Find the dispersion relation for these excitations and compute their velocity at small wavevector $k$. These "spin wave" excitations are analogous to phonons.
6. Now add an additional term $-B \sum_i \hat{\pi}_i$ to \eqref{eq:o-6}. Sketch a qualitative phase diagram for the model as a function of $J / u, B / u$.
7. Write down a Lagrangian in terms of $\theta_i$ and take the continuum limit. Note that when $B \neq 0$, the Lagrangian contains a term proportional to $\partial_t \theta_i$. While this term does not have much of an impact on the low-energy physics in the two phases, it has an important effect on the phase transition between them. Specifically, when this term is present, the phase transition is in the same universality class as the dilute Bose gas transition discussed in class, while when the term is absent, the phase transition is in a different universality class ( the $(d+1)$-dimensional classical XY universality class).
8. Actually, this term is also absent for special, nonzero values of $B$. Show that for certain values of $B$, the term linear in $\partial_t \theta_i$ does not affect the partition function $Z=\operatorname{Tr} e^{-\beta \hat{H}}$ and can thus be dropped. Find the corresponding values of $B / u$, and mark on your rotor model phase diagram the special points where the phase transition is in the XY universality class. Remark: in terms of the Bose-Hubbard model, these special points correspond to points where $\left\langle n_i\right\rangle$ is an integer.

\paragraph{Solution} \begin{itemize}
\item[1.] The energy now can be exactly evaluated ($N$ is the number of sites):
\begin{equation}
    E = \frac{U N}{2} (M^2 - M) - \mu NM
    = \frac{N}{2} (U M^2 - (U + 2\mu) M).
\end{equation}
At the ground state, $E$ is minimized.
If $M$ were continuous, we would have
\begin{equation}
    M = \frac{U + 2\mu}{2 U} = \frac{1}{2} + \frac{\mu}{U}, 
    \label{eq:continuous-m}
\end{equation}
but it's not. So we need to find the closest integer to \eqref{eq:continuous-m}.
Note that since
\[
    \frac{1}{2} \leq \frac{1}{2} + \frac{\mu}{U} - \floor*{\frac{\mu}{U}} < \frac{3}{2},
\]
the following $M$ is always a minimum point:
\begin{equation}
    M = \floor*{\frac{\mu}{U}} + 1.
\end{equation}
When $\mu / U$ is an integer, both 
\begin{equation}
    M = \frac{\mu}{U} 
\end{equation}
and 
\begin{equation}
    M = \floor*{\frac{\mu}{U}} + 1 = \frac{\mu}{U} + 1
\end{equation}
can be found in ground states.

The energy gap is 
\begin{equation}
    \begin{aligned}
        \Delta E &= \min( E|_{n_i = M+1 \text{ on one site}} - E|_M, E|_{n_i = M-1 \text{ on one site}} - E|_M ) \\
        &= \min( UM - \mu, U(- M + 1) + \mu ) \\
        &= \begin{cases}
            0 \text{ or } U, \quad &\text{$\mu / U$ integer}, \\
            \min\left( 
                U \floor*{\frac{\mu}{U}} + U - \mu,
                \mu - U \floor*{\frac{\mu}{U}} \right),\quad &\text{otherwise}.
        \end{cases}
    \end{aligned}
\end{equation}
So
\begin{equation}
    \frac{\Delta E}{U} 
    = \begin{cases}
        \frac{\mu}{U} - \floor*{\frac{\mu}{U}} , &\quad \frac{\mu}{U} - \floor*{\frac{\mu}{U}} \leq \frac{1}{2} , \\
        1 + \floor*{\frac{\mu}{U}} - \frac{\mu}{U} , &\quad \frac{\mu}{U} - \floor*{\frac{\mu}{U}} \geq \frac{1}{2}. 
    \end{cases}
\end{equation}
The energy gap and the phase diagram are shown in \prettyref{fig:mott-phase}.

\begin{figure}
    \centering
    \begin{tikzpicture}[x=0.75pt,y=0.75pt,yscale=-1,xscale=1]
    %uncomment if require: \path (0,300); %set diagram left start at 0, and has height of 300
    
    %Shape: Circle [id:dp6029649089269968] 
    \draw  [color={rgb, 255:red, 255; green, 255; blue, 255 }  ,draw opacity=1 ][fill={rgb, 255:red, 255; green, 255; blue, 255 }  ,fill opacity=1 ] (253.77,105.58) .. controls (253.77,103.86) and (255.17,102.46) .. (256.9,102.46) .. controls (258.63,102.46) and (260.02,103.86) .. (260.02,105.58) .. controls (260.02,107.31) and (258.63,108.71) .. (256.9,108.71) .. controls (255.17,108.71) and (253.77,107.31) .. (253.77,105.58) -- cycle ;
    %Shape: Circle [id:dp2540551743017936] 
    \draw  [color={rgb, 255:red, 255; green, 255; blue, 255 }  ,draw opacity=1 ][fill={rgb, 255:red, 255; green, 255; blue, 255 }  ,fill opacity=1 ] (201.48,105.58) .. controls (201.48,103.86) and (202.87,102.46) .. (204.6,102.46) .. controls (206.33,102.46) and (207.73,103.86) .. (207.73,105.58) .. controls (207.73,107.31) and (206.33,108.71) .. (204.6,108.71) .. controls (202.87,108.71) and (201.48,107.31) .. (201.48,105.58) -- cycle ;
    %Shape: Circle [id:dp16263056125963904] 
    \draw  [color={rgb, 255:red, 255; green, 255; blue, 255 }  ,draw opacity=1 ][fill={rgb, 255:red, 255; green, 255; blue, 255 }  ,fill opacity=1 ] (149.17,105.58) .. controls (149.17,103.86) and (150.57,102.46) .. (152.3,102.46) .. controls (154.03,102.46) and (155.42,103.86) .. (155.42,105.58) .. controls (155.42,107.31) and (154.03,108.71) .. (152.3,108.71) .. controls (150.57,108.71) and (149.17,107.31) .. (149.17,105.58) -- cycle ;
    %Straight Lines [id:da4206170110777874] 
    \draw    (100,155) -- (100,58.17) ;
    \draw [shift={(100,56.17)}, rotate = 90] [fill={rgb, 255:red, 0; green, 0; blue, 0 }  ][line width=0.08]  [draw opacity=0] (12,-3) -- (0,0) -- (12,3) -- cycle    ;
    %Straight Lines [id:da3506585102222235] 
    \draw    (100,155) -- (152.3,155) ;
    %Straight Lines [id:da895220782085971] 
    \draw    (152.3,155) -- (204.6,155) ;
    %Straight Lines [id:da8373948832562332] 
    \draw    (204.6,155) -- (256.9,155) ;
    %Straight Lines [id:da27683478112896753] 
    \draw    (256.9,155) -- (309.2,155) ;
    %Straight Lines [id:da32246801352481147] 
    \draw    (309.2,155) -- (359.5,155) ;
    \draw [shift={(361.5,155)}, rotate = 180] [fill={rgb, 255:red, 0; green, 0; blue, 0 }  ][line width=0.08]  [draw opacity=0] (12,-3) -- (0,0) -- (12,3) -- cycle    ;
    %Straight Lines [id:da8662027660822462] 
    \draw    (100,223) -- (152.3,223) ;
    \draw [shift={(152.3,223)}, rotate = 45] [color={rgb, 255:red, 0; green, 0; blue, 0 }  ][line width=0.75]    (-5.59,0) -- (5.59,0)(0,5.59) -- (0,-5.59)   ;
    %Straight Lines [id:da34431304247304895] 
    \draw    (152.3,223) -- (204.6,223) ;
    \draw [shift={(204.6,223)}, rotate = 45] [color={rgb, 255:red, 0; green, 0; blue, 0 }  ][line width=0.75]    (-5.59,0) -- (5.59,0)(0,5.59) -- (0,-5.59)   ;
    %Straight Lines [id:da12932606251036693] 
    \draw    (204.6,223) -- (256.9,223) ;
    \draw [shift={(256.9,223)}, rotate = 45] [color={rgb, 255:red, 0; green, 0; blue, 0 }  ][line width=0.75]    (-5.59,0) -- (5.59,0)(0,5.59) -- (0,-5.59)   ;
    %Straight Lines [id:da7717873736628669] 
    \draw    (256.9,223) -- (309.2,223) ;
    \draw [shift={(309.2,223)}, rotate = 45] [color={rgb, 255:red, 0; green, 0; blue, 0 }  ][line width=0.75]    (-5.59,0) -- (5.59,0)(0,5.59) -- (0,-5.59)   ;
    %Straight Lines [id:da2272617661619305] 
    \draw    (309.2,223) -- (359.5,223) ;
    \draw [shift={(361.5,223)}, rotate = 180] [fill={rgb, 255:red, 0; green, 0; blue, 0 }  ][line width=0.08]  [draw opacity=0] (12,-3) -- (0,0) -- (12,3) -- cycle    ;
    %Straight Lines [id:da5719434187257975] 
    \draw    (100,155.08) -- (126.15,105.5) ;
    %Straight Lines [id:da8364485711287293] 
    \draw    (152.3,155.08) -- (178.45,105.5) ;
    %Straight Lines [id:da26183198244473704] 
    \draw    (204.6,155.08) -- (230.75,105.5) ;
    %Straight Lines [id:da5177088170253232] 
    \draw    (256.9,155.08) -- (283.05,105.5) ;
    %Straight Lines [id:da3086727485831666] 
    \draw    (152.3,155) -- (126.15,105.5) ;
    %Straight Lines [id:da18228756210556907] 
    \draw    (204.6,155.08) -- (178.45,105.58) ;
    %Straight Lines [id:da12272885828721636] 
    \draw    (256.9,155) -- (230.75,105.5) ;
    %Straight Lines [id:da004516495867717518] 
    \draw    (309.2,155) -- (283.05,105.5) ;
    %Straight Lines [id:da4670816501266424] 
    \draw  [dash pattern={on 0.84pt off 2.51pt}]  (100,105.58) -- (332.5,105.58) ;
    %Straight Lines [id:da974860414493175] 
    \draw    (309.2,155) -- (335.35,105.42) ;
    
    % Text Node
    \draw (363.5,155) node [anchor=west] [inner sep=0.75pt]    {$\mu /U$};
    % Text Node
    \draw (100,53.17) node [anchor=south] [inner sep=0.75pt]    {$\Delta E/U$};
    % Text Node
    \draw (363.5,223) node [anchor=west] [inner sep=0.75pt]    {$\mu /U$};
    % Text Node
    \draw (98,105.58) node [anchor=east] [inner sep=0.75pt]   [align=left] {1/2};
    % Text Node
    \draw (100,158) node [anchor=north] [inner sep=0.75pt]   [align=left] {0};
    % Text Node
    \draw (151.3,158) node [anchor=north] [inner sep=0.75pt]   [align=left] {1};
    % Text Node
    \draw (205.6,158) node [anchor=north] [inner sep=0.75pt]   [align=left] {2};
    % Text Node
    \draw (256.9,158) node [anchor=north] [inner sep=0.75pt]   [align=left] {3};
    % Text Node
    \draw (309.2,157.83) node [anchor=north] [inner sep=0.75pt]   [align=left] {4};
    % Text Node
    \draw (102,226) node [anchor=north west][inner sep=0.75pt]    {$M=1$};
    % Text Node
    \draw (154.3,226) node [anchor=north west][inner sep=0.75pt]    {$M=2$};
    % Text Node
    \draw (206.6,226) node [anchor=north west][inner sep=0.75pt]    {$M=3$};
    % Text Node
    \draw (258.9,226) node [anchor=north west][inner sep=0.75pt]    {$M=4$};
    
    
    \end{tikzpicture}
    
    \caption{The energy gap plot and the phase diagram when $t = 0$}
    \label{fig:mott-phase}
\end{figure}

\item[2.] The gapless points in \prettyref{fig:mott-phase} can only be connected to the superfluid phase,
and therefore we get \prettyref{fig:boson-hubbard}.

\begin{figure}
    \centering
    \begin{tikzpicture}[x=0.75pt,y=0.75pt,yscale=-1,xscale=1]
    %uncomment if require: \path (0,300); %set diagram left start at 0, and has height of 300
    
    %Straight Lines [id:da07533110977734148] 
    \draw    (112,245) -- (399.5,245) ;
    \draw [shift={(401.5,245)}, rotate = 180] [fill={rgb, 255:red, 0; green, 0; blue, 0 }  ][line width=0.08]  [draw opacity=0] (12,-3) -- (0,0) -- (12,3) -- cycle    ;
    %Straight Lines [id:da16123173761944054] 
    \draw    (112,245) -- (112,198.21) ;
    %Straight Lines [id:da25470462677223615] 
    \draw    (112,198.21) -- (112,151.42) ;
    %Straight Lines [id:da9203516859237628] 
    \draw    (112,151.42) -- (112,104.62) ;
    %Straight Lines [id:da8858272174506854] 
    \draw    (112,104.62) -- (112,59.83) ;
    \draw [shift={(112,57.83)}, rotate = 90] [fill={rgb, 255:red, 0; green, 0; blue, 0 }  ][line width=0.08]  [draw opacity=0] (12,-3) -- (0,0) -- (12,3) -- cycle    ;
    %Curve Lines [id:da690865334975969] 
    \draw    (112,245) .. controls (351.5,235.5) and (347.5,199.5) .. (112,198.21) ;
    %Curve Lines [id:da49915852555779927] 
    \draw    (112,198.21) .. controls (320.5,191.5) and (280.5,151.5) .. (112,151.42) ;
    %Curve Lines [id:da3871736130183383] 
    \draw    (112,151.42) .. controls (243.5,149.5) and (253.5,109.5) .. (112,104.62) ;
    %Curve Lines [id:da8267537682584203] 
    \draw    (112,104.62) .. controls (230.5,86.17) and (156.5,69.17) .. (145.5,66.17) ;
    
    % Text Node
    \draw (403.5,245) node [anchor=west] [inner sep=0.75pt]    {$t/U$};
    % Text Node
    \draw (112,54.83) node [anchor=south] [inner sep=0.75pt]    {$\mu /U$};
    % Text Node
    \draw (114,221.6) node [anchor=west] [inner sep=0.75pt]    {$M=1\ \text{Mott}$};
    % Text Node
    \draw (114,174.81) node [anchor=west] [inner sep=0.75pt]    {$M=2\ \text{Mott}$};
    % Text Node
    \draw (114,128.02) node [anchor=west] [inner sep=0.75pt]    {$M=3\ \text{Mott}$};
    % Text Node
    \draw (292,84) node [anchor=north west][inner sep=0.75pt]   [align=left] {superfluid};
    
    
    \end{tikzpicture}
    
    \caption{Schematic phase diagram of the boson Hubbard model}
    \label{fig:boson-hubbard}
\end{figure}

\item[3.] We have 
\begin{equation}
    \mel{n_0 + k'}{a}{n_0 + k} = \sqrt{n_0 + k} \braket{n_0 + k'}{n_0 + k - 1} 
    = \sqrt{n_0 + k} \delta_{k', k-1},
\end{equation}
and 
\begin{equation}
    \mel{k'}{\ee^{- \ii \theta}}{k} 
    = \int_{0}^{2\pi} \frac{\dd{\theta}}{2\pi} \ee^{- \ii k' \theta} \ee^{- \ii \theta} \ee^{\ii k \theta}
    = \delta_{k', k - 1}.
\end{equation}
When $k \ll n_0$, we have 
\begin{equation}
    \mel{n_0 + k'}{a}{n_0 + k} \approx \sqrt{n_0} \mel{k'}{\ee^{- \ii \theta}}{k} \Rightarrow
    a \approx \sqrt{n_0} \ee^{- \ii \theta}.
\end{equation}
And similarly we have 
\begin{equation}
    \mel{n_0 + k'}{a^\dagger}{n_0 + k} = \sqrt{n_0 + k + 1} \delta_{k', k+1},
\end{equation}
and 
\begin{equation}
    \mel{k'}{\ee^{\ii \theta}}{k} 
    = \int_0^{2\pi} \frac{\dd{\theta}}{2\pi} \ee^{- \ii k \theta} \ee^{\ii \theta} \ee^{\ii k' \theta}
    = \delta_{k', k+1},
\end{equation}
and in the $k \ll n_0$ limit we have 
\begin{equation}
    a^{\dagger} \approx \sqrt{n_0} \ee^{\ii \theta}.
\end{equation}
Also, 
\begin{equation}
    \mel{n_0 + k'}{n}{n_0 + k} = (n_0 + k) \braket{n_0 + k}{n_0 + k'} = (n_0 + k) \delta_{k k'},
\end{equation}
and 
\begin{equation}
    \mel{k'}{\pi}{k} = \int_0^{2\pi} \dd{\theta} \ee^{- \ii k' \theta} (-\ii \partial_\theta) \ee^{\ii k \theta}
    = k \delta_{k k'},
\end{equation}
so 
\begin{equation}
    n = n_0 + \pi.
\end{equation}

\item[4.] When the above approximation works, 
\[
    \begin{aligned}
        H &= - t n_0 \sum_{\pair{i, j}} \ee^{\ii \theta_i} \ee^{- \ii \theta_j} + \text{h.c.}
        + \frac{U}{2} \sum_i (n_0 + \pi_i) (n_0 + \pi_i - 1) - \mu \sum_i (n_0 + \pi_i) \\
        &= - 2 t n_0 \sum_{\pair{i, j}} \cos(\theta_i - \theta_j) 
        + \frac{U}{2} \sum_i \pi_i^2 + \text{linear terms in $\pi$} + f(n_0).
    \end{aligned}
\]
The linear terms of $\pi$ can be safely ignored 
because we are working around $n = n_0$ that minimize the energy,
and $\pdv{E}{n} = 0$ at $n = n_0$.
Since $\pi$ is the fluctuation of $n$,
linear terms in $\pi$ correspond to first order Taylor terms and are also bound to be zero.
Indeed, this condition connects $n_0$ and $\mu$:
\begin{equation}
    \frac{U}{2} (2 n_0 \pi_i - \pi_i) - \mu \pi_0 = 0 \Rightarrow
    n_0 = \frac{1}{2} + \frac{\mu}{U}.
    \label{eq:n0-mu}
\end{equation}
So 
\begin{equation}
    H = - J \sum_{\pair{i, j}} \cos(\theta_i - \theta_j) 
    + u \sum_i \pi_i^2, \quad J = 2n_0 t, \quad u = \frac{U}{2}.
\end{equation}

\item[5.] In the limit of slow spatial varying,
we have 
\begin{equation}
    H = -J \sum_{\pair{i, j}} \left(1 - \frac{1}{2} (\theta_i - \theta_j)^2 \right) + u \sum_i \pi_i^2. 
\end{equation}
This Hamiltonian has exactly the same form of the lattice phonon Hamiltonian.
The operator EOMs are 
\begin{equation}
    \begin{aligned}
        \dv{\pi_i}{t} &= \frac{1}{\ii} \comm{\pi_i}{H} = 
        - J \sum_{\pair{i, j}} (\theta_i - \theta_j) , \\
        \dv{\theta_i}{t} &= \frac{1}{\ii} \comm{\theta_i}{H} = 2 u \pi_i,
    \end{aligned}
\end{equation}
and therefore 
\begin{equation}
    \dv[2]{\theta_i}{t} = 2 J u \sum_{\pair{i, j}} (\theta_j - \theta_i).
\end{equation}
Suppose the bond length of the lattice is $a$.
In a normal mode
\begin{equation}
    \theta_i \propto \ee^{\ii (\vb*{k} \cdot \vb*{r} - \omega_{\vb*{k}} t)},
\end{equation}
we have 
\[
    \begin{aligned}
        - \omega_{\vb*{k}}^2 &= 2 Ju \sum_{i=1}^d (
            \ee^{\ii \vb*{k} \cdot a \vu*{x}_i} + \ee^{- \ii \vb*{k} \cdot a \vu*{x}_i} - 2
        ) \\
        &= 2 J u\sum_{i=1}^d (2 \cos (\vb*{k} \cdot a \vu*{x}_i) - 2) \\
        &\approx - 2 J u \sum_{i=1}^d (\vb*{k} \cdot a \vu*{x}_i)^2 \\
        &= - 2 J u a^2 \vb*{k}^2,
    \end{aligned}
\]
so
\begin{equation}
    \omega_{\vb*{k}} = \sqrt{2 J u} a \abs{\vb*{k}}.
\end{equation}

\item[6.] When $J / u = 0$, we have
\begin{equation}
    H = \sum_i (u \pi_i^2 - B \pi_i).
\end{equation}
Again we apply the same procedure used to derive \prettyref{fig:boson-hubbard}.
In the ground state, we have 
\begin{equation}
    \pi_i = \begin{cases}
        \floor*{\frac{B}{2u}}, \quad & 0 \leq \frac{B}{2u} - \floor*{\frac{B}{2u}} \leq \frac{1}{2}, \\
        \floor*{\frac{B}{2u}} + 1, \quad & \frac{1}{2} \leq \frac{B}{2u} - \floor*{\frac{B}{2u}} < 1, \\
    \end{cases}
\end{equation}
and when 
\begin{equation}
    \frac{B}{2u} - \floor*{\frac{B}{2u}} = \frac{1}{2},
    \label{eq:gapless-cond-2}
\end{equation}
changing $\pi_i$ on one site doesn't change the energy,
and we get a gapless system.
So on the phase diagram, points defined by \eqref{eq:gapless-cond-2} are connected to the superfluid phase,
and the phase diagram of 
\begin{equation}
    H = - J \sum_{\pair{i, j}} \cos(\theta_i - \theta_j) 
    + u \sum_i \pi_i^2 - B \sum_i \pi_i , 
    \label{eq:eft-with-b}
\end{equation}
is given in \prettyref{fig:phase-eft-b}.

\begin{figure}
    \centering
    \begin{tikzpicture}[x=0.75pt,y=0.75pt,yscale=-1,xscale=1]
    %uncomment if require: \path (0,300); %set diagram left start at 0, and has height of 300
    
    %Straight Lines [id:da8276792481381465] 
    \draw    (132,166.42) -- (419.5,166.42) ;
    \draw [shift={(421.5,166.42)}, rotate = 180] [fill={rgb, 255:red, 0; green, 0; blue, 0 }  ][line width=0.08]  [draw opacity=0] (12,-3) -- (0,0) -- (12,3) -- cycle    ;
    %Straight Lines [id:da3789296587395721] 
    \draw    (132,270.83) -- (132,213.21) ;
    %Straight Lines [id:da39064345829352454] 
    \draw    (132,213.21) -- (132,166.42) ;
    %Straight Lines [id:da9582593367972863] 
    \draw    (132,166.42) -- (132,119.62) ;
    %Straight Lines [id:da6187299560771233] 
    \draw    (132,119.62) -- (132,72.83) ;
    %Curve Lines [id:da2247461850664918] 
    \draw    (132,189.81) .. controls (263.5,187.9) and (258.5,146.5) .. (132,143.02) ;
    %Curve Lines [id:da49099872678475487] 
    \draw    (132,143.02) .. controls (233.5,129.5) and (233.5,105.9) .. (132,96.23) ;
    %Straight Lines [id:da10020596191876074] 
    \draw    (132,72.83) -- (132,49.83) ;
    \draw [shift={(132,47.83)}, rotate = 90] [fill={rgb, 255:red, 0; green, 0; blue, 0 }  ][line width=0.08]  [draw opacity=0] (12,-3) -- (0,0) -- (12,3) -- cycle    ;
    %Curve Lines [id:da7773919720341762] 
    \draw    (132,236.6) .. controls (235.5,228.5) and (233.5,199.48) .. (132,189.81) ;
    %Curve Lines [id:da27110231454944866] 
    \draw    (189.5,264.08) .. controls (193.5,263.08) and (209.5,251.08) .. (132,236.6) ;
    %Straight Lines [id:da4792276182131656] 
    \draw    (208,118) ;
    \draw [shift={(208,118)}, rotate = 45] [color={rgb, 255:red, 0; green, 0; blue, 0 }  ][line width=0.75]    (-5.59,0) -- (5.59,0)(0,5.59) -- (0,-5.59)   ;
    %Straight Lines [id:da7026766645440017] 
    \draw    (229,166) ;
    \draw [shift={(229,166)}, rotate = 45] [color={rgb, 255:red, 0; green, 0; blue, 0 }  ][line width=0.75]    (-5.59,0) -- (5.59,0)(0,5.59) -- (0,-5.59)   ;
    %Straight Lines [id:da3147496528253195] 
    \draw    (208,214) ;
    \draw [shift={(208,214)}, rotate = 45] [color={rgb, 255:red, 0; green, 0; blue, 0 }  ][line width=0.75]    (-5.59,0) -- (5.59,0)(0,5.59) -- (0,-5.59)   ;
    
    % Text Node
    \draw (423.5,166.42) node [anchor=west] [inner sep=0.75pt]    {$J/u$};
    % Text Node
    \draw (132,44.83) node [anchor=south] [inner sep=0.75pt]    {$B/u$};
    % Text Node
    \draw (312,99) node [anchor=north west][inner sep=0.75pt]   [align=left] {superfluid};
    % Text Node
    \draw (140,111) node [anchor=north west][inner sep=0.75pt]   [align=left] {Mott};
    % Text Node
    \draw (142,161) node [anchor=north west][inner sep=0.75pt]   [align=left] {Mott};
    % Text Node
    \draw (144,201) node [anchor=north west][inner sep=0.75pt]   [align=left] {Mott};
    % Text Node
    \draw (130,143.02) node [anchor=east] [inner sep=0.75pt]   [align=left] {1/2};
    % Text Node
    \draw (130,96.23) node [anchor=east] [inner sep=0.75pt]   [align=left] {3/2};
    % Text Node
    \draw (130,189.81) node [anchor=east] [inner sep=0.75pt]   [align=left] {\mbox{-}1/2};
    % Text Node
    \draw (130,236.6) node [anchor=east] [inner sep=0.75pt]   [align=left] {\mbox{-}3/2};
    
    
    \end{tikzpicture}
    
    \caption{Phase diagram of \eqref{eq:eft-with-b}}
    \label{fig:phase-eft-b}
\end{figure}

\item[7.] We have 
\begin{equation}
    \dot{\theta}_i = \pdv{H}{\pi_i} = 2 u \pi_i - B, \quad 
    \pi_i = \frac{\dot{\theta}_i + B}{2u}.
\end{equation}
So
\begin{equation}
    \begin{aligned}
        L &= \sum_i \dot{\theta}_i \pi_i - H \\
        &= \sum_i \dot{\theta}_i \frac{\dot{\theta}_i + B}{2u} 
        + J \sum_{\pair{i, j}} \cos(\theta_i - \theta_j) 
        - u \sum_i \left( \frac{\dot{\theta}_i + B}{2u} \right)^2
        + B \sum_i \frac{\dot{\theta}_i + B}{2u} \\
        &= \frac{1}{4u} \sum_i (\dot{\theta}_i + B)^2 + J \sum_{\pair{i, j}} \cos(\theta_i - \theta_j) .
    \end{aligned}
\end{equation}
In the continuous limit, we have 
\begin{equation}
    \sum_{\pair{i, j}} \cos(\theta_i - \theta_j) 
    \to \sum_{\pair{i, j}} \left( 1 - \frac{1}{2} (\theta_i - \theta_j)^2 \right) 
    \simeq - \sum_{\pair{i, j}} \frac{1}{2} (\theta_i - \theta_j)^2 = - \frac{1}{2} a^2 (\grad{\theta})^2,
\end{equation}
and 
\begin{equation}
    \begin{aligned}
        L &= \int \frac{\dd[d]{\vb*{x}}}{a^d} \frac{1}{4 u} (\dot{\theta} + B)^2 
        - \int \frac{\dd[d]{\vb*{x}}}{a^d} \frac{J}{2} a^2 (\grad{\theta})^2  \\
        &= \int \dd[d]{\vb*{x}} \left(
            \frac{1}{4 u a^d} (\dot{\theta} + B)^2
            - \frac{J}{2} a^{d-2} (\grad{\theta})^2
        \right).
    \end{aligned}
    \label{eq:continuous-eft}
\end{equation}

\item[8.] In the imaginary time path integral,
\begin{equation}
    \ii \int \dd{t} \frac{B}{2u} \dot{\theta}_i \longrightarrow 
    \ii \int_0^{\beta} \dd{\tau} \frac{B}{2u} \partial_\tau \theta 
    = \frac{B}{2u} \ii (\theta_i(\beta) - \theta_i(0)).
    \label{eq:theta-dot-term}
\end{equation}
Since in the imaginary time integral,
the final state and the initial state have to be the same,
$\theta_i(\beta)$ and $\theta_i(0)$ have to be equivalent to each other,
and therefore 
\begin{equation}
    \theta_i(\beta) - \theta_i(0) = 2\pi n, \quad n \in \mathbb{Z}.
\end{equation}
So when $B / 2u$ is an integer, \eqref{eq:theta-dot-term} contributes nothing to the partition function.
The points are plotted on \prettyref{fig:phase-eft-b}.
Comparing the positions of these points on \prettyref{fig:phase-eft-b} and
on the original model \prettyref{fig:boson-hubbard},
we find the condition that the $\partial_\tau \theta$ term in \eqref{eq:continuous-eft} can be ignored 
is equivalent to 
\[
    \frac{\mu}{U} = \frac{1}{2} + n, \quad n = 0, 1, 2, \ldots,
\]
and from \eqref{eq:n0-mu} we know this is equivalent to the condition that $n_0$ is an integer.

\end{itemize}

\bibliographystyle{plain}
\bibliography{data}

\end{document}