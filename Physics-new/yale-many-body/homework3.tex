\documentclass[hyperref, a4paper]{article}

\usepackage{geometry}
\usepackage{titling}
\usepackage{titlesec}
% No longer needed, since we will use enumitem package
% \usepackage{paralist}
\usepackage{enumitem}
\usepackage{footnote}
%\usepackage{enumerate}
\usepackage{amsmath, amssymb, amsthm}
\usepackage{mathtools}
\usepackage{bbm}
\usepackage{cite}
\usepackage{graphicx}
\usepackage{subfigure}
\usepackage{physics}
\usepackage{tensor}
\usepackage{siunitx}
\usepackage[version=4]{mhchem}
\usepackage{tikz}
\usepackage{xcolor}
\usepackage{listings}
\usepackage{autobreak}
\usepackage[ruled, vlined, linesnumbered]{algorithm2e}
\usepackage{nameref,zref-xr}
\zxrsetup{toltxlabel}
\usepackage[colorlinks,unicode]{hyperref} % , linkcolor=black, anchorcolor=black, citecolor=black, urlcolor=black, filecolor=black
\usepackage[most]{tcolorbox}
\usepackage{prettyref}

% Page style
\geometry{left=3.18cm,right=3.18cm,top=2.54cm,bottom=2.54cm}
\titlespacing{\paragraph}{0pt}{1pt}{10pt}[20pt]
\setlength{\droptitle}{-5em}
%\preauthor{\vspace{-10pt}\begin{center}}
%\postauthor{\par\end{center}}

% More compact lists 
\setlist[itemize]{
    itemindent=17pt, 
    leftmargin=1pt,
    listparindent=\parindent,
    parsep=0pt,
}

% Math operators
\DeclareMathOperator{\timeorder}{\mathcal{T}}
\DeclareMathOperator{\diag}{diag}
\DeclareMathOperator{\legpoly}{P}
\DeclareMathOperator{\primevalue}{P}
\DeclareMathOperator{\sgn}{sgn}
\newcommand*{\ii}{\mathrm{i}}
\newcommand*{\ee}{\mathrm{e}}
\newcommand*{\const}{\mathrm{const}}
\newcommand*{\suchthat}{\quad \text{s.t.} \quad}
\newcommand*{\argmin}{\arg\min}
\newcommand*{\argmax}{\arg\max}
\newcommand*{\normalorder}[1]{: #1 :}
\newcommand*{\pair}[1]{\langle #1 \rangle}
\newcommand*{\fd}[1]{\mathcal{D} #1}
\DeclareMathOperator{\bigO}{\mathcal{O}}

% TikZ setting
\usetikzlibrary{arrows,shapes,positioning}
\usetikzlibrary{arrows.meta}
\usetikzlibrary{decorations.markings}
\tikzstyle arrowstyle=[scale=1]
\tikzstyle directed=[postaction={decorate,decoration={markings,
    mark=at position .5 with {\arrow[arrowstyle]{stealth}}}}]
\tikzstyle ray=[directed, thick]
\tikzstyle dot=[anchor=base,fill,circle,inner sep=1pt]

% Algorithm setting
% Julia-style code
\SetKwIF{If}{ElseIf}{Else}{if}{}{elseif}{else}{end}
\SetKwFor{For}{for}{}{end}
\SetKwFor{While}{while}{}{end}
\SetKwProg{Function}{function}{}{end}
\SetArgSty{textnormal}

\newcommand*{\concept}[1]{{\textbf{#1}}}

% Embedded codes
\lstset{basicstyle=\ttfamily,
  showstringspaces=false,
  commentstyle=\color{gray},
  keywordstyle=\color{blue}
}

% Reference formatting
\newrefformat{fig}{Figure~\ref{#1}}

% Color boxes
\tcbuselibrary{skins, breakable, theorems}
\newtcbtheorem[number within=section]{warning}{Warning}%
  {colback=orange!5,colframe=orange!65,fonttitle=\bfseries, breakable}{warn}
\newtcbtheorem[number within=section]{note}{Note}%
  {colback=green!5,colframe=green!65,fonttitle=\bfseries, breakable}{note}
\newtcbtheorem[number within=section]{info}{Info}%
  {colback=blue!5,colframe=blue!65,fonttitle=\bfseries, breakable}{info}

\newenvironment{shelldisplay}{\begin{lstlisting}}{\end{lstlisting}}

\title{Homework 3}
\author{Jinyuan Wu}

\begin{document}

\maketitle

\paragraph{Problem 2} 

\paragraph{Solution} \begin{itemize}
\item[(a)] Repeating the procedure used in ordinary superfluid, we do the decomposition
\begin{equation}
    \varphi = \sqrt{\rho} \ee^{\ii \theta} = \sqrt{\rho_0 + \var{\rho}} \ee^{\ii \theta},
\end{equation}
and therefore 
\begin{equation}
    - \frac{\varphi^* \laplacian \varphi}{2m} = 
    \frac{\rho}{2m} (\grad{\theta})^2 + \frac{(\grad{\rho})^2}{8 \rho m} ,
\end{equation}
\begin{equation}
    \varphi^* \partial_\tau \varphi 
    = \underbrace{\frac{1}{2} \partial_\tau \rho}_{\text{time derivative, ignored}} + 
    \ii \rho \partial_\tau \theta,
\end{equation}
\begin{equation}
    \abs{\varphi(\vb*{x})} U(\vb*{x} - \vb*{y}) \abs{\varphi(\vb*{y})} = 
    \rho(\vb*{x}) U(\vb*{x} - \vb*{y}) \rho(\vb*{y}),
\end{equation}
the theory is now 
\begin{equation}
    S = \int \dd{\tau} \left(
        \int \dd[d]{\vb*{x}} \left(
            \ii \rho \partial_\tau \theta 
            + \frac{\rho}{2m} (\grad{\theta})^2 + \frac{(\grad{\rho})^2}{8 \rho m}
            - \mu \rho
        \right) + 
        \frac{1}{2} \int \dd[d]{\vb*{x}} \int \dd[d]{\vb*{y}}
        \rho(\vb*{x}) U(\vb*{x} - \vb*{y}) \rho(\vb*{y})
    \right).
\end{equation}
Around the ground state, we have 
(note that since we are around a saddle point,
the sum of all terms containing $\var{\rho}$ only is always zero;
the resulting theory has the form of $c_1 \var{\rho} \partial_\tau \theta + c_2 \var{\rho}^2$;
the chemical potential term is therefore missing in the theory around the saddle point)
\[
    \ii \rho \partial_\tau \theta = \underbrace{\ii \rho_0 \partial_\tau \theta}_{\text{time derivative}} 
    + \ii \var{\rho} \partial_\tau \theta,
\]
and since $\grad{\rho} = \grad{\var{\rho}}$,
we have 
\[
    \frac{(\grad{\rho})^2}{8 \rho m} \approx \frac{(\grad{\var{\rho}})^2}{8 \rho_0 m},
\]
ignoring the fluctuation of the $\rho$ in the denominator.
Similarly, since we are working on a low energy theory,
the fluctuation of $\theta$ shouldn't be large,
and we have 
\[
    \frac{\rho}{2m} (\grad{\theta})^2 \approx \frac{\rho_0}{2m} (\grad{\theta})^2.
\]
The theory is then 
\begin{equation}
    S = \int \dd[d+1]{x} \left(
        \frac{\rho_0}{2m} (\grad{\theta})^2 
        + \ii \var{\rho} \partial_\tau \theta  
        + \frac{(\grad{\var{\rho}})^2}{8 \rho_0 m} 
        + \frac{1}{2} \var{\rho}(\vb*{x}) \int \dd[d]{\vb*{y}} U(\vb*{x} - \vb*{y}) \var{\rho}(\vb*{y})
    \right) + S_{\text{saddle}}.
\end{equation}
Integrating out $\var{\rho}$, we get 
\begin{equation}
    \begin{aligned}
        S_{\text{eff}} &= \int \dd[d+1]{x} \frac{\rho_0}{2m} (\grad{\theta})^2 
        - \frac{1}{2} \int \dd{\tau} \int \dd[d]{\vb*{x}} \dd[d]{\vb*{y}} 
        \ii \partial_\tau \theta(\vb*{x}, \tau)
            \frac{1}{
                \int \dd[d]{\vb*{y}} U(\vb*{x} - \vb*{y}) - \frac{1}{4 \rho_0 m} \laplacian
            } \ii \partial_\tau \theta (\vb*{y}, \tau) \\
        &= \int \dd[d+1]{x} \frac{\rho_0}{2m} (\grad{\theta})^2 + 
        \frac{1}{2} \int \dd{\tau} \int \dd[d]{\vb*{x}} \dd[d]{\vb*{y}}
        \partial_\tau \theta(\vb*{x}, \tau) G(\vb*{x} - \vb*{y}) \partial_\tau \theta(\vb*{y}),
    \end{aligned}
\end{equation}
where 
\begin{equation}
    \int \dd[d]{\vb*{y}} U(\vb*{x} - \vb*{y}) G(\vb*{y} - \vb*{z}) 
    - \frac{1}{4 \rho_0 m} \laplacian_{\vb*{x}} G(\vb*{x} - \vb*{z})  = \delta(\vb*{x} - \vb*{z}).
\end{equation}
Similar to the procedure in dealing with ordinary superfluid,
since we are only interested in the long wave length behaviors of $\theta$,
the $\laplacian$ term can be thrown away,
and we have 
\[
    \begin{aligned}
        \int \frac{\dd[d]{\vb*{p}}}{(2\pi)^d} \ee^{\ii \vb*{p} \cdot (\vb*{x} - \vb*{z})}
        &= \int \frac{\dd[d]{\vb*{p}}}{(2\pi)^d} \int \dd[d]{\vb*{y}} U(\vb*{x} - \vb*{y}) 
        G(\vb*{p}) \ee^{\ii \vb*{p} \cdot (\vb*{y} - \vb*{z})} \\
        &= \int \frac{\dd[d]{\vb*{p}}}{(2\pi)^d} 
        \int \dd[d]{\vb*{r}} U(\vb*{r}) \ee^{- \ii \vb*{p} \cdot \vb*{r}} 
        G(\vb*{p}) \ee^{\ii \vb*{p} \cdot (\vb*{x} - \vb*{z})} \quad (\vb*{r} = \vb*{x} - \vb*{y}) ,
    \end{aligned}
\]
so 
\begin{equation}
    G(\vb*{r}) = \int \frac{\dd[d]{\vb*{p}}}{(2\pi)^d} \ee^{\ii \vb*{p} \cdot \vb*{r}} G(\vb*{p}), \quad 
    G(\vb*{p}) = \frac{1}{U(\vb*{p})} 
    = \frac{1}{\int \dd[d]{\vb*{r}} U(\vb*{r}) \ee^{- \ii \vb*{p} \cdot \vb*{r}}}.
\end{equation}
To evaluate $G(\vb*{p})$, we need to find 
\begin{equation}
    U(\vb*{p}) = \int_0^\infty \dd{r} \frac{2\pi^{d/2}}{\Gamma(d / 2)} r^{d-1} \frac{U_0}{r^{d - \epsilon}}
\end{equation}

\end{itemize}

\paragraph{Problem 3}

\paragraph{Solution} \begin{itemize}
\item[(a)] The energy now can be exactly evaluated ($N$ is the number of sites):
\begin{equation}
    E = \frac{U N}{2} (M^2 - M) - \mu NM
    = \frac{N}{2} (U M^2 - (U + 2\mu) M).
\end{equation}
At the ground state, $E$ is minimized, and we have
\begin{equation}
    M = \frac{U + 2\mu}{2 U}, 
\end{equation}
and the energy gap is 
\begin{equation}
    \Delta E = E|_{M+1} - E|_M = \frac{N}{2} (2 U M - 2 \mu)
    = \frac{1}{2} NU.
\end{equation}

\end{itemize}

\end{document}