\documentclass[hyperref, a4paper]{article}

\usepackage{geometry}
\usepackage{titling}
\usepackage{titlesec}
% No longer needed, since we will use enumitem package
% \usepackage{paralist}
\usepackage{enumitem}
\usepackage{footnote}
\usepackage{marginnote}
\usepackage{enumerate}
\usepackage{amsmath, amssymb, amsthm}
\usepackage{mathtools}
\usepackage{bbm}
\usepackage{cite}
\usepackage{graphicx}
\usepackage{subfigure}
\usepackage{physics}
\usepackage{tensor}
\usepackage{siunitx}
\usepackage[version=4]{mhchem}
\usepackage{tikz}
\usepackage{xcolor}
\usepackage{listings}
\usepackage{autobreak}
\usepackage[ruled, vlined, linesnumbered]{algorithm2e}
\usepackage{nameref,zref-xr}
\zxrsetup{toltxlabel}
\zexternaldocument*[optics-]{../optics/optics}[optics.pdf]
\zexternaldocument*[solid-]{../solid/solid}[solid.pdf]
\zexternaldocument*[oldfluid-]{fluid}[fluid.pdf]
\zexternaldocument*[kt-]{../topological-phases-reading-notes/kt}[kt.pdf]
\usepackage[colorlinks,unicode]{hyperref} % , linkcolor=black, anchorcolor=black, citecolor=black, urlcolor=black, filecolor=black
\usepackage[most]{tcolorbox}
\usepackage{prettyref}

% Page style
\geometry{left=3.18cm,right=3.18cm,top=2.54cm,bottom=2.54cm}
\titlespacing{\paragraph}{0pt}{1pt}{10pt}[20pt]
\setlength{\droptitle}{-5em}
\preauthor{\vspace{-10pt}\begin{center}}
\postauthor{\par\end{center}}

% More compact lists 
%\setlist[itemize]{
%    itemindent=17pt, 
%    leftmargin=1pt,
%    listparindent=\parindent,
%    parsep=0pt,
%}

% Math operators
\DeclareMathOperator{\timeorder}{\mathcal{T}}
\DeclareMathOperator{\diag}{diag}
\DeclareMathOperator{\legpoly}{P}
\DeclareMathOperator{\primevalue}{P}
\DeclareMathOperator{\sgn}{sgn}
\newcommand*{\ii}{\mathrm{i}}
\newcommand*{\ee}{\mathrm{e}}
\newcommand*{\const}{\mathrm{const}}
\newcommand*{\suchthat}{\quad \text{s.t.} \quad}
\newcommand*{\argmin}{\arg\min}
\newcommand*{\argmax}{\arg\max}
\newcommand*{\normalorder}[1]{: #1 :}
\newcommand*{\pair}[1]{\langle #1 \rangle}
\newcommand*{\fd}[1]{\mathcal{D} #1}
\DeclareMathOperator{\bigO}{\mathcal{O}}

% TikZ setting
\usetikzlibrary{arrows,shapes,positioning}
\usetikzlibrary{arrows.meta}
\usetikzlibrary{decorations.markings}
\tikzstyle arrowstyle=[scale=1]
\tikzstyle directed=[postaction={decorate,decoration={markings,
    mark=at position .5 with {\arrow[arrowstyle]{stealth}}}}]
\tikzstyle ray=[directed, thick]
\tikzstyle dot=[anchor=base,fill,circle,inner sep=1pt]

% Algorithm setting
% Julia-style code
\SetKwIF{If}{ElseIf}{Else}{if}{}{elseif}{else}{end}
\SetKwFor{For}{for}{}{end}
\SetKwFor{While}{while}{}{end}
\SetKwProg{Function}{function}{}{end}
\SetArgSty{textnormal}

\newcommand*{\concept}[1]{{\textbf{#1}}}

% Support for tensor double arrows.
\renewcommand{\tensor}[1]{ \stackrel{\leftrightarrow}{\vb*{#1}}}

% Embedded codes
\lstset{basicstyle=\ttfamily,
  showstringspaces=false,
  commentstyle=\color{gray},
  keywordstyle=\color{blue}
}

% Reference formatting
\newrefformat{fig}{Figure~\ref{#1} on page~\pageref{#1}}

% Color boxes
\tcbuselibrary{skins, breakable, theorems}
\newtcbtheorem[number within=section]{warning}{Warning}%
  {colback=orange!5,colframe=orange!65,fonttitle=\bfseries, breakable}{warn}
\newtcbtheorem[number within=section]{note}{Note}%
  {colback=green!5,colframe=green!65,fonttitle=\bfseries, breakable}{note}

% Correctly displaying equation number in section titles
\pdfstringdefDisableCommands{\def\eqref#1{(\ref{#1})}}

\newcommand{\oldfluid}{\href{fluid.pdf}{this note}}
\newcommand{\ktnote}{\href{../topological-phases-reading-notes/kt.pdf}{this note}}

\title{Superfluid and Vortices}
\author{Jinyuan Wu}

\begin{document}

\maketitle

This article is a reading note of Wen's famous textbook \cite{Wen2007}. 
It is mainly a reconstruction of material related to the KT phase transition covered in Chapter~3 . 

\section{A simplest interacting boson system}

Eq. (3.3.1) in Section~3.3.1 provides a simplest interacting bosonic system with a complex scalar field \marginnote{Section~3.3.1}
\begin{equation}
    S = \int \dd[d]{\vb*{x}} \dd{t} (\ii \varphi^* \partial_t \varphi - \frac{1}{2m} \partial_{\vb*{x}} \varphi^* \partial_{\vb*{x}} \varphi + \mu \abs*{\varphi}^2 - \frac{V_0}{2} \abs*{\varphi}^4).
    \label{eq:bosonic}
\end{equation}
\marginnote{Eq. (3.3.1)}
The prefactor of the interaction term makes the corresponding term in the EOM of $\varphi$ and $\varphi^*$ not 
have a numerical factor, but it introduces a numerical factor in the vertex in Feynman diagrams. The sign of 
the mass term is derived as follows: first we have a $- \varphi^* \laplacian \varphi / 2m$ term in the 
Hamiltonian, and therefore we have a $\varphi^* \laplacian \varphi / 2m$ term in the Lagrangian, and by 
integration by parts we have $\varphi^* \laplacian \varphi / 2m \simeq - \partial_{\vb*{x}} \varphi^* 
\partial_{\vb*{x}} \varphi / 2m$. 

The semiclassical approximation from Eq. (3.3.1) to Eq. (3.3.2) can be justified when the temperature is high 
and therefore the most economical path does not have imaginary time evolution at all. It can also 
be derived using the ideas behind Eq. (3.4.1), where with a finite temperature, we can always integrate 
out modes with non-zero Matsubara frequencies. This gives a physical picture behind dynamic density 
functional theories and also explains why ``classical'' statistical physics is still relevant today.
It should be noted that when the temperature is low, parameters in the ``classical'' theory obtained 
by integrating out non-zero Matsubara frequencies are different with the truly classical approximation,
though they have exactly the same form. The fact can also be seen in the comparison between classical DFT 
and DFT in the context of hard condensed matter physics. 

This, actually, demonstrates the fact that the \emph{interpretation} of what ``quantum'' actually means 
is still confusing. In standard textbooks we learned that ``a quantum theory uses operators as basic 
degrees of freedom, while a classical theory do not'' and ``the Lagrangian of a quantum theory is to 
be placed into a partition function, while we just need to minimize the Lagrangian in a classical theory''.
But no one has ever \emph{seen} an operator or the quasi-probability distribution defined in path integral.
The idea to integrate out non-zero Matsubara frequencies to modify the parameters in the classical theory 
gives us a strange sense, that somehow the quantum effect is really just certain additional fluctuation
added to a classical theory. 

Note that we can only integrate out non-zero Matsubara frequencies when the temperature is not too low. 
When $T=0$, we need to integrate out \emph{uncountably} infinite degrees of freedom, 
which may have some subtlety. This gives rise to some uniquely zero-temperature quantum phenomena. 

The following contents from Eq.~(3.3.3) to Eq.~(3.3.4) are also covered in \oldfluid. \marginnote{Section~3.3.2}
The discussion between Eq. (3.3.4) to the end of Section~3.3.2 is important, which illustrates the 
Ginzburg-Landau paradigm and why it is almost always associated with symmetries (or otherwise it is 
highly unlikely that we have several minima of the energy functional that share the same energy, so 
that we have a smooth phase transition shown in Fig.~3.5), though the concept of order parameters 
can also be used in a first-order phase transition (see \href{../quasicrystal-fd/main.pdf}{here}, 
for example).  

\section{Quantum XY model from \eqref{eq:bosonic}}\label{sec:quantum-xy}

Note that there is no BEC when $\mu < 0$. \marginnote{Section~3.3.3} 
This is often justified by the argument that the 
Bose-Einstein distribution function is not well-defined when $\epsilon < \mu$, and if $\mu > 0$,
there must be a finite number of particles condensed on the ground state. The fact is also shown 
in the discussion around \eqref{oldfluid-eq:bec-ground} in \oldfluid.
The derivation of (3.3.10) is also done in \eqref{oldfluid-eq:quantum-xy} in \oldfluid, in the form 
of an imaginary time field theory. 

We call (3.3.10) a quantum \emph{XY} model. It is easy to see that its classical approximation is 
exactly the coarse-grained version of a classical XY model, and since $\theta$ is a real bosonic field,
the simplest way to give it time evolution (``quantum fluctuation'') is to add a $(\partial_t \theta)^2$
term. An interesting question may be how can we write down a quantum version of the \emph{lattice} XY 
model whose continuum field theory is (3.3.10). TODO: is this just \cite{latticemodel,Z_iga_2014}?

The method the author used to derive (3.3.13) and (3.3.14) is not really necessary in this case, 
since we all know what $\theta$ actually means, but it is a general method to connect the original 
degrees of freedom to the degrees of freedom in the effective theory so that we are able to calculate 
physical quantities defined in the original theory.

The quasiparticles corresponding to the small perturbation of the $\theta$ field have linear dispersion. They are spin waves when 
the field $\varphi \sim S^x + \ii S^y$. In Section~\ref{oldfluid-sec:intro-superfluid} of \oldfluid,
we introduced a physical picture which says if there are ``phonons'' in a bosonic liquid (and in this case 
$\varphi$ is the field operator of the bosons forming the fluid), then there may be a superfluid phase.
Therefore, sometimes we also call the quasiparticles connected to $\theta$ \emph{phonons}.

Since the target space of $\theta$ is $S^1$, we still have vortices in the quantum XY model, and these 
vortices have quantum fluctuations, i.e. time evolution. Note that in the superfluid phase, the amplitude 
of $\varphi$ is more rigid and $\vb*{j} \propto \grad \theta$, so we have 
\[
    \oint_C \dd{\vb*{l}} \cdot \vb*{v} \propto \oint_C \dd{\vb*{l}} \cdot \grad{\theta} = 2 \pi n \in \pi_1(S^1),
\]
and the vortices in the quantum XY model are just \emph{hydrodynamic vortices} in the bosonic fluid.

\section{Superfluid as a toy universe}

\marginnote{Section 3.3.5}
If there are ``phonon charges'' in a superfluid, then we can always fix a test charge at one particular 
point, and a static electric field surrounds it. Note, however, that if we regard $\theta$ as the potential,
then the electric field $- \grad \theta$ is just the current of the $\varphi$ bosons. In a static $\grad{\theta}$
configuration, we have stale non-zero $\grad{\theta}$, meaning that $\varphi$ bosons continuously flow from 
the position of the test charge to infinity. This definitely is not allowed in 

TODO: Still I'm not sure about what a roton is in this picture \ldots

\section{Equivalence to a sine-Gordon model}

In this section we show the duality between the 2D XY model and the 2D sine-Gordon model, which can also 
be thought as an effective model of \emph{clock model} (Wen used this name for the sine-Gordon model, just 
like using the name ``XY model'' for the effective field theory of the XY model). \marginnote{Section 3.5.2}
Sine-Gordon model is easier to study using RG, and \ktnote{} discusses the KT phase transition in the 
\emph{classical} XY model using the sine-Gordon model -- which is most frequently 
seen in introduction to KT phase transition itself. It should be noticed that the 
phase transition is not restricted to the classical region -- we may define quantum XY models, just as
is the case in \prettyref{sec:quantum-xy}. 
TODO: For example, the $\theta$ degrees of freedom in \ktnote{} can be realized with quantum rotors, which are 
introduced in the first chapter in \cite{Sachdev2009}. 

The duality is shown by Eq. (3.5.4). Here we show some details in the derivation. \marginnote{Eq. (3.5.4)} 
From Eq.~(3.5.3) we get the first line of Eq. (3.5.4): 
\[
    \begin{aligned}
        Z &= \int \fd{\theta} \  \sum_{k=0}^\infty \frac{1}{k!} \left( g \int \dd[2]{\vb*{x}} \frac{\ee^{\ii \theta} + \ee^{- \ii \theta}}{2} \right)^k \ee^{- \int \dd[2]{\vb*{x}} \frac{\kappa}{2} (\grad \theta)^2} \\
        &= \int \fd{\theta} \  \sum_{k=0}^\infty \frac{1}{k!} \left(\frac{g}{2}\right)^k \prod_{i=1}^k \int \dd[2]{\vb*{x}_i} \sum_{\{q_i = \pm 1 \}} \ee^{\sum_{i=1}^k \ii q_i \theta(\vb*{x}_i)} \ee^{ \int \dd[2]{\vb*{x}} \frac{\kappa}{2} \theta \laplacian \theta} .
    \end{aligned}
\]
Now, note that 
\[
    \exp(\sum_{i=1}^k \ii q_i \theta(\vb*{x}_i)) = \exp(\int \dd[2]{\vb*{x}} \theta(\vb*{x}) \sum_{i=1}^k \ii q_i \delta(\vb*{x} - \vb*{x}_i)),
\]
we can integrate out $\theta$ and get 
\[
    \begin{aligned}
        &\quad \int \fd{\theta} \ \ee^{\sum_{i=1}^k \ii q_i \theta(\vb*{x}_i)} \ee^{ \int \dd[2]{\vb*{x}} \frac{\kappa}{2} \theta \laplacian \theta} \\
        &= Z_0 \exp(- \frac{1}{2} \int \dd[2]{\vb*{x}} \int \dd[2]{\vb*{y}} \frac{1}{2 \pi \kappa} \ln \abs*{\vb*{x} - \vb*{y}} \sum_{i=1}^k \ii q_i \delta(\vb*{x} - \vb*{x}_i) \cdot \sum_{j=1}^k \ii q_j \delta(\vb*{y} - \vb*{x}_j)) \\
        &= Z_0 \exp(\frac{1}{4 \pi \kappa} \sum_{i=1}^k \sum_{j=1}^k q_i q_j \ln r_{ij}),
    \end{aligned}
\]
where we have used the Green function of $\laplacian$, which is 
\[
    \frac{1}{\kappa \laplacian} = \frac{1}{2 \pi \kappa} \ln r_{ij},
\]
and $Z_0$ is the partition function of the action $- \int \dd[2]{\vb*{x}} (\grad \theta)^2$. 
Therefore, we have 
\begin{equation}
    Z = Z_0 \sum_{k=0}^\infty \frac{1}{k!} \left(\frac{g}{2}\right)^k \prod_{i=1}^k \int \dd[2]{\vb*{x}_i} 
    \sum_{\{q_i = \pm 1 \}} \exp(\frac{1}{4 \pi \kappa} \sum_{i=1}^k \sum_{j=1}^k q_i q_j \ln r_{ij}).
    \label{eq:proto-sine-gordon-to-xy-1}
\end{equation}
Note that when $i=j$, $\ln r_{ij}$ diverges, and there are $k$ diverging terms. We can collect these divergence 
and $(g/2)^k$ together and define 
\begin{equation}
    \ee^{-k S_\text{core}} = \left(\frac{g}{2}\right)^k \exp(\frac{1}{4 \pi \kappa} k \ln 0 ),
\end{equation}  
and now we have 
\begin{equation}
    Z = \sum_{k=0}^\infty \frac{1}{k!} \ee^{- k S_\text{c}} \prod_{i=1}^k \int \dd[2]{\vb*{x}_i} \sum_{\{q_i = \pm 1\}} \exp(\frac{1}{4 \pi \kappa} 2 \sum_{i < j}^k q_i q_j \ln r_{ij}).
\end{equation}
This is already very close to the second line of Eq. (3.5.4), and we already get the 2D Coulomb interaction. 
What we need to do to get the final version of Eq. (3.5.4) is to notice that the low energy configurations 
contain the same number of positive and negative $q_i$'s, so we only have to consider the even $k$ terms. 
Defining
\begin{equation}
    h = \frac{1}{4 \pi \kappa},
\end{equation}
we have 
\[
    Z \approx \sum_{k=0}^\infty \frac{1}{(2k)!} \ee^{- 2 k S_\text{c}} \int \prod_{i=1}^{2k} \dd[2]{\vb*{x}_i} \sum_{q_i = \pm 1} \ee^{\sum_{i < j}^k 2 q_i q_j \ln r_{ij}}.
\]
Currently we imposes no restriction on $\{q_i\}$ and we just sum over all possible configurations, but since 
we can just rearrange $\{\vb*{x}_i\}$, all $\{q_i\}$ configurations with the same $k$ actually give the same 
weight in the partition function. There are $C_{2k}^k$ possible $\{q_i\}$ configurations with a given $k$ (since
we need exactly $k$ positive charges in $2k$ charges), and therefore we have 
\[
    \begin{aligned}
        Z &\approx \sum_{k=0}^\infty \frac{1}{(2k)!} \ee^{- 2 k S_\text{c}} \int \prod_{i=1}^{2k} \dd[2]{\vb*{x}_i} \underbrace{\frac{(2k!)}{k! k!}}_{C_{2k}^k} \ee^{\sum_{i < j}^k 2 q_i q_j \ln r_{ij}} \\
        &= \sum_{k=0}^\infty \frac{1}{k! k!} \ee^{- 2 k S_\text{c}} \int \prod_{i=1}^{2k} \dd[2]{\vb*{x}_i} \ee^{\sum_{i < j}^k 2 q_i q_j \ln r_{ij}}.
    \end{aligned}
\]
So now we have completed the derivation of Eq. (3.5.4), which is \eqref{kt-eq:coulomb-gas} in \ktnote.

In \ktnote{} we showed the equivalence between the XY model and the sine-Gordon model using Hubbard-Stratonovich
transformation (see Section~\ref{kt-sec:sine-gordon}), and here we show the equivalence directly by analyzing 
the Taylor expansion of the partition function. What we are doing here is actually a path integral version of 
second quantization, where we have shown that indeed the partition function of a many-body system is the 
same as the partition function of a field theory. 

Now we discuss the counterpart of vortices in the sine-Gordon model. Since vortices are non-local objects,
there is no such thing as ``correspondence of a vortex operator''. However, we can still get some insight 
with the energy of a configuration. From the above derivation, we see the weight of a configuration with 
$2k$ vortices placed at $\vb*{x}_1, \ldots, \vb*{x}_{2k}$ is
\[
    \sim \int \fd{\theta} \int \dd[2]{\vb*{x}} \prod_{i=1}^{2k} \ee^{\ii q_i \theta(\vb*{x}_i)} \ee^{- \int \dd[2]{\vb*{x}} \frac{\kappa}{2} (\grad \theta)^2},
\]
so roughly we have  
\begin{equation}
    \text{a vortex with winding number $q$ at $\vb*{x}$} \sim \text{$\ee^{\ii q \theta(\vb*{x})}$ in sine-Gordon}
    \label{eq:corresponding-1}
\end{equation}
in the sense of energy. This can also be seen in \eqref{kt-eq:mixed-sine-gordon-vortices} in \ktnote. 

From the discussion at the end of Section~\ref{kt-sec:sine-gordon}, we know there is no vortex in the 
sine-Gordon model corresponding to the XY model. Now if we are to work in a sine-Gordon model which 
allows vortices, what is the corresponding XY-like theory? From our experience deriving the effective 
theory of vortices in the XY model, we see that there is a divergent ``core'' energy in the $(\grad \theta)^2$ 
term, and we can easily find that $\cos \theta$ does not have such contributions. Therefore, 
the energy of vortices in the sine-Gordon model mainly comes from the $(\grad \theta)^2$ term. 
Then we can % TODO: it seems that since we can do the H-S transformation inversely, we can really get an XY model. Still, details are needed
So we have 
\begin{equation}
    \text{a vortex in sine-Gordon model} \sim \text{$\ee^{\ii q_i \theta(\vb*{x}_i)}$ in XY model}.
    \label{eq:corresponding-2}
\end{equation}
\eqref{eq:corresponding-2} has a direct implication: since the degrees of freedom in the XY model we are 
interested in at first are representations of the compact $U(1)$ group, we are able to have configurations 
with a non-zero winding number, and this symmetry blocks any $\ee^{\ii \theta}$ terms in the partition 
function of the XY model, and therefore there should be no vortices in the corresponding sine-Gordon 
model, i.e. the corresponding sine-Gordon model should be the \emph{non-compact} version.

\eqref{eq:corresponding-1} and \eqref{eq:corresponding-2} are examples of the so-called \concept{particle-vortex duality}.
An introduction of this can be found in \cite{tong2016lectures}. \marginnote{David Tong QHE Sec.~5.3}
It is possible to write down two continuous field theories, the particles of a field in one of which is 
vortices in the other.

\section{KT phase transition}


\bibliographystyle{plain}
\bibliography{xy,../tqft-cond-th/topological-order} 

\end{document}
