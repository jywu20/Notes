\documentclass[hyperref, a4paper]{article}

\usepackage{geometry}
\usepackage{titling}
\usepackage{titlesec}
% No longer needed, since we will use enumitem package
% \usepackage{paralist}
\usepackage{enumitem}
\usepackage{footnote}
\usepackage{marginnote}
\usepackage{enumerate}
\usepackage{amsmath, amssymb, amsthm}
\usepackage{mathtools}
\usepackage{bbm}
\usepackage{cite}
\usepackage{graphicx}
\usepackage{subfigure}
\usepackage{physics}
\usepackage{tensor}
\usepackage{siunitx}
\usepackage[version=4]{mhchem}
\usepackage{tikz}
\usepackage{xcolor}
\usepackage{listings}
\usepackage{autobreak}
\usepackage[ruled, vlined, linesnumbered]{algorithm2e}
\usepackage{nameref,zref-xr}
\zxrsetup{toltxlabel}
\zexternaldocument*[optics-]{../optics/optics}[optics.pdf]
\zexternaldocument*[solid-]{../solid/solid}[solid.pdf]
\usepackage[colorlinks,unicode]{hyperref} % , linkcolor=black, anchorcolor=black, citecolor=black, urlcolor=black, filecolor=black
\usepackage[most]{tcolorbox}
\usepackage{prettyref}

% Page style
\geometry{left=3.18cm,right=3.18cm,top=2.54cm,bottom=2.54cm}
\titlespacing{\paragraph}{0pt}{1pt}{10pt}[20pt]
\setlength{\droptitle}{-5em}
\preauthor{\vspace{-10pt}\begin{center}}
\postauthor{\par\end{center}}

% More compact lists 
%\setlist[itemize]{
%    itemindent=17pt, 
%    leftmargin=1pt,
%    listparindent=\parindent,
%    parsep=0pt,
%}

% Math operators
\DeclareMathOperator{\timeorder}{\mathcal{T}}
\DeclareMathOperator{\diag}{diag}
\DeclareMathOperator{\legpoly}{P}
\DeclareMathOperator{\primevalue}{P}
\DeclareMathOperator{\sgn}{sgn}
\newcommand*{\ii}{\mathrm{i}}
\newcommand*{\ee}{\mathrm{e}}
\newcommand*{\const}{\mathrm{const}}
\newcommand*{\suchthat}{\quad \text{s.t.} \quad}
\newcommand*{\argmin}{\arg\min}
\newcommand*{\argmax}{\arg\max}
\newcommand*{\normalorder}[1]{: #1 :}
\newcommand*{\pair}[1]{\langle #1 \rangle}
\newcommand*{\fd}[1]{\mathcal{D} #1}
\DeclareMathOperator{\bigO}{\mathcal{O}}

% TikZ setting
\usetikzlibrary{arrows,shapes,positioning}
\usetikzlibrary{arrows.meta}
\usetikzlibrary{decorations.markings}
\tikzstyle arrowstyle=[scale=1]
\tikzstyle directed=[postaction={decorate,decoration={markings,
    mark=at position .5 with {\arrow[arrowstyle]{stealth}}}}]
\tikzstyle ray=[directed, thick]
\tikzstyle dot=[anchor=base,fill,circle,inner sep=1pt]

% Algorithm setting
% Julia-style code
\SetKwIF{If}{ElseIf}{Else}{if}{}{elseif}{else}{end}
\SetKwFor{For}{for}{}{end}
\SetKwFor{While}{while}{}{end}
\SetKwProg{Function}{function}{}{end}
\SetArgSty{textnormal}

\newcommand*{\concept}[1]{{\textbf{#1}}}

% Support for tensor double arrows.
\renewcommand{\tensor}[1]{ \stackrel{\leftrightarrow}{\vb*{#1}}}

% Embedded codes
\lstset{basicstyle=\ttfamily,
  showstringspaces=false,
  commentstyle=\color{gray},
  keywordstyle=\color{blue}
}

% Reference formatting
\newrefformat{fig}{Figure~\ref{#1} on page~\pageref{#1}}

% Color boxes
\tcbuselibrary{skins, breakable, theorems}
\newtcbtheorem[number within=section]{warning}{Warning}%
  {colback=orange!5,colframe=orange!65,fonttitle=\bfseries, breakable}{warn}
\newtcbtheorem[number within=section]{note}{Note}%
  {colback=green!5,colframe=green!65,fonttitle=\bfseries, breakable}{note}

\title{Bosonic Field Theories in Condensed Matter Physics}
\author{Jinyuan Wu}

\begin{document}

\maketitle

This article is a reading note of Chapter 3 of Wen's famous textbook. 

\section{A simplest interacting boson system}

Section~3.3.1 provides a simplest interacting bosonic system with a complex scalar field \marginnote{Eq. (3.3.1)}
\begin{equation}
    S = \int \dd[d]{\vb*{x}} \dd{t} (\ii \varphi^* \partial_t \varphi - \frac{1}{2m} \partial_{\vb*{x}} \varphi^* \partial_{\vb*{x}} \varphi + \mu \abs*{\varphi}^2 - \frac{V_0}{2} \abs*{\varphi}^4).
    \label{eq:bosonic}
\end{equation}
The prefactor of the interaction term makes the corresponding term in the EOM of $\varphi$ and $\varphi^*$ not 
have a numerical factor, but it introduces a numerical factor in the vertex in Feynman diagrams. The sign of 
the mass term is derived as follows: first we have a $- \varphi^* \laplacian \varphi / 2m$ term in the 
Hamiltonian, and therefore we have a $\varphi^* \laplacian \varphi / 2m$ term in the Lagrangian, and by 
integration by parts we have $\varphi^* \laplacian \varphi / 2m \simeq - \partial_{\vb*{x}} \varphi^* 
\partial_{\vb*{x}} \varphi / 2m$. 

The semiclassical approximation from (3.3.1) to (3.3.2) can be justified when the temperature is high 
and therefore the most economical path does not have imaginary time evolution at all. It can also 
be derived using the ideas behind (3.4.1), where with a finite temperature, we can always integrate 
out modes with non-zero Matsubara frequencies. This gives a physical picture behind dynamic density 
functional theories and also explains why ``classical'' statistical physics is still relevant today.
It should be noted that the 

The following contents from Eq.~(3.3.3) to Eq.~(3.3.4) are also covered \href{fluid.pdf}{here}. 
The discussion between Eq. (3.3.4) to the end of Section~3.3.2 is important, which illustrates the 
Ginzburg-Landau paradigm and why it is almost always associated with symmetries (or otherwise it is 
highly unlikely that we have several minima of the energy functional that share the same energy, so 
that we have a smooth phase transition shown in Fig.~3.5), though the concept of order parameters 
can also be used in a first-order phase transition (see \href{../quasicrystal-fd/main.pdf}{here}, 
for example).  

\section{Quantum XY model from \eqref{eq:bosonic}}



\end{document}
