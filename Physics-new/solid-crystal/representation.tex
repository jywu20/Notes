\documentclass[hyperref, a4paper, 12pt]{article}

\usepackage{geometry}
\usepackage{float}
\usepackage{titling}
\usepackage{titlesec}
\usepackage{booktabs}
% No longer needed, since we will use enumitem package
% \usepackage{paralist}
\usepackage{enumitem}
\usepackage{footnote}
\usepackage{amsmath, amssymb, amsthm}
\usepackage{mathtools}
\usepackage{bbm}
\usepackage{graphicx}
\usepackage{subfigure}
\usepackage{simpler-wick}
\usepackage{physics}
\usepackage{tensor}
\usepackage{siunitx}
\usepackage[version=4]{mhchem}
\usepackage{tikz}
\usepackage{xcolor}
\usepackage{listings}
\usepackage{underscore}
\usepackage{autobreak}
\usepackage[ruled, vlined, linesnumbered]{algorithm2e}
\usepackage{nameref,zref-xr}
\zxrsetup{toltxlabel}
\usepackage[colorlinks,unicode, bookmarksnumbered]{hyperref} % , linkcolor=black, anchorcolor=black, citecolor=black, urlcolor=black, filecolor=black
\usepackage[most]{tcolorbox}
\usepackage{prettyref}

% Page style
\geometry{left=3.18cm,right=3.18cm,top=2.54cm,bottom=2.54cm}
\titlespacing{\paragraph}{0pt}{1pt}{10pt}[20pt]
\setlength{\droptitle}{-5em}

% More compact lists 
\setlist[itemize]{
    %itemindent=17pt, 
    %leftmargin=1pt,
    listparindent=\parindent,
    parsep=0pt,
}

\setlist[enumerate]{
    %itemindent=17pt, 
    %leftmargin=1pt,
    listparindent=\parindent,
    parsep=0pt,
}

% Math operators
\DeclareMathOperator{\timeorder}{\mathcal{T}}
\DeclareMathOperator{\diag}{diag}
\DeclareMathOperator{\legpoly}{P}
\DeclareMathOperator{\primevalue}{P}
\DeclareMathOperator{\sgn}{sgn}
\DeclareMathOperator{\res}{Res}
\newcommand*{\ii}{\mathrm{i}}
\newcommand*{\ee}{\mathrm{e}}
\newcommand*{\const}{\mathrm{const}}
\newcommand*{\suchthat}{\quad \text{s.t.} \quad}
\newcommand*{\argmin}{\arg\min}
\newcommand*{\argmax}{\arg\max}
\newcommand*{\normalorder}[1]{: #1 :}
\newcommand*{\pair}[1]{\langle #1 \rangle}
\newcommand*{\fd}[1]{\mathcal{D} #1}
\DeclareMathOperator{\bigO}{\mathcal{O}}

% TikZ setting
\usetikzlibrary{arrows,shapes,positioning}
\usetikzlibrary{arrows.meta}
\usetikzlibrary{decorations.markings}
\usetikzlibrary{calc}
\tikzstyle arrowstyle=[scale=1]
\tikzstyle directed=[postaction={decorate,decoration={markings,
    mark=at position .5 with {\arrow[arrowstyle]{stealth}}}}]
\tikzstyle ray=[directed, thick]
\tikzstyle dot=[anchor=base,fill,circle,inner sep=1pt]

% Algorithm setting
% Julia-style code
\SetKwIF{If}{ElseIf}{Else}{if}{}{elseif}{else}{end}
\SetKwFor{For}{for}{}{end}
\SetKwFor{While}{while}{}{end}
\SetKwProg{Function}{function}{}{end}
\SetArgSty{textnormal}

\newcommand*{\concept}[1]{{\textbf{#1}}}

% Embedded codes
\lstset{basicstyle=\ttfamily,
  showstringspaces=false,
  commentstyle=\color{gray},
  keywordstyle=\color{blue}
}

\lstdefinestyle{console}{
    basicstyle=\footnotesize\ttfamily,
    breaklines=true,
    postbreak=\mbox{\textcolor{red}{$\hookrightarrow$}\space}
}

% Reference formatting
\newrefformat{fig}{Figure~\ref{#1}}

% Color boxes
\tcbuselibrary{skins, breakable, theorems}

\newtcbtheorem[auto counter]{quotebox}{}{
    %title empty,
    enhanced,
    boxrule=0pt,
    %colback=blue!5,
    %colframe=blue!5,
    colback=white,
    colframe=white,
    coltitle=black!65,
    borderline west={2pt}{0pt}{black!65},
    %top=0pt,
    sharp corners,
    fonttitle=\bfseries, 
    breakable,
    }{box}

% Displaying texts in bookmarkers

\pdfstringdefDisableCommands{%
  \def\\{}%
  \def\ce#1{<#1>}%
}

\pdfstringdefDisableCommands{%
  \def\texttt#1{<#1>}%
  \def\mathbb#1{#1}%
}
\pdfstringdefDisableCommands{\def\eqref#1{(\ref{#1})}}

\makeatletter
\pdfstringdefDisableCommands{\let\HyPsd@CatcodeWarning\@gobble}
\makeatother

\newenvironment{shelldisplay}{\begin{lstlisting}}{\end{lstlisting}}

\newcommand*{\citesec}[1]{\S~{#1}}
\newcommand*{\citechap}[1]{Ch.~{#1}}
\newcommand*{\citefig}[1]{Fig.~{#1}}
\newcommand*{\citetable}[1]{Table~{#1}}
\newcommand*{\citepage}[1]{p.~{#1}}
\newcommand*{\citepages}[1]{pp.~{#1}}
\newcommand*{\citefootnote}[1]{fn.~{#1}}
\newcommand*{\citechapsec}[2]{\citechap{#1}.\citesec{#2}}
\newcommand{\literature}[1]{\textit{#1}}

\newrefformat{sec}{\citesec{\ref{#1}}}
\newrefformat{fig}{\citefig{\ref{#1}}}
\newrefformat{tbl}{\citetable{\ref{#1}}}
\newrefformat{chap}{\citechap{\ref{#1}}}
\newrefformat{fn}{\citefootnote{\ref{#1}}}
\newrefformat{box}{Box~\ref{#1}}
\newrefformat{ex}{\ref{#1}}

\newcommand{\shortcode}[1]{\texttt{#1}}

\lstset{style = console}


\newcommand*{\abinitio}{\textit{ab initio}}
\newcommand*{\kB}{k_{\text{B}}}
\newcommand*{\epsr}{\epsilon_{\text{r}}}

\title{Representation of point and space groups}
\author{Jinyuan Wu}

\begin{document}

\maketitle

\section{Overview of representations of finite groups}

\subsection{The characteristic table}

The \concept{character} of a group element in a representation
is defined as its trace.
Thus the character of $e$ (i.e. the identity transform) is 
the dimension of the representation space.

Because $\trace{g a g^{-1}} = \trace{g^{-1} g a} = \trace{a}$, we have

\begin{quotebox*}{Character of conjugacy class}{}
    The characters of all elements in a conjugacy class are the same.
    (But having the same character doesn't mean two elements are in one conjugacy class.)
\end{quotebox*}

The \concept{character table} is square, because

\begin{quotebox*}{Number of irreducible representations}{}
    The number of conjugacy classes is the same as the number of irreducible representations.
\end{quotebox*}

Notation:
\begin{itemize}
    \item The order of group $G$, i.e. the total number of elements it contains: $n$.
    \item $p, q$ refer to labels of irreducible representations.
    \item $g$ refers to a group element.
    \item $\chi^{(p)}(g)$ is the character of $g$ in representation $p$.
    \item $[g]$ is the conjugacy class containing $g$.
\end{itemize}

\begin{quotebox*}{The great orthogonality theorem}{}
    First,
    \[
        \frac{1}{n} \sum_{g \in G} \chi^{(p)}(g)^* \chi^{(q)}(g) = \delta_{pq}.
    \]
    Second, when $g$ and $g'$ are in the same conjugacy class,
    \[
        \frac{1}{n} \sum_p \chi^{(p)}(g)^* \chi^{(p)}(g') = \frac{1}{\abs*{[g]}},
    \]
    or otherwise the LHS vanishes.

    The second equation consequently means that the character table,
    if viewed as a matrix, is full rank.
\end{quotebox*}

The size of the conjugacy class of $e$ is always 1,
and therefore from the second equation above, we have 

\begin{quotebox*}{Burnside theorem}{}
    Suppose $d_p$ is the dimension of the representation space of representation $p$.
    \[
        \sum_p d_p^2 = n.
    \]
\end{quotebox*}

One usage of the character table:

\begin{quotebox*}{Equivalent representations}
    Two representations are equivalent if and only if they have the same characters.
\end{quotebox*}

\subsection{Representation of a group on a vector space}

In physics we work with wave functions, operators, Hamiltonians, etc.
We should note that when we say that a system has a symmetry $G$,
it means for all operations $U$ in $G$ (or more precisely,
in the representation of $G$ on the Hilbert space),
\begin{equation}
    \mel{\psi}{H}{\varphi} = \mel{\psi}{U^\dag H U}{\varphi} \text{ for all $\ket*{\psi}, \ket*{\varphi}$} \Leftrightarrow H = U^\dag H U,
    \label{eq:sym-of-h}
\end{equation}
and \emph{not}
\[
    \mel{\psi}{H}{\varphi} = \mel{\psi}{U^\dag U H U^\dag U}{\varphi},
\]
because the latter is trivial true:
it's like saying ``when I rotate the system (i.e. $H$) and I rotate my experimental configurations (i.e. $\ket*{\psi}$), the responses stay the same'' --
of course they \emph{have to} say the same.
A non-trivial claim would be something like ``when I rotate my system, while \emph{not} rotating my lab setup, magically all outputs are the same''
-- which is precisely \eqref{eq:sym-of-h}.

\subsection{Scalar, vector, tensor representations}\label{sec:tensor-group-rep}

\begin{quotebox*}{Decomposition of tensor products}{}
    Because $\trace{D_1(g) \otimes D_2(g)} = \trace{D_1(g)} \trace{D_2(g)}$
    for any group element $g$,
    the coefficients in 
    \begin{equation*}
        D_1(g) \otimes D_2(g) = \sum_p a_p D_p(g)
    \end{equation*}
    can be calculated by solving the equation system
    \begin{equation*}
        \chi_{D_1}(g) \chi_{D_2}(g) = \sum_{p} a_p \chi_p(g)
    \end{equation*}
    for all conjugacy classes $[g]$.
    Note that the RHS is full-rank, and the equation has a unique solution.
\end{quotebox*}

\section{Mulliken symbols}

Representations of a group may be labeled as $A_1$, $B_g$, or things like that.
This is known as the \concept{Mulliken symbols}.

The A, B, E symbols mean the follows:

\begin{table}[H]
    \caption{Letter notation of dimension}
    \centering
    \begin{tabular}{@{}ll@{}}
        \toprule
        \textbf{dimension} & \textbf{Mulliken symbol} \\ \midrule
        1                  & A and B                  \\
        2                  & E                        \\
        3                  & T                        \\
        4                  & G                        \\
        5                  & H                        \\ \bottomrule
    \end{tabular}
\end{table}

The distinction between A and B shows the sign of $\chi(c_n)$,
where $c_n$ represents rotation along the principal axis.

When an $c_2$ axis or a palne of reflection or a $\sigma_v$ plane of reflection,
we use subscripts $1,2$ to indicate if a sign change follows after this operation.
Similarly, $g$ and $u$ means $\chi(i) = \pm 1$, respectively.
Finally, $\chi(\sigma_h)$ is represented by $'$ ($+1$) or $''$ ($-1$).

\section{How things changes under point group operations}\label{sec:how-things-change}

Point group operations naturally have a representation in the 3D Euclidean space.
This representation may be reducible:
in this case, different irreducible representations act on different subspaces of $\mathbb{R}^3$.

We cannot know which subspace of $\mathbb{R}^3$ an irreducible representation acts on.
Therefore in reference books, the standard character table
is often augmented by what carries a representation.

Moreover, we note that some irreducible representations \emph{cannot} be 
written as components of the natural representation on $\mathbb{R}^3$.
The trivial identical representation of $D_{3h}$,
for instance, doesn't correspond to how any subspace of $\mathbb{R}^3$ changes:
we have $\sigma_h$ and it's not possible to let $z$ stay unchanged.
These representations however can be carried by polynomials of $x, y, z$.

\href{http://symmetry.constructor.university/}{This website} 
contains the extended character table of each point group.

\section{Symmetry of dielectric tensor}

In optics, $\vb*{q}$ is usually small,
and TODO: analyticity of $\epsilon$

\section{Optics in hexagonal crystals}

Suppose we are dealing with a hexagonal system with $D_{6h}$.
From the (extended; \prettyref{sec:how-things-change}) character table,
we find that the $A_{2u}$ irreducible representation acts on the $z$ component,
while the $E_{1u}$ irreducible representation acts on the $x, y$ components in $\mathbb{R}^3$.
These can be known by consulting the character table \href{http://symmetry.constructor.university/cgi-bin/group.cgi?group=606&option=4}{here}.
Thus the representation of $D_{6h}$ on $\mathbb{R}^3$ is 
\begin{equation}
    D = E_{1u} \oplus A_{2u}.
    \label{eq:d-6h-r3-rep}
\end{equation}
This is also given by the \shortcode{dipole (p)} section of 
\href{http://symmetry.constructor.university/cgi-bin/group.cgi?group=606&option=4}{the webpage above}.
Note that strictly speaking, \eqref{eq:d-6h-r3-rep} does not track which components the representations act on:
we need to keep that in mind ourselves.

Because we're working with optics, the wave length is assumed to be long enough,
and in utilizing \eqref{eq:sym-of-h} (where the wave functions are replaced by
electric field configurations, or ``single-photon wave functions''),
we can ignore the real space transform and focus on how components of $\vb*{E}$
are mixed together.
Thus \eqref{eq:sym-of-h} means that for all $g \in D_{6h}$,
\begin{equation}
    \vb*{\epsilon} = D^\dag(g) \vb*{\epsilon} D(g).
\end{equation}

The RHS transforms as a rank-2 tensor. (TODO: covariant and contravariant indices?)
The representation of $D_{6h}$ it carries is $D \otimes D$.
Note that the two $D$'s cannot be exchanged,
as the first acts on the first indices of $\epsilon_{ij}$,
while the second acts on the second indices.
Keeping the meaning of being before and after $\otimes$,
and that $E_{1u}$ acts on the $x, y$ component and $A_{2u}$ acts on the $z$ component
(note that we have a natural $z$ direction,
which is parallel to the $C_6$ axis),
we have 
\begin{equation}
    D \otimes D = (A_{2u} \otimes A_{2u}) \oplus 
    (A_{2u} \otimes E_{1u}) \oplus
    (E_{1u} \otimes A_{2u}) \oplus
    (E_{1u} \otimes E_{1u}).
\end{equation}

The next step is to see how the four tensor products 
act on the nine components of $\vb*{\epsilon}$,
following the procedure in \prettyref{sec:tensor-group-rep}.

\paragraph*{The $\epsilon_{zz}$ subspace}
We first notice that the absolute values of all characters of $A_{2u}$ is 1,
and hence the characters of $A_{2u} \otimes A_{2u}$ are all 1,
meaning that 
\begin{equation}
    A_{2u} \otimes A_{2u} = A_{1g},
\end{equation}
i.e. the trivial representation.
This immediately means that the $\epsilon_{zz}$ component is constant
regardless of whatever operations applied to the system.

\paragraph*{The $(\epsilon_{xz}, \epsilon_{yz})$ subspace}
By calculating the characters we also find 
\begin{equation}
    E_{1u} \otimes A_{2u} = E_{1g}.
\end{equation}
This means the $\epsilon_{xz}, \epsilon_{yz}$ components transform as $E_{1g}$.
Actually, once we realize that one irreducible representation acts on the $x, y$ coordinates
and another acts on the $z$ coordinate,
we should realize that $\epsilon_{xz}, \epsilon_{yz}$
transforms in the same way as $xz, yz$ does:
in the latter, the first variable ($x$ or $y$) feels the action of $E_{1u}$,
and the second variable ($z$) feels the action of $A_{2u}$.
So indeed in \href{http://symmetry.constructor.university/cgi-bin/group.cgi?group=606&option=4}{the character table}
we find that $E_{1g}$ acts on the space spanned by 
linearly recombining quadratic functions $(xz, yz)$.

Now, from \eqref{eq:sym-of-h}, we have 
\begin{equation}
    E_{1g}(g) \pmqty{\epsilon_{xz} \\ \epsilon_{yz}} = \pmqty{\epsilon_{xz} \\ \epsilon_{yz}},
    \label{eq:hexagonal-xz-yz}
\end{equation}
for all $g \in D_{6h}$.
Because we know $E_{1g}$ is $D_{6h}$'s action on the linear space spanned by $(xz, yz)$,
it immediately follows that 
\begin{equation}
    E_{1g}(c_6) = \pmqty{
        \frac{1}{2} & - \frac{\sqrt{3}}{2} \\
        \frac{\sqrt{3}}{2} & \frac{1}{2}
    },
\end{equation}
and from that we find that the linear equation system in \eqref{eq:hexagonal-xz-yz}
only has a vanishing solution.
This means in a hexagonal system, $\epsilon_{xz} = \epsilon_{yz} = 0$.

Following the same logic it can be shown that 
\begin{equation}
    \epsilon_{xz} = \epsilon_{yz} = \epsilon_{zx} = \epsilon_{zy} = 0.
\end{equation}
What we have shown here is that the subspace of $\vb*{\epsilon}$ that carries the 
$E_{1u} \otimes A_{2u}$ representation can't satisfy so many requirements from \eqref{eq:sym-of-h},
and everything has to vanish.

\paragraph*{The $\epsilon_{x/y, x/y}$ subspace}



\end{document}