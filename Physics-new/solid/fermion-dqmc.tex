\begin{back}{费米子系统的行列式蒙特卡洛模拟}{fermion-dqmc}
    为了和时间演化算符的通常写法保持一致,本文默认
    \[
        \prod_{i=1}^n a_i = a_n a_{n-1} \cdots a_1.
    \]
    如无特殊说明,使用$\tau$表示虚时间,$n, m$等表示离散的虚时间点的编号;使用$i, j$表示格点坐标,使用$\sigma$表示自旋,使用$x, y$等表示电子在坐标表象下的量子数,即$x=(i, \sigma)$。
    设$\hat{c}^\dagger$表示适当表象下的费米子产生算符排成的行向量,$\vb{s}_n$和$\vb{s}_\tau$表示第$n$个虚时间采样点(也即,$\tau=n\Delta \tau$)处的辅助场构型,如无特殊说明$\vb{s}$就表示整个辅助场时间线。

    由于费米子自由度难以直接在计算机中表示(格拉斯曼变量本质上是算符),我们需要使用一个玻色场来等效处理它。
    一般来说费米子的哈密顿量可以写成自由哈密顿量(通常是某个动能项)加上四次型相互作用哈密顿量(即只有二体相互作用,这是合理的,因为基本上固体理论中的相互作用几乎总是来自库伦相互作用)的情况。

    虚时间路径积分实际上就是要计算一个$\ee$指数矩阵的迹。
    
    使用Trotter-Suzuki近似,我们如下将虚时间路径积分写成自由部分和相互作用部分分离的形式:
    \[
        \ee^{-\beta {H}} = \prod_{n=1}^{m} \ee^{-\Delta \tau {H}_\text{I}} \ee^{-\Delta \tau {H}_0}.
    \]
    这里用到了Trotter-Suzuki近似,这个近似的误差控制在$\Delta \tau^2$量级。
    实际上也可以使用不同的Trotter分解顺序,比如说把自由哈密顿量写在前面,而把相互作用哈密顿量写在后面。
    如果没有相互作用哈密顿量,则能够将上式写成
    \[
        {H} = {H}_0 = \sum_{i, j} {c}_i^\dagger A_{ij} {c}_j
    \]
    的形式,那么这就比较容易,因为
    \begin{equation}
        \trace(\ee^{- \sum_{i, j} {c}_i^\dagger A_{ij} {c}_j}) = \det(1 + \ee^{- \vb{A}}).
    \end{equation}
    很容易通过对角化验证上式。实际上,更加一般的,我们有
    \begin{equation}
        \trace(\ee^{- \sum_{i, j} {c}_i^\dagger A_{ij} {c}_j} \ee^{- \sum_{i, j} {c}_i^\dagger B_{ij} {c}_j} \cdots) = \det(1 + \ee^{- \vb{A}}\ee^{- \vb{B}} \cdots),
        \label{eq:trace-to-det}
    \end{equation}
    甚至更一般的情况。
    总之,自由费米子哈密顿量的路径积分可以很容易地将费米子算符积掉,留下一个(可以使用标准的线性代数方法计算的)行列式。

    实际的哈密顿量为
    \[
        {H} = {H}_0 + {H}_\text{I}, 
    \]
    我们就需要设法将${H}_\text{I}$转化为单粒子算符的形式,也就是说要引入一个辅助场,让费米子之间的相互作用等效为费米子和这个辅助场的相互作用。
    可以选择适当的Hubbard-Stratonovich参量,使用离散H-S变换引入这个辅助场,然后积掉费米子自由度而剩下辅助场自由度。辅助场自由度是一整个辅助场世界线,也就是说如果有$d$个空间维度,那么辅助场世界线就有$d+1$个维度。
    这是量子统计的普遍特征:$d$维的量子系统等价于$d+1$维的经典系统,多出来的一个维度是(有限大小的)虚时间。最后就使用关于一系列辅助场构型的行列式之和写出了虚时间路径积分。
    由于辅助场的形式往往需要按照相互作用哈密顿量的形式确定,目前只是指出辅助场的存在性,而不具体讨论怎么做H-S变换。

    虚时间路径积分可以写成
    \begin{equation}
        \ee^{-\beta {H}} = \sum_{\vb{s}} C(\vb{s}) \prod_{n=1}^m \ee^{H_\text{I}(\vb{s}_n)} \ee^{-\Delta \tau {H}_0}, 
        \label{eq:imaginary-time-path-integral-with-aux-field}
    \end{equation}
    应注意其中$H_\text{I}(\vb{s}_n)$和之前定义的${H}_\text{I}$未必相等,它现在是积掉了费米子自由度之后的哈密顿量。
    由于${H}_0$是二次型,而由H-S变换的性质,${H}_\text{I}$也是二次型,则可以设
    \begin{equation}
        {H}_0 = {c}^\dagger \vb{h}_0 {c}, \quad {H}_\text{I} = {c}^\dagger \vb{h}_\text{I} {c},
    \end{equation}
    其中$\vb{h}_\text{I}$和$\vb{h}_0$是系数矩阵。
    为了简写我们显式地引入虚时间演化算符(由于虚时间演化算符含有全部动力学,我们这里采取的是虚时间的薛定谔绘景:不需要考虑算符的变动)
    \begin{equation}
        {U}_{\vb{s}}(\tau_2, \tau_1) = \prod_{n=n_1+1}^{n_2} \ee^{ {H}_\text{I}(\vb{s}_n)} \ee^{-\Delta \tau {H}_0}, \quad \vb{B}_{\vb{s}}(\tau_2, \tau_1) = \prod_{n=n_1+1}^{n_2} \underbrace{\ee^{\vb{h}_\text{I}(\vb{s}_n)} \ee^{-\Delta \tau \vb{h}_0}}_{\vb{B}_{\vb{s}}(\tau_1+\Delta \tau, \tau_1)},
    \end{equation}
    使用这些记号并考虑到\eqref{eq:imaginary-time-path-integral-with-aux-field},
    \[
        \trace(\ee^{-\beta {H}}) = \sum_{\vb{s}} C(\vb{s}) \trace {U}_{\vb{s}}(\beta, 0) = \sum_{\vb{s}} C(\vb{s}) \det(1 + \vb{B}_{\vb{s}}(\beta, 0)),
    \]
    而
    \[
        \begin{aligned}
            \trace(\ee^{-\beta {H}} {O}) &= \sum_{\vb{s}} C(\vb{s}) \trace({O} \ee^{-\beta {H}}) \\
            &= \sum_{\vb{s}} C(\vb{s}) \trace{U}_{\vb{s}}(\beta, 0) \frac{\trace({O} {U}_{\vb{s}}(\beta, 0))}{\trace {U}_{\vb{s}}(\beta, 0)},
        \end{aligned} 
    \]
    于是我们就得到
    \begin{equation}
        \expval*{{O}} = \sum_{\vb{s}} C(\vb{s}) \det(1 + \vb{B}_{\vb{s}}(\beta, 0)) \frac{\trace({O} {U}_{\vb{s}}(\beta, 0))}{\trace {U}_{\vb{s}}(\beta, 0)}.
    \end{equation}
    为了略微增大一般性,例如,为了计算响应函数之类的东西,我们引入完整的虚时间轴上各点的期望值:
    \begin{equation}
        \expval*{{O}}(\tau) = \frac{\trace({U}_{\vb{s}}(\beta, \tau) {O} {U}_{\vb{s}}(\tau, 0))}{\trace {U}_{\vb{s}}(\beta, 0)}.
    \end{equation}

    原则上,我们可以直接使用以上方法现在构造一个经典的随机现象,使得
    \begin{equation}
        p(\vb{s}) = \frac{ C(\vb{s}) \det(1 + \vb{B}_{\vb{s}}(\beta, 0))}{\sum_{\vb{s}} C(\vb{s}) \det(1 + \vb{B}_{\vb{s}}(\beta, 0))}, \quad \expval{O}_{\vb{s}} = \frac{\trace({O} {U}_{\vb{s}}(\beta, 0))}{\trace {U}_{\vb{s}}(\beta, 0)},
        \label{eq:ftqmc-prob}
    \end{equation}
    并且使用标准的有限温度平衡态场论的方法,也就是
    \begin{equation}
        \expval{O}_{\vb{s}} = \eval{\pdv{\ln \trace(\ee^{\eta {O}} {U}_{\vb{s}}(\beta, 0))}{\eta}}_{\eta=0},
    \end{equation}
    计算出$\expval{O}_{\vb{s}}$(由于此时费米子相互作用已经积掉了,而$\vb{s}$又固定死了,仅有的自由度就是费米子自由度,而且只有二次型,则Wick定理适用),我们就可以使用\eqref{eq:classical-expectation}得到各种物理量的期望值了。

    例如在${O}$是单体算符时,设
    \begin{equation}
        {O} = {c}^\dagger \vb{A} {c},
    \end{equation}
    则容易看出
    \begin{equation}
        \expval{O}_{\vb{s}}(\tau) = \trace((1 - (1+ \vb{B}_{\vb{s}}(\tau, 0) \vb{B}_{\vb{s}}(\beta, \tau))^{-1}) \vb{A}).
    \end{equation}
    一旦单体算符的期望确定了,多体算符的期望可以使用单体算符的期望算出来。

    从计算单体算符的期望的方式可以马上看出,格点坐标表象下的等时格林函数就是
    \begin{equation}
        \expval*{{c}_x(\tau) {c}^\dagger_y(\tau)}_{\vb{s}} = (1+ \vb{B}_{\vb{s}}(\tau, 0) \vb{B}_{\vb{s}}(\beta, \tau))^{-1}_{xy},
    \end{equation}
    相应的记其系数矩阵为$\vb{G}(\tau, \tau)$,或者简写为$\vb{G}(\tau)$,则有
    \begin{equation}
        \vb{G}(\tau, \tau) = \vb{G}(\tau) = (1+ \vb{B}_{\vb{s}}(\tau, 0) \vb{B}_{\vb{s}}(\beta, \tau))^{-1}.
    \end{equation}
    需要说明的是这里其实没有用到任何关于标签$x, y$的特殊性质——它们可以是$(i, \sigma)$,即格点坐标和自旋的组合,或者,如果能够保证自旋旋转不变性,那么不同的自旋实际上是解耦的,则可以取$x, y$为格点坐标,而单独附加标注自旋。
    容易看出等时格林函数满足以下递推关系:
    \begin{equation}
        \vb{G}_{\vb{s}}(\tau + 1) = \vb{B}_{\vb{s}}(\tau+1, \tau) \vb{G}_{\vb{s}}(\tau) \vb{B}_{\vb{s}}(\tau+1, \tau)^{-1}.
        \label{eq:eq-time-green-function-iter}
    \end{equation}

    进一步,我们考虑不等时格林函数。这些格林函数实际上对蒙特卡洛更新非常重要(见下一节)。
    设$\tau_1 > \tau_2$,则
    \[
        \begin{aligned}
            G_{\vb{s}}(x, \tau_1; y, \tau_2) &= \expval*{{c}_x(\tau_1) {c}^\dagger_y(\tau_2)} \\
            &= \frac{\trace ({U}_{\vb{s}}(\beta, \tau_2) {U}_{\vb{s}}^{-1}(\tau_1, \tau_2) {c}_x {U}_{\vb{s}}(\tau_1, \tau_2) {c}_y^\dagger {U}_{\vb{s}}(\tau_2, 0))}{\trace {U}_{\vb{s}}(\beta, 0)},
        \end{aligned}
    \]
    因此问题的核心就是要计算${U}_{\vb{s}}^{-1}(\tau_1, \tau_2) {c}_x {U}_{\vb{s}}(\tau_1, \tau_2)$。
    我们首先容易证明恒等式
    \[
        \ee^{\Delta \tau {c}^\dagger \vb{A} {c}} {c} \ee^{- \Delta \tau {c}^\dagger \vb{A} {c}} = \ee^{- \Delta \tau \vb{A}} {c},
    \]
    从而容易看出
    \[
        {U}_{\vb{s}}^{-1}(\tau_1, \tau_2) {c} {U}_{\vb{s}}(\tau_1, \tau_2) = \vb{B}_{\vb{s}}(\tau_1, \tau_2) {c},
    \]
    对上式取共轭转置,并重新定义${U}$,就得到
    \[
        {U}_{\vb{s}}^{-1}(\tau_1, \tau_2) {c}^\dagger {U}_{\vb{s}}(\tau_1, \tau_2) = {c}^\dagger \vb{B}^{-1}(\tau_1, \tau_2).
    \]
    而$\vb{B}$矩阵中不显含任何电子自由度,于是可以把它提到迹运算的外面,并且使用矩阵形式,就得到
    \[
        \vb{G}_{\vb{s}}(\tau_1, \tau_2) = \vb{B}_{\vb{s}}(\tau_1, \tau_2) \vb{G}_{\vb{s}}(\tau_2), \quad \tau_1 > \tau_2.
    \]
    在$\tau_2 > \tau_1$时可以得到类似的结果,最后
    \begin{equation}
        \vb{G}_{\vb{s}}(\tau_1, \tau_2) = \begin{cases}
            \vb{B}_{\vb{s}}(\tau_1, \tau_2) \vb{G}_{\vb{s}}(\tau_2), \quad &\tau_1 > \tau_2, \\
            - (1 - \vb{G}_{\vb{s}}(\tau_1)) \vb{B}^{-1}_{\vb{s}}(\tau_1, \tau_2) , \quad &\tau_1 < \tau_2.
        \end{cases}
    \end{equation}

    原则上\eqref{eq:ftqmc-prob}就足够计算接受率了,从而可以直接用于蒙特卡洛模拟,但是这样在计算上是非常不经济的,因为需要真的计算一系列矩阵的$\ee$指数的积,计算上非常耗时,并且由于每个矩阵都不大,乘起来会造成很大的误差。
    下面我们将讨论怎么高效、精确地计算更新率。

    设我们一次只更新$(i, n)$位置的$\vb{s}$格点,也就是,一次只更新一个固定虚时间点上的辅助场的一个格点。
    显然这只会影响一个$\vb{B}$帧,于是设
    % TODO:是一般结果吗?
    更新之后的$\vb{B}$矩阵是
    \begin{equation}
        \vb{B}_{\vb{s}'}(\beta, 0) = \vb{B}_{\vb{s}}(\beta, \tau) (1 + \vb{\Delta}^{(i)}) \vb{B}_{\vb{s}}(\tau, 0), \quad \tau = n \Delta \tau.
        \label{eq:updated-b-matrix}
    \end{equation}
    这样更新前后的$\vb{B}$矩阵只需要动一帧就够了,大大简化了计算。
    同时可以计算出接受率为
    \begin{equation}
        \begin{aligned}
            R &= \frac{\det(1 + \vb{B}_{\vb{s}}(\beta, \tau) (1 + \vb{\Delta}^{(i)}) \vb{B}_{\vb{s}}(\tau, 0))}{\det(1 + \vb{B}_{\vb{s}}(\beta, 0))} \\
            &= \det \left( 1 + \vb{\Delta}^{(i)} (1 - \vb{G}_{\vb{s}}(\tau)) \right).
        \end{aligned}
    \end{equation}
    可以看出,接受率实际上完全由$(i, n)$的位置以及等时格林函数确定。

    实际上,我们还可以一并把更新之后的格林函数使用更新之前的格林函数计算出来,这样就避免了在更新后重新计算格林函数而需要连乘$\vb{B}$矩阵。
    我们需要用到公式
    \[
        (\vb{A} + \vb{u} \vb{v}^\top)^{-1} = \vb{A}^{-1} - \frac{\vb{A}^{-1} \vb{u} \vb{v}^\top \vb{A}^{-1}}{1 + \vb{v}^\top \vb{A}^{-1} \vb{u}}.
    \]
    按照\eqref{eq:updated-b-matrix},我们有

    然后再应用\eqref{eq:eq-time-green-function-iter}。% TODO

    然而,即使等时格林函数也不容易计算,因为它还是涉及$\vb{B}$矩阵的连续相乘和取逆。
    为了保证精度,我们必须分析什么地方可能出现数值不稳定性,并使用适当的矩阵算法规避这些不稳定性。
    请注意系统的能谱一般来说是比较宽的,即${H}$最大的本征值和最小的本征值可以差很多,从而每一个$\exp(-\Delta \tau \vb{h}_0) \exp(\vb{h}_\text{I})$的最大和最小的特征值可以相差很多,它们连乘会产生一个条件数特别大的、几乎是奇异的矩阵,因此$\vb{B}$矩阵是非常病态的,稍有误差就会产生很大影响。
    考虑到计算机的精度不可能无限制提高,我们应该尽可能避免计算真的显式计算$\vb{B}$矩阵,而应该使用一些不那么病态的对象代替它。
    至少我们应该把不同的能量尺度分开来考虑,来避免条件数没完没了地增大。

    于是我们先来看一看怎么把“不同的能量尺度”(或者说不同的本征值)分开。
    设$\vb*{M}$是这样一个非常病态的矩阵,记其维数为$N_p \times N_p$,设
    \begin{equation}
        \vb{M} = \pmqty{\vb{v}_1 & \vb{v}_2 & \cdots & \vb{v}_{N_p}},
    \end{equation}
    对它们做Gram-Schimidt正交化,得到$\{\vb{v}_i'\}$,并设两者之间的转换矩阵由下式给出:
    \begin{equation}
        \pmqty{\vb{v}'_1 & \vb{v}'_2 & \cdots & \vb{v}'_{N_p}} = \pmqty{\vb{v}_1 & \vb{v}_2 & \cdots & \vb{v}_{N_p}} \vb{V}_R^{-1}.
    \end{equation}
    容易验证,存在对角矩阵$\vb{D}_R$,使得
    \begin{equation}
        \vb{M} = \underbrace{\pmqty{\vb{v}_1 / \abs*{\vb{v}_1} & \vb{v}_2 / \abs*{\vb{v}_2} & \cdots & \vb{v}_{N_p} / \abs*{\vb{v}_{N_p}}}}_{\vb{U}_R} \vb{D}_R \vb{V}_R,
    \end{equation}
    可以估计,$\vb{U}_R$和$\vb{V}_R$都不非常病态,则$\vb{D}_R$是比较病态的——这就是说,其对角线上的元素大小相差很大。
    于是我们就将$\vb{M}$中病态的部分收集到了一个对角矩阵中,以便妥善处理(例如要尽可能少用它们做任何计算,特别是矩阵连乘)。

    以上整套方法就称为\concept{行列式量子蒙特卡洛(DQMC)}。
    在行列式蒙特卡洛模拟中等时格林函数是核心。一方面它可以用于计算我们想要的所有物理量,一方面它决定了蒙特卡洛模拟的更新:更新前后的等时格林函数有闭形式的关系,并且接受率由等时格林函数确定。
    由于涉及大量矩阵运算,适当的优化对高精度、高效率的DQMC是非常关键的。
    Hubbard-Stratonovich参量的选择同样非常重要。
    
    DQMC的主要弱点是量子蒙特卡洛方法常常具有的符号问题。
\end{back}