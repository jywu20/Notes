\chapter{拓扑能带论}

本节要讨论的系统和\autoref{chap:conventional-metal}的哈密顿量基本上是差不多的:系统的基本自由度是某种电子,可以完全使用能带理论刻画它,相互作用可以忽略。
然而,特殊的拓扑性质会让这些系统展现出和普通的金属、绝缘体非常不同的行为。

能带绝缘体中化学势位于两条能带中间,因此有明确的、彼此之间不连续的价带和导带,换而言之,载流子存在能隙(见\autoref{sec:conductor-classification})。另一方面,导体中载流子不存在能隙。
单电子能谱是否存在能隙也决定了\emph{电子态}具有或者不具有能隙,因为多电子态能够发生的最小偏移就是多一个或者少一个电子。

既然连续地、局域地调节哈密顿量不会改变能隙的有无,我们可以据此定义一个拓扑等价类。
一种比较粗糙的方法是将
所有的传统绝缘体(就是\autoref{chap:conventional-metal}中出现的那些)都可以归入一类,这个类别中也包括真空,因为真空中的电子同样可以认为有一个价带(电子)和一个满带(正电子),两者之间存在能隙。

有能隙系统的低能行为和费米能级以上的能带无关,或者说能带绝缘体中的电子态似乎可以认为是平凡的。
然而,这不是必然的:具有能隙的电子态同样可以因为拓扑而有非平凡的行为。
这就是所谓\concept{拓扑绝缘体}。

\begin{back}{哈密顿量的拓扑分类}{hamiltonian-topological-calssification}
    哈密顿量变化,系统的基态和激发态也会随之变化。哈密顿量的局域、连续的变化\emph{不能}让基态和激发态遍历所有的可能。
    例如,基态和激发态之间是否存在能隙这件事在哈密顿量的连续局域变化下是不会变的。
    这些诸如能隙有无之类的东西可以看成拓扑不变量,据此我们可以对哈密顿量做拓扑分类。
\end{back}

\section{整数量子霍尔效应}

整数量子霍尔效应是另一个拓扑导致的非常新奇的效应。
经典霍尔效应预言的霍尔电导为\eqref{eq:classical-hall-conductivity},于是霍尔电阻正比于$B$。
然而在低温、强磁场下,观察到$\rho$和$B$之间的关系\emph{存在平台}(所谓\concept{电导平台}),稳定时的$\nu$取值为
\[
    \nu = 1, 2, 3, \cdots, \frac{1}{3}, \frac{2}{3}, \frac{2}{5}, \cdots.
\]
$\nu$取整数的情况称为\concept{整数量子霍尔效应},取分数的情况为\concept{分数量子霍尔效应}。

事实证明,整数量子霍尔效应仅仅在拓扑能带理论的框架下即可得到很好的解释,即它实质上还是一个短程纠缠系统。
分数量子霍尔效应则涉及强关联效应,而且存在拓扑序,即有长程纠缠。

\subsection{整数量子霍尔效应的基本现象}

% 图有空画了,先把想法写下来再说

在低温($\sim \SI{1.5}{K}$)、强磁场($\sim \SI{20}{T}$)时对材料做霍尔测量,可能观察到如下的现象:霍尔电阻随着磁场的增强会有增长,但是这个增长不是线性的,而是会出现一系列平台;在霍尔电阻的两个平台交界处,纵向电阻出现高峰,而在平台内部,纵向电阻则是零。
应当注意,这里由于磁场的存在,就电导率矩阵$\sigma$本身而言,时间反演对称性是\emph{破缺}的,即它不是一个厄米的矩阵。
事实上,注意到我们有\autoref{fig:hall-response},电导率张量应该写成
\begin{equation}
    \sigma = \pmqty{\sigma_{xx} & \sigma_{xy} \\ - \sigma_{xy} & \sigma_{xx}}.
\end{equation}
电动力学的论证告诉我们,体系中的能量损耗是
\[
    \vb*{j} \cdot \vb*{E} = \vb*{j} \cdot \sigma^{-1} \cdot \vb*{j},
\]
于是我们可以定义\concept{电阻率张量}
\begin{equation}
    \rho = \sigma^{-1} = \pmqty{\rho_{xx} & \rho_{xy} \\ - \rho_{xy} & \rho_{xx}}, \quad \sigma_{xx} = \frac{\rho_{xx}}{\rho_{xx}^2 + \rho_{xy}^2}, \quad \sigma_{xy} = - \frac{\rho_{xy}}{\rho_{xx}^2 + \rho_{xy}^2}.
    \label{eq:hall-conductivity-r-relation}
\end{equation}
这里有一个看起来很奇怪的地方,就是存在霍尔电阻的时候,$\sigma{xx} = 0$对应着$\rho_{xx} = 0$,但是在普通的导体中,电导率为零应该电阻率为无穷大才对。
实际上,我们可以看到,$\rho_{xx} \to \infty$时确实有$\sigma_{xx} \to 0$。$\sigma{xx} = 0$和$\rho_{xx} = 0$同时成立实际上对应着这样一种情况:由于某些巧妙的安排,材料里面是真的一点电流也没有,从而也没有能耗。

\begin{figure}
    \centering
    

\tikzset{every picture/.style={line width=0.75pt}} %set default line width to 0.75pt        

\begin{tikzpicture}[x=0.75pt,y=0.75pt,yscale=-1,xscale=1]
%uncomment if require: \path (0,300); %set diagram left start at 0, and has height of 300

%Shape: Square [id:dp7594515374681072] 
\draw  [draw opacity=0][fill={rgb, 255:red, 248; green, 231; blue, 28 }  ,fill opacity=0.5 ] (100,121) -- (193,121) -- (193,214) -- (100,214) -- cycle ;
%Shape: Square [id:dp3427181984144141] 
\draw  [draw opacity=0][fill={rgb, 255:red, 248; green, 231; blue, 28 }  ,fill opacity=0.5 ] (347,121) -- (440,121) -- (440,214) -- (347,214) -- cycle ;
\draw   (63.84,80.95) .. controls (67.18,77.6) and (72.6,77.6) .. (75.95,80.95) .. controls (79.29,84.29) and (79.29,89.71) .. (75.95,93.05) .. controls (72.6,96.4) and (67.18,96.4) .. (63.84,93.05) .. controls (60.5,89.71) and (60.5,84.29) .. (63.84,80.95) -- cycle ; \draw   (63.84,80.95) -- (75.95,93.05) ; \draw   (75.95,80.95) -- (63.84,93.05) ;
%Straight Lines [id:da9866503409418597] 
\draw [color={rgb, 255:red, 74; green, 144; blue, 226 }  ,draw opacity=1 ]   (146.5,204.5) -- (146.5,132.5) ;
\draw [shift={(146.5,130.5)}, rotate = 90] [fill={rgb, 255:red, 74; green, 144; blue, 226 }  ,fill opacity=1 ][line width=0.08]  [draw opacity=0] (12,-3) -- (0,0) -- (12,3) -- cycle    ;
%Straight Lines [id:da13595332114098047] 
\draw    (146.5,204.5) -- (86,204.5) ;
\draw [shift={(84,204.5)}, rotate = 360] [fill={rgb, 255:red, 0; green, 0; blue, 0 }  ][line width=0.08]  [draw opacity=0] (12,-3) -- (0,0) -- (12,3) -- cycle    ;
%Straight Lines [id:da5681213668237113] 
\draw [color={rgb, 255:red, 80; green, 227; blue, 194 }  ,draw opacity=1 ]   (146.5,204.5) -- (211,204.5) ;
\draw [shift={(213,204.5)}, rotate = 180] [fill={rgb, 255:red, 80; green, 227; blue, 194 }  ,fill opacity=1 ][line width=0.08]  [draw opacity=0] (12,-3) -- (0,0) -- (12,3) -- cycle    ;
%Straight Lines [id:da9071388501116191] 
\draw [color={rgb, 255:red, 74; green, 144; blue, 226 }  ,draw opacity=1 ]   (354.75,167.5) -- (430.25,167.5) ;
\draw [shift={(432.25,167.5)}, rotate = 180] [fill={rgb, 255:red, 74; green, 144; blue, 226 }  ,fill opacity=1 ][line width=0.08]  [draw opacity=0] (12,-3) -- (0,0) -- (12,3) -- cycle    ;
%Straight Lines [id:da36070135647852175] 
\draw [color={rgb, 255:red, 80; green, 227; blue, 194 }  ,draw opacity=1 ]   (354.75,167.5) -- (354.75,225) ;
\draw [shift={(354.75,227)}, rotate = 270] [fill={rgb, 255:red, 80; green, 227; blue, 194 }  ,fill opacity=1 ][line width=0.08]  [draw opacity=0] (12,-3) -- (0,0) -- (12,3) -- cycle    ;

% Text Node
\draw (89,78) node [anchor=north west][inner sep=0.75pt]    {$\boldsymbol{B}$};
% Text Node
\draw (148.5,127.5) node [anchor=south west] [inner sep=0.75pt]    {$\boldsymbol{j}$};
% Text Node
\draw (84,201.5) node [anchor=south] [inner sep=0.75pt]   [align=left] {Lorentz force};
% Text Node
\draw (215,201.5) node [anchor=south west] [inner sep=0.75pt]    {$\boldsymbol{E}$};
% Text Node
\draw (434.25,164.5) node [anchor=south west] [inner sep=0.75pt]    {$\boldsymbol{j}$};
% Text Node
\draw (356.75,230) node [anchor=north west][inner sep=0.75pt]    {$\boldsymbol{E}$};


\end{tikzpicture}

    \caption{电导率张量的不对称性:在$x$方向加一个电场可以保持一个$y$方向的稳定的霍尔电流,可是需要在$-y$方向加一个电场才能够保持$x$方向的稳定霍尔电流。}
    \label{fig:hall-response}
\end{figure}

\subsection{整数电导平台的定性分析}

\subsubsection{朗道能级}

在\autoref{sec:quantum-magnetic-field}中我们已经考虑过了朗道能级。
设被填充的最高的朗道能级编号为$n$,且填充在朗道能级中的电子总数为$N_\text{e}$,则有
\[
    \frac{N_\text{e}}{2} = (n-1) \frac{\Phi}{\Phi_0} + n_\text{high}, \quad 0 < n_\text{high} \leq \frac{\Phi}{\Phi_0},
\]
于是随着$1/B$上升(即$B$下降),$n$能级上的电子轨道占据数$n_\text{high}$线性上升。
如果我们能够保证电导大体上正比于电子填充数,那么以上机制就给出了电导平台的一种可能成因。以下我们解释为什么。

如果材料是非常理想的,那么其实不应该出现电导平台,因为此时$n$朗道能级被填充之后,立刻会开始填充$n+1$朗道能级,两个朗道能级之间由于不存在任何态,化学势可以发生不连续变化。
然而实际的体系中有杂质,因此朗道能级会出现展宽,并且两个朗道能级之间的态基本上是局域化态,而朗道能级中的态的能量分布则舒展一些(见\autoref{sec:anderson-localization})。
虽然直观地看,朗道能级附近的电子似乎是原地打转,但加入一个电场之后,它们是可以自由移动的,能够形成延展态(比较没有磁场的晶体:价电子也只是出现在原子附近,原地打转,但是形成晶体后,晶格势场还是会让价电子形成延展态)。因此,化学势在它们附近时,它们可以贡献纵向电导,而两个朗道能级之间的电子高度定域,任何时候都无法贡献纵向电导。
这解释了为什么纵向电导(从而,纵向电阻——见\eqref{eq:hall-conductivity-r-relation}附近的讨论)会在特定位置出现峰,因为只有在化学势在朗道能级附近时才可能有纵向电流。
这就造成了一个非常矛盾的情况:要观察到量子霍尔效应,体系必须比较“脏”,这样才能够有明确的定域态,从而形成前述现象,否则,化学势可以从一个朗道能级快速跳转到另一个朗道能级。
但实际上,在体系很脏时朗道能级附近的延展态也被破坏了,因此太脏的体系全然没有纵向电流,观察不到纵向电流随着磁场变化出现的峰。
因此明显的量子霍尔效应需要体系有些脏但又不太脏。

实际上,如果我们暂且认可化学势以下的朗道能级上的电子全部能够对霍尔电导有贡献,我们还能够解释霍尔电阻的行为:
在填充延展态时电导关于$1/B$线性上升,而在填充定域态时电导没有变化,出现平台。
然而我们马上会注意到一个不对头的地方,就是化学势都不在朗道能级上,而是在两个朗道能级之间,即化学势位于能隙之中时,体系是有能隙的,化学势以下的朗道能级被填满,要形成电流需要在这个朗道能级上形成空穴,因此需要消耗有限大的能量。
因此,我们实际上是在要求有能隙的电子模式——朗道能级——产生电流。这真是不可理喻。
但是实际上我们有一种方法绕过这个限制,那就是要求体态有能隙,但边界态没有能隙,从而边界态导电。
如果真的能够形成能导电的边界态,体态有能隙倒不是坏事,因为这意味着体态内部的关联长度有限,因此边界态和体态相对独立。
我们指望通过边界态电流来解释$\sigma_{xy}$的平台行为。

朗道能级确实能够形成稳定的边界态电流。
可以从一个经典图像看到这一点:在体态中电子可以不停做圆周运动,而在边界附近电子做完半个圆周运动后被反弹,而又往前做半个圆周运动。
因此,可以形成绕着整个边界运行的(手性的)电流。边界态上如果有杂质,电子可以潜到体态中,绕过杂质,形成一个新的边界态。因此边界态上的电流是无损耗的。
由于磁场的存在,我们还会发现这个边界态电流是\emph{手征}的(见\eqref{eq:one-dimension-linear-model}后面的讨论)。
现在考虑我们给材料施加一个电场,如\autoref{fig:hall-from-boundary}所示。如果材料在和电场垂直的方向上和外界导通,这会让和电场平行的两条边上的电子运动速度不同,由于电荷守恒,上边的电流要比下边的电流大,其结果就是外界能够测到一个从右往左的净电流。
当然,这就是霍尔电流。

\begin{figure}
    \centering
    

\tikzset{every picture/.style={line width=0.75pt}} %set default line width to 0.75pt        

\begin{tikzpicture}[x=0.75pt,y=0.75pt,yscale=-1,xscale=1]
%uncomment if require: \path (0,300); %set diagram left start at 0, and has height of 300

%Shape: Square [id:dp8110413682744386] 
\draw  [draw opacity=0][fill={rgb, 255:red, 248; green, 231; blue, 28 }  ,fill opacity=0.5 ] (196,90) -- (381,90) -- (381,275) -- (196,275) -- cycle ;
%Straight Lines [id:da548073548119657] 
\draw [color={rgb, 255:red, 80; green, 227; blue, 194 }  ,draw opacity=1 ][line width=1.5]    (288.5,225.25) -- (288.5,143.75) ;
\draw [shift={(288.5,139.75)}, rotate = 90] [fill={rgb, 255:red, 80; green, 227; blue, 194 }  ,fill opacity=1 ][line width=0.08]  [draw opacity=0] (15.6,-3.9) -- (0,0) -- (15.6,3.9) -- cycle    ;
%Rounded Rect [id:dp9220949409043817] 
\draw  [color={rgb, 255:red, 74; green, 144; blue, 226 }  ,draw opacity=1 ] (208,134.2) .. controls (208,116.42) and (222.42,102) .. (240.2,102) -- (336.8,102) .. controls (354.58,102) and (369,116.42) .. (369,134.2) -- (369,230.8) .. controls (369,248.58) and (354.58,263) .. (336.8,263) -- (240.2,263) .. controls (222.42,263) and (208,248.58) .. (208,230.8) -- cycle ;
%Straight Lines [id:da6028379127452343] 
\draw [color={rgb, 255:red, 74; green, 144; blue, 226 }  ,draw opacity=1 ]   (290,102) -- (285,102) ;
\draw [shift={(283,102)}, rotate = 360] [fill={rgb, 255:red, 74; green, 144; blue, 226 }  ,fill opacity=1 ][line width=0.08]  [draw opacity=0] (12,-3) -- (0,0) -- (12,3) -- cycle    ;
%Straight Lines [id:da16991469576308726] 
\draw [color={rgb, 255:red, 74; green, 144; blue, 226 }  ,draw opacity=1 ]   (208,182) -- (208,189) ;
\draw [shift={(208,191)}, rotate = 270] [fill={rgb, 255:red, 74; green, 144; blue, 226 }  ,fill opacity=1 ][line width=0.08]  [draw opacity=0] (12,-3) -- (0,0) -- (12,3) -- cycle    ;
%Straight Lines [id:da0056621577609063944] 
\draw [color={rgb, 255:red, 74; green, 144; blue, 226 }  ,draw opacity=1 ]   (293,263) ;
\draw [shift={(293,263)}, rotate = 180] [fill={rgb, 255:red, 74; green, 144; blue, 226 }  ,fill opacity=1 ][line width=0.08]  [draw opacity=0] (12,-3) -- (0,0) -- (12,3) -- cycle    ;
%Straight Lines [id:da45045547035779476] 
\draw [color={rgb, 255:red, 74; green, 144; blue, 226 }  ,draw opacity=1 ]   (369,188) -- (369,178) ;
\draw [shift={(369,176)}, rotate = 90] [fill={rgb, 255:red, 74; green, 144; blue, 226 }  ,fill opacity=1 ][line width=0.08]  [draw opacity=0] (12,-3) -- (0,0) -- (12,3) -- cycle    ;

% Text Node
\draw (297.75,169) node [anchor=north west][inner sep=0.75pt]    {$\boldsymbol{E}$};
% Text Node
\draw (202.5,112.5) node [anchor=south west] [inner sep=0.75pt]    {$\boldsymbol{j}$};


\end{tikzpicture}

    \caption{霍尔电流如何从边界态电流中形成}
    \label{fig:hall-from-boundary}
\end{figure}

在磁场比较高时,只有几个朗道能级上面填充了电子,从而霍尔电导比较低,霍尔电阻比较高,从而能够容易地观察到量子霍尔效应。

\subsubsection{霍尔电导的形式和Laughlin论证}

然而以上论证无法解释一件事,就是为什么电导平台的值不多不少,就是几个基本物理常数确定的
\[
    \sigma_\text{H} = n \frac{e^2}{h}, \quad n = 0, 1, 2, \ldots.
\]
通常的电导涉及关于材料的复杂性质,是不可能如此简洁的。
\concept{Laughlin论证}是一种直观理解关于为什么霍尔电导形式非常简洁的方法。
设有一个半径为$L$的圆柱体,其体态有能隙而边界态导电。
沿着它的轴向加入一个磁场,总磁通量为$\Phi$。让$\Phi$缓慢地发生变化,从零变化到$\frac{h c}{e}$,则$\Phi$变化前后圆柱体均处于同样的状态,因为绕着大小为$h / e$的磁通量转一圈什么也不会发生。
磁通量的变化会导致一个电场,在边界上,我们有
\[
    2 \pi L E = - \dv{\Phi}{t},
\]
从而
\[
    E(t) = \frac{1}{2\pi L} \dv{\Phi}{t}.
\]
边界上的电场方向垂直轴向,从而产生一个平行于轴向的霍尔电流
\[
    I = 2 \pi L j = 2 \pi L \sigma_\text{H} E = {\sigma_\text{H}} \dv{\Phi}{t}.
\]
这个电流造成的电荷量变化为
\[
    \Delta Q = \int \dd{t} I = {\sigma_\text{H}} \Delta \Phi = \sigma_\text{H} \frac{h}{e},
\]
而由于电荷由电子携带,有
\[
    \Delta Q = me, \quad m = 0, 1, 2, \ldots,
\]
于是
\[
    \sigma_\text{H} = m \frac{e^2}{h}, \quad m = 0, 1, 2, \ldots.
\]
这就是整数阶量子霍尔效应的来源:它和体系的结构完全无关,只要体系体态有能隙,就能够得出存在整数霍尔效应的结论。

\subsection{线性响应理论和拓扑不变量}

以上的定性分析能够解释整数量子霍尔效应,不过我们当然想要一个更加一般的、能够适用于各种类似的系统的解释。



\begin{back}{Berry相位}{berry-phase}
    设含时哈密顿量$H$依赖一系列连续参数$\vb{R}(t) = (R_1(t), \ldots, R_n(t))$,用$\ket{n(\vb{R}(t))}$标记该哈密顿量的本征态。
    假定$\vb{R}(t)$变化得(相比于$H$的能谱的最小能隙)充分缓慢,以至于如果系统以$\ket{n(\vb{R}(t))}$为初态,那么经过一段时间的演化之后不会跃迁到其他本征态上(\concept{绝热演化})。
    
    设系统初态为$\ket{n(\vb{R}(t))}$,经过一段时间后系统状态为
    \begin{equation}
        \ket{\psi_n(t)} = \ee^{\ii \gamma_n(t)} \exp(-\frac{\ii}{\hbar} \int_0^t \dd{t'} E_n(\vb{R}(t'))) \ket{n(\vb{R}(t))},
    \end{equation}
    将上式代入
    \[
        \ii \hbar \dv{t} \ket{\psi_n(t)} = H(\vb{R}(t)) \ket{\psi_n(t)},
    \]
    得到
    \begin{equation}
        \gamma_n(t) = \ii \int_0^t \dd{t'} \mel*{\psi_n(t)}{\dv{t}}{\psi_n(t)} = \ii \int_{\vb{R}(0)}^{\vb{R}(t)} \dd{\vb{R}} \cdot \mel*{\psi_n(t)}{\grad_{\vb{R}}}{\psi_n(t)} .
    \end{equation}
    注意到$\gamma_n(t)$不含有任何$\hbar$,因此实际上这\emph{不是}一个动力学相位,而是所谓的几何相位。
    定义
    \begin{equation}
        \vb{A} = \ii \mel*{\psi_n(t)}{\grad_{\vb{R}}}{\psi_n(t)} 
    \end{equation}
    为\concept{Berry联络},相应的相位称为\concept{Berry相位}。
    这里的底流形是参数空间,其上的协变导数定义要求相邻两个$\vb{R}$对应的态可以通过绝热演化连接。
    Berry联络就是满足这个条件的联络。从Berry联络自然可以得到Berry曲率张量
    \begin{equation}
        \Omega_{n, \mu \nu} = \partial_{R_\mu} \partial_{R_\nu} A_n - \partial_{R_\nu} \partial_{R_\mu} A_n.
    \end{equation}
    
    Berry联络是规范可变的,因为容易验证,如果做变换
    \begin{equation}
        \ket{n(\vb{R})} \longrightarrow \ee^{\ii \zeta(\vb{R})} \ket{n(\vb{R})}, 
    \end{equation}
    就有
    \begin{equation}
        \vb{A}(\vb{R}) \longrightarrow \vb{A}(\vb{R}) - \grad_{\vb{R}} \zeta(\vb{R}).
    \end{equation}
    在时间从$0$演化到$T$的闭合的$\vb{R}(t)$回路上,由于单值性,有
    \[
        \zeta(\vb{R}(T)) - \zeta(\vb{R}(0)) = 2 \pi n, \quad n \in \mathbb{Z},
    \]
    从而积分
    \begin{equation}
        \oint \dd{\vb{R}} \cdot \vb{A}
    \end{equation}
    在模$2\pi$的意义下是规范不变的。

    以上讨论的问题中参数空间都是外加参数,但是实际上也可以用系统内的物理量做参数。
    现在向系统中加入一个变化缓慢的外场,用它“牵引”系统做绝热演化,由于
    因此此时$\mel{g}{\partial_t}{g}$就给出一个Berry相位,参数空间是外场的参数空间,而由于是绝热演化,外场通常会让系统中的某一些物理量非常确定,那么我们就能够用系统中的物理量做参数了。
    例如,可以在空间中假象地引入一个势阱,用它让粒子每一时刻都几乎位于一个确定的位置,则这个势阱的位置——实际上就是粒子的位置——的取值范围(比如说一圆圈)就构成了一个参数空间,其上就可以定义Berry联络。
    A-B效应就是一个典型例子,在那里参数就是粒子位置$\vb*{r}$。
    在拓扑能带论中,参数则可以是晶格动量$\vb*{k}$。

    Berry相位可以体现在路径积分中。回顾各种路径积分的推导方式,都可以找到类似于$\mel{g}{\partial_t}{g}$的项,其中$\ket{g}$通常是某种“相干态”。
    设相干态是关于某些变量$q$的,回顾路径积分的推导过程,$\mel{g}{\partial_t}{g}$项实际上就对应着经典的从哈密顿量到拉格朗日量的勒让德变换中$p \dot{q}$这一项,因此如果有非平庸的Berry相位那么它一定会被收集在这一项中。
    因此拉氏量中的$p \dot{q}$项常常被称为Berry相位项,虽然一般情况下它也会容纳一些动力学演化的信息。
    $q$高度确定的绝热演化过程中,$p$中显含$\dot{q}$的项都可以丢弃,因为演化速度很慢(实际上$\dot{q}$通常和$q$是不对易的,不过无所谓,反正都是要丢弃的),于是$\int \dd{t} p \dot{q}$可以写成$\int p(q) \dd{q}$,此时它就是Berry相位。
\end{back}

\subsection{整数量子霍尔效应的Chern-Simons场论描述}

整数量子霍尔效应实际上可以使用Chern-Simons理论描述\cite{topological_insulator_concepts}。实际上分数量子霍尔效应也和Chern-Simons理论有关,见\autoref{sec:chern-simons-theory-fqhe}。
这样做的好处在于,只要我们能够从得到的Chern-Simons理论中找到电子对应的自由度,那么就可以很容易地分析电子间相互作用比较重要时拓扑绝缘体的各种性质是否还能够维持。

我们要做的事情大体上是这样:既然整数量子霍尔效应中的体态电子有能隙而边界态能产生的物理效应无非是导电,实际上可以通过写出体态中的电磁场的低能有效理论来得到整数量子霍尔效应的低能有效理论。

整数量子霍尔效应还可以被赋予流体动力学意义\cite{Chan_2016}。
% https://www.pks.mpg.de/~esicqw12/Talks_pdf/Fradkin.pdf

\section{Su-Schrieffer-Heeger模型}

\concept{Su-Schrieffer-Heeger(SSH)模型}是一个定义在一维格子上的紧束缚模型,在这个一维格子中有A和B两套子格\cite{Asb_th_2016}。
与之前不同,我们取开放边界条件,SSH模型为
\begin{equation}
    H = v \sum_{m=1}^N (c^\dagger_{m \text{A}} c_{m \text{B}} + \text{h.c.}) + w \sum_{m=1}^{N-1} (c^\dagger_{m+1, \text{A}} c_{m \text{B}} +  \text{h.c.}).
    \label{eq:ssh-hamiltonian}
\end{equation}
我们忽略了自旋指标,即只考虑一种自旋的电子,忽略自旋翻转的可能性;使用实际的分子实现SSH模型时,会得到两份SSH模型,我们只考虑其中一份。

在分析SSH模型时通常将电子在A、B子格上的哪一个看成内禀自由度而非坐标自由度或者说外部自由度,这实际上就是紧束缚模型中把能带标签当成抽象的一个指标而忽略它对电子空间分布的影响,而将$\vb*{i}$看成完全代表了电子空间位置的标签的做法。
在这种约定下,\eqref{eq:ssh-hamiltonian}可以写成
\begin{equation}
    H = v \sum_{m=1}^N \dyad{m} \otimes \sigma^x + w \sum_{m=1}^{N-1} \dyad{m+1}{m} \otimes \frac{\sigma^x + \ii \sigma^y}{2} + \text{h.c.}. 
    \label{eq:ssh-pseudospin-hamiltonian}
\end{equation}
这里算符$\sigma^{x, y, z}$作用于A,B两套子格张成的二维空间上——或者说对应的两条能带的能带编号张成的二维空间上(注意能带编号和子格编号未必是同样的表象)。

\begin{figure}
    \centering
    \documentclass[hyperref, UTF8, a4paper]{ctexart}

\usepackage{geometry}
\usepackage{titling}
\usepackage{titlesec}
\usepackage{paralist}
\usepackage{footnote}
\usepackage{enumerate}
\usepackage{amsmath, amssymb, amsthm}
\usepackage{bbm}
\usepackage{cite}
\usepackage{graphicx}
\usepackage{subfigure}
\usepackage{physics}
\usepackage{tikz}
\usepackage{autobreak}
\usepackage[ruled, vlined, linesnumbered, noend]{algorithm2e}
\usepackage[colorlinks, linkcolor=black, anchorcolor=black, citecolor=black]{hyperref}
\usepackage{prettyref}

% Page style
\geometry{left=3.18cm,right=3.18cm,top=2.54cm,bottom=2.54cm}
\titlespacing{\paragraph}{0pt}{1pt}{10pt}[20pt]
\setlength{\droptitle}{-5em}
\preauthor{\vspace{-10pt}\begin{center}}
\postauthor{\par\end{center}}

% Math operators
\DeclareMathOperator{\timeorder}{T}
\DeclareMathOperator{\diag}{diag}
\DeclareMathOperator{\legpoly}{P}
\DeclareMathOperator{\primevalue}{P}
\DeclareMathOperator{\sgn}{sgn}
\newcommand*{\ii}{\mathrm{i}}
\newcommand*{\ee}{\mathrm{e}}
\newcommand*{\const}{\mathrm{const}}
\newcommand*{\comment}{\paragraph{注记}}
\newcommand*{\suchthat}{\quad \text{s.t.} \quad}
\newcommand*{\argmin}{\arg\min}
\newcommand*{\argmax}{\arg\max}
\newcommand*{\normalorder}[1]{: #1 :}
\newcommand*{\pair}[1]{\langle #1 \rangle}
\newcommand*{\fd}[1]{\mathcal{D} #1}
\DeclareMathOperator{\bigO}{\mathcal{O}}

% prettyref setting
\newrefformat{sec}{第\ref{#1}节}
\newrefformat{note}{注\ref{#1}}
\newrefformat{fig}{图\ref{#1}}
\newrefformat{alg}{算法\ref{#1}}
\renewcommand{\autoref}{\prettyref}

% TikZ setting
\usetikzlibrary{arrows,shapes,positioning}
\usetikzlibrary{arrows.meta}
\usetikzlibrary{decorations.markings}
\tikzstyle arrowstyle=[scale=1]
\tikzstyle directed=[postaction={decorate,decoration={markings,
    mark=at position .5 with {\arrow[arrowstyle]{stealth}}}}]
\tikzstyle ray=[directed, thick]
\tikzstyle dot=[anchor=base,fill,circle,inner sep=1pt]

% Algorithm setting
\renewcommand{\algorithmcfname}{算法}
% Python-style code
\SetKwIF{If}{ElseIf}{Else}{if}{:}{elif:}{else:}{}
\SetKwFor{For}{for}{:}{}
\SetKwFor{While}{while}{:}{}
\SetKwInput{KwData}{输入}
\SetKwInput{KwResult}{输出}
\SetArgSty{textnormal}

\renewcommand{\emph}[1]{\textbf{#1}}
\newcommand*{\concept}[1]{\underline{\textbf{#1}}}
\newcommand*{\Ztwo}{$\mathbb{Z}_2$}

\title{常见格点模型}
\author{吴何友}

\begin{document}

\maketitle

\section{相互作用体系}

\subsection{Hubbard模型}

\begin{equation}
    \hat{H} = - t \sum_{\pair{i, j}, \sigma} \hat{c}^\dagger_{i \sigma} \hat{c}_{j \sigma} + U \sum_i \hat{n}_\uparrow \hat{n}_\downarrow,
\end{equation}

\section{磁场}

将电子和一个满足库伦规范的磁矢势$\vb*{A}$耦合,那么会出现动量的一个修正,这个修在在波函数上引入如下的相位变化:
\begin{equation}
    \theta = \int \dd{\vb*{l}} \cdot \vb*{A}.
\end{equation}
在格点模型中,电子仅仅出现在格点上。我们知道紧束缚模型的哈密顿量(即跃迁项)实际上就是动能,因此加入磁场意味着紧束缚模型的$t_{ij}$出现变化,考虑相位变化,则磁场会导致以下修正:
\begin{equation}
    t_{ij} \longrightarrow \ee^{\ii e \int_j^i \dd{\vb*{l}} \cdot \vb*{A} } t_{ij}.
\end{equation}
相应的,设一个格点上的闭合路径为$C$,通过它的磁通量为$\Phi$,则
\begin{equation}
    \ee^{\ii \Phi} = \prod_{C} t_{ij}.
\end{equation}

\end{document}
    \caption{SSH模型所在的格点,红色和蓝色圆圈分别代表A子格和B子格,只有相邻格点之间有跃迁}
\end{figure}

\subsection{体态}

\subsubsection{能谱}

一个有限大小的一维格子的体态是其内部,而边界则是两个点。本节分析SSH模型的体态。取热力学极限$N \to \infty$并施加周期性边界条件,可以求解得到普通的能带。
做傅里叶变换
\begin{equation}
    c_{m \alpha}^\dagger = \frac{1}{\sqrt{N}} \sum_k \ee^{- \ii k m a} c_{k \alpha}^\dagger, \quad \alpha = \text{A}, \text{B},
\end{equation}
得到
\begin{equation}
    H = \sum_k \pmqty{c^\dagger_{k \text{A}} & c^\dagger_{k \text{B}}} \pmqty{ 0 & v + w \ee^{- \ii k a} \\ v + w \ee^{\ii k a} & 0 } \pmqty{ c_{k \text{A}} \\ c_{k \text{B}} },
\end{equation}
对角化给出
\begin{equation}
    \epsilon_{k} = \pm \sqrt{ v^2 + w^2 + 2 wv \cos(ka) }.
    \label{eq:ssh-energy-band}
\end{equation}
两条能带之间的间距为
\begin{equation}
    2 \Delta = \abs{u - v}.
\end{equation}
容易看出在化学势为零时电子半填充,且电子和空穴的能隙(相对于费米面的距离)均为$\Delta$。

在$u=v$时我们得到通常的最为简单的紧束缚模型,系统是导体,因为载流子能隙为零;其余情况下,系统为绝缘体。
换而言之,跃迁系数的交错排列(staggering)让SSH模型的体态是绝缘体。

\subsubsection{动量空间哈密顿量和卷绕数}

在\eqref{eq:ssh-pseudospin-hamiltonian}中我们看到SSH模型的哈密顿量正比于作用于A,B两套子格张成的二维希尔伯特空间上的$\sigma$矩阵,从而动量空间中的哈密顿量可以写成
\begin{equation}
    H_k = \sum_{k} (h_0(k) \sigma^0 + h_i(k) \sigma^i),
\end{equation}
这里我们以子格编号而不是能带编号为表象。显然,二能带模型都能够写成这种形式,因为$\mathbb{C}^2$中的算符一定能够写成四个$\sigma$矩阵的线性组合。
在SSH模型中$h_0 = 0$,将\eqref{eq:ssh-hamiltonian}的坐标部分对角化,可以计算得到
\begin{equation}
    h_x(k) = v + w \cos(ka), \quad h_y(k) = w \sin(ka), \quad h_z(k) = 0.
    \label{eq:ssh-band-curve}
\end{equation}

对每个一维的二能带模型,我们都有一个这样的关系:
\begin{equation}
    k \mapsto (h_x(k), h_y(k), h_z(k)) : \mathbb{R} / 2 \pi \to \mathbb{R}^3.
    \label{eq:k-and-h-insulator}
\end{equation}
由于第一布里渊区的周期性,$k = 0$和$k = 2\pi$处,各个$h$值应该是一样的,即\eqref{eq:k-and-h-insulator}。
如果我们在处理一个体态为绝缘体的系统,还不能有一个$k$点让所有$h$都是零,否则载流子存在能隙,就是导体了。
这样\eqref{eq:k-and-h-insulator}就是一个$\mathbb{R}^3$中的闭合曲线,并且不经过原点。

对这条曲线我们很自然地会想到要计算卷绕数。对\eqref{eq:ssh-band-curve},$w > v$时卷绕数为1,而$w < v$时卷绕数为零。
这两种参数选取的交界点满足$w = v$,此时SSH模型体态为导体。
更一般的,对曲线\eqref{eq:k-and-h-insulator}的小幅变动不会改变卷绕数:要改变卷绕数,必须对$v$和$w$做有限大小的调整,让曲线\eqref{eq:k-and-h-insulator}的其中一段通过原点。
换而言之,$(v, w)$构成的参数空间中测度非零的区域都是绝缘体,但是这些区域中的卷绕数可能是不同的;不同卷绕数的区域的交界线一定是导体。
这给出了一种新型相变的机制:拓扑量子相变。

然后我们马上会发现这样一个事实:宏观下我们可以同时有比较确定的坐标和动量,例如我们可以讨论坐标$\vb*{r}$处的固体微团的能带结构,这样,实际上可以构造一个系统,其中每个固体微团都是一个SSH模型,其$v$和$w$是空间坐标$\vb*{r}$的函数,如果一个区域中$v < w$而另一个区域中$v > w$,那么这两个区域的交界处——如果有的话——一定是导体。
现实中我们就有这样的模型——一个有限大小的固体,如果其体态中有非平凡卷绕数,则它和空气接触的边界必定是导体!

这意味着对体态中的电子能带有非平凡的卷绕数的二能带绝缘体,其边界上一定有一些和体态中不同的无能隙电子模式。
我们在下一节中将讨论,SSH模型中的这种边界态到底是怎么来的。

\subsubsection{电子的二聚体}

\begin{figure}
    \centering
    \subfigure[$v=1, w= 0$,没有边界态]{
        

\tikzset{every picture/.style={line width=0.75pt}} %set default line width to 0.75pt        

\begin{tikzpicture}[x=0.75pt,y=0.75pt,yscale=-1,xscale=1]
%uncomment if require: \path (0,300); %set diagram left start at 0, and has height of 300

%Straight Lines [id:da9339444330477464] 
\draw [line width=1.5]    (87.6,148.94) -- (121.11,128.45) ;
%Shape: Circle [id:dp9507441294855312] 
\draw  [color={rgb, 255:red, 208; green, 2; blue, 27 }  ,draw opacity=1 ][fill={rgb, 255:red, 255; green, 255; blue, 255 }  ,fill opacity=1 ] (77.83,150.14) .. controls (77.72,144.74) and (82.01,140.27) .. (87.41,140.15) .. controls (92.82,140.04) and (97.29,144.33) .. (97.4,149.73) .. controls (97.51,155.13) and (93.23,159.61) .. (87.82,159.72) .. controls (82.42,159.83) and (77.95,155.54) .. (77.83,150.14) -- cycle ;
%Shape: Circle [id:dp8643130963730341] 
\draw  [color={rgb, 255:red, 74; green, 144; blue, 226 }  ,draw opacity=1 ][fill={rgb, 255:red, 255; green, 255; blue, 255 }  ,fill opacity=1 ] (111.32,128.65) .. controls (111.21,123.25) and (115.5,118.78) .. (120.9,118.67) .. controls (126.31,118.55) and (130.78,122.84) .. (130.89,128.24) .. controls (131.01,133.65) and (126.72,138.12) .. (121.31,138.23) .. controls (115.91,138.35) and (111.44,134.06) .. (111.32,128.65) -- cycle ;
%Straight Lines [id:da5498324199028166] 
\draw [line width=1.5]    (152.62,149.57) -- (186.14,129.09) ;
%Shape: Circle [id:dp4328369196955488] 
\draw  [color={rgb, 255:red, 208; green, 2; blue, 27 }  ,draw opacity=1 ][fill={rgb, 255:red, 255; green, 255; blue, 255 }  ,fill opacity=1 ] (142.84,149.78) .. controls (142.73,144.38) and (147.02,139.9) .. (152.42,139.79) .. controls (157.82,139.68) and (162.29,143.97) .. (162.41,149.37) .. controls (162.52,154.77) and (158.23,159.25) .. (152.83,159.36) .. controls (147.43,159.47) and (142.95,155.18) .. (142.84,149.78) -- cycle ;
%Shape: Circle [id:dp3960559998743858] 
\draw  [color={rgb, 255:red, 74; green, 144; blue, 226 }  ,draw opacity=1 ][fill={rgb, 255:red, 255; green, 255; blue, 255 }  ,fill opacity=1 ] (176.35,129.29) .. controls (176.24,123.89) and (180.53,119.42) .. (185.93,119.3) .. controls (191.33,119.19) and (195.81,123.48) .. (195.92,128.88) .. controls (196.03,134.29) and (191.74,138.76) .. (186.34,138.87) .. controls (180.94,138.98) and (176.47,134.7) .. (176.35,129.29) -- cycle ;
%Straight Lines [id:da7739710572922798] 
\draw [line width=1.5]    (219.61,148.17) -- (253.12,127.68) ;
%Shape: Circle [id:dp7636608333524304] 
\draw  [color={rgb, 255:red, 208; green, 2; blue, 27 }  ,draw opacity=1 ][fill={rgb, 255:red, 255; green, 255; blue, 255 }  ,fill opacity=1 ] (207.87,150.42) .. controls (207.75,145.01) and (212.04,140.54) .. (217.45,140.43) .. controls (222.85,140.32) and (227.32,144.6) .. (227.43,150.01) .. controls (227.55,155.41) and (223.26,159.88) .. (217.86,160) .. controls (212.45,160.11) and (207.98,155.82) .. (207.87,150.42) -- cycle ;
%Shape: Circle [id:dp34118834977139567] 
\draw  [color={rgb, 255:red, 74; green, 144; blue, 226 }  ,draw opacity=1 ][fill={rgb, 255:red, 255; green, 255; blue, 255 }  ,fill opacity=1 ] (243.34,127.89) .. controls (243.22,122.49) and (247.51,118.01) .. (252.92,117.9) .. controls (258.32,117.79) and (262.79,122.08) .. (262.9,127.48) .. controls (263.02,132.88) and (258.73,137.35) .. (253.33,137.47) .. controls (247.92,137.58) and (243.45,133.29) .. (243.34,127.89) -- cycle ;
%Straight Lines [id:da34572467796985484] 
\draw [color={rgb, 255:red, 0; green, 0; blue, 0 }  ,draw opacity=1 ][fill={rgb, 255:red, 208; green, 2; blue, 27 }  ,fill opacity=1 ][line width=1.5]    (351.64,148.81) -- (385.15,128.32) ;
%Shape: Circle [id:dp0427927859835342] 
\draw  [color={rgb, 255:red, 208; green, 2; blue, 27 }  ,draw opacity=1 ][fill={rgb, 255:red, 255; green, 255; blue, 255 }  ,fill opacity=1 ] (341.85,149.02) .. controls (341.74,143.61) and (346.03,139.14) .. (351.43,139.03) .. controls (356.83,138.91) and (361.31,143.2) .. (361.42,148.61) .. controls (361.53,154.01) and (357.24,158.48) .. (351.84,158.59) .. controls (346.44,158.71) and (341.97,154.42) .. (341.85,149.02) -- cycle ;
%Shape: Circle [id:dp133324728312036] 
\draw  [color={rgb, 255:red, 74; green, 144; blue, 226 }  ,draw opacity=1 ][fill={rgb, 255:red, 255; green, 255; blue, 255 }  ,fill opacity=1 ] (375.36,128.53) .. controls (375.25,123.12) and (379.54,118.65) .. (384.94,118.54) .. controls (390.35,118.43) and (394.82,122.71) .. (394.93,128.12) .. controls (395.05,133.52) and (390.76,137.99) .. (385.35,138.11) .. controls (379.95,138.22) and (375.48,133.93) .. (375.36,128.53) -- cycle ;
%Straight Lines [id:da8224850866746714] 
\draw [line width=1.5]    (416.66,149.45) -- (450.18,128.96) ;
%Shape: Circle [id:dp741311466978223] 
\draw  [color={rgb, 255:red, 208; green, 2; blue, 27 }  ,draw opacity=1 ][fill={rgb, 255:red, 255; green, 255; blue, 255 }  ,fill opacity=1 ] (406.88,149.65) .. controls (406.77,144.25) and (411.06,139.78) .. (416.46,139.66) .. controls (421.86,139.55) and (426.33,143.84) .. (426.45,149.24) .. controls (426.56,154.65) and (422.27,159.12) .. (416.87,159.23) .. controls (411.47,159.35) and (406.99,155.06) .. (406.88,149.65) -- cycle ;
%Shape: Circle [id:dp4961341695350736] 
\draw  [color={rgb, 255:red, 74; green, 144; blue, 226 }  ,draw opacity=1 ][fill={rgb, 255:red, 255; green, 255; blue, 255 }  ,fill opacity=1 ] (440.39,129.17) .. controls (440.28,123.76) and (444.57,119.29) .. (449.97,119.18) .. controls (455.37,119.06) and (459.85,123.35) .. (459.96,128.76) .. controls (460.07,134.16) and (455.78,138.63) .. (450.38,138.74) .. controls (444.98,138.86) and (440.51,134.57) .. (440.39,129.17) -- cycle ;
%Rounded Rect [id:dp9434888839128048] 
\draw   (74.85,157.53) .. controls (70.44,150.62) and (72.47,141.45) .. (79.37,137.03) -- (113.36,115.34) .. controls (120.26,110.92) and (129.44,112.95) .. (133.85,119.86) -- (133.85,119.86) .. controls (138.26,126.76) and (136.24,135.94) .. (129.33,140.35) -- (95.35,162.05) .. controls (88.44,166.46) and (79.26,164.44) .. (74.85,157.53) -- cycle ;
%Rounded Rect [id:dp17221804627544235] 
\draw   (139.85,157.53) .. controls (135.44,150.62) and (137.47,141.45) .. (144.37,137.03) -- (178.36,115.34) .. controls (185.26,110.92) and (194.44,112.95) .. (198.85,119.86) -- (198.85,119.86) .. controls (203.26,126.76) and (201.24,135.94) .. (194.33,140.35) -- (160.35,162.05) .. controls (153.44,166.46) and (144.26,164.44) .. (139.85,157.53) -- cycle ;
%Rounded Rect [id:dp3726670045223497] 
\draw   (205.85,157.53) .. controls (201.44,150.62) and (203.47,141.45) .. (210.37,137.03) -- (244.36,115.34) .. controls (251.26,110.92) and (260.44,112.95) .. (264.85,119.86) -- (264.85,119.86) .. controls (269.26,126.76) and (267.24,135.94) .. (260.33,140.35) -- (226.35,162.05) .. controls (219.44,166.46) and (210.26,164.44) .. (205.85,157.53) -- cycle ;
%Rounded Rect [id:dp9826226372275961] 
\draw   (338.85,157.53) .. controls (334.44,150.62) and (336.47,141.45) .. (343.37,137.03) -- (377.36,115.34) .. controls (384.26,110.92) and (393.44,112.95) .. (397.85,119.86) -- (397.85,119.86) .. controls (402.26,126.76) and (400.24,135.94) .. (393.33,140.35) -- (359.35,162.05) .. controls (352.44,166.46) and (343.26,164.44) .. (338.85,157.53) -- cycle ;
%Rounded Rect [id:dp7285905275692397] 
\draw   (403.85,157.53) .. controls (399.44,150.62) and (401.47,141.45) .. (408.37,137.03) -- (442.36,115.34) .. controls (449.26,110.92) and (458.44,112.95) .. (462.85,119.86) -- (462.85,119.86) .. controls (467.26,126.76) and (465.24,135.94) .. (458.33,140.35) -- (424.35,162.05) .. controls (417.44,166.46) and (408.26,164.44) .. (403.85,157.53) -- cycle ;

% Text Node
\draw (288,130.4) node [anchor=north west][inner sep=0.75pt]    {$\cdots $};
% Text Node
\draw (71,174.4) node [anchor=north west][inner sep=0.75pt]    {$m=1$};
% Text Node
\draw (141,174.4) node [anchor=north west][inner sep=0.75pt]    {$m=2$};
% Text Node
\draw (405,173.4) node [anchor=north west][inner sep=0.75pt]    {$m=N$};


\end{tikzpicture}
 
        \label{fig:ssh-model-trivial-limit}  
    }
    \subfigure[$w = 1, v = 0$,有边界态]{
        

\tikzset{every picture/.style={line width=0.75pt}} %set default line width to 0.75pt        

\begin{tikzpicture}[x=0.75pt,y=0.75pt,yscale=-1,xscale=1]
%uncomment if require: \path (0,300); %set diagram left start at 0, and has height of 300

%Rounded Rect [id:dp9215995225762108] 
\draw   (182.36,184.85) .. controls (186.78,177.94) and (184.75,168.76) .. (177.84,164.35) -- (143.86,142.65) .. controls (136.95,138.24) and (127.78,140.27) .. (123.37,147.18) -- (123.37,147.18) .. controls (118.96,154.08) and (120.98,163.26) .. (127.89,167.67) -- (161.87,189.37) .. controls (168.78,193.78) and (177.95,191.76) .. (182.36,184.85) -- cycle ;
%Rounded Rect [id:dp004547789490766618] 
\draw   (247.39,185.49) .. controls (251.8,178.58) and (249.78,169.4) .. (242.87,164.99) -- (208.89,143.29) .. controls (201.98,138.88) and (192.81,140.91) .. (188.39,147.81) -- (188.39,147.81) .. controls (183.98,154.72) and (186.01,163.9) .. (192.92,168.31) -- (226.9,190.01) .. controls (233.81,194.42) and (242.98,192.39) .. (247.39,185.49) -- cycle ;
%Straight Lines [id:da4395270133325633] 
\draw [line width=1.5]    (137.11,155.45) -- (168.62,176.57) ;
%Straight Lines [id:da5748378659869535] 
\draw [line width=1.5]    (202.14,156.09) -- (233.65,177.21) ;
%Straight Lines [id:da8933616909770259] 
\draw [line width=1.5]    (269.12,154.68) -- (300.64,175.81) ;
%Straight Lines [id:da1313579068666484] 
\draw [line width=1.5]    (399.18,155.96) -- (430.69,177.09) ;
%Shape: Circle [id:dp06218610801583213] 
\draw  [color={rgb, 255:red, 208; green, 2; blue, 27 }  ,draw opacity=1 ][fill={rgb, 255:red, 255; green, 255; blue, 255 }  ,fill opacity=1 ] (93.83,177.14) .. controls (93.72,171.74) and (98.01,167.27) .. (103.41,167.15) .. controls (108.82,167.04) and (113.29,171.33) .. (113.4,176.73) .. controls (113.51,182.13) and (109.23,186.61) .. (103.82,186.72) .. controls (98.42,186.83) and (93.95,182.54) .. (93.83,177.14) -- cycle ;
%Shape: Circle [id:dp37903459495294767] 
\draw  [color={rgb, 255:red, 74; green, 144; blue, 226 }  ,draw opacity=1 ][fill={rgb, 255:red, 255; green, 255; blue, 255 }  ,fill opacity=1 ] (127.32,155.65) .. controls (127.21,150.25) and (131.5,145.78) .. (136.9,145.67) .. controls (142.31,145.55) and (146.78,149.84) .. (146.89,155.24) .. controls (147.01,160.65) and (142.72,165.12) .. (137.31,165.23) .. controls (131.91,165.35) and (127.44,161.06) .. (127.32,155.65) -- cycle ;
%Shape: Circle [id:dp16465523379151348] 
\draw  [color={rgb, 255:red, 208; green, 2; blue, 27 }  ,draw opacity=1 ][fill={rgb, 255:red, 255; green, 255; blue, 255 }  ,fill opacity=1 ] (158.84,176.78) .. controls (158.73,171.38) and (163.02,166.9) .. (168.42,166.79) .. controls (173.82,166.68) and (178.29,170.97) .. (178.41,176.37) .. controls (178.52,181.77) and (174.23,186.25) .. (168.83,186.36) .. controls (163.43,186.47) and (158.95,182.18) .. (158.84,176.78) -- cycle ;
%Shape: Circle [id:dp8930426798219266] 
\draw  [color={rgb, 255:red, 74; green, 144; blue, 226 }  ,draw opacity=1 ][fill={rgb, 255:red, 255; green, 255; blue, 255 }  ,fill opacity=1 ] (192.35,156.29) .. controls (192.24,150.89) and (196.53,146.42) .. (201.93,146.3) .. controls (207.33,146.19) and (211.81,150.48) .. (211.92,155.88) .. controls (212.03,161.29) and (207.74,165.76) .. (202.34,165.87) .. controls (196.94,165.98) and (192.47,161.7) .. (192.35,156.29) -- cycle ;
%Shape: Circle [id:dp1747805400646203] 
\draw  [color={rgb, 255:red, 208; green, 2; blue, 27 }  ,draw opacity=1 ][fill={rgb, 255:red, 255; green, 255; blue, 255 }  ,fill opacity=1 ] (223.87,177.42) .. controls (223.75,172.01) and (228.04,167.54) .. (233.45,167.43) .. controls (238.85,167.32) and (243.32,171.6) .. (243.43,177.01) .. controls (243.55,182.41) and (239.26,186.88) .. (233.86,187) .. controls (228.45,187.11) and (223.98,182.82) .. (223.87,177.42) -- cycle ;
%Shape: Circle [id:dp5019762571183983] 
\draw  [color={rgb, 255:red, 74; green, 144; blue, 226 }  ,draw opacity=1 ][fill={rgb, 255:red, 255; green, 255; blue, 255 }  ,fill opacity=1 ] (259.34,154.89) .. controls (259.22,149.49) and (263.51,145.01) .. (268.92,144.9) .. controls (274.32,144.79) and (278.79,149.08) .. (278.9,154.48) .. controls (279.02,159.88) and (274.73,164.35) .. (269.33,164.47) .. controls (263.92,164.58) and (259.45,160.29) .. (259.34,154.89) -- cycle ;
%Shape: Circle [id:dp6707626137558458] 
\draw  [color={rgb, 255:red, 208; green, 2; blue, 27 }  ,draw opacity=1 ][fill={rgb, 255:red, 255; green, 255; blue, 255 }  ,fill opacity=1 ] (290.85,176.02) .. controls (290.74,170.61) and (295.03,166.14) .. (300.43,166.03) .. controls (305.83,165.91) and (310.31,170.2) .. (310.42,175.61) .. controls (310.53,181.01) and (306.24,185.48) .. (300.84,185.59) .. controls (295.44,185.71) and (290.97,181.42) .. (290.85,176.02) -- cycle ;
%Shape: Circle [id:dp014513623695553424] 
\draw  [color={rgb, 255:red, 74; green, 144; blue, 226 }  ,draw opacity=1 ][fill={rgb, 255:red, 255; green, 255; blue, 255 }  ,fill opacity=1 ] (389.39,156.17) .. controls (389.28,150.76) and (393.57,146.29) .. (398.97,146.18) .. controls (404.37,146.06) and (408.85,150.35) .. (408.96,155.76) .. controls (409.07,161.16) and (404.78,165.63) .. (399.38,165.74) .. controls (393.98,165.86) and (389.51,161.57) .. (389.39,156.17) -- cycle ;
%Shape: Circle [id:dp17073024403224846] 
\draw  [color={rgb, 255:red, 208; green, 2; blue, 27 }  ,draw opacity=1 ][fill={rgb, 255:red, 255; green, 255; blue, 255 }  ,fill opacity=1 ] (420.91,177.29) .. controls (420.79,171.89) and (425.08,167.42) .. (430.49,167.3) .. controls (435.89,167.19) and (440.36,171.48) .. (440.48,176.88) .. controls (440.59,182.29) and (436.3,186.76) .. (430.9,186.87) .. controls (425.49,186.98) and (421.02,182.7) .. (420.91,177.29) -- cycle ;
%Shape: Circle [id:dp7137518684766175] 
\draw  [color={rgb, 255:red, 74; green, 144; blue, 226 }  ,draw opacity=1 ][fill={rgb, 255:red, 255; green, 255; blue, 255 }  ,fill opacity=1 ] (454.42,156.8) .. controls (454.31,151.4) and (458.6,146.93) .. (464,146.82) .. controls (469.4,146.7) and (473.87,150.99) .. (473.99,156.39) .. controls (474.1,161.8) and (469.81,166.27) .. (464.41,166.38) .. controls (459.01,166.5) and (454.53,162.21) .. (454.42,156.8) -- cycle ;
%Rounded Rect [id:dp5471282050025372] 
\draw   (314.38,184.08) .. controls (318.79,177.18) and (316.76,168) .. (309.86,163.59) -- (275.88,141.89) .. controls (268.97,137.48) and (259.79,139.5) .. (255.38,146.41) -- (255.38,146.41) .. controls (250.97,153.32) and (252.99,162.5) .. (259.9,166.91) -- (293.88,188.6) .. controls (300.79,193.02) and (309.97,190.99) .. (314.38,184.08) -- cycle ;
%Rounded Rect [id:dp6574130908699356] 
\draw   (444.43,185.36) .. controls (448.84,178.45) and (446.82,169.28) .. (439.91,164.86) -- (405.93,143.17) .. controls (399.02,138.76) and (389.85,140.78) .. (385.44,147.69) -- (385.44,147.69) .. controls (381.02,154.6) and (383.05,163.77) .. (389.96,168.18) -- (423.94,189.88) .. controls (430.85,194.29) and (440.02,192.27) .. (444.43,185.36) -- cycle ;

% Text Node
\draw (91,194.4) node [anchor=north west][inner sep=0.75pt]    {$m=1$};
% Text Node
\draw (161,194.4) node [anchor=north west][inner sep=0.75pt]    {$m=2$};
% Text Node
\draw (425,193.4) node [anchor=north west][inner sep=0.75pt]    {$m=N$};
% Text Node
\draw (337,159.4) node [anchor=north west][inner sep=0.75pt]    {$\cdots $};


\end{tikzpicture}

        \label{fig:ssh-model-topological-limit}
    }
    \caption{SSH模型的两个极限}
\end{figure}

我们考虑两个极限,$w = 1, v = 0$和$v = 1, w = 0$。$v = 1, w = 0$时卷绕数为0,哈密顿量为
\begin{equation}
    H = v \sum_{m=1}^N \dyad{m} \otimes \sigma^x ,
\end{equation}
即不同$m$之间没有任何跃迁,但是电子能够在同一个$m$的两个子格点之间跃迁。换而言之,每个$m$的两个子格点之间“成键”,形成电子二聚体。
对角化某个$m$就是要将$\sigma^x$在$\sigma^z$表象下对角化,低能本征态为$(\ket*{\uparrow} - \ket*{\downarrow}) / \sqrt{2}$,或者用A和B来编号就是$(\ket*{\text{A}} - \ket*{\text{B}}) / \sqrt{2}$,从而$v = 1, w = 0$的SSH模型的基态就是
\begin{equation}
    \ket*{\text{ground}} = \otimes_{m} \frac{1}{\sqrt{2}} (\ket*{m \text{A}} - \ket*{m \text{B}}),
\end{equation}
画成图就是\autoref{fig:ssh-model-trivial-limit}。
可以看到这种情况下不存在任何非平庸的边界态:位于边界上的格点被“吸收”进了体态中的电子二聚体中。

另一方面,$w = 1, v = 0$时哈密顿量为
\begin{equation}
    H = \sum_{m=1}^{N-1} \dyad{m+1}{m} \otimes \frac{\sigma^x + \ii \sigma^y}{2} + \text{h.c.},
\end{equation}
此时仔细观察,会发现只有$m \text{B}$和$m+1, \text{A}$之间能够出现跃迁,因此同样会出现电子二聚体态,但是将它画成图,是\autoref{fig:ssh-model-topological-limit}。
此时只能够形成$N-1$个二聚体,$m=1, \text{A}$和$m=N, \text{B}$两个态不属于任何二聚体。
由于它们在边界上,此时存在边界态。

通过\eqref{eq:ssh-energy-band},我们发现无论是$v=1, w=0$还是$w=1,v=0$时电子能带都是平带,两者的体态行为看上去完全一样。
然而,$w=1, v=0$时存在边界态,而$v=1, w=0$时没有边界态。
我们之间已经说明过,$w=1, v=0$时体态中有非零卷绕数,从而的确应该有边界态,因此所有东西都符合得很好。

\subsection{边界态}

\subsection{对称性保护}

下面我们要分析在什么扰动下SSH模型的拓扑性质会完全消失。

\section{一维Kitaev链拓扑超导体}

\subsection{Kitaev链及其解析解}

以下一维模型称为\concept{Kitaev链}:
\begin{equation}
    {H} = - t \sum_{i} ({c}_{i}^\dagger {c}_{i+1} + \text{h.c.}) - \mu \sum_{i} {c}_{i}^\dagger {c}_{i} + \sum_{i} (\Delta {c}_{i} {c}_{i+1} + \text{h.c.} ).
    \label{eq:kitaev-chain-hamiltonian}
\end{equation}
\eqref{eq:kitaev-chain-hamiltonian}是一个p波超导模型,这个模型通常是这么来的:一个一维纳米线被放置在一个超导体上,两者的相互作用诱导前者也发生$U(1)$对称性破缺,然后我们使用平均场理论分析问题而引入一个$\Delta$参量。
\eqref{eq:kitaev-chain-hamiltonian}是一个紧束缚模型,
对\eqref{eq:kitaev-chain-hamiltonian}做傅里叶变换,可以得到
\begin{equation}
    {H} = \frac{1}{2} \sum_{\vb*{k}} \underbrace{\pmqty{{c}^\dagger_{\vb*{k}} & {c}_{-\vb*{k}}}}_{{\Psi}^\dagger} \pmqty{\epsilon_{\vb*{k}} - \mu & -2 \ii \Delta^* \sin k \\ 2 \ii \Delta \sin k & - \epsilon_{\vb*{k}} + \mu} \underbrace{\pmqty{{c}_{\vb*{k}} \\ {c}^\dagger_{-\vb*{k}}}}_{{\Psi}},
\end{equation}
然后再做Bogoliubov变换,计算出以下能谱:
\begin{equation}
    E_{\vb*{k}} = \pm \sqrt{(2t \cos k - \mu)^2 + 4 \abs{\Delta}^2 \sin^2 k}.
\end{equation}

\eqref{eq:kitaev-chain-hamiltonian}具有粒子-空穴对称性。% TODO
总之就有一个约束就是设$P$为粒子空穴变换,我们有
\[
    P {\Psi}_{\vb*{k}} P^{-1} = \tau^* {\Psi}_{-\vb*{k}}^*
\]

Kitaev链不存在对称性自发破缺,但能隙可开可闭。当
\begin{equation}
    \mu = \pm 2t
    \label{eq:kitaev-gap-point}
\end{equation}
时,能隙会关闭。除此之外任何参数的变动都只会引起连续的变化。
因此,如果体系发生相变,那么只能是在\eqref{eq:kitaev-gap-point}处发生一个和对称性无关的相变。在化学势很低时,即$\mu$趋于负无穷时,根本就没有电子,因此从$-\infty$到$-2t$的部分肯定是平庸的。
化学势非常高时(大于$2t$时)电子全满,同样是平庸的。
因此有趣的行为集中在$-2t$到$2t$之间。下面会看到,当$\mu$越过\eqref{eq:kitaev-gap-point}这两个点时,会发生一个拓扑相变。

$W = \pm 1$,这是一个定义在立体中的量?

\subsection{Kitaev链中的拓扑不变量}

下面定义一个能带的拓扑不变量。

总之,当$\mu$扫过$\mu=-2t$时,我们有
\[
    {c}_{\vb*{i}} = \frac{1}{2} ({\gamma}_{i\text{A}} + \ii {\gamma}_{i \text{B}}),
\]
容易验证均为
\begin{equation}
    \acomm*{{\gamma}_\alpha}{{\gamma}_\beta^\dagger} = 2 \delta_{\alpha \beta},
\end{equation}

\subsection{时间反演对称性保护的拓扑超导}

刚才描述的拓扑超导和对称性没有特别明确的对称性。当然可以说它有粒子-空穴对称性,但是这完全是一个数学上的处理。