\documentclass[hyperref, a4paper]{article}

\usepackage{geometry}
\usepackage{titling}
\usepackage{titlesec}
% No longer needed, since we will use enumitem package
% \usepackage{paralist}
\usepackage{enumitem}
\usepackage{footnote}
\usepackage{amsmath, amssymb, amsthm}
\usepackage{mathtools}
\usepackage{bbm}
\usepackage{graphicx}
\usepackage{subfigure}
\usepackage{physics}
\usepackage{tensor}
\usepackage{siunitx}
\usepackage[version=4]{mhchem}
\usepackage{tikz}
\usepackage{xcolor}
\usepackage{listings}
\usepackage{underscore}
\usepackage{autobreak}
\usepackage[ruled, vlined, linesnumbered]{algorithm2e}
\usepackage[sorting=none]{biblatex}
\addbibresource{td-agw.bib}
\usepackage[colorlinks,unicode]{hyperref} % , linkcolor=black, anchorcolor=black, citecolor=black, urlcolor=black, filecolor=black
\usepackage{prettyref}

% Page style
\geometry{left=3.18cm,right=3.18cm,top=2.54cm,bottom=2.54cm}
\titlespacing{\paragraph}{0pt}{1pt}{10pt}[20pt]
\setlength{\droptitle}{-5em}

% Math operators
\DeclareMathOperator{\timeorder}{\mathcal{T}}
\DeclareMathOperator{\diag}{diag}
\DeclareMathOperator{\legpoly}{P}
\DeclareMathOperator{\primevalue}{P}
\DeclareMathOperator{\sgn}{sgn}
\DeclareMathOperator{\res}{Res}
\newcommand*{\ii}{\mathrm{i}}
\newcommand*{\ee}{\mathrm{e}}
\newcommand*{\const}{\mathrm{const}}
\newcommand*{\suchthat}{\quad \text{s.t.} \quad}
\newcommand*{\argmin}{\arg\min}
\newcommand*{\argmax}{\arg\max}
\newcommand*{\normalorder}[1]{: #1 :}
\newcommand*{\pair}[1]{\langle #1 \rangle}
\newcommand*{\fd}[1]{\mathcal{D} #1}
\DeclareMathOperator{\bigO}{\mathcal{O}}

% TikZ setting
\usetikzlibrary{arrows,shapes,positioning}
\usetikzlibrary{arrows.meta}
\usetikzlibrary{decorations.markings}
\usetikzlibrary{calc}
\tikzstyle arrowstyle=[scale=1]
\tikzstyle directed=[postaction={decorate,decoration={markings,
    mark=at position .5 with {\arrow[arrowstyle]{stealth}}}}]
\tikzstyle ray=[directed, thick]
\tikzstyle dot=[anchor=base,fill,circle,inner sep=1pt]

% Algorithm setting
% Julia-style code
\SetKwIF{If}{ElseIf}{Else}{if}{}{elseif}{else}{end}
\SetKwFor{For}{for}{}{end}
\SetKwFor{While}{while}{}{end}
\SetKwProg{Function}{function}{}{end}
\SetArgSty{textnormal}

\newcommand*{\concept}[1]{{\textbf{#1}}}

% Embedded codes
\lstset{basicstyle=\ttfamily,
  showstringspaces=false,
  commentstyle=\color{gray},
  keywordstyle=\color{blue}
}

\lstdefinestyle{console}{
    basicstyle=\footnotesize\ttfamily,
    breaklines=true,
    postbreak=\mbox{\textcolor{red}{$\hookrightarrow$}\space}
}

% Reference formatting
\newrefformat{fig}{Figure~\ref{#1}}

% Displaying texts in bookmarkers

\pdfstringdefDisableCommands{%
  \def\\{}%
  \def\ce#1{<#1>}%
}

\pdfstringdefDisableCommands{%
  \def\texttt#1{<#1>}%
  \def\mathbb#1{#1}%
}
\pdfstringdefDisableCommands{\def\eqref#1{(\ref{#1})}}

\makeatletter
\pdfstringdefDisableCommands{\let\HyPsd@CatcodeWarning\@gobble}
\makeatother

\newenvironment{shelldisplay}{\begin{lstlisting}}{\end{lstlisting}}

\newcommand{\shortcode}[1]{\texttt{#1}}

\lstset{style = console}

\title{Time-dependent adiabatic $GW$}
\author{Jinyuan Wu}

\begin{document}

\maketitle

\section{From Keldysh formalism to quantum master equation}

Existing researches \cite{chan2021giant} utilizing TD-aGW carry out the calculation 
within the framework of quantum master equation (QME)
\begin{equation}
    \dot{\rho} + \ii \comm*{\rho}{H} = \int_{-\infty}^{t} \dd{t'} F[\rho(t')]
\end{equation}
which is the equation of motion of the 
reduced single-electron density matrix 
\begin{equation}
    \rho(\vb*{r}, \vb*{r}', t) = \expval{\psi^\dagger(\vb*{r}') \psi(\vb*{r})}
    = - \ii G^<(\vb*{r}, t; \vb*{r}', t).
\end{equation}
QME has clear physical meaning 
when the system contains well-defined quasiparticles.
The generality of QME and the flow from the most generalized Keldysh formalism 
to quantum kinetic theories is discussed in 
\cite{rammer1986quantum,vspivcka2005long,vspivcka2005long2,vspivcka2005long3}.
In this section we give a brief summary of the derivation 
and coverage as well as possible simplifications of QME.

In Keldysh formalism we still have the Dyson equation
\begin{equation}
    G = G_0 + G_0 \Sigma G = G_0 + G \Sigma G_0,
    \label{eq:keldysh-sigma}
\end{equation}
although now $G$, $G_0$ and $\Sigma$
are $2 \times 2$ matrices
due to the need of anti-time-ordered, lesser and greater Green functions 
besides the time-ordered Green function 
in deriving a diagrammatic perturbation theory 
for non-equilibrium Green functions \cite{lifschitz1983physical};
alternatively they can be defined on the Keldysh contour \cite{haug2008quantum}.
Note that the choice of $H_0$ has some arbitrariness at this point:
we may split the full many-body self-energy correction $\Sigma^{\text{full}}$
into an ``effective field'' which has no retardation and is inserted into $H_0$
and a ``real'' interaction part $\Sigma$.
By Lengreth's rules \cite{haug2008quantum},
we reduce \eqref{eq:keldysh-sigma} on the Keldysh contour 
into the following equation on the physical time axis:
\begin{equation}
    G^< = G^<_0 + G^{\text{R}}_0 \Sigma^{\text{R}} G^{<}_0
    + G^{\text{R}} \Sigma^{<} G^{\text{A}}_0 
    + G^{<} \Sigma^{\text{A}} G^\text{A} , \quad 
    G^{\text{A}, \text{R}} = G^{\text{A}, \text{R}}_0 + G^{\text{A}, \text{R}}_0 \Sigma^{\text{A}, \text{R}} G^{\text{A}, \text{R}}.
\end{equation}
By applying 
\begin{equation}
    \overleftarrow{G}_0^{-1} = - \ii \overleftarrow{\partial}_t - H_0 , \quad 
    \overrightarrow{G}_0^{-1} = \ii \overrightarrow{\partial}_t - H_0
\end{equation}
to the left and right hand sides of the equation about $G^<$
and utilizing the fact that $G_0^< \overleftarrow{G}_0^{-1} = \overrightarrow{G}_0^{-1} G_0^< = 0$,
we obtain 
\begin{equation}
    \ii \partial_T G^< - \comm*{H_0}{G^<} = 
    \Sigma^{\text{R}} G^< + \Sigma^< G^{\text{A}} - G^{\text{R}} \Sigma^< - G^< \Sigma^{\text{A}},
    \label{eq:gkb}
\end{equation}
where we have redefined the time variables as 
\begin{equation}
    T = \frac{t_1 + t_2}{2}, \quad t = t_1 - t_2.
\end{equation}
Roughly speaking, $T$ is related to the time variance 
of the external driving field or the relaxation of the system,
while $t$ corresponds to the time variable 
in equilibrium Green function,
and often Fourier transform is done in $t$,
and thus $G^< = G^<(T, \omega)$. 

\eqref{eq:gkb} is already very close to QME,
its left hand side being the left hand side of QME 
after integrating over $\omega$.
Three approximations are needed however to close the equation:
the explicit form of $\Sigma[G]$,
representation of $G^{\text{A}, \text{R}}$ with $G^<$ and $\Sigma^{\text{A}, \text{R}}$,
and reconstruction of $G^<$ using $\rho(T)$ i.e. $G^<(t_1 = t_2 = T)$.
It can be proved that the last is always possible \cite{vspivcka2005long},
with the lowest order approximation being the 
Generalized Kadanoff–Baym Ansatz,
and a closed QME can therefore be obtained \cite{vspivcka2005long2,haug2008quantum}.
Note that in principle this procedure does not require 
the system to have well-defined electron-like quasiparticle.

Two further approximations reduce QME to quantum Boltzmann equation (QBE)
\cite{rammer1986quantum,haug2008quantum}.
Under the assumption that well-defined quasiparticles always exist and hence we have 
\begin{equation}
    G^<(\vb*{X}, \vb*{p}, T, \omega) = 2\pi \delta(\omega - E_{\vb*{p}}(\vb*{X})) f(\vb*{X}, \vb*{p}, T),
\end{equation}
where Wigner transform has been applied.
Using the condition of smooth external driving force,
by gradient expansion, the left hand side of QME becomes 
the usual form of collisionless Boltzmann equation.
With both the quasiparticle assumption the right hand side of QME 
becomes the Fermi golden rule, 
which is prevalently used as the collision integral \cite{chen2022first}.
Deviations from the quasiparticle approximation,
e.g. strong electron-phonon interaction-induced renormalization, 
may still lead to QBE,
although with modified collision integral \cite{rammer1986quantum,wais2018quantum}.

\section{The COHSEX approximation} 

\printbibliography

\end{document}
