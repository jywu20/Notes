\documentclass[hyperref, a4paper]{article}

\usepackage{geometry}
\usepackage{titling}
\usepackage{titlesec}
% No longer needed, since we will use enumitem package
% \usepackage{paralist}
\usepackage{enumitem}
\usepackage{footnote}
\usepackage{amsmath, amssymb, amsthm}
\usepackage{mathtools}
\usepackage{bbm}
\usepackage{graphicx}
\usepackage{subfigure}
\usepackage{physics}
\usepackage{tensor}
\usepackage{siunitx}
\usepackage[version=4]{mhchem}
\usepackage{tikz}
\usepackage{xcolor}
\usepackage{listings}
\usepackage{underscore}
\usepackage{autobreak}
\usepackage[ruled, vlined, linesnumbered]{algorithm2e}
\usepackage[sorting=none]{biblatex}
\addbibresource{td-agw.bib}
\usepackage[colorlinks,unicode]{hyperref} % , linkcolor=black, anchorcolor=black, citecolor=black, urlcolor=black, filecolor=black
\usepackage{prettyref}

% Page style
\geometry{left=3.18cm,right=3.18cm,top=2.54cm,bottom=2.54cm}
\titlespacing{\paragraph}{0pt}{1pt}{10pt}[20pt]
\setlength{\droptitle}{-5em}

% Math operators
\DeclareMathOperator{\timeorder}{\mathcal{T}}
\DeclareMathOperator{\diag}{diag}
\DeclareMathOperator{\legpoly}{P}
\DeclareMathOperator{\primevalue}{P}
\DeclareMathOperator{\sgn}{sgn}
\DeclareMathOperator{\res}{Res}
\newcommand*{\ii}{\mathrm{i}}
\newcommand*{\ee}{\mathrm{e}}
\newcommand*{\const}{\mathrm{const}}
\newcommand*{\suchthat}{\quad \text{s.t.} \quad}
\newcommand*{\argmin}{\arg\min}
\newcommand*{\argmax}{\arg\max}
\newcommand*{\normalorder}[1]{: #1 :}
\newcommand*{\pair}[1]{\langle #1 \rangle}
\newcommand*{\fd}[1]{\mathcal{D} #1}
\DeclareMathOperator{\bigO}{\mathcal{O}}

% TikZ setting
\usetikzlibrary{arrows,shapes,positioning}
\usetikzlibrary{arrows.meta}
\usetikzlibrary{decorations.markings}
\usetikzlibrary{calc}
\tikzstyle arrowstyle=[scale=1]
\tikzstyle directed=[postaction={decorate,decoration={markings,
    mark=at position .5 with {\arrow[arrowstyle]{stealth}}}}]
\tikzstyle ray=[directed, thick]
\tikzstyle dot=[anchor=base,fill,circle,inner sep=1pt]

% Algorithm setting
% Julia-style code
\SetKwIF{If}{ElseIf}{Else}{if}{}{elseif}{else}{end}
\SetKwFor{For}{for}{}{end}
\SetKwFor{While}{while}{}{end}
\SetKwProg{Function}{function}{}{end}
\SetArgSty{textnormal}

\newcommand*{\concept}[1]{{\textbf{#1}}}

% Embedded codes
\lstset{basicstyle=\ttfamily,
  showstringspaces=false,
  commentstyle=\color{gray},
  keywordstyle=\color{blue}
}

\lstdefinestyle{console}{
    basicstyle=\footnotesize\ttfamily,
    breaklines=true,
    postbreak=\mbox{\textcolor{red}{$\hookrightarrow$}\space}
}

% Reference formatting
\newrefformat{fig}{Figure~\ref{#1}}

% Displaying texts in bookmarkers

\pdfstringdefDisableCommands{%
  \def\\{}%
  \def\ce#1{<#1>}%
}

\pdfstringdefDisableCommands{%
  \def\texttt#1{<#1>}%
  \def\mathbb#1{#1}%
}
\pdfstringdefDisableCommands{\def\eqref#1{(\ref{#1})}}

\makeatletter
\pdfstringdefDisableCommands{\let\HyPsd@CatcodeWarning\@gobble}
\makeatother

\newenvironment{shelldisplay}{\begin{lstlisting}}{\end{lstlisting}}

\newcommand{\shortcode}[1]{\texttt{#1}}

\lstset{style = console}

\title{Time-dependent adiabatic $GW$}
\author{Jinyuan Wu}

\begin{document}

\maketitle

\section{From Keldysh formalism to quantum master equation}

Existing researches \cite{chan2021giant} utilizing TD-aGW carry out the calculation 
within the framework of quantum master equation (QME)
\begin{equation}
    \dot{\rho} + \ii \comm*{\rho}{H} = \int_{-\infty}^{t} \dd{t'} F[\rho(t')]
\end{equation}
which is the equation of motion of the 
reduced single-electron density matrix 
\begin{equation}
    \rho(\vb*{r}, \vb*{r}', t) = \expval{\psi^\dagger(\vb*{r}') \psi(\vb*{r})}
    = - \ii G^<(\vb*{r}, t; \vb*{r}', t).
\end{equation}
QME can be derived heuristically 
when the system contains well-defined quasiparticles.
The generality of QME and the flow from the most generalized Keldysh formalism 
to quantum kinetic theories is discussed in 
\cite{rammer1986quantum,vspivcka2005long,vspivcka2005long2,vspivcka2005long3}.
In this section we give a brief summary of the derivation 
and coverage as well as possible simplifications of QME.

In Keldysh formalism we still have the Dyson equation
\begin{equation}
    G = G_0 + G_0 \Sigma G = G_0 + G \Sigma G_0,
    \label{eq:keldysh-sigma}
\end{equation}
although now $G$, $G_0$ and $\Sigma$
are $2 \times 2$ matrices
due to the need of anti-time-ordered, lesser and greater Green functions 
besides the time-ordered Green function 
in deriving a diagrammatic perturbation theory 
for non-equilibrium Green functions \cite{lifschitz1983physical};
alternatively they can be defined on the Keldysh contour \cite{haug2008quantum}.
Note that the choice of $H_0$ has some arbitrariness at this point:
we may split the full many-body self-energy correction $\Sigma^{\text{full}}$
into an ``effective field'' which has no retardation and is inserted into $H_0$
and a ``real'' interaction part $\Sigma$.
By Lengreth's rules \cite{haug2008quantum},
we reduce \eqref{eq:keldysh-sigma} on the Keldysh contour 
into the following equation on the physical time axis:
\begin{equation}
    G^< = G^<_0 + G^{\text{R}}_0 \Sigma^{\text{R}} G^{<}_0
    + G^{\text{R}} \Sigma^{<} G^{\text{A}}_0 
    + G^{<} \Sigma^{\text{A}} G^\text{A} , \quad 
    G^{\text{A}, \text{R}} = G^{\text{A}, \text{R}}_0 + G^{\text{A}, \text{R}}_0 \Sigma^{\text{A}, \text{R}} G^{\text{A}, \text{R}}.
\end{equation}
By applying 
\begin{equation}
    \overleftarrow{G}_0^{-1} = - \ii \overleftarrow{\partial}_t - H_0 , \quad 
    \overrightarrow{G}_0^{-1} = \ii \overrightarrow{\partial}_t - H_0
\end{equation}
to the left and right hand sides of the equation about $G^<$
and utilizing the fact that $G_0^< \overleftarrow{G}_0^{-1} = \overrightarrow{G}_0^{-1} G_0^< = 0$,
we obtain the so-called Kadanoff-Baym equation;
a similar equation can be obtained for $G^>$.
Here I present a form \cite{haug2008quantum} that is closer to 
what is often seen in equilibrium Green function theory:
\begin{equation}
    \left[G_0^{-1}-U-\Sigma, A\right]-[\Gamma, G]=0,
    \label{eq:gkb-for-a}
\end{equation}
where 
and $A$, $\Gamma$, $G$, and $\Sigma$ are defined as 
\begin{equation}
    \Sigma = \frac{1}{2}\left(\Sigma^{\mathrm{R}}+\Sigma^{\mathrm{A}}\right), \quad  
    G = \frac{1}{2}\left(G^{\mathrm{R}}+G^{\mathrm{A}}\right), \quad 
    A = \mathrm{i}\left(G^{\mathrm{R}}-G^{\mathrm{A}}\right), \quad 
    \Gamma = \mathrm{i}\left(\Sigma^{\mathrm{R}}-\Sigma^{\mathrm{A}}\right).
\end{equation}
The quantities $A$ and $G$  
is the generalization of its equilibrium counterpart, 
namely the spectral function;
$\Sigma$ and $\Gamma$ are generalizations of 
$\Re \Sigma$ and $\Im \Sigma$ in the equilibrium formalism, respectively;
note that in the most general case 
there is no guarantee that $\Sigma$ is predominantly a real number.
The form of \eqref{eq:gkb-for-a} also shows 
that the division of labor between $\Sigma$ and $U_0$
is only artificial, 
and with no further approximation done, 
different divisions give the same EOM for $A$,
as is expected.
The Kadanoff-Baym equation about $G^<$ can be written in to the following more readable form
\begin{equation}
    \ii \partial_T G^< - \comm*{H_0}{G^<} = 
    \Sigma^{\text{R}} G^< + \Sigma^< G^{\text{A}} - G^{\text{R}} \Sigma^< - G^< \Sigma^{\text{A}},
    \label{eq:gkb}
\end{equation}
where we have redefined the time variables as 
\begin{equation}
    T = \frac{t_1 + t_2}{2}, \quad t = t_1 - t_2.
\end{equation}
Roughly speaking, $T$ is related to the time variance 
of the external driving field or the relaxation of the system,
while $t$ corresponds to the time variable 
in equilibrium Green function,
and often Fourier transform is done in $t$,
and thus $G^< = G^<(T, \omega)$. 


The ``precursor'' equation \eqref{eq:gkb} is already very close to QME,
its left hand side being the left hand side of QME 
after integrating over $\omega$.
Three approximations are needed however to close the equation:
the explicit form of $\Sigma[G]$,
representation of $G^{\text{A}, \text{R}}$ with $G^<$ and $\Sigma^{\text{A}, \text{R}}$,
and reconstruction of $G^<$ using $\rho(T)$ i.e. $G^<(t_1 = t_2 = T)$.
It can be proved that the last is always possible \cite{vspivcka2005long},
with the lowest order approximation being the 
Generalized Kadanoff–Baym Ansatz,
and a closed QME can therefore be obtained \cite{vspivcka2005long2,haug2008quantum}.
Note that in principle this procedure does not require 
the system to have well-defined electron-like quasiparticle.
Note that we may be able to adjust the division of labor between $H_0$ and $\Sigma$,
as is said above, and 
if we want to place as much content as we want into $H_0$,
there is no guarantee that it stays real,
as is seen in the Lindblad equation 
widely used in quantum optics.

QME can be used to study higher-order correlations in a system 
without explicitly constructing 
the relevant Green functions.
The linear response, for example, reveals $G^{(4)}$,
since the first order response can be diagrammatically obtained 
by choosing an arbitrary propagator 
and plant an external field line into it (\prettyref{fig:linear-response} (a)).
Since the single-electron self-energy can always 
be written into the convolution 
of the two-electron kernel function and 
the renormalized Green function (\prettyref{fig:linear-response} (b)),
the letter containing its own self-energy correction 
and hence a two-electron kernel function,
by taking the linear response i.e. breaking a propagator in a single-electron self-energy,
we get a two-electron kernel function 
with the same level of accuracy (\prettyref{fig:linear-response} (c, d)). 

\begin{figure}
    \centering
    \begin{tikzpicture}[x=0.75pt,y=0.75pt,yscale=-1,xscale=1]
    %uncomment if require: \path (0,394); %set diagram left start at 0, and has height of 394
    
    %Shape: Square [id:dp8080521882731848] 
    \draw  [fill={rgb, 255:red, 155; green, 155; blue, 155 }  ,fill opacity=1 ] (86.17,32.33) -- (136.17,32.33) -- (136.17,82.33) -- (86.17,82.33) -- cycle ;
    %Straight Lines [id:da8056001438667917] 
    \draw    (78,32.33) -- (78.06,32.33) ;
    \draw [shift={(81.06,32.33)}, rotate = 180] [fill={rgb, 255:red, 0; green, 0; blue, 0 }  ][line width=0.08]  [draw opacity=0] (7.14,-3.43) -- (0,0) -- (7.14,3.43) -- (4.74,0) -- cycle    ;
    %Straight Lines [id:da42224269220484056] 
    \draw    (136.17,32.33) -- (157,32.33) ;
    %Straight Lines [id:da5200731935911023] 
    \draw    (146.58,32.33) -- (146.64,32.33) ;
    \draw [shift={(149.64,32.33)}, rotate = 180] [fill={rgb, 255:red, 0; green, 0; blue, 0 }  ][line width=0.08]  [draw opacity=0] (7.14,-3.43) -- (0,0) -- (7.14,3.43) -- (4.74,0) -- cycle    ;
    %Straight Lines [id:da37223848255403347] 
    \draw    (75.75,82.33) ;
    \draw [shift={(72.88,82.33)}, rotate = 360] [fill={rgb, 255:red, 0; green, 0; blue, 0 }  ][line width=0.08]  [draw opacity=0] (7.14,-3.43) -- (0,0) -- (7.14,3.43) -- (4.74,0) -- cycle    ;
    %Straight Lines [id:da02411483054413366] 
    \draw    (146.58,82.33) ;
    \draw [shift={(143.71,82.33)}, rotate = 360] [fill={rgb, 255:red, 0; green, 0; blue, 0 }  ][line width=0.08]  [draw opacity=0] (7.14,-3.43) -- (0,0) -- (7.14,3.43) -- (4.74,0) -- cycle    ;
    %Straight Lines [id:da8990286287188887] 
    \draw    (136.17,82.33) -- (157,82.33) ;
    %Shape: Rectangle [id:dp25208278024150643] 
    \draw  [draw opacity=0][fill={rgb, 255:red, 208; green, 2; blue, 27 }  ,fill opacity=0.1 ] (156.58,19.75) -- (178.08,19.75) -- (178.08,95) -- (156.58,95) -- cycle ;
    %Curve Lines [id:da6362174770069333] 
    \draw    (62.03,58.19) .. controls (62.03,27.69) and (72.28,32.44) .. (86.17,32.33) ;
    %Curve Lines [id:da6052356328137927] 
    \draw    (62.03,58.19) .. controls (60.53,86.19) and (72.28,82.23) .. (86.17,82.33) ;
    %Straight Lines [id:da17458344465924025] 
    \draw    (62.03,58.19) .. controls (60.36,59.86) and (58.7,59.86) .. (57.03,58.19) .. controls (55.36,56.52) and (53.7,56.52) .. (52.03,58.19) .. controls (50.36,59.86) and (48.7,59.86) .. (47.03,58.19) .. controls (45.36,56.52) and (43.7,56.52) .. (42.03,58.19) -- (38.78,58.19) -- (38.78,58.19) ;
    \draw [shift={(38.78,58.19)}, rotate = 225] [color={rgb, 255:red, 0; green, 0; blue, 0 }  ][line width=0.75]    (-5.59,0) -- (5.59,0)(0,5.59) -- (0,-5.59)   ;
    %Straight Lines [id:da3868484502445457] 
    \draw [color={rgb, 255:red, 208; green, 2; blue, 27 }  ,draw opacity=1 ]   (228.5,30.39) -- (250.61,30.39) ;
    %Straight Lines [id:da8447618514561495] 
    \draw [color={rgb, 255:red, 208; green, 2; blue, 27 }  ,draw opacity=1 ]   (284.36,30.39) -- (306.06,30.39) ;
    %Straight Lines [id:da4003858404269218] 
    \draw [color={rgb, 255:red, 208; green, 2; blue, 27 }  ,draw opacity=1 ]   (239.56,30.39) -- (239.61,30.39) ;
    \draw [shift={(242.61,30.39)}, rotate = 180] [fill={rgb, 255:red, 208; green, 2; blue, 27 }  ,fill opacity=1 ][line width=0.08]  [draw opacity=0] (7.14,-3.43) -- (0,0) -- (7.14,3.43) -- (4.74,0) -- cycle    ;
    %Straight Lines [id:da63663603215392] 
    \draw [color={rgb, 255:red, 208; green, 2; blue, 27 }  ,draw opacity=1 ]   (295.21,30.39) -- (295.27,30.39) ;
    \draw [shift={(298.27,30.39)}, rotate = 180] [fill={rgb, 255:red, 208; green, 2; blue, 27 }  ,fill opacity=1 ][line width=0.08]  [draw opacity=0] (7.14,-3.43) -- (0,0) -- (7.14,3.43) -- (4.74,0) -- cycle    ;
    %Shape: Square [id:dp10457466461274345] 
    \draw   (250.61,30.39) -- (284.36,30.39) -- (284.36,64.14) -- (250.61,64.14) -- cycle ;
    %Curve Lines [id:da9308118030712733] 
    \draw [color={rgb, 255:red, 74; green, 144; blue, 226 }  ,draw opacity=1 ][line width=1.5]    (250.61,64.14) .. controls (250.61,82.19) and (283.86,82.69) .. (284.36,64.14) ;
    %Straight Lines [id:da6820775689849501] 
    \draw [color={rgb, 255:red, 74; green, 144; blue, 226 }  ,draw opacity=1 ]   (268.81,77.39) -- (268.86,77.39) ;
    \draw [shift={(271.86,77.39)}, rotate = 180] [fill={rgb, 255:red, 74; green, 144; blue, 226 }  ,fill opacity=1 ][line width=0.08]  [draw opacity=0] (7.14,-3.43) -- (0,0) -- (7.14,3.43) -- (4.74,0) -- cycle    ;
    %Straight Lines [id:da04804602852086326] 
    \draw    (277.08,75.25) .. controls (279.23,74.28) and (280.79,74.86) .. (281.76,77.01) .. controls (282.74,79.16) and (284.3,79.74) .. (286.45,78.76) .. controls (288.6,77.79) and (290.16,78.37) .. (291.13,80.52) .. controls (292.1,82.67) and (293.66,83.25) .. (295.81,82.28) .. controls (297.96,81.3) and (299.52,81.88) .. (300.49,84.03) -- (301.88,84.55) -- (301.88,84.55) ;
    \draw [shift={(301.88,84.55)}, rotate = 65.57] [color={rgb, 255:red, 0; green, 0; blue, 0 }  ][line width=0.75]    (-5.59,0) -- (5.59,0)(0,5.59) -- (0,-5.59)   ;
    %Straight Lines [id:da7583379686906808] 
    \draw [color={rgb, 255:red, 208; green, 2; blue, 27 }  ,draw opacity=1 ]   (134.55,153.03) -- (156.67,153.03) ;
    \draw [shift={(148.21,153.03)}, rotate = 180] [fill={rgb, 255:red, 208; green, 2; blue, 27 }  ,fill opacity=1 ][line width=0.08]  [draw opacity=0] (7.14,-3.43) -- (0,0) -- (7.14,3.43) -- (4.74,0) -- cycle    ;
    %Straight Lines [id:da39448885367690245] 
    \draw [color={rgb, 255:red, 208; green, 2; blue, 27 }  ,draw opacity=1 ]   (190.17,153.03) -- (212.28,153.03) ;
    \draw [shift={(203.82,153.03)}, rotate = 180] [fill={rgb, 255:red, 208; green, 2; blue, 27 }  ,fill opacity=1 ][line width=0.08]  [draw opacity=0] (7.14,-3.43) -- (0,0) -- (7.14,3.43) -- (4.74,0) -- cycle    ;
    %Straight Lines [id:da6515356693161518] 
    \draw [color={rgb, 255:red, 208; green, 2; blue, 27 }  ,draw opacity=1 ]   (241.28,153.03) -- (263.4,153.03) ;
    \draw [shift={(254.94,153.03)}, rotate = 180] [fill={rgb, 255:red, 208; green, 2; blue, 27 }  ,fill opacity=1 ][line width=0.08]  [draw opacity=0] (7.14,-3.43) -- (0,0) -- (7.14,3.43) -- (4.74,0) -- cycle    ;
    %Shape: Circle [id:dp6868798252470776] 
    \draw   (263.4,153.03) .. controls (263.4,145.99) and (269.1,140.28) .. (276.15,140.28) .. controls (283.19,140.28) and (288.9,145.99) .. (288.9,153.03) .. controls (288.9,160.07) and (283.19,165.78) .. (276.15,165.78) .. controls (269.1,165.78) and (263.4,160.07) .. (263.4,153.03) -- cycle ;
    %Straight Lines [id:da10772229105331421] 
    \draw [color={rgb, 255:red, 208; green, 2; blue, 27 }  ,draw opacity=1 ][fill={rgb, 255:red, 208; green, 2; blue, 27 }  ,fill opacity=1 ]   (288.9,153.03) -- (311.01,153.03) ;
    \draw [shift={(302.55,153.03)}, rotate = 180] [fill={rgb, 255:red, 208; green, 2; blue, 27 }  ,fill opacity=1 ][line width=0.08]  [draw opacity=0] (7.14,-3.43) -- (0,0) -- (7.14,3.43) -- (4.74,0) -- cycle    ;
    %Straight Lines [id:da5473610425168207] 
    \draw [color={rgb, 255:red, 208; green, 2; blue, 27 }  ,draw opacity=1 ]   (43.55,153.28) -- (65.67,153.28) ;
    \draw [shift={(57.21,153.28)}, rotate = 180] [fill={rgb, 255:red, 208; green, 2; blue, 27 }  ,fill opacity=1 ][line width=0.08]  [draw opacity=0] (7.14,-3.43) -- (0,0) -- (7.14,3.43) -- (4.74,0) -- cycle    ;
    %Straight Lines [id:da8282046632541067] 
    \draw [color={rgb, 255:red, 208; green, 2; blue, 27 }  ,draw opacity=1 ]   (91.17,153.28) -- (113.28,153.28) ;
    \draw [shift={(104.82,153.28)}, rotate = 180] [fill={rgb, 255:red, 208; green, 2; blue, 27 }  ,fill opacity=1 ][line width=0.08]  [draw opacity=0] (7.14,-3.43) -- (0,0) -- (7.14,3.43) -- (4.74,0) -- cycle    ;
    %Shape: Circle [id:dp45441295779646795] 
    \draw   (65.67,153.28) .. controls (65.67,146.24) and (71.38,140.53) .. (78.42,140.53) .. controls (85.46,140.53) and (91.17,146.24) .. (91.17,153.28) .. controls (91.17,160.32) and (85.46,166.03) .. (78.42,166.03) .. controls (71.38,166.03) and (65.67,160.32) .. (65.67,153.28) -- cycle ;
    %Shape: Square [id:dp848075025730741] 
    \draw   (156.67,153.03) -- (190.42,153.03) -- (190.42,186.78) -- (156.67,186.78) -- cycle ;
    %Curve Lines [id:da7443735678084988] 
    \draw [color={rgb, 255:red, 74; green, 144; blue, 226 }  ,draw opacity=1 ][line width=1.5]    (156.61,187.14) .. controls (156.61,205.19) and (189.86,205.69) .. (190.36,187.14) ;
    %Straight Lines [id:da1714601872678736] 
    \draw [color={rgb, 255:red, 74; green, 144; blue, 226 }  ,draw opacity=1 ]   (174.81,200.39) -- (174.86,200.39) ;
    \draw [shift={(177.86,200.39)}, rotate = 180] [fill={rgb, 255:red, 74; green, 144; blue, 226 }  ,fill opacity=1 ][line width=0.08]  [draw opacity=0] (7.14,-3.43) -- (0,0) -- (7.14,3.43) -- (4.74,0) -- cycle    ;
    %Straight Lines [id:da25202689596161787] 
    \draw    (183.08,198.25) .. controls (185.23,197.28) and (186.79,197.86) .. (187.76,200.01) .. controls (188.74,202.16) and (190.3,202.74) .. (192.45,201.76) .. controls (194.6,200.79) and (196.16,201.37) .. (197.13,203.52) .. controls (198.1,205.67) and (199.66,206.25) .. (201.81,205.28) .. controls (203.96,204.3) and (205.52,204.88) .. (206.49,207.03) -- (207.88,207.55) -- (207.88,207.55) ;
    \draw [shift={(207.88,207.55)}, rotate = 65.57] [color={rgb, 255:red, 0; green, 0; blue, 0 }  ][line width=0.75]    (-5.59,0) -- (5.59,0)(0,5.59) -- (0,-5.59)   ;
    %Shape: Square [id:dp15920393863477011] 
    \draw   (138.04,297.29) -- (104.29,297.21) -- (104.37,263.46) -- (138.12,263.54) -- cycle ;
    %Straight Lines [id:da7447672969533063] 
    \draw [color={rgb, 255:red, 208; green, 2; blue, 27 }  ,draw opacity=1 ][line width=1.5]    (275.82,263.59) -- (254.12,263.54) ;
    %Straight Lines [id:da6191293038856294] 
    \draw [color={rgb, 255:red, 208; green, 2; blue, 27 }  ,draw opacity=1 ]   (267.47,263.57) ;
    \draw [shift={(268.42,263.58)}, rotate = 180.14] [fill={rgb, 255:red, 208; green, 2; blue, 27 }  ,fill opacity=1 ][line width=0.08]  [draw opacity=0] (7.14,-3.43) -- (0,0) -- (7.14,3.43) -- (4.74,0) -- cycle    ;
    %Straight Lines [id:da18190348813272927] 
    \draw [color={rgb, 255:red, 74; green, 144; blue, 226 }  ,draw opacity=1 ][line width=1.5]    (104.37,263.46) -- (82.68,263.41) ;
    %Straight Lines [id:da8771236317207696] 
    \draw [color={rgb, 255:red, 74; green, 144; blue, 226 }  ,draw opacity=1 ]   (93.53,263.43) -- (95.47,263.44) ;
    \draw [shift={(98.47,263.44)}, rotate = 180.14] [fill={rgb, 255:red, 74; green, 144; blue, 226 }  ,fill opacity=1 ][line width=0.08]  [draw opacity=0] (7.14,-3.43) -- (0,0) -- (7.14,3.43) -- (4.74,0) -- cycle    ;
    %Straight Lines [id:da26856250076334187] 
    \draw [color={rgb, 255:red, 208; green, 2; blue, 27 }  ,draw opacity=1 ][line width=1.5]    (275.74,297.34) -- (254.04,297.29) ;
    %Straight Lines [id:da27322314760032107] 
    \draw [color={rgb, 255:red, 208; green, 2; blue, 27 }  ,draw opacity=1 ]   (258.2,297.3) -- (262.14,297.31) ;
    \draw [shift={(261.14,297.31)}, rotate = 0.14] [fill={rgb, 255:red, 208; green, 2; blue, 27 }  ,fill opacity=1 ][line width=0.08]  [draw opacity=0] (7.14,-3.43) -- (0,0) -- (7.14,3.43) -- (4.74,0) -- cycle    ;
    %Straight Lines [id:da6138709318326319] 
    \draw [color={rgb, 255:red, 74; green, 144; blue, 226 }  ,draw opacity=1 ][line width=1.5]    (104.49,297.21) -- (82.79,297.16) ;
    %Straight Lines [id:da7013755344567185] 
    \draw [color={rgb, 255:red, 74; green, 144; blue, 226 }  ,draw opacity=1 ]   (86.95,297.17) -- (90.89,297.18) ;
    \draw [shift={(89.89,297.17)}, rotate = 0.14] [fill={rgb, 255:red, 74; green, 144; blue, 226 }  ,fill opacity=1 ][line width=0.08]  [draw opacity=0] (7.14,-3.43) -- (0,0) -- (7.14,3.43) -- (4.74,0) -- cycle    ;
    %Shape: Rectangle [id:dp8582573901110206] 
    \draw  [draw opacity=0][fill={rgb, 255:red, 208; green, 2; blue, 27 }  ,fill opacity=0.1 ] (275.58,242.08) -- (297.08,242.08) -- (297.08,317.33) -- (275.58,317.33) -- cycle ;
    %Shape: Rectangle [id:dp142232164849736] 
    \draw  [draw opacity=0][fill={rgb, 255:red, 74; green, 144; blue, 226 }  ,fill opacity=0.1 ] (61.08,242.08) -- (82.58,242.08) -- (82.58,317.33) -- (61.08,317.33) -- cycle ;
    %Straight Lines [id:da7076537356513151] 
    \draw [color={rgb, 255:red, 0; green, 0; blue, 0 }  ,draw opacity=1 ][line width=1.5]    (160.07,263.59) -- (138.37,263.54) ;
    %Straight Lines [id:da2678443139011346] 
    \draw [color={rgb, 255:red, 0; green, 0; blue, 0 }  ,draw opacity=1 ]   (151.72,263.57) ;
    \draw [shift={(152.67,263.58)}, rotate = 180.14] [fill={rgb, 255:red, 0; green, 0; blue, 0 }  ,fill opacity=1 ][line width=0.08]  [draw opacity=0] (7.14,-3.43) -- (0,0) -- (7.14,3.43) -- (4.74,0) -- cycle    ;
    %Straight Lines [id:da14726088384328118] 
    \draw [color={rgb, 255:red, 0; green, 0; blue, 0 }  ,draw opacity=1 ][line width=1.5]    (159.99,297.34) -- (138.29,297.29) ;
    %Straight Lines [id:da3669712826712628] 
    \draw [color={rgb, 255:red, 0; green, 0; blue, 0 }  ,draw opacity=1 ]   (142.45,297.3) -- (146.39,297.31) ;
    \draw [shift={(145.39,297.31)}, rotate = 0.14] [fill={rgb, 255:red, 0; green, 0; blue, 0 }  ,fill opacity=1 ][line width=0.08]  [draw opacity=0] (7.14,-3.43) -- (0,0) -- (7.14,3.43) -- (4.74,0) -- cycle    ;
    %Shape: Square [id:dp9363940295277975] 
    \draw   (254.04,297.29) -- (220.29,297.21) -- (220.37,263.46) -- (254.12,263.54) -- cycle ;
    %Straight Lines [id:da9502681725910218] 
    \draw [color={rgb, 255:red, 0; green, 0; blue, 0 }  ,draw opacity=1 ][line width=1.5]    (220.07,263.59) -- (198.37,263.54) ;
    %Straight Lines [id:da40191522732757323] 
    \draw [color={rgb, 255:red, 0; green, 0; blue, 0 }  ,draw opacity=1 ]   (211.72,263.57) ;
    \draw [shift={(212.67,263.58)}, rotate = 180.14] [fill={rgb, 255:red, 0; green, 0; blue, 0 }  ,fill opacity=1 ][line width=0.08]  [draw opacity=0] (7.14,-3.43) -- (0,0) -- (7.14,3.43) -- (4.74,0) -- cycle    ;
    %Straight Lines [id:da9875453009310811] 
    \draw [color={rgb, 255:red, 0; green, 0; blue, 0 }  ,draw opacity=1 ][line width=1.5]    (219.99,297.34) -- (198.29,297.29) ;
    %Straight Lines [id:da397706151002144] 
    \draw [color={rgb, 255:red, 0; green, 0; blue, 0 }  ,draw opacity=1 ]   (202.45,297.3) -- (206.39,297.31) ;
    \draw [shift={(205.39,297.31)}, rotate = 0.14] [fill={rgb, 255:red, 0; green, 0; blue, 0 }  ,fill opacity=1 ][line width=0.08]  [draw opacity=0] (7.14,-3.43) -- (0,0) -- (7.14,3.43) -- (4.74,0) -- cycle    ;
    
    % Text Node
    \draw (167.33,57.38) node  [rotate=-270] [align=left] {density};
    % Text Node
    \draw (111.17,57.33) node    {$G^{( 4)}$};
    % Text Node
    \draw (103,109) node [anchor=north west][inner sep=0.75pt]   [align=left] {(a)};
    % Text Node
    \draw (122,145) node [anchor=north west][inner sep=0.75pt]   [align=left] {$ $};
    % Text Node
    \draw (267.49,47.26) node  [rotate=-270]  {$K$};
    % Text Node
    \draw (258,109) node [anchor=north west][inner sep=0.75pt]   [align=left] {(b)};
    % Text Node
    \draw (214.28,153.03) node [anchor=west] [inner sep=0.75pt]  [color={rgb, 255:red, 208; green, 2; blue, 27 }  ,opacity=1 ] [align=left] {$\displaystyle \cdots $};
    % Text Node
    \draw (132.55,153.03) node [anchor=east] [inner sep=0.75pt]  [color={rgb, 255:red, 208; green, 2; blue, 27 }  ,opacity=1 ] [align=left] {$\displaystyle \cdots $};
    % Text Node
    \draw (276.15,153.03) node   [align=left] {$\displaystyle \Sigma $};
    % Text Node
    \draw (78.42,153.28) node   [align=left] {$\displaystyle \Sigma $};
    % Text Node
    \draw (173.49,170.26) node  [rotate=-270]  {$K$};
    % Text Node
    \draw (171,223) node [anchor=north west][inner sep=0.75pt]   [align=left] {(c)};
    % Text Node
    \draw (121.21,280.38) node  [rotate=-0.14]  {$K$};
    % Text Node
    \draw (286.33,279.71) node  [rotate=-270] [align=left] {density};
    % Text Node
    \draw (71.83,279.71) node  [rotate=-270] [align=left] {driving};
    % Text Node
    \draw (237.21,280.38) node  [rotate=-0.14]  {$K$};
    % Text Node
    \draw (167.53,279.35) node [anchor=west] [inner sep=0.75pt]  [color={rgb, 255:red, 0; green, 0; blue, 0 }  ,opacity=1 ] [align=left] {$\displaystyle \cdots $};
    % Text Node
    \draw (171,313) node [anchor=north west][inner sep=0.75pt]   [align=left] {(d)};
    
    
    \end{tikzpicture}
    
    \caption{Feynman diagrams representing linear response. 
    (a) Linear response of electron density (red rectangle) is given by 
    plating a line representing the external field
    into one bare propagator. 
    (b) Single-electron self-energy,
    rewritten as a convolution between 
    the four-point kernel function 
    (i.e. self-energy of renormalized electron-hole pair)
    and the renormalized Green function (thick blue line)
    with the influence of the driving field.
    The electron lines that are connected to the external legs in (a)
    are colored red.
    Note that the blue line has its own self-energy correction,
    and the external field line is connected to one bare propagator 
    within the thick blue line.
    (c) Inserting (b) into (a). 
    (d) The diagram we get after evaluating the derivative with respect to the external field 
    i.e. after removing the external field line.
    Since the thick blue line in (b) and (c) has its own self-energy correction
    which is also represented by (b), 
    we get accurately diagrams representing renormalized electron-hole pair;
    note that $K$ can be obtained by removing a renormalized Green function from $\Sigma$.} 
    \label{fig:linear-response}
\end{figure}

Two further approximations reduce QME to quantum Boltzmann equation (QBE)
\cite{rammer1986quantum,haug2008quantum}.
Under the assumption that well-defined quasiparticles always exist and hence we have 
\begin{equation}
    G^<(\vb*{X}, \vb*{p}, T, \omega) = 2\pi \delta(\omega - E_{\vb*{p}}(\vb*{X})) f(\vb*{X}, \vb*{p}, T),
\end{equation}
where Wigner transform has been applied.
Using the condition of smooth external driving force,
by gradient expansion, the left hand side of QME becomes 
the usual form of collisionless Boltzmann equation.
With both the quasiparticle assumption the right hand side of QME 
becomes the Fermi golden rule, 
which is prevalently used as the collision integral \cite{chen2022first}.
Deviations from the quasiparticle approximation,
e.g. strong electron-phonon interaction-induced renormalization, 
may still lead to QBE,
although with modified collision integral \cite{rammer1986quantum,wais2018quantum}.

\section{The COHSEX approximation} 

\printbibliography

\end{document}
