\chapter{液滴}

与大片的流体不同,液滴的行为和它周围的环境和表面的性质有很强的联系。

\section{飞溅}

在低气压环境中,液体落在表面上不会出现飞溅现象。这意味着飞溅来自液体和周围气体的接触:如果这种接触导致液体表面不稳定,那么飞溅就能够发生。
在气压不变时,将液体落到的表面替换成多孔板,从而液体和表面之间的气体能够随着液体下降被快速导走,同样不会出现飞溅现象。
因此,液体和表面之间的气体膜决定了飞溅是否出现。

\section{浸润}

设接触角为$\theta$,则
\[
    \dd{F} = (\gamma_\text{SL} - \gamma_\text{SV}) 2 \pi r \dd{r} + \gamma_\text{VL} \cos \theta \times 2 \pi r \dd{r},
\]
因此
\begin{equation}
    \cos \theta = \frac{\gamma_\text{SL} - \text{SV}}{\gamma_\text{VL}}.
\end{equation}

\subsection{荷叶效应}

固体表面上的坑坑洼洼能够改变液滴和固体表面的接触面积。可以想象这里有两种可能的接触状态。
\concept{Wenzel状态}为液滴下部充分填充坑坑洼洼的状态。此时液滴与界面接触的有效面积大于表观面积,液体与气体接触的有效面积不变。
我们有
\[
    \dd{F} = r (\gamma_\text{SL} - \gamma_\text{SV}) 2 \pi r \dd{r} + \gamma_\text{VL} \cos \theta^* \times 2 \pi r \dd{r},
\]
即
\begin{equation}
    \cos \theta^* = r \cos \theta.
\end{equation}
\concept{Cassie状态}为液滴下部不充分填满坑坑洼洼的状态。这时液滴与界面接触的有效面积比表观面积\emph{小},但是相应的,液体与气体接触的有效面积比表观面积大。
我们有
\[
    \dd{F} = \phi_\text{s} (\gamma_\text{SL} - \gamma_\text{SV}) 2 \pi r \dd{r} + (1 - \phi_\text{s}) \gamma \times 2 \pi r \dd{r} + \gamma \times 2 \pi r \dd{r} \cos \theta^*,
\]
从而
\begin{equation}
    \cos\theta^* = - 1 + \phi_\text{s} (\cos \theta + 1).
\end{equation}

高压下Cassie状态会转化为Wenzel状态。Cassie状态能够稳定存在是因为。