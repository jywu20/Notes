\documentclass[hyperref, UTF8, a4paper]{ctexart}

\usepackage{geometry}
\usepackage{titling}
\usepackage{titlesec}
\usepackage{paralist}
\usepackage{footnote}
\usepackage{enumerate}
\usepackage{amsmath, amssymb, amsthm}
\usepackage{bbm}
\usepackage{cite}
\usepackage{graphicx}
\usepackage{subfigure}
\usepackage{physics}
\usepackage{tikz}
\usepackage{tikz-feynhand}
\usepackage[colorlinks, linkcolor=black, anchorcolor=black, citecolor=black]{hyperref}
\usepackage{prettyref}

\geometry{left=3.18cm,right=3.18cm,top=2.54cm,bottom=2.54cm}
\titlespacing{\paragraph}{0pt}{1pt}{10pt}[20pt]
\setlength{\droptitle}{-5em}
\preauthor{\vspace{-10pt}\begin{center}}
\postauthor{\par\end{center}}

\DeclareMathOperator{\timeorder}{T}
\DeclareMathOperator{\diag}{diag}
\DeclareMathOperator{\legpoly}{P}
\DeclareMathOperator{\primevalue}{P}
\DeclareMathOperator{\sgn}{sgn}
\newcommand*{\ii}{\mathrm{i}}
\newcommand*{\ee}{\mathrm{e}}
\newcommand*{\const}{\mathrm{const}}
\newcommand*{\comment}{\paragraph{注记}}
\newcommand*{\suchthat}{\quad \text{s.t.} \quad}
\newcommand*{\argmin}{\arg\min}
\newcommand*{\argmax}{\arg\max}
\newcommand*{\normalorder}[1]{: #1 :}
\newcommand*{\pair}[1]{\langle #1 \rangle}
\newcommand*{\fd}[1]{\mathcal{D} #1}

\newrefformat{sec}{第\ref{#1}节}
\newrefformat{note}{注\ref{#1}}
\newrefformat{fig}{图\ref{#1}}
\renewcommand{\autoref}{\prettyref}

\usetikzlibrary{arrows,shapes,positioning}
\usetikzlibrary{arrows.meta}
\usetikzlibrary{decorations.markings}
\tikzstyle arrowstyle=[scale=1]
\tikzstyle directed=[postaction={decorate,decoration={markings,
    mark=at position .5 with {\arrow[arrowstyle]{stealth}}}}]
\tikzstyle ray=[directed, thick]
\tikzstyle dot=[anchor=base,fill,circle,inner sep=1pt]

\renewcommand{\emph}[1]{\textbf{#1}}
\newcommand*{\concept}[1]{\underline{\textbf{#1}}}

\title{张量网络}
\author{吴晋渊}

\begin{document}

\maketitle

数学上我们可以非常轻松地定义非常抽象的结构,但是实际能够“计算”的东西基本上不是整数就是实数,而涉及后者的模型大抵总是能够离散化的。
在量子多体理论中我们谈论连续空间中的波函数,是否空间可以自然地被离散化呢?
凝聚态系统提供了一个非常自然的方案。我们所知道的晶格系统中的自由度基本上都可以“被放在晶格上”。
考虑下面的几种自由度:
\begin{itemize}
    \item 自旋自由度无需多言,总是放在格点(或是边)上,因为纯自旋自由度的形成就是靠电子定域在某处,从而其轨道完全不重要,而只有自旋可以发生涨落。
    实际上,一个自旋$1/2$自由度就是一个量子比特。
    \item 玻色子
\end{itemize}

在晶格系统中分析局域性是非常容易的,因为此时判断一个算符是否是局域的非常简单:只需要看它涉及的格点是不是能够圈在一个圈里就行了。
分析纠缠也容易很多。

量子线路模型理论上能够模拟任意的有限时间的幺正变换,只要有足够多的层数。
量子线路模型是局域的,从而

张量网络和图形演算之间有比较紧密的关系,但是它们并不完全是一回事。
一个很容易看出的区别是引脚位置。
在图形演算中,有明确物理意义的量——量子态,算符等——的指标或者说引脚是分前后顺序的(因为需要区分逆变和协变),图形上,有的向左有的向右。
张量网络中不存在这样的区分。不过,就像一个区分逆变协变的微分几何方程可以用一个只有矩阵运算的线性代数库做数值计算一样,一个图形演算中的图形原则上也总是可以画成张量网络——简单地将指标向左或是向右的意义忘掉,并且在适当的地方加一个复共轭即可。
然而,一旦我们需要在某一组基矢量下展开态矢量等——这件事经常需要做,比如在我们谈论局域性的时候——没有更多结构的图形演算是很难用的。
一个张量网络中很容易表示的MPS在一个没有附加结构的图形演算中要如何表示呢?
在一些数值计算问题中,我们需要自动地快速计算一些图形演算地图,如何将它们转化成特定基矢量下的、能够快速计算的张量网络图是非常非平凡的问题。
还有一些时候张量网络无法展示图形演算中的结构,我们需要根据图形演算的一些数据来约束张量网络的形式。

\end{document}