\documentclass[hyperref, a4paper]{article}

\usepackage{geometry}
\usepackage{titling}
\usepackage{titlesec}
\usepackage{paralist}
\usepackage{footnote}
\usepackage{enumerate}
\usepackage{amsmath, amssymb, amsthm}
\usepackage{mathtools}
\usepackage{bbm}
\usepackage{cite}
\usepackage{graphicx}
\usepackage{subfigure}
\usepackage{physics}
\usepackage{tensor}
\usepackage{siunitx}
\usepackage[version=4]{mhchem}
\usepackage{tikz}
\usepackage{xcolor}
\usepackage{listings}
\usepackage{autobreak}
\usepackage[ruled, vlined, linesnumbered]{algorithm2e}
\usepackage{xr-hyper}
\usepackage[colorlinks,unicode]{hyperref} % , linkcolor=black, anchorcolor=black, citecolor=black, urlcolor=black, filecolor=black
\usepackage{prettyref}

% Page style
\geometry{left=3.18cm,right=3.18cm,top=2.54cm,bottom=2.54cm}
\titlespacing{\paragraph}{0pt}{1pt}{10pt}[20pt]
\setlength{\droptitle}{-5em}
\preauthor{\vspace{-10pt}\begin{center}}
\postauthor{\par\end{center}}

% Math operators
\DeclareMathOperator{\timeorder}{T}
\DeclareMathOperator{\diag}{diag}
\DeclareMathOperator{\legpoly}{P}
\DeclareMathOperator{\primevalue}{P}
\DeclareMathOperator{\sgn}{sgn}
\newcommand*{\ii}{\mathrm{i}}
\newcommand*{\ee}{\mathrm{e}}
\newcommand*{\const}{\mathrm{const}}
\newcommand*{\suchthat}{\quad \text{s.t.} \quad}
\newcommand*{\argmin}{\arg\min}
\newcommand*{\argmax}{\arg\max}
\newcommand*{\normalorder}[1]{: #1 :}
\newcommand*{\pair}[1]{\langle #1 \rangle}
\newcommand*{\fd}[1]{\mathcal{D} #1}
\DeclareMathOperator{\bigO}{\mathcal{O}}

% TikZ setting
\usetikzlibrary{arrows,shapes,positioning}
\usetikzlibrary{arrows.meta}
\usetikzlibrary{decorations.markings}
\tikzstyle arrowstyle=[scale=1]
\tikzstyle directed=[postaction={decorate,decoration={markings,
    mark=at position .5 with {\arrow[arrowstyle]{stealth}}}}]
\tikzstyle ray=[directed, thick]
\tikzstyle dot=[anchor=base,fill,circle,inner sep=1pt]

% Algorithm setting
% Julia-style code
\SetKwIF{If}{ElseIf}{Else}{if}{}{elseif}{else}{end}
\SetKwFor{For}{for}{}{end}
\SetKwFor{While}{while}{}{end}
\SetKwProg{Function}{function}{}{end}
\SetArgSty{textnormal}

\newcommand*{\concept}[1]{{\textbf{#1}}}

\lstset{basicstyle=\ttfamily,
  showstringspaces=false,
  commentstyle=\color{gray},
  keywordstyle=\color{blue}
}

\title{QFT I, Homework 1}
\author{Jinyuan Wu}

\begin{document}

\maketitle

\paragraph{1. Lorentz invariance} This is problem $2.6$ on p. 28 of Schwartz.
(a) Show that
\begin{equation}
    \int_{-\infty}^{\infty} \dd k^{0} \delta\left(k^{2}-m^{2}\right) \theta\left(k^{0}\right)=\frac{1}{2 \omega_{k}}.
    \label{eq:sch-2.6-1}
\end{equation}
where $\omega_{\vb*{k}}=\sqrt{|\vb*{k}|^{2}+m^{2}}$
(b) Show the integration measure $\dd^{4} k$ is Lorentz invariant.
(c) Show that
\[
\int \frac{\dd^{3} \vb*{k}}{2 \omega_{k}}
\]
is Lorentz invariant.

\paragraph{Solution} 
\begin{itemize}
    \item[(a)] \[
        \begin{aligned}
        \int_{-\infty}^{\infty} \dd k^{0} \delta\left(k^{2}-m^{2}\right) \theta\left(k^{0}\right) &= \int_{0}^{\infty} \dd k^{0} \delta\left((k^0)^2 - \omega_k^2 \right)  \\
        &= \int_{-\infty}^\infty \dd{((k^0)^2 - \omega_k^2)} \dv{k^0}{((k^0)^2 - \omega_k^2)} \delta((k^0)^2 - \omega_k^2) \\
        &= \int_{-\infty}^\infty \dd{((k^0)^2 - \omega_k^2)} \frac{1}{2 k^0} \delta((k^0)^2 - \omega_k^2) \\
        &= \int_{-\omega_k^2}^\infty \dd{x} \frac{1}{2 \sqrt{x + \omega_k^2}} \delta(x) \\
        &= \frac{1}{2 \omega_{\vb*{k}}},
        \end{aligned}
    \]
    which is exactly \eqref{eq:sch-2.6-1}.
    \item[(b)] A Lorentz transformation changes the momentum in this way:
    \[
        k^\mu \longrightarrow k'^\mu = \Lambda\indices{^\mu_\nu} k^\mu.
    \]
    Under such a transformation, a differential form changes according to the Jacobian determinant in the way of
    \[
        \dd[4]{k} \longrightarrow \dd[4]{k'} = \abs{\pdv{k'^\mu}{k^\nu}} \dd[4]{k} = \abs*{\Lambda\indices{^\mu_\nu}} \dd[4]{k}.
    \]
    We know the determinant of any Lorentz transformation matrix is 1, and thus $\dd[4]{k} = \dd[4]{k'}$, so it is a Lorentz invariant.
    \item[(c)] Since $\dd[4]{k}$ is Lorentz invariant, as well as $\delta(k^2 - m^2)$, if $\theta(k^0)$ is also Lorentz invariant then due to 
    \[
        \int_{k^0 = -\infty}^\infty \dd[4]{k} \delta(k^2 - m^2) \theta(k^0) = \int_{k^0 = -\infty}^\infty \dd{k^0} \delta(k^2 - m^2) \theta(k^0) \dd[3]{\vb*{k}} = \int \frac{\dd[3]{\vb*{k}}}{2 \omega_{\vb*{k}}},
    \] 
    we find that $\int \dd[3]{\vb*{k}} / 2 \omega_{\vb*{k}}$ is also Lorentz invariant.
    So the key point is to verify the invariance of $\theta(k^0)$.
\end{itemize}

\paragraph{2. Yukawa potential} This is problem $3.6$ on p.~43 of Schwartz.
(a) Calculate the equation of motion for a massive vector $A_{\mu}$ from the Lagrangian
\[
\mathcal{L}=-\frac{1}{4} F_{\mu \nu}^{2}+\frac{1}{2} m^{2} A_{\mu}^{2}-A_{\mu} J^{\mu}
\]
where $F_{\mu \nu}=\partial_{\mu} A_{\nu}-\partial_{\nu} A_{\mu}$. Assuming $\partial_{\mu} J^{\mu}=0$, use the equations to find a constraint on $A_{\mu}$.
(b) For $J_{\mu}$ the current of a point charge, show that the equations of motion for $A_{0}$ reduces to
\[
A_{0}(r)=\frac{e}{4 \pi^{2} \ii r} \int_{-\infty}^{\infty} \frac{k \dd k}{k^{2}+m^{2}} \ee^{\ii k \cdot r}.
\]
(c) Evaluate this integral with contour integration to get an explicit form for $A_{0}(r)$
(d) Show that as $m \rightarrow 0$ you reproduce the Coulomb potential.

\paragraph{Solution} \begin{itemize}
    \item[(a)] By the Euler-Lagrangian equation we have
    \[
        \pdv{\mathcal{L}}{A_\mu} - \partial_\nu \pdv{\mathcal{L}}{\partial_\nu A_\mu} = 0.
    \] 
    The first term is 
    \[
        \pdv{\mathcal{L}}{A_\mu} = m^2 A^\mu - J^\mu.
    \]
    As for the second term, we have
    \[
        \begin{aligned}
            \pdv{\mathcal{L}}{\partial_\nu A_\mu} &= - \frac{1}{2} F^{\rho \sigma} \pdv{(\partial_\rho F_\sigma - \partial_\sigma F_\rho)}{\partial_\nu A_\mu} \\
            &= - \frac{1}{2} F^{\rho \sigma} (\delta_{\rho \nu} \delta_{\sigma \mu} - \delta_{\sigma \nu} \delta_{\rho \mu}) \\
            &= - \frac{1}{2} (F^{\nu \mu} - F^{\mu \nu}) \\
            &= F^{\mu \nu},
        \end{aligned}
    \]
    and therefore
    \[
        \partial_\nu \pdv{\mathcal{L}}{\partial_\nu A_\mu} = \partial_\nu F^{\mu \nu} = - \partial_\nu \partial^\nu A^\mu + \partial_\nu \partial^\mu A^\nu.
    \]
    So the equation of motion is
    \[
        m^2 A^\mu - J^\mu + \partial_\nu \partial^\nu A^\mu - \partial^\mu \partial_\nu A^\nu = 0,
    \]
    or 
    \begin{equation}
        \partial_\nu \partial^\nu A^\mu - \partial^\mu \partial_\nu A^\nu + m^2 A^\mu = J^\mu.
    \end{equation}

    Under the assumption that $\partial_\mu J^\mu = 0$, we have
    \[
        \partial_\mu \partial_\nu \partial^\nu A^\mu - \partial_\mu \partial^\mu \partial_\nu A^\nu + m^2 \partial_\mu A^\mu = \partial_\mu J^\mu = 0,
    \]
    where the first two terms cancels, and we arrive at a constraint on $A^\mu$ that
    \begin{equation}
        \partial_\mu A^\mu = 0.
    \end{equation}
    This, in turn, simplifies the equation of motion into 
    \begin{equation}
        \partial_\nu \partial^\nu A^\mu = J^\mu.
        \label{eq:yukawa-simplified-motion}
    \end{equation}
    \item[(b)] Note that vector components in \eqref{eq:yukawa-simplified-motion} are decoupled and for $A^0$ we have
    \[
        \partial_\nu \partial^\nu A^0 = J^0,
    \] 
    \item[(c)]
\end{itemize}

\paragraph{Lorentz currents} This is problem $3.2$ on p. 42 of Schwartz.
(a) Calculate the conserved currents associated wit Lorentz transformations. Express the currents in terms of the energy momentum tensor.
(b) Evaluate the currents for $\mathcal{L}=\frac{1}{2} \phi\left(\partial^{2}+m^{2}\right) \phi$.

\paragraph{Solution} 

\end{document}