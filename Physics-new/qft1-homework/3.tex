\documentclass[hyperref, a4paper]{article}

\usepackage{geometry}
\usepackage{titling}
\usepackage{titlesec}
% No longer needed, since we will use enumitem package
% \usepackage{paralist}
\usepackage{enumitem}
\usepackage{footnote}
\usepackage{enumerate}
\usepackage{amsmath, amssymb, amsthm}
\usepackage{mathtools}
\usepackage{bbm}
\usepackage{cite}
\usepackage{graphicx}
\usepackage{subfigure}
\usepackage{physics}
\usepackage{tensor}
\usepackage{siunitx}
\usepackage[version=4]{mhchem}
\usepackage{tikz}
\usepackage{xcolor}
\usepackage{listings}
\usepackage{autobreak}
\usepackage[ruled, vlined, linesnumbered]{algorithm2e}
\usepackage{xr-hyper}
\usepackage[colorlinks,unicode]{hyperref} % , linkcolor=black, anchorcolor=black, citecolor=black, urlcolor=black, filecolor=black
\usepackage{prettyref}

% Page style
\geometry{left=3.18cm,right=3.18cm,top=2.54cm,bottom=2.54cm}
\titlespacing{\paragraph}{0pt}{1pt}{10pt}[20pt]
\setlength{\droptitle}{-5em}
\preauthor{\vspace{-10pt}\begin{center}}
\postauthor{\par\end{center}}

% More compact lists 
\setlist[itemize]{
    itemindent=17pt, 
    leftmargin=1pt,
    listparindent=\parindent,
    parsep=0pt,
}

% Math operators
\DeclareMathOperator{\timeorder}{\mathcal{T}}
\DeclareMathOperator{\diag}{diag}
\DeclareMathOperator{\legpoly}{P}
\DeclareMathOperator{\primevalue}{P}
\DeclareMathOperator{\sgn}{sgn}
\newcommand*{\ii}{\mathrm{i}}
\newcommand*{\ee}{\mathrm{e}}
\newcommand*{\const}{\mathrm{const}}
\newcommand*{\suchthat}{\quad \text{s.t.} \quad}
\newcommand*{\argmin}{\arg\min}
\newcommand*{\argmax}{\arg\max}
\newcommand*{\normalorder}[1]{: #1 :}
\newcommand*{\pair}[1]{\langle #1 \rangle}
\newcommand*{\fd}[1]{\mathcal{D} #1}
\DeclareMathOperator{\bigO}{\mathcal{O}}

% TikZ setting
\usetikzlibrary{arrows,shapes,positioning}
\usetikzlibrary{arrows.meta}
\usetikzlibrary{decorations.markings}
\tikzstyle arrowstyle=[scale=1]
\tikzstyle directed=[postaction={decorate,decoration={markings,
    mark=at position .5 with {\arrow[arrowstyle]{stealth}}}}]
\tikzstyle ray=[directed, thick]
\tikzstyle dot=[anchor=base,fill,circle,inner sep=1pt]

% Algorithm setting
% Julia-style code
\SetKwIF{If}{ElseIf}{Else}{if}{}{elseif}{else}{end}
\SetKwFor{For}{for}{}{end}
\SetKwFor{While}{while}{}{end}
\SetKwProg{Function}{function}{}{end}
\SetArgSty{textnormal}

\newcommand*{\concept}[1]{{\textbf{#1}}}

% Embedded codes
\lstset{basicstyle=\ttfamily,
  showstringspaces=false,
  commentstyle=\color{gray},
  keywordstyle=\color{blue}
}

\newrefformat{fig}{Figure~\ref{#1} on page~\pageref{#1}}

\title{QFT I, Homework 3}
\author{Jinyuan Wu}

\begin{document}

\maketitle

\paragraph{Feynman propagator in position space} Calculate the Feynman propagator in position space. To get the pole structure correct, you may find it helpful to use Schwinger parameters (see Schwartz Appendix B). Take the $m \rightarrow 0$ limit of your result to find [This is problem $6.1$ on p. 77 of Schwartz.]
\begin{equation}
    \left\langle 0\left|\timeorder\left\{\phi_{0}\left(x_{1}\right) \phi_{0}\left(x_{2}\right)\right\}\right| 0\right\rangle=-\frac{1}{4 \pi^{2}} \frac{1}{\left(x_{1}-x_{2}\right)^{2}-i \varepsilon}.
\end{equation}

\paragraph{Solution} The Feynman propagator in the momentum space is $\ii / (p^2 - m^2 + \ii 0^+)$, and by Fourier transformation we have 
\begin{equation}
    \timeorder \expval*{\phi_0(x_1) \phi_0(x_2)}{0} = \int \frac{\dd[4]{p}}{(2\pi)^4} \ee^{- \ii p \cdot (x_1 - x_2)} \frac{\ii}{p^2 - m^2 + \ii 0^+} .
\end{equation}
By Schwinger parametrization 
\[
    \frac{\ii}{A} = \int_0^\infty \dd{u} \ee^{\ii u A}
\]
we have 
\[
    \begin{aligned}
        \timeorder \expval*{\phi_0(x_1) \phi_0(x_2)}{0}& = \int \frac{\dd[4]{p}}{(2\pi)^4} \ee^{- \ii p \cdot (x_1 - x_2)} \int_0^\infty \dd{u} \ee^{\ii u (p^2 - m^2 + \ii 0^+)} \\
        &= \int_0^\infty \dd{u} \ee^{\ii u (- m^2 + \ii 0^+)} \int \frac{\dd[4]{p}}{(2\pi)^4} \ee^{- \ii p \cdot (x_1 - x_2) + \ii u p^2}.
    \end{aligned}
\]

\paragraph{}

\paragraph{$\phi^3$ theory} Consider the Lagrangian for $\phi^{3}$ theory, [This is problem $7.1$ on p. 103 of Schwartz.]
\[
\mathcal{L}=-\frac{1}{2} \phi\left(\square+m^{2}\right) \phi+\frac{g}{3 !} \phi^{3}
\]
(a) Draw a tree-level Feynman diagram for the decay $\phi \rightarrow \phi \phi$. Write down the corresponding amplitude using the Feynman rules.
(b) Now consider the one-loop correction, given by
\[
\begin{tikzpicture}[x=0.75pt,y=0.75pt,yscale=-1,xscale=1]
%uncomment if require: \path (0,300); %set diagram left start at 0, and has height of 300

%Shape: Circle [id:dp7778676702936163] 
\draw   (184,131.35) .. controls (184,116.25) and (196.25,104) .. (211.35,104) .. controls (226.46,104) and (238.71,116.25) .. (238.71,131.35) .. controls (238.71,146.46) and (226.46,158.71) .. (211.35,158.71) .. controls (196.25,158.71) and (184,146.46) .. (184,131.35) -- cycle ;
%Straight Lines [id:da5919235108742498] 
\draw    (129.29,131.35) -- (184,131.35) ;
%Straight Lines [id:da3411135593402974] 
\draw    (229.29,111.35) -- (260.71,67.56) ;
%Straight Lines [id:da049732723888284314] 
\draw    (227.29,154.56) -- (258.71,198.35) ;

% Text Node
\draw (127.29,131.35) node [anchor=east] [inner sep=0.75pt]    {$\phi $};
% Text Node
\draw (262.71,64.16) node [anchor=south west] [inner sep=0.75pt]    {$\phi $};
% Text Node
\draw (260.71,201.75) node [anchor=north west][inner sep=0.75pt]    {$\phi $};
\end{tikzpicture}
\]

Write down the corresponding amplitude using the Feynman rules.
(c) Now start over and write down the diagram from part (b) in position space, in terms of integrals over the intermediate points and Wick contractions, represented with factors of $D_{F}$.
(d) Show that after you apply LSZ, what you got in (c) reduces to what you got in (b), by integrating the phases into $\delta$-functions, and integrating over those $\delta$-functions.

\paragraph{Solution}

\paragraph{}

\paragraph{Example of differential cross section} Use the Lagrangian [This is problem $7.6$ on p. 104 of Schwartz.]
\[
\mathcal{L}=-\frac{1}{2} \phi_{1} \square \phi_{1}-\frac{1}{2} \phi_{2} \square \phi_{2}+\frac{\lambda}{2} \phi_{1}\left(\partial_{\mu} \phi_{2}\right)\left(\partial_{\mu} \phi_{2}\right)+\frac{g}{2} \phi_{1}^{2} \phi_{2}
\]
to calculate the differential cross section
\[
\frac{d \sigma}{d \Omega}\left(\phi_{1} \phi_{2} \rightarrow \phi_{1} \phi_{2}\right)
\]
at tree level.

\paragraph{Solution}

\paragraph{}

\paragraph{Decay of a scalar particle} This is problem $4.2$ on p. 127 of Peskin. Consider the following Lagrangian, involving two real scalar fields $\Phi$ and $\phi$:
\[
\mathcal{L}=\frac{1}{2}\left(\partial_{\mu} \Phi\right)^{2}-\frac{1}{2} M^{2} \Phi^{2}+\frac{1}{2}\left(\partial_{\mu} \phi\right)^{2}-\frac{1}{2} m^{2} \phi^{2}-\mu \Phi \phi \phi
\]
The last term is an interaction that allows a $\Phi$ particle to decay into two $\phi$ 's, provided that $M>2 m$. Assuming that this condition is met, calculate the lifetime of the $\Phi$ to lowest order in $\mu$.

\paragraph{Solution}

\paragraph{}

\end{document}