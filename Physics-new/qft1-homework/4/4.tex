\documentclass[hyperref, a4paper]{article}

\usepackage{geometry}
\usepackage{titling}
\usepackage{titlesec}
% No longer needed, since we will use enumitem package
% \usepackage{paralist}
\usepackage{enumitem}
\usepackage{footnote}
\usepackage{enumerate}
\usepackage{amsmath, amssymb, amsthm}
\usepackage{mathtools}
\usepackage{bbm}
\usepackage{cite}
\usepackage{graphicx}
\usepackage{subfigure}
\usepackage{physics}
\usepackage{tensor}
\usepackage{siunitx}
\usepackage[version=4]{mhchem}
\usepackage{tikz}
\usepackage{xcolor}
\usepackage{listings}
\usepackage{autobreak}
\usepackage[ruled, vlined, linesnumbered]{algorithm2e}
\usepackage{nameref,zref-xr}
\zxrsetup{toltxlabel}
\zexternaldocument*[hw2-]{../2/2}[2.pdf]
\usepackage[colorlinks,unicode]{hyperref} % , linkcolor=black, anchorcolor=black, citecolor=black, urlcolor=black, filecolor=black
\usepackage[most]{tcolorbox}
\usepackage{prettyref}

% Page style
\geometry{left=3.18cm,right=3.18cm,top=2.54cm,bottom=2.54cm}
\titlespacing{\paragraph}{0pt}{1pt}{10pt}[20pt]
\setlength{\droptitle}{-5em}
\preauthor{\vspace{-10pt}\begin{center}}
\postauthor{\par\end{center}}

% More compact lists 
\setlist[itemize]{
    itemindent=17pt, 
    leftmargin=1pt,
    listparindent=\parindent,
    parsep=0pt,
}

% Math operators
\DeclareMathOperator{\timeorder}{\mathcal{T}}
\DeclareMathOperator{\diag}{diag}
\DeclareMathOperator{\legpoly}{P}
\DeclareMathOperator{\primevalue}{P}
\DeclareMathOperator{\sgn}{sgn}
\newcommand*{\ii}{\mathrm{i}}
\newcommand*{\ee}{\mathrm{e}}
\newcommand*{\const}{\mathrm{const}}
\newcommand*{\suchthat}{\quad \text{s.t.} \quad}
\newcommand*{\argmin}{\arg\min}
\newcommand*{\argmax}{\arg\max}
\newcommand*{\normalorder}[1]{: #1 :}
\newcommand*{\pair}[1]{\langle #1 \rangle}
\newcommand*{\fd}[1]{\mathcal{D} #1}
\DeclareMathOperator{\bigO}{\mathcal{O}}

% TikZ setting
\usetikzlibrary{arrows,shapes,positioning}
\usetikzlibrary{arrows.meta}
\usetikzlibrary{decorations.markings}
\tikzstyle arrowstyle=[scale=1]
\tikzstyle directed=[postaction={decorate,decoration={markings,
    mark=at position .5 with {\arrow[arrowstyle]{stealth}}}}]
\tikzstyle ray=[directed, thick]
\tikzstyle dot=[anchor=base,fill,circle,inner sep=1pt]

% Algorithm setting
% Julia-style code
\SetKwIF{If}{ElseIf}{Else}{if}{}{elseif}{else}{end}
\SetKwFor{For}{for}{}{end}
\SetKwFor{While}{while}{}{end}
\SetKwProg{Function}{function}{}{end}
\SetArgSty{textnormal}

\newcommand*{\concept}[1]{{\textbf{#1}}}

% Embedded codes
\lstset{basicstyle=\ttfamily,
  showstringspaces=false,
  commentstyle=\color{gray},
  keywordstyle=\color{blue}
}

% Reference formatting
\newrefformat{fig}{Figure~\ref{#1} on page~\pageref{#1}}

% Color boxes
\tcbuselibrary{skins, breakable, theorems}
\newtcbtheorem[number within=section]{warning}{Warning}%
  {colback=orange!5,colframe=orange!65,fonttitle=\bfseries, breakable}{warn}
\newtcbtheorem[number within=section]{note}{Note}%
  {colback=green!5,colframe=green!65,fonttitle=\bfseries, breakable}{note}

\newcommand{\hwtwo}{\href{../2/2.pdf}{Homework 2}}

\title{QFT I, Homework 4}
\author{Jinyuan Wu}

\begin{document}

\maketitle

\paragraph{Scalar QED} Consider the theory of a complex scalar field $\phi$ interacting with the electromagnetic field $A^{\mu}$. The Lagrangian is
\begin{equation}
    \mathcal{L}=-\frac{1}{4} F_{\mu \nu} F^{\mu \nu}+\left(D_{\mu} \phi\right)^{*} D^{\mu} \phi-m^{2} \phi^{*} \phi.
    \label{eq:scalar-qed}
\end{equation}
where $D_{\mu}=\partial_{\mu}+ \ii e A_{\mu}$ is the usual gauge covaraint derivative.
\begin{itemize}
    \item[(a)] Show the Lagrangian is invariant under the gauge transformations
    \begin{equation}
        \phi(x) \rightarrow \ee^{-\ii \alpha(x)} \phi(x), \quad A_{\mu}(x) \rightarrow A_{\mu}(x)+\frac{1}{e} \partial_{\mu} \alpha(x).
        \label{eq:gauge}
    \end{equation}
    \item[(b)] Derive the Feynman rules for the interaction between photons and scalar particles.
    \item[(c)] Draw all the leading-order Feynman diagrams and compute the amplitude for the process $\gamma \gamma \rightarrow \phi \phi^{*}$.
    \item[(d)] Compute the differential cross section $\dd \sigma / \dd \cos \theta$. You can take an average over all initial state polarizations. For simplicity, you can restrict your calculation in the limit $m=0$.
    \item[(e)] Draw all leading order Feynman diagrams, that contribute to the Compton scattering process $\gamma \phi \rightarrow \gamma \phi$ and compute the differential cross section $\dd \sigma / \dd \cos \theta$ with $m=0$.
\end{itemize}

\paragraph{Solution} \begin{itemize}
\item[(a)] Under the gauge transformation \eqref{eq:gauge}, we have 
\[
    F_{\mu \nu}  \to 
    F'_{\mu \nu} = \partial_\mu A'_\nu - \partial_\nu A'_\mu
    = \partial_\mu \left(A_\nu + \frac{1}{e} \partial_\nu \alpha\right) 
    - \partial_\nu \left(A_\mu + \frac{1}{e} \partial_\mu \alpha\right)
    = \partial_\mu A_\nu - \partial_\nu A_\mu = F_{\mu \nu},
\] 
so the first term in \eqref{eq:scalar-qed} remains the same. It is obvious that under \eqref{eq:gauge}
\[
    \phi^* \phi \to \phi'^* \phi' = \ee^{\ii \alpha} \phi^* \ee^{- \ii \alpha} \phi = \phi^* \phi,
\]
so the third term in \eqref{eq:scalar-qed} is also invariant. Also we have
\[
    \begin{aligned}
        D^\mu \phi \to (\partial^\mu + \ii e A'^\mu) \phi' 
        &= (\partial^\mu + \ii e A^\mu + \ii \partial^\mu \alpha) \ee^{- \ii \alpha} \phi  \\
        &= \ee^{- \ii \alpha} (\partial^\mu - \ii \partial^\mu \alpha 
        + \ii e A^\mu + \ii \partial^\mu \alpha) \phi \\
        &= \ee^{- \ii \alpha} D^\mu \phi,
    \end{aligned}
\]
and also 
\[
    (D^\mu \phi)^* = \ee^{\ii \alpha} D^\mu \phi, 
\]
so $D^\mu \phi (D^\mu \phi)^*$ is also invariant. 
Therefore \eqref{eq:scalar-qed} is invariant under \eqref{eq:gauge}.

\item[(b)] Expanding \eqref{eq:gauge} we have 
\begin{equation}
    \mathcal{L} = \mathcal{L}_\text{scalar} + \mathcal{L}_\text{vector} + \mathcal{L}_\text{scalarQED},
\end{equation} 
where $\mathcal{L}_\text{scalar}$ and $\mathcal{L}_\text{vector}$ are Lagrangians of free scalar field and 
free massless vector field, and 
\begin{equation}
    \begin{aligned}
        \mathcal{L}_\text{scalarQED} &= (D_{\mu} \phi)^{*} D^{\mu} \phi - (\partial_\mu \phi)^* \partial^\mu \phi \\
        &= e^2 A_\mu A^\mu \phi \phi^* - \ii e A_\mu \phi^* \partial^\mu \phi + \ii e \partial_\mu \phi^* A^\mu \phi .
    \end{aligned}
\end{equation}

We make the following expansion of Fourier transformation. For the complex scalar field we have 
\begin{equation}
    \phi(x) = \int \frac{\dd[3]{\vb*{p}}}{(2\pi)^3} \frac{1}{\sqrt{2 \omega_{\vb*{p}}}} (a_{\vb*{p}} \ee^{- \ii p \cdot x} + b^\dagger_{\vb*{p}} \ee^{\ii p \cdot x}).
\end{equation}
which was proved in \eqref{hw2-eq:complex-scalar-expansion} in \hwtwo. The vector field is expanded as 
\begin{equation}
    A_\mu(x) = \int \frac{\dd[3]{\vb*{p}}}{(2\pi)^3} \frac{1}{\sqrt{2 \omega_{\vb*{p}}}} \sum_{r=1}^2 \epsilon_\mu^r(\vb*{p}) \left({a}_{\vb*{p}, r}^\dagger \ee^{ \ii p \cdot x} + {a}_{\vb*{p}, r} \ee^{ - \ii p \cdot x} \right).
\end{equation}
The first term gives the following (momentum space) diagram:
\begin{equation}
    \begin{gathered}
        \begin{tikzpicture}[x=0.75pt,y=0.75pt,yscale=-1,xscale=1]
            %uncomment if require: \path (0,300); %set diagram left start at 0, and has height of 300
            
            %Straight Lines [id:da20598725730998768] 
            \draw    (100,124) .. controls (102.36,124) and (103.54,125.18) .. (103.54,127.54) .. controls (103.54,129.89) and (104.72,131.07) .. (107.07,131.07) .. controls (109.43,131.07) and (110.61,132.25) .. (110.61,134.61) .. controls (110.61,136.96) and (111.79,138.14) .. (114.14,138.14) .. controls (116.5,138.14) and (117.68,139.32) .. (117.68,141.68) .. controls (117.68,144.03) and (118.86,145.21) .. (121.21,145.21) .. controls (123.57,145.21) and (124.75,146.39) .. (124.75,148.75) .. controls (124.75,151.1) and (125.93,152.28) .. (128.28,152.28) .. controls (130.64,152.28) and (131.82,153.46) .. (131.82,155.82) .. controls (131.82,158.18) and (133,159.36) .. (135.36,159.36) .. controls (137.71,159.36) and (138.89,160.54) .. (138.89,162.89) .. controls (138.89,165.25) and (140.07,166.43) .. (142.43,166.43) .. controls (144.78,166.43) and (145.96,167.61) .. (145.96,169.96) .. controls (145.96,172.32) and (147.14,173.5) .. (149.5,173.5) -- (153,177) -- (153,177) ;
            %Straight Lines [id:da2719113994092306] 
            \draw    (153,177) .. controls (153,179.36) and (151.82,180.54) .. (149.46,180.54) .. controls (147.11,180.54) and (145.93,181.72) .. (145.93,184.07) .. controls (145.93,186.43) and (144.75,187.61) .. (142.39,187.61) .. controls (140.04,187.61) and (138.86,188.79) .. (138.86,191.14) .. controls (138.86,193.5) and (137.68,194.68) .. (135.32,194.68) .. controls (132.97,194.68) and (131.79,195.86) .. (131.79,198.21) .. controls (131.79,200.57) and (130.61,201.75) .. (128.25,201.75) .. controls (125.9,201.75) and (124.72,202.93) .. (124.72,205.28) .. controls (124.72,207.64) and (123.54,208.82) .. (121.18,208.82) .. controls (118.82,208.82) and (117.64,210) .. (117.64,212.36) .. controls (117.64,214.71) and (116.46,215.89) .. (114.11,215.89) .. controls (111.75,215.89) and (110.57,217.07) .. (110.57,219.43) .. controls (110.57,221.78) and (109.39,222.96) .. (107.04,222.96) .. controls (104.68,222.96) and (103.5,224.14) .. (103.5,226.5) -- (102,228) -- (102,228) ;
            %Straight Lines [id:da11896663722164491] 
            \draw    (153,177) -- (205,229) ;
            %Straight Lines [id:da6755259756022818] 
            \draw    (205,125) -- (153,177) ;
            
            % Text Node
            \draw (98,120.6) node [anchor=south east] [inner sep=0.75pt]    {$\mu $};
            % Text Node
            \draw (100,231.4) node [anchor=north east] [inner sep=0.75pt]    {$\nu $};
            \end{tikzpicture}
    \end{gathered} = 2 \ii e^2 \eta_{\mu \nu},
    \label{eq:vertex-1}
\end{equation}
The second and the third term gives 
\input{vertex-2.tex}
\begin{equation}
    \begin{gathered}
        \begin{tikzpicture}[x=0.75pt,y=0.75pt,yscale=-1,xscale=1]
            %uncomment if require: \path (0,300); %set diagram left start at 0, and has height of 300
            
            %Straight Lines [id:da26529668927633354] 
            \draw    (123,138) .. controls (124.67,136.33) and (126.33,136.33) .. (128,138) .. controls (129.67,139.67) and (131.33,139.67) .. (133,138) .. controls (134.67,136.33) and (136.33,136.33) .. (138,138) .. controls (139.67,139.67) and (141.33,139.67) .. (143,138) .. controls (144.67,136.33) and (146.33,136.33) .. (148,138) .. controls (149.67,139.67) and (151.33,139.67) .. (153,138) .. controls (154.67,136.33) and (156.33,136.33) .. (158,138) .. controls (159.67,139.67) and (161.33,139.67) .. (163,138) .. controls (164.67,136.33) and (166.33,136.33) .. (168,138) .. controls (169.67,139.67) and (171.33,139.67) .. (173,138) .. controls (174.67,136.33) and (176.33,136.33) .. (178,138) .. controls (179.67,139.67) and (181.33,139.67) .. (183,138) .. controls (184.67,136.33) and (186.33,136.33) .. (188,138) -- (190,138) -- (190,138) ;
            %Straight Lines [id:da8078184451432662] 
            \draw    (190,138) -- (242,190) ;
            \draw [shift={(216,164)}, rotate = 225] [fill={rgb, 255:red, 0; green, 0; blue, 0 }  ][line width=0.08]  [draw opacity=0] (12,-3) -- (0,0) -- (12,3) -- cycle    ;
            %Straight Lines [id:da015273692780729986] 
            \draw    (242,86) -- (190,138) ;
            \draw [shift={(216,112)}, rotate = 315] [fill={rgb, 255:red, 0; green, 0; blue, 0 }  ][line width=0.08]  [draw opacity=0] (12,-3) -- (0,0) -- (12,3) -- cycle    ;
            %Straight Lines [id:da8273592738769799] 
            \draw    (217.59,95.41) -- (200,113) ;
            \draw [shift={(219,94)}, rotate = 135] [fill={rgb, 255:red, 0; green, 0; blue, 0 }  ][line width=0.08]  [draw opacity=0] (12,-3) -- (0,0) -- (12,3) -- cycle    ;
            %Straight Lines [id:da2568084829615431] 
            \draw    (198.41,160.41) -- (217,179) ;
            \draw [shift={(197,159)}, rotate = 45] [fill={rgb, 255:red, 0; green, 0; blue, 0 }  ][line width=0.08]  [draw opacity=0] (12,-3) -- (0,0) -- (12,3) -- cycle    ;
            
            % Text Node
            \draw (121,138) node [anchor=east] [inner sep=0.75pt]    {$\mu $};
            % Text Node
            \draw (207.5,100.1) node [anchor=south east] [inner sep=0.75pt]    {$p$};
            % Text Node
            \draw (202,169.4) node [anchor=north east] [inner sep=0.75pt]    {$q$};
            \end{tikzpicture}
    \end{gathered} = \ii e (p_\mu + q_\mu).
\end{equation}

\begin{note*}{}{}
    Here we follow the notation of Peskin, i.e. using the arrow \emph{on} a particle line to show whether this line 
    represents a particle or a antiparticle and using the \emph{momentum} arrow to denote whether this line
    represents creation or annihilation. The direction and sign of a 4-momentum is \emph{not} represented 
    in any arrow.
\end{note*} 

\item[(c)] 
\end{itemize}

\end{document}