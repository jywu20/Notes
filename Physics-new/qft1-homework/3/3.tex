\documentclass[hyperref, a4paper]{article}

\usepackage{geometry}
\usepackage{titling}
\usepackage{titlesec}
% No longer needed, since we will use enumitem package
% \usepackage{paralist}
\usepackage{enumitem}
\usepackage{footnote}
\usepackage{enumerate}
\usepackage{amsmath, amssymb, amsthm}
\usepackage{mathtools}
\usepackage{bbm}
\usepackage{cite}
\usepackage{graphicx}
\usepackage{subfigure}
\usepackage{physics}
\usepackage{tensor}
\usepackage{siunitx}
\usepackage[version=4]{mhchem}
\usepackage{tikz}
\usepackage{xcolor}
\usepackage{listings}
\usepackage{autobreak}
\usepackage[ruled, vlined, linesnumbered]{algorithm2e}
\usepackage{nameref,zref-xr}
\zxrsetup{toltxlabel}
\zexternaldocument*[optics-]{../../optics/optics}[optics.pdf]
\zexternaldocument*[solid-]{../../solid/solid}[solid.pdf]
\usepackage[colorlinks,unicode]{hyperref} % , linkcolor=black, anchorcolor=black, citecolor=black, urlcolor=black, filecolor=black
\usepackage[most]{tcolorbox}
\usepackage{prettyref}

% Page style
\geometry{left=3.18cm,right=3.18cm,top=2.54cm,bottom=2.54cm}
\titlespacing{\paragraph}{0pt}{1pt}{10pt}[20pt]
\setlength{\droptitle}{-5em}
\preauthor{\vspace{-10pt}\begin{center}}
\postauthor{\par\end{center}}

% More compact lists 
\setlist[itemize]{
    itemindent=17pt, 
    leftmargin=1pt,
    listparindent=\parindent,
    parsep=0pt,
}

% Math operators
\DeclareMathOperator{\timeorder}{\mathcal{T}}
\DeclareMathOperator{\diag}{diag}
\DeclareMathOperator{\legpoly}{P}
\DeclareMathOperator{\primevalue}{P}
\DeclareMathOperator{\sgn}{sgn}
\newcommand*{\ii}{\mathrm{i}}
\newcommand*{\ee}{\mathrm{e}}
\newcommand*{\const}{\mathrm{const}}
\newcommand*{\suchthat}{\quad \text{s.t.} \quad}
\newcommand*{\argmin}{\arg\min}
\newcommand*{\argmax}{\arg\max}
\newcommand*{\normalorder}[1]{: #1 :}
\newcommand*{\pair}[1]{\langle #1 \rangle}
\newcommand*{\fd}[1]{\mathcal{D} #1}
\DeclareMathOperator{\bigO}{\mathcal{O}}

% TikZ setting
\usetikzlibrary{arrows,shapes,positioning}
\usetikzlibrary{arrows.meta}
\usetikzlibrary{decorations.markings}
\tikzstyle arrowstyle=[scale=1]
\tikzstyle directed=[postaction={decorate,decoration={markings,
    mark=at position .5 with {\arrow[arrowstyle]{stealth}}}}]
\tikzstyle ray=[directed, thick]
\tikzstyle dot=[anchor=base,fill,circle,inner sep=1pt]

% Algorithm setting
% Julia-style code
\SetKwIF{If}{ElseIf}{Else}{if}{}{elseif}{else}{end}
\SetKwFor{For}{for}{}{end}
\SetKwFor{While}{while}{}{end}
\SetKwProg{Function}{function}{}{end}
\SetArgSty{textnormal}

\newcommand*{\concept}[1]{{\textbf{#1}}}

% Embedded codes
\lstset{basicstyle=\ttfamily,
  showstringspaces=false,
  commentstyle=\color{gray},
  keywordstyle=\color{blue}
}

% Reference formatting
\newrefformat{fig}{Figure~\ref{#1} on page~\pageref{#1}}

% Color boxes
\tcbuselibrary{skins, breakable, theorems}
\newtcbtheorem[number within=section]{warning}{Warning}%
  {colback=orange!5,colframe=orange!65,fonttitle=\bfseries, breakable}{warn}

\title{QFT I, Homework 3}
\author{Jinyuan Wu}

\begin{document}

\maketitle

\paragraph{Feynman propagator in position space} Calculate the Feynman propagator in position space. To get the pole structure correct, you may find it helpful to use Schwinger parameters (see Schwartz Appendix B). Take the $m \rightarrow 0$ limit of your result to find [This is problem $6.1$ on p. 77 of Schwartz.]
\begin{equation}
    \left\langle 0\left|\timeorder\left\{\phi_{0}\left(x_{1}\right) \phi_{0}\left(x_{2}\right)\right\}\right| 0\right\rangle=-\frac{1}{4 \pi^{2}} \frac{1}{\left(x_{1}-x_{2}\right)^{2}-i \varepsilon}.
    \label{eq:massless-propagator}
\end{equation}

\paragraph{Solution} The Feynman propagator in the momentum space is $\ii / (p^2 - m^2 + \ii 0^+)$, and by Fourier transformation we have 
\begin{equation}
    \expval*{\timeorder\phi_0(x_1) \phi_0(x_2)}{0} = \int \frac{\dd[4]{p}}{(2\pi)^4} \ee^{- \ii p \cdot (x_1 - x_2)} \frac{\ii}{p^2 - m^2 + \ii 0^+} .
\end{equation}
By Schwinger parametrization 
\[
    \frac{\ii}{A} = \int_0^\infty \dd{u} \ee^{\ii u A}
\]
we have 
\[
    \begin{aligned}
        \expval*{\timeorder \phi_0(x_1) \phi_0(x_2)}{0}& = \int \frac{\dd[4]{p}}{(2\pi)^4} \ee^{- \ii p \cdot (x_1 - x_2)} \int_0^\infty \dd{u} \ee^{\ii u (p^2 - m^2 + \ii 0^+)} \\
        &= \int_0^\infty \dd{u} \ee^{\ii u (- m^2 + \ii 0^+)} \int \frac{\dd[4]{p}}{(2\pi)^4} \ee^{- \ii p \cdot (x_1 - x_2) + \ii u p^2}.
    \end{aligned}
\]
By the $n$-dimensional Gaussian integral 
\[
    \int \dd[n]{\vb{x}} \ee^{- \frac{1}{2} \vb{x}^\top \vb{A} \vb{x} + \vb{b}^\top \vb{x}} = \sqrt{\frac{(2\pi)^n}{\det \vb{A}}} \ee^{\frac{1}{2} \vb{b}^\top \vb{A}^{-1} \vb{b}}
\]
we have 
\[
    \begin{aligned}
        \int \dd[4]{p} \ee^{- \ii p \cdot (x_1 - x_2) + \ii u p^2} 
        &= \sqrt{\frac{(2\pi)^4}{\det(- 2 \ii u \eta_{\mu \nu})}} \ee^{\frac{1}{2} (- \ii (x_1 - x_2)_\mu) \frac{1}{- 2 \ii u} \eta^{\mu \nu} (- \ii (x_1 - x_2)_\nu)} \\
        &= \frac{ (2\pi)^2 }{\ii (2u)^2} \ee^{- \frac{\ii}{4u} (x_1 - x_2)^2},
    \end{aligned}
\]
where we set
\[
    \vb{x} = p^\mu, \quad \vb{A} = - 2 \ii u \eta_{\mu \nu}, \quad \vb{b} = - \ii (x_1 - x_2)_\mu.
\]
The Feynman propagator is now 
\[
    \begin{aligned}
        \expval*{\timeorder \phi_0(x_1) \phi_0(x_2)}{0} &= \int_0^\infty \dd{u} \ee^{\ii u (- m^2 + \ii 0^+)} \frac{1}{(2\pi)^4} \times 
        \frac{ (2\pi)^2 }{\ii (2u)^2} \ee^{- \frac{\ii}{4u} (x_1 - x_2)^2} \\
        &= - \frac{\ii}{16 \pi^2} \int_0^\infty \frac{\dd{u}}{u^2} \ee^{- \frac{\ii}{4u} (x_1 - x_2)^2 - \ii u (m^2 - \ii 0^+)} \\
        &= - \frac{\ii}{16 \pi^2} \int_0^\infty \frac{\dd{u}}{u^2} \ee^{- \ii \left( \frac{1}{4u} (x_1 - x_2)^2 + m^2 u \right) - u 0^+}.
    \end{aligned}
\]
The integral in the last line is actually a modified Bessel function. 
%According to Section~10.32.10 in \cite{DLMF1032IntegralRepresentations} we have 
%\begin{equation}
%    K_1(z) = \frac{z}{4} \int_0^\infty \ee^{- t - \frac{z^2}{4t}} \frac{\dd{t}}{t^2},
%    \label{eq:bessel-1}
%\end{equation}
%where $\abs*{\arg z} < \pi / 2$.
Section~3.324 in \cite{table-integral} gives 
\[
    \int_{0}^{\infty} \exp \left(-\frac{\beta}{4 x}-\gamma x\right) \dd{x} x=\sqrt{\frac{\beta}{\gamma}} K_{1}(\sqrt{\beta \gamma}) \quad\text{where } \operatorname{Re} \beta \geq 0, \quad \operatorname{Re} \gamma>0,
\]
and by integration by substitution we have 
\begin{equation}
    \int_0^\infty \exp(- A t - \frac{B}{4t}) \frac{\dd{t}}{t^2} = 4 \sqrt{\frac{A}{B}} K_1(\sqrt{AB}),
    \label{eq:bessel-2}
\end{equation}
where $\Re A \geq 0$ and $\Re B > 0$.
By rewriting the Feynman propagator into  
\[
    \expval*{\timeorder \phi_0(x_1) \phi_0(x_2)}{0} = - \frac{\ii}{16 \pi^2} \int_0^\infty \frac{\dd{u}}{u^2} \ee^{- \ii \left( \frac{1}{4u} (x_1 - x_2)^2 + m^2 u \right) - \frac{1}{4u} 0^+}
\]
and taking 
\[
    A = \ii m^2, \quad B = \ii (x_1 - x_2)^2 + 0^+, 
\]
we have 
\[
    \begin{aligned}
        \expval*{\timeorder \phi_0(x_1) \phi_0(x_2)}{0} &= - \frac{\ii}{16 \pi^2} \times \lim_{\epsilon \to 0} 4 \sqrt{\frac{\ii m^2}{\ii (x_1 - x_2)^2 + \epsilon}} K_1(\sqrt{ \ii m^2 (\ii (x_1 - x_2)^2 + \epsilon) }) \\
        &= - \frac{\ii}{4 \pi^2} \lim_{\epsilon \to 0} \sqrt{\frac{m^2}{ (x_1 - x_2)^2 - \ii \epsilon}} K_1(\sqrt{- m (x_1 - x_2)^2 + \ii \epsilon}),
    \end{aligned}
\]
so we obtain the Feynman propagator with the pole structure taken into account:
\begin{equation}
    \expval*{\timeorder \phi_0(x_1) \phi_0(x_2)}{0} = 
    \frac{m}{4 \pi^2 \sqrt{ - (x_1 - x_2)^2 + \ii 0^+ }} K_1(\sqrt{- m (x_1 - x_2)^2 + \ii 0^+}).
\end{equation}

The expansion of the Bessel $K$ function can be obtained using Mathematica.
We have 
\[
    K_1(z) = \frac{1}{z} + \bigO(z),
\]
so the massless limit is 
\[
    \begin{aligned}
        \expval*{\timeorder \phi_0(x_1) \phi_0(x_2)}{0} &= 
        \frac{m}{4 \pi^2 \sqrt{ - (x_1 - x_2)^2 + \ii 0^+ }} \left( \frac{1}{\sqrt{ - m (x_1 - x_2)^2 + \ii 0^+ }} + \bigO(\sqrt{m}) \right) \\
        &= \frac{m}{4\pi^2} \frac{1}{- m (x_1 - x_2)^2 + \ii 0^+ } + \bigO(m^{3/2}) \\
        &\to - \frac{1}{4 \pi^2 (x_1 - x_2)^2 - \ii 0^+} \quad \text{as $m \to 0$},
    \end{aligned}
\]
which is just \eqref{eq:massless-propagator}.

\paragraph{}

\paragraph{$\phi^3$ theory} Consider the Lagrangian for $\phi^{3}$ theory, [This is problem $7.1$ on p. 103 of Schwartz.]
\[
\mathcal{L}=-\frac{1}{2} \phi\left(\square+m^{2}\right) \phi+\frac{g}{3 !} \phi^{3}
\]
(a) Draw a tree-level Feynman diagram for the decay $\phi \rightarrow \phi \phi$. Write down the corresponding amplitude using the Feynman rules.
(b) Now consider the one-loop correction, given by
\[
\begin{tikzpicture}[x=0.75pt,y=0.75pt,yscale=-1,xscale=1]
%uncomment if require: \path (0,300); %set diagram left start at 0, and has height of 300

%Shape: Circle [id:dp7778676702936163] 
\draw   (184,131.35) .. controls (184,116.25) and (196.25,104) .. (211.35,104) .. controls (226.46,104) and (238.71,116.25) .. (238.71,131.35) .. controls (238.71,146.46) and (226.46,158.71) .. (211.35,158.71) .. controls (196.25,158.71) and (184,146.46) .. (184,131.35) -- cycle ;
%Straight Lines [id:da5919235108742498] 
\draw    (129.29,131.35) -- (184,131.35) ;
%Straight Lines [id:da3411135593402974] 
\draw    (229.29,111.35) -- (260.71,67.56) ;
%Straight Lines [id:da049732723888284314] 
\draw    (227.29,154.56) -- (258.71,198.35) ;

% Text Node
\draw (127.29,131.35) node [anchor=east] [inner sep=0.75pt]    {$\phi $};
% Text Node
\draw (262.71,64.16) node [anchor=south west] [inner sep=0.75pt]    {$\phi $};
% Text Node
\draw (260.71,201.75) node [anchor=north west][inner sep=0.75pt]    {$\phi $};
\end{tikzpicture}
\]

Write down the corresponding amplitude using the Feynman rules.
(c) Now start over and write down the diagram from part (b) in position space, in terms of integrals over the intermediate points and Wick contractions, represented with factors of $D_{F}$.
(d) Show that after you apply LSZ, what you got in (c) reduces to what you got in (b), by integrating the phases into $\delta$-functions, and integrating over those $\delta$-functions.

\paragraph{Solution} \begin{itemize}
    \item[(a)] There is only one tree-level diagram for $\phi \to \phi \phi$ which is 
    \[
        \protect\begin{tikzpicture}[x=0.75pt,y=0.75pt,yscale=-1,xscale=1]
%uncomment if require: \path (0,300); %set diagram left start at 0, and has height of 300

%Straight Lines [id:da02427116566220855] 
\draw    (149.29,151.35) -- (239.71,151.35) ;
%Straight Lines [id:da37714185398141264] 
\draw    (239.71,151.28) -- (280.71,87.56) ;
%Straight Lines [id:da46636975643705836] 
\draw    (239.71,151.35) -- (278.71,223.28) ;
%Straight Lines [id:da37869167281550453] 
\draw    (175,140) -- (201.71,140) ;
\draw [shift={(203.71,140)}, rotate = 180] [fill={rgb, 255:red, 0; green, 0; blue, 0 }  ][line width=0.08]  [draw opacity=0] (12,-3) -- (0,0) -- (12,3) -- cycle    ;
%Straight Lines [id:da13087609447271187] 
\draw    (257.71,106.28) -- (271.64,84.25) ;
\draw [shift={(272.71,82.56)}, rotate = 482.3] [fill={rgb, 255:red, 0; green, 0; blue, 0 }  ][line width=0.08]  [draw opacity=0] (12,-3) -- (0,0) -- (12,3) -- cycle    ;
%Straight Lines [id:da7350093932250426] 
\draw    (267.71,187.56) -- (281.64,209.59) ;
\draw [shift={(282.71,211.28)}, rotate = 237.7] [fill={rgb, 255:red, 0; green, 0; blue, 0 }  ][line width=0.08]  [draw opacity=0] (12,-3) -- (0,0) -- (12,3) -- cycle    ;

% Text Node
\draw (147.29,151.35) node [anchor=east] [inner sep=0.75pt]    {$\phi $};
% Text Node
\draw (282.71,84.16) node [anchor=south west] [inner sep=0.75pt]    {$\phi $};
% Text Node
\draw (280.71,221.75) node [anchor=north west][inner sep=0.75pt]    {$\phi $};
\end{tikzpicture} = \ii g.
    \]
    The tree-level amplitude is therefore $g$ since $\ii \mathcal{M} = \ii g$.
    \item[(b)] The (amputated) one-loop diagram, before integrating over all inner momenta, is
    \begin{equation}
    \begin{gathered}
        \begin{tikzpicture}[x=0.75pt,y=0.75pt,yscale=-1,xscale=1]
            %uncomment if require: \path (0,300); %set diagram left start at 0, and has height of 300
            
            %Shape: Circle [id:dp004805450328669414] 
            \draw   (204,151.35) .. controls (204,136.25) and (216.25,124) .. (231.35,124) .. controls (246.46,124) and (258.71,136.25) .. (258.71,151.35) .. controls (258.71,166.46) and (246.46,178.71) .. (231.35,178.71) .. controls (216.25,178.71) and (204,166.46) .. (204,151.35) -- cycle ;
            %Straight Lines [id:da15453068055760366] 
            \draw    (149.29,151.35) -- (204,151.35) ;
            %Straight Lines [id:da6670631824254709] 
            \draw    (249.29,131.35) -- (280.71,87.56) ;
            %Straight Lines [id:da8512297070729149] 
            \draw    (247.29,174.56) -- (287.71,226.53) ;
            %Straight Lines [id:da12241887785336236] 
            \draw    (147.94,141.35) -- (174.65,141.35) ;
            \draw [shift={(176.65,141.35)}, rotate = 180] [fill={rgb, 255:red, 0; green, 0; blue, 0 }  ][line width=0.08]  [draw opacity=0] (12,-3) -- (0,0) -- (12,3) -- cycle    ;
            %Straight Lines [id:da04852909152903351] 
            \draw    (256.71,106.28) -- (271.55,85.19) ;
            \draw [shift={(272.71,83.56)}, rotate = 485.15] [fill={rgb, 255:red, 0; green, 0; blue, 0 }  ][line width=0.08]  [draw opacity=0] (12,-3) -- (0,0) -- (12,3) -- cycle    ;
            %Straight Lines [id:da8349858256899048] 
            \draw    (280,201.46) -- (294.81,221.57) ;
            \draw [shift={(296,223.18)}, rotate = 233.63] [fill={rgb, 255:red, 0; green, 0; blue, 0 }  ][line width=0.08]  [draw opacity=0] (12,-3) -- (0,0) -- (12,3) -- cycle    ;
            %Shape: Arc [id:dp6371346371514501] 
            \draw  [draw opacity=0] (196.08,138.27) .. controls (196.56,135.45) and (197.49,132.69) .. (198.91,130.07) .. controls (202.29,123.8) and (207.87,119.41) .. (214.36,117.2) -- (226.47,144.96) -- cycle ; \draw   (196.08,138.27) .. controls (196.56,135.45) and (197.49,132.69) .. (198.91,130.07) .. controls (202.29,123.8) and (207.87,119.41) .. (214.36,117.2) ;
            %Straight Lines [id:da025947757177437136] 
            \draw    (214.36,117.2) -- (218.86,115.3) ;
            \draw [shift={(220.71,114.53)}, rotate = 517.11] [fill={rgb, 255:red, 0; green, 0; blue, 0 }  ][line width=0.08]  [draw opacity=0] (12,-3) -- (0,0) -- (12,3) -- cycle    ;
            
            %Shape: Arc [id:dp055990585045730734] 
            \draw  [draw opacity=0] (267.88,140.82) .. controls (268.94,143.47) and (269.59,146.31) .. (269.75,149.29) .. controls (270.13,156.4) and (267.66,163.06) .. (263.27,168.33) -- (238.47,150.96) -- cycle ; \draw   (267.88,140.82) .. controls (268.94,143.47) and (269.59,146.31) .. (269.75,149.29) .. controls (270.13,156.4) and (267.66,163.06) .. (263.27,168.33) ;
            %Straight Lines [id:da03941082214307445] 
            \draw    (263.27,168.33) -- (260.42,172.29) ;
            \draw [shift={(259.26,173.92)}, rotate = 305.67] [fill={rgb, 255:red, 0; green, 0; blue, 0 }  ][line width=0.08]  [draw opacity=0] (12,-3) -- (0,0) -- (12,3) -- cycle    ;
            
            %Shape: Arc [id:dp9228997543552979] 
            \draw  [draw opacity=0] (236.87,186.62) .. controls (234.19,187.61) and (231.33,188.19) .. (228.35,188.27) .. controls (221.23,188.47) and (214.64,185.84) .. (209.48,181.32) -- (227.47,156.96) -- cycle ; \draw   (236.87,186.62) .. controls (234.19,187.61) and (231.33,188.19) .. (228.35,188.27) .. controls (221.23,188.47) and (214.64,185.84) .. (209.48,181.32) ;
            %Straight Lines [id:da046155827300820684] 
            \draw    (209.48,181.32) -- (205.59,178.37) ;
            \draw [shift={(203.99,177.16)}, rotate = 397.11] [fill={rgb, 255:red, 0; green, 0; blue, 0 }  ][line width=0.08]  [draw opacity=0] (12,-3) -- (0,0) -- (12,3) -- cycle    ;
            
            
            % Text Node
            \draw (162.29,136.95) node [anchor=south] [inner sep=0.75pt]    {$p$};
            % Text Node
            \draw (270.71,80.16) node [anchor=south east] [inner sep=0.75pt]    {$q_{1}$};
            % Text Node
            \draw (298,219.78) node [anchor=south west] [inner sep=0.75pt]    {$q_{2}$};
            % Text Node
            \draw (218.71,111.13) node [anchor=south east] [inner sep=0.75pt]    {$k_{1}$};
            % Text Node
            \draw (269.27,165.73) node [anchor=north west][inner sep=0.75pt]    {$k_{2}$};
            % Text Node
            \draw (187.27,181.73) node [anchor=north west][inner sep=0.75pt]    {$k_{3}$};
            \end{tikzpicture}
    \end{gathered} = (\ii g)^3 \frac{\ii}{k_1^2 - m^2 + \ii 0^+} \frac{\ii}{k_2^2 - m^2 + \ii 0^+} \frac{\ii}{k_3^2 - m^2 + \ii 0^+},
    \label{eq:phi-phiphi-no-integral-one-loop}
\end{equation}
    and the momentum conservation equations are 
    \[
        k_1 = q_1 + k_2, \quad k_2 = k_3 + q_2 , \quad p + k_3 = k_1.
    \]
    It can be seen that $k_1, k_2$ and $k_3$ cannot be determined completely using these equations, and if we denote $k_1$ as $k$, then 
    \[
        k_2 = k_1 - q_1, \quad k_3 = k_1 - p.
    \]
    There are three momentum conservation factors and three inner momentum integrals, 
    each of the former contributing a $(2\pi)^3$ factor and each the latter contributing a $1 / (2\pi)^3$ factor.
    One $(2\pi)^3$ factor is absorbed into the definition of $\mathcal{M}$, so finally, we have a remaining $1 / (2\pi)^3$ factor and should integrate the $k$ variable.
    The one-loop amplitude is therefore 
    \[
        \ii \mathcal{M}^{(1)}(p \to q_1 + q_2) 
        = \int \frac{\dd[4]{k}}{(2\pi)^3} (\ii g)^3 \frac{\ii}{k^2 - m^2 + \ii 0^+} \frac{\ii}{(k - q_1)^2 - m^2 + \ii 0^+} \frac{\ii}{(k - p)^2 - m^2 + \ii 0^+},
    \]
    or 
    \begin{equation}
        \mathcal{M}^{(1)}(p \to q_1 + q_2) = \ii g^3 \int \frac{\dd[4]{k}}{(2\pi)^3} 
        \frac{1}{k^2 - m^2 + \ii 0^+} \frac{1}{(k - q_1)^2 - m^2 + \ii 0^+} \frac{1}{(k - p)^2 - m^2 + \ii 0^+}.
        \label{eq:phi-phiphi-one-loop-amplitude}
    \end{equation}
    \item[(c)] Now we regard (non-amputated) \eqref{eq:phi-3-one-loop} as a term in the correlation function in the position space.
    The third order perturbation in (below all so-called $\int_{-\infty}^\infty \dd{t}$ integrations are actually $\lim_{T \to \infty(1 - \ii \epsilon)} \int_{-T}^T \dd{t}$) the numerator of 
    \[
        \expval*{\timeorder \phi(x) \phi(y) \phi(z) }{\Omega} =  
            \frac{
                \expval*{\timeorder \phi_\text{I}(x) \phi_\text{I}(y) \phi_\text{I}(z) \exp(- \ii \int_{-\infty}^\infty \dd{t} H_\text{I})}{\Omega}
            }{
                \expval*{\timeorder \exp(- \ii \int_{-\infty}^\infty \dd{t} H_\text{I})}{\Omega}
            }
    \]
    is 
    \begin{equation}
        \begin{aligned}
            &\quad \frac{1}{3!} \expval*{\timeorder \phi_\text{I}(x) \phi_\text{I}(y) \phi_\text{I}(z) \left( - \ii \int_{-\infty}^\infty \dd{t} H_\text{I} \right)^3 }{\Omega} \\
            &= \quad \frac{1}{3!} \expval*{\timeorder \phi_\text{I}(x) \phi_\text{I}(y) \phi_\text{I}(z) \left( \ii \int \dd[4]{w} \frac{g}{3!} \phi_\text{I}(w)^3 \right)^3 }{\Omega}.
        \end{aligned}
        \label{eq:phi-3-third-perturbation}
    \end{equation}
    For the sake of convenience we switch to the interaction picture and write $\phi(x)$ instead of $\phi_\text{I}(x)$.
    We want to find the terms in \eqref{eq:phi-3-third-perturbation} that correspond to \eqref{eq:phi-3-one-loop}.
    We label the variables of integration in the three $\int \dd[4]{w} \phi(w)^3$ factors as $w_1, w_2$ and $w_3$.
    The structure of \eqref{eq:phi-3-one-loop} means the corresponding terms must satisfy the following conditions:
    \begin{itemize}
        \item $\phi(x)$, $\phi(y)$ and $\phi(z)$ contract with a field in different $\int \dd[4]{w} \phi(w)^3$ factors.
        \item The remaining fields in the three $\int \dd[4]{w} \phi(w)^3$ factors contract with each other.
        \item Two fields in one $\int \dd[4]{w} \phi(w)^3$ do not contract.
    \end{itemize}
    Combinatorics tells us that there are $3!$ choices for $\phi(x)$, $\phi(y)$ and $\phi(z)$ to choose the $\int \dd[4]{w} \phi(w)^3$ factors they are to contract with.
    Furthermore, there are an additional factor $3^3$ for $\phi(x)$, $\phi(y)$ and $\phi(z)$ to choose exactly which field to contract with.
    The remaining choices are how the rest of $\phi(w_1), \phi(w_2)$ and $\phi(w_3)$ contract. 
    There are 8 possible choices: we can first pick out a $\phi(w_1)$ and it may contract with 4 possible fields, 
    and the second $\phi(w_2)$ may contract with 2 possible fields and then everything is fixed.
    So finally the terms in \eqref{eq:phi-3-third-perturbation} corresponding to \eqref{eq:phi-3-one-loop} are 
    \[
        \begin{aligned}
            3! \times 3^3 \times 4 \times 2 \times \left(\frac{\ii g}{3!}\right)^3 & \int \dd[4]{w_1} \int \dd[4]{w_2} \int \dd[4]{w_3} D_F(x - w_1) D_F(y - w_2) \\
            &\quad \times D_F(z - w_3) D_F(w_1 - w_2) D_F(w_2 - w_3) D_F(w_3 - w_1).
        \end{aligned}
    \]
    We see this agrees with the result obtained by applying the position space Feynman rules on \eqref{eq:phi-3-one-loop}, which is 
    \begin{equation}
    \begin{gathered}
        \begin{tikzpicture}[x=0.75pt,y=0.75pt,yscale=-1,xscale=1]
            %uncomment if require: \path (0,300); %set diagram left start at 0, and has height of 300
            
            %Shape: Circle [id:dp7977208390713126] 
            \draw   (224,171.35) .. controls (224,156.25) and (236.25,144) .. (251.35,144) .. controls (266.46,144) and (278.71,156.25) .. (278.71,171.35) .. controls (278.71,186.46) and (266.46,198.71) .. (251.35,198.71) .. controls (236.25,198.71) and (224,186.46) .. (224,171.35) -- cycle ;
            %Straight Lines [id:da24233957833824893] 
            \draw    (169.29,171.35) -- (224,171.35) ;
            %Straight Lines [id:da32762511162194086] 
            \draw    (269.29,151.35) -- (300.71,107.56) ;
            %Straight Lines [id:da9552561758852673] 
            \draw    (267.29,194.56) -- (307.71,246.53) ;
            
            % Text Node
            \draw (167.29,171.35) node [anchor=east] [inner sep=0.75pt]    {$z$};
            % Text Node
            \draw (302.71,104.16) node [anchor=south west] [inner sep=0.75pt]    {$x$};
            % Text Node
            \draw (309.71,249.93) node [anchor=north west][inner sep=0.75pt]    {$y$};
            \end{tikzpicture}            
    \end{gathered} = 
    \begin{aligned}
        &\int \dd[4]{w_1} \int \dd[4]{w_2} \int \dd[4]{w_3} (\ii g)^3 D_F(x - w_1) D_F(y - w_2) D_F(z - w_3) \\
        &\quad \times D_F(w_1 - w_2) D_F(w_2 - w_3) D_F(w_3 - w_1).
    \end{aligned}
    \label{eq:phi-phiphi-position-feynman}
\end{equation}

    \item[(d)] Now we check whether the LSZ reduction formula connects \eqref{eq:phi-phiphi-position-feynman} and \eqref{eq:phi-phiphi-one-loop-amplitude}.
    The Fourier transformation of \eqref{eq:phi-phiphi-position-feynman} is 
    \[
        \begin{aligned}
            &\quad \int \ee^{\ii q_1 \cdot x} \dd{x} \int \ee^{\ii q_2 \cdot y} \dd{y} \int \ee^{- \ii p \cdot z} \dd{z} 
            \int \dd[4]{w_1} \int \dd[4]{w_2} \int \dd[4]{w_3} (\ii g)^3 \\
            &\quad \quad \times  D_F(x - w_1) D_F(y - w_2) D_F(z - w_3) D_F(w_1 - w_2) D_F(w_2 - w_3) D_F(w_3 - w_1) \\
            &= \int \ee^{\ii q_1 \cdot x} \dd{x} \int \ee^{\ii q_2 \cdot y} \dd{y} \int \ee^{- \ii p \cdot z} \dd{z} 
            \int \dd[4]{w_1} \int \dd[4]{w_2} \int \dd[4]{w_3} (\ii g)^3 \\
            &\quad \quad \times \int \frac{\dd[4]{k_1}}{(2\pi)^4} \frac{\ii  \ee^{- \ii k_1 \cdot (x - w_1)}}{k_1^2 - m^2 + \ii 0^+} 
            \int \frac{\dd[4]{k_2}}{(2\pi)^4} \frac{\ii  \ee^{- \ii k_2 \cdot (y - w_2)}}{k_2^2 - m^2 + \ii 0^+} 
            \int \frac{\dd[4]{k_3}}{(2\pi)^4} \frac{\ii  \ee^{- \ii k_3 \cdot (z - w_3)}}{k_3^2 - m^2 + \ii 0^+} \\
            &\quad \quad \times
            \int \frac{\dd[4]{k_4}}{(2\pi)^4} \frac{\ii  \ee^{- \ii k_4 \cdot (w_1 - w_2)}}{k_4^2 - m^2 + \ii 0^+}
            \int \frac{\dd[4]{k_5}}{(2\pi)^4} \frac{\ii  \ee^{- \ii k_5 \cdot (w_2 - w_3)}}{k_5^2 - m^2 + \ii 0^+}
            \int \frac{\dd[4]{k_6}}{(2\pi)^4} \frac{\ii  \ee^{- \ii k_6 \cdot (w_3 - w_1)}}{k_6^2 - m^2 + \ii 0^+} \\
            &= \int \frac{\dd[4]{k_1}}{(2\pi)^4} \int \dd{x} \ee^{\ii (q_1 - k_1) \cdot x}
            \int \frac{\dd[4]{k_2}}{(2\pi)^4} \int \dd{y} \ee^{\ii (q_2 - k_2) \cdot y} 
            \int \frac{\dd[4]{k_3}}{(2\pi)^4} \int \dd{z} \ee^{- \ii (p + k_3) \cdot z} \\
            &\quad  \times \int \frac{\dd[4]{k_4}}{(2\pi)^4} \int \dd{w_1} \ee^{\ii w_1 \cdot (k_1 - k_4 + k_6)} 
            \int \frac{\dd[4]{k_5}}{(2\pi)^4} \int \dd{w_2} \ee^{\ii w_2 \cdot (k_2 + k_4 - k_5)} 
            \int \frac{\dd[4]{k_6}}{(2\pi)^4} \int \dd{w_3} \ee^{\ii w_3 \cdot (k_5 + k_3 - k_6)} \\
            &\quad \times (\ii g)^3 \frac{\ii}{k_1^2 - m^2 + \ii 0^+} \frac{\ii}{k_2^2 - m^2 + \ii 0^+} \frac{\ii}{k_3^2 - m^2 + \ii 0^+}  \\
            &\quad \times \frac{\ii}{k_4^2 - m^2 + \ii 0^+} \frac{\ii}{k_5^2 - m^2 + \ii 0^+}  \frac{\ii}{k_6^2 - m^2 + \ii 0^+} .
        \end{aligned}
    \]
    We can then integrating $k_1, x, k_2, y, k_3, z$, and obtain 
    \[
        \begin{aligned}
            & \int \frac{\dd[4]{k_4}}{(2\pi)^4} \int \frac{\dd[4]{k_5}}{(2\pi)^4} \int \frac{\dd[4]{k_6}}{(2\pi)^4} 
            (2\pi)^4 \delta^{(4)}(q_1 - k_4 + k_6) (2\pi)^4 \delta^{(4)}(q_2 + k_4 - k_5) \delta^{(4)}(k_5 - p - k_6) \\
            &\quad \times (\ii g)^3 \frac{\ii}{q_1^2 - m^2 + \ii 0^+} \frac{\ii}{q_2^2 - m^2 + \ii 0^+} \frac{\ii}{p^2 - m^2 + \ii 0^+} \\
            &\quad \times \frac{\ii}{k_4^2 - m^2 + \ii 0^+} \frac{\ii}{k_5^2 - m^2 + \ii 0^+}  \frac{\ii}{k_6^2 - m^2 + \ii 0^+}.
        \end{aligned}
    \]
    This is exactly what we encountered in \eqref{eq:phi-phiphi-no-integral-one-loop} except the additional three propagators 
    corresponding to the external legs with momenta $q_1, q_2$ and $p$.
    Repeating the procedure in (b) and integrating $k_4$ and $k_5$ and renaming $k_6$ to $-k$, we have 
    \[
        \begin{aligned}
            &\quad \int \frac{\dd[4]{k_4}}{(2\pi)^4} \int \frac{\dd[4]{k_5}}{(2\pi)^4} \int \frac{\dd[4]{k_6}}{(2\pi)^4} 
            (2\pi)^4 \delta^{(4)}(q_1 - k_4 + k_6) (2\pi)^4 \delta^{(4)}(q_2 + k_4 - k_5) (2\pi)^4 \delta^{(4)}(k_5 - p - k_6) \\
            &\quad \quad \times (\ii g)^3 \frac{\ii}{q_1^2 - m^2 + \ii 0^+} \frac{\ii}{q_2^2 - m^2 + \ii 0^+} \frac{\ii}{p^2 - m^2 + \ii 0^+} \\
            &\quad \quad  \times \frac{\ii}{k_4^2 - m^2 + \ii 0^+} \frac{\ii}{k_5^2 - m^2 + \ii 0^+}  \frac{\ii}{k_6^2 - m^2 + \ii 0^+} \\
            &= \int \frac{\dd[4]{k}}{(2\pi)^4}  (2\pi)^4 \delta^{(4)}(q_1 + q_2 - p) (\ii g)^3 \frac{\ii}{q_1^2 - m^2 + \ii 0^+} \frac{\ii}{q_2^2 - m^2 + \ii 0^+} \frac{\ii}{p^2 - m^2 + \ii 0^+} \\
            &\quad \quad  \times \frac{\ii}{k^2 - m^2 + \ii 0^+} \frac{\ii}{(k - q_1)^2 - m^2 + \ii 0^+}  \frac{\ii}{(k - p)^2 - m^2 + \ii 0^+} ,
        \end{aligned}
    \]
    so finally, we obtain
    \begin{equation}
    \begin{aligned}
        &\quad \int \ee^{\ii q_1 \cdot x} \dd{x} \int \ee^{\ii q_2 \cdot y} \dd{y} \int \ee^{- \ii p \cdot z} \dd{z} 
    \begin{gathered}
        \begin{tikzpicture}[x=0.75pt,y=0.75pt,yscale=-1,xscale=1]
            %uncomment if require: \path (0,300); %set diagram left start at 0, and has height of 300
            
            %Shape: Circle [id:dp7977208390713126] 
            \draw   (224,171.35) .. controls (224,156.25) and (236.25,144) .. (251.35,144) .. controls (266.46,144) and (278.71,156.25) .. (278.71,171.35) .. controls (278.71,186.46) and (266.46,198.71) .. (251.35,198.71) .. controls (236.25,198.71) and (224,186.46) .. (224,171.35) -- cycle ;
            %Straight Lines [id:da24233957833824893] 
            \draw    (169.29,171.35) -- (224,171.35) ;
            %Straight Lines [id:da32762511162194086] 
            \draw    (269.29,151.35) -- (300.71,107.56) ;
            %Straight Lines [id:da9552561758852673] 
            \draw    (267.29,194.56) -- (307.71,246.53) ;
            
            % Text Node
            \draw (167.29,171.35) node [anchor=east] [inner sep=0.75pt]    {$z$};
            % Text Node
            \draw (302.71,104.16) node [anchor=south west] [inner sep=0.75pt]    {$x$};
            % Text Node
            \draw (309.71,249.93) node [anchor=north west][inner sep=0.75pt]    {$y$};
            \end{tikzpicture}            
    \end{gathered} \\
    &= \frac{\ii}{q_1^2 - m^2 + \ii 0^+} \frac{\ii}{q_2^2 - m^2 + \ii 0^+} \frac{\ii}{p^2 - m^2 + \ii 0^+} (2\pi)^4 \delta^{(4)}(q_1 + q_2 - p)
    \begin{gathered}
        \begin{tikzpicture}[x=0.75pt,y=0.75pt,yscale=-1,xscale=1]
            %uncomment if require: \path (0,300); %set diagram left start at 0, and has height of 300
            
            %Shape: Circle [id:dp7977208390713126] 
            \draw   (224,171.35) .. controls (224,156.25) and (236.25,144) .. (251.35,144) .. controls (266.46,144) and (278.71,156.25) .. (278.71,171.35) .. controls (278.71,186.46) and (266.46,198.71) .. (251.35,198.71) .. controls (236.25,198.71) and (224,186.46) .. (224,171.35) -- cycle ;
            %Straight Lines [id:da24233957833824893] 
            \draw    (194.71,171.35) -- (224,171.35) ;
            %Straight Lines [id:da32762511162194086] 
            \draw    (269.29,151.35) -- (286.71,126.56) ;
            %Straight Lines [id:da9552561758852673] 
            \draw    (267.29,194.56) -- (281.71,216.53) ;
            %Straight Lines [id:da02229538950783616] 
            \draw    (195.94,163.35) -- (218.65,163.35) ;
            \draw [shift={(220.65,163.35)}, rotate = 180] [fill={rgb, 255:red, 0; green, 0; blue, 0 }  ][line width=0.08]  [draw opacity=0] (12,-3) -- (0,0) -- (12,3) -- cycle    ;
            %Straight Lines [id:da8847128858390558] 
            \draw    (267.71,141.28) -- (282.55,120.19) ;
            \draw [shift={(283.71,118.56)}, rotate = 485.15] [fill={rgb, 255:red, 0; green, 0; blue, 0 }  ][line width=0.08]  [draw opacity=0] (12,-3) -- (0,0) -- (12,3) -- cycle    ;
            %Straight Lines [id:da23645625787636826] 
            \draw    (274,193.46) -- (288.81,213.57) ;
            \draw [shift={(290,215.18)}, rotate = 233.63] [fill={rgb, 255:red, 0; green, 0; blue, 0 }  ][line width=0.08]  [draw opacity=0] (12,-3) -- (0,0) -- (12,3) -- cycle    ;
            
            % Text Node
            \draw (251.35,171.35) node   [align=left] {Amp.};
            % Text Node
            \draw (208.29,157.95) node [anchor=south] [inner sep=0.75pt]    {$p$};
            % Text Node
            \draw (280.71,120.16) node [anchor=south east] [inner sep=0.75pt]    {$q_{1}$};
            % Text Node
            \draw (292,211.78) node [anchor=south west] [inner sep=0.75pt]    {$q_{2}$};
            \end{tikzpicture}                     
    \end{gathered} \\
    &= \frac{\ii}{q_1^2 - m^2 + \ii 0^+} \frac{\ii}{q_2^2 - m^2 + \ii 0^+} \frac{\ii}{p^2 - m^2 + \ii 0^+} \mel{q_1, q_2}{S}{p} .
    \end{aligned}
\end{equation}
    It is clear that we have verified the LSZ reduction formula for the diagram \eqref{eq:phi-3-one-loop}.
\end{itemize}

\paragraph{}

\paragraph{Example of differential cross section} Use the Lagrangian [This is problem $7.6$ on p. 104 of Schwartz.]
\[
\mathcal{L}=-\frac{1}{2} \phi_{1} \square \phi_{1}-\frac{1}{2} \phi_{2} \square \phi_{2}+\frac{\lambda}{2} \phi_{1}\left(\partial_{\mu} \phi_{2}\right)\left(\partial_{\mu} \phi_{2}\right)+\frac{g}{2} \phi_{1}^{2} \phi_{2}
\]
to calculate the differential cross section
\[
\frac{\dd \sigma}{\dd \Omega}\left(\phi_{1} \phi_{2} \rightarrow \phi_{1} \phi_{2}\right)
\]
at tree level.

\paragraph{Solution} The propagators of both $\phi_1$ and $\phi_2$ are the massless $\ii / (p^2 + \ii 0^+)$.
The vertices are 
\begin{equation}
    \begin{gathered}
        \begin{tikzpicture}[x=0.75pt,y=0.75pt,yscale=-1,xscale=1]
            %uncomment if require: \path (0,300); %set diagram left start at 0, and has height of 300
            
            %Straight Lines [id:da07908696186365982] 
            \draw    (203.29,171.35) -- (259.71,171.35) ;
            %Straight Lines [id:da07554445249070696] 
            \draw    (259.71,171.28) -- (287.71,126.56) ;
            %Straight Lines [id:da3572919825774752] 
            \draw    (259.71,171.35) -- (286.71,218.28) ;
            
            % Text Node
            \draw (201.29,171.35) node [anchor=east] [inner sep=0.75pt]    {$\phi _{2}$};
            % Text Node
            \draw (289.71,123.16) node [anchor=south west] [inner sep=0.75pt]    {$\phi _{1}$};
            % Text Node
            \draw (288.71,221.68) node [anchor=north west][inner sep=0.75pt]    {$\phi _{1}$};
            \end{tikzpicture}            
    \end{gathered} = \ii g,
\end{equation}
and
\begin{equation}
    \begin{gathered}
        \begin{tikzpicture}[x=0.75pt,y=0.75pt,yscale=-1,xscale=1]
            %uncomment if require: \path (0,300); %set diagram left start at 0, and has height of 300
            
            %Straight Lines [id:da30930268851238485] 
            \draw    (223.29,191.35) -- (279.71,191.35) ;
            %Straight Lines [id:da5056201632849171] 
            \draw    (279.71,191.28) -- (307.71,146.56) ;
            %Straight Lines [id:da17652677159384544] 
            \draw    (279.71,191.35) -- (306.71,238.28) ;
            %Straight Lines [id:da4233642174705481] 
            \draw    (227.94,185.35) -- (250.65,185.35) ;
            \draw [shift={(252.65,185.35)}, rotate = 180] [fill={rgb, 255:red, 0; green, 0; blue, 0 }  ][line width=0.08]  [draw opacity=0] (12,-3) -- (0,0) -- (12,3) -- cycle    ;
            %Straight Lines [id:da02355902941078103] 
            \draw    (287.71,167.28) -- (301.64,145.25) ;
            \draw [shift={(302.71,143.56)}, rotate = 482.3] [fill={rgb, 255:red, 0; green, 0; blue, 0 }  ][line width=0.08]  [draw opacity=0] (12,-3) -- (0,0) -- (12,3) -- cycle    ;
            %Straight Lines [id:da37516165265108103] 
            \draw    (299,213.46) -- (311.98,235.46) ;
            \draw [shift={(313,237.18)}, rotate = 239.45] [fill={rgb, 255:red, 0; green, 0; blue, 0 }  ][line width=0.08]  [draw opacity=0] (12,-3) -- (0,0) -- (12,3) -- cycle    ;
            
            % Text Node
            \draw (221.29,191.35) node [anchor=east] [inner sep=0.75pt]    {$\phi _{1}$};
            % Text Node
            \draw (309.71,143.16) node [anchor=south west] [inner sep=0.75pt]    {$\phi _{2}$};
            % Text Node
            \draw (308.71,241.68) node [anchor=north west][inner sep=0.75pt]    {$\phi _{2}$};
            % Text Node
            \draw (240.29,179.95) node [anchor=south] [inner sep=0.75pt]    {$p$};
            % Text Node
            \draw (293.21,152.02) node [anchor=south east] [inner sep=0.75pt]    {$q_{1}$};
            % Text Node
            \draw (308,221.92) node [anchor=south west] [inner sep=0.75pt]    {$q_{2}$};
            \end{tikzpicture}        
    \end{gathered} = - \ii \lambda q_1 \cdot q_2. 
\end{equation}

We need to calculate the amplitude
\[
    \mathcal{M}(\phi_1 p_1, \phi_2 p_2 \to \phi_1 p_3, \phi_2 p_4).
\]
Since the two vertices all have three legs, each of the tree level diagrams of such a process is made of two vertices, one inner propagator and four external lines.
Both of $\phi_1 \phi_2 \to \phi_1$ and $\phi_1 \phi_2 \to \phi_2$ vertices are available here, 
so the intermediate propagator can be either $\phi_1$ or $\phi_2$.
We also have the $s$-, $t$-, and $u$-channels as we have to divide the four external legs into two pairs.
So there are $2 \times 3 = 6$ diagrams in total. 
We introduce Mandelstam variables
\begin{equation}
    \begin{aligned}
        &s=\left(p_{1}+p_{2}\right)^{2}=\left(p_{3}+p_{4}\right)^{2} \\
        &t=\left(p_{1}-p_{3}\right)^{2}=\left(p_{4}-p_{2}\right)^{2} \\
        &u=\left(p_{1}-p_{4}\right)^{2}=\left(p_{3}-p_{2}\right)^{2},
        \end{aligned}
\end{equation} 
and now we enumerate the diagrams (we have omitted factors like $(2\pi)^4 \delta^{(4)}(p_1 + p_2 - p_3 - p_4)$).
The diagrams with $\phi_1$ propagators are 
\begin{equation}
    \begin{gathered}
        \begin{tikzpicture}[x=0.75pt,y=0.75pt,yscale=-0.75,xscale=0.75]
            %uncomment if require: \path (0,300); %set diagram left start at 0, and has height of 300
            
            %Straight Lines [id:da40464806146453514] 
            \draw    (131.71,79) -- (151.41,134.74) ;
            %Straight Lines [id:da29677111237577014] 
            \draw    (129.71,190.48) -- (151.41,134.74) ;
            %Straight Lines [id:da6874090210216235] 
            \draw    (145.41,177.47) -- (154.66,154.59) ;
            \draw [shift={(155.41,152.74)}, rotate = 472.02] [fill={rgb, 255:red, 0; green, 0; blue, 0 }  ][line width=0.08]  [draw opacity=0] (12,-3) -- (0,0) -- (12,3) -- cycle    ;
            %Straight Lines [id:da30750692746256036] 
            \draw    (130,102.46) -- (139.25,125.33) ;
            \draw [shift={(140,127.18)}, rotate = 247.98000000000002] [fill={rgb, 255:red, 0; green, 0; blue, 0 }  ][line width=0.08]  [draw opacity=0] (12,-3) -- (0,0) -- (12,3) -- cycle    ;
            %Straight Lines [id:da624592681631982] 
            \draw    (151.41,134.74) -- (245.71,134.74) ;
            %Straight Lines [id:da3695208600103763] 
            \draw    (265.41,79) -- (245.71,134.74) ;
            %Straight Lines [id:da5169890936624615] 
            \draw    (267.41,190.48) -- (245.71,134.74) ;
            %Straight Lines [id:da9821639217825291] 
            \draw    (276.96,185.61) -- (267.71,162.74) ;
            \draw [shift={(277.71,187.47)}, rotate = 247.98] [fill={rgb, 255:red, 0; green, 0; blue, 0 }  ][line width=0.08]  [draw opacity=0] (12,-3) -- (0,0) -- (12,3) -- cycle    ;
            %Straight Lines [id:da9298848590018269] 
            \draw    (267.43,99.33) -- (259.12,122.18) ;
            \draw [shift={(268.12,97.46)}, rotate = 110] [fill={rgb, 255:red, 0; green, 0; blue, 0 }  ][line width=0.08]  [draw opacity=0] (12,-3) -- (0,0) -- (12,3) -- cycle    ;
            
            % Text Node
            \draw (129.71,75.6) node [anchor=south east] [inner sep=0.75pt]    {$\phi _{1}$};
            % Text Node
            \draw (127.71,193.88) node [anchor=north east] [inner sep=0.75pt]    {$\phi _{2}$};
            % Text Node
            \draw (128,105.86) node [anchor=north east] [inner sep=0.75pt]    {$p_{1}$};
            % Text Node
            \draw (152.41,168.5) node [anchor=north west][inner sep=0.75pt]    {$p_{2}$};
            % Text Node
            \draw (198.56,131.34) node [anchor=south] [inner sep=0.75pt]    {$\phi _{1}$};
            % Text Node
            \draw (273.62,106.42) node [anchor=south west] [inner sep=0.75pt]    {$p_{3}$};
            % Text Node
            \draw (280.71,173.1) node [anchor=west] [inner sep=0.75pt]    {$p_{4}$};
            % Text Node
            \draw (267.41,75.6) node [anchor=south west] [inner sep=0.75pt]    {$\phi _{1}$};
            % Text Node
            \draw (269.41,193.88) node [anchor=north west][inner sep=0.75pt]    {$\phi _{2}$};
            \end{tikzpicture}
    \end{gathered} \eqqcolon \ii \mathcal{M}_{s1} = 
    \frac{\ii}{(p_1 + p_2)^2  + \ii 0^+} (\ii g)^2 = - \ii \frac{g^2}{s + \ii 0^+} ,
\end{equation}
% Note that the out state consists of distinguishable particles
%\begin{equation}
    \begin{gathered}
        \begin{tikzpicture}[x=0.75pt,y=0.75pt,yscale=-1,xscale=1]
            %uncomment if require: \path (0,300); %set diagram left start at 0, and has height of 300
            
            %Straight Lines [id:da968688619535284] 
            \draw    (279.45,82.74) -- (223.71,102.45) ;
            %Straight Lines [id:da6740832290961962] 
            \draw    (167.97,80.74) -- (223.71,102.45) ;
            %Straight Lines [id:da7441488382972461] 
            \draw    (186.98,78.45) -- (209.85,87.7) ;
            \draw [shift={(211.71,88.45)}, rotate = 202.02] [fill={rgb, 255:red, 0; green, 0; blue, 0 }  ][line width=0.08]  [draw opacity=0] (12,-3) -- (0,0) -- (12,3) -- cycle    ;
            %Straight Lines [id:da6192843702040114] 
            \draw    (258.14,80.78) -- (235.27,90.03) ;
            \draw [shift={(259.99,80.03)}, rotate = 157.98] [fill={rgb, 255:red, 0; green, 0; blue, 0 }  ][line width=0.08]  [draw opacity=0] (12,-3) -- (0,0) -- (12,3) -- cycle    ;
            %Straight Lines [id:da5061127031046193] 
            \draw    (223.71,102.45) -- (223.71,168.74) ;
            %Straight Lines [id:da9802616713197088] 
            \draw    (279.45,188.45) -- (223.71,168.74) ;
            %Straight Lines [id:da37815238333918666] 
            \draw    (167.97,190.45) -- (223.71,168.74) ;
            %Straight Lines [id:da8760769937547084] 
            \draw    (184.98,193.74) -- (207.85,184.49) ;
            \draw [shift={(209.71,183.74)}, rotate = 517.98] [fill={rgb, 255:red, 0; green, 0; blue, 0 }  ][line width=0.08]  [draw opacity=0] (12,-3) -- (0,0) -- (12,3) -- cycle    ;
            %Straight Lines [id:da297806715922456] 
            \draw    (259.11,191.47) -- (236.27,183.15) ;
            \draw [shift={(260.99,192.15)}, rotate = 200] [fill={rgb, 255:red, 0; green, 0; blue, 0 }  ][line width=0.08]  [draw opacity=0] (12,-3) -- (0,0) -- (12,3) -- cycle    ;
            
            % Text Node
            \draw (165.97,77.34) node [anchor=south east] [inner sep=0.75pt]    {$\phi _{1}$};
            % Text Node
            \draw (165.97,193.85) node [anchor=north east] [inner sep=0.75pt]    {$\phi _{2}$};
            % Text Node
            \draw (188.98,75.05) node [anchor=south west] [inner sep=0.75pt]    {$p_{1}$};
            % Text Node
            \draw (186.98,197.14) node [anchor=north west][inner sep=0.75pt]    {$p_{2}$};
            % Text Node
            \draw (225.71,135.59) node [anchor=west] [inner sep=0.75pt]    {$\phi _{1}$};
            % Text Node
            \draw (257.99,76.63) node [anchor=south east] [inner sep=0.75pt]    {$p_{3}$};
            % Text Node
            \draw (258.99,195.55) node [anchor=north east] [inner sep=0.75pt]    {$p_{4}$};
            % Text Node
            \draw (281.45,79.34) node [anchor=south west] [inner sep=0.75pt]    {$\phi _{1}$};
            % Text Node
            \draw (281.45,191.85) node [anchor=north west][inner sep=0.75pt]    {$\phi _{2}$};
            \end{tikzpicture}            
    \end{gathered} \eqqcolon \ii \mathcal{M}_{t1} = \frac{\ii}{(p_1 - p_3)^2 + \ii 0^+} (\ii g)^2 = - \ii \frac{g^2}{t + \ii 0^+},
\end{equation}
\begin{equation}
    \begin{gathered}
        \begin{tikzpicture}[x=0.75pt,y=0.75pt,yscale=-1,xscale=1]
            %uncomment if require: \path (0,300); %set diagram left start at 0, and has height of 300
            
            %Straight Lines [id:da3426510440775925] 
            \draw    (248.45,195.45) -- (192.71,109.45) ;
            %Straight Lines [id:da5923518951291684] 
            \draw    (136.97,87.74) -- (192.71,109.45) ;
            %Straight Lines [id:da8812815107092951] 
            \draw    (155.98,85.45) -- (178.85,94.7) ;
            \draw [shift={(180.71,95.45)}, rotate = 202.02] [fill={rgb, 255:red, 0; green, 0; blue, 0 }  ][line width=0.08]  [draw opacity=0] (12,-3) -- (0,0) -- (12,3) -- cycle    ;
            %Straight Lines [id:da30564429983765695] 
            \draw    (254.54,104.69) -- (240.71,124.07) ;
            \draw [shift={(255.71,103.07)}, rotate = 125.54] [fill={rgb, 255:red, 0; green, 0; blue, 0 }  ][line width=0.08]  [draw opacity=0] (12,-3) -- (0,0) -- (12,3) -- cycle    ;
            %Straight Lines [id:da6306489775151924] 
            \draw    (192.71,109.45) -- (192.71,175.74) ;
            %Straight Lines [id:da668839367321165] 
            \draw    (210.71,152.07) -- (192.71,175.74) ;
            %Straight Lines [id:da1836661274128617] 
            \draw    (136.97,197.45) -- (192.71,175.74) ;
            %Straight Lines [id:da7473359384686846] 
            \draw    (153.98,200.74) -- (176.85,191.49) ;
            \draw [shift={(178.71,190.74)}, rotate = 517.98] [fill={rgb, 255:red, 0; green, 0; blue, 0 }  ][line width=0.08]  [draw opacity=0] (12,-3) -- (0,0) -- (12,3) -- cycle    ;
            %Straight Lines [id:da5965521376969758] 
            \draw    (253.35,188.78) -- (241.71,171.07) ;
            \draw [shift={(254.45,190.45)}, rotate = 236.68] [fill={rgb, 255:red, 0; green, 0; blue, 0 }  ][line width=0.08]  [draw opacity=0] (12,-3) -- (0,0) -- (12,3) -- cycle    ;
            %Straight Lines [id:da6225912631978587] 
            \draw    (251.71,96.07) -- (246.17,103.82) -- (221.71,138.07) ;
            
            % Text Node
            \draw (134.97,84.34) node [anchor=south east] [inner sep=0.75pt]    {$\phi _{1}$};
            % Text Node
            \draw (134.97,200.85) node [anchor=north east] [inner sep=0.75pt]    {$\phi _{2}$};
            % Text Node
            \draw (157.98,82.05) node [anchor=south west] [inner sep=0.75pt]    {$p_{1}$};
            % Text Node
            \draw (155.98,204.14) node [anchor=north west][inner sep=0.75pt]    {$p_{2}$};
            % Text Node
            \draw (190.71,142.59) node [anchor=east] [inner sep=0.75pt]    {$\phi _{1}$};
            % Text Node
            \draw (257.71,106.47) node [anchor=north west][inner sep=0.75pt]    {$p_{3}$};
            % Text Node
            \draw (256.45,187.05) node [anchor=south west] [inner sep=0.75pt]    {$p_{4}$};
            % Text Node
            \draw (250.45,86.34) node [anchor=south west] [inner sep=0.75pt]    {$\phi _{1}$};
            % Text Node
            \draw (250.45,198.85) node [anchor=north west][inner sep=0.75pt]    {$\phi _{2}$};
            \end{tikzpicture}            
    \end{gathered} \eqqcolon \ii \mathcal{M}_{u1} = \frac{\ii}{(p_1 - p_4)^2 + \ii 0^+} (\ii g)^2 = - \ii \frac{g^2}{u + \ii 0^+}.
\end{equation}
The expressions of diagrams with $\phi_2$ propagators are slightly more complicated. We have 
\begin{equation}
    \begin{gathered}
        \begin{tikzpicture}[x=0.75pt,y=0.75pt,yscale=-0.75,xscale=0.75]
        %uncomment if require: \path (0,300); %set diagram left start at 0, and has height of 300
        
        %Straight Lines [id:da5433544960650969] 
        \draw    (151.71,99) -- (171.41,154.74) ;
        %Straight Lines [id:da2211889438504424] 
        \draw    (149.71,210.48) -- (171.41,154.74) ;
        %Straight Lines [id:da8914945534164957] 
        \draw    (165.41,197.47) -- (174.66,174.59) ;
        \draw [shift={(175.41,172.74)}, rotate = 472.02] [fill={rgb, 255:red, 0; green, 0; blue, 0 }  ][line width=0.08]  [draw opacity=0] (12,-3) -- (0,0) -- (12,3) -- cycle    ;
        %Straight Lines [id:da8502419018444731] 
        \draw    (150,122.46) -- (159.25,145.33) ;
        \draw [shift={(160,147.18)}, rotate = 247.98000000000002] [fill={rgb, 255:red, 0; green, 0; blue, 0 }  ][line width=0.08]  [draw opacity=0] (12,-3) -- (0,0) -- (12,3) -- cycle    ;
        %Straight Lines [id:da13030876069840058] 
        \draw    (171.41,154.74) -- (265.71,154.74) ;
        %Straight Lines [id:da12916587008636737] 
        \draw    (285.41,99) -- (265.71,154.74) ;
        %Straight Lines [id:da8014677527354779] 
        \draw    (287.41,210.48) -- (265.71,154.74) ;
        %Straight Lines [id:da25455304099872467] 
        \draw    (296.96,205.61) -- (287.71,182.74) ;
        \draw [shift={(297.71,207.47)}, rotate = 247.98] [fill={rgb, 255:red, 0; green, 0; blue, 0 }  ][line width=0.08]  [draw opacity=0] (12,-3) -- (0,0) -- (12,3) -- cycle    ;
        %Straight Lines [id:da09919924539968838] 
        \draw    (287.43,119.33) -- (279.12,142.18) ;
        \draw [shift={(288.12,117.46)}, rotate = 110] [fill={rgb, 255:red, 0; green, 0; blue, 0 }  ][line width=0.08]  [draw opacity=0] (12,-3) -- (0,0) -- (12,3) -- cycle    ;
        
        % Text Node
        \draw (149.71,95.6) node [anchor=south east] [inner sep=0.75pt]    {$\phi _{1}$};
        % Text Node
        \draw (147.71,213.88) node [anchor=north east] [inner sep=0.75pt]    {$\phi _{2}$};
        % Text Node
        \draw (148,125.86) node [anchor=north east] [inner sep=0.75pt]    {$p_{1}$};
        % Text Node
        \draw (172.41,188.5) node [anchor=north west][inner sep=0.75pt]    {$p_{2}$};
        % Text Node
        \draw (218.56,151.34) node [anchor=south] [inner sep=0.75pt]    {$\phi _{2}$};
        % Text Node
        \draw (293.62,126.42) node [anchor=south west] [inner sep=0.75pt]    {$p_{3}$};
        % Text Node
        \draw (300.71,193.1) node [anchor=west] [inner sep=0.75pt]    {$p_{4}$};
        % Text Node
        \draw (287.41,95.6) node [anchor=south west] [inner sep=0.75pt]    {$\phi _{1}$};
        % Text Node
        \draw (289.41,213.88) node [anchor=north west][inner sep=0.75pt]    {$\phi _{2}$};
        \end{tikzpicture}
    \end{gathered} \quad 
    \begin{aligned}[t]
         &\eqqcolon \ii \mathcal{M}_{s2} = \frac{\ii}{(p_1 + p_2)^2 + \ii 0^+} ( \ii \lambda (p_1 + p_2) \cdot p_2) ( \ii \lambda (p_1 + p_2) \cdot p_4) \\
         &= - \ii \frac{\lambda^2 ((p_1 + p_2) \cdot p_2)((p_1 + p_2) \cdot p_4)}{s + \ii 0^+ },
    \end{aligned}
\end{equation}
%\begin{equation}
    \begin{gathered}
        \begin{tikzpicture}[x=0.75pt,y=0.75pt,yscale=-1,xscale=1]
            %uncomment if require: \path (0,300); %set diagram left start at 0, and has height of 300
            
            %Straight Lines [id:da3658497276365975] 
            \draw    (299.45,102.74) -- (243.71,122.45) ;
            %Straight Lines [id:da1392939324999145] 
            \draw    (187.97,100.74) -- (243.71,122.45) ;
            %Straight Lines [id:da5792004960946098] 
            \draw    (206.98,98.45) -- (229.85,107.7) ;
            \draw [shift={(231.71,108.45)}, rotate = 202.02] [fill={rgb, 255:red, 0; green, 0; blue, 0 }  ][line width=0.08]  [draw opacity=0] (12,-3) -- (0,0) -- (12,3) -- cycle    ;
            %Straight Lines [id:da5743221216400951] 
            \draw    (278.14,100.78) -- (255.27,110.03) ;
            \draw [shift={(279.99,100.03)}, rotate = 157.98] [fill={rgb, 255:red, 0; green, 0; blue, 0 }  ][line width=0.08]  [draw opacity=0] (12,-3) -- (0,0) -- (12,3) -- cycle    ;
            %Straight Lines [id:da927376856169188] 
            \draw    (243.71,122.45) -- (243.71,188.74) ;
            %Straight Lines [id:da8947002419253893] 
            \draw    (299.45,208.45) -- (243.71,188.74) ;
            %Straight Lines [id:da2196394430244235] 
            \draw    (187.97,210.45) -- (243.71,188.74) ;
            %Straight Lines [id:da06264839191963989] 
            \draw    (204.98,213.74) -- (227.85,204.49) ;
            \draw [shift={(229.71,203.74)}, rotate = 517.98] [fill={rgb, 255:red, 0; green, 0; blue, 0 }  ][line width=0.08]  [draw opacity=0] (12,-3) -- (0,0) -- (12,3) -- cycle    ;
            %Straight Lines [id:da7986552586213247] 
            \draw    (279.11,211.47) -- (256.27,203.15) ;
            \draw [shift={(280.99,212.15)}, rotate = 200] [fill={rgb, 255:red, 0; green, 0; blue, 0 }  ][line width=0.08]  [draw opacity=0] (12,-3) -- (0,0) -- (12,3) -- cycle    ;
            
            % Text Node
            \draw (185.97,97.34) node [anchor=south east] [inner sep=0.75pt]    {$\phi _{1}$};
            % Text Node
            \draw (185.97,213.85) node [anchor=north east] [inner sep=0.75pt]    {$\phi _{2}$};
            % Text Node
            \draw (208.98,95.05) node [anchor=south west] [inner sep=0.75pt]    {$p_{1}$};
            % Text Node
            \draw (206.98,217.14) node [anchor=north west][inner sep=0.75pt]    {$p_{2}$};
            % Text Node
            \draw (245.71,155.59) node [anchor=west] [inner sep=0.75pt]    {$\phi _{2}$};
            % Text Node
            \draw (277.99,96.63) node [anchor=south east] [inner sep=0.75pt]    {$p_{3}$};
            % Text Node
            \draw (278.99,215.55) node [anchor=north east] [inner sep=0.75pt]    {$p_{4}$};
            % Text Node
            \draw (301.45,99.34) node [anchor=south west] [inner sep=0.75pt]    {$\phi _{1}$};
            % Text Node
            \draw (301.45,211.85) node [anchor=north west][inner sep=0.75pt]    {$\phi _{2}$};
            \end{tikzpicture}            
    \end{gathered} \quad 
    \begin{aligned}[t]
         &\eqqcolon \ii \mathcal{M}_{t2} = \frac{\ii}{(p_1 - p_3)^2 + \ii 0^+} (- \ii \lambda p_3 \cdot (p_1 - p_3)) ( \ii \lambda (p_1 - p_3) \cdot p_4) \\
         &= \ii \frac{\lambda^2 (p_3 \cdot (p_1 - p_3))((p_1 - p_3) \cdot p_4)}{t + \ii 0^+ },
    \end{aligned}
\end{equation}
\begin{equation}
    \begin{gathered}
        \begin{tikzpicture}[x=0.75pt,y=0.75pt,yscale=-1,xscale=1]
            %uncomment if require: \path (0,300); %set diagram left start at 0, and has height of 300
            
            %Straight Lines [id:da7163777291340228] 
            \draw    (268.45,215.45) -- (212.71,129.45) ;
            %Straight Lines [id:da9665644235261202] 
            \draw    (156.97,107.74) -- (212.71,129.45) ;
            %Straight Lines [id:da46015634659975735] 
            \draw    (175.98,105.45) -- (198.85,114.7) ;
            \draw [shift={(200.71,115.45)}, rotate = 202.02] [fill={rgb, 255:red, 0; green, 0; blue, 0 }  ][line width=0.08]  [draw opacity=0] (12,-3) -- (0,0) -- (12,3) -- cycle    ;
            %Straight Lines [id:da45449423366142727] 
            \draw    (274.54,124.69) -- (260.71,144.07) ;
            \draw [shift={(275.71,123.07)}, rotate = 125.54] [fill={rgb, 255:red, 0; green, 0; blue, 0 }  ][line width=0.08]  [draw opacity=0] (12,-3) -- (0,0) -- (12,3) -- cycle    ;
            %Straight Lines [id:da5734602899087693] 
            \draw    (212.71,129.45) -- (212.71,195.74) ;
            %Straight Lines [id:da3140324014019511] 
            \draw    (230.71,172.07) -- (212.71,195.74) ;
            %Straight Lines [id:da5132193871688708] 
            \draw    (156.97,217.45) -- (212.71,195.74) ;
            %Straight Lines [id:da42914137190821] 
            \draw    (173.98,220.74) -- (196.85,211.49) ;
            \draw [shift={(198.71,210.74)}, rotate = 517.98] [fill={rgb, 255:red, 0; green, 0; blue, 0 }  ][line width=0.08]  [draw opacity=0] (12,-3) -- (0,0) -- (12,3) -- cycle    ;
            %Straight Lines [id:da9510920186382812] 
            \draw    (273.35,208.78) -- (261.71,191.07) ;
            \draw [shift={(274.45,210.45)}, rotate = 236.68] [fill={rgb, 255:red, 0; green, 0; blue, 0 }  ][line width=0.08]  [draw opacity=0] (12,-3) -- (0,0) -- (12,3) -- cycle    ;
            %Straight Lines [id:da1510064266893021] 
            \draw    (271.71,116.07) -- (266.17,123.82) -- (241.71,158.07) ;
            
            % Text Node
            \draw (154.97,104.34) node [anchor=south east] [inner sep=0.75pt]    {$\phi _{1}$};
            % Text Node
            \draw (154.97,220.85) node [anchor=north east] [inner sep=0.75pt]    {$\phi _{2}$};
            % Text Node
            \draw (177.98,102.05) node [anchor=south west] [inner sep=0.75pt]    {$p_{1}$};
            % Text Node
            \draw (175.98,224.14) node [anchor=north west][inner sep=0.75pt]    {$p_{2}$};
            % Text Node
            \draw (210.71,162.59) node [anchor=east] [inner sep=0.75pt]    {$\phi _{2}$};
            % Text Node
            \draw (277.71,126.47) node [anchor=north west][inner sep=0.75pt]    {$p_{3}$};
            % Text Node
            \draw (276.45,207.05) node [anchor=south west] [inner sep=0.75pt]    {$p_{4}$};
            % Text Node
            \draw (270.45,106.34) node [anchor=south west] [inner sep=0.75pt]    {$\phi _{1}$};
            % Text Node
            \draw (270.45,218.85) node [anchor=north west][inner sep=0.75pt]    {$\phi _{2}$};
            \end{tikzpicture}            
    \end{gathered} \quad \begin{aligned}[t]
        &\eqqcolon \ii \mathcal{M}_{u2} = \frac{\ii}{(p_1 - p_4)^2 + \ii 0^+} ( \ii \lambda (p_4 - p_1) \cdot p_2) ( \ii \lambda (p_4 - p_1) \cdot p_4) \\
        &= - \ii \frac{\lambda^2 ((p_4 - p_1) \cdot p_2)((p_4 - p_1) \cdot p_4)}{u + \ii 0^+ }.
   \end{aligned}
\end{equation}
Summing all diagrams up, we get the tree-level amplitude
\[
    \ii \mathcal{M}^{(1)} = \ii \mathcal{M}_{s1} + \ii \mathcal{M}_{s2} + \ii \mathcal{M}_{u1} + \ii \mathcal{M}_{u2} ,
\]
or 
\begin{equation}
    \begin{aligned}
        \mathcal{M}^{(1)} = &- g^2 \left( \frac{1}{s} + \frac{1}{u}  \right) - \lambda^2 \Biggl( \frac{ ((p_1 + p_2) \cdot p_2)((p_1 + p_2) \cdot p_4)}{s } + \frac{ ((p_4 - p_1) \cdot p_2)((p_4 - p_1) \cdot p_4)}{u } \Biggr).
    \end{aligned}
\end{equation}
Since the particles are all massless, we can use Eq.~(4.85) in Peskin, which is 
\begin{equation}
    \left(\dv{\sigma}{\Omega}\right)_\text{CM} = \frac{\abs*{\mathcal{M}}^2}{64 \pi^2 E_\text{CM}^2},  
\end{equation}
to calculate the cross section in the center-of-mass frame.

\begin{warning*}{}{}
    Note that not all the three $t$, $s$ and $u$ channels appear in this problem, 
    because the out state in this problem consists of two \emph{distinguishable} particles.
    The fact that for distinguishable particles only two channels appear can be verified using Wick theorem.
    \cite{stackexchange-answer} discusses another example of lack of a channel for an out state with 
    distinguishable particles. Note that in \cite{stackexchange-answer} the vertex connects two external 
    lines with the \emph{same} type, so we have the $s$-channel and the $t$-channel, while in this problem 
    the vertex connects one $\phi_1$ external line and one $\phi_2$ external line, 
    so we have the $s$-channel and the $u$-channel.
\end{warning*}

\paragraph{}

\paragraph{Decay of a scalar particle} This is problem $4.2$ on p. 127 of Peskin. 
Consider the following Lagrangian, involving two real scalar fields $\Phi$ and $\phi$:
\[
\mathcal{L}=\frac{1}{2}\left(\partial_{\mu} \Phi\right)^{2}-\frac{1}{2} M^{2} \Phi^{2}+\frac{1}{2}\left(\partial_{\mu} \phi\right)^{2}-\frac{1}{2} m^{2} \phi^{2}-\mu \Phi \phi \phi
\]
The last term is an interaction that allows a $\Phi$ particle to decay into two $\phi$'s, provided that $M>2 m$.
Assuming that this condition is met, calculate the lifetime of the $\Phi$ to lowest order in $\mu$.

\paragraph{Solution} There is only one vertex 
\begin{equation}
    \begin{gathered}
        \begin{tikzpicture}[x=0.75pt,y=0.75pt,yscale=-1,xscale=1]
            %uncomment if require: \path (0,300); %set diagram left start at 0, and has height of 300
            
            %Straight Lines [id:da2968754575447763] 
            \draw    (210.41,77) -- (185.71,139.74) ;
            %Straight Lines [id:da394889874260383] 
            \draw    (185.71,139.74) -- (120,139.74) ;
            %Straight Lines [id:da967293945652087] 
            \draw    (210.41,202.48) -- (185.71,139.74) ;
            %Straight Lines [id:da5057235711144976] 
            \draw    (125.94,131.35) -- (148.65,131.35) ;
            \draw [shift={(150.65,131.35)}, rotate = 180] [fill={rgb, 255:red, 0; green, 0; blue, 0 }  ][line width=0.08]  [draw opacity=0] (12,-3) -- (0,0) -- (12,3) -- cycle    ;
            %Straight Lines [id:da28162487571439243] 
            \draw    (190.71,101.28) -- (199.96,78.41) ;
            \draw [shift={(200.71,76.56)}, rotate = 472.02] [fill={rgb, 255:red, 0; green, 0; blue, 0 }  ][line width=0.08]  [draw opacity=0] (12,-3) -- (0,0) -- (12,3) -- cycle    ;
            %Straight Lines [id:da07064018933048888] 
            \draw    (207,169.46) -- (216.25,192.33) ;
            \draw [shift={(217,194.18)}, rotate = 247.98000000000002] [fill={rgb, 255:red, 0; green, 0; blue, 0 }  ][line width=0.08]  [draw opacity=0] (12,-3) -- (0,0) -- (12,3) -- cycle    ;
            
            % Text Node
            \draw (118,139.74) node [anchor=east] [inner sep=0.75pt]    {$\Phi $};
            % Text Node
            \draw (212.41,73.6) node [anchor=south west] [inner sep=0.75pt]    {$\phi $};
            % Text Node
            \draw (212.41,205.88) node [anchor=north west][inner sep=0.75pt]    {$\phi $};
            % Text Node
            \draw (138.29,126.95) node [anchor=south] [inner sep=0.75pt]    {$p$};
            % Text Node
            \draw (193.71,85.52) node [anchor=south east] [inner sep=0.75pt]    {$q_{1}$};
            % Text Node
            \draw (214,178.42) node [anchor=south west] [inner sep=0.75pt]    {$q_{2}$};
            \end{tikzpicture}
    \end{gathered} = - 2 \ii \mu.
\end{equation}
The tree level amplitude is therefore $\ii \mathcal{M} = - \ii \mu$.
The relation between the amplitude and the decay rate for a particle at rest is 
\begin{equation}
    \dd \Gamma=\frac{1}{2 m_{\mathcal{A}}}\left(\prod_{f} \frac{\dd^{3} p_{f}}{(2 \pi)^{3}} \frac{1}{2 E_{f}}\right)\left|\mathcal{M}\left(m_{\mathcal{A}} \rightarrow\left\{p_{f}\right\}\right)\right|^{2}(2 \pi)^{4} \delta^{(4)}\left(p_{\mathcal{A}}-\sum p_{f}\right),
\end{equation}
by Peskin Eq.~(4.86), where $p_\mathcal{A}$ and $p_f$ in the $\delta$-function factor are on-shell.
In this case the total rate is 
\[
    \begin{aligned}
        {\Gamma} &= \frac{1}{2} \times \frac{1}{2M} \int \frac{\dd[3]{\vb*{q}_1}}{(2\pi)^3} \frac{1}{2 E_{\vb*{q}_1}} \int \frac{\dd[3]{\vb*{q}_2}}{(2\pi)^3} \frac{1}{2 E_{\vb*{q}_2}} \abs*{-\ii 2 \mu}^2 (2\pi)^4 \delta(M - E_{\vb*{q}_1} - E_{\vb*{q}_2}) \delta^{(3)}(\vb*{q}_1 + \vb*{q}_2),
    \end{aligned}
\]
where the factor $1/2$ comes from the fact that the two output particles are identical and the phase space has to be shrunk to half 
(or from another perspective, when the two output particles are identical we get an extra $1 / \sqrt{2}$ factor
in the output wave function, and therefore we have an extra $1/2$ factor in the probability).
Evaluating this expression we have 
\[
    \begin{aligned}
        \Gamma &= \frac{1}{4M} \frac{1}{(2\pi)^2} 4 \mu^2 \int \frac{\dd[3]{\vb*{q}_1}}{2 E_{\vb*{q}_1}} \int \frac{\dd[3]{\vb*{q}_2}}{2 E_{\vb*{q}_2}} \delta^{(3)}(\vb*{q}_1 + \vb*{q}_2) \delta(M - E_{\vb*{q}_1} - E_{\vb*{q}_2}) \\
        &= \frac{\mu^2}{M (2\pi)^2} \int \frac{\dd[3]{\vb*{q}}}{(2 E_{\vb*{q}})^2} \delta(M - 2 E_{\vb*{q}}) \\
        &= \frac{\mu^2}{M (2\pi)^2} \int \frac{\dd[3]{\vb*{q}}}{M^2} \delta(M - 2 E_{\vb*{q}}) \\
        &= \frac{\mu^2}{M (2\pi)^2} \eval{\frac{4 \pi \abs*{\vb*{q}}^2}{M^2}}_{M = 2 E_{\vb*{q}}} \int_0^\infty \dd{\abs*{\vb*{q}}} \delta(M - 2 \sqrt{\abs*{\vb*{q}}^2 + m^2}) \\
        &= \frac{\mu^2}{M (2\pi)^2} \frac{4\pi}{M^2} \left( \frac{M^2}{4} - m^2 \right) \frac{1}{\eval{\pdv{(2 \sqrt{\abs*{\vb*{q}}^2 + m^2} - M)}{\abs*{\vb*{q}}}}_{M = 2 E_{\vb*{q}}} } \\
        &= \frac{\mu^2}{M (2\pi)^2} \frac{4\pi}{M^2} \left( \frac{M^2}{4} - m^2 \right) \frac{1}{ \frac{2}{M/2} \sqrt{\frac{M^2}{4} - m^2} } \\
        &= \frac{\mu^2}{8 \pi M^2} \sqrt{M^2 - 4m^2}.
    \end{aligned}
\]
So we have explicit expressions for the decay rate and the lifetime:
\begin{equation}
    \Gamma = \frac{\mu^2}{8 \pi M^2} \sqrt{M^2 - 4m^2}, \quad \tau = \frac{1}{\Gamma} = \frac{8 \pi M^2}{\mu^2 \sqrt{M^2 - 4 m^2}}.
\end{equation}

\paragraph{}

\bibliographystyle{plain}
\bibliography{3} 

\end{document}