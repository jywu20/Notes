%%%%%%%%%%%%%%%%%%%%%%%%%%%%%%%%%%%%%%%%%%%%%%%%%%%%%

Homework 1, problem 1

Consider a Lorentz transformation
\[
    x^\mu \longrightarrow {x'}^{\mu} = \Lambda\indices{^{\mu}_\nu} x^\nu,
\] 
under which we have
\[
    \begin{aligned}
        \dd[4]{x'} &= \dd{{x'}^0} \wedge \dd{{x'}^1} \wedge \dd{{x'}^2} \wedge \dd{{x'}^3} \\
        &= \left( \Lambda\indices{^0_\mu} \dd{x^\mu} \right) \wedge \left( \Lambda\indices{^1_\nu} \dd{x^\nu} \right) \wedge \left( \Lambda\indices{^2_\rho} \dd{x^\rho} \right) \wedge \left( \Lambda\indices{^3_\sigma} \dd{x^\sigma} \right).
    \end{aligned}
\]
Since $\dd{x^\mu} \wedge \dd{x^\mu} = 0$, only when $\mu, \nu, \rho, \sigma$ all take different values can the second line have a non-zero value, so we can assume that $\mu, \nu, \rho, \sigma$ is a permutation of $0, 1, 2, 3$.
Also, by the definition of wedging, if $\mu, \nu, \rho, \sigma$ is an even permutation of $0, 1, 2, 3$, then
\[
    \dd{x^\mu} \wedge \dd{x^\nu} \wedge \dd{x^\rho} \wedge \dd{x^\sigma} = \dd{x^0} \wedge \dd{x^1} \wedge \dd{x^2} \wedge \dd{x^3},
\]
and if $\mu, \nu, \rho, \sigma$ is an odd permutation of $0, 1, 2, 3$, then
\[
    \dd{x^\mu} \wedge \dd{x^\nu} \wedge \dd{x^\rho} \wedge \dd{x^\sigma} = - \dd{x^0} \wedge \dd{x^1} \wedge \dd{x^2} \wedge \dd{x^3},
\]
So $\dd[4]{x'}$ can be rewritten as 
\[
    \begin{aligned}
        \dd[4]{x'} &= \Lambda\indices{^0_\mu} \Lambda\indices{^1_\nu} \Lambda\indices{^2_\rho} \Lambda\indices{^3_\sigma} \epsilon^{\mu \nu \rho \sigma} \dd{x^0} \wedge \dd{x^1} \wedge \dd{x^2} \wedge \dd{x^3} \\
        &= \Lambda\indices{^0_\mu} \Lambda\indices{^1_\nu} \Lambda\indices{^2_\rho} \Lambda\indices{^3_\sigma} \epsilon^{\mu \nu \rho \sigma} \dd[4]{x}.
    \end{aligned}
\]
We immediately find that $\Lambda\indices{^0_\mu} \Lambda\indices{^1_\nu} \Lambda\indices{^2_\rho} \Lambda\indices{^3_\sigma} \epsilon^{\mu \nu \rho \sigma}$ is the determinant of $\Lambda\indices{^\mu_\nu}$, and hence is $1$, so we have
\[
    \dd[4]{x'} = \det \Lambda\indices{^\mu_\nu} \dd[4]{x} = \dd[4]{x},
\]
and therefore $\dd[4]{x}$ is Lorentz invariant.

Actually $\Lambda\indices{^0_\mu} \Lambda\indices{^1_\nu} \Lambda\indices{^2_\rho} \Lambda\indices{^3_\sigma} \epsilon^{\mu \nu \rho \sigma}$ is just the Jacobian determinant of the Lorentz transformation.
We have
\[
    \dd[4]{x'} = \abs{\pdv{x'^\mu}{x^\nu}} \dd[4]{x},
\]
and by definition 
\[
    \pdv{x'^\mu}{x^\nu} = \Lambda\indices{^\mu_\nu}.    
\]
Since the determinant of any Lorentz transformation matrix is $1$, we have $\dd[4]{x} = \dd[4]{x'}$.
Also, differential geometry guarantees that 
\[
    \epsilon = \sqrt{\abs*{g}} \dd{x^1} \wedge \cdots \dd{x^n}
\]
is a (pseudo) tensor and so is its Hodge star, and in Minkowski spacetime $g = 1$, so $\dd[4]{x}$ indeed is a scalar. 
%%%%%%%%%%%%%%%%%%%%%%%%%%%%%%%%%%%%%%%%%%%%%%%%%%%%%