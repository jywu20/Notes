\begin{equation}
    \begin{gathered}
        \begin{tikzpicture}[x=0.75pt,y=0.75pt,yscale=-1,xscale=1]
            %uncomment if require: \path (0,300); %set diagram left start at 0, and has height of 300
            
            %Shape: Circle [id:dp7977208390713126] 
            \draw   (224,171.35) .. controls (224,156.25) and (236.25,144) .. (251.35,144) .. controls (266.46,144) and (278.71,156.25) .. (278.71,171.35) .. controls (278.71,186.46) and (266.46,198.71) .. (251.35,198.71) .. controls (236.25,198.71) and (224,186.46) .. (224,171.35) -- cycle ;
            %Straight Lines [id:da24233957833824893] 
            \draw    (169.29,171.35) -- (224,171.35) ;
            %Straight Lines [id:da32762511162194086] 
            \draw    (269.29,151.35) -- (300.71,107.56) ;
            %Straight Lines [id:da9552561758852673] 
            \draw    (267.29,194.56) -- (307.71,246.53) ;
            
            % Text Node
            \draw (167.29,171.35) node [anchor=east] [inner sep=0.75pt]    {$z$};
            % Text Node
            \draw (302.71,104.16) node [anchor=south west] [inner sep=0.75pt]    {$x$};
            % Text Node
            \draw (309.71,249.93) node [anchor=north west][inner sep=0.75pt]    {$y$};
            \end{tikzpicture}            
    \end{gathered} = 
    \begin{aligned}
        &\int \dd[4]{w_1} \int \dd[4]{w_2} \int \dd[4]{w_3} (\ii g)^3 D_F(x - w_1) D_F(y - w_2) D_F(z - w_3) \\
        &\quad \times D_F(w_1 - w_2) D_F(w_2 - w_3) D_F(w_3 - w_1).
    \end{aligned}
    \label{eq:phi-phiphi-position-feynman}
\end{equation}