\documentclass[hyperref, a4paper]{article}

\usepackage{geometry}
\usepackage{titling}
\usepackage{titlesec}
% No longer needed, since we will use enumitem package
% \usepackage{paralist}
\usepackage{enumitem}
\usepackage{footnote}
\usepackage{marginnote}
\usepackage{enumerate}
\usepackage{amsmath, amssymb, amsthm}
\usepackage{mathtools}
\usepackage{bbm}
\usepackage{cite}
\usepackage{graphicx}
\usepackage{subfigure}
\usepackage{physics}
\usepackage{tensor}
\usepackage{siunitx}
\usepackage[version=4]{mhchem}
\usepackage{tikz}
\usepackage{xcolor}
\usepackage{listings}
\usepackage{autobreak}
\usepackage[ruled, vlined, linesnumbered]{algorithm2e}
\usepackage{nameref,zref-xr}
\zxrsetup{toltxlabel}
\zexternaldocument*[optics-]{../optics/optics}[optics.pdf]
\zexternaldocument*[solid-]{../solid/solid}[solid.pdf]
%\zexternaldocument*[oldfluid-]{fluid}[fluid.pdf]
\usepackage[colorlinks,unicode]{hyperref} % , linkcolor=black, anchorcolor=black, citecolor=black, urlcolor=black, filecolor=black
\usepackage[most]{tcolorbox}
\usepackage{prettyref}

% Page style
\geometry{left=3.18cm,right=3.18cm,top=2.54cm,bottom=2.54cm}
\titlespacing{\paragraph}{0pt}{1pt}{10pt}[20pt]
\setlength{\droptitle}{-5em}
\preauthor{\vspace{-10pt}\begin{center}}
\postauthor{\par\end{center}}

% More compact lists 
%\setlist[itemize]{
%    itemindent=17pt, 
%    leftmargin=1pt,
%    listparindent=\parindent,
%    parsep=0pt,
%}

% Math operators
\DeclareMathOperator{\timeorder}{\mathcal{T}}
\DeclareMathOperator{\diag}{diag}
\DeclareMathOperator{\legpoly}{P}
\DeclareMathOperator{\primevalue}{P}
\DeclareMathOperator{\sgn}{sgn}
\newcommand*{\ii}{\mathrm{i}}
\newcommand*{\ee}{\mathrm{e}}
\newcommand*{\const}{\mathrm{const}}
\newcommand*{\suchthat}{\quad \text{s.t.} \quad}
\newcommand*{\argmin}{\arg\min}
\newcommand*{\argmax}{\arg\max}
\newcommand*{\normalorder}[1]{: #1 :}
\newcommand*{\pair}[1]{\langle #1 \rangle}
\newcommand*{\fd}[1]{\mathcal{D} #1}
\DeclareMathOperator{\bigO}{\mathcal{O}}

% TikZ setting
\usetikzlibrary{arrows,shapes,positioning}
\usetikzlibrary{arrows.meta}
\usetikzlibrary{decorations.markings}
\tikzstyle arrowstyle=[scale=1]
\tikzstyle directed=[postaction={decorate,decoration={markings,
    mark=at position .5 with {\arrow[arrowstyle]{stealth}}}}]
\tikzstyle ray=[directed, thick]
\tikzstyle dot=[anchor=base,fill,circle,inner sep=1pt]

% Algorithm setting
% Julia-style code
\SetKwIF{If}{ElseIf}{Else}{if}{}{elseif}{else}{end}
\SetKwFor{For}{for}{}{end}
\SetKwFor{While}{while}{}{end}
\SetKwProg{Function}{function}{}{end}
\SetArgSty{textnormal}

\newcommand*{\concept}[1]{{\textbf{#1}}}

% Support for tensor double arrows.
\renewcommand{\tensor}[1]{ \stackrel{\leftrightarrow}{\vb*{#1}}}

% Embedded codes
\lstset{basicstyle=\ttfamily,
  showstringspaces=false,
  commentstyle=\color{gray},
  keywordstyle=\color{blue}
}

% Reference formatting
\newrefformat{fig}{Figure~\ref{#1} on page~\pageref{#1}}

% Color boxes
\tcbuselibrary{skins, breakable, theorems}
\newtcbtheorem[number within=section]{warning}{Warning}%
  {colback=orange!5,colframe=orange!65,fonttitle=\bfseries, breakable}{warn}
\newtcbtheorem[number within=section]{note}{Note}%
  {colback=green!5,colframe=green!65,fonttitle=\bfseries, breakable}{note}

% Correctly displaying equation number in section titles
\pdfstringdefDisableCommands{\def\eqref#1{(\ref{#1})}}

%\newcommand{\ktnote}{\href{../topological-phases-reading-notes/kt.pdf}{this note}}

\title{General Relativity as an Effective Field Theory}
\author{Jinyuan Wu}

\begin{document}

\maketitle

This article is a reading note of the chapters in Schwartz about how general Relativity (henceforth GR) can
be thought as an effective field theory. The idea was first shown by \cite{1970,boulware1975193}. 

\section{Spin-2 fields}

TODO: representations

\section{The free spin-2 theory}

We first construct a spin-2 field theory with the approach of Schwartz Section 8.7. \marginnote{Sec.~8.7.2}
Consider a symmetric rank 2 tensor $h_{\mu \nu}$. We try to write down a free theory of the theory, which 
must be quadratic in $h_{\mu \nu}$, and either quadratic or zeroth order in $\partial_\mu$. From $h_{\mu \nu}$ 
we can construct a list of objects that are first order in $h_{\mu \nu}$ and do not contain $\bigO(\partial^3)$:
\[
    h_{\mu \nu},\  h_{\alpha \alpha}, \ \Box h_{\mu \nu}, \ \partial_\mu \partial_\nu h_{\mu \nu}, \ 
    \partial_\mu \partial_\nu h_{\mu \alpha}, \ \Box h_{\alpha \alpha}, 
\]
and we can construct terms that are possible to appear in the free Lagrangian:
\[
    h_{\mu \nu}^2, \ h_{\mu \nu} \Box h_{\mu \nu}, \ h_{\nu \alpha} \partial_\mu \partial_\nu h_{\mu \alpha}, 
    \ h_{\alpha \alpha}^2, \ h_{\alpha \alpha} \Box h_{\beta \beta}, \ h_{\alpha \alpha} \partial_\mu \partial_\nu h_{\mu \nu},
\]
and hence we get (8.126) \marginnote{(8.126)}
\begin{equation}
    \mathcal{L}=a h_{\mu \nu} \square h_{\mu \nu}+b h_{\mu \nu} \partial_{\mu} \partial_{\alpha} h_{\nu \alpha}+c h \square h + d h \partial_{\mu} \partial_{\nu} h_{\mu \nu}+m^{2}\left(x h_{\mu \nu}^{2}+y h^{2}\right),
    \label{eq:general-lag}
\end{equation}
where we define $h = h_{\alpha \alpha}$. 

Now we consider the ``inner structure'' of $h_{\mu \nu}$. We first do the decomposition \marginnote{(8.124) to (8.125)}
\begin{equation}
    h_{\mu \nu} = h_{\mu\nu}^\text{T} + \partial_\mu \pi_\nu + \partial_\nu \pi_\mu,
\end{equation}
where we require  
\begin{equation}
    h_{\mu \nu}^\text{T} = h_{\nu \mu}^\text{T}, \quad \partial^\mu h_{\mu \nu}^\text{T} = 0.
\end{equation}
Again we can decompose $\pi_\mu$ into 
\begin{equation}
    \pi_{\mu}=\pi_{\mu}^\text{T}+\partial_{\mu} \pi^\text{L}, \quad \partial^\mu \pi_\mu^\text{T} = 0.
\end{equation}

Now we find the decomposition actually imposes strong constraints on \eqref{eq:general-lag}. First, 
since 
\[
    x h_{\mu \nu}^2 + y h^2 = 2 x (\Box \pi^\text{L})^2 + 2 y (\partial_\mu \partial_\nu \pi^\text{L})^2 \simeq 
    - 2x \pi^\text{L} \Box^2 \pi^\text{L} - 2y \partial_\mu \partial_\nu \partial^\mu \partial^\nu \pi^\text{L} 
    = - 2 (x+y) \pi^\text{L} \Box^2 \pi^\text{L},
\]
and there should be no $\bigO(\partial^4)$ terms in \eqref{eq:general-lag}, we find $x + y = 0$. 
TODO: derive 
\begin{equation} \marginnote{(8.128)}
    \mathcal{L}=\frac{1}{2} h_{\mu \nu} \square h_{\mu \nu}-h_{\mu \nu} \partial_{\mu} \partial_{\alpha} h_{\nu \alpha}+h \partial_{\mu} \partial_{\nu} h_{\mu \nu}-\frac{1}{2} h \square h+\frac{1}{2} m^{2}\left(h_{\mu \nu}^{2}-h^{2}\right)
\end{equation}
The massless case is 
\begin{equation}
    \mathcal{L}=\frac{1}{2} h_{\mu \nu} \square h^{\mu \nu}-h_{\mu \nu} \partial^{\mu} \partial_{\alpha} h^{\nu \alpha}+h \partial_{\mu} \partial_{\nu} h^{\mu \nu}-\frac{1}{2} h \square h. 
    \label{eq:massless}
\end{equation}

From now on we will focus on the massless case, as gravitation seems to have some shared properties with 
electromagnetism. For an effective theory of gravitation, We require that the basic degrees of freedom is 
a symmetric spin-2 tensor, and the free part of our theory is \eqref{eq:massless}. We will find that 
the almost only possibility is general relativity.

\section{Coupling with another field}

Now we try to couple \eqref{eq:massless} to other fields. Note that the $\pi$ modes never appear 
in \eqref{eq:massless} TODO: show this 
and therefore they cannot be coupled to external degrees of freedom. This \marginnote{(8.115) to (8.116)}
gives us a hint that these $\pi$ modes might be \emph{gauge redundancies}. Therefore, we introduce the minimal 
coupling between $h_{\mu \nu}$ and another tensor $T_{\mu \nu}$ made up by external fields, and the 
interaction Lagrangian is 
\begin{equation}
    \mathcal{L}_\text{int} \propto h_{\mu \nu} T^{\mu \nu} .
    \label{eq:coupling}
\end{equation}
Since we have to ensure that 
\[
    h_{\mu \nu}^\text{T} T^{\mu \nu} \simeq h_{\mu \nu} T^{\mu \nu} = (h_{\mu \nu}^\text{T} + \partial_\mu \pi_\nu + \partial_\nu \pi_\mu) T^{\mu \nu},
\] 
we must take advantages of integration by parts and assume that 
\[
    (\partial_\mu + \partial_\nu) T^{\mu \nu} = 0.
\]
Since $h_{\mu \nu}$ is symmetric, without loss of generality, we assume that $T_{\mu \nu}$ is symmetric, and 
therefore the above condition is equivalent to $\partial_\mu T^{\mu \nu} = 0$. This in turn means that 
under a transformation 
\begin{equation}
    h_{\mu \nu} \longrightarrow h_{\mu \nu} + \partial_\mu \xi_\nu + \partial_\nu \xi_\mu,
    \label{eq:gauge-transform}
\end{equation}
the theory - both the free part \eqref{eq:massless} and the coupling part with external fields \eqref{eq:coupling},
are invariant, confirming the claim that the $\pi$ degrees of freedom are indeed gauge degrees of freedom, and 
\eqref{eq:gauge-transform} is the gauge transformation. Now we already see something familiar in GR here: 
\eqref{eq:gauge-transform} seems to be how the \emph{metric} transforms, and the gauge group seems to be 
a local TODO: what group is this? 

The fact that $T^{\mu \nu}$ is a conservation current in turn poses an additional constraint. The 
\concept{Coleman-Mandula theorem} tells us that the \emph{energy-momentum tensor} is the only rank 2 tensor 
conservation current, so $T^{\mu \nu}$ (up to a constant) must be the energy-momentum tnesor.

\section{Self interaction}

We will soon find, however, that \eqref{eq:massless} plus \eqref{eq:coupling} is still not a self-consistent 
theory. TODO: free graviton cannot be coupled to an external field

\section{GR comes out}

\section{The non-renormalizable nature of GR} 

It has been traditionally claimed that GR is incompatible with quantum mechanics, because GR is not renormalizable. \marginnote{Sec.~22.4}
Schwartz says this claim is wrong - non-renormalizable theories are as useful as renormalizable ones. They just 
tell us themselves when they start to break. 
The conclusion is that there is, actually, \emph{no} conflict between general relativity and quantum mechanics.
Here we 

What is really concerning is that GR, in the context of QFT, is not 

\bibliographystyle{plain}
\bibliography{gr-effective} 

\end{document}