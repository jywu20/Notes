\documentclass[hyperref, a4paper]{article}

\usepackage{geometry}
\usepackage{titling}
\usepackage{titlesec}
% No longer needed, since we will use enumitem package
% \usepackage{paralist}
\usepackage{enumitem}
\usepackage{footnote}
\usepackage{marginnote}
\usepackage{enumerate}
\usepackage{amsmath, amssymb, amsthm}
\usepackage{mathtools}
\usepackage{bbm}
\usepackage{cite}
\usepackage{graphicx}
\usepackage{subfigure}
\usepackage{physics}
\usepackage{tensor}
\usepackage{siunitx}
\usepackage[version=4]{mhchem}
\usepackage{tikz}
\usepackage{xcolor}
\usepackage{listings}
\usepackage{autobreak}
\usepackage[ruled, vlined, linesnumbered]{algorithm2e}
\usepackage{nameref,zref-xr}
\zxrsetup{toltxlabel}
\zexternaldocument*[optics-]{../optics/optics}[optics.pdf]
\zexternaldocument*[solid-]{../solid/solid}[solid.pdf]
%\zexternaldocument*[oldfluid-]{fluid}[fluid.pdf]
\usepackage[colorlinks,unicode]{hyperref} % , linkcolor=black, anchorcolor=black, citecolor=black, urlcolor=black, filecolor=black
\usepackage[most]{tcolorbox}
\usepackage{prettyref}

% Page style
\geometry{left=3.18cm,right=3.18cm,top=2.54cm,bottom=2.54cm}
\titlespacing{\paragraph}{0pt}{1pt}{10pt}[20pt]
\setlength{\droptitle}{-5em}
\preauthor{\vspace{-10pt}\begin{center}}
\postauthor{\par\end{center}}

% More compact lists 
%\setlist[itemize]{
%    itemindent=17pt, 
%    leftmargin=1pt,
%    listparindent=\parindent,
%    parsep=0pt,
%}

% Math operators
\DeclareMathOperator{\timeorder}{\mathcal{T}}
\DeclareMathOperator{\diag}{diag}
\DeclareMathOperator{\legpoly}{P}
\DeclareMathOperator{\primevalue}{P}
\DeclareMathOperator{\sgn}{sgn}
\newcommand*{\ii}{\mathrm{i}}
\newcommand*{\ee}{\mathrm{e}}
\newcommand*{\const}{\mathrm{const}}
\newcommand*{\suchthat}{\quad \text{s.t.} \quad}
\newcommand*{\argmin}{\arg\min}
\newcommand*{\argmax}{\arg\max}
\newcommand*{\normalorder}[1]{: #1 :}
\newcommand*{\pair}[1]{\langle #1 \rangle}
\newcommand*{\fd}[1]{\mathcal{D} #1}
\DeclareMathOperator{\bigO}{\mathcal{O}}

% TikZ setting
\usetikzlibrary{arrows,shapes,positioning}
\usetikzlibrary{arrows.meta}
\usetikzlibrary{decorations.markings}
\tikzstyle arrowstyle=[scale=1]
\tikzstyle directed=[postaction={decorate,decoration={markings,
    mark=at position .5 with {\arrow[arrowstyle]{stealth}}}}]
\tikzstyle ray=[directed, thick]
\tikzstyle dot=[anchor=base,fill,circle,inner sep=1pt]

% Algorithm setting
% Julia-style code
\SetKwIF{If}{ElseIf}{Else}{if}{}{elseif}{else}{end}
\SetKwFor{For}{for}{}{end}
\SetKwFor{While}{while}{}{end}
\SetKwProg{Function}{function}{}{end}
\SetArgSty{textnormal}

\newcommand*{\concept}[1]{{\textbf{#1}}}

% Support for tensor double arrows.
\renewcommand{\tensor}[1]{ \stackrel{\leftrightarrow}{\vb*{#1}}}

% Embedded codes
\lstset{basicstyle=\ttfamily,
  showstringspaces=false,
  commentstyle=\color{gray},
  keywordstyle=\color{blue}
}

% Reference formatting
\newrefformat{fig}{Figure~\ref{#1} on page~\pageref{#1}}

% Color boxes
\tcbuselibrary{skins, breakable, theorems}
\newtcbtheorem[number within=section]{warning}{Warning}%
  {colback=orange!5,colframe=orange!65,fonttitle=\bfseries, breakable}{warn}
\newtcbtheorem[number within=section]{note}{Note}%
  {colback=green!5,colframe=green!65,fonttitle=\bfseries, breakable}{note}

% Correctly displaying equation number in section titles
\pdfstringdefDisableCommands{\def\eqref#1{(\ref{#1})}}

%\newcommand{\ktnote}{\href{../topological-phases-reading-notes/kt.pdf}{this note}}

\title{General Relativity as an Effective Field Theory}
\author{Jinyuan Wu}

\begin{document}

\maketitle

This article is a reading note of the chapters in Schwartz about how general Relativity (henceforth GR) can
be thought as an effective field theory. 

\section{The free spin-2 theory}

We first construct a spin-2 field theory with the approach of Schwartz Section 8.7. \marginnote{Sec.~8.7}
Consider a symmetric rank 2 tensor $h_{\mu \nu}$. We try to write down a free theory of the theory, which 
must be quadratic in $h_{\mu \nu}$, and either quadratic or zeroth order in $\partial_\mu$. From $h_{\mu \nu}$ 
we can construct a list of objects that are first order in $h_{\mu \nu}$ and do not contain $\bigO(\partial^3)$:
\[
    h_{\mu \nu},\  h_{\alpha \alpha}, \ \Box h_{\mu \nu}, \ \partial_\mu \partial_\nu h_{\mu \nu}, \ 
    \partial_\mu \partial_\nu h_{\mu \alpha}, \ \Box h_{\alpha \alpha}, 
\]
and we can construct terms that are possible to appear in the free Lagrangian:
\[
    h_{\mu \nu}^2, \ h_{\mu \nu} \Box h_{\mu \nu}, \ h_{\nu \alpha} \partial_\mu \partial_\nu h_{\mu \alpha}, 
    \ h_{\alpha \alpha}^2, \ h_{\alpha \alpha} \Box h_{\beta \beta}, \ h_{\alpha \alpha} \partial_\mu \partial_\nu h_{\mu \nu},
\]
and hence we get (8.126) \marginnote{(8.126)}
\begin{equation}
    \mathcal{L}=a h_{\mu \nu} \square h_{\mu \nu}+b h_{\mu \nu} \partial_{\mu} \partial_{\alpha} h_{\nu \alpha}+c h \square h + d h \partial_{\mu} \partial_{\nu} h_{\mu \nu}+m^{2}\left(x h_{\mu \nu}^{2}+y h^{2}\right),
    \label{eq:general-lag}
\end{equation}
where we define $h = h_{\alpha \alpha}$. 

Now we consider the ``inner structure'' of $h_{\mu \nu}$. We first do the decomposition \marginnote{(8.124) to (8.125)}
\begin{equation}
    h_{\mu \nu} = h_{\mu\nu}^\text{T} + \partial_\mu \pi_\nu + \partial_\nu \pi_\mu,
\end{equation}
where we require  
\begin{equation}
    h_{\mu \nu}^\text{T} = h_{\nu \mu}^\text{T}, \quad \partial^\mu h_{\mu \nu}^\text{T} = 0.
\end{equation}
Again we can decompose $\pi_\mu$ into 
\begin{equation}
    \pi_{\mu}=\pi_{\mu}^\text{T}+\partial_{\mu} \pi^\text{L}, \quad \partial^\mu \pi_\mu^\text{T} = 0.
\end{equation}

Now we find the decomposition actually imposes strong constraints on \eqref{eq:general-lag}. First, 
since 
\[
    x h_{\mu \nu}^2 + y h^2 = 2 x (\Box \pi^\text{L})^2 + 2 y (\partial_\mu \partial_\nu \pi^\text{L})^2 \simeq 
    - 2x \pi^\text{L} \Box^2 \pi^\text{L} - 2y \partial_\mu \partial_\nu \partial^\mu \partial^\nu \pi^\text{L} 
    = - 2 (x+y) \pi^\text{L} \Box^2 \pi^\text{L},
\]
and there should be no $\bigO(\partial^4)$ terms in \eqref{eq:general-lag}, we find $x + y = 0$. 

\section{Coupling with another field and the self interaction}

The conclusion is that there is, actually, \emph{no} conflict between general relativity and quantum mechanics.
\marginnote{Sec.~22.4}
What is really concerning is that GR, in the context of QFT, is not 

\end{document}