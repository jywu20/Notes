\documentclass[hyperref, UTF8]{ctexart}

\usepackage{geometry}
\usepackage{titling}
\usepackage{titlesec}
\usepackage{paralist}
\usepackage{footnote}
\usepackage{marginnote}
\usepackage{enumerate}
\usepackage{autobreak}
\usepackage{amsmath, amssymb, amsthm}
\usepackage{mathtools}
\usepackage{bbm}
\usepackage{cite}
\usepackage{graphicx}
\usepackage{subfigure}
\usepackage{physics}
\usepackage{siunitx}
\usepackage{tikz}
\usepackage[compat=1.1.0]{tikz-feynhand}
\usepackage[ruled, vlined, linesnumbered, noend]{algorithm2e}
\usepackage[colorlinks]{hyperref} % linkcolor=black, anchorcolor=black, citecolor=black, filecolor=black
\usepackage[most]{tcolorbox}
\usepackage{caption}
\usepackage{prettyref}

\geometry{left=3.18cm,right=3.18cm,top=2.54cm,bottom=2.54cm}
\titlespacing{\paragraph}{0pt}{1pt}{10pt}[20pt]
\setlength{\droptitle}{-5em}

\DeclareMathOperator{\timeorder}{\mathcal{T}}
\DeclareMathOperator{\diag}{diag}
\DeclareMathOperator{\legpoly}{P}
\DeclareMathOperator{\primevalue}{P}
\DeclareMathOperator{\sgn}{sgn}
\newcommand*{\ii}{\mathrm{i}}
\newcommand*{\ee}{\mathrm{e}}
\newcommand*{\const}{\mathrm{const}}
\newcommand*{\suchthat}{\quad \text{s.t.} \quad}
\newcommand*{\argmin}{\arg\min}
\newcommand*{\argmax}{\arg\max}
\newcommand*{\normalorder}[1]{: #1 :}
\newcommand*{\pair}[1]{\langle #1 \rangle}
\newcommand*{\fd}[1]{\mathcal{D} #1}

\newrefformat{sec}{第\ref{#1}节}
\newrefformat{note}{注\ref{#1}}
\newrefformat{fig}{图\ref{#1}}
\newrefformat{alg}{算法\ref{#1}}
\newrefformat{back}{背景知识\ref{#1}}
\newrefformat{info}{资料框\ref{#1}}
\newrefformat{warn}{注意事项\ref{#1}}

\usetikzlibrary{arrows,shapes,positioning}
\usetikzlibrary{arrows.meta}
\usetikzlibrary{decorations.markings}
\tikzstyle arrowstyle=[scale=1]
\tikzstyle directed=[postaction={decorate,decoration={markings,
    mark=at position .5 with {\arrow[arrowstyle]{stealth}}}}]
\tikzstyle ray=[directed, thick]
\tikzstyle dot=[anchor=base,fill,circle,inner sep=1pt]

% Algorithm setting
\renewcommand{\algorithmcfname}{算法}
% Python-style code
\SetKwIF{If}{ElseIf}{Else}{if}{:}{elif:}{else:}{}
\SetKwFor{For}{for}{:}{}
\SetKwFor{While}{while}{:}{}
\SetKwInput{KwData}{输入}
\SetKwInput{KwResult}{输出}
\SetArgSty{textnormal}

\tcbuselibrary{skins, breakable, theorems}

\renewcommand{\emph}[1]{\textbf{#1}}
\newcommand*{\concept}[1]{\underline{\textbf{#1}}}


\newcommand{\Ztwo}{$\mathbb{Z}_2$}

% Cite: superscript, [1]
%\makeatletter
%\renewcommand\@citess[1]{\textsuperscript{[#1]}}
%\makeatother

\title{凝聚态物理中新奇物态的理论和实验}
\author{吴晋渊 18307110155}

\begin{document}

\maketitle

\section{铁磁性和低维材料}

铁磁性和反铁磁性可以粗略地分为巡游磁性和局域磁性,前者和材料中电子的费米面有关,后者涉及的电子是高度局域的,从而在能带结构中不在费米面附近。
两种机制都和电子相互作用有关\cite{annurev.ms.14.080184.000245,Cleveland_1976},直观地看,由于库仑相互作用电子气具有自旋翻转对称性,需要一定的相互作用破缺此对称性,形成铁磁或反铁磁磁性序。
此过程中允许出现的对称性破缺方式、可能出现的各种强关联效应均为值得研究的话题。

\subsection{一维铁磁性自旋链}

从一维Ising模型的初次提出和求解\cite{ising1925beitrag},到一维Heisenberg模型的提出和求解\cite{Bethe_1931},再到二维Ising模型\cite{onsager1944crystal},再到二维Heisenberg模型不存在铁磁相的证明\cite{PhysRevLett.17.1133},对局域磁性的理论研究经历了从经典到量子,从低维到高维,从简单的Ising模型到更加符合实际材料的Heisenberg模型的演进历程,并见证了Bethe拟设、统计场论、Mermin–Wagner定理等一系列适用范围远远超过磁性系统的理论工作的发展。

一般而言,低维体系更难发生对称性自发破缺,因为直观地看,低维下序参量的涨落的实空间关联函数$\sim \int \dd[d]{k} 1/k$具有更低的$k$幂次,从而更加容易出现红外发散,即原本可能出现的序会因为热涨落而比较容易被破坏掉。
Mermin-Wagner定理禁止在$d \leq 2$时出现任何连续对称性的破缺,因为连续对称性破缺必定出现Goldstone模,而Goldstone模的涨落在$d \leq 2$时足以破坏该连续对称性破缺导致的序——这也导致低维晶体的不稳定性,因为晶格位移的局部涨落会非常明显\cite{landau-symmetry}。%
\footnote{
    虽然$d$维量子系统等价于$d+1$维经典统计系统,但是在有限温的情况下,这个$d+1$维经典统计系统多出来的维度大小有限,从而在相变时,普适类和$d$维经典系统仍然是一样的。
    只有零温量子相变时,体系的量子性才非常重要。
}%
离散对称性不受到这个限制,二维Ising模型仍然有铁磁-顺磁相变,但一维Ising模型不存在有限温相变,在任何温度下均为顺磁性。
这个事实同样和一个更为一般的结论——van Hove定理——有关:相互作用足够局域的一维体系确实不能够发生对称性自发破缺\cite{Cuesta_2004}。
理论和数值上可以证明,自旋相互作用代数衰减的一维Ising型模型
\begin{equation}
    H = - J \sum_{\pair{i, j}} \sigma_i \sigma_j - \sum_{i, j} \frac{1}{\abs{r_i - r_j}^\alpha} \sigma_i \sigma_j
\end{equation}
由于长程相互作用,确实能够发生相变\cite{PhysRevB.54.R12661}。

因此,一个自然的问题是,一维自旋链中是否能够通过某种方式引入长程相互作用而实现铁磁序。这里的思路和前述理论工作正好相反:理论上,要严格求解一个低维模型要比严格求解一个高维模型容易,但低维系统反而在对称性自发破缺现象上有更加有趣的行为。
实际上,从提取有效模型的角度出发,三维的局域磁性主要来自交换相互作用,但低维材料中,导致局域磁性的相互作用通道还要包括偶极相互作用,并且由于衬底存在显著的各向异性,各向异性更加明显\cite{shen1997magnetism}。

文献\cite{PhysRevLett.73.898}报道说,Fe/W(110)一维链上存在铁磁序。这如果是正确的,显然是非同寻常的发现。
然而,至少两个干扰因素没有排除:实验时,在衬底上制备了多条Fe链,链和链之间可能存在偶极相互作用,而观察到的铁磁序有一定可能是非平衡现象:等待一段时间,磁性序参量会出现衰减。
\cite{shen1997magnetism}制备了一个链之间的相互作用基本可以忽略的Fe/Cu(111)链,发现其中的铁磁序确实存在时间相关的衰减,并且其变化规律可以使用以Glauber动力学更新的一维各向异性Ising模型很好地描述。
\cite{gambardella2002ferromagnetism}通过一个Co/Pt链进一步确认了这一点。因此,如果不考虑链之间的相互作用,我们获得的是所谓的“超顺磁性”,表面上的加入磁场后迅速出现磁矩的铁磁行为来自动力学因素而非平衡态因素。
实际上,\cite{PhysRevLett.73.898}报道的体系中涉及的机制比这更加有趣。对该体系进一步的研究\cite{PhysRevB.57.R677}显示,这个体系中不同的Fe链之间存在偶极相互作用,从而其低能有效模型是一个二维Ising模型,其中的每一个Ising自由度都对应单条链上由于动力学障碍形成的“短程磁畴”,从而导致一个“超铁磁序”。

总之,\cite{PhysRevLett.73.898}讨论的体系并非具有真正的铁磁序,但是它展示了远比强性加入长程相互作用更加有趣的物理机制。

\section{高温超导}

传统的BCS超导需要通过虚声子交换在电子之间形成等效吸引相互作用,来诱导电子配对,这样的等效吸引相互作用是比较弱的,从而库伯对非常容易受到热涨落破坏,超导临界温度较低。
铜氧超导能够获得高得多的超导临界温度,已经到达液氮温区,从而已有一些工程应用,如超导输电线,超导磁体(在超导线圈中产生巨大的电流,然后用它来做电磁铁)等。
然而,铜氧超导体的机制仍然是一个谜团。铜氧超导体的发现已经过去了30年,而至今理论家们还在为其机制争论不休。
除了铜基超导体以外,铁基超导体是另一类高温超导体,其行为和铜氧超导体类似但也有不同,同样没有明确的理论解释。
这些超导体的机制和之前发现的基于BCS理论的超导体显然非常不同,一方面是因为较高的临界温度,一方面是因为违反了传统超导体的Matthias规则\cite{Conder_2016}。

\section{表面物理的实验方法}

\bibliographystyle{plain}
\bibliography{low-dimension-magnetization,surface-experiment}

\end{document}