\documentclass[hyperref, a4paper]{article}

\usepackage{geometry}
\usepackage{float}
\usepackage{titling}
\usepackage{titlesec}
% No longer needed, since we will use enumitem package
% \usepackage{paralist}
\usepackage{enumitem}
\usepackage{footnote}
\usepackage{enumerate}
\usepackage{amsmath, amssymb, amsthm}
\usepackage{mathtools}
\usepackage{bbm}
\usepackage{cite}
\usepackage{graphicx}
\usepackage{subcaption}
\usepackage{physics}
\usepackage{tensor}
\usepackage{siunitx}
\usepackage{booktabs}
\usepackage[version=4]{mhchem}
\usepackage{tikz}
\usepackage{xcolor}
\usepackage{listings}
\usepackage{autobreak}
\usepackage[ruled, vlined, linesnumbered]{algorithm2e}
\usepackage{xr-hyper}
\usepackage[colorlinks,unicode]{hyperref} % , linkcolor=black, anchorcolor=black, citecolor=black, urlcolor=black, filecolor=black
\usepackage{prettyref}

% Page style
\geometry{left=3.18cm,right=3.18cm,top=2.54cm,bottom=2.54cm}
\titlespacing{\paragraph}{0pt}{1pt}{10pt}[20pt]
\setlength{\droptitle}{-5em}
\preauthor{\vspace{-10pt}\begin{center}}
\postauthor{\par\end{center}}

% More compact lists 
\setlist[itemize]{itemindent=17pt, leftmargin=1pt}

% Math operators
\DeclareMathOperator{\timeorder}{T}
\DeclareMathOperator{\diag}{diag}
\DeclareMathOperator{\legpoly}{P}
\DeclareMathOperator{\primevalue}{P}
\DeclareMathOperator{\sgn}{sgn}
\newcommand*{\ii}{\mathrm{i}}
\newcommand*{\ee}{\mathrm{e}}
\newcommand*{\const}{\mathrm{const}}
\newcommand*{\suchthat}{\quad \text{s.t.} \quad}
\newcommand*{\argmin}{\arg\min}
\newcommand*{\argmax}{\arg\max}
\newcommand*{\normalorder}[1]{: #1 :}
\newcommand*{\pair}[1]{\langle #1 \rangle}
\newcommand*{\fd}[1]{\mathcal{D} #1}
\DeclareMathOperator{\bigO}{\mathcal{O}}
\DeclareMathOperator{\object}{Ob}
\DeclareMathOperator{\morphism}{Hom}

% TikZ setting
\usetikzlibrary{arrows,shapes,positioning}
\usetikzlibrary{arrows.meta}
\usetikzlibrary{decorations.markings}
\tikzstyle arrowstyle=[scale=1]
\tikzstyle directed=[postaction={decorate,decoration={markings,
    mark=at position .5 with {\arrow[arrowstyle]{stealth}}}}]
\tikzstyle ray=[directed, thick]
\tikzstyle dot=[anchor=base,fill,circle,inner sep=1pt]

% Algorithm setting
% Julia-style code
\SetKwIF{If}{ElseIf}{Else}{if}{}{elseif}{else}{end}
\SetKwFor{For}{for}{}{end}
\SetKwFor{While}{while}{}{end}
\SetKwProg{Function}{function}{}{end}
\SetArgSty{textnormal}

\newcommand*{\concept}[1]{{\textbf{#1}}}

\newrefformat{fig}{Figure~\ref{#1}}

% Embedded codes
\lstset{basicstyle=\ttfamily,
  showstringspaces=false,
  commentstyle=\color{gray},
  keywordstyle=\color{blue}
}

% Disable unsupported commands in bookmark titles 
\pdfstringdefDisableCommands{%
  \def\\{}%
  \def\texttt#1{<#1>}%
  \def\mathbb#1{#1}%
}
\pdfstringdefDisableCommands{\def\eqref#1{(\ref{#1})}}

\makeatletter
\pdfstringdefDisableCommands{\let\HyPsd@CatcodeWarning\@gobble}
\makeatother

\title{Fibonacci Anyons and Topological Quantum Computation by Prof. Yidun Wan}
\author{Jinyuan Wu}

\begin{document}

\maketitle

\href{./2021-10-29}{This lecture} covered fusion and braiding in an intuitive and non-mathematical way.
Among topological orders, the \concept{Fibonacci anyons} are of specific significance because they can be 
used to implement universal quantum computation.
The fusion rules are 
\begin{equation}
    1 \times \tau = \tau, \quad \tau \times \tau = \mathbbm{1} + \tau,
\end{equation}
and the topological spin of $\tau$ is $\ee^{\ii 2 \pi \frac{2}{5}}$.
Like the case in particle physics or in spin systems, $\tau$ may be treated as a Hilbert subspace.

The dimension of the Hilbert space of a $n$-$\tau$ system can be visualize via the \concept{Bratelli diagram}.
By counting the number of fusion paths we find that the dimension of a $n$-$\tau$ system is the $n$th term of 
the Fibonacci sequence. 
Note that a fusion path labels a state, as is the case in spin systems: we may label the spin singlet state $\ket*{0, 0}$ as $\ket{\frac{1}{2} \otimes \frac{1}{2} \to 0}$.
The dynamics are not relevant here, provided that the Hamiltonian allows a stable state with multiple anyons.
The quantum dimension is defined as 
\begin{equation}
    d_\tau^{N-2} \simeq \dim \mathcal{H}_{N \tau} \text{\quad as $N \to \infty$}
\end{equation} 
and we have 
\begin{equation}
    d_\tau = \frac{1 + \sqrt{5}}{2},
\end{equation}
which is not an integer and implies that $\tau$ is a highly non-local object.
The fact can also be seen from that $\tau \times \tau$ is two-dimensional, where there is a hidden degree of freedom of the composite particle.

Solving the pentagon identity we have 
\begin{equation}
    F^1_{\tau\tau\tau} = 1, \quad F^\tau_{\tau\tau\tau} = \pmqty{\phi^{-1} & \sqrt{\phi^{-1}} \\ \sqrt{\phi^{-1}} & \phi^{-1}}.
\end{equation}

We define 
\begin{equation}
    \begin{gathered}
        \begin{tikzpicture}[x=0.75pt,y=0.75pt,yscale=-1,xscale=1]
            %uncomment if require: \path (0,300); %set diagram left start at 0, and has height of 300
            
            %Straight Lines [id:da5003387477590115] 
            \draw    (100,102) -- (127.03,129.03) ;
            %Straight Lines [id:da9032290840095405] 
            \draw    (154.03,102.03) -- (127.03,129.03) ;
            %Straight Lines [id:da7183431712337873] 
            \draw    (127.03,129.03) -- (154.06,156.06) ;
            %Straight Lines [id:da23250397191492755] 
            \draw    (208.09,102.03) -- (154.06,156.06) ;
            %Straight Lines [id:da5713501558527598] 
            \draw    (154.06,156.06) -- (154.06,202.03) ;
            
            % Text Node
            \draw (100,98.6) node [anchor=south] [inner sep=0.75pt]    {$\tau $};
            % Text Node
            \draw (154.03,98.63) node [anchor=south] [inner sep=0.75pt]    {$\tau $};
            % Text Node
            \draw (142.55,139.15) node [anchor=south west] [inner sep=0.75pt]    {$1$};
            % Text Node
            \draw (208.09,98.63) node [anchor=south] [inner sep=0.75pt]    {$\tau $};
            % Text Node
            \draw (154.06,205.43) node [anchor=north] [inner sep=0.75pt]    {$\tau $};
            \end{tikzpicture}            
    \end{gathered} \coloneqq \ket*{0},
\end{equation}
and 
\begin{equation}
    \begin{gathered}
        \begin{tikzpicture}[x=0.75pt,y=0.75pt,yscale=-1,xscale=1]
            %uncomment if require: \path (0,300); %set diagram left start at 0, and has height of 300
            
            %Straight Lines [id:da9576962371333875] 
            \draw    (120,122) -- (147.03,149.03) ;
            %Straight Lines [id:da4174020335474997] 
            \draw    (174.03,122.03) -- (147.03,149.03) ;
            %Straight Lines [id:da5762727318644585] 
            \draw    (147.03,149.03) -- (174.06,176.06) ;
            %Straight Lines [id:da2944296235282442] 
            \draw    (228.09,122.03) -- (174.06,176.06) ;
            %Straight Lines [id:da54062847176088] 
            \draw    (174.06,176.06) -- (174.06,222.03) ;
            
            % Text Node
            \draw (120,118.6) node [anchor=south] [inner sep=0.75pt]    {$\tau $};
            % Text Node
            \draw (174.03,118.63) node [anchor=south] [inner sep=0.75pt]    {$\tau $};
            % Text Node
            \draw (162.55,159.15) node [anchor=south west] [inner sep=0.75pt]    {$\tau $};
            % Text Node
            \draw (228.09,118.63) node [anchor=south] [inner sep=0.75pt]    {$\tau $};
            % Text Node
            \draw (174.06,225.43) node [anchor=north] [inner sep=0.75pt]    {$\tau $};
            \end{tikzpicture}
    \end{gathered} \coloneqq \ket*{1}.
\end{equation}
Other states are non-computational states.
The generators of braiding operations that may be applied to these states include $\sigma_1$ and $\sigma_2$.
The operations form the three-strand braiding group $B_3$.
It can be proved that a representation of $B_3$ is % TODO

We find that $\rho(\sigma_1)$ and $\rho(\sigma_2)$ are block-diagonal, where the non-computational 
states are not coupled with $\ket*{0}$ and $\ket*{1}$, while the blocks of $\rho(\sigma_1)$ and $\rho(\sigma_2)$
that apply on the subspace spanned by $\ket*{0}$ and $\ket*{1}$ cannot be diagonalized simultaneously.
Therefore we can throw away the non-computational states and focus on the subspace spanned by $\ket*{0}$ and $\ket*{1}$
and the representation of $B_3$ on this space has quantum fluctuation.
It can then be found that the representation of $B_3$ on this subspace $\mathcal{H}_{0, 1}$ is a dense subset of the representation 
of $SU(2)$ on $\mathbb{C}^{2 \times 2}$.
Therefore, each operation on a qubit is either in the representation of $B_3$ on $\mathcal{H}_{0, 1}$ or is a limit point of a sequence 
of operators in the representation of $B_3$ on $\mathcal{H}_{0, 1}$.
So approximately by braiding we can implement any $SU(2)$ operations - or in other words single-qubit quantum gates - 
on $\mathcal{H}_{0, 1}$.

We can extend the above procedure to the braiding of multiple qubits, so any multiple-qubit quantum gates can be implemented as 
accurate as we want using braiding. 

In conclusion, the Fibonacci anyon system is the simplest system that is universal for quantum computing, and it is truly 
\emph{topological}: both the quantum states and the quantum gates are robust in that local perturbation cannot cause any error.

It should be noted that when $\rho(\sigma_1)$ and $\rho(\sigma_2)$ are implemented, we are not sure whether we are performing the 
version with $\mathcal{H}_{0, 1}$ and the non-computational states decoupled.
If the non-computational states are involved, we say there is \concept{leakage}. Topological quantum computing on Fibonacci anyons
are not robust against leakage. Quantum error correcting is one way to eliminate leakage. Another approach is to use long braiding
sequences to lower the leakage, because we can make the leakage to the non-computational states bounded and after a long operation
sequence the effective leakage is small. Anyway, since we have never experimentally see a Fibonacci anyon system, discussing how to 
make the topological quantum computing on such systems may be too early.

\end{document}