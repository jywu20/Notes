\documentclass[hyperref, a4paper]{article}

\usepackage{geometry}
\usepackage{float}
\usepackage{titling}
\usepackage{titlesec}
% No longer needed, since we will use enumitem package
% \usepackage{paralist}
\usepackage{enumitem}
\usepackage{footnote}
\usepackage{enumerate}
\usepackage{amsmath, amssymb, amsthm}
\usepackage{mathtools}
\usepackage{bbm}
\usepackage{cite}
\usepackage{graphicx}
\usepackage{subcaption}
\usepackage{physics}
\usepackage{tensor}
\usepackage{siunitx}
\usepackage{booktabs}
\usepackage[version=4]{mhchem}
\usepackage{tikz}
\usepackage{xcolor}
\usepackage{listings}
\usepackage{autobreak}
\usepackage[ruled, vlined, linesnumbered]{algorithm2e}
\usepackage{xr-hyper}
\usepackage[colorlinks,unicode]{hyperref} % , linkcolor=black, anchorcolor=black, citecolor=black, urlcolor=black, filecolor=black
\usepackage{prettyref}

% Page style
\geometry{left=3.18cm,right=3.18cm,top=2.54cm,bottom=2.54cm}
\titlespacing{\paragraph}{0pt}{1pt}{10pt}[20pt]
\setlength{\droptitle}{-5em}
\preauthor{\vspace{-10pt}\begin{center}}
\postauthor{\par\end{center}}

% More compact lists 
\setlist[itemize]{itemindent=17pt, leftmargin=1pt}

% Math operators
\DeclareMathOperator{\timeorder}{T}
\DeclareMathOperator{\diag}{diag}
\DeclareMathOperator{\legpoly}{P}
\DeclareMathOperator{\primevalue}{P}
\DeclareMathOperator{\sgn}{sgn}
\newcommand*{\ii}{\mathrm{i}}
\newcommand*{\ee}{\mathrm{e}}
\newcommand*{\const}{\mathrm{const}}
\newcommand*{\suchthat}{\quad \text{s.t.} \quad}
\newcommand*{\argmin}{\arg\min}
\newcommand*{\argmax}{\arg\max}
\newcommand*{\normalorder}[1]{: #1 :}
\newcommand*{\pair}[1]{\langle #1 \rangle}
\newcommand*{\fd}[1]{\mathcal{D} #1}
\DeclareMathOperator{\bigO}{\mathcal{O}}
\DeclareMathOperator{\object}{Ob}
\DeclareMathOperator{\morphism}{Hom}

% TikZ setting
\usetikzlibrary{arrows,shapes,positioning}
\usetikzlibrary{arrows.meta}
\usetikzlibrary{decorations.markings}
\tikzstyle arrowstyle=[scale=1]
\tikzstyle directed=[postaction={decorate,decoration={markings,
    mark=at position .5 with {\arrow[arrowstyle]{stealth}}}}]
\tikzstyle ray=[directed, thick]
\tikzstyle dot=[anchor=base,fill,circle,inner sep=1pt]

% Algorithm setting
% Julia-style code
\SetKwIF{If}{ElseIf}{Else}{if}{}{elseif}{else}{end}
\SetKwFor{For}{for}{}{end}
\SetKwFor{While}{while}{}{end}
\SetKwProg{Function}{function}{}{end}
\SetArgSty{textnormal}

\newcommand*{\concept}[1]{{\textbf{#1}}}

\newrefformat{fig}{Figure~\ref{#1}}

% Embedded codes
\lstset{basicstyle=\ttfamily,
  showstringspaces=false,
  commentstyle=\color{gray},
  keywordstyle=\color{blue}
}

% Disable unsupported commands in bookmark titles 
\pdfstringdefDisableCommands{%
  \def\\{}%
  \def\texttt#1{<#1>}%
  \def\mathbb#1{#1}%
}
\pdfstringdefDisableCommands{\def\eqref#1{(\ref{#1})}}

\makeatletter
\pdfstringdefDisableCommands{\let\HyPsd@CatcodeWarning\@gobble}
\makeatother

\title{Invertible and Non-invertible Topological Orders by Prof. Yang Qi}
\author{Jinyuan Wu}

\begin{document}

\maketitle

Classification of topological phases in condensed matter physics:
\begin{itemize}
    \item Long-range entanglement: intrinsic topological order
    \item Short-range entanglement: 
    \begin{itemize}
        \item SPT
        \item Invertible topological phases
    \end{itemize}
\end{itemize}
The concept of \concept{invertible topological phases} needs some clarification. 
When Xiaogang Wen started to study topological phases, he called topological phases with ground state degeneracy and anyons \emph{intrinsic} topological phases, 
and in contrary were SPTs without ground state degeneracy and anyons and whose topological properties come from the quantum phase protected by certain symmetry.
Some topological phases also lack anyons and ground state degeneracy, but they are not protected by any symmetry, making them hard to classify.
Examples of such phases include the integer quantum Hall effect and one dimensional Kitaev chain.
Note that though Kitaev chains have particle-hole duality, it arises from technical reasons and does not protect the topological phase.
These phases are obviously closer to SPTs as they have no anyons and cannot be classified by tensor categories, but they are not SPTs.

It should be noted that fermion systems have a hidden symmetry - \concept{fermion-parity symmetry}, that if we add a minus symbol for each species of fermions, the Hamiltonian remains the same.
The symmetry group is 
\begin{equation}
    \mathbb{Z}_2^f = \{ 1, (-1)^{P_f} \}, \quad P_f = \pm 1.
\end{equation}
Therefore the interplay between two kinds of symmetric elements that occurs in the double group construction in the band theory of electrons happens again here: If the ``bosonic'' symmetry group of a system is $G_b$, there is no guarantee that the complete symmetry group $G_f$ is $G_b \times \mathbb{Z}_2^f$.
For example, if $G_b = \mathbb{Z}_2^T$, the compatibility conditions requires 
\begin{equation}
    T^2 = (-1)^{P_f},
\end{equation}
so $G_f$ is either isomorphism to $\mathbb{Z}_2 \times \mathbb{Z}_2$ or isomorphism to $\mathbb{Z}_4$.

\end{document}