\documentclass[hyperref, a4paper]{article}

\usepackage{geometry}
\usepackage{titling}
\usepackage{titlesec}
% No longer needed, since we will use enumitem package
% \usepackage{paralist}
\usepackage{enumitem}
\usepackage{footnote}
\usepackage{amsmath, amssymb, amsthm}
\usepackage{mathtools}
\usepackage{bbm}
\usepackage{cite}
\usepackage{graphicx}
\usepackage{subfigure}
\usepackage{physics}
\usepackage{tensor}
\usepackage{siunitx}
\usepackage[version=4]{mhchem}
\usepackage{tikz}
\usepackage{xcolor}
\usepackage{listings}
\usepackage{underscore}
\usepackage{autobreak}
\usepackage[ruled, vlined, linesnumbered]{algorithm2e}
\usepackage{nameref,zref-xr}
\zxrsetup{toltxlabel}
\usepackage[colorlinks,unicode]{hyperref} % , linkcolor=black, anchorcolor=black, citecolor=black, urlcolor=black, filecolor=black
\usepackage[most]{tcolorbox}
\usepackage{prettyref}

% Page style
\geometry{left=3.18cm,right=3.18cm,top=2.54cm,bottom=2.54cm}
\titlespacing{\paragraph}{0pt}{1pt}{10pt}[20pt]
\setlength{\droptitle}{-5em}

% More compact lists 
\setlist[itemize]{
    %itemindent=17pt, 
    %leftmargin=1pt,
    listparindent=\parindent,
    parsep=0pt,
}

\setlist[enumerate]{
    %itemindent=17pt, 
    %leftmargin=1pt,
    listparindent=\parindent,
    parsep=0pt,
}

% Math operators
\DeclareMathOperator{\timeorder}{\mathcal{T}}
\DeclareMathOperator{\diag}{diag}
\DeclareMathOperator{\legpoly}{P}
\DeclareMathOperator{\primevalue}{P}
\DeclareMathOperator{\sgn}{sgn}
\DeclareMathOperator{\res}{Res}
\newcommand*{\ii}{\mathrm{i}}
\newcommand*{\ee}{\mathrm{e}}
\newcommand*{\const}{\mathrm{const}}
\newcommand*{\suchthat}{\quad \text{s.t.} \quad}
\newcommand*{\argmin}{\arg\min}
\newcommand*{\argmax}{\arg\max}
\newcommand*{\normalorder}[1]{: #1 :}
\newcommand*{\pair}[1]{\langle #1 \rangle}
\newcommand*{\fd}[1]{\mathcal{D} #1}
\DeclareMathOperator{\bigO}{\mathcal{O}}

% TikZ setting
\usetikzlibrary{arrows,shapes,positioning}
\usetikzlibrary{arrows.meta}
\usetikzlibrary{decorations.markings}
\usetikzlibrary{calc}
\tikzstyle arrowstyle=[scale=1]
\tikzstyle directed=[postaction={decorate,decoration={markings,
    mark=at position .5 with {\arrow[arrowstyle]{stealth}}}}]
\tikzstyle ray=[directed, thick]
\tikzstyle dot=[anchor=base,fill,circle,inner sep=1pt]

% Algorithm setting
% Julia-style code
\SetKwIF{If}{ElseIf}{Else}{if}{}{elseif}{else}{end}
\SetKwFor{For}{for}{}{end}
\SetKwFor{While}{while}{}{end}
\SetKwProg{Function}{function}{}{end}
\SetArgSty{textnormal}

\newcommand*{\concept}[1]{{\textbf{#1}}}

% Embedded codes
\lstset{basicstyle=\ttfamily,
  showstringspaces=false,
  commentstyle=\color{gray},
  keywordstyle=\color{blue}
}

\lstdefinestyle{console}{
    basicstyle=\footnotesize\ttfamily,
    breaklines=true,
    postbreak=\mbox{\textcolor{red}{$\hookrightarrow$}\space}
}

% Reference formatting
\newrefformat{fig}{Figure~\ref{#1}}

% Color boxes
\tcbuselibrary{skins, breakable, theorems}
\newtcbtheorem[number within=section]{warning}{Warning}%
  {colback=orange!5,colframe=orange!65,fonttitle=\bfseries, breakable}{warn}
\newtcbtheorem[number within=section]{note}{Note}%
  {colback=green!5,colframe=green!65,fonttitle=\bfseries, breakable}{note}
\newtcbtheorem[number within=section]{info}{Info}%
  {colback=blue!5,colframe=blue!65,fonttitle=\bfseries, breakable}{info}

% Displaying texts in bookmarkers

\pdfstringdefDisableCommands{%
  \def\\{}%
  \def\ce#1{<#1>}%
}

\pdfstringdefDisableCommands{%
  \def\texttt#1{<#1>}%
  \def\mathbb#1{#1}%
}
\pdfstringdefDisableCommands{\def\eqref#1{(\ref{#1})}}

\makeatletter
\pdfstringdefDisableCommands{\let\HyPsd@CatcodeWarning\@gobble}
\makeatother

\newenvironment{shelldisplay}{\begin{lstlisting}}{\end{lstlisting}}

\newcommand{\shortcode}[1]{\texttt{#1}}

\lstset{style = console}

% Make subsubsection labeled
\setcounter{secnumdepth}{4}
\setcounter{tocdepth}{4}

\newcommand{\kB}{k_{\text{B}}}
\newcommand*{\muB}{\mu_{\text{B}}}
\newcommand*{\muN}{\mu_{\text{N}}}

\title{Homework 3}
\author{Jinyuan Wu}

\begin{document}

\maketitle

\section{}

\subsection{}

In order to keep the matrix elements invariant, 
in the Heisenberg picture we have 
\begin{equation}
    O_H = \ee^{\ii H t / \hbar} O_S \ee^{- \ii H t / \hbar}, 
\end{equation}
so that 
\begin{equation}
    {}_H \mel{\psi'}{O_H}{\psi}_H = {}_S \mel{\psi'}{O_S}{\psi}_S.
\end{equation}
The differential time evolution equation is
\begin{equation}
    \dv{O_H}{t} = \frac{\ii}{\hbar} H O_H - O_H \cdot \frac{\ii}{\hbar} H
    + \ee^{\ii H t /\hbar} \pdv{O_S}{t} \ee^{- \ii H t / \hbar},
\end{equation}
and therefore 
\begin{equation}
    \begin{aligned}
        \dv{\expval{O}}{t} &= \frac{1}{\ii \hbar} \expval{\comm*{O_H}{H}}{\psi_H}
        + \expval*{\ee^{\ii H t /\hbar} \pdv{O_S}{t} \ee^{- \ii H t / \hbar}}{\psi_H} \\
        &= \frac{1}{\ii \hbar} \expval{\comm*{O_H}{H}}{\psi_H}
        + \expval*{\pdv{O_S}{t}}{\psi_S} \\
        &= \frac{1}{\ii \hbar} \expval{\comm*{O}{H}} + \expval{\pdv{O}{t}}. 
    \end{aligned}
\end{equation}
Note: here we assume $H$ has no time dependence,
and thus the time evolution operator assumes the simple form $\ee^{- \ii H t / \hbar}$.
The condition that $H$ has no time dependence 
also means in the Heisenberg picture, $H_H(t) = H_H(0)$,
and on the other hand, $H_H(t)$ can be obtained from $H_H(0)$
by replacing the values of all operators at $t = 0$
to the corresponding values at $t$,
and therefore the commutation relation $\comm*{O_H(t)}{H}$
can be obtained by replacing the occurrences of all operators  
in $\comm*{O_H(t=0)}{H} = \comm*{O_S}{H}$
with their values at $t$.
That's why in the third line, 
we omit the Heisenberg/Schrodinger picture labels.

\subsection{}

Since the Hamiltonian 
\begin{equation}
    H = \frac{p^2}{2m} + V 
\end{equation}
is real (and not just Hermitian), 
we can always obtain real eigenfunctions.
Thus
\[
    \begin{aligned}
        \int \dd{x} \psi^* x (- \ii \hbar) \partial_x \psi 
        &= - \ii \hbar \eval{\psi^* x \psi}^{\infty}_{x = - \infty}
        + \ii \hbar \int \dd{x} \partial_x (\psi^* x) \psi \\
        &= \ii \hbar \int \dd{x} \partial_x(x \psi) \psi 
        = \ii \hbar \int \dd{x} \psi^2 + \ii \hbar \int \dd{x} x (\partial_x \psi) \psi \\
        &= \ii \hbar \int \dd{x} \abs{\psi}^2 
        - \int \dd{x} \psi^* x (- \ii \hbar \partial_x) \psi ,
    \end{aligned}
\]
and therefore 
\[
    2 \int \dd{x} \psi^* x p \psi = \ii \hbar ,
\]
\begin{equation}
    \expval{x p} =  \ii \hbar / 2.
\end{equation}
Therefore 
\[
    0 = \dv{t} \expval{xp}
    = \frac{1}{\ii \hbar} \expval{\comm*{xp}{H}}.
\]
The commutation relation can be evaluated as 
(note the correspondence between Poisson brackets 
and commutators)
\[
    \begin{aligned}
        \comm*{xp}{H} &= \frac{1}{2m} \comm*{xp}{p^2} + \comm*{xp}{V(x)} \\
        &= \frac{1}{2m} \comm*{x}{p^2} p + x \comm*{p}{V(x)} \\
        &= \frac{1}{2m} \cdot 2 \ii \hbar p \cdot p 
        - x \cdot \ii \hbar \pdv{V(x)}{x},
    \end{aligned}
\]
so we find 
\[
    0 = \ii \hbar \expval{\frac{p^2}{m}} - \ii \hbar \expval{x \pdv{V(x)}{x}},
\]
and therefore
\begin{equation}
    2 \expval{T} = \expval{x \pdv{V(x)}{x}}.
\end{equation}

\subsection{}

The proof can be repeated by replacing $x (- \ii \hbar) \partial_x$ 
by $\vb*{x} \cdot (- \ii \hbar) \grad$.
We then have 
\[
    \begin{aligned}
        \int \dd[3]{\vb*{x}} \psi^* \vb*{x} \cdot (- \ii \hbar) \grad \psi 
        &= - \ii \hbar \oint \dd{\vb*{S}} \cdot (\psi^* \vb*{x} \psi )
        + \ii \hbar \int \dd[3]{\vb*{x}} \div(\psi^* \vb*{x}) \psi \\
        &= \ii \hbar \int \dd[3]{\vb*{x}} \div(\psi \vb*{x}) \psi \\
        &= \ii \hbar \int \dd[3]{\vb*{x}} \psi^2 \underbrace{\div{\vb*{x}} }_{3}
        + \ii \hbar \int \dd[3]{\vb*{x}} \psi \vb*{x} \cdot \grad \psi ,
    \end{aligned}
\]
and we find 
\begin{equation}
    2 \expval{\vb*{x} \cdot \vb*{p}} = 3 \ii \hbar.
\end{equation}
Then repeating the derivation above, 
we get 
\begin{equation}
    2 \expval{T} = \expval{\vb*{x} \cdot \pdv{V(\vb*{x})}{\vb*{x}}}.
\end{equation}

\subsection{}

When $V = - 1 / r$, we have 
\begin{equation}
    E = \expval{V} + \frac{1}{2} \expval{r \pdv{V}{r}} 
    = - \expval{\frac{1}{r}} + \frac{1}{2} \expval{r \cdot \frac{1}{r^2}}
    = - \frac{1}{2} \expval{\frac{1}{r}} < 0,
\end{equation}
and therefore bound states are possible.
When $V = - 1 / r^2$, however, we have 
\begin{equation}
    E = \expval{V} + \frac{1}{2} \expval{r \pdv{V}{r}} 
    = - \expval{\frac{1}{r^2}} + \frac{1}{2} \expval{r \cdot \frac{2}{r^3}} = 0,
\end{equation} 
which causes a contradiction:
if a bound state exists, then its energy is not below $E = 0$.
Thus the potential $V(r) = - 1 / r^2$ doesn't assume bound states.

\section{}

We want to calculate 
\begin{equation}
    P_{\text{stay}} = \abs{\mel{\text{old ground}}{U}{\text{new ground}}}^2.
\end{equation}
If the Hamiltonian of a system undergoes a sudden change
at $t = 0$ and assumes now time evolution beside this point,
the time evolution operator $U(t, 0)$ is 
trivial both when $t > 0$ and $t < 0$,
and the probability is 
\begin{equation}
    P_{\text{say}} = \abs{\braket{\text{old ground}}{\text{new ground}}}^2.
\end{equation}

Now suppose $\psi_0$ is the 1s orbital of the electron in a hydrogen atom.
The hydrogen atom obtains a momentum $- k \vu*{z}$ 
after scattering with an $\alpha$ particle.
In the frame of reference attached to the hydrogen atom
after scattering, 
the old ground state is moving with momentum $k \vu*{z}$, 
and its wave function therefore should be 
\begin{equation}
    \psi_{k0}(\vb*{r}) = \ee^{\ii k z} \psi_0(\vb*{r}).
\end{equation}
The possibility for the electron to say at the ground state 
after the collision 
-- in other words, 
the possibility for the electron 
to go from the old ground state to the new ground state -- 
is therefore 
\begin{equation}
    P_{\text{say}} = \abs*{\braket{\psi_0}{\psi_{k0}}}^2 
    = \abs{\int \dd[3]{\vb*{r}} \ee^{\ii k z} \abs{\psi_0(\vb*{r})}^2 }^2.
\end{equation}

\section{}

The two angular momentum degrees of freedom 
are the electron spin $\vb*{S}$
and the nucleus spin $\vb*{I}$.
Since we are at $n = 0$,
the electron has no orbital angular momentum.
The magnetic coupling Hamiltonian is 
\begin{equation}
    H_{\text{magnetic coupling}} = - \vb*{\mu} \cdot \vb*{B}
    = \frac{1}{\hbar} (2 \muB \vb*{S} - g_p \muN \vb*{I}) \cdot \vb*{B}.
\end{equation}
Considering the component of $\vb*{\mu}$ along $\vb*{F} = \vb*{S} + \vb*{I}$ only, we have 
\begin{equation}
    \begin{aligned}
        H_{\text{magnetic coupling}} = 
        \frac{1}{\hbar} 
        \frac{
            (2 \muB \vb*{S} - g_p \muN \vb*{I}) \cdot (\vb*{S} + \vb*{I}) 
        }{\vb*{F}^2} \vb*{F} \cdot \vb*{B} .
    \end{aligned}
\end{equation}
Expanding $\vb*{S}^2, \vb*{I}^2$ and $\vb*{S} \cdot \vb*{I}$, we have
\begin{equation}
    \frac{E}{m_F B} = \frac{1}{\hbar} \frac{
        2 \muB S(S+1) 
        - g_p \muN I(I+1)
        + (2 \muB - g_p \muN) \cdot 
        \frac{1}{2} (F(F+1) - S(S+1) - I(I+1))
    }{
        F(F+1)
    }.
\end{equation}
Since we are working with hydrogen, $I = 1/2$, $S = 1/2$, 
and $F = 0, 1$, and therefore 
\begin{equation}
    E = \frac{1}{\hbar} \frac{2 \muB - g_p \muN}{2} m_F B.
\end{equation}
So we have 
\begin{equation}
    E_{F=1, m_F} = E_0 + E_g + \frac{2 \muB - g_p \muN}{2 \hbar} m_F B,
\end{equation}
and 
\begin{equation}
    E_{F=0, m_F = 0} = E_0,
\end{equation}
where $E_0 = \SI{13.6}{eV}$.

\section{}

\subsection{}

When the electric field $\vb*{E}$ is along the $z$ axis, 
the dipole interaction matrix is 
\begin{equation}
    H = - e r \mathcal{E} \cos \theta .
    \label{eq:dipole}
\end{equation}
When the initial state is 
\begin{equation}
    \psi_0(\vb*{r}) = \frac{1}{\sqrt{\pi}} \ee^{- r},
\end{equation}
the first-order energy correction is
\begin{equation}
    E_0^{(1)} \propto \int \dd{\Omega} r^2 \dd{r} \ee^{- 2 r} r 
    \propto \int_{0}^{\infty} r^3 \ee^{- 2 r} 
    = 0
\end{equation}
by integration by parts.

\subsection{}

We have (note that $\psi_0(\vb*{r})$ has no $\theta$ or $\varphi$ dependence)
\begin{equation}
    \begin{aligned}
        \mel{n l m}{e \vb*{r} \cdot \vb*{E}}{0} 
        &\propto \int \dd[3]{\vb*{r}} \psi_{n l m}^*(\vb*{r}) r \cos \theta \psi_0(\vb*{r}) \\
        &\propto \int_{0}^{\pi} \sin \theta \dd{\theta} \int_{0}^{2\pi} \dd{\varphi}
        \ee^{- \ii m \varphi} \legpoly_l^{\abs{m}}(\cos \theta) \cos \theta .
    \end{aligned}
\end{equation}
The radial part always has a non-zero contribution, 
since the $r$ factor is not in the $\{R_{nl}\}$ basis 
and therefore 
\[
    \int_{0}^{\infty} r^2 \dd{r} R_{00}(r)^* r R_{nl}(r)
\]
doesn't vanish because the orthogonal conditions 
are not available here.
So in order to know when the transition is not possible,
we only need to focus on the $\theta$ and $\varphi$ parts.
For the $\varphi$ part
\[
    \int_{0}^{2\pi} \ee^{- \ii m \varphi} \dd{\varphi}
\]
obviously vanishes except when $m = 0$.
When $m = 0$, the $\theta$ part is 
\[
    \int_{0}^{\pi} \sin \theta \dd{\theta} \legpoly_l^{\abs{m}}(\cos \theta) \cos \theta 
    = \int_{-1}^{1} x \legpoly_l(x) \dd{x} ,
\]
and since $\legpoly_1(x) = x$, 
due to orthogonal relations, only when $l = 1$, 
the above expression has a non-zero value.
Thus if a static electric field is added along the $z$ axis, 
the only possible final state
is $\psi_{n = 1, m = 0}$.

\subsection{}

We want to use the Dalgarno-Lewis method and solve 
\begin{equation}
    \left(H_0-E_n^{(0)}\right)\left|n^{(1)}\right\rangle=-\left(H^{\prime}-E_n^{(1)}\right)\left|n^{(0)}\right\rangle, \quad 
    E_n^{(2)}=\left\langle n^{(0)}\left|H^{\prime}\right| n^{(1)}\right\rangle-E_n^{(1)}\left\langle n^{(0)} \mid n^{(0)}\right\rangle
\end{equation}
for the hydrogen ground state,
where $H'$ is given by \eqref{eq:dipole}.
Since we work in the Hartree atomic units, 
we can set $e = - 1$,
and ground state energy is 
\begin{equation}
    E_0^{(0)} = - \frac{1}{2}.
\end{equation}
Therefore the PDE about $\psi_0^{(1)}$ is 
\begin{equation}
    \left(
        - \frac{\laplacian}{2} - \frac{1}{r} + \frac{1}{2}
    \right) \psi_0^{(1)} = 
    - r \mathcal{E} \cos \theta \psi^{(0)} (r)
    = - r \mathcal{E} \cos \theta \frac{1}{\sqrt{\pi}} \ee^{- r}.
\end{equation}
Taking the ansatz 
\begin{equation}
    \psi^{(1)}_0(\vb*{r}) = (A + B r + C r^2) \ee^{-r} \cos \theta,
\end{equation}
we have 
\[
    \text{LHS} = \ee^{-r} \cos \theta \frac{A + (B - 2C) r^2 + 2 C r^3}{r^2}.
\]
Comparing this with the RHS, we find 
\[
    A = 0, \quad B - 2 C = 0, \quad 2 C = - \mathcal{E} \frac{1}{\sqrt{\pi}},
\]
and therefore 
\begin{equation}
    \psi^{(1)}_0(\vb*{r}) = - \frac{\mathcal{E}}{2 \sqrt{\pi}} (r^2 + 2 r) \ee^{-r} \cos \theta.
\end{equation}
Since $E_n^{(1)}$ vanishes, 
the second-order perturbation of the energy is 
\begin{equation}
    \begin{aligned}
        E_0^{(2)} &= \int \dd[3]{\vb*{r}} \frac{1}{\sqrt{\pi}} \ee^{- r} 
        \cdot (r \mathcal{E} \cos \theta)
        \cdot \frac{- \mathcal{E}}{2 \sqrt{\pi}} (r^2 + 2 r) \ee^{-r} \cos \theta \\
        &= - \frac{\mathcal{E}^2}{2 \pi} \cdot 2\pi \cdot 
        \int_{0}^{\infty} r^2 \dd{r} (r^2 + 2 r) \ee^{-2r} \cdot 
        \int_{0}^{\pi} \sin \theta \dd{\theta} \cos^2 \theta \\
        &= - \frac{\mathcal{E}^2}{2 \pi} \cdot 2\pi \cdot \frac{3}{2} \cdot \frac{2}{3} 
        = - \mathcal{E}^2.
    \end{aligned}
\end{equation}
This expression is a specific case of 
\begin{equation}
    E^{(2)} = - \frac{1}{2} \alpha_{\text{p}} \mathcal{E}^2.
    \label{eq:energy-alpha}
\end{equation}
The expression can be shown by the following argument:
due to selection rules (generalization of last section)
for any atomic orbital, $E_n^{(1)} = 0$,
and therefore 
\begin{equation}
    E^{(2)} = \mel*{nlm^{(0)}}{ r \mathcal{E} \cos \theta}{nlm^{(1)}},
\end{equation}
while we have 
\begin{equation}
    d = \expval{- e r \cos \theta}{nlm}
    \approx (\bra*{nlm^{(0)}} + \bra*{nlm^{(1)}}) (- e r \cos \theta) (\ket*{nlm^{(0)}} + \ket*{nlm^{(1)}})
    \eqqcolon \alpha_{\text{p}} \mathcal{E}. 
\end{equation}
Ignoring the second order terms, 
and employing the condition that the eigenfunctions are all real,
we find 
\[
    \mel*{nlm^{(0)}}{ r\cos \theta}{nlm^{(1)}} = - \alpha_{\text{p}} \mathcal{E},
\]
and thus \eqref{eq:energy-alpha}.

\end{document}