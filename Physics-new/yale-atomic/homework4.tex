\documentclass[hyperref, a4paper]{article}

\usepackage{geometry}
\usepackage{titling}
\usepackage{titlesec}
% No longer needed, since we will use enumitem package
% \usepackage{paralist}
\usepackage{enumitem}
\usepackage{footnote}
\usepackage{amsmath, amssymb, amsthm}
\usepackage{mathtools}
\usepackage{bbm}
\usepackage{cite}
\usepackage{graphicx}
\usepackage{subfigure}
\usepackage{physics}
\usepackage{tensor}
\usepackage{siunitx}
\usepackage[version=4]{mhchem}
\usepackage{tikz}
\usepackage{xcolor}
\usepackage{listings}
\usepackage{underscore}
\usepackage{autobreak}
\usepackage[ruled, vlined, linesnumbered]{algorithm2e}
\usepackage{nameref,zref-xr}
\zxrsetup{toltxlabel}
\usepackage[colorlinks,unicode]{hyperref} % , linkcolor=black, anchorcolor=black, citecolor=black, urlcolor=black, filecolor=black
\usepackage[most]{tcolorbox}
\usepackage{prettyref}

% Page style
\geometry{left=3.18cm,right=3.18cm,top=2.54cm,bottom=2.54cm}
\titlespacing{\paragraph}{0pt}{1pt}{10pt}[20pt]
\setlength{\droptitle}{-5em}

% More compact lists 
\setlist[itemize]{
    %itemindent=17pt, 
    %leftmargin=1pt,
    listparindent=\parindent,
    parsep=0pt,
}

\setlist[enumerate]{
    %itemindent=17pt, 
    %leftmargin=1pt,
    listparindent=\parindent,
    parsep=0pt,
}

% Math operators
\DeclareMathOperator{\timeorder}{\mathcal{T}}
\DeclareMathOperator{\diag}{diag}
\DeclareMathOperator{\legpoly}{P}
\DeclareMathOperator{\primevalue}{P}
\DeclareMathOperator{\sgn}{sgn}
\DeclareMathOperator{\res}{Res}
\newcommand*{\ii}{\mathrm{i}}
\newcommand*{\ee}{\mathrm{e}}
\newcommand*{\const}{\mathrm{const}}
\newcommand*{\suchthat}{\quad \text{s.t.} \quad}
\newcommand*{\argmin}{\arg\min}
\newcommand*{\argmax}{\arg\max}
\newcommand*{\normalorder}[1]{: #1 :}
\newcommand*{\pair}[1]{\langle #1 \rangle}
\newcommand*{\fd}[1]{\mathcal{D} #1}
\DeclareMathOperator{\bigO}{\mathcal{O}}

% TikZ setting
\usetikzlibrary{arrows,shapes,positioning}
\usetikzlibrary{arrows.meta}
\usetikzlibrary{decorations.markings}
\usetikzlibrary{calc}
\tikzstyle arrowstyle=[scale=1]
\tikzstyle directed=[postaction={decorate,decoration={markings,
    mark=at position .5 with {\arrow[arrowstyle]{stealth}}}}]
\tikzstyle ray=[directed, thick]
\tikzstyle dot=[anchor=base,fill,circle,inner sep=1pt]

% Algorithm setting
% Julia-style code
\SetKwIF{If}{ElseIf}{Else}{if}{}{elseif}{else}{end}
\SetKwFor{For}{for}{}{end}
\SetKwFor{While}{while}{}{end}
\SetKwProg{Function}{function}{}{end}
\SetArgSty{textnormal}

\newcommand*{\concept}[1]{{\textbf{#1}}}

% Embedded codes
\lstset{basicstyle=\ttfamily,
  showstringspaces=false,
  commentstyle=\color{gray},
  keywordstyle=\color{blue}
}

\lstdefinestyle{console}{
    basicstyle=\footnotesize\ttfamily,
    breaklines=true,
    postbreak=\mbox{\textcolor{red}{$\hookrightarrow$}\space}
}

% Reference formatting
\newrefformat{fig}{Figure~\ref{#1}}

% Color boxes
\tcbuselibrary{skins, breakable, theorems}
\newtcbtheorem[number within=section]{warning}{Warning}%
  {colback=orange!5,colframe=orange!65,fonttitle=\bfseries, breakable}{warn}
\newtcbtheorem[number within=section]{note}{Note}%
  {colback=green!5,colframe=green!65,fonttitle=\bfseries, breakable}{note}
\newtcbtheorem[number within=section]{info}{Info}%
  {colback=blue!5,colframe=blue!65,fonttitle=\bfseries, breakable}{info}

% Displaying texts in bookmarkers

\pdfstringdefDisableCommands{%
  \def\\{}%
  \def\ce#1{<#1>}%
}

\pdfstringdefDisableCommands{%
  \def\texttt#1{<#1>}%
  \def\mathbb#1{#1}%
}
\pdfstringdefDisableCommands{\def\eqref#1{(\ref{#1})}}

\makeatletter
\pdfstringdefDisableCommands{\let\HyPsd@CatcodeWarning\@gobble}
\makeatother

\newenvironment{shelldisplay}{\begin{lstlisting}}{\end{lstlisting}}

\newcommand{\shortcode}[1]{\texttt{#1}}

\lstset{style = console}

% Make subsubsection labeled
\setcounter{secnumdepth}{4}
\setcounter{tocdepth}{4}

\newcommand{\kB}{k_{\text{B}}}
\newcommand*{\muB}{\mu_{\text{B}}}
\newcommand*{\muN}{\mu_{\text{N}}}

\title{Homework 4}
\author{Jinyuan Wu}

\begin{document}

\maketitle

\section{Casimir-Polder Force}

\subsection{Interaction potential between two harmonic oscillators}

The classical polarizability of a harmonic oscillator is 
\begin{equation}
    \alpha = \frac{ e^2}{\epsilon_0 k}.
\end{equation}
Since $\omega_0^2 = k / m$, the effective interaction potential 
\begin{equation}
    V(R) = - \frac{1}{8} \frac{\hbar}{m^2 \omega_0^3} \left(\frac{e^2}{2 \pi \epsilon_0}\right)^2 \frac{1}{R^6}
\end{equation}
can be rewritten into 
\begin{equation}
    V(R) = - \frac{1}{32 \pi^2} \hbar \omega_0 \frac{\alpha^2}{R^6}.
\end{equation}

\subsection{An oscillator and a conducting wall}

If we do angle average to the dipole interaction potential
\begin{equation}
    \begin{aligned}
        V(r) &= -\frac{d_1 d_2}{4 \pi \epsilon_0 r_{12}^3}\left(\cos \theta_{12}-3 \cos \theta_1 \cos \theta_2\right) \\
        &= -\frac{d^2}{4 \pi \epsilon_0 r_{12}^3} (\cos 2 \theta - 3 \cos^2 \theta) \\
        &= -\frac{d^2}{4 \pi \epsilon_0 r_{12}^3} \left(
            - \frac{3}{2} - \frac{1}{2} \cos 2 \theta
        \right),
    \end{aligned}
\end{equation}
since the $\cos 2 \theta$ term vanishes, we get 
\begin{equation}
    \expval{V(r)} = \frac{3}{2} \frac{\expval*{d^2}}{4 \pi \epsilon_0 r_{12}^3} ,
\end{equation}
where 
\begin{equation}
    d = e x 
\end{equation}
is the dipole for one oscillator.
This means in order to obtain the effective potential 
between the oscillator and the mirror, 
we just need to do the following substitution:
\begin{equation}
    d_1 d_2 \longrightarrow - \frac{3}{2} e^2 \expval*{x^2}.
\end{equation}
Ignoring the back action to the internal state of the oscillator,
$\expval{x^2}$ can be evaluated using the property of a free oscillator:
\begin{equation}
    \frac{1}{2} k \expval*{x^2} = \frac{1}{2} \hbar \omega_0
    \Rightarrow \expval*{d^2} = e^2 \frac{\hbar \omega_0}{k}.
\end{equation}

\subsection{Relativistic effect in field propagation}

From the boundary condition, 
in a static field we have 
(note that the electric field is doubled 
because of the contribution of the image charge)
\begin{equation}
    \frac{\sigma}{\epsilon_0} = \vb*{n} \cdot \vb*{E} \Rightarrow
    \frac{\sigma(r, \theta)}{\epsilon_0} = - 2 \cdot \frac{e}{4\pi \epsilon_0 (R^2 + r^2)} \frac{R}{\sqrt{R^2 + r^2}}, 
\end{equation}
\begin{equation}
    \sigma(r) = - \frac{e R}{2 \pi (R^2 + r^2)^{3/2}} .
\end{equation}
The total charge is 
\begin{equation}
    \int_{0}^{\infty} 2\pi r \dd{r} \sigma(r)
    = - e \int_{0}^{\infty} \frac{r}{R} \frac{\dd{r}}{R} \frac{R^3}{(R^2 + r^2)^{3/2}}
    = - e \int_{0}^{\infty} \frac{\dd{x^2}}{2} \frac{1}{(x^2 + 1)^{3/2}} \\
    = - e ,
\end{equation}
the expected result.

When $R$ is large enough, 
it takes time for the electric field to propagate from the oscillator to the mirror 
and then back to the oscillator.
The time cost is 
\begin{equation}
    T = \frac{2 \sqrt{R^2 + r^2}}{c},
\end{equation}
so at $t$, the oscillator sees the electric field 
it spread out at $t - T$;
note that at different $T$ values,  
the oscillator has different phases:
the phase factor of $\vb*{d}$ is $\ee^{- \ii \omega_0 (t - T)}$.
So finally the phase factor of the contribution of charges at $r$
to the electric field felt by the oscillator at $t$ 
is $\ee^{\ii 2 \omega_0 T}$.
We may then take the real part (since we are working with linear electrodynamics) and get 
\begin{equation}
    \begin{aligned}
        V_{\text{d}}(R)
        &= \int_{0}^{\infty} 2 \pi r \dd{r} \sigma(r)
        \cdot \cos (2 \omega_0 T) \cdot e \varphi(r \to R)  \\
        &= \int_{0}^{\infty} 2 \pi r \dd{r} \sigma(r)
        \cdot \cos(4 k_0 \sqrt{R^2 + r^2}) \frac{e}{4 \pi \epsilon_0 \sqrt{R^2 + r^2}}.
    \end{aligned}
    \label{eq:retarded-v}
\end{equation}

\subsection{Evaluation of the retarded interaction potential}

We evaluate \eqref{eq:retarded-v}. 


\section{$p$ to $x$ matrix element}

We know 
\begin{equation}
    \comm*{\vb*{x}}{H} = \ii \hbar \frac{\vb*{p}}{m},
\end{equation}
and therefore 
\[
    \frac{\ii \hbar}{m} \mel{i}{\vb*{p}}{j} = \mel{i}{\comm*{\vb*{x}}{H}}{j}
    = \mel{i}{\vb*{x} E_j - E_i \vb*{x}}{j}
    = (\hbar\omega_j - \hbar \omega_i) \mel{i}{\vb*{x}}{j},
\]
and thus 
\begin{equation}
    \mel{i}{\vb*{p}}{j} = \ii m \underbrace{(\omega_i - \omega_j)}_{\omega_{ij}} \mel{i}{\vb*{x}}{j}.
\end{equation}

\section{Sum rules}

\subsection{The $x$-closure sum rule}

We have
\begin{equation}
    \sum_{k} \abs*{\mel**{n}{x_i}{k}}^2 = \bra{n} x_i \sum_k \dyad{k} x_i \ket*{n} = \abs*{\expval*{x_i^2}{n}},
\end{equation}
and therefore 
\begin{equation}
    \begin{aligned}
        \sum_{k} \abs*{\mel**{n}{\vb*{x}}{k}}^2 &=
        \sum_k \sum_i  \abs*{\mel**{n}{x_i}{k}}^2 \\
        &= \sum_i \abs*{\expval*{x_i^2}{n}} = \abs*{\expval*{\vb*{x}^2}{n}}.
    \end{aligned}
\end{equation}

\subsection{The Thomas-Reiche-Huhn (TRK) sum rule}

From 
\begin{equation}
    \frac{\hbar}{\ii} = \expval*{px - xp}{n} 
    = \sum_k (
        \mel{n}{p}{k} \mel{k}{x}{n}
        - \mel{n}{x}{k} \mel{k}{p}{n}
    )
\end{equation}
and 
\begin{equation}
    \comm*{H}{x} = \frac{\hbar}{m \ii}p,
\end{equation}
we have 
\[
    \begin{aligned}
        \frac{\hbar}{\ii} &= \sum_k (
            \frac{m \ii}{\hbar} \mel{n}{\comm*{H}{x}}{k} \mel{k}{x}{n}
            - \mel{n}{x}{k} \cdot \frac{m \ii}{\hbar} \mel{k}{\comm*{H}{x}}{n}
        ) \\
        &= \frac{m \ii}{\hbar} \sum_k (
            \mel{n}{E_n x - x E_k}{k} \mel{k}{x}{n}
            - \mel{n}{x}{k} \mel{k}{E_k x - x E_n}{n}
        ) \\
        &= \frac{m \ii}{\hbar} \sum_k (
            2 (E_n - E_k)  \mel{n}{x}{k} \mel{k}{x}{n}
        ) ,
    \end{aligned}
\]
and therefore
\begin{equation}
    \sum_k (E_k - E_n) \abs*{\mel{n}{x}{k}}^2 = \frac{\hbar^2}{2m}.
\end{equation}

\section{Hydrogen maser}

The structure of the device is shown in Foot Fig. 6.4.

\subsection{The initial state}

There are four states: $F = 0, M_F = 0$, 
and $F = 1, M_F = 0, \pm 1$.
The initial state is 
\begin{equation}
    \rho_0 = \frac{1}{Z} (\dyad{0, 0} + \ee^{- \beta \Delta E} (\dyad{1, -1} + \dyad{1, 0} + \dyad{1, 1})),
\end{equation}
where 
\begin{equation}
    Z = 1 + 3 \ee^{- \beta \Delta E},
\end{equation}
\begin{equation}
    \Delta E = \frac{2 \muB - g_p \muN}{2 \hbar} B.
\end{equation}
When the external field is turned off we have 
\begin{equation}
    \rho_0 = \frac{1}{4} \diag(1, 1, 1, 1).
\end{equation}
The order of basis vectors is the order in the above discussion.

\subsection{State selector}

As is said in Fig. 6.4, 
all atoms in $F = 0, M_F = 0$ and $F = 1, M_F = 1$ states are thrown away.
Now we add a very strong magnetic field in the $x$ direction,
and the interaction Hamiltonian $- \vb*{\mu} \cdot \vb*{B}$
doesn't introduce any split because of $M_F$,
since the latter is along the $z$ direction.
Since we only have two states, 
the Hamiltonian has to take the form 
\begin{equation}
    H \propto - \sigma_x ,
\end{equation} 
and the low energy eigenstate is 
\begin{equation}
    \ket*{\text{ground}} = \frac{1}{\sqrt{2}} (\ket*{1, 0} + \ket*{1, 1}),
\end{equation}
and since the transverse field $B$ is very strong, 
almost all Hydrogen atoms fall to this ground state, 
and the density matrix therefore is 
\begin{equation}
    \rho_{\text{in selector}} = \dyad{\text{ground}} = \frac{1}{2} \pmqty{1 & 1 \\ 1 & 1},
\end{equation}
or in the full basis, 
\begin{equation}
    \rho_{\text{in selector}} = \frac{1}{2} \pmqty{
        0 & 0 & 0 & 0 \\
        0 & 0 & 0 & 0 \\
        0 & 0 & 1 & 1 \\
        0 & 0 & 1 & 1
    }.
\end{equation}

\subsection{Transition between $F$ states}



\subsection{The state selector}

The state selector 

\section{Diffracted limited beam}

We are dealing with linear optics 
so the power of the laser beam is irrelevant.
The beam radius of a Gaussian beam is 
\begin{equation}
    w(z) = w_0 \sqrt{1 + \left(\frac{z}{z_{\text{R}}}\right)^2}, \quad 
    z_{\text{R}} = \frac{\pi w_0^2 n}{\lambda}.
\end{equation}
When $z$ is very large, we have 
\begin{equation}
    w(z) = w_0 \frac{z}{z_{\text{R}}} = \frac{\lambda}{\pi w_0 n} \cdot z.
\end{equation}
The wave length of green laser is \SI{532}{nm}.
The distance between earth and moon is \SI{382500}{km},
and we can take $w_0$ to be half of the \SI{1}{mm} diameter of the laser beam 
when it leaves the laser,
and thus on the moon the radius of the beam is 
\[
    w_{\text{moon}} = \frac{\lambda}{\pi w_0} \cdot R_{\text{earth-moon}} = 
    \SI{129}{km}.
\]
This can also be seen as an instance of uncertainty principle: 
the above equation is equivalent to 
\begin{equation}
    \underbrace{w_0}_{\Delta x} \cdot 
    \underbrace{\frac{\pi}{\lambda} \cdot \underbrace{\frac{w(z)}{z}}_{\tan \theta}}_{\Delta k} = 1.
\end{equation}

\end{document}