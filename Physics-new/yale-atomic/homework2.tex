\documentclass[hyperref, a4paper]{article}

\usepackage{geometry}
\usepackage{titling}
\usepackage{titlesec}
% No longer needed, since we will use enumitem package
% \usepackage{paralist}
\usepackage{enumitem}
\usepackage{footnote}
\usepackage{amsmath, amssymb, amsthm}
\usepackage{mathtools}
\usepackage{bbm}
\usepackage{cite}
\usepackage{graphicx}
\usepackage{subfigure}
\usepackage{physics}
\usepackage{tensor}
\usepackage{siunitx}
\usepackage[version=4]{mhchem}
\usepackage{tikz}
\usepackage{xcolor}
\usepackage{listings}
\usepackage{underscore}
\usepackage{autobreak}
\usepackage[ruled, vlined, linesnumbered]{algorithm2e}
\usepackage{nameref,zref-xr}
\zxrsetup{toltxlabel}
\usepackage[colorlinks,unicode]{hyperref} % , linkcolor=black, anchorcolor=black, citecolor=black, urlcolor=black, filecolor=black
\usepackage[most]{tcolorbox}
\usepackage{prettyref}

% Page style
\geometry{left=3.18cm,right=3.18cm,top=2.54cm,bottom=2.54cm}
\titlespacing{\paragraph}{0pt}{1pt}{10pt}[20pt]
\setlength{\droptitle}{-5em}

% More compact lists 
\setlist[itemize]{
    %itemindent=17pt, 
    %leftmargin=1pt,
    listparindent=\parindent,
    parsep=0pt,
}

\setlist[enumerate]{
    %itemindent=17pt, 
    %leftmargin=1pt,
    listparindent=\parindent,
    parsep=0pt,
}

% Math operators
\DeclareMathOperator{\timeorder}{\mathcal{T}}
\DeclareMathOperator{\diag}{diag}
\DeclareMathOperator{\legpoly}{P}
\DeclareMathOperator{\primevalue}{P}
\DeclareMathOperator{\sgn}{sgn}
\DeclareMathOperator{\res}{Res}
\newcommand*{\ii}{\mathrm{i}}
\newcommand*{\ee}{\mathrm{e}}
\newcommand*{\const}{\mathrm{const}}
\newcommand*{\suchthat}{\quad \text{s.t.} \quad}
\newcommand*{\argmin}{\arg\min}
\newcommand*{\argmax}{\arg\max}
\newcommand*{\normalorder}[1]{: #1 :}
\newcommand*{\pair}[1]{\langle #1 \rangle}
\newcommand*{\fd}[1]{\mathcal{D} #1}
\DeclareMathOperator{\bigO}{\mathcal{O}}

% TikZ setting
\usetikzlibrary{arrows,shapes,positioning}
\usetikzlibrary{arrows.meta}
\usetikzlibrary{decorations.markings}
\usetikzlibrary{calc}
\tikzstyle arrowstyle=[scale=1]
\tikzstyle directed=[postaction={decorate,decoration={markings,
    mark=at position .5 with {\arrow[arrowstyle]{stealth}}}}]
\tikzstyle ray=[directed, thick]
\tikzstyle dot=[anchor=base,fill,circle,inner sep=1pt]

% Algorithm setting
% Julia-style code
\SetKwIF{If}{ElseIf}{Else}{if}{}{elseif}{else}{end}
\SetKwFor{For}{for}{}{end}
\SetKwFor{While}{while}{}{end}
\SetKwProg{Function}{function}{}{end}
\SetArgSty{textnormal}

\newcommand*{\concept}[1]{{\textbf{#1}}}

% Embedded codes
\lstset{basicstyle=\ttfamily,
  showstringspaces=false,
  commentstyle=\color{gray},
  keywordstyle=\color{blue}
}

\lstdefinestyle{console}{
    basicstyle=\footnotesize\ttfamily,
    breaklines=true,
    postbreak=\mbox{\textcolor{red}{$\hookrightarrow$}\space}
}

% Reference formatting
\newrefformat{fig}{Figure~\ref{#1}}

% Color boxes
\tcbuselibrary{skins, breakable, theorems}
\newtcbtheorem[number within=section]{warning}{Warning}%
  {colback=orange!5,colframe=orange!65,fonttitle=\bfseries, breakable}{warn}
\newtcbtheorem[number within=section]{note}{Note}%
  {colback=green!5,colframe=green!65,fonttitle=\bfseries, breakable}{note}
\newtcbtheorem[number within=section]{info}{Info}%
  {colback=blue!5,colframe=blue!65,fonttitle=\bfseries, breakable}{info}

% Displaying texts in bookmarkers

\pdfstringdefDisableCommands{%
  \def\\{}%
  \def\ce#1{<#1>}%
}

\pdfstringdefDisableCommands{%
  \def\texttt#1{<#1>}%
  \def\mathbb#1{#1}%
}
\pdfstringdefDisableCommands{\def\eqref#1{(\ref{#1})}}

\makeatletter
\pdfstringdefDisableCommands{\let\HyPsd@CatcodeWarning\@gobble}
\makeatother

\newenvironment{shelldisplay}{\begin{lstlisting}}{\end{lstlisting}}

\newcommand{\shortcode}[1]{\texttt{#1}}

\lstset{style = console}

% Make subsubsection labeled
\setcounter{secnumdepth}{4}
\setcounter{tocdepth}{4}

\newcommand{\kB}{k_{\text{B}}}
\newcommand*{\muB}{\mu_{\text{B}}}

\title{Homework 2}
\author{Jinyuan Wu}

\begin{document}

\maketitle

\section{Polarization of electromagnetic field}

\subsection{The general form of a pure state}

We have  (assuming $\vu*{k} = \vu*{z}$)
\[
    \vb*{E} = E_x \pmqty{1 \\ 0} + E_y \pmqty{0 \\ 1}
    = \sqrt{\abs*{E_x}^2 + \abs*{E_y}^2} \ee^{\ii \varphi_x}
    \pmqty{
        \frac{\abs*{E_x}}{\sqrt{\abs*{E_x}^2 + \abs*{E_y}^2}} \\
        \ee^{\ii (\varphi_y - \varphi_x)} \frac{\abs*{E_y}}{\sqrt{\abs*{E_x}^2 + \abs*{E_y}^2}}
    }, 
\]
and by defining 
\begin{equation}
    E_0 = \sqrt{\abs*{E_x}^2 + \abs*{E_y}^2} \ee^{\ii \varphi_x}, 
\end{equation}
\begin{equation}
    \cos \theta = \frac{\abs*{E_x}}{\sqrt{\abs*{E_x}^2 + \abs*{E_y}^2}} ,
\end{equation}
and 
\begin{equation}
    \phi = \varphi_ y - \varphi_x, 
\end{equation}
we find 
\begin{equation}
    \vb*{E} = E_0 \pmqty{\cos \theta \\ \ee^{\ii \phi} \sin \theta}.
\end{equation}
We have 
\begin{equation}
    \ket*{H} = \pmqty{1 \\ 0}, \quad 
    \ket*{V} = \pmqty{0 \\ 1},
\end{equation}
and therefore after normalization, we have 
\begin{equation}
    \rho = (\cos \theta \ket*{H} + \ee^{\ii \phi} \sin \theta \ket*{V})
    (\cos \theta \bra*{H} + \ee^{- \ii \phi} \sin \theta \bra*{V})
    = \pmqty{
        \cos^2 \theta & \ee^{- \ii \phi} \sin \theta \cos \theta \\
        \ee^{\ii \phi} \sin \theta \cos \theta & \sin^2 \theta
    }.
\end{equation}

\subsection{The pure state $\rho^2 = \rho$ condition}

We can prove the pure state condition $\rho^2 = \rho$ explicitly:
\begin{equation}
    \begin{aligned}
        \rho^2 &= \pmqty{
            \cos^4 \theta + \sin^2 \theta \cos^2 \theta &
            \ee^{- \ii \phi} \sin \theta \cos^3 \theta + 
            \ee^{- \ii \phi} \sin^3 \theta \cos \theta \\
            \ee^{\ii \phi} \sin \theta \cos^3 \theta + 
            \ee^{\ii \phi} \sin^3 \theta \cos \theta &
            \sin^2 \theta \cos^2 \theta + \sin^4 \theta 
        }  \\
        &= \pmqty{
            \cos^2 \theta & \ee^{- \ii \phi} \sin \theta \cos \theta \\
            \ee^{\ii \phi} \sin \theta \cos \theta & \sin^2 \theta
        } = \rho.
    \end{aligned}
\end{equation}

\subsection{Mixed state}

The condition that $\rho$ is Hermite means it can be written as 
\[
    \rho = R (\sigma^0 + x \sigma^x + y \sigma^y + z \sigma^z), 
\]
where $R, x, y, z \in \mathbb{R}$, 
because the $\sigma$ matrices constitute a basis 
for all Hermite matrices in $\mathbb{C}^{2 \times 2}$.
Since $\sigma^{x, y, z}$ are traceless, 
from the condition $\trace \rho = 1$, we have 
\[
    1 = \trace \rho = R \trace \sigma^0 = 2 R \Rightarrow R = \frac{1}{2},
\]
so 
\begin{equation}
    \rho = \frac{1}{2} (\sigma^0 + x \sigma^x + y \sigma^y + z \sigma^z).
    \label{eq:rho-decomposition}
\end{equation}
In the matrix form, we have 
\[
    \rho = \pmqty{
        \frac{1 + z}{2} & \frac{x - \ii y}{2} \\
        \frac{x + \ii y}{2} & \frac{1 - z}{2}
    },
\]
and by substitution of variables 
(this is a three variables to three variables mapping, 
and therefore is valid)
\[
    \frac{1}{2} (1 - p) + p \cos^2 \theta = \frac{1 + z}{2}, 
    \quad x = p \cos \phi \sin 2 \theta, \quad y = p \sin \phi \sin 2 \theta,
\]
we get 
\[
    \frac{1 - z}{2} = \frac{1}{2} (1 - p) + p \sin^2 \theta, 
\]
and therefore
\begin{equation}
    \rho = (1 - p) \pmqty{\dmat{\frac{1}{2}, \frac{1}{2}}} + p \pmqty{
        \cos^2 \theta & \ee^{- \ii \phi} \sin \theta \cos \theta \\
        \ee^{\ii \phi} \sin \theta \cos \theta & \sin^2 \theta
    }.
    \label{eq:general-light-polarization}
\end{equation}

\subsection{Jones parameters and Stokes formalism}

The definition of Stokes parameters are 
\begin{equation}
    \begin{aligned}
        I &= \expval*{E_x^2} + \expval*{E_y^2}, \\
        Q &= \expval*{E_x^2} - \expval*{E_y^2}, \\
        U &= \expval*{E_a^2} - \expval*{E_b^2}, \\
        V &= \expval*{E_l^2} - \expval*{E_r^2},
    \end{aligned}
\end{equation}
where 
\begin{equation}
    \vu*{a} = \frac{1}{\sqrt{2}} (\vu*{x} + \vu*{y}), \quad 
    \vu*{b} = \frac{1}{\sqrt{2}} (\vu*{x} - \vu*{y}),
\end{equation}
and 
\begin{equation}
    \vu*{l} = \frac{1}{\sqrt{2}} (\vu*{x} + \ii \vu*{y}), \quad 
    \vu*{r} = \frac{1}{\sqrt{2}} (\vu*{x} - \ii \vu*{y}).
\end{equation}
Note that $E_{x, y}$ etc. above may be regarded as operators, 
and we have 
\begin{equation}
    \begin{aligned}
        E_x^2 + E_y^2 &= \sigma^0, \\
        E_x^2 - E_y^2 &= \sigma^z, \\
        E_a^2 - E_b^2 &= \sigma^x, \\
        E_l^2 - E_r^2 &= \sigma^y.
    \end{aligned}
\end{equation}
From these definitions
and the fact that $(\sigma^i)^2 = \sigma^0$ and 
all other products of $\sigma$ matrices are traceless, 
we find 
\begin{equation}
    I = \expval*{\sigma^0} = 1, 
\end{equation}
\begin{equation}
    Q = \expval*{\sigma^z} = \frac{1}{2} z \cdot 2 = 2 p \cos^2 \theta - p = p \cos 2 \theta,
\end{equation}
\begin{equation}
    U = \frac{1}{2} x \cdot 2 = p \cos \phi \sin 2 \theta,
\end{equation}
and 
\begin{equation}
    V = \frac{1}{2} y \cdot 2 = p \sin \phi \sin 2 \theta.
\end{equation}
Here $I$ is constantly 1, 
because we are working with the single-photon density matrix.

\subsection{Mueller calculus}

As is shown above, 
Mueller calculus actually works on the coefficients in \eqref{eq:rho-decomposition},
and therefore a Mueller matrix 
essentially gives the coefficients of $\rho \to U \rho U^\dagger$.
It makes sense as long as after its application, 
the $\sigma^0$ component in $\rho$ is still $1/2$.

\subsection{Transformation and measurement}

We have 
\begin{equation}
    \begin{aligned}
        |45\rangle & =\frac{1}{\sqrt{2}}\left(\begin{array}{l}
        1 \\
        1
        \end{array}\right) \\
        |\text{rcp} \rangle & =\frac{1}{\sqrt{2}}\left(\begin{array}{c}
        1 \\
        -\ii
        \end{array}\right)
        \end{aligned}.
\end{equation}
The correspond density matrices are 
\begin{equation}
    \rho_{45} = \frac{1}{2} \pmqty{1 & 1 \\ 1 & 1},
\end{equation}
and 
\begin{equation}
    \rho_{\text{rcp}} = \frac{1}{2} \pmqty{ 1 & \ii \\ - \ii & 1 }.
\end{equation}

The operator 
\begin{equation}
    U_{\text{rcp}} = \pmqty{
        \dmat{1 , - \ii}
    }
\end{equation}
then turns $\rho_{45}$ to $\rho_{\text{rcp}}$: 
\begin{equation}
    U_{\text{rcp}} \rho_{45} U_{\text{rcp}}^\dagger 
    = \frac{1}{2} \pmqty{\dmat{1, - \ii}}
    \pmqty{1 & 1 \\ 1 & 1}
    \pmqty{\dmat{1, \ii}}
    = \frac{1}{2} \pmqty{
        1 & \ii \\ - \ii & 1
    } = \rho_{\text{rcp}}.
\end{equation}

The horizontal polarizer operator 
\begin{equation}
    \mathcal{O} = \pmqty{
        1 & 0 \\ 0 & 0
    }
\end{equation}
is not unitary, because it has non-unitary eigenvalue $0$.
It is a projection operator: 
it takes in a beam polarized light 
and returns its $x$ component. 
It also represents a measurement:
we can use it in a projective measurement setting.
In a projective measurement with operator $\mathcal{O}$, 
$\trace (\rho \mathcal{O})$
is the probability that after measurement, 
the final state of the system falls into the subspace determined by $\mathcal{O}$.
In our case, the subspace determined by $\mathcal{O}$ 
is the subspace of horizontal polarization, 
so $\trace (\rho \mathcal{O})$ is the probability 
that after measurement, we find $\rho$ to be a horizontally polarized state.

Application of $\mathcal{O}$ on \eqref{eq:general-light-polarization} is 
\begin{equation}
    \mathcal{O} \rho \mathcal{O}^\dagger 
    = (1 - p) \pmqty{\frac{1}{2} & 0 \\ 0 & 0}
    + p \pmqty{
        \cos^2 \theta & 0 \\ 0 & 0
    }
    = \left( \frac{1}{2} + \frac{p}{2} \cos 2 \theta \right) \mathcal{O}.
\end{equation}
So after the application of $\mathcal{O}$, 
we get a horizontally polarized state, as is expected. 
The result is not normalized; 
the factor before $\mathcal{O}$ is just $\trace(\rho \mathcal{O})$,
which is the probability that 
after measurement, we find $\rho$ to be horizontally polarized.
When $p = 0$, it's $1/2$, 
which is expected for the unpolarized state; 
when $p = 1$, 
it's $\cos^2 \theta$, again the correct answer. 

\section{The $\rho^2 = \rho$ condition for pure states}

Suppose 
\begin{equation}
    \rho = \dyad{\psi}, \quad \ket*{\psi} = \sum_m a_m \ket*{m}.
\end{equation}
We have 
\begin{equation}
    \begin{aligned}
        \rho^2 & =\sum_{m, n} a_m^* a_n \ket*{n} \bra*{m} \sum_{j, k} a_j^* a_k \ket*{k} \bra*{j} \\
        & =\sum_{m, n, j, k} a_m^* a_n a_j^* \ket*{n} \braket*{m}{k} \bra*{j} \\
        & =\sum_{m, n, j, k} a_m^* a_n a_j^* a_k \dyad*{n}{j} \delta_{m k} \\
        & = \underbrace{\sum_m a_m^* a_m}_{= \braket*{\psi}{\psi} = 1} \sum_{n, j} a_n a_j^* \dyad*{n}{j} \\ 
        &= \sum_{n, j} a_n a_j^* \dyad*{n}{j} = \rho.
    \end{aligned}
\end{equation}

\section{Ammonia molecule}

The Hamiltonian of the two low-energy states of ammonia is 
\begin{equation}
    H = \pmqty{
        0 & \Delta / 2 \\
        \Delta / 2 & 0
    },
\end{equation}
where we set 
\begin{equation}
    \ket*{L} = \pmqty{1 \\ 0}, \quad
    \ket*{R} = \pmqty{0 \\ 1}.
\end{equation}
This Hamiltonian is just a scaled $\sigma^x$ matrix, 
and its eigenstates are straightforwardly given by 
\begin{equation}
    \ket*{+} = \frac{1}{\sqrt{2}} (\ket*{L} + \ket*{R}), \quad 
    \ket*{-} = \frac{1}{\sqrt{2}} (\ket*{L} - \ket*{R}),
\end{equation}
and the energies are 
\begin{equation}
    E_+ = \Delta / 2, \quad E_- = - \Delta / 2.
\end{equation}

After an electric field is added, 
in $\ket*{L}$ we have an additional energy contribution, 
and since the molecular configuration in $\ket*{R}$ is the opposite 
of the one in $\ket*{L}$, 
we have 
\begin{equation}
    H = \pmqty{
        dE / 2 & \Delta / 2 \\
        \Delta / 2 & -dE / 2
    }.
\end{equation}
Solving 
\[
    \det \pmqty{
        dE / 2 - \lambda & \Delta / 2 \\
        \Delta / 2 & -dE / 2 - \lambda
    }    = 0, 
\]
we get 
\begin{equation}
    E_\pm = \pm \frac{1}{2} \sqrt{\Delta^2 + d^2 E^2}, 
\end{equation}
and hence 
\begin{equation}
    \ket*{-} = \frac{dE - \sqrt{d^2 E^2 + \Delta^2}}{\sqrt{
        \Delta^2 + (\sqrt{d^2 E^2 + \Delta^2} - d E)^2
    }} \ket*{L} 
    + \frac{\Delta}{\sqrt{
        \Delta^2 + (\sqrt{d^2 E^2 + \Delta^2} - d E)^2}
    } \ket*{R}, 
\end{equation}
and 
\begin{equation}
    \ket*{+} = \frac{dE + \sqrt{d^2 E^2 + \Delta^2}}{\sqrt{
        \Delta^2 + (\sqrt{d^2 E^2 + \Delta^2} + d E)^2
    }} \ket*{L} 
    + \frac{\Delta}{\sqrt{
        \Delta^2 + (\sqrt{d^2 E^2 + \Delta^2} + d E)^2}
    } \ket*{R}.
\end{equation}

When $E$ is large, 
we have 
\[
    d E - \sqrt{d^2 E^2 + \Delta^2} \approx 0, 
\]
\[
    d E + \sqrt{d^2 E^2 + \Delta^2} \approx 2 d E,   
\]
and therefore 
\begin{equation}
    E_+ = \frac{1}{2} d E, \quad 
    \ket*{+} = \ket*{L},
\end{equation}
and 
\begin{equation}
    E_- = - \frac{1}{2} d E, \quad 
    \ket*{-} = \ket*{R}.
\end{equation}
This is expected, because when $E d\gg \Delta$, 
the non-diagonal terms in the Hamiltonian can be safely ignored. 

\section{Kaptiza’s pendulum}

\subsection{Integrating out the fast variable}

In the $\omega \to \infty$, $F_0 \to 0$ limit, 
the high-frequency part and the low-frequency part of 
the solution of 
\begin{equation}
    m R \ddot{\theta}=\left(-m g+F_0 \sin \omega t\right) \sin \theta
    \label{eq:kapitza-eq}
\end{equation}
are not strongly coupled and the high-frequency degree of freedom can be integrated out 
to get an effective theory of the low-frequency part. 
We do the decomposition 
\begin{equation}
    \theta = \theta_f + \theta_s, 
\end{equation}
where $\theta_f$ is the fast variable. 
Observing \eqref{eq:kapitza-eq}, 
we find the EOM of $\theta_f$ should be 
\begin{equation}
    mR \ddot{\theta}_f = F_0 \sin \omega t \sin (\theta_f + \theta_s),
    \label{eq:fast-part-exact}
\end{equation}
because the first term on the RHS of \eqref{eq:kapitza-eq} 
has a much lower frequency magnitude compared with $\omega$.
We take the first order approximation of \eqref{eq:fast-part-exact} 
and ignore the $\theta_f$ dependency on the RHS,
and this gives 
\begin{equation}
    \theta_f = - \frac{F_0}{m R \omega^2} \sin \theta_s \sin \omega t.
\end{equation}
Putting this back to \eqref{eq:kapitza-eq}, 
we get 
\[
    \begin{aligned}
        m R \left(
            \frac{F_0}{mR} \sin \theta_s \sin \omega t + \ddot{\theta}_s
        \right) &= 
        (- mg + F_0 \sin \omega t) 
        \sin(
            \theta_s - \frac{F_0}{m R \omega^2} \sin \theta_s \sin \omega t
        ) \\
        &= (- mg + F_0 \sin \omega t) 
        \left(
            \sin \theta_s - \cos \theta_s \cdot \frac{F_0}{m R \omega^2} \sin \theta_s \sin \omega t
        \right).
    \end{aligned}
\]
Now we average over all high-frequency time dependencies. 
The first term on the LHS averages zero, 
and so do the $- mg \sin \omega t$ term and 
the $F_0 \sin \omega t \sin \theta_s$ term on the RHS. 
On the other hand, 
the $\sin^2 \omega t$ term on the RHS averages 
\[
    - \frac{F_0^2}{m R \omega^2} \sin \theta_s \cos \theta_s \expval{\sin^2 \omega t}
    = - \frac{1}{2} \frac{F_0^2}{m R \omega^2} \sin \theta_s \cos \theta_s,
\]
so the final EOM for $\theta_s$ is 
\begin{equation}
    m R \ddot{\theta}_s = - m g \sin \theta_s 
    - \frac{1}{2} \frac{F_0^2}{m R \omega^2} \sin \theta_s \cos \theta_s.
    \label{eq:theta-s-eq}
\end{equation}

\subsection{Stable positions of $\theta_s$}

We let the LHS of \eqref{eq:theta-s-eq} be zero, 
and the equation becomes 
\[
    \sin \theta_s \left(
        mg + \frac{1}{2} \frac{F_0^2}{m R \omega^2} \cos \theta_s
    \right) = 0.
\]
Since $F_0 \to 0$, the second factor on the LHS can't be zero, 
so the equation becomes 
\begin{equation}
    \sin \theta_s = 0 \Rightarrow \theta_s = 0, \pi.
\end{equation}

Around $\theta_s = 0$, \eqref{eq:theta-s-eq} is approximately 
\[
    m R \ddot{\theta}_s = - mg \theta_s - \frac{1}{2} \frac{F_0^2}{m R \omega^2} \theta_s, 
\]
and therefore 
\begin{equation}
    \omega_{\theta = 0} = \sqrt{
        \frac{g}{R} + \frac{F_0^2}{m^2 R^2 \omega^2}
    }.
\end{equation}
This is always real, and therefore the $\theta_s = 0$ position is always stable.

Around $\theta_s = \pi$, we rewrite \eqref{eq:theta-s-eq} in terms of $\theta_s' = \pi - \theta_s$, 
and get 
\[
    - m R \ddot{\theta}'_s = mg \theta'_s - \frac{1}{2} \frac{F_0^2}{m R \omega^2} \theta_s' = 0,
\]
and 
\begin{equation}
    \omega_{\theta_s = \pi} = \sqrt{\frac{1}{2} \frac{F_0^2}{m^2 R^2 \omega^2} - \frac{g}{R}}.
\end{equation}
It can be seen that when $F_0 = 0$, 
the frequency is imaginary and therefore $\theta_s = \pi$ 
is not a stable position. 
However, when 
\begin{equation}
    \frac{1}{2} \frac{F_0^2}{m^2 R^2 \omega^2} \geq \frac{g}{R},
\end{equation}
we do have oscillation behavior around $\theta_s = \pi$.

\section{Relaxation of a spin polarization due to an electric field}

\subsection{The spin-magnetic field coupling}

We have 
\begin{equation}
    H = - \vb*{\mu} \cdot \vb*{B}, \quad \vb*{\mu} = - g \muB \vb*{S},
\end{equation}
and this means 
\begin{equation}
    H = \frac{1}{2} g \muB \vb*{\sigma} \cdot \vb*{B}.
\end{equation}
This gives
\begin{equation}
    \dv{\vb*{S}}{t} = \frac{g \muB}{\hbar} \vb*{B} \times \vb*{S}.
\end{equation}
When $\vb*{B}$ is fixed (instead of the motion magnetic field in \eqref{eq:motion-magnetic}),
we find the oscillation frequency is 
\begin{equation}
    \omega = \frac{g \muB}{\hbar} B,
\end{equation}
and therefore 
\begin{equation}
    \gamma = \frac{g \muB}{\hbar}.
\end{equation}

\subsection{The $\sigma_{\pm}$ representation}

We define 
\begin{equation}
    \sigma_\pm = \frac{\sigma_x \pm \ii \sigma_y}{2}, 
\end{equation}
and the commutation relations are 
\begin{equation}
    \comm*{\sigma_+}{\sigma_-} =  \sigma_z, \quad 
    \comm*{\sigma^z}{\sigma_\pm} = \pm 2 \sigma_\pm.
\end{equation}
The Hamiltonian is then 
\begin{equation}
    H = b^* \sigma_+ + b \sigma_- + b_z \sigma_z, 
\end{equation}
where 
\begin{equation}
    b = \frac{1}{2} g \muB (B_x + \ii B_y), \quad 
    b_z = \frac{1}{2} g \muB B_z.
    \label{eq:b-def}
\end{equation}

\subsection{Eliminating the static magnetic field}

The magnetic field felt by an electron wit velocity $\vb*{v}$ 
when an electric field $\vb*{E}$ is present is 
\begin{equation}
    \vb*{B}_1 =-\vb*{v} \times \vb*{E} / c^2.
    \label{eq:motion-magnetic}
\end{equation}
We also have a static magnetic field $\vb*{B}_0$ pointing towards $\vu*{z}$.
Here we assume $\vb*{B}_0$ and the motion magnetic field 
are orthogonal to each other, 
and this means the Hamiltonian is 
\begin{equation}
    H = b^* \sigma_+ + b \sigma_- + \frac{1}{2} \hbar \gamma B_0
    = b^* \sigma_+ + b \sigma_- + \underbrace{\frac{1}{2} \hbar \omega_0 \sigma_z}_{H_0}.
\end{equation}
The time evolution corresponding to that $\omega_0$ term is trivial, 
and using the interaction picture we can get rid of it. 
We have 
\begin{equation}
    \ee^{\ii H_0 t / \hbar} = \pmqty{
        \dmat{
            \ee^{\ii \omega_0 t / 2}, 
            \ee^{- \ii \omega_0 t / 2}
        }
    }.
\end{equation}
The new Hamiltonian is now 
\begin{equation}
    \begin{aligned}
        H &= \ee^{\ii H_0 t / \hbar} (b^* \sigma_+ + b \sigma_-) \ee^{- \ii H_0  t/ \hbar} \\
        &= \ee^{\ii \omega_0 t} b^* \sigma_+ + \ee^{- \ii \omega_0 t} b \sigma_-.
    \end{aligned}
\end{equation}
We will work with this Hamiltonian later.

\subsection{Time evolution}

Now we use 
\begin{equation}
    \frac{\partial \rho(t)}{\partial t}=-\frac{1}{\hbar^2} \int_0^t \dd t^{\prime}\left[H(t),\left[H\left(t^{\prime}\right), \rho(t)\right]\right] 
    = - \frac{1}{\hbar^2} \int_{0}^t \dd{t'} \comm*{H(t)}{\comm*{H(t - t')}{\rho(t)}}
\end{equation}
to find the time evolution of $\rho$, 
considering only second-order correlation 
in the random variable in $H$.
%TODO: how to improve this?
We again do decomposition
\begin{equation}
    \rho = \frac{1}{2} \sigma_0 + \rho_+ \sigma_+ + \rho_- \sigma_- + \rho_z \sigma_z,
\end{equation}
and we have (here $b'$ and $b'^*$ are values taken at $t - t'$)
\[
    \comm*{H(t - t')}{\rho(t)}
    = - \rho_+ \ee^{- \ii \omega_0 (t - t')} b' \sigma_z 
    + \rho_- \ee^{\ii \omega_0 (t - t')} b'^* \sigma_z 
    - 2 \rho_z \ee^{\ii \omega_0 (t - t')} b'^* \sigma_+ 
    + 2 \rho_z \ee^{- \ii \omega_0 (t - t')} b' \sigma_-,
\]
and by component, we get 
\begin{equation}
    \dot{\rho}_{+} = - \frac{1}{\hbar^2} \int_{0}^{t} \dd{t'} \left(
        2 \rho_+ \ee^{\ii \omega_0 t'} b' b^* 
        - 2 \rho_- \ee^{2 \ii \omega_0 t - \ii \omega_0 t'} b^* b'^*
    \right),
\end{equation}
\begin{equation}
    \dot{\rho}_{-} = - \frac{1}{\hbar^2} \int_{0}^{t} \dd{t'} \left(
        2 \rho_- \ee^{- \ii \omega_0 t'} b b'^* 
        - 2 \rho_+ \ee^{- 2 \ii \omega_0 t + \ii \omega_0 t'} b b'
    \right),
\end{equation}
and 
\begin{equation}
    \dot{\rho}_z = - \frac{1}{\hbar^2} \int_{0}^{t} \dd{t'} \left(
        2 \rho_z \ee^{- \ii \omega_0 t'} b b'^* 
        + 2 \rho_z \ee^{\ii \omega_0 t'} b^* b'
    \right).
\end{equation}

\subsection{Statistical average}

We now need to average the above three equations 
over all possible $b(t)$ configurations. 
Since we have \eqref{eq:b-def},
we find 
\[
    \expval*{b b'} \propto \expval*{B_x B_x'} - \expval*{B_y B_y'} + 2 \ii \expval*{B_x B_y' + B_y B_x'}.
\]
The imaginary part of this vanishes, 
because there is no correlation between orthogonal directions, 
and the real part also vanishes 
because of spatial rotational symmetry.
On the other hand, we have 

\end{document}