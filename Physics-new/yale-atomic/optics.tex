\documentclass[hyperref, a4paper]{article}

\usepackage{geometry}
\usepackage{titling}
\usepackage{titlesec}
% No longer needed, since we will use enumitem package
% \usepackage{paralist}
\usepackage{enumitem}
\usepackage{footnote}
\usepackage{amsmath, amssymb, amsthm}
\usepackage{mathtools}
\usepackage{bbm}
\usepackage{cite}
\usepackage{graphicx}
\usepackage{subfigure}
\usepackage{physics}
\usepackage{tensor}
\usepackage{siunitx}
\usepackage[version=4]{mhchem}
\usepackage{tikz}
\usepackage{xcolor}
\usepackage{listings}
\usepackage{underscore}
\usepackage{autobreak}
\usepackage[ruled, vlined, linesnumbered]{algorithm2e}
\usepackage{nameref,zref-xr}
\zxrsetup{toltxlabel}
\usepackage[colorlinks,unicode]{hyperref} % , linkcolor=black, anchorcolor=black, citecolor=black, urlcolor=black, filecolor=black
\usepackage[most]{tcolorbox}
\usepackage{prettyref}

% Page style
\geometry{left=3.18cm,right=3.18cm,top=2.54cm,bottom=2.54cm}
\titlespacing{\paragraph}{0pt}{1pt}{10pt}[20pt]
\setlength{\droptitle}{-5em}

% More compact lists 
\setlist[itemize]{
    %itemindent=17pt, 
    %leftmargin=1pt,
    listparindent=\parindent,
    parsep=0pt,
}

\setlist[enumerate]{
    %itemindent=17pt, 
    %leftmargin=1pt,
    listparindent=\parindent,
    parsep=0pt,
}

% Math operators
\DeclareMathOperator{\timeorder}{\mathcal{T}}
\DeclareMathOperator{\diag}{diag}
\DeclareMathOperator{\legpoly}{P}
\DeclareMathOperator{\primevalue}{P}
\DeclareMathOperator{\sgn}{sgn}
\DeclareMathOperator{\res}{Res}
\newcommand*{\ii}{\mathrm{i}}
\newcommand*{\ee}{\mathrm{e}}
\newcommand*{\const}{\mathrm{const}}
\newcommand*{\suchthat}{\quad \text{s.t.} \quad}
\newcommand*{\argmin}{\arg\min}
\newcommand*{\argmax}{\arg\max}
\newcommand*{\normalorder}[1]{: #1 :}
\newcommand*{\pair}[1]{\langle #1 \rangle}
\newcommand*{\fd}[1]{\mathcal{D} #1}
\DeclareMathOperator{\bigO}{\mathcal{O}}

% TikZ setting
\usetikzlibrary{arrows,shapes,positioning}
\usetikzlibrary{arrows.meta}
\usetikzlibrary{decorations.markings}
\usetikzlibrary{calc}
\tikzstyle arrowstyle=[scale=1]
\tikzstyle directed=[postaction={decorate,decoration={markings,
    mark=at position .5 with {\arrow[arrowstyle]{stealth}}}}]
\tikzstyle ray=[directed, thick]
\tikzstyle dot=[anchor=base,fill,circle,inner sep=1pt]

% Algorithm setting
% Julia-style code
\SetKwIF{If}{ElseIf}{Else}{if}{}{elseif}{else}{end}
\SetKwFor{For}{for}{}{end}
\SetKwFor{While}{while}{}{end}
\SetKwProg{Function}{function}{}{end}
\SetArgSty{textnormal}

\newcommand*{\concept}[1]{{\textbf{#1}}}

% Embedded codes
\lstset{basicstyle=\ttfamily,
  showstringspaces=false,
  commentstyle=\color{gray},
  keywordstyle=\color{blue}
}

\lstdefinestyle{console}{
    basicstyle=\footnotesize\ttfamily,
    breaklines=true,
    postbreak=\mbox{\textcolor{red}{$\hookrightarrow$}\space}
}

% Reference formatting
\newrefformat{fig}{Figure~\ref{#1}}

% Color boxes
\tcbuselibrary{skins, breakable, theorems}
\newtcbtheorem[number within=section]{warning}{Warning}%
  {colback=orange!5,colframe=orange!65,fonttitle=\bfseries, breakable}{warn}
\newtcbtheorem[number within=section]{note}{Note}%
  {colback=green!5,colframe=green!65,fonttitle=\bfseries, breakable}{note}
\newtcbtheorem[number within=section]{info}{Info}%
  {colback=blue!5,colframe=blue!65,fonttitle=\bfseries, breakable}{info}

% Displaying texts in bookmarkers

\pdfstringdefDisableCommands{%
  \def\\{}%
  \def\ce#1{<#1>}%
}

\pdfstringdefDisableCommands{%
  \def\texttt#1{<#1>}%
  \def\mathbb#1{#1}%
}
\pdfstringdefDisableCommands{\def\eqref#1{(\ref{#1})}}

\makeatletter
\pdfstringdefDisableCommands{\let\HyPsd@CatcodeWarning\@gobble}
\makeatother

\newenvironment{shelldisplay}{\begin{lstlisting}}{\end{lstlisting}}

\newcommand{\shortcode}[1]{\texttt{#1}}

\lstset{style = console}

% Make subsubsection labeled
\setcounter{secnumdepth}{4}
\setcounter{tocdepth}{4}

\title{Atomic physics: theories of laser beams and atoms}
\author{Jinyuan Wu}

\begin{document}

\maketitle

\section{Overview of atoms}

Atomic physics relies on the existence of atoms.
The idea that matter consists of atoms 
was seen in ancient Greek,
probably earlier in ancient India.
In the West, people usually attribute the idea to Democritus (450 BC),
who claimed that there are only atoms and ``void''.
The modern idea of atoms arose to explain the behavior of gases:
if we assume the basic degrees of freedom of a gas systems are 
$\vb*{r}_i$'s and $\vb*{p}_i$'s,
then everything works well -- 
except we need to use a constant $\hbar$ to decide the correct entropy of the gas:
\begin{equation}
    \dd{\Omega} = \frac{\dd[3]{\vb*{r}} \dd[3]{\vb*{p}}}{h^3}, \quad h = 2\pi \hbar.
\end{equation}
The origin of $\hbar$ led to the discovery of quantum mechanics.

The quantum nature of atoms is best demonstrated by beam splitters.
Beam splitters are widely used in optics:
they can be used to make a Mach-Zehnder interferometer, 
and the observed intensity takes the form of 
\[
    (\cos(\omega t + \phi_1) + \cos(\omega t + \phi_2))^2,
\]
and by moving the mirrors, 
we change $\phi_{1, 2}$,
and thus peaks and valleys occur in the relation between $I$ and $\phi_{1, 2}$.

\begin{figure}
    \centering

    \begin{tikzpicture}[x=0.75pt,y=0.75pt,yscale=-1,xscale=1]
    %uncomment if require: \path (0,300); %set diagram left start at 0, and has height of 300
    
    %Shape: Rectangle [id:dp4592102294328868] 
    \draw   (242.28,104.79) -- (192.79,154.28) -- (185.72,147.21) -- (235.21,97.72) -- cycle ;
    %Straight Lines [id:da8168249070848908] 
    \draw    (110,122) -- (206.83,122) ;
    \draw [shift={(209.83,122)}, rotate = 180] [fill={rgb, 255:red, 0; green, 0; blue, 0 }  ][line width=0.08]  [draw opacity=0] (10.72,-5.15) -- (0,0) -- (10.72,5.15) -- (7.12,0) -- cycle    ;
    %Straight Lines [id:da15292644630172703] 
    \draw    (209.83,122) -- (209.83,39) ;
    \draw [shift={(209.83,36)}, rotate = 90] [fill={rgb, 255:red, 0; green, 0; blue, 0 }  ][line width=0.08]  [draw opacity=0] (10.72,-5.15) -- (0,0) -- (10.72,5.15) -- (7.12,0) -- cycle    ;
    %Straight Lines [id:da76766140394506] 
    \draw    (218.83,128) -- (315.67,128) ;
    \draw [shift={(318.67,128)}, rotate = 180] [fill={rgb, 255:red, 0; green, 0; blue, 0 }  ][line width=0.08]  [draw opacity=0] (10.72,-5.15) -- (0,0) -- (10.72,5.15) -- (7.12,0) -- cycle    ;
    %Straight Lines [id:da5571278289677006] 
    \draw    (218.83,214) -- (218.83,131) ;
    \draw [shift={(218.83,128)}, rotate = 90] [fill={rgb, 255:red, 0; green, 0; blue, 0 }  ][line width=0.08]  [draw opacity=0] (10.72,-5.15) -- (0,0) -- (10.72,5.15) -- (7.12,0) -- cycle    ;
    
    % Text Node
    \draw (108,122) node [anchor=east] [inner sep=0.75pt]    {$E_{1}$};
    % Text Node
    \draw (218.83,217.4) node [anchor=north] [inner sep=0.75pt]    {$E_{2}$};
    % Text Node
    \draw (209.83,32.6) node [anchor=south] [inner sep=0.75pt]    {$E_{4}$};
    % Text Node
    \draw (320.67,128) node [anchor=west] [inner sep=0.75pt]    {$E_{3}$};
    
    
    \end{tikzpicture}
    \caption{Beam splitter example}
    \label{fig:beam-splitter}    
\end{figure}

NMore generally, in \prettyref{fig:beam-splitter}, we have 
\begin{equation}
    \begin{aligned}
        \abs*{\vb*{E}_3}^2 + \abs{\vb*{E}_4}^2 &= 
        (t \vb*{E}_1 + r \vb*{E}_2) (t \vb*{E}_1 + r \vb*{E}_2)^*
        + (t \vb*{E}_2 + r \vb*{E}_1) (t \vb*{E}_2 + r \vb*{E}_1)^* \\
        &= (\abs*{\vb*{E}_1}^2 + \abs*{\vb*{E}_2}^2) (\abs*{t}^2 + \abs*{r}^2)
        + (\vb*{E}_1^* \cdot \vb*{E}_2 + \text{h.c.}) (t^* r + r^* t).
    \end{aligned}
\end{equation}
To ensure energy conservation, we get 
\begin{equation}
    \abs*{t}^2 + \abs*{r}^2 = 1,
\end{equation}
which is expected, and 
\begin{equation}
    t^* r + r^* t = 0.
\end{equation}
The last equation means there has to be a phase shift caused by the beam splitter. 
In the \SI{50}{\percent}-\SI{50}{\percent} case, 
we have $r = t \ee^{\ii \phi}$,
and solving the above equation, 
we find $\phi = \SI{90}{\degree}$.
Thus a \SI{50}{\percent}-\SI{50}{\percent} beam splitter introduces a \SI{90}{\degree} phase difference 
between the two output beams 
if one of the input beam is absent, 
i.e. the corresponding input port is a dark port.

A comment: when we deal with actual light,
not just atom beams, 
polarization also influences the phase shift.

\end{document}