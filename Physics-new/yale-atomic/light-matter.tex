\documentclass[hyperref, a4paper]{article}

\usepackage{geometry}
\usepackage{titling}
\usepackage{titlesec}
% No longer needed, since we will use enumitem package
% \usepackage{paralist}
\usepackage{enumitem}
\usepackage{footnote}
\usepackage{amsmath, amssymb, amsthm}
\usepackage{mathtools}
\usepackage{bbm}
\usepackage{cite}
\usepackage{graphicx}
\usepackage{subfigure}
\usepackage{physics}
\usepackage{tensor}
\usepackage{siunitx}
\usepackage[version=4]{mhchem}
\usepackage{tikz}
\usepackage{xcolor}
\usepackage{listings}
\usepackage{underscore}
\usepackage{autobreak}
\usepackage[ruled, vlined, linesnumbered]{algorithm2e}
\usepackage{nameref,zref-xr}
\zxrsetup{toltxlabel}
\usepackage[colorlinks,unicode]{hyperref} % , linkcolor=black, anchorcolor=black, citecolor=black, urlcolor=black, filecolor=black
\usepackage[most]{tcolorbox}
\usepackage{prettyref}

% Page style
\geometry{left=3.18cm,right=3.18cm,top=2.54cm,bottom=2.54cm}
\titlespacing{\paragraph}{0pt}{1pt}{10pt}[20pt]
\setlength{\droptitle}{-5em}

% More compact lists 
\setlist[itemize]{
    %itemindent=17pt, 
    %leftmargin=1pt,
    listparindent=\parindent,
    parsep=0pt,
}

\setlist[enumerate]{
    %itemindent=17pt, 
    %leftmargin=1pt,
    listparindent=\parindent,
    parsep=0pt,
}

% Math operators
\DeclareMathOperator{\timeorder}{\mathcal{T}}
\DeclareMathOperator{\diag}{diag}
\DeclareMathOperator{\legpoly}{P}
\DeclareMathOperator{\primevalue}{P}
\DeclareMathOperator{\sgn}{sgn}
\DeclareMathOperator{\res}{Res}
\newcommand*{\ii}{\mathrm{i}}
\newcommand*{\ee}{\mathrm{e}}
\newcommand*{\const}{\mathrm{const}}
\newcommand*{\suchthat}{\quad \text{s.t.} \quad}
\newcommand*{\argmin}{\arg\min}
\newcommand*{\argmax}{\arg\max}
\newcommand*{\normalorder}[1]{: #1 :}
\newcommand*{\pair}[1]{\langle #1 \rangle}
\newcommand*{\fd}[1]{\mathcal{D} #1}
\DeclareMathOperator{\bigO}{\mathcal{O}}

% TikZ setting
\usetikzlibrary{arrows,shapes,positioning}
\usetikzlibrary{arrows.meta}
\usetikzlibrary{decorations.markings}
\usetikzlibrary{calc}
\tikzstyle arrowstyle=[scale=1]
\tikzstyle directed=[postaction={decorate,decoration={markings,
    mark=at position .5 with {\arrow[arrowstyle]{stealth}}}}]
\tikzstyle ray=[directed, thick]
\tikzstyle dot=[anchor=base,fill,circle,inner sep=1pt]

% Algorithm setting
% Julia-style code
\SetKwIF{If}{ElseIf}{Else}{if}{}{elseif}{else}{end}
\SetKwFor{For}{for}{}{end}
\SetKwFor{While}{while}{}{end}
\SetKwProg{Function}{function}{}{end}
\SetArgSty{textnormal}

\newcommand*{\concept}[1]{{\textbf{#1}}}

% Embedded codes
\lstset{basicstyle=\ttfamily,
  showstringspaces=false,
  commentstyle=\color{gray},
  keywordstyle=\color{blue}
}

\lstdefinestyle{console}{
    basicstyle=\footnotesize\ttfamily,
    breaklines=true,
    postbreak=\mbox{\textcolor{red}{$\hookrightarrow$}\space}
}

% Reference formatting
\newrefformat{fig}{Figure~\ref{#1}}

% Color boxes
\tcbuselibrary{skins, breakable, theorems}
\newtcbtheorem[number within=section]{warning}{Warning}%
  {colback=orange!5,colframe=orange!65,fonttitle=\bfseries, breakable}{warn}
\newtcbtheorem[number within=section]{note}{Note}%
  {colback=green!5,colframe=green!65,fonttitle=\bfseries, breakable}{note}
\newtcbtheorem[number within=section]{info}{Info}%
  {colback=blue!5,colframe=blue!65,fonttitle=\bfseries, breakable}{info}

% Displaying texts in bookmarkers

\pdfstringdefDisableCommands{%
  \def\\{}%
  \def\ce#1{<#1>}%
}

\pdfstringdefDisableCommands{%
  \def\texttt#1{<#1>}%
  \def\mathbb#1{#1}%
}
\pdfstringdefDisableCommands{\def\eqref#1{(\ref{#1})}}

\makeatletter
\pdfstringdefDisableCommands{\let\HyPsd@CatcodeWarning\@gobble}
\makeatother

\newenvironment{shelldisplay}{\begin{lstlisting}}{\end{lstlisting}}

\newcommand{\shortcode}[1]{\texttt{#1}}

\lstset{style = console}

% Make subsubsection labeled
\setcounter{secnumdepth}{4}
\setcounter{tocdepth}{4}

\newcommand{\kB}{k_{\text{B}}}

\title{Atomic physics: theories of laser beams and atoms}
\author{Jinyuan Wu}

\begin{document}

\maketitle

\section{Overview of atoms}

Atomic physics relies on the existence of atoms.
The idea that matter consists of atoms 
was seen in ancient Greek,
probably earlier in ancient India.
In the West, people usually attribute the idea to Democritus (450 BC),
who claimed that there are only atoms and ``void''.
The modern idea of atoms arose to explain the behavior of gases:
if we assume the basic degrees of freedom of a gas systems are 
$\vb*{r}_i$'s and $\vb*{p}_i$'s,
then everything works well -- 
except we need to use a constant $\hbar$ to decide the correct entropy of the gas:
\begin{equation}
    \dd{\Omega} = \frac{\dd[3]{\vb*{r}} \dd[3]{\vb*{p}}}{h^3}, \quad h = 2\pi \hbar.
\end{equation}
The origin of $\hbar$ led to the discovery of quantum mechanics.

The quantum nature of atoms is best demonstrated by beam splitters.
Beam splitters are widely used in optics:
they can be used to make a Mach-Zehnder interferometer, 
and the observed intensity takes the form of 
\[
    (\cos(\omega t + \phi_1) + \cos(\omega t + \phi_2))^2,
\]
and by moving the mirrors, 
we change $\phi_{1, 2}$,
and thus peaks and valleys occur in the relation between $I$ and $\phi_{1, 2}$.

\begin{figure}
    \centering

    \begin{tikzpicture}[x=0.75pt,y=0.75pt,yscale=-1,xscale=1]
    %uncomment if require: \path (0,300); %set diagram left start at 0, and has height of 300
    
    %Shape: Rectangle [id:dp4592102294328868] 
    \draw   (242.28,104.79) -- (192.79,154.28) -- (185.72,147.21) -- (235.21,97.72) -- cycle ;
    %Straight Lines [id:da8168249070848908] 
    \draw    (110,122) -- (206.83,122) ;
    \draw [shift={(209.83,122)}, rotate = 180] [fill={rgb, 255:red, 0; green, 0; blue, 0 }  ][line width=0.08]  [draw opacity=0] (10.72,-5.15) -- (0,0) -- (10.72,5.15) -- (7.12,0) -- cycle    ;
    %Straight Lines [id:da15292644630172703] 
    \draw    (209.83,122) -- (209.83,39) ;
    \draw [shift={(209.83,36)}, rotate = 90] [fill={rgb, 255:red, 0; green, 0; blue, 0 }  ][line width=0.08]  [draw opacity=0] (10.72,-5.15) -- (0,0) -- (10.72,5.15) -- (7.12,0) -- cycle    ;
    %Straight Lines [id:da76766140394506] 
    \draw    (218.83,128) -- (315.67,128) ;
    \draw [shift={(318.67,128)}, rotate = 180] [fill={rgb, 255:red, 0; green, 0; blue, 0 }  ][line width=0.08]  [draw opacity=0] (10.72,-5.15) -- (0,0) -- (10.72,5.15) -- (7.12,0) -- cycle    ;
    %Straight Lines [id:da5571278289677006] 
    \draw    (218.83,214) -- (218.83,131) ;
    \draw [shift={(218.83,128)}, rotate = 90] [fill={rgb, 255:red, 0; green, 0; blue, 0 }  ][line width=0.08]  [draw opacity=0] (10.72,-5.15) -- (0,0) -- (10.72,5.15) -- (7.12,0) -- cycle    ;
    
    % Text Node
    \draw (108,122) node [anchor=east] [inner sep=0.75pt]    {$E_{1}$};
    % Text Node
    \draw (218.83,217.4) node [anchor=north] [inner sep=0.75pt]    {$E_{2}$};
    % Text Node
    \draw (209.83,32.6) node [anchor=south] [inner sep=0.75pt]    {$E_{4}$};
    % Text Node
    \draw (320.67,128) node [anchor=west] [inner sep=0.75pt]    {$E_{3}$};
    
    
    \end{tikzpicture}
    \caption{Beam splitter example}
    \label{fig:beam-splitter}    
\end{figure}

NMore generally, in \prettyref{fig:beam-splitter}, we have 
\begin{equation}
    \begin{aligned}
        \abs*{\vb*{E}_3}^2 + \abs{\vb*{E}_4}^2 &= 
        (t \vb*{E}_1 + r \vb*{E}_2) (t \vb*{E}_1 + r \vb*{E}_2)^*
        + (t \vb*{E}_2 + r \vb*{E}_1) (t \vb*{E}_2 + r \vb*{E}_1)^* \\
        &= (\abs*{\vb*{E}_1}^2 + \abs*{\vb*{E}_2}^2) (\abs*{t}^2 + \abs*{r}^2)
        + (\vb*{E}_1^* \cdot \vb*{E}_2 + \text{h.c.}) (t^* r + r^* t).
    \end{aligned}
\end{equation}
To ensure energy conservation, we get 
\begin{equation}
    \abs*{t}^2 + \abs*{r}^2 = 1,
\end{equation}
which is expected, and 
\begin{equation}
    t^* r + r^* t = 0.
\end{equation}
The last equation means there has to be a phase shift caused by the beam splitter. 
In the \SI{50}{\percent}-\SI{50}{\percent} case, 
we have $r = t \ee^{\ii \phi}$,
and solving the above equation, 
we find $\phi = \SI{90}{\degree}$.
Thus a \SI{50}{\percent}-\SI{50}{\percent} beam splitter introduces a \SI{90}{\degree} phase difference 
between the two output beams 
if one of the input beam is absent, 
i.e. the corresponding input port is a dark port.

A comment: when we deal with actual light,
not just atom beams, 
polarization also influences the phase shift.

\section{Blackbody radiation}

Blackbody radiation is where atomic physics started.
Both the candle and the incandescent light bulb
give light when heated up.
What makes it interesting is,
if we heat up a black object, 
radiation also occurs:
this can be demonstrated by 
putting a light mill before a piece of heated black object:
the vanes rotate in the same way that 
they rotate when placed before a light bulb.
When the temperature isn't too high,
we can put a infrared screener between the light mill and the heated black object
and we find the rotation rate of the light mill goes down.
This means the relation between radiation and temperature of many objects 
is at least quantitatively modeled by the ideal black body,
and therefore studying its behavior is important.

A blackbody can be modeled as a hole on a closed optical cavity:
once something goes inside, 
it's highly unlikely to be reflected back.
The possible optical modes between two walls are like 
\[
    \sin(\frac{n \pi x}{L}) \quad \text{or} \quad 
    \sin(\frac{n \pi x}{L}),
\]
and $n = 0, 1, 2, 3, \dots$.
In a cubic box, we have $n_1, n_2, n_3$ to label all the modes, 
and TODO 

We will skip the tedious historical attempts to derive the correct blackbody radiation formula,
and just jump to the final answer:
the energy of each light mode has to be quantized.
The first try is to assume 
\begin{equation}
    E_{\text{mode $\omega$}} = \hbar \omega n.
\end{equation}
Thus, for each light mode, we have 
\begin{equation}
    Z = \sum_{n=0}^\infty \ee^{- n \hbar \omega / \kB T} = \frac{1}{1 - \ee^{- \hbar \omega / \kB T}} ,
\end{equation}
and therefore 
\begin{equation}
    \expval{E} = \frac{1}{Z} \sum_{n=0}^\infty n \hbar \omega \ee^{- n \hbar \omega / \kB T} 
    = \frac{\hbar \omega}{\ee^{\hbar \omega / \kB T} - 1}.
\end{equation}
The high temperature limit would we 
\[
    \expval{E} = \frac{\hbar \omega}{
        \frac{\hbar \omega}{\kB T} + \frac{1}{2} \left(\frac{\hbar \omega}{\kB T}\right)^2 + \cdots 
    } \to 
    \kB T \left(
        1 - \frac{1}{2} \frac{\hbar \omega}{\kB T} 
    \right) = \kB T - \frac{1}{2} \hbar \omega.
\]
Thus, in order to go back to the classical prediction when $T \to \infty$, 
the energy of a single mode has to be 
\begin{equation}
    E = \hbar \omega \left( n + \frac{1}{2} \right),
\end{equation}
and the correct expected energy is 
\begin{equation}
    \expval{E} = \frac{\hbar \omega}{
        \frac{\hbar \omega}{\kB T} + \frac{1}{2} \left(\frac{\hbar \omega}{\kB T}\right)^2 + \cdots 
    } \to 
    \kB T \left(
        1 - \frac{1}{2} \frac{\hbar \omega}{\kB T} 
    \right) = \kB T - \frac{1}{2} \hbar \omega
    + \frac{1}{2} \hbar \omega.
\end{equation}

Now the total energy of the cavity can be found: 
it's 
\begin{equation}
    U = \int_0^\infty \expval{E_\omega} n(\omega) \dd{\omega}
    = L^3 
\end{equation}
The energy jet can then be found:
\begin{equation}
    J = \frac{1}{4} c U = \underbrace{\frac{\pi^2 \kB^4}{60 \hbar^3 c^3}}_{\sigma} T^4.
\end{equation}
TODO: whether we need to multiply a 2 factor to $J$;
the direction of radiation; 
the vibrational $1/2$ factor; 
the 

Blackbody radiation is related to the temperature of the Earth, actually. 
The radiation power of the sun is 
\begin{equation}
    P_\text{s} = 4\pi r_{\text{s}}^2 \cdot \sigma T^4 \cdot \epsilon_{\text{s}}, 
\end{equation}
where $\epsilon_{\text{s}}$ is a (rather coarse) estimation 
of how much the Sun deviates from an ideal blackbody.
The energy flow at the position of the Earth is 
\begin{equation}
    I_{\text{e}} = \frac{P_{\text{s}}}{4 \pi R_{\text{e}}^2},
\end{equation}
and the power received by the Earth is 
\begin{equation}
    P_{\text{e}} = \pi r_{\text{e}}^2 I_{\text{e}} \epsilon_{\text{e} 1}.
\end{equation}
Here note that we are using the cross section of the Earth 
as the ``surface'', 
because actually what we have is 
\[
    P = \int \dd{\vb*{S}} \cdot \vb*{I}.
\]
The balance of energy, therefore, means 
\begin{equation}
    P_{\text{e}} = \pi r_{\text{e}}^2 I_{\text{e}} \epsilon_{\text{e} 1}
    = 4 \pi r_{\text{e}}^2 \sigma T_{\text{e}}^4 \epsilon_{\text{e} 2}.
    \label{eq:earth-balence-1}
\end{equation}
Albedo analysis means the absorption efficiency of the Earth is $\epsilon_{\text{e} 1} = 0.65$,
and if $\epsilon_{\text{e} 2} \approx 1$, 
we find $T_{\text{e}} = \SI{251}{K}$.
This of course goes against the observed result $\SI{300}{K}$.
The point here is the Earth is not really a blackbody ideal enough when it comes to radiation,
if we view the atmosphere as a part of the Earth:
heat radiation is reflected back,
and the amount of energy flowing into the space is reduced. 
This means the temperature of the Earth has to be higher than what is estimated 
using \eqref{eq:earth-balence-1}.

Einstein noticed one deficiency of the traditional picture of atomic radiation:
because of the spatial rotational symmetry, 
what we get is a spherical wave,
going out infinitely. 
Now this means an atom undergoes random walking because of this radiation.

Suppose $n_2$ is the occupation of the excited state, 
and $n_1$ is the occupation of the ground state. 
Then we have 
TODO: see my old notes 

We actually have 
\begin{equation}
    A = \rho(\text{zero point energy}) \dd{\omega} B = \frac{1}{2} \frac{\hbar \omega^3}{\pi^2 c^2}.
\end{equation}

\end{document}