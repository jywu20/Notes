\documentclass[hyperref, a4paper]{article}

\usepackage{geometry}
\usepackage{titling}
\usepackage{titlesec}
% No longer needed, since we will use enumitem package
% \usepackage{paralist}
\usepackage{enumitem}
\usepackage{footnote}
\usepackage{amsmath, amssymb, amsthm}
\usepackage{mathtools}
\usepackage{bbm}
\usepackage{cite}
\usepackage{graphicx}
\usepackage{subfigure}
\usepackage{physics}
\usepackage{tensor}
\usepackage{siunitx}
\usepackage[version=4]{mhchem}
\usepackage{tikz}
\usepackage{xcolor}
\usepackage{listings}
\usepackage{underscore}
\usepackage{autobreak}
\usepackage[ruled, vlined, linesnumbered]{algorithm2e}
\usepackage{nameref,zref-xr}
\zxrsetup{toltxlabel}
\usepackage[colorlinks,unicode]{hyperref} % , linkcolor=black, anchorcolor=black, citecolor=black, urlcolor=black, filecolor=black
\usepackage[most]{tcolorbox}
\usepackage{prettyref}

% Page style
\geometry{left=3.18cm,right=3.18cm,top=2.54cm,bottom=2.54cm}
\titlespacing{\paragraph}{0pt}{1pt}{10pt}[20pt]
\setlength{\droptitle}{-5em}

% More compact lists 
\setlist[itemize]{
    %itemindent=17pt, 
    %leftmargin=1pt,
    listparindent=\parindent,
    parsep=0pt,
}

\setlist[enumerate]{
    %itemindent=17pt, 
    %leftmargin=1pt,
    listparindent=\parindent,
    parsep=0pt,
}

% Math operators
\DeclareMathOperator{\timeorder}{\mathcal{T}}
\DeclareMathOperator{\diag}{diag}
\DeclareMathOperator{\legpoly}{P}
\DeclareMathOperator{\primevalue}{P}
\DeclareMathOperator{\sgn}{sgn}
\DeclareMathOperator{\res}{Res}
\newcommand*{\ii}{\mathrm{i}}
\newcommand*{\ee}{\mathrm{e}}
\newcommand*{\const}{\mathrm{const}}
\newcommand*{\suchthat}{\quad \text{s.t.} \quad}
\newcommand*{\argmin}{\arg\min}
\newcommand*{\argmax}{\arg\max}
\newcommand*{\normalorder}[1]{: #1 :}
\newcommand*{\pair}[1]{\langle #1 \rangle}
\newcommand*{\fd}[1]{\mathcal{D} #1}
\DeclareMathOperator{\bigO}{\mathcal{O}}

% TikZ setting
\usetikzlibrary{arrows,shapes,positioning}
\usetikzlibrary{arrows.meta}
\usetikzlibrary{decorations.markings}
\usetikzlibrary{calc}
\tikzstyle arrowstyle=[scale=1]
\tikzstyle directed=[postaction={decorate,decoration={markings,
    mark=at position .5 with {\arrow[arrowstyle]{stealth}}}}]
\tikzstyle ray=[directed, thick]
\tikzstyle dot=[anchor=base,fill,circle,inner sep=1pt]

% Algorithm setting
% Julia-style code
\SetKwIF{If}{ElseIf}{Else}{if}{}{elseif}{else}{end}
\SetKwFor{For}{for}{}{end}
\SetKwFor{While}{while}{}{end}
\SetKwProg{Function}{function}{}{end}
\SetArgSty{textnormal}

\newcommand*{\concept}[1]{{\textbf{#1}}}

% Embedded codes
\lstset{basicstyle=\ttfamily,
  showstringspaces=false,
  commentstyle=\color{gray},
  keywordstyle=\color{blue}
}

\lstdefinestyle{console}{
    basicstyle=\footnotesize\ttfamily,
    breaklines=true,
    postbreak=\mbox{\textcolor{red}{$\hookrightarrow$}\space}
}

% Reference formatting
\newrefformat{fig}{Figure~\ref{#1}}

% Color boxes
\tcbuselibrary{skins, breakable, theorems}
\newtcbtheorem[number within=section]{warning}{Warning}%
  {colback=orange!5,colframe=orange!65,fonttitle=\bfseries, breakable}{warn}
\newtcbtheorem[number within=section]{note}{Note}%
  {colback=green!5,colframe=green!65,fonttitle=\bfseries, breakable}{note}
\newtcbtheorem[number within=section]{info}{Info}%
  {colback=blue!5,colframe=blue!65,fonttitle=\bfseries, breakable}{info}

% Displaying texts in bookmarkers

\pdfstringdefDisableCommands{%
  \def\\{}%
  \def\ce#1{<#1>}%
}

\pdfstringdefDisableCommands{%
  \def\texttt#1{<#1>}%
  \def\mathbb#1{#1}%
}
\pdfstringdefDisableCommands{\def\eqref#1{(\ref{#1})}}

\makeatletter
\pdfstringdefDisableCommands{\let\HyPsd@CatcodeWarning\@gobble}
\makeatother

\newenvironment{shelldisplay}{\begin{lstlisting}}{\end{lstlisting}}

\newcommand{\shortcode}[1]{\texttt{#1}}
\newcommand{\muB}{\mu_{\text{B}}}

\lstset{style = console}

% Make subsubsection labeled
\setcounter{secnumdepth}{4}
\setcounter{tocdepth}{4}

\newcommand{\kB}{k_{\text{B}}}

\title{Theory of atom}
\author{Jinyuan Wu}

\begin{document}

\maketitle

\section{Hydrogen}

The energy unit Hartree is defined as twice of $E_1$.
That's to say, one Hartree is equal to \SI{27.6}{eV},
and if the Hartree unit is used, 
then the energy levels of the hydrogen atom are 
$1 / 2n^2$.

\subsection{Stability}

The hydrogen atom is bound together by Coulomb potential $1 / r$.
From Virial theorem, we will find not all attractive potentials lead to stable bound states. 
Specifically, $1 / r^2$ or $1 / r^3$ doesn't give us bound states in 3D.
This can be shown by explicitly calculating $T + V$: 
if it's greater than zero for the whole spectrum, 
then of course we don't have stable bound states. 

\subsection{Finite size effects}

\subsubsection{Finite nucleus mass}

In reality, the nucleus has a finite mass 
and therefore also moves together with the electron.
Fortunately this is a two-body problem 
and we can work in the center of mass frame,
and the EOM of $\vb*{r}_{\text{electron}} - \vb*{r}_{\text{nucleus}}$
is governed by the usual Newton's second law 
with the mass being
\begin{equation}
    \mu = \frac{m_1 m_2}{m_1 + m_2} \approx m_1 - \frac{m_1^2}{m_2}, \quad m_2 \to \infty.
\end{equation}
Thus 
\begin{equation}
    \frac{\var{E_n}}{E_n} = - \frac{m_1}{m_2} = - \frac{1}{1850}
\end{equation}
for hydrogen.
This relative error is the same for all energy levels; 
it immediately leads to the \concept{isotope shift},
in which adding a neutron to the nucleus 
changes the energy levels.
This is the most important correction.

\subsubsection{Going into the nucleus}

When $r$ is \emph{smaller} than the radius of the nucleus, 
it can be verified by Gauss's theorem that 
\begin{equation}
    V(r) = \frac{1}{2} \frac{r^2}{R_{\text{n}}^3} - \frac{3}{2 R_{\text{n}}}.
\end{equation}
To see why, just calculate the force using this potential 
and check the force obtained by 
\begin{equation}
    4 \pi r^2 \cdot F(r) = \int_{0}^{r} \frac{Z e}{\frac{4}{3} \pi R_{\text{n}}^3} \cdot 4 \pi r'^2 \dd{r'}.
\end{equation}
The constant term is there 
to guarantee continuity at $r = R_{\text{n}}$.
So 
\begin{equation}
    V(r) = \begin{cases}
        - \frac{1}{r}, &r > R_{\text{n}}, \\
        \frac{1}{2} \frac{r^2}{R_{\text{n}}^3} - \frac{3}{2 R_{\text{n}}}, & r < R_{\text{n}}.
    \end{cases}
\end{equation}
So, we find the existence of a finite-size nucleus means 
we have a perturbation Hamiltonian 
\begin{equation}
    V(r) - V_0(r) = \frac{1}{r} + \frac{1}{2} \frac{r^2}{R_{\text{n}}^3} - \frac{3}{2 R_{\text{n}}}.
\end{equation}
The first-order energy correction can therefore be determined.
The magnitude is $\num{1.6e-10} E_{\text{H}}$.
It's small, but is already observable using existing spectrography techniques.

\subsection{Relativistic corrections}

\subsubsection{Spin-orbital coupling}

The first order perturbation of the SOC Hamiltonian is 
\[
    E^{(1)} = \frac{e^2}{8 \pi \epsilon_0} \frac{1}{m^2 c} \expval{\frac{\vb*{L} \cdot \vb*{S}}{r^3}}.
\]
Note that $\vb*{L} \cdot \vb*{S}$ extracts information about $m$ and $m_s$
(which are good quantum numbers)
in the wave function:
we have 
\begin{equation}
    \vb*{L} \cdot \vb*{S} = \frac{\hbar^2}{2} (j (j + 1) - l (l + 1) - s (s + 1)).
\end{equation}
So the energy perturbation is just 
\[
    E^{(1)} = \frac{e^2}{8 \pi \epsilon_0} \frac{1}{m^2 c} \expval{\frac{1}{r^3}} \cdot 
    \frac{\hbar^2}{2} (j (j + 1) - l (l + 1) - s (s + 1)).
\]
Now we just add $E^{(1)}$ to $T + V$, 
and we find the influence of SOC can be seen as adding 
\begin{equation}
    H = \frac{e^2}{8 \pi \epsilon_0} \frac{1}{m^2 c} \expval{\frac{1}{r^3}} \vb*{L} \cdot \vb*{S},
\end{equation}
to the total Hamiltonian, where 
\begin{equation}
    \expval{\frac{1}{r^3}} = \frac{1}{l (l + 1/2) (l + 1) n^3}.
\end{equation}
Formally, this means we have averaged over $1 / r^3$ only;
but note that strictly speaking $\vb*{L}$ is no longer 
the old $\vb*{L}$ obtained from $V + T$, 
because after perturbation of SOC, 
the eigenstates themselves are changed, 
and so is $L_z = \sum m \dyad*{n, l, m, m_s}{n, l, m, m_s}$.

We can estimate the magnitude of SOC correction: 
we have 
\begin{equation}
    \frac{\var{E_n}}{E_n} = \frac{E_n}{m c^2} = \frac{10^{-5} Z^2}{n^2}.
\end{equation}

\subsubsection{Relativistic kinetic energy}

Another relativistic effect, apart from SOC, 
is the kinetic energy of an electron is actually 
\begin{equation}
    T = \sqrt{m^2 c^4 + p^2 c^2},
\end{equation}
and not just $p^2 / 2m$. 
So now we have a perturbation term in the kinetic energy.
The Taylor expansion gives 
\begin{equation}
    T = m c^2 + \frac{p^2}{2m} - \frac{p^4}{8 m^3 c^2} + \cdots,
\end{equation}
and the first relativistic correction is 
\begin{equation}
    H = - \frac{p^4}{8 m^3 c^2}.
\end{equation}
Its expectation can be found using the following trick:
\begin{equation}
    \begin{aligned}
        \expval*{H}{\psi} &\propto (\bra*{\psi} p^2) (p^2 \ket*{\psi}) \\
        &= \bra*{\psi} 2 (E - V(r)) \cdot 2 (E - V(r)) \ket*{\psi} ,
    \end{aligned}
\end{equation}
and this can be further simplified using Virial's theorem 
\begin{equation}
    2 \expval{T} = \expval{\vb*{r} \cdot \vb*{v} V},
\end{equation}
which, in the Coulomb case, means for \emph{all} eigenstates 
(and not just the ground state), we have 
\begin{equation}
    \expval{T} = - E_n, \quad \expval{V} = 2 E_n.
\end{equation}
The $\expval{V^2}$ term can be evaluated using Feynman-Hellmun theorem
\begin{equation}
    \pdv{E_n(\lambda)}{\lambda} = \expval{\pdv{H(\lambda)}{\lambda}}{\psi(\lambda)}. 
\end{equation}
Recall that there is a 
\[
    \frac{l (l + 1)}{2 r^2}
\]
term in the Hamiltonian, 
and we find 
\begin{equation}
    \pdv{l} \left( - \frac{1}{(n_r + l)^2} \right)^2 = \expval{\frac{2l + 1}{r^2}},
\end{equation} 
and $\expval{1 / r^2}$ can then be found by taking the derivative of $E_{n} = E_{n_r + l}$.
The final expression is 
\begin{equation}
    E^{(1)}_{\text{rel}} = \frac{1}{2 m c^2} \frac{(E_n^{(0)})^2}{mc^2} \left(
        \frac{4n}{l+1/2} - 3
    \right).
\end{equation}
This term is extremely small. 
It can be easily seen that when $l = 0$, 
the term takes its maximum,
but even at the maximum its magnitude is still only 
$\sim \SI{1}{MHz}$.

Thus the main relativistic correction to atomic energy levels is SOC. 
The energy level splitting caused by SOC 
is called the \concept{fine structure}.
But not that the relative magnitude of it compared 
with Coulomb interaction between electrons 
is not known:
it's possible that the single-body SOC is more important,
but the opposite case -- that Coulomb interaction between electrons is stronger 
-- is also possible.

\subsubsection{The Darwin term}



\subsection{Electron-nucleus interaction beyond Coulomb potential}

The interaction between the electron and the nucleus 
is of course not restricted to Coulomb interaction.
The nucleus can have a spin, 
and then magnetic dipole interaction becomes a perturbation.
We can link spin and magnetic moment by 
\begin{equation}
    \vb*{\mu} = \frac{g}{2m} \vb*{S},
\end{equation}
and we also have 
\begin{equation}
    \vb*{B} = \frac{\mu_0}{4 \pi r^3} (3 (\vb*{\mu} \cdot \vu*{r}) \vu*{r} - \vb*{\mu})
    + \frac{2}{3} \mu_0 \vb*{\mu} \delta^3(\vb*{r}),
\end{equation}
and therefore the Hamiltonian is 
\begin{equation}
    H = \frac{\mu_0 g_{\text{n}} e^2}{8 \pi m_{\text{n}} m_e} 
    \left(
        \frac{3 (\vb*{I} \cdot \vu*{r}) (\vb*{S} \cdot \vu*{r})}{r^3}
        - \vb*{S} \cdot \vb*{I}
    \right)
    + \frac{\mu_0 g_{\text{n}} e^2}{3 \pi m_{\text{n}} m_e} \vb*{I} \cdot \vb*{S}.
\end{equation}

When $l = 0$, the first-order perturbation caused by this Hamiltonian 
is proportion to $\expval{\vb*{S} \cdot \vb*{I}}$,
because the $(\vb*{I} \cdot \vu*{r}) (\vb*{S} \cdot \vu*{r})$
vanishes after $\int \dd{\Omega}$.
We define 
\begin{equation}
    \vb*{F} = \vb*{I} + \vb*{S},
\end{equation}
so that 
\begin{equation}
    \vb*{I} \cdot \vb*{S} = F(F+1) - I(I+1) - S(S+1),
\end{equation}
and we can use $F$ to label 

The energy level splitting caused by this effect 
is \concept{hyperfine structure}.
We have 
\begin{equation}
    \frac{E_{\text{HFS}}}{E_n} \sim \num{1e-7},
\end{equation}
which is still larger than the finite size nucleus effect.

\subsection{Fluctuation of electromagnetic field}

\concept{Lamb shift} can be estimated using the following approach:
we can obtain a ``vacuum electric field strength''
from the zero-point energy, 
and then apply this electric field to the atom.

After finding $\var{\vb*{r}}$, 
the perturbation of the atomic energy can be estimated by 
Taylor expansion of the Coulomb potential.
We have
\begin{equation}
    V(\vb*{r} + \var{\vb*{r}}) = V(\vb*{r}) + 
    \var{\vb*{r}} \cdot \grad{V}
    + \frac{1}{2} (\var{\vb*{r}} \cdot \grad)^2 V + \cdots,
\end{equation}
and we can find its average and get 
\begin{equation}
    \Delta V = \alpha^5 m c^2 \frac{1}{6 \pi} \ln \frac{2}{\pi a}. 
\end{equation}
This correction goes beyond all single-atom corrections discussed before; 
historically it was a strong support of quantum electrodynamics
because it revealed that in a space without photons, 
we still can't completely ignore the presence of the electromagnetic field.

This sometimes is called a proof of the existence of the zero-point energy,
although what it really proves is the quantum nature of the electromagnetic field.

\section{Energy level splitting in external fields}

\begin{equation}
    E_n = \frac{-\SI{13.6}{eV}}{n^2} + \muB (2m_s + m_l).
\end{equation}

\end{document}