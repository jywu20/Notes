\documentclass[hyperref, a4paper]{article}

\usepackage{geometry}
\usepackage{titling}
\usepackage{titlesec}
% No longer needed, since we will use enumitem package
% \usepackage{paralist}
\usepackage{enumitem}
\usepackage{footnote}
\usepackage{amsmath, amssymb, amsthm}
\usepackage{mathtools}
\usepackage{bbm}
\usepackage{cite}
\usepackage{graphicx}
\usepackage{subfigure}
\usepackage{physics}
\usepackage{tensor}
\usepackage{siunitx}
\usepackage[version=4]{mhchem}
\usepackage{tikz}
\usepackage{xcolor}
\usepackage{listings}
\usepackage{underscore}
\usepackage{autobreak}
\usepackage[ruled, vlined, linesnumbered]{algorithm2e}
\usepackage{nameref,zref-xr}
\zxrsetup{toltxlabel}
\usepackage[colorlinks,unicode]{hyperref} % , linkcolor=black, anchorcolor=black, citecolor=black, urlcolor=black, filecolor=black
\usepackage[most]{tcolorbox}
\usepackage{prettyref}

% Page style
\geometry{left=3.18cm,right=3.18cm,top=2.54cm,bottom=2.54cm}
\titlespacing{\paragraph}{0pt}{1pt}{10pt}[20pt]
\setlength{\droptitle}{-5em}

% More compact lists 
\setlist[itemize]{
    %itemindent=17pt, 
    %leftmargin=1pt,
    listparindent=\parindent,
    parsep=0pt,
}

\setlist[enumerate]{
    %itemindent=17pt, 
    %leftmargin=1pt,
    listparindent=\parindent,
    parsep=0pt,
}

% Math operators
\DeclareMathOperator{\timeorder}{\mathcal{T}}
\DeclareMathOperator{\diag}{diag}
\DeclareMathOperator{\legpoly}{P}
\DeclareMathOperator{\primevalue}{P}
\DeclareMathOperator{\sgn}{sgn}
\DeclareMathOperator{\res}{Res}
\newcommand*{\ii}{\mathrm{i}}
\newcommand*{\ee}{\mathrm{e}}
\newcommand*{\const}{\mathrm{const}}
\newcommand*{\suchthat}{\quad \text{s.t.} \quad}
\newcommand*{\argmin}{\arg\min}
\newcommand*{\argmax}{\arg\max}
\newcommand*{\normalorder}[1]{: #1 :}
\newcommand*{\pair}[1]{\langle #1 \rangle}
\newcommand*{\fd}[1]{\mathcal{D} #1}
\DeclareMathOperator{\bigO}{\mathcal{O}}

% TikZ setting
\usetikzlibrary{arrows,shapes,positioning}
\usetikzlibrary{arrows.meta}
\usetikzlibrary{decorations.markings}
\usetikzlibrary{calc}
\tikzstyle arrowstyle=[scale=1]
\tikzstyle directed=[postaction={decorate,decoration={markings,
    mark=at position .5 with {\arrow[arrowstyle]{stealth}}}}]
\tikzstyle ray=[directed, thick]
\tikzstyle dot=[anchor=base,fill,circle,inner sep=1pt]

% Algorithm setting
% Julia-style code
\SetKwIF{If}{ElseIf}{Else}{if}{}{elseif}{else}{end}
\SetKwFor{For}{for}{}{end}
\SetKwFor{While}{while}{}{end}
\SetKwProg{Function}{function}{}{end}
\SetArgSty{textnormal}

\newcommand*{\concept}[1]{{\textbf{#1}}}

% Embedded codes
\lstset{basicstyle=\ttfamily,
  showstringspaces=false,
  commentstyle=\color{gray},
  keywordstyle=\color{blue}
}

\lstdefinestyle{console}{
    basicstyle=\footnotesize\ttfamily,
    breaklines=true,
    postbreak=\mbox{\textcolor{red}{$\hookrightarrow$}\space}
}

% Reference formatting
\newrefformat{fig}{Figure~\ref{#1}}

% Color boxes
\tcbuselibrary{skins, breakable, theorems}
\newtcbtheorem[number within=section]{warning}{Warning}%
  {colback=orange!5,colframe=orange!65,fonttitle=\bfseries, breakable}{warn}
\newtcbtheorem[number within=section]{note}{Note}%
  {colback=green!5,colframe=green!65,fonttitle=\bfseries, breakable}{note}
\newtcbtheorem[number within=section]{info}{Info}%
  {colback=blue!5,colframe=blue!65,fonttitle=\bfseries, breakable}{info}

% Displaying texts in bookmarkers

\pdfstringdefDisableCommands{%
  \def\\{}%
  \def\ce#1{<#1>}%
}

\pdfstringdefDisableCommands{%
  \def\texttt#1{<#1>}%
  \def\mathbb#1{#1}%
}
\pdfstringdefDisableCommands{\def\eqref#1{(\ref{#1})}}

\makeatletter
\pdfstringdefDisableCommands{\let\HyPsd@CatcodeWarning\@gobble}
\makeatother

\newenvironment{shelldisplay}{\begin{lstlisting}}{\end{lstlisting}}

\newcommand{\shortcode}[1]{\texttt{#1}}
\newcommand{\muB}{\mu_{\text{B}}}

\lstset{style = console}

% Make subsubsection labeled
\setcounter{secnumdepth}{4}
\setcounter{tocdepth}{4}

\newcommand{\kB}{k_{\text{B}}}

\title{Theory of atom}
\author{Jinyuan Wu}

\begin{document}

\maketitle

\section{Hydrogen}

The energy unit Hartree is defined as twice of $E_1$.
That's to say, one Hartree is equal to \SI{27.6}{eV},
and if the Hartree unit is used, 
then the energy levels of the hydrogen atom are 
$1 / 2n^2$.

\subsection{Stability}

The hydrogen atom is bound together by Coulomb potential $1 / r$.
From Virial theorem, we will find not all attractive potentials lead to stable bound states. 
Specifically, $1 / r^2$ or $1 / r^3$ doesn't give us bound states in 3D.
This can be shown by explicitly calculating $T + V$: 
if it's greater than zero for the whole spectrum, 
then of course we don't have stable bound states. 

\subsection{Finite size effects}

\subsubsection{Finite nucleus mass}

In reality, the nucleus has a finite mass 
and therefore also moves together with the electron.
Fortunately this is a two-body problem 
and we can work in the center of mass frame,
and the EOM of $\vb*{r}_{\text{electron}} - \vb*{r}_{\text{nucleus}}$
is governed by the usual Newton's second law 
with the mass being
\begin{equation}
    \mu = \frac{m_1 m_2}{m_1 + m_2} \approx m_1 - \frac{m_1^2}{m_2}, \quad m_2 \to \infty.
\end{equation}
Thus 
\begin{equation}
    \frac{\var{E_n}}{E_n} = - \frac{m_1}{m_2} = - \frac{1}{1850}
\end{equation}
for hydrogen.
This relative error is the same for all energy levels; 
it immediately leads to the \concept{isotope shift},
in which adding a neutron to the nucleus 
changes the energy levels.
This is the most important correction.

\subsubsection{Going into the nucleus}

When $r$ is \emph{smaller} than the radius of the nucleus, 
it can be verified by Gauss's theorem that 
\begin{equation}
    V(r) = \frac{1}{2} \frac{r^2}{R_{\text{n}}^3} - \frac{3}{2 R_{\text{n}}}.
\end{equation}
To see why, just calculate the force using this potential 
and check the force obtained by 
\begin{equation}
    4 \pi r^2 \cdot F(r) = \int_{0}^{r} \frac{Z e}{\frac{4}{3} \pi R_{\text{n}}^3} \cdot 4 \pi r'^2 \dd{r'}.
\end{equation}
The constant term is there 
to guarantee continuity at $r = R_{\text{n}}$.
So 
\begin{equation}
    V(r) = \begin{cases}
        - \frac{1}{r}, &r > R_{\text{n}}, \\
        \frac{1}{2} \frac{r^2}{R_{\text{n}}^3} - \frac{3}{2 R_{\text{n}}}, & r < R_{\text{n}}.
    \end{cases}
\end{equation}
So, we find the existence of a finite-size nucleus means 
we have a perturbation Hamiltonian 
\begin{equation}
    V(r) - V_0(r) = \frac{1}{r} + \frac{1}{2} \frac{r^2}{R_{\text{n}}^3} - \frac{3}{2 R_{\text{n}}}.
\end{equation}
The first-order energy correction can therefore be determined.
The magnitude is $\num{1.6e-10} E_{\text{H}}$.
It's small, but is already observable using existing spectrography techniques.

\subsection{Relativistic corrections}

\subsubsection{Spin-orbital coupling}

The first order perturbation of the SOC Hamiltonian is 
\[
    E^{(1)} = \frac{e^2}{8 \pi \epsilon_0} \frac{1}{m^2 c} \expval{\frac{\vb*{L} \cdot \vb*{S}}{r^3}}.
\]
Note that $\vb*{L} \cdot \vb*{S}$ extracts information about $m$ and $m_s$
(which are good quantum numbers)
in the wave function:
we have 
\begin{equation}
    \vb*{L} \cdot \vb*{S} = \frac{\hbar^2}{2} (j (j + 1) - l (l + 1) - s (s + 1)).
\end{equation}
So the energy perturbation is just 
\[
    E^{(1)} = \frac{e^2}{8 \pi \epsilon_0} \frac{1}{m^2 c} \expval{\frac{1}{r^3}} \cdot 
    \frac{\hbar^2}{2} (j (j + 1) - l (l + 1) - s (s + 1)).
\]
Now we just add $E^{(1)}$ to $T + V$, 
and we find the influence of SOC can be seen as adding 
\begin{equation}
    H = \frac{e^2}{8 \pi \epsilon_0} \frac{1}{m^2 c} \expval{\frac{1}{r^3}} \vb*{L} \cdot \vb*{S},
\end{equation}
to the total Hamiltonian, where 
\begin{equation}
    \expval{\frac{1}{r^3}} = \frac{1}{l (l + 1/2) (l + 1) n^3}.
\end{equation}
Formally, this means we have averaged over $1 / r^3$ only;
but note that strictly speaking $\vb*{L}$ is no longer 
the old $\vb*{L}$ obtained from $V + T$, 
because after perturbation of SOC, 
the eigenstates themselves are changed, 
and so is $L_z = \sum m \dyad*{n, l, m, m_s}{n, l, m, m_s}$.

We can estimate the magnitude of SOC correction: 
we have 
\begin{equation}
    \frac{\var{E_n}}{E_n} = \frac{E_n}{m c^2} = \frac{10^{-5} Z^2}{n^2}.
\end{equation}

\subsubsection{Relativistic kinetic energy}

Another relativistic effect, apart from SOC, 
is the kinetic energy of an electron is actually 
\begin{equation}
    T = \sqrt{m^2 c^4 + p^2 c^2},
\end{equation}
and not just $p^2 / 2m$. 
So now we have a perturbation term in the kinetic energy.
The Taylor expansion gives 
\begin{equation}
    T = m c^2 + \frac{p^2}{2m} - \frac{p^4}{8 m^3 c^2} + \cdots,
\end{equation}
and the first relativistic correction is 
\begin{equation}
    H = - \frac{p^4}{8 m^3 c^2}.
\end{equation}
Its expectation can be found using the following trick:
\begin{equation}
    \begin{aligned}
        \expval*{H}{\psi} &\propto (\bra*{\psi} p^2) (p^2 \ket*{\psi}) \\
        &= \bra*{\psi} 2 (E - V(r)) \cdot 2 (E - V(r)) \ket*{\psi} ,
    \end{aligned}
\end{equation}
and this can be further simplified using Virial's theorem 
\begin{equation}
    2 \expval{T} = \expval{\vb*{r} \cdot \vb*{v} V},
\end{equation}
which, in the Coulomb case, means for \emph{all} eigenstates 
(and not just the ground state), we have 
\begin{equation}
    \expval{T} = - E_n, \quad \expval{V} = 2 E_n.
\end{equation}
The $\expval{V^2}$ term can be evaluated using Feynman-Hellmun theorem
\begin{equation}
    \pdv{E_n(\lambda)}{\lambda} = \expval{\pdv{H(\lambda)}{\lambda}}{\psi(\lambda)}. 
\end{equation}
Recall that there is a 
\[
    \frac{l (l + 1)}{2 r^2}
\]
term in the Hamiltonian, 
and we find 
\begin{equation}
    \pdv{l} \left( - \frac{1}{(n_r + l)^2} \right)^2 = \expval{\frac{2l + 1}{r^2}},
\end{equation} 
and $\expval{1 / r^2}$ can then be found by taking the derivative of $E_{n} = E_{n_r + l}$.
The final expression is 
\begin{equation}
    E^{(1)}_{\text{rel}} = \frac{1}{2 m c^2} \frac{(E_n^{(0)})^2}{mc^2} \left(
        \frac{4n}{l+1/2} - 3
    \right).
\end{equation}
This term is extremely small. 
It can be easily seen that when $l = 0$, 
the term takes its maximum,
but even at the maximum its magnitude is still only 
$\sim \SI{1}{MHz}$.

Thus the main relativistic correction to atomic energy levels is SOC. 
The energy level splitting caused by SOC 
is called the \concept{fine structure}.
But not that the relative magnitude of it compared 
with Coulomb interaction between electrons 
is not known:
it's possible that the single-body SOC is more important,
but the opposite case -- that Coulomb interaction between electrons is stronger 
-- is also possible.

\subsubsection{The Darwin term}



\subsection{Electron-nucleus interaction beyond Coulomb potential}

The interaction between the electron and the nucleus 
is of course not restricted to Coulomb interaction.
The nucleus can have a spin, 
and then magnetic dipole interaction becomes a perturbation.
We can link spin and magnetic moment by 
\begin{equation}
    \vb*{\mu} = \frac{g}{2m} \vb*{S},
\end{equation}
and we also have 
\begin{equation}
    \vb*{B} = \frac{\mu_0}{4 \pi r^3} (3 (\vb*{\mu} \cdot \vu*{r}) \vu*{r} - \vb*{\mu})
    + \frac{2}{3} \mu_0 \vb*{\mu} \delta^3(\vb*{r}),
\end{equation}
and therefore the Hamiltonian is 
\begin{equation}
    H = \frac{\mu_0 g_{\text{n}} e^2}{8 \pi m_{\text{n}} m_e} 
    \left(
        \frac{3 (\vb*{I} \cdot \vu*{r}) (\vb*{S} \cdot \vu*{r})}{r^3}
        - \vb*{S} \cdot \vb*{I}
    \right)
    + \frac{\mu_0 g_{\text{n}} e^2}{3 \pi m_{\text{n}} m_e} \vb*{I} \cdot \vb*{S}.
\end{equation}

When $l = 0$, the first-order perturbation caused by this Hamiltonian 
is proportion to $\expval{\vb*{S} \cdot \vb*{I}}$,
because the $(\vb*{I} \cdot \vu*{r}) (\vb*{S} \cdot \vu*{r})$
vanishes after $\int \dd{\Omega}$.
We define 
\begin{equation}
    \vb*{F} = \vb*{I} + \vb*{S},
\end{equation}
so that 
\begin{equation}
    2 \vb*{I} \cdot \vb*{S} = F(F+1) - I(I+1) - S(S+1),
\end{equation}
and we can use $F$ to label 

The energy level splitting caused by this effect 
is \concept{hyperfine structure}.
We have 
\begin{equation}
    \frac{E_{\text{HFS}}}{E_n} \sim \num{1e-7},
\end{equation}
which is still larger than the finite size nucleus effect.

\subsection{Fluctuation of electromagnetic field}

\concept{Lamb shift} can be estimated using the following approach:
we can obtain a ``vacuum electric field strength''
from the zero-point energy, 
and then apply this electric field to the atom.

After finding $\var{\vb*{r}}$, 
the perturbation of the atomic energy can be estimated by 
Taylor expansion of the Coulomb potential.
We have
\begin{equation}
    V(\vb*{r} + \var{\vb*{r}}) = V(\vb*{r}) + 
    \var{\vb*{r}} \cdot \grad{V}
    + \frac{1}{2} (\var{\vb*{r}} \cdot \grad)^2 V + \cdots,
\end{equation}
and we can find its average and get 
\begin{equation}
    \Delta V = \alpha^5 m c^2 \frac{1}{6 \pi} \ln \frac{2}{\pi a}. 
\end{equation}
This correction goes beyond all single-atom corrections discussed before; 
historically it was a strong support of quantum electrodynamics
because it revealed that in a space without photons, 
we still can't completely ignore the presence of the electromagnetic field.

This sometimes is called a proof of the existence of the zero-point energy,
although what it really proves is the quantum nature of the electromagnetic field.

\section{$CG$ coefficients and energy level splitting in external fields}

We may want to know the relation between $\ket*{l, m_l} \otimes \ket*{s, m_s}$
and $\ket*{j, m_j}$.
Let's start with an example. 
We have 
\begin{equation}
    J_- \ket{j = \frac{3}{2}, m_j = \frac{3}{2}} = \sqrt{
        j(j+1) - m_j (m_j - 1)
    } \ket{j, m_j - 1} = 
    \sqrt{3} \ket{j = \frac{3}{2}, m_j = \frac{1}{2}}.
\end{equation}
Since the only possibility to have $j = 3/2$, $m_j = 3/2$ 
is that $l = 1, m_l = 1$ and $s = 1/2, m_s = 1/2$, 
we have 
\begin{equation}
    \ket{j = \frac{3}{2}, m_j = \frac{3}{2}}
    = \ket{l = 1, m_l = 1} \otimes \ket{s = \frac{1}{2}, m_s = \frac{1}{2}},
\end{equation}
and we also know that 
\begin{equation}
    J_- = S_- + L_-,
\end{equation}
and we get 
\begin{equation}
    \begin{aligned}
        \sqrt{3} \ket{j = \frac{3}{2}, m_j = \frac{1}{2}} &= 
        (L_- + S_-) \ket{l = 1, m_l = 1} \otimes \ket{s = \frac{1}{2}, m_s = \frac{1}{2}} \\
        &= \sqrt{1(1+1) - 1(1-1)} \ket{l = 1, m_l = 0} \otimes \ket{s = \frac{1}{2}, m_s = \frac{1}{2}} \\
        &\quad + \ket{l = 1, m_l = 1} \otimes \sqrt{\frac{1}{2} (\frac{1}{2} + 1) - \frac{1}{2} (\frac{1}{2} - 1)} 
        \ket{s = \frac{1}{2}, m_s = - \frac{1}{2}},
    \end{aligned}
\end{equation}
and we find 
\begin{equation}
    \begin{aligned}
        \ket{j = \frac{3}{2}, m_j = \frac{1}{2}}
        &= \frac{1}{\sqrt{3}} \Big(
        \sqrt{2} \ket{l = 1, m_l = 0} \otimes \ket{s = \frac{1}{2}, m_s = \frac{1}{2}} \\
        &\quad + \ket{l = 1, m_l = 1} \otimes \ket{s = \frac{1}{2}, m_s = - \frac{1}{2}}
    \Big).
    \end{aligned}
\end{equation}
This routine can be repeated for other states. 
To save space, below we use $\ket{\frac{3}{2}, \ket{\frac{3}{2}}}$ to refer to $\ket{j, m_l}$,
and use $\ket{1, 1} \otimes \ket{\frac{1}{2}, - \frac{1}{2}}$ to refer to $\ket{l, m_l} \otimes \ket{s, m_s}$.

Now we are able to evaluate the energy correction caused by external fields
from first principles.
For example, we have
\begin{equation}
    \mel{\frac{3}{2}, \frac{1}{2}}{L_z + 2S_z}{\frac{1}{2}, \frac{1}{2}}
    = 
\end{equation}
The fact that the non-diagonal matrix elements don't vanish 
means the external fields mix different $\ket{j, m_j}$ states.
This then means that generally, 
the energy change caused by external magnetic fields is not linear.
When the external magnetic field is weak, 
only first order perturbation is important 
and we can use the effective Lande $g$ factor approach,
and $\Delta E \propto B$;
but when the external magnetic field is stronger, 
this no longer works. 

\section{Atomic interaction}

\subsection{Model: the interaction between two oscillators}

\begin{figure}
    \centering
    \begin{tikzpicture}[x=0.75pt,y=0.75pt,yscale=-1,xscale=1]
    %uncomment if require: \path (0,300); %set diagram left start at 0, and has height of 300
    
    %Straight Lines [id:da9381488273380147] 
    \draw    (100,115) .. controls (101.67,113.33) and (103.33,113.33) .. (105,115) .. controls (106.67,116.67) and (108.33,116.67) .. (110,115) .. controls (111.67,113.33) and (113.33,113.33) .. (115,115) .. controls (116.67,116.67) and (118.33,116.67) .. (120,115) .. controls (121.67,113.33) and (123.33,113.33) .. (125,115) .. controls (126.67,116.67) and (128.33,116.67) .. (130,115) -- (131.33,115) -- (131.33,115) ;
    \draw [shift={(131.33,115)}, rotate = 0] [color={rgb, 255:red, 0; green, 0; blue, 0 }  ][fill={rgb, 255:red, 0; green, 0; blue, 0 }  ][line width=0.75]      (0, 0) circle [x radius= 3.35, y radius= 3.35]   ;
    \draw [shift={(100,115)}, rotate = 0] [color={rgb, 255:red, 0; green, 0; blue, 0 }  ][fill={rgb, 255:red, 0; green, 0; blue, 0 }  ][line width=0.75]      (0, 0) circle [x radius= 3.35, y radius= 3.35]   ;
    %Straight Lines [id:da07242510029961391] 
    \draw    (204.33,115) .. controls (206,113.33) and (207.66,113.33) .. (209.33,115) .. controls (211,116.67) and (212.66,116.67) .. (214.33,115) .. controls (216,113.33) and (217.66,113.33) .. (219.33,115) .. controls (221,116.67) and (222.66,116.67) .. (224.33,115) .. controls (226,113.33) and (227.66,113.33) .. (229.33,115) .. controls (231,116.67) and (232.66,116.67) .. (234.33,115) -- (235.67,115) -- (235.67,115) ;
    \draw [shift={(235.67,115)}, rotate = 0] [color={rgb, 255:red, 0; green, 0; blue, 0 }  ][fill={rgb, 255:red, 0; green, 0; blue, 0 }  ][line width=0.75]      (0, 0) circle [x radius= 3.35, y radius= 3.35]   ;
    \draw [shift={(204.33,115)}, rotate = 0] [color={rgb, 255:red, 0; green, 0; blue, 0 }  ][fill={rgb, 255:red, 0; green, 0; blue, 0 }  ][line width=0.75]      (0, 0) circle [x radius= 3.35, y radius= 3.35]   ;
    %Straight Lines [id:da38639705579891603] 
    \draw [color={rgb, 255:red, 155; green, 155; blue, 155 }  ,draw opacity=1 ]   (100,81.4) -- (100,115) ;
    %Straight Lines [id:da47837788925550906] 
    \draw [color={rgb, 255:red, 155; green, 155; blue, 155 }  ,draw opacity=1 ]   (204.33,81.4) -- (204.33,115) ;
    %Straight Lines [id:da8463192173973635] 
    \draw    (102,96.4) -- (202.33,96.4) ;
    \draw [shift={(204.33,96.4)}, rotate = 180] [fill={rgb, 255:red, 0; green, 0; blue, 0 }  ][line width=0.08]  [draw opacity=0] (12,-3) -- (0,0) -- (12,3) -- cycle    ;
    \draw [shift={(100,96.4)}, rotate = 0] [fill={rgb, 255:red, 0; green, 0; blue, 0 }  ][line width=0.08]  [draw opacity=0] (12,-3) -- (0,0) -- (12,3) -- cycle    ;
    %Straight Lines [id:da7143303294528669] 
    \draw [color={rgb, 255:red, 208; green, 2; blue, 27 }  ,draw opacity=1 ]   (204.33,115) ;
    \draw [shift={(204.33,115)}, rotate = 0] [color={rgb, 255:red, 208; green, 2; blue, 27 }  ,draw opacity=1 ][fill={rgb, 255:red, 208; green, 2; blue, 27 }  ,fill opacity=1 ][line width=0.75]      (0, 0) circle [x radius= 3.35, y radius= 3.35]   ;
    %Straight Lines [id:da04628233779071622] 
    \draw [color={rgb, 255:red, 208; green, 2; blue, 27 }  ,draw opacity=1 ]   (100,115) ;
    \draw [shift={(100,115)}, rotate = 0] [color={rgb, 255:red, 208; green, 2; blue, 27 }  ,draw opacity=1 ][fill={rgb, 255:red, 208; green, 2; blue, 27 }  ,fill opacity=1 ][line width=0.75]      (0, 0) circle [x radius= 3.35, y radius= 3.35]   ;
    %Straight Lines [id:da9621227173324463] 
    \draw [color={rgb, 255:red, 74; green, 144; blue, 226 }  ,draw opacity=1 ]   (131.33,115) ;
    \draw [shift={(131.33,115)}, rotate = 0] [color={rgb, 255:red, 74; green, 144; blue, 226 }  ,draw opacity=1 ][fill={rgb, 255:red, 74; green, 144; blue, 226 }  ,fill opacity=1 ][line width=0.75]      (0, 0) circle [x radius= 3.35, y radius= 3.35]   ;
    %Straight Lines [id:da31457274147436154] 
    \draw [color={rgb, 255:red, 74; green, 144; blue, 226 }  ,draw opacity=1 ]   (235.67,115) ;
    \draw [shift={(235.67,115)}, rotate = 0] [color={rgb, 255:red, 74; green, 144; blue, 226 }  ,draw opacity=1 ][fill={rgb, 255:red, 74; green, 144; blue, 226 }  ,fill opacity=1 ][line width=0.75]      (0, 0) circle [x radius= 3.35, y radius= 3.35]   ;
    
    % Text Node
    \draw (108,129.9) node [anchor=north west][inner sep=0.75pt]    {$x_{1}$};
    % Text Node
    \draw (210,129.9) node [anchor=north west][inner sep=0.75pt]    {$x_{2}$};
    % Text Node
    \draw (152.17,93) node [anchor=south] [inner sep=0.75pt]    {$R$};
    
    
    \end{tikzpicture}
    
    \caption{Two dipoles}
    \label{fig:two-dipole-1}
\end{figure}

Consider two harmonic oscillators, 
each of which is a electric dipole (\prettyref{fig:two-dipole-1}).
The free Hamiltonian is
\begin{equation}
    H_0 = \frac{p_1^2}{2m} + \frac{1}{2} k x_1^2 + 
    \frac{p_2^2}{2m} + \frac{1}{2} k x_2^2,
    \label{eq:unperturbed-oscillators}
\end{equation}
and the state of the system is represented by 
\begin{equation}
    \braket{\vb*{r}_1, \vb*{r}_2}{\psi} = \psi_1(\vb*{r}_1) \psi_2(\vb*{r}_2).
\end{equation}
Now we introduce the Coulomb interaction Hamiltonian 
\begin{equation}
    H_1 = \frac{1}{4 \pi \epsilon_0} \left(
        \frac{e^2}{R} - \frac{e^2}{R - x_1} - \frac{e^2}{R + x_2} + \frac{e^2}{R - x_1 + x_2}
    \right).
\end{equation}
In the $x_1, x_2 \ll R$ limit, 
we have 
\begin{equation}
    H_1 = \frac{e^2}{4 \pi \epsilon_0 R} \left(
        - \left(\frac{x_1}{R}\right)^2
        - \left(\frac{x_2}{R}\right)^2 
        + \left(\frac{x_1 - x_2}{R}\right)^2
    \right)
    = - \frac{e^2}{4 \pi \epsilon_0 R^3} \cdot 2 x_1 x_2 .
    \label{eq:x1x2-interaction}
\end{equation}
Since the cross term appears, 
we define 
\begin{equation}
    x_\pm = \frac{x_1 \pm x_2}{\sqrt{2}}, \quad 
    p_\pm = \frac{p_1 \pm p_2}{\sqrt{2}},
    \label{eq:x-p-pm}
\end{equation}
and the total Hamiltonian becomes 
\begin{equation}
    H = \frac{p_+^2}{2m} + \frac{p_-^2}{2m} + \frac{1}{2} k x_+^2 + \frac{1}{2} k x_-^2 
    - \frac{e^2}{4 \pi \epsilon_0} \frac{x_+^2}{R^3} 
    + \frac{e^2}{4 \pi \epsilon_0} \frac{x_-^2}{R^3},
\end{equation}
so we find we have two corrected modes,
the frequencies of which are
\begin{equation}
    \omega_\pm = \sqrt{
        \frac{k \mp e^2 / 2 \pi \epsilon_0 R^3}{m}
    }.
\end{equation}

Now as a demonstration, suppose we don't find the trick \eqref{eq:x-p-pm}, 
and want to solve the Hamiltonian \eqref{eq:x1x2-interaction}
by perturbation theory.
Since $x_1, x_2$ creation and annihilation operators,
the first order perturbation is bound to be zero, 
because the eigenstates of \eqref{eq:unperturbed-oscillators}
are labeled by $n_1$ and $n_2$, 
and $\mel{n}{a+a^\dagger}{n} = 0$.
The second order perturbation of ground state energy is 
\begin{equation}
    \Delta E_0^{(2)} = \sum_{n_1, n_2 \neq 0} 
    \left(\frac{e^2}{4 \pi \epsilon_0 R^3}\right)^2 \frac{
        \abs*{\mel{n_1, n_2}{x_1 x_2}{0, 0}}^2
    }{
        E_0 - (E_{n_1} + E_{n_2})
    } = - \frac{1}{4 m^2 \omega_0^3} \frac{1}{2} \left(\frac{e^2}{2 \pi \epsilon_0}\right)^2 \frac{1}{R^6}.
    \label{eq:van-der-waals}
\end{equation}
This is a crude demonstration of the idea of 
the $1/R^6$ van der Waals force.

Now suppose we put three harmonic oscillators into the system;
The transition matrix is 
\begin{equation}
    \mel{1, 1, 1}{x_1 x_2 + x_3 x_1 + x_2 x_3}{0, 0, 0},
\end{equation}
which is three times as large as its counterpart 
with just two oscillators;
putting this into \eqref{eq:van-der-waals},
we find the numerator is 9 times as large as 
its two-body counterpart 
and the denominator is 3 times as large as the 
two-body counterpart,
and therefore we find the non-additivity of van der Waals force.

\section{Casimir force}

\begin{equation}
    E(a) = \sum_{n \geq 0} n = \frac{\hbar}{2} \frac{1}{2\pi} \int_{0}^{\infty} \log g(\xi) \dd{\xi}
\end{equation}

\begin{equation}
    \frac{1}{2\pi \ii} \oint z \frac{f(z)}{f'(z)} \dd{z} = \sum
\end{equation}

TODO: 
\begin{equation}
    F(d) = \frac{23 \hbar c}{40 d^4} \left(\frac{3}{4\pi}\right)^2
    \left(\frac{\varepsilon_{\text{r}} - 1}{\varepsilon_{\text{r}} + 2}\right)^2.
\end{equation}

\section{Atomic structure}

For multiple-electron atoms, 
repulsion between electrons significantly increases 
the energy at every energy level.
The ground state is \SI{-24.5}{eV},
while the value predicted by the 
hydrogen model is \SI{-54.4}{eV}.

The energy levels of multiple-electron atoms 
are usually shown with Grotrian diagrams.
Since 

\section{Spherical tensor operators}

The Pauli matrices are actually specific instances of 
spherical tensor operators.
We can expand Hamiltonian and/or density matrix 
into a linear combination of spherical tensor operators.

\section{Thomas-Fermi approximation}

\begin{equation}
    2 \cdot \frac{4 \pi}{3} p_{\text{F}}^3 \cdot \frac{V}{(2\pi)^3} = N \Rightarrow 
    \frac{1}{2} p_{\text{F}}^2 = \frac{1}{2} (3 \pi^2 n)^{3/2}.
\end{equation}
\begin{equation}
    \frac{1}{2} p^2 - \phi = E.
\end{equation}
We define 
\begin{equation}
    \frac{1}{2} p_0^2 = \phi - \phi_0,
\end{equation}
so that $\phi_0$ is the potential at the infinity 
and when $\phi = \phi_0$, $n = 0$,
which means we have indeed gone to the infinity.
Then we can redefine $\phi - \phi_0$ to $\phi$ and get
\begin{equation}
    \laplacian \phi = 4\pi n = 4\pi \cdot \frac{(2\phi)^{3/2}}{3 \pi^2}.
\end{equation}
\begin{equation}
    \phi(r) = \frac{Z}{r} \chi(r Z^{1/3} / b), \quad 
    b = \frac{1}{2} \left(\frac{3 \pi}{4}\right)^{2/3}, \quad 
    \sqrt{y} \pdv[2]{\chi}{y} = \chi^{3/2}.
\end{equation}
When $r \to \infty$ we find $\phi$ damps to zero, 
so this is indeed correct.

\section{Light-atom interaction}

By considering the system in a beam of light, we have
\begin{equation}
    \text{radiation} \propto \rho_{22} = \frac{\Omega^2 / 4}{(\omega - \omega_0)^2 + \Omega^2 / 2 + \Gamma^2 / 4},
\end{equation}
which has the limit 
\begin{equation}
    I_{\text{out}} = \frac{1}{2} \quad \text{as $\Omega \to \infty$},
\end{equation} 
which is known as \concept{saturation}: 
when the driving electric force is strong enough, 
the output light doesn't show any frequency dependence. 
When the input power isn't that strong, 
we get \concept{power broadening}.

We can then define the absorption cross section 
\begin{equation}
    \rho \sigma \dd{l} = \text{\# of scattering within $\dd{l}$}.
\end{equation}

There are of course other types of broadening.
In a real world experimental setting 
atoms have thermal Fluctuation,
and if we ignore the quantum nature of atoms, 
we have 
\begin{equation}
    P(\vb*{v}) \dd[3]{\vb*{v}} = \left(\frac{M}{2 \pi \kB T}\right)^{1/2}
    \exp(- \frac{M \vb*{v}^2}{2 \kB T} ) \dd[3]{\vb*{v}},
\end{equation}
and then Doppler shift passes this fluctuation 
to the observed frequency.
Actually the fluctuation caused by Doppler shift 
is about 1000 times larger than the broadening caused by damping
when $v \sim \SI{100}{m/s}$.

We can also apply two beams of lasers to the system,
and if one beam (we call the \concept{pumping laser}) 
is stronger than the other (the \concept{detection laser}),
but still much smaller than the saturation intensity, 
then we will see a dip in the intensity-frequency curve 
at the frequency of the pump,
if we measure the output caused by the detection beam,
because the pumping beam drives
atoms to the excited state and doesn't let them go down.
This can be used to do
\concept{saturation-absorption spectroscopy},
in which TODO 
and when the 



\end{document}