\documentclass[hyperref, a4paper]{article}

\usepackage{geometry}
\usepackage{titling}
\usepackage{titlesec}
% No longer needed, since we will use enumitem package
% \usepackage{paralist}
\usepackage{enumitem}
\usepackage{footnote}
\usepackage{amsmath, amssymb, amsthm}
\usepackage{mathtools}
\usepackage{bbm}
\usepackage{cite}
\usepackage{graphicx}
\usepackage{subfigure}
\usepackage{physics}
\usepackage{tensor}
\usepackage{siunitx}
\usepackage[version=4]{mhchem}
\usepackage{tikz}
\usepackage{xcolor}
\usepackage{listings}
\usepackage{underscore}
\usepackage{autobreak}
\usepackage[ruled, vlined, linesnumbered]{algorithm2e}
\usepackage{nameref,zref-xr}
\zxrsetup{toltxlabel}
\usepackage[colorlinks,unicode]{hyperref} % , linkcolor=black, anchorcolor=black, citecolor=black, urlcolor=black, filecolor=black
\usepackage[most]{tcolorbox}
\usepackage{prettyref}

% Page style
\geometry{left=3.18cm,right=3.18cm,top=2.54cm,bottom=2.54cm}
\titlespacing{\paragraph}{0pt}{1pt}{10pt}[20pt]
\setlength{\droptitle}{-5em}

% More compact lists 
\setlist[itemize]{
    %itemindent=17pt, 
    %leftmargin=1pt,
    listparindent=\parindent,
    parsep=0pt,
}

\setlist[enumerate]{
    %itemindent=17pt, 
    %leftmargin=1pt,
    listparindent=\parindent,
    parsep=0pt,
}

% Math operators
\DeclareMathOperator{\timeorder}{\mathcal{T}}
\DeclareMathOperator{\diag}{diag}
\DeclareMathOperator{\legpoly}{P}
\DeclareMathOperator{\primevalue}{P}
\DeclareMathOperator{\sgn}{sgn}
\DeclareMathOperator{\res}{Res}
\newcommand*{\ii}{\mathrm{i}}
\newcommand*{\ee}{\mathrm{e}}
\newcommand*{\const}{\mathrm{const}}
\newcommand*{\suchthat}{\quad \text{s.t.} \quad}
\newcommand*{\argmin}{\arg\min}
\newcommand*{\argmax}{\arg\max}
\newcommand*{\normalorder}[1]{: #1 :}
\newcommand*{\pair}[1]{\langle #1 \rangle}
\newcommand*{\fd}[1]{\mathcal{D} #1}
\DeclareMathOperator{\bigO}{\mathcal{O}}

% TikZ setting
\usetikzlibrary{arrows,shapes,positioning}
\usetikzlibrary{arrows.meta}
\usetikzlibrary{decorations.markings}
\usetikzlibrary{calc}
\tikzstyle arrowstyle=[scale=1]
\tikzstyle directed=[postaction={decorate,decoration={markings,
    mark=at position .5 with {\arrow[arrowstyle]{stealth}}}}]
\tikzstyle ray=[directed, thick]
\tikzstyle dot=[anchor=base,fill,circle,inner sep=1pt]

% Algorithm setting
% Julia-style code
\SetKwIF{If}{ElseIf}{Else}{if}{}{elseif}{else}{end}
\SetKwFor{For}{for}{}{end}
\SetKwFor{While}{while}{}{end}
\SetKwProg{Function}{function}{}{end}
\SetArgSty{textnormal}

\newcommand*{\concept}[1]{{\textbf{#1}}}

% Embedded codes
\lstset{basicstyle=\ttfamily,
  showstringspaces=false,
  commentstyle=\color{gray},
  keywordstyle=\color{blue}
}

\lstdefinestyle{console}{
    basicstyle=\footnotesize\ttfamily,
    breaklines=true,
    postbreak=\mbox{\textcolor{red}{$\hookrightarrow$}\space}
}

% Reference formatting
\newrefformat{fig}{Figure~\ref{#1}}

% Color boxes
\tcbuselibrary{skins, breakable, theorems}
\newtcbtheorem[number within=section]{warning}{Warning}%
  {colback=orange!5,colframe=orange!65,fonttitle=\bfseries, breakable}{warn}
\newtcbtheorem[number within=section]{note}{Note}%
  {colback=green!5,colframe=green!65,fonttitle=\bfseries, breakable}{note}
\newtcbtheorem[number within=section]{info}{Info}%
  {colback=blue!5,colframe=blue!65,fonttitle=\bfseries, breakable}{info}

% Displaying texts in bookmarkers

\pdfstringdefDisableCommands{%
  \def\\{}%
  \def\ce#1{<#1>}%
}

\pdfstringdefDisableCommands{%
  \def\texttt#1{<#1>}%
  \def\mathbb#1{#1}%
}
\pdfstringdefDisableCommands{\def\eqref#1{(\ref{#1})}}

\makeatletter
\pdfstringdefDisableCommands{\let\HyPsd@CatcodeWarning\@gobble}
\makeatother

\newenvironment{shelldisplay}{\begin{lstlisting}}{\end{lstlisting}}

\newcommand{\shortcode}[1]{\texttt{#1}}

\lstset{style = console}

% Make subsubsection labeled
\setcounter{secnumdepth}{4}
\setcounter{tocdepth}{4}

\newcommand{\kB}{k_{\text{B}}}

\title{Homework 1}
\author{Jinyuan Wu}

\begin{document}

\maketitle 

\section{Problem 1: The Beam Splitter}

Since $\abs*{t}^2 = \abs*{r}^2 = 1/2$,
we have 
\begin{equation}
    \pmqty{E_c \\ E_d} = 
    \underbrace{
        \pmqty{
        \ee^{\ii \phi_{ta}} & \ee^{\ii \phi_{rb}} \\
        \ee^{\ii \phi_{ra}} & \ee^{\ii \phi_{tb}} 
        }
    }_M \pmqty{E_{a} \\ E_b}.
\end{equation}
The unitary condition means 
\begin{equation}
    M M^\dagger = I,
\end{equation}
which in turns means 
\[
    \begin{aligned}
        I &= \frac{1}{2} \pmqty{
            \ee^{\ii \phi_{ta}} & \ee^{\ii \phi_{rb}} \\
            \ee^{\ii \phi_{ra}} & \ee^{\ii \phi_{tb}} 
        } 
        \pmqty{
            \ee^{- \ii \phi_{ta}} & \ee^{-\ii \phi_{ra}} \\
            \ee^{- \ii \phi_{rb}} & \ee^{-\ii \phi_{tb}} 
        } \\ 
        &= \frac{1}{2} \pmqty{
            2 & \ee^{\ii (\phi_{ta} - \phi_{ra})} + \ee^{\ii (\phi_{rb} - \phi_{tb})} \\
            \ee^{- \ii (\phi_{ta} - \phi_{ra})} + \ee^{- \ii (\phi_{rb} - \phi_{tb})} & 2
        },
    \end{aligned}
\]
and this is equivalent to 
\[
    \ee^{\ii (\phi_{ta} - \phi_{ra})} + \ee^{\ii (\phi_{rb} - \phi_{tb})} = 0,
\]
or in other words 
\begin{equation}
    \phi_{ta} - \phi_{ra} = \phi_{rb} - \phi_{tb} + \pi n, \quad \text{$n$ odd}.
\end{equation}

\section{Problem 2: Interferometers}



\section{Correlation function and Other Properties of the Blackbody Field}

\subsection{Energy at $\omega$; Total Energy}

\subsubsection{Energy of an electromagnetic mode}\label{sec:correlation.energy.mode}

From 
\[
    \curl{\vb*{E}} = - \pdv{\vb*{B}}{t},
\]
we have 
\[
    \ii \vb*{k} \times \vb*{E}_\omega = \ii \omega \vb*{B}_\omega,
\]
and therefore 
\[
    \abs*{\vb*{B}_\omega} = \frac{k}{\omega} \abs{\vb*{E}_\omega} = \frac{1}{c} \abs{\vb*{E}},
\]
so 
\begin{equation}
    \begin{aligned}
        u_\omega &= \frac{\epsilon_0}{2} \abs*{\vb*{E}_\omega^2 }
        + \frac{1}{2 \mu_0} \abs*{\vb*{B}_\omega}^2 \\
        &= \frac{\epsilon_0}{2} \abs*{\vb*{E}_\omega^2} 
        + \frac{1}{2 \mu_0} \underbrace{\frac{1}{c^2}}_{\mu_0 \epsilon_0} \abs*{\vb*{E}}_\omega^2  \\
        &= \epsilon_0 \abs*{\vb*{E}_\omega}^2.
    \end{aligned}
\end{equation}

\subsubsection{Energy density}

Now we derive the energy at $\omega$.
Between $\omega$ and $\omega + \dd{\omega}$, we have 
\[
    \text{\# of $\vb*{k}$ per $\dd{\omega}$} = 
    \frac{V}{(2\pi)^3} 4 \pi k^2 \dd{k}, \quad k = \frac{\omega}{c}.
\]
Since there are two polarizations for each $\vb*{k}$,
the number of states per $\dd{\omega}$ is 
\begin{equation}
    \text{\# of state per $\dd{\omega}$} 
    = 2 \cdot \text{\# of $\vb*{k}$ per $\dd{\omega}$} 
    = \frac{V}{\pi^2 c^3} \omega^2 \dd{\omega}.
\end{equation}
Now since the total energy in the cavity is 
\begin{equation}
    U = \int \text{\# of state per $\dd{\omega}$}  
    \cdot \hbar \omega \cdot \frac{1}{\ee^{\hbar \omega / \kB T} - 1},
\end{equation}
the total energy density -- the amount of energy per $\dd[3]{\vb*{r}}$ -- is 
\begin{equation}
    u = \int \dd{\omega} \frac{\hbar \omega^3}{\pi^2 c^3} 
    \frac{1}{\ee^{\hbar \omega / \kB T} - 1}.
\end{equation}
Using 
\[
    \int_0^\infty \frac{x^3 \dd{x}}{\ee^{x} - 1} = \frac{\pi^4}{15}, 
\]
we get 
\begin{equation}
    u = \frac{\hbar}{\pi^2 c^3} \left(\frac{\kB T}{\hbar}\right)^4 \cdot \frac{\pi^4}{15}.
\end{equation}
The intensity of radiation out of the cavity is 
\[
    I = \sum_{\text{$m$ outgoing}} A \vb*{n} \cdot \vb*{S}_{m}, 
    \quad \vb*{S}_m = u_m c \vu*{k},
\]
where $\vb*{n}$ is the normal vector of the hole between the cavity and the outside word, 
$m$ is the index of optical modes within the cavity,
$\vb*{S}_m$ is the Poynting vector of mode $m$.
We can make use of the spherical symmetry of radiation:
suppose $\dd{\Omega}$ is the solid angle element of $\vu*{k}$,
we have 
\[
    \begin{aligned}
        J = \frac{I}{A} &= 
        \underbrace{\frac{1}{4\pi}}_{\text{total solid angle}} 
        \int_{\text{$\vu*{k}$ outgoing}} \dd{\Omega}
        \vb*{n} \cdot u c \vu*{k} \\
        &= uc \cdot \frac{1}{4\pi} \int_{\theta \leq \pi / 2} \sin \theta \dd{\theta} \dd{\varphi}
        \cos \theta \\
        &= uc \cdot \frac{1}{4\pi} \cdot \frac{1}{2} \cdot 2 \pi = \frac{1}{4} u c,
    \end{aligned}
\]
and finally we get 
\begin{equation}
    J = \underbrace{
        \frac{\pi^2 \kB^4}{60 \hbar^3 c^2}
    }_\sigma T^4.
\end{equation}

\subsection{Correlation Function of the Black Body Field}

The experimental definition of the correlation function is 
\begin{equation}
    R_{xx}(\tau) = \frac{1}{T} \int_{-T/2}^{T/2} \dd{t} E_x(t+\tau) E_x(t),
\end{equation}
and so on. 
Using the ergodic condition, this is equivalent to 
\begin{equation}
    R_{xx} (\tau) = \expval{E_x(\tau) E_x(0)}.
\end{equation}

According to \prettyref{sec:correlation.energy.mode}, 
we have 
\begin{equation}
    u = \epsilon_0 \abs*{\vb*{E}}^2 = 
\end{equation}
\begin{equation}
    R_{xx}(0) = \expval{E_x(0)^2} = 
\end{equation}

\end{document}