\documentclass[hyperref, a4paper]{article}

\usepackage{geometry}
\usepackage{titling}
\usepackage{titlesec}
% No longer needed, since we will use enumitem package
% \usepackage{paralist}
\usepackage{enumitem}
\usepackage{footnote}
\usepackage{amsmath, amssymb, amsthm}
\usepackage{mathtools}
\usepackage{bbm}
\usepackage{cite}
\usepackage{graphicx}
\usepackage{subfigure}
\usepackage{physics}
\usepackage{tensor}
\usepackage{siunitx}
\usepackage[version=4]{mhchem}
\usepackage{tikz}
\usepackage{xcolor}
\usepackage{listings}
\usepackage{underscore}
\usepackage{autobreak}
\usepackage[ruled, vlined, linesnumbered]{algorithm2e}
\usepackage{nameref,zref-xr}
\zxrsetup{toltxlabel}
\usepackage[colorlinks,unicode]{hyperref} % , linkcolor=black, anchorcolor=black, citecolor=black, urlcolor=black, filecolor=black
\usepackage[most]{tcolorbox}
\usepackage{prettyref}

% Page style
\geometry{left=3.18cm,right=3.18cm,top=2.54cm,bottom=2.54cm}
\titlespacing{\paragraph}{0pt}{1pt}{10pt}[20pt]
\setlength{\droptitle}{-5em}

% More compact lists 
\setlist[itemize]{
    %itemindent=17pt, 
    %leftmargin=1pt,
    listparindent=\parindent,
    parsep=0pt,
}

\setlist[enumerate]{
    %itemindent=17pt, 
    %leftmargin=1pt,
    listparindent=\parindent,
    parsep=0pt,
}

% Math operators
\DeclareMathOperator{\timeorder}{\mathcal{T}}
\DeclareMathOperator{\diag}{diag}
\DeclareMathOperator{\legpoly}{P}
\DeclareMathOperator{\primevalue}{P}
\DeclareMathOperator{\sgn}{sgn}
\DeclareMathOperator{\res}{Res}
\newcommand*{\ii}{\mathrm{i}}
\newcommand*{\ee}{\mathrm{e}}
\newcommand*{\const}{\mathrm{const}}
\newcommand*{\suchthat}{\quad \text{s.t.} \quad}
\newcommand*{\argmin}{\arg\min}
\newcommand*{\argmax}{\arg\max}
\newcommand*{\normalorder}[1]{: #1 :}
\newcommand*{\pair}[1]{\langle #1 \rangle}
\newcommand*{\fd}[1]{\mathcal{D} #1}
\DeclareMathOperator{\bigO}{\mathcal{O}}

% TikZ setting
\usetikzlibrary{arrows,shapes,positioning}
\usetikzlibrary{arrows.meta}
\usetikzlibrary{decorations.markings}
\usetikzlibrary{calc}
\tikzstyle arrowstyle=[scale=1]
\tikzstyle directed=[postaction={decorate,decoration={markings,
    mark=at position .5 with {\arrow[arrowstyle]{stealth}}}}]
\tikzstyle ray=[directed, thick]
\tikzstyle dot=[anchor=base,fill,circle,inner sep=1pt]

% Algorithm setting
% Julia-style code
\SetKwIF{If}{ElseIf}{Else}{if}{}{elseif}{else}{end}
\SetKwFor{For}{for}{}{end}
\SetKwFor{While}{while}{}{end}
\SetKwProg{Function}{function}{}{end}
\SetArgSty{textnormal}

\newcommand*{\concept}[1]{{\textbf{#1}}}

% Embedded codes
\lstset{basicstyle=\ttfamily,
  showstringspaces=false,
  commentstyle=\color{gray},
  keywordstyle=\color{blue}
}

\lstdefinestyle{console}{
    basicstyle=\footnotesize\ttfamily,
    breaklines=true,
    postbreak=\mbox{\textcolor{red}{$\hookrightarrow$}\space}
}

% Reference formatting
\newrefformat{fig}{Figure~\ref{#1}}

% Color boxes
\tcbuselibrary{skins, breakable, theorems}
\newtcbtheorem[number within=section]{warning}{Warning}%
  {colback=orange!5,colframe=orange!65,fonttitle=\bfseries, breakable}{warn}
\newtcbtheorem[number within=section]{note}{Note}%
  {colback=green!5,colframe=green!65,fonttitle=\bfseries, breakable}{note}
\newtcbtheorem[number within=section]{info}{Info}%
  {colback=blue!5,colframe=blue!65,fonttitle=\bfseries, breakable}{info}

% Displaying texts in bookmarkers

\pdfstringdefDisableCommands{%
  \def\\{}%
  \def\ce#1{<#1>}%
}

\pdfstringdefDisableCommands{%
  \def\texttt#1{<#1>}%
  \def\mathbb#1{#1}%
}
\pdfstringdefDisableCommands{\def\eqref#1{(\ref{#1})}}

\makeatletter
\pdfstringdefDisableCommands{\let\HyPsd@CatcodeWarning\@gobble}
\makeatother

\newenvironment{shelldisplay}{\begin{lstlisting}}{\end{lstlisting}}

\newcommand{\shortcode}[1]{\texttt{#1}}

\lstset{style = console}

% Make subsubsection labeled
\setcounter{secnumdepth}{4}
\setcounter{tocdepth}{4}

\newcommand{\kB}{k_{\text{B}}}

\title{Homework 1}
\author{Jinyuan Wu}

\begin{document}

\maketitle 

\section{Problem 1: The Beam Splitter}

Since $\abs*{t}^2 = \abs*{r}^2 = 1/2$,
we have 
\begin{equation}
    \pmqty{E_c \\ E_d} = 
    \underbrace{
        \pmqty{
        \ee^{\ii \phi_{ta}} & \ee^{\ii \phi_{rb}} \\
        \ee^{\ii \phi_{ra}} & \ee^{\ii \phi_{tb}} 
        }
    }_M \pmqty{E_{a} \\ E_b}.
\end{equation}
The unitary condition means 
\begin{equation}
    M^\dagger M = I,
\end{equation}
which in turns means 
\[
    \begin{aligned}
        I &= \frac{1}{2} 
        \pmqty{
            \ee^{- \ii \phi_{ta}} & \ee^{-\ii \phi_{ra}} \\
            \ee^{- \ii \phi_{rb}} & \ee^{-\ii \phi_{tb}} 
        } 
        \pmqty{
            \ee^{\ii \phi_{ta}} & \ee^{\ii \phi_{rb}} \\
            \ee^{\ii \phi_{ra}} & \ee^{\ii \phi_{tb}} 
        } \\ 
        &= \frac{1}{2} \pmqty{
            2 & \ee^{\ii (\phi_{rb} - \phi_{ta})} + \ee^{\ii (\phi_{tb} - \phi_{ra})} \\
            \ee^{\ii (\phi_{ta} - \phi_{rb})} + \ee^{\ii (\phi_{ra} - \phi_{tb})} & 2
        },
    \end{aligned}
\]
and this is equivalent to 
\[
    \ee^{\ii (\phi_{rb} - \phi_{ta})} + \ee^{\ii (\phi_{tb} - \phi_{ra})} = 0,
\]
or in other words 
\begin{equation}
    \phi_{rb} - \phi_{ta} = \phi_{tb} - \phi_{ra} + \pi n, \quad \text{$n$ odd}.
\end{equation}

\section{Problem 2: Interferometers}

\begin{figure}
    \centering
    \begin{tikzpicture}[x=0.75pt,y=0.75pt,yscale=-1,xscale=1]
        %uncomment if require: \path (0,300); %set diagram left start at 0, and has height of 300
        
        %Straight Lines [id:da024073586307176376] 
        \draw    (285.33,191.21) -- (338.67,191.21) ;
        %Straight Lines [id:da00754509431769268] 
        \draw [color={rgb, 255:red, 245; green, 166; blue, 35 }  ,draw opacity=1 ]   (127.46,124.62) -- (297,124.62) ;
        \draw [shift={(212.23,124.62)}, rotate = 180] [fill={rgb, 255:red, 245; green, 166; blue, 35 }  ,fill opacity=1 ][line width=0.08]  [draw opacity=0] (7.14,-3.43) -- (0,0) -- (7.14,3.43) -- (4.74,0) -- cycle    ;
        %Shape: Rectangle [id:dp10948146523699043] 
        \draw  [draw opacity=0][fill={rgb, 255:red, 155; green, 155; blue, 155 }  ,fill opacity=0.3 ] (339.21,75.61) -- (256.06,174.7) -- (254.79,173.64) -- (337.94,74.54) -- cycle ;
        %Straight Lines [id:da3293184384942547] 
        \draw [color={rgb, 255:red, 245; green, 166; blue, 35 }  ,draw opacity=1 ]   (297,124.62) -- (401.21,124.62) ;
        \draw [shift={(351.7,124.62)}, rotate = 180] [fill={rgb, 255:red, 245; green, 166; blue, 35 }  ,fill opacity=1 ][line width=0.08]  [draw opacity=0] (7.14,-3.43) -- (0,0) -- (7.14,3.43) -- (4.74,0) -- cycle    ;
        %Straight Lines [id:da36464990166378697] 
        \draw [color={rgb, 255:red, 144; green, 19; blue, 254 }  ,draw opacity=1 ] [dash pattern={on 4.5pt off 4.5pt}]  (297,124.62) -- (297,233.18) ;
        %Straight Lines [id:da14386930212480187] 
        \draw  [dash pattern={on 4.5pt off 4.5pt}]  (251.06,170.56) -- (342.94,78.68) ;
        %Shape: Arc [id:dp38829661669018756] 
        \draw  [draw opacity=0] (333.09,80.87) .. controls (334.14,81.52) and (335.15,82.23) .. (336.13,83.03) .. controls (336.7,83.48) and (337.25,83.96) .. (337.77,84.45) -- (317.29,106.37) -- cycle ; \draw   (333.09,80.87) .. controls (334.14,81.52) and (335.15,82.23) .. (336.13,83.03) .. controls (336.7,83.48) and (337.25,83.96) .. (337.77,84.45) ;  
        %Straight Lines [id:da5224488685141762] 
        \draw [color={rgb, 255:red, 245; green, 166; blue, 35 }  ,draw opacity=1 ]   (297,124.62) -- (309.58,191) ;
        \draw [shift={(303.78,160.37)}, rotate = 259.27] [fill={rgb, 255:red, 245; green, 166; blue, 35 }  ,fill opacity=1 ][line width=0.08]  [draw opacity=0] (7.14,-3.43) -- (0,0) -- (7.14,3.43) -- (4.74,0) -- cycle    ;
        %Shape: Arc [id:dp8153827577665933] 
        \draw  [draw opacity=0] (306.66,177.63) .. controls (305.22,177.85) and (303.75,177.96) .. (302.25,177.96) .. controls (300.63,177.96) and (299.04,177.83) .. (297.49,177.58) -- (302.25,147.96) -- cycle ; \draw  [color={rgb, 255:red, 245; green, 166; blue, 35 }  ,draw opacity=1 ] (306.66,177.63) .. controls (305.22,177.85) and (303.75,177.96) .. (302.25,177.96) .. controls (300.63,177.96) and (299.04,177.83) .. (297.49,177.58) ;  
        %Straight Lines [id:da2341731700142713] 
        \draw    (268.53,50.76) -- (320.14,40.72) ;
        %Straight Lines [id:da000712146693267357] 
        \draw    (401.21,102.75) -- (401.21,230.84) ;
        %Straight Lines [id:da3125918413760984] 
        \draw [color={rgb, 255:red, 245; green, 166; blue, 35 }  ,draw opacity=1 ]   (401.21,124.62) -- (304.08,124.62) ;
        \draw [shift={(350.05,124.62)}, rotate = 360] [fill={rgb, 255:red, 245; green, 166; blue, 35 }  ,fill opacity=1 ][line width=0.08]  [draw opacity=0] (7.14,-3.43) -- (0,0) -- (7.14,3.43) -- (4.74,0) -- cycle    ;
        %Straight Lines [id:da6256146356477137] 
        \draw [color={rgb, 255:red, 184; green, 233; blue, 134 }  ,draw opacity=1 ]   (306.92,112.12) -- (322.28,191.47) ;
        \draw [shift={(315.09,154.35)}, rotate = 259.04] [fill={rgb, 255:red, 184; green, 233; blue, 134 }  ,fill opacity=1 ][line width=0.08]  [draw opacity=0] (7.14,-3.43) -- (0,0) -- (7.14,3.43) -- (4.74,0) -- cycle    ;
        %Straight Lines [id:da9550632258492229] 
        \draw [color={rgb, 255:red, 184; green, 233; blue, 134 }  ,draw opacity=1 ]   (294.33,45.74) -- (306.92,112.12) ;
        \draw [shift={(299.86,74.9)}, rotate = 79.27] [fill={rgb, 255:red, 184; green, 233; blue, 134 }  ,fill opacity=1 ][line width=0.08]  [draw opacity=0] (7.14,-3.43) -- (0,0) -- (7.14,3.43) -- (4.74,0) -- cycle    ;
        %Straight Lines [id:da3596043853538955] 
        \draw [color={rgb, 255:red, 189; green, 16; blue, 224 }  ,draw opacity=1 ] [dash pattern={on 4.5pt off 4.5pt}]  (236.61,45.74) -- (322.71,45.74) ;
        %Straight Lines [id:da5083419032584222] 
        \draw [color={rgb, 255:red, 189; green, 16; blue, 224 }  ,draw opacity=1 ]   (251.61,47.74) -- (251.61,122.82) ;
        \draw [shift={(251.61,124.82)}, rotate = 270] [fill={rgb, 255:red, 189; green, 16; blue, 224 }  ,fill opacity=1 ][line width=0.08]  [draw opacity=0] (12,-3) -- (0,0) -- (12,3) -- cycle    ;
        \draw [shift={(251.61,45.74)}, rotate = 90] [fill={rgb, 255:red, 189; green, 16; blue, 224 }  ,fill opacity=1 ][line width=0.08]  [draw opacity=0] (12,-3) -- (0,0) -- (12,3) -- cycle    ;
        %Straight Lines [id:da9784246302104025] 
        \draw [color={rgb, 255:red, 184; green, 233; blue, 134 }  ,draw opacity=1 ]   (306.92,112.12) -- (127.61,112.12) ;
        \draw [shift={(221.36,112.12)}, rotate = 180] [fill={rgb, 255:red, 184; green, 233; blue, 134 }  ,fill opacity=1 ][line width=0.08]  [draw opacity=0] (7.14,-3.43) -- (0,0) -- (7.14,3.43) -- (4.74,0) -- cycle    ;
        %Straight Lines [id:da607897315469625] 
        \draw [color={rgb, 255:red, 74; green, 144; blue, 226 }  ,draw opacity=1 ]   (398.61,223.18) -- (299,223.18) ;
        \draw [shift={(297,223.18)}, rotate = 360] [fill={rgb, 255:red, 74; green, 144; blue, 226 }  ,fill opacity=1 ][line width=0.08]  [draw opacity=0] (12,-3) -- (0,0) -- (12,3) -- cycle    ;
        \draw [shift={(400.61,223.18)}, rotate = 180] [fill={rgb, 255:red, 74; green, 144; blue, 226 }  ,fill opacity=1 ][line width=0.08]  [draw opacity=0] (12,-3) -- (0,0) -- (12,3) -- cycle    ;
        
        % Text Node
        \draw (339.77,87.85) node [anchor=north west][inner sep=0.75pt]    {$\theta $};
        % Text Node
        \draw (309.58,194.4) node [anchor=north] [inner sep=0.75pt]  [color={rgb, 255:red, 245; green, 166; blue, 35 }  ,opacity=1 ]  {$2\theta $};
        % Text Node
        \draw (295,157.91) node [anchor=east] [inner sep=0.75pt]  [color={rgb, 255:red, 144; green, 19; blue, 254 }  ,opacity=1 ]  {$d$};
        % Text Node
        \draw (249.61,85.28) node [anchor=east] [inner sep=0.75pt]  [color={rgb, 255:red, 189; green, 16; blue, 224 }  ,opacity=1 ]  {$l_{1}$};
        % Text Node
        \draw (348.81,219.78) node [anchor=south] [inner sep=0.75pt]  [color={rgb, 255:red, 74; green, 144; blue, 226 }  ,opacity=1 ]  {$l_{2}$};
        
        
        \end{tikzpicture}
        
    \caption{Michelson interferometer with tilted mirrors} 
    \label{fig:michelson}
\end{figure}

Consider a Michelson interferometer, and rotate the beam splitter with an angle of $\theta$,
and also rotate one mirror with an angle of $2 \theta$,
and we get \prettyref{fig:michelson}.
The change of the optical path of the green ray is 
\begin{equation}
    \Delta L_{\text{green}} = \frac{l_1 + d}{\cos 2 \theta} - (l_1 + d) 
    = (l_1 + d) \left(
        1 + \frac{1}{2} (2 \theta)^2 + \cdots - 1
    \right) = 2(l_1 + d) \theta^2 + \cdots,
\end{equation}
and the change of the optical path of the orange ray is 
\begin{equation}
    \Delta L_{\text{orange}} = l_2 + \frac{d}{\cos 2 \theta} - (l_2 + d) 
    = d \left(
        1 + \frac{1}{2} (2\theta)^2 + \cdots - 1
    \right) = 2d \theta^2 + \cdots.
\end{equation}
Thus the changes of both paths are $\bigO(\theta^2)$.

When the potential -- in optics, the refractive index -- is changed, 
the path of the beam may be changed, 
but as is outlined above, slight change of the angle of propagation 
only causes a $\bigO(\theta^2)$ change on the optical path, 
so the main contribution of the change of the refractive index 
is the correction factor to terms like $l_1$ or $d$ in 
$\Delta L_{\text{green}}$ or $\Delta L_{\text{orange}}$.
If, for example, a sample is placed on $l_1$, 
then we have 
\begin{equation}
    \Delta L_{\text{green}} = n \frac{l_1 + d}{\cos 2 \theta} - (l_1 + d)
    = (l_1 + d) \left(
        n \left( 1 + \frac{1}{2} (2 \theta)^2 + \cdots \right) - 1 
    \right) 
    = (l_1 + d) (n - 1 + 2 n \theta^2 + \cdots),
\end{equation}
and the first order variance of $\Delta L_{\text{green}}$ comes from the $n$ factor in the 
$n (l_1 + d) / \cos 2 \theta$ term. 

\section{Problem 3: Correlation function and Other Properties of the Blackbody Field}

\subsection{Energy at $\omega$; Total Energy}

\subsubsection{Energy of an electromagnetic mode}\label{sec:correlation.energy.mode}

From 
\[
    \curl{\vb*{E}} = - \pdv{\vb*{B}}{t},
\]
we have 
\[
    \ii \vb*{k} \times \vb*{E}_\omega = \ii \omega \vb*{B}_\omega,
\]
and therefore 
\[
    \abs*{\vb*{B}_\omega} = \frac{k}{\omega} \abs{\vb*{E}_\omega} = \frac{1}{c} \abs{\vb*{E}_\omega},
\]
so 
\begin{equation}
    \begin{aligned}
        u_\omega &= \frac{\epsilon_0}{2} \abs*{\vb*{E}_\omega^2 }
        + \frac{1}{2 \mu_0} \abs*{\vb*{B}_\omega}^2 \\
        &= \frac{\epsilon_0}{2} \abs*{\vb*{E}_\omega^2} 
        + \frac{1}{2 \mu_0} \underbrace{\frac{1}{c^2}}_{\mu_0 \epsilon_0} \abs*{\vb*{E}_\omega}^2  \\
        &= \epsilon_0 \abs*{\vb*{E}_\omega}^2.
    \end{aligned}
\end{equation}

Here the notation $u_\omega$ may be slightly confusing.
What we want is 
\begin{equation}
    u = \int \dd{\omega} u_\omega.
\end{equation}
If we interpret it as 
the energy density (spatial density) of \emph{one} photon mode with frequency $\omega$, 
and the energy density contributed by \emph{all} photon modes 
with the frequency being between $\omega$ and $\omega + \dd{\omega}$
is $n(\omega) \dd{\omega} \cdot u_\omega$,
where $n(\omega)$ is the density of states. 
In this way, we get the expressions in the beginning of \prettyref{sec:correlation.energy.density}.

We can also define $\vb*{E}_\omega$ according to the standard 
time-domain Fourier transformation:
\begin{equation}
    \vb*{E}_{\omega} = \int \ee^{\ii \omega t} \vb*{E}(\vb*{r}, t) \dd{t} , \quad
    \vb*{E}(\vb*{r}, t) = \sum_{\vb*{k}, \sigma = 1, 2}
    \ii \sqrt{\frac{\hbar \omega_{\vb*{k}}}{2 \epsilon_0 V}} 
    a_{\vb*{k} \sigma} \vu*{e}_\sigma \ee^{\ii \vb*{k} \cdot \vb*{r} - \ii \omega_{\vb*{k}} t} + \text{h.c.},
\end{equation}
and we have 
\begin{equation}
    \begin{aligned}
        \expval*{\vb*{E}(\vb*{r}, 0) \cdot \vb*{E}(\vb*{r}, t)} &= 
        \int \frac{\dd{\omega'}}{2\pi} \int \frac{\dd{\omega}}{2\pi}
        \expval*{\vb*{E}(\vb*{r}, \omega') \cdot \vb*{E}(\vb*{r}, \omega)}
        \ee^{- \ii \omega t}.
    \end{aligned}
\end{equation}
Under this definition, 
we have two $2\pi \delta (\omega - \omega_{\vb*{k}})$ factors in the RHS;
one of them may be understood as imposing the energy conservation condition $\omega + \omega' = 0$,
which is then canceled by the integration $\int \dd{\omega'} / 2\pi$, 
and another of them becomes the density of states, 
because there are more than one $(\vb*{k}, \sigma)$ pair with which  
$\omega_{\vb*{k} \sigma} = \omega$, 
and we sum over all $\vb*{k}$'s and $\sigma$'s.
(Note that due to the momentum conservation condition 
and the orthogonal relation concerning $\vu*{e}_\sigma$,
although in the RHS we have two sums over $\vb*{k}$ and $\sigma$,
only one of them is kept.)
So the eventual expression of the correlation function looks like 
\begin{equation}
    \begin{aligned}
        \expval*{\vb*{E}(\vb*{r}, 0) \cdot \vb*{E}(\vb*{r}, t)}
        &= \int \frac{\dd{\omega}}{2\pi} \ee^{- \ii \omega t} \int \frac{\dd{\omega'}}{2\pi} 
        \underbrace{S({\omega}) 2\pi \delta(\omega + \omega') }_{
            \expval*{\vb*{E}(\vb*{r}, \omega') \cdot \vb*{E}(\vb*{r}, \omega)}
        } \\ 
        &= \int \frac{\dd{\omega}}{2\pi} \ee^{- \ii \omega t} S(\omega),
    \end{aligned}
\end{equation}
where 
\begin{equation}
    \begin{aligned}
        S(\omega) &= 
        \frac{1}{\epsilon_0} \cdot 
        \underbrace{\frac{1}{V} \sum_{\vb*{k}, \sigma} 2\pi \delta(\omega - \omega_{\vb*{k} \sigma})}_{
            \text{density of states per volume}
        }
        \hbar \omega_{\vb*{k}} 
        \cdot \underbrace{\frac{1}{2}  \expval*{a_{\vb*{k} \sigma} a^\dagger_{\vb*{k} \sigma} + \text{h.c.}}}_{
            n_{\vb*{k} \sigma} + \frac{1}{2}
        } \\
        &= \frac{1}{\epsilon_0} 2\pi n(\omega) \hbar \omega \cdot \left(
            f(\omega) + \frac{1}{2}
        \right),
    \end{aligned}
\end{equation}
where $f(\omega)$ is the occupation on energy level $\omega$,
which is the Bose-Einstein distribution in an equilibrium state. 
Putting these together, we get 
\begin{equation}
    \expval*{\vb*{E}(\vb*{r}, 0) \cdot \vb*{E}(\vb*{r}, t)} = 
    \frac{1}{\epsilon_0} \int \dd{t} \ee^{- \ii \omega t} \underbrace{
        n(\omega) \cdot \hbar {\omega} \left(
        f(\omega) + \frac{1}{2}
        \right)
    }_{\eqqcolon u_\omega}.
    \label{eq:u-omega-to-correlation}
\end{equation}
Multiplying $\epsilon_0$ on both sides of the equation and 
take $t = 0$, 
and we arrive at the desired expression of $u_\omega$. 
Its relation with $\expval*{\vb*{E}(\vb*{r}, \omega') \cdot \vb*{E}(\vb*{r}, \omega)}$ however 
involves some normalization factors: 
what we do have is 
\begin{equation}
    \expval*{\vb*{E}(\vb*{r}, \omega') \cdot \vb*{E}(\vb*{r}, \omega)} = 2 \pi \delta(\omega + \omega') S(\omega), \quad 
    S(\omega) = \frac{2\pi}{\epsilon_0} u_\omega.  
\end{equation}
But this doesn't create much trouble: 
we can always find $u_\omega$ 
using the density of states and the occupation, 
and then the correlation find is known
after a Fourier transformation. 

\subsubsection{Energy density}\label{sec:correlation.energy.density}

Now we derive the energy at $\omega$.
Between $\omega$ and $\omega + \dd{\omega}$, we have 
\[
    \text{\# of $\vb*{k}$ per $\dd{\omega}$} = 
    \frac{V}{(2\pi)^3} 4 \pi k^2 \dd{k}, \quad k = \frac{\omega}{c}.
\]
Since there are two polarizations for each $\vb*{k}$,
the number of states per $\dd{\omega}$ is 
\begin{equation}
    \text{\# of state per $\dd{\omega}$} 
    = 2 \cdot \text{\# of $\vb*{k}$ per $\dd{\omega}$} 
    = \frac{V}{\pi^2 c^3} \omega^2 \dd{\omega}.
\end{equation}
Now since the total energy in the cavity is 
\begin{equation}
    U = \int \text{\# of state per $\dd{\omega}$}  
    \cdot \hbar \omega \cdot \frac{1}{\ee^{\hbar \omega / \kB T} - 1},
\end{equation}
the total energy density -- the amount of energy per $\dd[3]{\vb*{r}}$ -- is 
\begin{equation}
    u = \int_0^\infty \dd{\omega} \frac{\hbar \omega^3}{\pi^2 c^3} 
    \frac{1}{\ee^{\hbar \omega / \kB T} - 1}.
\end{equation}
Note that here we choose $\omega \geq 0$,
because when $\omega < 0$, 
the density of states vanishes. 
Using 
\[
    \int_0^\infty \frac{x^3 \dd{x}}{\ee^{x} - 1} = \frac{\pi^4}{15}, 
\]
we get 
\begin{equation}
    u = \frac{\hbar}{\pi^2 c^3} \left(\frac{\kB T}{\hbar}\right)^4 \cdot \frac{\pi^4}{15}.
\end{equation}
The intensity of radiation out of the cavity is 
\[
    I = \sum_{\text{$m$ outgoing}} A \vb*{n} \cdot \vb*{S}_{m}, 
    \quad \vb*{S}_m = u_m c \vu*{k},
\]
where $\vb*{n}$ is the normal vector of the hole between the cavity and the outside word, 
$m$ is the index of optical modes within the cavity,
$\vb*{S}_m$ is the Poynting vector of mode $m$.
We can make use of the spherical symmetry of radiation:
suppose $\dd{\Omega}$ is the solid angle element of $\vu*{k}$,
we have 
\[
    \begin{aligned}
        J = \frac{I}{A} &= 
        \underbrace{\frac{1}{4\pi}}_{\text{total solid angle}} 
        \int_{\text{$\vu*{k}$ outgoing}} \dd{\Omega}
        \vb*{n} \cdot u c \vu*{k} \\
        &= uc \cdot \frac{1}{4\pi} \int_{\theta \leq \pi / 2} \sin \theta \dd{\theta} \dd{\varphi}
        \cos \theta \\
        &= uc \cdot \frac{1}{4\pi} \cdot \frac{1}{2} \cdot 2 \pi = \frac{1}{4} u c,
    \end{aligned}
\]
and finally we get 
\begin{equation}
    J = \underbrace{
        \frac{\pi^2 \kB^4}{60 \hbar^3 c^2}
    }_\sigma T^4.
\end{equation}

\subsection{Correlation Function of the Black Body Field}

The experimental definition of the correlation function is 
\begin{equation}
    R_{xx}(\tau) = \frac{1}{T} \int_{-T/2}^{T/2} \dd{t} E_x(t+\tau) E_x(t),
\end{equation}
and so on. 
Using the ergodic condition, this is equivalent to 
\begin{equation}
    R_{xx} (\tau) = \expval{E_x(\tau) E_x(0)}.
\end{equation}
The same applies for $R_{xy}$, etc. 

Now since we are dealing with linear optics, 
there is no SHG process, etc., 
and each state in the density matrix $\rho = \sum_n \dyad*{n}{n} \ee^{- E_n / \kB t}$
is a photon Fock state.
We know $E_x$ contains photon modes 
for which the polarization vector $\vu*{e}$ is in the $x$ direction,
while $E_y$ contains photon modes 
for which the polarization vector $\vu*{e}$ is in the $y$ direction.
So for each $\ket*{n}$ state that is an eigenstate of the density matrix, 
we have 
\[
    \expval{E_x E_y}{n} = C_1 \expval*{a_{\vu*{x}} a_{\vu*{y}}}{n} + 
    C_2 \expval*{a_{\vu*{x}} a^\dagger_{\vu*{y}}}{n} + 
    C_3 \expval*{a^\dagger_{\vu*{x}} a_{\vu*{y}}}{n} +
    C_4 \expval*{a^\dagger_{\vu*{x}} a^\dagger_{\vu*{y}}}{n},
\]
and each term vanishes because 
after the operators $a_{\vu*{x}} a_{\vu*{y}}$ etc. are applied to the ket vectors, 
the photon occupation configurations on the right and the left are different.
So for each $\ket*{n}$ in $\rho$,
$\expval*{E_x E_y} = 0$,
and therefore $\expval{E_x E_y}_\rho$ also vanishes. 
The same applies for $R_{yz}$ or $R_{zx}$.

According to \prettyref{sec:correlation.energy.mode}, 
we have 
\begin{equation}
    u = \epsilon_0 \abs*{\vb*{E}}^2 .
\end{equation}
To relate $\expval*{E_x^2}$ to $\expval*{\vb*{E}^2}$, note that 
\[
    \expval{\vb*{E}^2} \propto {\vu*{e}_1}^2 + {\vu*{e}_2}^2 = 2, 
\]
and 
\[
    \expval{E_x^2} \propto (\vu*{e}_1 \cdot \vu*{x})^2 +  (\vu*{e}_2 \cdot \vu*{x})^2
    = 1 - (\vu*{k} \cdot \vu*{x})^2.
\]
The coordinates needed to determine $\vu*{k}$, $\vu*{e}_1$, and $\vu*{e}_2$ 
are the polar angle $\theta$ and the azimuthal angle $\varphi$ of $\vu*{k}$.
To go over all possible polarizations,
we need an additional parameter $\psi$ specifying the direction of $\vu*{e}_1$,
and once $\vu*{k}$ and $\vu*{e}_1$ are determined,
we can get the orientation of $\vu*{e}_2$.
To go over \emph{independent} polarization modes, 
no further parameter is needed.
So we have 
\[
    \begin{aligned}
        \frac{\expval{E_x^2}}{\expval{\vb*{E}^2}}
        &= \frac{
            \int \dd{\Omega} (1 - (\vu*{k} \cdot \vu*{x})^2)
        }{ \int \dd{\Omega} \times 2 } = \frac{1}{8\pi} \int \sin \theta \dd{\theta} \dd{\varphi} 
        \left(
            1 - \sin^2 \theta \cos^2 \varphi
        \right) \\
        &= \frac{1}{8\pi} \left(
            4\pi - \int_0^\pi \sin^3 \theta \dd{\theta} \int_0^{2\pi} \cos^2 \varphi \dd{\varphi}
        \right) \\
        &= \frac{1}{3},
    \end{aligned}
\]
and therefore 
\begin{equation}
    R_{xx}(0) = \expval{E_x(0)^2} = \frac{1}{3 \epsilon_0} u,  
\end{equation}
and similarly 
\begin{equation}
    R_{xx}(0) = R_{yy}(0) = R_{zz}(0) = \frac{1}{3 \epsilon_0} u.
\end{equation}

Following \eqref{eq:u-omega-to-correlation} and the last equation, we have 
\begin{equation}
    R_{xx} (t) = \frac{1}{3 \epsilon_0} \int \dd{\omega}  \ee^{- \ii \omega t} u_\omega
    = \frac{\hbar}{3 \epsilon_0 \pi^2 c^3} 
    \left(\frac{\kB T}{\hbar}\right)^4
    \int_0^\infty \dd{x} x^3 
    \frac{
        \ee^{- x \left(1 + \ii t \frac{\kB T}{\hbar} \right)}
    }{1 - \ee^{-x}} .
\end{equation}
Again, we choose $\omega \geq 0$,
or otherwise $n(\omega)$ is zero and $u_\omega$ therefore vanishes.
Since we have 
\begin{equation}
    \psi^{(m)}(z)=(-1)^{m+1} \int_0^{\infty} \frac{t^m e^{-z t}}{1-e^{-t}} \mathrm{~d} t,
\end{equation}
where $\psi^{(m)}(z)$ is the so-called polygamma function, 
we have 
\begin{equation}
    R_{xx}(t) = \frac{\hbar}{3 \epsilon_0 \pi^2 c^3} 
    \left(\frac{\kB T}{\hbar}\right)^4
    \psi^{(m)}\left(
        1 + \ii t \frac{\kB T}{\hbar}
    \right).
\end{equation}


\subsection{Properties of 300 K black body field}

Since 
\begin{equation}
    \expval{I} = \sigma T^4 = \frac{1}{4} c \epsilon_0 \vb*{E}^2,
\end{equation}
when $T = \SI{300}{K}$, we have 
$I = \SI{459}{W/m^2}$,
and $\abs{\vb*{E}} = \SI{832}{V/m}$.
Although \SI{100}{V/m} can cause an electric shock,
this ``field strength'' can't really be felt,
because the strength and direction of $\vb*{E}$ is constantly changing 
and a stable electric field toward a static direction is never established.

The position of the peak of blackbody radiation 
depends on whether we are working with 
the frequency spectrum or the wavelength spectrum. 
When we are working with the frequency spectrum,
from 
\[
    \dv{u_\omega}{\omega} = 0 
\]
we get 
\begin{equation}
    \frac{\hbar \omega}{3 \kB T} = 1- \ee^{- \frac{\hbar \omega}{\kB T}},
\end{equation}
so 
\begin{equation}
    \frac{\hbar \omega}{\kB T} = 2.82, 
\end{equation}
and therefore when $T = \SI{300}{K}$,
we get $\nu = \omega / 2\pi = \SI{1.76e13}{Hz}$,
and therefore $\lambda = c / \nu = \SI{1.7e-5}{m}$.
If we are working with the wavelength spectrum, 
the nonlinear scaling factor $\dv*{\omega}{\lambda}$ changes the equation into 
\begin{equation}
    \frac{\hbar \omega}{5 \kB T} = 1- \ee^{- \frac{\hbar \omega}{\kB T}},
\end{equation}
and the same procedure can be repeated again. 

\end{document}