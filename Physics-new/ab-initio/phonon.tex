\documentclass[hyperref, a4paper]{article}

\usepackage{geometry}
\usepackage{titling}
\usepackage{titlesec}
% No longer needed, since we will use enumitem package
% \usepackage{paralist}
\usepackage{enumitem}
\usepackage{footnote}
\usepackage{amsmath, amssymb, amsthm}
\usepackage{mathtools}
\usepackage{bbm}
\usepackage{graphicx}
\usepackage{subfigure}
\usepackage{physics}
\usepackage{tensor}
\usepackage{siunitx}
\usepackage[version=4]{mhchem}
\usepackage{tikz}
\usepackage{xcolor}
\usepackage{listings}
\usepackage{underscore}
\usepackage{autobreak}
\usepackage[ruled, vlined, linesnumbered]{algorithm2e}
\usepackage{nameref,zref-xr}
\zxrsetup{toltxlabel}
\usepackage[sorting=none]{biblatex}
\addbibresource{phonon.bib}
\usepackage[colorlinks,unicode]{hyperref} % , linkcolor=black, anchorcolor=black, citecolor=black, urlcolor=black, filecolor=black
\usepackage[most]{tcolorbox}
\usepackage{prettyref}

% Page style
\geometry{left=3.18cm,right=3.18cm,top=2.54cm,bottom=2.54cm}
\titlespacing{\paragraph}{0pt}{1pt}{10pt}[20pt]
\setlength{\droptitle}{-5em}

% More compact lists 
\setlist[itemize]{
    %itemindent=17pt, 
    %leftmargin=1pt,
    listparindent=\parindent,
    parsep=0pt,
}

\setlist[enumerate]{
    %itemindent=17pt, 
    %leftmargin=1pt,
    listparindent=\parindent,
    parsep=0pt,
}

% Math operators
\DeclareMathOperator{\timeorder}{\mathcal{T}}
\DeclareMathOperator{\diag}{diag}
\DeclareMathOperator{\legpoly}{P}
\DeclareMathOperator{\primevalue}{P}
\DeclareMathOperator{\sgn}{sgn}
\DeclareMathOperator{\res}{Res}
\newcommand*{\ii}{\mathrm{i}}
\newcommand*{\ee}{\mathrm{e}}
\newcommand*{\const}{\mathrm{const}}
\newcommand*{\suchthat}{\quad \text{s.t.} \quad}
\newcommand*{\argmin}{\arg\min}
\newcommand*{\argmax}{\arg\max}
\newcommand*{\normalorder}[1]{: #1 :}
\newcommand*{\pair}[1]{\langle #1 \rangle}
\newcommand*{\fd}[1]{\mathcal{D} #1}
\DeclareMathOperator{\bigO}{\mathcal{O}}

% TikZ setting
\usetikzlibrary{arrows,shapes,positioning}
\usetikzlibrary{arrows.meta}
\usetikzlibrary{decorations.markings}
\usetikzlibrary{calc}
\tikzstyle arrowstyle=[scale=1]
\tikzstyle directed=[postaction={decorate,decoration={markings,
    mark=at position .5 with {\arrow[arrowstyle]{stealth}}}}]
\tikzstyle ray=[directed, thick]
\tikzstyle dot=[anchor=base,fill,circle,inner sep=1pt]

% Algorithm setting
% Julia-style code
\SetKwIF{If}{ElseIf}{Else}{if}{}{elseif}{else}{end}
\SetKwFor{For}{for}{}{end}
\SetKwFor{While}{while}{}{end}
\SetKwProg{Function}{function}{}{end}
\SetArgSty{textnormal}

\newcommand*{\concept}[1]{{\textbf{#1}}}

% Embedded codes
\lstset{basicstyle=\ttfamily,
  showstringspaces=false,
  commentstyle=\color{gray},
  keywordstyle=\color{blue}
}

\lstdefinestyle{console}{
    basicstyle=\footnotesize\ttfamily,
    breaklines=true,
    postbreak=\mbox{\textcolor{red}{$\hookrightarrow$}\space}
}

% Reference formatting
\newrefformat{fig}{Figure~\ref{#1}}

% Color boxes
\tcbuselibrary{skins, breakable, theorems}

\newtcbtheorem{infobox}{Box}{
    enhanced,
    boxrule=0pt,
    colback=blue!5,
    colframe=blue!5,
    coltitle=blue!50,
    borderline west={4pt}{0pt}{blue!65},
    sharp corners,
    fonttitle=\bfseries, 
    breakable,
    before upper={\parindent15pt\noindent}}{box}
\newtcbtheorem[use counter from=infobox]{theorybox}{Box}{
    enhanced,
    boxrule=0pt,
    colback=orange!5, 
    colframe=orange!5, 
    coltitle=orange!50,
    borderline west={4pt}{0pt}{orange!65},
    sharp corners,
    fonttitle=\bfseries, 
    breakable,
    before upper={\parindent15pt\noindent}}{box}
\newtcbtheorem[use counter from=infobox]{learnbox}{Box}{
    enhanced,
    boxrule=0pt,
    colback=green!5,
    colframe=green!5,
    coltitle=green!50,
    borderline west={4pt}{0pt}{green!65},
    sharp corners,
    fonttitle=\bfseries, 
    breakable,
    before upper={\parindent15pt\noindent}}{box}

% Displaying texts in bookmarkers

\pdfstringdefDisableCommands{%
  \def\\{}%
  \def\ce#1{<#1>}%
}

\pdfstringdefDisableCommands{%
  \def\texttt#1{<#1>}%
  \def\mathbb#1{#1}%
}
\pdfstringdefDisableCommands{\def\eqref#1{(\ref{#1})}}

\makeatletter
\pdfstringdefDisableCommands{\let\HyPsd@CatcodeWarning\@gobble}
\makeatother

\newenvironment{shelldisplay}{\begin{lstlisting}}{\end{lstlisting}}

\newcommand{\shortcode}[1]{\texttt{#1}}

\lstset{style = console}

% Make subsubsection labeled
\setcounter{secnumdepth}{4}
\setcounter{tocdepth}{4}

\newcommand*{\abinitio}{\textit{ab initio}}

\title{First-principle phonon calculation}
\author{Jinyuan Wu}

\begin{document}

\maketitle

\section{Feynman diagrams for phonons}

\subsection{Single-phonon Green function}

The general definition of the Green function for a real bosonic field is 
\begin{equation}
    D(\vb*{q}, t - t') = - \ii \expval*{\timeorder A_{\vb*{q}}(t) A_{\vb*{q}}(t')},
\end{equation}
where 
\begin{equation}
    A_{\vb*{q}} = a_{\vb*{q}} + a^\dagger_{- \vb*{q}}.
\end{equation}
In the non-interactive case, we have
\begin{equation}
    \begin{aligned}
        D^{(0)}(\vb*{q}, t - t') &= - \ii \expval*{\timeorder
            (a_{\vb*{q}} \ee^{- \ii \omega_{\vb*{q}} t} + \text{h.c.})
            (a_{\vb*{q}} \ee^{- \ii \omega_{\vb*{q}} t'} + \text{h.c.})
        } \\
        &= - \ii (\theta(t - t') \ee^{- \ii \omega_{\vb*{q}} (t - t')}
        + \theta(t' - t) \ee^{\ii \omega_{\vb*{q}} (t - t')}).
    \end{aligned}
\end{equation}
Then we can evaluate the Fourier transform of it and get 
\begin{equation}
    D^{(0)}(\vb*{q}, \omega) = \int_{-\infty}^{\infty} \ee^{\ii \omega (t - t')} D^{(0)}(\vb*{q}, t - t')
    = \frac{1}{\omega - \omega_{\vb*{q}} + \ii 0^+} - \frac{1}{\omega - \omega_{\vb*{q}} - \ii 0^+}
    = \frac{2 \omega_{\vb*{q}}}{\omega^2 - \omega_{\vb*{q}}^2 + \ii 0^+}.
\end{equation}
This also works for phonons.
Note the fact that even when we have zero phonon in the ground state, 
we still have two terms in the propagator:
it comes from the fact that $A_{\vb*{q}}$ always contains 
both annihilation and creation operators 
and thus $D^{(0)}$ is always non-zero regardless of the sign of $t - t'$.
Also note the minus sign between $1 / (\omega \pm \omega_{\vb*{q}})$: 
it comes from the fact that we are dealing with bosons and not fermions.
For phonons, $A_{\vb*{q}}$ is proportional to $X_{\vb*{q}}$;
this enables us to link phonon Green function with displacement correlation function.

Just like the case for electrons, 
we can do self-energy correction to the phonon propagator.
Suppose we have already done a Dyson resummation 
and find 
\begin{equation}
    D^{-1}(\vb*{q}, \omega) = (D^{(0)}(\vb*{q}, \omega))^{-1} - \Pi(\vb*{q}, \omega).
\end{equation}
We have 
\begin{equation}
    D(\vb*{q}, \omega) = \left(
        \frac{\omega^2 - \omega_{\vb*{q}}^2 - 2 \omega_{\vb*{q}} \Pi + \ii 0^+}{2 \omega_{\vb*{q}}}
    \right)^{-1}
    = 
\end{equation}

\section{The molecular dynamics approach}

See \cite{zhang2022finite}. 
The idea is rather simple: 
the phonon spectrum just illustrates 
the vibration modes of atoms,
so if we just let atoms move 
(with an average kinetic energy that agrees 
with the temperature $T$ in question),
then the Fourier transform 
of the orbitals of the atoms 
gains high intensity 
on the dispersion relations.
This seems to be the most \abinitio{} method I've ever learned about.

A more efficient way is the so-called temperature-dependent effective potential,
which is obtained by curve-fitting of the atomic force fields in MD.

\printbibliography

\end{document}