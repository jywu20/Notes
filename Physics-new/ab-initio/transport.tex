\documentclass[hyperref, a4paper]{article}

\usepackage{geometry}
\usepackage{titling}
\usepackage{titlesec}
% No longer needed, since we will use enumitem package
% \usepackage{paralist}
\usepackage{enumitem}
\usepackage{footnote}
\usepackage{amsmath, amssymb, amsthm}
\usepackage{mathtools}
\usepackage{bbm}
\usepackage{cite}
\usepackage{graphicx}
\usepackage{subfigure}
\usepackage{physics}
\usepackage{tensor}
\usepackage{siunitx}
\usepackage[version=4]{mhchem}
\usepackage{tikz}
\usepackage{xcolor}
\usepackage{listings}
\usepackage{underscore}
\usepackage{autobreak}
\usepackage[ruled, vlined, linesnumbered]{algorithm2e}
\usepackage{nameref,zref-xr}
\zxrsetup{toltxlabel}
\usepackage[colorlinks,unicode]{hyperref} % , linkcolor=black, anchorcolor=black, citecolor=black, urlcolor=black, filecolor=black
\usepackage[most]{tcolorbox}
\usepackage{prettyref}

% Page style
\geometry{left=3.18cm,right=3.18cm,top=2.54cm,bottom=2.54cm}
\titlespacing{\paragraph}{0pt}{1pt}{10pt}[20pt]
\setlength{\droptitle}{-5em}

% More compact lists 
\setlist[itemize]{
    %itemindent=17pt, 
    %leftmargin=1pt,
    listparindent=\parindent,
    parsep=0pt,
}

\setlist[enumerate]{
    %itemindent=17pt, 
    %leftmargin=1pt,
    listparindent=\parindent,
    parsep=0pt,
}

% Math operators
\DeclareMathOperator{\timeorder}{\mathcal{T}}
\DeclareMathOperator{\diag}{diag}
\DeclareMathOperator{\legpoly}{P}
\DeclareMathOperator{\primevalue}{P}
\DeclareMathOperator{\sgn}{sgn}
\DeclareMathOperator{\res}{Res}
\newcommand*{\ii}{\mathrm{i}}
\newcommand*{\ee}{\mathrm{e}}
\newcommand*{\const}{\mathrm{const}}
\newcommand*{\suchthat}{\quad \text{s.t.} \quad}
\newcommand*{\argmin}{\arg\min}
\newcommand*{\argmax}{\arg\max}
\newcommand*{\normalorder}[1]{: #1 :}
\newcommand*{\pair}[1]{\langle #1 \rangle}
\newcommand*{\fd}[1]{\mathcal{D} #1}
\DeclareMathOperator{\bigO}{\mathcal{O}}

% TikZ setting
\usetikzlibrary{arrows,shapes,positioning}
\usetikzlibrary{arrows.meta}
\usetikzlibrary{decorations.markings}
\usetikzlibrary{calc}
\tikzstyle arrowstyle=[scale=1]
\tikzstyle directed=[postaction={decorate,decoration={markings,
    mark=at position .5 with {\arrow[arrowstyle]{stealth}}}}]
\tikzstyle ray=[directed, thick]
\tikzstyle dot=[anchor=base,fill,circle,inner sep=1pt]

% Algorithm setting
% Julia-style code
\SetKwIF{If}{ElseIf}{Else}{if}{}{elseif}{else}{end}
\SetKwFor{For}{for}{}{end}
\SetKwFor{While}{while}{}{end}
\SetKwProg{Function}{function}{}{end}
\SetArgSty{textnormal}

\newcommand*{\concept}[1]{{\textbf{#1}}}

% Embedded codes
\lstset{basicstyle=\ttfamily,
  showstringspaces=false,
  commentstyle=\color{gray},
  keywordstyle=\color{blue}
}

\lstdefinestyle{console}{
    basicstyle=\footnotesize\ttfamily,
    breaklines=true,
    postbreak=\mbox{\textcolor{red}{$\hookrightarrow$}\space}
}

% Reference formatting
\newrefformat{fig}{Figure~\ref{#1}}

% Color boxes
\tcbuselibrary{skins, breakable, theorems}
\newtcbtheorem[number within=section]{warning}{Warning}%
  {colback=orange!5,colframe=orange!65,fonttitle=\bfseries, breakable}{warn}
\newtcbtheorem[number within=section]{note}{Note}%
  {colback=green!5,colframe=green!65,fonttitle=\bfseries, breakable}{note}
\newtcbtheorem[number within=section]{info}{Info}%
  {colback=blue!5,colframe=blue!65,fonttitle=\bfseries, breakable}{info}

% Displaying texts in bookmarkers

\pdfstringdefDisableCommands{%
  \def\\{}%
  \def\ce#1{<#1>}%
}

\pdfstringdefDisableCommands{%
  \def\texttt#1{<#1>}%
  \def\mathbb#1{#1}%
}
\pdfstringdefDisableCommands{\def\eqref#1{(\ref{#1})}}

\makeatletter
\pdfstringdefDisableCommands{\let\HyPsd@CatcodeWarning\@gobble}
\makeatother

\newenvironment{shelldisplay}{\begin{lstlisting}}{\end{lstlisting}}

\newcommand{\shortcode}[1]{\texttt{#1}}

\lstset{style = console}

\title{Boltzmann equation and the like}
\author{Jinyuan Wu}

\begin{document}

\maketitle

\section{Boltzmann equation and Fermi golden rule}

We define
\[
    \text{\# of band $n$ electrons in $\Delta \Omega, \Delta V$}
    = \frac{\Delta V}{V} \sum_{\vb*{k} \in \Delta \Omega} f_{n} (\vb*{r}, \vb*{k}, t),
    = \Delta V \int_{\Delta \Omega} \frac{\dd[3]{\vb*{k}}}{(2\pi)^3} f_n (\vb*{r}, \vb*{k}, t),
\]
and therefore when the $\vb*{k}$-grid is dense enough, i.e. when $V \to \infty$, '
the single-electron distribution function $f(\vb*{r}, \vb*{k}, t)$ is 
normalized as
\begin{equation}
    \text{\# of band $n$ electrons in $\dd[3]{\vb*{r}} \dd[3]{\vb*{k}}$} 
    = f_n(\vb*{r}, \vb*{k}, t) \dd[3]{\vb*{r}} \frac{\dd[3]{\vb*{k}}}{(2\pi)^3}.
\end{equation}

The Boltzmann equation can be derived from the following intuitive notion:
\[
    \dv{t} \frac{\Delta V}{V} f_n (\vb*{r}, \vb*{k}, t) 
    = \sum_{\text{initial states}} \Gamma_{\text{initial states} \to n \vb*{k}} 
    - \sum_{\text{final states}} \Gamma_{n, \vb*{k} \to \text{final states}} ,
\]
where we have (according to Fermi golden rule)
\begin{equation}
    \Gamma_{1 \to 2} = \frac{2\pi}{\hbar} 
    \abs*{\mel{2}{H_{\text{int}}}{1}}^2 
    \delta(E_2 - E_1).
\end{equation}

\section{Relation with Green function EOM}

Note that in order to obtain a well-defined 
quasiparticle distribution function $f(\vb*{r}, \vb*{k}, t)$,
we assume that the Green function of the system 
is almost without dissipation,
but then there is the collision integral
which clearly introduces dissipation. 
So the real approximation we made here is that
only the real part of the self-energy 
is important in the the diffusion term on the LHS, 
while the main effect of the imaginary part is the collision integral.

It should be noted that the imaginary part is present with $T = 0$.
The physical picture is, 
the eigenstates of a Coulomb-interactive electron gas 
are all compositions of Fock states with varying electron distributions, 
and therefore any single-particle description of the system 
suffers from the loss of information to 2-electron, 3-electron, etc. subspaces,
which is effectively modeled as a finite lifetime of the particle.
This doesn't come from any thermal fluctuation.
Note that this also means the ground state of the system is ``boiling''
in the independent-electron picture;
we therefore may say ``electrons are scattering each other even at the ground state''.
Note, however, that this scattering rate decreases as $\tau \sim 1/(k - k_{\text{F}})^2$ in RPA,
and the single-electron picture 
at least works good enough for bands near the Fermi surface 
in systems for which $GW$ works.

\section{From quantum Boltzmann equation to classical Boltzmann equation}

\begin{equation}
    \left(\pdv{f}{t}\right)_{\text{c}} = 2\pi 
    \sum_{\vb*{k}, \vb*{q}} \frac{\abs*{V(q)}^2}{V^2} 
    \delta\left(
        \frac{
            (\vb*{k} - \vb*{q})^2 + (\vb*{k}_1 + \vb*{q})^2 
            - \vb*{k}_1^2 - \vb*{k}^2
        }{2m}
    \right) \cdot (f_{\vb*{k}} f_{\vb*{k}_1} - f_{\vb*{k} - \vb*{q}} f_{\vb*{k}_1 + \vb*{q}})
\end{equation}

\[
    (\vb*{k} - \vb*{q})^2 + (\vb*{k}_1 + \vb*{q})^2 
    - \vb*{k}_1^2 - \vb*{k}^2
    = 2 q^2 + \vu*{e}_{\vb*{k}_1 - \vb*{k}} \cdot \vu*{q} \abs*{\vb*{k} - \vb*{k}_1} q
\]

\[
    \frac{1}{V^2} \sum_{\vb*{k}, \vb*{q}} \longrightarrow 
    \frac{1}{(2\pi)^6} \int \dd[3]{\vb*{k}} \int \dd[3]{\vb*{q}},
\]

\begin{equation}
    \begin{aligned}
        &\quad \left(\pdv{f}{t}\right)_{\text{c}} \\
        &= \frac{1}{(2\pi)^5 } \int \dd[3]{\vb*{k}_1} \int \dd{\Omega_{\vb*{q}}} q^2 \dd{q}  
        \abs*{V(q)}^2 \delta \left(
            \frac{2 q^2 + \vu*{e}_{\vb*{k}_1 - \vb*{k}} \cdot \vu*{q} \cdot \abs*{\vb*{k} - \vb*{k}_1} q}{m} 
        \right) (f_1' f' - f_1 f) \\
        &= \frac{1}{(2\pi)^5 } \int \dd[3]{\vb*{k}_1} \int \dd{\Omega_{\vb*{q}}} q^2 \cdot \abs*{V(q)}^2 (f_1' f' - f_1 f)  \cdot \eval{\frac{m}{4 q + \vu*{e}_{\vb*{k}_1 - \vb*{k}} \cdot \vu*{q} 
        \abs*{\vb*{k} - \vb*{k}_1}} }_{2 q^2 + \vu*{e}_{\vb*{k}_1 - \vb*{k}} \cdot \vu*{q} \cdot \abs*{\vb*{k} - \vb*{k}_1} q = 0} \\
        &= \frac{1}{(2\pi)^5 } \int \dd[3]{\vb*{k}_1} \int \dd{\Omega_{\vb*{q}}} \abs*{V(q)}^2 
        \cdot \frac{q}{2} m (f_1' f' - f_1 f) |_{2 q = - \vu*{e}_{\vb*{k}_1 - \vb*{k}} \cdot \vu*{q} \cdot \abs*{\vb*{k} - \vb*{k}_1}},
    \end{aligned}
\end{equation}

\begin{equation}
    \sigma = \left(\frac{m V}{2\pi}\right)^2 \abs*{V(q)}^2 
\end{equation}
and since 

\begin{equation}
    \frac{1}{(2\pi)^5} \int \dd[3]{\vb*{k}_1} \int \dd{\Omega_{\vb*{q}}} \sigma 
    \cdot \frac{(2\pi)^2}{m^2} \cdot m \cdot \vu*{e}_{\vb*{k} - \vb*{k}_1} \cdot \vu*{q} \abs*{\vb*{k} - \vb*{k}_1}
\end{equation}

Now we choose $\vu*{e}_{\vb*{k} - \vb*{k}_1}$ as the direction of the $z$ axis; 
the direction of $\vb*{k}' - \vb*{k}_1'$ is described by $(\theta, \varphi)$.

\begin{equation}
    \dd{\Omega_{\vb*{q}}} = \dd{\varphi} \sin \frac{\pi + \theta}{2} \dd \frac{\pi + \theta}{2} 
    = \dd{\varphi} \cdot \frac{1}{4} \cdot \frac{\sin \theta}{\vu*{e}_{\vb*{k} - \vb*{k}'} \cdot \vu*{q}} \dd{\theta}
    = \frac{1}{4} \frac{1}{\vu*{e}_{\vb*{k} - \vb*{k}'} \cdot \vu*{q}} \dd{\Omega}
\end{equation}

so now we arrive at the usual classical Boltzmann equation, 
where $\dd{\Omega}$ corresponds to the direction of $\vb*{k}' - \vb*{k}'_1$.

But now we have a 1/16 factor \dots

Also see Landau vol. 10 (2.2), 
where the equivalence between the two kinds of collision integrals 
is displayed without proof

\section{Phonon}

The vertex of electron-phonon interaction has the form $\abs*{g / \sqrt{V}}^2$,
where $g$ is a volume independent quantity;
the collision integral is therefore 
\begin{equation}
    \left(\pdv{f}{t}\right)_{\text{c}} = 
    \frac{2\pi}{V} \sum_{\vb*{k}, \lambda} \abs*{g}^2 
\end{equation}
Similarly, for Coulomb scattering, 
the vertex looks like $V(q) / V$, where the $V(q)$ factor 
is the $4 \pi e^2 / q^2$ interaction potential, 
and the Fermi golden rule -- and therefore the collision integral -- 
has a $1/ V^2$ factor.

\end{document}