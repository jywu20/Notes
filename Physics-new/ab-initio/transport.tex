\documentclass[hyperref, a4paper]{article}

\usepackage{geometry}
\usepackage{titling}
\usepackage{titlesec}
% No longer needed, since we will use enumitem package
% \usepackage{paralist}
\usepackage{enumitem}
\usepackage{footnote}
\usepackage{amsmath, amssymb, amsthm}
\usepackage{mathtools}
\usepackage{bbm}
\usepackage{graphicx}
\usepackage{subfigure}
\usepackage{physics}
\usepackage{tensor}
\usepackage{siunitx}
\usepackage[version=4]{mhchem}
\usepackage{tikz}
\usepackage{xcolor}
\usepackage{listings}
\usepackage{underscore}
\usepackage{autobreak}
\usepackage[ruled, vlined, linesnumbered]{algorithm2e}
\usepackage[sorting=none]{biblatex}
\addbibresource{boltzmann.bib}
\usepackage{nameref,zref-xr}
\zxrsetup{toltxlabel}
\usepackage[colorlinks,unicode]{hyperref} % , linkcolor=black, anchorcolor=black, citecolor=black, urlcolor=black, filecolor=black
\usepackage[most]{tcolorbox}
\usepackage{prettyref}

% Page style
\geometry{left=3.18cm,right=3.18cm,top=2.54cm,bottom=2.54cm}
\titlespacing{\paragraph}{0pt}{1pt}{10pt}[20pt]
\setlength{\droptitle}{-5em}

% More compact lists 
\setlist[itemize]{
    %itemindent=17pt, 
    %leftmargin=1pt,
    listparindent=\parindent,
    parsep=0pt,
}

\setlist[enumerate]{
    %itemindent=17pt, 
    %leftmargin=1pt,
    listparindent=\parindent,
    parsep=0pt,
}

% Math operators
\DeclareMathOperator{\timeorder}{\mathcal{T}}
\DeclareMathOperator{\diag}{diag}
\DeclareMathOperator{\legpoly}{P}
\DeclareMathOperator{\primevalue}{P}
\DeclareMathOperator{\sgn}{sgn}
\DeclareMathOperator{\res}{Res}
\newcommand*{\ii}{\mathrm{i}}
\newcommand*{\ee}{\mathrm{e}}
\newcommand*{\const}{\mathrm{const}}
\newcommand*{\suchthat}{\quad \text{s.t.} \quad}
\newcommand*{\argmin}{\arg\min}
\newcommand*{\argmax}{\arg\max}
\newcommand*{\normalorder}[1]{: #1 :}
\newcommand*{\pair}[1]{\langle #1 \rangle}
\newcommand*{\fd}[1]{\mathcal{D} #1}
\DeclareMathOperator{\bigO}{\mathcal{O}}

% TikZ setting
\usetikzlibrary{arrows,shapes,positioning}
\usetikzlibrary{arrows.meta}
\usetikzlibrary{decorations.markings}
\usetikzlibrary{calc}
\tikzstyle arrowstyle=[scale=1]
\tikzstyle directed=[postaction={decorate,decoration={markings,
    mark=at position .5 with {\arrow[arrowstyle]{stealth}}}}]
\tikzstyle ray=[directed, thick]
\tikzstyle dot=[anchor=base,fill,circle,inner sep=1pt]

% Algorithm setting
% Julia-style code
\SetKwIF{If}{ElseIf}{Else}{if}{}{elseif}{else}{end}
\SetKwFor{For}{for}{}{end}
\SetKwFor{While}{while}{}{end}
\SetKwProg{Function}{function}{}{end}
\SetArgSty{textnormal}

\newcommand*{\concept}[1]{{\textbf{#1}}}

% Embedded codes
\lstset{basicstyle=\ttfamily,
  showstringspaces=false,
  commentstyle=\color{gray},
  keywordstyle=\color{blue}
}

\lstdefinestyle{console}{
    basicstyle=\footnotesize\ttfamily,
    breaklines=true,
    postbreak=\mbox{\textcolor{red}{$\hookrightarrow$}\space}
}

% Reference formatting
\newrefformat{fig}{Figure~\ref{#1}}

% Color boxes
\tcbuselibrary{skins, breakable, theorems}
\newtcbtheorem[number within=section]{warning}{Warning}%
  {colback=orange!5,colframe=orange!65,fonttitle=\bfseries, breakable}{warn}
\newtcbtheorem[number within=section]{note}{Note}%
  {colback=green!5,colframe=green!65,fonttitle=\bfseries, breakable}{note}
\newtcbtheorem[number within=section]{info}{Info}%
  {colback=blue!5,colframe=blue!65,fonttitle=\bfseries, breakable}{info}

% Displaying texts in bookmarkers

\pdfstringdefDisableCommands{%
  \def\\{}%
  \def\ce#1{<#1>}%
}

\pdfstringdefDisableCommands{%
  \def\texttt#1{<#1>}%
  \def\mathbb#1{#1}%
}
\pdfstringdefDisableCommands{\def\eqref#1{(\ref{#1})}}

\makeatletter
\pdfstringdefDisableCommands{\let\HyPsd@CatcodeWarning\@gobble}
\makeatother

\newenvironment{shelldisplay}{\begin{lstlisting}}{\end{lstlisting}}

\newcommand{\shortcode}[1]{\texttt{#1}}

\lstset{style = console}

\title{Boltzmann equation and the like}
\author{Jinyuan Wu}

\begin{document}

\maketitle

\section{Kadanoff-Baym equation}

In the non-equilibrium case, 
we still have 
\begin{equation}
    \begin{aligned}
        \expval{A_{\text{H}}} &= \expval{  U(-\infty, t) A_{\text{I}}(t) U(t, -\infty) }_{0} \\
        &= \expval*{  \underbrace{U(-\infty, \infty)}_{S^{-1}} U(\infty, t) A_{\text{I}}(t) U(t, -\infty) }_0 \\
        &= \expval*{S^{-1} \timeorder_t ( S A_{\text{I}}(t) )}_0,
    \end{aligned}
\end{equation}
where the subscript $_0$ means the initial state.
Therefore we can still define the time-ordered Green function 
\begin{equation}
    \ii G_{12} = \expval*{S^{-1} \timeorder_t (\psi_1 \psi_2^\dagger S) },
\end{equation}
in hope of building something similar to the equilibrium field theory.
In the equilibrium case, 
we can always alternate $S$ into such a time evolution operator 
in which the interaction is turned off when $t = \pm \infty$,
and is only turned on near $t_{1,2}$,
and since the initial state is the ground state, 
we expect 
\begin{equation}
    S \ket*{\text{unperturbed GS}} = \ee^{\ii \cdot \text{some factor}} \ket*{\text{unperturbed GS}},
\end{equation}
and therefore we can get $S^{-1}$ out of the expectation brackets 
and collect its contribution into vacuum bubble diagrams.
In the non-equilibrium case, however,
the initial state may be a mixed state, 
which contains some components of excited states, 
and if we apply $S$ to such an state, 
the result is a mixture of several excited states 
due to scattering.
Thus, after Wick theorem is established, 
we need to deal with contributions from both $S$ and $S^{-1}$.
Contributions from $S^{-1}$ essentially involve 
\emph{anti}-time ordering,
and contraction between terms in $S^{-1}$ and $S$ also appears, 
which involves no time ordering.
Thus in the field theory pertaining to the non-equilibrium case, 
we need \emph{four} kinds of propagators:
the time ordered Green function, 
the anti-time ordered Green function,
and the ``lesser'' and ``greater'' Green functions,
which are proportional to $\expval*{\psi_2^\dagger \psi_1}$ and $\expval*{\psi_1 \psi_2^\dagger}$;
to label these Green functions, 
we just need to add an additional $\pm$ label 
at each end of the propagator.

So now the Dyson equation looks like 
\begin{equation}
    G = G_0 + G_0 \Sigma G, 
\end{equation}
where 
\begin{equation}
    G = \pmqty{
        G^{--} & G^{-+} \\ 
        G^{+-} & G^{++}
    }   ,
\end{equation}
and $G^{--}$ is the time-ordered Green function,
$G^{++}$ the anti-time ordered Green function,
$G^{-+}$ the lesser Green function,
and $G^{+-}$ the greater Green function.
Note that an interaction vertex is either completely connected to $-$ ends
or completely connected to to $+$ ends:
the former comes from $S$,
and the letter comes from $S^{-1}$;
we can also expect they differ with their signs.
The input line of $\Sigma_{+-}$ must bear the label $-$,
and the output end must bear the label $+$.

Now we can further simply the above scheme by working on 
not the physical time axis from $-\infty$ to $\infty$,
but the Keldysh contour that goes from $-\infty$ to $\infty$ 
and then goes back \cite{kremp2005quantum}.
We give the branch from $-\infty$ to $\infty$ an infinitesimal \emph{positive} imaginary part,
and the branch from $\infty$ to $-\infty$ an infinitesimal \emph{negative} imaginary part.
Now the information in $G$ is faithfully reconstructed by 
\begin{equation}
    \ii G = \expval*{S^{-1} \timeorder_{\mathcal{C}} \psi_1 \psi_2^\dagger},
\end{equation}
where $\timeorder_{\mathcal{C}}$ does time ordering on the Keldysh contour,
not the physical time axis:
we can show that 
\begin{equation}
    G^{--}_{12} = G(t_1^+, t_2^+), 
\end{equation}
and so on.
So now we can also obtain the Dyson equation for this new $G_{12}$,
in which the convolution between two diagrammatic components 
involves complex integral on the Keldysh contour 
and not just the physical time axis.
Based on the complex contour and the complex time order operator,
we are able to prove the Lengreth theorem \cite{vspivcka2005long}:
\begin{equation}
    (AB)^< = A^{\text{R}} B^< + A^< B^{\text{A}}, \quad 
    (AB)^> = A^{\text{R}} B^> + A^> B^{\text{A}}, \quad 
    (AB)^{\text{R}} = A^{\text{R}} B^{\text{R}}, \quad 
    (AB)^{\text{A}} = A^{\text{A}} B^{\text{A}},
\end{equation}
where 
\begin{equation}
    A^>(t_1, t_2) = A(t_1^+, t_2^-), \quad 
    A^<(t_1, t_2) = A(t_1^-, t_2^+), \quad 
    A^{\text{R}}(t_1, t_2) = \theta(t_1 - t_2) (A^> - A^<),
    \label{eq:def-l-g-a-r}
\end{equation}
following the usual definition in equilibrium field theory 
(just replace $A$ by $G$ and all equations make sense).
In each of the four equations above, 
the product in the LHS corresponds to 
convolution on the Keldysh contour,
and the product in the RHS corresponds to 
convolution on the physical time axis.

Note that the Lengreth theorem is a rather non-trivial result: 
it means, for example, we have 
\begin{equation}
    G^{\text{A}} = G_0^{ \text{A}} + G_0^{ \text{A}} \Sigma^{\text{A}} G^{\text{A}},
\end{equation}
and both $G^{\text{A}}$ and $\Sigma^{\text{A}}$
can be constructed using the equation in \eqref{eq:def-l-g-a-r} about $A^<$.
This can be explicitly verified by 
doing a linear transform to the matrix version of the Dyson equation 
\cite{rammer1986quantum}.
Specifically, we have Eq. (2.71) in \cite{rammer1986quantum}.

By applying the Lengreth theorem to the Dyson equation, we get 
\cite{rammer1986quantum,vspivcka2005long}
\begin{equation}
    G^< = G^<_0 + G^<_0 \Sigma^{\text{A}} G^{\text{A}}
    + G_0^{\text{R}} \Sigma^{\text{R}} G^< 
    + G_0^{\text{R}} \Sigma^{<} G^\text{A} ,
\end{equation}
and if we apply (note that different Green functions are reduced to 
the choices used in contour integral when doing Fourier transform of $1 / (E - H_0)$)
\begin{equation}
    G_0^{-1} = \ii \pdv{t} - H
\end{equation}
to the above equation, we get 
\begin{equation}
    G_0^{-1} G^< = \Sigma^{\text{R}} G^< + \Sigma^< \Sigma^{\text{A}},
\end{equation}
since $G_0^{-1}$ kills $G^<$ and $G^>$.
A similar equation can be obtained for $G^>$.
As for $G^{\text{A}}$, we have 
\begin{equation}
    G^{\text{A}} = G^{\text{A}}_0 + G^{\text{A}}_0 \Sigma^{\text{A}} G^{\text{A}}.
\end{equation}
Thus we have already gotten EOMs for non-equilibrium Green functions 
which are always right
and form a close system of equations
when the self-energy contains no higher correlation: 
$\Sigma^{\text{A}, \text{R}}$ can all be obtained from $\Sigma^{<,>}$,
which then can be obtained from $G^{<, >}$.
Of course, numerically solving the equation system can be tedious. 
Having an approximate self-energy that is good enough
is also not a trivial task and is not possible 
when higher-order electron correlation is important.

Usually several further approximations -- apart from 
the condition that no higher correlation is taken into account --
are needed to simplify the equations.
One major approximation is to discard all retardation effects in the self-energy,
and the EOMs then only contain one time variable,
instead of two.
This of course breaks when the external driving frequency is high.

The Kadanoff-Baym equation can be further made into a quantum master equation 
governing the reduced single-electron density matrix, 
using some version of Kadanoff-Baym ansatz 
to get rid of the double time dependence of $\Sigma$
\cite{lipavsky1986generalized,PhysRevB.92.205304,vspivcka2005long,hermanns2013few}.
At this stage, all retardation effects in the self-energy 
are collected into the ``dissipation'' part;
the temporally local part of the self-energy 
becomes a part of the effective Hamiltonian,
the latter, for example, may be the static COHSEX approximation,
which is the static limit of the (accurate) COHSEX decomposition
\cite{faber2014electronic}.
If this approximate form of $\Sigma$ is used,
then we get a quantum master equation 
with no quantum jump channels,
which is known as time-dependent adiabatic $GW$ \cite{attaccalite2011real,chan2021giant}.

Here I have a question: 
there seems to be two types of approximations 
if we want to have a quantum master equation without dissipation terms:
$\Im \Sigma = 0$, 
and $\Sigma$ has no retardation.
Are the two conditions equivalent to each other?
The no retardation condition is equivalent to the no frequency dependence condition,

Problem reading \cite{vspivcka2005long}:
in (44) and (45), 
the LHS contains only $H_0$,
but for a realistic quantum master equation,
the LHS should also contain the ``real'' part of $\Sigma$;
on the other hand in (61),
the (time-dependent) lesser Green function 
has poles at the self-energy corrected dispersion line.
But then see (19), where the ``mean-field part'' of $\Sigma$
is included into $H_0$?
It seems mass renormalization is also included in the RHS of the 

Also note that mass renormalization isn't as simple as I previously thought
-- see discussions around the role of $\sigma$ in the paper.

\section{Boltzmann equation and Fermi golden rule}

We define
\[
    \text{\# of band $n$ electrons in $\Delta \Omega, \Delta V$}
    = \frac{\Delta V}{V} \sum_{\vb*{k} \in \Delta \Omega} f_{n} (\vb*{r}, \vb*{k}, t),
    = \Delta V \int_{\Delta \Omega} \frac{\dd[3]{\vb*{k}}}{(2\pi)^3} f_n (\vb*{r}, \vb*{k}, t),
\]
and therefore when the $\vb*{k}$-grid is dense enough, i.e. when $V \to \infty$, '
the single-electron distribution function $f(\vb*{r}, \vb*{k}, t)$ is 
normalized as
\begin{equation}
    \text{\# of band $n$ electrons in $\dd[3]{\vb*{r}} \dd[3]{\vb*{k}}$} 
    = f_n(\vb*{r}, \vb*{k}, t) \dd[3]{\vb*{r}} \frac{\dd[3]{\vb*{k}}}{(2\pi)^3}.
\end{equation}

The Boltzmann equation can be derived from the following intuitive notion:
\[
    \dv{t} \frac{\Delta V}{V} f_n (\vb*{r}, \vb*{k}, t) 
    = \sum_{\text{initial states}} \Gamma_{\text{initial states} \to n \vb*{k}} 
    - \sum_{\text{final states}} \Gamma_{n, \vb*{k} \to \text{final states}} ,
\]
where we have (according to Fermi golden rule)
\begin{equation}
    \Gamma_{1 \to 2} = \frac{2\pi}{\hbar} 
    \abs*{\mel{2}{H_{\text{int}}}{1}}^2 
    \delta(E_2 - E_1).
\end{equation}

\section{Relation with Green function EOM}

Note that in order to obtain a well-defined 
quasiparticle distribution function $f(\vb*{r}, \vb*{k}, t)$,
we assume that the Green function of the system 
is almost without dissipation,
but then there is the collision integral
which clearly introduces dissipation. 
So the real approximation we made here is that
only the real part of the self-energy 
is important in the the diffusion term on the LHS, 
while the main effect of the imaginary part is the collision integral.

It should be noted that the imaginary part is present with $T = 0$.
The physical picture is, 
the eigenstates of a Coulomb-interactive electron gas 
are all compositions of Fock states with varying electron distributions, 
and therefore any single-particle description of the system 
suffers from the loss of information to 2-electron, 3-electron, etc. subspaces,
which is effectively modeled as a finite lifetime of the particle.
This doesn't come from any thermal fluctuation.
Note that this also means the ground state of the system is ``boiling''
in the independent-electron picture;
we therefore may say ``electrons are scattering each other even at the ground state''.
Note, however, that this scattering rate decreases as $\tau \sim 1/(k - k_{\text{F}})^2$ in RPA,
and the single-electron picture 
at least works good enough for bands near the Fermi surface 
in systems for which $GW$ works.

\section{From quantum Boltzmann equation to classical Boltzmann equation}

\begin{equation}
    \left(\pdv{f}{t}\right)_{\text{c}} = 2\pi 
    \sum_{\vb*{k}, \vb*{q}} \frac{\abs*{V(q)}^2}{V^2} 
    \delta\left(
        \frac{
            (\vb*{k} - \vb*{q})^2 + (\vb*{k}_1 + \vb*{q})^2 
            - \vb*{k}_1^2 - \vb*{k}^2
        }{2m}
    \right) \cdot (f_{\vb*{k}} f_{\vb*{k}_1} - f_{\vb*{k} - \vb*{q}} f_{\vb*{k}_1 + \vb*{q}})
\end{equation}

\[
    (\vb*{k} - \vb*{q})^2 + (\vb*{k}_1 + \vb*{q})^2 
    - \vb*{k}_1^2 - \vb*{k}^2
    = 2 q^2 + \vu*{e}_{\vb*{k}_1 - \vb*{k}} \cdot \vu*{q} \abs*{\vb*{k} - \vb*{k}_1} q
\]

\[
    \frac{1}{V^2} \sum_{\vb*{k}, \vb*{q}} \longrightarrow 
    \frac{1}{(2\pi)^6} \int \dd[3]{\vb*{k}} \int \dd[3]{\vb*{q}},
\]

\begin{equation}
    \begin{aligned}
        &\quad \left(\pdv{f}{t}\right)_{\text{c}} \\
        &= \frac{1}{(2\pi)^5 } \int \dd[3]{\vb*{k}_1} \int \dd{\Omega_{\vb*{q}}} q^2 \dd{q}  
        \abs*{V(q)}^2 \delta \left(
            \frac{2 q^2 + \vu*{e}_{\vb*{k}_1 - \vb*{k}} \cdot \vu*{q} \cdot \abs*{\vb*{k} - \vb*{k}_1} q}{m} 
        \right) (f_1' f' - f_1 f) \\
        &= \frac{1}{(2\pi)^5 } \int \dd[3]{\vb*{k}_1} \int \dd{\Omega_{\vb*{q}}} q^2 \cdot \abs*{V(q)}^2 (f_1' f' - f_1 f)  \cdot \eval{\frac{m}{4 q + \vu*{e}_{\vb*{k}_1 - \vb*{k}} \cdot \vu*{q} 
        \abs*{\vb*{k} - \vb*{k}_1}} }_{2 q^2 + \vu*{e}_{\vb*{k}_1 - \vb*{k}} \cdot \vu*{q} \cdot \abs*{\vb*{k} - \vb*{k}_1} q = 0} \\
        &= \frac{1}{(2\pi)^5 } \int \dd[3]{\vb*{k}_1} \int \dd{\Omega_{\vb*{q}}} \abs*{V(q)}^2 
        \cdot \frac{q}{2} m (f_1' f' - f_1 f) |_{2 q = - \vu*{e}_{\vb*{k}_1 - \vb*{k}} \cdot \vu*{q} \cdot \abs*{\vb*{k} - \vb*{k}_1}},
    \end{aligned}
\end{equation}

\begin{equation}
    \sigma = \left(\frac{m V}{2\pi}\right)^2 \abs*{V(q)}^2 
\end{equation}
and since 

\begin{equation}
    \frac{1}{(2\pi)^5} \int \dd[3]{\vb*{k}_1} \int \dd{\Omega_{\vb*{q}}} \sigma 
    \cdot \frac{(2\pi)^2}{m^2} \cdot m \cdot \vu*{e}_{\vb*{k} - \vb*{k}_1} \cdot \vu*{q} \abs*{\vb*{k} - \vb*{k}_1}
\end{equation}

Now we choose $\vu*{e}_{\vb*{k} - \vb*{k}_1}$ as the direction of the $z$ axis; 
the direction of $\vb*{k}' - \vb*{k}_1'$ is described by $(\theta, \varphi)$.

\begin{equation}
    \dd{\Omega_{\vb*{q}}} = \dd{\varphi} \sin \frac{\pi + \theta}{2} \dd \frac{\pi + \theta}{2} 
    = \dd{\varphi} \cdot \frac{1}{4} \cdot \frac{\sin \theta}{\vu*{e}_{\vb*{k} - \vb*{k}'} \cdot \vu*{q}} \dd{\theta}
    = \frac{1}{4} \frac{1}{\vu*{e}_{\vb*{k} - \vb*{k}'} \cdot \vu*{q}} \dd{\Omega}
\end{equation}

so now we arrive at the usual classical Boltzmann equation, 
where $\dd{\Omega}$ corresponds to the direction of $\vb*{k}' - \vb*{k}'_1$.

But now we have a 1/16 factor \dots

Also see Landau vol. 10 (2.2), 
where the equivalence between the two kinds of collision integrals 
is displayed without proof

\section{Phonon}

The vertex of electron-phonon interaction has the form $\abs*{g / \sqrt{V}}^2$,
where $g$ is a volume independent quantity;
the collision integral is therefore 
\begin{equation}
    \left(\pdv{f}{t}\right)_{\text{c}} = 
    \frac{2\pi}{V} \sum_{\vb*{k}, \lambda} \abs*{g}^2 
\end{equation}
Similarly, for Coulomb scattering, 
the vertex looks like $V(q) / V$, where the $V(q)$ factor 
is the $4 \pi e^2 / q^2$ interaction potential, 
and the Fermi golden rule -- and therefore the collision integral -- 
has a $1/ V^2$ factor.

\printbibliography

\end{document}