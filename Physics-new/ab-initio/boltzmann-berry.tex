\documentclass[hyperref, a4paper]{article}

\usepackage{geometry}
\usepackage{titling}
\usepackage{titlesec}
% No longer needed, since we will use enumitem package
% \usepackage{paralist}
\usepackage{enumitem}
\usepackage{footnote}
\usepackage{amsmath, amssymb, amsthm}
\usepackage{mathtools}
\usepackage{bbm}
\usepackage{graphicx}
\usepackage{subfigure}
\usepackage{physics}
\usepackage{tensor}
\usepackage{siunitx}
\usepackage[version=4]{mhchem}
\usepackage{tikz}
\usepackage{xcolor}
\usepackage{listings}
\usepackage{underscore}
\usepackage{autobreak}
\usepackage[ruled, vlined, linesnumbered]{algorithm2e}
\usepackage[sorting=none]{biblatex}
\addbibresource{boltzmann.bib}
\usepackage{nameref,zref-xr}
\zxrsetup{toltxlabel}
\usepackage[colorlinks,unicode]{hyperref} % , linkcolor=black, anchorcolor=black, citecolor=black, urlcolor=black, filecolor=black
\usepackage[most]{tcolorbox}
\usepackage{prettyref}

% Page style
\geometry{left=3.18cm,right=3.18cm,top=2.54cm,bottom=2.54cm}
\titlespacing{\paragraph}{0pt}{1pt}{10pt}[20pt]
\setlength{\droptitle}{-5em}

% More compact lists 
\setlist[itemize]{
    %itemindent=17pt, 
    %leftmargin=1pt,
    listparindent=\parindent,
    parsep=0pt,
}

\setlist[enumerate]{
    %itemindent=17pt, 
    %leftmargin=1pt,
    listparindent=\parindent,
    parsep=0pt,
}

% Math operators
\DeclareMathOperator{\timeorder}{\mathcal{T}}
\DeclareMathOperator{\diag}{diag}
\DeclareMathOperator{\legpoly}{P}
\DeclareMathOperator{\primevalue}{P}
\DeclareMathOperator{\sgn}{sgn}
\DeclareMathOperator{\res}{Res}
\newcommand*{\ii}{\mathrm{i}}
\newcommand*{\ee}{\mathrm{e}}
\newcommand*{\const}{\mathrm{const}}
\newcommand*{\suchthat}{\quad \text{s.t.} \quad}
\newcommand*{\argmin}{\arg\min}
\newcommand*{\argmax}{\arg\max}
\newcommand*{\normalorder}[1]{: #1 :}
\newcommand*{\pair}[1]{\langle #1 \rangle}
\newcommand*{\fd}[1]{\mathcal{D} #1}
\DeclareMathOperator{\bigO}{\mathcal{O}}

% TikZ setting
\usetikzlibrary{arrows,shapes,positioning}
\usetikzlibrary{arrows.meta}
\usetikzlibrary{decorations.markings}
\usetikzlibrary{calc}
\tikzstyle arrowstyle=[scale=1]
\tikzstyle directed=[postaction={decorate,decoration={markings,
    mark=at position .5 with {\arrow[arrowstyle]{stealth}}}}]
\tikzstyle ray=[directed, thick]
\tikzstyle dot=[anchor=base,fill,circle,inner sep=1pt]

% Algorithm setting
% Julia-style code
\SetKwIF{If}{ElseIf}{Else}{if}{}{elseif}{else}{end}
\SetKwFor{For}{for}{}{end}
\SetKwFor{While}{while}{}{end}
\SetKwProg{Function}{function}{}{end}
\SetArgSty{textnormal}

\newcommand*{\concept}[1]{{\textbf{#1}}}

% Embedded codes
\lstset{basicstyle=\ttfamily,
  showstringspaces=false,
  commentstyle=\color{gray},
  keywordstyle=\color{blue}
}

\lstdefinestyle{console}{
    basicstyle=\footnotesize\ttfamily,
    breaklines=true,
    postbreak=\mbox{\textcolor{red}{$\hookrightarrow$}\space}
}

% Reference formatting
\newrefformat{fig}{Figure~\ref{#1}}

% Color boxes
\tcbuselibrary{skins, breakable, theorems}
\newtcbtheorem[number within=section]{warning}{Warning}%
  {colback=orange!5,colframe=orange!65,fonttitle=\bfseries, breakable}{warn}
\newtcbtheorem[number within=section]{note}{Note}%
  {colback=green!5,colframe=green!65,fonttitle=\bfseries, breakable}{note}
\newtcbtheorem[number within=section]{info}{Info}%
  {colback=blue!5,colframe=blue!65,fonttitle=\bfseries, breakable}{info}

% Displaying texts in bookmarkers

\pdfstringdefDisableCommands{%
  \def\\{}%
  \def\ce#1{<#1>}%
}

\pdfstringdefDisableCommands{%
  \def\texttt#1{<#1>}%
  \def\mathbb#1{#1}%
}
\pdfstringdefDisableCommands{\def\eqref#1{(\ref{#1})}}

\makeatletter
\pdfstringdefDisableCommands{\let\HyPsd@CatcodeWarning\@gobble}
\makeatother

\newenvironment{shelldisplay}{\begin{lstlisting}}{\end{lstlisting}}

\newcommand{\shortcode}[1]{\texttt{#1}}

\lstset{style = console}

\title{Bloch wave function in quantum Boltzmann equation}
\author{Jinyuan Wu}

\begin{document}

\maketitle

There are several ways to build QBE in a crystal: 
\begin{enumerate}
    \item The most naive way is to insert the 
    band structure $\epsilon_{n \vb*{k}}$ into the LHS of QBE: 
    thus we get corrected velocity.
    The main problem of this approach is it  
    ignores the influence of the Bloch wave function; 
    since the eigenstates in the crystal are no longer plane waves, 
    defining a distribution function about $\vb*{r}$ and $\vb*{k}$
    can't be done by naively doing Wigner transform.
    \item One way to improve the situation is to 
    start from the so-called Bloch semi-classic EOM about $\vb*{r}$ and $\vb*{k}$; 
    we insert $\dot{\vb*{r}}$ and $\dot{\vb*{k}}$ given by the Bloch equations 
    into the LHS of QBE. 

    This level of approximation is equivalent to building the QBE 
    based on the effective Hamiltonian about the wave packet envelope. 
    However, it doesn't include the effects of the Bloch wave functions 
    to the collision term. 

    \item How to incorporate the effects of the Bloch wave functions into the collision term?
    Some kind of ``wave packet envelope'' seems to be needed, 
    or otherwise gradient expansion is ill-defined; 
    should we work in this basis (mapping Bloch state to plane waves, etc.)?

    \item Further questions: hopping between bands 
        (how frequent are they? Do we need $f_{n_1 n_2}(\vb*{r}, \vb*{k}, t)$? etc.)
\end{enumerate}

\section{Describing the wave packet in one band}

In this section we ignore Coulomb scattering.
Also, below we consider wave packet in one band first; 
the band index $n$ is therefore fixed and not summed over.

\subsection{No electric field coupling}

Suppose we have a wave packet 
\begin{equation}
    \psi(\vb*{r}, t) = \sum_{\vb*{k}} c_{\vb*{k}}(t) \psi_{n \vb*{k}}(\vb*{r}).
    = \sum_{\vb*{k}} c_{\vb*{k}}(t) u_{n \vb*{k}}(\vb*{r}) \ee^{\ii \vb*{k} \cdot \vb*{r}}.
\end{equation}
We consider the information of the spatial structure of $\psi$
within a primitive unit cell as unimportant; 
in other words, we consider Fourier components with 
the wave vector being outside the 1BZ as unimportant.
This filtering means the envelope of the wave packet is 
\begin{equation}
    \psi^{\text{envelope}}(\vb*{r}, t)
    = \sum_{\vb*{k}} c_{\vb*{k}}(t) \ee^{\ii \vb*{k} \cdot \vb*{r}}, 
\end{equation}
where formally we set $u_{n \vb*{k}}$ to unity.

We first deal with the case with no external driving electric field. 
In this case the total Hamiltonian is just the band Hamiltonian, 
and we have 
\begin{equation}
    \ii \partial_t \psi = H \psi \Rightarrow
    \ii \partial_t c_{\vb*{k}} = \varepsilon_{n \vb*{k}} c_{\vb*{k}} \Rightarrow
    \ii \partial_t \psi^{\text{envelope}} = \varepsilon_{n \vb*{k}}(\vb*{k} \to - \ii \grad) \psi^{\text{envelope}} .
\end{equation}
The last line can be proved by properties of Fourier transform.
Thus the effective Hamiltonian of the envelope function 
can be obtained by replacing $\vb*{k}$ with $- \ii \grad$ 
in the band energy $\varepsilon_{n \vb*{k}}$.
Thus, we conclude that when scattering is not considered 
and the effects of the external driving field is small, 
the naive approach of treating $\vb*{k}$ as the real momentum 
and replacing $\vb*{k}^2 / 2m$ by $\varepsilon_{n \vb*{k}}$
is accurate enough.

\subsection{Electric field coupling}

When we do have electric field coupling things are different, 
since the coupling term $e \vb*{r} \cdot \vb*{E}$%
\footnote{
    Here the convention is $e > 0$, 
    and thus 
    \begin{equation}
        q \varphi \approx q \vb*{r} \cdot \grad \varphi 
        = - q \vb*{r} \cdot \vb*{E} = e \vb*{r} \cdot \vb*{E}.
    \end{equation}
}
doesn't necessarily appear to be the same 
in the effective Hamiltonian of the wave packets. 
The Schrodinger equation now is 
\begin{equation}
    \ii \partial_t \psi = H^{\text{band}} \psi + e \vb*{r} \cdot \vb*{E} \psi
    \Rightarrow \ii \partial_t c_{\vb*{k}} = \varepsilon_{n \vb*{k}} c_{\vb*{k}}
    + \sum_{\vb*{k}'} e \vb*{E} \cdot 
    \mel**{\psi_{n \vb*{k}}}{\vb*{r}}{\psi_{n \vb*{k}'}} c_{\vb*{k}'},
\end{equation}
where we have assumed that $\vb*{E}$ changes slowly enough 
so it can be extracted out of the expectation brackets;
this condition is needed for Boltzmann transportation anyway, 
or otherwise gradient expansion fails. 
The second term in the RHS of the second equation 
is notorious for being ill-defined:
searching for a sensible definition of this term 
gave rise to the theory of band topology.
To make things easier at the first step, 
let's do decomposition 
\begin{equation}
    \vb*{r} \to \vb*{r}^{\text{envelope}} + \vb*{x},
\end{equation}
where $\vb*{r}^{\text{envelope}}$ is the center of the wave packet. 
So we have 
\begin{equation}
    \sum_{\vb*{k}'} e \vb*{E} \cdot 
    \mel**{\psi_{n \vb*{k}}}{\vb*{r}}{\psi_{n \vb*{k}'}} c_{\vb*{k}'}
    = e \vb*{E} \cdot \vb*{r}^{\text{envelope}} c_{\vb*{k}}
    + \sum_{\vb*{k}'} e \vb*{E} \cdot 
    \mel**{\psi_{n \vb*{k}}}{\vb*{x}}{\psi_{n \vb*{k}'}} c_{\vb*{k}'},
\end{equation}
where only the first term is non-zero in the free electron case.
Since the $\vb*{x}$ term vanishes in a free electron gas, 
we can evaluate this term using the same techniques 
used when establishing relation between 
polarization and Berry curvature. 
We also change the normalization convention so that 
\begin{equation}
    \psi_{n \vb*{k}} = \frac{1}{\sqrt{N}} \ee^{\ii \vb*{k} \cdot \vb*{r}} u_{n \vb*{k}}.
\end{equation}
We have 
\begin{equation}
    \begin{aligned}
        \mel**{\psi_{n \vb*{k}}}{\vb*{x}}{\psi_{n \vb*{k}'}}
        &= \frac{1}{N} \sum_{\vb*{R}} \int_{\text{u.c.}} \dd[d]{\vb*{x}} 
        \ee^{- \ii \vb*{k} \cdot (\vb*{x} + \vb*{R})} u^*_{n \vb*{k}}(\vb*{x})
        \cdot (\vb*{x} + \vb*{R}) \cdot 
        \ee^{\ii \vb*{k}' \cdot (\vb*{x} + \vb*{R})} u_{n \vb*{k}'}(\vb*{x}) \\
        &= \frac{1}{N} \sum_{\vb*{R}} \ee^{\ii \vb*{R} \cdot (\vb*{k}' - \vb*{k})}
        \int_{\text{u.c.}} \dd[d]{\vb*{x}} 
        \ee^{- \ii \vb*{k} \cdot \vb*{x} } u^*_{n \vb*{k}}(\vb*{x})
        \cdot (\vb*{x} + \vb*{R}) \cdot 
        \ee^{\ii \vb*{k}' \cdot \vb*{x} } u_{n \vb*{k}'}(\vb*{x})  .
    \end{aligned}
    \label{eq:erE-inter-1}
\end{equation}
The first line is obtained by expanding $\vb*{x}$ into $\vb*{x} + \vb*{R}$, 
where $\vb*{x}$ is confined in the primitive unit cell. 
The integral over $\vb*{x}$ can be split into one term with
an $\vb*{x}$ factor and another with $\vb*{R}$. 
The $\vb*{R}$ term has to vanish TODO: why? 

The $\vb*{x}$ part in \eqref{eq:erE-inter-1} is therefore the only non-zero part, 
and it reads 
\begin{equation}
    \begin{aligned}
        \mel**{\psi_{n \vb*{k}}}{\vb*{x}}{\psi_{n \vb*{k}'}}
        &= \delta_{\vb*{k} \vb*{k}'} \int_{\text{u.c.}} \dd[d]{\vb*{x}}
        \ee^{- \ii \vb*{k} \cdot \vb*{x} } u^*_{n \vb*{k}}(\vb*{x})
        u_{n \vb*{k}'}(\vb*{x})
        \cdot (- \ii \partial_{\vb*{k}'})
        \ee^{\ii \vb*{k}' \cdot \vb*{x} }   \\
        &= \delta_{\vb*{k} \vb*{k}'} \int_{\text{u.c.}} \dd[d]{\vb*{x}}
        \ee^{- \ii \vb*{k} \cdot \vb*{x} } u^*_{n \vb*{k}}(\vb*{x})
        \cdot (- \ii \partial_{\vb*{k}'})
        u_{n \vb*{k}'}(\vb*{x}) \ee^{\ii \vb*{k}' \cdot \vb*{x} }  \\
        &\quad + \delta_{\vb*{k} \vb*{k}'} \int_{\text{u.c.}} \dd[d]{\vb*{x}}
        \ee^{- \ii \vb*{k} \cdot \vb*{x} } u^*_{n \vb*{k}}(\vb*{x})
        \cdot \ee^{\ii \vb*{k}' \cdot \vb*{x} }  (\ii \partial_{\vb*{k}'})
        u_{n \vb*{k}'}(\vb*{x}) \\
        &= \delta_{\vb*{k} \vb*{k}'} (- \ii \partial_{\vb*{k}'}) 
        \int_{\text{u.c.}} \dd[d]{\vb*{x}}
        \ee^{- \ii \vb*{k} \cdot \vb*{x} } u^*_{n \vb*{k}}(\vb*{x})
        \cdot 
        u_{n \vb*{k}'}(\vb*{x}) \ee^{\ii \vb*{k}' \cdot \vb*{x} }  \\
        &\quad + \delta_{\vb*{k} \vb*{k}'} \int_{\text{u.c.}} \dd[d]{\vb*{x}}
         u^*_{n \vb*{k}}(\vb*{x})
        \cdot   (\ii \partial_{\vb*{k}})
        u_{n \vb*{k}}(\vb*{x}).  \\
    \end{aligned}
\end{equation}
The first term may be non-zero, but it vanishes once placed in a sum over $\vb*{k}'$:
we have 
\[
    \sum_{\vb*{k}'} c_{\vb*{k}'} 
    \delta_{\vb*{k} \vb*{k}'} (- \ii \partial_{\vb*{k}'}) 
        \int_{\text{u.c.}} \dd[d]{\vb*{x}}
        \ee^{- \ii \vb*{k} \cdot \vb*{x} } u^*_{n \vb*{k}}(\vb*{x})
        \cdot 
        u_{n \vb*{k}'}(\vb*{x}) \ee^{\ii \vb*{k}' \cdot \vb*{x} } 
\]
TODO 

The final conclusion is we have 
\begin{equation}
    \ii \partial_t c_{\vb*{k}}
    = \varepsilon_{n \vb*{k}} c_{\vb*{k}}
    + e \vb*{E} \cdot (\vb*{r}^{\text{envelope}} + A_{n \vb*{k}}), \quad 
    A_{n \vb*{k}} = \ii \mel**{u_{n \vb*{k}}}{\partial_{\vb*{k}}}{u_{n \vb*{k}}},
\end{equation}
and the effective Hamiltonian is now 
\begin{equation}
    H^{\text{envelope}} = (\varepsilon_{n \vb*{k}} 
    + (\vb*{r}^{\text{envelope}} + \vb*{A}_{n \vb*{k}}) \cdot \vb*{E})|_{\vb*{k} \to - \grad}.
\end{equation}
This agrees with the effective Hamiltonian obtained 
from path integral formalism; 
in the latter, however, we don't know if the ``position'' variable 
is indeed the center of the wave packet, 
while here we explicitly show that the center of the wave packet 
does look like a position variable in quantum mechanics.

\end{document}