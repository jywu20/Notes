\documentclass[hyperref, a4paper]{article}

\usepackage{geometry}
\usepackage{titling}
\usepackage{titlesec}
% No longer needed, since we will use enumitem package
% \usepackage{paralist}
% \usepackage{enumitem}
\usepackage{footnote}
\usepackage{marginnote}
\usepackage{enumerate}
\usepackage{amsmath, amssymb, amsthm}
\usepackage{mathtools}
\usepackage{bbm}
\usepackage{cite}
\usepackage{graphicx}
\usepackage{subfigure}
\usepackage{physics}
\usepackage{tensor}
\usepackage{siunitx}
\usepackage[version=4]{mhchem}
\usepackage{tikz}
\usepackage{xcolor}
\usepackage{listings}
\usepackage{autobreak}
\usepackage[ruled, vlined, linesnumbered]{algorithm2e}
\usepackage{nameref,zref-xr}
\zxrsetup{toltxlabel}
\zexternaldocument*[gr-]{../relativity/relativistic}[relativistic.pdf]
\zexternaldocument*[optics-]{../optics/optics}[optics.pdf]
\zexternaldocument*[solid-]{../solid/solid}[solid.pdf]
\zexternaldocument*[kt-]{../topological-phases-reading-notes/kt}[kt.pdf]
\usepackage[colorlinks,unicode]{hyperref} % , linkcolor=black, anchorcolor=black, citecolor=black, urlcolor=black, filecolor=black
\usepackage[most]{tcolorbox}
\usepackage{prettyref}

% Page style
\geometry{left=3.18cm,right=3.18cm,top=2.54cm,bottom=2.54cm}
\titlespacing{\paragraph}{0pt}{1pt}{10pt}[20pt]
\setlength{\droptitle}{-5em}
\preauthor{\vspace{-10pt}\begin{center}}
\postauthor{\par\end{center}}

% More compact lists 
%\setlist[itemize]{
%    itemindent=17pt, 
%    leftmargin=1pt,
%    listparindent=\parindent,
%    parsep=0pt,
%}

% Math operators
\DeclareMathOperator{\timeorder}{\mathcal{T}}
\DeclareMathOperator{\diag}{diag}
\DeclareMathOperator{\legpoly}{P}
\DeclareMathOperator{\primevalue}{P}
\DeclareMathOperator{\sgn}{sgn}
\newcommand*{\ii}{\mathrm{i}}
\newcommand*{\ee}{\mathrm{e}}
\newcommand*{\const}{\mathrm{const}}
\newcommand*{\suchthat}{\quad \text{s.t.} \quad}
\newcommand*{\argmin}{\arg\min}
\newcommand*{\argmax}{\arg\max}
\newcommand*{\normalorder}[1]{: #1 :}
\newcommand*{\pair}[1]{\langle #1 \rangle}
\newcommand*{\fd}[1]{\mathcal{D} #1}
\DeclareMathOperator{\bigO}{\mathcal{O}}

% TikZ setting
\usetikzlibrary{arrows,shapes,positioning}
\usetikzlibrary{arrows.meta}
\usetikzlibrary{decorations.markings}
\tikzstyle arrowstyle=[scale=1]
\tikzstyle directed=[postaction={decorate,decoration={markings,
    mark=at position .5 with {\arrow[arrowstyle]{stealth}}}}]
\tikzstyle ray=[directed, thick]
\tikzstyle dot=[anchor=base,fill,circle,inner sep=1pt]
\usetikzlibrary{fadings}
\usetikzlibrary{patterns}
\usetikzlibrary{shadows.blur}
\usetikzlibrary{shapes}

% Algorithm setting
% Julia-style code
\SetKwIF{If}{ElseIf}{Else}{if}{}{elseif}{else}{end}
\SetKwFor{For}{for}{}{end}
\SetKwFor{While}{while}{}{end}
\SetKwProg{Function}{function}{}{end}
\SetArgSty{textnormal}

\newcommand*{\concept}[1]{{\textbf{#1}}}

% Support for tensor double arrows.
\renewcommand{\tensor}[1]{ \stackrel{\leftrightarrow}{\vb*{#1}}}

% Embedded codes
\lstset{basicstyle=\ttfamily,
  showstringspaces=false,
  commentstyle=\color{gray},
  keywordstyle=\color{blue}
}

% Reference formatting
\newrefformat{fig}{Figure~\ref{#1} on page~\pageref{#1}}

% Color boxes
\tcbuselibrary{skins, breakable, theorems}
\newtcbtheorem[number within=section]{warning}{Warning}%
  {colback=orange!5,colframe=orange!65,fonttitle=\bfseries, breakable}{warn}
\newtcbtheorem[number within=section]{note}{Note}%
  {colback=green!5,colframe=green!65,fonttitle=\bfseries, breakable}{note}

% Correctly displaying equation number in section titles
\pdfstringdefDisableCommands{\def\eqref#1{(\ref{#1})}}

\newcommand{\grnote}{\href{../relativity/relativistic.pdf}{this note}}

\title{Prof. Bambi on General Relativity}
\author{Jinyuan Wu}

\begin{document}

\maketitle

This is a note about Prof. Cosimo Bambi's lecture on general relativity from April 15 based on \cite{bambi2018introduction}.

\section{Schwarzschild spacetime}

In \href{./2022-4-15.pdf}{the previous lectures} we have find the equations describing the spacetime.
In principle, what we need to do is just to solve these equations. But it's hard -- actually, we are 
only able to obtain explicit solutions in very specific cases with strong symmetry. In the following 
lectures, we will discuss some of these solutions.

The derivation of the Schwarzschild solution \marginnote{Sec. 8.1 to 8.3} is completely covered from Sec. 8.1 to Sec. 8.3,
and here we just clarify some tricky details:
\begin{itemize}
    \item The logic when deriving (8.8) \marginnote{(8.1) to (8.8)} can be articulated in a clearer way.
    First, in (8.5), we define 
    \begin{equation} \marginnote{(8.6)}
        \tilde{r}^2 = g(t, r) r^2,
    \end{equation}
    and therefore 
    \[
        \dd{s^2} = - f'(t, \tilde{r}) c^2 \dd{t^2} + g_1(t, \tilde{r}) \dd{\tilde{r}^2} + h(t, \tilde{r}) \dd{t} \dd{\tilde{r}} + \tilde{r}^2 (\dd{\theta}^2 + \sin^2 \theta \dd{\phi^2}).
    \]
    Now we can diagonalize the prt of the metric tensor involving $t$ and $r$, which result in a coordinate transformation like 
    \[
        \pmqty{r' \\ t'} = U \pmqty{\tilde{r} \\ t}.
    \]
    We can rescale $U$ at each spacetime point such that $\tilde{r} = r'$ while still keeping $g_{t' r'} = 0$, so 
    \begin{equation}
        \dd{s^2} = - f''(t', \tilde{r}) c^2 \dd{t'^2} + g_1''(t', \tilde{r}) \dd{\tilde{r}^2} + \tilde{r}^2 (\dd{\theta}^2 + \sin^2 \theta \dd{\phi^2}),
        \label{eq:original-sch-metric}
    \end{equation}
    which is just (8.8), in which $r$ is not $r$ in (8.2) but $\tilde{r}$,  \marginnote{(8.8)}
    and $g$ is not $g$ in (8.3).

    Now we discuss the geometrical meaning of $\tilde{r}$. \eqref{eq:original-sch-metric} means the 
    line element on a sphere is 
    \[
        \dd{l^2}_\text{sphere} = \tilde{r}^2 (\dd{\theta^2} + \sin^2 \theta \dd{\phi^2}),
    \]
    and therefore the surface is of area $4 \pi \tilde{r}^2$. But since $g_1''(t', \tilde{r})$ is not a 
    uniform one, the distance between the center of the sphere and the boundary of the sphere, i.e. \marginnote{discussion between (8.7) and (8.8)}
    \[
        \int_0^{\tilde{r}} \dd{\tilde{r}}' g_1''(t', \tilde{r}),
    \]
    is not necessarily $\tilde{r}$. 
    \item (8.8) may be time dependent. However, in Einstein's gravity this is impossible -- the equation 
    $R_{tr} = 0$ blocks this possibility \marginnote{(8.27)} and simplifies the following derivation.
    \item The Schwarzschild metric is derived \emph{in vacuum}. It \emph{always} holds if we are sure that 
    the gravitational field is spherically symmetric enough, no matter what radical physical processes is going on 
    in the center. Rotation creates a special spacial direction (the axis) and therefore breaks the symmetry,
    but if an object rotates slow enough, the Schwarzschild solution is still a good approximation.
    The standard of ``slow'' here is to be determined by comparing a more realistic model with the Schwarzschild
    metric. 
\end{itemize}

We define the \concept{Schwarzschild radius} $r_\text{s}$ as the radius where the Schwarzschild metric 
has singularity. We have 
\begin{equation} \marginnote{(8.42)}
    r_\text{s} = \frac{2 GM}{c^2}.
\end{equation}
This means the proper time of a static object in a Schwarzschild gravitational field is 
\begin{equation}
    \dd{\tau} = \sqrt{1 + \frac{2 \Phi}{c^2}} \dd{t} = \sqrt{1 - \frac{2 G M}{c^2 r}} \dd{t} = \sqrt{1 - \frac{r_\text{s}}{r}} \dd{t} < \dd{t}.
\end{equation}

\bibliographystyle{plain}
\bibliography{books-used}

\end{document}