\documentclass[hyperref, a4paper]{article}

\usepackage{geometry}
\usepackage{titling}
\usepackage{titlesec}
% No longer needed, since we will use enumitem package
% \usepackage{paralist}
% \usepackage{enumitem}
\usepackage{footnote}
\usepackage{marginnote}
\usepackage{enumerate}
\usepackage{amsmath, amssymb, amsthm}
\usepackage{mathtools}
\usepackage{bbm}
\usepackage{cite}
\usepackage{graphicx}
\usepackage{subfigure}
\usepackage{physics}
\usepackage{tensor}
\usepackage{siunitx}
\usepackage[version=4]{mhchem}
\usepackage{tikz}
\usepackage{xcolor}
\usepackage{listings}
\usepackage{autobreak}
\usepackage[ruled, vlined, linesnumbered]{algorithm2e}
\usepackage{nameref,zref-xr}
\zxrsetup{toltxlabel}
\zexternaldocument*[gr-]{../relativity/relativistic}[relativistic.pdf]
\zexternaldocument*[optics-]{../optics/optics}[optics.pdf]
\zexternaldocument*[solid-]{../solid/solid}[solid.pdf]
\zexternaldocument*[kt-]{../topological-phases-reading-notes/kt}[kt.pdf]
\usepackage[colorlinks,unicode]{hyperref} % , linkcolor=black, anchorcolor=black, citecolor=black, urlcolor=black, filecolor=black
\usepackage[most]{tcolorbox}
\usepackage{prettyref}

% Page style
\geometry{left=3.18cm,right=3.18cm,top=2.54cm,bottom=2.54cm}
\titlespacing{\paragraph}{0pt}{1pt}{10pt}[20pt]
\setlength{\droptitle}{-5em}
\preauthor{\vspace{-10pt}\begin{center}}
\postauthor{\par\end{center}}

% More compact lists 
%\setlist[itemize]{
%    itemindent=17pt, 
%    leftmargin=1pt,
%    listparindent=\parindent,
%    parsep=0pt,
%}

% Math operators
\DeclareMathOperator{\timeorder}{\mathcal{T}}
\DeclareMathOperator{\diag}{diag}
\DeclareMathOperator{\legpoly}{P}
\DeclareMathOperator{\primevalue}{P}
\DeclareMathOperator{\sgn}{sgn}
\newcommand*{\ii}{\mathrm{i}}
\newcommand*{\ee}{\mathrm{e}}
\newcommand*{\const}{\mathrm{const}}
\newcommand*{\suchthat}{\quad \text{s.t.} \quad}
\newcommand*{\argmin}{\arg\min}
\newcommand*{\argmax}{\arg\max}
\newcommand*{\normalorder}[1]{: #1 :}
\newcommand*{\pair}[1]{\langle #1 \rangle}
\newcommand*{\fd}[1]{\mathcal{D} #1}
\DeclareMathOperator{\bigO}{\mathcal{O}}

% TikZ setting
\usetikzlibrary{arrows,shapes,positioning}
\usetikzlibrary{arrows.meta}
\usetikzlibrary{decorations.markings}
\tikzstyle arrowstyle=[scale=1]
\tikzstyle directed=[postaction={decorate,decoration={markings,
    mark=at position .5 with {\arrow[arrowstyle]{stealth}}}}]
\tikzstyle ray=[directed, thick]
\tikzstyle dot=[anchor=base,fill,circle,inner sep=1pt]
\usetikzlibrary{fadings}
\usetikzlibrary{patterns}
\usetikzlibrary{shadows.blur}
\usetikzlibrary{shapes}

% Algorithm setting
% Julia-style code
\SetKwIF{If}{ElseIf}{Else}{if}{}{elseif}{else}{end}
\SetKwFor{For}{for}{}{end}
\SetKwFor{While}{while}{}{end}
\SetKwProg{Function}{function}{}{end}
\SetArgSty{textnormal}

\newcommand*{\concept}[1]{{\textbf{#1}}}

% Support for tensor double arrows.
\renewcommand{\tensor}[1]{ \stackrel{\leftrightarrow}{\vb*{#1}}}

% Embedded codes
\lstset{basicstyle=\ttfamily,
  showstringspaces=false,
  commentstyle=\color{gray},
  keywordstyle=\color{blue}
}

% Reference formatting
\newrefformat{fig}{Figure~\ref{#1} on page~\pageref{#1}}

% Color boxes
\tcbuselibrary{skins, breakable, theorems}
\newtcbtheorem[number within=section]{warning}{Warning}%
  {colback=orange!5,colframe=orange!65,fonttitle=\bfseries, breakable}{warn}
\newtcbtheorem[number within=section]{note}{Note}%
  {colback=green!5,colframe=green!65,fonttitle=\bfseries, breakable}{note}

% Correctly displaying equation number in section titles
\pdfstringdefDisableCommands{\def\eqref#1{(\ref{#1})}}

\newcommand{\grnote}{\href{../relativity/relativistic.pdf}{this note}}

\title{Prof. Bambi on General Relativity}
\author{Jinyuan Wu}

\begin{document}

\maketitle

This is a note about Prof. Cosimo Bambi's lecture on general relativity from February 25 to April 8, 2022.

This lecture is about the 1-2 chapters in \cite{bambi2018introduction}. Nothing quite interesting.
\cite{bambi2018introduction} itself is very detailed and it seems I don't really need to take much notes.

\section{Relativistic kinematics}

\subsection{The Christoffel symbols of spherical coordinates}

The Christoffel symbol of spherical coordinates, given in (1.83) in \cite{bambi2018introduction}, can be 
calculated automatically in \href{spherical-coordinate-differential-geometry.nb}{this Mathematica notebook}.

\subsection{Derivation of special relativity}

Section~2.1 and 2.2 seem to be based on Chapter 1 and 2 of Landau's book about field theory. 
The arguments have been summarized in Section~\ref{gr-sec:principle-special-relativity} in \grnote.

Section~2.3 derives the Lorentz transformations by Wick rotation of Euclidean rotation in $d = 4$
-- (2.18) is actually just $\tau = \ii t$ in condensed matter physics.
A rotation on $xy$ has the form of (2.20) and we have $C_4^2 = 6$ rotations. Adding 4 translations,
we get the total 10 generators of the rotation group in $\mathbb{R}^4$, and hence the Lorentz transformations
in the (3+1)-dimensional Minkowski spacetime.

Equations from (2.22) to (2.27) are trying to relate the parameter of $\mathbb{R}^4$ rotations to the relative 
velocity of the two reference frames.

Note that since the rotation group $SO(4)$ is not Abelian, Lorentz transformations do not commute in general. \marginnote{Problem~2.3}

The time coordinate in the reference frame attached to a particle (i.e. in the reference frame the particle 
is at rest) is called the \concept{proper time} of the particle, usually denoted as $\tau$.
Since 
\[ \marginnote{(2.36) and (2.37)}
    \dd{s^2} = - \left( c^2 - \left(\pdv{x}{t}\right)^2 - \left(\pdv{y}{t}\right)^2 - \left(\pdv{z}{t}\right)^2 \right) \dd{t^2} = - \left( 1 - \frac{v^2}{c^2} \right) c^2 \dd{t^2} = - \frac{c^2}{\gamma^2} \dd{t^2} = - c^2 \dd{\tau}^2,
\] 
we have 
\begin{equation}
    \dd{t} = \gamma \dd{\tau}.
\end{equation}
Since a Lorentz transformation keeps the metrics, we have 
\[
    \dd{t} \dd{V} = \dd{\tau} \dd{V_0},
\]
where $\dd{V_0}$ is the volume metrics in the reference frame attached to the particle. This can also by checked 
by noting that (here $\vb*{v}$ is along the $x$ axis)
\[ \marginnote{from (2.45)}
    \begin{aligned}
        \dd{x'} \wedge \dd{t'} &= (-\gamma v \dd{t} + \gamma \dd{x}) \wedge (\gamma \dd{t} - \gamma v \dd{x} / c^2) \\
        &= \gamma^2 \dd{x} \wedge \dd{t} + \frac{\gamma^2 v^2}{c^2} \dd{t} \wedge \dd{x} \\
        &= \gamma^2 \left( 1 - \frac{v^2}{c^2} \right) \dd{x} \wedge \dd{t} = \dd{x} \wedge \dd{t}.
    \end{aligned}
\]
Therefore, 
\begin{equation} \marginnote{(2.44)}
    \dd{V} = \frac{\dd{V_0}}{\gamma}.
\end{equation}

\subsection{Aberration of light and superluminal motion}

We should note that \emph{Lorentz contraction} of length (see for example (2.43)) is often \emph{not} 
\marginnote{(2.48) to (2.57)} the length change we \emph{observe} when an object is moving fast.
The point here is that to measure a line we need to detect light signals from its two ends, and generally 
speaking, two signals arriving at our detector start their journeys at different time points, while on the 
other hand, from the way we derive Lorentz contraction ((2.41) and (2.42)) \marginnote{(2.41) and (2.42)}
we are dealing with events happening at the same time point in the laboratory frame of reference. 
The conclusion is $l$ in (2.43) is often not the length we \emph{see} of a moving object.

The fact that the ``distance'' we see is actually not the authentic space distance between two events with the 
same time in a given frame of reference means that when calculating the velocity, we are taking the time
derivative of two points at different time points, or in other words, taking the derivative of a distance 
with respect to a time not coherent with the current frame of reference. In this way, some superfluous
``superluminal'' movement can occur. Consider, for example, the case of \prettyref{fig:superluminal}. 
Suppose at $t$ a beam of light is emitted from the ejected material, and it arrives at the detector at $t'$.
We therefore have 
\begin{equation} \marginnote{(2.51), (2.52)}
    \begin{aligned}
        c(t' - t) &= L = \sqrt{(D - vt \cos \varphi)^2 + v^2 t^2 \sin^2 \varphi} \\
        &= D - vt \cos \varphi + \bigO(v^2 t^2 / D^2).
    \end{aligned}
\end{equation}
Therefore, we have 
\begin{equation} \marginnote{(2.53)}
    t' = \frac{D}{c} + t (1 - \beta \cos \varphi).
\end{equation}
Now we try to evaluate the \emph{apparent} velocity on the $y$ direction, which is $\dv*{y}{t'}$.
Note that only $\dv*{y}{t}$ is bounded by $c$, while $\dv*{y}{t'}$ does not have an upper bound.
Actually, we have 
\begin{equation} \marginnote{(2.54)}
    \dv{y}{t'} = \dv{y}{t} \dv{t}{t'} = v \sin \varphi \times \frac{1}{1 - \beta \cos \varphi}.
\end{equation}
The maximum is shown to be $v \gamma$. Here we can see the apparent velocity is obtained using $t'$ \marginnote{(2.57)}
as the time, which is not $x^0$ in the frame of coordinate attached to the observer. 

\begin{figure}
    \centering
    

% Gradient Info
  
\tikzset {_f6md6ajgy/.code = {\pgfsetadditionalshadetransform{ \pgftransformshift{\pgfpoint{0 bp } { 0 bp }  }  \pgftransformscale{1 }  }}}
\pgfdeclareradialshading{_jk0eqd11o}{\pgfpoint{0bp}{0bp}}{rgb(0bp)=(0.97,0.91,0.11);
rgb(0bp)=(0.97,0.91,0.11);
rgb(25bp)=(1,1,1);
rgb(400bp)=(1,1,1)}
\tikzset{every picture/.style={line width=0.75pt}} %set default line width to 0.75pt        

\begin{tikzpicture}[x=0.75pt,y=0.75pt,yscale=-1,xscale=1]
%uncomment if require: \path (0,315); %set diagram left start at 0, and has height of 315

%Straight Lines [id:da8791392372886593] 
\draw [color={rgb, 255:red, 0; green, 0; blue, 0 }  ,draw opacity=0.75 ]   (184,118.23) -- (312,242) ;
%Straight Lines [id:da6990399911927212] 
\draw [color={rgb, 255:red, 0; green, 0; blue, 0 }  ,draw opacity=0.5 ]   (96,242) -- (96,283) ;
%Straight Lines [id:da929753289055431] 
\draw [color={rgb, 255:red, 0; green, 0; blue, 0 }  ,draw opacity=1 ]   (96,242) -- (214.84,75.63) ;
\draw [shift={(216,74)}, rotate = 125.54] [fill={rgb, 255:red, 0; green, 0; blue, 0 }  ,fill opacity=1 ][line width=0.08]  [draw opacity=0] (12,-3) -- (0,0) -- (12,3) -- cycle    ;
%Straight Lines [id:da9476652899394022] 
\draw    (96,242) -- (396,242) ;
\draw [shift={(398,242)}, rotate = 180] [fill={rgb, 255:red, 0; green, 0; blue, 0 }  ][line width=0.08]  [draw opacity=0] (12,-3) -- (0,0) -- (12,3) -- cycle    ;
%Straight Lines [id:da12075393106418608] 
\draw    (96,242) -- (96,52) ;
\draw [shift={(96,50)}, rotate = 90] [fill={rgb, 255:red, 0; green, 0; blue, 0 }  ][line width=0.08]  [draw opacity=0] (12,-3) -- (0,0) -- (12,3) -- cycle    ;
%Shape: Ellipse [id:dp7501822937817628] 
\draw  [draw opacity=0][shading=_jk0eqd11o,_f6md6ajgy] (65.63,224.61) .. controls (71.12,215.02) and (89.17,215.04) .. (105.94,224.64) .. controls (122.71,234.25) and (131.86,249.81) .. (126.37,259.39) .. controls (120.88,268.98) and (102.83,268.96) .. (86.06,259.36) .. controls (69.29,249.75) and (60.14,234.19) .. (65.63,224.61) -- cycle ;
%Shape: Arc [id:dp14808297636810597] 
\draw  [draw opacity=0] (125.4,201.77) .. controls (137.65,210.74) and (145.66,225.17) .. (145.82,241.5) -- (96,242) -- cycle ; \draw  [color={rgb, 255:red, 0; green, 0; blue, 0 }  ,draw opacity=0.5 ] (125.4,201.77) .. controls (137.65,210.74) and (145.66,225.17) .. (145.82,241.5) ;
%Shape: Ellipse [id:dp5550484242786045] 
\draw  [color={rgb, 255:red, 80; green, 227; blue, 194 }  ,draw opacity=1 ][fill={rgb, 255:red, 80; green, 227; blue, 194 }  ,fill opacity=0.75 ] (191.07,102.77) .. controls (187.81,101.28) and (181.99,106.99) .. (178.09,115.53) .. controls (174.18,124.06) and (173.66,132.2) .. (176.93,133.69) .. controls (180.19,135.18) and (186.01,129.47) .. (189.91,120.93) .. controls (193.82,112.39) and (194.34,104.26) .. (191.07,102.77) -- cycle ;
%Straight Lines [id:da06332288321674429] 
\draw [color={rgb, 255:red, 0; green, 0; blue, 0 }  ,draw opacity=0.5 ]   (312,242) -- (312,283) ;
%Straight Lines [id:da6458514363662675] 
\draw [color={rgb, 255:red, 0; green, 0; blue, 0 }  ,draw opacity=0.5 ]   (98,269) -- (310,269) ;
\draw [shift={(312,269)}, rotate = 180] [fill={rgb, 255:red, 0; green, 0; blue, 0 }  ,fill opacity=0.5 ][line width=0.08]  [draw opacity=0] (12,-3) -- (0,0) -- (12,3) -- cycle    ;
\draw [shift={(96,269)}, rotate = 0] [fill={rgb, 255:red, 0; green, 0; blue, 0 }  ,fill opacity=0.5 ][line width=0.08]  [draw opacity=0] (12,-3) -- (0,0) -- (12,3) -- cycle    ;
%Straight Lines [id:da4496754806264811] 
\draw [color={rgb, 255:red, 0; green, 0; blue, 0 }  ,draw opacity=0.5 ] [dash pattern={on 4.5pt off 4.5pt}]  (184,118.23) -- (184,242) ;
%Straight Lines [id:da29868680463289254] 
\draw [color={rgb, 255:red, 0; green, 0; blue, 0 }  ,draw opacity=0.5 ]   (184,76) -- (184,118.23) ;
\draw [shift={(184,74)}, rotate = 90] [fill={rgb, 255:red, 0; green, 0; blue, 0 }  ,fill opacity=0.5 ][line width=0.08]  [draw opacity=0] (12,-3) -- (0,0) -- (12,3) -- cycle    ;

% Text Node
\draw (400,242) node [anchor=west] [inner sep=0.75pt]    {$x$};
% Text Node
\draw (94,46.6) node [anchor=south east] [inner sep=0.75pt]    {$y$};
% Text Node
\draw (150,210.4) node [anchor=north west][inner sep=0.75pt]    {$\varphi $};
% Text Node
\draw (204,272.4) node [anchor=north] [inner sep=0.75pt]    {$D$};
% Text Node
\draw (250,176.71) node [anchor=south west] [inner sep=0.75pt]    {$L$};
% Text Node
\draw (98,149.4) node [anchor=north west][inner sep=0.75pt]    {$R=vt$};
% Text Node
\draw (218,70.6) node [anchor=south west] [inner sep=0.75pt]    {$v$};
% Text Node
\draw (184,70.6) node [anchor=south] [inner sep=0.75pt]    {$v_{y}$};
% Text Node
\draw (308,217) node [anchor=north west][inner sep=0.75pt]   [align=left] {detector};


\end{tikzpicture}

    \caption{Superluminal motion of a ejected material from a galaxy}
    \label{fig:superluminal}
\end{figure}

\subsection{Time dilation and cosmic ray muons}

The lifetime of an unstable particle is measured according to its proper time, and this causes the difference \marginnote{Sec.~2.6}
between the prediction of Newtonian and relativistic theories of the flux of the particle after traveling a 
certain distance. This is actually a piece of strong evidence of special relativity.

\section{Relativistic mechanics (single particle)}

\subsection{The action}

We know the Lagrangian of a Newtonian free particle is 
\begin{equation} \marginnote{(3.1)}
    L = \frac{1}{2} m g_{ij} \dot{x}^i \dot{x}^j.
    \label{eq:newton}
\end{equation}
The natural generalization is 
\begin{equation}
    L = \frac{1}{2} m g_{\mu \nu} \dot{x}^\mu \dot{x}^\nu.
    \label{eq:generalized-lagrangian}
\end{equation}
Note, however, the meaning of $\dot{x}$ is not clear here: $g_{\mu \nu} \dd{x^\mu} \dd{x^\nu}$ is already a 
covariant value, and to make the whole expression covariant, the ``time'' used in the time derivative $\dot{x}$
should also be a relativistic scalar, which can only be the proper time $\tau$. Similarly, when calculating 
the action, we need to integrate \eqref{eq:generalized-lagrangian} over $\tau$. So the final action is 
\begin{equation}
    S = \int \dd{\tau} L, \quad L = \frac{1}{2} m g_{\mu \nu} \dv{x^\mu}{\tau} \dv{x^\nu}{\tau}.
    \label{eq:gr-action-1}
\end{equation}
We can repeat the process in (1.71) \marginnote{(1.71)} and find that the EOM is 
\begin{equation} \marginnote{(3.6)}
    \dv[2]{x^\mu}{\tau} + \Gamma\indices{^\mu_\nu _\rho} \dv{x^\nu}{\tau} \dv{x^\rho}{\tau} = 0.
    \label{eq:eom-sr}
\end{equation}
Actually Newton's second law can also be written into this form (again see the discussion around (1.71)),
but this time we are working with $\mu, \nu, \rho = 0, 1, 1, 3$, not just $1, 2, 3$.
Note that \eqref{eq:eom-sr} is actually the geodesic equation. 

\eqref{eq:eom-sr} is about $\dv*{x^\mu}{\tau}$, which is the tangent vector of the trajectory of the particle
and is manifestly a 4-vector. Its components are 
\begin{equation}
    \dv{x^\mu}{\tau} = \dv{x^\mu}{t} \dv{t}{\tau} = (\gamma c, \gamma \vb*{v}).
    \label{eq:4-velocity-single}
\end{equation}

\begin{note*}{}
    When we say some expression is \concept{manifestly} covariant, we mean the expression is built up by 
    tensor and Einstein summation notation and can be automatically decided as covariant.
\end{note*}

We have a more ``geometric'' version of \eqref{eq:newton}. Note that 
\[
    \dd{l} = \sqrt{g_{ij} \dot{x}^i \dot{x}^j} \dd{t}
\]
is an increasing function of the integrand of \eqref{eq:newton}. In general, this \emph{doesn't} mean that 
the EOM of $\dv*{l}{\tau}$ is the same as \eqref{eq:newton}. The Euler-Lagrangian equation of $\sqrt{L}$ is 
\begin{equation}
    0 = - \frac{1}{4 L^{3/2}} \dv{L}{t} \pdv{L}{\dot{x}^k} + \frac{1}{2 \sqrt{L}} \dv{t} \pdv{L}{\dot{x}^k} - \frac{1}{2 \sqrt{L}} \pdv{L}{x^k},
\end{equation}
which is generally not equivalent to the original EOM of $L$. However, from theorems in differential geometry,
we know that the curve parameter that makes the geodesic equation hold is \emph{always} an affine function of 
the line length parameter, and since $L$ is the square of the line length element, $\dv*{L}{t} = 0$, and 
therefore a solution of the EOM of \eqref{eq:newton} is also a solution of the EOM of
\begin{equation}
    S = \int \dd{l}.
    \label{eq:newton-line}
\end{equation}
The EOM of \eqref{eq:newton} has yet more solutions, which can be verified to be \emph{reparameterized} 
geodesic equations. \marginnote{Canbin Liang, discussion after (3-3-3)} This fact -- that the solutions of 
$L$ and $\sqrt{L}$ only differ with reparameterization-- is not a general fact, but we don't talk about 
$\sqrt{L}$ in systems other than relativistic free particles, either. For more discussion about the physical 
meaning of the two Lagrangians, see \cite{sqrt-lag}.

Similarly, we may guess the version of \eqref{eq:newton-line} corresponding to \eqref{eq:gr-action-1} 
is 
\begin{equation}
    S = \int \abs*{\dd{s}} = \int \sqrt{- \dd{s^2}}.
\end{equation}
This is indeed the case, since actually by the definition of proper time, we have 
\begin{equation} \marginnote{(3.4)}
    c^2 \dd{\tau^2} = - \dd{s^2} = - g_{\mu \nu} \dd{x^\mu} \dd{x^\nu},
\end{equation}
and therefore $L$ in \eqref{eq:gr-action-1} is just a constant, and we have $S \propto \int \dd{\tau}$.

A question is what is the relativistic version of mechanics of \emph{massless} particles. \marginnote{Sec.~3.3}
Since $m$ in the action is just a constant, the geodesic equation \eqref{eq:eom-sr} still works. 
The Lagrangian therefore can still be written as 
\begin{equation} \marginnote{(3.38)}
    L = - mc \sqrt{- g_{\mu \nu} \dot{x}^\mu \dot{x}^\nu},
    \label{eq:gr-action-2}
\end{equation}
but now $m$ should be a constant interpreted as a \emph{coupling constant} with the dimension of mass.

\subsection{Comparison between Lagrangians}

\begin{figure}
    \centering
    

\tikzset{every picture/.style={line width=0.75pt}} %set default line width to 0.75pt        

\begin{tikzpicture}[x=0.75pt,y=0.75pt,yscale=-1,xscale=1]
%uncomment if require: \path (0,384); %set diagram left start at 0, and has height of 384

%Straight Lines [id:da7623325543703428] 
\draw    (243,148.67) -- (243,217.67) ;
\draw [shift={(243,219.67)}, rotate = 270] [fill={rgb, 255:red, 0; green, 0; blue, 0 }  ][line width=0.08]  [draw opacity=0] (12,-3) -- (0,0) -- (12,3) -- cycle    ;
\draw [shift={(243,146.67)}, rotate = 90] [fill={rgb, 255:red, 0; green, 0; blue, 0 }  ][line width=0.08]  [draw opacity=0] (12,-3) -- (0,0) -- (12,3) -- cycle    ;
%Straight Lines [id:da2569465613685389] 
\draw    (472,148.67) -- (472,217.67) ;
\draw [shift={(472,219.67)}, rotate = 270] [fill={rgb, 255:red, 0; green, 0; blue, 0 }  ][line width=0.08]  [draw opacity=0] (12,-3) -- (0,0) -- (12,3) -- cycle    ;
\draw [shift={(472,146.67)}, rotate = 90] [fill={rgb, 255:red, 0; green, 0; blue, 0 }  ][line width=0.08]  [draw opacity=0] (12,-3) -- (0,0) -- (12,3) -- cycle    ;
%Rounded Rect [id:dp8999512925345985] 
\draw  [draw opacity=0][fill={rgb, 255:red, 80; green, 227; blue, 194 }  ,fill opacity=0.3 ] (34,104.93) .. controls (34,99.45) and (38.45,95) .. (43.93,95) -- (587.07,95) .. controls (592.55,95) and (597,99.45) .. (597,104.93) -- (597,134.73) .. controls (597,140.22) and (592.55,144.67) .. (587.07,144.67) -- (43.93,144.67) .. controls (38.45,144.67) and (34,140.22) .. (34,134.73) -- cycle ;
%Rounded Rect [id:dp555270186695668] 
\draw  [draw opacity=0][fill={rgb, 255:red, 184; green, 233; blue, 134 }  ,fill opacity=0.3 ] (36,231.93) .. controls (36,226.45) and (40.45,222) .. (45.93,222) -- (587.07,222) .. controls (592.55,222) and (597,226.45) .. (597,231.93) -- (597,261.73) .. controls (597,267.22) and (592.55,271.67) .. (587.07,271.67) -- (45.93,271.67) .. controls (40.45,271.67) and (36,267.22) .. (36,261.73) -- cycle ;
%Straight Lines [id:da6078259469899183] 
\draw    (472,219.67) -- (244.91,147.27) ;
\draw [shift={(243,146.67)}, rotate = 17.68] [fill={rgb, 255:red, 0; green, 0; blue, 0 }  ][line width=0.08]  [draw opacity=0] (12,-3) -- (0,0) -- (12,3) -- cycle    ;
%Rounded Rect [id:dp6948384732062591] 
\draw  [draw opacity=0][fill={rgb, 255:red, 144; green, 19; blue, 254 }  ,fill opacity=0.2 ] (176,28.57) .. controls (176,22) and (181.33,16.67) .. (187.91,16.67) -- (291.09,16.67) .. controls (297.67,16.67) and (303,22) .. (303,28.57) -- (303,287.76) .. controls (303,294.34) and (297.67,299.67) .. (291.09,299.67) -- (187.91,299.67) .. controls (181.33,299.67) and (176,294.34) .. (176,287.76) -- cycle ;
%Rounded Rect [id:dp2732755345951299] 
\draw  [draw opacity=0][fill={rgb, 255:red, 74; green, 144; blue, 226 }  ,fill opacity=0.2 ] (400,26.67) .. controls (400,21.14) and (404.48,16.67) .. (410,16.67) -- (561,16.67) .. controls (566.52,16.67) and (571,21.14) .. (571,26.67) -- (571,289.67) .. controls (571,295.19) and (566.52,299.67) .. (561,299.67) -- (410,299.67) .. controls (404.48,299.67) and (400,295.19) .. (400,289.67) -- cycle ;

% Text Node
\draw (424,100.9) node [anchor=north west][inner sep=0.75pt]    {$L=\frac{1}{2} mg_{\mu \nu }\dot{x}^{\mu }\dot{x}^{\nu }$};
% Text Node
\draw (408,230.4) node [anchor=north west][inner sep=0.75pt]    {$L=-mc\sqrt{-g_{\mu \nu }\dot{x}^{\mu }\dot{x}^{\nu }}$};
% Text Node
\draw (185,100.9) node [anchor=north west][inner sep=0.75pt]    {$L=\frac{1}{2} mg_{ij}\dot{x}^{i}\dot{x}^{j}$};
% Text Node
\draw (189.5,230.4) node [anchor=north west][inner sep=0.75pt]    {$L=\sqrt{g_{ij}\dot{x}^{i}\dot{x}^{j}}$};
% Text Node
\draw (153,176) node [anchor=north west][inner sep=0.75pt]   [align=left] {equivalent};
% Text Node
\draw (493,176) node [anchor=north west][inner sep=0.75pt]   [align=left] {equivalent};
% Text Node
\draw (293.76,144.3) node [anchor=north west][inner sep=0.75pt]  [rotate=-18] [align=left] {low-speed approximation};
% Text Node
\draw (53,101) node [anchor=north west][inner sep=0.75pt]   [align=left] {quadratic\\formalism};
% Text Node
\draw (55,229) node [anchor=north west][inner sep=0.75pt]   [align=left] {trajectory \\formalism};
% Text Node
\draw (201,34) node [anchor=north west][inner sep=0.75pt]   [align=left] {Newtonian};
% Text Node
\draw (448,35) node [anchor=north west][inner sep=0.75pt]   [align=left] {Relativistic};


\end{tikzpicture}

    \caption{The four Lagrangians discussed}
    \label{fig:four-lag}
\end{figure}

\prettyref{fig:four-lag} compares the four Lagrangians we discussed above. There are a few points worth noting.

First, although \eqref{eq:gr-action-1} is talked about in 3.2.2 and and \eqref{eq:gr-action-2} is talked about in 
3.2.1, the former being ``4-dimensional'' and the latter being ``3-dimensional'', the trajectory in \eqref{eq:gr-action-2} can 
be parameterized using \emph{both} $t$ and $\tau$. \eqref{eq:gr-action-2} is talked about in 3.2.1 because 
only when we use $t$ as the curve parameter can we have the concise expression $\sqrt{c^2 - \dot{\vb*{x}}^2}$ in (3.12). \marginnote{Sec. 3.2.1-3.2.2} \eqref{eq:gr-action-1} also doesn't impose any constraint on its curve 
parameter $\tau$, so when taking variation of \eqref{eq:gr-action-1}, we \emph{don't} impose the constraint that 
$\tau$ is the proper time. We \emph{know} $\tau$ is the proper time \emph{after} we find the Euler-Lagrangian
equation \eqref{eq:eom-sr}, which only holds when $\tau$ is an affine function of the line length parameter.

Second, the easiest way to go back to Newtonian mechanics is to start from 
\[
    L = - mc \sqrt{- g_{\mu \nu} \dot{x}^\mu \dot{x}^\nu}
\]
and take Taylor expansion, and we get the first two terms 
\[
    L = - mc^2 \sqrt{1 - \frac{\vb*{v}^2}{c^2}} = - mc^2 \left(1 - \frac{\vb*{v}^2}{2 c^2}\right) + \bigO(\vb*{v}^4 / c^4), 
\]
which, after throwing away the constant term, is precisely 
\[
    L = \frac{1}{2} m g_{ij} \dot{x}^i \dot{x}^j.
\]

\subsection{Particle collision}

Relativistic scattering theory is important in particle physics. Consider a typical reaction:
\begin{equation} \marginnote{(3.41)}
    A + B \longrightarrow C + D.
    \label{eq:reaction-abcd}
\end{equation}
Suppose we are working in a frame of reference where $A$ is at rest. Suppose $B$ moves along the $x$ axis.
Then we have 
\begin{equation} \marginnote{(3.43)}
    p_A^\mu = (m_A c, 0, 0, 0), \quad p_B^\mu = (m_B \gamma c, m_B \gamma v, 0, 0).
    \label{eq:p-a-b-4-mom}
\end{equation}
After the reaction, the energies and the momenta of $C$ and $D$ can be quite complicated, so we just 
try to find some general constraints imposed on them. A reaction is possible if and only if both 
energy conservation and momentum conservation hold. The momentum conservation condition can be 
satisfied by working in a reference frame where the total 3-momentum of the system vanishes, and 
this dictates 
\begin{equation} \marginnote{(3.44)}
    p_C^\mu = (\sqrt{m_C^2 c^2 + \vb*{p}_C^2}, \vb*{p}_C), \quad p_D^\mu = (\sqrt{m_D^2 c^2 + \vb*{p}_C^2}, - \vb*{p}_C).
    \label{eq:p-c-d-4-mom}
\end{equation}
Now we just need to impose the energy conservation constraint. Naively doing so is hard because energy itself 
is not a scalar. However, there \emph{is} a conserved relativistic scalar: we have  
\begin{equation}
    p^\text{i}_\mu p^{\text{i} \mu} = p^\text{f}_\mu p^{\text{f} \mu} \eqqcolon M^2 c^2,
\end{equation}
where $M$ is named the \concept{invariant mass}. Note that we can evaluate the LHS in the reference frame of 
\eqref{eq:p-a-b-4-mom} and the RHS in the reference frame of \eqref{eq:p-c-d-4-mom}, and this gives 
\begin{equation} \marginnote{(3.48)}
    \left(m_{A} c+m_{B} \gamma c\right)^{2}-m_{B}^{2} \gamma^{2} v^{2}=\left( \sqrt{m_C^2 c^2 + \vb*{p}_C^2} + \sqrt{m_D^2 c^2 + \vb*{p}_C^2} \right)^{2} ,
\end{equation}
which takes the minimum when $\vb*{p}_C = 0$, and we find the minimum energy of $B$ is 
\begin{equation} \marginnote{(3.50)}
    E_B^\text{th} = p_B^0 c = m_B \gamma c^2 = \frac{\left(m_{C}^{2}+m_{D}^{2}+2 m_{C} m_{D}-m_{A}^{2}-m_{B}^{2}\right) c^{2}}{2 m_{A}}.
\end{equation}
This is called the \concept{threshold energy}, because if $E_B < E_B^\text{th}$, \eqref{eq:reaction-abcd} cannot happen.

The procedure can be repeated for different processes and from this we can find another fact about collision that 
head-on collision is more effective than fixed-target collision.

\section{Relativistic perfect fluid}

The topic of relativistic idea fluid is discussed in Problem 2.1 and 2.2. \marginnote{Problem 2.1, 2.2}
First of all, we always have 
\[
    T^{00} = \epsilon = \rho_m c^2,
\]
where $\epsilon$ is the energy density of the fluid in the rest frame, and $\rho_m$ is the mass density.
In a perfect fluid, when we are in the rest-frame, there is no flow, and since momentum is carried by fluid flow,
the density of momentum is also zero, i.e.
\[
    T^{i0} = 0.
\]
The argument used in non-relativistic perfect fluid can be transplanted here for $T^{ij}$: since an idea fluid 
is isotropic, and it cannot hold shear force, as long as the time scale we are interested is long enough to 
hide how the fluid responds to an external shear force, we can assume all shear force components in $T^{ij}$
are zero. Thus we have 
\[
    T^{ij} = \pmqty{\dmat{p, p, p}}.
\]
Putting everything together, we get 
\begin{equation} \marginnote{(2.60)}
    T^{\mu \nu} = \pmqty{\dmat{\epsilon, p, p, p}}.
    \label{eq:rest-frame-perfect-fluid-t}
\end{equation}

Now we can get $T^{\mu \nu}$ in any coordinate system with a Lorentz transformation.
Applying the Lorentz transformation on $x$ direction:
\begin{equation} \marginnote{(2.28)}
    \Lambda\indices{^\mu_\nu} = \pmqty{\dmat{\gamma & - \gamma \beta \\ - \gamma \beta & \gamma, 1, 1}},
    \label{eq:lorentz-x}
\end{equation}
we get (the process can be found in \href{relativistic-ideal-fluid.nb}{this Mathematica notebook})
\begin{equation}
    T'^{\mu \nu} = \pmqty{\frac{p \beta ^2}{1-\beta ^2}+\frac{\epsilon }{1-\beta ^2} & -\frac{\epsilon  \beta }{1-\beta
    ^2}-\frac{p \beta }{1-\beta ^2} \\
  -\frac{\epsilon  \beta }{1-\beta ^2}-\frac{p \beta }{1-\beta ^2} & \frac{\epsilon  \beta ^2}{1-\beta
    ^2}+\frac{p}{1-\beta ^2} }.
    \label{eq:x-moving-fluid-t}
\end{equation}
Suppose the speed of the frame of reference after \eqref{eq:lorentz-x} in the rest frame of the fluid is $v$,
we find \eqref{eq:x-moving-fluid-t} is the energy-momentum tensor of a fluid moving with the velocity of 
$- v \vu*{e}_x$.  \marginnote{Problem~2.2}

\eqref{eq:x-moving-fluid-t} is not covariant. We need to generalize it into a \marginnote{Liang Sec.~6.5} 
covariant version. Solely with information provided in \eqref{eq:x-moving-fluid-t}, the covariant version 
cannot be decided, because systems other than a perfect fluid can also has a energy-momentum tensor like 
\eqref{eq:rest-frame-perfect-fluid-t}. Another way to see the point is to note that velocity of the fluid is 
different on different points, and a global Lorentz transformation cannot turn the fluid into the state of rest. 
We can do local Lorentz transformation, but this distorts the components of $\eta^{\mu \nu}$, but when deriving 
\eqref{eq:rest-frame-perfect-fluid-t} we have $\eta = \diag(-1, 1, 1, 1)$.  

The generic covariant energy-momentum tensor of a perfect fluid is % TODO: derivation
\begin{equation}
    T^{\mu \nu} = \left( \rho_m + \frac{p}{c^2} \right) U^\mu U^\nu + p \eta^{\mu \nu} = \frac{1}{c^2} \left( \epsilon + p \right) U^\mu U^\nu + p \eta^{\mu \nu}.
    \label{eq:general-fluid}
\end{equation}
Note that $p$ and $\epsilon$ in \eqref{eq:general-fluid} are defined in a special frame of reference, but this does 
not eliminate the covariance of \eqref{eq:general-fluid}, because for a fluid there is indeed a special frame of reference, i.e. the rest frame of itself. 

Finally we check whether \eqref{eq:general-fluid} reduces to \eqref{eq:x-moving-fluid-t} if the velocity 
of the fluid is globally $- v \vu*{e}$. When in the rest frame of reference, we have 
\begin{equation}
    U^\mu = \pmqty{c \\ 0 \\ 0 \\ 0},
\end{equation}
and \eqref{eq:general-fluid} reads 
\[
    T^{\mu \nu} = \frac{1}{c^2} (\epsilon + p) \pmqty{\dmat{c^2, 0, 0, 0}} + p \pmqty{\dmat{-1, 1, 1, 1}}, 
\]
which is just \eqref{eq:rest-frame-perfect-fluid-t}. After a global Lorentz transformation \eqref{eq:lorentz-x}, 
we have 
\[
    U'^\mu = \Lambda\indices{^\mu_\nu} U^\nu = \pmqty{\gamma c \\ - \gamma \beta c \\ 0 \\ 0}.
\]
Substituting this into \eqref{eq:general-fluid}, we indeed come back to \eqref{eq:x-moving-fluid-t}. (The process is in \href{relativistic-ideal-fluid.nb}{this Mathematica notebook}). 

\section{Electromagnetism}

\subsection{The action}

In this section we try to establish a relativistic covariant version of electromagnetism.
First, the EOM of particles 
\begin{equation} \marginnote{(4.1)}
    m \ddot{\vb*{r}} = e \vb*{E} + \frac{e}{c} \dot{\vb*{r}} \times \vb*{B} 
\end{equation}
has to change, because it allows particles to be accelerated without an upper bound. 
Second, the Maxwell equations must be written into a covariant form. \marginnote{(4.2) to (4.5)}
We've already done this in \href{../advanced-electrodynamics/lecture-12-15.pdf}{this note}, 
but here we need to repeat the procedure with $g_{\mu \nu} = \diag(-1, 1, 1, 1)$ and in the Gaussian 
unit system. 

We will find the action 
\begin{equation}
    \begin{aligned}
        S&=S_{\mathrm{m}}+S_{\mathrm{int}}+S_{\mathrm{em}}, \\
        S_{\mathrm{m}} &=-m c \int_{\Gamma} \sqrt{-\dd s^{2}}, \\
        S_{\mathrm{int}} &=\frac{e}{c} \int_{\Gamma} A_{\mu} \dd x^{\mu}, \\
        S_{\mathrm{em}} &=-\frac{1}{16 \pi c} \int_{\Omega} F^{\mu v} F_{\mu v} \dd^{4} \Omega
    \end{aligned}
\end{equation}
recovers to the relativistic version of Newton's second law and Maxwell equations.  \marginnote{(4.15) and (4.12)}
Here $\dd[4]{\Omega} = c \dd{t} \dd[3]{\vb*{r}}$.
Note that the definition of $F_{\mu \nu}$ here differs with $F_{\mu \nu}$ with $g_{\mu \nu} = (1, -1, -1, -1)$
with a global minus sign.

The interaction action $S_{\text{int}}$ can also be written into a ``hydrodynamic'' form. \marginnote{(4.19)}
Before doing so, we need to find a relativistic description of flowing. We start from a collective quantity that 
is invariant between frames of reference, which is total electric charge here. 
The density is 
\begin{equation}
    \rho = \dv{Q}{V} = \gamma \dv{Q}{V_0}
\end{equation}
and is not a Lorentz scalar (because $\dv*{Q}{V_0}$ is one). We tentatively define a manifestly 4-vector 
\begin{equation}
    J^\mu = \dv{Q}{V_0} U^\mu = \frac{\rho}{\gamma} (\gamma c, \gamma \vb*{v}) = (\rho c, \rho \vb*{v}),
\end{equation}
where $U^\mu$ in the many-body case is the coarse-grained 4-velocity as defined in \eqref{eq:4-velocity-single},
and in the case where there is only one particle is just \eqref{eq:4-velocity-single}. 
It can be immediately found that $\partial_\mu J^\mu = 0$, and therefore $J^\mu$ is a good definition of 4-current.

Now in the case with only one particle in the electrodynamic field, we have 
\[
    \rho(\vb*{r}', t) = e \delta(\vb*{r}' - \vb*{r}(t)), \quad \Gamma = \{ (t, \vb*{r}(t)) \}_t,
\]
so
\[
    \begin{aligned}
        S_\text{int} &= \frac{1}{c} \int \rho \dd{V}  \int_\Gamma \dd{x^\mu} A_\mu = \frac{1}{c} \int \rho \dd{V} \int_{t_1}^{t_2} A_\mu \dv{x^\mu}{t} \dd{t} \\
        &= \frac{1}{c^2} \int_\Omega \underbrace{c \dd{t} \dd{V}}_{\dd[4]{x}} \rho \dv{x^\mu}{t} A_\mu \\
        &= \frac{1}{c^2} \int_\Omega \dd[4]{x} J_\text{single particle}^\mu A_\mu .
    \end{aligned}
\]
Since $J^\mu$ in a continuum is just the coarse-grained version of $\sum J^\mu_\text{single particle}$,
the many-body version of $S_\text{int}$ is 
\begin{equation} \marginnote{(4.19)}
    S_\text{int} = \frac{1}{c^2} \int_\Omega \dd[4]{x} J^\mu A_\mu.
\end{equation}

\subsection{Maxwell equations}

Maxwell equations don't need to be generalized, because they are already invariant.

\section{Riemannian geometry}

\subsection{Christoffel symbol and covariant derivative}

Here we briefly list some important formulae:
\begin{equation} \marginnote{(5.15)}
    \nabla_{\nu} u^{\mu}= \partial_\nu u^\mu +\Gamma_{\nu \rho}^{\mu} u^{\rho},
    \label{eq:u-derivative}
\end{equation}
and from this and $\nabla_\nu (u^\mu w_\mu) = 0$, recursively we have 
\begin{equation} \marginnote{(5.50)}
    \begin{aligned}
        \nabla_{\lambda} T_{\nu_{1} \nu_{2} \ldots \nu_{s}}^{\mu_{1} \mu_{2} \ldots \mu_{r}}=\frac{\partial}{\partial x^{\lambda}} T_{\nu_{1} \nu_{2} \ldots \nu_{s}}^{\mu_{1} \mu_{2} \ldots \mu_{r}}
        &\underbrace{+\Gamma_{\lambda \sigma}^{\mu_{1}} T_{\nu_{1} \nu_{2} \ldots \nu_{s}}^{\sigma \mu_{2} \ldots \mu_{r}}+\Gamma_{\lambda \sigma}^{\mu_{2}} T_{\nu_{1} \nu_{2} \ldots \nu_{s}}^{\mu_{1} \sigma \ldots \mu_{r}}+\ldots+\Gamma_{\lambda \sigma}^{\mu_{r}} T_{\nu_{1} \nu_{2} \ldots \nu_{s}}^{\mu_{1} \mu_{2} \ldots \sigma}}_{r \text { terms }} \\
        &\underbrace{-\Gamma_{\lambda \nu_{1}}^{\sigma} T_{\sigma \nu_{2} \ldots \nu_{s}}^{\mu_{1} \mu_{2} \ldots \mu_{r}}-\Gamma_{\lambda \nu_{2}}^{\sigma} T_{\nu_{1} \sigma \ldots \nu_{s}}^{\mu_{1} \mu_{2} \ldots \mu_{r}}-\ldots-\Gamma_{\lambda \nu_{s}}^{\sigma} T_{\nu_{1} \nu_{2} \ldots \sigma}^{\mu_{1} \mu_{2} \ldots \mu_{r}}}_{s \text { terms }},
        \end{aligned}
        \label{eq:multiple-index-derivative}
\end{equation}
and specifically, we have 
\begin{equation} \marginnote{(5.49)}
    \nabla_{\mu} W_{\nu}=\frac{\partial W_{\nu}}{\partial x^{\mu}}-\Gamma_{\mu \nu}^{\rho} W_{\rho}.
    \label{eq:w-derivative}
\end{equation}
These formulae can be remembered using the following key points:
\begin{itemize}
    \item $\nabla_\nu u^\mu$ is used as a definition of covariant derivative and hence the sign before the $\Gamma$ term is +. 
    \item There is only one upper index $\mu$, and it can't be on $u$ in the $\Gamma$ term. 
    The lower index $\nu$ must be reflected on the Christoffel symbol. 
    So the $\Gamma$ term has to be the contraction of $\Gamma^\mu_{\nu \cdot}$ and $u$, and hence $\Gamma^\mu_{\nu \rho} u^\rho$.
    \item Then from $\nabla_\nu (u^\mu w_\mu) = 0$ we get \eqref{eq:w-derivative}. The $-$ sign comes from the fact 
    that the two $\Gamma$ terms in \eqref{eq:u-derivative} and \eqref{eq:w-derivative} must cancel with each other.
    There are now two lower indices $\mu, \nu$, and neither of them should be on $W$ (or otherwise the upper index
    of $\Gamma$ has no lower index to contract with). So both of them should be on $\Gamma$, and the $\Gamma$ 
    term is therefore $- \Gamma_{\mu \nu}^\rho W_\rho$.
    \item Then \eqref{eq:multiple-index-derivative} can be obtained by remembering that $\lambda$ is always on the $\Gamma$ symbol, $r$ terms are like \eqref{eq:u-derivative}, and $s$ terms are like \eqref{eq:w-derivative}.
\end{itemize}

The Christoffel symbol is given by 
\begin{equation} \marginnote{Conventions}
    \Gamma_{\nu \rho}^{\mu}=\frac{1}{2} g^{\mu \lambda}\left(\frac{\partial g_{\lambda \rho}}{\partial x^{\nu}}+\frac{\partial g_{\nu \lambda}}{\partial x^{\rho}}-\frac{\partial g_{\nu \rho}}{\partial x^{\lambda}}\right).
\end{equation}
This formula can be remembered by noticing  
\begin{itemize}
    \item it starts with $\frac{1}{2} g^{\mu \lambda}$. The $\mu$ index is an upper index, so it must come with 
    another upper index $\lambda$, which, then, have to be contracted with a lower index in the brackets.
    \item The first and the second term can be obtained by placing the lower $\lambda$ on $g$. The third term 
    is obtained by placing $\lambda$ on $x$.
    \item There are two positive terms, so the signs have to be $+, +, -$, because after $\nu \leftrightarrow \rho$,
    the first and the second terms swaps, so they must bear the same sign.
\end{itemize}
From this we find (5.52). \marginnote{(5.52)} Here we give a step by step derivation of $\nabla_\sigma g^{\mu \nu}$.
\begin{itemize}
    \item Using the tricks described above, we have 
    \[
        \nabla_\sigma g^{\mu \nu} = \partial_\sigma g^{\mu \nu} + \Gamma^\mu_{\sigma \delta} g^{\delta \nu} + \Gamma^\nu_{\sigma \delta} g^{\delta \mu}.
    \]
    Note that the second and third terms only differ with $\mu \leftrightarrow \nu$.
    \item Using the tricks described above, we have 
    \[
        \Gamma^\mu_{\sigma \delta} = \frac{1}{2} g^{\mu \lambda} (\partial_\sigma g_{\lambda \delta} + \partial_\delta g_{\lambda \sigma} - \partial_\lambda g_{\sigma \delta}).
    \]
    \item Both $\partial_\sigma g^{\mu \nu}$ and $\partial_\sigma g_{\mu \nu}$ appear. Since 
    \[
        \partial_\sigma (g_{\lambda \delta} g^{\delta \alpha}) = \partial_\sigma \delta^\alpha_\lambda = 0,
    \]
    we have 
    \[
        g_{\lambda \delta} \partial_\sigma g^{\delta \alpha} + g^{\delta \alpha} \partial_\sigma g_{\lambda \delta} = 0.
    \]
    By multiply $g_{\alpha \beta}$ to the equation (i.e. taking the inverse of $g^{\delta \alpha}$), we have 
    \[
        \partial_\sigma g_{\lambda \beta} = - g_{\alpha \beta} g_{\lambda \delta} \partial_\sigma g^{\delta \alpha}.
    \]
    \item Therefore \[
        \begin{aligned}
            \Gamma^\mu_{\sigma \delta} g^{\delta \nu} &= \frac{1}{2} g^{\mu \lambda} g^{\delta \nu} (- g_{\alpha \delta} g_{\lambda \beta} \partial_\sigma g^{\beta \alpha} - g_{\alpha \sigma} g_{\lambda \gamma} \partial_{\delta} g^{\gamma \alpha} + g_{\alpha \delta} g_{\sigma \gamma} \partial_\lambda g^{\lambda \alpha}) \\
            &= - \frac{1}{2} \partial_\sigma g^{\mu \nu} - \frac{1}{2} g^{\delta \nu} g_{\alpha \sigma} \partial_\delta g^{\mu \alpha} + \frac{1}{2} g^{\mu \lambda} g_{\sigma \gamma} \partial_\lambda g^{\gamma \nu} .
        \end{aligned}
    \] 
    Swapping $\mu$ and $\nu$, we have 
    \[
        \Gamma^\nu_{\sigma \delta} g^{\delta \mu} = - \frac{1}{2} \partial_\sigma g^{\mu \nu} - \frac{1}{2} g^{\delta \mu} g_{\alpha \sigma} \partial_\delta g^{\nu \alpha} + \frac{1}{2} g^{\nu \lambda} g_{\sigma \gamma} \partial_\lambda g^{\gamma \mu}.
    \]
    So we find 
    \[
        \Gamma^\mu_{\sigma \delta} g^{\delta \nu} + \Gamma^\nu_{\sigma \delta} g^{\delta \mu} = - \partial_\sigma g^{\mu \nu}, 
    \]
    and therefore we complete the proof.
\end{itemize}

\subsection{The geodesic equation}

If a curve $x^\mu(t)$ (here $t$ is just a parameter and not necessarily the time) is a geodesic, the following 
equivalent conditions hold:
\begin{itemize}
    \item The \concept{geodesic equation} 
    \begin{equation}
        \partial_t^2 x^\mu + \Gamma^\mu_{\nu \sigma} \partial_t x^\nu \partial_t x^\sigma = 0
        \label{eq:geodesic-eq}
    \end{equation}
    holds.
    \item The tangent vector satisfies 
    \begin{equation}
        T^\mu \nabla_\mu T^\nu = 0.
        \label{eq:geodesic-tangent}
    \end{equation}
    Note that $T^\mu = \partial_t x^\mu$, because $T^a$ is just $\partial_t$, and 
    \begin{equation}
        T^\mu = (\dd{x^\mu})_a T^a = \partial_t \text{ acting on } x^\mu = \partial_t x^\mu.
    \end{equation}
    \item The curve is a stationary solution of the variational problem 
    \begin{equation}
        0 = \var{\int \dd{s}} = \var{\int \sqrt{ \abs{g_{\mu \nu} \dd{x^\mu} \dd{x^\nu}} } }.
    \end{equation}
\end{itemize}

\eqref{eq:geodesic-tangent} is just a precise version of \eqref{eq:geodesic-eq}, because 
\[
    \begin{aligned}
        0 &= T^\mu \nabla_\mu T^\nu = T^\mu (\partial_\mu T^\nu + \Gamma^{\nu}_{\mu \sigma} T^\sigma) \\
        &= T^\mu \partial_\mu T^\nu + \Gamma^{\nu}_{\mu \sigma} T^\mu T^\sigma \\
        &= \partial_t T^\nu  + \Gamma^{\nu}_{\mu \sigma} T^\mu T^\sigma \\
        &= \partial_t^2 x^\nu + \Gamma^{\nu}_{\mu \sigma} \partial_t x^\mu \partial_t x^\sigma.
    \end{aligned}
\]

It's not hard to remember these equations. \eqref{eq:geodesic-tangent} is quite a clear one, and so is 

\subsection{Riemann tensor}

Now we go on to Riemann tensor. \marginnote{(5.80) to (5.83)} First we explain the equivalence between (5.67) and (5.83) (if we take the 
latter as a definition of the Riemann tensor and ignore its explicit expression in the second equation).
The transport of $u^\mu$ along an infinitesimal vector $p^\mu$, (5.80), is derived from 
\[
    \begin{aligned}
        0 &= p^\mu \nabla_\mu u^\nu = p^\mu (\partial_\mu u^\nu + \Gamma^\nu_{\mu \sigma} u^\sigma) \\
        &= u^\nu_{A \to B} - u^\nu + p^\mu \Gamma^\nu_{\mu \sigma} u^\sigma
    \end{aligned}
\]
and therefore 
\begin{equation} \marginnote{(5.80)}
    u^\nu_{A \to B} = u^\nu - p^\mu \Gamma^\nu_{\mu \sigma} u^\sigma.
\end{equation}
Now we have 
\[
    u^\mu_{A \to B \to D} = (1 + q^\nu \nabla_\nu) u^\mu_{A \to B} = (1 + q^\nu \nabla_\nu) (1 + p^\rho \nabla_\rho) u^\mu,
\]
and similarly
\[
    u^\mu_{A \to C \to D} = (1 + p^\rho \nabla_\rho) (1 + q^\nu \nabla_\nu) u^\mu.
\]
So 
\[
    \begin{aligned}
        u^\mu_{A \to B \to D} - u^\mu_{A \to C \to D} &= q^\nu p^\rho (\nabla_\nu \nabla_\rho - \nabla_\rho \nabla_\nu) u^\mu \\
        &= q^\nu p^\rho (\nabla_\nu \nabla_\rho - \nabla_\rho \nabla_\nu) u_\sigma g^{\mu \sigma} \\ 
        &= q^\nu p^\rho g^{\mu \sigma} R\indices{^\lambda_{\sigma \rho \nu}} u_\lambda .
    \end{aligned}
\]
The equivalence between (5.67) and (5.83) is therefore reduced to the following equation:
\begin{equation}
    R\indices{^\mu_{\tau \nu \rho}} u^\tau q^\nu p^\rho = q^\nu p^\rho g^{\mu \sigma} R\indices{^\lambda_{\sigma \rho \nu}} u_\lambda.
\end{equation}
We have 
\[
    \begin{aligned}
        \text{LHS} &= g^{\mu \sigma} R_{\sigma \tau \nu \rho} u^\tau q^\nu p^\rho = g^{\mu \sigma} R_{\tau \sigma \rho \nu} u^\tau q^\nu p^\rho, \marginnote{(5.77)} \\
        \text{RHS} &= q^\nu p^\rho g^{\mu \sigma} g^{\tau \lambda} R\indices{_\tau_{\sigma \rho \nu}} u_\lambda = q^\nu p^\rho g^{\mu \sigma} R\indices{_\tau_{\sigma \rho \nu}} u^\tau = \text{LHS},
    \end{aligned}
\]
so indeed (5.67) and (5.83) are equivalent as definitions of the Riemann tensor.

From the definition of the Riemann tensor, we have 
\begin{equation}
    R_{\mu \nu \rho \sigma} = - R_{\nu \mu \rho \sigma} = - R_{\mu \nu \sigma \rho} = R_{\rho \sigma \mu \nu}.
    \label{eq:r-tensor-symmetry}
\end{equation}
This equation helps us remember the definition of the Riemann tensor $R\indices{^\mu_{\nu \rho \sigma}}$:
\begin{itemize}
    \item There are two $\partial_\Box \Gamma^\mu_{\nu \Box}$ terms. Here we are sure that both $\mu$ and $\nu$
    appear on $\Gamma$, because the $\partial \Gamma$ term comes from the $\partial_\rho \Gamma^\mu_{\sigma \nu} A_\mu$ term in $\nabla_\rho \nabla_\sigma A_\mu$. 
    The signs before the two terms are opposite, because $R_{\mu \nu \rho \sigma} = - R_{\mu \nu \sigma \rho}$.
    \item The two boxes are to be filled with $\rho$ and $\sigma$ conforming to the order $\mu \nu \rho \sigma$.
    So we have 
    \[
        \partial_\rho \Gamma^\mu_{\nu \sigma} - \partial_\sigma \Gamma^\mu_{\nu \rho}.
    \]
    \item The third and fourth term definitely involve an upper $\mu$ index. Since the two $\Gamma$ symbols 
    have to contract with each other, they are in the form of $\Gamma^\mu_{\lambda \Box} \Gamma^\lambda_{\nu \Box}$.
    Here we are sure $\mu$ and $\nu$ are on different $\Gamma$'s because if they are on the same $\Gamma$, then 
    the only possible term is $\Gamma^{\mu}_{\lambda \nu} \Gamma^{\lambda}_{\rho \sigma}$, which is symmetric and 
    not antisymmetric.
    \item Again we fill the boxes according to the order $\rho \sigma$, and get 
    \[
        \Gamma^\mu_{\lambda \rho} \Gamma^\lambda_{\nu \sigma} - \Gamma^\mu_{\lambda \sigma} \Gamma^\lambda_{\nu \rho}.
    \]
    \item So here we get 
    \begin{equation}
        R\indices{^\mu_{\nu \rho \sigma}} = \partial_\rho \Gamma^\mu_{\nu \sigma} - \partial_\sigma \Gamma^\mu_{\nu \rho} + \Gamma^\mu_{\lambda \rho} \Gamma^\lambda_{\nu \sigma} - \Gamma^\mu_{\lambda \sigma} \Gamma^\lambda_{\nu \rho}.
    \end{equation}
\end{itemize}

\subsection{Ricci tensor, scalar curvature and Einstein tensor}

Since \eqref{eq:r-tensor-symmetry} holds, if we want to contract two indices of $R\indices{^\mu_{\nu \rho \sigma}}$
to obtain a rank-2 tensor, the only choice is to contract $\mu$ and $\rho$: contracting $\mu$ and $\nu$
or $\rho$ and $\sigma$ results in zero, and we have 
\[
    \underbrace{g^{\mu \rho} R_{\mu \nu \rho \sigma}}_{R\indices{^\mu_{\nu \mu \rho}}} = 
    g^{\mu \rho} R_{\nu \mu \sigma \rho} ,
\]
the RHS can be seen as contracting $\nu$ and $\sigma$. So we define the \concept{Ricci tensor}
\begin{equation} \marginnote{(5.84), (5.85)}
    R_{\nu \sigma} = R\indices{^\mu_{\nu \mu \sigma}}, \quad R_{\mu \nu} = R_{\nu \mu}.
\end{equation}
The trace of the Ricci tensor 
\begin{equation} \marginnote{(5.86)}
    R = R\indices{^\mu_\mu}.
\end{equation}
is called the \concept{scalar curvature}.

Through long and tedious proof we get the \concept{first and second Bianchi idensities}:
\begin{equation}
    R\indices{^\mu_{\nu \rho \sigma}} + R\indices{^\mu_{\sigma \nu \rho}} + R\indices{^\mu_{\rho \sigma \nu}} = 0,
\end{equation}
and 
\begin{equation}
    \nabla_\mu R\indices{^\kappa_{\lambda \nu \rho}} + \nabla_\rho R\indices{^\kappa_{\lambda \mu \nu }} + \nabla_\nu R\indices{^\kappa_{\lambda \rho \mu}} = 0.
    \label{eq:second-bianchi}
\end{equation}
They can be remembered by rotating the indices. 

From \eqref{eq:second-bianchi}, we have 
\[ \marginnote{(5.96)}
    \begin{aligned}
        0 &= \delta^\kappa_\nu (\nabla_\mu R\indices{^\kappa_{\lambda \nu \rho}} + \underbrace{\nabla_\rho R\indices{^\kappa_{\lambda \mu \nu }}}_{- \nabla_\rho R\indices{^\kappa_{\lambda \nu \mu}}} + \nabla_\nu R\indices{^\kappa_{\lambda \rho \mu}}) \\
        &= \nabla_\mu R_{\lambda \rho} - \nabla_\rho R_{\lambda \mu} + \nabla_\kappa R\indices{^\kappa_{\lambda \rho \mu}},
    \end{aligned}
\]
and therefore 
\[ \marginnote{(5.97)}
    \begin{aligned}
        0 &= g^{\lambda \rho} (\nabla_\mu R_{\lambda \rho} - \nabla_\rho R_{\lambda \mu} + \nabla_\kappa R\indices{^\kappa_{\lambda \rho \mu}}) \\
        &= \nabla_\mu R - \nabla_\lambda R\indices{^\lambda_\mu} + g^{\lambda \rho} \nabla_\kappa g^{\kappa \sigma} R_{\sigma \lambda \rho \mu} \\
        &= \nabla_\mu R - \nabla_\lambda R\indices{^\lambda_\mu} - \nabla_\kappa g^{\kappa \sigma} R_{\sigma 
        \mu} \\
        &=  \underbrace{\nabla_\mu R}_{\nabla_\lambda \delta^\lambda_\mu R} - \nabla_\lambda R\indices{^\lambda_\mu} - \nabla_\kappa R\indices{^\kappa_\mu} ,
    \end{aligned}
\]
\[
    \nabla_{\mu} (g^{\mu \nu} R - 2 R^{\mu \nu}) = 0,
\]
so 
\begin{equation}
    \nabla_\mu \underbrace{\left( R^{\mu \nu} - \frac{1}{2} g^{\mu \nu} R \right)}_{\eqqcolon G^{\mu \nu}} = 0.
    \label{eq:einstein-tensor}
\end{equation}
We name $G^{\mu \nu}$ the \concept{Einsten tensor}.

\section{Physics with a background gravitational field}

Note that \cite{bambi2018introduction} uses the term \emph{general relativity} to denote any metric theory of 
gravity. 

\subsection{Absorbing inertial forces into the metric}

\subsection{Absorbing Newtonian gravity into the metric}

In the Newtonian limit, suppose the gravitational potential is $\Phi$, the Lagrangian is 
\begin{equation} \marginnote{(5.3)}
    L = - m c^2 + \frac{1}{2} m \vb*{v}^2 - m \Phi.
\end{equation}
Note that $m c^2 \gg m \vb*{v}^2 / 2, m \Phi$. \marginnote{(5.4)}
This Lagrangian has a natural high-speed completion:
\begin{equation}
    L = - m c \sqrt{c^2 - \vb*{v}^2 + 2 \Phi} = - m c^2 \left( 1 + \frac{- \vb*{\vb*{v}^2} + 2 \Phi}{c^2} + \bigO(\vb*{v}^4 / c^4, \Phi^2 / c^4) \right) .
\end{equation}
Note that this is the Lagrangian of a free-falling particle, and if it's a metric theory, it can be rephrased into 
\begin{equation}
    S = - m c \int \sqrt{- g_{\mu \nu} \dot{x^\mu} \dot{x^\nu}} \dd{t}, \quad \dot{x}^\mu \coloneqq \dv{x^\mu}{t}.
\end{equation}
Now obviously, we have 
\begin{equation} \marginnote{(5.7)}
    g_{\mu \nu} = \pmqty{\dmat{ -\left(1 + \frac{2\Phi}{c^2}\right), 1, 1, 1 }}. \label{eq:metric-with-phi}
\end{equation}
So we find Newtonian gravity is a low-speed, low gravitational force limit of a metric theory with metrics
\eqref{eq:metric-with-phi}. 

Note that it's not necessary that in the low speed limit of the particle and weak gravitational field limit
(actually, since gravitational force can be used to accelerate the particle, a low particle speed already 
implies a weak gravitational field) the only kind of gravitational potential is the Newtonian one.
For example, see \href{https://en.wikipedia.org/wiki/Parameterized\_post-Newtonian\_formalism}{the PPN formalism}.
This is because it's possible that $g_{\mu \nu}$ has components deviate from the Lorentz metrics other 
than $g_{00}$. However, even \emph{with} the presence of such components, Newtonian gravity is \emph{still} 
the low-speed weak-field theory of gravity, because a particle moving slow enough is \emph{unable to feel} 
these components. This is what is actually shown in Section 6.3. \marginnote{(6.4) to (6.14)}
We see that the derivation ends in (6.14), which is the $g_{00}$ component of \eqref{eq:metric-with-phi} 
and is only about $g_{tt}$ but not other components -- but other components are \emph{not relevant}, anyway.

To see so, we repeat the derivation in Sec. 6.3. 
Consider a particle that moves much slower than $c$ in a weak, static gravitational field. \marginnote{(6.4) to (6.6)}
The geodesic equation is 
\[
    \ddot{x}^\mu + \Gamma^\mu_{\sigma \rho} \dot{x}^\sigma \dot{x}^\rho = 0.
\]
Since $\dot{x^i} \ll c$, and $\dot{x^0} = c \dot{t}$, approximately 
\begin{equation} \marginnote{(6.7)}
    \ddot{x}^\mu + \Gamma^{\mu}_{tt} c^2 \dot{t}^2 = 0.
    \label{eq:slow-speed-eom}
\end{equation}
By definition, and considering $\partial_t g_{\mu \nu} = 0$ since the gravitational field is static, we have 
\[
    \begin{aligned}
        \Gamma^\mu_{tt} &= \frac{1}{2} g^{\mu \nu} (\partial_t g_{\nu t} + \partial_t g_{t \nu} - \partial_\nu g_{tt}) \\
        &= - \frac{1}{2} g^{\mu \nu} \partial_\nu g_{tt} = - \frac{1}{2} g^{\mu i} \partial_i g_{tt} \\
        &=  - \frac{1}{2} g^{\mu i} \partial_i h_{tt},
    \end{aligned}
\]
so 
\begin{equation} \marginnote{(6.8)}
    \Gamma^{t}_{tt} = - \frac{1}{2} g^{t i} \partial_{i} h_{tt} = - \frac{1}{2} h^{ti} \partial_i h_{tt} = \bigO(h^2),
\end{equation}
and 
\begin{equation} \marginnote{(6.8)}
    \Gamma^{i}_{tt} = - \frac{1}{2} g^{i j} \partial_j h_{tt} = - \frac{1}{2} \eta^{ij} \partial_j h_{tt} = - \frac{1}{2} \partial_i h_{tt}.
    \label{eq:gamma-i-tt}
\end{equation}
The EOM \eqref{eq:slow-speed-eom} is therefore 
\begin{equation} \marginnote{(6.9), (6.10)}
    \ddot{t} = 0, \quad \ddot{x}^i - \frac{1}{2} c^2 \dot{t}^2 \partial_i h_{tt} = 0.
\end{equation}
Since 
\[
    \begin{aligned}
        \ddot{x}^i &= \dv{t}{\tau} \dv{t} \left( \dv{t}{\tau} \dv{x^i}{t} \right) = \dot{t}^2 \dv[2]{x^i}{t} + \dv{x^i}{t} \dv{t}{\tau} \dv{t} \dv{t}{\tau} \\
        &= \dot{t}^2 \dv[2]{x^i}{t} + \dv{x^i}{t} \dv[2]{t}{\tau} = \dot{t}^2 \dv[2]{x^i}{t},
    \end{aligned}
\]
we have 
\begin{equation}
    \dv[2]{x^i}{t} = \frac{1}{2} c^2 \partial_i h_{tt}.
\end{equation}
Compare this with Newton's second law of an object in an external gravitational field 
\begin{equation}
    \dv[2]{x^i}{t} = - \pdv{\Phi}{x^i},
\end{equation}
and the fact that when $h_{\mu \nu} = 0$, there is no gravity and $\Phi = 0$, we have 
\begin{equation}
    h_{tt} = - \frac{2 \Phi}{c^2}. \label{eq:htt-and-gravity}
\end{equation}
This is just \eqref{eq:metric-with-phi}. Note that in the step of (6.7), other components of $\Gamma_{\mu \nu}^\sigma$
don't have the change to go into the EOM, because the corresponding $\dot{x^\mu}$ factors are so small.
In other words, these components can't be felt by a particle moving slowly enough.

\subsection{Local inertial reference frame and moving frame}

In this section we show the existence of a local inertial reference frame, i.e. to show that a metric theory 
always satisfy Einstein's equivalence principle. First, we can always diagonalize the metrics at a certain point
by the following coordinate transformation:
\begin{equation}
    \dd{x^\mu} \to \dd{x'^\mu} = e^{\mu'}_\nu \dd{x^\nu},
\end{equation}
where $e^{\mu'}_\nu$'s are obtained by the following diagonalization 
(the eigenvalues are absorbed into $e^{\mu'}_\nu$)
\begin{equation}
    g_{\mu \nu} = e^{\alpha'}_\mu e^{\beta'}_\nu \eta_{\alpha' \beta'}.
\end{equation}
So without loss of generality, we assume 
\begin{equation}
    g_{\mu \nu}(0) = \eta_{\mu \nu}.
\end{equation}
The metric tensor can be expanded as 
\begin{equation} \marginnote{(6.21)}
    g_{\mu v}(x)=g_{\mu v}(0)+\left.\frac{\partial g_{\mu v}}{\partial x^{\rho}}\right|_{0} x^{\rho}+\left.\frac{1}{2} \frac{\partial^{2} g_{\mu v}}{\partial x^{\rho} \partial x^{\sigma}}\right|_{0} x^{\rho} x^{\sigma}+\cdots.
\end{equation}
Now suppose we do the coordinate transformation 
\begin{equation} \marginnote{(6.22)}
    x^{\mu} \rightarrow x^{\prime \mu}=x^{\mu}+\frac{1}{2} \Gamma_{\rho \sigma}^{\mu}(0) x^{\rho} x^{\sigma} + \cdots.
\end{equation}
The inverse is 
\begin{equation}
    x^{\mu}=x^{\prime \mu}-\frac{1}{2} \Gamma_{\rho \sigma}^{\mu}(0) x^{\prime \rho} x^{\prime \sigma}+\cdots,
\end{equation}
so we have 
\begin{equation}
    \frac{\partial x^{\alpha}}{\partial x^{\prime \mu}}=\delta_{\mu}^{\alpha}-\Gamma_{\mu \nu}^{\alpha}(0) x^{\prime \nu}+\cdots.
\end{equation}
Under these transformations we find (6.29), so (6.26) evaluates to zero. \marginnote{(6.26) to (6.29)}
By the definition of $\Gamma^\sigma_{\mu \nu}$, after the transformation, we see all Christoffel symbols become zero, so locally, we get $\nabla_\mu \to g_\mu$, $g_{\mu \nu} \to \eta_{\mu \nu}$, and hence $\{x'^\mu\}$
is a local inertial reference frame.

The next question is what $\{x'^\mu\}$ actually is. Actually it's just the free-falling reference frame 
in the gravitational field without rotation. % TODO

\subsection{Time slowing down in gravitational field}

The general covariance principle implies that the time $\dd \tau$ measured by a 
clock (i.e. how many period a system with periodic behavior undergoes from one event on the system's world 
line to another event, which may also be called as the time ``felt by an object with identical trajectory 
with the clock'', because the EOM of the object with its parameter being the time measured by the clock in 
the inertial frame of reference reserves its form when in a gravitational field) has to be given by 
\begin{equation}
    c^2 \dd{\tau}^2 = - \dd{s^2} = - \eta_{\mu \nu} \dd{x^\mu} \dd{x^\nu},
    \label{eq:clock-inertial}
\end{equation}
because the only degrees of freedom are the coordinates of the clock, and we see $\dd{\tau}$ is just the proper time. 
By the equivalence principle, the local behavior of an object with an arbitrary trajectory in an arbitrary
gravitational field in its local inertial frame of reference is the same as the behavior in a real inertial 
frame of reference (without gravity). So for a clock in a gravitational field, in its local inertial frame 
of reference, the time it measures has to resemble \eqref{eq:clock-inertial}, and we have 
\[
    - c^2 \dd{\tau^2} = \dd{s^2} = \underbrace{g_{\mu \nu} \dd{x^\mu} \dd{x^\nu} = \eta_{\mu \nu} \dd{x^\mu} \dd{x^\nu} }_{\text{in local inertial frame of reference}},
\]
so again by the principle of general covariance, the time measured by the clock is 
\begin{equation}
    c^2 \dd{\tau^2} = - \dd{s^2} = - g_{\mu \nu} \dd{x^\mu} \dd{x^\nu}
\end{equation}
in \emph{any} coordinates. We also call this quantity the proper time.

As long as gravity is created by the metrics, clocks placed in a gravitational field is slower than a clock 
far from any gravity source. For example, with \eqref{eq:metric-with-phi}, we have 
\begin{equation} \marginnote{(6.34)}
    \dd{\tau}^2 = \left(1 + \frac{2 \Phi}{c^2} \right) \dd{t}^2. \label{eq:time-slow-grativity}
\end{equation} 
When the metrics is not that simple, but the clock moves very slowly, we still have \eqref{eq:time-slow-grativity}.
Note that since gravity is always attractive, $\Phi < 0$, so $\dd{\tau} < \dd{t}$. This means with the 
presence of a gravitational field, in the eye of the lab frame of reference, the clock is slowed down:
when a long period of time has gone, the clock pointer only moves a little.
Note that $\dd{t}$ is \emph{independent} of $g_{\mu \nu}$, i.e. ``the time felt in one particular frame of reference''
is independent of the gravitational field and can be thought as the time felt by a remote observer away from 
all gravitational sources.

This is essential to keep the clock on a GPS satellite synchronous with the clock on the ground. 
Suppose the satellite moves with velocity $\vb*{v}$ at a particular time point. We have 
\[
    \dd{\tau_\text{satellite}} = \sqrt{1 + \frac{2 \Phi_\text{satellite}}{c^2}} \dd{t}_\text{space},
\]
where $\dd{t_\text{space}}$ is the time step of the reference system $S'$ moving with velocity $\vb*{v}$ with respect 
to the ground. A signal receiver on the ground is also in the gravitational field, and therefore 
\[
    \dd{\tau_\text{ground}} = \sqrt{1 + \frac{2 \Phi_\text{ground}}{c^2}} \dd{t_\text{ground}}.
\]
Now since $\dd{t_\text{space}}$ and $\dd{t_\text{ground}}$ are time steps in different coordinates, 
they can't be compared directly. However, since a GPS receiver works by receiving messages from satellites,
which record the time stamp when the message was sent, and comparing the current time and the time stamps 
to calculate the distance between the receiver and satellites to calculate the position of the receiver (the 
positions of satellites are given by the orbital information), and two time stamps sent at $t_\text{satellite}$ and
$t_\text{satellite} + \dd{t_\text{satellite}}$ in $S'$ are received by the receiver at $t_0$ and 
$t_0 + \gamma \dd{t_\text{satellite}}$, respectively, where 
\begin{equation}
    t_0 = t_\text{ground} + \frac{R}{c},
    \label{eq:time-to-r-gps}
\end{equation}
we find that if two time stamps are sent with time delay $\dd{\tau_\text{satellite}}$ with respect to the time 
felt by the satellite, then the time difference between the two time stamps is perceived by the receiver as 
\[
    \begin{aligned}
        \dd{\tau_\text{ground}} &= \sqrt{1 + \frac{2 \Phi_\text{ground}}{c^2}} \dd{t_\text{ground}} \\
        &=  \sqrt{1 + \frac{2 \Phi_\text{ground}}{c^2}} \frac{1}{\sqrt{1 - \frac{v^2}{c^2}}} \dd{t_\text{space}} \\
        &= \sqrt{1 + \frac{2 \Phi_\text{ground}}{c^2}} \frac{1}{\sqrt{1 - \frac{v^2}{c^2}}} \frac{1}{\sqrt{1 + \frac{2 \Phi_\text{ground}}{c^2}} } \dd{\tau_\text{satellite}} ,
    \end{aligned}
\]
and since the gravitational field of earth is very weak and so is $v / c$, we have 
\begin{equation} \marginnote{(6.39), (6.40)}
    \dd{\tau}_\text{ground} = \left( 1 + \frac{\Phi_\text{ground}}{c^2} - \frac{\Phi_\text{ground}}{c^2} + \frac{v^2}{2 c^2} \right) \dd{\tau_\text{satellite}}.
\end{equation}
So we find the time difference between messages sent by the satellite and the time difference between the 
messages received by the receiver are not synchronous, which is a combination of \emph{both} the special 
relativity effect and the gravitational effect. If we ignore this fact, a considerable systematic error will 
be introduced, because in this case, the receiver will just wrongly predict the position of the satellite
according to $\tau_\text{ground}$ 

% TODO: whether we need to use $t_\text{ground}$ to find the position of the satellite and use $\tau_\text{ground}$ to check the time stamp

The error between the two time differences will 
be wrongly included into $R$ in \eqref{eq:time-to-r-gps}. Since 

\section{Einstein's gravity}

It's possible to derive Einstein's gravity by field theoretic methods (see \grnote), 
but it's actually more convenient to derive the EOM directly. 

Here we briefly review the line of argumentation in Section 7.1. \marginnote{(Sec. 7.1)}
The list of assumptions is 
\begin{enumerate}
    \item The gravitational field is completely described by the metric tensor.
    \item The field EOMs are tensor equations: general covariance principle.
    \item The field EOMs are at most of second order.
    \item Newtonian limit
    \item $T^{\mu \nu}$ is the source.
    \item In the absence of matter, $g_{\mu \nu} \to \eta_{\mu \nu}$.
\end{enumerate}

From conditions 2 and 5 we have 
\begin{equation} \marginnote{(7.1)}
    G^{\mu \nu} = \kappa T^{\mu \nu},
\end{equation}
where $G^{\mu \nu}$ is a function of the metric tensor (according to condition 1). So now the problem is how to 
assemble a second order tensor with no more than second order derivative of $g_{\mu \nu}$ (according to 
condition 3). The simplest term is $\Lambda g_{\mu \nu}$. Since $R$ is a scalar assembled by 
$R\indices{^\mu_{\nu \sigma \delta}}$, which is made up by first and second order derivatives of the metric tensor,
$R g_{\mu \nu}$ is also a possible term. The so-called Einstein tensor 
\begin{equation}
    G^{\mu \nu} = R^{\mu \nu} - \frac{1}{2} R g^{\mu \nu}
\end{equation}
also satisfies these conditions. It can be proved that the linear combination of $g_{\mu \nu}$ and $R_{\mu \nu}$ are the only possibilities
in a four-dimensional spacetime with linear dependency on the second order derivative of $g_{\mu \nu}$ \cite{lovelock1972four}.
Absorbing the prefactor of $R^{\mu \nu}$ into $\kappa$, we get 
\begin{equation} \marginnote{(7.8)}
    R_{\mu \nu}-\frac{1}{2} g_{\mu \nu} R+\Lambda g_{\mu \nu} = \kappa T_{\mu \nu}.
    \label{eq:cosmos-constant}
\end{equation}
This is the EOM allowed by conditions 2 ,5 and 1, 3, and the assumption that the EOM is linear 
in the second order derivative of $g_{\mu \nu}$. In our $d=4$ spacetime without the cosmological
constant, we have 
\[
    R - 2 R = \kappa T,
\]
where $T$ is defined as the trace of $T^{\mu \nu}$, so 
\begin{equation}
    R_{\mu \nu} = \kappa \left(T_{\mu \nu} - \frac{1}{2} T g_{\mu \nu} \right).
    \label{eq:t-to-r}
\end{equation}

Due to \eqref{eq:einstein-tensor}, we have 
\begin{equation}
    \nabla_\mu T^{\mu \nu} = 0,
\end{equation}
which is just the generally covariant version of energy and momentum conservation. 

There are yet two conditions to be satisfied. Condition 6 is not satisfied if $\Lambda \neq 0$, 
because when $T_{\mu \nu} = 0$ and $g_{\mu \nu} = \eta_{\mu \nu}$, the whole Riemann tensor 
vanishes, and the LHS of \eqref{eq:cosmos-constant} is $\Lambda g_{\mu \nu}$, while the RHS vanishes.
So $\Lambda$ has to be zero if we want $g_{\mu \nu}$ to go back to the Lorentz metrics when 
there is no matter in the universe. So here we see the effect introduced by the $\Lambda$ term: 
it is just like a never-vanishing, homogenous energy-momentum. So $\Lambda$ is called the 
\concept{cosmological constant}, which distorts spacetime together with $T_{\mu \nu}$. 
Formally, we can absorb the $\Lambda g_{\mu \nu}$ into $T_{\mu \nu}$ to make the following derivation 
looks clearer. 

Now we derive the Newtonian limit to decide the value of $\kappa$. We have 
\[
    \begin{aligned}
        R_{tt} &= \partial_\mu \Gamma^\mu_{tt} - \partial_t \Gamma^\mu_{t \mu} + \Gamma^\mu_{\mu \lambda} \Gamma^\lambda_{tt} - \Gamma^\mu_{t \lambda} \Gamma^\lambda_{t \mu} \\
        &= \partial_i \Gamma^{i}_{tt} + \Gamma^\mu_{\mu \lambda} \Gamma^\lambda_{tt} - \Gamma^\mu_{t \lambda} \Gamma^\lambda_{t \mu}.
    \end{aligned}
\]
The second term in the first line vanishes because the Christoffel symbol only contains the metric tensor, 
which vanishes under derivative. Actually, it can be directly verified that $\Gamma^\mu_{\mu t} = 0$:
\[
    \begin{aligned}
        \Gamma^\mu_{t \mu} &= \frac{1}{2} g^{\mu \sigma} (\partial_t g_{\sigma \mu} + \partial_{\mu} g_{\sigma t} - \partial_\sigma g_{t \mu}) \\
        &= \frac{1}{2} g^{\mu \sigma} (\partial_{\mu} g_{\sigma t} - \partial_\sigma g_{t \mu}) = 0.
    \end{aligned}
\]
Similarly we have
\[
    \begin{aligned}
        \Gamma^\mu_{\mu i} &= \frac{1}{2} g^{\mu \sigma} (\partial_i g_{\sigma \mu} + \partial_\mu g_{\sigma i} - \partial_\sigma g_{i \mu}) \\
        &= \frac{1}{2} g^{\mu \sigma} \partial_i g_{\sigma \mu} = \frac{1}{2} g^{\mu \sigma} \partial_i h_{\sigma \mu}.
    \end{aligned}
\]
This means the third term is 
\[
    \Gamma^\mu_{\mu \lambda} \Gamma^\lambda_{tt} = \Gamma^\mu_{\mu i} \Gamma^i_{tt} = - \frac{1}{2} \partial_i h_{tt} \cdot \frac{1}{2} g^{\mu \sigma} \partial_i h_{\sigma \mu} = \bigO(h^2).
\]
The explicit form of the fourth term is hard to find, but since all Christoffel symbols involve derivatives
of the metric tensor, and $\partial_i g^{\mu \nu} = \bigO(h)$ and $\partial_t g^{\mu \nu} = 0$,
we know $\Gamma \sim h$, so the fourth term is also of order $\bigO(h^2)$ and can be thrown away. 
So finally we get (recall \eqref{eq:gamma-i-tt} and \eqref{eq:htt-and-gravity}) 
\begin{equation} \marginnote{(7.15), (7.16)}
    R_{tt} = \partial_i \Gamma^i_{tt} + \bigO(h^2) = - \frac{1}{2} \laplacian h_{tt} + \bigO(h^2) = \frac{1}{c^2} \laplacian \Phi.
\end{equation}
In the Newtonian limit, all gravitational sources move slowly, so the only important component of 
the energy-momentum tensor is 
\begin{equation} \marginnote{(7.13)}
    T^{tt} = \rho c^2.
\end{equation}
This, together with \eqref{eq:t-to-r}, means 
\[
    \frac{1}{c^2} \laplacian \Phi = \frac{1}{2} \kappa \rho c^2,
\]
so 
\begin{equation}
    \laplacian \Phi = \frac{c^4 \kappa}{2} \rho = 4 \pi G \rho, \quad \kappa = \frac{8 \pi G}{c^4}.
\end{equation}
So indeed the Newtonian limit of Einstein's gravity is the Newton's gravity, and we have also linked 
$\kappa$ and $G$ together.

\bibliographystyle{plain}
\bibliography{books-used}

\end{document}