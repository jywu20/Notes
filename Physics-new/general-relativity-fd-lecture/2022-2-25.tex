\documentclass[hyperref, a4paper]{article}

\usepackage{geometry}
\usepackage{titling}
\usepackage{titlesec}
% No longer needed, since we will use enumitem package
% \usepackage{paralist}
\usepackage{enumitem}
\usepackage{footnote}
\usepackage{marginnote}
\usepackage{enumerate}
\usepackage{amsmath, amssymb, amsthm}
\usepackage{mathtools}
\usepackage{bbm}
\usepackage{cite}
\usepackage{graphicx}
\usepackage{subfigure}
\usepackage{physics}
\usepackage{tensor}
\usepackage{siunitx}
\usepackage[version=4]{mhchem}
\usepackage{tikz}
\usepackage{xcolor}
\usepackage{listings}
\usepackage{autobreak}
\usepackage[ruled, vlined, linesnumbered]{algorithm2e}
\usepackage{nameref,zref-xr}
\zxrsetup{toltxlabel}
\zexternaldocument*[gr-]{../relativity/relativistic}[relativistic.pdf]
\zexternaldocument*[optics-]{../optics/optics}[optics.pdf]
\zexternaldocument*[solid-]{../solid/solid}[solid.pdf]
\zexternaldocument*[kt-]{../topological-phases-reading-notes/kt}[kt.pdf]
\usepackage[colorlinks,unicode]{hyperref} % , linkcolor=black, anchorcolor=black, citecolor=black, urlcolor=black, filecolor=black
\usepackage[most]{tcolorbox}
\usepackage{prettyref}

% Page style
\geometry{left=3.18cm,right=3.18cm,top=2.54cm,bottom=2.54cm}
\titlespacing{\paragraph}{0pt}{1pt}{10pt}[20pt]
\setlength{\droptitle}{-5em}
\preauthor{\vspace{-10pt}\begin{center}}
\postauthor{\par\end{center}}

% More compact lists 
%\setlist[itemize]{
%    itemindent=17pt, 
%    leftmargin=1pt,
%    listparindent=\parindent,
%    parsep=0pt,
%}

% Math operators
\DeclareMathOperator{\timeorder}{\mathcal{T}}
\DeclareMathOperator{\diag}{diag}
\DeclareMathOperator{\legpoly}{P}
\DeclareMathOperator{\primevalue}{P}
\DeclareMathOperator{\sgn}{sgn}
\newcommand*{\ii}{\mathrm{i}}
\newcommand*{\ee}{\mathrm{e}}
\newcommand*{\const}{\mathrm{const}}
\newcommand*{\suchthat}{\quad \text{s.t.} \quad}
\newcommand*{\argmin}{\arg\min}
\newcommand*{\argmax}{\arg\max}
\newcommand*{\normalorder}[1]{: #1 :}
\newcommand*{\pair}[1]{\langle #1 \rangle}
\newcommand*{\fd}[1]{\mathcal{D} #1}
\DeclareMathOperator{\bigO}{\mathcal{O}}

% TikZ setting
\usetikzlibrary{arrows,shapes,positioning}
\usetikzlibrary{arrows.meta}
\usetikzlibrary{decorations.markings}
\tikzstyle arrowstyle=[scale=1]
\tikzstyle directed=[postaction={decorate,decoration={markings,
    mark=at position .5 with {\arrow[arrowstyle]{stealth}}}}]
\tikzstyle ray=[directed, thick]
\tikzstyle dot=[anchor=base,fill,circle,inner sep=1pt]
\usetikzlibrary{fadings}
\usetikzlibrary{patterns}
\usetikzlibrary{shadows.blur}
\usetikzlibrary{shapes}

% Algorithm setting
% Julia-style code
\SetKwIF{If}{ElseIf}{Else}{if}{}{elseif}{else}{end}
\SetKwFor{For}{for}{}{end}
\SetKwFor{While}{while}{}{end}
\SetKwProg{Function}{function}{}{end}
\SetArgSty{textnormal}

\newcommand*{\concept}[1]{{\textbf{#1}}}

% Support for tensor double arrows.
\renewcommand{\tensor}[1]{ \stackrel{\leftrightarrow}{\vb*{#1}}}

% Embedded codes
\lstset{basicstyle=\ttfamily,
  showstringspaces=false,
  commentstyle=\color{gray},
  keywordstyle=\color{blue}
}

% Reference formatting
\newrefformat{fig}{Figure~\ref{#1} on page~\pageref{#1}}

% Color boxes
\tcbuselibrary{skins, breakable, theorems}
\newtcbtheorem[number within=section]{warning}{Warning}%
  {colback=orange!5,colframe=orange!65,fonttitle=\bfseries, breakable}{warn}
\newtcbtheorem[number within=section]{note}{Note}%
  {colback=green!5,colframe=green!65,fonttitle=\bfseries, breakable}{note}

% Correctly displaying equation number in section titles
\pdfstringdefDisableCommands{\def\eqref#1{(\ref{#1})}}

\newcommand{\grnote}{\href{../relativity/relativistic.pdf}{this note}}

\title{Prof. Bambi on General Relativity}
\author{Jinyuan Wu}

\begin{document}

\maketitle

This is a note about Prof. Cosimo Bambi's lecture on general relativity on February 25 and March 3, 2022.

This lecture is about the 1-2 chapters in \cite{bambi2018introduction}. Nothing quite interesting.
\cite{bambi2018introduction} itself is very detailed and it seems I don't really need to take much notes.

\section{The Christoffel symbols of spherical coordinates}

The Christoffel symbol of spherical coordinates, given in (1.83) in \cite{bambi2018introduction}, can be 
calculated automatically in \href{spherical-coordinate-differential-geometry.nb}{this Mathematica notebook}.

\section{Derivation of special relativity}

Section~2.1 and 2.2 seem to be based on Chapter 1 and 2 of Landau's book about field theory. 
The arguments have been summarized in Section~\ref{gr-sec:principle-special-relativity} in \grnote.

Section~2.3 derives the Lorentz transformations by Wick rotation of Euclidean rotation in $d = 4$
-- (2.18) is actually just $\tau = \ii t$ in condensed matter physics.
A rotation on $xy$ has the form of (2.20) and we have $C_4^2 = 6$ rotations. Adding 4 translations,
we get the total 10 generators of the rotation group in $\mathbb{R}^4$, and hence the Lorentz transformations
in the (3+1)-dimensional Minkowski spacetime.

Equations from (2.22) to (2.27) are trying to relate the parameter of $\mathbb{R}^4$ rotations to the relative 
velocity of the two reference frames.

Note that since the rotation group $SO(4)$ is not Abelian, Lorentz transformations do not commute in general. \marginnote{Problem~2.3}

\section{Aberration of light and superluminal motion}

We should note that \emph{Lorentz contraction} of length (see for example (2.43)) is often \emph{not} 
\marginnote{(2.48) to (2.57)} the length change we \emph{observe} when an object is moving fast.
The point here is that to measure a line we need to detect light signals from its two ends, and generally 
speaking, two signals arriving at our detector start their journeys at different time points, while on the 
other hand, from the way we derive Lorentz contraction ((2.41) and (2.42)) \marginnote{(2.41) and (2.42)}
we are dealing with events happening at the same time point in the laboratory frame of reference. 
The conclusion is $l$ in (2.43) is often not the length we \emph{see} of a moving object.

The fact that the ``distance'' we see is actually not the authentic space distance between two events with the 
same time in a given frame of reference means that when calculating the velocity, we are taking the time
derivative of two points at different time points, or in other words, taking the derivative of a distance 
with respect to a time not coherent with the current frame of reference. In this way, some superfluous
``superluminal'' movement can occur. Consider, for example, the case of \prettyref{fig:superluminal}. 
Suppose at $t$ a beam of light is emitted from the ejected material, and it arrives at the detector at $t'$.
We therefore have 
\begin{equation} \marginnote{(2.51), (2.52)}
    \begin{aligned}
        c(t' - t) &= L = \sqrt{(D - vt \cos \varphi)^2 + v^2 t^2 \sin^2 \varphi} \\
        &= D - vt \cos \varphi + \bigO(v^2 t^2 / D^2).
    \end{aligned}
\end{equation}
Therefore, we have 
\begin{equation} \marginnote{(2.53)}
    t' = \frac{D}{c} + t (1 - \beta \cos \varphi).
\end{equation}
Now we try to evaluate the \emph{apparent} velocity on the $y$ direction, which is $\dv*{y}{t'}$.
Note that only $\dv*{y}{t}$ is bounded by $c$, while $\dv*{y}{t'}$ does not have an upper bound.
Actually, we have 
\begin{equation} \marginnote{(2.54)}
    \dv{y}{t'} = \dv{y}{t} \dv{t}{t'} = v \sin \varphi \times \frac{1}{1 - \beta \cos \varphi}.
\end{equation}
The maximum is shown to be $v \gamma$. Here we can see the apparent velocity is obtained using $t'$ \marginnote{(2.57)}
as the time, which is not $x^0$ in the frame of coordinate attached to the observer. 

\begin{figure}
    \centering
    

% Gradient Info
  
\tikzset {_f6md6ajgy/.code = {\pgfsetadditionalshadetransform{ \pgftransformshift{\pgfpoint{0 bp } { 0 bp }  }  \pgftransformscale{1 }  }}}
\pgfdeclareradialshading{_jk0eqd11o}{\pgfpoint{0bp}{0bp}}{rgb(0bp)=(0.97,0.91,0.11);
rgb(0bp)=(0.97,0.91,0.11);
rgb(25bp)=(1,1,1);
rgb(400bp)=(1,1,1)}
\tikzset{every picture/.style={line width=0.75pt}} %set default line width to 0.75pt        

\begin{tikzpicture}[x=0.75pt,y=0.75pt,yscale=-1,xscale=1]
%uncomment if require: \path (0,315); %set diagram left start at 0, and has height of 315

%Straight Lines [id:da8791392372886593] 
\draw [color={rgb, 255:red, 0; green, 0; blue, 0 }  ,draw opacity=0.75 ]   (184,118.23) -- (312,242) ;
%Straight Lines [id:da6990399911927212] 
\draw [color={rgb, 255:red, 0; green, 0; blue, 0 }  ,draw opacity=0.5 ]   (96,242) -- (96,283) ;
%Straight Lines [id:da929753289055431] 
\draw [color={rgb, 255:red, 0; green, 0; blue, 0 }  ,draw opacity=1 ]   (96,242) -- (214.84,75.63) ;
\draw [shift={(216,74)}, rotate = 125.54] [fill={rgb, 255:red, 0; green, 0; blue, 0 }  ,fill opacity=1 ][line width=0.08]  [draw opacity=0] (12,-3) -- (0,0) -- (12,3) -- cycle    ;
%Straight Lines [id:da9476652899394022] 
\draw    (96,242) -- (396,242) ;
\draw [shift={(398,242)}, rotate = 180] [fill={rgb, 255:red, 0; green, 0; blue, 0 }  ][line width=0.08]  [draw opacity=0] (12,-3) -- (0,0) -- (12,3) -- cycle    ;
%Straight Lines [id:da12075393106418608] 
\draw    (96,242) -- (96,52) ;
\draw [shift={(96,50)}, rotate = 90] [fill={rgb, 255:red, 0; green, 0; blue, 0 }  ][line width=0.08]  [draw opacity=0] (12,-3) -- (0,0) -- (12,3) -- cycle    ;
%Shape: Ellipse [id:dp7501822937817628] 
\draw  [draw opacity=0][shading=_jk0eqd11o,_f6md6ajgy] (65.63,224.61) .. controls (71.12,215.02) and (89.17,215.04) .. (105.94,224.64) .. controls (122.71,234.25) and (131.86,249.81) .. (126.37,259.39) .. controls (120.88,268.98) and (102.83,268.96) .. (86.06,259.36) .. controls (69.29,249.75) and (60.14,234.19) .. (65.63,224.61) -- cycle ;
%Shape: Arc [id:dp14808297636810597] 
\draw  [draw opacity=0] (125.4,201.77) .. controls (137.65,210.74) and (145.66,225.17) .. (145.82,241.5) -- (96,242) -- cycle ; \draw  [color={rgb, 255:red, 0; green, 0; blue, 0 }  ,draw opacity=0.5 ] (125.4,201.77) .. controls (137.65,210.74) and (145.66,225.17) .. (145.82,241.5) ;
%Shape: Ellipse [id:dp5550484242786045] 
\draw  [color={rgb, 255:red, 80; green, 227; blue, 194 }  ,draw opacity=1 ][fill={rgb, 255:red, 80; green, 227; blue, 194 }  ,fill opacity=0.75 ] (191.07,102.77) .. controls (187.81,101.28) and (181.99,106.99) .. (178.09,115.53) .. controls (174.18,124.06) and (173.66,132.2) .. (176.93,133.69) .. controls (180.19,135.18) and (186.01,129.47) .. (189.91,120.93) .. controls (193.82,112.39) and (194.34,104.26) .. (191.07,102.77) -- cycle ;
%Straight Lines [id:da06332288321674429] 
\draw [color={rgb, 255:red, 0; green, 0; blue, 0 }  ,draw opacity=0.5 ]   (312,242) -- (312,283) ;
%Straight Lines [id:da6458514363662675] 
\draw [color={rgb, 255:red, 0; green, 0; blue, 0 }  ,draw opacity=0.5 ]   (98,269) -- (310,269) ;
\draw [shift={(312,269)}, rotate = 180] [fill={rgb, 255:red, 0; green, 0; blue, 0 }  ,fill opacity=0.5 ][line width=0.08]  [draw opacity=0] (12,-3) -- (0,0) -- (12,3) -- cycle    ;
\draw [shift={(96,269)}, rotate = 0] [fill={rgb, 255:red, 0; green, 0; blue, 0 }  ,fill opacity=0.5 ][line width=0.08]  [draw opacity=0] (12,-3) -- (0,0) -- (12,3) -- cycle    ;
%Straight Lines [id:da4496754806264811] 
\draw [color={rgb, 255:red, 0; green, 0; blue, 0 }  ,draw opacity=0.5 ] [dash pattern={on 4.5pt off 4.5pt}]  (184,118.23) -- (184,242) ;
%Straight Lines [id:da29868680463289254] 
\draw [color={rgb, 255:red, 0; green, 0; blue, 0 }  ,draw opacity=0.5 ]   (184,76) -- (184,118.23) ;
\draw [shift={(184,74)}, rotate = 90] [fill={rgb, 255:red, 0; green, 0; blue, 0 }  ,fill opacity=0.5 ][line width=0.08]  [draw opacity=0] (12,-3) -- (0,0) -- (12,3) -- cycle    ;

% Text Node
\draw (400,242) node [anchor=west] [inner sep=0.75pt]    {$x$};
% Text Node
\draw (94,46.6) node [anchor=south east] [inner sep=0.75pt]    {$y$};
% Text Node
\draw (150,210.4) node [anchor=north west][inner sep=0.75pt]    {$\varphi $};
% Text Node
\draw (204,272.4) node [anchor=north] [inner sep=0.75pt]    {$D$};
% Text Node
\draw (250,176.71) node [anchor=south west] [inner sep=0.75pt]    {$L$};
% Text Node
\draw (98,149.4) node [anchor=north west][inner sep=0.75pt]    {$R=vt$};
% Text Node
\draw (218,70.6) node [anchor=south west] [inner sep=0.75pt]    {$v$};
% Text Node
\draw (184,70.6) node [anchor=south] [inner sep=0.75pt]    {$v_{y}$};
% Text Node
\draw (308,217) node [anchor=north west][inner sep=0.75pt]   [align=left] {detector};


\end{tikzpicture}

    \caption{Superluminal motion of a ejected material from a galaxy}
    \label{fig:superluminal}
\end{figure}

\section{Time dilation and cosmic ray muons}

The lifetime of an unstable particle is measured according to its proper time, and this causes the difference \marginnote{Sec.~2.6}
between the prediction of Newtonian and relativistic theories of the flux of the particle after traveling a 
certain distance. This is actually a piece of strong evidence of special relativity.

\section{Relativistic mechanics}

We know the Lagrangian of a Newtonian free particle is 
\begin{equation} \marginnote{(3.1)}
    L = \frac{1}{2} m g_{ij} \dot{x}^i \dot{x}^j.
    \label{eq:newton}
\end{equation}
The natural generalization is 
\begin{equation}
    L = \frac{1}{2} m g_{\mu \nu} \dot{x}^\mu \dot{x}^\nu.
    \label{eq:generalized-lagrangian}
\end{equation}
Note, however, the meaning of $\dot{x}$ is not clear here: $g_{\mu \nu} \dd{x^\mu} \dd{x^\nu}$ is already a 
covariant value, and to make the whole expression covariant, the ``time'' used in the time derivative $\dot{x}$
should also be a relativistic scalar, which can only be the proper time $\tau$. Similarly, when calculating 
the action, we need to integrate \eqref{eq:generalized-lagrangian} over $\tau$. So the final action is 
\begin{equation}
    S = \int \dd{\tau} L, \quad L = \frac{1}{2} m g_{\mu \nu} \dv{x^\mu}{\tau} \dv{x^\nu}{\tau}.
    \label{eq:gr-action-1}
\end{equation}
We can repeat the process in (1.71) \marginnote{(1.71)} and find that the EOM is 
\begin{equation} \marginnote{(3.6)}
    \dv[2]{x^\mu}{\tau} + \Gamma\indices{^\mu_\nu _\rho} \dv{x^\nu}{\tau} \dv{x^\rho}{\tau} = 0.
    \label{eq:eom-sr}
\end{equation}
Actually Newton's second law can also be written into this form (again see the discussion around (1.71)),
but this time we are working with $\mu, \nu, \rho = 0, 1, 1, 3$, not just $1, 2, 3$.
Note that \eqref{eq:eom-sr} is actually the geodesic equation. 

We have a more ``geometric'' version of \eqref{eq:newton}. Since 
\[
    \dd{l} = \sqrt{g_{ij} \dot{x}^i \dot{x}^j} \dd{t}
\]
is an increasing function of the integrand of \eqref{eq:newton}, the action corresponding to \eqref{eq:newton} is equivalent to 
\begin{equation}
    S = \int \dd{l}.
    \label{eq:newton-line}
\end{equation}
Similarly, we may guess the version of \eqref{eq:newton-line} corresponding to \eqref{eq:gr-action-1} 
is 
\begin{equation}
    S = \int \abs*{\dd{s}} = \int \sqrt{- \dd{s^2}}.
\end{equation}
This is indeed the case, since actually by the definition of proper time, we have 
\begin{equation} \marginnote{(3.4)}
    c^2 \dd{\tau^2} = - \dd{s^2} = - g_{\mu \nu} \dd{x^\mu} \dd{x^\nu},
\end{equation}
and therefore $L$ in \eqref{eq:gr-action-1} is just a constant, and we have $S \propto \int \dd{\tau}$.

A question is what is the relativistic version of mechanics of \emph{massless} particles. \marginnote{Sec.~3.3}
Since $m$ in the action is just a constant, the geodesic equation \eqref{eq:eom-sr} still works. 
The Lagrangian therefore can still be written as 
\begin{equation} \marginnote{(3.38)}
    L = - mc \sqrt{- g_{\mu \nu} \dot{x}^\mu \dot{x}^\nu},
\end{equation}
but now $m$ should be a constant interpreted as a \emph{coupling constant} with the dimension of mass.

\section{Particle collision}

Relativistic scattering theory is important in particle physics. Consider a typical reaction:
\begin{equation} \marginnote{(3.41)}
    A + B \longrightarrow C + D.
    \label{eq:reaction-abcd}
\end{equation}
Suppose we are working in a frame of reference where $A$ is at rest. Suppose $B$ moves along the $x$ axis.
Then we have 
\begin{equation} \marginnote{(3.43)}
    p_A^\mu = (m_A c, 0, 0, 0), \quad p_B^\mu = (m_B \gamma c, m_B \gamma v, 0, 0).
    \label{eq:p-a-b-4-mom}
\end{equation}
After the reaction, the energies and the momenta of $C$ and $D$ can be quite complicated, so we just 
try to find some general constraints imposed on them. A reaction is possible if and only if both 
energy conservation and momentum conservation hold. The momentum conservation condition can be 
satisfied by working in a reference frame where the total 3-momentum of the system vanishes, and 
this dictates 
\begin{equation} \marginnote{(3.44)}
    p_C^\mu = (\sqrt{m_C^2 c^2 + \vb*{p}_C^2}, \vb*{p}_C), \quad p_D^\mu = (\sqrt{m_D^2 c^2 + \vb*{p}_C^2}, - \vb*{p}_C).
    \label{eq:p-c-d-4-mom}
\end{equation}
Now we just need to impose the energy conservation constraint. Naively doing so is hard because energy itself 
is not a scalar. However, there \emph{is} a conserved relativistic scalar: we have  
\begin{equation}
    p^\text{i}_\mu p^{\text{i} \mu} = p^\text{f}_\mu p^{\text{f} \mu} \eqqcolon M^2 c^2,
\end{equation}
where $M$ is named the \concept{invariant mass}. Note that we can evaluate the LHS in the reference frame of 
\eqref{eq:p-a-b-4-mom} and the RHS in the reference frame of \eqref{eq:p-c-d-4-mom}, and this gives 
\begin{equation} \marginnote{(3.48)}
    \left(m_{A} c+m_{B} \gamma c\right)^{2}-m_{B}^{2} \gamma^{2} v^{2}=\left(m_{C} c+m_{D} c\right)^{2} - (2 \vb*{p}_C)^2,
\end{equation}
which can be simplified into 
\begin{equation} \marginnote{(3.49)}
    2 m_{A} m_{B} \gamma=m_{C}^{2}+m_{D}^{2}+2 m_{C} m_{D}-m_{A}^{2}-m_{B}^{2} - 4 \vb*{p}_C^2 / c^2.
\end{equation}
The minimum energy of $B$ is 
\begin{equation} \marginnote{(3.50)}
    E_B^\text{th} = p_B^0 c = m_B \gamma c^2 = \frac{\left(m_{C}^{2}+m_{D}^{2}+2 m_{C} m_{D}-m_{A}^{2}-m_{B}^{2}\right) c^{2}}{2 m_{A}}.
\end{equation}
This is called the \concept{threshold energy}, because if $E_B < E_B^\text{th}$, \eqref{eq:reaction-abcd} cannot happen.

The procedure can be repeated for different processes and from this we can find another fact about collision that 
head-on collision is more effective than fixed-target collision.

\section{Relativistic perfect fluid}

The topic of relativistic idea fluid is discussed in Problem 2.1 and 2.2. \marginnote{Problem 2.1, 2.2}
First of all, we always have 
\[
    T^{00} = \epsilon = \rho_m c^2,
\]
where $\epsilon$ is the energy density of the fluid in the rest frame, and $\rho_m$ is the mass density.
In a perfect fluid, when we are in the rest-frame, there is no flow, and since momentum is carried by fluid flow,
the density of momentum is also zero, i.e.
\[
    T^{i0} = 0.
\]
The argument used in non-relativistic perfect fluid can be transplanted here for $T^{ij}$: since an idea fluid 
is isotropic, and it cannot hold shear force, as long as the time scale we are interested is long enough to 
hide how the fluid responds to an external shear force, we can assume all shear force components in $T^{ij}$
are zero. Thus we have 
\[
    T^{ij} = \pmqty{\dmat{p, p, p}}.
\]
Putting everything together, we get 
\begin{equation} \marginnote{(2.60)}
    T^{\mu \nu} = \pmqty{\dmat{\epsilon, p, p, p}}.
    \label{eq:rest-frame-perfect-fluid-t}
\end{equation}

Now we can get $T^{\mu \nu}$ in any coordinate system with a Lorentz transformation.
Applying the Lorentz transformation on $x$ direction:
\begin{equation} \marginnote{(2.28)}
    \Lambda\indices{^\mu_\nu} = \pmqty{\dmat{\gamma & - \gamma \beta \\ - \gamma \beta & \gamma, 1, 1}},
    \label{eq:lorentz-x}
\end{equation}
we get (the process can be found in \href{relativistic-ideal-fluid.nb}{this Mathematica notebook})
\begin{equation}
    T'^{\mu \nu} = \pmqty{\frac{p \beta ^2}{1-\beta ^2}+\frac{\epsilon }{1-\beta ^2} & -\frac{\epsilon  \beta }{1-\beta
    ^2}-\frac{p \beta }{1-\beta ^2} \\
  -\frac{\epsilon  \beta }{1-\beta ^2}-\frac{p \beta }{1-\beta ^2} & \frac{\epsilon  \beta ^2}{1-\beta
    ^2}+\frac{p}{1-\beta ^2} }.
    \label{eq:x-moving-fluid-t}
\end{equation}
Suppose the speed of the frame of reference after \eqref{eq:lorentz-x} in the rest frame of the fluid is $v$,
we find \eqref{eq:x-moving-fluid-t} is the energy-momentum tensor of a fluid moving with the velocity of 
$- v \vu*{e}_x$.  \marginnote{Problem~2.2}

\eqref{eq:x-moving-fluid-t} is not covariant. We need to generalize it into a \marginnote{Liang Sec.~6.5} 
covariant version. Solely with information provided in \eqref{eq:x-moving-fluid-t}, the covariant version 
cannot be decided, because systems other than a perfect fluid can also has a energy-momentum tensor like 
\eqref{eq:rest-frame-perfect-fluid-t}. Another way to see the point is to note that velocity of the fluid is 
different on different points, and a global Lorentz transformation cannot turn the fluid into the state of rest. 
We can do local Lorentz transformation, but this distorts the components of $\eta^{\mu \nu}$, but when deriving 
\eqref{eq:rest-frame-perfect-fluid-t} we have $\eta = \diag(-1, 1, 1, 1)$.  

The generic covariant energy-momentum tensor of a perfect fluid is % TODO: derivation
\begin{equation}
    T^{\mu \nu} = \left( \rho_m + \frac{p}{c^2} \right) U^\mu U^\nu + p \eta^{\mu \nu} = \frac{1}{c^2} \left( \epsilon + p \right) U^\mu U^\nu + p \eta^{\mu \nu}.
    \label{eq:general-fluid}
\end{equation}
Note that $p$ and $\epsilon$ in \eqref{eq:general-fluid} are defined in a special frame of reference, but this does 
not eliminate the covariance of \eqref{eq:general-fluid}, because for a fluid there is indeed a special frame of reference, i.e. the rest frame of itself. When in the rest frame of reference, we have 
\begin{equation}
    U^\mu = \pmqty{\mu \\ 0 \\ 0 \\ 0},
\end{equation}
and \eqref{eq:general-fluid} reads 
\[
    T^{\mu \nu} = \frac{1}{c^2} (\epsilon + p) \pmqty{\dmat{c^2, 0, 0, 0}} + p \pmqty{\dmat{-1, 1, 1, 1}}, 
\]
which is just \eqref{eq:rest-frame-perfect-fluid-t}. After a global Lorentz transformation \eqref{eq:lorentz-x}, 
we have 
\[
    U'^\mu = \Lambda\indices{^\mu_\nu} U^\nu = \pmqty{\gamma c \\ - \gamma \beta c \\ 0 \\ 0}.
\]
Substituting this into \eqref{eq:general-fluid}, we get \eqref{eq:x-moving-fluid-t}. (The process is in \href{relativistic-ideal-fluid.nb}{this Mathematica notebook}). 

\bibliographystyle{plain}
\bibliography{books-used}

\end{document}