\documentclass[hyperref, a4paper]{article}

\usepackage{geometry}
\usepackage{titling}
\usepackage{titlesec}
% No longer needed, since we will use enumitem package
% \usepackage{paralist}
\usepackage{enumitem}
\usepackage{footnote}
\usepackage{marginnote}
\usepackage{enumerate}
\usepackage{amsmath, amssymb, amsthm}
\usepackage{mathtools}
\usepackage{bbm}
\usepackage{cite}
\usepackage{graphicx}
\usepackage{subfigure}
\usepackage{physics}
\usepackage{tensor}
\usepackage{siunitx}
\usepackage[version=4]{mhchem}
\usepackage{tikz}
\usepackage{xcolor}
\usepackage{listings}
\usepackage{autobreak}
\usepackage[ruled, vlined, linesnumbered]{algorithm2e}
\usepackage{nameref,zref-xr}
\zxrsetup{toltxlabel}
\zexternaldocument*[gr-]{../relativity/relativistic}[relativistic.pdf]
\zexternaldocument*[optics-]{../optics/optics}[optics.pdf]
\zexternaldocument*[solid-]{../solid/solid}[solid.pdf]
\zexternaldocument*[kt-]{../topological-phases-reading-notes/kt}[kt.pdf]
\usepackage[colorlinks,unicode]{hyperref} % , linkcolor=black, anchorcolor=black, citecolor=black, urlcolor=black, filecolor=black
\usepackage[most]{tcolorbox}
\usepackage{prettyref}

% Page style
\geometry{left=3.18cm,right=3.18cm,top=2.54cm,bottom=2.54cm}
\titlespacing{\paragraph}{0pt}{1pt}{10pt}[20pt]
\setlength{\droptitle}{-5em}
\preauthor{\vspace{-10pt}\begin{center}}
\postauthor{\par\end{center}}

% More compact lists 
%\setlist[itemize]{
%    itemindent=17pt, 
%    leftmargin=1pt,
%    listparindent=\parindent,
%    parsep=0pt,
%}

% Math operators
\DeclareMathOperator{\timeorder}{\mathcal{T}}
\DeclareMathOperator{\diag}{diag}
\DeclareMathOperator{\legpoly}{P}
\DeclareMathOperator{\primevalue}{P}
\DeclareMathOperator{\sgn}{sgn}
\newcommand*{\ii}{\mathrm{i}}
\newcommand*{\ee}{\mathrm{e}}
\newcommand*{\const}{\mathrm{const}}
\newcommand*{\suchthat}{\quad \text{s.t.} \quad}
\newcommand*{\argmin}{\arg\min}
\newcommand*{\argmax}{\arg\max}
\newcommand*{\normalorder}[1]{: #1 :}
\newcommand*{\pair}[1]{\langle #1 \rangle}
\newcommand*{\fd}[1]{\mathcal{D} #1}
\DeclareMathOperator{\bigO}{\mathcal{O}}

% TikZ setting
\usetikzlibrary{arrows,shapes,positioning}
\usetikzlibrary{arrows.meta}
\usetikzlibrary{decorations.markings}
\tikzstyle arrowstyle=[scale=1]
\tikzstyle directed=[postaction={decorate,decoration={markings,
    mark=at position .5 with {\arrow[arrowstyle]{stealth}}}}]
\tikzstyle ray=[directed, thick]
\tikzstyle dot=[anchor=base,fill,circle,inner sep=1pt]

% Algorithm setting
% Julia-style code
\SetKwIF{If}{ElseIf}{Else}{if}{}{elseif}{else}{end}
\SetKwFor{For}{for}{}{end}
\SetKwFor{While}{while}{}{end}
\SetKwProg{Function}{function}{}{end}
\SetArgSty{textnormal}

\newcommand*{\concept}[1]{{\textbf{#1}}}

% Support for tensor double arrows.
\renewcommand{\tensor}[1]{ \stackrel{\leftrightarrow}{\vb*{#1}}}

% Embedded codes
\lstset{basicstyle=\ttfamily,
  showstringspaces=false,
  commentstyle=\color{gray},
  keywordstyle=\color{blue}
}

% Reference formatting
\newrefformat{fig}{Figure~\ref{#1} on page~\pageref{#1}}

% Color boxes
\tcbuselibrary{skins, breakable, theorems}
\newtcbtheorem[number within=section]{warning}{Warning}%
  {colback=orange!5,colframe=orange!65,fonttitle=\bfseries, breakable}{warn}
\newtcbtheorem[number within=section]{note}{Note}%
  {colback=green!5,colframe=green!65,fonttitle=\bfseries, breakable}{note}

% Correctly displaying equation number in section titles
\pdfstringdefDisableCommands{\def\eqref#1{(\ref{#1})}}

\newcommand{\grnote}{\href{../relativity/relativistic.pdf}{this note}}

\title{Prof. Bambi on General Relativity}
\author{Jinyuan Wu}

\begin{document}

\maketitle

This is a note about Prof. Cosimo Bambi's lecture on general relativity on February 25 and March 3, 2022.

This lecture is about the 1-2 chapters in \cite{bambi2018introduction}. Nothing quite interesting.
\cite{bambi2018introduction} itself is very detailed and it seems I don't really need to take much notes.

\section{The Christoffel symbols of spherical coordinates}

The Christoffel symbol of spherical coordinates, given in (1.83) in \cite{bambi2018introduction}, can be 
calculated automatically in \href{spherical-coordinate-differential-geometry.nb}{this Mathematica notebook}.

\section{Derivation of special relativity}

Section~2.1 and 2.2 seem to be based on Chapter 1 and 2 of Landau's book about field theory. 
The arguments have been summarized in Section~\ref{gr-sec:principle-special-relativity} in \grnote.

Section~2.3 derives the Lorentz transformations by Wick rotation of Euclidean rotation in $d = 4$
-- (2.18) is actually just $\tau = \ii t$ in condensed matter physics.
A rotation on $xy$ has the form of (2.20) and we have $C_4^2 = 6$ rotations. Adding 4 translations,
we get the total 10 generators of the rotation group in $\mathbb{R}^4$, and hence the Lorentz transformations
in the (3+1)-dimensional Minkowski spacetime.

Equations from (2.22) to (2.27) are trying to relate the parameter of $\mathbb{R}^4$ rotations to the relative 
velocity of the two reference frames.

\section{Aberration of light and superluminal motion}

We should note that \emph{Lorentz contraction} of length (see for example (2.43)) is often \emph{not} 
\marginnote{(2.48) to (2.57)} the length change we observe when an object is moving fast.
This fact can be 

\section{Time dilation and cosmic ray muons}

The lifetime of an unstable particle is measured according to its proper time, and this causes the difference
between the prediction of Newtonian and relativistic theories of the flux of the particle after traveling a 
certain distance. This is actually a piece of strong evidence of special relativity.

\bibliographystyle{plain}
\bibliography{books-used}

\end{document}