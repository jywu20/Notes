\documentclass[hyperref, a4paper]{article}

\usepackage{geometry}
\usepackage{titling}
\usepackage{titlesec}
% No longer needed, since we will use enumitem package
% \usepackage{paralist}
% \usepackage{enumitem}
\usepackage{footnote}
\usepackage{marginnote}
\usepackage{enumerate}
\usepackage{amsmath, amssymb, amsthm}
\usepackage{mathtools}
\usepackage{bbm}
\usepackage{cite}
\usepackage{graphicx}
\usepackage{subfigure}
\usepackage{physics}
\usepackage{tensor}
\usepackage{siunitx}
\usepackage[version=4]{mhchem}
\usepackage{tikz}
\usepackage{xcolor}
\usepackage{listings}
\usepackage{autobreak}
\usepackage[ruled, vlined, linesnumbered]{algorithm2e}
\usepackage{nameref,zref-xr}
\zxrsetup{toltxlabel}
\zexternaldocument*[gr-]{../relativity/relativistic}[relativistic.pdf]
\zexternaldocument*[optics-]{../optics/optics}[optics.pdf]
\zexternaldocument*[solid-]{../solid/solid}[solid.pdf]
\zexternaldocument*[kt-]{../topological-phases-reading-notes/kt}[kt.pdf]
\usepackage[colorlinks,unicode]{hyperref} % , linkcolor=black, anchorcolor=black, citecolor=black, urlcolor=black, filecolor=black
\usepackage[most]{tcolorbox}
\usepackage{prettyref}

% Page style
\geometry{left=3.18cm,right=3.18cm,top=2.54cm,bottom=2.54cm}
\titlespacing{\paragraph}{0pt}{1pt}{10pt}[20pt]
\setlength{\droptitle}{-5em}

% More compact lists 
%\setlist[itemize]{
%    itemindent=17pt, 
%    leftmargin=1pt,
%    listparindent=\parindent,
%    parsep=0pt,
%}

% Math operators
\DeclareMathOperator{\timeorder}{\mathcal{T}}
\DeclareMathOperator{\diag}{diag}
\DeclareMathOperator{\legpoly}{P}
\DeclareMathOperator{\primevalue}{P}
\DeclareMathOperator{\sgn}{sgn}
\newcommand*{\ii}{\mathrm{i}}
\newcommand*{\ee}{\mathrm{e}}
\newcommand*{\const}{\mathrm{const}}
\newcommand*{\suchthat}{\quad \text{s.t.} \quad}
\newcommand*{\argmin}{\arg\min}
\newcommand*{\argmax}{\arg\max}
\newcommand*{\normalorder}[1]{: #1 :}
\newcommand*{\pair}[1]{\langle #1 \rangle}
\newcommand*{\fd}[1]{\mathcal{D} #1}
\DeclareMathOperator{\bigO}{\mathcal{O}}

% TikZ setting
\usetikzlibrary{arrows,shapes,positioning}
\usetikzlibrary{arrows.meta}
\usetikzlibrary{decorations.markings}
\tikzstyle arrowstyle=[scale=1]
\tikzstyle directed=[postaction={decorate,decoration={markings,
    mark=at position .5 with {\arrow[arrowstyle]{stealth}}}}]
\tikzstyle ray=[directed, thick]
\tikzstyle dot=[anchor=base,fill,circle,inner sep=1pt]
\usetikzlibrary{fadings}
\usetikzlibrary{patterns}
\usetikzlibrary{shadows.blur}
\usetikzlibrary{shapes}

% Algorithm setting
% Julia-style code
\SetKwIF{If}{ElseIf}{Else}{if}{}{elseif}{else}{end}
\SetKwFor{For}{for}{}{end}
\SetKwFor{While}{while}{}{end}
\SetKwProg{Function}{function}{}{end}
\SetArgSty{textnormal}

\newcommand*{\concept}[1]{{\textbf{#1}}}

% Support for tensor double arrows.
\renewcommand{\tensor}[1]{ \stackrel{\leftrightarrow}{\vb*{#1}}}

% Embedded codes
\lstset{basicstyle=\ttfamily,
  showstringspaces=false,
  commentstyle=\color{gray},
  keywordstyle=\color{blue}
}

% Reference formatting
\newrefformat{fig}{Figure~\ref{#1} on page~\pageref{#1}}

% Color boxes
\tcbuselibrary{skins, breakable, theorems}
\newtcbtheorem[number within=section]{warning}{Warning}%
  {colback=orange!5,colframe=orange!65,fonttitle=\bfseries, breakable}{warn}
\newtcbtheorem[number within=section]{note}{Note}%
  {colback=green!5,colframe=green!65,fonttitle=\bfseries, breakable}{note}

% Correctly displaying equation number in section titles
\pdfstringdefDisableCommands{\def\eqref#1{(\ref{#1})}}

\newcommand{\grnote}{\href{../relativity/relativistic.pdf}{this note}}

\title{Midterm project}
\author{Jinyuan Wu}

\begin{document}

\maketitle

\section{Part 1: ideal fluid in flat spacetime}

\subsection{Derivation of the ideal fluid energy-momentum tensor}

\paragraph{Problem} Show that the energy-momentum tensor of an ideal fluid is 
\begin{equation}
    T^{\mu \nu} = \frac{1}{c^2} (\epsilon + p) U^\mu U^\nu + p \eta^{\mu \nu},
    \label{eq:ideal-fluid-t}
\end{equation}
where 
\begin{equation}
    \epsilon = \rho_m c^2,
\end{equation}
$\rho_m$ is the density of the rest mass in the rest frame of reference, $U^\mu$ is the 4-velocity and hence 
\begin{equation}
    U_\mu U^\mu = - c^2,
\end{equation}
and $p$ is the pressure.
Hint: an ideal fluid is a isotropic system in which there is no dissipation, i.e. there is no hidden degrees 
of freedom other then the velocity of flow and the pressure, and which doesn't have any shear forces inside.

\paragraph{Solution} Consider a small control volume $V$ around $x^i$ in the ideal fluid at the time $x^0$, whose 
velocity is $\vb*{v}$ in the lab reference frame. Consider a frame of reference $S'$ that moves with 
velocity $\vb*{v}$. Suppose $x'$ is the coordinates in $S'$ corresponding to $x$ in the lab reference frame. 
In $S'$, the flow velocity of $V$ is zero, and therefore since it is only dependent on ``local'' information, 
$T'_{\mu \nu}(x')$ is the same as the energy-momentum tensor of an ideal fluid at rest. (Note that this step 
of of reasoning implicitly invokes the assumption that there is no hidden degree of freedom in the fluid -- 
if there is, then we have no guarantee that these degrees of freedom in $V$ have the same values in an ideal 
fluid.)

So now we need to write down the energy-momentum tensor for an ideal fluid at rest. By the physical meaning of 
the terms we immediately find $T^{00}$ is the ``rest energy density'' of the system, which is $\rho_m c^2$.
The $T^{ij}$ components are the stress tensor. The no-shear-force condition means $T^{ij}$ is diagonal, and the 
isotropic condition means all three diagonal elements of $T^{ij}$ are the same, which means $T^{ij}$ assumes 
the following form:
\begin{equation}
    T^{ij} = \pmqty{\dmat{p, p, p}}.
\end{equation}
Obviously, $p$ is just what is known as the \emph{pressure}. The components $T^{i0}$ are the density of momentum,
which are also zero. So $T^{\mu \nu}$ of an ideal fluid at rest is 
\begin{equation}
    T^{\mu \nu} = \pmqty{\dmat{\rho_m c^2, p, p, p}},
\end{equation}
and therefore for the fluid volume in the $S'$ reference frame, we have 
\begin{equation}
    T^{' \mu \nu} = \pmqty{\dmat{\rho_m c^2, p, p, p}} = \pmqty{\dmat{\epsilon, p, p, p}} .
\end{equation}

Now we can use a Lorentz transformation back into the lab reference frame. The lab reference frame moves with 
velocity $- \vb*{v}$ in $S'$. Without loss of generality, we assume $\vb*{v} = v \vu*{e}_x$, and and transformation
matrix (which is the inverse Lorentz transformation between the lab reference frame and $S'$) is 
\begin{equation}
    \Lambda\indices{^\mu_\nu} = \pmqty{\dmat{\gamma & \gamma \beta \\ \gamma \beta & \gamma, 1, 1}},
\end{equation}
where 
\begin{equation}
    \beta = \frac{v}{c}, \quad \gamma = \frac{1}{\sqrt{1 - v^2 / c^2}}.
\end{equation}
So the energy-momentum tensor in the lab reference frame at $x$ is 
\begin{equation}
    \begin{aligned}
        T^{\mu \nu}(x) &= \Lambda\indices{^\mu_\rho} \Lambda\indices{^\nu_\sigma} T^{' \rho \sigma}(x') \\
        &= \pmqty{\dmat{\gamma & \gamma \beta \\ \gamma \beta & \gamma, 1, 1}} \pmqty{\dmat{\epsilon, p, p, p}} \pmqty{\dmat{\gamma & \gamma \beta \\ \gamma \beta & \gamma, 1, 1}} \\
        &= \pmqty{\dmat{ \gamma^2 \epsilon + \gamma^2 \beta^2 p & \gamma^2 \beta \epsilon + \gamma^2 \beta p \\ \gamma^2 \beta \epsilon + \gamma^2 \beta p & \gamma^2 \beta^2 \epsilon + \gamma^2 p , p, p}}.
    \end{aligned}
    \label{eq:t-in-beta-gamma}
\end{equation} 

Now we need to rephrase this equation in terms of $U^\mu$. We have 
\begin{equation}
    U^\mu = \gamma (c, \vb*{v}), 
\end{equation}
and therefore we have 
\begin{equation}
    \begin{aligned}
        \frac{1}{c^2} (\epsilon + p) U^\mu U^\nu + p \eta^{\mu \nu} &= \frac{1}{c^2} (\epsilon + p) \pmqty{c \\ \vb*{v}} \pmqty{c & \vb*{v}} + p \pmqty{\dmat{-1, 1, 1, 1}} \\
        &= \frac{\epsilon + p}{c^2} \pmqty{ \dmat{\gamma^2 c^2 & \gamma^2 \vb*{v} \\ \gamma^2 \vb*{v} & \gamma^2 \vb*{v}^2 , , } } + + p \pmqty{\dmat{-1, 1, 1, 1}}  \\
        &= \pmqty{ \dmat{ \epsilon \gamma^2 + p (\gamma^2 - 1) & \gamma^2 \beta (\epsilon + p) \\ \gamma^2 \beta (\epsilon + p) & (\epsilon + p) \gamma^2 \beta^2 + p , p, p } }.
    \end{aligned}
    \label{eq:def-ideal-fluid-rhs}
\end{equation}
It's straightforward to verify that 
\[
    \gamma^2 \epsilon + p(\gamma^2 - 1) = \gamma^2 \epsilon + p \frac{v^2 / c^2}{1 - v^2 / c^2} = \gamma^2 \epsilon + p \gamma^2 \beta^2,
\]
\[
    (\epsilon + p) \gamma^2 \beta^2 + p = \epsilon \gamma^2 \beta^2 + p \left(1 + \frac{v^2/c^2}{1 - v^2 / c^2} \right) = \epsilon \gamma^2 \beta^2 + p \frac{1}{1 - v^2 / c^2} = \epsilon \gamma^2 \beta^2 + p \gamma^2.
\]
Therefore, the RHSs of \eqref{eq:t-in-beta-gamma} and \eqref{eq:def-ideal-fluid-rhs} are the same, and therefore 
we have shown \eqref{eq:ideal-fluid-t} for a single small fluid volume. Since our argument works at every point 
of the fluid, we have shown \eqref{eq:ideal-fluid-t} for the whole fluid system. Note that it's possible to have 
different $\epsilon$'s and different $p$'s at different points.

Until now we haven't decided the status of $p$, but since $T^{\mu \nu}$ should be a tensor, and $\epsilon$ is a 
scalar by definition, and \eqref{eq:ideal-fluid-t} is a covariant formula, the only possibility is that $p$ is a 
scalar. 

\subsection{The equations of movement}

\paragraph{Problem} Find equations of motion from \eqref{eq:ideal-fluid-t}. Is the resulting equation system 
closed? If not, why? 

\paragraph{Solution} From the conservation of energy and momentum, we have 
\begin{equation}
    0 = \partial_\mu T^{\mu \nu} = \frac{1}{c^2} U^\mu U^\nu \partial_\mu(\epsilon + p) + \frac{\epsilon + p}{c^2} (U^\nu \partial_\mu U^\mu + U^\mu \partial_\mu U^\nu) + \partial^\nu p.
    \label{eq:conservation}
\end{equation}
Multiplying $U_\nu$ to \eqref{eq:conservation}, we have 
\begin{equation}
    \begin{aligned}
        0 &= \frac{1}{c^2} \underbrace{U_\nu U^\nu}_{- c^2} U^\mu \partial_\mu (\epsilon + p) 
        + \frac{\epsilon + p}{c^2} ( \underbrace{U_\nu U^\nu}_{-c^2} \partial_\mu U^\mu + \underbrace{U_\nu U^\mu \partial_\mu U^\nu}_{=0} ) + U^\mu \partial_\mu p \\
        &= - U^\mu \partial_\mu \epsilon - (\epsilon + p) \partial_\mu U^\mu ,
    \end{aligned}
\end{equation}
where 
\begin{equation}
    U_\nu U^\mu \partial_\mu U^\nu = U^\mu \partial_\mu ( \underbrace{U^\nu U_\nu}_{-c^2} ) - U^\mu U^\nu \partial_\mu U_\nu = - U_\nu U^\mu \partial_\mu U^\nu 
    \label{eq:aux-1}
\end{equation}
and therefore is zero. 

Multiplying $U^\sigma U_\nu + c^2 \delta^\sigma_\nu$ to \eqref{eq:conservation},
we have 
\begin{equation}
    \begin{aligned}
        0 &= \frac{1}{c^2} U^\sigma U^\mu \underbrace{U_\nu U^\nu}_{- c^2} \partial_\mu(\epsilon + p) + \frac{\epsilon + p}{c^2} U^\sigma ( \underbrace{U_\nu U^\nu}_{-c^2} \partial_\mu U^\mu + \underbrace{U_\nu U^\mu \partial_\mu U^\nu}_{\eqref{eq:aux-1} = 0}) + U^\sigma U_\nu \partial^\nu p  \\
        &\quad + U^\sigma U^\mu \partial_\mu (\epsilon + p) + (\epsilon + p) (U^\sigma \partial_\mu U^\mu + U^\mu \partial_\mu U^\sigma) + c^2 \partial^\sigma p \\
        &= U^\sigma U_\nu \partial^\nu p + (\epsilon + p) U^\mu \partial_\mu U^\sigma + c^2 \partial^\sigma p.
    \end{aligned}
\end{equation}

So we get the EOMs:
\begin{equation}
    \begin{aligned}
        &U^\mu \partial_\mu \epsilon + (\epsilon + p) \partial_\mu U^\mu = 0, \\
        &U^\nu U_\mu \partial^\mu p + (\epsilon + p) U^\mu \partial_\mu U^\nu + c^2 \partial^\nu p = 0.
    \end{aligned}
    \label{eq:relativistic-eom}
\end{equation}

The EOMs are not closed if $\epsilon$ and $p$ are allowed to vary. In this case, we still need an additional
equation. This additional equation has to be derived from the property of the fluid in question to decide 
how the distribution of $\epsilon$ and $p$ evolve. For example, it may be an equation in the form of 
$p = p (\epsilon) = p(\rho_m)$, i.e. the inhomogeneous distribution of the density creates an inhomogeneous
pressure field.

\subsection{Low speed approximation}

\paragraph{Problem} Show that \eqref{eq:relativistic-eom} reduced to Euler's equations when the speed is low.

\paragraph{Solution} We start from the first equation. When the speed is slow, we have 
\begin{equation}
    U^\mu \approx (c, \vb*{v}), \quad c \gg \abs{\vb*{v}}, \quad \partial_\mu U^\mu \approx \div{\vb*{v}},
\end{equation}
and therefore 
\[
    \begin{aligned}
        0 &= (\partial_t + \vb*{v} \cdot \grad ) \rho_m c^2 + (\underbrace{\rho_m c^2 + p}_{\approx \rho_m c^2}) \div{\vb*{v}} \\
        &= c^2 (\partial_t \rho_m + \vb*{v} \cdot \grad \rho_m + \rho_m \div{\vb*{v}}) \\
        &= c^2 (\partial_t \rho_m + \div{(\rho_m \vb*{v})}), 
    \end{aligned}
\]
\begin{equation}
    \partial_t \rho_m + \div{(\rho_m \vb*{v})} = 0.
\end{equation}
This is the continuity equation. 

The second equation is 
\[
    \begin{aligned}
        0 = U^\nu (\partial_t + \vb*{v} \cdot \grad) p + (\rho_m c^2 + p) (\partial_t + \vb*{v} \cdot \grad) U^\nu + c^2 \partial^\nu p = 0.
    \end{aligned}
\]
We only consider the case of $\nu = i$, which gives 
\[
    \begin{aligned}
        0 &= \vb*{v} (\partial_t + \vb*{v} \cdot \grad) p + (\rho_m c^2 + p) (\partial_t + \vb*{v} \cdot \grad) \vb*{v} + c^2 \grad{p} \\ 
        &\approx c^2 ( \rho_m (\partial_t + \vb*{v} \cdot \grad) \vb*{v} + \grad{p} ), 
    \end{aligned}
\]
so 
\begin{equation}
    \rho_m (\partial_t + \vb*{v} \cdot \grad) \vb*{v} = - \grad{p}.
\end{equation}
This is the momentum equation, which can also be written more explicitly as a conservation law:
\begin{equation}
    \pdv{t} (\rho \vb*{v}) + \div{(\rho \vb*{v} \otimes \vb*{v})} = - \grad{p},
\end{equation} 
or as the Newton's second law on an infinitesimal control volume
\begin{equation}
    \dv{t} (\rho \vb*{v})  = -\grad{p} - \rho \vb*{v} \div{\vb*{v}} , \quad \dv{t} = \pdv{t} + \vb*{v} \cdot \grad.
\end{equation}

So we find the relativistic theory of ideal fluids proposed here reduces correctly to the non-relativistic theory.

\subsection{The ``dust''}

\paragraph{Problem} Ideal fluid without inner pressure is called dust. It can also be seen as a bunch of 
particles with very weak interactions. Discuss why the two definitions are compatible.

\paragraph{Solution} We substitute $p$ with zero in \eqref{eq:relativistic-eom}, and the second equation 
immediately becomes 
\begin{equation}
    U^\mu \partial_\mu U^\nu = 0.
\end{equation}
This is the geodesic equation, which is the EOM of free particles, so indeed the infinitesimal volumes in 
the fluid doesn't interact with each other. The (maybe coarse-grained) theory of a bundle of non-interacting 
particles is always an ideal fluid without inner pressure. 

\section{Part 2: idea fluid in curved spacetime}



\end{document}