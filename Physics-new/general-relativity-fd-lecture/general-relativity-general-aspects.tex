\documentclass[hyperref, a4paper]{article}

\usepackage{geometry}
\usepackage{titling}
\usepackage{titlesec}
% No longer needed, since we will use enumitem package
% \usepackage{paralist}
\usepackage{enumitem}
\usepackage{footnote}
\usepackage{marginnote}
\usepackage{enumerate}
\usepackage{amsmath, amssymb, amsthm}
\usepackage{mathtools}
\usepackage{bbm}
\usepackage{cite}
\usepackage{graphicx}
\usepackage{subfigure}
\usepackage{physics}
\usepackage{tensor}
\usepackage{siunitx}
\usepackage[version=4]{mhchem}
\usepackage{tikz}
\usepackage{xcolor}
\usepackage{listings}
\usepackage{autobreak}
\usepackage[ruled, vlined, linesnumbered]{algorithm2e}
\usepackage{nameref,zref-xr}
\zxrsetup{toltxlabel}
\zexternaldocument*[gr-]{../relativity/relativistic}[relativistic.pdf]
\zexternaldocument*[optics-]{../optics/optics}[optics.pdf]
\zexternaldocument*[solid-]{../solid/solid}[solid.pdf]
\zexternaldocument*[kt-]{../topological-phases-reading-notes/kt}[kt.pdf]
\usepackage[colorlinks,unicode]{hyperref} % , linkcolor=black, anchorcolor=black, citecolor=black, urlcolor=black, filecolor=black
\usepackage[most]{tcolorbox}
\usepackage{prettyref}

% Page style
\geometry{left=3.18cm,right=3.18cm,top=2.54cm,bottom=2.54cm}
\titlespacing{\paragraph}{0pt}{1pt}{10pt}[20pt]
\setlength{\droptitle}{-5em}
\preauthor{\vspace{-10pt}\begin{center}}
\postauthor{\par\end{center}}

% More compact lists 
%\setlist[itemize]{
%    itemindent=17pt, 
%    leftmargin=1pt,
%    listparindent=\parindent,
%    parsep=0pt,
%}

% Math operators
\DeclareMathOperator{\timeorder}{\mathcal{T}}
\DeclareMathOperator{\diag}{diag}
\DeclareMathOperator{\legpoly}{P}
\DeclareMathOperator{\primevalue}{P}
\DeclareMathOperator{\sgn}{sgn}
\newcommand*{\ii}{\mathrm{i}}
\newcommand*{\ee}{\mathrm{e}}
\newcommand*{\const}{\mathrm{const}}
\newcommand*{\suchthat}{\quad \text{s.t.} \quad}
\newcommand*{\argmin}{\arg\min}
\newcommand*{\argmax}{\arg\max}
\newcommand*{\normalorder}[1]{: #1 :}
\newcommand*{\pair}[1]{\langle #1 \rangle}
\newcommand*{\fd}[1]{\mathcal{D} #1}
\DeclareMathOperator{\bigO}{\mathcal{O}}

% TikZ setting
\usetikzlibrary{arrows,shapes,positioning}
\usetikzlibrary{arrows.meta}
\usetikzlibrary{decorations.markings}
\tikzstyle arrowstyle=[scale=1]
\tikzstyle directed=[postaction={decorate,decoration={markings,
    mark=at position .5 with {\arrow[arrowstyle]{stealth}}}}]
\tikzstyle ray=[directed, thick]
\tikzstyle dot=[anchor=base,fill,circle,inner sep=1pt]
\usetikzlibrary{fadings}
\usetikzlibrary{patterns}
\usetikzlibrary{shadows.blur}
\usetikzlibrary{shapes}

% Algorithm setting
% Julia-style code
\SetKwIF{If}{ElseIf}{Else}{if}{}{elseif}{else}{end}
\SetKwFor{For}{for}{}{end}
\SetKwFor{While}{while}{}{end}
\SetKwProg{Function}{function}{}{end}
\SetArgSty{textnormal}

\newcommand*{\concept}[1]{{\textbf{#1}}}

% Support for tensor double arrows.
\renewcommand{\tensor}[1]{ \stackrel{\leftrightarrow}{\vb*{#1}}}

% Embedded codes
\lstset{basicstyle=\ttfamily,
  showstringspaces=false,
  commentstyle=\color{gray},
  keywordstyle=\color{blue}
}

% Reference formatting
\newrefformat{fig}{Figure~\ref{#1} on page~\pageref{#1}}

% Color boxes
\tcbuselibrary{skins, breakable, theorems}
\newtcbtheorem[number within=section]{warning}{Warning}%
  {colback=orange!5,colframe=orange!65,fonttitle=\bfseries, breakable}{warn}
\newtcbtheorem[number within=section]{note}{Note}%
  {colback=green!5,colframe=green!65,fonttitle=\bfseries, breakable}{note}

% Correctly displaying equation number in section titles
\pdfstringdefDisableCommands{\def\eqref#1{(\ref{#1})}}

\newcommand{\grnote}{\href{../relativity/relativistic.pdf}{this note}}

\title{Prof. Cosimo Bambi on General Relativity}
\author{Jinyuan Wu}

\begin{document}

\maketitle

This is a note about Prof. Cosimo Bambi's lecture on general relativity from March 25, 2022.

\section{Metric tensor as gravitational field}

The idea of a metric field as a physical degree of freedom can also be motivated by Newtonian gravity.
Consider the well-known fact that \emph{gravitational mass} is the same as \emph{inertial mass}. 
This means the Lagrangian of a particle in a gravitational potential $\Phi$ is 
\begin{equation} \marginnote{(5.3)}
    L = \frac{1}{2} m \vb*{v}^2 - m \Phi,
\end{equation}
and this Lagrangian is a low energy approximation of 
\begin{equation} \marginnote{(5.5)}
    L = - m c \sqrt{c^2 - \vb*{v}^2 + 2 \Phi},
    \label{eq:metric-with-gravitational-potential}
\end{equation}
so the action is 
\begin{equation} \marginnote{(5.6), (5.7)}
    S = - mc \int \dd{t} \sqrt{- g_{\mu \nu} \dot{x}^\mu \dot{x}^\nu}, \quad g_{\mu \nu} = \pmqty{\dmat{- \left(1 + \frac{2 \Phi}{c}\right), 1, 1, 1}}.
\end{equation}
So we see that the gravitational potential is absorbed into the metric tensor. 

Of course, \eqref{eq:metric-with-gravitational-potential} is just an educated guess, but we will find it leads 
to the beautiful and well-tested theory of general relativity.

\section{Covariant derivative}

Sec. 5.2.1 checks the tensor nature of $\nabla_\mu u^\nu$. \marginnote{(5.25)}
Sec. 5.2.2 shows what it really does: parallel transport of vectors.

Here we briefly summarize the argument used in Sec. 5.2.2. From (5.28) to (5.35), the parallel transport 
of $V^\theta$ and $V^r$ is derived by definition. From (5.36) to (5.40), we calculated the Christoffel 
symbols using a short cut (compare two forms of the geodesic equations). From (5.41) and (5.43), 
it is shown that parallel transport can also be calculated using Christoffel symbols, and in (5.44) 
we show the direct connection between covariant derivative and vector parallel transport.

The information of the metric tensor enters the calculation from (5.28) to (5.35) when we calculate the 
parallel transport in the Cartesian coordinates, and it also enters the calculation from (5.36) to (5.40)
when we write down (5.36).

\section{Riemann tensor}

Note that different people use different definition of the Riemann tensor. 

\end{document}