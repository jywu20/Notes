\documentclass[hyperref, a4paper]{article}

\usepackage{geometry}
\usepackage{titling}
\usepackage{titlesec}
% No longer needed, since we will use enumitem package
% \usepackage{paralist}
\usepackage{enumitem}
\usepackage{footnote}
\usepackage{marginnote}
\usepackage{enumerate}
\usepackage{amsmath, amssymb, amsthm}
\usepackage{mathtools}
\usepackage{bbm}
\usepackage{cite}
\usepackage{graphicx}
\usepackage{subfigure}
\usepackage{physics}
\usepackage{tensor}
\usepackage{siunitx}
\usepackage[version=4]{mhchem}
\usepackage{tikz}
\usepackage{xcolor}
\usepackage{listings}
\usepackage{autobreak}
\usepackage[ruled, vlined, linesnumbered]{algorithm2e}
\usepackage{nameref,zref-xr}
\zxrsetup{toltxlabel}
\zexternaldocument*[optics-]{../optics/optics}[optics.pdf]
\zexternaldocument*[solid-]{../solid/solid}[solid.pdf]
\zexternaldocument*[oldfluid-]{fluid}[fluid.pdf]
\zexternaldocument*[kt-]{../topological-phases-reading-notes/kt}[kt.pdf]
\usepackage[colorlinks,unicode]{hyperref} % , linkcolor=black, anchorcolor=black, citecolor=black, urlcolor=black, filecolor=black
\usepackage[most]{tcolorbox}
\usepackage{prettyref}

% Page style
\geometry{left=3.18cm,right=3.18cm,top=2.54cm,bottom=2.54cm}
\titlespacing{\paragraph}{0pt}{1pt}{10pt}[20pt]
\setlength{\droptitle}{-5em}
\preauthor{\vspace{-10pt}\begin{center}}
\postauthor{\par\end{center}}

% More compact lists 
%\setlist[itemize]{
%    itemindent=17pt, 
%    leftmargin=1pt,
%    listparindent=\parindent,
%    parsep=0pt,
%}

% Math operators
\DeclareMathOperator{\timeorder}{\mathcal{T}}
\DeclareMathOperator{\diag}{diag}
\DeclareMathOperator{\legpoly}{P}
\DeclareMathOperator{\primevalue}{P}
\DeclareMathOperator{\sgn}{sgn}
\newcommand*{\ii}{\mathrm{i}}
\newcommand*{\ee}{\mathrm{e}}
\newcommand*{\const}{\mathrm{const}}
\newcommand*{\suchthat}{\quad \text{s.t.} \quad}
\newcommand*{\argmin}{\arg\min}
\newcommand*{\argmax}{\arg\max}
\newcommand*{\normalorder}[1]{: #1 :}
\newcommand*{\pair}[1]{\langle #1 \rangle}
\newcommand*{\fd}[1]{\mathcal{D} #1}
\DeclareMathOperator{\bigO}{\mathcal{O}}

% TikZ setting
\usetikzlibrary{arrows,shapes,positioning}
\usetikzlibrary{arrows.meta}
\usetikzlibrary{decorations.markings}
\tikzstyle arrowstyle=[scale=1]
\tikzstyle directed=[postaction={decorate,decoration={markings,
    mark=at position .5 with {\arrow[arrowstyle]{stealth}}}}]
\tikzstyle ray=[directed, thick]
\tikzstyle dot=[anchor=base,fill,circle,inner sep=1pt]

% Algorithm setting
% Julia-style code
\SetKwIF{If}{ElseIf}{Else}{if}{}{elseif}{else}{end}
\SetKwFor{For}{for}{}{end}
\SetKwFor{While}{while}{}{end}
\SetKwProg{Function}{function}{}{end}
\SetArgSty{textnormal}

\newcommand*{\concept}[1]{{\textbf{#1}}}

% Support for tensor double arrows.
\renewcommand{\tensor}[1]{ \stackrel{\leftrightarrow}{\vb*{#1}}}

% Embedded codes
\lstset{basicstyle=\ttfamily,
  showstringspaces=false,
  commentstyle=\color{gray},
  keywordstyle=\color{blue}
}

% Reference formatting
\newrefformat{fig}{Figure~\ref{#1} on page~\pageref{#1}}

% Color boxes
\tcbuselibrary{skins, breakable, theorems}
\newtcbtheorem[number within=section]{warning}{Warning}%
  {colback=orange!5,colframe=orange!65,fonttitle=\bfseries, breakable}{warn}
\newtcbtheorem[number within=section]{note}{Note}%
  {colback=green!5,colframe=green!65,fonttitle=\bfseries, breakable}{note}

% Correctly displaying equation number in section titles
\pdfstringdefDisableCommands{\def\eqref#1{(\ref{#1})}}
\makeatletter
\pdfstringdefDisableCommands{\let\HyPsd@CatcodeWarning\@gobble}
\makeatother

\title{$U(1)$ Gauge Theories in Condensed Matter Physics}
\author{Jinyuan Wu}

\begin{document}

\maketitle

This article is a reading note of Wen's famous textbook \cite{Wen2007}. 
It is mainly a reconstruction of material related to $U(1)$ gauge theories covered in Chapter~6. 

\section{$U(1)$ gauge theory in $2+1$ dimensions}

The most general form of a $U(1)$ theory is 
\begin{equation}
    \mathcal{L}_{U(1)} = - \frac{1}{4 g^2} f_{\mu \nu} f^{\mu \nu}, \quad 
    f_{\mu \nu} = \partial_\mu a_\nu - \partial_\nu a_\mu.
\end{equation}
In the case of a 2+1 dimensional spacetime, we have 
\[
    \begin{aligned}
        f_{\mu \nu} f^{\mu \nu} &= f^{i 0} f_{i 0} + f^{0 i} f_{0 i} + f^{ij} f_{ij} \\
        &= e^i e_i + e^i e_i + f^{12} f_{12} + f^{21} f_{21} \\
        &= 2 e^i e_i + 2 f^{21} f_{21},
    \end{aligned}
\]
where we define 
\begin{equation}
    e_i = \partial_0 a_i - \partial_i a_0, \quad b = \partial_1 a_2 - \partial_2 a_1,
\end{equation}
Note that since in the Minkowski space  
\[
    \partial_0 = \partial^0, \quad a_i = - a^i, \quad \partial_{1,2} = - \partial^{1, 2}, \quad a_{1, 2} = - a^{1, 2},
\]
we have 
\[
    f_{\mu \nu} f^{\mu \nu} = - 2 \vb*{e}^2 + 2 b^2,
\]
so the final Lagrangian is 
\begin{equation} \marginnote{Eq.~(6.3.2)}
    \mathcal{L}_{U(1)} = \frac{1}{2 g^2} (\vb*{e}^2 - b^2),
\end{equation}
where the inner product is now defined in the Euclidean space. 

Now we try to quantize the theory. \marginnote{Sec.~6.3.1} We impose the Coulomb gauge 
\begin{equation} \marginnote{Eq.~(6.3.3)}
    \div{\vb*{a}} = 0,
    \label{eq:coulomb-gauge}
\end{equation}
and since we are dealing with the free space electromagnetic field, we can actually additionally impose 
\begin{equation} \marginnote{Discussion between Eq.~(6.3.3) and Eq.~(6.3.4)}
    a_0 = 0.
\end{equation}
This \concept{radiation gauge} is well-known when dealing with 3D electromagnetic waves. Here we show explicitly that under 
\eqref{eq:coulomb-gauge}, $a_0$ is actually decoupled from other degrees of freedom and can be set to 
any constant value. Now in the Euclidean space, we have  
\begin{equation}
    e_i = e^i = - \partial_t a_i - \partial_i a_0, \quad b = \partial_1 a_2 - \partial_2 a_1,
\end{equation}
and terms involving $a_0$ Lagrangian all come from the $\vb*{e}^2$ term, and we have 
\[
    \begin{aligned}
        &\quad \partial_i a_0 \partial_i a_0 + \partial_0 a_i \partial_i a_0 + \partial_i a_0 \partial_0 a_i \\
        &\simeq (\partial_i a_0)^2 - 2 a_0 \partial_i \partial_0 a_i 
        = (\partial_i a_0)^2 ,
    \end{aligned}
\]
so we see that $a_0$ is decoupled from $\vb*{a}$. 

Under the radiation gauge, we can decompose $\vb*{a}$ into different modes. 

I feel I'm still unable to understand what he wanted to do \dots 

Some key points in Section 6.3:

When deriving the equation after Eq.~(6.3.4), pay attention to the fact that 
\[
    \int \dd[2]{\vb*{x}} \sim L_1 L_2,
\]
and the Fourier transformation of a single $\ee^{\ii \vb*{k} \cdot \vb*{x}}$ can be regarded as zero. 

The second term of $a_i$ can actually be extended into 
\[
    b_0 x^1 \delta_{i, 2} - b_0 x^2 \delta_{i, 1}
\]
or not? And why is $b_0$ not an operator?

\bibliographystyle{plain}
\bibliography{gauge} 

\end{document}