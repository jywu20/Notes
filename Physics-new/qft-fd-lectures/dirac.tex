\documentclass[hyperref, a4paper]{article}

\usepackage{geometry}
\usepackage{marginnote}
\usepackage{titling}
\usepackage{titlesec}
% No longer needed, since we will use enumitem package
% \usepackage{paralist}
\usepackage{enumitem}
\usepackage{footnote}
\usepackage{enumerate}
\usepackage{amsmath, amssymb, amsthm}
\usepackage{mathtools}
\usepackage{bbm}
\usepackage{cite}
\usepackage{graphicx}
\usepackage{subfigure}
\usepackage{physics}
\usepackage{tensor}
\usepackage{siunitx}
\usepackage{slashed}
\usepackage{centernot}
\usepackage[version=4]{mhchem}
\usepackage{tikz}
\usepackage{xcolor}
\usepackage{listings}
\usepackage{autobreak}
\usepackage[ruled, vlined, linesnumbered]{algorithm2e}
\usepackage{nameref,zref-xr}
\zxrsetup{toltxlabel}
\zexternaldocument*[rep-]{lorentz-rep}[lorentz-rep.pdf]
\usepackage[colorlinks,unicode]{hyperref} % , linkcolor=black, anchorcolor=black, citecolor=black, urlcolor=black, filecolor=black
\usepackage[most]{tcolorbox}
\usepackage{prettyref}

% Page style
\geometry{left=3.18cm,right=3.18cm,top=2.54cm,bottom=2.54cm}
\titlespacing{\paragraph}{0pt}{1pt}{10pt}[20pt]
\setlength{\droptitle}{-5em}
\preauthor{\vspace{-10pt}\begin{center}}
\postauthor{\par\end{center}}

% More compact lists 
%\setlist[itemize]{
%    itemindent=17pt, 
%    leftmargin=1pt,
%    listparindent=\parindent,
%    parsep=0pt,
%}

% Math operators
\DeclareMathOperator{\timeorder}{\mathcal{T}}
\DeclareMathOperator{\diag}{diag}
\DeclareMathOperator{\legpoly}{P}
\DeclareMathOperator{\primevalue}{P}
\DeclareMathOperator{\sgn}{sgn}
\newcommand*{\ii}{\mathrm{i}}
\newcommand*{\ee}{\mathrm{e}}
\newcommand*{\const}{\mathrm{const}}
\newcommand*{\suchthat}{\quad \text{s.t.} \quad}
\newcommand*{\argmin}{\arg\min}
\newcommand*{\argmax}{\arg\max}
\newcommand*{\normalorder}[1]{: #1 :}
\newcommand*{\pair}[1]{\langle #1 \rangle}
\newcommand*{\fd}[1]{\mathcal{D} #1}
\DeclareMathOperator{\bigO}{\mathcal{O}}

% Feynman slash
\newcommand{\fsl}[1]{{\centernot{#1}}}

% TikZ setting
\usetikzlibrary{arrows,shapes,positioning}
\usetikzlibrary{arrows.meta}
\usetikzlibrary{decorations.markings}
\tikzstyle arrowstyle=[scale=1]
\tikzstyle directed=[postaction={decorate,decoration={markings,
    mark=at position .5 with {\arrow[arrowstyle]{stealth}}}}]
\tikzstyle ray=[directed, thick]
\tikzstyle dot=[anchor=base,fill,circle,inner sep=1pt]

% Algorithm setting
% Julia-style code
\SetKwIF{If}{ElseIf}{Else}{if}{}{elseif}{else}{end}
\SetKwFor{For}{for}{}{end}
\SetKwFor{While}{while}{}{end}
\SetKwProg{Function}{function}{}{end}
\SetArgSty{textnormal}

\newcommand*{\concept}[1]{{\textbf{#1}}}

% Embedded codes
\lstset{basicstyle=\ttfamily,
  showstringspaces=false,
  commentstyle=\color{gray},
  keywordstyle=\color{blue}
}

% Reference formatting
\newrefformat{fig}{Figure~\ref{#1} on page~\pageref{#1}}

% Color boxes
\tcbuselibrary{skins, breakable, theorems}
\newtcbtheorem[number within=section]{warning}{Warning}%
  {colback=orange!5,colframe=orange!65,fonttitle=\bfseries, breakable}{warn}
\newtcbtheorem[number within=section]{note}{Note}%
  {colback=green!5,colframe=green!65,fonttitle=\bfseries, breakable}{note}
\newtcbtheorem[number within=section]{info}{Info}%
  {colback=blue!5,colframe=blue!65,fonttitle=\bfseries, breakable}{info}

\newcommand{\hwtwo}{\href{../2/2.pdf}{Homework 2}}

\newcommand{\lorentzrep}{\href{lorentz-rep.pdf}{this article}}

\title{Dirac Theory}
\author{Jinyuan Wu}

\begin{document}

\maketitle

\section{The Dirac equation}

We have found spinors are representations of Lorentz group in \lorentzrep.

\begin{equation}
    (\ii \gamma^\mu \partial_\mu - m) \psi(x) = 0.
    \label{eq:dirac-eq}
\end{equation}

\section{Free-particle solutions of the Dirac equation}
 
Now we solve \eqref{eq:dirac-eq}. \marginnote{Peskin Section~3.3.}
We work under the chiral basis. We search for a plane wave solution 
\begin{equation}
    \psi(x) = u(p) \ee^{- \ii p \cdot x}, \quad p^2 = m^2.
\end{equation}
In the rest frame, $p = p_0 = (m, 0)$, and \eqref{eq:dirac-eq} becomes 
\[
    (m \gamma^0 - m) u(p_0) = m \pmqty{-1 & 1 \\ 1 & -1} u(p_0) = 0,
\]
and the solutions are 
\begin{equation}
    u(p_0) = \sqrt{m} \pmqty{\xi \\ \xi}, 
\end{equation}
for \emph{any} Weyl spinor $\xi$. The normalization condition is $\xi^\dagger \xi = 1$, and 
\begin{equation}
    \xi_{S_z = \uparrow} = \pmqty{1 \\ 0}, \quad \xi_{S_z = \downarrow} = \pmqty{0 \\ 1}.
\end{equation}

We can find the solution for an arbitrary momentum by boosting. From \eqref{rep-eq:four-vec-lorentz} and 
\eqref{rep-eq:weyl-lorentz} in \lorentzrep, we have 
\[
    (\mathcal{J}^{03})\indices{^\alpha_\beta} = \ii (\eta^{0 \alpha} \delta\indices{^3_\beta} - \delta\indices{^0_\beta} \eta^{3 \alpha}) = \ii \pmqty{ 0 & 0 & 0 & 1 \\ 0 & 0 & 0 & 0 \\ 0 & 0 & 0 & 0 \\ 1 & 0 & 0 & 0 },
\]
and 
\[
    S^{03} = - \frac{\ii}{2} \pmqty{\dmat{\sigma^3, - \sigma^3}},
\]
and therefore these two transformations 
\[
    \exp\left( \eta \pmqty{ 0 & 0 & 0 & 1 \\ 0 & 0 & 0 & 0 \\ 0 & 0 & 0 & 0 \\ 1 & 0 & 0 & 0 } \right), \quad 
    \exp(\frac{- \eta}{2} \pmqty{\dmat{\sigma^3, - \sigma^3}} )
\]
are the same Lorentz group element acting on different objects. To boost $(m, 0)$ into $(E, p^3)$, one need 
an $eta$ defined by \marginnote{Peskin (3.48)}
\begin{equation}
    \begin{aligned}
        \left(\begin{array}{c}
        E \\
        p^{3}
        \end{array}\right) &=\exp \left[\eta\left(\begin{array}{ll}
        0 & 1 \\
        1 & 0
        \end{array}\right)\right]\left(\begin{array}{c}
        m \\
        0
        \end{array}\right) \\
        &=\left[\cosh \eta\left(\begin{array}{ll}
        1 & 0 \\
        0 & 1
        \end{array}\right)+\sinh \eta\left(\begin{array}{ll}
        0 & 1 \\
        1 & 0
        \end{array}\right)\right]\left(\begin{array}{c}
        m \\
        0
        \end{array}\right) \\
        &=\left(\begin{array}{c}
        m \cosh \eta \\
        m \sinh \eta
        \end{array}\right) ,
        \end{aligned}
    \label{eq:momentum-change}
\end{equation}
and therefore $u(p)$ is obtained by a boost on the $z$ direction with the same $\eta$, 
which is \marginnote{Peskin (3.49)}
\begin{equation}
    \begin{aligned}
        u(p) &=\exp \left[-\frac{1}{2} \eta\left(\begin{array}{cc}
        \sigma^{3} & 0 \\
        0 & -\sigma^{3}
        \end{array}\right)\right] \sqrt{m}\left(\begin{array}{l}
        \xi \\
        \xi
        \end{array}\right) \\
        &=\left[\cosh \left(\frac{1}{2} \eta\right)\left(\begin{array}{cc}
        1 & 0 \\
        0 & 1
        \end{array}\right)-\sinh \left(\frac{1}{2} \eta\right)\left(\begin{array}{cc}
        \sigma^{3} & 0 \\
        0 & -\sigma^{3}
        \end{array}\right)\right] \sqrt{m}\left(\begin{array}{l}
        \xi \\
        \xi
        \end{array}\right) \\
        &=\left(\begin{array}{cc}
        e^{\eta / 2}\left(\frac{1-\sigma^{3}}{2}\right)+e^{-\eta / 2}\left(\frac{1+\sigma^{3}}{2}\right) & 0 \\
        0 & e^{\eta / 2}\left(\frac{1+\sigma^{3}}{2}\right)+e^{-\eta / 2}\left(\frac{1-\sigma^{3}}{2}\right)
        \end{array}\right) \sqrt{m}\left(\begin{array}{l}
        \xi \\
        \xi
        \end{array}\right) \\
        &=\left(\begin{array}{l}
        {\left[\sqrt{E+p^{3}}\left(\frac{1-\sigma^{3}}{2}\right)+\sqrt{E-p^{3}}\left(\frac{1+\sigma^{3}}{2}\right)\right] \xi} \\
        {\left[\sqrt{E+p^{3}}\left(\frac{1+\sigma^{3}}{2}\right)+\sqrt{E-p^{3}}\left(\frac{1-\sigma^{3}}{2}\right)\right] \xi}
        \end{array}\right) ,
        \end{aligned}
\end{equation}
where the last line comes from the fact that since \eqref{eq:momentum-change}, we have 
\[
    E + p^3 = m (\cosh \eta + \sinh \eta) = m \ee^{\eta}, \quad 
    E - p^3 = m (\cosh \eta - \sinh \eta) = m \ee^{-\eta}.
\]
The last line gives \marginnote{Peskin (3.50)}
\begin{equation}
    u(p) = \left(\begin{array}{c}
        \sqrt{p \cdot \sigma} \xi \\
        \sqrt{p \cdot \bar{\sigma}} \xi
        \end{array}\right),
    \label{eq:u-general-p}
\end{equation}
where $\bar{\sigma} = (\sigma^0, - \vb*{\sigma})$. \marginnote{Peskin (3.41)}
To prove \eqref{eq:u-general-p}, we note that 
\[
    \left(\sqrt{E+p^{3}}\left(\frac{1-\sigma^{3}}{2}\right)+\sqrt{E-p^{3}}\left(\frac{1+\sigma^{3}}{2}\right)\right)^2 = 
    \pmqty{\dmat{E - p^3, E + p^3}} = p^0 \sigma^0 - p^3 \sigma^3 = p \cdot \sigma,
\]
and the same thing hold for $\sqrt{p \cdot \bar{\sigma}}$. 
It would be straightforward to find \marginnote{Peskin (3.51)}
\begin{equation}
    (p \cdot \sigma)(p \cdot \bar{\sigma})=p^{2}=m^{2}.
\end{equation}

It can be easily found that \marginnote{See the discussion in Peskin between (3.65) and (3.66)}
\begin{equation}
    (p \cdot \sigma) (p \cdot \bar{\sigma}) = (p^0)^2 - \vb*{p}^2 = m^2.
\end{equation}

\section{Discrete symmetries of the Dirac theory}

There are three discrete symmetry operations that are important: \marginnote{Peskin Section~3.6.}
\begin{itemize}
    \item \concept{Charge conjugation}, exchanging particles and antiparticles. The name comes from 
    the fact that in a gauge theory, particles and antiparticles carry charges that are equal in 
    the absolute value but different in sign. 
    \item \concept{Time reversal symmetry}, one discrete generator of the Lorentz group.
    \item \concept{Parity}, another discrete generator of the Lorentz group.
\end{itemize}



\end{document}