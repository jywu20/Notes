\documentclass[hyperref, a4paper]{article}

\usepackage{geometry}
\usepackage{float}
\usepackage{titling}
\usepackage{titlesec}
% No longer needed, since we will use enumitem package
% \usepackage{paralist}
\usepackage{enumitem}
\usepackage{footnote}
\usepackage{enumerate}
\usepackage{amsmath, amssymb, amsthm}
\usepackage{mathtools}
\usepackage{bbm}
\usepackage{cite}
\usepackage{graphicx}
\usepackage{subcaption}
\usepackage{physics}
\usepackage{tensor}
\usepackage{siunitx}
\usepackage{booktabs}
\usepackage[version=4]{mhchem}
\usepackage{tikz}
\usepackage{xcolor}
\usepackage{listings}
\usepackage{autobreak}
\usepackage[ruled, vlined, linesnumbered]{algorithm2e}
\usepackage{xr-hyper}
\usepackage[colorlinks,unicode]{hyperref} % , linkcolor=black, anchorcolor=black, citecolor=black, urlcolor=black, filecolor=black
\usepackage{prettyref}

% Page style
\geometry{left=3.18cm,right=3.18cm,top=2.54cm,bottom=2.54cm}
\titlespacing{\paragraph}{0pt}{1pt}{10pt}[20pt]
\setlength{\droptitle}{-5em}
\preauthor{\vspace{-10pt}\begin{center}}
\postauthor{\par\end{center}}

% More compact lists 
\setlist[itemize]{itemindent=17pt, leftmargin=1pt}

% Math operators
\DeclareMathOperator{\timeorder}{\mathcal{T}}
\DeclareMathOperator{\diag}{diag}
\DeclareMathOperator{\legpoly}{P}
\DeclareMathOperator{\primevalue}{P}
\DeclareMathOperator{\sgn}{sgn}
\newcommand*{\ii}{\mathrm{i}}
\newcommand*{\ee}{\mathrm{e}}
\newcommand*{\const}{\mathrm{const}}
\newcommand*{\suchthat}{\quad \text{s.t.} \quad}
\newcommand*{\argmin}{\arg\min}
\newcommand*{\argmax}{\arg\max}
\newcommand*{\normalorder}[1]{: #1 :}
\newcommand*{\pair}[1]{\langle #1 \rangle}
\newcommand*{\fd}[1]{\mathcal{D} #1}
\DeclareMathOperator{\bigO}{\mathcal{O}}
\DeclareMathOperator{\object}{Ob}
\DeclareMathOperator{\morphism}{Hom}

% TikZ setting
\usetikzlibrary{arrows,shapes,positioning}
\usetikzlibrary{arrows.meta}
\usetikzlibrary{decorations.markings}
\tikzstyle arrowstyle=[scale=1]
\tikzstyle directed=[postaction={decorate,decoration={markings,
    mark=at position .5 with {\arrow[arrowstyle]{stealth}}}}]
\tikzstyle ray=[directed, thick]
\tikzstyle dot=[anchor=base,fill,circle,inner sep=1pt]

% Algorithm setting
% Julia-style code
\SetKwIF{If}{ElseIf}{Else}{if}{}{elseif}{else}{end}
\SetKwFor{For}{for}{}{end}
\SetKwFor{While}{while}{}{end}
\SetKwProg{Function}{function}{}{end}
\SetArgSty{textnormal}

\newcommand*{\concept}[1]{{\textbf{#1}}}

\newrefformat{fig}{Figure~\ref{#1}}

% Embedded codes
\lstset{basicstyle=\ttfamily,
  showstringspaces=false,
  commentstyle=\color{gray},
  keywordstyle=\color{blue}
}

% Disable unsupported commands in bookmark titles 
\pdfstringdefDisableCommands{%
  \def\\{}%
  \def\texttt#1{<#1>}%
  \def\mathbb#1{#1}%
}
\pdfstringdefDisableCommands{\def\eqref#1{(\ref{#1})}}

\makeatletter
\pdfstringdefDisableCommands{\let\HyPsd@CatcodeWarning\@gobble}
\makeatother

\title{Scattering in Relativistic Quantum Field Theories by Prof. Dingyu Shao}
\author{Jinyuan Wu}

\begin{document}

\maketitle

\section{Demonstration of the relation between interaction Feynman propagators, free Feynman propagators, $S$-matrices using scalar field}

Suppose the vacuum state $\ket*{\Omega}$ of a field theory with interaction is not orthogonal to the vacuum state $\ket*{0}$ of the free theory.
In order to build connection between the two states, we imaginarily set $\ket*{0}$ to be the initial state and turn on the interaction, 
and make the excited components in $\ket*{0}$ ``relaxed'' back to $\ket*{\Omega}$.
We add an imaginary part to the time to make this happen and we have 
\[
    |\Omega\rangle=\lim _{T \rightarrow \infty(1-\ii \epsilon)}\left(e^{-i E_{0} T}\langle\Omega \mid 0\rangle\right)^{-1} e^{-\ii H T}|0\rangle.
\]
\begin{equation}
    \ket*{\Omega} = \lim_{T \to \infty(1 - \ii \epsilon)} (\ee^{- \ii E_0 (t - (-T))} \braket*{\Omega}{0})^{-1} U(t_0, -T) \ket*{0},
\end{equation}
Suppose $x^0 > y^0 > t_0$, we have 
\[
    \begin{aligned}
        \expval*{\timeorder[\phi(x) \phi(y)]}{\Omega} 
    \end{aligned}
\]

See Peskin 4.2 for details

So in the end we have
\begin{equation}
    \expval*{\timeorder[\phi(x) \phi(y)]}{\Omega} = \lim_{T \to \infty(1 - \ii \epsilon)} \frac{\expval*{\timeorder \phi_\text{I}(x) \phi_\text{I}(y) \exp (- \ii \int_{-T}^T \dd{t} H_\text{I}(t))}{0}}{\expval*{\timeorder \exp(- \ii \int_{-T}^T \dd{t} H_\text{I}(t))}{0}}.
\end{equation}

\begin{equation}
    S = \mathbbm{1} + \ii T ,
\end{equation}
and we define 
\begin{equation}
    \ _0\mel{p_1, p_2, \ldots, p_m}{T}{q_1, q_2, \ldots, q_n}_0 = (2\pi)^{m+n} \delta^{(4)}(\sum p - \sum q) \mathcal{M}(p_1, p_2, \ldots, p_m \to q_1, q_2, \ldots, q_n).
\end{equation}

\section{Scattering}

A scattering experiment involves processes like the following:
\begin{equation}
    p_1 + p_2 \longrightarrow \sum_{i} q_i.
\end{equation}
The 
\begin{equation}
    \dd{P} = 
\end{equation}

\end{document}