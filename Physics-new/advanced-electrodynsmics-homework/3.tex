\documentclass[hyperref, a4paper]{article}

\usepackage{geometry}
\usepackage{titling}
\usepackage{titlesec}
% No longer needed, since we will use enumitem package
% \usepackage{paralist}
\usepackage{enumitem}
\usepackage{footnote}
% Conflicts with enumitem
%\usepackage{enumerate}
\usepackage{amsmath, amssymb, amsthm}
\usepackage{mathtools}
\usepackage{bbm}
\usepackage{cite}
\usepackage{graphicx}
\usepackage{subfigure}
\usepackage{physics}
\usepackage{tensor}
\usepackage{siunitx}
\usepackage[version=4]{mhchem}
\usepackage{tikz}
\usepackage{xcolor}
\usepackage{listings}
\usepackage{autobreak}
\usepackage[ruled, vlined, linesnumbered]{algorithm2e}
\usepackage{nameref,zref-xr}
\zxrsetup{toltxlabel}
\zexternaldocument*[optics-]{../optics/optics}[optics.pdf]
\zexternaldocument*[solid-]{../solid/solid}[solid.pdf]
\usepackage[colorlinks,unicode]{hyperref} % , linkcolor=black, anchorcolor=black, citecolor=black, urlcolor=black, filecolor=black
\usepackage[most]{tcolorbox}
\usepackage{prettyref}

% Page style
\geometry{left=3.18cm,right=3.18cm,top=2.54cm,bottom=2.54cm}
\titlespacing{\paragraph}{0pt}{1pt}{10pt}[20pt]
\setlength{\droptitle}{-5em}
\preauthor{\vspace{-10pt}\begin{center}}
\postauthor{\par\end{center}}

% More compact lists 
\setlist[itemize]{
    itemindent=17pt, 
    leftmargin=1pt,
    listparindent=\parindent,
    parsep=0pt,
}

% Math operators
\DeclareMathOperator{\timeorder}{\mathcal{T}}
\DeclareMathOperator{\diag}{diag}
\DeclareMathOperator{\legpoly}{P}
\DeclareMathOperator{\primevalue}{P}
\DeclareMathOperator{\sgn}{sgn}
\newcommand*{\ii}{\mathrm{i}}
\newcommand*{\ee}{\mathrm{e}}
\newcommand*{\const}{\mathrm{const}}
\newcommand*{\suchthat}{\quad \text{s.t.} \quad}
\newcommand*{\argmin}{\arg\min}
\newcommand*{\argmax}{\arg\max}
\newcommand*{\normalorder}[1]{: #1 :}
\newcommand*{\pair}[1]{\langle #1 \rangle}
\newcommand*{\fd}[1]{\mathcal{D} #1}
\DeclareMathOperator{\bigO}{\mathcal{O}}

% TikZ setting
\usetikzlibrary{arrows,shapes,positioning}
\usetikzlibrary{calc}
\usetikzlibrary{arrows.meta}
\usetikzlibrary{decorations.markings}
\tikzstyle arrowstyle=[scale=1]
\tikzstyle directed=[postaction={decorate,decoration={markings,
    mark=at position .5 with {\arrow[arrowstyle]{stealth}}}}]
\tikzstyle ray=[directed, thick]
\tikzstyle dot=[anchor=base,fill,circle,inner sep=1pt]

% Algorithm setting
% Julia-style code
\SetKwIF{If}{ElseIf}{Else}{if}{}{elseif}{else}{end}
\SetKwFor{For}{for}{}{end}
\SetKwFor{While}{while}{}{end}
\SetKwProg{Function}{function}{}{end}
\SetArgSty{textnormal}

\newcommand*{\concept}[1]{{\textbf{#1}}}

% Embedded codes
\lstset{basicstyle=\ttfamily,
  showstringspaces=false,
  commentstyle=\color{gray},
  keywordstyle=\color{blue}
}

% Reference formatting
\newrefformat{fig}{Figure~\ref{#1} on page~\pageref{#1}}

% Color boxes
\tcbuselibrary{skins, breakable, theorems}
\newtcbtheorem[number within=section]{warning}{Warning}%
  {colback=orange!5,colframe=orange!65,fonttitle=\bfseries, breakable}{warn}
\newtcbtheorem[number within=section]{note}{Note}%
  {colback=green!5,colframe=green!65,fonttitle=\bfseries, breakable}{note}

\title{Advanced Electrodynamics, Homework 3}
\author{Jinyuan Wu}

\begin{document}

\maketitle

\paragraph{2D Green function} (a) Derive the 2D Green function in polar coordinates.

\paragraph{Solution} \begin{itemize}
\item[(a)] The 2D Green function is given by the solution of the two dimensional version of Helmholtz equation
with an external source: 
\begin{equation}
    (\laplacian + k^2) G_0(\vb*{r} - \vb*{r}') = - \delta^{(2)}(\vb*{r} - \vb*{r}').
\end{equation}  
The solution, in terms of Fourier transformation, is 
\[
    G_0(\vb*{R}) = - \int \frac{\dd[2]{\vb*{p}}}{(2\pi)^2} \frac{\ee^{\ii \vb*{p} \cdot \vb*{R}}}{k^2 - {\vb*{p}}^2 + \ii 0^+}.
\]
In polar coordinates where we consider the direction of $\vb*{R}$ to be the $\theta = 0$ axis, we have 
\[
    \begin{aligned}
        G_0(\vb*{R}) &= - \frac{1}{(2\pi)^2} \int_{0}^\infty p \dd{p} \int_{0}^{2\pi} \dd{\theta} 
        \frac{\ee^{\ii p \abs*{\vb*{R}} \cos \theta}}{k^2 - p^2 + \ii 0^+} \\
        &= \frac{1}{(2\pi)^2} \frac{1}{2} \int_{0}^{2\pi} \dd{\theta} \int_0^\infty \dd{p}
        \left( \frac{1}{p + k - \ii 0^+} + \frac{1}{p - k - \ii 0^+} \right) 
        \ee^{\ii p \abs*{\vb*{R}} \cos \theta} \\
        &= \frac{1}{2 (2\pi)^2} \left(
            \int_{-\pi/2}^{\pi/2} \dd{\theta} \times 2 \pi \ii \ee^{\ii k \abs*{\vb*{R}} \cos \theta}
            + \int_{\pi/2}^{3\pi/2} \dd{\theta} \times 2 \pi \ii \ee^{- \ii k \abs*{\vb*{R}} \cos \theta}
        \right) \\
        &= \frac{\ii}{4\pi} ((\pi J_0(k \abs*{\vb*{R}}) + \ii \pi \mathbf{H}(k \abs*{\vb*{R}}))
        + (\pi J_0(k \abs*{\vb*{R}}) - \ii \pi \mathbf{H}(k \abs*{\vb*{R}}))) \\
        &= \frac{\ii}{4\pi} \times 2 \pi J_0(k \abs*{\vb*{R}}).
    \end{aligned}
\]
So we get 
\begin{equation}
    G_0(\vb*{R}) = \frac{\ii}{2} 
\end{equation}

\item[(b)]
\end{itemize}

\paragraph{}

\paragraph{Dyadic green function in Fourier space} (a) Show that in vacuum the Maxwell equations can be rephrased 
into 
\begin{equation}
    \vb*{M}^{2}\left[\begin{array}{l}
        \vb*{E} \\
        \vb*{H}
        \end{array}\right]=\left[\begin{array}{cc}
        c^{2} \vb*{k} \cdot \vb*{k}-c^{2} \vb*{k} \vb*{k} & 0 \\
        0 & c^{2} \vb*{k} \cdot \vb*{k}-c^{2} \vb*{k} \vb*{k}
        \end{array}\right]\left[\begin{array}{c}
        \vb*{E} \\
        \vb*{H}
        \end{array}\right]=\omega^{2}\left[\begin{array}{c}
        \vb*{E} \\
        \vb*{H}
        \end{array}\right].
\end{equation}
(b) Find the eigenvalues and eigenvectors. (c) Derive the Green function in the Fourier space, 
and show why longitude modes are absent.

\paragraph{Solution} \begin{itemize}
\item[(a)] In the Fourier space the Maxwell equations are 
\[
    \begin{aligned}
        \vb*{k} \cdot \vb*{E} &= 0, \\ 
        \vb*{k} \times \vb*{E} &= \omega \vb*{B}, \\
        \vb*{k} \cdot \vb*{B} &= 0, \\
        \vb*{k} \times \vb*{B} &= - \frac{1}{c^2} \omega \vb*{E}.
    \end{aligned}
\]
where $\vb*{E}$ and $\vb*{B}$ are actually $\vb*{\mathcal{E}}(\vb*{k}, \omega)$ 
and $\vb*{\mathcal{B}}(\vb*{k}, \omega)$, respectively. 
 
\end{itemize}

\end{document}