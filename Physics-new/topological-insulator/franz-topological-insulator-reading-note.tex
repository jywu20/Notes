\documentclass[hyperref, a4paper]{article}

\usepackage{geometry}
\usepackage{titling}
\usepackage{titlesec}
% No longer needed, since we will use enumitem package
% \usepackage{paralist}
\usepackage{enumitem}
\usepackage{footnote}
\usepackage{enumerate}
\usepackage{amsmath, amssymb, amsthm}
\usepackage{mathtools}
\usepackage{bbm}
\usepackage{cite}
\usepackage{graphicx}
\usepackage[labelformat=simple]{subcaption}
\usepackage{physics}
\usepackage{tensor}
\usepackage{siunitx}
\usepackage[version=4]{mhchem}
\usepackage{tikz}
\usepackage{xcolor}
\usepackage{listings}
\usepackage{marginnote}
\usepackage{autobreak}
\usepackage[ruled, vlined, linesnumbered]{algorithm2e}
\usepackage{nameref,zref-xr}
\zxrsetup{toltxlabel}
\zexternaldocument*[optics-]{../optics/optics}[optics.pdf]
\zexternaldocument*[solid-]{../solid/solid}[solid.pdf]
\usepackage[colorlinks,unicode]{hyperref} % , linkcolor=black, anchorcolor=black, citecolor=black, urlcolor=black, filecolor=black
\usepackage{prettyref}

% Page style
\geometry{left=3.18cm,right=3.18cm,top=2.54cm,bottom=2.54cm}
\titlespacing{\paragraph}{0pt}{1pt}{10pt}[20pt]
\setlength{\droptitle}{-5em}
\preauthor{\vspace{-10pt}\begin{center}}
\postauthor{\par\end{center}}

% More compact lists 
\setlist[itemize]{
    itemindent=17pt, 
    leftmargin=1pt,
    listparindent=\parindent,
    parsep=0pt,
}

% Math operators
\DeclareMathOperator{\timeorder}{\mathcal{T}}
\DeclareMathOperator{\diag}{diag}
\DeclareMathOperator{\legpoly}{P}
\DeclareMathOperator{\primevalue}{P}
\DeclareMathOperator{\sgn}{sgn}
\newcommand*{\ii}{\mathrm{i}}
\newcommand*{\ee}{\mathrm{e}}
\newcommand*{\const}{\mathrm{const}}
\newcommand*{\suchthat}{\quad \text{s.t.} \quad}
\newcommand*{\argmin}{\arg\min}
\newcommand*{\argmax}{\arg\max}
\newcommand*{\normalorder}[1]{: #1 :}
\newcommand*{\pair}[1]{\langle #1 \rangle}
\newcommand*{\fd}[1]{\mathcal{D} #1}
\DeclareMathOperator{\bigO}{\mathcal{O}}

% TikZ setting
\usetikzlibrary{arrows,shapes,positioning}
\usetikzlibrary{arrows.meta}
\usetikzlibrary{decorations.markings}
\tikzstyle arrowstyle=[scale=1]
\tikzstyle directed=[postaction={decorate,decoration={markings,
    mark=at position .5 with {\arrow[arrowstyle]{stealth}}}}]
\tikzstyle ray=[directed, thick]
\tikzstyle dot=[anchor=base,fill,circle,inner sep=1pt]

% Algorithm setting
% Julia-style code
\SetKwIF{If}{ElseIf}{Else}{if}{}{elseif}{else}{end}
\SetKwFor{For}{for}{}{end}
\SetKwFor{While}{while}{}{end}
\SetKwProg{Function}{function}{}{end}
\SetArgSty{textnormal}

\newcommand*{\concept}[1]{{\textbf{#1}}}

% Embedded codes
\lstset{basicstyle=\ttfamily,
  showstringspaces=false,
  commentstyle=\color{gray},
  keywordstyle=\color{blue}
}

\newcommand{\soliddoc}{\href{../solid/solid.pdf}{this note}}

\renewcommand\thesubfigure{(\alph{subfigure})}
\newrefformat{fig}{Figure~\ref{#1}}

\title{Reading Note of Topological Insulators by M. Franz and L. Molenkamp}
\author{Jinyuan Wu}

\begin{document}

\maketitle

This is a reading note of \cite{franz2013topological}.

\section{SSH model and AKLT chain}

We have already done some calculation about the SSH model in Section \ref{solid-sec:ssh} in \soliddoc. Here we briefly review 
the model. \marginnote{Sec.~3.2} The model (spinless version) is 
\begin{equation} \marginnote{Sec.~3.2, (15)}
    H = \sum_i (t + \var{t}) c^\dagger_{i \text{A}} c_{i \text{B}} + (t - \var{t}) c^\dagger_{i+1, \text{A}} c_{i \text{B}} + \text{h.c.}
\end{equation}
After a Fourier transformation we have 
\begin{equation} \marginnote{(16), (17)}
    H = \sum_k c^\dagger_{ka} H_{ab} c_{kb}, \quad H(k) = \vb*{d}(k) \cdot \vb*{\sigma},
\end{equation}
where 
\begin{equation}
    \begin{aligned}
        d_x(k) &= (t + \var{t}) + (t - \var{t}) \cos ka, \\
        d_y(k) &= (t - \var{t}) \sin ka, \\
        d_z(k) &= 0 .
    \end{aligned}
    \label{eq:ssh-spectrum}
\end{equation}
All two-band models can be rewritten into $\sum_k d_\mu \sigma^\mu$, and here $d_0 = d_3 = 0$. 
Now we see \eqref{eq:ssh-spectrum} is a $S^1 \to S^1$ mapping, and it can be classified by the winding number 
around $\vb*{d} = 0$, which is either $0$ or $1$ when we adjust $t$ and $\var{t}$. The phase 
where the winding number is 0 is a trivial phase, with the same topological properties of the vacuum, so 
it does not have any edge state, which can be verified explicitly and is illustrated in \prettyref{fig:ssh-trivial}.
The phase with winding number being 1 is topologically non-trivial, and something must happen at the edge, 
so we predict there will be an edge state, which can also be explicitly verified and is shown in \prettyref{fig:ssh-edge}.

The existence of a topological phase transition between the two states relies on the fact that the curve of 
$\vb*{d}$ decided by \eqref{eq:ssh-spectrum} is $S^1$. If we can perturb the system with an external Hamiltonian
which provides a non-zero $d_z$, then we can smoothly switch from one phase to another. 
The problem is under which symmetry constraint we have $d_0 = d_3 = 0$. Note that 
\[
    \acomm*{\sigma^z}{\sigma^x} = \acomm*{\sigma^z}{\sigma^y} = 0, \quad 
    \acomm*{\sigma^z}{\sigma^0} = \acomm*{\sigma^z}{\sigma^z} = 2 \sigma^0,
\]
we have 
\begin{equation}
    \acomm*{\sigma^z}{d_\mu \sigma^\mu} = 2 (d_0 + d_3) \sigma^0 .
\end{equation}
Therefore, under the condition 
\begin{equation}
    \comm*{\sigma^z}{H} = 0,
\end{equation}
% Question: d_0 = 0???
a two-band system always

\begin{figure}
    \centering
    \begin{subfigure}{0.8\textwidth}
        \centering
        

\tikzset{every picture/.style={line width=0.75pt}} %set default line width to 0.75pt        

\begin{tikzpicture}[x=0.75pt,y=0.75pt,yscale=-1,xscale=1]
%uncomment if require: \path (0,300); %set diagram left start at 0, and has height of 300

%Straight Lines [id:da9339444330477464] 
\draw [line width=1.5]    (87.6,148.94) -- (121.11,128.45) ;
%Shape: Circle [id:dp9507441294855312] 
\draw  [color={rgb, 255:red, 208; green, 2; blue, 27 }  ,draw opacity=1 ][fill={rgb, 255:red, 255; green, 255; blue, 255 }  ,fill opacity=1 ] (77.83,150.14) .. controls (77.72,144.74) and (82.01,140.27) .. (87.41,140.15) .. controls (92.82,140.04) and (97.29,144.33) .. (97.4,149.73) .. controls (97.51,155.13) and (93.23,159.61) .. (87.82,159.72) .. controls (82.42,159.83) and (77.95,155.54) .. (77.83,150.14) -- cycle ;
%Shape: Circle [id:dp8643130963730341] 
\draw  [color={rgb, 255:red, 74; green, 144; blue, 226 }  ,draw opacity=1 ][fill={rgb, 255:red, 255; green, 255; blue, 255 }  ,fill opacity=1 ] (111.32,128.65) .. controls (111.21,123.25) and (115.5,118.78) .. (120.9,118.67) .. controls (126.31,118.55) and (130.78,122.84) .. (130.89,128.24) .. controls (131.01,133.65) and (126.72,138.12) .. (121.31,138.23) .. controls (115.91,138.35) and (111.44,134.06) .. (111.32,128.65) -- cycle ;
%Straight Lines [id:da5498324199028166] 
\draw [line width=1.5]    (152.62,149.57) -- (186.14,129.09) ;
%Shape: Circle [id:dp4328369196955488] 
\draw  [color={rgb, 255:red, 208; green, 2; blue, 27 }  ,draw opacity=1 ][fill={rgb, 255:red, 255; green, 255; blue, 255 }  ,fill opacity=1 ] (142.84,149.78) .. controls (142.73,144.38) and (147.02,139.9) .. (152.42,139.79) .. controls (157.82,139.68) and (162.29,143.97) .. (162.41,149.37) .. controls (162.52,154.77) and (158.23,159.25) .. (152.83,159.36) .. controls (147.43,159.47) and (142.95,155.18) .. (142.84,149.78) -- cycle ;
%Shape: Circle [id:dp3960559998743858] 
\draw  [color={rgb, 255:red, 74; green, 144; blue, 226 }  ,draw opacity=1 ][fill={rgb, 255:red, 255; green, 255; blue, 255 }  ,fill opacity=1 ] (176.35,129.29) .. controls (176.24,123.89) and (180.53,119.42) .. (185.93,119.3) .. controls (191.33,119.19) and (195.81,123.48) .. (195.92,128.88) .. controls (196.03,134.29) and (191.74,138.76) .. (186.34,138.87) .. controls (180.94,138.98) and (176.47,134.7) .. (176.35,129.29) -- cycle ;
%Straight Lines [id:da7739710572922798] 
\draw [line width=1.5]    (219.61,148.17) -- (253.12,127.68) ;
%Shape: Circle [id:dp7636608333524304] 
\draw  [color={rgb, 255:red, 208; green, 2; blue, 27 }  ,draw opacity=1 ][fill={rgb, 255:red, 255; green, 255; blue, 255 }  ,fill opacity=1 ] (207.87,150.42) .. controls (207.75,145.01) and (212.04,140.54) .. (217.45,140.43) .. controls (222.85,140.32) and (227.32,144.6) .. (227.43,150.01) .. controls (227.55,155.41) and (223.26,159.88) .. (217.86,160) .. controls (212.45,160.11) and (207.98,155.82) .. (207.87,150.42) -- cycle ;
%Shape: Circle [id:dp34118834977139567] 
\draw  [color={rgb, 255:red, 74; green, 144; blue, 226 }  ,draw opacity=1 ][fill={rgb, 255:red, 255; green, 255; blue, 255 }  ,fill opacity=1 ] (243.34,127.89) .. controls (243.22,122.49) and (247.51,118.01) .. (252.92,117.9) .. controls (258.32,117.79) and (262.79,122.08) .. (262.9,127.48) .. controls (263.02,132.88) and (258.73,137.35) .. (253.33,137.47) .. controls (247.92,137.58) and (243.45,133.29) .. (243.34,127.89) -- cycle ;
%Straight Lines [id:da34572467796985484] 
\draw [color={rgb, 255:red, 0; green, 0; blue, 0 }  ,draw opacity=1 ][fill={rgb, 255:red, 208; green, 2; blue, 27 }  ,fill opacity=1 ][line width=1.5]    (351.64,148.81) -- (385.15,128.32) ;
%Shape: Circle [id:dp0427927859835342] 
\draw  [color={rgb, 255:red, 208; green, 2; blue, 27 }  ,draw opacity=1 ][fill={rgb, 255:red, 255; green, 255; blue, 255 }  ,fill opacity=1 ] (341.85,149.02) .. controls (341.74,143.61) and (346.03,139.14) .. (351.43,139.03) .. controls (356.83,138.91) and (361.31,143.2) .. (361.42,148.61) .. controls (361.53,154.01) and (357.24,158.48) .. (351.84,158.59) .. controls (346.44,158.71) and (341.97,154.42) .. (341.85,149.02) -- cycle ;
%Shape: Circle [id:dp133324728312036] 
\draw  [color={rgb, 255:red, 74; green, 144; blue, 226 }  ,draw opacity=1 ][fill={rgb, 255:red, 255; green, 255; blue, 255 }  ,fill opacity=1 ] (375.36,128.53) .. controls (375.25,123.12) and (379.54,118.65) .. (384.94,118.54) .. controls (390.35,118.43) and (394.82,122.71) .. (394.93,128.12) .. controls (395.05,133.52) and (390.76,137.99) .. (385.35,138.11) .. controls (379.95,138.22) and (375.48,133.93) .. (375.36,128.53) -- cycle ;
%Straight Lines [id:da8224850866746714] 
\draw [line width=1.5]    (416.66,149.45) -- (450.18,128.96) ;
%Shape: Circle [id:dp741311466978223] 
\draw  [color={rgb, 255:red, 208; green, 2; blue, 27 }  ,draw opacity=1 ][fill={rgb, 255:red, 255; green, 255; blue, 255 }  ,fill opacity=1 ] (406.88,149.65) .. controls (406.77,144.25) and (411.06,139.78) .. (416.46,139.66) .. controls (421.86,139.55) and (426.33,143.84) .. (426.45,149.24) .. controls (426.56,154.65) and (422.27,159.12) .. (416.87,159.23) .. controls (411.47,159.35) and (406.99,155.06) .. (406.88,149.65) -- cycle ;
%Shape: Circle [id:dp4961341695350736] 
\draw  [color={rgb, 255:red, 74; green, 144; blue, 226 }  ,draw opacity=1 ][fill={rgb, 255:red, 255; green, 255; blue, 255 }  ,fill opacity=1 ] (440.39,129.17) .. controls (440.28,123.76) and (444.57,119.29) .. (449.97,119.18) .. controls (455.37,119.06) and (459.85,123.35) .. (459.96,128.76) .. controls (460.07,134.16) and (455.78,138.63) .. (450.38,138.74) .. controls (444.98,138.86) and (440.51,134.57) .. (440.39,129.17) -- cycle ;
%Rounded Rect [id:dp9434888839128048] 
\draw   (74.85,157.53) .. controls (70.44,150.62) and (72.47,141.45) .. (79.37,137.03) -- (113.36,115.34) .. controls (120.26,110.92) and (129.44,112.95) .. (133.85,119.86) -- (133.85,119.86) .. controls (138.26,126.76) and (136.24,135.94) .. (129.33,140.35) -- (95.35,162.05) .. controls (88.44,166.46) and (79.26,164.44) .. (74.85,157.53) -- cycle ;
%Rounded Rect [id:dp17221804627544235] 
\draw   (139.85,157.53) .. controls (135.44,150.62) and (137.47,141.45) .. (144.37,137.03) -- (178.36,115.34) .. controls (185.26,110.92) and (194.44,112.95) .. (198.85,119.86) -- (198.85,119.86) .. controls (203.26,126.76) and (201.24,135.94) .. (194.33,140.35) -- (160.35,162.05) .. controls (153.44,166.46) and (144.26,164.44) .. (139.85,157.53) -- cycle ;
%Rounded Rect [id:dp3726670045223497] 
\draw   (205.85,157.53) .. controls (201.44,150.62) and (203.47,141.45) .. (210.37,137.03) -- (244.36,115.34) .. controls (251.26,110.92) and (260.44,112.95) .. (264.85,119.86) -- (264.85,119.86) .. controls (269.26,126.76) and (267.24,135.94) .. (260.33,140.35) -- (226.35,162.05) .. controls (219.44,166.46) and (210.26,164.44) .. (205.85,157.53) -- cycle ;
%Rounded Rect [id:dp9826226372275961] 
\draw   (338.85,157.53) .. controls (334.44,150.62) and (336.47,141.45) .. (343.37,137.03) -- (377.36,115.34) .. controls (384.26,110.92) and (393.44,112.95) .. (397.85,119.86) -- (397.85,119.86) .. controls (402.26,126.76) and (400.24,135.94) .. (393.33,140.35) -- (359.35,162.05) .. controls (352.44,166.46) and (343.26,164.44) .. (338.85,157.53) -- cycle ;
%Rounded Rect [id:dp7285905275692397] 
\draw   (403.85,157.53) .. controls (399.44,150.62) and (401.47,141.45) .. (408.37,137.03) -- (442.36,115.34) .. controls (449.26,110.92) and (458.44,112.95) .. (462.85,119.86) -- (462.85,119.86) .. controls (467.26,126.76) and (465.24,135.94) .. (458.33,140.35) -- (424.35,162.05) .. controls (417.44,166.46) and (408.26,164.44) .. (403.85,157.53) -- cycle ;

% Text Node
\draw (288,130.4) node [anchor=north west][inner sep=0.75pt]    {$\cdots $};
% Text Node
\draw (71,174.4) node [anchor=north west][inner sep=0.75pt]    {$m=1$};
% Text Node
\draw (141,174.4) node [anchor=north west][inner sep=0.75pt]    {$m=2$};
% Text Node
\draw (405,173.4) node [anchor=north west][inner sep=0.75pt]    {$m=N$};


\end{tikzpicture}

        \subcaption{}
        \label{fig:ssh-trivial}
    \end{subfigure}    
    \vspace{1em}
    \begin{subfigure}{0.8\textwidth}
        \centering
        

\tikzset{every picture/.style={line width=0.75pt}} %set default line width to 0.75pt        

\begin{tikzpicture}[x=0.75pt,y=0.75pt,yscale=-1,xscale=1]
%uncomment if require: \path (0,300); %set diagram left start at 0, and has height of 300

%Rounded Rect [id:dp9215995225762108] 
\draw   (182.36,184.85) .. controls (186.78,177.94) and (184.75,168.76) .. (177.84,164.35) -- (143.86,142.65) .. controls (136.95,138.24) and (127.78,140.27) .. (123.37,147.18) -- (123.37,147.18) .. controls (118.96,154.08) and (120.98,163.26) .. (127.89,167.67) -- (161.87,189.37) .. controls (168.78,193.78) and (177.95,191.76) .. (182.36,184.85) -- cycle ;
%Rounded Rect [id:dp004547789490766618] 
\draw   (247.39,185.49) .. controls (251.8,178.58) and (249.78,169.4) .. (242.87,164.99) -- (208.89,143.29) .. controls (201.98,138.88) and (192.81,140.91) .. (188.39,147.81) -- (188.39,147.81) .. controls (183.98,154.72) and (186.01,163.9) .. (192.92,168.31) -- (226.9,190.01) .. controls (233.81,194.42) and (242.98,192.39) .. (247.39,185.49) -- cycle ;
%Straight Lines [id:da4395270133325633] 
\draw [line width=1.5]    (137.11,155.45) -- (168.62,176.57) ;
%Straight Lines [id:da5748378659869535] 
\draw [line width=1.5]    (202.14,156.09) -- (233.65,177.21) ;
%Straight Lines [id:da8933616909770259] 
\draw [line width=1.5]    (269.12,154.68) -- (300.64,175.81) ;
%Straight Lines [id:da1313579068666484] 
\draw [line width=1.5]    (399.18,155.96) -- (430.69,177.09) ;
%Shape: Circle [id:dp06218610801583213] 
\draw  [color={rgb, 255:red, 208; green, 2; blue, 27 }  ,draw opacity=1 ][fill={rgb, 255:red, 255; green, 255; blue, 255 }  ,fill opacity=1 ] (93.83,177.14) .. controls (93.72,171.74) and (98.01,167.27) .. (103.41,167.15) .. controls (108.82,167.04) and (113.29,171.33) .. (113.4,176.73) .. controls (113.51,182.13) and (109.23,186.61) .. (103.82,186.72) .. controls (98.42,186.83) and (93.95,182.54) .. (93.83,177.14) -- cycle ;
%Shape: Circle [id:dp37903459495294767] 
\draw  [color={rgb, 255:red, 74; green, 144; blue, 226 }  ,draw opacity=1 ][fill={rgb, 255:red, 255; green, 255; blue, 255 }  ,fill opacity=1 ] (127.32,155.65) .. controls (127.21,150.25) and (131.5,145.78) .. (136.9,145.67) .. controls (142.31,145.55) and (146.78,149.84) .. (146.89,155.24) .. controls (147.01,160.65) and (142.72,165.12) .. (137.31,165.23) .. controls (131.91,165.35) and (127.44,161.06) .. (127.32,155.65) -- cycle ;
%Shape: Circle [id:dp16465523379151348] 
\draw  [color={rgb, 255:red, 208; green, 2; blue, 27 }  ,draw opacity=1 ][fill={rgb, 255:red, 255; green, 255; blue, 255 }  ,fill opacity=1 ] (158.84,176.78) .. controls (158.73,171.38) and (163.02,166.9) .. (168.42,166.79) .. controls (173.82,166.68) and (178.29,170.97) .. (178.41,176.37) .. controls (178.52,181.77) and (174.23,186.25) .. (168.83,186.36) .. controls (163.43,186.47) and (158.95,182.18) .. (158.84,176.78) -- cycle ;
%Shape: Circle [id:dp8930426798219266] 
\draw  [color={rgb, 255:red, 74; green, 144; blue, 226 }  ,draw opacity=1 ][fill={rgb, 255:red, 255; green, 255; blue, 255 }  ,fill opacity=1 ] (192.35,156.29) .. controls (192.24,150.89) and (196.53,146.42) .. (201.93,146.3) .. controls (207.33,146.19) and (211.81,150.48) .. (211.92,155.88) .. controls (212.03,161.29) and (207.74,165.76) .. (202.34,165.87) .. controls (196.94,165.98) and (192.47,161.7) .. (192.35,156.29) -- cycle ;
%Shape: Circle [id:dp1747805400646203] 
\draw  [color={rgb, 255:red, 208; green, 2; blue, 27 }  ,draw opacity=1 ][fill={rgb, 255:red, 255; green, 255; blue, 255 }  ,fill opacity=1 ] (223.87,177.42) .. controls (223.75,172.01) and (228.04,167.54) .. (233.45,167.43) .. controls (238.85,167.32) and (243.32,171.6) .. (243.43,177.01) .. controls (243.55,182.41) and (239.26,186.88) .. (233.86,187) .. controls (228.45,187.11) and (223.98,182.82) .. (223.87,177.42) -- cycle ;
%Shape: Circle [id:dp5019762571183983] 
\draw  [color={rgb, 255:red, 74; green, 144; blue, 226 }  ,draw opacity=1 ][fill={rgb, 255:red, 255; green, 255; blue, 255 }  ,fill opacity=1 ] (259.34,154.89) .. controls (259.22,149.49) and (263.51,145.01) .. (268.92,144.9) .. controls (274.32,144.79) and (278.79,149.08) .. (278.9,154.48) .. controls (279.02,159.88) and (274.73,164.35) .. (269.33,164.47) .. controls (263.92,164.58) and (259.45,160.29) .. (259.34,154.89) -- cycle ;
%Shape: Circle [id:dp6707626137558458] 
\draw  [color={rgb, 255:red, 208; green, 2; blue, 27 }  ,draw opacity=1 ][fill={rgb, 255:red, 255; green, 255; blue, 255 }  ,fill opacity=1 ] (290.85,176.02) .. controls (290.74,170.61) and (295.03,166.14) .. (300.43,166.03) .. controls (305.83,165.91) and (310.31,170.2) .. (310.42,175.61) .. controls (310.53,181.01) and (306.24,185.48) .. (300.84,185.59) .. controls (295.44,185.71) and (290.97,181.42) .. (290.85,176.02) -- cycle ;
%Shape: Circle [id:dp014513623695553424] 
\draw  [color={rgb, 255:red, 74; green, 144; blue, 226 }  ,draw opacity=1 ][fill={rgb, 255:red, 255; green, 255; blue, 255 }  ,fill opacity=1 ] (389.39,156.17) .. controls (389.28,150.76) and (393.57,146.29) .. (398.97,146.18) .. controls (404.37,146.06) and (408.85,150.35) .. (408.96,155.76) .. controls (409.07,161.16) and (404.78,165.63) .. (399.38,165.74) .. controls (393.98,165.86) and (389.51,161.57) .. (389.39,156.17) -- cycle ;
%Shape: Circle [id:dp17073024403224846] 
\draw  [color={rgb, 255:red, 208; green, 2; blue, 27 }  ,draw opacity=1 ][fill={rgb, 255:red, 255; green, 255; blue, 255 }  ,fill opacity=1 ] (420.91,177.29) .. controls (420.79,171.89) and (425.08,167.42) .. (430.49,167.3) .. controls (435.89,167.19) and (440.36,171.48) .. (440.48,176.88) .. controls (440.59,182.29) and (436.3,186.76) .. (430.9,186.87) .. controls (425.49,186.98) and (421.02,182.7) .. (420.91,177.29) -- cycle ;
%Shape: Circle [id:dp7137518684766175] 
\draw  [color={rgb, 255:red, 74; green, 144; blue, 226 }  ,draw opacity=1 ][fill={rgb, 255:red, 255; green, 255; blue, 255 }  ,fill opacity=1 ] (454.42,156.8) .. controls (454.31,151.4) and (458.6,146.93) .. (464,146.82) .. controls (469.4,146.7) and (473.87,150.99) .. (473.99,156.39) .. controls (474.1,161.8) and (469.81,166.27) .. (464.41,166.38) .. controls (459.01,166.5) and (454.53,162.21) .. (454.42,156.8) -- cycle ;
%Rounded Rect [id:dp5471282050025372] 
\draw   (314.38,184.08) .. controls (318.79,177.18) and (316.76,168) .. (309.86,163.59) -- (275.88,141.89) .. controls (268.97,137.48) and (259.79,139.5) .. (255.38,146.41) -- (255.38,146.41) .. controls (250.97,153.32) and (252.99,162.5) .. (259.9,166.91) -- (293.88,188.6) .. controls (300.79,193.02) and (309.97,190.99) .. (314.38,184.08) -- cycle ;
%Rounded Rect [id:dp6574130908699356] 
\draw   (444.43,185.36) .. controls (448.84,178.45) and (446.82,169.28) .. (439.91,164.86) -- (405.93,143.17) .. controls (399.02,138.76) and (389.85,140.78) .. (385.44,147.69) -- (385.44,147.69) .. controls (381.02,154.6) and (383.05,163.77) .. (389.96,168.18) -- (423.94,189.88) .. controls (430.85,194.29) and (440.02,192.27) .. (444.43,185.36) -- cycle ;

% Text Node
\draw (91,194.4) node [anchor=north west][inner sep=0.75pt]    {$m=1$};
% Text Node
\draw (161,194.4) node [anchor=north west][inner sep=0.75pt]    {$m=2$};
% Text Node
\draw (425,193.4) node [anchor=north west][inner sep=0.75pt]    {$m=N$};
% Text Node
\draw (337,159.4) node [anchor=north west][inner sep=0.75pt]    {$\cdots $};


\end{tikzpicture}

        \subcaption{}
        \label{fig:ssh-edge}
    \end{subfigure}
    \caption{(a)}
\end{figure}

It is easy to notice that with half filling, we can regard a pair of 
ALKT chain, Jordan-Wigner transformation

\bibliographystyle{plain}
\bibliography{topological-band}

\end{document}