\documentclass[hyperref, a4paper]{article}

\usepackage{geometry}
\usepackage{titling}
\usepackage{titlesec}
% No longer needed, since we will use enumitem package
% \usepackage{paralist}
\usepackage{enumitem}
\usepackage{footnote}
\usepackage{enumerate}
\usepackage{amsmath, amssymb, amsthm}
\usepackage{mathtools}
\usepackage{bbm}
\usepackage{cite}
\usepackage{graphicx}
\usepackage{subfigure}
\usepackage{physics}
\usepackage{tensor}
\usepackage{siunitx}
\usepackage[version=4]{mhchem}
\usepackage{tikz}
\usepackage{xcolor}
\usepackage{listings}
\usepackage{autobreak}
\usepackage[ruled, vlined, linesnumbered]{algorithm2e}
\usepackage{nameref,zref-xr}
\zxrsetup{toltxlabel}
\zexternaldocument*[optics-]{../optics/optics}[optics.pdf]
\zexternaldocument*[solid-]{../solid/solid}[solid.pdf]
\usepackage[colorlinks,unicode]{hyperref} % , linkcolor=black, anchorcolor=black, citecolor=black, urlcolor=black, filecolor=black
\usepackage{prettyref}

% Page style
\geometry{left=3.18cm,right=3.18cm,top=2.54cm,bottom=2.54cm}
\titlespacing{\paragraph}{0pt}{1pt}{10pt}[20pt]
\setlength{\droptitle}{-5em}
\preauthor{\vspace{-10pt}\begin{center}}
\postauthor{\par\end{center}}

% More compact lists 
\setlist[itemize]{
    itemindent=17pt, 
    leftmargin=1pt,
    listparindent=\parindent,
    parsep=0pt,
}

% Math operators
\DeclareMathOperator{\timeorder}{\mathcal{T}}
\DeclareMathOperator{\diag}{diag}
\DeclareMathOperator{\legpoly}{P}
\DeclareMathOperator{\primevalue}{P}
\DeclareMathOperator{\sgn}{sgn}
\newcommand*{\ii}{\mathrm{i}}
\newcommand*{\ee}{\mathrm{e}}
\newcommand*{\const}{\mathrm{const}}
\newcommand*{\suchthat}{\quad \text{s.t.} \quad}
\newcommand*{\argmin}{\arg\min}
\newcommand*{\argmax}{\arg\max}
\newcommand*{\normalorder}[1]{: #1 :}
\newcommand*{\pair}[1]{\langle #1 \rangle}
\newcommand*{\fd}[1]{\mathcal{D} #1}
\DeclareMathOperator{\bigO}{\mathcal{O}}

% TikZ setting
\usetikzlibrary{arrows,shapes,positioning}
\usetikzlibrary{arrows.meta}
\usetikzlibrary{decorations.markings}
\tikzstyle arrowstyle=[scale=1]
\tikzstyle directed=[postaction={decorate,decoration={markings,
    mark=at position .5 with {\arrow[arrowstyle]{stealth}}}}]
\tikzstyle ray=[directed, thick]
\tikzstyle dot=[anchor=base,fill,circle,inner sep=1pt]

% Algorithm setting
% Julia-style code
\SetKwIF{If}{ElseIf}{Else}{if}{}{elseif}{else}{end}
\SetKwFor{For}{for}{}{end}
\SetKwFor{While}{while}{}{end}
\SetKwProg{Function}{function}{}{end}
\SetArgSty{textnormal}

\newcommand*{\concept}[1]{{\textbf{#1}}}

% Embedded codes
\lstset{basicstyle=\ttfamily,
  showstringspaces=false,
  commentstyle=\color{gray},
  keywordstyle=\color{blue}
}

\newcommand{\opticsdoc}{\href{../optics/optics}{the optics note}}
\newcommand{\soliddoc}{\href{../solid/solid}{the solid state physics note}}

\newrefformat{fig}{Figure~\ref{#1} on page~\pageref{#1}}
\newrefformat{sec}{Section~\ref{#1}}

\title{Clarification of Some Issues concerning KT Phase Transition}
\author{Jinyuan Wu}

\begin{document}

\maketitle

This is a note about some confusing issues related to the KT phase transition.

It is kind of confusing to call the number of vortices topological \emph{quantum} number, as we are currently dealing with a classical model.
It should be noted, however, that a quantum model may exhibit a KT universality class as well, and in this quantum model the number of vortices now labels different energy eigenstates and is really a quantum number.
In this case, we can now apply the standard procedure of second quantization and recast the theory about vortices into a field theory about the \emph{vortex field} (note that whenever there are particle-like objects labeling the states, we can always rephrase the theory into a quantum many-body theory which is essentially a field theory).
In principle the vortex field and the original field with topologically non-trivial configuration space describe the same degrees of freedom and therefore can be calculated from each other, but obtaining an explicit analytic form may be difficult.

People sometimes would argue that in the classical world fields (or ``waves'') have nothing to do with particles: the connection between them is something to be revealed in quantum field theories.
Now we know this claim is not quite true because a theory of particles can naturally emerges from a \emph{classical} field theory, and a particle, in the language of the field theory, is \emph{not} a peak of the field (which is the case in perturbative quantum field theory), but rather a topological defect, and applying a local field operator on a state does not create a particle.
This is an important lesson.
Actually, a similar idea which reduces particles to topological knots in a certain liquid was once thought to be a promising candidate of a theory of everything in the 19th century.

Another example where fields and particles correspond but not in the standard way as is shown in perturbative quantum field theories can be found in investigation of \concept{fractal quantum phase transitions}, where a field theory is proposed to elucidate the critical behavior of a system containing \concept{fractons}, but the field in the theory does not create or annihilate fractons \cite{zhou2021fractal}.

The topological charges in the KT phase transition are not directly related to the ``topological charges'' in topological orders like quantum spin liquids or fractional quantum Hall effect.
Actually, the term \emph{topological phase transition} as is used in KT phase transition is misleading, 
as in a phase where topological defects appear in the low energy region, i.e. the high temperature 
phase of 2D XY model, an \emph{intrinsic topological order} (anyons, ground state degeneracy, etc.) is unlikely to form. 
This is studied in \cite{wen2019}.

\bibliographystyle{plain}
\bibliography{kt-related} 

\end{document}