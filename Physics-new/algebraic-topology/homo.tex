\documentclass[hyperref, a4paper]{article}

\usepackage{geometry}
\usepackage{float}
\usepackage{titling}
\usepackage{titlesec}
% No longer needed, since we will use enumitem package
% \usepackage{paralist}
\usepackage{enumitem}
\usepackage{footnote}
\usepackage{enumerate}
\usepackage{amsmath, amssymb, amsthm}
\usepackage{mathtools}
\usepackage{bbm}
\usepackage{cite}
\usepackage{graphicx}
\usepackage{subcaption}
\usepackage{physics}
\usepackage{tensor}
\usepackage{siunitx}
\usepackage{marginnote}
\usepackage{booktabs}
\usepackage[version=4]{mhchem}
\usepackage{tikz}
\usepackage{xcolor}
\usepackage{listings}
\usepackage{autobreak}
\usepackage[ruled, vlined, linesnumbered]{algorithm2e}
\usepackage{xr-hyper}
\usepackage[colorlinks,unicode]{hyperref} % , linkcolor=black, anchorcolor=black, citecolor=black, urlcolor=black, filecolor=black
\usepackage{prettyref}

% Page style
\geometry{left=3.18cm,right=3.18cm,top=2.54cm,bottom=2.54cm}
\titlespacing{\paragraph}{0pt}{1pt}{10pt}[20pt]
\setlength{\droptitle}{-5em}
\preauthor{\vspace{-10pt}\begin{center}}
\postauthor{\par\end{center}}

% More compact lists 
\setlist[itemize]{itemindent=17pt, leftmargin=1pt}

% Math operators
\DeclareMathOperator{\timeorder}{T}
\DeclareMathOperator{\diag}{diag}
\DeclareMathOperator{\legpoly}{P}
\DeclareMathOperator{\primevalue}{P}
\DeclareMathOperator{\sgn}{sgn}
\DeclareMathOperator{\im}{im}
\newcommand*{\ii}{\mathrm{i}}
\newcommand*{\ee}{\mathrm{e}}
\newcommand*{\const}{\mathrm{const}}
\newcommand*{\suchthat}{\quad \text{s.t.} \quad}
\newcommand*{\argmin}{\arg\min}
\newcommand*{\argmax}{\arg\max}
\newcommand*{\normalorder}[1]{: #1 :}
\newcommand*{\pair}[1]{\langle #1 \rangle}
\newcommand*{\fd}[1]{\mathcal{D} #1}
\DeclareMathOperator{\bigO}{\mathcal{O}}
\DeclareMathOperator{\object}{Ob}
\DeclareMathOperator{\morphism}{Hom}

% TikZ setting
\usetikzlibrary{arrows,shapes,positioning}
\usetikzlibrary{arrows.meta}
\usetikzlibrary{decorations.markings}
\tikzstyle arrowstyle=[scale=1]
\tikzstyle directed=[postaction={decorate,decoration={markings,
    mark=at position .5 with {\arrow[arrowstyle]{stealth}}}}]
\tikzstyle ray=[directed, thick]
\tikzstyle dot=[anchor=base,fill,circle,inner sep=1pt]

% Algorithm setting
% Julia-style code
\SetKwIF{If}{ElseIf}{Else}{if}{}{elseif}{else}{end}
\SetKwFor{For}{for}{}{end}
\SetKwFor{While}{while}{}{end}
\SetKwProg{Function}{function}{}{end}
\SetArgSty{textnormal}

\newcommand*{\concept}[1]{{\textbf{#1}}}

\newrefformat{fig}{Figure~\ref{#1}}

% Embedded codes
\lstset{basicstyle=\ttfamily,
  showstringspaces=false,
  commentstyle=\color{gray},
  keywordstyle=\color{blue}
}

% Disable unsupported commands in bookmark titles 
\pdfstringdefDisableCommands{%
  \def\\{}%
  \def\texttt#1{<#1>}%
  \def\mathbb#1{#1}%
}
\pdfstringdefDisableCommands{\def\eqref#1{(\ref{#1})}}

\makeatletter
\pdfstringdefDisableCommands{\let\HyPsd@CatcodeWarning\@gobble}
\makeatother

\title{Homology and Homotopy Groups}
\author{Jinyuan Wu}

\begin{document}

\maketitle

This note is mainly based on \cite{nakahara}.

Topological field theories often emerge from condensed matter systems, and the topological invariants they give
are often connected to \concept{homotopy groups}, which, intuitively speaking, classify possible field 
configurations into different topological sectors. 

In practice homotopy groups are hard to calculate. That is why people often seek algebraic objects that 
are more easy to calculated and then connect them to homotopy groups (or other algebraic objects that we
are interested in). This approach is called \concept{algebraic topology}, and one most frequently used 
algebraic object is the \concept{homology group}. They do not have very intuitive meaning, but they are 
easier to deal with.

\section{Some basic facts about Abelian groups}

We use $+$ to denote the group operation of an Abelian group. \marginnote{Sec.~3.1}
The expression $x - y$ is defined as $x^{-1} \circ y$.
The unit is denoted as $0$.
The expression $n x$ where $n \in \mathbb{N}$ and $g \in G$ means $g^n$.

A map between Abelian groups $f: G_1 \to G_2$ is said
to be a \concept{homomorphism} if $f(x + y) = f(x) + f(y)$, i.e. it keeps the multiplication relations. 
An \concept{isomorphism} is a homomorphism that is also a bijection. 

Suppose $H$ is a subgroup of $G$. We have a equivalence relation $x \sim y$ if and only if $x - y \in H$.
The equivalence class of $x$ is denoted as $[x]$, and the set of all equivalence classes in $G$ is denoted 
as $G / H$, which is the \concept{the quotient space}. We can easily find that $G / H$ is also a group 
and the group operation $+$ in $G$ naturally induces the group operation $+$ in $G/H$.
We have 
\begin{equation}
    G / G = \{ [0] \} = \{ [h] \}, \quad h \in H, 
\end{equation}
and 
\begin{equation}
    G / \{0\} = G.
\end{equation}

Some examples of quotient spaces:
\begin{equation}
    \mathbb{Z} / k \mathbb{Z} \simeq \mathbb{Z}_k.
\end{equation}

If $f: G_1 \to G_2$ is a homomorphism, it can be found that $\ker f$ is a subgroup of $G_1$ 
and $\im f$ is a subgroup of $G_2$. This lemma can be proved almost directly by definition.

Now we state the \concept{fundamental theorem of homomorphism}: for a homomorphism $f: G_1 \to G_2$, we have 
\begin{equation}
    G_1 / \ker f \simeq \im f.
\end{equation}
This is Theorem~3.1 in \cite{nakahara}. 

If group elements of an Abelian group $G$ are all in the form of 
\begin{equation}
    n_{1} x_{1}+\cdots+n_{r} x_{r}, \quad n_{i} \in \mathbb{Z}, 1 \leq i \leq r,
\end{equation}
we say that $G$ is \concept{(finitely) generated} with \concept{generators} $x_1, \ldots, x_r$.

If $G$ is finitely generated with $r$ \emph{linear independent} generators we say it is a 
\concept{free Abelian group of rank $r$}. 
It should be noted that not every finitely generated Abelian group is a free Abelian group, 
because if $x_1, x_2, \ldots, x_n$ are not linear independent, 
i.e. there exist a integer sequence $\{n_r\}$ with some non-zero terms such that 
\begin{equation}
    n_1 x_1 + n_2 x_2 + \cdots n_r x_r = 0,
\end{equation}
it is generally \emph{insufficient} to obtain an explicit expression of $x_r$ (assuming that $n_r \neq 0$) 
in terms of $x_1, \ldots, x_{r-1}$, since we \emph{cannot} ``divide'' a group element with an integer.
An example is that $\mathbb{Z}_2$ is a finite generated group (the generator is 1) but is not a free group,
because $\{1\}$ is not a linear independent set since $1 + 1 = 0$.

And now we want to analyze the structure of finitely generated groups and free groups. 
First we analyze the simple case of cyclic groups.
\concept{Cyclic groups}, or in other words groups generated with a single element, are important in group theory.
It can be found using the fundamental theorem of homomorphism that a finite cyclic group is isomorphic to 
some $\mathbb{Z}_N$ while an infinite cyclic group is isomorphic to $\mathbb{Z}$. 
If $G$ is a cyclic group generated by $x$ and is finite, than there is a smallest positive integer $N$ 
such that $Nx=0$. We consider the map $f: \mathbb{Z} \to G$,
\[
    f(n) = nx,
\]
then clearly we have 
\[
    \ker f = \{0, \pm N, \pm 2 N, \ldots\} = N \mathbb{Z},
\]
so we have 
\[
    G = \im f \simeq \mathbb{Z} / \ker f = \mathbb{Z} / N \mathbb{Z} \simeq \mathbb{Z}_N.
\]
If 
For an infinite cyclic Abelian group, we have $\ker f = \{0\}$, so 
\[
    G = \im f \simeq \mathbb{Z} / \ker f = \mathbb{Z}.
\]

It can be seen that a finite cyclic Abelian group is not a free Abelian group because the fact that there exists a positive 
integer $N$ such that $Nx = 0$ means the generator set $\{x\}$ is not linear independent.
On the other hand, an infinite free cyclic group $G \simeq \mathbb{Z}$ is free.
Since we are discussing Abelian groups, we can imagine that an arbitrary finitely generated group $G$ can be 
rephrase as a direct sum of cyclic groups, each of which is generated by one of the generators of $G$.
This is indeed the case, and the fact is Theorem~3.2 in \cite{nakahara}: 
Let $G$ be a finitely generated Abelian group (not necessarily free) with $m$ generators. 
Then $G$ is isomorphic to the direct sum of cyclic groups,
\begin{equation}
    G \simeq \underbrace{\mathbb{Z} \oplus \cdots \oplus \mathbb{Z}}_{r} \oplus \mathbb{Z}_{k_{1}} \oplus \cdots \oplus \mathbb{Z}_{k_{p}},
    \label{eq:fundamental-thm-finitely-generated}
\end{equation}
where $m=r+p$. The number $r$ is called the rank of $G$.
This is called \concept{fundamental theorem of finitely generated Abelian groups}.
We can see that the rank defined in \eqref{eq:fundamental-thm-finitely-generated} is exactly the same as the rank defined for 
free Abelian groups.

\section{Simplexes and simplicial complexes}

We now consider how can a manifold be sliced into small ``cells''.  \marginnote{Sec.~3.2}
What we are going to introduce in this section 
are simplexes and simplicial complexes. The former are ``cells'', and the latter are a certain type of topological 
objects assembled with simplexes. 

The generalized definition for a \concept{$r$-simplex} is 
\begin{equation}
    \left\langle p_{0} p_{1} \cdots p_{r}\right\rangle=\left\{x \in \mathbb{R}^{m} \mid x=\sum_{i=0}^{r} c_{i} p_{i}, c_{i} \geq 0, \sum_{i=0}^{r} c_{i}=1\right\}.
\end{equation}
A 0-simplex is a point $\expval*{p_0}$ or $p_0$ for short. A 1-simplex is a line $\expval*{p_0 p_1}$. 
A 2-simplex $\expval*{p_0 p_1 p_2}$ is a triangle with its interior included 
and a 3-simplex $\expval*{p_0 p_1 p_2 p_3}$ is a solid tetrahedron.
We require an $r$-simplex to be $r$-dimensional, or in other words $p_0, p_1, \ldots, p_r$ are required 
to be geometrically independent.

Let $q$ be an integer such that $0 \leq q \leq r$. If we choose $q+1$ points $p_{i_{0}}, \ldots, p_{i_{q}}$ out of 
$p_{0}, \ldots, p_{r}$, these $q+1$ points define a $q$-simplex 
$\sigma_{q}=$ $\left\langle p_{i_{0}}, \ldots, p_{i_{q}}\right\rangle$, 
which is called a \concept{$\boldsymbol{q}$-face} of $\sigma_{r}$. We write $\sigma_{q} \leq \sigma_{r}$ 
if $\sigma_{q}$ is a face of $\sigma_r$. If $\sigma_{q} \neq \sigma_{r}$, we say $\sigma_{q}$ is a \concept{proper face} of $\sigma_{r}$, 
denoted as $\sigma_{q}<\sigma_{r}$.
The number of $q$-faces in an $r$-simplex is $\left(\begin{array}{c} r+1 \\ q+1\end{array}\right)$. A 0-simplex is defined to have no proper faces.

Let $K$ be a set of finite number of simplexes in $\mathbb{R}^{m}$. If these simplexes are nicely fitted together, $K$ is called a simplicial complex. By 'nicely' we mean:
\begin{itemize}
    \item[(i)] an arbitrary face of a simplex of $K$ belongs to $K$, that is, if $\sigma \in K$ and $\sigma^{\prime} \leq \sigma$ then $\sigma^{\prime} \in K $, and
    \item[(ii)] if $\sigma$ and $\sigma^{\prime}$ are two simplexes of $K$, the intersection $\sigma \cap \sigma^{\prime}$ is either empty or a common face of $\sigma$ and $\sigma^{\prime}$, that is, if $\sigma, \sigma^{\prime} \in K$ then either $\sigma \cap \sigma^{\prime}=\varnothing$ or $\sigma \cap \sigma^{\prime} \leq \sigma$ and $\sigma \cap \sigma^{\prime} \leq \sigma^{\prime}$.
\end{itemize}

A simplicial complex $K$ is a set whose elements are simplexes. 
If each simplex is regarded as a subset of $\mathbb{R}^{m}(m \geq \operatorname{dim} K)$, 
the union of all the simplexes becomes a subset of $\mathbb{R}^{m}$. 
This subset is called the \concept{polyhedron} $|K|$ of a simplicial complex $K $. 
The dimension of $|K|$ as a subset of $\mathbb{R}^{m}$ is the same as that of $K$; $\operatorname{dim}|K|=\operatorname{dim} K$.

Let $X$ be a topological space. If there exists a simplicial complex $K$ and a homeomorphism $f:|K| \rightarrow X, X$ is said to be triangulable and the pair $(K, f)$ is called a triangulation of $X$. Given a topological space $X$, its triangulation is far from unique. We will be concerned with triangulable spaces only.

\begin{figure}
    \centering
    

\tikzset{every picture/.style={line width=0.75pt}} %set default line width to 0.75pt        

\begin{tikzpicture}[x=0.75pt,y=0.75pt,yscale=-0.7,xscale=0.7]
%uncomment if require: \path (0,639); %set diagram left start at 0, and has height of 639

%Shape: Triangle [id:dp3110987434104029] 
\draw  [draw opacity=0][fill={rgb, 255:red, 208; green, 2; blue, 27 }  ,fill opacity=0.49 ] (435.02,431.93) -- (490.02,493.48) -- (343.02,493.48) -- cycle ;
%Straight Lines [id:da23071724630812596] 
\draw    (100,110) -- (168.02,110) ;
\draw [shift={(134.01,110)}, rotate = 180] [fill={rgb, 255:red, 0; green, 0; blue, 0 }  ][line width=0.08]  [draw opacity=0] (12,-3) -- (0,0) -- (12,3) -- cycle    ;
%Straight Lines [id:da8694748377726691] 
\draw    (115,135.39) -- (150.02,135.39) ;
\draw [shift={(152.02,135.39)}, rotate = 180] [fill={rgb, 255:red, 0; green, 0; blue, 0 }  ][line width=0.08]  [draw opacity=0] (12,-3) -- (0,0) -- (12,3) -- cycle    ;
%Straight Lines [id:da9404665890679174] 
\draw    (341,165.42) -- (385.51,86) ;
%Straight Lines [id:da5814469013082211] 
\draw    (385.51,86) -- (430.02,165.42) ;
\draw [shift={(407.76,125.71)}, rotate = 60.73] [fill={rgb, 255:red, 0; green, 0; blue, 0 }  ][line width=0.08]  [draw opacity=0] (12,-3) -- (0,0) -- (12,3) -- cycle    ;
%Straight Lines [id:da9287508175938255] 
\draw    (430.02,165.42) -- (341,165.42) ;
\draw [shift={(385.51,165.42)}, rotate = 180] [fill={rgb, 255:red, 0; green, 0; blue, 0 }  ][line width=0.08]  [draw opacity=0] (12,-3) -- (0,0) -- (12,3) -- cycle    ;
%Shape: Triangle [id:dp2801716619328658] 
\draw  [fill={rgb, 255:red, 255; green, 0; blue, 0 }  ,fill opacity=0.38 ] (531.03,84.42) -- (578.02,167) -- (484,167) -- cycle ;
%Shape: Arc [id:dp41495653998762627] 
\draw  [draw opacity=0] (545.54,128.47) .. controls (546.56,130.41) and (547.31,132.54) .. (547.72,134.83) .. controls (549.7,145.99) and (542.8,156.54) .. (532.31,158.4) .. controls (521.82,160.26) and (511.71,152.73) .. (509.73,141.57) .. controls (507.75,130.42) and (514.65,119.87) .. (525.14,118.01) .. controls (529.72,117.2) and (534.22,118.17) .. (538.01,120.46) -- (528.72,138.2) -- cycle ; \draw   (545.54,128.47) .. controls (546.56,130.41) and (547.31,132.54) .. (547.72,134.83) .. controls (549.7,145.99) and (542.8,156.54) .. (532.31,158.4) .. controls (521.82,160.26) and (511.71,152.73) .. (509.73,141.57) .. controls (507.75,130.42) and (514.65,119.87) .. (525.14,118.01) .. controls (529.72,117.2) and (534.22,118.17) .. (538.01,120.46) ;
%Straight Lines [id:da2465746489402889] 
\draw    (546.56,135.89) -- (545.81,130.45) ;
\draw [shift={(545.54,128.47)}, rotate = 82.18] [fill={rgb, 255:red, 0; green, 0; blue, 0 }  ][line width=0.08]  [draw opacity=0] (12,-3) -- (0,0) -- (12,3) -- cycle    ;
%Straight Lines [id:da9681452079628661] 
\draw    (362,185.35) -- (397.02,185.35) ;
\draw [shift={(399.02,185.35)}, rotate = 180] [fill={rgb, 255:red, 0; green, 0; blue, 0 }  ][line width=0.08]  [draw opacity=0] (12,-3) -- (0,0) -- (12,3) -- cycle    ;
%Straight Lines [id:da8825919820050203] 
\draw    (432,138.35) -- (413.99,106.16) ;
\draw [shift={(413.02,104.42)}, rotate = 60.78] [fill={rgb, 255:red, 0; green, 0; blue, 0 }  ][line width=0.08]  [draw opacity=0] (12,-3) -- (0,0) -- (12,3) -- cycle    ;
%Straight Lines [id:da6095517687938568] 
\draw    (121.02,493.39) -- (213.02,432.39) ;
%Straight Lines [id:da7215645251372842] 
\draw    (213.02,432.39) -- (268.02,493.39) ;
\draw [shift={(240.52,462.89)}, rotate = 47.96] [fill={rgb, 255:red, 0; green, 0; blue, 0 }  ][line width=0.08]  [draw opacity=0] (12,-3) -- (0,0) -- (12,3) -- cycle    ;
%Straight Lines [id:da9389285840705635] 
\draw    (268.02,493.39) -- (121.02,493.39) ;
\draw [shift={(194.52,493.39)}, rotate = 180] [fill={rgb, 255:red, 0; green, 0; blue, 0 }  ][line width=0.08]  [draw opacity=0] (12,-3) -- (0,0) -- (12,3) -- cycle    ;
%Straight Lines [id:da08098785410100162] 
\draw    (185,503.35) -- (220.02,503.35) ;
\draw [shift={(222.02,503.35)}, rotate = 180] [fill={rgb, 255:red, 0; green, 0; blue, 0 }  ][line width=0.08]  [draw opacity=0] (12,-3) -- (0,0) -- (12,3) -- cycle    ;
%Straight Lines [id:da7925741886726259] 
\draw    (239,475.35) -- (220.33,453.89) ;
\draw [shift={(219.02,452.39)}, rotate = 48.98] [fill={rgb, 255:red, 0; green, 0; blue, 0 }  ][line width=0.08]  [draw opacity=0] (12,-3) -- (0,0) -- (12,3) -- cycle    ;
%Straight Lines [id:da18630694174926754] 
\draw    (121.02,493.39) -- (189.02,317.39) ;
%Straight Lines [id:da2109679003597844] 
\draw    (268.02,493.39) -- (189.02,317.39) ;
%Straight Lines [id:da45837960590342774] 
\draw    (213.02,432.39) -- (189.02,317.39) ;
\draw [shift={(201.02,374.89)}, rotate = 78.21] [fill={rgb, 255:red, 0; green, 0; blue, 0 }  ][line width=0.08]  [draw opacity=0] (12,-3) -- (0,0) -- (12,3) -- cycle    ;
%Straight Lines [id:da44905213388577514] 
\draw    (197.02,401.39) -- (191.45,376.34) ;
\draw [shift={(191.02,374.39)}, rotate = 77.47] [fill={rgb, 255:red, 0; green, 0; blue, 0 }  ][line width=0.08]  [draw opacity=0] (12,-3) -- (0,0) -- (12,3) -- cycle    ;
%Straight Lines [id:da8557187797252179] 
\draw    (343.02,492.93) -- (435.02,431.93) ;
%Straight Lines [id:da49311673814324175] 
\draw    (435.02,431.93) -- (490.02,492.93) ;
%Straight Lines [id:da8741219616106455] 
\draw    (490.02,492.93) -- (343.02,492.93) ;
%Straight Lines [id:da27419828241382005] 
\draw    (407,502.9) -- (442.02,502.9) ;
\draw [shift={(444.02,502.9)}, rotate = 180] [fill={rgb, 255:red, 0; green, 0; blue, 0 }  ][line width=0.08]  [draw opacity=0] (12,-3) -- (0,0) -- (12,3) -- cycle    ;
%Straight Lines [id:da4141020718770121] 
\draw    (461,474.9) -- (442.33,453.44) ;
\draw [shift={(441.02,451.93)}, rotate = 48.98] [fill={rgb, 255:red, 0; green, 0; blue, 0 }  ][line width=0.08]  [draw opacity=0] (12,-3) -- (0,0) -- (12,3) -- cycle    ;
%Straight Lines [id:da8046331265053788] 
\draw    (343.02,492.93) -- (411.02,316.93) ;
%Straight Lines [id:da2955378202440655] 
\draw    (490.02,492.93) -- (411.02,316.93) ;
%Straight Lines [id:da35181341891439355] 
\draw    (435.02,431.93) -- (411.02,316.93) ;
%Straight Lines [id:da36255971221648187] 
\draw    (419.02,400.93) -- (413.45,375.88) ;
\draw [shift={(413.02,373.93)}, rotate = 77.47] [fill={rgb, 255:red, 0; green, 0; blue, 0 }  ][line width=0.08]  [draw opacity=0] (12,-3) -- (0,0) -- (12,3) -- cycle    ;
%Shape: Triangle [id:dp9087569317007258] 
\draw  [draw opacity=0][fill={rgb, 255:red, 255; green, 0; blue, 0 }  ,fill opacity=0.22 ] (410.47,316.93) -- (491.02,492.93) -- (343.02,492.93) -- cycle ;

% Text Node
\draw (98,110) node [anchor=east] [inner sep=0.75pt]    {$p_{0}$};
% Text Node
\draw (170.02,110) node [anchor=west] [inner sep=0.75pt]    {$p_{1}$};
% Text Node
\draw (133.51,138.79) node [anchor=north] [inner sep=0.75pt]    {$\mathrm{d}\boldsymbol{r} =\mu _{0}$};
% Text Node
\draw (523,126) node [anchor=north west][inner sep=0.75pt]   [align=left] {+};
% Text Node
\draw (478,171.4) node [anchor=north west][inner sep=0.75pt]    {$\mathrm{d}^{2} S=\frac{1}{2} \mu _{0} \land \mu _{1}$};
% Text Node
\draw (339,168.82) node [anchor=north east] [inner sep=0.75pt]    {$p_{0}$};
% Text Node
\draw (432.02,168.82) node [anchor=north west][inner sep=0.75pt]    {$p_{1}$};
% Text Node
\draw (385.51,82.6) node [anchor=south] [inner sep=0.75pt]    {$p_{2}$};
% Text Node
\draw (380.51,188.75) node [anchor=north] [inner sep=0.75pt]    {$\mu _{0}$};
% Text Node
\draw (438.89,101.75) node [anchor=north] [inner sep=0.75pt]    {$\mu _{1}$};
% Text Node
\draw (128,226) node [anchor=north west][inner sep=0.75pt]   [align=left] {(a)};
% Text Node
\draw (426,226) node [anchor=north west][inner sep=0.75pt]   [align=left] {(b)};
% Text Node
\draw (119.02,496.79) node [anchor=north east] [inner sep=0.75pt]    {$p_{0}$};
% Text Node
\draw (270.02,496.79) node [anchor=north west][inner sep=0.75pt]    {$p_{1}$};
% Text Node
\draw (225.02,429.99) node [anchor=south] [inner sep=0.75pt]    {$p_{2}$};
% Text Node
\draw (203.51,506.75) node [anchor=north] [inner sep=0.75pt]    {$\mu _{0}$};
% Text Node
\draw (219.02,455.79) node [anchor=north] [inner sep=0.75pt]    {$\mu _{1}$};
% Text Node
\draw (170.52,395.49) node [anchor=south west] [inner sep=0.75pt]    {$\mu _{2}$};
% Text Node
\draw (189.02,313.99) node [anchor=south] [inner sep=0.75pt]    {$p_{3}$};
% Text Node
\draw (425.51,506.3) node [anchor=north] [inner sep=0.75pt]    {$\mu _{0}$};
% Text Node
\draw (441.02,455.33) node [anchor=north] [inner sep=0.75pt]    {$\mu _{1}$};
% Text Node
\draw (392.52,395.03) node [anchor=south west] [inner sep=0.75pt]    {$\mu _{2}$};
% Text Node
\draw (291,598) node [anchor=north west][inner sep=0.75pt]   [align=left] {(c)};
% Text Node
\draw (356,536.75) node [anchor=north west][inner sep=0.75pt]    {$\mathrm{d} V=\frac{1}{6} \mu _{0} \land \mu _{1} \land \mu _{2}$};


\end{tikzpicture}

    \caption{Orientations of a 1-simplex, a 2-simplex, and a 3-simplex, 
    and the corresponding differential forms when the simplexes are small enough.}
    \label{fig:1-2-3-simplex}
\end{figure}

Now we assign \emph{orientations} to an $r$-simplex for $r > 1$. A oriented simplex is labeled as $(p_0 p_1 \cdots p_r)$. 
Note that since any oriented manifold has only two orientations, for $r$ points $p_0, p_1, \ldots, p_r$,
we only have two types of oriented $r$-simplex. We can use $(p_0 p_1 \cdots p_r)$ and its even permutation 
to label the ``positive'' orientation, and use an odd permutation of $p_0, p_1, \ldots, p_r$ to label the ``negative''
orientation. If $\sigma_r$ is a ``positively'' oriented $r$-simplex, we use $- \sigma_r$ to denote its negative 
version, and vice versa. 
In this way we have 
\begin{equation}
    \left(p_{i_{0}} p_{i_{1}} \ldots p_{i_{r}}\right)=\operatorname{sgn}(P)\left(p_{0} p_{1} \ldots p_{r}\right), \quad 
    P=\left(\begin{array}{cccc}
        0 & 1 & \ldots & r \\
        i_{0} & i_{1} & \ldots & i_{r}
        \end{array}\right).
    \label{eq:oriented-def}
\end{equation}

We can have some intuitive understanding of the orientation of an $r$-simplex using pictures 
from differential geometry. For an oriented $r$-simplex $(p_0 p_1 \cdots p_r)$ that is small enough, we define 
\[
    \mu_i = p_{i+1} - p_i,
\]
and the corresponding differential form is 
\[
    \frac{1}{r!} \mu_0 \wedge \mu_1 \wedge \cdots \wedge \mu_{r-1}.
\]
Then the definition \eqref{eq:oriented-def} starts to make sense.
A 1-simplex can be viewed as a line from one point to the other point, inducing a line element (see \prettyref{fig:1-2-3-simplex}(a)). 
The direction of the line determines 
the sign of the corresponding differential form, so we have 
We have 
\[
    (p_0 p_1) = - (p_1 p_0),
\]
which is the $r=1$ case of \eqref{eq:oriented-def}. The case of 2-complex (see \prettyref{fig:1-2-3-simplex}(b)) is 
almost the same, where the sign of the area element varies depending on whether the oriented boundary is clockwise or not,
and all clockwise routines share the same sign, i.e.
\[
    (p_1 p_2 p_3) = (p_2 p_3 p_1) = (p_3 p_1 p_2),
\]
and so do anticlockwise orientations, which is the $r=2$ case of \eqref{eq:oriented-def}.
In \prettyref{fig:1-2-3-simplex}(c), the bottom of the tetrahedron has the ``positive orientation'', and since the 
order of vertices is $p_0 p_1 p_2 p_3$, the volume of the tetrahedron is also positive.

We formally define $(p_0) = p_0$.

\section{Chain group, cycle group and boundary group}

Note that we can regard the $-$ operation on simplexes as an \emph{inverse} operation.
We also know that differential forms may be added.
Also, if two simplexes with the same orientation are sticked together, they just form a larger object (though not a simplex)
which shares the orientation.
Therefore, all oriented $r$-simplexes in a simplicial complex $K$ \emph{freely generate} an Abelian group,
which is defined as the \concept{$r$-chain group $C_r(K)$}.
When $r > \dim K$ we define $C_r(K)$ to be the trivial group denoted as 0.

\bibliographystyle{plain}
\bibliography{algebraic-topo} 

\end{document}