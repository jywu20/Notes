\documentclass[hyperref, a4paper]{article}

\usepackage{geometry}
\usepackage{titling}
\usepackage{titlesec}
% No longer needed, since we will use enumitem package
% \usepackage{paralist}
\usepackage{enumitem}
\usepackage{footnote}
\usepackage{marginnote}
\usepackage{enumerate}
\usepackage{amsmath, amssymb, amsthm}
\usepackage{mathtools}
\usepackage{bbm}
\usepackage{cite}
\usepackage{graphicx}
\usepackage{subfigure}
\usepackage{physics}
\usepackage{tensor}
\usepackage{siunitx}
\usepackage[version=4]{mhchem}
\usepackage{tikz}
\usepackage{xcolor}
\usepackage{listings}
\usepackage{autobreak}
\usepackage[ruled, vlined, linesnumbered]{algorithm2e}
\usepackage{nameref,zref-xr}
\zxrsetup{toltxlabel}
\zexternaldocument*[optics-]{../optics/optics}[optics.pdf]
\zexternaldocument*[solid-]{../solid/solid}[solid.pdf]
\zexternaldocument*[jellium-]{../band-metal-insulator/electron-gas}[electron-gas.pdf]
\usepackage[colorlinks,unicode]{hyperref} % , linkcolor=black, anchorcolor=black, citecolor=black, urlcolor=black, filecolor=black
\usepackage[most]{tcolorbox}
\usepackage{prettyref}

% Page style
\geometry{left=3.18cm,right=3.18cm,top=2.54cm,bottom=2.54cm}
\titlespacing{\paragraph}{0pt}{1pt}{10pt}[20pt]
\setlength{\droptitle}{-5em}
\preauthor{\vspace{-10pt}\begin{center}}
\postauthor{\par\end{center}}

% More compact lists 
\setlist[itemize]{
    itemindent=17pt, 
    leftmargin=1pt,
    listparindent=\parindent,
    parsep=0pt,
}

% Math operators
\DeclareMathOperator{\timeorder}{\mathcal{T}}
\DeclareMathOperator{\diag}{diag}
\DeclareMathOperator{\legpoly}{P}
\DeclareMathOperator{\primevalue}{P}
\DeclareMathOperator{\sgn}{sgn}
\newcommand*{\ii}{\mathrm{i}}
\newcommand*{\ee}{\mathrm{e}}
\newcommand*{\const}{\mathrm{const}}
\newcommand*{\suchthat}{\quad \text{s.t.} \quad}
\newcommand*{\argmin}{\arg\min}
\newcommand*{\argmax}{\arg\max}
\newcommand*{\normalorder}[1]{: #1 :}
\newcommand*{\pair}[1]{\langle #1 \rangle}
\newcommand*{\fd}[1]{\mathcal{D} #1}
\DeclareMathOperator{\bigO}{\mathcal{O}}

% TikZ setting
\usetikzlibrary{arrows,shapes,positioning}
\usetikzlibrary{arrows.meta}
\usetikzlibrary{decorations.markings}
\tikzstyle arrowstyle=[scale=1]
\tikzstyle directed=[postaction={decorate,decoration={markings,
    mark=at position .5 with {\arrow[arrowstyle]{stealth}}}}]
\tikzstyle ray=[directed, thick]
\tikzstyle dot=[anchor=base,fill,circle,inner sep=1pt]

% Algorithm setting
% Julia-style code
\SetKwIF{If}{ElseIf}{Else}{if}{}{elseif}{else}{end}
\SetKwFor{For}{for}{}{end}
\SetKwFor{While}{while}{}{end}
\SetKwProg{Function}{function}{}{end}
\SetArgSty{textnormal}

\newcommand*{\concept}[1]{{\textbf{#1}}}

% Embedded codes
\lstset{basicstyle=\ttfamily,
  showstringspaces=false,
  commentstyle=\color{gray},
  keywordstyle=\color{blue}
}

% Reference formatting
\newrefformat{fig}{Figure~\ref{#1} on page~\pageref{#1}}

% Color boxes
\tcbuselibrary{skins, breakable, theorems}
\newtcbtheorem[number within=section]{warning}{Warning}%
  {colback=orange!5,colframe=orange!65,fonttitle=\bfseries, breakable}{warn}
\newtcbtheorem[number within=section]{note}{Note}%
  {colback=green!5,colframe=green!65,fonttitle=\bfseries, breakable}{note}

\newcommand{\jellium}{\href{../band-metal-insulator/electron-gas.pdf}{this note}}

\title{Bosonization by Prof. Yue Yu}
\author{Jinyuan Wu}

\begin{document}

\maketitle

\section{An overview of CFTs}

This is a note about a lecture given by Prof. Yue Yu as a part of his course on CFT.

To not start from the abstract mathematics, we start from bosonization of (1+1)-dimensional 
interacting fermion systems. This results in a CFT with $c=1$ in (1+1)-dimensions.
We are especially interested in the (1+1)-dimensional case, because in higher dimensional systems, the conformal 
symmetry is a global symmetry, something like $\mathrm{SL}(2, \mathbb{C})$, while in the 
(1+1)-dimensional case, the conformal symmetry is a \emph{local} symmetry, and therefore it is actually a \emph{gauge symmetry}.

CFTs often describe critical points of 2-dimensional classical models and (1+1)-dimensional quantum models.
From CFT we can calculate correlation functions and critical exponents \emph{exactly}.
A classification of a certain group of CFTs is also a classification of the corresponding physical models.

Though $c=1$ CFTs are directly related to the Luttinger liquid, they contain infinite primary fields and therefore is too ``large'' to be handled.
The case when $c < 1$ is even worse: the number of primary fields can be infinite and the primary fields can be not well-defined.
Fortunately, we still have \concept{minimal models} where we have a finite number of primary fields, and the whole Hilbert space is well-defined.
(On the other hand, the Hilbert space of a generic gauge theory may contain states with negative norms and therefore is not well-defined before gauge fixing.)
Some of minimal models corresponds to well-known exactly solvable statistical models, e.g. Ising model, tricritical Ising model, Potts model.
The primary fields in a minimal model generally obeys anyon statistics, Abelian and non-Abelian.

CFTs may also describe systems with a finite characteristic length, if we place them on a finite Riemann surface (for example, a torus). 
This reduces the conformal symmetry to a smaller symmetry.
For example, on a torus, the conformal invariance reduces into modular invariance.
A finite size system can also be described by a CFT with finite size scaling, and the CFT can predict boundary effects.

For $c > 1$, we have the \concept{Wess-Zumino-Witten (WZW) models} with Lie group symmetry.
The most interesting WZW models are those with simple Lie group symmetry.
% TODO

CFTs with different central charges are connected by RG flows, which is shown by the \concept{c-theorem}. CFTs with definite central charges, therefore, are fixed points.

In higher dimensions, the naive generalization of CFTs is not interesting, but we also have some results.
There are higher dimensional generalization of the c-theorem.
AdS/CFT is a correspondence between gravity and CFTs and was once thought to be hopeful to solve some strongly correlated problems.
Conformal bootstrap can also be generalized to higher dimensions, though we are less confident about whether we take the constraints correctly there. 

Probably the most intriguing application of CFTs in condensed matter physics is about topological quantum computation,
where non-Abelian anyons are thought as qubits, and the braiding of anyons corresponds to the unitary transformations.
We can build quantum gates with braiding.
The simplest example is Ising non-Abelian anyon, which is the bound state of the spin and Majorana fermion.
Braidings of Ising non-Abelian anyons cannot represent all unitary transformations.
The Fibonacci anyon can achieve universal topological quantum computing.

Note that fermions can also be non-Abelian. For example, a charge fermion can be delocalized into two types of Majorana fermions.
% TODO: Kitaev superconducting chain
Majorana fermions can also achieve universal topological quantum computing.
Note, however, they can not be generated within a CFT. The theory describing them is a unitary tensor category, a generalization of CFTs.

\section{Bosonization of free fields}

This section is mainly based on \cite{onedfermi1995,PhysRevLett.75.890}. 
\cite{onedfermi1995} is an introduction. \cite{PhysRevLett.75.890} is a work by Prof. Yu about some details.

To make things easier, we redefine the unit of energy, and the Hamiltonian of a one-dimensional spinless fermion gas is 
\begin{equation}
    H = \int \dd{x} \abs*{\grad{\psi}}^2 + \frac{1}{2} \int \dd{x} \dd{y} V(x - y) \rho(x) \rho(y). 
    \label{eq:1d-ham}
\end{equation}
The free theory case is 
\begin{equation}
    H_\text{eff} = H - \mu N = \sum_k \xi_k c^\dagger_k c_k, \quad \xi_k = \begin{cases}
        (k^2 - k_\text{F}^2) / \lambda, \quad &\abs*{k} < k_\text{F}, \\
        k^2 - k_\text{F}^2, \quad &\abs*{k} > k_\text{F}.
    \end{cases}
    \label{eq:free-spectrum}
\end{equation}
Here we take a more generalized form. For \eqref{eq:1d-ham}, of course $\lambda = 1$. 
The theory with $\lambda = 3$ is actually about edge states of $\nu = 1 /3$ FQHE and is related to anyon statistics in the sense of exclusion. % ??? TODO

We only consider \eqref{eq:free-spectrum} near the two Fermi points, and the spectrum is now 
\begin{equation}
    \xi_{k, \pm} = \begin{cases}
        \pm v_\text{F} (k \mp k_\text{F}), \quad \abs*{k} > k_\text{F}, \\
        \pm v_\text{F} (k \mp k_\text{F}) / \lambda, \quad \abs*{k} < k_\text{F},
    \end{cases}
\end{equation}
where $\pm$ in the subscript of $\xi_{k, \pm}$ distinguishes the mode around $\pm k_\text{F}$.
Since we are only interested in the density fluctuation, around $k = k_\text{F}$, we define 
\begin{equation}
    \rho^+_q = \sum_{k > k_\text{F}} \normalorder{c^\dagger_{k+q} c_k} 
    + \sum_{k < k_\text{F} - \lambda q} \normalorder{c^\dagger_{k + \lambda q} c_k} 
    + \sum_{k_\text{F} - \lambda q < k < k_\text{F} } \normalorder{c^\dagger_{\frac{k - k_\text{F}}{\lambda}  + k_\text{F} +q} c_k} 
\end{equation}
for $0 <q \ll k_\text{F}$. When $\lambda = 1$, the definition goes back to the ordinary 
\begin{equation}
    \rho^+_q = \sum_k \normalorder{c^\dagger_{k+q} c_k}.
\end{equation}
Similarly, the density fluctuation around $k = - k_\text{F}$ is defined as 
\begin{equation}
    \rho^-_q = \sum_{k < -k_\text{F}} \normalorder{c^\dagger_{k-q} c_k} 
    + \sum_{k > - k_\text{F} + \lambda q} \normalorder{c^\dagger_{k - \lambda q} c_k} 
    + \sum_{- k_\text{F} + \lambda q > k > - k_\text{F} } \normalorder{c^\dagger_{\frac{k + k_\text{F}}{\lambda} - k_\text{F} - q} c_k} .
\end{equation}

Now we make the \concept{Tomonaga approximaton}: we replace $\comm*{\rho^\pm_q}{\rho^\pm_{q'}}$ by its ground state 
expectation value. The procedure we are taking here is similar to how we simply \eqref{jellium-eq:bosonic-eq} in \jellium. % TODO: hubbard-stratonovich
This gives 
\begin{equation}
    \comm*{\rho^\pm_q}{\rho^\pm_{q'}} \approx \expval*{\comm*{\rho^\pm_q}{\rho^\pm_{q'}}}{0} = \sum_{k_\text{F} - \lambda q < k < k_\text{F}} \expval*{c_{k+\lambda q} c^\dagger_{k + \lambda q'}}{0} = \frac{L}{2\pi} q \delta_{qq'}.
\end{equation}
Since there is no interaction between $c_+$ modes and $c_-$ modes, the Hamiltonian in \eqref{eq:free-spectrum} is  
\begin{equation}
    \comm*{H}{\rho_q^\pm} = \comm*{H_\pm}{\rho_q^\pm} \approx \pm v_\text{F} q \rho^\pm_q.
    \label{eq:free-boson-comm-ham}
\end{equation}
under the Tomonaga approximation. \eqref{eq:free-boson-comm-ham} is the same as the commutation relation in a free phonon system.
Defining 
\begin{equation}
    b_{q} = \sqrt{\frac{2\pi}{q}} \rho^+_q, \quad \tilde{b}_q = \sqrt{\frac{2\pi}{q}} (\rho_q^-)^\dagger,
\end{equation}
we find that the Hamiltonian
\begin{equation}
    H_\text{B} = v_\text{s} \sum_{q > 0} \left( q(b^\dagger_q b_q + \tilde{b}^\dagger_q \tilde{b}_q) + \frac{\pi}{2L} (\lambda M^2 + \frac{1}{\lambda} J^2) \right)
    \label{eq:luttinger-bosonized}
\end{equation}
has the same commutation relation with density fluctuations with $H_\text{eff}$. 
If only density fluctuations are important, \eqref{eq:luttinger-bosonized} is a good effective theory of the free fermion theory, which is usually called the \concept{Luttinger liquid theory}.

Note that there is a constant term in \eqref{eq:luttinger-bosonized} which does not affect the dynamics. 
It comes from the contribution of zero modes.
We will skip the discussion about these terms currently, but actually it is these terms that \emph{classifies} Luttinger liquids.

\section{Interaction between density fluctuations}

Now we take $\lambda = 1$ and add back the electron spins, which is merely to add an additional index to the density modes.
Consider the Hamiltonian (often called \concept{Tomonaga-Luttinger model}) introduced in \cite{onedfermi1995}:
\begin{equation}
    \begin{aligned}
        & H = H_0 + H_2 + H_4,  \\
        & H_2 = \frac{1}{L} \sum_{p, s, s'} \left( g_{2 \parallel} (p) \delta_{s s'} + g_{2 \bot }(p) \delta_{s, -s'} \right) \rho_{p + s} \rho_{-p, -, s'}, \\
        & H_4 = \frac{1}{2L} \sum_{p, r, s, s'} \left( g_{4 \parallel} (p) \delta_{s s'} + g_{4 \bot }(p) \delta_{s, -s'} \right) \normalorder{\rho_{p r s} \rho_{-p, r, s'}},
    \end{aligned}
    \label{eq:tomonaga-luttinger-model}
\end{equation} 
which only includes density-density interaction in the momentum space (or in other words, ``forward scattering'').
After bosonization of $H_0$, we find that the Hamiltonian is a bilinear of density fluctuations.
Therefore, after bosonization, the Tomonaga-Luttinger model is exactly solvable.
Despite being non-perturbative, \eqref{eq:tomonaga-luttinger-model} can be understood precisely.

The zero mode term is now 
\begin{equation}
    \frac{\pi}{2L} \sum_s (v_N (\underbrace{N_{+ s} + N_{-s}}_{N_s})^2  + v_{J} (\underbrace{N_{+s} - N_{-s}}_{J_s})^2 ),
\end{equation}
where $N_{rs} = \rho_{0, r, s}$. When there is no interaction, we have $v_\text{s} = v_N = v_J = v_\text{F}$, but after 
introducing forward scattering, generally speaking there will be charge-spin separation.

Now we can do diagonalization. We define charge variable 
\begin{equation}
    \rho_r(p) = \frac{1}{\sqrt{2}} (\rho_{p, r \uparrow})
\end{equation}

\bibliographystyle{plain}
\bibliography{../bosonization/bosonization}

\end{document}