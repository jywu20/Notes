\documentclass[hyperref, a4paper]{article}

\usepackage{geometry}
\usepackage{float}
\usepackage{titling}
\usepackage{titlesec}
% No longer needed, since we will use enumitem package
% \usepackage{paralist}
\usepackage{enumitem}
\usepackage{footnote}
\usepackage{enumerate}
\usepackage{amsmath, amssymb, amsthm}
\usepackage{mathtools}
\usepackage{bbm}
\usepackage{cite}
\usepackage{graphicx}
\usepackage{subcaption}
\usepackage{physics}
\usepackage{tensor}
\usepackage{siunitx}
\usepackage{booktabs}
\usepackage[version=4]{mhchem}
\usepackage{tikz}
\usepackage{xcolor}
\usepackage{listings}
\usepackage{autobreak}
\usepackage[ruled, vlined, linesnumbered]{algorithm2e}
\usepackage{nameref,zref-xr}
\zxrsetup{toltxlabel}
\zexternaldocument*[prev-]{./lecture-10-20}[lecture-10-20.pdf]
\zexternaldocument*[solid-]{../solid/solid}[solid.pdf]
\usepackage[colorlinks,unicode]{hyperref} % , linkcolor=black, anchorcolor=black, citecolor=black, urlcolor=black, filecolor=black
\usepackage{prettyref}

% Page style
\geometry{left=3.18cm,right=3.18cm,top=2.54cm,bottom=2.54cm}
\titlespacing{\paragraph}{0pt}{1pt}{10pt}[20pt]
\setlength{\droptitle}{-5em}
\preauthor{\vspace{-10pt}\begin{center}}
\postauthor{\par\end{center}}

% More compact lists 
\setlist[itemize]{itemindent=17pt, leftmargin=1pt}

% Math operators
\DeclareMathOperator{\timeorder}{\mathcal{T}}
\DeclareMathOperator{\diag}{diag}
\DeclareMathOperator{\legpoly}{P}
\DeclareMathOperator{\primevalue}{P}
\DeclareMathOperator{\sgn}{sgn}
\newcommand*{\ii}{\mathrm{i}}
\newcommand*{\ee}{\mathrm{e}}
\newcommand*{\const}{\mathrm{const}}
\newcommand*{\suchthat}{\quad \text{s.t.} \quad}
\newcommand*{\argmin}{\arg\min}
\newcommand*{\argmax}{\arg\max}
\newcommand*{\normalorder}[1]{: #1 :}
\newcommand*{\pair}[1]{\langle #1 \rangle}
\newcommand*{\fd}[1]{\mathcal{D} #1}
\DeclareMathOperator{\bigO}{\mathcal{O}}
\DeclareMathOperator{\object}{Ob}
\DeclareMathOperator{\morphism}{Hom}

% TikZ setting
\usetikzlibrary{arrows,shapes,positioning}
\usetikzlibrary{arrows.meta}
\usetikzlibrary{decorations.markings}
\tikzstyle arrowstyle=[scale=1]
\tikzstyle directed=[postaction={decorate,decoration={markings,
    mark=at position .5 with {\arrow[arrowstyle]{stealth}}}}]
\tikzstyle ray=[directed, thick]
\tikzstyle dot=[anchor=base,fill,circle,inner sep=1pt]

% Algorithm setting
% Julia-style code
\SetKwIF{If}{ElseIf}{Else}{if}{}{elseif}{else}{end}
\SetKwFor{For}{for}{}{end}
\SetKwFor{While}{while}{}{end}
\SetKwProg{Function}{function}{}{end}
\SetArgSty{textnormal}

\newcommand*{\concept}[1]{{\textbf{#1}}}

\newrefformat{fig}{Figure~\ref{#1}}

\newcommand{\soliddoc}{\href{../solid/solid}{the solid state physics note}}

% Embedded codes
\lstset{basicstyle=\ttfamily,
  showstringspaces=false,
  commentstyle=\color{gray},
  keywordstyle=\color{blue}
}

\title{Poor man's linear response theory by Prof. Kun Din}
\author{Jinyuan Wu}

\begin{document}

\maketitle

In \href{./lecture-10-20}{the previous lecture} we discussed the K-K relation and its generalization in metals, the analytic structure of electrodynamic response functions, and sum rules.
We also introduced a poor man's linear response theory.

\section{Poor man's linear response theory}

The ground state is determined by
\begin{equation}
    \eval{\fdv{G}{n}}_{0} + e \phi_0 = \mu, \quad \laplacian \phi_0 = \frac{e}{\epsilon_0} (n_\text{lattice} - n_0).
\end{equation}
The excitation is determined by 
\begin{equation}
    \pdv{\rho_1}{t} + \div{\vb*{j}_1} = 0, \quad \pdv{\vb*{j}_1}{t} = - \frac{e n_0}{m} \grad \left( \fdv{G}{n} \right)_1 + \frac{e^2 n_0}{m} \vb*{E}_1.
\end{equation}
Note that since the system can be charged (for example in the case of capacitors), maybe 
\begin{equation}
    \int \dd[3]{\vb*{r}} (n_\text{lattice} - n_0) \neq 0.
\end{equation}

By the Fourier transformation of $t$ we have 
\begin{equation}
    (- \ii \omega + \gamma) \vb*{j}_1 = - \frac{e n_0}{m} \grad \left( \fdv{G}{n} \right)_1 + \frac{e^2 n_0}{m} \vb*{E}(\omega), \quad \div{\vb*{j}_1} - \ii \omega \rho_1 = 0.
\end{equation}
Adding the wave function 
\begin{equation}
    \curl(\curl{\vb*{E}_1(\omega)}) - \left( \frac{\omega}{c} \right)^2 \vb*{E}_1(\omega) = \ii \omega \mu_0 \vb*{j}_1(\omega),
\end{equation}
the electrodynamic behavior can be found.

A specific case can be immediately seen from these equations. If the $G[n]$ term is not important, we already find that 
\begin{equation}
    \epsilon_\text{r}(\infty) = 1 - \frac{\omega_0^2}{\omega^2},
\end{equation}
which is the prediction of Drude model.
Let 
\begin{equation}
    G[n] = \int \dd[3]{\vb*{r}} g(n, \grad{n}, \ldots) .
\end{equation}
The leading term of $g$ is the famous \concept{Thomas-Fermi energy} 
\begin{equation}
    g(n, \grad{n}, \ldots) = \frac{3 \hbar^2}{10 m_\text{e}} (3\pi)^{2/3} n^{5/3} + \cdots,
\end{equation}
and we can add the \concept{von-Weizsäcker energy} and \concept{exchange-correlation terms} (the exact meaning of which can be found in Section~\ref{solid-sec:kohn-sham-no-spin} in \soliddoc), for example 
\begin{equation}
    g(n, \grad{n}, \ldots) = \frac{3 \hbar^2}{10 m_\text{e}} (3\pi)^{2/3} n^{5/3} + \underbrace{\frac{\lambda \hbar^2}{8 m_\text{e}} \frac{\abs*{\grad{n}}^2}{n}}_{E_\text{vW}} + E_\text{XC}.
    \label{eq:simple-functional}
\end{equation}
The exact expression of $g$ does not matter, since if the wave function does not appear explicitly, phenomena that are purely quantum mechanical are hard to repeat in our model.
For example, if somehow the \emph{phase} of the electron wave function influences the electric field, then the prediction of our fluid dynamic approach is \emph{qualitatively wrong}.

\section{Thomas-Fermi screening}

We consider a static example. The problem is 
\begin{equation}
    \div{\vb*{D}} = \rho_\text{ext}, \quad \laplacian \phi_1 = - \frac{\rho_\text{ext} + \rho_1}{\epsilon_0}, \quad \epsilon_0 \div{\vb*{E}} = \underbrace{\rho_\text{ext} + \rho_1}_{\rho_\text{total}}. 
\end{equation}
Switching to the momentum space, we have 
\[
    \ii \vb*{k} \cdot \vb*{\vb*{D}}(\vb*{k}) = \epsilon_0 \abs*{\vb*{k}}^2 \phi_\text{ext}(\vb*{k}) = \rho_\text{ext}(\vb*{k}), 
\]
and by definition we have 
\begin{equation}
    \epsilon_\text{r}(\vb*{k}, \omega = 0) = \frac{\phi_\text{ext}(\vb*{k})}{\phi_1(\vb*{k})} = \frac{\rho_\text{ext}}{\rho_\text{total}} = 1 - \frac{\rho_1(\vb*{k})}{\rho_\text{total}(\vb*{k})}.
\end{equation}
If we just consider the Thomas-Fermi term, we have 
\[
    \fdv{G}{n} = \pdv{g}{n} - \div{\pdv{g}{\grad{n}} } = \frac{\hbar^2}{2m_\text{e}} (3\pi^2 n)^{2/3} ,
\]
and therefore 
\[
    \left(\fdv{G}{n}\right)_1 = \frac{\hbar^2}{2m_\text{e}} (3\pi^2 )^{2/3} ((n_1 + n_0)^{3/2} - n_0^{3/2}) \approx \frac{\hbar^2}{m_\text{e}} (3\pi^2)^{2/3} \frac{n_1}{n_0} n_0^{2/3} = \frac{2 \epsilon_\text{F}^0}{3 n_0} n_1,
\]
where we define
\begin{equation}
    \epsilon_\text{F}^1 \coloneqq \frac{\hbar^2}{2m_\text{e}} (3\pi^2 n_0)^{2/3}.
\end{equation}
So finally we have 
\begin{equation}
    \epsilon(\vb*{k}, \omega=0) = 1 + \frac{k_\text{s}^2}{\abs*{\vb*{k}}^2},
\end{equation}
where 
\begin{equation}
    k_\text{s} = \frac{3 e^2 n_0}{ 2 \epsilon_\text{F}^0 \epsilon_0} 
    \label{eq:thomas-fermi-approx}
\end{equation}
is called the \concept{Thomas-Fermi wave number}.

Here is the magnitude of $k_\text{s}$: We have 
\begin{equation}
    \frac{k_\text{s}}{k_\text{F}} = \sqrt{r_\text{s} \left(\frac{16}{3\pi^2}\right)^{2/3}} = 0.814 \sqrt{r_\text{s}},
\end{equation}
where $r_\text{s}$ is usually between $2$ and $6$.
Therefore $k_\text{S}$ is of the same magnitude of $k_\text{F}$, and is quite a large momentum compared to ordinary electromagnetic waves.

Switching back to the real space, we have 
\[
    \begin{aligned}
        \phi_1(\vb*{r}) &= \frac{q}{\epsilon_0 (2\pi)^3} \int \dd[3]{\vb*{k}} \frac{\ee^{\ii \vb*{k} \cdot \vb*{r}}}{\abs*{\vb*{k}}^2 + k_\text{S}^2} \\
        &= \frac{q}{\epsilon_0 (2\pi)^2} \frac{1}{2 \ii r} \int_{-\infty}^\infty \dd{k} \frac{k (\ee^{\ii k r} - \ee^{- \ii k r})}{(k + \ii k_\text{s}) (k - \ii k_\text{s})} ,
    \end{aligned}
\]
Note that there is no causality in the spacial dimensions, so there may be poles on the upper plane.
Competing the contour integrals we have 
\begin{equation}
    \phi_1(\vb*{r}) = \frac{q}{4\pi \epsilon_0 r} \ee^{- k_\text{s} r},
\end{equation}
which is called the \concept{Yukawa potential}. Usually the screening length is $1 / k_\text{s} \sim \SI{1}{\angstrom}$, so the electric field in a metal is screened almost perfectly.

\section{Dynamic screening}

Now we consider a scenario where both the charge and the current are perturbed.
We have 
\begin{equation}
    (- \ii \omega + \gamma) \vb*{j}_1 + \frac{v_\text{p}^2}{3} \grad{\rho_1} = \epsilon_0 \omega_\text{p}^2 \vb*{E}_1(\omega),
\end{equation}
or to eliminate the dependence on $\rho_1$, we take the time derivative and obtain
\begin{equation}
    \omega(\omega + \ii \gamma) \vb*{j}_1 + \beta^2 \grad(\div{\vb*{j}_1}) = \ii \omega \epsilon_0 \omega_\text{p}^2 \vb*{E}_1(\omega).
\end{equation}
Now we perform the Helmholtz decomposition. Let 
\begin{equation}
    \vb*{E}_1 = \vb*{E}_\text{l} + \vb*{E}_\text{t}, \quad \vb*{j}_1 = \vb*{j}_\text{l} + \vb*{j}_\text{t},
\end{equation}
where the subscript l means ``longitude'' and t means ``transverse''.
The equations for the longitude modes are 
\begin{equation}
    \omega(\omega + \ii \gamma) \vb*{j}_\text{l} - \beta^2 \vb*{k} (\vb*{k} \cdot \vb*{j}_\text{l}) = \ii \omega \epsilon_0 \omega_\text{p}^2 \vb*{E}_\text{l},
\end{equation}
while the equations for the transverse modes are 
\begin{equation}
    \omega(\omega + \ii \gamma) \vb*{j}_\text{t} = \ii \omega \epsilon_0 \omega_\text{p}^2 \vb*{E}_\text{t}, \quad \vb*{k} \times (\vb*{k} \times \vb*{E}_\text{t}) + \left(\frac{\omega}{c}\right)^2 \vb*{E}_\text{t} = - \ii \omega \epsilon_0 \omega_\text{p}^2 \vb*{E}_\text{t}(\omega).
\end{equation}
By the routine mentioned in the last section we have 
\begin{equation}
    \epsilon_\text{l}(\vb*{k}, \omega) = 1 - \frac{\omega_\text{p}^2}{\omega(\omega + \ii \gamma) - \beta^2 \abs*{\vb*{k}}^2},
\end{equation}
and 
\begin{equation}
    \epsilon_\text{t}(\vb*{k}, \omega) = 1 - \frac{\omega_\text{p}^2}{\omega(\omega + \ii \gamma)}.
\end{equation}

The pole in $\epsilon_\text{l}$ is given by 
\begin{equation}
    \abs*{\vb*{k}}^2 = \left(\frac{\omega}{c}\right)^2 \left( 1 - \frac{\omega_\text{p}^2}{\omega(\omega + \ii \gamma)} \right),
\end{equation}
and when there is no dissipation we get 
\begin{equation}
    \omega^2 = \omega_\text{p}^2 + \vb*{k}^2 c^2,
\end{equation}
again the prediction of Drude model. 

When $k / k_\text{s}$ is small, there is almost no propagating modes in metals: the longitude mode is gapped, and the frequency of the transverse mode is just too low to be noticed.
Therefore, in the $k / k_\text{s} \ll 1$ limit, a metal essentially blocks any propagating modes, and screens any external sources.
When $k / k_\text{s} \gg 1$, the frequency of the longitude mode is just too high to be relevant, and now we have propagating transverse modes in the metal.
In this limit the metal looks ``transparent''.

\section{Boundaries and other inhomogeneous systems}

The Thomas-Fermi term is definitely not enough for inhomogeneous systems.
An extreme case is a boundary, where on one side we have no electrons at all and therefore the electron density has a sharp decay with a big $\grad{n}$, which must be included into the energy functional $G[n]$.
Actually in the \SI{10}{nm} to \SI{1}{nm} region, violations of continuum electromagnetism most frequently come from boundaries.
Violations of the continuum theory appear usually in the $< \SI{1}{nm}$ region in bulks.

With an energy functional $G[n]$ that is accurate enough, it can be found that the electron density has a peak near the boundary, and the electric field changes sharply.
In the macroscopic electrodynamics, we do a coarse-graining of all fields around the boundary and approximate all high wave number components of the fields with $\delta$-functions. 
\emph{Boundary conditions} are introduced to characterize the boundary, which is not necessary in the microscopic theory.

\section{A summary of continuum electromagnetism}

\begin{itemize}
    \item Microscopic and macroscopic Maxwell equations; continuum of approximation.
    \begin{itemize}
        \item Spacial averaging; test functions.
        \item Long wave length condition.
    \end{itemize}
    \item Constitutive relations.
    \begin{itemize}
        \item Macroscopic description of materials.
        \item Weak field approximation: only linear susceptibility.
        \item Low speed movement.
        \item Multipole expansion.
    \end{itemize}
    \item Response functions in the frequency domain.
    \begin{itemize}
        \item The K-K relations.
        \item Causality.
    \end{itemize}
    \item Poor man's linear response theory.
\end{itemize}

\end{document}