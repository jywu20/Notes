\documentclass[hyperref, a4paper]{article}

\usepackage{geometry}
\usepackage{titling}
\usepackage{titlesec}
% No longer needed, since we will use enumitem package
% \usepackage{paralist}
\usepackage{enumitem}
\usepackage{footnote}
\usepackage{enumerate}
\usepackage{amsmath, amssymb, amsthm}
\usepackage{mathtools}
\usepackage{bbm}
\usepackage{cite}
\usepackage{graphicx}
\usepackage{subfigure}
\usepackage{physics}
\usepackage{tensor}
\usepackage{siunitx}
\usepackage[version=4]{mhchem}
\usepackage{tikz}
\usepackage{xcolor}
\usepackage{listings}
\usepackage{autobreak}
\usepackage[ruled, vlined, linesnumbered]{algorithm2e}
\usepackage{nameref,zref-xr}
\zxrsetup{toltxlabel}
\zexternaldocument*[optics-]{../optics/optics}[optics.pdf]
\zexternaldocument*[solid-]{../solid/solid}[solid.pdf]
\usepackage[colorlinks,unicode]{hyperref} % , linkcolor=black, anchorcolor=black, citecolor=black, urlcolor=black, filecolor=black
\usepackage[most]{tcolorbox}
\usepackage{prettyref}

% Page style
\geometry{left=3.18cm,right=3.18cm,top=2.54cm,bottom=2.54cm}
\titlespacing{\paragraph}{0pt}{1pt}{10pt}[20pt]
\setlength{\droptitle}{-5em}
\preauthor{\vspace{-10pt}\begin{center}}
\postauthor{\par\end{center}}

% More compact lists 
%\setlist[itemize]{
%    itemindent=17pt, 
%    leftmargin=1pt,
%    listparindent=\parindent,
%    parsep=0pt,
%}

% Math operators
\DeclareMathOperator{\timeorder}{\mathcal{T}}
\DeclareMathOperator{\diag}{diag}
\DeclareMathOperator{\legpoly}{P}
\DeclareMathOperator{\primevalue}{P}
\DeclareMathOperator{\sgn}{sgn}
\newcommand*{\ii}{\mathrm{i}}
\newcommand*{\ee}{\mathrm{e}}
\newcommand*{\const}{\mathrm{const}}
\newcommand*{\suchthat}{\quad \text{s.t.} \quad}
\newcommand*{\argmin}{\arg\min}
\newcommand*{\argmax}{\arg\max}
\newcommand*{\normalorder}[1]{: #1 :}
\newcommand*{\pair}[1]{\langle #1 \rangle}
\newcommand*{\fd}[1]{\mathcal{D} #1}
\DeclareMathOperator{\bigO}{\mathcal{O}}

% TikZ setting
\usetikzlibrary{arrows,shapes,positioning}
\usetikzlibrary{arrows.meta}
\usetikzlibrary{decorations.markings}
\tikzstyle arrowstyle=[scale=1]
\tikzstyle directed=[postaction={decorate,decoration={markings,
    mark=at position .5 with {\arrow[arrowstyle]{stealth}}}}]
\tikzstyle ray=[directed, thick]
\tikzstyle dot=[anchor=base,fill,circle,inner sep=1pt]

% Algorithm setting
% Julia-style code
\SetKwIF{If}{ElseIf}{Else}{if}{}{elseif}{else}{end}
\SetKwFor{For}{for}{}{end}
\SetKwFor{While}{while}{}{end}
\SetKwProg{Function}{function}{}{end}
\SetArgSty{textnormal}

\newcommand*{\concept}[1]{{\textbf{#1}}}

% Support for tensor double arrows.
\renewcommand{\tensor}[1]{ \stackrel{\leftrightarrow}{\vb*{#1}}}

% Embedded codes
\lstset{basicstyle=\ttfamily,
  showstringspaces=false,
  commentstyle=\color{gray},
  keywordstyle=\color{blue}
}

% Reference formatting
\newrefformat{fig}{Figure~\ref{#1} on page~\pageref{#1}}

% Color boxes
\tcbuselibrary{skins, breakable, theorems}
\newtcbtheorem[number within=section]{warning}{Warning}%
  {colback=orange!5,colframe=orange!65,fonttitle=\bfseries, breakable}{warn}
\newtcbtheorem[number within=section]{note}{Note}%
  {colback=green!5,colframe=green!65,fonttitle=\bfseries, breakable}{note}

\title{Diffraction and Scattering in Electrodynamics by Prof. Kun Din}
\author{Jinyuan Wu}
\date{December 8, 2021}

\begin{document}

\maketitle

\section{Mie scattering and the partial wave expansion}

In \href{lecture-12-1.tex}{the previous lecture} we discussed about Mie scattering. Here we discuss more about 
the partial wave expansion. First we review the scalar version of partial wave approximation.
Suppose the input wave function is a plane wave 
\begin{equation}
    u_\text{in} = \ee^{\ii k z} = \sum_{n=0}^\infty (2n+1) \ii^n j_n(kr ) \legpoly_n(\cos \theta).
\end{equation}
A scattering eigen state can still be seen as a mixture of spherical functions at the infinity, and therefore 
it can be written as 
\begin{equation}
    u = \sum_n (2n+1) \ii^n \legpoly_n(\cos \theta) \frac{1}{2} (h_n^{(2)}(kr) + S_n h_n^{(1)}(kr)),
\end{equation}
where $S_n$ is a complex factor. We know that at the infinity we have 
\begin{equation}
    u = \ee^{\ii k z} + \frac{\ee^{\ii k r}}{r} f(\theta),
\end{equation}
and therefore we have 
\begin{equation}
    f(\theta) = \frac{1}{2 \ii k} \sum_n (2n+1) (S_n - 1) \legpoly_
    n(\cos \theta) \eqqcolon 
    \frac{1}{2 \ii k} \sum_n (2n+1) (\ee^{\ii 2 \delta_n} - 1) \legpoly_n(\cos \theta).
\end{equation}
The unitarity of scattering guarantees that $\abs*{S_n} = 1$, and therefore $\delta_n$ is a real number 
and is often called the \concept{phase shift}.

In the $ka \to 0$ limit, we have 
\begin{equation}
    a_n = \frac{\epsilon_1 - \epsilon}{\ii (\epsilon_1 n + \epsilon (n+1))} \frac{(ka)^{2n+1} (n+1)}{(2n+1)!! (2n-1)!!}, \quad 
    b_n = \frac{\mu_1 - \mu}{\ii (\mu_1 n + \mu (n+1))} \frac{(ka)^{2n+1} (n+1)}{(2n+1)!! (2n-1)!!},
\end{equation}
and if we just keep the $a_1$ term we find 
\begin{equation}
    Q_\text{sca} = \frac{8}{3} (ka)^4 \left(\frac{\epsilon_1 - \epsilon}{\epsilon_1 + 2 \epsilon}\right)^2,
    \label{eq:rayleigh-cross-section}
\end{equation}
which is just the Rayleigh cross section. 

\section{Effective medium}

This is actually a way to find the effective permittivity of a electric dipole or to find the effective electric
dipole of a sphere. Suppose 
\begin{equation}
    \vb*{p} = \vu*{e}_x \alpha E_0 \ee^{\ii k z},
\end{equation}
and we have 
\begin{equation}
    \vb*{E}(\vb*{r}, \omega) = \omega^2 \mu_0 \tensor{\vb*{G}} \cdot \vb*{p}, 
\end{equation}
where 
\begin{equation}
    \tensor{\vb*{G}} = \frac{\ee^{\ii k R}}{4 \pi R} \left( \tensor{\vb*{I}} - \frac{\vb*{R} \vb*{R}}{\vb*{R}^2} \right)
\end{equation}
is the dyadic Green function. Then we find 
\[
    \frac{k}{\epsilon} \Im \alpha = C_\text{ext}^\text{dipole} = C_\text{ext}^\text{Mie} = 4 \pi k a^3 \Im \left(\frac{\epsilon_1 - \epsilon}{\epsilon_1 + 2 \epsilon}\right),
\]
and therefore 
\begin{equation}
    \alpha = 4 \pi \epsilon a^3 \frac{\epsilon_1 - \epsilon}{\epsilon_1 + 2 \epsilon}.
\end{equation}
This is the basis of \concept{Maxwell Garnett theory}. Suppose some particles with permittivity $\epsilon_1$
are distributed in a matrix medium with permittivity $\epsilon_\text{m}$.
By the Clausius–Mossotti relation we have 
\[
    \frac{\epsilon_\text{eff} - 1}{\epsilon_\text{eff} + 2} = \frac{N}{3 \epsilon_\text{m}} \alpha_\text{sphere},
\]
and on the other hand we have 
\[
    \alpha_\text{sphere} = 4 \pi \epsilon_\text{m} a^3 \frac{\epsilon_1 - \epsilon_\text{m}}{\epsilon_1 + 2 \epsilon_\text{m}},
\]
and therefore 
\begin{equation}
    \frac{\epsilon_\text{eff} - 1}{\epsilon_\text{eff} + 2} = \frac{4}{3} \pi a^3 N \frac{\epsilon_1 - \epsilon_\text{m}}{\epsilon_1 + 2 \epsilon_\text{m}}.
    \label{eq:maxwell-garnett}
\end{equation}
\eqref{eq:maxwell-garnett} only works for spheres. A more general formulation is 
\begin{equation}
    \vb*{p} = V \beta (\epsilon_1 - \epsilon_\text{m}) \vb*{E},
\end{equation}
and in the case of \eqref{eq:maxwell-garnett} we have 
\begin{equation}
    \beta = \frac{3 \epsilon_\text{m}}{\epsilon_1 + 2 \epsilon_\text{m}} = \frac{\epsilon_\text{m}}{\epsilon_\text{m} + \frac{1}{3} (\epsilon_1 - \epsilon_m)}.
\end{equation}
More generally we have 
\begin{equation}
    \beta = \frac{\epsilon_\text{m}}{\epsilon_\text{m} + L (\epsilon_1 - \epsilon_\text{m})}.
\end{equation}
Note that $\beta$ and $L$ may be tensors, and for a sphere $L = 1/3$. We define 
\begin{equation}
    f_\text{r} = NV,
\end{equation}
where $V$ is the total volume of the medium. It can be seen that $f_\text{r}$ gives the volume fraction of the 
particles. Therefore we have 
\begin{equation}
    \frac{\epsilon_\text{eff} - 1}{\epsilon_\text{eff} + 2} = \frac{f_\text{r} \beta}{3 \epsilon_\text{m}} (\epsilon_1 - \epsilon_\text{m}),
\end{equation} 
or in other words 
\begin{equation}
    \epsilon_\text{eff} = \epsilon_\text{m} \left( 1 + \frac{3 f_\text{r} \frac{\epsilon_1 - \epsilon_\text{m}}{\epsilon_1 + 2 \epsilon_\text{m}}}{1 - f_\text{r} \frac{\epsilon_1 - \epsilon_\text{m}}{\epsilon_1 + 2 \epsilon_\text{m}}} \right).
\end{equation}
This equation works for particles in the shape of spheres, rectangles or cylinders.
For tablet-like particles, for example, we have $L_\parallel = 0, L_\bot = 1$, and therefore 
\[
    \beta_\parallel = 1, \quad \beta_\bot = \frac{\epsilon_\text{m}}{\epsilon_1},
\] 
and therefore 
\begin{equation}
    \epsilon_{\text{ext}, \parallel} = (1 - f_\text{r}) \epsilon_\text{m} + f_\text{r} \epsilon_1,
\end{equation}
and
\begin{equation}
    \epsilon_{\text{ext}, \bot} = \left( \frac{1 - f_\text{r}}{\epsilon_\text{m}} + \frac{f_\text{r}}{\epsilon_1} \right)^{-1}.
\end{equation}

Note that we have restricted ourselves on dielectrics, where the lowest electromagnetic modes are almost 
always gapless, and the effective medium approximation works well. This approximation may broke for metals,
where low frequency electromagnetic waves cannot pass because 
\[
    \epsilon = 1 - \frac{\omega_\text{p}^2}{\omega^2}.
\]
In the language of band theory, in metals, electromagnetic modes are gapped. An interesting case is 
when both $\epsilon$ and $\mu$ are negative. This is called \concept{negative-index metamaterial}.

\section{A brief summary}

\begin{itemize}
    \item Geometrical optics limit.
    \item Mie scattering.
    \begin{itemize}
        \item Special functions: vector spherical functions, $\pi_n, \tau_n$, Riccati-Bessel functions,
        \item Mie coefficients, and
        \item Cross sections.        
    \end{itemize}
    \item Partial wave expansion, and
    \item Effective permittivity.
\end{itemize}

\section{Special relativity}

If the form of an EOM is invariant under a certain transformation, we say that the objects involved 
in the equation are \concept{covariant}. They will change under the transformation, but they change 
in a certain way that in an EOM their changes somehow cancel with each other, leaving the form of the 
EOM invariant. Sometimes we also say the equation is \emph{covariant}, since only the form is invariant,
and the actually LHS and RHS change.

Under the Galilean transformation
\begin{equation}
    \vb*{r}' = \vb*{r} - \vb*{v} t, \quad t' = t,
\end{equation}
we have 
\begin{equation}
    \pdv{\vb*{r}} = \pdv{\vb*{r}'}, \quad \pdv{t} = \pdv{t'} - \vb*{v} \cdot \pdv{\vb*{r}'},
\end{equation}
and therefore 
\begin{equation}
    \pdv[2]{t} = \pdv[2]{{t'}} - 2 \vb*{v} \cdot \pdv{\vb*{r}'} \pdv{t'} + \vb*{v} \vb*{v} : \pdv{\vb*{r}'} \pdv{\vb*{r}'},
\end{equation}
so under the (passive) Galilean transformation, we see that the wave equation 
\begin{equation}
    \laplacian f - \frac{1}{c^2} \pdv[2]{f}{t} = 0
\end{equation}
turns into 
\begin{equation}
    \left( {\laplacian}' - \frac{1}{c^2} \pdv[2]{{t'}} + \frac{2}{c^2} (\vb*{v} \cdot \grad') \pdv{t'} - \frac{1}{c^2} \vb*{v} \vb*{v} : \grad' \grad' \right) f = 0.
\end{equation}
So the wave function is \emph{not} covariant under the Galilean transformation.


\end{document}