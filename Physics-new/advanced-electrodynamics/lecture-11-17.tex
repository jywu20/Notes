\documentclass[hyperref, a4paper]{article}

\usepackage{geometry}
\usepackage{float}
\usepackage{titling}
\usepackage{titlesec}
% No longer needed, since we will use enumitem package
% \usepackage{paralist}
\usepackage{enumitem}
\usepackage{footnote}
\usepackage{enumerate}
\usepackage{amsmath, amssymb, amsthm}
\usepackage{mathtools}
\usepackage{bbm}
\usepackage{cite}
\usepackage{graphicx}
\usepackage{subcaption}
\usepackage{physics}
\usepackage{tensor}
\usepackage{siunitx}
\usepackage{booktabs}
\usepackage[version=4]{mhchem}
\usepackage{tikz}
\usepackage{xcolor}
\usepackage{listings}
\usepackage{autobreak}
\usepackage[ruled, vlined, linesnumbered]{algorithm2e}
\usepackage{nameref,zref-xr}
\zxrsetup{toltxlabel}
%\zexternaldocument*[prev-]{./lecture-10-20}[lecture-10-20.pdf]
\zexternaldocument*[optics-]{../optics/optics}[optics.pdf]
\zexternaldocument*[solid-]{../solid/solid}[solid.pdf]
% \zexternaldocument*[prev-]{./lecture-11-3}[lecture-11-10.pdf]
\usepackage[colorlinks,unicode]{hyperref} % , linkcolor=black, anchorcolor=black, citecolor=black, urlcolor=black, filecolor=black
\usepackage{prettyref}

% Page style
\geometry{left=3.18cm,right=3.18cm,top=2.54cm,bottom=2.54cm}
\titlespacing{\paragraph}{0pt}{1pt}{10pt}[20pt]
\setlength{\droptitle}{-5em}
\preauthor{\vspace{-10pt}\begin{center}}
\postauthor{\par\end{center}}

% More compact lists 
\setlist[itemize]{itemindent=17pt, leftmargin=1pt}

% Math operators
\DeclareMathOperator{\timeorder}{\mathcal{T}}
\DeclareMathOperator{\diag}{diag}
\DeclareMathOperator{\legpoly}{P}
\DeclareMathOperator{\primevalue}{P}
\DeclareMathOperator{\sgn}{sgn}
\newcommand*{\ii}{\mathrm{i}}
\newcommand*{\ee}{\mathrm{e}}
\newcommand*{\const}{\mathrm{const}}
\newcommand*{\suchthat}{\quad \text{s.t.} \quad}
\newcommand*{\argmin}{\arg\min}
\newcommand*{\argmax}{\arg\max}
\newcommand*{\normalorder}[1]{: #1 :}
\newcommand*{\pair}[1]{\langle #1 \rangle}
\newcommand*{\fd}[1]{\mathcal{D} #1}
\DeclareMathOperator{\bigO}{\mathcal{O}}
\DeclareMathOperator{\object}{Ob}
\DeclareMathOperator{\morphism}{Hom}

% Support for tensor double arrows.
\renewcommand{\tensor}[1]{ \stackrel{\leftrightarrow}{\vb*{#1}}}

% TikZ setting
\usetikzlibrary{arrows,shapes,positioning}
\usetikzlibrary{arrows.meta}
\usetikzlibrary{decorations.markings}
\tikzstyle arrowstyle=[scale=1]
\tikzstyle directed=[postaction={decorate,decoration={markings,
    mark=at position .5 with {\arrow[arrowstyle]{stealth}}}}]
\tikzstyle ray=[directed, thick]
\tikzstyle dot=[anchor=base,fill,circle,inner sep=1pt]

% Algorithm setting
% Julia-style code
\SetKwIF{If}{ElseIf}{Else}{if}{}{elseif}{else}{end}
\SetKwFor{For}{for}{}{end}
\SetKwFor{While}{while}{}{end}
\SetKwProg{Function}{function}{}{end}
\SetArgSty{textnormal}

\newcommand*{\concept}[1]{{\textbf{#1}}}

\newrefformat{fig}{Figure~\ref{#1}}

\newcommand{\opticsddoc}{\href{../optics/optics}{the solid state physics note}}
\newcommand{\soliddoc}{\href{../solid/solid}{the solid state physics note}}
\newcommand{\prevdoc}{\href{./lecture-11-3}{the previous lecture}}

% Embedded codes
\lstset{basicstyle=\ttfamily,
  showstringspaces=false,
  commentstyle=\color{gray},
  keywordstyle=\color{blue}
}

\allowdisplaybreaks

\title{Green Function in Electrodynamics by Prof. Kun Din}
\author{Jinyuan Wu}

\begin{document}

\maketitle

\section{A review of the theory of basic linear operators}

It should be noted that the operators representing the system in linear optical systems are slightly more 
complicated that the case in quantum mechanics. For example, 
in Section~\ref{optics-sec:long-wavelength-photon-maxwell-general} in \opticsddoc, 
we find the orthogonal relation is
\[
    \int_V \dd[3]{\vb*{r}} \vb*{u}^*_m \cdot \vb*{\epsilon} \cdot \vb*{u}_n = \delta_{mn}
\]
instead of 
\[
    \int_V \dd[3]{\vb*{r}} \vb*{u}^*_m \cdot \vb*{u}_n = \delta_{mn}.
\]
This means the Maxwell equations in a general linear optical system is \emph{not} Hermitian in the sense in quantum mechanics.

Here we review basic aspects of linear operators in an attempt to create a ``generalized quantum mechanics'' for photons.

Given an inner product $\langle \cdot, \cdot \rangle$, which is a positive bilinear form,
we define the \emph{adjoint} of a linear operator $\mathcal{L}$ as $\bar{\mathcal{L}}$ such that for all vectors $u, v$, we have 
\begin{equation}
    \langle v , \mathcal{L} u \rangle = \langle \bar{\mathcal{L}} v, u \rangle.
\end{equation}
Then we have the following theorems:
\begin{itemize}
    \item \concept{Fredholm's theorem}: The orthogonal complement of the row space of $\mathcal{L}$ is the null space of $\mathcal{L}$.
    In other words, if there exists a vector $u$ such that $\mathcal{L} u = f$, and we have $\bar{\mathcal{L}} v = 0$, 
    then $\expval{v, f} = 0$.
    \item Each linear operator corresponds to a bilinear form.
    \item If $\mathcal{L} u_n = \lambda_n u_n$ and $\bar{\mathcal{L}} v_m = \lambda_m v_m$, then $\expval{v_m, u_n} = \delta_{mn}$.
    In other words the orthogonal relation now works for a \concept{right eigenvector} and a \concept{left eigenvector}.
\end{itemize}

Let us consider elementary linear algebra on $\mathbb{C}$, which is the quantum mechanics for lattice systems, like a tight-binding model.
Operators now correspond to matrices, and we no longer distinguish the two concepts.
We define 
\begin{equation}
    \expval{\vb{a}, \vb{b}} = \vb{a}^\dagger \vb{b},
\end{equation}
which is a positive bilinear form. The definition of adjoints now reads 
\[
    \vb{v}^\dagger (\mathcal{L} \vb{u}) = (\bar{\mathcal{L}} \vb{v})^\dagger \vb{u} = \vb{v}^\dagger (\bar{\mathcal{L}})^\dagger \vb{u},
\]
or in other words $\bar{\mathcal{L}} = \mathcal{L}^\dagger$.
If $\bar{\mathcal{L}} \vb{u} = \lambda \vb{u}$, i.e. $\vb{u}$ is a left eigenvector of $\mathcal{L}$, then 
\[
    \lambda^* \vb{u}^\dagger = (\bar{\mathcal{L}} \vb{u})^\dagger = \vb{u}^\dagger \mathcal{L}^{\dagger \dagger} 
    = \vb{u}^\dagger \mathcal{L},
\]
and we see the meaning of the term ``left eigenvector''.

Now we go to a infinite-dimensional case: \concept{Sturm-Liouville theory}. Consider a differential operator 
\begin{equation}
    \mathcal{L} = p_0 \dv[2]{x} + p_1 \dv{x} + p_2,
\end{equation}
and we define 
\begin{equation}
    \expval{f, g} = \int_{a}^b \dd{x} f(x)^* g(x).
\end{equation}
By integration by parts we have 
\[
    \begin{aligned}
        \expval{v, \mathcal{L}u} &= \int_a^b \dd{x} v^* \left( p_0 \dv[2]{u}{x} + p_1 \dv{u}{x} + p_2 u \right) \\
        &= \eval*{\left( (p_0 v^*) \dv{u}{x} - \dv{(p_0 v^*)}{x} u + p_1 v^* u \right) }_{a}^{b} 
        + \int_a^b \dd{x} \left(\dv[2]{(p_0 v^*)}{x} - \dv{(p_1 v^*)}{x} + p_2 v^*\right) u .
    \end{aligned}
\]
We say the \concept{formal adjoint} of $\mathcal{L}$ is defined by
\begin{equation}
    \bar{\mathcal{L}} f = \dv[2]{(p_0 f)}{x} - \dv{(p_1 f)}{x} + p_2 f.
    \label{eq:formal-adjoint}
\end{equation}
If the boundary condition makes 
\begin{equation}
    \eval*{\left( (p_0 v^*) \dv{u}{x} - \dv{(p_0 v^*)}{x} u + p_1 v^* u \right) }_{a}^{b} = 0
    \label{eq:sl-boundary}
\end{equation}
holds, then the formal adjoint of $\mathcal{L}$ defined in \eqref{eq:formal-adjoint} is the true adjoint.
We can see that \eqref{eq:sl-boundary} holds under Dirichlet boundary condition, Neumann boundary condition 
and Floquent boundary condition, so Sturm-Liouville theory works perfectly well for a large class of physical problems.
We usually define 
\begin{equation}
    \expval{v, \mathcal{L} u} - \expval*{\bar{\mathcal{L}} v, u} = J(v, u)
    \label{eq:generalized-green}
\end{equation}
as the \concept{bilinear concomitant}, where $\mathcal{L}$ is the formal adjoint in \eqref{eq:formal-adjoint}.
\eqref{eq:generalized-green} may be seen as the generalized version of the Green formula in electrostatics.
We can see that $\mathcal{L}$ is self-adjoint if and only if 
\begin{equation}
    p_1 = \dv{p_0}{x}.
    \label{eq:self-adjoint-condition}
\end{equation}

It can be soon realized that \eqref{eq:self-adjoint-condition} does not hold in electrodynamics, 
where $p_0$ may have spacial variation (i.e. $\epsilon$) but there is no $p_1$.
There is one way to make up for the fact. We define 
\begin{equation}
    w(x) = \frac{1}{p_0} \exp(\int \dd{x} \frac{p_1(x)}{p_0(x)}),
\end{equation}
\begin{equation}
    \bar{p}_0 = \exp(\int \dd{x} \frac{p_1(x)}{p_0(x)}),
\end{equation}
and 
\begin{equation}
    \bar{p}_1 = \frac{p_1}{p_0} \exp(\int \dd{x} \frac{p_1(x)}{p_0(x)}),
\end{equation}
then we have 
\begin{equation}
    \int_a^b \dd{x} v^* (w(x) \mathcal{L}) u = 
    \eval{(v^* \bar{p}_0 u' - (v^*)' \bar{p}_0 u)}_a^b + \int_a^b \dd{x} (w(x) \mathcal{L} v)^* u. 
\end{equation}
So here again, we define the inner product to be 
\begin{equation}
    \expval{u, v} = \int_a^b \dd{x} u^* v,
\end{equation}
and we define the formal adjoint to be $\mathcal{L}^*$, and if the boundary condition makes 
\begin{equation}
    \eval{(v^* \bar{p}_0 u' - (v^*)' \bar{p}_0 u)}_a^b = 0
\end{equation}
holds then $\mathcal{L}$'s adjoint is the formal adjoint.

\section{Maxwell equation}

We consider the wave equation 
\begin{equation}
    \left( \curl \curl - \left(\frac{\omega}{c}\right)^2 \epsilon_\text{r}(\vb*{r})\right) \vb*{E} = 0.
\end{equation}
This may be viewed as an eigenvalue problem, where the linear operator is 
\begin{equation}
    \mathcal{L}_E = \frac{1}{\epsilon_\text{r}(\vb*{r})} \curl \curl.
    \label{eq:wave-eigen-1}
\end{equation}
To make the equation more symmetric, we define 
\begin{equation}
    \vb*{Q}(\vb*{r}) = \sqrt{\epsilon_\text{r}(\vb*{r})} \vb*{E}(\vb*{r}),
\end{equation}
and now problem \eqref{eq:wave-eigen-1} reads 
\begin{equation}
    H \vb*{Q} \coloneqq \frac{1}{\sqrt{\epsilon_\text{r}(\vb*{r})}} 
    \curl \left( \curl \frac{1}{\sqrt{\epsilon_\text{r}(\vb*{r})}} \vb*{Q}(\vb*{r}) \right)
    = \left(\frac{\omega}{c}\right)^2 \vb*{Q}.
\end{equation}
We define inner product for $\vb*{Q}$ vectors as 
\[
    \expval{\vb*{f}, \vb*{g}} = \int \dd[3]{\vb*{r}} \vb*{f}^*(\vb*{r}) \cdot \vb*{g}(\vb*{r}),
\]
and then we have 
\[
    \int \dd[3]{\vb*{r}} \left( \frac{1}{\sqrt{\epsilon_\text{r}}} \curl \left( \curl \frac{1}{\sqrt{\epsilon_\text{r}}} \vb*{Q}_1^* \right) \right) \cdot \vb*{Q}_2
    = \oint \dd{\vb*{S}} \cdot \left( \left( \curl \frac{\vb*{Q}^*_1}{\sqrt{\epsilon_\text{r}}} \right) \times \frac{\vb*{Q}_2}{\sqrt{\epsilon_\text{r}}} \right)
    + , % TODO
\]
for Floquent boundary condition the surface term vanishes, and we fine \eqref{eq:wave-eigen-1} is self-adjoint.
Thus, under the definition of inner product 
\begin{equation}
    \expval*{\vb*{E}_1, \vb*{E}_2} = \expval*{\vb*{Q}_1, \vb*{Q}_2} = \int \dd[3]{\vb*{r}} \epsilon_\text{r}(\vb*{r}) \vb*{E}_1^* \cdot \vb*{E}_2.
\end{equation}

Now we consider the generalized linear constitutive relation 
\begin{equation}
    (\tensor{\vb*{D}} - \ii \omega \tensor{\vb*{K}}) \vb*{e} = - \vb*{J},
\end{equation}
where 
\begin{equation}
    \tensor{\vb*{D}} \coloneqq \pmqty{0 & - \ii \curl \\ - \ii \curl & 0}, 
    \quad \tensor{\vb*{K}} \coloneqq \pmqty{ \tensor{\vb*{\epsilon}} & \tensor{\vb*{\xi}} \\ \tensor{\vb*{\eta}} & \tensor{\vb*{\mu}} },
    \quad \vb*{e} \coloneqq \pmqty{\vb*{E} \\ \vb*{H}} , \quad \vb*{j} \coloneqq \pmqty{\vb*{J} \\ \vb*{M}}.
\end{equation}
Let 
\begin{equation}
    U = \pmqty{\dmat{1, -1}},
\end{equation}
and we have 

\section{Review}

\begin{itemize}
    \item Transforming differential equations into integral equations often makes things easier: the idea of Green functions.
    \item Green function as the inverse of the LHS of an inhomogeneous equation.
    \item From scalar Green function to the dyadic Green function.
    \item Retarded potential.
    \item Example: electric dipole.
    \item Green function and Fourier transformation.
    \item On-shell and off-shell properties.
    \item Weyl and Sommerfield representation.
    \item Fourier transformation of the dyadic Green function.
    \item Spectrum decomposition.
    \item The foundation of the spectrum decomposition: linear operators, adjoint.
\end{itemize}

These lectures do not cover 1D or 2D cases. The low dimensional cases may be derived in the same manner.
The 1D, 2D and 3D Green functions are related. For example, the 2D Green function may be viewed as a 
generalized 3D Green function with a line source instead of a point source.

\end{document}