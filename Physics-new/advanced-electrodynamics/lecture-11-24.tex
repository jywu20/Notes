\documentclass[hyperref, a4paper]{article}

\usepackage{geometry}
\usepackage{titling}
\usepackage{titlesec}
% No longer needed, since we will use enumitem package
% \usepackage{paralist}
\usepackage{enumitem}
\usepackage{footnote}
\usepackage{enumerate}
\usepackage{amsmath, amssymb, amsthm}
\usepackage{mathtools}
\usepackage{bbm}
\usepackage{cite}
\usepackage{graphicx}
\usepackage{subfigure}
\usepackage{physics}
\usepackage{tensor}
\usepackage{siunitx}
\usepackage[version=4]{mhchem}
\usepackage{tikz}
\usepackage{xcolor}
\usepackage{listings}
\usepackage{autobreak}
\usepackage[ruled, vlined, linesnumbered]{algorithm2e}
\usepackage{nameref,zref-xr}
\zxrsetup{toltxlabel}
\zexternaldocument*[optics-]{../optics/optics}[optics.pdf]
\zexternaldocument*[solid-]{../solid/solid}[solid.pdf]
\usepackage[colorlinks,unicode]{hyperref} % , linkcolor=black, anchorcolor=black, citecolor=black, urlcolor=black, filecolor=black
\usepackage[most]{tcolorbox}
\usepackage{prettyref}

% Page style
\geometry{left=3.18cm,right=3.18cm,top=2.54cm,bottom=2.54cm}
\titlespacing{\paragraph}{0pt}{1pt}{10pt}[20pt]
\setlength{\droptitle}{-5em}
\preauthor{\vspace{-10pt}\begin{center}}
\postauthor{\par\end{center}}

% More compact lists 
\setlist[itemize]{
    itemindent=17pt, 
    leftmargin=1pt,
    listparindent=\parindent,
    parsep=0pt,
}

% Math operators
\DeclareMathOperator{\timeorder}{\mathcal{T}}
\DeclareMathOperator{\diag}{diag}
\DeclareMathOperator{\legpoly}{P}
\DeclareMathOperator{\primevalue}{P}
\DeclareMathOperator{\sgn}{sgn}
\newcommand*{\ii}{\mathrm{i}}
\newcommand*{\ee}{\mathrm{e}}
\newcommand*{\const}{\mathrm{const}}
\newcommand*{\suchthat}{\quad \text{s.t.} \quad}
\newcommand*{\argmin}{\arg\min}
\newcommand*{\argmax}{\arg\max}
\newcommand*{\normalorder}[1]{: #1 :}
\newcommand*{\pair}[1]{\langle #1 \rangle}
\newcommand*{\fd}[1]{\mathcal{D} #1}
\DeclareMathOperator{\bigO}{\mathcal{O}}

% TikZ setting
\usetikzlibrary{arrows,shapes,positioning}
\usetikzlibrary{arrows.meta}
\usetikzlibrary{decorations.markings}
\tikzstyle arrowstyle=[scale=1]
\tikzstyle directed=[postaction={decorate,decoration={markings,
    mark=at position .5 with {\arrow[arrowstyle]{stealth}}}}]
\tikzstyle ray=[directed, thick]
\tikzstyle dot=[anchor=base,fill,circle,inner sep=1pt]

% Algorithm setting
% Julia-style code
\SetKwIF{If}{ElseIf}{Else}{if}{}{elseif}{else}{end}
\SetKwFor{For}{for}{}{end}
\SetKwFor{While}{while}{}{end}
\SetKwProg{Function}{function}{}{end}
\SetArgSty{textnormal}

\newcommand*{\concept}[1]{{\textbf{#1}}}

% Embedded codes
\lstset{basicstyle=\ttfamily,
  showstringspaces=false,
  commentstyle=\color{gray},
  keywordstyle=\color{blue}
}

% Reference formatting
\newrefformat{fig}{Figure~\ref{#1} on page~\pageref{#1}}

% Color boxes
\tcbuselibrary{skins, breakable, theorems}
\newtcbtheorem[number within=section]{warning}{Warning}%
  {colback=orange!5,colframe=orange!65,fonttitle=\bfseries, breakable}{warn}
\newtcbtheorem[number within=section]{note}{Note}%
  {colback=green!5,colframe=green!65,fonttitle=\bfseries, breakable}{note}
\newtcbtheorem[number within=section]{info}{Info}%
  {colback=blue!5,colframe=blue!65,fonttitle=\bfseries, breakable}{info}

\title{Diffraction and Scattering in Electrodynamics by Prof. Kun Din}
\author{Jinyuan Wu}
\date{November 24, 2021}

\begin{document}

\maketitle

In previous lectures we only discussed boundaries at infinity. The details of boundaries do not affect the 
electromagnetic waves. Their role - if any - is to ``guide'' the electromagnetic waves, as is the case in 
spherical optical cavities. 

In this lecture we are going to discuss the physics of boundaries in electrodynamics.
Suppose $\kappa$ to be the magnitude of the curvature of a boundary, which estimates its geometric details.
When $\kappa \lambda \ll 1$, the boundary is almost flat in the eye of the light, and the theory on 
what happens around the boundary is \emph{geometrical optics}. 
On the other hand, when $\kappa \lambda \gg 1$, the details of the wave fronts of electromagnetic waves 
are not clear in the eye of the boundary, and this case is called \concept{Rayleigh scattering}.
Rayleigh scattering usually happens when small particles are placed in light, and what the particles 
feel is almost a homogeneous variating potential. 

Between the two limits are \concept{scattering} and \concept{diffraction}. Scattering is defined as the 
phenomenon that the electrodynamic modes couple with inner modes of matter strongly, and the momenta of 
both change significantly. Diffraction, on the other hand, means that light beams are split by the boundary, and
there is no strong change on momentum, but the distribution in the momentum space changes (for example, if a
plane wave is injected into a system, the direction of the output light beam is still the same, but the intensity
now has spatial distribution, or in other words we can see a smooth distribution in the momentum space).

\section{The general theory of scattering}

\begin{figure}
    \centering
    

\tikzset{every picture/.style={line width=0.75pt}} %set default line width to 0.75pt        

\begin{tikzpicture}[x=0.75pt,y=0.75pt,yscale=-1,xscale=1]
%uncomment if require: \path (0,300); %set diagram left start at 0, and has height of 300

%Shape: Polygon Curved [id:ds2533540347281886] 
\draw   (165,71.5) .. controls (185,61.5) and (207.01,43.33) .. (255,71.5) .. controls (302.99,99.67) and (243.99,94.17) .. (255,131.5) .. controls (266.01,168.83) and (185,161.5) .. (165,131.5) .. controls (145,101.5) and (145,81.5) .. (165,71.5) -- cycle ;
%Straight Lines [id:da7727972044618603] 
\draw    (24,80) -- (120.01,80) ;
\draw [shift={(72.01,80)}, rotate = 180] [fill={rgb, 255:red, 0; green, 0; blue, 0 }  ][line width=0.08]  [draw opacity=0] (12,-3) -- (0,0) -- (12,3) -- cycle    ;
%Straight Lines [id:da38949556327200385] 
\draw    (24,100) -- (120.01,100) ;
\draw [shift={(72.01,100)}, rotate = 180] [fill={rgb, 255:red, 0; green, 0; blue, 0 }  ][line width=0.08]  [draw opacity=0] (12,-3) -- (0,0) -- (12,3) -- cycle    ;
%Straight Lines [id:da9522247337792025] 
\draw    (25,118) -- (121.01,118) ;
\draw [shift={(73.01,118)}, rotate = 180] [fill={rgb, 255:red, 0; green, 0; blue, 0 }  ][line width=0.08]  [draw opacity=0] (12,-3) -- (0,0) -- (12,3) -- cycle    ;
%Straight Lines [id:da7188532746511043] 
\draw    (291.5,149.5) -- (398.51,192.33) ;
\draw [shift={(345.01,170.92)}, rotate = 201.81] [fill={rgb, 255:red, 0; green, 0; blue, 0 }  ][line width=0.08]  [draw opacity=0] (12,-3) -- (0,0) -- (12,3) -- cycle    ;
%Straight Lines [id:da26404412726962834] 
\draw    (311.5,62.5) -- (392.01,24.83) ;
\draw [shift={(351.76,43.67)}, rotate = 154.93] [fill={rgb, 255:red, 0; green, 0; blue, 0 }  ][line width=0.08]  [draw opacity=0] (12,-3) -- (0,0) -- (12,3) -- cycle    ;

% Text Node
\draw (53,141.4) node [anchor=north west][inner sep=0.75pt]    {$\boldsymbol{E}_{\text{in}}$};
% Text Node
\draw (306,14.4) node [anchor=north west][inner sep=0.75pt]    {$\boldsymbol{E}_{\text{sca}}$};
% Text Node
\draw (156,47.4) node [anchor=north west][inner sep=0.75pt]    {$S$};
% Text Node
\draw (198.5,109.9) node [anchor=north west][inner sep=0.75pt]    {$V$};


\end{tikzpicture}

    \caption{Scattering system}
    \label{fig:scattering}
\end{figure}

Consider a optical system is placed in vacuum, and we investigate its behavior when light is shed on it.
Energy conservation gives 
\begin{equation}
    - \int \dd[3]{\vb*{r}} \div{\vb*{S}} - \pdv{u}{t} = \int \dd[3]{\vb*{r}} \vb*{j} \cdot \vb*{E},
    \label{eq:energy-conservation-eq}
\end{equation}
where the volume contains the volume of the system.
We make the decomposition 
\begin{equation}
    \vb*{E} = \vb*{E}_\text{in} + \vb*{E}_\text{sca}, \quad \vb*{H} = \vb*{H}_\text{in} + \vb*{H}_\text{sca},
\end{equation}
and from now on we work with monochromic light. 

\begin{note*}{}{}
    Working on monochromic light is equivalent to working on stationary states or eigenstates of an optical system.
    There is no guarantee that in a real experimental setting, the input light persists long enough 
    to make the system's state arbitrarily close to a stationary state.
    Generally speaking, when a wave pocket is injected into an optical system, 
    we will see two phenomena: 
    \begin{itemize}
        \item The wave pocket expands, because different frequency components in the wave pocket go through
        the optical system with different speeds, so frequency components in the output state get staggered.
        This can be called the temporal effects, though wave pocket expansion can also be observed in the space.
        \item Since the input wave pocket has a finite spacial distribution, it has lots of wave vector
        components, and each of them may be scattered in a slightly different way, so we may observe 
        diffraction in the output wave pocket. This may be called the spacial effects. 
    \end{itemize}
    Generally, if the duration of a pulse is longer than 50 or 100 light periods, we can view the input 
    pulse as an infinite one, and the output part of the stationary state accurately describe the output.
    If the duration of the input pulse is shorter, or the optical system has very strong dispersion,
    the scattering behavior changes dramatically.
    For these cases - often dubbed as \concept{ultrafast pulses} - solving Maxwell equations in the time 
    domain is much more convenient than solving the frequency domain equation.
    Thus, time-dependent solution of Maxwell equations usually happens in investigation of ultrafast pulses.
\end{note*}

We consider a system illustrated in \prettyref{fig:scattering}. We choose a large surface $S$ to surround the 
system. The surface $S$ must be large enough to ensure that it does not ``see'' the details of the scattering 
system. We expand the Poynting vector as  
\begin{equation}
    \expval*{\vb*{S}} = \frac{1}{2} \Re \vb*{E} \times \vb*{H}^* 
    = \frac{1}{2} \Re (\vb*{E}_\text{in} + \vb*{E}_\text{sca}) \times (\vb*{H}_\text{in}^* + \vb*{H}_\text{sca}^*),
\end{equation}
and we define 
\begin{equation}
    \expval*{\vb*{S}_\text{in}} = \frac{1}{2} \Re \vb*{E}_\text{in} \times \vb*{H}_\text{in}^*,
\end{equation}
\begin{equation}
    \expval*{\vb*{S}_\text{sca}} = \frac{1}{2} \Re \vb*{E}_\text{sca} \times \vb*{H}_\text{sca}^*,
\end{equation}
and 
\begin{equation}
    \expval*{\vb*{S}_\text{ext}} = \frac{1}{2} \Re (\vb*{E}_\text{sca} \times \vb*{H}_\text{in}^* + \vb*{E}_\text{in} \times \vb*{H}_\text{sca}^*).
\end{equation}
The last term is called \concept{extinction}, because it depicts how scattering interrupts the energy flow of 
the input light. We use $W$ to denote the power of energy flow $\vb*{S}$, i.e. the flux of the flow on a certain 
surface.

We often define 
\begin{equation}
    C = \frac{W}{I}
\end{equation}
as the \concept{cross section}. After evaluating the expectation value of \eqref{eq:energy-conservation-eq}, we have 
\[
    - \oint \dd{\vb*{S}} \cdot \expval*{\vb*{S}} = \int \dd[3]{\vb*{r}} \expval*{\vb*{j} \cdot \vb*{E}}.
\]
We find the RHS is just the rate of light absorption: We have 
\[
    \begin{aligned}
        W_\text{abs} &= - \oint \dd{\vb*{S}} \cdot \expval*{\vb*{S}} \\
        &= - \oint \dd{\vb*{S}} \cdot \expval*{(\vb*{S}_\text{in} + \vb*{S}_\text{sca} + \vb*{S}_\text{ext} )}.
    \end{aligned}
\]
We know that in the scattering components energy is flowing outward, and therefore 
\begin{equation}
    W_\text{sca} = \oint \dd{\vb*{S}} \cdot \expval*{\vb*{S}_\text{sca}}.
\end{equation}
It is possible that the background medium is dissipative, and in this case we use 
\begin{equation}
    W_\text{in} = - \oint \dd{\vb*{S}} \cdot \expval*{\vb*{S}_\text{in}}
\end{equation}
to denote the energy dissipation in $S$ caused by the dissipative nature of the background medium. 
The sign of the extinction term needs some thinking, but since it is the interference between two beams 
of light, it is also outward, and therefore 
\begin{equation}
    W_\text{ext} = - \oint \dd{\vb*{S}} \cdot \expval*{\vb*{S}_\text{ext}}.
\end{equation}
Therefore, we have 
\[
    W_\text{abs} = W_\text{in} - W_\text{sca} + W_\text{ext},
\]
and therefore 
\begin{equation}
    C_\text{ext} + C_\text{in} = C_\text{sca} + C_\text{abs}.
\end{equation}
Usually there is no dissipation in the background material, and in this case we have 
\begin{equation}
    C_\text{ext} = C_\text{sca} + C_\text{abs},
\end{equation}
so the extinction cross section is also the \concept{total cross section}, including both scattering 
and absorption effects. We can see from the above derivation that it is a natural result of energy conservation.

In geometrical optics, the total cross section cannot be larger than the geometric
cross section of the absorber, but when the wave nature of light is taken into account, the total cross 
section can be larger - the diameter of the total cross section may be several wavelengths larger than the 
geometric diameter. This is often quite hard in engineering. The scattering cross section, even in geometrical
optics, can be as large as twice of the geometric cross section. Actually in geometrical optics, 
the total cross section is always twice of the geometrical cross section. % ??? really

When $\vb*{E}_\text{in}$ is a plane wave, the asymptotic behavior of $\vb*{E}$ as $\abs*{\vb*{r}} \to \infty$ is 
\begin{equation}
    \vb*{E} = E_0 \vu*{e}_0 \ee^{\ii \vb*{k}_0 \cdot \vb*{r}} + E_0 \frac{\ee^{\ii k r}}{r} \vb*{f}(\vb*{k}).
\end{equation}
Here we use $\vb*{k}$ to denote the direction of the scattered electromagnetic wave, which depends on $\vb*{r}$.
We are sure that the spacial dependence is in the form of $\ee^{\ii k r} / r$ because when $\abs*{\vb*{r}} \to 0$
the electromagnetic waves can be seen as spherical waves (the center of the sphere being the optical system where 
scattering happens). Information about the optical system, i.e. the angular distribution of scattered 
electromagnetic waves, the amplitude and the polarization is stored in $\vb*{f}(\vb*{k})$.

Since outside the optical system there is no current, we have 
\begin{equation}
    \vb*{H} = \frac{\vu*{k} \times \vb*{E}}{Z_0}, \quad Z_0 = \sqrt{\frac{\mu}{\epsilon}}.
\end{equation}
Note that if the optical system in question is an array of smaller optical components, 
then $Z_0$ \emph{cannot} be obtained from the photon band structure of the array 
- we can only read the refractive index from the band structure. 
We have to evaluate $E / H$ in that array to decide $Z_0$.
Then we find that 
\begin{equation}
    \begin{aligned}
        W_\text{ext} &= - \frac{1}{2} \oint \dd{S} \vu*{k} \cdot 
            \Re (\vb*{E}_\text{in} \times (\vu*{k} \times \vb*{E}_\text{sca}^*) + \vb*{E}_\text{sca} \times (\vu*{k}_0 \times \vb*{E}_\text{in}^*)) \\
        &= - \frac{E_0^2}{2 Z_0} \oint \frac{1}{r} \Re (
        (\vu*{e}_0 \cdot \vb*{f}^* ) \ee^{\ii \vb*{k}_0 \cdot \vb*{r}} \ee^{- \ii k r} 
        + (\vu*{e}^*_0 \cdot \vb*{f}) (\vu*{k}_0 \cdot \vu*{k}) \ee^{- \ii \vb*{k}_0 \cdot \vb*{r}} \ee^{\ii k r} \\
        &\quad \quad  - (\vu*{k} \cdot \vu*{e}_0) (\vu*{k} \cdot \vb*{f}^*) \ee^{\ii \vb*{k}_0 \cdot \vb*{r}} \ee^{- \ii k r} 
         - (\vu*{k} \cdot \vu*{e}_0^*) (\vu*{k}_0 \cdot \vb*{f}) \ee^{- \ii \vb*{k}_0 \cdot \vb*{r}} \ee^{\ii k r}  ) \dd{S}.
    \end{aligned}
    \label{eq:w-ext-step-1}
\end{equation}
Note that the factor $E_0^2 / (2 Z_0)$ is just the intensity of the input light, i.e. 
\begin{equation}
    I_\text{in} = \frac{1}{2} \epsilon_0 E_0^2 c = \frac{E_0^2}{2 Z_0},
\end{equation}
and $\dd{S}$ contains $r^2$ and a solid angle element $\dd{\Omega}$, so we can evaluate   
\[
    \int_{-\pi}^\pi \cos \theta \dd{\theta} \ee^{\pm \ii \cos \theta k r}
\]
factors in \eqref{eq:w-ext-step-1}. Note that \eqref{eq:w-ext-step-1} is only correct when we are 
far from the optical system, so we just need to take the $kr \to \infty$ limit and use \eqref{eq:far-kr}.
\begin{info*}{}
    We have 
    \[
        \int_{-1}^1 \dd{\mu}  \ee^{\pm \ii k r \mu} g(\mu) 
        = \int_{-1}^1 \dd \ee^{\pm \ii k r \mu} \frac{g(\mu)}{\pm \ii k r},
    \]
    and when $kr \to \infty$, since $\ee^{\ii k r \mu}$ vibrates much faster than $g(\mu)$, we have 
    \[
        \int_{-1}^1 \dd{\mu}  \ee^{\pm \ii k r \mu} g(\mu) \approx \int_{-1}^1 \dd\left(\ee^{\pm \ii k r \mu} \frac{g(\mu)}{\pm \ii k r}\right) = \frac{g(+1) \ee^{\pm \mathrm{i} k r}-g(-1) \ee^{\mp \mathrm{i} k r}}{\pm \mathrm{i} k r}.
    \]
    In the language of $\delta$-functions, we have 
    \begin{equation}
        \lim _{k_{0} r \rightarrow \infty} \ee^{\mathrm{i} \vb*{k}_{0} \cdot \vb*{r}}=\frac{2 \pi \mathrm{i}}{k_{0} r}\left\{\ee^{-\mathrm{i} k_{0} r} \delta\left(\hat{\vb*{r}}+\hat{\vb*{k}}_{0}\right)-\ee^{\mathrm{i} k_{0} r} \delta\left(\hat{\vb*{r}}-\hat{\vb*{k}}_{0}\right)\right\}.
        \label{eq:far-kr}
    \end{equation}
\end{info*}
\noindent Then we find that the last two terms in \eqref{eq:w-ext-step-1} vanish, because $\vb*{k}$ is 
replaced by $\vb*{k}_0$, and the $\vu*{k} \cdot \vu*{e}_0$ factors just vanish. The rest terms are (here 
we use $\mu$ to denote $\cos \theta$)
\[
    \begin{aligned}
        W_\text{ext} &= - I_\text{in} \int \frac{r^2 \dd{\varphi} \dd{\mu}}{r} \Re ((\vu*{e}_0 \cdot \vb*{f}^*) \ee^{\ii \vb*{k}_0 \cdot \vb*{r}} \ee^{- \ii k r} + (\vu*{e}^*_0 \cdot \vb*{f}) (\vu*{k}_0 \cdot \vu*{k}) \ee^{- \ii \vb*{k}_0 \cdot \vb*{r}} \ee^{\ii k r}) \\
        &= - I_\text{in} \int r \dd{\varphi} \Re \Biggl( \frac{\vu*{e}_0 \cdot \vb*{f}^*(\vb*{k}_0) \ee^{\ii k r} \ee^{- \ii k r}}{\ii k r} - \frac{\vu*{e}_0 \cdot \vb*{f}^*(- \vb*{k}_0) \ee^{- \ii k r} \ee^{- \ii k r}}{\ii k r} \\
        & \quad \quad \quad + \frac{\vu*{e}_0^* \cdot \vb*{f}(\vb*{k}_0) (\vu*{k}_0 \cdot \vu*{k}_0) \ee^{- \ii k r} \ee^{\ii k r}}{- \ii k r} - \frac{\vu*{e}_0^* \cdot \vb*{f}(-\vb*{k}_0) (- \vu*{k}_0 \cdot \vu*{k}_0) \ee^{ \ii k r} \ee^{\ii k r}}{- \ii k r}  \Biggr) ,
    \end{aligned}
\]
and it can be seen that there is no $\varphi$ dependence in the integral, so the integral over $\varphi$ just means $2\pi$. Moreover, we can see the second and the fourth term, after added up, is a pure imaginary number and therefore do not contribute to the final result. We therefore find 
\[
    \begin{aligned}
        W_\text{ext} &= - I_\text{in} \cdot 2 \pi r \frac{2 \Im \vu*{e}_0 \cdot \vb*{f}^*(\vb*{k}_0)}{k r} \\
        &= \frac{4 \pi }{k} I_\text{in} \Im \vu*{e}^*_0 \cdot \vb*{f}(\vb*{k}_0).
    \end{aligned}
\]
The final result is 
\begin{equation}
    C_\text{ext} = \frac{W_\text{ext}}{I_\text{in}} = \frac{4 \pi}{k} \Im \vb*{e}_0^* \cdot \vb*{f}(\vb*{k}_0).
    \label{eq:optical-theorem}
\end{equation}
This formula is called the \concept{optical theorem}. This is a result of energy conservation, or 
more generally, a result of \emph{unitarity}. 
\eqref{eq:optical-theorem} is actually a specific version of the optical theorem for plane waves, 
and it breaks for a highly convergent light beam. The generalized version of the optical theorem 
can be found in Born's famous textbook.

Similar to \eqref{eq:w-ext-step-1}, we have 
\[
    \begin{aligned}
        W_\text{sca} &= \frac{1}{2} \oint \dd{\vb*{S}} \cdot \vb*{E}_\text{sca} \times \vb*{H}_\text{sca} \\
        &= \frac{1}{2 Z} \oint \dd{\vb*{S}} \cdot \vb*{E}_\text{sca} \times (\vu*{k} \vb*{E}^*_\text{sca}) \\
        &= \frac{1}{2 Z} \oint \dd{S} \vu*{k} \cdot (\abs*{\vb*{E}_\text{sca}}^2 \vu*{k} - (\vu*{k} \cdot \vb*{E}_\text{sca}) \vb*{E}^*_\text{sca}).
    \end{aligned}
\]
When $r$ is large, $\ee^{\ii k r}$ vibrates much faster than $\vb*{f}(\vb*{k}) / r$, and therefore the transverse field condition requires
\[
    0 = \div{\frac{\ee^{\ii k r}}{r} \vb*{f}(\vb*{k})} \approx \ii \vb*{k} \cdot \vb*{f}(\vb*{r}) \frac{\ee^{\ii k r}}{r},
\]
and therefore $\vu*{k} \cdot \vb*{E}_\text{sca} = 0$, so we have 
\[
    \begin{aligned}
        W_\text{sca} &= \frac{1}{2Z} \oint \dd{S} \abs*{\vb*{E}_\text{sca}}^2 \\
        &= \frac{1}{2Z} E_0^2 \int r^2 \dd{\Omega} \frac{\abs*{\vb*{f}(\vb*{k})}^2}{r^2} \\
        &= I_\text{in} \int \dd{\Omega} \abs*{\vb*{f}(\vb*{k})}^2,
    \end{aligned}
\]
and therefore 
\begin{equation}
    C_\text{sca} = \frac{1}{2 Z_0 I_\text{in}} \oint \dd{S} \vu*{k} \cdot \Re \vb*{E}_\text{sca} \times (\vu*{k} \times \vb*{E}_\text{sca}) = \int_{4\pi} \abs*{\vb*{f}(\vb*{k})}^2 \dd{\Omega}.
\end{equation}
Again we get something we are already familiar in quantum mechanics. 

\section{Mie scattering}

\begin{figure}
    \centering
    

\tikzset{every picture/.style={line width=0.75pt}} %set default line width to 0.75pt        

\begin{tikzpicture}[x=0.75pt,y=0.75pt,yscale=-1,xscale=1]
%uncomment if require: \path (0,300); %set diagram left start at 0, and has height of 300

%Straight Lines [id:da20104584041056972] 
\draw    (232.66,178.01) -- (162.74,247.93) ;
\draw [shift={(161.33,249.34)}, rotate = 315] [fill={rgb, 255:red, 0; green, 0; blue, 0 }  ][line width=0.08]  [draw opacity=0] (12,-3) -- (0,0) -- (12,3) -- cycle    ;
%Straight Lines [id:da7424478890669008] 
\draw    (232.66,178.01) -- (387.02,178.01) ;
\draw [shift={(389.02,178.01)}, rotate = 180] [fill={rgb, 255:red, 0; green, 0; blue, 0 }  ][line width=0.08]  [draw opacity=0] (12,-3) -- (0,0) -- (12,3) -- cycle    ;
%Straight Lines [id:da7190301587069228] 
\draw    (232.66,178.01) -- (232.66,49.34) ;
\draw [shift={(232.66,47.34)}, rotate = 90] [fill={rgb, 255:red, 0; green, 0; blue, 0 }  ][line width=0.08]  [draw opacity=0] (12,-3) -- (0,0) -- (12,3) -- cycle    ;
%Shape: Circle [id:dp02220990844122195] 
\draw  [fill={rgb, 255:red, 255; green, 0; blue, 0 }  ,fill opacity=0.24 ] (188.06,178.01) .. controls (188.06,153.38) and (208.03,133.41) .. (232.66,133.41) .. controls (257.29,133.41) and (277.26,153.38) .. (277.26,178.01) .. controls (277.26,202.64) and (257.29,222.61) .. (232.66,222.61) .. controls (208.03,222.61) and (188.06,202.64) .. (188.06,178.01) -- cycle ;
%Shape: Ellipse [id:dp31691702644238307] 
\draw  [dash pattern={on 4.5pt off 4.5pt}] (188.06,178.01) .. controls (188.06,167.15) and (208.03,158.34) .. (232.66,158.34) .. controls (257.29,158.34) and (277.26,167.15) .. (277.26,178.01) .. controls (277.26,188.87) and (257.29,197.67) .. (232.66,197.67) .. controls (208.03,197.67) and (188.06,188.87) .. (188.06,178.01) -- cycle ;
%Straight Lines [id:da25526492267747836] 
\draw [color={rgb, 255:red, 0; green, 0; blue, 0 }  ,draw opacity=0.41 ]   (232.66,178.01) -- (251.78,140.13) ;
\draw [shift={(252.68,138.34)}, rotate = 116.78] [color={rgb, 255:red, 0; green, 0; blue, 0 }  ,draw opacity=0.41 ][line width=0.75]    (10.93,-3.29) .. controls (6.95,-1.4) and (3.31,-0.3) .. (0,0) .. controls (3.31,0.3) and (6.95,1.4) .. (10.93,3.29)   ;
%Straight Lines [id:da00970045585010526] 
\draw    (20.67,146) -- (116.68,146) ;
\draw [shift={(68.67,146)}, rotate = 180] [fill={rgb, 255:red, 0; green, 0; blue, 0 }  ][line width=0.08]  [draw opacity=0] (12,-3) -- (0,0) -- (12,3) -- cycle    ;
%Straight Lines [id:da22731336451330875] 
\draw    (20.67,166) -- (119.68,166) ;
\draw [shift={(70.17,166)}, rotate = 180] [fill={rgb, 255:red, 0; green, 0; blue, 0 }  ][line width=0.08]  [draw opacity=0] (12,-3) -- (0,0) -- (12,3) -- cycle    ;
%Straight Lines [id:da9547860380958502] 
\draw    (21.67,184) -- (117.68,184) ;
\draw [shift={(69.67,184)}, rotate = 180] [fill={rgb, 255:red, 0; green, 0; blue, 0 }  ][line width=0.08]  [draw opacity=0] (12,-3) -- (0,0) -- (12,3) -- cycle    ;
%Straight Lines [id:da1868394895374128] 
\draw    (100,119) -- (121.02,119) ;
\draw [shift={(123.02,119)}, rotate = 180] [color={rgb, 255:red, 0; green, 0; blue, 0 }  ][line width=0.75]    (10.93,-3.29) .. controls (6.95,-1.4) and (3.31,-0.3) .. (0,0) .. controls (3.31,0.3) and (6.95,1.4) .. (10.93,3.29)   ;
%Shape: Wave [id:dp053768706962373125] 
\draw  [color={rgb, 255:red, 248; green, 231; blue, 28 }  ,draw opacity=1 ] (11.33,164.5) .. controls (15.41,175.52) and (19.31,186.01) .. (23.83,186.01) .. controls (28.36,186.01) and (32.26,175.52) .. (36.33,164.5) .. controls (40.41,153.49) and (44.31,143) .. (48.83,143) .. controls (53.36,143) and (57.26,153.49) .. (61.33,164.5) .. controls (65.41,175.52) and (69.31,186.01) .. (73.83,186.01) .. controls (78.36,186.01) and (82.26,175.52) .. (86.33,164.5) .. controls (90.41,153.49) and (94.31,143) .. (98.83,143) .. controls (102.99,143) and (106.62,151.86) .. (110.35,161.85) ;

% Text Node
\draw (232.66,43.94) node [anchor=south] [inner sep=0.75pt]    {$x$};
% Text Node
\draw (159.33,252.74) node [anchor=north east] [inner sep=0.75pt]    {$y$};
% Text Node
\draw (391.02,178.01) node [anchor=west] [inner sep=0.75pt]    {$z$};
% Text Node
\draw (250,135) node [anchor=south west] [inner sep=0.75pt]    {$a$};
% Text Node
\draw (108,88.07) node [anchor=north west][inner sep=0.75pt]    {$\hat{\boldsymbol{k}}$};


\end{tikzpicture}

    \caption{Mie scattering}
\end{figure}

\concept{Mie scattering} is among few examples that can be solved exactly. It studies a sphere made of 
dielectric. The scattered fields at infinity are expanded using spherical functions:
\begin{equation}
    \vb*{E}_\text{sca} = \sum_{n=1}^\infty E_n (\ii a_n \vb*{N}^{(3)}_{e1n} - b_n \vb*{M}^{(3)}_{o1n}), \quad 
    \vb*{H}_\text{sca} = \frac{k}{\omega \mu} \sum_{n=1}^\infty E_n (\ii b_n \vb*{N}^{(3)}_{o1n} + a_n \vb*{M}^{(3)}_{e1n}),
\end{equation}
where the superscript $(3)$ means % TODO
and the input light beam is described by 
\begin{equation}
    \vb*{E}_\text{in} = E_0 \sum_{n=1}^\infty \ii^n \frac{2n+1}{n(n+1)} (\vb*{M}^{(1)}_{o1n} - \vb*{N}^{(1)}_{e1n}),
\end{equation}
the proof of which can be found in \href{../advanced-electrodynsmics-homework/1.pdf}{this homework}.


\end{document}