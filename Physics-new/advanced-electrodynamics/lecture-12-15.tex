\documentclass[hyperref, a4paper]{article}

\usepackage{geometry}
\usepackage{titling}
\usepackage{titlesec}
% No longer needed, since we will use enumitem package
% \usepackage{paralist}
\usepackage{enumitem}
\usepackage{footnote}
\usepackage{enumerate}
\usepackage{amsmath, amssymb, amsthm}
\usepackage{mathtools}
\usepackage{bbm}
\usepackage{cite}
\usepackage{graphicx}
\usepackage{subfigure}
\usepackage{physics}
\usepackage{tensor}
\usepackage{siunitx}
\usepackage[version=4]{mhchem}
\usepackage{tikz}
\usepackage{xcolor}
\usepackage{listings}
\usepackage{autobreak}
\usepackage[ruled, vlined, linesnumbered]{algorithm2e}
\usepackage{nameref,zref-xr}
\zxrsetup{toltxlabel}
\zexternaldocument*[optics-]{../optics/optics}[optics.pdf]
\zexternaldocument*[solid-]{../solid/solid}[solid.pdf]
\usepackage[colorlinks,unicode]{hyperref} % , linkcolor=black, anchorcolor=black, citecolor=black, urlcolor=black, filecolor=black
\usepackage[most]{tcolorbox}
\usepackage{prettyref}

% Page style
\geometry{left=3.18cm,right=3.18cm,top=2.54cm,bottom=2.54cm}
\titlespacing{\paragraph}{0pt}{1pt}{10pt}[20pt]
\setlength{\droptitle}{-5em}
\preauthor{\vspace{-10pt}\begin{center}}
\postauthor{\par\end{center}}

% More compact lists 
%\setlist[itemize]{
%    itemindent=17pt, 
%    leftmargin=1pt,
%    listparindent=\parindent,
%    parsep=0pt,
%}

% Math operators
\DeclareMathOperator{\timeorder}{\mathcal{T}}
\DeclareMathOperator{\diag}{diag}
\DeclareMathOperator{\legpoly}{P}
\DeclareMathOperator{\primevalue}{P}
\DeclareMathOperator{\sgn}{sgn}
\newcommand*{\ii}{\mathrm{i}}
\newcommand*{\ee}{\mathrm{e}}
\newcommand*{\const}{\mathrm{const}}
\newcommand*{\suchthat}{\quad \text{s.t.} \quad}
\newcommand*{\argmin}{\arg\min}
\newcommand*{\argmax}{\arg\max}
\newcommand*{\normalorder}[1]{: #1 :}
\newcommand*{\pair}[1]{\langle #1 \rangle}
\newcommand*{\fd}[1]{\mathcal{D} #1}
\DeclareMathOperator{\bigO}{\mathcal{O}}

% TikZ setting
\usetikzlibrary{arrows,shapes,positioning}
\usetikzlibrary{arrows.meta}
\usetikzlibrary{decorations.markings}
\tikzstyle arrowstyle=[scale=1]
\tikzstyle directed=[postaction={decorate,decoration={markings,
    mark=at position .5 with {\arrow[arrowstyle]{stealth}}}}]
\tikzstyle ray=[directed, thick]
\tikzstyle dot=[anchor=base,fill,circle,inner sep=1pt]

% Algorithm setting
% Julia-style code
\SetKwIF{If}{ElseIf}{Else}{if}{}{elseif}{else}{end}
\SetKwFor{For}{for}{}{end}
\SetKwFor{While}{while}{}{end}
\SetKwProg{Function}{function}{}{end}
\SetArgSty{textnormal}

\newcommand*{\concept}[1]{{\textbf{#1}}}

% Support for tensor double arrows.
\renewcommand{\tensor}[1]{ \stackrel{\leftrightarrow}{\vb*{#1}}}

% Embedded codes
\lstset{basicstyle=\ttfamily,
  showstringspaces=false,
  commentstyle=\color{gray},
  keywordstyle=\color{blue}
}

% Reference formatting
\newrefformat{fig}{Figure~\ref{#1} on page~\pageref{#1}}

% Color boxes
\tcbuselibrary{skins, breakable, theorems}
\newtcbtheorem[number within=section]{warning}{Warning}%
  {colback=orange!5,colframe=orange!65,fonttitle=\bfseries, breakable}{warn}
\newtcbtheorem[number within=section]{note}{Note}%
  {colback=green!5,colframe=green!65,fonttitle=\bfseries, breakable}{note}

\title{Special Relativity by Prof. Kun Din}
\author{Jinyuan Wu}
\date{December 15, 2021}

\begin{document}

\maketitle

\begin{figure}
    \centering
    

\tikzset{every picture/.style={line width=0.75pt}} %set default line width to 0.75pt        

\begin{tikzpicture}[x=0.75pt,y=0.75pt,yscale=-1,xscale=1]
%uncomment if require: \path (0,300); %set diagram left start at 0, and has height of 300

%Straight Lines [id:da8681225059205384] 
\draw    (110,19) -- (110,124.4) ;
%Straight Lines [id:da3990104017619389] 
\draw    (129,19) -- (129,106.4) ;
%Straight Lines [id:da9785605717014227] 
\draw    (254,124.4) -- (110,124.4) ;
%Straight Lines [id:da7468169256628547] 
\draw    (274,106.4) -- (129,106.4) ;
%Straight Lines [id:da9260453532978494] 
\draw    (254,124.4) -- (254,190.4) ;
%Straight Lines [id:da7290989671749033] 
\draw    (274,106.4) -- (274,207.4) ;
%Straight Lines [id:da378131613174463] 
\draw    (254,190.4) -- (109,190.4) ;
%Straight Lines [id:da05126523351792489] 
\draw    (274,207.4) -- (129,207.4) ;
%Straight Lines [id:da1883737138564383] 
\draw    (129,207.4) -- (129,294.79) ;
%Straight Lines [id:da15436358091415991] 
\draw    (109,190.4) -- (109,295.79) ;
%Straight Lines [id:da043041854444031635] 
\draw    (119,258.4) -- (119,216.4) ;
\draw [shift={(119,214.4)}, rotate = 90] [fill={rgb, 255:red, 0; green, 0; blue, 0 }  ][line width=0.08]  [draw opacity=0] (12,-3) -- (0,0) -- (12,3) -- cycle    ;
%Straight Lines [id:da12988045570093476] 
\draw    (119,84.4) -- (119,42.4) ;
\draw [shift={(119,40.4)}, rotate = 90] [fill={rgb, 255:red, 0; green, 0; blue, 0 }  ][line width=0.08]  [draw opacity=0] (12,-3) -- (0,0) -- (12,3) -- cycle    ;
%Straight Lines [id:da20086243101913204] 
\draw    (263,173.4) -- (263,131.4) ;
\draw [shift={(263,129.4)}, rotate = 90] [fill={rgb, 255:red, 0; green, 0; blue, 0 }  ][line width=0.08]  [draw opacity=0] (12,-3) -- (0,0) -- (12,3) -- cycle    ;
%Shape: Circle [id:dp5922438727715782] 
\draw  [draw opacity=0][fill={rgb, 255:red, 255; green, 255; blue, 0 }  ,fill opacity=1 ] (35,114) .. controls (35,109.58) and (38.58,106) .. (43,106) .. controls (47.42,106) and (51,109.58) .. (51,114) .. controls (51,118.42) and (47.42,122) .. (43,122) .. controls (38.58,122) and (35,118.42) .. (35,114) -- cycle ;
%Straight Lines [id:da09861215455143957] 
\draw [color={rgb, 255:red, 255; green, 255; blue, 0 }  ,draw opacity=1 ]   (66,114) -- (306,114) ;
\draw [shift={(186,114)}, rotate = 180] [fill={rgb, 255:red, 255; green, 255; blue, 0 }  ,fill opacity=1 ][line width=0.08]  [draw opacity=0] (12,-3) -- (0,0) -- (12,3) -- cycle    ;
%Straight Lines [id:da11198846212653613] 
\draw [color={rgb, 255:red, 255; green, 255; blue, 0 }  ,draw opacity=1 ]   (306,114) -- (306,198.4) ;
\draw [shift={(306,156.2)}, rotate = 270] [fill={rgb, 255:red, 255; green, 255; blue, 0 }  ,fill opacity=1 ][line width=0.08]  [draw opacity=0] (12,-3) -- (0,0) -- (12,3) -- cycle    ;
%Straight Lines [id:da5280305392195277] 
\draw [color={rgb, 255:red, 255; green, 255; blue, 0 }  ,draw opacity=1 ]   (43,137.4) -- (43,197.4) ;
\draw [shift={(43,167.4)}, rotate = 270] [fill={rgb, 255:red, 255; green, 255; blue, 0 }  ,fill opacity=1 ][line width=0.08]  [draw opacity=0] (12,-3) -- (0,0) -- (12,3) -- cycle    ;
%Straight Lines [id:da24867776121526064] 
\draw [color={rgb, 255:red, 255; green, 255; blue, 0 }  ,draw opacity=1 ]   (43,198.4) -- (306,198.4) ;
\draw [shift={(174.5,198.4)}, rotate = 180] [fill={rgb, 255:red, 255; green, 255; blue, 0 }  ,fill opacity=1 ][line width=0.08]  [draw opacity=0] (12,-3) -- (0,0) -- (12,3) -- cycle    ;




\end{tikzpicture}

    \caption{Light in moving liquid}
    \label{fig:light-liquid}
\end{figure}

So the problem is whether the wave equation of light does transform under Galilean transformation.
An experiment shown in \prettyref{fig:light-liquid} can be used to verify the claim. If Galilean transformation
works well, then the two light beams undergo the following velocities change:
\begin{equation}
    w_+ = \frac{c}{n} + v, \quad w_- = \frac{c}{n} - v,
\end{equation}
where $v$ is the speed of the liquid. What was actually observed, however, was 
\begin{equation}
    w_+ = \frac{c}{n} + v \left( 1 - \frac{1}{n^2 } \right).
    \label{eq:wplus}
\end{equation}

\eqref{eq:wplus} means things are more complicated than a pure Galilean transformation.
This was not considered as a big problem, though, because at that time the only wave phenomenon observed 
was mechanical wave. Therefore, \eqref{eq:wplus} was initially thought as a piece of evidence for 
\concept{Luminiferous aether}, something that bore light and could have interaction with ordinary materials.

The claim of existence of aether, nonetheless, was rejected by the Michelson–Morley experiment.

Lorentz found that if the transformation (now named \concept{Lorentz transformation})
\begin{equation}
    \pmqty{c t' \\ z'} = \pmqty{\gamma & - \beta \gamma \\ - \beta & \gamma} \pmqty{ct \\ z}, 
    \quad \beta = \frac{v}{c}, \quad \gamma = \frac{1}{\sqrt{1 - \beta^2}}
    \label{eq:lorentz}
\end{equation}
was the correct rule for light, then everything was find. We can verify that 
\[
    \pdv{t} = - \beta \gamma c \pdv{z'} + \gamma \pdv{t'},
\] 
\[
    \pdv[2]{t} = \beta^2 \gamma^2 c^2 \pdv[2]{{z'}} - 2 \beta c \gamma^2 \pdv{t'} \pdv{z'} + \gamma^2 \pdv[2]{{t'}},
\]
\[
    \pdv{z} = \gamma =\pdv{z'} - \frac{\beta \gamma}{c} \pdv{t'},
\]
\[
    \pdv[2]{z} = \gamma^2 \pdv[2]{{z'}} - \frac{2 \beta \gamma^2}{c} \pdv{t'} \pdv{z'} + \frac{\beta^2 \gamma^2}{c^2} \pdv[2]{{t'}},
\]
and therefore 
\begin{equation}
    \pdv[2]{z} - \frac{1}{c^2} \pdv[2]{t} = \pdv[2]{{z'}} - \frac{1}{c^2} \pdv[2]{{t'}}. 
    \label{eq:invariant-wave}
\end{equation}
Therefore, the negative result in Michelson–Morley experiment can be explained.

If we accept \eqref{eq:lorentz} to be the correct time-space transformation, we need to define an inner product 
that is invariant under it. Since we have \eqref{eq:invariant-wave}, a wise idea is to define 
\begin{equation}
    \vb{a} \cdot \vb{b} = \vb{a}^\top \pmqty{\dmat{1, -1, -1, -1}} \vb{b}.
\end{equation}
This is called the \concept{Lorentz metric}. Usually, we use $a$ instead of $\vb{a}$ to denote a 4-vector. 
Note that here we need to pay attention to \emph{covariant} and \emph{contravariant} vectors.
A covariant vector transforms like 
\begin{equation}
    a^\mu \to a'^\mu = \Lambda\indices{^\mu_\nu} a_\nu,
\end{equation}
while a contravariant vector transforms like 
\begin{equation}
    a_\mu \to a'_\mu = a_\nu (\Lambda^{-1})\indices{^\nu_\mu}.
\end{equation}
A inner product is always performed between a contravariant vector and a covariant vector.

Now we are able to find all possible Lorentz transformations: they are just matrices that keep the 
Lorentz metric.%
\footnote{To be exact, we just need to keep the Lorentz metrics before and after the transformation 
\emph{congruent}, because a Lorentz transformation can rescale the coordinates. What really matter
are the \emph{inertia indices}.}
A detailed derivation can be found in Jackson Chapter~17. There we will find 
\begin{equation}
    \Lambda\indices{^\mu_\nu} = \pmqty{\gamma & - \gamma \vb{\beta}^\top \\ - \gamma \vb{\beta}^\top & \vb{I}_3 + (\gamma - 1) \frac{\vb{\beta} \vb{\beta}^\top}{\beta^2}},
\end{equation}
where 
\begin{equation}
    \vb{\beta} = (\beta_1, \beta_2, \beta_3).
\end{equation}

Now we rewrite electrodynamics in a relativistic way. We have 
\begin{equation}
    \partial_\mu = \pmqty{ \frac{1}{c} \pdv{t} & \grad }, \quad \partial^\mu = \pmqty{ \frac{1}{c} \pdv{t} \\ - \grad },
\end{equation}
and therefore the continuity equation is 
\begin{equation}
    \partial_\mu j^\mu = 0, \quad j^\mu = (\rho, \vb*{j}).
\end{equation}
We define 
\begin{equation}
    A^\mu = \pmqty{\frac{\phi}{c} \\ \vb*{A}}, 
\end{equation}
and we will find that the Lorentz gauge can now be written in a concise way:
\begin{equation}
    \partial_\mu A^\mu = 0.
\end{equation} 
Under the Lorentz gauge, the wave equations can be rewritten as
\begin{equation}
    \partial^\mu \partial_\mu A^\nu = \mu_0 j^\nu.
\end{equation}

The generalized (i.e. gauge independent) wave equation is 
\begin{equation}
    \partial_\mu (\underbrace{\partial^\mu A^\nu - \partial^\nu A^\mu}_{F^{\mu \nu}}) = \mu_0 j^\nu.
    \label{eq:maxwell}
\end{equation}
We can verify that the components of this equation reduce to Maxwell equations. 
We have 
\[
    F^{0(1,2,3)} = \pdv{(ct)} A^{(1,2,3)} - \left( - \pdv{x} (c^{-1} \phi) \right) = - \frac{1}{c} E^{(1,2,3)},
\]
\[
    F^{12} = \partial^1 A^2 - \partial^2 A^1 = \left( - \pdv{x} \right) A_y - \left( - \pdv{y} \right) A_x = - B_z
\]
and similarly 
\[
    F^{23} = - B_x, \quad F^{31} = - B_y,
\]
and therefore we find the equation
\[
    \partial_\mu F^{\mu 0} = \mu_0 j^0
\]
is equivalent to 
\[
    \div{(c^{-1} \vb*{E})} = \mu_0 \rho c,
\]
which is the first Maxwell equation. The equation
\[
    \partial_\mu F^{\mu 1} = \mu_0 j^1
\]
is equivalent to 
\[
    \pdv{(ct)} (- c^{-1} E_x) + \pdv{y} B_z - \pdv{z} B_y = \mu_0 j_x,
\]
which is the fourth Maxwell equation. The other two Maxwell equations - two the sourceless equations - cannot 
be derived from \eqref{eq:maxwell}. They actually come from the fact that $F^{\mu \nu}$ is generated from 
antisymmetrized $\partial^\mu A^\nu$ and the consequent Bianchi identity. Therefore, \eqref{eq:maxwell} is 
actually not exactly equivalent to all Maxwell equations. The tensor 
\begin{equation}
    F^{\mu \nu} = \pmqty{ 0 & - c^{-1} E_x & - c^{-1} E_y & - c^{-1} E_z \\
    c^{-1} E_x & 0 & - B_z & B_y \\
    c^{-1} E_y & B_z & 0 & - B_x \\
    c^{-1} E_z & - B_y & B_x & 0 }
\end{equation}
or in other words 
\begin{equation}
    F_{\mu \nu} = \pmqty{ 0 & c^{-1} E_x & c^{-1} E_y & c^{-1} E_z \\
    -c^{-1} E_x & 0 & - B_z & B_y \\
    -c^{-1} E_y & B_z & 0 & - B_x \\
    -c^{-1} E_z & - B_y & B_x & 0 }
\end{equation}
is called the \concept{field strength tensor}.

\begin{note*}{}{}
    Matrix representation of tensors may be misleading. For vectors, we can always use a column vector to 
    represent components of a covariant vector and use a row vector to represent components of a contravariant
    vector, but since $T^{\mu \nu}, T_{\mu \nu}, T\indices{^\mu_\nu}$ and $T\indices{_\mu^\nu}$ are all written
    into a rectangular matrix, it is impossible to distinguish which index is covariant or contravariant. 
\end{note*}

We can find that if we define 
\begin{equation}
    \mathcal{F}^{\mu \nu} = \pmqty{ 0 & -B_x & -B_y & -B_z \\
     - B_x & 0 & c^{-1} E_z & - c^{-1} E_y \\
     - B_y & - c^{-1} E_z & 0 & c^{-1} E_x \\
     - B_z & c^{-1} E_y & - c^{-1} E_x & 0},
\end{equation}
the equation 
\begin{equation}
    \partial_\mu \mathcal{F}^{\mu \nu} = 0
\end{equation}
gives the two sourceless Maxwell equations. Now we can find a duality between the electric field and the magnetic
field: when there is no source in the space, we have 
\[
    \partial_\mu \mathcal{F}^{\mu \nu} = \partial_\mu {F}^{\mu \nu} = 0 = 0,
\]
and therefore the transformation 
\begin{equation}
    - c^{-1} E_i \longrightarrow B_i
\end{equation}
leaves every equation the same.

\end{document}