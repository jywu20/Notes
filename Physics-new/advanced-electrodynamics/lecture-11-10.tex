\documentclass[hyperref, a4paper]{article}

\usepackage{geometry}
\usepackage{float}
\usepackage{titling}
\usepackage{titlesec}
% No longer needed, since we will use enumitem package
% \usepackage{paralist}
\usepackage{enumitem}
\usepackage{footnote}
\usepackage{enumerate}
\usepackage{amsmath, amssymb, amsthm}
\usepackage{mathtools}
\usepackage{bbm}
\usepackage{cite}
\usepackage{graphicx}
\usepackage{subcaption}
\usepackage{physics}
\usepackage{tensor}
\usepackage{siunitx}
\usepackage{booktabs}
\usepackage[version=4]{mhchem}
\usepackage{tikz}
\usepackage{xcolor}
\usepackage{listings}
\usepackage{autobreak}
\usepackage[ruled, vlined, linesnumbered]{algorithm2e}
\usepackage{nameref,zref-xr}
\zxrsetup{toltxlabel}
%\zexternaldocument*[prev-]{./lecture-10-20}[lecture-10-20.pdf]
\zexternaldocument*[solid-]{../solid/solid}[solid.pdf]
\zexternaldocument*[prev-]{./lecture-11-3}[lecture-11-3.pdf]
\usepackage[colorlinks,unicode]{hyperref} % , linkcolor=black, anchorcolor=black, citecolor=black, urlcolor=black, filecolor=black
\usepackage{prettyref}

% Page style
\geometry{left=3.18cm,right=3.18cm,top=2.54cm,bottom=2.54cm}
\titlespacing{\paragraph}{0pt}{1pt}{10pt}[20pt]
\setlength{\droptitle}{-5em}
\preauthor{\vspace{-10pt}\begin{center}}
\postauthor{\par\end{center}}

% More compact lists 
\setlist[itemize]{itemindent=17pt, leftmargin=1pt}

% Math operators
\DeclareMathOperator{\timeorder}{\mathcal{T}}
\DeclareMathOperator{\diag}{diag}
\DeclareMathOperator{\legpoly}{P}
\DeclareMathOperator{\primevalue}{P}
\DeclareMathOperator{\sgn}{sgn}
\newcommand*{\ii}{\mathrm{i}}
\newcommand*{\ee}{\mathrm{e}}
\newcommand*{\const}{\mathrm{const}}
\newcommand*{\suchthat}{\quad \text{s.t.} \quad}
\newcommand*{\argmin}{\arg\min}
\newcommand*{\argmax}{\arg\max}
\newcommand*{\normalorder}[1]{: #1 :}
\newcommand*{\pair}[1]{\langle #1 \rangle}
\newcommand*{\fd}[1]{\mathcal{D} #1}
\DeclareMathOperator{\bigO}{\mathcal{O}}
\DeclareMathOperator{\object}{Ob}
\DeclareMathOperator{\morphism}{Hom}

% Support for tensor double arrows.
\renewcommand{\tensor}[1]{ \stackrel{\leftrightarrow}{\vb*{#1}}}

% TikZ setting
\usetikzlibrary{arrows,shapes,positioning}
\usetikzlibrary{arrows.meta}
\usetikzlibrary{decorations.markings}
\tikzstyle arrowstyle=[scale=1]
\tikzstyle directed=[postaction={decorate,decoration={markings,
    mark=at position .5 with {\arrow[arrowstyle]{stealth}}}}]
\tikzstyle ray=[directed, thick]
\tikzstyle dot=[anchor=base,fill,circle,inner sep=1pt]

% Algorithm setting
% Julia-style code
\SetKwIF{If}{ElseIf}{Else}{if}{}{elseif}{else}{end}
\SetKwFor{For}{for}{}{end}
\SetKwFor{While}{while}{}{end}
\SetKwProg{Function}{function}{}{end}
\SetArgSty{textnormal}

\newcommand*{\concept}[1]{{\textbf{#1}}}

\newrefformat{fig}{Figure~\ref{#1}}

\newcommand{\soliddoc}{\href{../solid/solid}{the solid state physics note}}
\newcommand{\prevdoc}{\href{./lecture-11-3}{the previous lecture}}

% Embedded codes
\lstset{basicstyle=\ttfamily,
  showstringspaces=false,
  commentstyle=\color{gray},
  keywordstyle=\color{blue}
}

\allowdisplaybreaks

\title{Green Function in Electrodynamics by Prof. Kun Din}
\author{Jinyuan Wu}

\begin{document}

\maketitle

This article is a lecture note of Prof. Kun Ding's lecture on Advanced Electrodynamics on 10 November, 2021.

\section{Scalar Green function and density of states}

We have already known that Green functions relate the differential form of a certain equation of motion to its
integral form, that in electrodynamics the scalar Green function (i.e. the retarded potential) can be used to
derive the vector Green function, and that many quantities like the $- \vb*{p} / 3\epsilon_0$ electric field 
that are usually derived using heuristic way can be derived from the vector Green function directly.

Here we show another way to derive the scalar Green function. We will derive the scalar Green function with its spacial variables in the position space and its temporal variable in the frequency domain.
Suppose
\begin{equation}
    k = \frac{\omega}{c},
\end{equation}
we have 
\[
    \begin{aligned}
        G(\vb*{R}) &= \int \frac{\dd[3]{\vb*{q}}}{(2\pi)^3} \frac{\ee^{\ii \vb*{q} \cdot \vb*{R}}}{{\vb*{q}}^2 - k^2 \pm \ii 0^+} \\
        &= \frac{1}{(2\pi)^3} \int_0^\infty q^2 \dd{q} \int_0^\pi \sin \theta \dd{\theta} \int_0^\pi \dd{\phi}
        \frac{\ee^{\ii q R \cos \theta}}{q^2 - k^2 + \ii 0^+} \\
        &= \frac{1}{4 \pi^2 \ii R} \int_0^\infty \frac{q (\ee^{\ii q R} - \ee^{- \ii q R})}{q^2 - k^2 + \ii 0^+} \\
        &= \frac{\ee^{\mp \ii (k \mp \ii 0^+) R}}{4 \pi R}.
    \end{aligned}
\]
The retarded wave is 
\begin{equation}
    G(\vb*{R}) = \frac{\ee^{\ii k R}}{4 \pi R}.
    \label{eq:retarded-green}
\end{equation}
This is exactly \eqref{prev-eq:scalar-green} in \prevdoc.
\eqref{eq:retarded-green} is an \emph{off-shell} propagator as $\vb*{q}$ may not satisfy the 
momentum-energy relation $\vb*{q}^2 - k^2 = 0$. It need not to be on-shell, because a Green function 
describes how the electromagnetic field responds to an external stimulation, not how electromagnetic
waves propagate freely.

We know (in quantum field theories) that 
\[
    \frac{1}{\vb*{q}^2 - k^2 - \ii 0^+} = \primevalue \frac{1}{\vb*{q}^2 - k^2} + \ii \pi \delta(\vb*{q}^2 - k^2),
\]
and we can connect \eqref{eq:retarded-green} with the density of states. The definition of density of states 
for a generalized system whose dispersion relation is like $\abs*{\vb*{q}}^2 = k^2$ is 
\begin{equation}
    \rho(\omega) = \dv{k^2(\omega)}{\omega} \int \frac{\dd[3]{\vb*{q}}}{(2\pi)^3} \delta(k^2(\omega) - \abs*{\vb*{q}}^2).
    \label{eq:density-of-states-light}
\end{equation}
This can be derived using the following argument. Suppose $\dd{S}$ is the area element of the isoenergetic surface
in the momentum space, and there are $\dd{S} \dd{q_\bot} / ((2\pi)^3 / V)$ states in the volume element 
$\dd{S} \dd{q_\bot}$ in the momentum space. Therefore, the density of states - the number of states per energy unit
per volume unit - is 
\[
    \begin{aligned}
        \sum \frac{\dd{S} \dd{q_\bot} / ((2\pi)^3 / V)}{V \dd{\omega}} &= \int \frac{\dd{S} \dd{q_\bot}}{(2\pi)^3 \dd{\omega}} \\
        &=  \int \frac{\dd{S} }{(2\pi)^3 \dd{\omega}} \frac{\dd{(k^2 - \vb*{q}^2)}}{\abs*{\grad_{\vb*{q}} (k^2 - \vb*{q}^2)}} \\
        &= \dv{k^2}{\omega} \int \frac{\dd{S}}{(2\pi)^3} \frac{1}{\abs*{\grad_{\vb*{q}} (k^2 - \vb*{q}^2)}} \\
        &= \dv{k^2}{\omega} \int \frac{\dd[3]{\vb*{q}}}{(2\pi)^3} \delta(k^2 - \vb*{q}^2).
    \end{aligned}
\]
Then we find the relation between the imaginary part of the Green function and the density of states:
\begin{equation}
    \rho(\omega) = \dv{k(\omega)^2}{\omega} \frac{1}{\pi} \Im G(\vb*{R} = 0, \omega).
    \label{eq:density-of-state-from-green}
\end{equation}

\eqref{eq:density-of-state-from-green} is actually more general than the case of homogeneous material 
we have just discussed. For a non-homogeneous, it can be generalized to connect \emph{local density of states} 
with the Green function.

\section{The on-shell theory of Green function}

\eqref{eq:density-of-state-from-green} involves something that is proportional to $\delta(\vb*{q}^2 - p^2)$,
which means the imaginary part of the Green function is given by a \emph{on-shell} theory, where the input 
momentum $\vb*{q}$ is on-shell. Now we discuss how to calculate the Green function only with on-shell momentum. 
We define 
\begin{equation}
    k_z = \sqrt{k^2 - q_x^2 - q_y^2},
    \label{eq:kz-on-shell}
\end{equation}
and we have 
\[
    \begin{aligned}
        \frac{\ee^{\ii k R}}{4 \pi R} = G(\vb*{r}) &= 
        \int \frac{\dd{q_x} \dd{q_y} \dd{q_z}}{(2\pi)^3} \frac{\ee^{\ii q_x x + \ii q_y y} \ee^{\ii q_z z}}{(q_z - k_z) (q_z + k_z) - \ii 0^+} \\
        &= \frac{1}{(2\pi)^3} \int \dd{q_x} \int \dd{q_y} 2 \pi \ii \frac{\ee^{\ii (q_x x + q_y y)} \ee^{\ii k_z z}}{2 k_z}
    \end{aligned}
\]
when $z > 0$. The case when $z < 0$ can be checked to be 
\[
    \frac{\ee^{\ii k R}}{4 \pi R} = \frac{1}{(2\pi)^3} \int \dd{q_x} \int \dd{q_y} 2 \pi \ii \frac{\ee^{\ii (q_x x + q_y y)} \ee^{- \ii k_z z}}{2 k_z}.
\]
So in the end we have 
\begin{equation}
    \frac{\ee^{\ii k R}}{R} = \frac{\ii}{2\pi} \int \dd[2]{\vb*{q}_\bot} \frac{\ee^{\ii \vb*{q}_\bot \cdot \vb*{r}_\bot} \ee^{\ii k_z \abs*{z}}}{k_z},
    \label{eq:angular-spectrum}
\end{equation}
where $\vb*{q}_\bot$ and $\vb*{r}_\bot$ are $\vb*{q}$ and $\vb*{r}$'s projection on the $xy$ plane.
\eqref{eq:angular-spectrum} is called the \concept{angular spectrum decomposition} or \concept{Weyl decomposition}.
We can see the RHS of \eqref{eq:angular-spectrum} consists of all possible on-shell plane waves. 
This can be seen as a spacial version of Duhamel's principle, where the electromagnetic field created by an 
external point charge can be equivalently seen as the decomposition of several electromagnetic waves in free space.
In the quantum context, it means after a scattering event, the output states are decomposition of on-shell 
particle states. Note that \eqref{eq:angular-spectrum} contains \emph{evanescent waves}: $q_x$ and $q_y$ 
are integrated over all possible values, which means $k_z$ may be imaginary. Note that we require 
\begin{equation}
    \Re k_z > 0, \quad \Im k_z > 0,
\end{equation}
to get outward waves. 

\eqref{eq:angular-spectrum} can be further simplified. Suppose 
\[
    \vb*{q}_\bot = q_\bot (\cos \alpha, \sin \alpha), \quad \vb*{r}_\bot = r_\bot (\cos \phi, \sin \phi),
\]
we have 
\[
    \frac{\ee^{\ii k R}}{R} = \frac{\ii}{2\pi} \int q_\bot \dd{q_\bot} \int \dd{\alpha} \frac{\ee^{\ii q_\bot r_\bot \cos(\alpha - \phi)} \ee^{\ii k_z \abs*{z}}}{k_z},
\]
and since 
\[
    J_0(x) = \frac{1}{2\pi} \int_0^{2\pi} \dd{\alpha} \ee^{\ii x \cos(\alpha - \phi)},
\]
we have 
\begin{equation}
    \frac{\ee^{\ii k R}}{R} = \ii \int \dd{q_\bot} \frac{q_\bot}{k_z} J_0(q_\bot r_\bot) \ee^{\ii k_z \abs*{z}}.
    \label{eq:sommerfield-id}
\end{equation}
This is called the \concept{Sommerfield identity}.

\eqref{eq:angular-spectrum} and \eqref{eq:sommerfield-id} are all on-shell versions of the scalar Green function, as we have constraints like \eqref{eq:kz-on-shell} so that $k_z$ cannot vary freely.

\section{On-shell properties of the vector Green function}

Take the second order derivative of \eqref{eq:angular-spectrum}, we have 
\[
    \begin{aligned}
        \pdv[2]{z} G(\vb*{r}) &= 
        \left( - \frac{\ii}{4\pi} \int \dd[2]{\vb*{q}_\bot} \frac{k_z}{2} \ee^{\ii q_\bot r_\bot} \ee^{\ii k_z \abs*{z}} \right) \sgn(z)^2
        + \left( - \frac{1}{4 \pi^2} \int \dd[2]{\vb*{q}_\bot} \ee^{\ii q_\bot r_\bot} \ee^{\ii k_z \abs*{z}} \right) \delta(z) \\
        &= - \delta^{(3)}(\vb*{r}) - \frac{\ii}{4\pi^2} \int \dd[2]{\vb*{q}_\bot} \frac{k_z}{2} \ee^{\ii q_\bot r_\bot} \ee^{\ii k_z \abs*{z}}.
    \end{aligned}
\]
Repeating this process we get 
\begin{equation}
    \grad \grad G(\vb*{r}) = - \frac{\vu*{z} \vu*{z}}{k^2} \delta(\vb*{r}) + \frac{1}{8 \pi^2} \int \frac{\dd[2]{\vb*{q}_\bot}}{k_z} 
    \begin{cases}
        \left( \tensor{\vb*{I}} - \frac{\vb*{k}_+ \vb*{k}_+}{k^2} \right) \ee^{\ii \vb*{k}_+ \cdot \vb*{r}}, \quad z > 0, \\
        \left( \tensor{\vb*{I}} - \frac{\vb*{k}_- \vb*{k}_-}{k^2} \right) \ee^{\ii \vb*{k}_- \cdot \vb*{r}}, \quad z < 0,
    \end{cases}
\end{equation}
where 
\begin{equation}
    \vb*{k}_+ = (q_x, q_y, k_z), \quad \vb*{k}_- = (q_x, q_y, - k_z).
\end{equation}
We define 
\begin{equation}
    \vu*{e} = \frac{\vb*{k} \times \vu*{z}}{\abs*{\vb*{k} \times \vu*{z}}}, 
    \quad \vu*{h} = \frac{\vb*{k} \times (\vb*{k} \times \vu*{z})}{\abs*{\vb*{k} \times (\vb*{k} \times \vu*{z})}},
    \quad \vu*{l} = \frac{\vb*{k}}{\abs*{\vb*{k}}},
\end{equation}
and it can be seen that they form a complete and orthogonal basis of $\mathbb{R}^3$. 
We then find that 
\begin{equation}
    \tensor{\vb*{I}} - \frac{\vb*{k}_- \vb*{k}_-}{k^2} = \vu*{e} \vu*{e} + \vu*{h} \vu*{h},
\end{equation}

\section{Mode expansion of the Green function}

From 
\[
    (\mathcal{L} - \lambda) G(\vb*{r} - \vb*{r}') = \delta(\vb*{r} - \vb*{r}')
\]
we have 
\begin{equation}
    G(\vb*{r} - \vb*{r}') = \sum_n \frac{u^*_n(\vb*{r}') u_n(\vb*{r})}{\lambda_n - \lambda}.
\end{equation}

\end{document}