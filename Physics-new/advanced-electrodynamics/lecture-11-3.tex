\documentclass[hyperref, a4paper]{article}

\usepackage{geometry}
\usepackage{float}
\usepackage{titling}
\usepackage{titlesec}
% No longer needed, since we will use enumitem package
% \usepackage{paralist}
\usepackage{enumitem}
\usepackage{footnote}
\usepackage{enumerate}
\usepackage{amsmath, amssymb, amsthm}
\usepackage{mathtools}
\usepackage{bbm}
\usepackage{cite}
\usepackage{graphicx}
\usepackage{subcaption}
\usepackage{physics}
\usepackage{tensor}
\usepackage{siunitx}
\usepackage{booktabs}
\usepackage[version=4]{mhchem}
\usepackage{tikz}
\usepackage{xcolor}
\usepackage{listings}
\usepackage{autobreak}
\usepackage[ruled, vlined, linesnumbered]{algorithm2e}
\usepackage{nameref,zref-xr}
\zxrsetup{toltxlabel}
\zexternaldocument*[prev-]{./lecture-10-20}[lecture-10-20.pdf]
\zexternaldocument*[solid-]{../solid/solid}[solid.pdf]
\usepackage[colorlinks,unicode]{hyperref} % , linkcolor=black, anchorcolor=black, citecolor=black, urlcolor=black, filecolor=black
\usepackage{prettyref}

% Page style
\geometry{left=3.18cm,right=3.18cm,top=2.54cm,bottom=2.54cm}
\titlespacing{\paragraph}{0pt}{1pt}{10pt}[20pt]
\setlength{\droptitle}{-5em}
\preauthor{\vspace{-10pt}\begin{center}}
\postauthor{\par\end{center}}

% More compact lists 
\setlist[itemize]{itemindent=17pt, leftmargin=1pt}

% Math operators
\DeclareMathOperator{\timeorder}{\mathcal{T}}
\DeclareMathOperator{\diag}{diag}
\DeclareMathOperator{\legpoly}{P}
\DeclareMathOperator{\primevalue}{P}
\DeclareMathOperator{\sgn}{sgn}
\newcommand*{\ii}{\mathrm{i}}
\newcommand*{\ee}{\mathrm{e}}
\newcommand*{\const}{\mathrm{const}}
\newcommand*{\suchthat}{\quad \text{s.t.} \quad}
\newcommand*{\argmin}{\arg\min}
\newcommand*{\argmax}{\arg\max}
\newcommand*{\normalorder}[1]{: #1 :}
\newcommand*{\pair}[1]{\langle #1 \rangle}
\newcommand*{\fd}[1]{\mathcal{D} #1}
\DeclareMathOperator{\bigO}{\mathcal{O}}
\DeclareMathOperator{\object}{Ob}
\DeclareMathOperator{\morphism}{Hom}

% Support for tensor double arrows.
\renewcommand{\tensor}[1]{ \stackrel{\leftrightarrow}{\vb*{#1}}}

% TikZ setting
\usetikzlibrary{arrows,shapes,positioning}
\usetikzlibrary{arrows.meta}
\usetikzlibrary{decorations.markings}
\tikzstyle arrowstyle=[scale=1]
\tikzstyle directed=[postaction={decorate,decoration={markings,
    mark=at position .5 with {\arrow[arrowstyle]{stealth}}}}]
\tikzstyle ray=[directed, thick]
\tikzstyle dot=[anchor=base,fill,circle,inner sep=1pt]

% Algorithm setting
% Julia-style code
\SetKwIF{If}{ElseIf}{Else}{if}{}{elseif}{else}{end}
\SetKwFor{For}{for}{}{end}
\SetKwFor{While}{while}{}{end}
\SetKwProg{Function}{function}{}{end}
\SetArgSty{textnormal}

\newcommand*{\concept}[1]{{\textbf{#1}}}

\newrefformat{fig}{Figure~\ref{#1}}

\newcommand{\soliddoc}{\href{../solid/solid}{the solid state physics note}}

% Embedded codes
\lstset{basicstyle=\ttfamily,
  showstringspaces=false,
  commentstyle=\color{gray},
  keywordstyle=\color{blue}
}

\title{Green Function in Electrodynamics by Prof. Kun Din}
\author{Jinyuan Wu}

\begin{document}

\maketitle

This article is a lecture note of Prof. Kun Ding's lecture on Advanced Electrodynamics on 3 November, 2021.

\section{Green functions}

After switching to the frequency domain in the temporal direction (maybe together with some spacial dimensions), 
an linear equation with external stimulation is in the general form of 
\begin{equation}
    (\mathcal{L} - \lambda \rho(\vb*{r})) u(\vb*{r}) = f(\vb*{r}),
    \label{eq:general-linear}
\end{equation}
where $\mathcal{L}$ is a linear operator. The normal modes can be obtained by the generalized eigenvalue problem
\begin{equation}
    (\mathcal{L} - \lambda \rho(\vb*{r})) u(\vb*{r}) = 0,
\end{equation}
which is \eqref{eq:general-linear} without the external field.

We define the \concept{Green function} as 
\begin{equation}
    (\mathcal{L} - \lambda \rho(\vb*{r})) G(\vb*{r}, \vb*{r}') = \delta(\vb*{r} - \vb*{r}').
    \label{eq:green-def}
\end{equation} 
When the system has spacial translational symmetry, we have $G(\vb*{r} - \vb*{r}') = G(\vb*{r} - \vb*{r}')$.
Once the Green function is obtained, the result of the stimulation can, in principle, be obtained by convolution.
The existence of Green function can be proved using normal modes or eigenstates of \eqref{eq:general-linear}, which is also a general way to actually calculate the Green function.

If \eqref{eq:general-linear} is a vector equation, the Green function is a second-rank tensor. 
\eqref{eq:green-def}, in this case, should be written as 
\begin{equation}
    (\mathcal{L} - \lambda \rho(\vb*{r})) \tensor{G}(\vb*{r}, \vb*{r}') = \tensor{I} \delta(\vb*{r} - \vb*{r}').
\end{equation}

There are a large variety interesting problems stemming from \eqref{eq:general-linear}.
An important problem is the \concept{reverse problem}: how can we decide $\rho(\vb*{r})$ with known $u(\vb*{r})$ and $f(\vb*{r})$?
The problem is involved in CT, seismology, material characterization (for example, determine the structure of a sample using scattering experiments).
In physics the most important problem is how to accurately calculate the Green function.

\section{Examples of Green functions in electrodynamics}

In electrostatics the equation is 
\begin{equation}
    \laplacian \phi = - \frac{1}{\epsilon_0} \rho(\vb*{r}),
\end{equation}
and the Green function is just 
\begin{equation}
    G(\vb*{r} - \vb*{r}') = \frac{1}{4 \pi \epsilon_0} \frac{1}{\abs*{\vb*{r} - \vb*{r}'}},
\end{equation}
which is the potential around a test charge.
The $1/r$ Green function corresponds to the Laplacian operator $\laplacian$.

\end{document}