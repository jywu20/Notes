\documentclass[hyperref, a4paper]{article}

\usepackage{geometry}
\usepackage{marginnote}
\usepackage{titling}
\usepackage{titlesec}
% No longer needed, since we will use enumitem package
% \usepackage{paralist}
\usepackage{enumitem}
\usepackage{footnote}
\usepackage{enumerate}
\usepackage{amsmath, amssymb, amsthm}
\usepackage{mathtools}
\usepackage{bbm}
\usepackage{cite}
\usepackage{graphicx}
\usepackage{subfigure}
\usepackage{physics}
\usepackage{tensor}
\usepackage{siunitx}
\usepackage{slashed}
\usepackage{centernot}
\usepackage[version=4]{mhchem}
\usepackage{tikz}
\usepackage{xcolor}
\usepackage{listings}
\usepackage{autobreak}
\usepackage[ruled, vlined, linesnumbered]{algorithm2e}
\usepackage{nameref,zref-xr}
\zxrsetup{toltxlabel}
\zexternaldocument*[solid-]{../solid/solid}[solid.pdf]
\zexternaldocument*[electrongas-]{../band-metal-insulator/electron-gas}[electron-gas.pdf]
\usepackage[colorlinks,unicode]{hyperref} % , linkcolor=black, anchorcolor=black, citecolor=black, urlcolor=black, filecolor=black
\usepackage[most]{tcolorbox}
\usepackage{prettyref}

% Page style
\geometry{left=3.18cm,right=3.18cm,top=2.54cm,bottom=2.54cm}
\titlespacing{\paragraph}{0pt}{1pt}{10pt}[20pt]
\setlength{\droptitle}{-5em}
\preauthor{\vspace{-10pt}\begin{center}}
\postauthor{\par\end{center}}

% More compact lists 
%\setlist[itemize]{
%    itemindent=17pt, 
%    leftmargin=1pt,
%    listparindent=\parindent,
%    parsep=0pt,
%}

% Math operators
\DeclareMathOperator{\timeorder}{\mathcal{T}}
\DeclareMathOperator{\diag}{diag}
\DeclareMathOperator{\legpoly}{P}
\DeclareMathOperator{\primevalue}{P}
\DeclareMathOperator{\sgn}{sgn}
\newcommand*{\ii}{\mathrm{i}}
\newcommand*{\ee}{\mathrm{e}}
\newcommand*{\const}{\mathrm{const}}
\newcommand*{\suchthat}{\quad \text{s.t.} \quad}
\newcommand*{\argmin}{\arg\min}
\newcommand*{\argmax}{\arg\max}
\newcommand*{\normalorder}[1]{: #1 :}
\newcommand*{\pair}[1]{\langle #1 \rangle}
\newcommand*{\fd}[1]{\mathcal{D} #1}
\DeclareMathOperator{\bigO}{\mathcal{O}}

% Feynman slash
\newcommand{\fsl}[1]{{\centernot{#1}}}

% TikZ setting
\usetikzlibrary{arrows,shapes,positioning}
\usetikzlibrary{arrows.meta}
\usetikzlibrary{decorations.markings}
\tikzstyle arrowstyle=[scale=1]
\tikzstyle directed=[postaction={decorate,decoration={markings,
    mark=at position .5 with {\arrow[arrowstyle]{stealth}}}}]
\tikzstyle ray=[directed, thick]
\tikzstyle dot=[anchor=base,fill,circle,inner sep=1pt]

% Algorithm setting
% Julia-style code
\SetKwIF{If}{ElseIf}{Else}{if}{}{elseif}{else}{end}
\SetKwFor{For}{for}{}{end}
\SetKwFor{While}{while}{}{end}
\SetKwProg{Function}{function}{}{end}
\SetArgSty{textnormal}

\newcommand*{\concept}[1]{{\textbf{#1}}}

% Embedded codes
\lstset{basicstyle=\ttfamily,
  showstringspaces=false,
  commentstyle=\color{gray},
  keywordstyle=\color{blue}
}

\newcommand{\soliddoc}{\href{../solid/solid.pdf}{this solid state physics note}}

% Reference formatting
\newrefformat{fig}{Figure~\ref{#1}}

% Color boxes
\tcbuselibrary{skins, breakable, theorems}
\newtcbtheorem[number within=section]{warning}{Warning}%
  {colback=orange!5,colframe=orange!65,fonttitle=\bfseries, breakable}{warn}
\newtcbtheorem[number within=section]{note}{Note}%
  {colback=green!5,colframe=green!65,fonttitle=\bfseries, breakable}{note}
\newtcbtheorem[number within=section]{info}{Info}%
  {colback=blue!5,colframe=blue!65,fonttitle=\bfseries, breakable}{info}

\title{Bosonization in Electronic Systems}
\author{Jinyuan Wu}

\begin{document}
    
\maketitle

\section{The physical picture}

We have already discussed bosonization in Section.~\ref{solid-sec:one-dim-free} in \soliddoc.
Facts like that particle-hole pairs in 1D are much simpler than the case in higher dimensions and that the density-density correlation function have two clear poles, which means we have two bosonic density modes 
all hint the possibility of bosonization.
The first fact is the argument used in \soliddoc. Here we briefly discuss the second argument.
It is well known that the retarded density-density Green function is 
\begin{equation} \marginnote{Fradkin Eq.~(6.7)}
    D^\text{R}(\vb*{q}, \omega) = \int \frac{\dd[d]{\vb*{p}}}{(2\pi)^d} \frac{n_{\vb*{p}} - n_{\vb*{p} + \vb*{q}}}{\omega - \xi_{\vb*{p} + \vb*{q}} + \xi_{\vb*{p}} + \ii 0^+},
\end{equation}
and when $\vb*{q}$ is small, we have 
\[
    \xi_{\vb*{p} + \vb*{q}} - \xi_{\vb*{p}} \approx \vb*{q} \cdot \grad_{\vb*{p}} \xi_{\vb*{p}} = \vb*{q} \cdot \vb*{v},
\]
and 
\[
    n_{\vb*{p}} - n_{\vb*{p} + \vb*{q}} \approx - \vb*{q} \cdot \grad_{\vb*{p}}{n_{\vb*{p}}} = 
    - \vb*{q} \cdot \grad_{\vb*{p}} \theta(\mu - \epsilon_{\vb*{p}}) \approx \vb*{q} \cdot \vb*{v}_\text{F} \delta(- \xi_{\vb*{p}}),
\]
and therefore 
\begin{equation} \marginnote{Fradkin Eq.~(6.10), Wen below Eq.~(4.3.1)}
    \begin{aligned}
        D^\text{R}(\vb*{q}, \omega) &= \int \frac{\dd[d]{\vb*{p}}}{(2\pi)^d} \delta(-\xi_{\vb*{p}}) \frac{\vb*{q} \cdot \vb*{v}_\text{F}}{\omega - \vb*{q} \cdot \vb*{v} + \ii 0^+} \\
        &= \int_\text{FS} \frac{\dd[d-1]{S}}{(2\pi)^d} \frac{1}{v_\text{F}} \frac{\vb*{q} \cdot \vb*{v}_\text{F}}{\omega - \vb*{q} \cdot \vb*{v}_\text{F} + \ii 0^+}  \\
        &= \int_\text{FS} \frac{\dd[d-1]{S}}{(2\pi)^d} \frac{\vb*{q} \cdot \vu*{v}_\text{F}}{\omega - \vb*{q} \cdot \vb*{v}_\text{F} + \ii 0^+} \\
        &\sim \frac{p_\text{F}^{d-1}}{(2\pi)^d} \int_\text{FS} \dd[d-1]{\Omega} \frac{\vb*{q} \cdot \vu*{v}_\text{F}}{\omega - \vb*{q} \cdot \vb*{v}_\text{F} + \ii 0^+} ,
    \end{aligned}
\end{equation}
where FS means the Fermi surface. For a 1D electron gas we therefore get 
\begin{equation} \marginnote{Fradkin Eq.~(6.10), Wen Eq.~(4.3.2)}
    D^\text{R} \sim \frac{q}{2\pi} \left( \frac{1}{\omega - q v_\text{F} + \ii 0^+} - \frac{1}{\omega + q v_\text{F} + \ii 0^+} \right). 
\end{equation}
So we get two poles, instead of branch cuts. 

This article discusses some details in bosonization, summarizing important facts in famous textbooks and papers.
Bosonization can be viewed as one way to exactly solve a system, which we will see in Luttinger liquid and
bosonization of spin systems. A more ``physical'' motivation is to capture the bosonic modes in the system,
i.e. density and current fluctuations, or since we usually only deal with low-energy perturbations in 
condensed matter systems, to capture \emph{sound waves} \cite{Tomonaga1950RemarksOB,10.1143/PTPS.170.185}. \marginnote{Philip Phillips Chapter 10}
This is why the fields after bosonization are all density modes: We really do not know what to do other than
this. Sometimes, bosonization are known as the \emph{hydrodynamical approach} (for example in Wen's famous 
textbook), which reveals its physical nature. This name is slightly misleading, as \emph{hydrodynamical approaches}
often denote more ``kinetic'' approaches, which is more about deriving EOMs of expectation values instead of 
\marginnote{Wen Section 5.1.3, 5.3.3 and 5.3.4}
quantum field operators. This physical picture can also be justified using the following line of thinking: 
one motivation to study the electronic structure of a condensed matter system is to find its electromagnetic 
response, and $A_\mu$ is always coupled to a bilinear of the fermionic fields. What can be revealed directly 
by electromagnetic responses, therefore, are only bosonic modes (for example see Section~\ref{electrongas-sec:boson-modes} \href{../band-metal-insulator/electron-gas.pdf}{here}).

\begin{figure}
    \centering
    

\tikzset{every picture/.style={line width=0.75pt}} %set default line width to 0.75pt        

\begin{tikzpicture}[x=0.75pt,y=0.75pt,yscale=-0.85,xscale=0.85]
%uncomment if require: \path (0,300); %set diagram left start at 0, and has height of 300

%Straight Lines [id:da009219527738367317] 
\draw    (405,225) -- (617,225) ;
\draw [shift={(619,225)}, rotate = 180] [fill={rgb, 255:red, 0; green, 0; blue, 0 }  ][line width=0.08]  [draw opacity=0] (12,-3) -- (0,0) -- (12,3) -- cycle    ;
%Straight Lines [id:da2079961540240256] 
\draw    (405,225) -- (405,84.5) ;
\draw [shift={(405,82.5)}, rotate = 90] [fill={rgb, 255:red, 0; green, 0; blue, 0 }  ][line width=0.08]  [draw opacity=0] (12,-3) -- (0,0) -- (12,3) -- cycle    ;
%Straight Lines [id:da25299815998590813] 
\draw  [dash pattern={on 4.5pt off 4.5pt}]  (405,145.5) -- (512,145.5) ;
%Straight Lines [id:da23611716817469208] 
\draw  [dash pattern={on 4.5pt off 4.5pt}]  (512,145.5) -- (512,224.5) ;
%Curve Lines [id:da9173257888464759] 
\draw    (405,145.5) .. controls (500,143.5) and (519,226.5) .. (587,225) ;
%Curve Lines [id:da006174472132643993] 
\draw [draw opacity=0][fill={rgb, 255:red, 74; green, 144; blue, 226 }  ,fill opacity=0.5 ]   (405,146) .. controls (500,144) and (519,227) .. (587,225.5) .. controls (404,224) and (589,225) .. (405,225) .. controls (404,147) and (404,223) .. (405,146) -- cycle ;
%Straight Lines [id:da4367238433084937] 
\draw    (68,225) -- (280,225) ;
\draw [shift={(282,225)}, rotate = 180] [fill={rgb, 255:red, 0; green, 0; blue, 0 }  ][line width=0.08]  [draw opacity=0] (12,-3) -- (0,0) -- (12,3) -- cycle    ;
%Straight Lines [id:da3594146199392185] 
\draw    (68,225) -- (68,84.5) ;
\draw [shift={(68,82.5)}, rotate = 90] [fill={rgb, 255:red, 0; green, 0; blue, 0 }  ][line width=0.08]  [draw opacity=0] (12,-3) -- (0,0) -- (12,3) -- cycle    ;
%Straight Lines [id:da3068774329734081] 
\draw  [dash pattern={on 4.5pt off 4.5pt}]  (68,145.5) -- (175,145.5) ;
%Straight Lines [id:da14782967759901622] 
\draw  [dash pattern={on 4.5pt off 4.5pt}]  (175,145.5) -- (175,224.5) ;
%Curve Lines [id:da7552878456427861] 
\draw    (68,145.5) .. controls (110,144.5) and (137,146.5) .. (175,155.5) ;
%Curve Lines [id:da40499842494527183] 
\draw    (175,219.5) .. controls (195,222.5) and (217,224.5) .. (282,225) ;
%Straight Lines [id:da31095032234024567] 
\draw    (175,155.5) -- (175,219.5) ;
%Curve Lines [id:da9323659415170968] 
\draw [draw opacity=0][fill={rgb, 255:red, 74; green, 144; blue, 226 }  ,fill opacity=0.5 ]   (68,145.5) .. controls (131,145.5) and (120,144.5) .. (175,155.5) .. controls (176,220.5) and (174,159.5) .. (175,219.5) .. controls (227,224.5) and (229,225.5) .. (282,225) .. controls (83,223.5) and (262,226.5) .. (68,225) .. controls (69,151.5) and (67,226.5) .. (68,145.5) -- cycle ;
%Straight Lines [id:da2726570766053802] 
\draw [color={rgb, 255:red, 155; green, 155; blue, 155 }  ,draw opacity=1 ]   (175,155.5) -- (190,155.5) ;
%Straight Lines [id:da794346269199613] 
\draw [color={rgb, 255:red, 155; green, 155; blue, 155 }  ,draw opacity=1 ]   (175,219.5) -- (190,219.5) ;
%Straight Lines [id:da8221957579922665] 
\draw [color={rgb, 255:red, 155; green, 155; blue, 155 }  ,draw opacity=1 ]   (187,157.5) -- (187,217.5) ;
\draw [shift={(187,219.5)}, rotate = 270] [fill={rgb, 255:red, 155; green, 155; blue, 155 }  ,fill opacity=1 ][line width=0.08]  [draw opacity=0] (12,-3) -- (0,0) -- (12,3) -- cycle    ;
\draw [shift={(187,155.5)}, rotate = 90] [fill={rgb, 255:red, 155; green, 155; blue, 155 }  ,fill opacity=1 ][line width=0.08]  [draw opacity=0] (12,-3) -- (0,0) -- (12,3) -- cycle    ;

% Text Node
\draw (621,225) node [anchor=west] [inner sep=0.75pt]    {$k$};
% Text Node
\draw (405,79.5) node [anchor=south] [inner sep=0.75pt]    {$n_{k}$};
% Text Node
\draw (284,225) node [anchor=west] [inner sep=0.75pt]    {$k$};
% Text Node
\draw (68,79.5) node [anchor=south] [inner sep=0.75pt]    {$n_{k}$};
% Text Node
\draw (192,178) node [anchor=north west][inner sep=0.75pt]    {$Z$};
% Text Node
\draw (166,250) node [anchor=north west][inner sep=0.75pt]   [align=left] {(a)};
% Text Node
\draw (499,250) node [anchor=north west][inner sep=0.75pt]   [align=left] {(b)};


\end{tikzpicture}

    \caption{The electron distribution function of (a) a Fermi liquid and (b) of a Luttinger liquid. There is no discontinuity (i.e. no Fermi surface) in a Luttinger liquid.}
    \label{fig:distribution}
\end{figure}

The difference between a Fermi liquid and a Luttinger liquid can be illustrated in \prettyref{fig:distribution}.
The distribution function of a free Fermi gas is plotted in dotted lines. After the introduction of interaction,
if we still have well-defined electron-like excitations, then the electron Green function will be something like 
\[
    \frac{Z_{\vb*{k}}}{\omega - \xi_{\vb*{k}}^\text{renor} + \ii 0^+} + \cdots,
\]
where the superscript means ``renormalized'' (i.e. corrected by self-energy), and a renormalization constant 
$Z_{\vb*{k}}$ appears. The behavior of the contribution of the first term into the density distribution function 
is a step function times $Z_{\vb*{k}}$, which is shown in \prettyref{fig:distribution}(a) and defined the Fermi surface. Since the total 
number of electrons is conserved before and after introducing the interaction and $Z_{\vb*{k}} < 1$, 
additional terms must appear in the Green function to make up for the loss of $(1 - Z_{\vb*{k}}) N_\text{e}$
electrons caused by $Z_{\vb*{k}}$, which explains why \prettyref{fig:distribution}(a) is not a perfect 
step function. When in the Luttinger liquid, however, there is no step function behavior in the distribution
function, so $Z_{\vb*{k}} \to 0$ and the $Z/(\omega - \xi)$ term in the electron Green function disappears.
This means that there is no clear pole in the electron Green function, and that there are no electron-like excitations. Actually, electrons are unstable, non-perturbative \emph{solitons} in the Luttinger liquid, 
the fermionic behavior arising from string construction similar to the one in Jordan–Wigner transformation.

\section{Luttinger liquid from a Hubbard model}

In this section, we repeat the calculation done in Section~10.1 in Philip Phillips. \marginnote{Philip Phillips Section~10.1}

Since most modes we are interested in are near the two Fermi points, we can only keep the two $k = \pm k_\text{F}$
component in the spacial electron operator, i.e.
\begin{equation} \marginnote{Philip Phillips Eq.~(10.4)}
    \Psi_{n \sigma} = \ee^{\ii k_\text{F} n a} \Psi_{n \sigma +} + \ee^{- \ii k_\text{F} n a} \Psi_{n \sigma -},
    \label{eq:only-fermi-points}
\end{equation}
where the $-$ mode is the left-moving mode and the $+$ mode is the right-moving mode. We can also make the 
following first-order Taylor expansion:
\begin{equation}
    \Psi_{n+1, \sigma, \pm} = \Psi_{n \sigma \pm 1} + a \partial_x \Psi_{n \sigma \pm 1}.
    \label{eq:first-order-taylor-tb}
\end{equation}
Now according to \eqref{eq:only-fermi-points} and \eqref{eq:first-order-taylor-tb}, the tight-binding Hamiltonian can be expanded into 
\[
    \begin{aligned}
        &\quad - t \sum_{n, \sigma} \Psi_{n \sigma}^\dagger \Psi_{n+1, \sigma} \\
        &= - t \sum_{n, \sigma} (\ee^{\ii k_\text{F} a} \Psi^\dagger_{n \sigma +} \Psi_{n \sigma +} + \ee^{- \ii k_\text{F} a} \Psi^\dagger_{n \sigma -} \Psi_{n \sigma -} + \ee^{\ii k_\text{F} a} a \Psi^\dagger_{n \sigma +} \partial_x \Psi_{n \sigma +} + \ee^{- \ii k_\text{F} a} a \Psi^\dagger_{n \sigma -} \partial_x \Psi_{n \sigma -}),
    \end{aligned}
\]
where the non-zero-frequency terms vanish after summing up over $n$, i.e. after the inverse Fourier transformation.
Since the Hamiltonian is Hermitian, we have 
\[
    \begin{aligned}
        - t \sum_{n, \sigma} &\Psi_{n \sigma}^\dagger \Psi_{n+1, \sigma} \\
        = - \frac{t}{2} \sum_{n, \sigma} &(\ee^{\ii k_\text{F} a} \Psi^\dagger_{n \sigma +} \Psi_{n \sigma +} + \ee^{- \ii k_\text{F} a} \Psi^\dagger_{n \sigma -} \Psi_{n \sigma -} + \ee^{\ii k_\text{F} a} a \Psi^\dagger_{n \sigma +} \partial_x \Psi_{n \sigma +} + \ee^{- \ii k_\text{F} a} a \Psi^\dagger_{n \sigma -} \partial_x \Psi_{n \sigma -} \\
        & + \ee^{- \ii k_\text{F} a} \Psi^\dagger_{n \sigma +} \Psi_{n \sigma +} + \ee^{\ii k_\text{F} a} \Psi^\dagger_{n \sigma -} \Psi_{n \sigma -} + \ee^{- \ii k_\text{F} a} a \underbrace{\partial_x \Psi_{n \sigma +}^\dagger \Psi_{n \sigma +}}_{- \Psi_{n \sigma +}^\dagger \partial_x \Psi_{n \sigma +}} + \ee^{\ii k_\text{F} a} a \underbrace{\partial_x \Psi_{n \sigma -}^\dagger \Psi_{n \sigma -}}_{- \Psi_{n \sigma -}^\dagger \partial_x \Psi_{n \sigma -}} ) ,
    \end{aligned}
\]
so finally 
\begin{equation}
    \begin{aligned}
        H_t &= - t \sum_{n, \sigma} (\cos(k_\text{F} a) \Psi^\dagger_{n \sigma +} \Psi_{n \sigma +}
        + \cos(k_\text{F} a) \Psi^\dagger_{n \sigma -} \Psi_{n \sigma -}
        + \ii a\sin(k_\text{F} a) \Psi_{n \sigma +}^\dagger \partial_x \Psi_{n \sigma +} \\
        &\quad \quad - \ii a\sin(k_\text{F} a) \Psi_{n \sigma -}^\dagger \partial_x \Psi_{n \sigma -}).
    \end{aligned}
    \label{eq:linearized-proto}
\end{equation}
We already know that the energy spectrum of the full $H_t$ (i.e. taking into account not only the modes 
near $\pm k_\text{F}$, but also modes far from them) is 
\begin{equation}
    \epsilon_{k \sigma} = - 2 t \cos(k a), 
    \label{eq:full-spectrum-tb}
\end{equation}
so the chemical potential is 
\begin{equation}
    \mu = - 2 t \cos (k_\text{F} a),
\end{equation}
and therefore the first two terms in \eqref{eq:linearized-proto} will be canceled by the term $- \mu N$.
From \eqref{eq:full-spectrum-tb} we also know that 
\begin{equation}
    v = \grad_{k} \epsilon_{k} = 2 a t \sin(k a),
\end{equation}
and therefore 
\begin{equation}
    v_\text{F} = 2 a t \sin(k_\text{F} a),
\end{equation}
so \eqref{eq:linearized-proto} becomes 
\begin{equation}
    \begin{aligned}
        H_t - \mu N &= - v_\text{F} \sum_{n, \sigma} (\ii \Psi_{n \sigma +}^\dagger \partial_x \Psi_{n \sigma +} - \ii \Psi_{n \sigma -}^\dagger \partial_x \Psi_{n \sigma -})  \\
        &= - v_\text{F} \sum_{\sigma} \int \dd{x} (\ii \Psi_{\sigma +}^\dagger(x) \partial_x \Psi_{\sigma +}(x) - \ii \Psi_{\sigma -}^\dagger(x) \partial_x \Psi_{\sigma -}(x) ), \quad \sqrt{a} \Psi_{\sigma \pm} (x) = \Psi_{n \sigma \pm}.
    \end{aligned} 
\end{equation}

\section{The Luttinger model}

\bibliographystyle{plain}
\bibliography{../formalism/fluid-quantum,bosonization,../solid/fermi-liquid.bib} 

\end{document}