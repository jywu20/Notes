\documentclass[hyperref, UTF8, a4paper]{ctexart}

\usepackage{geometry}
\usepackage{titling}
\usepackage{titlesec}
\usepackage{paralist}
\usepackage{footnote}
\usepackage{enumerate}
\usepackage{autobreak}
\usepackage{amsmath, amssymb, amsthm}
\usepackage{mathtools}
\usepackage{bbm}
\usepackage[superscript]{cite}
\usepackage{graphicx}
\usepackage{subfigure}
\usepackage{physics}
\usepackage{siunitx}
\usepackage{tikz}
\usepackage[compat=1.1.0]{tikz-feynhand}
\usepackage[ruled, vlined, linesnumbered, noend]{algorithm2e}
\usepackage{xr-hyper}
\usepackage[colorlinks, linkcolor=black, anchorcolor=black, citecolor=black, filecolor=black]{hyperref}
\usepackage[most]{tcolorbox}
\usepackage{caption}
\usepackage{prettyref}

% Cite: superscript, [1]
\makeatletter
\renewcommand\@citess[1]{\textsuperscript{[#1]}}
\makeatother

\externaldocument[optics-]{../optics/optics}[optics.pdf]
\externaldocument[vasp-]{../cond-comp/vasp/vasp}[vasp.pdf]
\externaldocument[qft-]{../relativistic-qft/relativistic-qft}[relativistic-qft.pdf]
\externaldocument[soft-]{../soft/soft}[soft.pdf]

\geometry{left=3.18cm,right=3.18cm,top=2.54cm,bottom=2.54cm}
\titlespacing{\paragraph}{0pt}{1pt}{10pt}[20pt]
\setlength{\droptitle}{-5em}
\preauthor{\vspace{-10pt}\begin{center}}
\postauthor{\par\end{center}}

\DeclareMathOperator{\timeorder}{\mathcal{T}}
\DeclareMathOperator{\diag}{diag}
\DeclareMathOperator{\legpoly}{P}
\DeclareMathOperator{\primevalue}{P}
\DeclareMathOperator{\sgn}{sgn}
\newcommand*{\ii}{\mathrm{i}}
\newcommand*{\ee}{\mathrm{e}}
\newcommand*{\const}{\mathrm{const}}
\newcommand*{\suchthat}{\quad \text{s.t.} \quad}
\newcommand*{\argmin}{\arg\min}
\newcommand*{\argmax}{\arg\max}
\newcommand*{\normalorder}[1]{: #1 :}
\newcommand*{\pair}[1]{\langle #1 \rangle}
\newcommand*{\fd}[1]{\mathcal{D} #1}

\newrefformat{chap}{第\ref{#1}章}
\newrefformat{sec}{第\ref{#1}节}
\newrefformat{note}{注\ref{#1}}
\newrefformat{fig}{图\ref{#1}}
\newrefformat{alg}{算法\ref{#1}}
\renewcommand{\autoref}{\prettyref}

\usetikzlibrary{arrows,shapes,positioning}
\usetikzlibrary{arrows.meta}
\usetikzlibrary{decorations.markings}
\tikzstyle arrowstyle=[scale=1]
\tikzstyle directed=[postaction={decorate,decoration={markings,
    mark=at position .5 with {\arrow[arrowstyle]{stealth}}}}]
\tikzstyle ray=[directed, thick]
\tikzstyle dot=[anchor=base,fill,circle,inner sep=1pt]

% Algorithm setting
\renewcommand{\algorithmcfname}{算法}
% Python-style code
\SetKwIF{If}{ElseIf}{Else}{if}{:}{elif:}{else:}{}
\SetKwFor{For}{for}{:}{}
\SetKwFor{While}{while}{:}{}
\SetKwInput{KwData}{输入}
\SetKwInput{KwResult}{输出}
\SetArgSty{textnormal}

\tcbuselibrary{skins, breakable, theorems}

\renewcommand{\emph}[1]{\textbf{#1}}
\newcommand*{\concept}[1]{\underline{\textbf{#1}}}

\newcommand{\hmn}[1]{% Hermann-Maguin notation
  \ensuremath{\begingroup\setupHMN #1\endgroup}%
}

\newcommand{\setupHMN}{%
  \doHMN{-}{\HMNoverline}%
  \doHMN{*}{\HMNminverse}%
  \doHMN{i}{\infty}
}

\newcommand{\doHMN}[2]{%
  \begingroup\lccode`~=`#1
  \lowercase{\endgroup\let~}#2%
  \mathcode`#1="8000
}

\newcommand{\HMNminverse}[1]{\frac{#1}{m}}
\newcommand{\HMNoverline}[1]{\mkern1mu\overline{\mkern-1mu#1\mkern-1mu}\mkern1mu}

\newcommand{\Ztwo}{$\mathbb{Z}_2$}

\newcommand{\bigO}[1]{\mathcal{O}(#1)}

\title{准晶}
\author{吴晋渊 18307110155}
\date{}

\begin{document}

\maketitle

\section{准晶概述}

准晶的发现可以追溯到Shechtman等报道的快速冷却的Al-Mn合金的衍射图样中观察到的正二十面体对称性\cite{PhysRevLett.53.1951}。
正二十面体对称性包含一个$C_5$对称轴,而不可能有一个晶格具有这种类型的对称性——它不在晶体允许的32种点群中\cite{Johnston_1960}。


\section{准晶相变的金斯堡-朗道理论}

本节将以文献\cite{PhysRevB.32.5764}为例,介绍准晶相变的金斯堡-朗道理论。
我们采用金斯堡-朗道理论的标准处理方法,假定系统的状态可以使用一个空间中的连续、平滑的序参量描述,系统的行为可以使用一个仅仅关于序参量的自由能完整描述,通过对称性写下自由能的形式,并分析自由能中各参数变动时系统是否发生对称性自发破缺,以及发生后系统基态的性质。
虽然金斯堡-朗道理论通常是用于处理二级相变的,但是如果序参量在两相交界处变化足够平缓,从而能够保证系统在相变点附近的行为仍然可以使用。使用金斯堡-朗道理论处理固液相变已经成为常见的方法\cite{fabrizio2008,PhysRevB.90.104101}。
事实上,对有明确、不连续的两相交界的一级相变,基于金斯堡-朗道理论的相场方法\cite{provatas2011phase}也常常在数值模拟中被使用,以避免显式追踪相边界,节约计算资源\cite{boettinger2002phase}。

考虑一个具有平移不变性和(连续)旋转不变性的液体。液体结晶属于结构相变,故序参量大体上是密度。
对一个最一般的系统,序参量选取是否正确、系统自由能是否还依赖于序参量以外的(无法直接从系统的哈密顿量出发获得序参量),但对液体,将自由能写成密度的一个泛函已经属于常规操作\cite{Evans_2016,cdft2020}。
通常的液体的低能状态是均匀的,而无论是晶体还是准晶依照定义密度分布都不是完全均匀的,如果特定条件下能够形成准晶,那么准晶态必定相较其它状态在某种意义上更加稳定,即系统自由能最低的状态将不再是密度处处为常数的状态。
因此可以将密度的$\vb*{q} \neq 0$的傅里叶分量$\rho(\vb*{q})$视作结晶的序参量。根据涉及的波矢的个数,液体的自由能的展开式子形如下式:
\begin{equation}
    \begin{aligned}
        F[\rho] &= \sum_\text{all $\vb*{q}$'s} r \rho_{\vb*{q}} \rho_{- \vb*{q}} + u (\rho_{\vb*{q}} \rho_{- \vb*{q}})^2 
        + w \rho_{\vb*{q}} \rho_{- \vb*{q}} \rho_{\vb*{p}} \rho_{- \vb*{p}} 
        + v_3 \rho_{\vb*{q}_1} \rho_{\vb*{q}_2} \rho_{\vb*{q}_3} \delta^3(\vb*{q}_1 + \vb*{q}_2 + \vb*{q}_3) \\
        &\quad \quad + v_4 \rho_{\vb*{q}_1} \rho_{\vb*{q}_2} \rho_{\vb*{q}_3} \rho_{\vb*{q}_4} \delta^3(\sum_i \vb*{q}_i) 
        + v_5 \rho_{\vb*{q}_1} \rho_{\vb*{q}_2} \rho_{\vb*{q}_3} \rho_{\vb*{q}_4} \rho_{\vb*{q}_5}
        \delta^3(\sum_i \vb*{q}_i) + \cdots,
    \end{aligned}
    \label{eq:free-energy-static}
\end{equation}
其中的$\delta$函数保证了理论的空间平移不变性;空间旋转不变性保证了系数仅仅依赖于$\vb*{q}_i$的模长。
据此自由能可以计算$\rho(\vb*{r})$的期望值。如果发现出现非零的$\expval{\rho(\vb*{q})}$意味着出现对称性自发破缺,有某种序形成。
不同种类的序会贡献不同形式的项到$\rho(\vb*{r})$中。例如,一个完美的层列液晶序会贡献一个单独的
\begin{equation}
    \rho_\text{nematic} = \rho(\vb*{q}) (\ee^{\ii \vb*{q} \cdot \vb*{r} + \ii \theta } + \ee^{- \ii \vb*{q} \cdot \vb*{r} - \ii \theta } ) + \rho(2\vb*{q}) (\ee^{\ii \vb*{q} \cdot \vb*{r} + \ii \theta' } + \ee^{- \ii \vb*{q} \cdot \vb*{r} - \ii \theta' } ) + \cdots,
\end{equation}
项,它在一个方向上有连续平移对称性破缺,但是在其它方向上连续平移对称性仍然保持(见\prettyref{fig:smectic})。如果我们只考虑系统的长程行为,可以截断高次谐波。
使用$\rho(\vb*{q})$的语言,就是由液晶序贡献的那部分密度中有$\rho(\vb*{q}) = \text{phase factor} \times \rho(-\vb*{q})$,其中$\vb*{q}$方向取为液晶分子的指向,长度取为$2\pi n/ L$,因为液晶仍然保留了$C_2$对称性。这里$\theta$因子来自液晶序可以整体平移这一事实。
类似的,一个保留了$C_3$对称性的序中$\rho(\vb*{q}_1)$, $\rho(\vb*{q}_2)$, $\rho(\vb*{q}_3)$这三个量也只应该相差一个相因子,其中$\vb*{q}_2$是$\vb*{q}_1$绕着指定的轴旋转\SI{120}{\degree}得到的矢量,$\vb*{q}_3$是$\vb*{q}_1$绕着同一个轴旋转\SI{240}{\degree}得到的;$\rho(\vb*{q}_1)$, $\rho(\vb*{q}_2)$, $\rho(\vb*{q}_3)$这三个量的相位因子来自序中原子的平移,它们相差的相位因子则表明了$C_3$序的晶格常数。
因此,通过观察$\rho(\vb*{q})$的不同成分可以辨认出体系中的不同序。

\begin{figure}
    \centering
    

\tikzset{every picture/.style={line width=0.75pt}} %set default line width to 0.75pt        

\begin{tikzpicture}[x=0.75pt,y=0.75pt,yscale=-0.7,xscale=0.7]
%uncomment if require: \path (0,300); %set diagram left start at 0, and has height of 300

%Rounded Rect [id:dp05563000442277022] 
\draw  [fill={rgb, 255:red, 255; green, 255; blue, 255 }  ,fill opacity=1 ] (137,200.08) .. controls (138.7,200.08) and (140.08,198.7) .. (140.08,197) -- (140.08,170.33) .. controls (140.08,168.63) and (138.7,167.25) .. (137,167.25) -- (137,167.25) .. controls (135.3,167.25) and (133.92,168.63) .. (133.92,170.33) -- (133.92,197) .. controls (133.92,198.7) and (135.3,200.08) .. (137,200.08) -- cycle ;
%Rounded Rect [id:dp6531641548320384] 
\draw  [fill={rgb, 255:red, 255; green, 255; blue, 255 }  ,fill opacity=1 ] (144.17,200.08) .. controls (145.87,200.08) and (147.25,198.7) .. (147.25,197) -- (147.25,170.33) .. controls (147.25,168.63) and (145.87,167.25) .. (144.17,167.25) -- (144.17,167.25) .. controls (142.46,167.25) and (141.08,168.63) .. (141.08,170.33) -- (141.08,197) .. controls (141.08,198.7) and (142.46,200.08) .. (144.17,200.08) -- cycle ;
%Rounded Rect [id:dp7705246921925859] 
\draw  [fill={rgb, 255:red, 255; green, 255; blue, 255 }  ,fill opacity=1 ] (150.33,200.08) .. controls (152.04,200.08) and (153.42,198.7) .. (153.42,197) -- (153.42,170.33) .. controls (153.42,168.63) and (152.04,167.25) .. (150.33,167.25) -- (150.33,167.25) .. controls (148.63,167.25) and (147.25,168.63) .. (147.25,170.33) -- (147.25,197) .. controls (147.25,198.7) and (148.63,200.08) .. (150.33,200.08) -- cycle ;
%Rounded Rect [id:dp16514777594403607] 
\draw  [fill={rgb, 255:red, 255; green, 255; blue, 255 }  ,fill opacity=1 ] (180.33,194.92) .. controls (182.04,194.92) and (183.42,193.54) .. (183.42,191.83) -- (183.42,165.17) .. controls (183.42,163.46) and (182.04,162.08) .. (180.33,162.08) -- (180.33,162.08) .. controls (178.63,162.08) and (177.25,163.46) .. (177.25,165.17) -- (177.25,191.83) .. controls (177.25,193.54) and (178.63,194.92) .. (180.33,194.92) -- cycle ;
%Rounded Rect [id:dp9545252921582017] 
\draw  [fill={rgb, 255:red, 255; green, 255; blue, 255 }  ,fill opacity=1 ] (183.17,204.08) .. controls (184.87,204.08) and (186.25,202.7) .. (186.25,201) -- (186.25,174.33) .. controls (186.25,172.63) and (184.87,171.25) .. (183.17,171.25) -- (183.17,171.25) .. controls (181.46,171.25) and (180.08,172.63) .. (180.08,174.33) -- (180.08,201) .. controls (180.08,202.7) and (181.46,204.08) .. (183.17,204.08) -- cycle ;
%Rounded Rect [id:dp6380802078932455] 
\draw  [fill={rgb, 255:red, 255; green, 255; blue, 255 }  ,fill opacity=1 ] (192.17,194.08) .. controls (193.87,194.08) and (195.25,192.7) .. (195.25,191) -- (195.25,164.33) .. controls (195.25,162.63) and (193.87,161.25) .. (192.17,161.25) -- (192.17,161.25) .. controls (190.46,161.25) and (189.08,162.63) .. (189.08,164.33) -- (189.08,191) .. controls (189.08,192.7) and (190.46,194.08) .. (192.17,194.08) -- cycle ;
%Rounded Rect [id:dp49673386562771316] 
\draw  [fill={rgb, 255:red, 255; green, 255; blue, 255 }  ,fill opacity=1 ] (161.17,198.08) .. controls (162.87,198.08) and (164.25,196.7) .. (164.25,195) -- (164.25,168.33) .. controls (164.25,166.63) and (162.87,165.25) .. (161.17,165.25) -- (161.17,165.25) .. controls (159.46,165.25) and (158.08,166.63) .. (158.08,168.33) -- (158.08,195) .. controls (158.08,196.7) and (159.46,198.08) .. (161.17,198.08) -- cycle ;
%Rounded Rect [id:dp944726253294734] 
\draw  [fill={rgb, 255:red, 255; green, 255; blue, 255 }  ,fill opacity=1 ] (178.17,196.92) .. controls (179.87,196.92) and (181.25,195.54) .. (181.25,193.83) -- (181.25,167.17) .. controls (181.25,165.46) and (179.87,164.08) .. (178.17,164.08) -- (178.17,164.08) .. controls (176.46,164.08) and (175.08,165.46) .. (175.08,167.17) -- (175.08,193.83) .. controls (175.08,195.54) and (176.46,196.92) .. (178.17,196.92) -- cycle ;
%Rounded Rect [id:dp7634948824752461] 
\draw  [fill={rgb, 255:red, 255; green, 255; blue, 255 }  ,fill opacity=1 ] (164.25,201.17) .. controls (165.95,201.17) and (167.33,199.79) .. (167.33,198.08) -- (167.33,171.42) .. controls (167.33,169.71) and (165.95,168.33) .. (164.25,168.33) -- (164.25,168.33) .. controls (162.55,168.33) and (161.17,169.71) .. (161.17,171.42) -- (161.17,198.08) .. controls (161.17,199.79) and (162.55,201.17) .. (164.25,201.17) -- cycle ;
%Rounded Rect [id:dp842642877347413] 
\draw  [fill={rgb, 255:red, 255; green, 255; blue, 255 }  ,fill opacity=1 ] (204.17,194.08) .. controls (205.87,194.08) and (207.25,192.7) .. (207.25,191) -- (207.25,164.33) .. controls (207.25,162.63) and (205.87,161.25) .. (204.17,161.25) -- (204.17,161.25) .. controls (202.46,161.25) and (201.08,162.63) .. (201.08,164.33) -- (201.08,191) .. controls (201.08,192.7) and (202.46,194.08) .. (204.17,194.08) -- cycle ;
%Rounded Rect [id:dp2527241147768988] 
\draw  [fill={rgb, 255:red, 255; green, 255; blue, 255 }  ,fill opacity=1 ] (215,192.08) .. controls (216.7,192.08) and (218.08,190.7) .. (218.08,189) -- (218.08,162.33) .. controls (218.08,160.63) and (216.7,159.25) .. (215,159.25) -- (215,159.25) .. controls (213.3,159.25) and (211.92,160.63) .. (211.92,162.33) -- (211.92,189) .. controls (211.92,190.7) and (213.3,192.08) .. (215,192.08) -- cycle ;
%Rounded Rect [id:dp10611436324219059] 
\draw  [fill={rgb, 255:red, 255; green, 255; blue, 255 }  ,fill opacity=1 ] (221.17,192.08) .. controls (222.87,192.08) and (224.25,190.7) .. (224.25,189) -- (224.25,162.33) .. controls (224.25,160.63) and (222.87,159.25) .. (221.17,159.25) -- (221.17,159.25) .. controls (219.46,159.25) and (218.08,160.63) .. (218.08,162.33) -- (218.08,189) .. controls (218.08,190.7) and (219.46,192.08) .. (221.17,192.08) -- cycle ;
%Rounded Rect [id:dp6761737981434759] 
\draw  [fill={rgb, 255:red, 255; green, 255; blue, 255 }  ,fill opacity=1 ] (138,128.08) .. controls (139.7,128.08) and (141.08,126.7) .. (141.08,125) -- (141.08,98.33) .. controls (141.08,96.63) and (139.7,95.25) .. (138,95.25) -- (138,95.25) .. controls (136.3,95.25) and (134.92,96.63) .. (134.92,98.33) -- (134.92,125) .. controls (134.92,126.7) and (136.3,128.08) .. (138,128.08) -- cycle ;
%Rounded Rect [id:dp44018912567548596] 
\draw  [fill={rgb, 255:red, 255; green, 255; blue, 255 }  ,fill opacity=1 ] (145.17,128.08) .. controls (146.87,128.08) and (148.25,126.7) .. (148.25,125) -- (148.25,98.33) .. controls (148.25,96.63) and (146.87,95.25) .. (145.17,95.25) -- (145.17,95.25) .. controls (143.46,95.25) and (142.08,96.63) .. (142.08,98.33) -- (142.08,125) .. controls (142.08,126.7) and (143.46,128.08) .. (145.17,128.08) -- cycle ;
%Rounded Rect [id:dp049822785616503884] 
\draw  [fill={rgb, 255:red, 255; green, 255; blue, 255 }  ,fill opacity=1 ] (151.33,128.08) .. controls (153.04,128.08) and (154.42,126.7) .. (154.42,125) -- (154.42,98.33) .. controls (154.42,96.63) and (153.04,95.25) .. (151.33,95.25) -- (151.33,95.25) .. controls (149.63,95.25) and (148.25,96.63) .. (148.25,98.33) -- (148.25,125) .. controls (148.25,126.7) and (149.63,128.08) .. (151.33,128.08) -- cycle ;
%Rounded Rect [id:dp7275156706152337] 
\draw  [fill={rgb, 255:red, 255; green, 255; blue, 255 }  ,fill opacity=1 ] (181.33,122.92) .. controls (183.04,122.92) and (184.42,121.54) .. (184.42,119.83) -- (184.42,93.17) .. controls (184.42,91.46) and (183.04,90.08) .. (181.33,90.08) -- (181.33,90.08) .. controls (179.63,90.08) and (178.25,91.46) .. (178.25,93.17) -- (178.25,119.83) .. controls (178.25,121.54) and (179.63,122.92) .. (181.33,122.92) -- cycle ;
%Rounded Rect [id:dp7851150622633405] 
\draw  [fill={rgb, 255:red, 255; green, 255; blue, 255 }  ,fill opacity=1 ] (184.17,132.08) .. controls (185.87,132.08) and (187.25,130.7) .. (187.25,129) -- (187.25,102.33) .. controls (187.25,100.63) and (185.87,99.25) .. (184.17,99.25) -- (184.17,99.25) .. controls (182.46,99.25) and (181.08,100.63) .. (181.08,102.33) -- (181.08,129) .. controls (181.08,130.7) and (182.46,132.08) .. (184.17,132.08) -- cycle ;
%Rounded Rect [id:dp023929054330812827] 
\draw  [fill={rgb, 255:red, 255; green, 255; blue, 255 }  ,fill opacity=1 ] (193.17,122.08) .. controls (194.87,122.08) and (196.25,120.7) .. (196.25,119) -- (196.25,92.33) .. controls (196.25,90.63) and (194.87,89.25) .. (193.17,89.25) -- (193.17,89.25) .. controls (191.46,89.25) and (190.08,90.63) .. (190.08,92.33) -- (190.08,119) .. controls (190.08,120.7) and (191.46,122.08) .. (193.17,122.08) -- cycle ;
%Rounded Rect [id:dp8807448370596598] 
\draw  [fill={rgb, 255:red, 255; green, 255; blue, 255 }  ,fill opacity=1 ] (162.17,126.08) .. controls (163.87,126.08) and (165.25,124.7) .. (165.25,123) -- (165.25,96.33) .. controls (165.25,94.63) and (163.87,93.25) .. (162.17,93.25) -- (162.17,93.25) .. controls (160.46,93.25) and (159.08,94.63) .. (159.08,96.33) -- (159.08,123) .. controls (159.08,124.7) and (160.46,126.08) .. (162.17,126.08) -- cycle ;
%Rounded Rect [id:dp2406222646847067] 
\draw  [fill={rgb, 255:red, 255; green, 255; blue, 255 }  ,fill opacity=1 ] (179.17,124.92) .. controls (180.87,124.92) and (182.25,123.54) .. (182.25,121.83) -- (182.25,95.17) .. controls (182.25,93.46) and (180.87,92.08) .. (179.17,92.08) -- (179.17,92.08) .. controls (177.46,92.08) and (176.08,93.46) .. (176.08,95.17) -- (176.08,121.83) .. controls (176.08,123.54) and (177.46,124.92) .. (179.17,124.92) -- cycle ;
%Rounded Rect [id:dp8366661122103913] 
\draw  [fill={rgb, 255:red, 255; green, 255; blue, 255 }  ,fill opacity=1 ] (165.25,129.17) .. controls (166.95,129.17) and (168.33,127.79) .. (168.33,126.08) -- (168.33,99.42) .. controls (168.33,97.71) and (166.95,96.33) .. (165.25,96.33) -- (165.25,96.33) .. controls (163.55,96.33) and (162.17,97.71) .. (162.17,99.42) -- (162.17,126.08) .. controls (162.17,127.79) and (163.55,129.17) .. (165.25,129.17) -- cycle ;
%Rounded Rect [id:dp5979480085184197] 
\draw  [fill={rgb, 255:red, 255; green, 255; blue, 255 }  ,fill opacity=1 ] (205.17,122.08) .. controls (206.87,122.08) and (208.25,120.7) .. (208.25,119) -- (208.25,92.33) .. controls (208.25,90.63) and (206.87,89.25) .. (205.17,89.25) -- (205.17,89.25) .. controls (203.46,89.25) and (202.08,90.63) .. (202.08,92.33) -- (202.08,119) .. controls (202.08,120.7) and (203.46,122.08) .. (205.17,122.08) -- cycle ;
%Rounded Rect [id:dp5520431708688049] 
\draw  [fill={rgb, 255:red, 255; green, 255; blue, 255 }  ,fill opacity=1 ] (216,120.08) .. controls (217.7,120.08) and (219.08,118.7) .. (219.08,117) -- (219.08,90.33) .. controls (219.08,88.63) and (217.7,87.25) .. (216,87.25) -- (216,87.25) .. controls (214.3,87.25) and (212.92,88.63) .. (212.92,90.33) -- (212.92,117) .. controls (212.92,118.7) and (214.3,120.08) .. (216,120.08) -- cycle ;
%Rounded Rect [id:dp9248678649410391] 
\draw  [fill={rgb, 255:red, 255; green, 255; blue, 255 }  ,fill opacity=1 ] (222.17,120.08) .. controls (223.87,120.08) and (225.25,118.7) .. (225.25,117) -- (225.25,90.33) .. controls (225.25,88.63) and (223.87,87.25) .. (222.17,87.25) -- (222.17,87.25) .. controls (220.46,87.25) and (219.08,88.63) .. (219.08,90.33) -- (219.08,117) .. controls (219.08,118.7) and (220.46,120.08) .. (222.17,120.08) -- cycle ;
%Rounded Rect [id:dp11633727618214662] 
\draw  [fill={rgb, 255:red, 255; green, 255; blue, 255 }  ,fill opacity=1 ] (133,272.08) .. controls (134.7,272.08) and (136.08,270.7) .. (136.08,269) -- (136.08,242.33) .. controls (136.08,240.63) and (134.7,239.25) .. (133,239.25) -- (133,239.25) .. controls (131.3,239.25) and (129.92,240.63) .. (129.92,242.33) -- (129.92,269) .. controls (129.92,270.7) and (131.3,272.08) .. (133,272.08) -- cycle ;
%Rounded Rect [id:dp19991605793380307] 
\draw  [fill={rgb, 255:red, 255; green, 255; blue, 255 }  ,fill opacity=1 ] (140.17,272.08) .. controls (141.87,272.08) and (143.25,270.7) .. (143.25,269) -- (143.25,242.33) .. controls (143.25,240.63) and (141.87,239.25) .. (140.17,239.25) -- (140.17,239.25) .. controls (138.46,239.25) and (137.08,240.63) .. (137.08,242.33) -- (137.08,269) .. controls (137.08,270.7) and (138.46,272.08) .. (140.17,272.08) -- cycle ;
%Rounded Rect [id:dp42844763551577003] 
\draw  [fill={rgb, 255:red, 255; green, 255; blue, 255 }  ,fill opacity=1 ] (146.33,272.08) .. controls (148.04,272.08) and (149.42,270.7) .. (149.42,269) -- (149.42,242.33) .. controls (149.42,240.63) and (148.04,239.25) .. (146.33,239.25) -- (146.33,239.25) .. controls (144.63,239.25) and (143.25,240.63) .. (143.25,242.33) -- (143.25,269) .. controls (143.25,270.7) and (144.63,272.08) .. (146.33,272.08) -- cycle ;
%Rounded Rect [id:dp39463105589249836] 
\draw  [fill={rgb, 255:red, 255; green, 255; blue, 255 }  ,fill opacity=1 ] (176.33,266.92) .. controls (178.04,266.92) and (179.42,265.54) .. (179.42,263.83) -- (179.42,237.17) .. controls (179.42,235.46) and (178.04,234.08) .. (176.33,234.08) -- (176.33,234.08) .. controls (174.63,234.08) and (173.25,235.46) .. (173.25,237.17) -- (173.25,263.83) .. controls (173.25,265.54) and (174.63,266.92) .. (176.33,266.92) -- cycle ;
%Rounded Rect [id:dp5570828347495911] 
\draw  [fill={rgb, 255:red, 255; green, 255; blue, 255 }  ,fill opacity=1 ] (179.17,276.08) .. controls (180.87,276.08) and (182.25,274.7) .. (182.25,273) -- (182.25,246.33) .. controls (182.25,244.63) and (180.87,243.25) .. (179.17,243.25) -- (179.17,243.25) .. controls (177.46,243.25) and (176.08,244.63) .. (176.08,246.33) -- (176.08,273) .. controls (176.08,274.7) and (177.46,276.08) .. (179.17,276.08) -- cycle ;
%Rounded Rect [id:dp8351877025940617] 
\draw  [fill={rgb, 255:red, 255; green, 255; blue, 255 }  ,fill opacity=1 ] (188.17,266.08) .. controls (189.87,266.08) and (191.25,264.7) .. (191.25,263) -- (191.25,236.33) .. controls (191.25,234.63) and (189.87,233.25) .. (188.17,233.25) -- (188.17,233.25) .. controls (186.46,233.25) and (185.08,234.63) .. (185.08,236.33) -- (185.08,263) .. controls (185.08,264.7) and (186.46,266.08) .. (188.17,266.08) -- cycle ;
%Rounded Rect [id:dp8163905794152979] 
\draw  [fill={rgb, 255:red, 255; green, 255; blue, 255 }  ,fill opacity=1 ] (157.17,270.08) .. controls (158.87,270.08) and (160.25,268.7) .. (160.25,267) -- (160.25,240.33) .. controls (160.25,238.63) and (158.87,237.25) .. (157.17,237.25) -- (157.17,237.25) .. controls (155.46,237.25) and (154.08,238.63) .. (154.08,240.33) -- (154.08,267) .. controls (154.08,268.7) and (155.46,270.08) .. (157.17,270.08) -- cycle ;
%Rounded Rect [id:dp021154523060017638] 
\draw  [fill={rgb, 255:red, 255; green, 255; blue, 255 }  ,fill opacity=1 ] (174.17,268.92) .. controls (175.87,268.92) and (177.25,267.54) .. (177.25,265.83) -- (177.25,239.17) .. controls (177.25,237.46) and (175.87,236.08) .. (174.17,236.08) -- (174.17,236.08) .. controls (172.46,236.08) and (171.08,237.46) .. (171.08,239.17) -- (171.08,265.83) .. controls (171.08,267.54) and (172.46,268.92) .. (174.17,268.92) -- cycle ;
%Rounded Rect [id:dp6107290134756147] 
\draw  [fill={rgb, 255:red, 255; green, 255; blue, 255 }  ,fill opacity=1 ] (160.25,273.17) .. controls (161.95,273.17) and (163.33,271.79) .. (163.33,270.08) -- (163.33,243.42) .. controls (163.33,241.71) and (161.95,240.33) .. (160.25,240.33) -- (160.25,240.33) .. controls (158.55,240.33) and (157.17,241.71) .. (157.17,243.42) -- (157.17,270.08) .. controls (157.17,271.79) and (158.55,273.17) .. (160.25,273.17) -- cycle ;
%Rounded Rect [id:dp6749271405762176] 
\draw  [fill={rgb, 255:red, 255; green, 255; blue, 255 }  ,fill opacity=1 ] (200.17,266.08) .. controls (201.87,266.08) and (203.25,264.7) .. (203.25,263) -- (203.25,236.33) .. controls (203.25,234.63) and (201.87,233.25) .. (200.17,233.25) -- (200.17,233.25) .. controls (198.46,233.25) and (197.08,234.63) .. (197.08,236.33) -- (197.08,263) .. controls (197.08,264.7) and (198.46,266.08) .. (200.17,266.08) -- cycle ;
%Rounded Rect [id:dp9946230417394801] 
\draw  [fill={rgb, 255:red, 255; green, 255; blue, 255 }  ,fill opacity=1 ] (211,264.08) .. controls (212.7,264.08) and (214.08,262.7) .. (214.08,261) -- (214.08,234.33) .. controls (214.08,232.63) and (212.7,231.25) .. (211,231.25) -- (211,231.25) .. controls (209.3,231.25) and (207.92,232.63) .. (207.92,234.33) -- (207.92,261) .. controls (207.92,262.7) and (209.3,264.08) .. (211,264.08) -- cycle ;
%Rounded Rect [id:dp15697973709097868] 
\draw  [fill={rgb, 255:red, 255; green, 255; blue, 255 }  ,fill opacity=1 ] (217.17,264.08) .. controls (218.87,264.08) and (220.25,262.7) .. (220.25,261) -- (220.25,234.33) .. controls (220.25,232.63) and (218.87,231.25) .. (217.17,231.25) -- (217.17,231.25) .. controls (215.46,231.25) and (214.08,232.63) .. (214.08,234.33) -- (214.08,261) .. controls (214.08,262.7) and (215.46,264.08) .. (217.17,264.08) -- cycle ;
%Straight Lines [id:da5282684837420737] 
\draw    (70,176) -- (70,113.33) ;
\draw [shift={(70,111.33)}, rotate = 90] [color={rgb, 255:red, 0; green, 0; blue, 0 }  ][line width=0.75]    (10.93,-3.29) .. controls (6.95,-1.4) and (3.31,-0.3) .. (0,0) .. controls (3.31,0.3) and (6.95,1.4) .. (10.93,3.29)   ;
%Straight Lines [id:da8123690998660977] 
\draw    (238,283) -- (403,283) ;
\draw [shift={(405,283)}, rotate = 180] [fill={rgb, 255:red, 0; green, 0; blue, 0 }  ][line width=0.08]  [draw opacity=0] (12,-3) -- (0,0) -- (12,3) -- cycle    ;
%Straight Lines [id:da7045381375059749] 
\draw    (238,283) -- (238,60.33) ;
\draw [shift={(238,58.33)}, rotate = 90] [fill={rgb, 255:red, 0; green, 0; blue, 0 }  ][line width=0.08]  [draw opacity=0] (12,-3) -- (0,0) -- (12,3) -- cycle    ;
%Shape: Wave [id:dp6487755568120255] 
\draw   (272.03,62.33) .. controls (253.33,65.86) and (240,69.44) .. (240,73.37) .. controls (240,79.62) and (273.65,85) .. (309,90.62) .. controls (344.35,96.25) and (378,101.63) .. (378,107.87) .. controls (378,114.12) and (344.35,119.5) .. (309,125.12) .. controls (273.65,130.75) and (240,136.13) .. (240,142.37) .. controls (240,148.62) and (273.65,154) .. (309,159.62) .. controls (344.35,165.25) and (378,170.63) .. (378,176.87) .. controls (378,183.12) and (344.35,188.5) .. (309,194.12) .. controls (273.65,199.75) and (240,205.13) .. (240,211.37) .. controls (240,217.62) and (273.65,223) .. (309,228.62) .. controls (344.35,234.25) and (378,239.63) .. (378,245.87) .. controls (378,252.12) and (344.35,257.5) .. (309,263.12) .. controls (273.65,268.75) and (240,274.13) .. (240,280.37) .. controls (240,281.38) and (240.88,282.37) .. (242.5,283.33) ;

% Text Node
\draw (70,107.93) node [anchor=south] [inner sep=0.75pt]    {$\boldsymbol{q}$};
% Text Node
\draw (407,283) node [anchor=west] [inner sep=0.75pt]   [align=left] {density};


\end{tikzpicture}

    \caption{层状液晶的密度分布:一个方向上出现了空间平移对称性破缺,从而有一个波矢,但是其它方向上密度仍然是大体上均匀的}
    \label{fig:smectic}
\end{figure}

\begin{figure}
    \centering
    \includegraphics[width=0.6\textwidth]{wavevector.PNG}
    \caption{文献\cite{PhysRevB.32.5764}中的图1,不同序的密度的非零傅里叶分量的波矢。(a) 简单的层状液晶,(b) 三棱柱状液晶,(c) 体心立方晶格,具有正八面体对称性,图中标出了六个独立的波矢方向,(d) 五棱柱状液晶,在$z$方向上没有破缺平移对称性,但是在$xy$平面上构成彭罗斯镶嵌(e) 正二十面体对称性的准晶}
    \label{fig:wavevectors}
\end{figure}

容易看出,由于\eqref{eq:free-energy-static}中各项都有$\delta$函数,只有对称性匹配的序才能对\eqref{eq:free-energy-static}中的特定一项有贡献。
以下为简便起见,我们考虑一个液体,在其中$v_4$不重要,只有$v_3$和$v_5$项是重要的。显然,如果某种序中能够找到三个波矢和为零的傅里叶分量,那么这个序会对$v_3$项有贡献,否则就没有贡献。一个体心立方晶格(bcc)的点群是$O_h$,从而其对称性和正八面体完全相同;空间群在波矢上的作用仅有点群操作有非平庸的效果,因此一个bcc序对密度的贡献中,所有的波矢都可以放在一个正八面体的棱上,即
\begin{equation}
    \rho(\vb*{r}) = \sum_\text{$\vb*{q}_i$ in octahedron edge} \frac{\rho}{\sqrt{6}} \cos(\vb*{q}_i \cdot \vb*{r} + \theta_i) + \cdots,
\end{equation}
其中$\cdots$指的是bcc格点的内部结构导致的高次谐波,此处略去;一个正八面体有12条棱,但是为了保证空间倒转对称性,我们需要让正对的两条棱上的波矢符号相反,因此只需要求和6条棱;这里我们已经将$\vb*{q}_i$和$-\vb*{q}_i$项的贡献加了起来,并且做了正确的归一化。
计算得到最低的自由能为
\begin{equation}
    (F_3)_\text{min} = - \frac{2 \rho^3 v_3}{3 \sqrt{6}}.
    \label{eq:bcc-free-energy}
\end{equation}
另一种对$v_3$项有贡献的序是截面为三角形的棒状液晶,但是由于一个正八面体中有更多首尾相连构成三角形的波矢,一般来说这种棒状液晶不如bcc稳定。

类似的,$v_5$项只能由一个波矢中有五个波矢能够首尾相连成五边形的序“激活”,如具有$C_5$对称性的序。注意这个序不是周期的,因为五个波矢不可公度,但是既然我们是在讨论平滑化的密度,没有周期性毫无影响。一种可能的序是截面为五边形的棱柱状液晶,在$z$轴上是比较连续的,但是在$xy$平面上是非周期性密铺,其波矢结构如\prettyref{fig:wavevectors}(d),密度为
\begin{equation}
    \rho(\boldsymbol{r})=\sum_{i=1}^{5} \frac{\rho}{\sqrt{5}} \cos \left(\boldsymbol{q}_{i} \cdot \boldsymbol{r}+\theta_{i}\right),
\end{equation}
其自由能的极小值为
\begin{equation}
    (F_{5})_\text{min} =-\frac{v_{5}}{25 \sqrt{5}} \rho^{5}.
    \label{eq:rodlike-lyotropic-free-energy}
\end{equation}

还有一个更加结构更加复杂的序能够同时激活$v_3$项和$v_5$项:波矢结构为\prettyref{fig:wavevectors}(e)中的正二十面体的准晶。
可以看到,\prettyref{fig:wavevectors}(e)中的波矢既能够形成首尾相连的三角形,也能够形成首尾相连的五边形,因此对$v_3$项和$v_5$项都有贡献。
与前述两个序类似,可以计算出这种准晶序的最低的自由能是
\begin{equation}
    \left(F_{3}+F_{5}\right)_{\min }=-\frac{2 \rho^{3} v_{3}}{75 \sqrt{15}}-\frac{2 \rho^{5} v_{5}}{3 \sqrt{15}}.
    \label{eq:quasi-crystal-free-energy}
\end{equation}
比较\eqref{eq:bcc-free-energy},\eqref{eq:rodlike-lyotropic-free-energy}和\eqref{eq:quasi-crystal-free-energy},可以发现\eqref{eq:quasi-crystal-free-energy}总是低于\eqref{eq:rodlike-lyotropic-free-energy},即准晶序总是比五棱柱液晶序稳定,但是\eqref{eq:bcc-free-energy}和\eqref{eq:quasi-crystal-free-energy}的大小关系在$v_3$和$v_5$值没有给定时是不能确定的。
因此,调节$v_3$和$v_5$项,可以让正二十面体对称性的准晶序和正八面体对称性的bcc晶体序互相转化。

综上所述,我们可以看到,在结晶过程的金斯堡-朗道理论描述中,准晶序除了对称性和晶体序不同以外,其它没有任何不同;没有什么对称性上条件要求\eqref{eq:free-energy-static}中不能出现$v_5$之类的项,而有这种项出现,准晶序就能够产生。因此,在特定的参数下,准晶序出现是稳定且非常自然的。
事实上,基于常见的刻画电中性粒子间相互作用的兰纳德-琼斯势进行的平衡态蒙特卡洛模拟表明,一个简单的二成分系统的平衡态就是十重旋转对称的准晶(如\prettyref{fig:10fold}所示),而且甚至不是彭罗斯结构的\cite{PhysRevLett.58.706}。

\begin{figure}
    \centering
    \includegraphics[width=0.7\textwidth]{10fold.PNG}
    \caption{文献\cite{PhysRevLett.58.706}中的图1,二成分兰纳德-琼斯势体系的低能态。注意这里黑色原子和白色原子的位置关系相比彭罗斯镶嵌更加“随意”,没有明确的镶嵌规则。}
    \label{fig:10fold}
\end{figure}

\section{准晶的生长}

在说明了理论上确实可以存在稳定的准晶序之后,我们讨论准晶生长的动力学过程。对动力学过程的研究是非常重要的,因为实际的凝聚态系统中充斥着各种亚稳态和寿命较长的瞬态\cite{PhysRevB.75.064107},一些诸如玻璃化的重要物理现象本质上就是动力学的\cite{mct-primer}。
原则上,对结晶的研究完全可以通过分子动力学模拟实现,但由于计算资源的限制,能够模拟的时间和空间尺度都极其受限\cite{PhysRevLett.88.245701}。因而,现实的准晶生长的动力学必定需要一些低能有效模型。
动态密度泛函理论是一种常见的理论框架\cite{pfc2009,PhysRevB.75.064107},其中系统的自由能被写成系统密度的泛函,形如
\begin{equation}
    F[\rho(\boldsymbol{r})]=F_{\mathrm{id}}[\rho(\boldsymbol{r})]+F_{\mathrm{ex}}[\rho(\boldsymbol{r})]+F_{\mathrm{ext}}[\rho(\boldsymbol{r})],
\end{equation}
其中
\begin{equation}
    F_{\text {id }}[\rho(\boldsymbol{r})]=k_{B} T \int d \boldsymbol{r} \rho(\boldsymbol{r})\left\{\ln \left[\rho(\boldsymbol{r}) \Lambda^{d}\right]-1\right\}
\end{equation}
为理想气体的自由能密度泛函,而
\begin{equation}
    F_{\text {ext }}[\rho(\boldsymbol{r})]=\int \dd^3 \boldsymbol{r} \rho(\boldsymbol{r}) V(\boldsymbol{r}, t)
\end{equation}
为外界势场和密度的耦合,而
\begin{equation}
    \begin{aligned}
        {F}_\text{ex} / k_{B} T=& \int d \boldsymbol{x}\left\{\rho(\boldsymbol{r}) \ln \left[\rho(\boldsymbol{r}) / \rho_{l}\right]-\delta \rho(\boldsymbol{r})\right\} \\
        &-\sum_{n=2}^{\infty} \frac{1}{n !} \int \prod_{i=1}^{n} d \boldsymbol{r}_{i} \delta \rho\left(\boldsymbol{r}_{i}\right) C_{n}\left(\boldsymbol{r}_{1}, \boldsymbol{r}_{2}, \boldsymbol{r}_{3}, \ldots, \boldsymbol{r}_{n}\right)
        \end{aligned}
\end{equation}
为液体内相互作用引入的修正,其中$\var{\rho}(\vb*{r})$为$\rho(\vb*{r})$偏离平衡密度的多少,而$C_n$为$n$点关联函数。
这三项中的参数可以第一性原理地获得。
在假定系统过阻尼、阻尼相比其它动力学过程明显很多,以及动态的系统状态仍然可以完全使用密度刻画(而忽略流量等其它物理量)后,可以使用郎之万方程导出密度的运动方程\cite{pfc2009,PhysRevB.75.064107}
\begin{equation}
    \dot{\rho}(\mathbf{r}, t)=\gamma^{-1} \nabla \cdot\left[\rho(\mathbf{r}, t) \nabla \frac{\delta F[\rho(\mathbf{r}, t)]}{\delta \rho(\mathbf{r}, t)}\right].
    \label{eq:ddft}
\end{equation}
这是一个非线性的方程,通常称为动态密度泛函理论(dynamic density functional theory, DDFT),解之可获得结晶过程\cite{Neuhaus_2014},但仍然耗费较多计算资源。前一节介绍的金斯堡-朗道理论在唯象地引入动力学之后可以更加高效地模拟结晶,称为相场模型,已经在材料科学中取得了广泛应用,但一般的相场模型通常直接将周期性的密度场平滑化(如见\prettyref{fig:phase-field})到看不清其晶格结构,即假定晶相内部完全是均匀各向同性的,这会遗漏弹性各向异性、晶向等信息。
一种兼顾计算简单和物理图像完整的理论是相场晶体(phase field crystal, PFC)方法,它基于一个简化了的自由能密度泛函和将\eqref{eq:ddft}右边括号内的$\rho(\vb*{r})$用稳定时的密度替换得到的时间演化方程,形式上很像普通的相场理论,但是可以从动力学经典密度泛函理论推导出,且平衡时密度分布是(准)周期性的\cite{pfc2009,PhysRevB.75.064107}。

\begin{figure}
    \centering
    \includegraphics[width=0.6\textwidth]{phase-field.PNG}
    \caption{相场的物理解释,来自文献\cite{boettinger2002phase}中的图1,其中$\phi(x)$为相场}
    \label{fig:phase-field}
\end{figure}

我们做近似
我们考虑\cite{PhysRevB.19.2775}导出,用于一个第一性原理的结晶过程的静态密度泛函理论,

引入$\nabla^4$项的好处在于,此时\cite{PhysRevE.70.051605}

\section{准晶中的电子态}

到目前为止,我们都只是在讨论准晶的结构,而将电子视为提供原子间等效相互作用的中间媒介。但正如传统固体物理中,晶格为库仑相互作用电子气提供了周期性背景一样,准晶为电子提供了由不可公度的频率成分构成的非周期性背景。
准晶中的电子态因此值得特别讨论。

\section{结论}

\bibliographystyle{plain}
\bibliography{main} 

\end{document}