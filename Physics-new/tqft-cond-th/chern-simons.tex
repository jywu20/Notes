\documentclass[hyperref, a4paper]{article}

\usepackage{geometry}
\usepackage{titling}
\usepackage{titlesec}
% No longer needed, since we will use enumitem package
% \usepackage{paralist}
\usepackage{enumitem}
\usepackage{footnote}
\usepackage{marginnote}
\usepackage{enumerate}
\usepackage{amsmath, amssymb, amsthm}
\usepackage{mathtools}
\usepackage{bbm}
\usepackage{cite}
\usepackage{graphicx}
\usepackage{subfigure}
\usepackage{physics}
\usepackage{tensor}
\usepackage{siunitx}
\usepackage[version=4]{mhchem}
\usepackage{tikz}
\usepackage{xcolor}
\usepackage{listings}
\usepackage{autobreak}
\usepackage[ruled, vlined, linesnumbered]{algorithm2e}
\usepackage{nameref,zref-xr}
\zxrsetup{toltxlabel}
\zexternaldocument*[optics-]{../optics/optics}[optics.pdf]
\zexternaldocument*[solid-]{../solid/solid}[solid.pdf]
\usepackage[colorlinks,unicode]{hyperref} % , linkcolor=black, anchorcolor=black, citecolor=black, urlcolor=black, filecolor=black
\usepackage[most]{tcolorbox}
\usepackage{prettyref}

% Page style
\geometry{left=3.18cm,right=3.18cm,top=2.54cm,bottom=2.54cm}
\titlespacing{\paragraph}{0pt}{1pt}{10pt}[20pt]
\setlength{\droptitle}{-5em}
\preauthor{\vspace{-10pt}\begin{center}}
\postauthor{\par\end{center}}

% More compact lists 
\setlist[itemize]{
    itemindent=17pt, 
    leftmargin=1pt,
    listparindent=\parindent,
    parsep=0pt,
}

% Math operators
\DeclareMathOperator{\timeorder}{\mathcal{T}}
\DeclareMathOperator{\diag}{diag}
\DeclareMathOperator{\legpoly}{P}
\DeclareMathOperator{\primevalue}{P}
\DeclareMathOperator{\sgn}{sgn}
\newcommand*{\ii}{\mathrm{i}}
\newcommand*{\ee}{\mathrm{e}}
\newcommand*{\const}{\mathrm{const}}
\newcommand*{\suchthat}{\quad \text{s.t.} \quad}
\newcommand*{\argmin}{\arg\min}
\newcommand*{\argmax}{\arg\max}
\newcommand*{\normalorder}[1]{: #1 :}
\newcommand*{\pair}[1]{\langle #1 \rangle}
\newcommand*{\fd}[1]{\mathcal{D} #1}
\DeclareMathOperator{\bigO}{\mathcal{O}}

% TikZ setting
\usetikzlibrary{arrows,shapes,positioning}
\usetikzlibrary{arrows.meta}
\usetikzlibrary{decorations.markings}
\tikzstyle arrowstyle=[scale=1]
\tikzstyle directed=[postaction={decorate,decoration={markings,
    mark=at position .5 with {\arrow[arrowstyle]{stealth}}}}]
\tikzstyle ray=[directed, thick]
\tikzstyle dot=[anchor=base,fill,circle,inner sep=1pt]

% Algorithm setting
% Julia-style code
\SetKwIF{If}{ElseIf}{Else}{if}{}{elseif}{else}{end}
\SetKwFor{For}{for}{}{end}
\SetKwFor{While}{while}{}{end}
\SetKwProg{Function}{function}{}{end}
\SetArgSty{textnormal}

\newcommand*{\concept}[1]{{\textbf{#1}}}

% Embedded codes
\lstset{basicstyle=\ttfamily,
  showstringspaces=false,
  commentstyle=\color{gray},
  keywordstyle=\color{blue}
}

% Reference formatting
\newrefformat{fig}{Figure~\ref{#1} on page~\pageref{#1}}

% Color boxes
\tcbuselibrary{skins, breakable, theorems}
\newtcbtheorem[number within=section]{warning}{Warning}%
  {colback=orange!5,colframe=orange!65,fonttitle=\bfseries, breakable}{warn}
\newtcbtheorem[number within=section]{note}{Note}%
  {colback=green!5,colframe=green!65,fonttitle=\bfseries, breakable}{note}

\title{Field Theories for Anyons}
\author{Jinyuan Wu}

\begin{document}

\maketitle

\section{Abelian anyons and Chern-Simons theory}

This section reviews what is covered in \cite{viefers_anyons}.
Briefly speaking, what we are going to do is to encode the phases of anyons into a gauge theory.
We expect a gauge theory will do the job because we can always mimic the exchange phases of anyons 
by considering the anyons as particles created by a bosonic field coupled to a gauge field.
The particles created by the bosonic field carry both a flux and a charge, 
and thus the Aharonov–Bohm effect can reproduce the correct self and mutual phases. 
What we are going to talk about in this section are Abelian anyons, so things will not be too hard.
We claim after the exchange phases are correctly encoded into a gauge theory, 
we can obtained a \emph{complete} low energy effective theory of a topological order,
since we are usually not interested in the scattering of anyons, and the low energy dynamics of an 
anyon system has nothing more than the exchange phases.

Why we use a gauge field theory instead of something like a Coulomb interaction is discussed in Section~3.1
in \cite{viefers_anyons}. In summary, it is because with a Coulomb-like Hamiltonian we often run into 
duplicate phase factors. Another approach is to construct anyon field operators. It is possible, but 
the resulting theories are usually highly complicated and technical.  

Before discussing what gauge field theory we want to use we can first narrow down the candidates.
We would better use a \emph{topological quantum field theory (TQFT)}, which has no non-trivial particle dynamics.
A type of well-studied TQFT is \concept{Chern-Simons theories} -- and they are what we are going to use.
We consider, for example, the following theory where Schrödinger bosons are coupled to a single gauge field,
and there is a Chern-Simons term for the gauge field itself:
\begin{equation}
    \mathcal{L}= \ii \bar{\phi} D_{0} \phi+\frac{1}{2 m} \bar{\phi} \vb*{D}^{2} \phi
    +\underbrace{\frac{1}{4 \theta} \epsilon^{\mu \nu \lambda} a_{\mu} \partial_{\nu} a_{\lambda}}_{\mathcal{L}_\text{CS}},
    \label{eq:simplest-cs-theory}
\end{equation} 
where 
\begin{equation}
    D_{\mu}=\partial_{\mu}+ \ii a_{\mu}.
\end{equation}

We have not proved that \eqref{eq:simplest-cs-theory} is indeed gauge invariant yet. Under a local transformation
\[
    a_\mu \longrightarrow a_\mu + \partial_\mu h,
\]
we have
\[
    \begin{aligned}
        \frac{1}{4\theta} \epsilon^{\mu \nu \lambda} a_\mu \partial_\nu a_\lambda &\longrightarrow 
        \frac{1}{4\theta} \epsilon^{\mu \nu \lambda} (a_\mu + \partial_\mu h) \partial_\nu (a_\lambda + \partial_\lambda h) \\
        &= \frac{1}{4 \theta} (\epsilon^{\mu \nu \lambda} a_\mu \partial_\nu a_\lambda 
        + \epsilon^{\mu \nu \lambda} \partial_\mu h \partial_\nu a_\lambda
        + \epsilon^{\mu \nu \lambda} a_\mu \partial_\nu \partial_\lambda h
        + \epsilon^{\mu \nu \lambda} \partial_\mu h \partial_\nu \partial_\lambda h) \\
        &= \frac{1}{4 \theta} (\epsilon^{\mu \nu \lambda} a_\mu \partial_\nu a_\lambda 
        + \epsilon^{\mu \nu \lambda} \partial_\mu( h \partial_\nu a_\lambda) 
        - \epsilon^{\mu \nu \lambda} h \partial_\nu \partial_\mu a_\lambda ) \\
        &= \frac{1}{4 \theta} (\epsilon^{\mu \nu \lambda} a_\mu \partial_\nu a_\lambda 
        + \epsilon^{\mu \nu \lambda} \partial_\mu( h \partial_\nu a_\lambda)),
    \end{aligned}
\]
where the last two terms in the second line vanish because $\epsilon^{\mu \nu \lambda} a_\mu a_\nu = 0$, and 
this also how we get the last line. Note that though after the transformation the Lagrangian \emph{density}
changes, after integrating over the whole space the total Lagrangian does \emph{not} change because the second 
term in the last line above is a boundary term. So we find that in \eqref{eq:simplest-cs-theory}, the 
Chern-Simons part is indeed gauge invariant, though it cannot be written in terms of $\vb*{E}$ and $\vb*{B}$.
The coupling term in \eqref{eq:simplest-cs-theory} transforms in the same way of QED, where the bosonic 
field undergoes a local $U(1)$ transformation. Thus, \eqref{eq:simplest-cs-theory} is a 
\emph{$U(1)$ Chern-Simons theory}.

Now we evaluate the EOM of \eqref{eq:simplest-cs-theory}. The Chern-Simons terms gives 
\[
    \begin{aligned}
        \partial_\rho \pdv{\mathcal{L}}{\partial_\rho a_\sigma} - \pdv{\mathcal{L}}{a_\sigma} &= 
        \frac{1}{4 \theta} \partial_\rho \epsilon^{\mu \rho \sigma} a_\mu - \frac{1}{4\theta} \epsilon^{\sigma \nu \lambda} \partial_\nu a_\lambda \\
        &= -  \frac{1}{2 \theta} \epsilon^{\sigma \nu \lambda} \partial_\nu a_\lambda.
    \end{aligned}
\]
Another way to find the above equation is to use integration by parts to rephrase all terms 
containing $a_\mu \partial_\rho a_\sigma$ into $- \partial_\rho a_\mu a_\sigma$, so the first term in the 
LHS of the first line vanishes, and we get a $2$ factor.
In the Schrödinger field part, terms involving $a_\mu$ are 
\[
    \begin{aligned}
        &\quad - \bar{\phi} a_0 \phi + \frac{\ii}{2m} \bar{\phi} \partial_i (a_i \phi) 
        + \frac{\ii}{2m} \bar{\phi} a_i \partial_i \phi - \frac{1}{2m} a_i^2 \bar{\phi} \phi \\
        &= - \bar{\phi} a_0 \phi - \frac{\ii}{2m} (\partial_i \bar{\phi}) a_i \phi 
        + \frac{\ii}{2m} \bar{\phi} a_i \partial_i \phi - \frac{1}{2m} a_i^2 \bar{\phi} \phi  ,
    \end{aligned}
\]
and we have 
\[
    \partial_\rho \pdv{\mathcal{L}}{\partial_\rho a_0} - \pdv{\mathcal{L}}{a_0} = \bar{\phi} \phi,
\]
and 
\[
    \partial_\rho \pdv{\mathcal{L}}{\partial_\rho a_i} - \pdv{\mathcal{L}}{a_i} 
    = \frac{a_i}{m} \bar{\phi} \phi + \frac{\ii}{2m} (\phi \partial_i \bar{\phi} - \bar{\phi} \partial_i \phi).
\]
We soon find that the RHSs of the above two equations are just conservation flows of a Schrödinger field 
coupled to a gauge field, as is the case in QED. So the EOM of $a_\mu$ is just 
\begin{equation}
    j^\sigma - \frac{1}{2 \theta} \epsilon^{\sigma \nu \lambda} \partial_\nu a_\lambda = 0.
    \label{eq:simplest-eom}
\end{equation} 
Specifically, the temporal part of the EOM is 
\begin{equation}
    \epsilon^{i j} \partial_{i} a_{j}=2 \theta \rho, \quad \rho = \bar{\phi} \phi.
\end{equation}

\section{Chern-Simons theory as an effective theory of IQHE}

Now we follow the derivation of Section~4.4 in Wen's famous textbook. \marginnote{Wen Section~4.4}
Here we define 
\begin{equation}
    \theta = \frac{\pi}{K}.
\end{equation}


\bibliographystyle{plain}
\bibliography{../topological-seminar-fd/field-theory} 

\end{document}