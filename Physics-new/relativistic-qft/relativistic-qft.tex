\documentclass[hyperref, UTF8, a4paper]{ctexbook}

\usepackage{geometry}
\usepackage{titling}
\usepackage{titlesec}
\usepackage{paralist}
\usepackage{footnote}
\usepackage{enumerate}
\usepackage{amsmath, amssymb, amsthm}
\usepackage{mathtools}
\usepackage{simplewick}
\usepackage{cite}
\usepackage{graphicx}
\usepackage{subfigure}
\usepackage{physics}
\usepackage{tensor}
\usepackage{siunitx}
\usepackage{slashed}
\usepackage{centernot}
\usepackage{tikz}
\usepackage[compat=1.1.0]{tikz-feynhand}
\usepackage{nameref,zref-xr}
\usepackage{mathrsfs}
\usepackage{gensymb}
\usepackage{booktabs}
\zxrsetup{toltxlabel}
\zexternaldocument*[solid-]{../solid/solid}[solid.pdf]
\zexternaldocument*[homework1-]{../qft1-homework/1/1}[1.pdf]
\usepackage[colorlinks, linkcolor=black, anchorcolor=black, citecolor=black, filecolor=black]{hyperref}
\usepackage{prettyref}

\geometry{left=3.18cm,right=3.18cm,top=2.54cm,bottom=2.54cm}
\titlespacing{\paragraph}{0pt}{1pt}{10pt}[20pt]
\setlength{\droptitle}{-5em}
\preauthor{\vspace{-10pt}\begin{center}}
\postauthor{\par\end{center}}

\DeclareMathOperator{\timeorder}{T}
\DeclareMathOperator{\diag}{diag}
\newcommand*{\ii}{\mathrm{i}}
\newcommand*{\ee}{\mathrm{e}}
\newcommand*{\const}{\mathrm{const}}
\newcommand*{\comment}{\paragraph{注记}}
\newcommand{\fsl}[1]{{\centernot{#1}}}
\newcommand*{\reals}{\mathbb{R}}
\newcommand*{\complexes}{\mathbb{C}}

\newcommand*{\fd}[1]{\mathcal{D} #1}

\newcommand*{\bigO}[1]{\mathcal{O}{#1}}

\newrefformat{sec}{第\ref{#1}节}
\newrefformat{note}{注\ref{#1}}
\newrefformat{chap}{第\ref{#1}章}
\newrefformat{part}{第\ref{#1}部分}
\renewcommand{\autoref}{\prettyref}

\newenvironment{bigcase}{\left\{\quad\begin{aligned}}{\end{aligned}\right.}

\newcommand{\concept}[1]{\underline{\textbf{#1}}}
\renewcommand{\emph}{\textbf}

\newcommand{\normord}[1]{\vcentcolon\mathrel{#1}\vcentcolon}
\providecommand{\vcentcolon}{\mathrel{\mathop{:}}}

\tikzfeynhandset{
    every boldfermion@@/.style={
    /tikz/draw=none,
    /tikz/decoration={name=none},
    /tikz/postaction={
            /tikz/draw,
            /tikz/double,
            /tikz/line width = \feynhandlinesize,
            /tikzfeynhand/with arrow=0.5,
        },
    },
    every boldfermion/.style={/tikzfeynhand/every boldfermion@@/.append style={#1}},
    boldfermion/.style={
    /tikzfeynhand/every boldfermion@@,
    }
}

\newcommand{\soliddoc}{\href{../solid/solid}{固体物理笔记}}
\newcommand{\homeworkone}{\href{../qft1-homework/1}{作业1}}

\allowdisplaybreaks[4]

\title{相对论性量子场论}
\author{吴晋渊,郭家祺}

\begin{document}

\maketitle

\vspace{2em}

如无特殊说明,本文所谓的本征态指的都是归一化。
希腊字母的指标跑遍所有时空维度,而拉丁字母的指标仅仅跑遍空间维度,也就是$\mu, \nu, \ldots = 0, 1, 2, 3$而$i, j, \ldots = 1, 2, 3$。
常规斜体字母$x, y, p$等若经说明为多分量对象,默认为四维矢量,相应的,$\vb*{x}, \vb*{y}, \vb*{p}$等为它们的空间部分。
常规斜体字母的点乘表示四维矢量乘法,如
\[
    p^2 = \omega^2 - \abs{\vb*{p}}^2, \quad x \cdot y = x^0 y^0 - \vb*{x} \cdot \vb*{y} = t_x t_y - \vb*{x} \cdot \vb*{y}.
\]
指标$a,b,\ldots$也有可能指各种多分量对象的指标,未必正好取$1, 2, 3$。
$T$表示编时算符,$N$表示正规序算符。

张量的分量矩阵默认以排在前面的指标为行指标,以排在后面的指标为列指标。

本文将略去所有最为基础的formalism:量子力学的基本框架——包括路径积分和正则形式,无论是关于单粒子的还是关于场的;如何将多粒子态写成场的激发态(即所谓二次量子化);微扰论的形式理论等等。
这些东西可以在任何好的量子场论教科书中找到。

\part{相对论性量子场论}

\chapter{量子场论的形式理论框架}

物理实验表明存在一个时间轴,且存在\concept{粒子},它的自由度在不考虑引力导致的空间翘曲时包括一个三维欧氏空间,可能还有一些内禀自由度。
狭义相对论还告诉我们,移动参考系,时间和空间的变换是洛伦兹变换,这对应一个$3+1$维闵可夫斯基时空,即度规可以化为
\[
    \eta_{\mu\nu} = \diag (1, -1, -1, -1)
\]
的四维几何。通常使用$t, x, y, z$或者$x^0, x^1, x^2, x^3$来依次标记这4个坐标。
容易看出$x, y, z$或者说$x^1, x^2, x^3$就构成一个三维欧氏几何,它们是\concept{空间维}。%
$x^0$则是\concept{时间维}。

在狭义相对论中难以使用哈密顿动力学,因为此时“时间”的概念是不清楚的:应该使用固有时还是坐标时?
因此,简单地将单粒子量子力学移植到相对论时空中不是好的选择。
除此之外单粒子量子力学还存在一个很大的问题。为了让粒子有非平凡的运动,通常会往粒子的哈密顿量中引入一个势能,然而势能必须由另一些东西提供。
如果这个“东西”的参数发生了变化,似乎应该认为势能发生了瞬时的变化,但这样就有信号的瞬时传递了。这和狭义相对论当然是矛盾的。

场论是解决这个问题的一种方案。就刚才势场的问题,我们知道,实际上能够看到的大部分势场都是电磁相互作用产生的,的确,通过推迟势能够解决信号瞬时传输的问题,但是更加自然的做法是引入\emph{电磁场}的概念:物质激发出电磁场,电磁场反作用于物质。
大量的场论能够很自然地服从狭义相对论——例如,电动力学在洛伦兹变换下不变。
因此,将已知在相对论情况下运作良好的场论量子化似乎是一个相当吸引人的做法。

这个想法导致了我们称为\concept{量子场论}的一整套理论。在比较系统性的教科书上,我们会看到以下形式理论:
\begin{enumerate}
    \item 给定一个场论拉氏量$\mathcal{L}$和它做勒让德变换得到的哈密顿量密度$\mathcal{H}$。我们下面将单粒子量子力学中使用过的方案移植到场论中。识别出场$\phi$和它的正则动量$\pi$之后,我们施加正则对易关系
    \begin{equation}
        \comm*{\phi(\vb*{x}, t)}{\pi(\vb*{y}, t)} = \ii \delta^{(3)}(\vb*{x} - \vb*{y}),
    \end{equation}
    或者,为了后面我们将会看到的导出费米子的目的,施加反对易关系
    \begin{equation}
        \acomm*{\phi(\vb*{x}, t)}{\pi(\vb*{y}, t)} = \ii \delta^{(3)}(\vb*{x} - \vb*{y}).
    \end{equation}
    在这里,$\phi$的地位相当于量子力学中的$x$,而$\vb*{x}$则不是量子力学中的算符,而是区分不同的场自由度的标签,对应不同坐标$x_i$的标签$i$。
    在这个步骤之后我们其实已经得到了最为广义的“量子场论”:场的量子力学。
    \item 场算符的线性叠加自然给出了一组满足产生湮灭算符对易关系的$a$和$a^\dagger$;通常它们就是动量表象下的$a_{\vb*{p} \sigma}$和$a^\dagger_{\vb*{p} \sigma}$,其中$\sigma$是另一些标签,比如说标记不同场分量(“偏振”)的标签。
    于是,一个二次型的哈密顿量用这一组产生湮灭算符写出来就形如
    \begin{equation}
        H \propto \sum_{\vb*{p}, \sigma} \omega_{\vb*{p} \sigma} a^\dagger_{\vb*{p} \sigma} a_{\vb*{p} \sigma}.
    \end{equation}
    这正是谐振子的哈密顿量。经典场论中谐振子可以连续变化,而量子的谐振子的能级则是分立的;此时哈密顿量的本征态可以用$\{n_{\vb*{p} \sigma}\}$标记。
    我们认为这些量实际上就是粒子数——实际上,容易验证一个单粒子态服从的薛定谔方程正是将$\mathcal{L}$对应的波动方程中的场诠释为波函数而得到的方程。
    于是我们发现,在二次型哈密顿量下(称为\concept{自由场}),场的希尔伯特空间实际上是多粒子态的Fock空间,产生湮灭算符的标签——如动量、偏振等——正好也是单粒子的标签。
    \item 以上我们从场得到了多粒子Fock空间,当然也可以从多粒子Fock空间构造产生湮灭算符而得到场。
    \item 加入相互作用之后,可能出现粒子数生灭或者。
    因此我们发现“场的量子力学”实际上就是满足“场的哈密顿量是谐振子哈密顿量加上相互作用项之后得到的,场是粒子的产生湮灭算符,从而多粒子态可以看成场的激发态”的一个理论,也即,一种非常自然的量子多体理论。这是“量子场论”一词的更加通常的含义:通过场展开讨论的关于多粒子的量子多体理论。
    % TODO:微扰论
    \item 最后,我们可以使用量子力学中就已经做过的相干态路径积分方法,% TODO
\end{enumerate}
实际上,通常所说的“量子场论”的意义更加局限,指的是满足相对论协变性的量子场论,即高能物理的量子场论;物理中如果需要分析非相对论协变性的量子场论,一般都是在分析凝聚态系统,此时基本的场就是电子场、声子场而相互作用是库仑相互作用,称为\concept{凝聚态场论}。
凝聚态场论实际上已经非常复杂;在高能物理的量子场论中我们通常只分析少数几个粒子的散射过程,基态为真空态,而凝聚态场论中讨论含有大量粒子的基态、讨论束缚态问题都是非常常见的。
另一方面,凝聚态场论不受到相对论协变性约束,因此其中有更多可能性。虽说低速下的时空对称性是伽利略对称性,但是凝聚态场论很多是伽利略对称性也\emph{不遵守}的,例如晶格的存在本身就破坏了连续平移对称性和旋转对称性。
因此,很多时候高能物理的量子场论实际上反而更加简单,虽然凝聚态场论背后的基本物理机制实际上只有库仑相互作用。
实际上,一些人甚至认为依靠凝聚态物理中复杂的演生行为能够完全复刻出整个高能物理。

以上形式理论看起来非常优美:它将正则量子化和路径积分量子化都推广到了一般人能够想象的最广义的程度,成功统一了看起来完全不相干的粒子和场的理论。
然而我们必须承认这种优美实际上是一种假象。具体来说,以上形式理论面临如下挑战:
\begin{itemize}
    \item 首先,它难以数学上良定义。“场自由度”显然是无穷维的数学对象,一般都极其难以处理。
    此外,一般的量子场论——如QED或是QCD——中普遍存在的发散意味着实际上如上形式理论允许我们定义\emph{紫外不完备的}理论,即\emph{依照它自身的结构}就不能够适用于任意高能量的理论。
    因此,如果我们要将以上最为一般的形式理论数学化,重整化、有效理论、紫外截断等信息必须被纳入其中。
    不言而喻,这非常具有挑战性。
    \item 应当指出对场作用正则量子化是一个物理意义非常不明确的操作(这也是很多初学者无法接受量子场论的原因:完全无法理解将场量子化的动机)。
    在单粒子量子力学中,将物理量提升为算符,施加正则对易关系的目的是非常明确的:我们需要一个允许物理量不确定的理论,但是它又不是简单的经典概率理论,那么很快会发现单粒子量子力学是一种很好的备选理论;正则量子化之后粒子的波动性与被实验非常好地展现的经验事实完全一致。
    问题是,在量子场论中,我们几乎从来不讨论“波泛函”$\braket*{\phi}{\Psi}$——我们从来只考虑\emph{粒子}的行为。
    因此,量子场论中的场似乎只是起了一个辅助作用:我们用它来帮助构造一个多体理论而已。
    
    如果事情只是这样,那么还不算太糟糕。我们可以将“场的量子化”当成一个陈旧的历史术语,把它当成施加产生湮灭算符对易或反对易关系的一个简便写法。
    然而应当注意,量子场论取得的所有成功全部是关于粒子散射实验和$S$矩阵的——看着拉氏量写下费曼规则计算$S$矩阵的步骤是确定无疑非常有用的,即使它们可能未必是终极理论的形式理论。
    现在问题来了:从“场的量子力学”——以下将$\mathcal{L}$对应的场的量子力学记作$\mathrm{QMFT}(\mathcal{L})$——能够推导出“看着拉氏量写下$S$矩阵”需要的各种规则——以下记作$\mathrm{SMat}(\mathcal{L})$,但是真的被实验证实可靠的只有$\mathrm{SMat}(\mathcal{L})$,而它实际上\emph{不需要}$\mathcal{QMFT}(\mathcal{L})$就能够被定义。
    除了$\mathrm{SMat}(\mathcal{L})$以外的$\mathrm{QMFT}(\mathcal{L})$的形式理论反而变成了累赘——甚至于“$\phi$和$\pi$构成的哈密顿量”是否真的有物理意义我们都是不知道的。
    \item 例如,我们实际上不知道$\mathrm{QMFT}(\mathcal{L})$的束缚态预言是不是和实验符合。
    量子力学能够正确处理束缚态,但是没有人能够完整地证明,比如说,$\mathrm{QMFT}(\mathcal{L}_\text{QED})$和“彼此库伦排斥的电子”的束缚态量子力学理论是一致的。
    我们只能够证明$\mathrm{SMat}(\mathcal{L}_\text{QED})$中电子-电子散射的概率振幅在非相对论极限下和库伦散射一致,从而,勉强算是(说“勉强”见下一条疑难)证明了$\mathrm{QMFT}(\mathcal{L}_\text{QED})$的散射态行为和库伦散射一致,但是它的束缚态部分到底如何是不好说的。
    按理说,两个量子理论有一样的束缚态行为和不同的散射态行为是完全可能的事情,所以$\mathrm{QMFT}(\mathcal{L}_\text{QED})$对束缚态电子的预言居然不对——或者它对腔体中的光子的预言居然不对——并不是一个可以轻易排除的可能。
    \item 最后,实际上$\mathrm{QMFT}(\mathcal{L})$和$\mathrm{SMat}(\mathcal{L})$的散射态是不是能够说一样其实也成问题。
    我们知道一个一般的场论$\mathcal{L}$的$\mathrm{SMat}(\mathcal{L})$给出的微扰级数一般来说是不收敛的,是一个渐进级数,我们需要按照一定的准则来决定保留前几阶;然而如果$\mathrm{QMFT}(\mathcal{L})$是良定义的,至少我们得想出一种自圆其说的办法让整个微扰级数都有明确意义。
\end{itemize}

基于以上原因,最好还是将$\mathrm{QMFT}(\mathcal{L})$当成一种推导$\mathrm{SMat}(\mathcal{L})$的启发式方法,而不是什么巍峨堂皇的理论体系。
毕竟严格来说既然我们对其束缚态行为知之甚少,眼下它甚至不能用于从QED推导出凝聚态场论和量子力学。
它在高能物理中的唯一作用就是让人们能够接受$\mathrm{SMat}$的规则。
另一方面,凝聚态场论倒的确是定义良好的一个$\mathrm{QMFT}$意义下的场论:直接将多电子哈密顿量用二次量子化形式理论写出即可:
\begin{equation}
H = \int \dd[3]{\vb*{r}} \psi^\dagger_\sigma \left( - \frac{\laplacian}{2m} + V_\text{ion}(\vb*{r}) \right) \psi_\sigma + \frac{1}{2} \int \dd[3]{\vb*{r}} \dd[3]{\vb*{r}'} \psi^\dagger_{\sigma'}(\vb*{r}') \psi^\dagger_\sigma(\vb*{r}') \frac{e^2}{\abs{\vb*{r} - \vb*{r}'}} \psi_\sigma(\vb*{r}) \psi_{\sigma'}(\vb*{r}),
\label{eq:condense-qft}
\end{equation}
但是凝聚态场论中$\mathrm{SMat}$没有太大意义;因此,直接从凝聚态场论的第一性原理哈密顿量其实计算不出太多东西。
%我们将在\soliddoc中讨论简单的\eqref{eq:condense-qft}如何导致一整个学科为它建立——而能够解决的问题还只是九牛一毛。

\documentclass[UTF8, a4paper]{ctexart}

\usepackage{geometry}
\usepackage{titling}
\usepackage{titlesec}
\usepackage{paralist}
\usepackage{footnote}
\usepackage{enumerate}
\usepackage{amsmath, amssymb, amsthm}
\usepackage{cite}
\usepackage{graphicx}
\usepackage{subfigure}
\usepackage{physics}
\usepackage{tikz}
\usepackage[colorlinks, linkcolor=black, anchorcolor=black, citecolor=black]{hyperref}

\geometry{left=2.5cm,right=2.5cm,top=2.5cm,bottom=2.5cm}
\titlespacing{\paragraph}{0pt}{1pt}{10pt}[20pt]
\setlength{\droptitle}{-5em}
\preauthor{\vspace{-10pt}\begin{center}}
\postauthor{\par\end{center}}

\newcommand*{\diff}{\mathop{}\!\mathrm{d}}
\newcommand*{\st}{\quad \text{s.t.} \quad}
\newcommand*{\const}{\mathrm{const}}
\newcommand*{\comment}{\paragraph{注记}}
\newcommand*{\warning}{\paragraph{注意}}
\newcommand*{\ii}{\mathrm{i}}
\newcommand*{\ee}{\mathrm{e}}
\newcommand*{\reals}{\mathbb{R}}
\newcommand*{\complexes}{\mathbb{C}}
\newcommand{\average}[1]{\langle #1 \rangle}

\newenvironment{bigcase}{\left\{\quad\begin{aligned}}{\end{aligned}\right.}

\title{对称性}
\author{wujinq}

\begin{document}

\maketitle

\section{常见对称性}

$S_n$ n阶排序群,有$n!$个元素
$A_n$ n阶置换(偶排列)群,有$n!/2$个元素?
$Z_n$ n阶离散旋转群,可以使用复数$\ee^{\ii 2\pi j / n}$,有$n$个元素

$SL(n, \reals)$ $n$维实矩阵中行列式为$1$的矩阵组成的群
$SL(n, \complexes)$ $n$维实矩阵中行列式为$1$的矩阵组成的群
$U(n)$ $n$维酉变换组成的群
$O(n)$ $n$维正交变换组成的群
$SO(n)$ $n$维旋转组成的群

\end{document}

\chapter{三种常见的场和它们的量子化}

\section{概述}

\subsection{正则量子化的大致手续}\label{sec:canonical-general}

在相对论性量子场论中我们仍然要求自由粒子的拉氏量具有最高的对称性,也就是说,在庞加莱群作用下没有变化,
且拉氏量只含有二次项(从而给出线性的运动方程)。%
\footnote{虽然本文主要分析正则量子化,但写出运动方程还是用的是拉氏量。这是更加方便的做法,因为正则表述在理论框架上将时间和空间分开对待了,因此不容易观察哈密顿量在洛伦兹变化之下是不是给出恒定不变的动力学。}
空间平移不变性意味着拉氏量不能显含坐标;空间各向同性意味着拉氏量中的参数必须是标量,不能出现多分量的参数。

在已经写出额拉氏量之后,我们可以计算三种场的哈密顿量,为正则量子化做好准备。
由于我们通常在欧氏空间中写出哈密顿量而在闵可夫斯基时空中讨论拉氏量,需要格外注意一点:闵可夫斯基时空的度规为$(+, -, -, -)$而欧几里得空间的度规为$(+, +, +)$,因此
\[
    A_\mu A^\mu = (\dot{A}^0)^2 - \vb*{A}^2,
\]
上式左边为闵可夫斯基时空中的表达式,右边为欧几里得空间中的表达式。
换而言之,闵可夫斯基时空中的$A_i A^i$和欧几里得空间中的$A_i A^i$差一个负号。

在得到了场的哈密顿量之后,我们据此执行正则量子化。我们接下来需要对场算符施加正则对易或反对易关系。由于空间平移不变性,我们将在动量表象下工作。此外,为了让公式看起来好看一些,我们忽略算符的$\hat{\ }$帽子。
在处理单粒子量子力学坐标表象下的问题时我们同时需要讨论算符$\hat{\vb*{x}}$和可以随意变动的空间坐标$\vb*{x}$,但是在量子场论中几乎从来不需要讨论场变量的取值,因此忽略帽子并不会造成任何问题。
在对自由场论做量子化时为了突出场的作用,我们采用海森堡绘景,即让场发生时间演化。

场算符做傅里叶变换之后的产生湮灭算符的归一化涉及一个微妙的、和相对论特性有关的地方。
考虑到\eqref{eq:relativity-p},我们有
\[
    \ket{p, \sigma} = \sqrt{2 \omega_{\vb*{p}}} \ket{\vb*{p}, \sigma},
\]
于是设${a}_{\vb*{p}, \sigma}$和${a}^\dagger_{\vb*{p}, \sigma}$为单粒子态$\ket{\vb*{p}, \sigma}$对应的产生湮灭算符,则有
\[
    ({\alpha}_{\vb*{p}, \sigma})^\dagger = \sqrt{2\omega_{\vb*{p}}} {a}_{\vb*{p}, \sigma}^\dagger,
\]
这样场算符的展开式就是
\[
    {\varphi}(x) \propto \sum_\sigma \int \frac{\dd[3]{\vb*{p}}}{\sqrt{(2\pi)^3 2 \omega_{\vb*{p}}}} \left( {a}^\dagger_{\vb*{p}, \sigma} \ee^{\ii p_\mu x^\mu} + {a}_{\vb*{p}, \sigma} \ee^{- \ii p_\mu x^\mu} \right) e_\sigma,
\]
其中$e_\sigma$标记场$\phi$属于的场表示的有限维部分的基底。
或者,既然我们已经转而在三维空间中讨论问题,即已经不再要求洛伦兹协变性了,可以使用三维矢量更加清晰地写出
\begin{equation}
    {\varphi}(\vb*{x}, t) \propto \sum_\sigma \int \frac{\dd[3]{\vb*{p}}}{\sqrt{(2\pi)^3 2 \omega_{\vb*{p}}}} \left( {a}^\dagger_{\vb*{p}, \sigma} \ee^{- \ii \vb*{p} \cdot \vb*{x} + \ii \omega_{\vb*{p}} t} + {a}_{\vb*{p}, \sigma} \ee^{\ii \vb*{p} \cdot \vb*{x} - \ii \omega_{\vb*{p}} t} \right) e_\sigma. 
    \label{eq:expanding-field-operator}
\end{equation}
由于三种场都服从克莱因-高登方程,将\eqref{eq:expanding-field-operator}代入\eqref{eq:klein-gordon-eq}会发现$\omega_{\vb*{p}}$和$\vb*{p}$正好服从质壳关系\eqref{eq:mass-shell}。
另一方面,也可以使用相对论协变的积分测度,写出
\begin{equation}
    {\varphi}(\vb*{x}, t) \propto \sum_\sigma \int \frac{\dd[3]{\vb*{p}}}{2 \omega_{\vb*{p}} (2\pi)^{3/2}} \left( {\alpha}^\dagger_{\vb*{p}, \sigma} \ee^{ \ii p_\mu x^\mu} + {\alpha}_{\vb*{p}, \sigma} \ee^{- \ii \pi_\mu x^\mu} \right) e_\sigma. 
    \label{eq:expanding-field-operator-relativity}
\end{equation}

展开式\eqref{eq:expanding-field-operator-relativity}和\eqref{eq:expanding-field-operator}有各自的好处。
\eqref{eq:expanding-field-operator-relativity}给出的产生湮灭算符以及它们产生的单粒子态是洛伦兹协变的,但是在处理对易关系的时候会略有复杂,因为此时产生湮灭算符的对易关系必定也是协变的,因此必须指定
\[
    \comm*{{\alpha}_{\vb*{p}}}{{\alpha}^\dagger_{\vb*{p}'}} \sim \omega_{\vb*{p}} \delta^3 (\vb*{p} - \vb*{p}')
\]
这样的对易关系,或者类似的反对易关系;当然,因为此时使用的积分测度是$\int \dd[3]{\vb*{p}} / (2 \omega_{\vb*{p}})$,这样的对易关系是正确的——无非是$\delta$函数要修改为某种相对论形式而已。
\eqref{eq:expanding-field-operator}给出的产生湮灭算符以及它们产生的单粒子态不是洛伦兹协变的,但是可以简化对易关系以及归一化时使用的积分测度。
例如在对所有的动量模式求和时如果我们选取积分测度为$\int \dd[3]{\vb*{p}}$,那么就需要指定
\[
    \comm*{{a}_{\vb*{p}}}{{a}^\dagger_{\vb*{p}'}} = \delta^3 (\vb*{p} - \vb*{p}'),
\]
或者类似的反对易关系。而如果指定
\[
    \comm*{{a}_{\vb*{p}}}{{a}^\dagger_{\vb*{p}'}} = (2\pi)^3 \delta^3 (\vb*{p} - \vb*{p}'),
\]
此时只需要始终使用积分测度
\[
    \int \frac{\dd[3]{\vb*{p}}}{(2\pi)^3}
\]
对所有动量模式求和即可。
两种展开式都是傅里叶变换,因此都能够消除哈密顿量中的导数。

为了避免引起混乱,这里我们统一指定下面使用的归一化常数和记号。
本文采取Peskin和Quantum Field Theory for the Gifted Amateur的归一化常数和记号的一种混合。
前面的$\ket{p}$和$\ket{\vb*{p}}$的区分是Quantum Field Theory for the Gifted Amateur中使用的记号(Peskin中没有引入$\ket{p}$,Peskin中的$\ket{\vb*{p}}$就是本文的$\ket{p}$),而本文中的归一化常数则保持和Peskin一致。
对傅里叶变换,规定对动量求和时积分测度带因子$1/(2\pi)^3$而对坐标求和时积分测度就是$\dd[3]{\vb*{x}}$。
这样,产生湮灭算符的对易关系就是
\begin{equation}
    \comm*{{a}_{\vb*{p}}}{{a}^\dagger_{\vb*{p}'}} = (2\pi)^3 \delta^3 (\vb*{p} - \vb*{p}'),
\end{equation}
或者将相对论协变的$\alpha$算符转化为差一个常数的$a$算符,即令
\begin{equation}
    \alpha^\dagger_{\vb*{p}, \sigma} = \sqrt{2 \omega_{\vb*{p}}} a^\dagger_{\vb*{p}, \sigma},
\end{equation}
就有
\begin{equation}
    \comm*{{\alpha}_{\vb*{p}}}{{\alpha}^\dagger_{\vb*{p}'}} = (2\pi)^3 2 \omega_{\vb*{p}} \delta^3 (\vb*{p} - \vb*{p}').
\end{equation}
反对易关系只需要将$\comm*{\cdot}{\cdot}$改成$\acomm*{\cdot}{\cdot}$即可。
场算符的展开和\eqref{eq:strange-p-state}遵从一样的习惯,为
\begin{equation}
    \varphi(\vb*{x}, t) = \sum_{\sigma} \int \frac{\dd[3]{\vb*{p}}}{(2\pi)^3 2 \omega_{\vb*{p}}} \left( {\alpha}^\dagger_{\vb*{p}, \sigma} \ee^{- \ii \vb*{p} \cdot \vb*{x} + \ii \omega_{\vb*{p}} t} + {\alpha}_{\vb*{p}, \sigma} \ee^{\ii \vb*{p} \cdot \vb*{x} - \ii \omega_{\vb*{p}} t} \right) e_\sigma,
\end{equation}
或者
\begin{equation}
    {\varphi}(\vb*{x}, t) = \sum_\sigma \int \frac{\dd[3]{\vb*{p}}}{(2\pi)^3} \frac{1}{\sqrt{2 \omega_{\vb*{p}}}} \left( {a}^\dagger_{\vb*{p}, \sigma} \ee^{- \ii \vb*{p} \cdot \vb*{x} + \ii \omega_{\vb*{p}} t} + {a}_{\vb*{p}, \sigma} \ee^{\ii \vb*{p} \cdot \vb*{x} - \ii \omega_{\vb*{p}} t} \right) e_\sigma. 
    \label{eq:field-operator-fourier}
\end{equation}
$\ee$指数中的$\ii \omega t - \ii \vb*{p} \cdot \vb*{x}$也可以写成相对论协变形式$\ii p \cdot x$。

对单粒子态,有
\[
    \ket{p, \sigma} = \sqrt{2 \omega_{\vb*{p}}} {a}^\dagger_{\vb*{p}, \sigma} \ket{0},
\]
从而有
\begin{equation}
    \braket{p, \alpha}{q, \beta} = (2\pi)^3 2 \omega_{\vb*{p}} \delta_{\alpha \beta} \delta^3(\vb*{p} - \vb*{q}).
\end{equation}
在某一个给定的时间,将${\phi}(\vb*{x}, t)$作用在真空态上得到
\[
    \begin{aligned}
        {\varphi}(\vb*{x}, t) \ket{0} &= \sum_\sigma \int \frac{\dd[3]{\vb*{p}}}{(2\pi)^3} \frac{1}{\sqrt{2 \omega_{\vb*{p}}}} {a}^\dagger_{\vb*{p}, \sigma} \ee^{- \ii \vb*{p} \cdot \vb*{x} + \ii \omega_{\vb*{p}} t} e_{\sigma} \ket{0} \\
        &= \sum_\sigma \int \frac{\dd[3]{\vb*{p}}}{(2\pi)^3} \frac{1}{2 \omega_{\vb*{p}}} \ee^{- \ii \vb*{p} \cdot \vb*{x} + \ii \omega_{\vb*{p}} t} e_{\sigma} \ket{p, \sigma}.
    \end{aligned}
\]

这里说明一下为什么我们采取了上面所说的傅里叶变换,而没有,比如说,改变一下正负号选取。
我们知道
\[
    \varphi^\dagger_\sigma (\vb*{x}, t) \ket{0} \sim \ket{\vb*{x}(t), \sigma},
\]
从而
\[
    \mel{0}{\varphi_\sigma (\vb*{x}, t)}{\psi} \sim \text{single particle wavefunction}.
\]
我们希望\eqref{eq:x-p-trans}成立。可以验证,取上述傅里叶变换,可以得到
\[
    \mel{0}{\varphi_\sigma (\vb*{x}, t)}{\psi} \sim \ee^{-\ii p \cdot x} \propto \ee^{\ii \vb*{p} \cdot \vb*{x}},
\]
这正是\eqref{eq:x-p-trans}。因此本节给出的傅里叶变换是正确的。
表面上看这会导致一个疑难,就是
\[
    \varphi^\dagger_\sigma (\vb*{x}, t) \ket{0} \propto \ee^{\ii \omega t},
\]
但是一个单粒子态的演化应该以$\ee^{-\ii \omega t}$为时间演化因子,但是其实这里没有任何问题:此处的$\varphi$是海森堡绘景中的,$\ket*{\vb*{x}}$以$\ee^{\ii \omega t}$时间演化,而实际的单粒子态是没有时间演化的。
那么,现在我们做一个绘景变换,在所有态上乘以一个因子$\ee^{-\ii \omega t}$,那么$\ket*{\vb*{x}}$就没有时间演化了,而实际的单粒子态以$\ee^{-\ii \omega t}$时间演化,这正是正确的薛定谔绘景。

场算符(对应于湮灭算符的场算符)有时候会被粗略地当成“相对论量子力学中的波函数”。

判断应该使用对易关系来量子化场还是应该使用反对易关系来量子化场应当遵守几个条件:
\begin{itemize}
    \item 非平凡性。哈密顿量不应该给出平凡的结果。
    \item 因果性。在某一个时空点施加相互作用只应该产生局域的影响。特别的,在一个时空点做测量不应该对与之间隔(指的是闵可夫斯基时空中的“距离”)为正的时空点产生影响。
    \item 能量正定性。哈密顿量应该可以写成产生湮灭算符的正定二次型,以避免能量无限下降。
\end{itemize}

以上给出的步骤完全描述了场算符的量子化过程。这种使用傅里叶变换得到对角化的哈密顿量的方式有时也称为\concept{正则量子化},因为它是算符的正则量子化(即施加对易或反对易关系)之后立刻可以完成的。

关于与场算符配套的真空态要说一句:在自由场下,无论采取哪种绘景,真空态$\ket{0}$或者没有时间演化,或者时间演化只是乘上一个复数因子。这是因为真空态一定是哈密顿算符的本征态。
此外,本节采用的量子化方案也体现出了一个重要的物理图像:具有确定能量$E$的粒子在经典极限下就对应着以圆频率$E$振荡的场。

最后,以上将场做傅里叶变换以消去运动方程中的导数的做法在经典情况下当然也适用。容易看出,动量为$\vb*{p}$的粒子模式的经典极限就是波矢为$\vb*{p}$的平面波;相应的,位置为$\vb*{r}$的粒子模式的经典极限就是
\[
    \phi(\vb*{r}') = \delta(\vb*{r}' - \vb*{r}).
\]
以上两个模式的振幅均不确定;量子情况下振幅是分立的而经典情况下则不是。

即使在经典场中,也存在动量和位置不能同时确定的现象。场的量子化带来的不是动量和位置不能同时确定,而是场的振幅是离散化的——经典情况下,平面波的振幅可以任意变化,而量子情况下,${\phi}$(或者别的场算符)的本征值是离散的,此时才能够良定义“粒子”。

在本节剩下的部分中,我们将会看到,标量场和矢量场只能被量子化为玻色场,而旋量场只能被量子化为费米场。
这是\concept{自旋统计定理}的特例,这个定理说,对满足洛伦兹对称性的系统,半整数自旋对应着费米子,而整数自旋对应着玻色子。
在没有洛伦兹对称性时这个结论不一定成立,实际上,对很多具有实际意义的系统——如格点系统——我们甚至没有旋转对称性,所以也无从讨论自旋。

\subsection{经典近似}

一个相对论性量子场论的退化:
\begin{itemize}
    \item 量子性的单个粒子,即系统的基本自由度是坐标和一些额外的离散标签(自旋等),在这里,近似体现在缺乏粒子生灭,对那些本身缺乏粒子生灭的理论,从量子场论退化到单粒子理论没有任何近似。
    \begin{itemize}
        \item 进一步,相对论性单粒子理论通常并没有什么意义,因此非相对论性近似往往也是必要的。
        \item 退化得到的非相对论性单粒子理论也可以做二次量子化,得到一个非相对论性量子场论。这个理论当然可以通过原本的相对论性量子场论做非相对论近似一口气得到。
        \item 还可以完全忽略量子涨落得到经典单粒子理论。
    \end{itemize}
    \item 经典场论。如果原本的相对论性量子场论能够退化成单粒子理论,则它退化成的经典场论和单粒子理论的波动力学具有一样的形式。这就是“把波函数量子化”能够行得通的原因,也是麦克斯韦方程有时候看起来就好像一个波函数的原因——光子没有特别良定义的坐标表象下的单光子量子力学,但是如果硬是要定义一个,那么大体上就是麦克斯韦方程的样子;经典的麦克斯韦方程中的平面波在量子化之后真的就是动量本征态,可以得到坐标-动量不确定性关系,等等。
    \begin{itemize}
        \item 可以做WKB近似,或者说程函近似,得到的$\vb*{k}$和经典单粒子理论是一致的。
    \end{itemize}
\end{itemize}

\section{标量场}

\subsection{实标量场的克莱因-高登方程}\label{sec:k-g-eq}

标量场的拉氏量中只能够出现$\phi$和$\partial_\mu \phi$构成的一次或二次不变量。
$\phi$构造出的一次不变量有$\phi$,二次不变量有$\phi^2$,$\partial_\mu \phi$是矢量,$\partial_\mu \phi$不可能和$\phi$缩并,而由于拉氏量中的参数都是标量它也不可能和参数缩并,因此它只能和自己缩并,得到$\partial_\mu \phi \partial^\mu \phi$。
这样我们得到
\[
    \mathcal{L} = A \phi + B \phi^2 + C \partial_\mu \phi \partial^\mu \phi.
\]
拉氏量中的$\phi$项实际上无关紧要,因为完全可以通过重新定义一个$\phi' = \phi + \const$来把一次项消除掉,于是我们略去这一项。
最后,通过重新定义长度单位和$m$,可以得到
\begin{equation}
    \mathcal{L} = \frac{1}{2} (\partial_\mu \phi \partial^\mu \phi - m^2 \phi^2).
    \label{eq:klein-gordon-lagrangian}
\end{equation}
这个拉氏量导致下面的运动方程:
\begin{equation}
    (\partial_\mu \partial^\mu + m^2) \phi = 0.
    \label{eq:klein-gordon-eq}
\end{equation}
这就是\concept{克莱因-高登方程},标量场或者说自旋0场的基本运动方程。
可以证明,为了让\eqref{eq:klein-gordon-eq}给出有物理意义的预言(例如不出现无限下降的能量,等等),应当取$m \geq 0$。

实际上,所有场的运动方程均满足克莱因-高登方程。我们将在推导其它场的运动方程之后证明这一点。
这就导致了一个重要的结果。平移生成元在场表示中为\eqref{eq:transition-inf-rep},从而
\[
    P_\mu P^\mu = - \partial_\mu \partial^\mu,
\]
于是代入\eqref{eq:klein-gordon-eq},得到
\[
    P_\mu P^\mu \phi = - \partial_\mu \partial^\mu = m^2 \phi,
\]
其中$m$为克莱因-高等方程中出现的那个$m$。也即,通过$P_\mu P^\mu$的表示的本征值(实际上就是它和恒等变换之间差的倍数,因为$P_\mu P^\mu$是卡西米尔元)得到的$m$和场的运动方程得到的$m$是一样的。
这个$m$实际上就是粒子的静质量。在\autoref{sec:sch-eq-from-kg}中我们会看到它就对应着薛定谔方程中的质量。

现在导出标量场的哈密顿表述。计算得到
\begin{equation}
    \pi = \partial_0 \phi = \dot{\phi},
    \label{eq:klein-gordon-pi}
\end{equation}
相应的
\begin{equation}
    \mathcal{H} = \frac{1}{2} \dot{\phi}^2 + \frac{1}{2} (\grad{\phi})^2 + \frac{1}{2} m^2 \phi^2.
\end{equation}
如我们希望的那样,哈氏量密度是正定的。这当然是因为我们适当地选择了$\mathcal{L}$的正负号。

另外,注意到 % TODO:好像我们还从来没有严格定义过下式?
\[
    P_0 = E, \quad P_i \vb*{e}^i = \vb*{p},
\]
在场表示中我们可以写出
\[
    E^2 - \vb*{p}^2 = m^2
\]
或者说
\[
    E^2 = m^2 + \vb*{p}^2.
\]
这正是质壳关系\eqref{eq:mass-shell}。这就提示我们,还有另一种量子化方式:做替换
\[
    E \longrightarrow \hat{E} = \ii \partial_0, \quad \vb*{p} \longrightarrow \hat{\vb*{p}} = - \ii \grad,
\]
则从能量-动量关系就可以得到克莱因-高登方程。
我们不采用这种方案,因为它隐含地引入了太多的假设:算符$E, \vb*{p}$是作用在一个算符场上而不是态矢量上;$E,\vb*{p}$是厄米算符,也即,平移群在此算符场上取幺正表示(注意这一点并不一般成立!例如,场表示中的有限维表示就常常不是幺正的),等等。
为了和经典场论中的记号保持一致,我们后面将用$\omega$代替$E$。
$\omega$或者说场的傅里叶分量的频率在经典场论下有明确意义,而在量子场论的语境下,它实际上对应了这个场的傅里叶分量(粒子产生算符或者湮灭算符)产生/湮灭的粒子的能量$E$。

\subsection{实标量场的正则量子化}

\subsubsection{平面波模式}

无需额外考虑标量场的基,于是对实标量场,可以直接取
\begin{equation}
    {\phi}(\vb*{x}, t) = \int \frac{\dd[3]{\vb*{p}}}{(2\pi)^3} \frac{1}{\sqrt{2 \omega_{\vb*{p}}}} \left( {a}^\dagger_{\vb*{p}} \ee^{ - \ii \vb*{p} \cdot \vb*{x} + \ii \omega_{\vb*{p}} t} + {a}_{\vb*{p}} \ee^{ \ii \vb*{p} \cdot \vb*{x} - \ii \omega_{\vb*{p}} t} \right),
    \label{eq:expanding-klein-gordon-field}
\end{equation}
显然它是\eqref{eq:klein-gordon-eq}的一个解。相应的使用\eqref{eq:klein-gordon-pi},有
\begin{equation}
    {\pi}(\vb*{x}, t) = \int \frac{\dd[3]{\vb*{p}}}{(2\pi)^3} \  \ii \sqrt{\frac{\omega_{\vb*{p}}}{2}} \left( {a}^\dagger_{\vb*{p}} \ee^{ - \ii \vb*{p} \cdot \vb*{x} + \ii \omega_{\vb*{p}} t} - {a}_{\vb*{p}} \ee^{ \ii \vb*{p} \cdot \vb*{x} - \ii \omega_{\vb*{p}} t} \right)
\end{equation}
共轭动量不是洛伦兹协变的。这并不让人意外,因为其定义和时间维的选取有关。
计算得到
\[
    {H} = \int \frac{\dd[3]{\vb*{p}}}{(2\pi)^3} \frac{1}{2} \omega_{\vb*{p}} ({a}_{\vb*{p}}^\dagger {a}_{\vb*{p}} + {a}_{\vb*{p}} {a}^\dagger_{\vb*{p}}).
\]

下面把正则对易关系施加到标量场${\phi}$上。
通过计算可以得知,这等价于
\begin{equation}
    \comm*{{a}_{\vb*{p}}}{{a}^\dagger_{\vb*{p}'}} = (2\pi)^3 \delta^3 (\vb*{p} - \vb*{p}'), \quad \comm*{{a}_{\vb*{p}}}{{a}_{\vb*{p}'}} = 0.
    \label{eq:quantization-scalar}
\end{equation}
相应的,反对易关系等价于
\[
    \acomm*{{a}_{\vb*{p}}}{{a}^\dagger_{\vb*{p}'}} = (2\pi)^3 \delta^3 (\vb*{p} - \vb*{p}'), \quad \acomm*{{a}_{\vb*{p}}}{{a}_{\vb*{p}'}} = 0.
\]
将反对易关系代入哈密顿量表达式会导致哈密顿量变成常数,因此这是平凡解,舍去。
将对易关系代入哈密顿量的表达式,得到
\begin{equation}
    {H} = \int \frac{\dd[3]{\vb*{p}}}{(2\pi)^3} \omega_{\vb*{p}} \left({a}_{\vb*{p}}^\dagger {a}_{\vb*{p}}  + \frac{1}{2} \comm*{{a}_{\vb*{p}}}{{a}^\dagger_{\vb*{p}}} \right).
    \label{eq:hamiltonian-of-klein-gordon}
\end{equation}
容易看出第二项实际上是发散的。
产生这种发散的原因在于,相对论性量子场论不会被用于处理动量特别高的问题(在那里需要新的物理,通常称为“紫外端的物理”),因此所谓的对整个动量空间的积分实际上只是对动量空间中一块很大的区域的积分。
在这种意义下,\eqref{eq:hamiltonian-of-klein-gordon}中的第二项是一个很大的常数,称为\concept{真空零点能}。因此在讨论全空间内的问题时,可以丢弃它得到等效的哈密顿量(注意此时哈密顿量的正定性实际上被破坏了)%?真的吗?
\begin{equation}
    {H} = \int \frac{\dd[3]{\vb*{p}}}{(2\pi)^3} \omega_{\vb*{p}} {a}_{\vb*{p}}^\dagger {a}_{\vb*{p}}.
\end{equation}
这是一个福克空间上的$1$粒子算符。它表明自由场情况下单粒子携带能量为$\omega_{\vb*{p}}$。
通过反复使用对易关系\eqref{eq:quantization-scalar}以及真空态被湮灭算符作用后得到$0$这一事实,可以计算出
\begin{equation}
    {H} {a}_{\vb*{p}}^\dagger \ket{0} = \omega_{\vb*{p}} {a}_{\vb*{p}}^\dagger \ket{0}.
\end{equation}
因此正如我们预期的那样,单粒子态$\ket{\vb*{p}}$是哈密顿量的本征态。

真空零点能的出现实际上意味着原来的哈密顿量中的各个项是不对易的,因此真空态的能量不能是零,如果它是零,那么由哈密顿量的正定性,哈密顿量中的每一项作用在真空态上都会得到零,于是真空态是哈密顿量的每一项的本征态,这就产生了矛盾。
% TODO:实际上不对易的算符还是可以有共同本征态的,以上说法不正确,需要进一步说明
不对易性是纯粹的量子概念,因此真空零点能只有在量子场论中才能够得到良好的定义。
如果哈密顿量中所有的项都是彼此对易的,就不会有真空零点能。有时真空零点能的存在也称为量子涨落,因为即使在真空态,也不是所有的物理量都有完全确定的值。

需要注意的是如果我们讨论的问题不是定义在全空间上的,可能不能直接把真空零点能丢弃。例如,设有两块无穷大的金属板,它们施加的边界条件会让\eqref{eq:expanding-klein-gordon-field}中的一些模式为零,通过计算可以发现板间的真空零点能小于板外,从而产生一个板之间的吸引力。

总之,标量场需要使用正则对易关系来量子化,不能用反对易关系。因此标量场描述$0$自旋玻色子。

\subsubsection{守恒量}

场的动量为
\[
    P_i = \int \dd[3]{\vb*{x}} \pi \partial_i \phi,
\]
从而
\begin{equation}
    {\vb*{P}} = - \int \dd[3]{\vb*{x}} {\pi} \grad{{\phi}} = \int \dd[3]{\vb*{x}} \vb*{p} {a}^\dagger_{\vb*{p}} {a}_{\vb*{p}} + \text{vaccum zero-point item},
\end{equation}
正如我们预期的,场的动量也是单粒子算符,并且正好就是所有粒子的动量之和。

标量场没有内禀自由度,因此也不携带自旋角动量。我们无需讨论其自旋角动量。

\subsubsection{实标量场的传播子}

通过\eqref{eq:quantization-scalar}中的归一化常数我们可以写出动量空间中的同时间两点关联函数:
\begin{equation}
    \mel{0}{a_{\vb*{p}} a^\dagger_{\vb*{q}}}{0} = (2\pi)^3 \delta(\vb*{p} - \vb*{q}), \quad \mel{0}{a^\dagger_{\vb*{p}} a_{\vb*{q}}}{0} = 0.
\end{equation}
代入展开式\eqref{eq:expanding-klein-gordon-field},就得到实空间下的两点关联函数(这两个点可以不等时):
\begin{equation}
    D(x-y) = \mel{0}{\phi(x) \phi(y)}{0} = \int \frac{\dd[3]{\vb*{p}}}{(2\pi)^3} \frac{1}{2 \omega_{\vb*{p}}} \ee^{-\ii p \cdot (x - y)} |_{p^0 = E_{\vb*{p}}} .
\end{equation}
被积分的$\ee$指数是洛伦兹不变的,积分测度也是洛伦兹不变的,等式左边也是洛伦兹不变的——本该如此,既然标量场是洛伦兹不变的。
这个式子可以写成更加明显的洛伦兹不变的形式。
容易看出
\[
    \int \dd{\omega} \frac{1}{(\omega - \omega_{\vb*{p}})(\omega + \omega_{\vb*{p}}  - \ii 0^+)} \ee^{-\ii p \cdot (x - y)} = - \pi \ii \frac{1}{2 \omega_{\vb*{p}}} \ee^{-\ii p \cdot (x - y)} |_{p^0 = E_{\vb*{p}}} ,
\]
这里积分号表示计算积分主值,负号是由于$\ee^{-\ii \omega t}$在下半平面衰减而不是上半平面。
于是就有
\begin{equation}
    \begin{aligned}
        D(x-y) &= 2 \int \frac{\dd[4]{p}}{(2\pi)^4} \frac{\ii}{(\omega - \omega_{\vb*{p}})(\omega + \omega_{\vb*{p}}  - \ii 0^+)} \ee^{-\ii p \cdot (x - y)} \\
        &= 2 \int \frac{\dd[4]{p}}{(2\pi)^4} \frac{\ii}{p^2 - m^2 - \ii 0^+ (\omega - \omega_{\vb*{p}})} \ee^{-\ii p \cdot (x - y)}.
    \end{aligned}
\end{equation}
这个积分看起来非常不自然:我们需要忽略其中一个极点,并且取另一个极点的积分主值,然后再将结果乘以$2$。
这种不自然性暗示$\mel*{0}{\phi(x) \phi(y)}{0}$可能并不是非常好用的表示两点关联的方式——它的确不是,因为当$x^0 < y^0$时它还是会给出一个值,并没有很好地反映因果性。
对这个量的有意义的使用局限于$x^0 > y^0$时。

在等时情况下上式实际上可以进一步往下计算。设$x^0-y^0=0$,$\vb*{x} - \vb*{y} = \vb*{r}$,采用球坐标系,有
\[
    \begin{aligned}
        D(x-y) &= \int \frac{\dd[3]{\vb*{p}}}{(2\pi)^3} \frac{1}{2 \omega_{\vb*{p}}} \ee^{\ii \vb*{p} \cdot \vb*{r}} \\
        &= \frac{2\pi}{(2\pi)^3} \int \frac{p^2 \dd{p}}{2 \omega_{\vb*{p}}} \frac{\ee^{\ii p r} - \ee^{- \ii p r}}{\ii p r} \\
        &= - \frac{\ii}{2 (2\pi)^2 r} \int_{-\infty}^\infty \frac{p \dd{p}}{\sqrt{p^2 + m^2}} \ee^{\ii p r},
    \end{aligned}
\]
被积函数是一个多值函数,从$\ii m$到$\ii \infty$,$- \ii m$到$- \ii \infty$作割线,即确定一个单值分支。
将积分路径换成围绕从$\ii m$到$\ii \infty$的割线,并作变量代换$p = \ii \rho$,可将上式化为
\[
    D(x-y) = \frac{1}{4\pi^2 r} \int_{m}^\infty \dd{\rho} \frac{\rho \ee^{-\rho r}}{\sqrt{\rho^2 - m^2}} \stackrel{r \to \infty}{\sim} \ee^{- m r}.
\]
这就是说,即使是彼此之间有类空间隔的两点之间实际上也是有关联的,虽然随着举例增加关联函数快速下降;或者说,在光锥之外,跃迁振幅会指数下降但是不会一开始就降到零。

这实际上并没有直接破坏因果律,因为量子态并不是直接可观测的,真正有意义的是测量,如果具有类空间隔的两点之间的同种可观测量是对易的,因果律还是能够保持。
计算$\phi(x)$和$\phi(y)$的对易子:
\begin{equation}
    \begin{aligned}
        \comm*{\phi(x)}{\phi(y)} &= \int \frac{\dd[3]{\vb*{p}}}{(2\pi)^3} \frac{1}{2 \omega_{\vb*{p}}} (\ee^{\ii p \cdot (y - x)} - \ee^{\ii p \cdot (x - y)}) \\
        &= D(x-y) - D(y-x),
    \end{aligned}
\end{equation}
在$x$和$y$之间为类空间隔时,洛伦兹变换可以让$x-y$变成$y-x$,由于$D(x-y)$和$D(y-x)$各自是洛伦兹标量,$\comm*{\phi(x)}{\phi(y)}$就是零。
否则,洛伦兹变换不能让$x-y$变成$y-x$,因为此时时间的先后顺序是绝对的,从而$\phi(x)$和$\phi(y)$不对易。
因此因果律在这里确实是能够保持的。

我们看到$\phi(x)$和$\phi(y)$的对易子实际上就是一个普通的数,即

在实际计算中通常计算编时格林函数,因为其携带了足够多的信息,并且在自由场情况下有Wick定理。
仿照前述引入$\omega$积分的方法,可以验证我们有
\begin{equation}
    D_F(x-y) = \mel*{0}{T \phi(x) \phi(y)}{0} = \int \frac{\dd[4]{p}}{(2\pi)^4} \frac{\ii}{p^2 - m^2 + \ii 0^+} \ee^{- \ii p \cdot (x - y)}.
\end{equation}
这称为\concept{费曼传播子},可以看到它在动量空间中的形式是非常简单的。

\subsection{实标量场的路径积分量子化}

旋量场的路径积分需要额外提几句,因为此时所谓的“经典场”实际上是一些彼此反对易的算符(所谓\concept{格拉斯曼数}),而不是正常的数。
格拉斯曼数的积分基本上只用在写出配分函数上,

\[
    S = \int \dd[4]{x} \bar{\psi} (\ii G^{-1}) \psi
\]

\subsection{非相对论极限}

\subsubsection{克莱因-高登方程的退化形式}\label{sec:sch-eq-from-kg}

我们将讨论克莱因-高登方程的退化形式。旋量场和标量场由于也服从克莱因-高登方程,没有必要单独考虑——它们多出来的自由度可以使用其它方式,如自旋等,引入。
实际上我们讨论的应该是复的克莱因-高登方程,因为旋量场是复的,但本节的讨论并不会用到场是不是复的这个信息。

首先我们注意到一个事实:时谐波
\begin{equation}
    \phi = \ee^{- \ii m t}
    \label{eq:lowest-energy}
\end{equation}
是\eqref{eq:klein-gordon-eq}的解,并且它的能量最低,就是零。(代入哈氏量可得)因此,能量不高的场只是微微偏离\eqref{eq:lowest-energy},我们设其为
\begin{equation}
    \phi(\vb*{x}, t) = \psi(\vb*{x}, t) \ee^{- \ii m t},
    \label{eq:low-energy-ansatz}
\end{equation}
则
\[
    (\partial_\mu \partial^\mu + m^2) \phi = \ee^{- \ii m t} (-2 \ii m \partial_t \psi + \partial_t^2 \psi - \laplacian{\psi}).
\]
由于$\phi$只是略微偏离\eqref{eq:lowest-energy}%
\footnote{需要注意的是这个说法字面上实际上是不严谨的。${\phi}$是一个算符,它包含了所有可能的$\phi$的取值,不应该“只是略微偏离\eqref{eq:lowest-energy}”。
然而,${\phi}$的本征态中非常偏离\eqref{eq:lowest-energy}的那部分模式在我们的低能有效理论中并不会被涉及到。
换而言之,我们关心的那部分$\ket{\phi}$只是略微偏离\eqref{eq:lowest-energy},因此认为${\phi}$只是略微偏离\eqref{eq:lowest-energy}并不会显著改变我们的理论的行为。
}%
,可以预期$\psi$的时间部分振荡不会特别明显,于是取近似
\[
    \partial_t^2 \psi \ll m \partial_t \psi,
\]
就得到
\begin{equation}
    \ii \partial_t \psi + \frac{1}{2m} \laplacian{\psi} = 0.
    \label{eq:schodinger-eq}
\end{equation}
\eqref{eq:schodinger-eq}称为\concept{薛定谔场}的运动方程。容易看出它不是洛伦兹协变的,这是理所当然的,因为它描述的现象发生在低能近似下,此时伽利略对称性就足够了。
薛定谔场是复的,无论$\phi$是不是复场,因为拟设\eqref{eq:low-energy-ansatz}引入了一个复数因子。

方程\eqref{eq:schodinger-eq}是以下拉氏量%
\footnote{$\grad{\psi}^\dagger \cdot \grad{\psi}$代表将两个梯度算符做缩并,行向量$\psi^\dagger$和列向量$\psi$相乘,即$\partial_i \psi^\dagger \partial^i \psi$。混合使用不变量记号和矩阵记号是因为我们并不知道$\psi$的内部结构,只知道$\psi^\dagger \psi$是标量,因此把$\psi$当成一个整体,好像一个标量一样,来做计算。}
\begin{equation}
    \mathcal{L} = \frac{\ii}{2} \left( \psi^\dagger \dot{\psi} - \psi \dot{\psi}^\dagger \right) - \frac{1}{2m} \grad{\psi^\dagger} \cdot \grad{\psi}
    \label{eq:schodinger-lagrangian}
\end{equation}
的运动方程。把$\psi$和$\psi^\dagger$看成两个独立的场,分别应用欧拉-拉格朗日方程,就能够得到\eqref{eq:schodinger-eq}和其共轭转置。

容易看出,
\[
    \pi(\psi) = \pdv{\mathcal{L}}{\dot{\psi}} = \frac{\ii}{2} \psi^\dagger, \quad \pi(\psi^\dagger) = \pdv{\mathcal{L}}{\dot{\psi}^\dagger} = - \frac{\ii}{2} \psi^\top,
\]
从而可以计算出
\begin{equation}
    \mathcal{H} = \frac{1}{2m} \grad{\psi^\dagger} \cdot \grad{\psi}.
\end{equation}
这个哈氏量中出现了$\pi$的导数,处理起来会比较麻烦。为了规避这些麻烦,我们将不再讨论经典的哈密顿动力学,而直接开始做量子化。

电荷密度为
\begin{equation}
    \rho(\vb*{r}) = q \psi^\dagger(\vb*{r}) \psi(\vb*{r}),
\end{equation}
而且电流密度为
\begin{equation}
    \vb*{j}(\vb*{r}) = \frac{1}{2m\ii} (\psi^\dagger(\vb*{r}) \grad{\psi}(\vb*{r}) - \psi(\vb*{r}) \grad{\psi}^\dagger(\vb*{r}))
\end{equation}

\subsubsection{薛定谔场}

% TODO:可以看到,$j=0$的标量场给出的粒子的自旋角动量为0,$j=\pm \frac{1}{2}$的旋量场给出的粒子的自旋角动量为$\pm 1/2$,$j=1$的矢量场给出的粒子的自旋角动量为$\pm 1$。这并不让人意外,因为$j$决定了粒子的内禀自由度的维度($2j+1$)。无质量的情况比较特殊
实际上也可以通过量子化薛定谔场来得到非相对论性量子场论。薛定谔场实际上是标量场、旋量场、矢量场退化而来的场,因此它也有内禀自由度。使用自旋(或者螺旋度)标记这些内禀自由度。
由于薛定谔场不是实场,考虑对易关系
\[
    \comm*{{\psi}^i(\vb*{x}, t)}{{\pi}_j(\vb*{y}, t)} = \ii \delta^i_j \delta^3 (\vb*{x} - \vb*{y}),
\]
即
\[
    \comm*{{\psi}^i(\vb*{x}, t)}{({\psi}^j)^\dagger (\vb*{y}, t)} = 2 \delta^3 (\vb*{x} - \vb*{y}),
\]
这表明

总之,在不涉及粒子相互作用时,单粒子量子力学足以覆盖薛定谔场的情况,即“非相对论量子场论”就是量子力学。%
\footnote{
    量子场论和量子力学的对应实际上有两方面:首先,量子场论和量子力学都可以写成哈密顿动力学的形式,当然前者各个物理量可以使用空间位置作为标签而后者物理量的标签都是离散的;其次,量子场论和量子力学都能够描述多粒子态。
    当我们说非相对论性量子场论就是量子力学时我们是在说后者,当我们说量子场论是3+1维量子力学时我们是在说前者。
}%
我们再一次看到场自由度和数量可变的粒子自由度实际上就是一回事。
初等量子力学中可以直接定义S算符、单粒子费曼图(“原子吸收一个光子、释放一个光子”,等等),你可能会问为什么这些本来用于场论的概念也可以被用在单粒子量子力学上,毕竟前者是3+1维理论而后者是0+1维理论。但实际上,这些用在量子场论上的概念完全可以被应用在薛定谔场上,而由于薛定谔场不涉及粒子数变化,这些概念就可以被套用在单粒子态量子力学上。

\subsubsection{关于归一化的注记}

在将相对论性的相互作用顶角移植到非相对论极限下时需要注意归一化问题。
在非相对论性极限下我们更喜欢使用$\{\ket{\vb*{p}}\}$表象,而不是$\{\ket{p}\}$表象。
我们知道
\[
    \ket{p} = \sqrt{2 \omega_{\vb*{p}}} \ket{\vb*{p}},
\]
在非相对论极限下就有
\begin{equation}
    \ket{\vb*{p}} = \frac{1}{\sqrt{2m}} \ket{p},
\end{equation}
于是应有
\begin{equation}
    \mel*{\vb*{p}_1, \ldots, \vb*{p}_m}{S}{\vb*{q}_1, \ldots, \vb*{q}_n} = \frac{1}{(2m)^{(m+n)/2}} \mel*{p_1, \ldots, p_m}{S}{q_1, \ldots, q_n}.
\end{equation}

\section{旋量场}

\subsection{旋量场的狄拉克方程}

旋量场实际上几乎从来不会在经典情况下遇到,因为它们的场值是复数,因此不具有直接的物理意义。

本节讨论旋量的运动方程。使用凑拉氏量的方法处理旋量会比较困难,因为旋量的指标分带点的和不带点的,因此会频繁地涉及求共轭等运算,在拉格朗日动力学中讨论这些问题并不方便。
因此接下来我们尝试直接构造旋量的运动方程,从这些运动方程反推对应的拉氏量。
我们将尝试构造一阶运动方程。如果对魏尔旋量和狄拉克旋量都能够构造出一阶运动方程,那就没有必要考虑更高阶的运动方程。
% TODO:为什么?

\subsubsection{魏尔旋量的运动方程} 

首先讨论魏尔旋量的运动方程。满足平移不变性的方程形如
\[
    \partial_0 \psi = b^i \partial_i \psi + C \psi,
\]
其中$b^i$和$C$是常数,$C$可以是一个旋量矩阵。
由于我们同时还要求旋转不变性,$C$只能是一个标量。显然,这个方程中所有含有导数的项加在一起必然得到一个旋量,即
\[
    \partial_0 \psi - b^i \partial_i \psi = \text{a covariant term} = C \psi.
\]
梯度算符是矢量,按照\eqref{eq:vector-is-spin-tensor},我们可以写出作用在魏尔旋量上的导数算符
\begin{equation}
    \partial_{a \dot{b}} = \partial_\nu \sigma^\nu_{a \dot{b}}.
\end{equation}
从而魏尔旋量的梯度就是
\[
    \partial_{a \dot{b}} \psi^{\dot{b}} = \partial_\nu \sigma^\nu_{a \dot{b}} \psi^{\dot{b}}
\]
和
\[
    \partial^{\dot{a} b} \psi_b = \partial_\mu (\sigma^\mu)^{\dot{a} b} \psi_b.
\]
两个表达式中,$\sigma$的指标都一个带点一个不带点,这是为了保证梯度算符的协变性,因为矢量是一个左手旋量和一个右手旋量直积的结果。
$\partial_0 \psi - b^i \partial_i \psi$应该能够写成以上两种旋量梯度的函数。由于以上两种旋量梯度带一个指标,而$\partial_0 \psi - b^i \partial_i \psi$也是单指标对象,显然两者只应该差一个倍数。(当然,也可以将一个旋量张量参数和旋量梯度做缩并,但这样就没有旋转不变性了)这个倍数可以被吸收到$C$中。
从而运动方程形如
\[
    \mathrm{grad} \psi = C \psi.
\]
然而,注意到左手旋量的梯度是一个右手旋量,右手旋量的梯度是一个左手旋量,因此以上的方程会让一个右手旋量的各个分量等于一个左手旋量的各个分量,从而破坏了洛伦兹协变性。
消除这个矛盾的唯一可能就是让$C=0$,于是左手旋量的运动方程为
\[
    \partial_\mu (\sigma^\mu)^{\dot{a} b} \psi_b = 0,
\]
右手旋量的运动方程为
\[
    \partial_\nu \sigma^\nu_{a \dot{b}} \psi^{\dot{b}} = 0.
\]
由定义,$(\sigma^\mu)_{a \dot{b}}$的分量矩阵就是$\sigma$矩阵,而$(\sigma^\mu)^{\dot{a} b}$的分量矩阵则需要通过指标升降关系
\[
    (\sigma^\mu)^{\dot{a} b} = (\epsilon^{ac} (\sigma^\mu)_{c \dot{d}} \epsilon^{\dot{b} \dot{d}})^*
\]
得到。定义$(\sigma^\mu)^{\dot{a} b}$的分量矩阵为$\bar{\sigma}^\mu$,通过计算可以发现
\begin{equation}
    \bar{\sigma}^0 = \sigma^0 = I, \quad \bar{\sigma}^i = - \sigma^i,
\end{equation}
于是可以使用矩阵形式写出运动方程:
\begin{equation}
    \partial_0 \psi \pm \sigma^i \partial_i \psi = 0 .
    \label{eq:weyl-eq}
\end{equation}
负号为左手旋量,正号为右手旋量。从\eqref{eq:weyl-eq}立刻可以得到
\[
    \partial_\mu \partial^\mu \psi = 0,
\]
无论是右手旋量还是左手旋量。因此,单独一个旋量场一定是没有质量的。

然而,拉氏量中的质量项实际上是自由理论的一部分,因此一种非常简单的让旋量场获得质量而与此同时理论保持为自由理论的方法是让一对旋量耦合起来。
两个旋量场的耦合意味着拉氏量中要出现一个二次项,它由两个旋量的乘积构成,而由于这个二次项必须是标量,它应该包含一个左手旋量和一个右手旋量。
考虑到我们可以把一个左手旋量和一个右手旋量打包成一个狄拉克旋量,有必要分析狄拉克旋量的运动方程。
% TODO: majorana费米子

\subsubsection{狄拉克旋量的运动方程} 

\paragraph{魏尔表象} 我们现在让狄拉克旋量中的两个旋量之间有线性的相互作用(从而,关于狄拉克旋量的方程仍然是线性的)。
我们将狄拉克旋量写成一个左手旋量和一个右手旋量拼合的形式,称为\concept{魏尔表象}或\concept{手性基},虽然实际上有其它方式可以表示狄拉克旋量。
洛伦兹不变性的要求意味着,唯一可能的方程形式如下:
\begin{equation}
    \begin{aligned}
        (\partial_0 + \sigma^i \partial_i) \psi_R = - \ii m_1 \psi_L, \\
        (\partial_0 - \sigma^i \partial_i) \psi_L = - \ii m_2 \psi_R.
    \end{aligned}
    \label{eq:interacting-weyl-eq}
\end{equation}
其中我们为了节省符号,使用$\psi_L$和$\psi_R$分别代表狄拉克旋量$\psi$的左手部分和右手部分。
当然,$m=0$时就狄拉克旋量的运动方程就退化为了一对完全无关的左手旋量和右手旋量。
这也就是实际计算时没有必要单独讨论魏尔旋量的原因。
在\eqref{eq:interacting-weyl-eq}中各个方程的两边作用适当的算符来让两个场解耦,就得到
\[
    \partial_\mu \partial^\mu \psi_L = - m_1 m_2 \psi_L, \quad \partial_\mu \partial^\mu \psi_R = - m_1 m_2 \psi_R,
\]
也就是说狄拉克旋量也满足克莱因-高登方程,只需要我们令$m^2=m_1 m_2$。从而为了得到物理解,我们要求$m \geq 0$。
可以看到$m_1$和$m_2$的具体取值实际上是无关紧要的,重要的是它们的乘积,因此以下让它们都取$m$。

为了将\eqref{eq:interacting-weyl-eq}写成更加紧凑的形式,引入$\gamma$矩阵%
\footnote{这里给出的$\gamma$矩阵的形式实际上只是一种可能性。我们称这种将狄拉克旋量的左手部分和右手部分分开处理(或者等价地说,狄拉克旋量的基或者只含有左手旋量,或者只含有右手旋量),并且按照\eqref{eq:gamma-matrix}引入$\gamma$矩阵的方式为\concept{魏尔表象}。也可以取其它的旋量基,从而获得其它表象。}
\begin{equation}
    \gamma^\mu = \pmqty{0 & \sigma^\mu \\ \bar{\sigma}^\mu & 0}, \quad \gamma^5 = \pmqty{-I & 0 \\ 0 & I},
    \label{eq:gamma-matrix}
\end{equation}
从而
\begin{equation}
    \gamma_\mu = \eta_{\mu \nu} \gamma^\nu = \pmqty{ 0 & \bar{\sigma}^\mu \\ \sigma^\mu & 0 },
\end{equation}
则得到
\begin{equation}
    (\ii \gamma^\mu \partial_\mu - m) \psi = 0.
    \label{eq:dirac-eq}
\end{equation}
这就是\concept{狄拉克方程}。如前所述,它能够推导出克莱因-高登方程,并且在$m$取零时退化为一个左手旋量场和一个右手旋量场的简单组合。

现在我们尝试拼凑一个拉氏量出来。由于狄拉克场的运动方程是一阶的而它又是一个复场,需要通过$\psi$的复共轭拼凑出一个在运动方程意义下“独立”的场,然后构造一个同时包含$\psi$及其复共轭的拉氏量,由这个拉氏量给出关于$\psi$和它的复共轭的两个方程,并且这两个方程必须等价。
现在我们尝试寻找和\eqref{eq:dirac-eq}等价,但是仅仅包含其复共轭的方程。
由\eqref{eq:dirac-eq}取共轭转置%
\footnote{这里的共轭转置是指场的共轭转置,不需要对作用在场上的算符$\partial_\mu$取共轭转置。}
,得到
\[
    (-\ii) \partial_\mu \psi^\dagger (\gamma^\mu)^\dagger - m \gamma^\dagger = 0.
\]
容易验证$\gamma$矩阵具有下面的性质:
\[
    (\gamma^0)^\dagger = \gamma^0, \quad (\gamma^i)^\dagger = - \gamma^i, 
\]
以及
\[
    \gamma^i \gamma^0 = - \gamma^0 \gamma^i,
\]
我们发现
\[
    \ii \partial_0 \psi^\dagger \gamma^0 \gamma^0 + \ii \partial_i \psi^\dagger \gamma^0 \gamma^i + m \psi^\dagger \gamma^0 = 0.
\]
定义
\begin{equation}
    \bar{\psi} = \psi^\dagger \gamma^0,
\end{equation}
则其运动方程为
\begin{equation}
    \ii \partial_\mu \bar{\psi} \gamma^\mu + m \bar{\psi} = 0.
    \label{eq:cog-dirac-eq}
\end{equation}
这正是我们需要的另一个运动方程。
我们会发现,拉氏量
\begin{equation}
    \mathcal{L} = \bar{\psi} (\ii \gamma^\mu \partial_\mu - m) \psi
    \label{eq:dirac-lagrangian}
\end{equation}
分别对$\psi$和$\bar{\psi}$应用欧拉-拉格朗日方程,就得到\eqref{eq:dirac-eq}和\eqref{eq:cog-dirac-eq}。同时容易验证这是一个洛伦兹标量。这表明\eqref{eq:dirac-lagrangian}确实就是狄拉克场的拉氏量。
我们之后还会非常频繁地写$\gamma^\mu a_\mu$这样的量,可以将其简记为$\slashed{a}$。

从\eqref{eq:dirac-lagrangian}可以推导出对应的哈氏量。计算共轭动量可以得到
\begin{equation}
    \pi = \ii \bar{\psi} \gamma^0 = \ii \psi^\dagger,
\end{equation}
从而能够得到哈氏量密度
\begin{equation}
    \mathcal{H} = - \ii \bar{\psi} \gamma^i \partial_i \psi  + m \bar{\psi} \psi = - \pi \gamma^0 \gamma^i \partial_i \psi - \ii m \pi \gamma^0 \psi.
    \label{eq:dirac-hamiltonian}
\end{equation}

有时我们需要让一个狄拉克旋量中的左手部分或是右手部分单独拿出来与其它场耦合。
容易看出,矩阵
\begin{equation}
    P_L = \frac{1 - \gamma^5}{2}, \quad P_R = \frac{1 + \gamma^5}{2}
\end{equation}
分别是左手旋量和右手旋量的投影算符。

\paragraph{狄拉克表象} 以上我们都设$\psi$的各分量由一个左手旋量和一个右手旋量拼凑而成。这称为\concept{手性基}或者\concept{魏尔表象}。%
\footnote{
    虽然我们在讨论场算符,但是这些场算符可以直接激发出单粒子态,后者和前者之间的关系是线性的,所以“基底”或“表象”的术语确实是正确的:它们的确给出了单费米子的一组基底。
}%
在手性基当中,拉氏量的质量项为
\[
    - \bar{\psi} m \psi = - m (\chi_L^\dagger \xi_R + \xi_R^\dagger \chi_L),
\]
这不是一个对角化的二次型。若做分量变换
\begin{equation}
    \psi' = \frac{1}{\sqrt{2}} \underbrace{\pmqty{1 & 1 \\ -1 & 1}}_{U} \psi, \quad \gamma'^\mu = U \gamma^\mu U^\dagger,
\end{equation}
质量项就被对角化了。我们称这种分量选取为\concept{质量基}或者\concept{狄拉克表象}。
在后面做量子化时会看到,在魏尔表象中,$\psi$的$p^0 > 0$的解包含权重相等的一个左手旋量分量和一个右手旋量分量,$p^0 < 0$的解包含权重正好差一个负号的一个左手旋量分量和一个右手旋量分量,因此狄拉克表象下,$\psi$的前两个分量代表$p^0 > 0$的模式而后两个分量代表$p^0 < 0$的模式。
实际上,这就意味着狄拉克表象的前两个分量代表自旋一上一下的两种粒子,而后两个分量分别代表自旋一上一下的两种反粒子。

容易计算出质量基下
\begin{equation}
    \gamma^0 = \pmqty{I & 0 \\ 0 & -I}, \quad \gamma^i = \pmqty{0 & \sigma^i \\ - \sigma^i & 0 }, \quad \gamma^5 = \pmqty{0 & I \\ I & 0}.
\end{equation}
在质量基下,左手投影算符和右手投影算符分别是

\subsubsection{$\gamma$矩阵和Clifford代数}

\paragraph{$\gamma$矩阵的乘法规则}

\paragraph{$\gamma$矩阵的洛伦兹变换} %TODO:旋量场的洛伦兹标量

$\gamma$矩阵的洛伦兹变换需要一些特别的注记。如果我们做像本文定义的那样的洛伦兹变换,即让旋量场变换,那么$\gamma$矩阵在洛伦兹变换下根本就不应该发生变化。
但是,另一方面,$\bar{\psi} \gamma^\mu \psi$可以验证的确是矢量。
这种看似冲突的情况——$\gamma$矩阵同时看起来像标量和矢量——当然来自它们代表的是可以作用在单粒子态上的算符,或者作用在场算符上的“元算符”,而非场算符本身这一事实。
回顾引入狄拉克旋量的过程,$\gamma$矩阵的作用其实是“正确地将左手部分的魏尔旋量和右手部分的魏尔旋量混合起来”,它实际上提供了拉氏量关于旋量场的形式,自然在洛伦兹变换下可以没有变化。

这样在实际计算中我们其实可以在很多时候将旋量有关的量——旋量场和$\gamma$矩阵——当成没有内部结构的对象,即假装系统中只有标量和矢量,而通过正确地排列旋量场和$\gamma$矩阵的顺序来完成本应通过旋量指标完成的工作。
对矢量当然也可以这么做,实际上研究数值计算时我们将一切都写成矩阵就是在做这件事;在量子场论中不这么做的主要原因是,在量子场论中我们需要频繁讨论矢量的任意张量积,从而保留上下指标会让计算变得直观,而对旋量做的操作则相当有限。

\paragraph{$\gamma$矩阵的物理意义} 我们已经看到$(1-\gamma^5)/2$和$(1+\gamma^5)/2$分别是左手和右手的手征投影算符。
还可以验证,
\begin{equation}
    \gamma^5 P_L \psi = - P_L \psi, \quad \gamma^5 P_R \psi = P_R \psi,
\end{equation}
因此$\gamma^5$是\concept{手征算符},其负本征值对应左旋态而正本征值对应右旋态。
另一方面,$\gamma^0$则交换一个狄拉克旋量的左手和右手部分。

\subsection{旋量场的正则量子化}

\subsubsection{平面波模式}

我们还是例行公事地观察狄拉克方程的平面波解,并将其中的场变量用算符代替。接下来将使用手性基。
平面波旋量场的分量应该包括这些标记:首先是动量。三维动量给定之后,对时谐场有$p^0 = \pm \omega_{\vb*{p}}$,但由于旋量场不是实场,因此其正频率部分和负频率部分之间没有简单的关系,即不能统一用$a$和$a^\dagger$表示而需要引入两种模式。
在动量给定后,只需要求解
\[
    (\gamma^\mu p_\mu - m) \psi = 0.
\]
洛伦兹变换不会改变这个方程的解的结构,所以为了弄清楚方程的解大致是什么样的我们可以先求解
\[
    (\gamma^0 p_0 - m) \psi = 0.
\]
我们会发现$p^0 = \pm \omega_{\vb*{p}}$各自对应两个独立的解,分别是
\[
    \pmqty{1 \\ 0 \\ 1 \\ 0}, \quad \pmqty{0 \\ 1 \\ 0 \\ 1}, \quad \pmqty{1 \\ 0 \\ -1 \\ 0}, \quad \pmqty{0 \\ 1 \\ 0 \\ -1}.
\]
容易验证自旋算符和哈密顿量对易,且魏尔旋量的两个分量正好就代表两种不同的自旋。
因此,标记平面波旋量场的分量的独立标签包括:三维动量,频率的正负,以及自旋$1/2$。
这样,我们设
\begin{equation}
    \psi(x) = \int \frac{\dd[3]{\vb*{p}}}{(2\pi)^3} \frac{1}{\sqrt{2 \omega_{\vb*{p}}}} (a_{\vb*{p}, s} u^s(p) \ee^{-\ii p \cdot x} + b_{\vb*{p}, s} v^s(p) \ee^{\ii p \cdot x}),
\end{equation}
其中$s = \pm 1/2$,$u^s$和$v^s$分别满足(这里$p_i$仍然和$p^i$差一个负号,以和闵可夫斯基时空度规一致)
\begin{equation}
    (\gamma^0 \omega_{\vb*{p}} + \gamma^i p_i) u_s(p) = m u_s(p), \quad (- \gamma^0 \omega_{\vb*{p}} + \gamma^i p_i) v_s(p) = m v_s(p), 
    \label{eq:u-v-eigen}
\end{equation}
并且被归一化为
\begin{equation}
    (u^r(p))^\dagger u^s(p) = 2 \omega_{\vb*{p}} \delta^{rs}, \quad (v^r(p))^\dagger v^s(p) = 2 \omega_{\vb*{p}} \delta^{rs}, \quad (u^r(p))^\dagger v^s(p) = 0.
    \label{eq:dirac-norm}
\end{equation}
这里$2\omega_{\vb*{p}}$因子是为了和后面的正则量子化中的动量积分中的$1/2\omega_{\vb*{p}}$因子抵消,从而保证正确的归一化,让动量产生湮灭算符的对易子为$\delta^3(\vb*{p} - \vb*{p})$的同时,$\psi$和$\ii \psi^\dagger$的对易子为$\ii \delta(\vb*{x} - \vb*{y})$。

代入哈密顿量,可以得到(我们会发现哈密顿量\eqref{eq:dirac-hamiltonian}中的第一项和\eqref{eq:u-v-eigen}左边的第二项抵消了)
\[
    H = \sum_{s} \int \frac{\dd[3]{\vb*{p}}}{(2\pi)^3} \omega_{\vb*{p}} (a^\dagger_{\vb*{p}, s} a_{\vb*{p}, s} - b^\dagger_{\vb*{p}, s} b_{\vb*{p}, s}).
\]

看起来我们有麻烦了——$b$模式上的粒子产生得越多,能量越低,似乎出现了能量的无限下降。
然而,如果对$\psi$场施加反对易关系,就会得到
\[
    H = \sum_{s} \int \frac{\dd[3]{\vb*{p}}}{(2\pi)^3} \omega_{\vb*{p}} (a^\dagger_{\vb*{p}, s} a_{\vb*{p}, s} + b_{\vb*{p}, s} b^\dagger_{\vb*{p}, s} - 1),
\]
在略去发散但是是常数的真空零点能之后就得到
\begin{equation}
    H = \sum_{s} \int \frac{\dd[3]{\vb*{p}}}{(2\pi)^3} \omega_{\vb*{p}} (a^\dagger_{\vb*{p}, s} a_{\vb*{p}, s} + b_{\vb*{p}, s} b^\dagger_{\vb*{p}, s} ).
\end{equation}
现在如果我们将$b$看成某种粒子的\emph{产生}算符,而将$b^\dagger$看成\emph{湮灭}算符,那么我们就完成了量子化——哈密顿量变成了占据数的一次函数,并且有正确的反对易关系。
另一方面,如果施加对易关系,虽然也可以得到不会无限下降的哈密顿量,但是不能将$b$看成某种粒子的\emph{产生}算符,而将$b^\dagger$看成\emph{湮灭}算符,因为设$\tilde{b} = b^\dagger$,就有$\comm*{\tilde{b}}{\tilde{b}^\dagger} = -1$,而不是我们需要的$+1$。

于是,我们交换$b$和$b^\dagger$的位置,而要求
\begin{equation}
    \psi(x) = \int \frac{\dd[3]{\vb*{p}}}{(2\pi)^3} \frac{1}{\sqrt{2 \omega_{\vb*{p}}}} (a_{\vb*{p}, s} u^s(p) \ee^{-\ii p \cdot x} + b^\dagger_{\vb*{p}, s} v^s(p) \ee^{\ii p \cdot x}),
\end{equation}
并施加反对易关系
\begin{equation}
    \acomm*{\psi_a(\vb*{x})}{\psi^\dagger_b(\vb*{y})} = \delta^3(\vb*{x} - \vb*{y}) \delta_{ab},
\end{equation}
则哈密顿量就是
\begin{equation}
    H = \sum_{s} \int \frac{\dd[3]{\vb*{p}}}{(2\pi)^3} \omega_{\vb*{p}} (a^\dagger_{\vb*{p}, s} a_{\vb*{p}, s} + b^\dagger_{\vb*{p}, s} b_{\vb*{p}, s} ).
\end{equation}

$b$模式和$a$模式互为反粒子。通过将$\psi$场的傅里叶展开中的积分变量换成$-\vb*{p}$(不改变积分值),可以发现
% 有误:只是对应,还需要加上$u$和$v$矩阵的切换
\[
    a^\dagger_{\vb*{p}} \ee^{- \ii \omega_{\vb*{p}} t} = b_{-\vb*{p}} \ee^{\ii \omega_{\vb*{p}} t},
\]
即它们的确互为反粒子。我们通常以$b^\dagger$为反粒子,虽然这只是习惯问题。
在绘制含有反粒子的费曼图时应当令传播子上的箭头和与传播子平行的动量箭头方向相反,这样顶角的形式无需因为传入的是粒子还是反粒子而改变(例如,$\gamma^\mu A_\mu \bar{\psi} \psi$的形式在这种情况下永远是“一条玻色子线连接一条方向指向顶角的费米子线和一条方向远离顶角的费米子线”),而又提醒了我们一个传播子是反粒子,同时顶角的动量守恒关系也非常直观。
这可以直观地展示为
\[
    \begin{tikzpicture}
        \begin{feynhand}
            \vertex (a) at (0,0); \vertex (b) at (2,0);
            \propag[fer, mom={$-k$}] (a) to (b);
        \end{feynhand}
    \end{tikzpicture} a(-k) = 
    \begin{tikzpicture}
        \begin{feynhand}
            \vertex (a) at (0,0); \vertex (b) at (2,0);
            \propag[fer, revmom={$k$}] (a) to (b);
        \end{feynhand}
    \end{tikzpicture} =
    b^\dagger(k) \begin{tikzpicture}
        \begin{feynhand}
            \vertex (a) at (0,0); \vertex (b) at (2,0);
            \propag[anti fermion, mom={$-k$}] (a) to (b);
        \end{feynhand}
    \end{tikzpicture} .
\]

总之,对旋量场,由于负能量部分(即反粒子)的存在,对易关系是不适用的,因此必须选择反对易关系,这意味着旋量场一定是费米场。

\subsubsection{旋量场的偏振}

现在我们只是形式地写出了$u$和$v$,但是并没有真的计算出它们具体是多少。
计算它们具体是多少是很重要的,因为显然易见,计算偏振方向任意的散射振幅时要用到它们。

首先,对$\vb*{p}=0$的情况,有
\begin{equation}
    u(p) = \sqrt{m} \pmqty{\xi \\ \xi}, \quad \xi^\dagger \xi = 1,
\end{equation}
这样和归一化条件\eqref{eq:dirac-norm}一致。

不失一般性地,认为$\vb*{p}$指向$z$轴。我们的策略是这样的:首先写出动量和推动参数$\eta$之间的关系%
\footnote{
    这里有一个可能的问题,虽然我们确定与能量匹配、共同组成四维动量的那个$\vb*{p}$和标记了自由旋量场的稳定模式的那个$\vb*{p}$都是空间平移的生成元,万一它们没有被一起“定标”怎么办?比如说,如果正好差一个常数怎么办?自旋就有这样的情况。
    好在我们已经证明,对后者也有$E^2 = \abs*{\vb*{p}}^2 + m^2$,因此这两个$\vb*{p}$的的确确是同一个东西。
}%
,然后将$\eta$代入旋量场的洛伦兹变换计算出动量为$\vb*{p}$的模式。

设我们做了一个洛伦兹变换,空间动量方向指向$z$轴,即有
\[
    \pmqty{E \\ p^3} = \exp(\eta \pmqty{0 & 1 \\ 1 & 0}) \pmqty{m \\ 0} = \pmqty{m \cosh \eta \\ m \sinh \eta},
\]
另一方面,在$p=(E, 0, 0, p^3)$时
% TODO:以及统一记号

\begin{equation}
    u(p) = \pmqty{ \sqrt{p \cdot \sigma} \xi \\ \sqrt{p \cdot \bar{\sigma}} \xi },
\end{equation}
类似地可以得到
\begin{equation}
    v(p) = \pmqty{ \sqrt{p \cdot \sigma} \xi \\ - \sqrt{p \cdot \bar{\sigma}} \xi }.
\end{equation}

\begin{equation}
    \bar{u}^r(p) u^s(p) = 2 m \delta^{rs}, \quad u^{r \dagger}(p) u^s(p) = 2 \omega_{\vb*{p}} \delta^{rs},
\end{equation}
\begin{equation}
    \bar{v}^r(p) v^s(p) = - 2 m \delta^{rs}, \quad v^{r \dagger}(p) v^s(p) = 2 \omega_{\vb*{p}} \delta^{rs}.
\end{equation}

\begin{equation}
    \sum_s u^s(p) \bar{u}^s(p) = \slashed{p} + m, \quad \sum_s v^s(p) \bar{v}^s(p) = \slashed{p} - m.
\end{equation}

后面我们将会用$\xi$表示$(\xi^1, \xi^2)$,从而$\xi^\dagger \xi = \sigma^0$。

\concept{Gordon恒等式}

\begin{equation}
    \bar{u}(q_2) (q_1^\mu )
\end{equation}

\subsubsection{守恒量和洛伦兹不变量}

\begin{equation}
    \sigma^{\mu \nu} = \frac{\ii}{2} \comm*{\gamma^\mu}{\gamma^\nu}
\end{equation}

\subsubsection{旋量场的传播子}

现在计算旋量场的各种传播子。旋量场的传播子定义需要特别注意,因为一个$\psi$中含有一种粒子和一种反粒子。
所幸,一个只含有空间定域的正费米子的态还是可以通过$\psi^\dagger \ket{0}$创建,因为反费米子的湮灭算符作用在真空态上给出零,没有任何贡献。
看起来,用$\mel*{0}{\psi \psi^\dagger}{0}$做正费米子的传播子而用$\mel*{0}{\psi^\dagger \psi}{0}$做反费米子的传播子应该是正确的。
然而,可以发现,这样的传播子不是洛伦兹协变的。能够从相对论性量子场论的路径积分计算出来的关联函数肯定是洛伦兹协变的,既然$\mel*{0}{T \psi \psi^\dagger}{0}$不是洛伦兹协变,那么可能并不能很容易地计算它,可能Wick定理都不一定成立,等等。
因此我们应该计算和$\mel*{0}{T \psi \psi^\dagger}{0}$只差一个可逆矩阵变换但是是洛伦兹协变的$\mel*{0}{T \psi \bar{\psi}}{0}$。%
\footnote{
    没有什么规定了传播子一定要定义成$\mel*{0}{\psi \psi^\dagger}{0}$的形式;计算传播子的目的是获得一种简便易行的方式,可用于从一个拉氏量中提取时间演化算符在多粒子态(可以使用场的连乘构造)下的矩阵元,因此传播子只需要大体上长成$\mel*{0}{\psi \psi^\dagger}{0}$的形式即可,差一个线性变换是完全可以的,只要所有重要的信息都能够恢复出来。
}%

\subsubsection{旋量场的路径积分量子化}

其实用路径积分算这个更快
\begin{equation}
    D_F(x-y) = \mel{0}{T \psi(x) \bar{\psi}(y)}{0} = \int \frac{\dd[4]{p}}{(2\pi)^4} \frac{\ii (\slashed{p} + m)}{p^2 - m^2 + \ii 0^+} \ee^{- \ii p \cdot (x- y)}.
\end{equation}

% TODO:费曼图
费米子的费曼规则比较难写。但是在规范场论中其实很好写,因为一张图中不存在涉及一个以上费米子的顶角,即每个顶角中都只有一个费米子进去,一个费米子出来,矩阵$\gamma^\mu$,$u$和$v$(列矢量按照$s$指标排成矩阵)只需要按照这个费米子的“世界线”依次累乘即可。

\subsection{非相对论极限}

在非相对论极限下旋量场当然也会退化成薛定谔场。

\section{矢量场}


\subsection{矢量场的麦克斯韦方程和布洛卡方程}

\subsubsection{对称性分析}

% TODO:$(\partial_\mu A^\mu)^2$
由于自由场导数阶数的限制,出现在拉氏量中的只能是$A^\mu$和$\partial^\nu A^\mu$构成的一次或二次不变量。当然,实际上也可以出现$\partial_\mu A^\nu$或者$\partial_\mu A_\mu$这种,但因为它们都可以使用$\partial^\nu A^\mu$表示出来,故没有必要考虑它们。
只含有$A^\mu$二次不变量为$A^\mu A_\mu$,没有一次不变量;只含有$\partial^\mu A^\nu$的一次不变量是它自我缩并得到的$\partial^\mu A_\mu$,二次的不变量是两个$\partial^\mu A^\nu$缩并得到的$\partial^\mu A^\nu \partial_\mu A_\nu$和$\partial^\mu A^\nu \partial_\nu A_\mu$。
由于参数都是标量,$\partial^\mu A^\nu$不能和参数缩并,也不能和$A^\mu$缩并($C^\nu A^\mu \partial_\nu A_\mu$要求参数是矢量,$A^\mu A^\nu \partial_\mu A_\nu$是三次项),因此我们得到了所有可能的不变量。
从而拉氏量形如
\[
    \mathcal{L} = C_1 A^\mu A_\mu + C_2 \partial^\mu A_\mu + C_3 \partial^\mu A^\nu \partial_\mu A_\nu + C_4 \partial^\mu A^\nu \partial_\nu A_\mu.
\]
代入欧拉-拉格朗日方程可以看出,$C_2$项在运动方程中不会引入任何项,故略去。
于是
\[
    \mathcal{L} = C_1 A^\mu A_\mu + C_3 \partial^\mu A^\nu \partial_\mu A_\nu + C_4 \partial^\mu A^\nu \partial_\nu A_\mu.
\]
代入欧拉-拉格朗日方程,得到
\begin{equation}
    \partial_\mu (C_3 \partial^\mu A^\nu +  C_4 \partial^\nu A^\mu) = C_1 A^\nu.
    \label{eq:vector-motion-eq}
\end{equation}
我们首先考虑$C_3 = - C_4$时的特殊情况。重新定义各系数,使得
\begin{equation}
    \mathcal{L} = - \frac{1}{2} \partial^\mu A^\nu \partial_\mu A_\nu + \frac{1}{2} \partial^\mu A^\nu \partial_\nu A_\mu + \frac{m^2}{2} A_\mu A^\mu,
\end{equation}
对应的,
\begin{equation}
    \partial_\mu (\partial^\mu A^\nu - \partial^\nu A^\mu) + m^2 A^\nu = 0.
    \label{eq:proca-eq}
\end{equation}
常定义
\begin{equation}
    F^{\mu \nu} = \partial^\mu A^\nu - \partial^\nu A^\mu,
\end{equation}
于是就有
\begin{equation}
    \mathcal{L} = - \frac{1}{4} F_{\mu \nu} F^{\mu \nu} + \frac{1}{2} m^2 A_\mu A^\mu.
    \label{eq:proca-lagrangian}
\end{equation}
现在回到一般情况。我们指出这样一个结论:无论$C_3,C_4$取什么值,对应的场$A^\mu$都可以和$C_3 = - C_4$时的某个场${A'}^\mu$建立一一对应。
% TODO:证明
因此布洛卡方程\eqref{eq:proca-eq}就不失一般性地描写了所有的矢量场的运动方程。

\eqref{eq:proca-eq}在$m \neq 0$时可以推导出克莱因-高登方程。注意到
\[
    m^2 \partial_\nu A^\nu = \partial_\nu \partial_\mu \partial^\nu A^\mu - \partial_\mu \partial^\mu \partial_\nu A^\nu = 0,
\]
于是
\begin{equation}
    \partial_\mu A^\mu = 0.
    \label{eq:lorentz-gauge}
\end{equation}
回代入\eqref{eq:proca-eq},发现其左边第二项为零,于是
\[
    \partial_\mu \partial^\mu A^\nu + m^2 A^\nu = 0.
\]
于是\eqref{eq:proca-eq}就约化成了\eqref{eq:lorentz-gauge}和四个克莱因-高登方程。
而当$m=0$时,运动方程在规范变换
\begin{equation}
    A^\mu \longrightarrow {A'}^\mu = A^\mu + \partial^\mu \varphi
\end{equation}
下不变。这意味着矢量场$A^\mu$的四个自由度实际上是多余的。%
\footnote{显然,只要选定了一个$\varphi$,同一个时间点上的$A^\mu$和${A'}^\mu$之间必定可以建立起一一对应关系。形象地说,不同$\varphi$对应的$A'$的运行轨迹相互平行,因此只需要其中一条轨迹就能够确定所有轨迹。选取特定的一条轨迹就是选取一个规范。
规范自由度——也就是决定“实际的轨道是哪一条”的自由度——是一个隐藏的额外自由度。
这里的情况和对称性自发破缺有点类似,在后者中,隐藏的自由度是序参量。不同的隐藏的额外自由度取值将系统的态空间分成了互不相交的分支。
可以认为规范自由度不是物理的自由度,也就是说它仅仅出现在拉氏量中,而规范自由度取值不同的状态在希尔伯特空间中应该被认为是同样的状态。
选取一个规范意味着先假定规范自由度取值不同的状态真的是不一样的,然后取状态空间中的一个分支。}%
换而言之,存在\emph{规范冗余性}。

\subsubsection{重矢量场的哈密顿量}

矢量场的共轭动量为
\[
    \pi_\mu = \partial_\mu A^0 - \partial^0 A_\mu,
\]
或者写成
\begin{equation}
    \pi^\mu = \partial^\mu A^0 - \partial^0 A^\mu.
\end{equation}
注意到$\pi^0 = 0$,因此可以只讨论其空间部分$\vb*{\pi}$。
在质量$m$不为零时场没有规范不变性,可以直接做计算得到
\begin{equation}
    A^0 = - \frac{1}{m^2} \div{\vb*{\pi}},
\end{equation}
以及
\begin{equation}
    \partial_0 A^0 = - \partial_i A^i = - \div{\vb*{A}},
\end{equation}
哈氏量为
\begin{equation}
    \mathcal{H} = \frac{1}{2} \vb*{\pi}^2 + \frac{1}{2m^2} (\div{\vb*{\pi}})^2 + \frac{1}{2} (\curl{\vb*{A}})^2 + \frac{1}{2} m^2 \vb*{A}^2.
\end{equation}
$m$出现在了分母中,这意味着无质量的场需要额外处理。

\subsubsection{麦克斯韦理论的经典规范选取}

现在来处理无质量的场。其运动方程为
\[
    \partial_\mu (\partial^\mu A^\nu - \partial^\nu A^\mu) = 0.
\]
无论$\partial_\mu A^\mu$是什么,总可以找到一个$\varphi$使得
\[
    \partial_\mu \partial^\mu \varphi = - \partial_\mu A^\mu,
\]
从而对应的有
\[
    \partial_\mu {A'}^\mu = 0.
\]
于是我们不失一般性地强行要求\eqref{eq:lorentz-gauge}对$m=0$时的矢量场成立。这称为选取了\concept{洛伦兹规范}。选取了洛伦兹规范意味着,实际的场自由度只有三个。知道了$A$的三个分量就可以计算出第四个。
当然,这不是唯一的规范选取方式。例如可以直接要求$A^0 = 0$,称为\concept{辐射规范}。
选取洛伦兹规范的好处在于,方程\eqref{eq:lorentz-gauge}是洛伦兹协变的,因此在做量子化时能够直接套用正则量子化关系而不必担心场方程不是洛伦兹协变而产生的修正。

我们施加洛伦兹规范。当然也可以选取别的规范,但这可能会破坏洛伦兹协变性,从而导致我们得到的哈密顿动力学实际上是带有约束的,从而给之后做量子化带来麻烦。
此时运动方程为
\begin{equation}
    \partial_\mu \partial^\mu A^\nu = 0.
    \label{eq:massless-vector-eq}
\end{equation}
拉氏量\eqref{eq:proca-lagrangian}直接导出的不是这个方程,于是我们使用另一个能够直接导出\eqref{eq:massless-vector-eq}的拉氏量
\begin{equation}
    \mathcal{L} = - \frac{1}{4} F_{\mu \nu} F^{\mu \nu} - \frac{1}{2} (\partial_\mu A^\mu)^2.
    \label{eq:maxwell-lagrangian-fixed}
\end{equation}
可以看到这个拉氏量多出了一项,这个实际上就是所谓的规范固定项。
在给定了洛伦兹规范的前提下,这个拉氏量实际上就和$m=0$的\eqref{eq:proca-lagrangian}是等价的。
此时
\begin{equation}
    \pi^0 = -\partial_\mu A^\mu, \quad \pi^i = \partial^i A^0 - \partial^0 A^i.
\end{equation}
当然,由洛伦兹规范,$\pi^0$就是零,不过我们完全可以算出哈密顿量之后再施加洛伦兹规范。
哈密顿量为 % TODO:这一部分似乎不需要太多笔墨,反正量子化的时候都是重新算的 关键之处在于哈密顿量和规范是有关的
\begin{equation}
    \mathcal{H} = 
\end{equation}

\subsection{无质量矢量场的正则量子化}

无质量矢量场的正则量子化涉及很多棘手的细节。

\subsubsection{横场量子化}

使用\eqref{eq:field-operator-fourier}展开一个无质量矢量场为
\begin{equation}
    A_\mu (\vb*{x}, t) = \int \frac{\dd[3]{\vb*{p}}}{(2\pi)^3} \frac{1}{\sqrt{2 \omega_{\vb*{p}}}} \sum_{r=0}^3 \epsilon_\mu^r(\vb*{p}) \left({a}_{\vb*{p}, r}^\dagger \ee^{ - \ii \vb*{p} \cdot \vb*{x} + \ii \omega_{\vb*{p}} t} + {a}_{\vb*{p}, r} \ee^{ \ii \vb*{p} \cdot \vb*{x} - \ii \omega_{\vb*{p}} t} \right), 
    \label{eq:expanding-massless-vector-field}
\end{equation}
由于没有质量,
\begin{equation}
    \omega_{\vb*{p}} = \abs{\vb*{p}}.
\end{equation}
$\epsilon^r$为一组闵可夫斯基时空的基矢量,称它们为\concept{偏振矢量},也即,
\begin{equation}
    (\epsilon^r)_\mu (\epsilon^{r'})^\mu = \eta^{r r'}.
\end{equation}
为了确定偏振矢量,通常要求
\begin{equation}
    \epsilon^1 \cdot p = \epsilon^2 \cdot p = 0,
\end{equation}
并认为$\epsilon^0$是类时的,而$\epsilon^{1,2,3}$是类空的。这样,当$p^\mu \propto (1, 0, 0, 1)$,即$\vb*{p}$指向$z$轴时,我们有
\begin{equation}
    \epsilon^0 = \pmqty{1 \\ 0 \\ 0 \\ 0}, \quad \epsilon^1 = \pmqty{0 \\ 1 \\ 0 \\ 0}, \quad \epsilon^2 = \pmqty{0 \\ 0 \\ 1 \\ 0}, \quad \epsilon^3 = \pmqty{0 \\ 0 \\ 0 \\ 1}.
    \label{eq:z-axis-p-epsilon}
\end{equation}
% TODO:这是$\epsilon^\mu$还是$\epsilon_\mu$???
$p$取其它值时只需要对\eqref{eq:z-axis-p-epsilon}做洛伦兹变换即可,因为$\epsilon$的定义完全是洛伦兹协变的。

% TODO:为什么?这一片我都没有动手算过,
可以计算出
\begin{equation}
    \pi^\mu (\vb*{x}, t) = \int \frac{\dd[3]{\vb*{p}}}{(2\pi)^3} \sqrt{\frac{\omega_{\vb*{p}}}{2}} \ii \sum_{r=0}^3 (\epsilon^r)^\mu (\vb*{p}) \left( {a}_{\vb*{p}, r} \ee^{\ii \vb*{p} \cdot \vb*{x} - \ii \omega_{\vb*{p}} t} - {a}_{\vb*{p}, r}^\dagger \ee^{ - \ii \vb*{p} \cdot \vb*{x} + \ii \omega_{\vb*{p}} t} \right),
\end{equation}
施加正则对易关系,通过计算得到
% TODO:真的可以**等价**地得到下式吗??
\begin{equation}
    \comm*{{a}_{\vb*{p}, \lambda}}{{a}^\dagger_{\vb*{p}', \lambda'}} = - \eta_{\lambda \lambda'} (2\pi)^3 \delta^3(\vb*{p} - \vb*{p}'), \quad \comm*{{a}^\dagger_{\vb*{p}, \lambda}}{{a}^\dagger_{\vb*{p}', \lambda'}} = \comm*{{a}_{\vb*{p}, \lambda}}{{a}_{\vb*{p}', \lambda'}} = 0.
\end{equation}
$\lambda=1, 2, 3$时对易关系是正确的,但是$\lambda=0$给出了一个不正常的对易关系
\[
    \comm*{{a}_{\vb*{p}, 0}}{{a}^\dagger_{\vb*{p}', 0}} = - (2\pi)^3 \delta^3 (\vb*{p} - \vb*{p}').
\]
例如,它产生的同样的单粒子态的内积将会是一个负数,这和我们对单粒子态的通常认识不符。
此外,哈密顿量成为
\begin{equation}
    {H} = \int \frac{\dd[3]{\vb*{p}}}{(2\pi)^3} \omega_{\vb*{p}} \left( - {a}_{\vb*{p},0}^\dagger {a}_{\vb*{p}, 0} + \sum_{i=1}^3 {a}_{\vb*{p},i}^\dagger {a}_{\vb*{p}, i} \right),
\end{equation}
因此能量非正定。
显然这些问题都和${a}^\dagger_{\vb*{p},0}$有关,也就是说来自一个非物理的自由度。
会有非物理的自由度是显然的,因为我们在处理一个有规范不变性的场却从来没有选取过一个规范。
现在我们处理的是量子场,因此既可以直接对场做约束,也可以缩小态空间的范围。

我们先尝试直接将洛伦兹规范作用在场上,即要求对场算符有
\[
    \partial^\mu A_\mu = 0.
\]
然而,这是不可能的,
% TODO
经过检验,Gupia-Blenler量子化条件
\begin{equation}
    \partial^\mu {A}_\mu^{(+)} \ket{\psi} = 0
    \label{eq:gupia-blenlder}
\end{equation}
是一个可行的方案。
它实际上约束了态空间的范围。
代入\eqref{eq:expanding-massless-vector-field},并注意到$\epsilon^1$与$\epsilon^2$和四维动量做内积得到零,我们发现
\begin{equation}
    ({a}_{\vb*{p}, 0} - {a}_{\vb*{p}, 3}) \ket{\psi} = 0.
\end{equation}
这意味着在无质量矢量场的态空间中哈密顿量实际上是
\begin{equation}
    {H} = \int \frac{\dd[3]{\vb*{p}}}{(2\pi)^3} \omega_{\vb*{p}} ({a}_{\vb*{p},1}^\dagger {a}_{\vb*{p}, 1} + {a}_{\vb*{p},2}^\dagger {a}_{\vb*{p}, 2}).
\end{equation}
于是负能量问题也就解决了。哈密顿量中没有出现的量可以直接被略去,因为它们对系统的动力学不产生任何影响。
% TODO:严格说明
于是取
\begin{equation}
    A_\mu (\vb*{x}, t) = \int \frac{\dd[3]{\vb*{x}}}{(2\pi)^3} \frac{1}{\sqrt{2 \omega_{\vb*{p}}}} \sum_{r=1}^2 \epsilon_\mu^r(\vb*{p}) \left({a}_{\vb*{p}, r}^\dagger \ee^{ - \ii \vb*{p} \cdot \vb*{x} + \ii \omega_{\vb*{p}} t} + {a}_{\vb*{p}, r} \ee^{ \ii \vb*{p} \cdot \vb*{x} - \ii \omega_{\vb*{p}} t} \right),
\end{equation}
以及
\begin{equation}
    \pi^\mu (\vb*{x}, t) = \int \frac{\dd[3]{\vb*{p}}}{(2\pi)^3} \sqrt{\frac{\omega_{\vb*{p}}}{2}} \ii \sum_{r=1}^2 (\epsilon^r)^\mu (\vb*{p}) \left( {a}_{\vb*{p}, r} \ee^{\ii \vb*{p} \cdot \vb*{x} - \ii \omega_{\vb*{p}} t} - {a}_{\vb*{p}, r}^\dagger \ee^{ - \ii \vb*{p} \cdot \vb*{x} + \ii \omega_{\vb*{p}} t} \right),
\end{equation}
重新计算对易关系得到
\begin{equation}
    \comm*{{a}_{\vb*{p}, \lambda}}{{a}^\dagger_{\vb*{p}', \lambda'}} = \delta_{\lambda \lambda'} (2\pi)^3 \delta^3(\vb*{p} - \vb*{p}'), \quad \comm*{{a}^\dagger_{\vb*{p}, \lambda}}{{a}^\dagger_{\vb*{p}', \lambda'}} = \comm*{{a}_{\vb*{p}, \lambda}}{{a}_{\vb*{p}', \lambda'}} = 0, \quad \lambda = 1, 2.
\end{equation}

\subsubsection{守恒量}

下面我们推导动量和自旋角动量的公式。轨道角动量的由于是动量衍生出来的量,我们暂不考虑。
首先假设$p^\mu \propto (1, 0, 0, 1)$。
按照\eqref{eq:spin-angular-momentum}可以计算得到
\[
    {S}_3 = \int \dd[3]{\vb*{x}} \mathcal{S}_3 = \ii \int \frac{\dd[3]{\vb*{p}}}{(2\pi)^3} (- {a}_{\vb*{p},1} {a}^\dagger_{\vb*{p}, 2} + {a}^\dagger_{\vb*{p}, 1} {a}_{\vb*{p}, 2} + {a}_{\vb*{p}, 2} {a}_{\vb*{p}, 1}^\dagger - {a}_{\vb*{p}, 2}^\dagger {a}_{\vb*{p}, 1} ) ,
\]
另外两个方向上的自旋角动量都是零。

我们原本预期矢量场会有三个自由度(因为\eqref{eq:lorentz-gauge}消除掉了一个自由度),但是实际上无质量矢量场只有两个自由度。
导致这一切的原因当然是无质量这个事实——它使得四维动量$p$不再能够写成$(1, 0, 0, 0)$这样的形式,而只能够写成$(1,0,0,1)$这样,从而让$A^0$和$A^3$相互抵消了。
从洛伦兹群在态空间上的表示出发可以更好地看待这个问题:$m=0$时洛伦兹群保持动量不变的小群不再是旋转群。
以一种更加物理的视角,无质量矢量场对应的粒子一直在以光速运动,不能找到一个相对它静止的参考系,因此对一个这样的粒子,实际上总是有一个特定的空间方向即它的运动方向,为了保持协变性,其自旋只能够沿着这个方向。换而言之,此时有意义的实际上是螺旋度而不是三维的角动量${\vb*{S}}$,即其内禀自由度是平面旋转群(以运动方向为轴旋转)的表示而不是三维旋转群的表示。
而对有质量的粒子,总是可以找到一个相对它静止的参考系,在这个参考系中空间是各向同性的,因此可以应用$SO(3)$的表示。
这和经典电磁场的偏振只有两个方向是对应的。
% TODO:经典场的傅里叶分量就是量子的产生湮灭算符

% TODO:场实际上只有两个自由度,因此粒子也只有两个内禀自由度,因此螺旋度是粒子的内禀自由度空间的CSCO。

% TODO:所以总之就是,无质量矢量场的自旋只在动量的方向上有非零分量,因此描述无质量矢量场的粒子的内禀自由度需要的实际上是螺度

\subsubsection{传播子}

由于洛伦兹度规的空间部分全部都是负的,我们有
\begin{equation}
    D_F = \mel*{0}{T A_{\mu}(x) A_{\nu}(y)}{0} = - \int \frac{\dd[4]{p}}{(2\pi)^4} \frac{\ii \eta_{\mu \nu}}{p^2 + \ii 0^+} \ee^{- \ii p \cdot (x - y)} = \int \frac{\dd[4]{p}}{(2\pi)^4} \frac{\ii \delta_{\mu \nu}}{p^2 + \ii 0^+} \ee^{- \ii p \cdot (x - y)}.
\end{equation}

有一点应该指出:规范选取\eqref{eq:gupia-blenlder}是作用在态上的。
这就是说,% TODO:虚光子???

\subsection{重矢量场的正则量子化}

\subsection{路径积分量子化和规范固定}

我们现在从路径积分的角度量子化无质量矢量场。我们首先尝试朴素地直接将$m=0$的\eqref{eq:proca-lagrangian}放入$\ee$指数,通过分部积分,得到
\begin{equation}
    S = - \frac{1}{4} \int \dd[4]{x} F_{\mu \nu} F^{\mu \nu} = \frac{1}{2} \int \dd[4]{x} = \frac{1}{2} \int \dd[4]{x} A^\mu (\eta_{\mu \nu} \partial^2 - \partial_\mu \partial_\nu) A^\nu,
\end{equation}
那么,合乎情理的下一步就是取积分核$(\eta_{\mu \nu} \partial^2 - \partial_\mu \partial_\nu)$的逆。
现在问题出现了:这个积分核根本就没有逆——其行列式为零。
因此,直接将$m=0$的\eqref{eq:proca-lagrangian}放入$\ee$指数得到的结果是发散的。
这当然是因为我们没有做规范固定。对无相互作用的无质量矢量场,局域对称操作
\[
    A^\mu \longrightarrow A^\mu + \partial^\mu \alpha
\]
不改变哈密顿量。实际上这是一个规范对称性,即这个局域操作不会改变量子态。然而,朴素地做路径积分不会考虑这一点,从而路径积分会对本该只求和一遍的场构型求和无数多遍。

本节将使用\concept{Faddeev-Popov量子化}方法,大致的思路是通过适当手段插入一个$\delta$函数,对一族规范等价的场构型,只挑选其中一个纳入路径积分。
这一方法的不足之处在于可能还不能消除所有的规范冗余性,但是对微扰计算来说,由于场构型的变动总是充分小的,Faddeev-Popov量子化足够给出可靠的答案。
我们在朴素版本的配分函数的$A$的积分后面插入
\[
    1 = \int \fd{\alpha} \delta(G(A^\alpha)) \det\left( \fdv{G(A^\alpha)}{\alpha} \right),
\]
其中$A^\alpha$就是$A+\partial \alpha$。这样就有
\[
    \begin{aligned}
        Z &= \int \fd{A} \int \fd{\alpha} \delta(G(A^\alpha)) \det\left( \fdv{G(A^\alpha)}{\alpha} \right) \ee^{\ii S[A]} \\
        &= \int \fd{\alpha} \int \fd{A^\alpha} \delta(G(A^\alpha)) \det\left( \fdv{G(A^\alpha)}{\alpha} \right) \ee^{\ii S[A^\alpha]},
    \end{aligned}
\]
其中$A^\alpha$的积分测度和$A$完全一样,且规范对称性意味着$S[A^\alpha]$和$S[A]$也一样。
为了保持洛伦兹协变性,$G(A)$可以选择为这样:
\begin{equation}
    G(A^\alpha) = \partial^\mu A_\mu^\alpha - \omega(x) = \partial^\mu A_\mu + \partial^2 \alpha - \omega(x),
\end{equation}
其中$\omega$是一个任意的标量函数。这样,雅可比行列式因子就可以提出积分号外,得到
\[
    \begin{aligned}
        Z &= (\det \partial^2) \int \fd{\alpha} \int \fd{A^\alpha} \delta(\partial^\mu A_\mu^\alpha - \omega) \ee^{\ii S[A^\alpha]} \\
        &= (\det \partial^2) \int \fd{\alpha} \int \fd{A} \delta(\partial^\mu A_\mu - \omega)  \ee^{\ii S[A]},
    \end{aligned}
\]
这里我们已经重新标记$A^\alpha$为$A$了。既然$\omega$可以随意变动,我们不妨对所有的$\omega$做一次求和,得到
\[
    \begin{aligned}
        Z' &= \int \fd{\omega} \ee^{-\ii \int \dd[4]{x} \frac{\omega^2}{2 \xi}} (\det \partial^2) \int \fd{\alpha} \int \fd{A} \delta(\partial^\mu A_\mu - \omega)  \ee^{\ii S[A]} \\
        &= (\det \partial^2) \int \fd{\alpha} \int \fd{A} \ee^{\ii S[A]} \exp(- \ii \int \dd[4]{x} \frac{1}{2 \xi} (\partial^\mu A_\mu)^2).
    \end{aligned}
\]
现在可以把前面的一大堆因子全部扔掉,得到最后的配分函数:
\begin{equation}
    Z = \int \fd{A} \exp(\ii \int \dd[4]{x} \left( - \frac{1}{4} F_{\mu \nu} F^{\mu \nu} - \frac{1}{2 \xi} (\partial^\mu A_\mu)^2\right)).
\end{equation}
可以看到,这实际上就是向拉氏量加了一个\concept{规范固定项},破坏了拉氏量的规范对称性,但是得到的物理是完全一样的。
$\xi=1$称为\concept{费曼规范},和\eqref{eq:maxwell-lagrangian-fixed}是一致的;$\xi=0$称为\concept{朗道规范}。

做完规范固定之后,可以按照正常的手续求出传播子。通过分部积分将作用量写成
\[
    S = \int \dd[4]{x} (A^\mu (\eta_{\mu \nu} \partial^2 - (1 - \frac{1}{\xi}) \partial_\mu \partial_\nu) A^\nu),
\]
于是
\[
    \mel{0}{T A_\mu(x) A_\nu(y)}{0} = \int \frac{\dd[4]{x}}{(2\pi)^4} \frac{\ii}{-k^2 \eta^{\mu \nu} + (1 - \frac{1}{\xi}) k^\mu k^\nu},
\]
并需要加上一个无穷小虚部。我们会发现
\[
    \left( -k^2 \eta^{\mu \nu} + (1 - \frac{1}{\xi}) k^\mu k^\nu \right) \left(\eta_{\nu \rho} - (1 - \xi) \frac{k_\nu k_\rho}{k^2} \right) = - k^2 \delta^\mu_\rho,
\]
于是
\begin{equation}
    D_F(x - y)_{\mu \nu} = \mel*{0}{T A_\mu(x) A_\nu(y)}{0} = \int \frac{\dd[4]{k}}{(2\pi)^4} \frac{-\ii}{k^2 + \ii 0^+} \left( \eta_{\mu \nu} - (1 - \xi) \frac{k_\mu k_\nu}{k^2} \right).
\end{equation}
这里多出来的一个负号是因为,在标量场和旋量场中,拉氏量动能项的两个导数算符分别作用在两个场上,将一个场视为入射,一个视为出射,切换到动量空间,得到的是$- \ii k \cdot \ii k = k^2$,而在这里,两个导数算符都作用在右边的场上,得到的是$(\ii k)^2=-k^2$。
请注意上式已经是做完规范固定之后的结果,无需再额外做任何规范固定,这就是说,对一张费曼图的中间态求和时,$\mu$和$\nu$的确要取遍$0$到$4$。
只有入射光子外线和出射光子外线需要特别选择。

\chapter{用量子场论计算可观察量}

\section{散射}

高能物理实验中涉及的物理过程的时间尺度通常远远小于我们能够观察的时间尺度,并且一般很难研究束缚态(部分是因为束缚态问题基本上是凝聚态研究的范畴而此时除了库仑定律以外也不需要太多物理,部分是因为微扰计算束缚态问题非常困难)。
因此,大部分情况下我们都只需要考虑初末态均在无穷远处的散射问题即可。因此,需要计算的主要就是$S$矩阵——实际上是动量表象下的$S$矩阵,以下如无特殊说明,$S$矩阵指的就是动量表象下的$S$矩阵。
然而,实验中能够制备的含有一定量的动量确定的粒子的态是完整的哈密顿量的本征态(而不是自由理论的本征态),此时的多粒子态的能量不是其单粒子能量的简单相加。
原则上我们同样可以微扰计算完整的哈密顿量的本征态,但是这将是非常费力并且困难的。

然而,在实际的散射问题中,无论是入射态还是出射态的结构都相对简单,因为实际的物理粒子的波函数的分布不可能是遍布全空间的平面波,而只能是动量大体上确定的波包,而将时间推到$\infty$或是$-\infty$时这些波包会相隔得足够远,从而它们之间没有相互作用,整个量子态的能量也就是各个粒子的能量之和。(或者等价地说,入射和出射时相互作用可能认为是“关闭”的,而只有粒子相隔得足够近才被打开)
因此,入射态和出射态实际上仍然能够写成
\[
    \ket{p_1, p_2, \cdots, p_n} = \sqrt{2\omega_{\vb*{p}_1}} \cdots \sqrt{2\omega_{\vb*{p}_n}} a^\dagger_{\vb*{p}_1} a^\dagger_{\vb*{p}_2} \cdots a^\dagger_{\vb*{p}_n} \ket{\Omega}
\]
的形式,但是此时的$a$算符——可以记为$a_\text{in}$或$a_\text{out}$——已经不再是自由场(“裸的”场)的$a$算符了,因为它创建的单粒子态是实际的、加入了相互作用的哈密顿量的本征态,即它创建的单粒子态是物理粒子,而不是没有相互作用时的裸粒子;同样$\omega$也不再是裸的单粒子能量。%
\footnote{
    我们对入射态和出射态的要求是非常严格的:一方面,出现在这些态中的粒子一定是能够稳定存在的粒子,寿命有限的准粒子(对应完整的哈密顿量的展宽的能级,可以看成是近似的本征态,但是因为各种扰动而不是严格的本征态)不能出现在入射态和出射态中;我们还进一步要求这些粒子之间的距离足够远,以至于可以近似看成自由的;最后我们还要求这些态被正确地归一化了,从而引入场强重整化因子。

    凝聚态理论考虑的系统状态比这多得多:寿命有限的准粒子时常需要被讨论,并且通常不会一次重整化就把所有相互作用都去除而只留下带自能修正的粒子。
    例如,费米液体理论中的“电子”实际上是已经经过相互作用修正的能带电子,但是即使在系统基态中这些准粒子的相互碰撞仍然需要纳入考虑。
}%
在海森堡绘景下,$S$矩阵的矩阵元基本上就具有
\[
    \braket*{p_1, p_2}{k_1, k_2} = \sqrt{2\omega_{\vb*{p}_1}} \sqrt{2\omega_{\vb*{p}_2}} \sqrt{2\omega_{\vb*{k}_1}} \sqrt{2\omega_{\vb*{k}_2}} \mel*{\Omega}{a_{\vb*{p}_2} a_{\vb*{p}_1} a^\dagger_{\vb*{k}_1} a^\dagger_{\vb*{k}_2}}{\Omega}
\]
这样的形式,这意味着只要能够找到$a_\text{in}$,$a_\text{out}$和裸场$a$之间的关系,即可确定关联函数和$S$矩阵的关系。
为了看出$a_\text{in}$,$a_\text{out}$和裸场$a$之间的关系,我们可以尝试为这种近乎独立的$a_\text{in}$和$a_\text{out}$所描述的“自由粒子”写下一个有效理论,显然这个有效理论是完整的、带有相互作用的理论不断重整化的结果,相互作用会修正“自由理论”的参数。
由于只考虑单粒子,相互作用带来的修正实际上就是所谓的自能修正:能够调整的参数包括质量项和$\partial_\mu \phi \partial^\mu \phi$项,对后者的调节等价于对场本身的变换。(此外,由于只考虑单粒子,且四维动量$p$的安排非常接近在壳,粒子无法衰变,因此自能修正没有虚部)
因此我们得出结论:$a_\text{in}$,$a_\text{out}$和裸场$a$之间应该差一个常数,这个常数是场强重整化引入的因子。
这就确定了关联函数和$S$矩阵之间的关系:在统一到动量空间中之后,两者首先由于相互作用修正的原因会差一个场强重整化因子;动量空间中的关联函数的解析性质和自由场的关联函数非常相似,只不过前者的极点给出的“质量”会出现跑动;相比之下,$S$矩阵要光滑很多,因为没有极点。
可以从关联函数计算$S$矩阵,但是不能反过来,因为$S$矩阵仅仅包含初态位于$-\infty$而末态位于$\infty$的过程。
因此应当有:
\[
    \prod_{i} \frac{\text{renormalization factors}}{\omega - \omega_{\vb*{k}} + \ii 0^+} \times \mel*{p}{S}{k} \propto \text{Fourier transformation of} \mel*{\Omega}{\phi(x_1) \cdots \phi(x_n)}{\Omega}.
\]

虽然我们做出了以上形式的论证,完整地做Wilson重整化群计算而得到场强重整化因子当然还是非常繁琐的,因此本节尝试以一种更加简单的方法建立微扰计算$S$矩阵的方法。
本节将始终在以下对高能物理来说非常一般的假设下工作:系统具有完整的洛伦兹对称性;入射和出射态中各个粒子相隔足够远;只有接近关联函数极点时的动量-能量安排才是值得分析的,因为其它时候实验现象也不明显。

本节将首先介绍一些可以使用$S$矩阵计算的物理量,以展示计算$S$矩阵的必要性,然后给出微扰计算$S$矩阵的方法。

\subsection{可观察物理量}

\subsubsection{散射截面}

\begin{equation}
    \mel*{p_1, p_2, \ldots, p_m}{\ii T}{k_1, k_2, \ldots, k_n} = \ii (2\pi)^4 \delta^4(\sum_i p_i - \sum_j k_j) \mathcal{M}(k_1, k_2, \ldots, k_n \to p_1, p_2, \ldots, p_m).
\end{equation}

\subsubsection{跃迁率}

\subsubsection{非相对论极限}

我们知道$S$矩阵可以用所谓李普曼-施温格方程写出。非相对论性单粒子量子力学中通常采取这样的归一化方案:
\[
    \hat{T} = \hat{H}' + \hat{H}' \frac{1}{E - \hat{H}_0} \hat{H}' + \cdots, \quad \hat{S} = 1 - 2 \pi \ii \delta(E_f - E_i) \hat{T},
\]
% TODO:怎么计算等效势能

\subsection{S矩阵和关联函数的关系}

\subsubsection{标量场编时格林函数的解析结构和LSZ约化公式}

考虑一个$n$点关联函数$\mel{\Omega}{T \phi(x_1) \cdots \phi(x_n)}{\Omega}$。
我们对$x_1$做傅里叶变换,这是因为实际上用于计算$S$矩阵矩阵元的本征态都是动量本征态,因此我们首先应该在$n$点关联函数中引入动量的概念。
我们没有直接使用用动量标记的产生湮灭算符,因为我们还希望用频率代替时间,而“频率”并不是量子数,因此不是任何一套产生湮灭算符的标签的一部分,如果使用用动量标记的产生湮灭算符就还需要对时间做一次傅里叶变换,表达式就看起来不协变了。
反之,对四维坐标$x_1$做完傅里叶变换之后,我们就可以用四维动量$k_1$代替$x_1$作为标记了,洛伦兹协变性就很明显。
这样得到的动量空间关联函数不保证入射和出射的四维动量是在壳的。(另一方面,$S$矩阵肯定是在壳的)

我们有
\begin{equation}
    \begin{aligned}
        &\quad \int \dd[4]{x_1} \ee^{\ii k_1 \cdot x_1} \mel{\Omega}{T \phi(x_1) \phi(x_2) \cdots \phi(x_n)}{\Omega} \\
        &= \left( \int_{-\infty}^{T_-} + \int_{T_-}^{T_+} + \int_{T_+}^\infty \right) \dd{t_1} \int \dd[3]{\vb*{x}_1} \ee^{\ii \omega_1 t_1 - \ii \vb*{k}_1 \cdot \vb*{x}_1} \mel{\Omega}{T \phi(x_1) \phi(x_2) \cdots \phi(x_n)}{\Omega} \\
        &= \int_{-\infty}^{T_-} \dd{t_1} \ee^{\ii \omega_1 t_1} \int \dd[3]{\vb*{x}_1} \ee^{- \ii \vb*{k}_1 \cdot \vb*{x}_1} \mel{\Omega}{T [\phi(x_2) \cdots \phi(x_n)] \phi(x_1)}{\Omega} \\
        &+ \int_{T_+}^{\infty} \dd{t_1} \ee^{\ii \omega_1 t_1} \int \dd[3]{\vb*{x}_1} \ee^{- \ii \vb*{k}_1 \cdot \vb*{x}_1} \mel{\Omega}{\phi(x_1) T [\phi(x_2) \cdots \phi(x_n)]}{\Omega} \\
        &+ \int_{T_-}^{T_+} \dd{t_1} \ee^{\ii \omega_1 t_1} \int \dd[3]{\vb*{x}_1} \ee^{- \ii \vb*{k}_1 \cdot \vb*{x}_1} \mel{\Omega}{T [\phi(x_1) \phi(x_2) \cdots \phi(x_n)]}{\Omega}.
    \end{aligned}
    \label{eq:correlation-after-fourier}
\end{equation}
最后一个等号只需要令$T^+$充分大,$T_-$充分小即可成立。
做完傅里叶变换之后的结果肯定具有奇异性:在相互作用下稳定的单粒子态有确定的动量和能量,因此在$\omega_1$和$\vb*{k}_1$之间满足正确的色散关系时,上式发散。
(多粒子态由于能量给定时其中某个动量可以连续变化,不会产生奇异性,但是会产生有限高度的峰)

高能物理实验中实际会测量的区域通常就在这种会导致奇异性的$\omega$和$\vb*{k}$的取值附近,因为这里效应最明显。
在\eqref{eq:correlation-after-fourier}的极点附近,$T_-$到$T_+$区段的积分基本上没有什么作用,因为其奇异性弱于$T_+$到$\infty$的积分和$-\infty$到$T_-$的积分。
因此之后我们不再考虑$T_-$到$T_+$区段的积分。

现在使用常用的技巧:在算符连乘积序列中插入一个完备关系,从而将$n$点关联函数转化为$n-1$点关联函数。
即使在加入相互作用之后,动量仍然是好量子数,因此可以有完备关系
\[
    1 = \sum_{\Lambda} \int \frac{\dd[3]{\vb*{p}}}{(2\pi)^3} \dyad*{\Lambda_{\vb*{p}}}, 
\]
其中$\Lambda$标记了冗余的量子数。对\eqref{eq:correlation-after-fourier}最后一步的第二项,插入完备关系得到
\[
    \begin{aligned}
        &\quad \int_{T_+}^{\infty} \dd{t_1} \ee^{\ii \omega_1 t_1} \int \dd[3]{\vb*{x}_1} \ee^{- \ii \vb*{k}_1 \cdot \vb*{x}_1} \mel{\Omega}{\phi(x_1) T [\phi(x_2) \cdots \phi(x_n)]}{\Omega} \\
        &= \int_{T_+}^{\infty} \dd{t_1} \ee^{\ii \omega_1 t_1} \int \dd[3]{\vb*{x}_1} \ee^{- \ii \vb*{k}_1 \cdot \vb*{x}_1} \mel*{\Omega}{\phi(x_1) \sum_{\Lambda} \int \frac{\dd[3]{\vb*{p}}}{(2\pi)^3} \dyad*{\Lambda_{\vb*{p}}} T [\phi(x_2) \cdots \phi(x_n)]}{\Omega} \\
        &= \sum_{\Lambda} \int \frac{\dd[3]{\vb*{p}}}{(2\pi)^3} \int_{T_+}^{\infty} \dd{t_1} \ee^{\ii \omega_1 t_1} \int \dd[3]{\vb*{x}_1} \ee^{- \ii \vb*{k}_1 \cdot \vb*{x}_1} \mel*{\Omega}{\phi(x_1)}{\Lambda_{\vb*{p}}} \mel*{\Lambda_{\vb*{p}}}{T [\phi(x_2) \cdots \phi(x_n)]}{\Omega}.
    \end{aligned}
\]
因子$\mel*{\Omega}{\phi(x_1)}{\Lambda_{\vb*{p}}}$可以进一步展开。
首先,因子$\mel*{\Omega}{\phi(0)}{\Lambda_{\vb*{k}_1}}$实际上只在$\ket{\Lambda_{\vb*{k}_1}}$是单粒子态时才能够有非零值(因为它可以写成一个经过场强重整化的产生算符作用在真空态上,而入射态和出射态可以近似看成自由理论,从而被夹在$\mel*{\Omega}{\cdot}{\Omega}$中的产生算符和湮灭算符数目必须一致)。
其次,我们注意到,实际上可以将$\phi(x_1)$写成$\ee^{\ii P \cdot x} \phi(0) \ee^{- \ii P \cdot x}$,其中$P$是全空间的动量算符,或者说平移群的生成元。在有相互作用的情况下,这样作用的结果并不方便计算,因为其中包含一个时间演化算符,但是我们总是可以形式地写出这样的式子。
于是
\[
    \mel*{\Omega}{\phi(x)}{\Lambda_{\vb*{p}}} = \mel*{\Omega}{\phi(x)}{\Lambda_{\vb*{p}}} \ee^{- \ii p \cdot x} |_{p^0 = E_{\vb*{p}}}. 
\]
于是就有
\[
    \begin{aligned}
        &\quad \int_{T_+}^{\infty} \dd{t_1} \ee^{\ii \omega_1 t_1} \int \dd[3]{\vb*{x}_1} \ee^{- \ii \vb*{k}_1 \cdot \vb*{x}_1} \mel{\Omega}{\phi(x_1) T [\phi(x_2) \cdots \phi(x_n)]}{\Omega} \\
        &= \sum_{\Lambda} \int \frac{\dd[3]{\vb*{p}}}{(2\pi)^3} \int_{T_+}^{\infty} \dd{t_1} \ee^{\ii (\omega_1 - E_{\vb*{p}}) t_1} \int \dd[3]{\vb*{x}_1} \ee^{- \ii (\vb*{k}_1 - \vb*{p}) \cdot \vb*{x}_1} \mel*{\Omega}{\phi(0)}{\Lambda_{\vb*{p}}} \mel*{\Lambda_{\vb*{p}}}{T [\phi(x_2) \cdots \phi(x_n)]}{\Omega} \\
        &= \sum_{\Lambda} \int \frac{\dd[3]{\vb*{p}}}{(2\pi)^3} \left( - \frac{1}{\ii (\omega_1 - E_{\vb*{p}} + \ii 0^+)} \ee^{\ii (\omega_1 - E_{\vb*{p}}) T_+} \right) (2\pi)^3 \delta(\vb*{k}_1 - \vb*{p}) \\
        & \quad \times \mel*{\Omega}{\phi(0)}{\Lambda_{\vb*{p}}} \mel*{\Lambda_{\vb*{p}}}{T [\phi(x_2) \cdots \phi(x_n)]}{\Omega} \\
        &= \sum_{\Lambda} \frac{\ii}{\omega_1 - E_{\vb*{k}_1}(\Lambda) + \ii 0^+} \ee^{\ii (\omega_1 - E_{\vb*{p}}) T_+} \mel*{\Omega}{\phi(0)}{\Lambda_{\vb*{k}_1}} \mel*{\Lambda_{\vb*{k}_1}}{T [\phi(x_2) \cdots \phi(x_n)]}{\Omega}.
    \end{aligned}
\]
上式中(以及之后),$E_{\vb*{k}}$代表的都是$\ket{\Lambda_{\vb*{k}_1}}$的能量,不是裸粒子的能量,已经加入了自能修正。
在这里我们使用标准的在对时间的积分中引入无穷小虚部,让$\omega_1$的奇点出现在下半平面的方法;实际上,如果不这样,积分也无法收敛。
我们现在稍微改变一下$\ket{\Lambda_{\vb*{p}}}$的归一化方式。目前使用的归一化方案是不协变的,对应于$\ket{\vb*{p}}$,现在我们转而使用
\[
    \ket{\lambda_{\vb*{p}}} = \sqrt{2 E_{\vb*{p}}} \ket{\Lambda_{\vb*{p}}},
\]
于是就有
\[
    \begin{aligned}
        &\quad \int_{T_+}^{\infty} \dd{t_1} \ee^{\ii \omega_1 t_1} \int \dd[3]{\vb*{x}_1} \ee^{- \ii \vb*{k}_1 \cdot \vb*{x}_1} \mel{\Omega}{\phi(x_1) T [\phi(x_2) \cdots \phi(x_n)]}{\Omega} \\
        &= \sum_{\lambda} \frac{1}{2E_{\vb*{k}_1}(\Lambda)} \frac{\ii}{\omega_1 - E_{\vb*{k}_1}(\Lambda) + \ii 0^+} \ee^{\ii (\omega_1 - E_{\vb*{p}}) T_+} \mel*{\Omega}{\phi(0)}{\lambda_{\vb*{k}_1}} \mel*{\lambda_{\vb*{k}_1}}{T [\phi(x_2) \cdots \phi(x_n)]}{\Omega}.
    \end{aligned}
\]
在接近\eqref{eq:correlation-after-fourier}的极点时,就有
\[
    \begin{aligned}
        &\quad \int_{T_+}^{\infty} \dd{t_1} \ee^{\ii \omega_1 t_1} \int \dd[3]{\vb*{x}_1} \ee^{- \ii \vb*{k}_1 \cdot \vb*{x}_1} \mel{\Omega}{\phi(x_1) T [\phi(x_2) \cdots \phi(x_n)]}{\Omega} \\
        &= \sum_{\lambda} \frac{\ii}{(\omega_1)^2 - (E_{\vb*{k}_1}(\lambda))^2 + \ii 0^+} \mel*{\Omega}{\phi(0)}{\lambda_{\vb*{k}_1}} \mel*{\lambda_{\vb*{k}_1}}{T [\phi(x_2) \cdots \phi(x_n)]}{\Omega}.
    \end{aligned}
\]
类似地可以得到
\[
    \begin{aligned}
        &\quad \int_{-\infty}^{T_-} \dd{t_1} \ee^{\ii \omega_1 t_1} \int \dd[3]{\vb*{x}_1} \ee^{- \ii \vb*{k}_1 \cdot \vb*{x}_1} \mel{\Omega}{T [\phi(x_2) \cdots \phi(x_n)] \phi(x_1)}{\Omega} \\
        &= - \sum_{\lambda} \frac{1}{2E_{\vb*{k}_1}(\lambda)} \frac{\ii}{\omega_1 + E_{\vb*{k}_1}(\lambda) + \ii 0^+} \ee^{\ii (\omega_1 + E_{\vb*{p}}) T_-} \mel*{\Omega}{T [\phi(x_2) \cdots \phi(x_n)]}{\lambda_{\vb*{k}_1}} \mel*{\lambda_{\vb*{k}_1}}{\phi(0)}{\Omega},
    \end{aligned}
\]
其中由于$\phi(x_1)$的位置发生了变化,一些量的正负号和左右矢的顺序发生了变化。不过,这一项并不产生极点。
因此最终我们得到
\[
    \begin{aligned}
        &\quad \int_{T_+}^{\infty} \dd{t_1} \ee^{\ii \omega_1 t_1} \int \dd[3]{\vb*{x}_1} \ee^{- \ii \vb*{k}_1 \cdot \vb*{x}_1} \mel{\Omega}{\phi(x_1) T [\phi(x_2) \cdots \phi(x_n)]}{\Omega} \\
        &\stackrel{\omega_1 \to E_{\vb*{k}_1}(\Lambda)}{\sim} \sum_{\lambda} \frac{\ii}{(\omega_1)^2 - (E_{\vb*{k}_1}(\lambda))^2 + \ii 0^+} \mel*{\Omega}{\phi(0)}{\lambda_{\vb*{k}_1}} \mel*{\lambda_{\vb*{k}_1}}{T [\phi(x_2) \cdots \phi(x_n)]}{\Omega}.
    \end{aligned}
\]
此外,设$U$是一个让三维动量减小$\vb*{k}_1$的洛伦兹变换,则
\[
    \begin{aligned}
        \mel*{\Omega}{\phi(0)}{\lambda_{\vb*{k}_1}} &= \mel*{\Omega}{\phi(0)U^{-1} U}{\lambda_{\vb*{k}_1}} \\
        &= \mel*{\Omega}{U \phi(0)U^{-1} U}{\lambda_{\vb*{k}_1}} \\
        &= \mel*{\Omega}{\phi(0)}{\lambda_0},
    \end{aligned}
\]
第二个等号是因为真空态在洛伦兹变换下显然是不变的,第三个等号用到了$U \phi(0) U^{-1} = \phi(0)$这一事实(将$\phi(0)$展开为傅里叶级数就能看出为什么)。
于是就得到
\[
    \begin{aligned}
        &\quad \int \dd[4]{x_1} \ee^{\ii k_1 \cdot x_1} \mel{\Omega}{\phi(x_1) T [\phi(x_2) \cdots \phi(x_n)]}{\Omega} \\
        &\stackrel{\omega_1 \to E_{\vb*{k}_1}(\lambda)}{\sim} \sum_{\lambda} \frac{\ii}{(\omega_1)^2 - (E_{\vb*{k}_1}(\lambda))^2 + \ii 0^+} \mel*{\Omega}{\phi(0)}{\lambda_{0}} \mel*{\lambda_{\vb*{k}_1}}{T [\phi(x_2) \cdots \phi(x_n)]}{\Omega}.
    \end{aligned}
\]
上面的结果的意义非常明显了:$x_1$变量做了傅里叶变换的关联函数会有一系列极点,这些极点的具体位置由在带相互作用的场论下的本征态的能谱(而不是自由粒子的能谱)决定;并且,会多出来一个因子$\mel*{\Omega}{\phi(0)}{\lambda_{0}}$。
由于$\ket{\lambda_{\vb*{k}_1}}$是经过相互作用修正的单粒子态,实际上只需要一个动量参数就足够标记它,于是我们去掉对$\lambda$的求和(因为显然只有一个可能的$\lambda$),并且用$\omega$代替$E$(再次提醒:这是已经经过相互作用修正的单粒子能量),就得到
\begin{equation}
    \begin{aligned}
        &\quad \int \dd[4]{x_1} \ee^{\ii k_1 \cdot x_1} \mel{\Omega}{\phi(x_1) T [\phi(x_2) \cdots \phi(x_n)]}{\Omega} \\
        &\stackrel{\omega_1 \to \omega_{\vb*{k}_1}}{\sim} \frac{\ii}{\omega_1^2 - \omega_{\vb*{k}_1}^2 + \ii 0^+} \mel*{\Omega}{\phi(0)}{p={0}} \mel*{k_1}{T [\phi(x_2) \cdots \phi(x_n)]}{\Omega}.
    \end{aligned}
    \label{eq:scalar-correlation-pole-single}
\end{equation}
这里我们已经用$\ket{k_1}$代替了$\ket{\lambda_{\vb*{k}_1}}$,前者就表示无穷远处的单粒子态。

现在设想我们对$\mel*{\Omega}{T \phi(x_1) \phi(x_n)}{\Omega}$中的每一个位置变量都做傅里叶变换。
由\eqref{eq:scalar-correlation-pole-single},会发现每个$k_i$实际上都有极点。
我们实际上可以在一个公式内把所有这些极点都反映出来。考虑对\eqref{eq:scalar-correlation-pole-single}中的$x_2$做傅里叶变换,由于$x_2$仅仅包含在最后一个因子中,只需要计算
\[
    \begin{aligned}
        &\quad \int \dd[4]{x_2} \ee^{\ii k_2 \cdot x_2} \mel*{k_1}{T [\phi(x_2) \cdots \phi(x_n)]}{\Omega} \\
        &= \left( \int_{-\infty}^{T_-} + \int_{T_-}^{T_+} + \int_{T_+}^\infty \right) \dd{t_2} \ee^{\ii \omega_2 t_2} \int \dd[3]{\vb*{x}_2} \ee^{-\ii \vb*{k}_2 \cdot \vb*{x}_2} \mel*{k_1}{T [\phi(x_2) \cdots \phi(x_n)]}{\Omega}.
    \end{aligned}
\]
仿照我们先前做的操作,只需要计算$T_+$到$\infty$的积分即可得到最为奇异的部分,于是可以将$\phi(x_2)$提出编时算符的作用域内,放在最左边,然后仿照前面的操作,在$\phi(x_2)$和编时算符序列之间插入完备性关系,得到
\[
    \begin{aligned}
        &\quad \int_{T_+}^\infty \dd{t_2} \ee^{\ii \omega_2 t_2} \int \dd[3]{\vb*{x}_2} \ee^{-\ii \vb*{k}_2 \cdot \vb*{x}_2} \mel*{k_1}{\phi(x_2) T [\phi(x_3) \cdots \phi(x_n)]}{\Omega} \\
        &= \int \frac{\dd[3]{\vb*{p}_1}}{(2\pi)^3} \frac{1}{2 \omega_{\vb*{p}_1}} \int \frac{\dd[3]{\vb*{p}_2}}{(2\pi)^3} \frac{1}{2 \omega_{\vb*{p}_2}} \int_{T_+}^\infty \dd{t_2} \ee^{\ii \omega_2 t_2} \int \dd[3]{\vb*{x}_2} \ee^{-\ii \vb*{k}_2 \cdot \vb*{x}_2} \\ 
        &\quad \quad \times \mel*{k_1}{\phi(x_2)}{p_1, p_2} \mel*{p_1, p_2}{T [\phi(x_3) \cdots \phi(x_n)]}{\Omega} \\
        &= \int \frac{\dd[3]{\vb*{p}}}{(2\pi)^3} \frac{1}{2 \omega_{\vb*{p}}} \int_{T_+}^\infty \dd{t_2} \ee^{\ii \omega_2 t_2} \int \dd[3]{\vb*{x}_2} \ee^{-\ii \vb*{k}_2 \cdot \vb*{x}_2}  \mel*{\Omega}{\phi(x_2)}{p} \mel*{k_1, p}{T [\phi(x_3) \cdots \phi(x_n)]}{\Omega} \\
    \end{aligned}
\]
这一次只有$\mel*{k_1}{\phi(x_2)}{p_1, p_2}$有非零值(注意$\ket{p_1, p_2}$是入射/反射态,可以用经过场重整化的产生算符作用两次产生),于是我们引入了两个动量积分;第二个等号还是因为$\ket{p_1, p_2}$可以拆分,从而其中一个必须和$k_1$相等。
对因子$\mel*{\Omega}{\phi(x_2)}{p}$施加如前所述的插入四维平移算符的操作,得到
\[
    \mel*{\Omega}{\phi(x_2)}{p} = \mel*{\Omega}{\phi(0)}{p=0} \ee^{-\ii p \cdot x_2}|_{p^0=E_{\vb*{p}}},
\]
然后积掉$t_2$和$\vb*{x}_2$,就得到
\begin{equation}
    \begin{aligned}
        &\quad \int \dd[4]{x_1} \ee^{\ii k_1 \cdot x_1} \int \dd[4]{x_2} \ee^{\ii k_2 \cdot x_2} \mel{\Omega}{\phi(x_1) T [\phi(x_2) \cdots \phi(x_n)]}{\Omega} \\
        &\stackrel{\omega_1 \to \omega_{\vb*{k}_1}, \omega_2 \to \omega_{\vb*{k}_2}}{\sim} \frac{\ii \mel*{\Omega}{\phi(0)}{p={0}}}{\omega_1^2 - \omega_{\vb*{k}_1}^2 + \ii 0^+} \frac{\ii \mel*{\Omega}{\phi(0)}{p={0}}}{\omega_2^2 - \omega_{\vb*{k}_2}^2 + \ii 0^+} \mel*{k_1, k_2}{T [\phi(x_2) \cdots \phi(x_n)]}{\Omega}.
    \end{aligned}
    \label{eq:scalar-correlation-pole-double}
\end{equation}
如此重复——实际上,我们还可以反过来做以上步骤,此时为了让极点出现,傅里叶变换的$\ee$指数要加上一个减号,在本节处理的标量场中这无所谓,但是对有粒子-准粒子区别的场,代表入射粒子的场算符和代表出射粒子的场算符差一个负号,从而一些要正着做傅里叶变换一些要反着做傅里叶变换——就得到
\begin{equation}
    \begin{aligned}
        &\quad \prod_{i=1}^m \int \dd[4]{x_i} \ee^{\ii p_i \cdot x_i} \prod_{j=1}^n \int \dd[4]{y_j} \ee^{- \ii k_j \cdot y_j} \mel{\Omega}{T [\phi(x_1) \cdots \phi(x_n) \phi(y_1) \cdots \phi(y_n)]}{\Omega} \\
        &\stackrel{p_i^0 \to \omega_{\vb*{p}_i}, \; k_j^0 \to \omega_{\vb*{k}_j}}{\sim} \prod_{i=1}^m \frac{\ii \sqrt{Z}}{\omega_i^2 - \omega_{\vb*{p}_i}^2 + \ii 0^+} \prod_{j=1}^n \frac{\ii \sqrt{Z}}{\omega_j^2 - \omega_{\vb*{k}_j}^2 + \ii 0^+} \braket*{p_1, p_2, \ldots, p_m}{k_1, k_2, \ldots, k_n},
    \end{aligned}
    \label{eq:lsz-reduction-scalar}
\end{equation}
其中
\begin{equation}
    Z = \abs{\mel*{\Omega}{\phi(0)}{p=0}}^2.
    \label{eq:z-factor-def}
\end{equation}
以上所有的推导都是在海森堡绘景下完成的,因此上式右边的因子就是$S$矩阵。
于是我们就得到了联系$S$矩阵和关联函数的公式,其形式和之前的分析完全一样,$Z$正是场强重整化因子。%
\eqref{eq:lsz-reduction-scalar}就是标量场的\concept{LSZ约化公式},其形式和之前我们预期的完全一样。
实际上,LSZ约化公式说明重整化后的格林函数是重整化前的$1 / Z^{(m+n)/2}$倍。

\subsubsection{$S$矩阵的微扰计算}

在得到了\eqref{eq:lsz-reduction-scalar}之后就可以微扰计算$S$矩阵了,因为可以微扰计算关联函数。
实际上,计算关联函数比计算$S$矩阵更加困难。这本质上是因为$S$矩阵丢弃了初末态均为有限时间的信息($S$矩阵仅仅保留了动量空间关联函数那些四维动量均在壳的那部分初末态),并且在费曼图的语言下有非常显然的解释。
下面我们来分析关联函数的费曼图,并给出直接微扰计算$S$矩阵的方法。

计算关联函数的任何一张费曼图都具有这样的形式:一个amputated diagram居于中间,外线和它之间连接有自能修正图。
在动量空间下,一张自能修正图可以等价地看成
\[
    \int \dd[4]{x} \ee^{\ii p \cdot x} \int \dd[4]{y} \ee^{- \ii k \cdot y} \mel{\Omega}{\phi(x) \phi(y)}{\Omega},
\]
按照\eqref{eq:z-factor-def},$\sqrt{Z}$对应$\mel{\Omega}{\phi(x_1) a^\dagger_\text{in}}{\Omega}$,这个关联函数中的两个场中,一个做了场强重整化而另一个没有。
既然$\mel{\Omega}{a_\text{out} a^\dagger_\text{in}}{\Omega}$就是$1$,$\mel{\Omega}{\phi(x_1) \phi(x_2)}{\Omega}$的场强重整化因子为$Z$。

关联函数$\mel{\Omega}{\phi(x_1) \phi(x_2)}{\Omega}$可以直接微扰计算,从而在理论已经给定的情况下,我们可以把$Z$到底是什么写出来。
设$- \ii M^2(p^2)$是单粒子不可约图(对应自能修正),那么就有
\[
    \begin{aligned}
        &\quad \int \dd[4]{x} \ee^{\ii p \cdot x} \int \dd[4]{y} \ee^{- \ii k \cdot y} \mel{\Omega}{\phi(x_1) \phi(x_2)}{\Omega} \\
        &= \frac{\ii}{p^2 - m_0^2} + \frac{\ii}{p^2 - m_0^2} (- \ii M^2(p^2)) \frac{\ii}{p^2 - m_0^2} + \cdots \\
        &= \frac{\ii}{p^2 - m_0^2 - M^2(p^2)},
    \end{aligned}
\]
其中$m_0$为粒子裸质量。自能$M^2$关于$p^2$的最低阶项当然是$p^2$自己,即
\[
    M^2 = c_0 + c_1 p^2 + \cdots.
\]
在低阶近似下$c_1=1$,高阶下则会有可见的修正,实际上这就对应着对$(\partial_\mu \phi)^2$的修正。
因此,在极点附近,我们有
\begin{equation}
    \int \dd[4]{x} \ee^{\ii p \cdot x} \int \dd[4]{y} \ee^{- \ii k \cdot y} \mel{\Omega}{\phi(x_1) \phi(x_2)}{\Omega} \stackrel{p^0 \to E_{\vb*{p}}}{\sim} \frac{\ii Z}{p^2 - m^2} + \text{regular}. 
\end{equation}
因此这给出了计算$Z$和有效质量$m$的方法:微扰计算自能修正,然后按照上式化简,观察极点位置就得到了$m$,在极点附近比较关联函数和$\ii / (p^2 - m^2)$就得到了$Z$。
将上式代入\eqref{eq:lsz-reduction-scalar},就发现
\begin{equation}
    \braket*{p_1, p_2, \ldots, p_m}{k_1, k_2, \ldots, k_n} \stackrel{p_i^0 \to \omega_{\vb*{p}_i}, \; k_j^0 \to \omega_{\vb*{k}_j}}{\sim} (\sqrt{Z})^{m+n} \times \text{amputated diagrams}.
    \label{eq:amputated-diagram-z-factor}
\end{equation}
因此,计算自能修正并得到$Z$之后,只需要计算amputated diagrams就能够微扰计算得到$S$矩阵。

我们还可以更加简化一些。注意到,所有只含有单粒子自能修正的图——即有$n$个入射粒子,$n$个出射粒子,总共有$n$个连通子图,不同粒子对应不同连通子图的图——在amputate之后就是简单地将入射端和出射端连接起来,而这些图正好对应$S=1 + \ii T$中的$1$,因此如果只计算$\mathcal{M}$,无需计算这些图。

如果只计算树图,那么显然$Z=1$,因为此时没有任何自能修正,从而无论是物理质量还是场强都没有做重整化。
因此,计算树图时只将amputated tree diagram求和即可。
圈图的计算一般要用到重整化,此时为了方便看出场强重整化因子,并不会使用\eqref{eq:amputated-diagram-z-factor}。
但是,在圈图计算中,场强重整化的那些$\sqrt{Z}$通常是作为抵消项被引入了,在施加重整化条件之后,它们和圈图计算中的发散抵消了,从而其实也无需显式计算$Z$。
总之我们实际上不会直接使用LSZ约化公式计算散射振幅:如果只计算树图,那么$Z=1$;而如果计算圈图,那么$Z$是作为抵消项被考虑进去的,圈图计算的最终结果是对树图中的传播子的物理参数(基本上是质量)和顶角函数做了一定修正,因此$Z$同样不会出现在最终的计算结果中。

% TODO:路径积分量子化中的LSZ

\section{正规化和重整化}

如果朴素地做圈图计算,通常会得到紫外发散。%
\footnote{
    相对论性量子场论中的红外发散一般是因为少考虑了一些过程,如由于光子无能隙,可以任意地产生和消灭,从而一个过程的概率分散在很多个有入射和出射“软光子”(能量很低并无可观测效应的光子)。
    当求和所有这些过程后,发散一般就消失了,并不具有特别的意义。

    凝聚态场论中的情况正好相反:紫外发散是没有关系的,因为凝聚态系统中有最小的特征长度,如果出现了发散,可以引入一个物理可观测的紫外截断消除这个发散(如BCS理论中的$\omega_\text{D}$)。
    红外发散反而不那么好处理。
}%
这并不特别令人意外,因为没有什么保证了我们的理论在任意能标下都一定成立,从而,将动量积分的上限推到无穷大大约是不合适的。
要想计算出有意义的结果通常要求我们知道更高能标处的物理的细节;然而,如果我们的理论实际上具有低能标下的一个不动点,那么更高能标处的物理实际上无关紧要,只要我们只关心低能的现象。

在这种情况下,可以做下面的操作来消除发散:
\begin{enumerate}
    \item 在圈图计算中寻找一个能够标记重整化群流的参数,可能是和Wilson重整化群流一致的动量积分上限$\Lambda$,也可以将维数延拓到实数中,从而用维数偏离$4$的程度$\epsilon$做这个参数。此时积分的发散部分可以被分离出来,这就是\concept{正规化}。
    \item 令理论中的各个参数(所谓“裸”参数)跑动起来,包括场强,即引入\concept{抵消项},其大小暂时未知。
    \item 计算若干个可以实际观察到的物理量(通常具有和裸参数类似的物理意义,从而它们可以称为\concept{物理参数},如自由场的关联函数的极点给出裸质量,而有相互作用的关联函数的极点给出物理质量),将它们写成裸参数、参数跑动(即抵消项中的参数)和重整化群流参数的形式。
    \item 在低能不动点处,物理参数应当和重整化群流参数无关,因此可以反过来将参数跑动写成重整化群流参数和物理参数的函数,从而求解出所有抵消项。
    用于确定抵消项的条件即为\concept{重整化条件}。
    如果理论可重整,此时所有其它可以实际观察到的物理量中的发散都会相互抵消,从而我们成功地将一些可以实际观察到的物理量写成了另一些可以实际观察到的物理量(即物理参数)的函数,即给出了实验预言。
\end{enumerate}
在最终的计算结果中裸参数都没有出现;这是正确的,因为实际上并没有什么能够真的“观察”到裸参数——无法确定实际测到的物理参数有多少来自裸参数,多少来自相互作用修正。
实际上,为了和发散抵消,裸参数一般都是反向发散的。

在实际的计算中,还有以下技巧:
\begin{itemize}
    \item 由于圈图数目非常多,通常我们会以需要计算的散射振幅的树图当成骨架,将圈图当成对骨架图中各个成分的修正。
    这样的好处是,做完全部修正的骨架图的各个成分通常足够给出物理参数了,如做完自能修正的传播子可以给出物理质量,做完顶角修正的amputated vertex diagram可以给出有效相互作用强度。
    \item 由于实际有意义的关联函数、散射振幅等均已经做过场强重整化,可以用场强重整化之后的场来做拉氏量中的基本自由度;这样与场强重整化有关的抵消项会自动出现在拉氏量中,并且做完重整化之后直接计算amputated diagram即可得到散射振幅,没有必要再显式计算$Z$。
    \item 抵消项可以被显式给出。如果不希望引入太多顶角,也可以首先形式地写出含有未知的裸参数的散射振幅的形式,然后用未知的裸参数去拼凑出物理参数。
    如果显式地使用抵消项,由于我们在重整化不动点附近工作,其实可以将抵消项设置为“裸参数偏离物理参数的多少”,而直接将物理参数放进拉氏量中。
\end{itemize}

表面上,任何一个理论都可以做这样的操作——对称性允许的拉氏量中的项是无限多的,我们可以引入任意多的抵消项来消除发散。
但是,如果需要引入无数多的抵消项,那么重整化操作就是无法完成的。

为了估计发散的程度,我们可以引入一些指标。\concept{原始发散图}指的是只要切断一根内线(即不计算这根内线的积分),就能够收敛的图。发散的图是用原始发散图组装起来的。
如果一个理论中的原始发散图的个数有限,

由于紫外发散来自动量积分有太多重,一张图$\Gamma$中的动量的幂次——所谓\concept{表观发散度}——为
\begin{equation}
    D(\Gamma) = \sum_i n_i d_i + 2 I_\text{B} + 3 I_\text{F} - 4(\sum_i n_i - 1),
\end{equation}
其中$n_i$指的是某一类型的顶角的个数,$d_i$是类型$i$的顶角中的动量幂次,$I_\text{B}$和$I_\text{F}$分别表示玻色子和费米子线的个数,因为费米子传播子的分母中只有一个$k$,做完四维动量积分之后动量幂次为3,而玻色子传播子的分母中有两个$k$,做完四维动量积分之后动量幂次为2。
最后一项是因为顶角会引入一个动量守恒条件;我们故意减去了$1$,因为费曼图最终的计算结果也肯定满足动量守恒条件,即有一个$\delta(\sum \vb*{k})$并没有被积分掉,而是留到了计算结果中。
现在我们进一步设$i$类型顶角中有$b_i$个玻色子线,$f_i$个费米子线,并设有$E_\text{F}$条费米子外线,$E_\text{B}$条玻色子外线,则
\[
    E_\text{F} + 2 I_\text{F} = \sum_i n_i f_i, \quad E_\text{B} + 2 I_\text{B} = \sum_i n_i b_i,
\]
因为一条内线连接两个顶角。这样就有
\begin{equation}
    D(\Gamma) = \sum_i n_i \left( d_i + b_i + \frac{3}{2} f_i - 4  \right) + 4 - E_\text{B} - \frac{3}{2} E_\text{F}.
\end{equation}

表观发散度实际上就是在做量纲分析,而且做的是朴素的工程量纲分析,因此是不尽然可靠的。
大体上说,如果表观发散度大于零,那么这张图发散,如果小于零,那么这张图收敛,如果等于零,那么这张图应该对数发散,但是这只是一个非常粗略,可能不准确的估计。
不过,\concept{Weinberg power counting theorem}保证了,当且仅当一张图及其子图的表观发散度都是小于零,它收敛。

为了保证尽可能多的图的表观发散度小于等于零,我们会要求
\begin{equation}
    d_i + b_i + \frac{3}{2} f_i - 4 \leq 0,
\end{equation}
因此比较安全的安排是,一个顶角最多有四条玻色子线、两条费米子线,否则有可能产生无穷多种发散的图,理论可能不能重整化。

\subsection{维数正规化}

注:经常用Wick转动来化简此处的积分,但是由于$p^0 > 0$时极点在下半平面而$p^0 < 0$时极点在上半平面,为了避免撞上奇点,Wick转动应该将积分路径顺时针旋转\SI{90}{\degree},从而可以设$l = \ii l^\text{E}$。
TODO:这和我们后面做的将整个理论做Wick转动时用的记号似乎不一样?

将内线动量从$4$维扩充为$n$维度,外线动量保持不变。

让维数变化时,没有必要让$\gamma$矩阵的维数发生变化,因此对$\gamma$矩阵的乘积的迹计算无需做任何调整,即自旋指标不需要做任何调整,而对坐标指标($\mu$这种)的迹计算(如$\gamma^\mu \gamma_\mu$)则需要调整。

\subsection{骨架图的修正}

重整化条件:
\begin{itemize}
    \item 在修正后的单粒子格林函数的极点处,有质量粒子的自能修正对$p^2$(玻色子)或者$\slashed{p}$(费米子)的一阶导数为零;无质量粒子的自能修正为零。
    这是为了确保没有场强重整化。
    动量远离单粒子格林函数极点时它们当然可以不是零;本应如此,否则圈图修正无法体现。
    \item 在修正后的单粒子格林函数的极点处,极点给出的质量(通过$p^2=m^2$解出)就是我们设定的物理质量;在显式引入自能修正时,有质量粒子的自能修正为零。
    这是为了确保质量的修正为零。
    \item 顶角函数和物理相互作用强度相同,这是为了确保顶角修正为零。
    具体什么是“物理相互作用强度”取决于探测方式,如量子电动力学中通常是使用静电学方法测定电磁相互作用的强度,于是我们要求顶角函数在光子动量为零时和静电学方法测得的电磁相互作用强度(其实就是元电荷)相同。
    其它时候顶角函数可以有偏离,以展现高阶过程的修正。
\end{itemize}

\section{单位制,指标记号和度规选取}

\subsection{度规的圣战}

时间在整个四维矢量中的位置,一些人取为$x^0$,一些人取为$x^4$。

\subsubsection{$-+++$和$+---$}

在以上所有的讨论中,我们都在使用度规$+---$。度规$-+++$在一些文献中也是常用的,并且在做Wick转动时更加方便(见下一节)。

概述:两种度规的协变矢量保持一致,点乘差一个负号。

\subsubsection{Wick转动}

为了免去洛伦兹度规的麻烦,一些人会做Wick转动,即令$\tau=\ii t$,这样就不需要区分逆变协变了,并且很多积分的性质会变得良好。
为了尽可能减少需要改动的地方,最方便的做法是在Wick转动后的理论(“欧氏空间度规理论”)和$-+++$度规之间切换,需要时再切换到$+---$度规。

如果单纯是做替换$\tau = \ii t$,那么没有太多可说,但实际上其它物理量也需要变化,而这些变化有很多自由发挥的空间。
例如,原本在相对论性量子场论中含有傅里叶变换的表达式在做完Wick转动之后是否需要修改为拉普拉斯变换?
格林函数的定义是否需要更动?这些都是需要指定的。

我们于是施加以下条件:
\begin{itemize}
    \item Wick转动后的理论,如果将时间的积分区域设为实数,应该给出原理论对应的有限温度场论。
    这就是说,对应关系$\tau \leftrightarrow \ii t$和$\ii \omega_n \leftrightarrow \omega$应当可以在Wick转动的过程中找到,虽然由于$t$和$\omega$区分逆变和协变,有待进一步澄清以上关系式中的$t$和$\omega$指的是什么。
    \item Wick转动前后标量尽可能不变。
    \item Wick转动将原本的傅里叶变换映射为$\ee^{\ii k_i x_i}$形式的欧氏空间傅里叶变换。
    \item 闵氏时空下,矢量的欧氏部分的协变分量和欧氏四维空间中对应的分量需要完全一样,从而我们无需在$(x^\text{M})^i$和$x^\text{E}_i$之间做任何区分。这里上标M和E分别表示闵氏时空和欧氏四维空间,下同。
\end{itemize}

我们来看一下这些条件意味着什么。首先,第二个和第三个条件意味着应有
\[
    (x^\text{M})^0 p^\text{M}_0 + (x^\text{M})^1 p^\text{M}_1 + (x^\text{M})^2 p^\text{M}_2 + (x^\text{M})^3 p^\text{M}_3 = x^\text{E}_0 p^\text{E}_0 + x^\text{E}_1 p^\text{E}_1 + x^\text{E}_2 p^\text{E}_2 + x^\text{E}_3 p^\text{E}_3,
\]
因为我们有
\[
    \ee^{\ii k^M \cdot x^M} = \ee^{\ii k^E \cdot x^E}.
\]
请注意在$-+++$度规下$p_i$和$p^i$并无差别,于是按照第四个条件,我们有
\[
    (x^\text{M})^1 p^\text{M}_1 + (x^\text{M})^2 p^\text{M}_2 + (x^\text{M})^3 p^\text{M}_3 = x^\text{E}_1 p^\text{E}_1 + x^\text{E}_2 p^\text{E}_2 + x^\text{E}_3 p^\text{E}_3,
\]
即
\[
    (x^\text{M})^0 p^\text{M}_0 = x^\text{E}_0 p^\text{E}_0.
\]
在相对论性量子场论中我们通常认为$t$就是$x^0$而$\omega$就是$p^0$,它和$p_0$正好差了一个负号,而显然我们应该指定$\tau=\ii t$为$x^\text{E}_0$,因此就有
\[
    p_0^\text{E} = \ii \omega,
\]
即应有
\begin{equation}
    p_0^\text{E} = - \omega_n = \ii \omega, \quad x^\text{E}_0 = \tau = \ii t.
\end{equation}
换句话说,四维欧氏空间中的$p^\text{E}$的时间分量和松原频率差了一个负号。
这其实是正确的,因为在凝聚态场论中有
\[
    \phi(\tau) \propto \sum_n \ee^{-\ii \omega_n \tau} \phi_n,
\]
加入动量之后就是
\[
    \phi(\tau, \vb*{x}) \propto \sum_n \int \dd[3]{\vb*{p}} \phi_n(\vb*{p}) \ee^{-\ii \omega_n \tau + \ii \vb*{p} \cdot \vb*{x}} , 
\]
考虑到$\omega_n \tau = \omega t$,做了反Wick转动之后这正好就是$-+++$度规,即使用$\omega_n$表示的傅里叶变换的$\ee$指数本身遵循$-+++$度规而不是$++++$度规,自然会导致$\omega_n$和$p^\text{E}_0$差一个负号。

任何使用爱因斯坦求和得到的洛伦兹标量在Wick转动后形式均不变,因为Wick转动相当于做了一次坐标变换,而满足上下同指标求和规则的量在坐标变换下形式不变。
由于Wick转动后是欧氏度规,我们不必再区分逆变和协变。
需要修改的主要是积分测度,即需要加上或减少一个因子$\ii$。
因此以上给出的关于坐标和频率的Wick转动足够让我们完成标量场论的配分函数的Wick转动。
对矢量场只需要对$A^0$做代换即可,即取
\begin{equation}
    A^\text{E}_0 = (A^\text{E})^0 = \ii (A^\text{M})^0.
\end{equation}

在完成配分函数的Wick转动之后还需要注意格林函数也需要做一些调整。
从闵可夫斯基时空的理论中得到关联函数的方式是做变分导数$\fdv*{Z}{(\ii J)}$,$J$是在以
\[
    \exp(\ii \int \dd[4]{x} J \phi)
\]
形式引入的。在Wick转动后,以上激励项变成了
\[
    \exp(\int \dd[4]{x} J \phi).
\]
在此过程中场$J$和$\phi$没有发生任何变化,所以用四维欧氏时空计算闵氏时空中的格林函数,只需要先计算$\fdv*{Z^\text{E}}{(\ii J)}$然后做反Wick转动即可。
然而,需要注意一件事:四维欧氏时空中本身也定义有格林函数,即所谓虚时间格林函数,而当激励项以
\[
    \exp(\int \dd[4]{x} J \phi)
\]
形式给出时,计算虚时间格林函数使用的泛函导数是$\fdv*{Z^\text{E}}{J}$,没有$\ii$。(这和松原格林函数又差了一个负号)
在四维欧氏时空中计算虚时间格林函数,使用Wick定理,画费曼图等使用的都是$\fdv*{Z^\text{E}}{J}$方法求出的格林函数,但是最后切换回$-+++$度规的闵氏时空时需要把$\ii$加回去,格林函数中涉及几个场加几个。
例如,对二体格林函数,四维欧氏时空中的虚时间格林函数计算出来之后,需要做反Wick转动(在此过程中由于$\tau = \ii t$,闵氏时空格林函数的分子上会多出$-\ii$),然后加上一个负号(由于泛函导数而导致的$\ii$有两个,因为有两个场,而$\ii^2=-1$),才能得到$-+++$度规的闵氏时空中的格林函数。

对旋量场,事情略微复杂一些,因为$\gamma$矩阵也需要做同样的变换——当然其实可以不做任何变换,但是这样很多公式会看起来很奇怪。
考虑配分函数:
\[
    \begin{aligned}
        Z &= \int \fd{\psi} \int \fd{\bar{\psi}} \exp(\ii \int \dd[4]{x} \bar{\psi} (\ii \gamma^\mu \partial_\mu - m) \psi) \\
        &= \int \fd{\psi} \int \fd{\bar{\psi}} \exp(\int \dd{\tau} \int \dd[3]{\vb*{x}} \bar{\psi} (\ii \gamma^\mu \partial_\mu - m) \psi) \\
        &= \int \fd{\psi} \int \fd{\bar{\psi}} \exp(\int \dd{\tau} \int \dd[3]{\vb*{x}} \bar{\psi} (- \gamma^0 \pdv{\tau} + \ii \gamma^i \partial_i - m) \psi).
    \end{aligned}
\]
如果我们做变换
\begin{equation}
    (\gamma^\text{M})^0 = (\gamma^\text{E})^0, \quad - \ii (\gamma^\text{M})^i = (\gamma^\text{E})^i,
\end{equation}
并且,由于到了欧氏空间中,不再区分上下标,就得到
\begin{equation}
    Z = \int \fd{\psi} \int \fd{\bar{\psi}} \exp(- \int \dd[4]{x^\text{E}} \bar{\psi} (\gamma^\text{E}_\mu \partial_\mu^\text{E} + m) \psi)
\end{equation}
这就是旋量场的Wick转动。相应的$\gamma$矩阵的代数需要做调整,但是旋量本身并不需要做调整(除了用$-\ii \tau$代替$t$以外)。
$\gamma$矩阵的变动意味着,$\gamma^\mu a_\mu$形式的量即$\slashed{a}$虽然看起来像是点乘,但是在Wick转动下会发生变化:$a_0^\text{E}$相比于$(a^\text{M})^0$多出来了一个$\ii$,而$\gamma_i^\text{E}$相比于$(\gamma^\text{M})^i$多出来了一个$-\ii$,再加上度规从$-+++$变成了$++++$,我们就得到
\begin{equation}
    \gamma^\text{E}_\mu a^\text{E}_\mu = \slashed{a}^\text{E} = - \ii \slashed{a}^\text{M} = -\ii {\gamma^\text{M}}^\mu a^\text{M}_\mu.
\end{equation}

以上推导都是针对相对论情况下的。非相对论情况下的理论大多都是一个相对论情况下的理论的低能有效理论,因此所有的矢量分量的Wick转动规则仍然适用。
例如,电磁场和非相对论性电子的耦合给出的拉氏量不具有洛伦兹协变性,但是电磁分量

\part{规范场论}

在单粒子量子力学和前面作为例子计算过的一些场论中,哈密顿量或是拉格朗日量中出现了一些动力学变量,我们从这些动力学变量中挑选出一些来,它们不多也不少地可以标记希尔伯特空间的一组基矢量。
略微推广一下,我们其实可以研究这样的理论:从其中的彼此对易的动力学变量中挑选出一组,它们标记的基矢量张成的空间要\emph{大于}我们要研究的希尔伯特空间。
这允许在理论中引入更加丰富的行为,而与此同时保持物理的自由度数目正确。
一种获得这样的理论的构造方式是通过所谓的“规范对称性”。
物理学中的对称性通常包括时空对称性(即将物理事件的时空坐标做一个变换,一般来说,是洛伦兹变换)和内部对称性(即某个参数空间中的变换,通常是各点上场的变换)。
\concept{规范对称性}指的则是变换参数依赖局域时空坐标的对称性,即与定域的变换相关的对称性;通常我们在\emph{每一个}空间点都放置某个群$G$的副本,即在每个空间点都放置一个$G$的群元$g(x)$,我们要求理论在任何一个$g(x)$场的变换(即所谓\concept{规范变换})下都保持不变,或者说,它具有\emph{局域的}$G$对称性。
选取规范变换作用在希尔伯特空间上的轨道的一个截面就称为\concept{选取一个规范}:选取了一个规范后的一个波函数做时间演化绝对不会演化到另一个规范下的某个波函数中。
这暗示我们,规范对称性实际上并不是真正的对称性:到头来,我们只不过是在(不必要地)让希尔伯特空间扩张了若干倍的代价下,让理论可以具有的形式更加多样而已:一个规范理论总是可以使用一个非规范理论重写,但是如果前者是局域的,那么后者一般不是局域的。
一个典型的例子见\soliddoc中的第\ref{solid-sec:z2-dual-ising-model}节,其中我们发现,虽然$\mathbb{Z}_2$规范理论本身的规范冗余自由度可以完全去掉,可是这样的代价是,$\mathbb{Z}_2$规范场和其它东西耦合的哈密顿量就不是局域的了。
这和普通的场被引入的动机类似:不引入场,就会出现粒子间的超距作用,则必须适当地说明粒子间的哪种超距作用是允许的。
实际上我们总是在已经选取了一个规范以后的希尔伯特空间中工作:如果算符$U$是一个真正的对称性操作,那么$\ket{\psi}$和$U \ket{\psi}$是\emph{不同的}波函数,而如果$U$是一个规范对称性,那么如果$\ket{\psi}$在希尔伯特空间中,$U \ket{\psi}$就\emph{不在}希尔伯特空间中;或者,我们可以要求$\ket{\psi}$和$U \ket{\psi}$认同。

这里其实有一个含混之处:如果一个理论具有规范结构,那么它肯定具有表面上的局域对称性;但是如果一个理论具有局域对称性,它是否只能是一个规范理论?
Elitzur定理\cite{Elitzur_1975}表明,一个理论的局域对称性无法被破缺:一个普通的对称性总是可以通过改变拉氏量而被自发破缺,但是局域对称性没有这种现象。
% TODO

显然规范理论的方法是大有好处的:它能够构造出原本我们想象不到的理论,还能够让我们\emph{批量}构造这种理论。
我们可以强迫一个自由理论具有规范对称性,或者说给它赋予一个规范结构,然后引入适当的规范场。这样的做法是非常套路性的,以至于我们可以发明出若干套方案,对每套方案,给定一个规范群就能够\emph{自动产生}一个规范场论。

此外,如果我们总是通过最小耦合将物质和规范玻色子耦合在一起,那么规范玻色场的运动方程的外源总是物质场的守恒流,只要物质场的运动方程只含有一阶导数(否则会出现玻色子与规范场耦合的那种$\phi^2 A^2$的顶角;但是,我们的世界中物质场总是费米的)。
这和我们在电动力学(实际上也包括经典力学和量子力学:“粒子在外源中运动”这样的问题)中的直观——物质场的“密度”驱动了规范玻色场,规范玻色场反作用于物质场的密度——是非常一致的。
在二次量子化的语言中,密度是费米场的二次型,即我们可以将费米型的物质场看成经典电动力学中的物质密度场开平方根得到的,做了这个开平方操作后费曼图能够直观地反映“物质粒子吸收或发射一个规范玻色子之后进入另一个模式”:
\[
    \begin{tikzpicture}[x=0.75pt,y=0.75pt,yscale=-1,xscale=1]
        %uncomment if require: \path (0,300); %set diagram left start at 0, and has height of 300
        
        %Straight Lines [id:da1633599723274024] 
        \draw    (182,200.25) -- (212,143.92) ;
        \draw [shift={(197,172.08)}, rotate = 118.04] [fill={rgb, 255:red, 0; green, 0; blue, 0 }  ][line width=0.08]  [draw opacity=0] (12,-3) -- (0,0) -- (12,3) -- cycle    ;
        %Straight Lines [id:da16154280042763247] 
        \draw    (187,190.67) -- (177,209.83) ;
        
        %Straight Lines [id:da6047779860192122] 
        \draw    (207,134.25) -- (177,77.92) ;
        \draw [shift={(192,106.08)}, rotate = 61.96] [fill={rgb, 255:red, 0; green, 0; blue, 0 }  ][line width=0.08]  [draw opacity=0] (12,-3) -- (0,0) -- (12,3) -- cycle    ;
        %Straight Lines [id:da8492954390028673] 
        \draw    (202,124.67) -- (212,143.83) ;
        
        %Straight Lines [id:da6189148266726483] 
        \draw    (212,144) .. controls (213.67,142.33) and (215.33,142.33) .. (217,144) .. controls (218.67,145.67) and (220.33,145.67) .. (222,144) .. controls (223.67,142.33) and (225.33,142.33) .. (227,144) .. controls (228.67,145.67) and (230.33,145.67) .. (232,144) .. controls (233.67,142.33) and (235.33,142.33) .. (237,144) .. controls (238.67,145.67) and (240.33,145.67) .. (242,144) .. controls (243.67,142.33) and (245.33,142.33) .. (247,144) .. controls (248.67,145.67) and (250.33,145.67) .. (252,144) .. controls (253.67,142.33) and (255.33,142.33) .. (257,144) .. controls (258.67,145.67) and (260.33,145.67) .. (262,144) .. controls (263.67,142.33) and (265.33,142.33) .. (267,144) .. controls (268.67,145.67) and (270.33,145.67) .. (272,144) .. controls (273.67,142.33) and (275.33,142.33) .. (277,144) .. controls (278.67,145.67) and (280.33,145.67) .. (282,144) .. controls (283.67,142.33) and (285.33,142.33) .. (287,144) -- (290,144) -- (290,144) ;
        \end{tikzpicture}      
\]
只有规范玻色子和密度场的理论是没有这种图的;这正是经典电动力学中的情况,在那里我们只能讨论电磁场如何激发出密度模式,密度模式再如何激发出电磁场。

出于某些原因,大自然为基本粒子赋予的物理是非常节俭的。我们将看到,目前已知的除了引力以外的\emph{所有}相互作用都是通过\emph{一套}这样的方案——杨-米尔斯理论——产生的。
这一事实——即所谓\concept{规范原理}——是量子场论历史上所谓“改变人心的转换”,它被系统应用之前,各个场的相互作用基本上只能唯象确定,它被系统应用之后,只需要写出规范群(即局域对称性的对称群)就能够确定相互作用。

本文将首先介绍电动力学,分析其性质,然后通过考虑其自然推广而得到杨-米尔斯理论。

\documentclass[hyperref, UTF8, a4paper]{ctexart}

\usepackage{geometry}
\usepackage{titling}
\usepackage{titlesec}
\usepackage{paralist}
\usepackage{footnote}
\usepackage{enumerate}
\usepackage{amsmath, amssymb, amsthm}
\usepackage{simplewick}
\usepackage{cite}
\usepackage{graphicx}
\usepackage{subfigure}
\usepackage{physics}
\usepackage{centernot}
\usepackage{tikz}
\usepackage{tikz-feynhand}
\usepackage[colorlinks, linkcolor=black, anchorcolor=black, citecolor=black]{hyperref}
\usepackage{prettyref}

\geometry{left=3.18cm,right=3.18cm,top=2.54cm,bottom=2.54cm}
\titlespacing{\paragraph}{0pt}{1pt}{10pt}[20pt]
\setlength{\droptitle}{-5em}
\preauthor{\vspace{-10pt}\begin{center}}
\postauthor{\par\end{center}}

\DeclareMathOperator{\timeorder}{T}
\DeclareMathOperator{\diag}{diag}
\newcommand*{\ii}{\mathrm{i}}
\newcommand*{\ee}{\mathrm{e}}
\newcommand*{\const}{\mathrm{const}}
\newcommand*{\comment}{\paragraph{注记}}
\newcommand{\fsl}[1]{{\centernot{#1}}}
\newcommand*{\reals}{\mathbb{R}}
\newcommand*{\complexes}{\mathbb{C}}

\newrefformat{sec}{第\ref{#1}节}
\newrefformat{note}{注\ref{#1}}
\renewcommand{\autoref}{\prettyref}

\newenvironment{bigcase}{\left\{\quad\begin{aligned}}{\end{aligned}\right.}

\newcommand{\concept}[1]{\underline{\textbf{#1}}}
\renewcommand{\emph}{\textbf}

\allowdisplaybreaks[4]

\title{量子电动力学的具体计算}
\author{吴晋渊}

\begin{document}

\maketitle

\section{非相对论极限}

\subsection{电子,光子和电场}

光子无法做非相对论近似,因为无论如何,麦克斯韦方程都应该成立,而这个方程就是洛伦兹协变的。
需要做非相对论近似的只有电子。在非相对论近似下,一切有质量的场都退化为薛定谔场,电子也不例外。
因此在QED的非相对论极限下,基本的粒子包括电子和光子,光子无任何变化,电子场则是动能为$\vb*{k}^2 / 2m$,不再满足相对论协变性,由动量和自旋标记的场。

\subsection{树图阶的相互作用}

在非相对论情况下,光子的能量不足以激发出电子-正电子对,真空极化并不重要。
因此,光子虽然是无能隙的,一些时候仍然可以积掉光子而得到电子-电子等效相互作用,这个过程中只需要考虑纯光子传播子即可因为其它场可以看成背景场。
积掉光子还意味着电子自己需要做自能修正,让质量什么的发生变化,即由于电子和光子的相互作用,电子带上了“电磁质量”。
这是在经典电动力学中也已经知道的一个现象,但是在经典电动力学中不足以处理自能导致的发散。

\subsubsection{库伦相互作用}

不满足$\mu=1, 2$的光子可以出现在中间过程中,但是不会出现在外线中。
既然真空极化不重要,本节直接积掉这些光子,这只会导致树图(因为QED中没有光子-光子相互作用顶角)。
我们将得到两张图:
\[
    \begin{tikzpicture}
        \begin{feynhand}
            \vertex (a) at (-1.5, 0.8);
            \vertex (b) at (-1.5, -0.8);
            \vertex (c) at (-1, 0);
            \vertex (d) at (0, 0);
            \vertex (e) at (0.5, 0.8);
            \vertex (f) at (0.5, -0.8);

            \propag[anti fermion] (a) to (c);
            \propag[fermion] (b) to (c);
            \propag[photon] (c) to (d);
            \propag[fermion] (d) to (e);
            \propag[anti fermion] (d) to (f);
        \end{feynhand}
    \end{tikzpicture}, \quad \quad 
    \begin{tikzpicture}
        \begin{feynhand}
            \vertex (a) at (-1.5, 0.8);
            \vertex (b) at (-1.5, -0.8);
            \vertex (c) at (-1, 0);
            \vertex (d) at (0, 0);
            \vertex (e) at (0.5, 0.8);
            \vertex (f) at (0.5, -0.8);

            \propag[anti fermion] (e) to (c);
            \propag[fermion] (b) to (c);
            \propag[photon] (c) to (d);
            \propag[fermion] (d) to (a);
            \propag[anti fermion] (d) to (f);
        \end{feynhand}
    \end{tikzpicture}
\]
如果参与散射的两个粒子是可以分辨的(即除了动量和自旋以外还有别的标签可以区分它们),那么第二张图和第一张图不在一个相互作用通道中。
低能过程由于动量低,相应的特征尺度很大,即粒子不会离得很近,这种情况下粒子的“位置”近似起到了区分两个粒子的标签的作用。这意味着第二张图可以忽略。

于是我们计算第一张图,它给出
\[
    \begin{gathered}
        \begin{tikzpicture}
            \begin{feynhand}
                \vertex (a) at (-1.5, 0.8);
                \vertex (b) at (-1.5, -0.8);
                \vertex (c) at (-1, 0);
                \vertex (d) at (0, 0);
                \vertex (e) at (0.5, 0.8);
                \vertex (f) at (0.5, -0.8);
    
                \propag[fermion] (c) to [edge label={$p'$}] (a);
                \propag[fermion] (b) to [edge label={$p$}] (c);
                \propag[photon] (c) to (d);
                \propag[fermion] (d) to [edge label={$k'$}] (e);
                \propag[fermion] (f) to [edge label={$k$}] (d);
            \end{feynhand}
        \end{tikzpicture}
    \end{gathered} = (-\ii e)^2 \bar{u}(p') \gamma^\mu u(p) \frac{-\ii \eta_{\mu \nu}}{(p' - p)^2 + \ii 0^+} \bar{u}(k') \gamma^\nu u(k).
\]
我们考虑$p, p', k, k'$都几乎是零的情况,并且只对$\mu=\nu=0, 3$的情况求和——其实我们会看到,$\mu = \nu = 1, 2$两种情况并不会有贡献。
此时计算会发现
\[
    \bar{u}(p') \gamma^0 u(p) = u^\dagger(p') u(p) \approx m \pmqty{\xi^\dagger & \xi^\dagger} \pmqty{\xi \\ \xi} = 2m \sigma^0,
\]
而
\[
    \bar{u}(p') \gamma^i u(p) = u^\dagger(p') \pmqty{\dmat{- \sigma^i, \sigma^i}} u(p) \approx m \pmqty{\xi^\dagger & \xi^\dagger} \pmqty{\dmat{- \sigma^i, \sigma^i}} \pmqty{\xi \\ \xi} = 0.
\]
于是我们就得到
\[
    \begin{aligned}
        \begin{gathered}
            \begin{tikzpicture}
                \begin{feynhand}
                    \vertex (a) at (-1.5, 0.8);
                    \vertex (b) at (-1.5, -0.8);
                    \vertex (c) at (-1, 0);
                    \vertex (d) at (0, 0);
                    \vertex (e) at (0.5, 0.8);
                    \vertex (f) at (0.5, -0.8);
        
                    \propag[fermion] (c) to [edge label={$p'$}] (a);
                    \propag[fermion] (b) to [edge label={$p$}] (c);
                    \propag[photon] (c) to (d);
                    \propag[fermion] (d) to [edge label={$k'$}] (e);
                    \propag[fermion] (f) to [edge label={$k$}] (d);
                \end{feynhand}
            \end{tikzpicture}
        \end{gathered} &= (-\ii e)^2 \bar{u}(p') \gamma^\mu u(p) \frac{-\ii \eta_{\mu \nu}}{(p' - p)^2 + \ii 0^+} \bar{u}(k') \gamma^\nu u(k) \\
        &= \ii e^2 2 m (\sigma^0)_p \frac{1}{(p' - p)^2 + \ii 0^+} 2m (\sigma^0)_k \\
        &= - \frac{\ii e^2 (2m)^2 \sigma^0}{\abs*{\vb*{p}' - \vb*{p}}^2 - \ii 0^+}.
    \end{aligned}
\]
上式是$\{\ket*{p}\}$表象下的相互作用哈密顿量矩阵元;下标$p$和$k$用于区分作用在不同单粒子态上的矩阵。
矩阵$\sigma^0$给出了自旋的变化情况,可以看到以上相互作用通道不挑选入射自旋,也不改变入射自旋。
我们要做非相对论近似,所以要转换到$\{\ket*{\vb*{p}}\}$表象下,由于有四条外线,要除以因子$(\sqrt{2m})^{4}$。
于是非相对论极限下,我们获得相互作用顶角
\begin{equation}
    \begin{gathered}
        \begin{tikzpicture}
            \begin{feynhand}
                \vertex (a) at (-1.5, 0.8);
                \vertex (b) at (-1.5, -0.8);
                \vertex (c) at (-1, 0);
                \vertex (d) at (0, 0);
                \vertex (e) at (0.5, 0.8);
                \vertex (f) at (0.5, -0.8);
    
                \propag[fermion] (c) to [edge label={$p', \alpha$}] (a);
                \propag[fermion] (b) to [edge label={$p, \alpha$}] (c);
                \propag[photon] (c) to (d);
                \propag[fermion] (d) to [edge label={$k', \beta$}] (e);
                \propag[fermion] (f) to [edge label={$k, \beta$}] (d);
            \end{feynhand}
        \end{tikzpicture}
    \end{gathered} = -\ii \frac{e^2}{\abs*{\vb*{p} - \vb*{p}'}^2} (2\pi)^4 \delta^4(k' + p' - p - k).
\end{equation}
这个相互作用顶角的形式实际上正是动量空间中的库伦定律。
为了更加清晰地看出库伦定律,我们将上式切换回实空间,做傅里叶变换
\[
    \begin{aligned}
        \int \frac{\dd[4]{p'}}{(2\pi)^4} \ee^{\ii p' \cdot x_1} \int \frac{\dd[4]{k'}}{(2\pi)^4} \ee^{\ii k' \cdot x_2} \int \frac{\dd[4]{p}}{(2\pi)^4} \ee^{- \ii p \cdot x_3} \int \frac{\dd[4]{p}}{(2\pi)^4} \ee^{- \ii p \cdot x_4},
    \end{aligned}
\]
计算发现
\begin{equation}
    \begin{aligned}
        \begin{gathered}
            \begin{tikzpicture}
                \begin{feynhand}
                    \vertex (a) at (-1.5, 0.8) {$x_1, \alpha$};
                    \vertex (b) at (-1.5, -0.8) {$x_3, \alpha$};
                    \vertex (c) at (-1, 0);
                    \vertex (d) at (0, 0);
                    \vertex (e) at (0.5, 0.8) {$x_2, \beta$};
                    \vertex (f) at (0.5, -0.8) {$x_4, \beta$};
        
                    \propag[fermion] (c) to (a);
                    \propag[fermion] (b) to (c);
                    \propag[photon] (c) to (d);
                    \propag[fermion] (d) to (e);
                    \propag[fermion] (f) to (d);
                \end{feynhand}
            \end{tikzpicture}
        \end{gathered} &= -\ii e^2 \delta(t_4 - t_1) \delta^4(x_1 - x_3) \delta^4(x_2 - x_4) \int \frac{\dd[3]{\vb*{q}}}{(2\pi)^3} \frac{\ee^{-\ii \vb*{q} \cdot (\vb*{x}_4 - \vb*{x}_1)}}{\abs*{\vb*{q}}^2 - \ii 0^+} \\
        &= -\ii \delta(t_4 - t_1) \delta^4(x_1 - x_3) \delta^4(x_2 - x_4) \frac{e^2}{4\pi \abs*{\vb*{x}_4 - \vb*{x}_1}}.
    \end{aligned}
    \label{eq:coulomb-interaction}
\end{equation}
因此我们的确得到了库伦相互作用。在计算时有一个细节:计算\eqref{eq:coulomb-interaction}的第一个等号右边的积分时,我们有
\[
    \begin{aligned}
        \int \frac{\dd[3]{\vb*{q}}}{(2\pi)^3} \frac{\ee^{-\ii \vb*{q} \cdot (\vb*{x}_4 - \vb*{x}_1)}}{\abs*{\vb*{q}}^2 - \ii 0^+} &= \frac{2\pi}{(2\pi)^3} \int_0^{\pi} \sin \theta \dd{\theta} \int_0^\infty q^2 \dd{q} \frac{\ee^{- \ii q \abs*{\vb*{x}_4 - \vb*{x}_1} \cos \theta}}{\abs*{\vb*{q}}^2 - \ii 0^+} \\
        &= \frac{1}{4\pi^2} \int_0^\infty \frac{q^2}{q^2 - \ii 0^+} \dd{q} \frac{\ee^{- \ii q \abs*{\vb*{x}_4 - \vb*{x}_1}} - \ee^{\ii q \abs*{\vb*{x}_4 - \vb*{x}_1}}}{- \ii q \abs*{\vb*{x}_4 - \vb*{x}_1}} \\
        &= \frac{1}{4\pi^2 \ii} \int_{-\infty}^\infty \frac{q^2}{q^2 - \ii 0^+} \dd{q} \frac{\ee^{\ii q \abs*{\vb*{x}_4 - \vb*{x}_1}}}{q \abs*{\vb*{x}_4 - \vb*{x}_1}}.
    \end{aligned}
\]
如果我们将因子$q^2/(q^2 - \ii 0^+)$直接当成$1$,上式就没有确定的值了,因为极点直接出现在了积分路径上。
不过,我们有
\[
    \frac{q^2}{q^2 - \ii \epsilon} = \frac{1}{1 - \frac{\ii \epsilon}{q^2}} \to 0 \text{as $q \to 0$},
\]
因此我们应该取积分主值,即取
\[
    \int \frac{\dd[3]{\vb*{q}}}{(2\pi)^3} \frac{\ee^{-\ii \vb*{q} \cdot (\vb*{x}_4 - \vb*{x}_1)}}{\abs*{\vb*{q}}^2 - \ii 0^+} = \frac{1}{4\pi^2 \ii} \text{P} \int_{-\infty}^\infty \dd{q} \frac{\ee^{\ii q \abs*{\vb*{x}_4 - \vb*{x}_1}}}{q \abs*{\vb*{x}_4 - \vb*{x}_1}} = \frac{1}{4\pi^2 \ii} \frac{\pi \ii}{\abs*{\vb*{x}_4 - \vb*{x}_1}} = \frac{1}{4\pi \abs*{\vb*{x}_4 - \vb*{x}_1}}. 
\]

单粒子量子力学中的散射理论相当于梯形图近似。

\subsubsection{磁矩}



\section{辐射修正}

\end{document}

\documentclass[hyperref, UTF8, a4paper]{ctexart}

\usepackage{geometry}
\usepackage{titling}
\usepackage{titlesec}
\usepackage{paralist}
\usepackage{footnote}
\usepackage{enumerate}
\usepackage{amsmath, amssymb, amsthm}
\usepackage{mathtools}
\usepackage{simplewick}
\usepackage{cite}
\usepackage{graphicx}
\usepackage{subfigure}
\usepackage{physics}
\usepackage{tikz-feynhand}
\usepackage{centernot}
\usepackage{slashed}
\usepackage{tikz}
\usepackage[colorlinks, linkcolor=black, anchorcolor=black, citecolor=black]{hyperref}
\usepackage{prettyref}

\geometry{left=3.18cm,right=3.18cm,top=2.54cm,bottom=2.54cm}
\titlespacing{\paragraph}{0pt}{1pt}{10pt}[20pt]
\setlength{\droptitle}{-5em}
\preauthor{\vspace{-10pt}\begin{center}}
\postauthor{\par\end{center}}

\DeclareMathOperator{\timeorder}{T}
\DeclareMathOperator{\diag}{diag}
\newcommand*{\ii}{\mathrm{i}}
\newcommand*{\ee}{\mathrm{e}}
\newcommand*{\const}{\mathrm{const}}
\newcommand*{\comment}{\paragraph{注记}}
\newcommand{\fsl}[1]{{\centernot{#1}}}
\newcommand*{\reals}{\mathbb{R}}
\newcommand*{\complexes}{\mathbb{C}}
\newcommand*{\fd}[1]{{\mathcal{D} #1}}

\newcommand{\normord}[1]{\vcentcolon\mathrel{#1}\vcentcolon}
\providecommand{\vcentcolon}{\mathrel{\mathop{:}}}

\newrefformat{sec}{第\ref{#1}节}
\newrefformat{note}{注\ref{#1}}
\renewcommand{\autoref}{\prettyref}

\newenvironment{bigcase}{\left\{\quad\begin{aligned}}{\end{aligned}\right.}

\newcommand{\concept}[1]{\underline{\textbf{#1}}}
\renewcommand{\emph}{\textbf}

\tikzfeynhandset{
    every boldfermion@@/.style={
    /tikz/draw=none,
    /tikz/decoration={name=none},
    /tikz/postaction={
            /tikz/draw,
            /tikz/double,
            /tikz/line width = \feynhandlinesize,
            /tikzfeynhand/with arrow=0.5,
        },
    },
    every boldfermion/.style={/tikzfeynhand/every boldfermion@@/.append style={#1}},
    boldfermion/.style={
    /tikzfeynhand/every boldfermion@@,
    }
}

\allowdisplaybreaks[4]

\title{规范场论}
\author{吴晋渊}

\begin{document}

\maketitle

物理学中的对称性通常包括时空对称性(即将物理事件的时空坐标做一个变换,一般来说,是洛伦兹变换)和内部对称性(即某个参数空间中的变换,通常是各点上场的变换)。
\concept{规范对称性}指的则是变换参数依赖场和物理量的局域时空坐标的对称性,即与定域的变换相关的对称性。
实际上,规范对称性的要求足够确定系统中各个场的相互作用方式!这一事实——即所谓\concept{规范原理}——是量子场论历史上所谓“改变人心的转换”,它被系统应用之前,各个场的相互作用基本上只能唯象确定,它被系统应用之后,只需要写出规范群(即局域对称性的对称群)就能够确定相互作用。

本文将首先介绍电动力学,分析其性质,然后通过考虑其自然推广而得到杨-米尔斯理论。

我们将经常用到李代数。可以采用下面的约定: % TODO
\begin{equation}
    [T^a, T^b] = \ii f^{abc} T^c.
\end{equation}
\begin{equation}
    \trace{(T^a T^b)} = \frac{1}{2} \delta^{ab}, \quad \trace{T^a} = 0.
\end{equation}

\section{电动力学}

\subsection{规范场和狄拉克旋量场的最小耦合}

在相对论性量子场论中——也即,在以闵可夫斯基时空为底流形的量子场论中,我们尝试将一个自旋$1/2$的狄拉克旋量场和一个无质量矢量场耦合起来。
旋量场的拉氏量为
\begin{equation}
    \mathcal{L}_\text{spin} = - m \bar{\psi} \psi + \ii \bar{\psi} \gamma_\mu \partial^\mu \psi,
    \label{eq:spin-lagrangian}
\end{equation}
而矢量场的拉氏量为
\begin{equation}
    \mathcal{L}_\text{vec} = - \frac{1}{2} (\partial^\mu A^\nu \partial_\mu A_\nu - \partial^\mu A^\nu \partial_\nu A_\mu).
    \label{eq:vec-lagrangian}
\end{equation}
很容易看出\eqref{eq:spin-lagrangian}具有全局$U(1)$对称性:它在变换
\[
    \psi \longrightarrow \psi' = \psi \ee^{\ii \alpha}
\]
下保持不变。同样,\eqref{eq:vec-lagrangian}具有场的全局平移不变性(自由无质量矢量场的规范对称性),它在变换
\[
    A^\mu \longrightarrow A'^\mu = A^\mu + a^\mu
\]
下保持不变。这两个对称性都是全局的:如果$\alpha$或$a^\mu$依赖于坐标,由于导数的链式法则,会多出来一些项。
具体来说,我们有
\begin{equation}
    \mathcal{L}_\text{spin} \longrightarrow \mathcal{L}_\text{spin}' = \mathcal{L}_\text{spin} - \bar{\psi} \gamma_\mu \psi \partial^\mu \alpha.
    \label{eq:psi-change}
\end{equation}
对于矢量场,在$a^\mu$的形式任意的情况下,$\mathcal{L}_\text{vec}$的变换无规律可循,但是如果我们用某个标量的梯度$\partial^\mu a$代替$a^\mu$,那么有
\[
    A^\mu \longrightarrow A'^\mu = A^\mu + \partial^\mu a, \quad
    \mathcal{L}_\text{vec} \longrightarrow \mathcal{L}_\text{vec}' = \mathcal{L}_\text{vec}.
\]
也就是说,矢量场的场的平移对称性实际上可以稍加推广而仍然成立。

$\psi$的变换的相位因子是一个标量;$A^\mu$的场的平移量也是一个标量的梯度。很容易想到的尝试是,我们是否可以将两个场耦合起来,并要求整个系统在局域变换($e$是常数而$a(\vb*{x})$依赖于坐标)
\begin{equation}
    \psi \longrightarrow \psi' = \psi \ee^{\ii e a}, \quad A^\mu \longrightarrow A'^\mu = A^\mu + \partial^\mu a
    \label{eq:gauge-transformation}
\end{equation}
下保持不变?%
\footnote{早期的物理学家会认为,一个变换应该是物理上可行的,因此它不应该是全局的,而\emph{只能}是局域的(例如我们可以让$a$在很小的范围内才不为零)。
但是实际上这种观点是错误的——使得我们想要从头写下一个拉氏量的原因实际上是我们希望从一个非常简洁的源头推导出麦克斯韦方程,但是后面会看到,在经典情况下描述了一切电磁现象的麦克斯韦方程本身在$U(1)$规范变换下不变,这意味着\eqref{eq:gauge-transformation}展示的对称性实际上是一种冗余,即系统中存在非物理、可以略去的自由度。
这些自由度不参与和实际观测值有关的任何相互作用,作用在它们上面的变换完全没有必要是局域的。
我们要求系统的动力学在局域$U(1)$变换下不变,归根到底还是满足实验观测结论的需要。
}%
自由矢量场部分肯定是不变的,那么就要适当设计相互作用项的形式,把$\psi$做局域$U(1)$变换之后拉氏量多出来的一项吸收掉。
当然,如果相互作用项是$- e A^\mu \bar{\psi} \gamma_\mu \psi$,那就正好,因为
\[
    - e A^\mu \bar{\psi} \gamma_\mu \psi - \bar{\psi} \psi \gamma_\mu \partial^\mu (e a) = - e A'^\mu \bar{\psi'} \gamma_\mu \psi'.
\]
于是我们得出结论:拉氏量
\begin{equation}
    \mathcal{L} = 
    \underbrace{- m \bar{\psi} \psi + \ii \bar{\psi} \gamma_\mu \partial^\mu \psi }_{\mathcal{L}_\text{spin}}
    \underbrace{- \frac{1}{2} (\partial^\mu A^\nu \partial_\mu A_\nu - \partial^\mu A^\nu \partial_\nu A_\mu)}_{\mathcal{L}_\text{vec}}
    \underbrace{- e A^\mu \bar{\psi} \gamma_\mu \psi}_\text{interaction}
    \label{eq:qed-lagrangian}
\end{equation}
具有局域$U(1)$不变性。推导出\eqref{eq:qed-lagrangian}的方法就是\concept{最小耦合}。%
\footnote{需要注意的是最小耦合实际上并不是唯一的能够让理论满足局域$U(1)$对称性的方案。理论中有哪些规范场、相互作用的形式如何,归根到底都需要实验上的提示。
例如,电磁理论中最小耦合适用是因为它能够导出麦克斯韦方程,而麦克斯韦方程是已经验证了的在经典情况下正确的电磁定律。}%

于是,我们得出结论:一个自旋$1/2$狄拉克旋量场和一个无质量矢量场耦合,并要求理论具有\emph{局域}$U(1)$对称性,那么就会得到
我们将会看到,这个理论实际上就是电动力学。

还可以引入一些记号来简化\eqref{eq:qed-lagrangian}。首先引入反对称张量\concept{电磁张量}
\begin{equation}
    F^{\mu \nu} = \partial^\mu A^\nu - \partial^\nu A^\mu,
\end{equation}
则自由矢量场拉氏量为
\[
    \mathcal{L}_\text{vec} = - \frac{1}{4} F_{\mu \nu} F^{\mu \nu}.
\]
另一方面,相互作用项和含有旋量场的导数的项形式非常接近,因此可以定义\concept{协变导数}
\begin{equation}
    \ii D^\mu = \ii \partial^\mu - e A^\mu,
\end{equation}
最后将\eqref{eq:qed-lagrangian}写成
\begin{equation}
    \begin{aligned}
        \mathcal{L} &= \bar{\psi} (\ii \gamma^\mu D_\mu - m) \psi - \frac{1}{4} F_{\mu \nu} F^{\mu \nu} \\
        &= \bar{\psi} (\ii \slashed{D} - m) \psi - \frac{1}{4} F_{\mu \nu} F^{\mu \nu}. 
    \end{aligned}
    \label{eq:short-qed-lagrangian}
\end{equation}
这里我们用斜杠记号表示$\gamma_\mu A^\mu$。

\subsection{运动方程和守恒量}\label{sec:four-eqs}

从\eqref{eq:qed-lagrangian}马上可以使用欧拉-拉格朗日方程写出运动方程。对$\psi$我们有
\[
    \ii \partial_\mu \bar{\psi} \gamma^\mu + e \bar{\psi} \gamma_\mu A^\mu + m \bar{\psi} = 0,
\]
对其取共轭,或者对$\bar{\psi}$应用欧拉-拉格朗日方程,就得到
\begin{equation}
    \ii \gamma^\mu \partial_\mu \psi - m \psi = e \gamma_\mu A^\mu \psi.
    \label{eq:movement-eq-1}
\end{equation}
对$A^\mu$应用欧拉-拉格朗日方程,则有
\begin{equation}
    \partial_\mu F^{\mu \nu} = \partial_\mu (\partial^\mu A^\nu - \partial^\nu A^\mu) = e \bar{\psi} \gamma^\nu \psi.
    \label{eq:movement-eq-2}
\end{equation}
以上两个方程给出了\eqref{eq:qed-lagrangian}的运动方程。(如前所述,$\psi$和$\bar{\psi}$虽然是独立的场,但它们的运动方程并不独立,因为运动方程是一阶的)

现在我们分析局域$U(1)$对称性带来的守恒量。在局域$U(1)$变换下,我们有
\[
    \var{\psi} = \ii e \psi \var{a}, \quad \var{A^\mu} = \partial^\mu \var{a},
\]
则守恒流为
\[
    \begin{aligned}
        J^\mu \var{a} &= - e \bar{\psi} \gamma^\mu \psi \var{a} + (-\partial^\mu A^\nu + \partial^\nu A^\mu) \partial_\nu \var{a} \\
        &= - e \bar{\psi} \gamma^\mu \psi \var{a} - \partial_\nu (\partial^\nu A^\mu - \partial^\mu A^\nu) \var{a},
    \end{aligned}
\]
第二个等号实际上是忽略了一个边界项后得到的结果。%
\footnote{考虑到$\var{a}$在每一点都可以独立地变化,$\int A\var{a} = \int B \var{a}$意味着$A=B$。}%
无论如何,这个守恒流的第二项是平凡的,因为它就是电磁张量的一个指标求散度之后的结果,它的散度当然是零。
那么,我们就有以下守恒荷:
\begin{equation}
    J^\mu = e \bar{\psi} \gamma^\mu \psi , \quad \partial_\mu J^\mu = 0.
    \label{eq:four-current}
\end{equation}
回过头看,实际上这是\emph{全局$U(1)$对称性}的守恒荷——全局$U(1)$对称性中$a$在时空上是均匀的,那么$\partial_\mu \var{a}$就是零,正好让含有$A$的那个平凡的项消失。
实际上从\eqref{eq:movement-eq-2}中我们也可以得到这个守恒流。由于电磁张量是反对称的,我们有:
\[
    0 = \partial_\mu \partial_\nu F^{\mu \nu} = \partial_\mu (e \bar{\psi} \gamma^\mu \psi).
\]
这就导出了\eqref{eq:four-current}。
使用\eqref{eq:four-current}可以将\eqref{eq:movement-eq-2}写成
\begin{equation}
    \partial_\mu F^{\mu \nu} = J^\nu.
    \label{eq:four-maxwell}
\end{equation}

我们最后评论一下以上守恒量和运动方程的意义。\eqref{eq:four-maxwell}实际上就是麦克斯韦方程的一部分。构成麦克斯韦方程另外一部分的是以下恒等式
\begin{equation}
    \partial_\mu F_{\nu \rho} + \partial_\nu F_{\rho \mu} + \partial_\rho F_{\mu \nu} = 0,
    \label{eq:bianchi-identity}
\end{equation}
它是$F_{\mu \nu}$定义为$A^\mu$的梯度的反对称化导致的结果。%
$J^\mu$给出了麦克斯韦方程中的$\rho$和$\vb*{j}$,即它正是\concept{电荷}的守恒流,电荷是全局$U(1)$对称性对应的守恒荷。
\eqref{eq:four-current}就是$e$乘以粒子数密度,因此$e$就是$\psi$场激发的粒子携带的电荷量。对电子,它是$e = -\abs*{e}$,$\abs*{e}$是元电荷。

\subsection{规范}\label{sec:gauge-def}

电动力学在局域$U(1)$变换下的对称性实际上是一个\emph{规范对称性},也就是说,做任意的局域$U(1)$变换,不会有任何可以观察到的变化,也就是说\eqref{eq:qed-lagrangian}中实际上有多余的自由度。
我们需要对$A$和$\psi$施加适当的约束,以确保满足这个约束的$A$和$\psi$取值可以覆盖所有物理上可能产生的状态,同时不含有任何冗余的自由度,也即要\emph{选取一个规范}。
既然局域$U(1)$对称性是规范对称性,任何反映系统实际状态的物理量都应该在$U(1)$规范变换下不变。

形式最漂亮的应该是\concept{洛伦兹规范},也就是
\begin{equation}
    \partial_\mu A^\mu = 0,
\end{equation}
在这个规范下\eqref{eq:four-maxwell}转化为
\begin{equation}
    \Box^2 A = J, \quad \Box^2 = \partial_\mu \partial^\mu.
\end{equation}
我们得到了一个四维波动方程,当然,这就是\concept{电磁波}。
一个很自然的问题是,洛伦兹规范是否不失一般性?是否存在一组$A^\mu$不能够通过一个规范变换变换为一组满足洛伦兹规范的$A'^\mu$?
实际上洛伦兹规范确实是不失一般性的,因为波动方程的性质很良好,对一个给定的标量场$C$,总是可以找到一个标量场$a$,使得
\[
    \partial_\mu \partial^\mu a = C,
\]
这样不论原本$A^\mu$取什么值,只需要解出一个$a^\mu$使得
\[
    \partial_\mu \partial^\mu a = \partial_\mu A^\mu,
\]
然后做规范变换
\[
    A'^\mu = A^\mu - \partial^\mu a, \quad \psi' = \psi \ee^{-\ii e a},
\]
得到的$A'^\mu$就是服从洛伦兹规范的——并且表示和$A^\mu$完全一样的物理状态。
因此洛伦兹规范确实是不失一般性的。

容易看出$F^{\mu \nu}$是一个规范不变量。实际上,在选定了规范之后,可以从它恢复出$A^\mu$。
不失一般性地选择洛伦兹规范,则我们有
\[
    \partial_\mu F^{\mu \nu} = \partial_\mu \partial^\mu A^\nu - \partial^\nu \partial_\mu A^\mu = \partial_\mu \partial^\mu A^\nu,
\]
由于波动方程的良好性质,我们就从上式反解出$A^\mu$了。
如果是别的规范,就按照它转换到洛伦兹规范的方式,从洛伦兹规范转换到原有规范即可。
总之,原则上任何规范不变量都可以通过$F^{\mu \nu}$求导、积分等得到。

\section{量子电动力学}

本节我们讨论量子化之后的电动力学,即\concept{量子电动力学},或者简称\concept{QED}。
自由无质量矢量场本身体现出规范对称性,因此对它的量子化和QED的量子化是紧密相连的。实际上,这让一些量子场论教科书——如Peskin——甚至直接将矢量场的有关内容作为QED的一部分介绍。

\subsection{正则量子化}

\subsection{路径积分量子化}

\subsection{可重整性}

\subsubsection{Ward-Takahashi恒等式}

考虑仅仅对$\psi$和$\bar{\psi}$做局域$U(1)$变换而不对矢量场做变换的一个情况。
此时根据\eqref{eq:psi-change},有
\begin{equation}
    \var{\psi} = \ii e \alpha \psi, \quad \var{\bar{\psi}} = - \ii e \alpha \bar{\psi}, \quad \var{\mathcal{L}} = - e \bar{\psi} \gamma^\mu \psi \partial_\mu \alpha.
    \label{eq:psi-change-only}
\end{equation}
这不是一个经典意义下的对称操作,因为作用量会发生变化,但是我们知道量子情况下只要保持路径积分测度不变,有时仍然能够得到类似于诺特定理的结果。
此外如果没有适当的规范对称性,也无法写出如此漂亮的“拉氏量的变化就是$\partial_\mu \alpha$乘以电流”的变换。
因此,应当记住\eqref{eq:psi-change-only}实际上是规范对称性的产物。
Ward-Takahashi恒等式是量子版本的诺特定理,而这里我们要导出它关于某些散射振幅的特定形式。
对关联函数$\mel{\Omega}{T \psi(x_1) \bar{\psi}(x_2)}{\Omega}$做以上变换,会发现
\[
    \begin{aligned}
        0 &= \int \fd{\psi} \fd{\bar{\psi}} \fd{A} \ee^{\ii S} \big( - \ii (\int \dd[4]{x} \partial_\mu \alpha(x)) (j^\mu(x) \psi(x_1) \bar{\psi}(x_2) ) \\
        & \quad \quad \quad + (\ii e \alpha(x_1) \psi(x_1)) \bar{\psi}(x_2) + \psi(x_1)(- \ii e \alpha(x_2) \bar{\psi}(x_2)) \big).
    \end{aligned}
\]
其中$j^\mu$就是$e \bar{\psi} \gamma^\mu \psi$。对第一项做分部积分,并且将后两项写成$\delta$函数的形式,就得到
\begin{equation}
    \partial_\mu \mel{\Omega}{T j^\mu(x) \psi(x_1) \bar{\psi}(x_2)}{\Omega} = - e \delta(x - x_1) \mel{\Omega}{T \psi(x_1) \bar{\psi}(x_2)}{\Omega} + e \delta(x - x_2) \mel{\Omega}{T \psi(x_1) \bar{\psi}(x_2)}{\Omega}.
\end{equation}
这就是量子版本的电荷守恒方程。

现在我们做傅里叶变换
\[
    \int \dd[4]{x} \ee^{-\ii k \cdot x} \int \dd[4]{x_1} \ee^{\ii q \cdot x_1} \int \dd[4]{x_2} \ee^{- \ii p \cdot x_2} ,
\]
等式右边没有太多可说,等式左边的$\partial_\mu j^\mu(x)$变成了
\[
    \int \dd[4]{x} \ee^{-\ii k \cdot x} \partial_\mu j^\mu(x) = \int \frac{\dd[4]{k_1}}{(2\pi)^4} \ii k_\mu e \bar{\psi}(k_1 + k) \gamma^\mu \psi(k_1).
\]
值得注意的是,组成$j^\mu(x)$的两个费米子算符现在都具有不确定的动量,即它们对应内线或者说传播子。
此外应当注意$-\ii e \gamma^\mu$正是光子-电子相互作用顶角,因此上式在去掉$k_\mu$之后乘上因子$\epsilon_\mu$就得到子图“一个光子入射,打出一对电子和正电子”。
这样我们就有
\[
    \begin{aligned}
        &\quad \int \dd[4]{x} \ee^{-\ii k \cdot x} \int \dd[4]{x_1} \ee^{\ii q \cdot x_1} \int \dd[4]{x_2} \ee^{- \ii p \cdot x_2} \partial_\mu \mel{\Omega}{T j^\mu(x) \psi(x_1) \bar{\psi}(x_2)}{\Omega} \\
        &= - k_\mu \begin{gathered}
            \begin{tikzpicture}
                \begin{feynhand}
                    \vertex (a) at (-1.5, 0) {$\mu$};
                    \vertex [grayblob] (b) at (0, 0) {};
                    \vertex (c) at (0, 1.5);
                    \vertex (d) at (0, -1.5);
                    \propag [photon, mom={$k$}] (a) to (b); 
                    \propag [fermion] (b) to [edge label={$q$}] (c);
                    \propag [fermion] (d) to [edge label={$p$}] (b);
                \end{feynhand}
            \end{tikzpicture}
        \end{gathered}.
    \end{aligned}
\]
这里有一个微妙的地方:费曼图实际上就是微扰计算某种“格林矩阵”或是“跃迁矩阵”的矩阵元的图形,其外线就好像矩阵元的指标;通常,外线要么代表$S$矩阵的入射态和出射态(此时外线没有传播子),要么代表关联函数中出现的场(此时外线有传播子,并且通常会有自能修正)。
然而,上式中的图形中的外线实际上不止一种:电子线是关联函数中的外线,有传播子,但光子线实际上是(给定偏振,从而不和$\epsilon_\mu$点乘的)外场引入的。
很容易验证,上式左边没有光子传播子,那么右边当然也没有,因此光子线不是关联函数中的外线;但是这里的光子线同样不是$S$矩阵的外线,因为光子不必在壳(物理地说,外场意味着有一个非常重的系统在和我们讨论的系统交换光子,这些光子当然无需在壳)。
实际上,下图
\[
    \begin{tikzpicture}
        \begin{feynhand}
            \vertex [crossdot] (a) at (-3.5, 0) {};
            \vertex (e) at (-2, 0);
            \vertex [grayblob] (b) at (0, 0) {};
            \vertex (c) at (0, 1.5);
            \vertex (d) at (0, -1.5);
            \propag [photon, mom={$k, \mu$}] (a) to (e); 
            \propag [fermion] (b) to [in=60, out=150, looseness=1.5] (e);
            \propag [fermion] (e) to [in=210, out=300, looseness=1.5] (b);
            \propag [fermion] (b) to [edge label={$q$}] (c);
            \propag [fermion] (d) to [edge label={$p$}] (b);
        \end{feynhand}
    \end{tikzpicture}
\]
是更加合理的画法,光子线与$\otimes$相连代表它没有传播子这一事实,虽然一些教科书(如Peskin并没有采取这种画法)。
所以最后我们得到
\begin{equation}
    - k_\mu \times \begin{gathered}
        \begin{tikzpicture}
            \begin{feynhand}
                \vertex [crossdot] (a) at (-3.5, 0) {};
                \vertex (e) at (-2, 0);
                \vertex [grayblob] (b) at (0, 0) {};
                \vertex (c) at (0, 1.5);
                \vertex (d) at (0, -1.5);
                \propag [photon, mom={$k, \mu$}] (a) to (e); 
                \propag [fermion] (b) to [in=60, out=150, looseness=1.5] (e);
                \propag [fermion] (e) to [in=210, out=300, looseness=1.5] (b);
                \propag [fermion] (b) to [edge label={$q$}] (c);
                \propag [fermion] (d) to [edge label={$p$}] (b);
            \end{feynhand}
        \end{tikzpicture}
    \end{gathered} 
    = - e \begin{gathered}
        \begin{tikzpicture}
            \begin{feynhand}
                \vertex [grayblob] (b) at (0, 0) {};
                \vertex (c) at (0, 1.5);
                \vertex (d) at (0, -1.5);
                \propag [fermion] (b) to [edge label={$q-k$}] (c);
                \propag [fermion] (d) to [edge label={$p$}] (b);
            \end{feynhand}
        \end{tikzpicture}
    \end{gathered}
    \quad + \quad e \begin{gathered}
        \begin{tikzpicture}
            \begin{feynhand}
                \vertex [grayblob] (b) at (0, 0) {};
                \vertex (c) at (0, 1.5);
                \vertex (d) at (0, -1.5);
                \propag [fermion] (b) to [edge label={$q$}] (c);
                \propag [fermion] (d) to [edge label={$p+k$}] (b);
            \end{feynhand}
        \end{tikzpicture}
    \end{gathered}\ .
    \label{eq:ward-takahashi-qed}
\end{equation}
这就是量子电动力学中通常所说的\concept{Ward-Takahashi恒等式}。
实际上,由一般的Ward-Takahashi恒等式(即“量子诺特定理”)可以看出,在变换\eqref{eq:psi-change-only}下,我们在上式两边的关联函数中添加更多的场(即加入更多入射和出射电子线),上式也是满足的。
注意以上公式中从空间做傅里叶变换得到的动量空间的关联函数并不是在壳的,且\eqref{eq:ward-takahashi-qed}中的$e$实际上是裸参数$e_0$。

\eqref{eq:ward-takahashi-qed}看起来非常不自然,但应当注意到它的左边包括两个电子传播子而右边有一个电子传播子,从而在左边留下因子$Z_2$;此外,它显然含有顶角函数,即含有$Z_1$。
光子没有传播子,但是仍然有自能修正(因为项链图还是存在的;在算$S$矩阵矩阵元时无需考虑项链图,因为只需要算amputated的图,在算关联函数时项链图自动被纳入了光子传播子的自能修正;这里没有光子传播子,但是项链图还在)。
但是,在$k \to 0$时光子不再有自能修正,因为电子有质量,一个动量几乎为零的光子无法激发出正负电子对,因此可以忽略因子$Z_3$。
因此,等式\eqref{eq:ward-takahashi-qed}意味着在量子电动力学中$Z_1$和$Z_2$之间有某种约束关系。
在$k \to 0$时,由重整化条件,顶角函数退化为“物理电荷”乘以$\gamma^\mu$。
因此\eqref{eq:ward-takahashi-qed}左边在$k \to 0$时可以写成
\[
    \begin{aligned}
        &\quad - k_\mu \times \begin{gathered}
            \begin{tikzpicture}
                \begin{feynhand}
                    \vertex [crossdot] (a) at (-3.5, 0) {};
                    \vertex (e) at (-2, 0);
                    \vertex [grayblob] (b) at (0, 0) {};
                    \vertex (c) at (0, 1.5);
                    \vertex (d) at (0, -1.5);
                    \propag [photon, mom={$k, \mu$}] (a) to (e); 
                    \propag [fermion] (b) to [in=60, out=150, looseness=1.5] (e);
                    \propag [fermion] (e) to [in=210, out=300, looseness=1.5] (b);
                    % TODO:粗线表示自能修正
                    \propag [fermion] (b) to [edge label={$q$}] (c);
                    \propag [fermion] (d) to [edge label={$p$}] (b);
                \end{feynhand}
            \end{tikzpicture}
        \end{gathered} 
        = - k_\mu \times \begin{gathered}
            \begin{tikzpicture}
                \begin{feynhand}
                    \vertex [crossdot] (a) at (-1.5, 0) {};
                    \vertex [grayblob] (b) at (0, 0) {$-\ii e \Gamma$};
                    \vertex (c) at (0, 1.5);
                    \vertex (d) at (0, -1.5);
                    \propag [photon, mom={$k, \mu$}] (a) to (b); 
                    \propag [boldfermion] (b) to [edge label={$q$}] (c);
                    \propag [boldfermion] (d) to [edge label={$p$}] (b);
                \end{feynhand}
            \end{tikzpicture}
        \end{gathered} \\
        & = 
        - k_\mu \frac{\ii Z_2}{\slashed{p} - m + \ii 0^+} \frac{\ii Z_2}{\slashed{q} - m + \ii 0^+} (- \ii e \gamma^\mu) (2\pi)^4 \delta^4(q-k-p), \quad \text{as $k \to 0$} .
    \end{aligned}
\]
$\sqrt{Z_3}$因子就是$1$,$e$是重整化之后的电荷。
另一方面,\eqref{eq:ward-takahashi-qed}的右边则是(请注意推导\eqref{eq:ward-takahashi-qed}时用的电荷就是裸电荷,这里记作$e_0$)
\[
    - e_0 \begin{gathered}
        \begin{tikzpicture}
            \begin{feynhand}
                \vertex [grayblob] (b) at (0, 0) {};
                \vertex (c) at (0, 1.5);
                \vertex (d) at (0, -1.5);
                \propag [fermion] (b) to [edge label={$q-k$}] (c);
                \propag [fermion] (d) to [edge label={$p$}] (b);
            \end{feynhand}
        \end{tikzpicture}
    \end{gathered}
    \quad + \quad e_0 \begin{gathered}
        \begin{tikzpicture}
            \begin{feynhand}
                \vertex [grayblob] (b) at (0, 0) {};
                \vertex (c) at (0, 1.5);
                \vertex (d) at (0, -1.5);
                \propag [fermion] (b) to [edge label={$q$}] (c);
                \propag [fermion] (d) to [edge label={$p+k$}] (b);
            \end{feynhand}
        \end{tikzpicture}
    \end{gathered} 
    = - e_0 (2\pi)^4 \delta(q-k-p) \left( \frac{\ii Z_2}{\slashed{p} - m + \ii 0^+} - \frac{\ii Z_2}{\slashed{p} + \slashed{k} - m + \ii 0^+} \right).
\]
考虑到电荷的重整化为$e_0 Z_2 Z_3^{1/2} = e Z_1$而$Z_3$在$k \to 0$时为$1$,有
\[
    - k_\mu \frac{\ii Z_2}{\slashed{p} - m + \ii 0^+} \frac{\ii Z_2}{\slashed{p} + \slashed{k} - m + \ii 0^+} (- \ii e_0 Z_2 Z_1^{-1} \gamma^\mu) = - e_0 \left( \frac{\ii Z_2}{\slashed{p} - m + \ii 0^+} - \frac{\ii Z_2}{\slashed{p} + \slashed{k} - m + \ii 0^+} \right),
\]
从而最终得到
\begin{equation}
    Z_1 = Z_2.
\end{equation}
这个结论是严格成立的;换句话说,计算无穷阶微扰之后,重整化因子$Z_1$和$Z_2$一定是一样的。

\eqref{eq:ward-takahashi-qed}还有另一个用处。\eqref{eq:ward-takahashi-qed}中的电子外线全部是关联函数中的外线,带有传播子,四维动量可以不在壳。
然而,如果我们考虑四维动量在壳的那些情况,那么\eqref{eq:ward-takahashi-qed}右边的两个关联函数中均有四维动量是离壳的(例如容易验证,如果$q$和$k$在壳那么$q-k$肯定是离壳的),因此这些关联函数对$S$矩阵没有贡献。
因此,对那些所有入射和出射外线——无论是电子还是光子——都是$S$矩阵型外线的单光子图——其实就是单光子过程的$\mathcal{M}$——我们有
\begin{equation}
    k_\mu \mathcal{M}^\mu = 0,
\end{equation}
其中$\mathcal{M} = \epsilon_\mu \mathcal{M}^\mu$。这称为\concept{Ward恒等式}。这是Ward-Takahashi恒等式的在壳情况。

总之,规范对称性、所之而来的电荷守恒、Ward恒等式基本上具有同样的来源。

\section{杨-米尔斯理论}

\subsection{杨-米尔斯理论的拉氏量}

\subsubsection{规范场的引入和协变导数}

电动力学的Maxwell理论是一个比较简单的规范理论,其中旋量场可以做任意的局域$U(1)$变换
\[
    \Omega(x) = \ee^{\ii a(x)},
\]
为了让拉氏量在此变换下保持不变,一个额外的矢量场被耦合到旋量场上,当旋量场做$U(1)$变换时矢量场的场值发生一个平移,从而拉氏量在局域$U(1)$变换下不变。
$U(1)$群的特性决定了系统中存在矢量场这一事实,并且决定了相互作用的形式。
以一种系统性的方式决定一个理论中应该有什么场,以及相互作用应该取什么形式,显然是非常有吸引力的。
因此,非常自然地,我们希望用一个更加复杂的李群做规范变换,并且开发一套看着一个李群就能够写下一个具有规范对称性的理论的方式。
对$U(1)$规范理论的推广看起来似乎有非常多可能的选择,但所幸我们已经有一个成熟的、和微分几何紧密相关的、已经在描述强相互作用和弱相互作用的方面大获成功的理论框架:\concept{杨-米尔斯理论}。

下面我们将用一种启发式的方法去导出杨-米尔斯理论,主要是通过模仿电动力学中的概念。
我们只讨论紧致的李群$G$,此时其一定具有幺正表示,这正是我们想要的。
对标量场,看起来唯一能够作用在其上的操作就是乘以一个复因子。
因此如果我们想要让$G$有一个$n$维幺正矩阵表示就需要引入$n$个标量场。($n$和李代数维数没有必然关系)此时,自由理论形如
\[
    \mathcal{L} = \frac{1}{2} \partial_\mu \phi^\dagger \partial^\mu \phi - \frac{1}{2} m^2 \phi^2,
\]
其中$\phi$是$n$个标量场排成的一个列矢量。我们用$t^a$标记$G$的$n$维幺正表示的李代数成员。如果我们想让拉氏量规范不变,即在变换
\[
    \phi(x) \to \Omega(x) \phi(x), \quad \phi^\dagger(x) \to \phi^\dagger(x) \Omega^\dagger(x), \quad \Omega(x) = \ee^{- \ii g \theta_a(x) t^a},
\]
下不变,那么无需调整质量项,因为显然
\[
    \phi^\dagger \Omega^\dagger \Omega \phi = \phi^2.
\]
含有导数的项则需要修正,具体来说,我们需要找到某种协变导数,使得
\begin{equation}
    D_\mu (\Omega(x) \phi(x)) = \Omega(x) D_\mu \phi(x).
    \label{eq:covariant-derivative}
\end{equation}
$\partial_\mu$肯定不满足这个条件,因为
\[
    \partial_\mu (\Omega(x) \phi(x)) = \Omega(x) \partial_\mu \phi + (\partial_\mu \Omega) \phi.
\]
对狄拉克旋量,表面上看如果有$n$个旋量场,$G$可以有$4n$维表示,但是这实际上是行不通的,因为旋量场的拉氏量为
\[
    \mathcal{L} = \bar{\psi} (\ii \gamma^\mu \partial_\mu - m) \psi,
\]
如果$\Omega$是$4n$维的,那么可以让$\Omega$作用到旋量内部的各个分量上。
然而,此时$\bar{\psi}$的变换方式为
\[
    \bar{\psi} \to \psi^\dagger \Omega^\dagger \gamma^0,
\]
没有什么能够保证$\gamma$和$\Omega^\dagger$一定对易,但是$\gamma$和$\Omega^\dagger$最好是对易的,否则简单地接受\eqref{eq:covariant-derivative}并不能让拉氏量在规范变换下不变。
一种比较方便的做法是在将$\Omega$作用于旋量场上的时候将旋量(以及各个$\gamma$矩阵)看成一个整体,不允许对其分量单独进行操作,于是$\Omega$应该是$n$维的,并且因为$\gamma^0$此时相当于一个标量,它和$\Omega$肯定是对易的。
同样,所有$n$个旋量场的质量必须一样,否则$m$将成为一个任意的对角矩阵,而未必和$\Omega$对易,那么$\bar{\psi} m \psi$就不是不变的了。
接受这个做法之后,在旋量场的情况下同样只需要设法找到某种协变导数使得\eqref{eq:covariant-derivative}成立即可。

现在我们分析协变导数的具体形式。在杨-米尔斯理论中,我们模仿电动力学,直接引入矢量场$A_\mu$并要求
\begin{equation}
    D_\mu = \partial_\mu - \ii g A_\mu,
\end{equation}
并以此为依据决定$A_\mu$在规范变换下如何变动,即让$A_\mu$“吸收掉”多余的$(\partial_\mu \Omega) \phi$项。
这种协变导数的形式和微分几何中的协变导数非常一致,$A$就是一个联络(我们将在\autoref{sec:transition}中看到它确确实实就是几何上那种“平移时矢量分量跟着转”的联络),即所谓\concept{规范联络}或者说\concept{规范场}。
显然应有
\[
    \partial_\mu - \ii g A_\mu' = \Omega(x) (\partial_\mu - \ii g A_\mu) \Omega^{-1}(x),
\]
容易看出$A_\mu$的变换规则应为
\begin{equation}
    A_\mu(x) \to \Omega(x) A_\mu(x) \Omega^{-1}(x) + \frac{\ii}{g} \Omega (\partial_\mu \Omega^{-1}(x)). 
\end{equation}
请注意$\Omega(x)$是$n$维矩阵,因此为了避免得到平庸的结果,实际上我们也需要让$A_\mu$变成一个$n$维矩阵,也就是除了时空指标$\mu=0, 1, 1, 3$以外还需要让$A$带两个从$1$跑到$n$的矩阵指标。
在杨-米尔斯理论中我们实际上会将每个时空点、每个时空分量上都是$n$维矢量的$A$场限制为李代数$\{t^a\}$的成员,因为$A_\mu$的变换规则的无穷小版本为
\[
    \var{A_\mu} = \ii g \theta_a \comm*{A_\mu}{t^a} - t^a \partial_\mu \theta_a,
\]
因此如果我们要求$A_\mu(x)$是李代数$\{t^a\}$的成员那么变换之后它还是李代数的成员。
这样,$A$可以用三个标签标记,一个是时空点$x$,一个是矢量指标$\mu$,还有一个是规范指标$a$,写成
\begin{equation}
    A_\mu(x) = A_\mu^a(x) t^a.
\end{equation}
规范场和旋量场(或标量场)现在都带上了一个规范指标$a$,其
于是就有
\begin{equation}
    \begin{aligned}
        \var{A_\mu} &= \ii g \theta_a \comm*{A_\mu^b t^b}{t^a} - t^a \partial_\mu \theta_a \\
        &= - t^a D_\mu \theta_a,
    \end{aligned}
\end{equation}
其中
\begin{equation}
    D_\mu \theta^a = \partial_\mu \theta^a + g f^{abc} A^b_\mu \theta^c.
\end{equation}
这里需要解释一下记号$D_\mu$。我们之前定义的$D_\mu$作用在$G$的一个$n$维表示上;然而,规范场$A_\mu^a$是将$A_\mu$以$t^a$为基底展开得到的分量。
作用在$A_\mu$上的变换并不是$n$维表示空间上的线性变换,而是李代数的伴随表示的表示空间上的线性变换。
此处的协变导数$D_\mu$在后者中而不在前者中,它和作用在$\psi$或是$\phi$上的$D_\mu$的具体形式是不同的(虽然都代表“协变的平移”)。
此外$D_\mu \theta_a$实际上是$(D_\mu \theta)_a$,即它会将$\theta_a$的各个分量混合起来,正如微分几何中的情况那样,即我们有$D_\mu \theta_a = D_{\mu \ ac} \theta_c$。

然后我们就会发现,如果李群是非阿贝尔的,那么$A_\mu$的无穷小变换不仅仅是场值做一个平移,还需要加上一个对易子。
电动力学仅讨论$U(1)$变换,属于阿贝尔规范理论,杨-米尔斯理论则是非阿贝尔规范场论。
今后我们称$A_\mu$为\concept{规范场},而称$\psi$或是$\phi$为\concept{物质场},因为和电动力学中的图像类似,似乎“物质场通过携带动量的规范玻色子发生相互作用”。
当然,规范玻色子——如光子——其实也是一种物质,所以这个说法有不准确之处。

\subsubsection{场强张量}

规范场可以有它自己的动能项。在杨-米尔斯理论中,这个动能项大体上仍然应该和电动力学一致,即大体上仍有
\[
    \mathcal{L}_A = - \frac{1}{4} F_{\mu \nu} F^{\mu \nu}
\]
成立。在电动力学中我们有
\[
    \comm*{D_\mu}{D_\nu} = \ii e F_{\mu \nu},
\]
而在杨-米尔斯理论中,$\comm*{D_\mu}{D_\nu}$在规范变换下为
\[
    \comm*{D_\mu}{D_\nu} \to \Omega \comm*{D_\mu}{D_\nu} \Omega^{-1},
\]
因此可以定义
\begin{equation}
    \comm*{D_\mu}{D_\nu} = - \ii g F_{\mu \nu} = - \ii g (\partial_\mu A_\nu - \partial_\nu A_\mu - \ii g \comm*{A_\mu}{A_\nu}),
\end{equation}
作为电动力学中的场强张量的推广。由于$A$实际上是$n$维矩阵,$F_{\mu \nu} F^{\mu \nu}$也是$n$维矩阵,因此我们还需要加上一个求迹操作就能够得到规范不变而同时洛伦兹不变的拉氏量:
\begin{equation}
    \mathcal{L}_A = - \frac{1}{2} \trace(F_{\mu \nu} F^{\mu \nu}) = - \frac{1}{4} F_{\mu \nu}^a F^{a \ \mu \nu},
\end{equation}
如果$G$是$U(1)$,那么上式就自动退化为了电动力学。

可以看到在这种思路下面规范场本身是不能有质量的,因为质量项$m^2 A_\mu A^\mu$无论如何没法变得规范不变。
但是,通过希格斯机制,实际上可以给规范场引入一个等效的质量。本节暂时不讨论这些内容。

因此我们现在就得到了杨-米尔斯理论的拉氏量:如果规范场和旋量场耦合,那么拉氏量就是
\begin{equation}
    \mathcal{L} = \bar{\psi} (\ii \slashed{D} - m) \psi - \frac{1}{4} F_{\mu \nu}^a F^{a \ \mu \nu}.
    \label{eq:yang-mills-lagrangian}
\end{equation}
$F_{\mu \nu} F^{\mu \nu}$项前面的系数本来可以有变化,但是我们完全可以将其吸收到$A$中,然后用调整$g$来保持协变导数不变。
这个拉氏量看起来和电动力学基本上一样,但是因为$F$中的非线性部分,其经典行为实际上就非常有趣。

前面已经说明过,所有旋量场的质量都是一样的,并且从\eqref{eq:yang-mills-lagrangian}也可以看出,规范玻色子没有质量。
因此表面上,杨-米尔斯理论是非常局限的。
但实际上并不是这样:通过希格斯机制是可以引入质量的。

\subsubsection{关于李代数的限制}

% TODO:对李代数的限制

\subsection{对称性和守恒量}

\subsubsection{平移}\label{sec:transition}

规范场的平移操作值得特别讨论。简单地令
\begin{equation}
    x \longrightarrow x + \var{x}, \quad \psi'(x') = \psi(x), \quad A'(x') = A(x)
    \label{eq:naive-transition}
\end{equation}
的确可以保持拉氏量不变,也的确能够据此计算出一个诺特守恒流,但是这样的诺特守恒流不是规范不变的。
这就是说,应该从中“删去”一些虽然平移不变,但是并非规范不变的东西,才能够得到定义良好的能量-动量张量。
我们知道使用\eqref{eq:naive-transition}计算能动张量实际上同时做了一个从拉氏量到哈密顿量的切换,其中我们取
\[
    \pi^\mu = \pdv{\mathcal{L}}{\partial \partial_\mu \phi}
\]
为正则动量,所以其实\eqref{eq:naive-transition}得不到有意义的结果是非常正确的——规范场论本身含有冗余自由度,不能指望此时朴素的勒让德变换仍然成立。
还可以从另一个角度出发看这个问题:变换\eqref{eq:naive-transition}和规范变换不对易,其生成元自然和规范变换也不对易。

注意到,协变导数$D_\mu$是和规范变换对易的,并且它的确代表某种平移,因此我们尝试以它做无穷小变换,或者等价地说,先做一个纯粹的平移变换,再做一个规范变换,即取
\begin{equation}
    \begin{aligned}
        x^\mu &\longrightarrow x^\mu + \var{x^\mu}, \\
        \phi(x) &\longrightarrow \phi'(x') = \exp(\ii g \var{x^\mu} A_\mu) \phi(x), \\
        A_\mu(x) &\longrightarrow A_{\mu}'(x') = \exp(\ii g \var{x^\nu} A_\nu) A_\mu(x) \exp(- \ii g \var{x^\rho} A_\rho) \\
        &+ \frac{\ii}{g} \exp(\ii g \var{x^\rho} A_\rho) \partial_\mu \exp(- \ii g \var{x^\nu} A_\nu),
    \end{aligned}
\end{equation}
计算得到
\begin{equation}
    \begin{aligned}
        \var{\phi} &= - \var x^\mu D_\mu \phi, \\
        \var{A_\mu} &= - \var{x^\nu} F_{\nu \mu},
    \end{aligned} 
\end{equation}
于是守恒流就是

\subsubsection{规范对称性和规范荷}

费米场和每一种规范玻色子之间的相互作用项都有一个不同的规范荷,这个规范荷由规范群的表示给出。

\subsection{Wick转动}

在做完Wick转动

\subsection{微分几何的观点}

\section{杨-米尔斯理论的量子化}

\subsection{Faddeev–Popov量子化}

\subsubsection{规范固定和鬼场}

规范对称性会导致正则量子化变得比较困难,因为需要做复杂的规范选取来消除多余的自由度,而在路径积分量子化中则可以通过Faddeev–Popov量子化比较容易地解决。
我们已经在自由无质量矢量场的量子化中使用过了Faddeev–Popov量子化,这回我们如法炮制。
和自由无质量矢量场的情况不同,此时不仅需要引入规范固定项,还需要引入鬼场。

我们通过在$\int \fd{A_\mu}$之后插入
\[
    1 = \int \fd{\alpha} \delta(G(A^\alpha)) \det(\fdv{G(A^\alpha)}{\alpha})
\]
来设法将对只相差一个规范变换的场重复计数导致的因子提取出来,其中$\alpha$标记规范变换的参数,它带有一个规范指标;规范固定为$G(A)=0$,$G$定义为洛伦兹协变的
\begin{equation}
    G(A^\alpha) = \partial^\mu (A^\alpha)_\mu - \omega(x),
\end{equation}
其中$\omega(x)$是任意的标量场。我们让$\alpha$取无穷小量,则有
\[
    G(A^\alpha) = \partial^\mu A_\mu + \frac{1}{g} \partial^\mu D_\mu \alpha^a - \omega(x),
\]
于是
\[
    \begin{aligned}
        Z &= \int \fd{A} \fd{\psi} \ee^{\ii S[A, \psi]} \\
        &= \int \fd{A} \fd{\psi} \int \fd{\alpha} \delta(G(A^\alpha)) \det(\fdv{G(A^\alpha)}{\alpha}) \ee^{\ii S[A, \psi]} \\
        &= \frac{1}{g} \det(\partial^\mu D_\mu) \int \fd{A} \int \fd{\psi} \int \fd{\alpha} \delta(G(A^\alpha)) \ee^{\ii S[A, \psi]} \\
        &= \frac{1}{g} \det(\partial^\mu D_\mu) \int \fd{\psi} \int \fd{\alpha} \int \fd{A^\alpha} \delta(G(A^\alpha)) \ee^{\ii S[A^\alpha, \psi]} \\
        &= \frac{1}{g} \det(\partial^\mu D_\mu) \int \fd{\psi} \int \fd{\alpha} \int \fd{A} \delta(G(A)) \ee^{\ii S[A, \psi]},
    \end{aligned}
\]
倒数第二个等号是因为$\fd{A}$和$\fd{A^\alpha}$相同,倒数第一个等号是我们重新标记了场。
既然$\omega(x)$可以任意取值,我们不妨重新定义配分函数,去掉无用的因子$g$,并对所有的$\omega(x)$求和,得到
\[
    \begin{aligned}
        Z &= \det(\partial^\mu D_\mu) \int \fd{\omega} \ee^{-\ii \int \dd[4]{x} \frac{\omega^2}{2 \xi}} \int \fd{A} \int \fd{\psi} \int \fd{\alpha} \delta(\partial^\mu A_\mu - \omega(x)) \ee^{\ii S[A, \psi]} \\
        &= \det(\partial^\mu D_\mu) \int \fd{\alpha} \int \fd{A} \int \fd{\psi} \exp(-\ii \int \dd[4]{x} \frac{(\partial^\mu A_\mu^a)^2}{2 \xi}) \ee^{\ii S[A, \psi]}.
    \end{aligned}
\]
无用的$\int \fd{\alpha}$因子可以略去。与自由场的情况不同,此时因子$\det(\partial^\mu D_\mu)$中仍然含有场变量,不能直接丢弃。
为此,可以引入一个\concept{鬼场}$c$,它是一个复标量场,但是是格拉斯曼数,这样就能够满足
\[
    \int \fd{c} \int \fd{\bar{c}} \exp(\ii \int \dd[4]{x} \bar{c} (- \partial^\mu D_\mu) c) = \det(\partial^\mu D_\mu),
\]
如果$c$是普通标量场,那么行列式会出现在分母上。显然$c$并没有什么物理意义,在计算协变的物理量时也没有外线。

现在我们就完成了规范场论\eqref{eq:yang-mills-lagrangian}的量子化:只需要用等效的拉氏量
\begin{equation}
    \mathcal{L} = \bar{\psi} (\ii \slashed{D} - m) \psi - \frac{1}{4} F_{\mu \nu}^a F^{a \ \mu \nu} \underbrace{- \frac{(\partial^\mu A_\mu^a)^2}{2 \xi}}_{\text{gauge fixing}} + \underbrace{\bar{c} (- \partial^\mu D_\mu) c}_{\text{ghost}}
\end{equation}
做路径积分即可。注意规范场和旋量场都有$n$个,因此协变导数$D_\mu$是一个$n$维矩阵,鬼场$c$也是$n$维的,多出来一个规范指标标记这$n$个维度。

\subsubsection{传播子和顶角}

关于费米子交换加负号这件事:规范理论中相互作用项中费米场均呈现为$\bar{\psi} \psi$形式,若干个相互作用项的乘积形如$\bar{\psi} \psi \cdots \bar{\psi} \psi$。
容易验证,这样一个算符序列和外部的算符缩并,缩并线的交叉次数一定是偶数,从而不会因为费米子算符的交换而产生负号。
因此,负号只应该存在于这样一个算符序列内部的缩并,即出现在费米子线形成一个圈的时候。
而容易验证,此时负号仅仅存在于$\expval*{\bar{\psi} \psi}$一个因子中。
因此,费米子交换加负号这件事在规范场论的费曼图中体现为:但凡费米子线形成了一个圈,加负号,否则什么都不做。
在单条费米子线形成一个圈时可以直接把这个圈算出来,此时负号已经体现在这个圈的值当中了,因此无需做任何额外处理。

从规范场论演生出的低能有效理论——如库伦相互作用——虽然不再有规范玻色子传播子,但是顶角的形式仍然保持不变,因此费米子交换加负号这一特点同样可以通过“但凡费米子线形成了圈,加负号”完全描述。

\subsection{BRST对称性}

引入鬼场之后,有效作用量就不再是规范不变的了,因为规范冗余性已经消除,规范不变性被规范固定项和鬼场去掉了。
不过,BRST四人发现,之前定义的规范变换如果补充上一个鬼场的变换,能够有一个整体的对称性。
这个对称性显然不是通常意义上的整体对称性,因为鬼场等都是非物理的,从而,毫不意外的,这个对称性对应的守恒荷是一个格拉斯曼数,其平方为零。
这种变换——\concept{BRST变换}——可以很好地描述规范场的拓扑性质。
通过BRST变换还可以得到Ward-Takahashi恒等式和Slavnov-Taylor恒等式。

首先,我们引入一个玻色辅助场$B$——实际上是一系列玻色辅助场$B^a$,带有一个规范指标,其总数和李代数的维数一致,和$n$没有特别直接的关系——让规范固定项消失,得到
\begin{equation}
    \mathcal{L} = \bar{\psi} (\ii \slashed{D} - m) \psi - \frac{1}{4} F_{\mu \nu}^a F^{a \ \mu \nu} + \bar{c}^a (- \partial^\mu D_\mu^{ac}) c^c + \frac{\xi}{2} (B^a)^2 + B^a  \partial^\mu A_\mu^a. 
    \label{eq:gauge-fixed-with-b}
\end{equation}
现在考虑如下无穷小变换:
\begin{equation}
    \begin{aligned}
        \var{\psi} &= \ii g \epsilon c^a t^a \psi, \\
        \var{A^a_\mu} &= \epsilon D_\mu^{ac} c^c, \\
        \var{c^a} &= - \frac{1}{2} g \epsilon f^{abc} c^b c^c, \\
        \var{\bar{c}^a} &= \epsilon B^a, \\
        \var{B^a} &= 0,
    \end{aligned}
    \label{eq:brst}
\end{equation}
整个拉氏量\eqref{eq:gauge-fixed-with-b}在变换\eqref{eq:brst}之下完全就是不变的。
首先,$A_\mu$和$\psi$的变换就是以$\epsilon c^c$为位移的规范变换,因此\eqref{eq:gauge-fixed-with-b}的头两项不变。
第四项也不变,因为$B$根本就没有发生任何变化。很容易注意到
\[
    \var{B^a \partial^\mu A_\mu^a} = \epsilon B^a \partial^\mu D_\mu^{ac} c^c = - (\var{\bar{c}^a}) (- \partial^\mu D_\mu^{ac}) c^c,
\]
因此只需要检验
\[
    \var{(D_\mu^{ac} c^c)} = 0
\]
即可。将上式展开,
\[
    \begin{aligned}
        \var{(D_\mu^{ac} c^c)} &= D^{ac}_\mu (- \frac{1}{2} g \epsilon f^{cbd} c^b c^d) + g f^{abc} (\var{A_\mu^b}) c^c \\
        &= - \frac{1}{2} g \epsilon \partial_\mu (f^{aed} c^e c^d) - \frac{1}{2} g^2 \epsilon f^{abc} f^{ced} A_\mu^b c^e c^d + \epsilon g f^{abc} (\partial_\mu c^b + g f^{bed} A_\mu^e c^d) c^c ,
    \end{aligned}
\]
展开第一项并使用鬼场的反交换性可以得到
\[
    - \frac{1}{2} g \epsilon \partial_\mu (f^{aed} c^e c^d) = g \epsilon f^{ade} c^e \partial_\mu c^d,
\]
于是
\[
    \var{(D_\mu^{ac} c^c)} = - \frac{1}{2} g^2 \epsilon f^{abc} f^{ced} A_\mu^b c^e c^d + \epsilon g^2 f^{abc} f^{bed} A_\mu^e c^d c^c.
\]
我们将上式右边第二项写得更加对称一些(又一次用到了鬼场的反对称性):
\[
    \begin{aligned}
        \var{(D_\mu^{ac} c^c)} &= - \frac{1}{2} g^2 \epsilon f^{abc} f^{ced} A_\mu^b c^e c^d + \epsilon g^2 f^{abc} f^{bed} A_\mu^e c^d c^c \\
        &= - \frac{1}{2} g^2 \epsilon f^{abc} f^{ced} (A_\mu^b c^e c^d + A^e_\mu c^d c^b + A_\mu^d c^b c^e),
    \end{aligned}
\]
重新排列指标,得到
\[
    \var{(D_\mu^{ac} c^c)} = - \frac{1}{2} g^2 \epsilon (f^{abc} f^{ced} + f^{adc} f^{cbe} + f^{aec} f^{cdb}) A_\mu^b c^e c^d.
\]
由雅可比恒等式发现上式为零。这就表明\eqref{eq:gauge-fixed-with-b}在变换\eqref{eq:brst}之下不变。这称为\concept{BRST}对称性。

设BRST变换的无穷小生成元为$Q$,则由于\eqref{eq:brst}中的每一条变换都正比于鬼场而鬼场是反对易的,我们立刻发现$Q^2=0$。
如果我们对\eqref{eq:gauge-fixed-with-b}做正则量子化,那么它一定有一个幂零的守恒荷$Q$。
一个幂零算符会给出希尔伯特空间的如下分割:
\begin{itemize}
    \item 将被$Q$作用后不为零的那些态组成的子空间记作$\mathcal{H}_1$;
    \item 将$Q$作用在$\mathcal{H}_1$上得到的子空间记作$\mathcal{H}_2$;
    \item 将除此以外的部分记作$\mathcal{H}_0$。
\end{itemize}
由于$Q$和$H$对易,任何一个本征态被作用了$Q$之后得到的还是本征态。$\mathcal{H}_1$中的本征态被作用了$Q$之后就得到$\mathcal{H}_2$中的本征态。
$\mathcal{H}_0$中的本征态被作用了$Q$之后得到的本征态还是在$\mathcal{H}_0$中。
也即,$\mathcal{H}_0$中有一个二重简并,$\mathcal{H}_1$和$\mathcal{H}_2$放在一起构成二重简并。

为了获得定义良好的单粒子态,我们暂时令$g=0$,此时会发现,在$Q$的作用下反鬼场$\bar{c}$变为规范玻色子(注意根据\eqref{eq:gauge-fixed-with-b},$B^a$和$A^a_\mu$满足的算符方程是线性的,它们实际上对应同一种激发),
% TODO

还有一个微妙的细节需要说明:BRST变换保持路径积分的积分测度不变,即这里没有量子反常。
这件事的证明如下。首先,不难验证
\[
    \pdv{B^{a'}}{B^a} = \pdv{\bar{c}^{a'}}{\bar{c}^{a}} = \pdv{c^{a'}}{c^a} = 0,
\]
因此只需要计算$\psi$和$A$两组变量的雅可比行列式即可。我们有
\[
    \pdv{\psi'}{\psi} = \ii g \epsilon c^a t^a, \quad \pdv{\bar{\psi}'}{\bar{\psi}} = - \ii g \epsilon c^a t^a,
\]
以及
\[
    \pdv{A^{a'}_\mu}{A^b_\mu} = g f^{abc} c^c.
\]
路径积分测度的变换因子为
\[
    \frac{\det (\pdv*{A'}{A})}{\det (\pdv*{\psi'}{\psi}) \det (\pdv*{\bar{\psi}'}{\bar{\psi}})} = \frac{\prod_\mu (1 + g f^{abc} c^c)}{(\det (1 - \ii g \epsilon c^a t^a)) (\det (1 + \ii g \epsilon c^a t^a))} = 1 + \mathcal{O}(\epsilon^2),
\]
因此BRST变换下路径积分测度的确是几乎不变的。

\subsection{微扰计算}

微扰计算使用费曼图,费曼图中我们要对内线上的动量做四重积分。

\section{正规化和重整化}

本节将花大量篇幅分析微扰计算时如何做重整化。
乍一看,基于费曼图的分析是不必要的,因为重整化群流具有紫外不动点这件事足够保证一个量子场论的可重整性。
但是应当注意,重整化群流的计算本身常用微扰论,一阶或是二阶微扰论给出的重整化群流是不是可靠是不好说的;
此外,一个理论可重整化和它的每个费曼图都能够消除所有发散并不是一回事——可能存在这样的情况,使得需要重求和一些图才能够消除所有发散。
因此,基于费曼图和抵消项来分析理论的可重整性还是必要的。

\subsection{圈图发散及其来源}

树图计算通常不会发散。如果我们要考虑更高阶的微扰,即要计算圈图,则通常会导致发散,因为我们需要对一个传播子做四维动量积分,得到一个类似于
\[
    \int \frac{\dd[4]{k}}{(2\pi)^4} \frac{1}{k^2 - m^2} \sim \int \dd{k} k
\]
的式子,就出现了发散,具体来说是紫外发散。
红外发散可以通过给所有场加上一个小的质量而轻松解决,计算完成之后再让此质量趋于零,或者也可以通过重求和含有能量可以任意低、数量可以任意多的出射和入射无质量玻色子来解决,无需用正式的重整化消除它们。
不是所有的圈图都发散,一个例子是梯形图,其中确实有一个圈,但是没有发散。
因此,我们需要先判断什么图会发散。

\subsubsection{用表观发散度做估计}

从\eqref{eq:yang-mills-lagrangian}可以看出,各个物理量的量纲满足
\[
    [\dd[4]{x}] + [F^2] = 0, \quad [F] = [\partial A] = 1 + [A],
\]
于是
\[
    [A] = 1 = [\partial].
\]
另一方面,根据协变导数的定义,$\partial$和$g A$的量纲一致,因此只有一种可能:$g$根本就没有量纲。

量纲分析对表观发散度计算很重要。我们知道一张图的表观发散度是
\begin{equation}
    D(\Gamma) = \sum_i n_i \left( d_i + b_i + \frac{3}{2} f_i - 4  \right) + 4 - E_\text{B} - \frac{3}{2} E_\text{F},
\end{equation}
其中$d_i$,$b_i$和$f_i$分别表示$i$类型顶角的动量幂次,玻色子线数目(鬼场也算玻色子,因为其动能项形式和玻色子一致;量纲分析和对易还是反对易无关)和费米子线数目。
玻色场的量纲是$1$(前面刚刚证明过),费米场的量纲是$3/2$,而如前所述耦合常数无量纲,因此对来自协变导数的规范场-旋量场耦合项,有
\[
    [\dd[4]{x}] + d_i + b_i + \frac{3}{2} f_i = 0,
\]
即
\[
    d_i + b_i + \frac{3}{2} f_i - 4 = 0.
\]
其它的顶角来自$F_{\mu \nu} F^{\mu \nu}$项,其中的三玻色子顶角的耦合常数$\sim g$而四玻色子顶角的耦合常数$\sim g^2$,因此同样耦合常数量纲为零,同样有上式成立。
这就意味着,在杨-米尔斯理论中,我们有
\begin{equation}
    D(\Gamma) = 4 - E_\text{B} - \frac{3}{2} E_\text{F}.
\end{equation}
考虑让$D(\Gamma) \geq 0$的情况,会发现以下几种图按照表观发散度的估计会发散:
\begin{itemize}
    \item 四条玻色子外线,表观发散度为$0$;
    \item 三条玻色子外线,表观发散度为$1$;
    \item 两条玻色子外线,表观发散度为$2$;
    \item 一条玻色子外线,两条费米子外线,表观发散度为$0$;
    \item 两条费米子外线,表观发散度为$1$。
\end{itemize}
还有一些图,如一条费米子外线和两条玻色子外线的图,如果存在,也会发散,但是杨-米尔斯理论的顶角形式意味着这样的图是不可能出现的。

\subsubsection{发散子图}

表观发散度不能立刻用于判断一个图是不是发散。一个简单的费米子或是玻色子线按照上一节的说法可能会发散,但是它们显然没有发散。不过,这比较容易排除:我们只需要只讨论amputated diagram即可,此时表观发散度为零的图一定涉及对数发散。
表观发散度为负数的图似乎不应该发散,但是实际上也可能会发散,例如我们在梯形图的任意一个电子线上加一个自能修正,整张图立刻就发散了。
换而言之,一个表观发散度小于零的图仍然可能发散,此时发散来自其子图。

一些图如果切断任意一根内线(即:不对它求积分),那么就不发散。这样的图称为\concept{原始发散图}。
原始发散图的发散程度显然由表观发散度给定。
如果一个图含有原始发散图那么它肯定发散。但是,在引入一个抵消项消除原始发散图之后,它可能还是发散,因为此时的那个原始发散子图虽然被重整化了,变成了一个内部结构无需讨论的点,但是留下来的骨架仍然可能含有另一个原始发散图。
我们可以继续适当调整抵消项来让这个剩下来的原始发散图也被消除掉。
但是,现在又有一个新的问题:一张图中可能有两个原始发散图,且它们共享一个传播子。
此时的积分往往会给出一些难以处理的含有$\log p$之类的因子的发散。
正比于$p^n$的多项式型发散(称为\concept{局域发散},因为它在实空间是非常定域的)比较容易用抵消项消除,因为各个场的自能修正会给出$p^2$项,拼凑一下就可以得到一个多项式。
正比于$\log p$之类的因子的发散(称为\concept{非局域发散},因为它在动量空间是长程的)则无法消除,因为这要求形如$\log \partial_\mu$的抵消项。

因此用抵消项减除发散是非常非平凡的。\autoref{sec:bphz}给出了这么做的一种系统做法。

\subsection{传统方法:维数正规化和抵消项}

一种可行的重整化方法是,首先可以以树图为骨架而对传播子做自能修正而把顶角修正为顶角函数。
这样,所有圈图修正都可以归入自能修正和顶角函数修正之一。我们要做的就是先做正规化,赋予发散的圈图积分一个定义,然后引入系数未知的抵消项,最后使用重整化条件引入物理参数,计算得到所有的东西。

使用这种办法不容易分析一个理论是否在每一阶都是可重整的。\autoref{sec:bphz}给出了一种更加系统的方法。

重整化条件为
\begin{itemize}
    \item 在修正后的单粒子格林函数的极点处,有质量粒子的自能修正对$p^2$(玻色子)或者$\slashed{p}$(费米子)的一阶导数为零;无质量粒子的自能修正为零。
    这是为了确保没有场强重整化。
    动量远离单粒子格林函数极点时它们当然可以不是零;本应如此,否则圈图修正无法体现。
    \item 在修正后的单粒子格林函数的极点处,极点给出的质量(通过$p^2=m^2$解出)就是我们设定的物理质量;在显式引入自能修正时,有质量粒子的自能修正为零。
    这是为了确保质量的修正为零。
    \item 顶角函数和物理相互作用强度相同,这是为了确保顶角修正为零。
    具体什么是“物理相互作用强度”取决于探测方式,如量子电动力学中通常是使用静电学方法测定电磁相互作用的强度,于是我们要求顶角函数在光子动量为零时和静电学方法测得的电磁相互作用强度(其实就是元电荷)相同。
    当然,也可以在顶角函数在光子动量为一个有限非零值的时候做计算。
    这件事实际上是非常非平凡的,因为在不同能标下耦合常数实际上会有小的跑动。详见\autoref{sec:rg}。
\end{itemize}

\subsection{BPHZ重整化}\label{sec:bphz}

\concept{BPHZ重整化}是一种系统的分析基于费曼图的可重整性的方法。前面看到,系统地做任意阶图的发散减除的主要问题在于交缠发散。
BPHZ重整化中交缠发散可以自动地被消去。

以下我们引入一些记号:$\Gamma$等表示一张费曼图,直接计算它,会得到
\begin{equation}
    F_\Gamma = \int \prod_i \dd[4]{k_i} I_\Gamma,
\end{equation}
$I_\Gamma$是被积函数,整个积分发散。$I_\Gamma$具有这样的一般形式:
\begin{equation}
    I_\Gamma = \prod_{ab} \Delta_{ab} \prod_c P_c,
\end{equation}
其中$\Delta$和$P$分别代表传播子和顶角的值,下标$a, b, c$代表顶角的编号。
我们设
\begin{equation}
    J_\Gamma = \int \prod_i \dd[4]{k_i} R_\Gamma
\end{equation}
为$F_\Gamma$的有限部分。
用$\bar{R}_\Gamma$表示$\Gamma$给出的,所有子图的发散均已经减除(从而子图可以用一个内部结构无需考虑的点代替)的被积函数,这个被积函数做了内线积分之后仍然可能发散。
用$t^\Gamma \Gamma$表示去除$\Gamma$的骨架(即其所有发散子图都被用一个点代替之后留下来东西)的发散的方法。
显然如果$\Gamma$已经是原始发散图了,那么$\bar{R}_\Gamma = R_\Gamma$,否则还需要做最后一步减除$R_\Gamma = t^\Gamma \bar{R}_\Gamma$。

我们还将用$p$表示外动量,而用$k$表示内线动量。

大体上说在BPHZ重整化中我们要做这么几件事:
\begin{enumerate}
    \item 首先,给出一种机械的,基于表观发散度的,拿到一张图立刻提取出其不发散的部分和发散的部分的方法,无论是整体的发散还是子图的发散。
    在这一步我们暂时不考虑抵消项的具体形式,也无需做传统意义上的带一个参数($\Lambda$或者$\epsilon$)正规化。
    具体来说,我们需要:
    \begin{enumerate}
        \item 对每个原始发散图,指定一种方式减除其发散部分,即对每个原始发散图$\Gamma$都指定$t^\Gamma \Gamma$。
        \item 对非原始发散图,指定一种方式系统地减除其所有子图的发散,这么做了之后,再减除其最外层骨架的发散。
        \item 说明交缠发散能够被消除。
    \end{enumerate}
    \item 第一步中得到了一系列发散部分。这些发散部分的形式五花八门,但是还是可以分类的:一个$\bar{R}_\gamma$如果还是发散,那么它只能是一个原始发散图。因此每一个$t^\gamma \bar{R}_\gamma$实际上都相当于用某个抵消项抵消了和它具有同样的外线的一个原始发散图产生的发散。
    
    很显然如果理论中有一些原始发散图给出的发散需要形如$\log \partial_\mu$之类的抵消项,或者原始发散图有无数多个,等等,那么这一步就是不现实的。
    因此这就给出了一种简便的方式,来看着一个理论的拉氏量,判断它是否可重整。
    \item 引入重整化条件。如果我们只是要证明理论可以重整化,这一步可以省去。
\end{enumerate}

\paragraph{原始发散图的去发散} 对原始发散图$\Gamma$,设有$E$条外线,则有$E-1$个独立的外线动量。
一个直截了当的去发散方案是在$p=0$(其实不在$p=0$附近展开也是可以的,这基本上就是“在不同能标附近做计算”——见\autoref{sec:rg})附近泰勒展开到$D(\Gamma)$阶:
\begin{equation}
    t^\Gamma I_\Gamma(p_1, p_2, \ldots, p_E) = f(0, \ldots, 0) + \cdots + \frac{1}{D(\Gamma)!} \sum_{i_1, \ldots, i_{D(\Gamma)} = 1}^{E-1} p_{i_1} \cdots p_{i_{D(\Gamma)}} \frac{\partial^{D(\Gamma)} I_\Gamma}{\partial p_{i_1} \cdots \partial p_{i_{D(\Gamma)}}}.
    \label{eq:t-gamma}
\end{equation}
也就是说我们丢弃$D(\Gamma)$阶及以下的所有项。稍微考虑一下会发现这是合理的选择:\eqref{eq:t-gamma}囊括了所有的发散,$(1-t^\Gamma) I_\Gamma$应当是收敛的。
一个发散的内线积分的积分变量或者和外线没有关系,或者是外线动量的某个线性组合(因为动量守恒关系)。
如果是前者,它导致的发散和外线没有关系,那就会被收集到$f(0, \ldots, 0)$当中,于是就被减除了。
如果是后者,那么对$p$求导实际上就是对$k$求导(因为$p$和$k$总是加在一起),那么$I_\Gamma$的一阶到$D(\Gamma)$阶中,$k$的次数加上$4$是大于等于零的(依照表观发散度即可看出这一点),更高阶项中,$k$的次数加上$4$是小于零的。
因此$I_\Gamma$的一阶到$D(\Gamma)$阶做了内线积分会发散而其它阶则会收敛。

\paragraph{非原始发散图的去发散} 消除原始发散图的发散时引入的那些抵消项当然也会出现在非原始发散图的微扰计算当中。
现在如果用重整化之后的、包含抵消项的那一套费曼图微扰计算非原始发散图$\Gamma$,那么被积函数除了$I_\Gamma$以外还包括将$\Gamma$中的一些子图用抵消项代替而得到的一些图给出的被积函数——如果$\Gamma$有彼此不相交(无共同顶角和粒子线)的发散子图$\{\gamma_1, \gamma_2, \ldots, \gamma_n\}$(无论它是否是原始发散图),那么使用重整化后的微扰论计算$\Gamma$时,计算$\{\gamma_1, \gamma_2, \ldots, \gamma_n\}$时做的减除也会被作用在$\Gamma$上。
一些组合数学的论证让我们发现,用重整化之后的微扰论计算$\Gamma$时的被积函数是
\[
    I_\Gamma - \sum_{\{\gamma_1, \ldots, \gamma_n\}} I_{\Gamma/\{\gamma_1, \ldots, \gamma_n\}} \prod_{i=1}^n \bar{R}_{\gamma_i},
\]
其中$\Gamma/\{\gamma_1, \ldots, \gamma_n\}$指的就是将所有被提到的子图都缩成一个点之后得到的图产生的被积函数,这些子图彼此不交(当然,也不彼此嵌套)。$n$大小不定,因为总是可以丢掉一些子图。
很显然,这样由于不相交的子图导致的发散应该都消去了,尚待解决的是,是否将交缠发散也消去了?
然而要注意到,将交缠在一起的两个圈图中的一个断开,得到的就是非常普通的原始发散图,计算这个图之后把断开的圈连上,当成内线积分,就得到了交缠发散。
我们用抵消项取代其中一个圈图,两者具有正好相反的发散,因此用抵消项取代其中一个圈图之后算另一个圈图会得到一个反向的交缠发散!
交缠发散就是如此被消去的。

这样我们就得到了一套系统地、递归减除所有发散子图的方法:
\begin{equation}
    R_\Gamma = (1 - t^\Gamma)(I_\Gamma - \sum_{\{\gamma_1, \ldots, \gamma_n\}} I_{\Gamma/\{\gamma_1, \ldots, \gamma_n\}} \prod_{i=1}^n \bar{R}_{\gamma_i})
\end{equation}
最外层的$(1-t^\Gamma)$因子消去了$\Gamma$中所有发散子图都被重整化为一个点之后,整体作为原始发散图而产生的发散。
或者,换句话说,我们有
\begin{equation}
    \bar{R}_\Gamma = I_\Gamma - \sum_{\{\gamma_1, \ldots, \gamma_n\}} I_{\Gamma/\{\gamma_1, \ldots, \gamma_n\}} \prod_{i=1}^n \bar{R}_{\gamma_i}.
\end{equation}

最后我们来分析杨-米尔斯理论的可重整性。通过量纲分析得到的几种发散图中的原始发散图的外线和可能的顶角的外线完全一致(实际上,正好一一对应),因此只需要说明没有与$\log p$之类的无法直接抵消的项相乘的发散即可。


\subsection{重整化群}\label{sec:rg}

我们知道朴素地做圈图计算会出现发散的原因是杨-米尔斯理论并不是在任何能标下都完全成立的普适理论,而是某个自洽的万有理论的一个低能近似。
不过,在我们能够探测的所有情况下,杨-米尔斯理论都是适用的,即动量截断$\Lambda$是非常、非常高的。
这也就是我们没有必要知道高能标处的物理具体是什么的原因:我们只需要知道它会让低能标处的参数发生跑动,从而抵消掉所有发散就可以了。
所有远离$\Lambda$的地方的所有物理过程都可以通过标准的微扰重整化的方法——无论是传统方法还是BPHZ——完全计算出来。
不言而喻,相较于$\Lambda$的不同“低能标”之间仍然可以有很大的差异:几乎处于绝对零度的凝聚态系统和对撞机中的电子都服从QED,但是显然它们的能标非常不同。

微扰重整化方法能够处理的区域是非常宽广的。我们经常直觉性地说:“在这个能标下,某某效应是重要的,在那个能标下,另一些效应是重要的”。
我们能否指出,什么叫做“某个能标下(不是说这个能标以下,而是说这个能标附近)重要的物理是什么”?
我们需要给“能标”下一个操作性的定义。在凝聚态场论中直接使用$\Lambda$没有任何问题,因为它反比于可观测的晶格常数,而在高能物理中没有这种东西。
在高能物理中有一个地方可以自然地引入“能标”的概念:顶角函数通常依赖于入射粒子的动量的某些线性组合,我们通常会将有效耦合常数定义为特定入射粒子动量下的顶角函数大小。例如,QED中的有效耦合常数就定义为光子动量为零时的顶角函数。在规范场论中这个定义可以直接推广。
我们把诸如此类的“电子-光子散射过程中的光子动量”或是“四玻色子过程中入射玻色子动量差”用符号$\mu$表示。
自能的重整化条件是确定的,顶角函数的重整化条件则依赖于$\mu$的取值。
因此我们也可以将$\mu$称为\concept{减除点},即我们在这个能标附近减除发散。

不同$\mu$下的物理质量或是物理耦合常数等显然存在不同。再仔细考虑一下就会发现这里实际上有一个重整化:$\mu=0$——即通常选取的重整化条件下原则上是可以计算具有任意的入射动量的过程的,但是在$\mu \neq 0$时同样原则上可以计算具有任意的入射动量的过程。
这两个不同的$\mu$对应的两套重整化条件——从而两套物理质量和耦合常数——实际上彼此互为有效理论。
因此在高能物理中,虽然我们并不会去调整$\Lambda$,积掉一些自由度来导致参数跑动,但这不是说不存在参数跑动的现象。
参数跑动是有的,只不过由于这实际上是由于“在一个$\mu$下一些图被用于修正质量而在另一个$\mu$下一些图被用于修正耦合常数”,对应的重整化群真的就是群,因为没有任何信息损失。
两种不同的重整化条件之间,相差一个重整化群操作。%
\footnote{
    这不是说高能物理中不存在需要真的积掉一些高动量自由度的情况。QCD在低能区域的研究基本上就是强子的凝聚态物理,低能有效理论是经常要用到的。
}%

我们尝试写出这种重整化群的重整化群方程。用$g$泛泛地指代理论中的各种常数,用$p$泛泛地指代各个动量。设$G$是一个包含$f$种场的关联函数,其中类型为$i$的场出现了$n_i$次。用$G_0$指代用裸量计算的关联函数。
$G$依赖于$\mu$(因为微扰计算$G$时用到了含有$\mu$的重整化条件),也依赖于$g$(因为用到了含有$\mu$的重整化条件必然会用到这个条件下的物理参数),而$g$也是依赖于$\mu$的(实际上,由于$g$无非可以由一些物理过程的振幅确定,我们知道了某个$\mu$下的$g$之后就可以计算另一些$\mu$下的$g$)。
这样,就有
\[
    G^{n_1 \cdots n_f}(p, g, \mu) = G_0^{n_1 \cdots n_f}(p, g_0) \prod_{i=1}^f Z_i^{n_i / 2}.
\]
由于
\[
    \dv{G_0}{\mu} = 0
\]
而
\begin{equation}
    \mu \dv{\mu} = \mu \pdv{\mu} + \mu \pdv{g}{\mu} \pdv{g} = \pdv{\log \mu} + \pdv{g}{\log \mu} \pdv{g},
\end{equation}
我们有\concept{Callan–Symanzik方程}
\begin{equation}
    \left( \mu \pdv{\mu} + \beta(g) \pdv{g} + \sum_{i=1}^f n_i \gamma_i \right) G^{n_1 \cdots n_f}(p, g, \mu) = 0,
    \label{eq:c-s-eq}
\end{equation}
其中
\begin{equation}
    \beta(g) = \pdv{g}{\log \mu}, \quad \gamma_i = - \pdv{Z_i^{1/2}}{\log \mu}
\end{equation}
分别称为\concept{$\beta$函数}和\concept{反常量纲}。前者给出物理常数的跑动,后者给出经过场强重整化修正的场的量纲。

重整化群\eqref{eq:c-s-eq}允许我们在知道了一个能标下的关联函数之后去预测另一个能标下的关联函数。
很容易产生一个问题:如果取$\mu=0$就足够计算所有问题的话,为什么还需要取不同的$\mu$?
最为简单的答案是,这样更加方便:例如,设我们需要计算某个能量尺度大体上是$\mu$的过程,而手头边正好有一个能标$\mu$下的物理常数,那么使用基于$\mu$的重整化条件显然更加方便,并且可能只需要计算树图就足够得到非常精确的结果。
相反,从$\mu=0$处的物理常数出发做计算就可能要计算一些圈图。
因此某个能量尺度大体上是$\mu$的过程中间会发生什么,使用基于$\mu$的重整化条件是最能够清楚地展现的。
这就体现出了重整化群的一个作用:它可以用于定性(以及在需要的时候,随时可以定量)地、直观地分析“某个能标下哪些效应明显”:
\begin{enumerate}
    \item 如果随着重整化群流,某个物理参数在高能标下发散(此时称为出现\concept{朗道极点}),那么这个理论基本上不要想能够预测比这更高的能标处的物理,即一个自然的动量截断被给出了。QED是一个典型的例子。
    \item 如果随着重整化群流,某个物理参数在高能标下变得很小,即出现\concept{渐进自由},那么它在高能标下就肯定是非常好处理的(只要使用了正确的重整化条件)。QCD是一个典型的例子。实际上到目前为止,只有非阿贝尔的杨-米尔斯理论在四维闵可夫斯基时空中具有渐近自由。
    实验观测到强子在高能标下的确有渐进自由,因此这可以
    \item 随着重整化群流一些相互作用耦合常数可能会有时候正有时候负,即有时候有吸引有时候排斥,这通常意味着存在相变。
    \item 如果随着重整化群流,某个吸引相互作用在低能标下变得很大,那就出现了\concept{禁闭}。QCD也是这方面的一个例子。
\end{enumerate}

在凝聚态物理中实际上也可以看到类似的处理,如我们分析Kondo效应时就对散射过程的耦合常数做了一个重整化,实际上就是把对散射贡献比较大的过程先求和到一起,那么看着散射过程的耦合常数随着能标的变化就能够大概知道不同能标下散射振幅大小的变化。
温度对应能标,所以我们就看到了不同温度下散射振幅大小的变化。

在具体做计算时,经常使用维数正规化中引入的$\mu$作为参数来计算重整化群流。这种方式得到的重整化群方程、用动量截断和Wilson重整化群计算得到的重整化群方程、用重整化条件中光子动能为能标计算得到的重整化群方程都是一样的。
后两者是一样的已经说明了,第一种方法和Wilson重整化群的等价性是由于,我们关心的区域总的来说都是低能的,从而在两种方案下,我们能看到的只是“一个能标带着一些参数在跑动”,而由于$\mu$和$\Lambda$在拉氏量中的地位是完全相同的,它们给出同样的重整化群方程。

\section{量子色动力学}

本节给出非阿贝尔杨-米尔斯理论的一个例子,著名的描述了强相互作用的量子理论,\concept{量子色动力学}或者简称\concept{QCD}。

QCD是通过$SU(3)$规范对称性得到的理论。我们取$SU(3)$的维数最小的幺正表示,此时李代数的表示的基底就是泡利矩阵。
这是一个三维表示,从而我们需要三个放在一起的旋量场做表示空间,记作
\begin{equation}
    \psi = \pmqty{\psi_1 \\ \psi_2},
\end{equation}
称为\concept{三重态}。

\begin{equation}
    f^{acd} f^{bcd} = \delta^{ab} N
\end{equation}

八种不同的胶子

\section{拓扑和整体微分几何}

一些非线性方程能够给出\concept{孤子解}。一个例子是
\begin{equation}
    \mathcal{L} = \frac{1}{2} \partial_\mu - \frac{\lambda}{4} \left(\phi^2 - \frac{m^2}{\lambda} \right)^2
\end{equation}

\begin{equation}
    (-\partial_t^2 + \partial_x^2) \phi + \phi ( 1- \phi) = 0,
\end{equation}
\begin{equation}
    \phi(x) = \pm \tanh \frac{x - x_0}{\sqrt{2}}
\end{equation}
做洛伦兹变换,就得到一个在空间中持续匀速运动的波包
\begin{equation}
    \phi(x, t) = \pm \tanh \frac{x - vt - x_0}{\sqrt{2(1-v^2)}}.
\end{equation}



\section{共形场论}

\subsection{局域场的算符代数}

凝聚态系统在相变点附近关联长度发散,从而可以预期,能够使用一个无能隙的理论描述它,并且很大一类系统的这种理论将会具有非常好的性质。
如果这个理论正好就是一个自由理论,那么对任何的关联函数的计算都是显然的:我们总是可以找到一组场,记之为$\phi(x)$,任何一个局域的算符$A(x)$均可以展开为$\phi^n$的正规序的线性组合:
\begin{equation}
    A(x) = \sum_{n \geq 0} a_n \normord{\phi(x)^n},
\end{equation}
其中
\[
    \normord{\phi(x)^n} = \phi(x)^n - \expval*{\phi(x)^n},
\]
在这么定义会产生疑难的时候,只需要让各个$x$略微差一些,然后计算完成后让它们相等即可。
由于自由理论中场的量纲就是工程量纲,无需做任何特殊的考虑,就能够得到
\begin{equation}
    \expval*{\phi(x) \phi(y)} \sim \frac{1}{r^{2 d_\phi}}, \quad r = \abs*{x - y}.
\end{equation}
于是我们可以据此估计出任何一个关联函数的衰减趋势,并实际上真的计算出它。

现在我们考虑相互作用系统。此时随意选择一个$\varphi(x)$,一般来说是不能有以上操作了,因为此时$\phi$未必有完全确定的反常量纲,例如场论中$\phi^2$项的$\beta$函数可能同时显含其它很多项的参数。
然而,\emph{假定}我们确实找到了一组可数的场$\{\varphi_i(x)\}$,使得$\varphi_i^2$项真的就是临界点附近重整化群流的本征方向,从而
\begin{equation}
    \varphi_i(x) = \lambda^{d_i} \varphi_i(\lambda x),
    \label{eq:primary-field-scaling}
\end{equation}
并且,进一步,任何一个局域算符都可以写成这组场的线性组合(或者至少,在关联函数的括号$\expval*{\cdot}$中可以这么做——这种线性展开可能并不一般地成立;后文中很多类似的线性展开也需要如此理解),即
\begin{equation}
    A(x) = \sum_{i} a_i \varphi_i(x).
\end{equation}
对自由理论,显然
\begin{equation}
    \varphi_i(x) = \normord{\varphi(x)^i},
\end{equation}
对相互作用体系我们尚不清楚$\varphi_i(x)$是什么。
无论如何,在\eqref{eq:primary-field-scaling}严格成立时,我们有
\begin{equation}
    \expval*{\varphi_n(x)} = 0.
\end{equation}

现在我们得到了任何局域的算符的代数:就是一个线性代数。$\varphi_n(x)$是局域的算符的基底。
现在考虑两个相隔了有限距离的点上的局域算符的乘积,即$A(x_1) B(x_2)$。
当$x_2 \to x_1$时,这个形式会变成一个单一的局域算符,从而可以做算符展开,
% TODO:适用条件
\begin{equation}
    A(x_1) B(x_2) = \sum_{k} \beta(x_1, x_2) \varphi_k(x_2).
\end{equation}
特别的,如果$A(x)$和$B(x)$实际上都是$\phi(x)$场,
\begin{equation}
    \varphi_p(x_1) \varphi_q(x_2) = \sum_{r} C_{pq}^r(x_1, x_2) \varphi_r(x_2). 
\end{equation}

\begin{equation}
    C_{pq}^r(x_1, x_2) = c_{pq}^r \frac{1}{\abs*{x_1 - x_2}^{d_p + d_q - d_r}}.
\end{equation}
其中$c_{pq}^r$是\concept{算符代数的结构常数}。

似乎我们有两种方法定义共形场论,其一是通过局部的、在尺度变换下协变的算符代数,其二是假定度规在局域尺度变换下只差一个常数。

\begin{equation}
    x' = x + \epsilon,
\end{equation}
\begin{equation}
    \partial_\mu \epsilon_\nu + \partial_\ni \epsilon_\mu = \rho(x) g_{\mu \nu},
\end{equation}
对上式两边求迹,得到
\[
    2 \partial_\mu \epsilon^\mu = D \rho(x),
\]
于是
\begin{equation}
    \partial_\mu \epsilon_\nu + \partial_\nu \epsilon_\mu = \frac{2}{D} g_{\mu \nu} \partial \cdot \epsilon
\end{equation}

% TODO: g_\mu^\mu = D这件事

\subsection{共性对称性}

我们称\concept{共形群}为空间尺度变换和洛伦兹变换共同组成的群。具有共形不变性的场论就是\concept{共形场论}。
维数大于2的共形场论是一个有限维的普通李群,有有限个生成元。高维共性不变性是整体的,没有局域不变性,从而也不是规范理论。
然而,对二维系统,共形变换是局域变换群,共性不变的场论也是一种规范理论。此时的共性群的生成元有无限多个,描述它们的代数不再是普通的李代数,而是Virasora代数。

很多一维系统——比如一维电子气演生出来的Luttinger液体——看起来非常“简单”,且时间和空间对应得非常好,使得做完Wick转动之后我们几乎就得到了一个定义在\emph{复平面上的}场论。
这样的场论具有很多非常有趣的性质。

二维系统的$\epsilon_1$和$\epsilon_2$满足柯西-黎曼条件,从而二维的共性变换就是一个解析函数。

\begin{equation}
    \partial_z = \frac{1}{2} (\partial_1 - \ii \partial_2), \quad \partial_{\bar{z}} = \frac{1}{2} (\partial_1 + \ii \partial_2),
\end{equation}
\begin{equation}
    \partial_1 = \partial_z + \partial_{\bar{z}}, \quad \partial_2 = \ii (\partial_z - \partial_{\bar{z}}).
\end{equation}

\begin{equation}
    \dd{s^2} = \dd{x^2} + \dd{y^2} = \dd{z} \dd{\bar{z}}.
\end{equation}

\begin{equation}
    g_{zz} = g_{\bar{z} \bar{z}} = 0, \quad g_{z \bar{z}} = g_{\bar{z} z} = \frac{1}{2}.
\end{equation}

严格的共形不变性会对$T_{\mu \nu}$做出非常强的限制:旋转对称性意味着
\begin{equation}
    T_{\mu \nu} = T_{\nu \mu},
\end{equation}
标度不变性意味着
\begin{equation}
    T_{\mu}^\nu = 0.
\end{equation}

$x^\mu = (x^1, x^2)$, $x^{\mu'} = (z, \bar{z})$,则
\[
    \left[\pdv{x^\mu}{x^{\mu'}}\right]_{\mu \mu'} = \pmqty{\frac{1}{2} & \frac{1}{2} \\ - \frac{\ii}{2} & \frac{\ii}{2}},
\]
由于
\[
    T_{\mu' \nu'} = \pdv{x^\mu}{x^{\mu'}} T_{\mu \nu} \pdv{x^\nu}{x^{\nu'}},
\]
有
\[
    \left[T_{\mu' \nu'}\right]_{\mu' \nu'} = \left[\pdv{x^\mu}{x^{\mu'}}\right]_{\mu \mu'}^\top \left[T_{\mu \nu}\right]_{\mu \nu} \left[\pdv{x^\mu}{x^{\mu'}}\right]_{\mu \mu'},
\]
计算得到
\begin{equation}
    \begin{aligned}
        T_{zz} &= \frac{1}{4} (T_{11} - T_{22} + 2\ii T_{12}) \eqqcolon T(z, \bar{z}) , \\
        T_{\bar{z} \bar{z}} &= \frac{1}{4} (T_{11} - T_{22} - 2\ii T_{12}) \eqqcolon \bar{T}(z, \bar{z}) \\
        T_{z \bar{z}} &= T_{\bar{z} z} = \frac{1}{4} (T_{11} + T_{22}) = \frac{1}{4} T_\mu^\mu \eqqcolon \frac{1}{4} \Theta(z, \bar{z}).
    \end{aligned}
\end{equation}
使用这些记号,守恒律$\partial_\mu T^{\mu \nu} = 0$变成
% TODO
在共形对称性严格成立的时候,或者说在临界点上,有
\begin{equation}
    \partial_z \bar{T}(z, \bar{z}) = \partial_{\bar{z}} T(z, \bar{z}) = 0.
\end{equation}

\subsection{二维无穷小共性变换}

既然二维共形变换实际上就是一个解析函数,

无穷小生成元:
\begin{equation}
    l_n = - z^{n+1} \partial, \quad \bar{l}_n = - \bar{z}^{n+1} \bar{\partial}.
\end{equation}



\end{document}

\chapter{量子色动力学}

本节给出非阿贝尔杨-米尔斯理论的一个例子,著名的描述了强相互作用的量子理论,\concept{量子色动力学}或者简称\concept{QCD}。

QCD是通过$SU(3)$规范对称性得到的理论。$SU(3)$的李代数的结构常数为
\begin{equation}
    f^{123}=1\ ,\quad f^{147}=f^{165}=f^{246}=f^{257}=f^{345}=f^{376}={\frac {1}{2}}\ ,\quad f^{458}=f^{678}={\frac {\sqrt {3}}{2}}.
\end{equation}
我们取$SU(3)$的维数最小的幺正表示,此时李代数的表示的基底是
\begin{equation}
    T^a = \frac{1}{2} \lambda^a,
\end{equation}
$\lambda^a$是所谓的\concept{盖尔曼矩阵},它是泡利矩阵在$SU(3)$中的对应物,定义为
\begin{equation}
    \lambda^1 = 
\end{equation}
这是一个三维表示,从而我们需要三个放在一起的旋量场做表示空间,记作
\begin{equation}
    \psi = \pmqty{\psi_1 \\ \psi_2 \\ \psi_3},
\end{equation}
称为\concept{三重态}。

我们将$\psi$场给出的粒子称为\concept{夸克}。QCD中夸克的标签包括旋量场一定有的动量、自旋、手性,以及$i=1, 2, 3$提供的\concept{色指标}。
在标准模型中还有更多夸克的标签,包括三代夸克,每代有两种。
$A^\mu$场给出的粒子称为\concept{胶子},QCD中胶子的标签包括无质量矢量场一定有的动量和偏振,以及$a=1$到$8$,这里$a$也称为\concept{色指标}。


\part{粒子物理和标准模型}

\part{共形场论}

\chapter{二维共形场论的基本结构}

提出共形场论的一个动机是相变现象。凝聚态系统在相变点附近关联长度发散,从而可以预期,能够使用一个无能隙的理论描述它,并且很大一类系统的这种理论将会具有非常好的性质。
如果这个理论正好就是一个自由理论,那么对任何的关联函数的计算都是显然的:我们总是可以找到一组场,记之为$\phi(x)$,任何一个局域的算符$A(x)$均可以展开为$\phi^n$的正规序的线性组合:
\begin{equation}
    A(x) = \sum_{n \geq 0} a_n \normord{\phi(x)^n},
\end{equation}
其中
\[
    \normord{\phi(x)^n} = \phi(x)^n - \expval*{\phi(x)^n},
\]
在这么定义会产生疑难的时候,只需要让各个$x$略微差一些,然后计算完成后让它们相等即可。
由于自由理论中场的量纲就是工程量纲,无需做任何特殊的考虑,就能够得到
\begin{equation}
    \expval*{\phi(x) \phi(y)} \sim \frac{1}{r^{2 d_\phi}}, \quad r = \abs*{x - y}.
\end{equation}
于是我们可以据此估计出任何一个关联函数的衰减趋势,并实际上真的计算出它。

一个非常自然的问题是,能否将这种做法——仅仅通过一个理论的\emph{尺度不变性}就确定重要的关联函数的形式而无需使用常规的费曼图等技术,甚至通过形式由对称性确定的若干关联函数\emph{定义}一个理论,完全绕开拉氏量(这称为\concept{bootstrap})——推广到一些不那么平凡、带有相互作用的量子场论中。
为了尽可能减少理论中其它的结构,从而让bootstrap可以通过非常清晰的方式实现(例如只通过对称性信息是无法确定下来QED的),我们下面将要研究一大类具有非常完美的对称性的场论——\concept{共形场论}。

\section{共形对称性和共形群}

\subsection{任意维度的共形群}

简单地说,\concept{共形群}为空间尺度变换和洛伦兹变换共同组成的群(允许局域的操作)。具有共形不变性的场论就是\concept{共形场论}。
更加清楚的定义是这样的:如果一个坐标变换$x \longrightarrow x'$会导致
\begin{equation}
    g'(x') = \Lambda(x) g(x),
    \label{eq:conformal-def}
\end{equation}
那么这就是一个共形变换;全体共形变换组成共形群。
洛伦兹群显然是共形群的一个子群;局域的坐标缩放——让原本均匀的坐标网格变得一些区域大一些区域小——也是共形变换。
显然,共形场论是相对论性量子场论的一个特例。

我们现在分析共形群里面有什么操作。无穷小坐标变换
\[
    x^\mu \longrightarrow x^\mu + \epsilon^\mu
\]
下度规变化为
\begin{equation}
    g_{\mu \nu} \longrightarrow g_{\mu \nu} - (\partial_\mu \epsilon_\nu + \partial_\nu \epsilon_\mu),
\end{equation}
共形变换的定义\eqref{eq:conformal-def}等价于能够找到函数$f(x)$使得
\begin{equation}
    \partial_\mu \epsilon_\nu + \partial_\nu \epsilon_\mu = g_{\mu \nu} f(x).
    \label{eq:small-conformal-f-def}
\end{equation}
实际上我们还能够将$f(x)$关于$\epsilon$的形式确定下来——只需要在上式两边取迹即可,我们有
\[
    2 \partial_\mu \epsilon^\mu = d f(x),
\]
即
\begin{equation}
    f(x) = \frac{2}{d} \partial_\mu \epsilon^\mu,
\end{equation}
其中$d$为时空总维数。这里要注意$g\indices{_\mu^\nu}$实际上就是$\delta_\mu^\nu$,从而求迹之后一定是$d$,而不是$\eta_{\mu \nu}$矩阵求迹后的$2-d$。

下面为了方便起见我们只考虑$g_{\mu \nu}$是欧氏空间度规的情况;也即,我们做一个Wick转动,将所有的相对论性量子场论切换到虚时间下。
我们接着可以尝试确定可能的$\epsilon$的形式。
对\eqref{eq:small-conformal-f-def}两边求导数,我们有


维数大于2的共形场论是一个有限维的普通李群,有有限个生成元。高维共性不变性是整体的,没有局域不变性,从而也不是规范理论。
然而,对二维系统,共形变换是局域变换群,共性不变的场论也是一种规范理论。此时的共性群的生成元有无限多个,且量子化时会出现反常。

\subsection{二维共形群和共形场论}

二维共形场论的重要性一方面来自它自身的奇特性质——对称群有无穷多个生成元,具有很多非常有趣的性质——一方面也有很强的物理意义。
很多一维系统——比如一维电子气演生出来的Luttinger液体(见\soliddoc的第\ref{solid-chap:luttinger-liquid}章)——的低能有效理论看起来非常“简单”,且时间和空间对应得非常好,使得做完Wick转动之后我们几乎就得到了一个定义在\emph{复平面上的}且具有共形对称性的场论。

二维系统的$\epsilon_1$和$\epsilon_2$满足柯西-黎曼条件,从而二维的共性变换就是一个解析函数。

\begin{equation}
    \partial_z = \frac{1}{2} (\partial_1 - \ii \partial_2), \quad \partial_{\bar{z}} = \frac{1}{2} (\partial_1 + \ii \partial_2),
\end{equation}
\begin{equation}
    \partial_1 = \partial_z + \partial_{\bar{z}}, \quad \partial_2 = \ii (\partial_z - \partial_{\bar{z}}).
\end{equation}

\begin{equation}
    \dd{s^2} = \dd{x^2} + \dd{y^2} = \dd{z} \dd{\bar{z}}.
\end{equation}

\begin{equation}
    g_{zz} = g_{\bar{z} \bar{z}} = 0, \quad g_{z \bar{z}} = g_{\bar{z} z} = \frac{1}{2}.
\end{equation}

严格的共形不变性会对$T_{\mu \nu}$做出非常强的限制:旋转对称性意味着
\begin{equation}
    T_{\mu \nu} = T_{\nu \mu},
\end{equation}
标度不变性意味着
\begin{equation}
    T_{\mu}^\nu = 0.
\end{equation}

$x^\mu = (x^1, x^2)$, $x^{\mu'} = (z, \bar{z})$,则
\[
    \left[\pdv{x^\mu}{x^{\mu'}}\right]_{\mu \mu'} = \pmqty{\frac{1}{2} & \frac{1}{2} \\ - \frac{\ii}{2} & \frac{\ii}{2}},
\]
由于
\[
    T_{\mu' \nu'} = \pdv{x^\mu}{x^{\mu'}} T_{\mu \nu} \pdv{x^\nu}{x^{\nu'}},
\]
有
\[
    \left[T_{\mu' \nu'}\right]_{\mu' \nu'} = \left[\pdv{x^\mu}{x^{\mu'}}\right]_{\mu \mu'}^\top \left[T_{\mu \nu}\right]_{\mu \nu} \left[\pdv{x^\mu}{x^{\mu'}}\right]_{\mu \mu'},
\]
计算得到
\begin{equation}
    \begin{aligned}
        T_{zz} &= \frac{1}{4} (T_{11} - T_{22} + 2\ii T_{12}) \eqqcolon T(z, \bar{z}) , \\
        T_{\bar{z} \bar{z}} &= \frac{1}{4} (T_{11} - T_{22} - 2\ii T_{12}) \eqqcolon \bar{T}(z, \bar{z}) \\
        T_{z \bar{z}} &= T_{\bar{z} z} = \frac{1}{4} (T_{11} + T_{22}) = \frac{1}{4} T_\mu^\mu \eqqcolon \frac{1}{4} \Theta(z, \bar{z}).
    \end{aligned}
\end{equation}
使用这些记号,守恒律$\partial_\mu T^{\mu \nu} = 0$变成
% TODO
在共形对称性严格成立的时候,或者说在临界点上,有
\begin{equation}
    \partial_z \bar{T}(z, \bar{z}) = \partial_{\bar{z}} T(z, \bar{z}) = 0.
\end{equation}

\subsection{二维无穷小共性变换}

既然二维共形变换实际上就是一个解析函数,

无穷小生成元:
\begin{equation}
    l_n = - z^{n+1} \partial, \quad \bar{l}_n = - \bar{z}^{n+1} \bar{\partial}.
\end{equation}


\section{共形场论中的关联函数和算符代数}

\subsection{局域场的算符代数}

现在我们考虑相互作用系统。此时随意选择一个$\varphi(x)$,一般来说是不能有以上操作了,因为此时$\phi$未必有完全确定的反常量纲,例如场论中$\phi^2$项的$\beta$函数可能同时显含其它很多项的参数。
然而,\emph{假定}我们确实找到了一组可数的场$\{\varphi_i(x)\}$,使得$\varphi_i^2$项真的就是临界点附近重整化群流的本征方向,从而
\begin{equation}
    \varphi_i(x) = \lambda^{d_i} \varphi_i(\lambda x),
    \label{eq:primary-field-scaling}
\end{equation}
并且,进一步,任何一个局域算符都可以写成这组场的线性组合(或者至少,在关联函数的括号$\expval*{\cdot}$中可以这么做——这种线性展开可能并不一般地成立;后文中很多类似的线性展开也需要如此理解),即
\begin{equation}
    A(x) = \sum_{i} a_i \varphi_i(x).
\end{equation}
对自由理论,显然
\begin{equation}
    \varphi_i(x) = \normord{\varphi(x)^i},
\end{equation}
对相互作用体系我们尚不清楚$\varphi_i(x)$是什么。
无论如何,在\eqref{eq:primary-field-scaling}严格成立时,我们有
\begin{equation}
    \expval*{\varphi_n(x)} = 0.
\end{equation}

现在我们得到了任何局域的算符的代数:就是一个线性代数。$\varphi_n(x)$是局域的算符的基底。
现在考虑两个相隔了有限距离的点上的局域算符的乘积,即$A(x_1) B(x_2)$。
当$x_2 \to x_1$时,这个形式会变成一个单一的局域算符,从而可以做算符展开,
% TODO:适用条件
\begin{equation}
    A(x_1) B(x_2) = \sum_{k} \beta(x_1, x_2) \varphi_k(x_2).
\end{equation}
特别的,如果$A(x)$和$B(x)$实际上都是$\phi(x)$场,
\begin{equation}
    \varphi_p(x_1) \varphi_q(x_2) = \sum_{r} C_{pq}^r(x_1, x_2) \varphi_r(x_2). 
\end{equation}

\begin{equation}
    C_{pq}^r(x_1, x_2) = c_{pq}^r \frac{1}{\abs*{x_1 - x_2}^{d_p + d_q - d_r}}.
\end{equation}
其中$c_{pq}^r$是\concept{算符代数的结构常数}。

似乎我们有两种方法定义共形场论,其一是通过局部的、在尺度变换下协变的算符代数,其二是假定度规在局域尺度变换下只差一个常数。

\begin{equation}
    x' = x + \epsilon,
\end{equation}
\begin{equation}
    \partial_\mu \epsilon_\nu + \partial_\ni \epsilon_\mu = \rho(x) g_{\mu \nu},
\end{equation}
对上式两边求迹,得到
\[
    2 \partial_\mu \epsilon^\mu = D \rho(x),
\]
于是
\begin{equation}
    \partial_\mu \epsilon_\nu + \partial_\nu \epsilon_\mu = \frac{2}{D} g_{\mu \nu} \partial \cdot \epsilon
\end{equation}

% TODO: g_\mu^\mu = D这件事


描述它们的代数不再是普通的李代数,而是Virasora代数。

\bibliographystyle{plain}
\bibliography{relativistic-qft,../formalism/gauge-local-sym} 

\end{document}