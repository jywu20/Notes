\documentclass[hyperref, a4paper]{article}

\usepackage{geometry}
\usepackage{titling}
\usepackage{titlesec}
% No longer needed, since we will use enumitem package
% \usepackage{paralist}
\usepackage{enumitem}
\usepackage{footnote}
\usepackage{enumerate}
\usepackage{amsmath, amssymb, amsthm}
\usepackage{mathtools}
\usepackage{bbm}
\usepackage{graphicx}
\usepackage{subcaption}
\usepackage{physics}
\usepackage{tensor}
\usepackage{siunitx}
\usepackage[version=4]{mhchem}
\usepackage{tikz}
\usepackage{xcolor}
\usepackage{listings}
\usepackage{autobreak}
\usepackage[ruled, vlined, linesnumbered]{algorithm2e}
\usepackage{nameref,zref-xr}
\zxrsetup{toltxlabel}
\usepackage[sorting=none]{biblatex}
\addbibresource{theory.bib}
\addbibresource{experiments.bib}
\usepackage[colorlinks,unicode]{hyperref} % , linkcolor=black, anchorcolor=black, citecolor=black, urlcolor=black, filecolor=black
\usepackage[most]{tcolorbox}
\usepackage{prettyref}

% Page style
\geometry{left=3.18cm,right=3.18cm,top=2.54cm,bottom=2.54cm}
\titlespacing{\paragraph}{0pt}{1pt}{10pt}[20pt]
\setlength{\droptitle}{-5em}

% More compact lists 
\setlist[itemize]{
    itemindent=17pt, 
    leftmargin=1pt,
    listparindent=\parindent,
    parsep=0pt,
}

% Math operators
\DeclareMathOperator{\timeorder}{\mathcal{T}}
\DeclareMathOperator{\diag}{diag}
\DeclareMathOperator{\legpoly}{P}
\DeclareMathOperator{\primevalue}{P}
\DeclareMathOperator{\sgn}{sgn}
\DeclareMathOperator{\res}{Res}
\newcommand*{\ii}{\mathrm{i}}
\newcommand*{\ee}{\mathrm{e}}
\newcommand*{\const}{\mathrm{const}}
\newcommand*{\suchthat}{\quad \text{s.t.} \quad}
\newcommand*{\argmin}{\arg\min}
\newcommand*{\argmax}{\arg\max}
\newcommand*{\normalorder}[1]{: #1 :}
\newcommand*{\pair}[1]{\langle #1 \rangle}
\newcommand*{\fd}[1]{\mathcal{D} #1}
\DeclareMathOperator{\bigO}{\mathcal{O}}

% TikZ setting
\usetikzlibrary{arrows,shapes,positioning}
\usetikzlibrary{arrows.meta}
\usetikzlibrary{decorations.markings}
\tikzstyle arrowstyle=[scale=1]
\tikzstyle directed=[postaction={decorate,decoration={markings,
    mark=at position .5 with {\arrow[arrowstyle]{stealth}}}}]
\tikzstyle ray=[directed, thick]
\tikzstyle dot=[anchor=base,fill,circle,inner sep=1pt]

% Algorithm setting
% Julia-style code
\SetKwIF{If}{ElseIf}{Else}{if}{}{elseif}{else}{end}
\SetKwFor{For}{for}{}{end}
\SetKwFor{While}{while}{}{end}
\SetKwProg{Function}{function}{}{end}
\SetArgSty{textnormal}

\newcommand*{\concept}[1]{{\textbf{#1}}}

% Embedded codes
\lstset{basicstyle=\ttfamily,
  showstringspaces=false,
  commentstyle=\color{gray},
  keywordstyle=\color{blue}
}

% Reference formatting
\newrefformat{fig}{Fig.~\ref{#1}}
\newcommand*{\term}[1]{\textit{#1}}

% Color boxes
\tcbuselibrary{skins, breakable, theorems}

\newtcbtheorem{infobox}{Box}{
    enhanced,
    boxrule=0pt,
    colback=blue!5,
    colframe=blue!5,
    coltitle=blue!50,
    borderline west={4pt}{0pt}{blue!65},
    sharp corners,
    fonttitle=\bfseries, 
    breakable,
    before upper={\parindent15pt\noindent}}{box}
\newtcbtheorem[use counter from=infobox]{theorybox}{Box}{
    enhanced,
    boxrule=0pt,
    colback=orange!5, 
    colframe=orange!5, 
    coltitle=orange!50,
    borderline west={4pt}{0pt}{orange!65},
    sharp corners,
    fonttitle=\bfseries, 
    breakable,
    before upper={\parindent15pt\noindent}}{box}
\newtcbtheorem[use counter from=infobox]{learnbox}{Box}{
    enhanced,
    boxrule=0pt,
    colback=green!5,
    colframe=green!5,
    coltitle=green!50,
    borderline west={4pt}{0pt}{green!65},
    sharp corners,
    fonttitle=\bfseries, 
    breakable,
    before upper={\parindent15pt\noindent}}{box}


\newenvironment{shelldisplay}{\begin{lstlisting}}{\end{lstlisting}}

\newcommand*{\kB}{k_{\text{B}}}
\newcommand*{\muB}{\mu_{\text{B}}}
\newcommand*{\efermi}{\varepsilon_{\text{F}}}
\newcommand*{\pfermi}{p_{\text{F}}}
\newcommand*{\vfermi}{v_{\text{F}}}
\newcommand*{\sA}{\text{A}}
\newcommand*{\sB}{\text{B}}
\newcommand*{\Tc}{T_{\text{c}}}
\newcommand*{\hethree}{$^3$He}
\newcommand*{\hefour}{$^4$He}

\title{Bosonic modes in Fermi liquid}
\author{Jinyuan Wu}

\begin{document}

\maketitle

\section{Introduction}

The Fermi liquid theory can be justified by diagrammatic resummation: 
by summing up a certain family of self-energy diagrams
that are believed to be important, 
we get a correction to the electron band dispersion relation,
as well as a finite lifetime.
The electron density-dependent part of the self-energy correction 
is often known as ``forward scattering'',
which has the form of $f_{\vb*{p} \vb*{k}'} \var{n}(\vb*{p}) \var{n}(\vb*{p}')$ 
in the energy functional.
But interaction channels beside forward scattering 
that come from Coulomb interaction do not just disappear;
they are still a part of the Hamiltonian
and will contribute to the specific heat 
when the system is heated up.
Therefore, it can be expected that a real condensed matter system 
that is said to be in a Fermi liquid phase 
contains \emph{more} than electron-like quasiparticles. 

Characterization of the full spectrum of a system is 
generally only possible for exactly solvable systems. 
This report is constrained on bosonic modes in Fermi liquid,%
\footnote{
    There is a terminological confusion here: 
    the term \term{Fermi liquid} may refer to 
    a system whose Hamiltonian is exactly in the shape of 
    Fermi liquid energy functional,
    or it may refer to a system 
    in which the behavior of electron Green function 
    follows the Fermi liquid theory, 
    but may contain other excitations.
    This note uses the latter definition;
    thus the phrase ``a Fermi liquid'' 
    is a shorthand for ``a real-world condensed matter system 
    demonstrating Fermi liquid behaviors in its single-electron part''.
}
or to be specific, on excitations 
for which a quantum is essentially a renormalized electron-hole pair.
In other words, in this report 
we are interested in oscillation modes of 
operators with the shape of $c^\dagger_{\vb*{k} + \vb*{q} / 2} c_{\vb*{k} - \vb*{q} / 2}$.
Three-electron behaviors do exist \cite{patton2003trion,singh2016trion}
but are beyond the scope of this report.

\section{The formalism}

In principle all electro-hole bosonic modes can be found by 
looking at poles of the four-point Green function,
or in other words, 
by diagonalizing the four-point kernel.
This indeed is the usual method in first-principle calculations 
\cite{berkeleygw},
but is not feasible for semi-quantitative analytical purposes. 

\begin{figure}
    \centering
    \tikzset{every picture/.style={line width=0.75pt}} %set default line width to 0.75pt        

\begin{tikzpicture}[x=0.75pt,y=0.75pt,yscale=-1,xscale=1]
%uncomment if require: \path (0,300); %set diagram left start at 0, and has height of 300

%Straight Lines [id:da9882880739712865] 
\draw [color={rgb, 255:red, 0; green, 0; blue, 0 }  ,draw opacity=1 ][line width=0.75]    (60.17,139) -- (76.83,139) ;
\draw [shift={(71.7,139)}, rotate = 180] [fill={rgb, 255:red, 0; green, 0; blue, 0 }  ,fill opacity=1 ][line width=0.08]  [draw opacity=0] (8.04,-3.86) -- (0,0) -- (8.04,3.86) -- (5.34,0) -- cycle    ;
%Straight Lines [id:da42392425232719333] 
\draw  [dash pattern={on 0.84pt off 2.51pt}]  (76.83,118.41) -- (76.83,139) ;
%Shape: Circle [id:dp7049494622344523] 
\draw  [color={rgb, 255:red, 155; green, 155; blue, 155 }  ,draw opacity=1 ][line width=1.5]  (62.07,103.64) .. controls (62.07,95.49) and (68.68,88.87) .. (76.83,88.87) .. controls (84.99,88.87) and (91.6,95.49) .. (91.6,103.64) .. controls (91.6,111.8) and (84.99,118.41) .. (76.83,118.41) .. controls (68.68,118.41) and (62.07,111.8) .. (62.07,103.64) -- cycle ;
%Straight Lines [id:da8083498580412369] 
\draw [color={rgb, 255:red, 155; green, 155; blue, 155 }  ,draw opacity=1 ][line width=1.5]    (90.83,99.52) -- (91.07,107.24) ;
\draw [shift={(91.11,108.38)}, rotate = 268.22] [fill={rgb, 255:red, 155; green, 155; blue, 155 }  ,fill opacity=1 ][line width=0.08]  [draw opacity=0] (9.91,-4.76) -- (0,0) -- (9.91,4.76) -- (6.58,0) -- cycle    ;
%Straight Lines [id:da19225743957281805] 
\draw  [dash pattern={on 0.84pt off 2.51pt}]  (76.83,67.01) -- (76.83,88.87) ;
%Straight Lines [id:da07506393201557948] 
\draw [color={rgb, 255:red, 155; green, 155; blue, 155 }  ,draw opacity=1 ][line width=1.5]    (63.22,109.08) -- (62.07,103.64) ;
\draw [shift={(61.6,101.47)}, rotate = 77.99] [fill={rgb, 255:red, 155; green, 155; blue, 155 }  ,fill opacity=1 ][line width=0.08]  [draw opacity=0] (9.91,-4.76) -- (0,0) -- (9.91,4.76) -- (6.58,0) -- cycle    ;
%Straight Lines [id:da15812365367400316] 
\draw [color={rgb, 255:red, 0; green, 0; blue, 0 }  ,draw opacity=1 ][line width=0.75]    (166.38,87.33) -- (175.41,111.61) ;
\draw [shift={(172.01,102.47)}, rotate = 249.61] [fill={rgb, 255:red, 0; green, 0; blue, 0 }  ,fill opacity=1 ][line width=0.08]  [draw opacity=0] (8.04,-3.86) -- (0,0) -- (8.04,3.86) -- (5.34,0) -- cycle    ;
%Straight Lines [id:da28581830346964154] 
\draw  [dash pattern={on 0.84pt off 2.51pt}]  (210.26,111.57) -- (175.41,111.61) ;
%Straight Lines [id:da12541785389989157] 
\draw [color={rgb, 255:red, 155; green, 155; blue, 155 }  ,draw opacity=1 ][line width=1.5]    (41,206) -- (77.83,206) ;
\draw [shift={(64.42,206)}, rotate = 180] [fill={rgb, 255:red, 155; green, 155; blue, 155 }  ,fill opacity=1 ][line width=0.08]  [draw opacity=0] (9.91,-4.76) -- (0,0) -- (9.91,4.76) -- (6.58,0) -- cycle    ;
%Straight Lines [id:da8921512092815866] 
\draw  [dash pattern={on 0.84pt off 2.51pt}]  (77.83,206) -- (77.83,230.74) ;
%Shape: Arc [id:dp27478538585863044] 
\draw  [draw opacity=0][dash pattern={on 1.69pt off 2.76pt}][line width=1.5]  (43.2,204.81) .. controls (43.2,204.69) and (43.21,204.57) .. (43.21,204.45) .. controls (43.49,185.28) and (59.72,169.98) .. (79.46,170.27) .. controls (99.19,170.56) and (114.96,186.34) .. (114.68,205.51) .. controls (114.68,205.67) and (114.67,205.84) .. (114.67,206) -- (78.94,204.98) -- cycle ; \draw  [dash pattern={on 1.69pt off 2.76pt}][line width=1.5]  (43.2,204.81) .. controls (43.2,204.69) and (43.21,204.57) .. (43.21,204.45) .. controls (43.49,185.28) and (59.72,169.98) .. (79.46,170.27) .. controls (99.19,170.56) and (114.96,186.34) .. (114.68,205.51) .. controls (114.68,205.67) and (114.67,205.84) .. (114.67,206) ;  
%Straight Lines [id:da8668834992558518] 
\draw [color={rgb, 255:red, 0; green, 0; blue, 0 }  ,draw opacity=1 ][line width=0.75]    (21.17,205.57) -- (41.01,205.57) ;
\draw [shift={(34.29,205.57)}, rotate = 180] [fill={rgb, 255:red, 0; green, 0; blue, 0 }  ,fill opacity=1 ][line width=0.08]  [draw opacity=0] (8.04,-3.86) -- (0,0) -- (8.04,3.86) -- (5.34,0) -- cycle    ;
%Straight Lines [id:da20240648481719448] 
\draw [color={rgb, 255:red, 0; green, 0; blue, 0 }  ,draw opacity=1 ][line width=0.75]    (114.67,206) -- (132.5,206) ;
\draw [shift={(126.78,206)}, rotate = 180] [fill={rgb, 255:red, 0; green, 0; blue, 0 }  ,fill opacity=1 ][line width=0.08]  [draw opacity=0] (8.04,-3.86) -- (0,0) -- (8.04,3.86) -- (5.34,0) -- cycle    ;
%Straight Lines [id:da5837921094324088] 
\draw [color={rgb, 255:red, 155; green, 155; blue, 155 }  ,draw opacity=1 ][line width=1.5]    (77.83,206) -- (114.67,206) ;
\draw [shift={(101.25,206)}, rotate = 180] [fill={rgb, 255:red, 155; green, 155; blue, 155 }  ,fill opacity=1 ][line width=0.08]  [draw opacity=0] (9.91,-4.76) -- (0,0) -- (9.91,4.76) -- (6.58,0) -- cycle    ;
%Straight Lines [id:da9594705332247448] 
\draw [color={rgb, 255:red, 0; green, 0; blue, 0 }  ,draw opacity=1 ][line width=0.75]    (76.83,139) -- (93.5,139) ;
\draw [shift={(88.37,139)}, rotate = 180] [fill={rgb, 255:red, 0; green, 0; blue, 0 }  ,fill opacity=1 ][line width=0.08]  [draw opacity=0] (8.04,-3.86) -- (0,0) -- (8.04,3.86) -- (5.34,0) -- cycle    ;
%Flowchart: Summing Junction [id:dp8110382690818376] 
\draw   (80.53,63.31) .. controls (80.53,61.27) and (78.87,59.62) .. (76.83,59.62) .. controls (74.79,59.62) and (73.14,61.27) .. (73.14,63.31) .. controls (73.14,65.35) and (74.79,67.01) .. (76.83,67.01) .. controls (78.87,67.01) and (80.53,65.35) .. (80.53,63.31) -- cycle ; \draw   (79.45,60.7) -- (74.22,65.92) ; \draw   (74.22,60.7) -- (79.45,65.92) ;
%Straight Lines [id:da16604078947596346] 
\draw [color={rgb, 255:red, 0; green, 0; blue, 0 }  ,draw opacity=1 ][line width=0.75]    (175.41,111.61) -- (166.89,135.25) ;
\draw [shift={(170.06,126.44)}, rotate = 289.81] [fill={rgb, 255:red, 0; green, 0; blue, 0 }  ,fill opacity=1 ][line width=0.08]  [draw opacity=0] (8.04,-3.86) -- (0,0) -- (8.04,3.86) -- (5.34,0) -- cycle    ;
%Straight Lines [id:da01479137916014972] 
\draw [color={rgb, 255:red, 155; green, 155; blue, 155 }  ,draw opacity=1 ][line width=1.5]    (220.23,87.29) -- (210.26,111.57) ;
\draw [shift={(213.35,104.06)}, rotate = 292.34] [fill={rgb, 255:red, 155; green, 155; blue, 155 }  ,fill opacity=1 ][line width=0.08]  [draw opacity=0] (9.91,-4.76) -- (0,0) -- (9.91,4.76) -- (6.58,0) -- cycle    ;
%Straight Lines [id:da7480591369957916] 
\draw [color={rgb, 255:red, 155; green, 155; blue, 155 }  ,draw opacity=1 ][line width=1.5]    (210.26,111.57) -- (219.29,135.85) ;
\draw [shift={(216.52,128.4)}, rotate = 249.61] [fill={rgb, 255:red, 155; green, 155; blue, 155 }  ,fill opacity=1 ][line width=0.08]  [draw opacity=0] (9.91,-4.76) -- (0,0) -- (9.91,4.76) -- (6.58,0) -- cycle    ;
%Straight Lines [id:da5196339191405468] 
\draw [color={rgb, 255:red, 0; green, 0; blue, 0 }  ,draw opacity=1 ][line width=0.75]    (171.89,166.23) -- (200.34,166.23) ;
\draw [shift={(189.31,166.23)}, rotate = 180] [fill={rgb, 255:red, 0; green, 0; blue, 0 }  ,fill opacity=1 ][line width=0.08]  [draw opacity=0] (8.04,-3.86) -- (0,0) -- (8.04,3.86) -- (5.34,0) -- cycle    ;
%Straight Lines [id:da8901634782972021] 
\draw [color={rgb, 255:red, 155; green, 155; blue, 155 }  ,draw opacity=1 ][line width=1.5]    (200.34,166.23) -- (229.22,166.23) ;
\draw [shift={(219.78,166.23)}, rotate = 180] [fill={rgb, 255:red, 155; green, 155; blue, 155 }  ,fill opacity=1 ][line width=0.08]  [draw opacity=0] (9.91,-4.76) -- (0,0) -- (9.91,4.76) -- (6.58,0) -- cycle    ;
%Straight Lines [id:da6436116206103029] 
\draw [color={rgb, 255:red, 0; green, 0; blue, 0 }  ,draw opacity=1 ][line width=0.75]    (172.22,205.57) -- (200.67,205.57) ;
\draw [shift={(181.75,205.57)}, rotate = 0] [fill={rgb, 255:red, 0; green, 0; blue, 0 }  ,fill opacity=1 ][line width=0.08]  [draw opacity=0] (8.04,-3.86) -- (0,0) -- (8.04,3.86) -- (5.34,0) -- cycle    ;
%Straight Lines [id:da7570806690081875] 
\draw [color={rgb, 255:red, 155; green, 155; blue, 155 }  ,draw opacity=1 ][line width=1.5]    (200.67,205.57) -- (229.56,205.57) ;
\draw [shift={(208.61,205.57)}, rotate = 0] [fill={rgb, 255:red, 155; green, 155; blue, 155 }  ,fill opacity=1 ][line width=0.08]  [draw opacity=0] (9.91,-4.76) -- (0,0) -- (9.91,4.76) -- (6.58,0) -- cycle    ;
%Straight Lines [id:da722260533343094] 
\draw [line width=1.5]  [dash pattern={on 1.69pt off 2.76pt}]  (200.34,166.23) -- (200.67,205.57) ;
%Flowchart: Summing Junction [id:dp12707351253689114] 
\draw   (81.53,234.44) .. controls (81.53,232.4) and (79.87,230.74) .. (77.83,230.74) .. controls (75.79,230.74) and (74.14,232.4) .. (74.14,234.44) .. controls (74.14,236.48) and (75.79,238.13) .. (77.83,238.13) .. controls (79.87,238.13) and (81.53,236.48) .. (81.53,234.44) -- cycle ; \draw   (80.45,231.83) -- (75.22,237.05) ; \draw   (75.22,231.83) -- (80.45,237.05) ;




\end{tikzpicture}

    \caption{Linear response of two-point Green function gives four-point Green function: 
    the two Feynman diagrams on the left are $GW$ diagrams with the Green function 
    modified by one external field line, 
    representing linear response of the system;
    the linear susceptibility can be obtained 
    formally by erasing the external field line 
    and we get the two Feynman diagrams on the right,
    which describes the most frequently considered 
    two terms in the Bethe–Salpeter equation formalism \cite{berkeleygw}.}
    \label{fig:two-point-response}
\end{figure}

One way to proceed is to notice that 
linear response of two-point Green function 
to an external field coupled to electrons 
gives us four-point Green function
(\prettyref{fig:two-point-response}).
This again is a first-principle approach 
not feasible for analytical studies
\cite{attaccalite2011real},
but further simplification is possible.
For a direct connection between physical observables and the Green function,
we work with the so-called \emph{lesser Green function}
\begin{equation}
    G^<(\vb*{x}_1, t_1, \vb*{x}_2, t_2) = \ii \expval{\psi^\dagger(2) \psi(1)}.
\end{equation}
Wigner transform of the lesser Green function reads
\begin{equation}
    \begin{aligned}
        G^<(\vb*{X}, \vb*{p}, T, \omega) &= 
        \int \dd{t} \ee^{\ii \omega t}
        \int \dd[3]{\vb*{r}} \ee^{- \ii \vb*{p} \cdot \vb*{x}} 
        G^<(T + t / 2, \vb*{X} + \vb*{x} / 2, T - t/2, \vb*{X} - \vb*{x} / 2), \quad \\
        \vb*{x} &= \vb*{x}_1 - \vb*{x}_2, \quad 
        \vb*{X} = \frac{\vb*{x}_1 + \vb*{x}_2}{2}, \quad 
        t = t_1 - t_2, \quad T = \frac{t_1 + t_2}{2}.
    \end{aligned}
    \label{eq:g-to-f}
\end{equation}
The Wigner transform
defines the position and momentum variables;
note that similar to the single-electron Wigner function,
usual positivity conditions expected in the classical case 
do not hold in general for $G^<(\vb*{X}, \vb*{p})$.
We then introduce two additional assumptions.
The first is the validity of \emph{gradient expansion}:
physical quantities involved in the calculation 
should not have very high order dependence on either $\vb*{X}$ or $\vb*{p}$.
The second is that the quasiparticle picture works well in the system 
so that the peak in the spectral function is sharp enough, and we have 
the following \concept{Kadanoff–Baym ansatz}:
\begin{equation}
    G^<(\vb*{X}, \vb*{p}, T, \omega) = - 2 \pi \ii \delta(\omega - \xi(\vb*{X}, \vb*{p})) \cdot 
    f(\vb*{X}, \vb*{p}, T),
    \label{eq:spectral-peak}
\end{equation}
where $\xi(\vb*{X}, \vb*{p})$ is the 
single-electron Hamiltonian plus the real part of the self-energy 
and thus is not necessarily diagonal in the momentum space 
and has thus undergone Wigner transform.
Note that $\xi$ is corrected by the self-energy,
which may contain $T$ since the system may be subject to an external driving field;
thus, in order for \eqref{eq:spectral-peak} to make sense, 
we should also stipulate that the driving frequency
is small compared to the internal time scale of the system.
The two assumptions are sufficient to lead to the \emph{quantum Boltzmann equation}
\begin{equation}
    \pdv{f}{T} + \grad_{\vb*{p}} \xi \cdot \grad_{\vb*{R}} f 
    - \grad_{\vb*{R}} \xi \cdot \grad_{\vb*{p}} f 
    = \left(\pdv{f}{t}\right)_{\text{c}}, 
    \label{eq:quantum-boltzmann}
\end{equation}
where the collision integral on the right-hand side is decided by Fermi golden rule 
and is 
\begin{equation}
    \left(\pdv{f}{t}\right)_{\text{c}} = 
\end{equation}
Note that the only difference between the quantum Boltzmann equation 
and the classical Boltzmann equation 
is the $(1 \pm f)$ factors coming from fermionic/bosonic statistics.
The gradient expansion condition is intuitively reflected 
by the fact that the collision integral depends on $\vb*{R}$ only;
also note that the imaginary part of the self-energy is ignored 
in the spectral function
but is picked up back to the collision integral.
Below, we replace $\vb*{R}$ by $\vb*{r}$ 
and $T$ by $t$
for the sake of convenience.
Of course, in order for \eqref{eq:quantum-boltzmann} to be a closed equation,
we need an implicitly assumption:
the self-energy correction to $f$ should only contain explicitly $f(\vb*{r}, \vb*{p}, t)$
and have no explicit dependence on higher order Green functions.

The most generalized derivation of \eqref{eq:quantum-boltzmann} 
involves Keldysh field theory 
\cite{RevModPhys.58.323}
and is beyond the scope of this report.
As a proof of concept,
a collision-free linearized quantum Boltzmann equation can be derived 
using random phase approximation (RPA) 
from the equation of motion of the electron-hole pair creation operator
\begin{equation}
    n_{\vb*{k} \vb*{q}}^\dagger = c^\dagger_{\vb*{k} + \vb*{q}} c_{\vb*{k}},
\end{equation}
where $\vb*{q}$ is the total momentum of the pair 
and $\vb*{k}$ is the ``internal'' or relative momentum;
recall that $c_{\vb*{k}}$ creates a hole with momentum $- \vb*{k}$.
By ignoring $\bigO(n^3)$ terms in the equation of motion of $n_{\vb*{k} \vb*{q}}$
and switching the equation to the frequency domain
we have \cite{pines2018theory}
\begin{equation}
    - \ii \dot{n}_{\vb*{p} \vb*{q}}^\dagger
    = (\epsilon_{\vb*{p} + \vb*{q}} - \epsilon_{\vb*{p}}) n^\dagger_{\vb*{p} \vb*{q}}
    + \sum_{\vb*{k}} n_{\vb*{k}} (V_{\vb*{k} - \vb*{p}} - V_{\vb*{k} - \vb*{p} - \vb*{q}}) n^\dagger_{\vb*{p} \vb*{q}}
    - \sum_{\vb*{k}} (V_{\vb*{q}} - V_{\vb*{k} - \vb*{p}})
    (n_{\vb*{p} + \vb*{q}} - n_{\vb*{p}})
    n^\dagger_{\vb*{k} \vb*{q}}.
    \label{eq:eom-density}
\end{equation}
Note that 
\begin{equation}
    \begin{aligned}
        \int \dd[3]{\vb*{x}} \ee^{- \ii \vb*{p} \cdot \vb*{x}} \psi^\dagger(\vb*{x}_2) \psi(\vb*{x}_1) 
        &= \int \dd[3]{\vb*{x}} \ee^{- \ii \vb*{p} \cdot \vb*{x}}
        \frac{1}{\sqrt{V}} \sum_{\vb*{k}} \ee^{- \ii \vb*{k} \cdot (\vb*{X} - \vb*{x} /2 )} c_{\vb*{k}}^\dagger
        \cdot \frac{1}{\sqrt{V}} \sum_{\vb*{k}'} \ee^{\ii \vb*{k}' \cdot (\vb*{X} + \vb*{x} / 2)} c_{\vb*{k}'} \\
        &= \sum_{\vb*{k}, \vb*{k}'} \ee^{\ii \vb*{X} \cdot (\vb*{k}' - \vb*{k})}
        \delta_{\vb*{p}, \frac{\vb*{k} + \vb*{k}'}{2}} c^\dagger_{\vb*{k}} c_{\vb*{k}'},
    \end{aligned}
\end{equation}
and therefore%
\footnote{
    The EOM of the weight of 
    $\ket*{\text{electron at $\vb*{p} + \vb*{q} / 2$}, \text{hole at $\vb*{p} - \vb*{q} / 2$}}$,
    is the same as 
    the \emph{annihilation operator} of the electron-hole pair,
    and therefore there is nothing wrong with the equation below.
}
\begin{equation}
    \begin{aligned}
        \int \dd[3]{\vb*{r}} \ee^{- \ii \vb*{q} \cdot \vb*{r}} f(\vb*{r}, \vb*{p}, t)
        &\simeq \expval*{c^\dagger_{\vb*{k} - \vb*{q} / 2} c_{\vb*{k} + \vb*{q} / 2}} \\
        &\simeq \braket*{\text{electron at $\vb*{p} + \vb*{q} / 2$}, \text{hole at $\vb*{p} - \vb*{q} / 2$}}{\text{single excitation}} .
    \end{aligned}
\end{equation}
The last equation can be verified by comparing 
the EOM in Schr\"{o}dinger picture \cite{egri1985excitons} and in Heisenberg picture.
The gradient expansion condition is equivalent to the condition that 
the characteristic length scale of $f$ with respect to the $\vb*{r}$ variable is large,
which then is equivalent to the condition that $\vb*{q}$ is small.
Then, by noticing 
\begin{equation}
    \xi_{\vb*{p}}(\vb*{r}) = \varepsilon^0_{\vb*{p}}
    + \sum_{\vb*{k}} (V_{0} - V_{\vb*{k} - \vb*{p}} ) n_{\vb*{k}}(\vb*{r}) - \mu,
    \label{eq:simple-fermi-liquid-xi}
\end{equation}
taking the $\vb*{q} \to 0$ limit and keep only the $\bigO(q)$ term in all Taylor expansions 
with respect to $\vb*{q}$
and noticing that 
\[
    \ii \vb*{q} \simeq \pdv{\vb*{r}},
\]
we get a linearized quantum Boltzmann equation with a vanishing collision integral.

The second assumption is by definition satisfied for a Fermi liquid.
The first assumption -- that gradient expansion works -- is at the first glance 
broken in condensed matter systems, 
since the crystal potential has a very small characteristic length scale.
For a single-band problem, however, 
we can manually find a ``position'' operator $\vb*{x}$ as the conjugate variable 
of the lattice momentum $\vb*{k}$,
which represents the center of the wave packet 
and in the coarse-grained macroscopic limit 
appears to be the commonly known position.
The gradient expansion condition therefore is equivalent to the $\vb*{q} \to 0$ limit.
When a uniform electric field is applied, 
it influences $\vb*{x}_1 - \vb*{x}_2$, not $(\vb*{x}_1 + \vb*{x}_2) / 2$;
accordingly, it influences $\vb*{k}$, 
which is now to be understood as the \emph{relative} momentum 
between the electron and the hole 
(absence of an electron with momentum $\vb*{k}$ 
is equivalent to existence of a hole with momentum $- \vb*{k}$).
If, however, the electric field has very strong spatial variance, 
the electron at $\vb*{r}_1$ feels a different force 
from that felt at $\vb*{r}_2$,
and the electron-hole pair gets driven as a whole,
giving a non-zero value to $\vb*{q}$.
Thus eventually, 
the small-$\vb*{q}$ condition is equivalent to 
the condition that the wave length of the driving electric field 
should be small compared with the atomic length scale.

There is yet one more caveat pertaining to the quantum Boltzmann equation:
the naive equation \eqref{eq:quantum-boltzmann} faces a fundamental constraint in multi-band systems
(including band splitting caused by an external magnetic field),
because, as an example, the counterpart of \eqref{eq:eom-density}
contains $\varepsilon^{\text{c}}_{\vb*{k} + \vb*{q}} - \varepsilon^{\text{v}}_{\vb*{k}}$.
Should the two energies be both from a single band, 
gradience expansion would possible when $\vb*{q}$ is small, 
which leads to the $\pdv{\varepsilon_{\vb*{k}}}{\vb*{k}} \pdv{n}{\vb*{r}}$ term;
but for the multi-band case we still need a constant and finite 
$\varepsilon^{\text{c}}_{\vb*{k}} - \varepsilon^{\text{v}}_{\vb*{k}}$ term
even when $\vb*{q} \to 0$.
The way to solve this is to introduce $\ii \comm*{\varepsilon}{f}$ 
in the left-hand side of \eqref{eq:quantum-boltzmann},
where $f_{n n'}$ and $\varepsilon_{n n'}$
are seen as matrices (but the commutator does not count $\vb*{r}$ and $\vb*{p}$ as 
quantum operators, since the corresponding effects are already considered 
by existing terms in \eqref{eq:quantum-boltzmann}).
The rest of the terms on the left-hand side of \eqref{eq:quantum-boltzmann}
should also be replaced by corresponding matrix forms.

\section{Landau kinetic theory of neutral Fermi liquid and zero sound} 

\begin{figure}
    \centering
    

\tikzset{every picture/.style={line width=0.75pt}} %set default line width to 0.75pt        

\begin{tikzpicture}[x=0.75pt,y=0.75pt,yscale=-1,xscale=1]
%uncomment if require: \path (0,300); %set diagram left start at 0, and has height of 300

%Straight Lines [id:da9979613690985549] 
\draw [color={rgb, 255:red, 0; green, 0; blue, 0 }  ,draw opacity=1 ][line width=0.75]    (191.89,155.8) -- (242.35,155.8) ;
\draw [shift={(220.32,155.8)}, rotate = 180] [fill={rgb, 255:red, 0; green, 0; blue, 0 }  ,fill opacity=1 ][line width=0.08]  [draw opacity=0] (8.04,-3.86) -- (0,0) -- (8.04,3.86) -- (5.34,0) -- cycle    ;
%Straight Lines [id:da5231803030380069] 
\draw [color={rgb, 255:red, 0; green, 0; blue, 0 }  ,draw opacity=1 ][line width=0.75]    (242.35,155.8) -- (293.58,155.8) ;
\draw [shift={(271.16,155.8)}, rotate = 180] [fill={rgb, 255:red, 0; green, 0; blue, 0 }  ,fill opacity=1 ][line width=0.08]  [draw opacity=0] (8.04,-3.86) -- (0,0) -- (8.04,3.86) -- (5.34,0) -- cycle    ;
%Straight Lines [id:da4079322973600161] 
\draw [color={rgb, 255:red, 0; green, 0; blue, 0 }  ,draw opacity=1 ][line width=0.75]    (192.48,225.57) -- (242.94,225.57) ;
\draw [shift={(213.01,225.57)}, rotate = 0] [fill={rgb, 255:red, 0; green, 0; blue, 0 }  ,fill opacity=1 ][line width=0.08]  [draw opacity=0] (8.04,-3.86) -- (0,0) -- (8.04,3.86) -- (5.34,0) -- cycle    ;
%Straight Lines [id:da9026612092672825] 
\draw [color={rgb, 255:red, 0; green, 0; blue, 0 }  ,draw opacity=1 ][line width=0.75]    (242.94,225.57) -- (294.17,225.57) ;
\draw [shift={(263.85,225.57)}, rotate = 0] [fill={rgb, 255:red, 0; green, 0; blue, 0 }  ,fill opacity=1 ][line width=0.08]  [draw opacity=0] (8.04,-3.86) -- (0,0) -- (8.04,3.86) -- (5.34,0) -- cycle    ;
%Straight Lines [id:da7016774699366952] 
\draw [line width=1.5]  [dash pattern={on 1.69pt off 2.76pt}]  (242.35,155.8) -- (242.94,225.57) ;
%Rounded Rect [id:dp7248725342845543] 
\draw   (139,138) .. controls (139,130.27) and (145.27,124) .. (153,124) -- (195,124) .. controls (202.73,124) and (209,130.27) .. (209,138) -- (209,240.85) .. controls (209,248.59) and (202.73,254.85) .. (195,254.85) -- (153,254.85) .. controls (145.27,254.85) and (139,248.59) .. (139,240.85) -- cycle ;
%Straight Lines [id:da7227711421802434] 
\draw    (154,105) -- (198.17,105) ;
\draw [shift={(200.17,105)}, rotate = 180] [fill={rgb, 255:red, 0; green, 0; blue, 0 }  ][line width=0.08]  [draw opacity=0] (12,-3) -- (0,0) -- (12,3) -- cycle    ;
%Rounded Rect [id:dp10838249510257558] 
\draw   (285,138) .. controls (285,130.27) and (291.27,124) .. (299,124) -- (341,124) .. controls (348.73,124) and (355,130.27) .. (355,138) -- (355,240.85) .. controls (355,248.59) and (348.73,254.85) .. (341,254.85) -- (299,254.85) .. controls (291.27,254.85) and (285,248.59) .. (285,240.85) -- cycle ;
%Straight Lines [id:da6110642333116789] 
\draw    (299,105) -- (343.17,105) ;
\draw [shift={(345.17,105)}, rotate = 180] [fill={rgb, 255:red, 0; green, 0; blue, 0 }  ][line width=0.08]  [draw opacity=0] (12,-3) -- (0,0) -- (12,3) -- cycle    ;
%Straight Lines [id:da09763037399523022] 
\draw    (256,175.85) -- (256,207.85) ;
\draw [shift={(256,209.85)}, rotate = 270] [fill={rgb, 255:red, 0; green, 0; blue, 0 }  ][line width=0.08]  [draw opacity=0] (12,-3) -- (0,0) -- (12,3) -- cycle    ;

% Text Node
\draw (189.89,155.8) node [anchor=east] [inner sep=0.75pt]    {$\boldsymbol{p}' +\boldsymbol{q}$};
% Text Node
\draw (295.58,155.8) node [anchor=west] [inner sep=0.75pt]    {$\boldsymbol{p} +\boldsymbol{q}$};
% Text Node
\draw (190.48,225.57) node [anchor=east] [inner sep=0.75pt]    {$\boldsymbol{p}'$};
% Text Node
\draw (296.17,225.57) node [anchor=west] [inner sep=0.75pt]    {$\boldsymbol{p}$};
% Text Node
\draw (177.08,102) node [anchor=south] [inner sep=0.75pt]    {$\boldsymbol{q}$};
% Text Node
\draw (322.08,102) node [anchor=south] [inner sep=0.75pt]    {$\boldsymbol{q}$};
% Text Node
\draw (261,182.85) node [anchor=west] [inner sep=0.75pt]    {$\boldsymbol{p} -\boldsymbol{p}'$};


\end{tikzpicture}

    \caption{Deriving the quasiparticle interaction function.
    In the first-order correction the TODO.
    As long as the corrected vertex is well-defined in the $\vb*{q} \to 0$ limit, 
    we can simply set $\vb*{q} = 0$ when $\vb*{q}$ appears 
    in any of the internal interaction lines;
    the corrected interaction vertex therefore only depends on $\vb*{p} - \vb*{k}$,
    as is shown in the figure above.
    The vertex then can be seen as an interaction channel 
    that keeps the total momentum $\vb*{q}$ of an electron-hole pair 
    but constantly changes its internal momentum $\vb*{p}$.}
\end{figure}

$\Re \Sigma$ can have explicit dependence on $G(\vb*{r}, \vb*{r}', t, t')$
i.e. $f(\vb*{r}, \vb*{p}, t)$,
as is shown in \eqref{eq:simple-fermi-liquid-xi}.
Below we change the notation once again and use $n_{\vb*{p}}(\vb*{r})$ 
to refer to the distribution function in place of $f(\vb*{r}, \vb*{p})$
to follow the established convention in Fermi liquid theory,
and also to imply that 
the normalization scheme of the Boltzmann distribution function $n_{\vb*{p}}(\vb*{r})$
follows the same scheme of the density operator $n_{\vb*{p}}$:
at the ground state, we have 
\begin{equation}
    n_{\vb*{p}}(\vb*{r}) = \theta(\efermi - \varepsilon_{\vb*{p} \sigma}),
\end{equation}
and it is easy to verify that 
\begin{equation}
    \int \frac{\dd[3]{\vb*{p}}}{(2\pi)^3} \int \dd[3]{\vb*{r}} n_{\vb*{p}}(\vb*{r})  = 
    \text{\# of occupied states} = N.
\end{equation}
For a general Fermi liquid we can insert
\begin{equation}
    \varepsilon_{\vb*{p} \sigma}(\vb*{r}) = \varepsilon^0_{\vb*{p} \sigma} + 
    \frac{1}{V} \sum_{\vb*{p}', \sigma'} f_{\vb*{p} \vb*{p}' \sigma \sigma'} \var{n}_{\vb*{p}' \sigma'} (\vb*{r})
    \label{eq:landau-eq-assumption}
\end{equation}
into the quantum Boltzmann equation;
the resulting equation system is called \term{Landau equation}.
Since now $\var{n}$ has both $\vb*{p}$ and $\vb*{r}$ dependence,
$\varepsilon_{\vb*{p}}$ also has $\vb*{r}$ dependence.
The $1/V$ factor comes from the normalization constant 
of a two-body interaction, 
as is seen in 
\begin{equation}
    \var{E} = \sum_{\vb*{p}, \sigma} \varepsilon^0_{\vb*{p}} \var{n}_{\vb*{p} \sigma}
    + \frac{1}{2V} \sum_{\vb*{p}', \sigma'} f_{\vb*{p} \vb*{p}' \sigma \sigma'}
    \var{n}_{\vb*{p} \sigma} \var{n}_{\vb*{p}' \sigma'}.
    \label{eq:fermi-liquid-energy}
\end{equation}
No spatial dependence (i.e. $\vb*{q}$ dependence) 
is added to $f_{\vb*{p} \vb*{p}'}$:
the fact that $n_{\vb*{k} \vb*{q}}$ has a non-zero value when $\vb*{q} \neq 0$
is due to external electromagnetic driving, 
which at the linear level does not change
the momentum conservation condition 
where an interaction line meets with two electron lines.
Note that \eqref{eq:landau-eq-assumption} is already beyond 
the Fermi liquid energy functional \eqref{eq:fermi-liquid-energy},
since the energy functional is unable to change the internal relative momentum
i.e. $\vb*{p}$ for an electron-hole pair,
but that is exactly what happens in \eqref{eq:landau-eq-assumption}.
\eqref{eq:landau-eq-assumption}, then, implicitly assumes 
that the evolution of $\vb*{p}$ is irrelevant to the value of $\vb*{q}$;
since we are working in the small-$\vb*{q}$ region where Boltzmann equation holds,
as long as we have a well-defined $f_{\vb*{p} \vb*{p}'}$ function 
in the first place.
Unscreened Coulomb interaction however breaks this condition
(\prettyref{sec:charged-fermi-liquid}).

The inclusion of the Fermi liquid self-energy correction 
immediately leads to an important consequence 
of Fermi liquid: 
that when the temperature is zero and 
no collision is possible for quasiparticles, 
we still have density modes which resemble 
ordinary sound wave in some aspects. 
This mode is known as \emph{zero sound}.
To show this, we go to Fourier space where 

When the temperature is non-zero,
$\tau \propto 1 / T^2$ is finite 
and zero sound faces strong damping 
when its frequency is too slow.
In the low frequency domain, 
where thermal equilibrium is almost always established,
we get ordinary sound or ``first sound''.
The first sound can be derived 
by calculating mechanical properties of the Fermi liquid in question 
and inserting the compressibility 
into $v = \sqrt{\pdv*{p}{\rho}}$ \cite{lifshitz2013statistical}.
This approach assumes the usual framework of near-equilibrium hydrodynamics 
(Navier-Stokes equation, response function from derivative of free energy, etc.)
works for Fermi liquid when the frequency is low enough; 
a direct verification can be found in \cite{belitz2022soft}.

We note that for many thermalized systems it's safe to assume 
a hydrodynamic-like non-equilibrium low-energy effective theory,
without going into details about the microscopic Hamiltonian.
Often, the microscopic degrees of freedom of the system are all thermalized,
and therefore \emph{at each spatial point},
the low-energy degrees of freedom are attached to some sort of thermal bath.
If a physical quantity is not protected by conservation laws,
its excitation soon becomes incoherent because of coupling with the thermal bath,
and since the system is thermalized,
the eventual incoherent reduced density matrix for that excitation is a thermal state,
and in other words, the behaviors of physical quantities are well-predicted 
by ordinary equilibrium statistical physics,
like Fermi or Bose distribution.
On the other hand, excitation of conserved quantities can't be thermalized this easily:
usually some sort of relaxation is needed.
This means for thoroughly thermalized systems,
it's likely that the only relevant variables in its low-energy non-equilibrium theory 
are quantities like energy, momentum and particle number,
and all other quantities are to be described by local equilibrium.
This gives us hydrodynamic-like behaviors 
(although this is still far from concrete EOMs, like the Navier-Stokes equations).
On the other hand, when the temperature is low enough,
the most relevant variables are almost always gapless bosonic modes.
These two states of matter are illustrated in Fermi liquids:
in fully thermalized Fermi liquid, we expect to see hydrodynamic behaviors,
where the primary excitation in the system is ordinary sound wave,
while in zero-temperature Fermi liquid,
we expect to see gapless bosonic modes that can be derived within zero-temperature quantum mechanics,
which is just the zero sound.

The spectrum of first sound 
is connected to the spectrum of zero sound:
the zero sound and the first sound can be derived 
in a unified way \cite{khalatnikov1958dispersion},
and we may say first sound is 
merely zero sound with finite temperature correction, 
but this correction is so severe that the qualitative physical picture 
is radically changed.
First sound can be derived with macroscopic conservation equation 
usually used in fluid dynamics,
while this is no longer possible for zero sound;
the two types of sounds also have different dissipation mechanism:
zero sound is damped because of the non-zero collision integral,
while first sound is damped because the collision is not strong enough,
so as a first sound wave propagates,
it excites electron-hole pairs 
and loses energy
\cite{abel1966propagation,belitz2022soft}. 
The sound spectrum of Fermi liquid is therefore summarized in \prettyref{fig:sound-comparison}.

\begin{figure}
    \centering
    
% Gradient Info
  
\tikzset {_d56l9pf75/.code = {\pgfsetadditionalshadetransform{ \pgftransformshift{\pgfpoint{0 bp } { 0 bp }  }  \pgftransformrotate{-146 }  \pgftransformscale{2 }  }}}
\pgfdeclarehorizontalshading{_9togdt637}{150bp}{rgb(0bp)=(1,1,1);
rgb(56.84058598109654bp)=(1,1,1);
rgb(59.8930413382394bp)=(0.61,0.61,0.61);
rgb(62.5bp)=(0.29,0.29,0.29);
rgb(100bp)=(0.29,0.29,0.29)}

% Gradient Info
  
\tikzset {_jbj0scktb/.code = {\pgfsetadditionalshadetransform{ \pgftransformshift{\pgfpoint{0 bp } { 0 bp }  }  \pgftransformrotate{-129 }  \pgftransformscale{2 }  }}}
\pgfdeclarehorizontalshading{_gxn22itp6}{150bp}{rgb(0bp)=(0.29,0.29,0.29);
rgb(37.5bp)=(0.29,0.29,0.29);
rgb(43.7797600882394bp)=(1,1,1);
rgb(100bp)=(1,1,1)}
\tikzset{every picture/.style={line width=0.75pt}} %set default line width to 0.75pt        

\begin{tikzpicture}[x=0.75pt,y=0.75pt,yscale=-1,xscale=1]
%uncomment if require: \path (0,300); %set diagram left start at 0, and has height of 300

%Straight Lines [id:da6006990507896193] 
\draw    (257.92,94.17) -- (325.92,33.1) ;
%Straight Lines [id:da8463599501793553] 
\draw    (103,235) -- (344.17,235) ;
\draw [shift={(346.17,235)}, rotate = 180] [fill={rgb, 255:red, 0; green, 0; blue, 0 }  ][line width=0.08]  [draw opacity=0] (12,-3) -- (0,0) -- (12,3) -- cycle    ;
%Straight Lines [id:da2275445831732943] 
\draw    (103,235) -- (103,21.19) ;
\draw [shift={(103,19.19)}, rotate = 90] [fill={rgb, 255:red, 0; green, 0; blue, 0 }  ][line width=0.08]  [draw opacity=0] (12,-3) -- (0,0) -- (12,3) -- cycle    ;
%Straight Lines [id:da6099852885108679] 
\draw    (103,235) -- (184.26,201.02) ;
%Shape: Polygon Curved [id:ds3933722880021524] 
\draw  [draw opacity=0][shading=_9togdt637,_d56l9pf75] (294.51,202.92) .. controls (242.83,207.29) and (203.28,197.08) .. (165.26,208.96) .. controls (197.68,188.2) and (225.06,176.75) .. (250.35,126.15) .. controls (261.8,137.72) and (295.13,166.87) .. (294.51,202.92) -- cycle ;
%Straight Lines [id:da7618060721275304] 
\draw  [dash pattern={on 0.84pt off 2.51pt}]  (103,235) -- (325.17,141.52) ;
%Shape: Polygon Curved [id:ds9419710356804576] 
\draw  [draw opacity=0][shading=_gxn22itp6,_jbj0scktb] (171.32,89.99) .. controls (221.15,75.65) and (263.94,80.05) .. (296.92,58.92) .. controls (278.92,90.85) and (239.92,111.85) .. (229.57,156.7) .. controls (216.08,147.58) and (177.72,125.47) .. (171.32,89.99) -- cycle ;
%Straight Lines [id:da4615538031472426] 
\draw  [dash pattern={on 0.84pt off 2.51pt}]  (103,235) -- (326.17,32.85) ;
%Curve Lines [id:da7468246422953644] 
\draw  [dash pattern={on 4.5pt off 4.5pt}]  (184.26,201.02) .. controls (286.17,136.19) and (206.17,142.19) .. (291.92,63.64) ;
%Straight Lines [id:da9243960083758465] 
\draw  [dash pattern={on 0.84pt off 2.51pt}]  (104,127) -- (337.17,127) ;

% Text Node
\draw (328.17,35.85) node [anchor=north west][inner sep=0.75pt]   [align=left] {zero sound};
% Text Node
\draw (327.17,144.52) node [anchor=north west][inner sep=0.75pt]   [align=left] {first sound};
% Text Node
\draw (348.17,235) node [anchor=west] [inner sep=0.75pt]    {$q$};
% Text Node
\draw (101,19.19) node [anchor=east] [inner sep=0.75pt]    {$\omega $};
% Text Node
\draw (193,47) node [anchor=north west][inner sep=0.75pt]   [align=left] {collision};
% Text Node
\draw (233,210) node [anchor=north west][inner sep=0.75pt]   [align=left] {lack of collision};
% Text Node
\draw (101,127.09) node [anchor=east] [inner sep=0.75pt]    {$\sim \frac{1}{\tau }$};


\end{tikzpicture}

    \caption{Comparison between zero sound and first sound: 
    they can be seen as one continuous branch 
    on the spectrum, 
    where a large imaginary part is present when $\omega \tau \sim 1$,
    they have different physical pictures.
    Zero sound only exists in the $\omega \tau \gg 1$ region,
    while first sound only exists in the $\omega \tau \ll 1$ region;
    damping of zero sound is due to collision,
    while damping of first sound is due to the fact that
    collision is not strong enough to establish thermal equilibrium.
    First sound is a finite temperature effect.}
    \label{fig:sound-comparison}
\end{figure}

\section{Damping mechanisms}



\section{Charged Fermi liquid and the plasmon}\label{sec:charged-fermi-liquid}

If we try to calculate $f_{\vb*{p} \vb*{p}'}$ by 
comparing \eqref{eq:landau-eq-assumption} and \eqref{eq:simple-fermi-liquid-xi}
in a real condensed matter system,
we immediately find that the Hartree term $V_0$ diverges.
This radically long-range nature of the Hartree term therefore 
is better captured by 
putting the Hartree term to the 
\emph{spatial} dependence of the corrected single-electron energy 
instead of the momentum dependence.
The complete kinetic theory therefore involves three instead of two equations:
the Hartree correction (below $e > 0$)
\begin{equation}
    \laplacian \varphi = \frac{e}{\epsilon_0} 
    \cdot \underbrace{
        \frac{1}{V} \sum_{\vb*{p}, \sigma} \var{n}_{\vb*{p} \sigma}(\vb*{r})
    }_{\rho(\vb*{r}) / (-e)},
\end{equation}
the single-electron energy corrected by 
the short-range part of the self-energy
\begin{equation}
    \varepsilon_{\vb*{p} \sigma}(\vb*{r}) = \varepsilon^0_{\vb*{p}} 
        + \frac{1}{V} \sum_{\vb*{p}', \sigma'} 
        f_{\vb*{p} \vb*{p}' \sigma \sigma'} \var{n}_{\vb*{p}' \sigma'}(\vb*{r}),
\end{equation}
and the quantum Boltzmann equation 
\begin{equation}
    \pdv{{n}_{\vb*{p} \sigma}}{t}  
    + \pdv{\varepsilon_{\vb*{p} \sigma}}{\vb*{p}} \cdot \pdv{{n}_{\vb*{p} \sigma}}{\vb*{r}}
    - \pdv{{n}_{\vb*{p} \sigma}}{\vb*{p}} \cdot \pdv{(\varepsilon_{\vb*{p} \sigma} - e \varphi)}{\vb*{r}} = 0.
\end{equation}
The equation system is known as \emph{Landau-Silin equation}.
Frequently, the $-e \varphi$ term is the \emph{only} term 
considered in a classical picture of charged electrons:
$f_{\vb*{p} \vb*{p}'}$ contains the Fock term,
which does not have a clear classical picture.

The Landau-Silin equation, compared with the Landau equation 
for neutral Fermi liquid, 
introduces an energy gap to the zero sound mode, 
and since this radically changes the properties of the mode 
in the long wavelength limit, 
the established terminology is to name 
this corrected zero sound mode the \emph{plasmon}.
Repeating the linearization procedure used before, we have
\begin{equation}
    \pdv{t} \var{n}_{\vb*{p} \sigma} 
    + \pdv{\varepsilon_{\vb*{p} \sigma}}{\vb*{p}} \cdot \pdv{\var{n}_{\vb*{p} \sigma}}{\vb*{r}}
    - \pdv{{n}^0_{\vb*{p} \sigma}}{\vb*{p}} \cdot \pdv{(\var \varepsilon_{\vb*{p} \sigma} - e \var \varphi)}{\vb*{r}} = 0,
\end{equation}
For simplicity, here we do not attempt to find the full dispersion relation 
but rather constrain ourself to the long wavelength limit, 
where $\var{\varphi}$ dominates the last term.
Ignoring $\var{\varepsilon}_{\vb*{p} \sigma} \propto f$ 
and switching to Fourier space, we have 
(note that the $\vb*{q} \cdot \vb*{v} \var{n}_{\vb*{p} \sigma}$ term 
has to be kept or otherwise transportational behaviors 
are completely eliminated)
\begin{equation}
    - \ii \omega \var{n_{\vb*{p} \sigma}} + \ii \vb*{q} \cdot \vb*{v} \var{n_{\vb*{p} \sigma}} + \pdv{{n}_{\vb*{p} \sigma}}{\vb*{p}} \cdot (- e \vb*{E}) = 0,
    \label{eq:e-to-linear-var-n}
\end{equation}
\begin{equation}
    \ii \vb*{q} \cdot \vb*{E} = \frac{1}{\epsilon_0} (-e) \cdot \frac{1}{V} \sum_{\vb*{p} , \sigma}
    \var{n}_{\vb*{p} \sigma}.
    \label{eq:var-n-to-e}
\end{equation}
Putting \eqref{eq:e-to-linear-var-n} into \eqref{eq:var-n-to-e}, we get 
\begin{equation}
    \vb*{q} \cdot \vb*{E} = \vb*{E}  \cdot \frac{e^2}{\epsilon_0} \cdot \frac{1}{V} \sum_{\vb*{p}, \sigma}
    \frac{1}{\vb*{q} \cdot \vb*{v} - \omega} \pdv{n_{\vb*{p} \sigma}}{\vb*{p}}.
\end{equation}
Note that since 
\[
    \vb*{E} = - \grad{\varphi} = - \ii \vb*{q} \varphi,
\]
$\vb*{q}$ is parallel to $\vb*{E}$, and thus we get 
an implicit expression of $\omega$:
\begin{equation}
    \begin{aligned}
        q &= \vu*{q} \cdot \frac{e^2}{\epsilon_0} \sum_\sigma \int \frac{\dd[3]{\vb*{p}}}{(2\pi)^3} 
        \frac{1}{\vb*{q} \cdot \vb*{v} - \omega} \pdv{n_{\vb*{p} \sigma}}{\vb*{p}} \\
        &= - \frac{e^2}{\epsilon_0} \sum_\sigma \int \frac{\dd[3]{\vb*{p}}}{(2\pi)^3} 
        n_{\vb*{p} \sigma}
        \vu*{q} \cdot \pdv{\vb*{p}} \frac{1}{\vb*{q} \cdot \vb*{v} - \omega} \\
        &= \frac{e^2}{\epsilon_0} \sum_\sigma \int \frac{\dd[3]{\vb*{p}}}{(2\pi)^3} 
        n_{\vb*{p} \sigma}
        \frac{\vu*{q} \cdot \vb*{q}}{m} \frac{1}{(\vb*{q} \cdot \vb*{v} - \omega)^2} ,
    \end{aligned}
\end{equation}
and therefore
\begin{equation}
    1 = \frac{e^2}{m \epsilon_0} \sum_\sigma \int \frac{\dd[3]{\vb*{p}}}{(2\pi)^3} 
    n_{\vb*{p} \sigma}
    \frac{1}{(\vb*{q} \cdot \vb*{v} - \omega)^2}
\end{equation}
Here we have used the relation $\vb*{p} = m \vb*{v}$;
for a generalized condensed matter system this is not correct, 
but frequently what matters is only the dispersion relation around $k_{\text{F}}$,
where we can do parabolic expansion 
and define an effective mass.
We now take the $\vb*{q} \to 0$ limit thoroughly, and we get 
\begin{equation}
    1 = \frac{e^2}{m \epsilon_0} \sum_{\sigma} \int \frac{\dd[3]{\vb*{p}}}{(2\pi)^3} n_{\vb*{p} \sigma}
    \cdot \frac{1}{\omega^2}
    = \frac{e^2}{m \epsilon_0 \omega^2} \cdot n,
\end{equation} 
where $n = N / V$ is the number density of electrons.
This means when $\vb*{q} = 0$, the frequency of zero sound is 
\begin{equation}
    \omega_{\text{p}} = \sqrt{\frac{e^2 n}{m \epsilon_0}}.
\end{equation}


Strictly speaking, the above derivation may not be quantitatively correct, 
since the quantum Boltzmann equation assumes 
a retardation-free self-energy 
and does not work when the frequency is too high, 
and whether at the region of $\omega_{\text{p}}$
the quantum Boltzmann equation is still quantitatively correct 
is at least not universally justified.

It can be seen that plasmon only occurs with a divergent interaction potential.

\section{Excitons}\label{sec:microscopic-bosonic-modes}

Excitons exist in multi-band systems.
When the energy of two bands are different, 
a $\comm*{\varepsilon}{f}$ term should be added to the quantum Boltzmann equation.
In a homogeneous electron gas 
we only have a spin-up band and a spin-down band, 
and we can apply an external magnetic field to create a band gap,
and then a new type of spin wave -- essentially an exciton mode 
between the spin-up band and the spin-down band -- emerges \cite{lifshitz2013statistical}.
This concept can be generalized to other discrete indices, 
like the band index in a realistic condensed matter system.

It can be seen that the quantum Boltzmann equation of electrons 
does not provide us with any approximation scheme on top of TODO:
it is just the gradient expansion version of (TODO).
This does not mean the quantum Boltzmann equation is useless for excitons:
excitons are usually modeled as independent particles themselves
\emph{besides} electrons,
and exciton formation is modeled as 
a three-particle vertex \cite{klimontovich1981quantum},
while 

In this way we can handle exciton-phonon interaction with an acceptable time cost 
and

Of course, the validity of this is only guaranteed 
when the exciton lifetime is long enough, 
exciton distribution is smooth in space, etc. 

\section{Discussion}

This report reviews the quantum Boltzmann equation approach to Fermi liquid.
Despite its semiclassical appearance, 
we show that with appropriate modifications,
the validity of quantum Boltzmann equation 
can be reduced to three conditions:
that we can have a close equation (or system of equation) about 
the single-electron Green function,
that the characteristic length scale of how things change is 
small compared to the atomic length scale, 
and that the peaks in the spectral function are sharp enough.
The last condition blocks strong frequency dependence (i.e. retardation) of $\Sigma$;
such a strong frequency dependence may come from 
a large external driving frequency as well as strong correlation;
capturing frequency-dependent effects of the latter requires 
a more complicated formalism, 
like dynamic mean-field theory (DMFT) \cite{georges1996dynamical}.
To go beyond the small-$\vb*{q}$ conditions,
a single-time (i.e. with respect to $T = (t_1 + t_2) / 2$)
single-electron Green function equation of motion
(usually known as the \emph{Kadanoff-Baym equation}) is needed,
which is equivalent to a single-electron density matrix equation of motion
\cite{attaccalite2011real}
and reduces to the quantum Boltzmann equation with \eqref{eq:g-to-f} 
and gradient expansion.
Neither formalisms may be face problems when the frequency 
(with respect to $T$) is too high;
in this case retardation may be important 
and the complete double-time Green function equation of motion is needed.

Note that we can find counterparts of the conditions 
in the derivation of classical Boltzmann equation:
classical Boltzmann equation can be derived from 
truncating the BBGKY hierarchy,
which corresponds to the close-equation condition;
the validity of classical Boltzmann equation 
also depends on that the characteristic length scale 
of $f(\vb*{r}, \vb*{p} , t)$ and the external driving field 
is small compared with the microscopic scales,
or otherwise the collision integral cannot be localized 
at one spatial site; 
classical Boltzmann equation also breaks 
when the external driving frequency is comparable
to the scattering time.
The main difference between the classical and the quantum case, 
therefore, is whether $f(\vb*{r}, \vb*{p}, t)$ is positive everywhere;
in the classical case this is true, 
while in the quantum case, 
the space of possible $f(\vb*{r}, \vb*{p}, t)$
is just the space of possible Wigner functions. 

When the temperature is moderately high 
so that zero sound is suppressed,
the behaviors of the Fermi liquid 
are close enough to those of an ``ordinary'' liquid, like water.
When the temperature is low enough so that 
the usual Fermi liquid theory 
where collision is ignorable 
begins to work,
the usual fluid dynamics description breaks.
It is possible to redeem a hydrodynamic description in the zero-temperature limit;
the price is now there are infinite conserved quantities in the theory
\cite{Wen2007}. 
We therefore conclude that a Fermi liquid is indeed a liquid: 
in the high temperature limit it is a usual liquid,
and in the low temperature limit it is an exotic one,
but still describable with generalized density modes.
Apart from the sound modes,
the distinction between the $\omega \tau \ll 1$ hydrodynamic region
and the $\omega \tau \gg 1$ collision-less region 
can also be demonstrated by transportational behaviors of electrons
\cite{lucas2015hydrodynamic,lucas2018hydrodynamics,sulpizio2019visualizing}.

\printbibliography

\end{document}