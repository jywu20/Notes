\documentclass[hyperref, a4paper]{article}

\usepackage{geometry}
\usepackage{titling}
\usepackage{titlesec}
% No longer needed, since we will use enumitem package
% \usepackage{paralist}
\usepackage{enumitem}
\usepackage{footnote}
\usepackage{enumerate}
\usepackage{amsmath, amssymb, amsthm}
\usepackage{mathtools}
\usepackage{bbm}
\usepackage{graphicx}
\usepackage{subcaption}
\usepackage{physics}
\usepackage{tensor}
\usepackage{siunitx}
\usepackage[version=4]{mhchem}
\usepackage{tikz}
\usepackage{xcolor}
\usepackage{listings}
\usepackage{autobreak}
\usepackage[ruled, vlined, linesnumbered]{algorithm2e}
\usepackage{nameref,zref-xr}
\zxrsetup{toltxlabel}
\usepackage[sorting=none]{biblatex}
\addbibresource{theory.bib}
\addbibresource{experiments.bib}
\usepackage[colorlinks,unicode]{hyperref} % , linkcolor=black, anchorcolor=black, citecolor=black, urlcolor=black, filecolor=black
\usepackage[most]{tcolorbox}
\usepackage{prettyref}

% Page style
\geometry{left=3.18cm,right=3.18cm,top=2.54cm,bottom=2.54cm}
\titlespacing{\paragraph}{0pt}{1pt}{10pt}[20pt]
\setlength{\droptitle}{-5em}

% More compact lists 
\setlist[itemize]{
    itemindent=17pt, 
    leftmargin=1pt,
    listparindent=\parindent,
    parsep=0pt,
}

% Math operators
\DeclareMathOperator{\timeorder}{\mathcal{T}}
\DeclareMathOperator{\diag}{diag}
\DeclareMathOperator{\legpoly}{P}
\DeclareMathOperator{\primevalue}{P}
\DeclareMathOperator{\sgn}{sgn}
\DeclareMathOperator{\res}{Res}
\newcommand*{\ii}{\mathrm{i}}
\newcommand*{\ee}{\mathrm{e}}
\newcommand*{\const}{\mathrm{const}}
\newcommand*{\suchthat}{\quad \text{s.t.} \quad}
\newcommand*{\argmin}{\arg\min}
\newcommand*{\argmax}{\arg\max}
\newcommand*{\normalorder}[1]{: #1 :}
\newcommand*{\pair}[1]{\langle #1 \rangle}
\newcommand*{\fd}[1]{\mathcal{D} #1}
\DeclareMathOperator{\bigO}{\mathcal{O}}

% TikZ setting
\usetikzlibrary{arrows,shapes,positioning}
\usetikzlibrary{arrows.meta}
\usetikzlibrary{decorations.markings}
\tikzstyle arrowstyle=[scale=1]
\tikzstyle directed=[postaction={decorate,decoration={markings,
    mark=at position .5 with {\arrow[arrowstyle]{stealth}}}}]
\tikzstyle ray=[directed, thick]
\tikzstyle dot=[anchor=base,fill,circle,inner sep=1pt]

% Algorithm setting
% Julia-style code
\SetKwIF{If}{ElseIf}{Else}{if}{}{elseif}{else}{end}
\SetKwFor{For}{for}{}{end}
\SetKwFor{While}{while}{}{end}
\SetKwProg{Function}{function}{}{end}
\SetArgSty{textnormal}

\newcommand*{\concept}[1]{{\textbf{#1}}}

% Embedded codes
\lstset{basicstyle=\ttfamily,
  showstringspaces=false,
  commentstyle=\color{gray},
  keywordstyle=\color{blue}
}

% Reference formatting
\newrefformat{fig}{Fig.~\ref{#1}}
\newcommand*{\term}[1]{\textit{#1}}

% Color boxes
\tcbuselibrary{skins, breakable, theorems}

\newtcbtheorem{infobox}{Box}{
    enhanced,
    boxrule=0pt,
    colback=blue!5,
    colframe=blue!5,
    coltitle=blue!50,
    borderline west={4pt}{0pt}{blue!65},
    sharp corners,
    fonttitle=\bfseries, 
    breakable,
    before upper={\parindent15pt\noindent}}{box}
\newtcbtheorem[use counter from=infobox]{theorybox}{Box}{
    enhanced,
    boxrule=0pt,
    colback=orange!5, 
    colframe=orange!5, 
    coltitle=orange!50,
    borderline west={4pt}{0pt}{orange!65},
    sharp corners,
    fonttitle=\bfseries, 
    breakable,
    before upper={\parindent15pt\noindent}}{box}
\newtcbtheorem[use counter from=infobox]{learnbox}{Box}{
    enhanced,
    boxrule=0pt,
    colback=green!5,
    colframe=green!5,
    coltitle=green!50,
    borderline west={4pt}{0pt}{green!65},
    sharp corners,
    fonttitle=\bfseries, 
    breakable,
    before upper={\parindent15pt\noindent}}{box}


\newenvironment{shelldisplay}{\begin{lstlisting}}{\end{lstlisting}}

\newcommand*{\kB}{k_{\text{B}}}
\newcommand*{\muB}{\mu_{\text{B}}}
\newcommand*{\efermi}{E_{\text{F}}}
\newcommand*{\pfermi}{p_{\text{F}}}
\newcommand*{\vfermi}{v_{\text{F}}}
\newcommand*{\sA}{\text{A}}
\newcommand*{\sB}{\text{B}}
\newcommand*{\Tc}{T_{\text{c}}}
\newcommand*{\hethree}{$^3$He}
\newcommand*{\hefour}{$^4$He}

\title{Bosonic modes in Fermi liquid}
\author{Jinyuan Wu}

\begin{document}

\maketitle

\section{Introduction}

The Fermi liquid theory can be justified by diagrammatic resummation: 
by summing up a certain family of self-energy diagrams
that are believed to be important, 
we get a correction to the electron band dispersion relation,
as well as a finite lifetime.
The electron density-dependent part of the self-energy correction 
is often known as ``forward scattering'',
which has the form of $f_{\vb*{p} \vb*{k}'} \var{n}(\vb*{p}) \var{n}(\vb*{p}')$ 
in the energy functional.
But interaction channels beside forward scattering 
that come from Coulomb interaction do not just disappear;
they are still a part of the Hamiltonian
and will contribute to the specific heat 
when the system is heated up.
Therefore, it can be expected that a real condensed matter system 
that is said to be in a Fermi liquid phase 
contains \emph{more} than electron-like quasiparticles. 

Characterization of the full spectrum of a system is 
generally only possible for exactly solvable systems. 
This report is constrained on bosonic modes in Fermi liquid,%
\footnote{
    There is a terminological confusion here: 
    the term \term{Fermi liquid} may refer to 
    a system whose Hamiltonian is exactly in the shape of 
    Fermi liquid energy functional,
    or it may refer to a system 
    in which the behavior of electron Green function 
    follows the Fermi liquid theory, 
    but may contain other excitations.
    This note uses the latter definition;
    thus the phrase ``a Fermi liquid'' 
    is a shorthand for ``a real-world condensed matter system 
    demonstrating Fermi liquid behaviors in its single-electron part''.
}
or to be specific, on excitations 
for which a quantum is essentially a renormalized electron-hole pair.
In other words, in this report 
we are interested in oscillation modes of 
operators with the shape of $c^\dagger_{\vb*{k} + \vb*{q} / 2} c_{\vb*{k} - \vb*{q} / 2}$.
Three-electron behaviors do exist \cite{patton2003trion,singh2016trion}
but are beyond the scope of this report.

\section{The formalism}

In principle all electro-hole bosonic modes can be found by 
looking at poles of the four-point Green function,
or in other words, 
by diagonalizing the four-point kernel.
This indeed is the usual method in first-principle calculations 
\cite{berkeleygw},
but is not feasible for semi-quantitative analytical purposes. 

\begin{figure}
    \centering
    \tikzset{every picture/.style={line width=0.75pt}} %set default line width to 0.75pt        

\begin{tikzpicture}[x=0.75pt,y=0.75pt,yscale=-1,xscale=1]
%uncomment if require: \path (0,300); %set diagram left start at 0, and has height of 300

%Straight Lines [id:da9882880739712865] 
\draw [color={rgb, 255:red, 0; green, 0; blue, 0 }  ,draw opacity=1 ][line width=0.75]    (60.17,139) -- (76.83,139) ;
\draw [shift={(71.7,139)}, rotate = 180] [fill={rgb, 255:red, 0; green, 0; blue, 0 }  ,fill opacity=1 ][line width=0.08]  [draw opacity=0] (8.04,-3.86) -- (0,0) -- (8.04,3.86) -- (5.34,0) -- cycle    ;
%Straight Lines [id:da42392425232719333] 
\draw  [dash pattern={on 0.84pt off 2.51pt}]  (76.83,118.41) -- (76.83,139) ;
%Shape: Circle [id:dp7049494622344523] 
\draw  [color={rgb, 255:red, 155; green, 155; blue, 155 }  ,draw opacity=1 ][line width=1.5]  (62.07,103.64) .. controls (62.07,95.49) and (68.68,88.87) .. (76.83,88.87) .. controls (84.99,88.87) and (91.6,95.49) .. (91.6,103.64) .. controls (91.6,111.8) and (84.99,118.41) .. (76.83,118.41) .. controls (68.68,118.41) and (62.07,111.8) .. (62.07,103.64) -- cycle ;
%Straight Lines [id:da8083498580412369] 
\draw [color={rgb, 255:red, 155; green, 155; blue, 155 }  ,draw opacity=1 ][line width=1.5]    (90.83,99.52) -- (91.07,107.24) ;
\draw [shift={(91.11,108.38)}, rotate = 268.22] [fill={rgb, 255:red, 155; green, 155; blue, 155 }  ,fill opacity=1 ][line width=0.08]  [draw opacity=0] (9.91,-4.76) -- (0,0) -- (9.91,4.76) -- (6.58,0) -- cycle    ;
%Straight Lines [id:da19225743957281805] 
\draw  [dash pattern={on 0.84pt off 2.51pt}]  (76.83,67.01) -- (76.83,88.87) ;
%Straight Lines [id:da07506393201557948] 
\draw [color={rgb, 255:red, 155; green, 155; blue, 155 }  ,draw opacity=1 ][line width=1.5]    (63.22,109.08) -- (62.07,103.64) ;
\draw [shift={(61.6,101.47)}, rotate = 77.99] [fill={rgb, 255:red, 155; green, 155; blue, 155 }  ,fill opacity=1 ][line width=0.08]  [draw opacity=0] (9.91,-4.76) -- (0,0) -- (9.91,4.76) -- (6.58,0) -- cycle    ;
%Straight Lines [id:da15812365367400316] 
\draw [color={rgb, 255:red, 0; green, 0; blue, 0 }  ,draw opacity=1 ][line width=0.75]    (166.38,87.33) -- (175.41,111.61) ;
\draw [shift={(172.01,102.47)}, rotate = 249.61] [fill={rgb, 255:red, 0; green, 0; blue, 0 }  ,fill opacity=1 ][line width=0.08]  [draw opacity=0] (8.04,-3.86) -- (0,0) -- (8.04,3.86) -- (5.34,0) -- cycle    ;
%Straight Lines [id:da28581830346964154] 
\draw  [dash pattern={on 0.84pt off 2.51pt}]  (210.26,111.57) -- (175.41,111.61) ;
%Straight Lines [id:da12541785389989157] 
\draw [color={rgb, 255:red, 155; green, 155; blue, 155 }  ,draw opacity=1 ][line width=1.5]    (41,206) -- (77.83,206) ;
\draw [shift={(64.42,206)}, rotate = 180] [fill={rgb, 255:red, 155; green, 155; blue, 155 }  ,fill opacity=1 ][line width=0.08]  [draw opacity=0] (9.91,-4.76) -- (0,0) -- (9.91,4.76) -- (6.58,0) -- cycle    ;
%Straight Lines [id:da8921512092815866] 
\draw  [dash pattern={on 0.84pt off 2.51pt}]  (77.83,206) -- (77.83,230.74) ;
%Shape: Arc [id:dp27478538585863044] 
\draw  [draw opacity=0][dash pattern={on 1.69pt off 2.76pt}][line width=1.5]  (43.2,204.81) .. controls (43.2,204.69) and (43.21,204.57) .. (43.21,204.45) .. controls (43.49,185.28) and (59.72,169.98) .. (79.46,170.27) .. controls (99.19,170.56) and (114.96,186.34) .. (114.68,205.51) .. controls (114.68,205.67) and (114.67,205.84) .. (114.67,206) -- (78.94,204.98) -- cycle ; \draw  [dash pattern={on 1.69pt off 2.76pt}][line width=1.5]  (43.2,204.81) .. controls (43.2,204.69) and (43.21,204.57) .. (43.21,204.45) .. controls (43.49,185.28) and (59.72,169.98) .. (79.46,170.27) .. controls (99.19,170.56) and (114.96,186.34) .. (114.68,205.51) .. controls (114.68,205.67) and (114.67,205.84) .. (114.67,206) ;  
%Straight Lines [id:da8668834992558518] 
\draw [color={rgb, 255:red, 0; green, 0; blue, 0 }  ,draw opacity=1 ][line width=0.75]    (21.17,205.57) -- (41.01,205.57) ;
\draw [shift={(34.29,205.57)}, rotate = 180] [fill={rgb, 255:red, 0; green, 0; blue, 0 }  ,fill opacity=1 ][line width=0.08]  [draw opacity=0] (8.04,-3.86) -- (0,0) -- (8.04,3.86) -- (5.34,0) -- cycle    ;
%Straight Lines [id:da20240648481719448] 
\draw [color={rgb, 255:red, 0; green, 0; blue, 0 }  ,draw opacity=1 ][line width=0.75]    (114.67,206) -- (132.5,206) ;
\draw [shift={(126.78,206)}, rotate = 180] [fill={rgb, 255:red, 0; green, 0; blue, 0 }  ,fill opacity=1 ][line width=0.08]  [draw opacity=0] (8.04,-3.86) -- (0,0) -- (8.04,3.86) -- (5.34,0) -- cycle    ;
%Straight Lines [id:da5837921094324088] 
\draw [color={rgb, 255:red, 155; green, 155; blue, 155 }  ,draw opacity=1 ][line width=1.5]    (77.83,206) -- (114.67,206) ;
\draw [shift={(101.25,206)}, rotate = 180] [fill={rgb, 255:red, 155; green, 155; blue, 155 }  ,fill opacity=1 ][line width=0.08]  [draw opacity=0] (9.91,-4.76) -- (0,0) -- (9.91,4.76) -- (6.58,0) -- cycle    ;
%Straight Lines [id:da9594705332247448] 
\draw [color={rgb, 255:red, 0; green, 0; blue, 0 }  ,draw opacity=1 ][line width=0.75]    (76.83,139) -- (93.5,139) ;
\draw [shift={(88.37,139)}, rotate = 180] [fill={rgb, 255:red, 0; green, 0; blue, 0 }  ,fill opacity=1 ][line width=0.08]  [draw opacity=0] (8.04,-3.86) -- (0,0) -- (8.04,3.86) -- (5.34,0) -- cycle    ;
%Flowchart: Summing Junction [id:dp8110382690818376] 
\draw   (80.53,63.31) .. controls (80.53,61.27) and (78.87,59.62) .. (76.83,59.62) .. controls (74.79,59.62) and (73.14,61.27) .. (73.14,63.31) .. controls (73.14,65.35) and (74.79,67.01) .. (76.83,67.01) .. controls (78.87,67.01) and (80.53,65.35) .. (80.53,63.31) -- cycle ; \draw   (79.45,60.7) -- (74.22,65.92) ; \draw   (74.22,60.7) -- (79.45,65.92) ;
%Straight Lines [id:da16604078947596346] 
\draw [color={rgb, 255:red, 0; green, 0; blue, 0 }  ,draw opacity=1 ][line width=0.75]    (175.41,111.61) -- (166.89,135.25) ;
\draw [shift={(170.06,126.44)}, rotate = 289.81] [fill={rgb, 255:red, 0; green, 0; blue, 0 }  ,fill opacity=1 ][line width=0.08]  [draw opacity=0] (8.04,-3.86) -- (0,0) -- (8.04,3.86) -- (5.34,0) -- cycle    ;
%Straight Lines [id:da01479137916014972] 
\draw [color={rgb, 255:red, 155; green, 155; blue, 155 }  ,draw opacity=1 ][line width=1.5]    (220.23,87.29) -- (210.26,111.57) ;
\draw [shift={(213.35,104.06)}, rotate = 292.34] [fill={rgb, 255:red, 155; green, 155; blue, 155 }  ,fill opacity=1 ][line width=0.08]  [draw opacity=0] (9.91,-4.76) -- (0,0) -- (9.91,4.76) -- (6.58,0) -- cycle    ;
%Straight Lines [id:da7480591369957916] 
\draw [color={rgb, 255:red, 155; green, 155; blue, 155 }  ,draw opacity=1 ][line width=1.5]    (210.26,111.57) -- (219.29,135.85) ;
\draw [shift={(216.52,128.4)}, rotate = 249.61] [fill={rgb, 255:red, 155; green, 155; blue, 155 }  ,fill opacity=1 ][line width=0.08]  [draw opacity=0] (9.91,-4.76) -- (0,0) -- (9.91,4.76) -- (6.58,0) -- cycle    ;
%Straight Lines [id:da5196339191405468] 
\draw [color={rgb, 255:red, 0; green, 0; blue, 0 }  ,draw opacity=1 ][line width=0.75]    (171.89,166.23) -- (200.34,166.23) ;
\draw [shift={(189.31,166.23)}, rotate = 180] [fill={rgb, 255:red, 0; green, 0; blue, 0 }  ,fill opacity=1 ][line width=0.08]  [draw opacity=0] (8.04,-3.86) -- (0,0) -- (8.04,3.86) -- (5.34,0) -- cycle    ;
%Straight Lines [id:da8901634782972021] 
\draw [color={rgb, 255:red, 155; green, 155; blue, 155 }  ,draw opacity=1 ][line width=1.5]    (200.34,166.23) -- (229.22,166.23) ;
\draw [shift={(219.78,166.23)}, rotate = 180] [fill={rgb, 255:red, 155; green, 155; blue, 155 }  ,fill opacity=1 ][line width=0.08]  [draw opacity=0] (9.91,-4.76) -- (0,0) -- (9.91,4.76) -- (6.58,0) -- cycle    ;
%Straight Lines [id:da6436116206103029] 
\draw [color={rgb, 255:red, 0; green, 0; blue, 0 }  ,draw opacity=1 ][line width=0.75]    (172.22,205.57) -- (200.67,205.57) ;
\draw [shift={(181.75,205.57)}, rotate = 0] [fill={rgb, 255:red, 0; green, 0; blue, 0 }  ,fill opacity=1 ][line width=0.08]  [draw opacity=0] (8.04,-3.86) -- (0,0) -- (8.04,3.86) -- (5.34,0) -- cycle    ;
%Straight Lines [id:da7570806690081875] 
\draw [color={rgb, 255:red, 155; green, 155; blue, 155 }  ,draw opacity=1 ][line width=1.5]    (200.67,205.57) -- (229.56,205.57) ;
\draw [shift={(208.61,205.57)}, rotate = 0] [fill={rgb, 255:red, 155; green, 155; blue, 155 }  ,fill opacity=1 ][line width=0.08]  [draw opacity=0] (9.91,-4.76) -- (0,0) -- (9.91,4.76) -- (6.58,0) -- cycle    ;
%Straight Lines [id:da722260533343094] 
\draw [line width=1.5]  [dash pattern={on 1.69pt off 2.76pt}]  (200.34,166.23) -- (200.67,205.57) ;
%Flowchart: Summing Junction [id:dp12707351253689114] 
\draw   (81.53,234.44) .. controls (81.53,232.4) and (79.87,230.74) .. (77.83,230.74) .. controls (75.79,230.74) and (74.14,232.4) .. (74.14,234.44) .. controls (74.14,236.48) and (75.79,238.13) .. (77.83,238.13) .. controls (79.87,238.13) and (81.53,236.48) .. (81.53,234.44) -- cycle ; \draw   (80.45,231.83) -- (75.22,237.05) ; \draw   (75.22,231.83) -- (80.45,237.05) ;




\end{tikzpicture}

    \caption{Linear response of two-point Green function gives four-point Green function: 
    the two Feynman diagrams on the left are $GW$ diagrams with the Green function 
    modified by one external field line, 
    representing linear response of the system;
    the linear susceptibility can be obtained 
    formally by erasing the external field line 
    and we get the two Feynman diagrams on the right,
    which describes the most frequently considered 
    two terms in the Bethe–Salpeter equation formalism \cite{berkeleygw}.}
    \label{fig:two-point-response}
\end{figure}

One way to proceed is to notice that 
linear response of two-point Green function 
to an external field coupled to electrons 
gives us four-point Green function
(\prettyref{fig:two-point-response}).
This again is a first-principle approach 
not feasible for analytical studies
\cite{PhysRevB.84.245110},
but further simplification is possible.
For a direct connection between physical observables and the Green function,
we work with the so-called \emph{lesser Green function}
\begin{equation}
    G^<(\vb*{r}_1, t_1, \vb*{r}_2, t_2) = \ii \expval{\psi^\dagger(2) \psi(1)}.
\end{equation}
Wigner transform of the lesser Green function reads
\begin{equation}
    \begin{aligned}
        G^<(\vb*{X}, \vb*{p}, T, \omega) &= 
        \int \dd{t} \ee^{\ii \omega t}
        \int \dd[3]{\vb*{r}} \ee^{- \ii \vb*{p} \cdot \vb*{x}} 
        G^<(T + t / 2, \vb*{X} + \vb*{x} / 2, T - t/2, \vb*{X} - \vb*{x} / 2), \quad \\
        \vb*{x} &= \vb*{x}_1 - \vb*{x}_2, \quad 
        \vb*{X} = \frac{\vb*{x}_1 + \vb*{x}_2}{2}, \quad 
        t = t_1 - t_2, \quad T = \frac{t_1 + t_2}{2}.
    \end{aligned}
\end{equation}
The Wigner transform
defines the position and momentum variables;
note that similar to the single-electron Wigner function,
usual positivity conditions expected in the classical case 
do not hold in general for $G^<(\vb*{X}, \vb*{p})$.
We then introduce two additional assumptions.
The first is the validity of \emph{gradient expansion}:
physical quantities involved in the calculation 
should not have very high order dependence on either $\vb*{X}$ or $\vb*{p}$.
The second is that the quasiparticle picture works well in the system 
so that the peak in the spectral function is sharp enough, and we have 
\begin{equation}
    G^<(\vb*{X}, \vb*{p}, T, \omega) = - 2 \pi \ii \delta(\omega - \xi(\vb*{X}, \vb*{p})) \cdot 
    f(\vb*{X}, \vb*{p}, T),
\end{equation}
where $\xi(\vb*{X}, \vb*{p})$ is the 
single-electron Hamiltonian plus the real part of the self-energy 
and thus is not necessarily diagonal in the momentum space 
and has thus undergone Wigner transform.
The two assumptions are sufficient to lead to the \emph{quantum Boltzmann equation}
\begin{equation}
    \pdv{f}{T} + \grad_{\vb*{p}} \xi \cdot \grad_{\vb*{R}} f 
    - \grad_{\vb*{R}} \xi \cdot \grad_{\vb*{p}} f 
    = \left(\pdv{f}{t}\right)_{\text{c}}, 
\end{equation}
where the collision integral on the right-hand side is decided by Fermi golden rule 
and is 
\begin{equation}
    \left(\pdv{f}{t}\right)_{\text{c}} = 
\end{equation}
Note that the only difference between the quantum Boltzmann equation 
and the classical Boltzmann equation 
is the $(1 \pm f)$ factors coming from fermionic/bosonic statistics.
The gradient expansion condition is intuitively reflected 
by the fact that the collision integral depends on $\vb*{R}$ only;
also note that the imaginary part of the self-energy is ignored 
in the spectral function
but is picked up back to the collision integral.
Below, we replace $\vb*{R}$ by $\vb*{r}$ for the sake of convenience.
The most generalized derivation involves Keldysh field theory 
\cite{RevModPhys.58.323}
and is beyond the scope of this report;
as a proof of concept, 
the collision-free quantum Boltzmann equation can be derived 
using random phase approximation (RPA) 
from the equation of motion of $n_{\vb*{k} \vb*{q}}$:

Classical Boltzmann equation can be derived using 
BBGKY hierarchy,
if we make the assumption that $n$-th order correlation functions ($n > 2$) 
are ignorable; 
here a similar procedure is applied in the quantum region, 
giving us the equation of motion 
of two-point (i.e. single-electron) Green function;
it is the further two conditions that
finally leads to the quantum Boltzmann equation.
The second assumption is by definition satisfied for a Fermi liquid.
The first assumption -- that gradient expansion works -- is at the first glance 
broken in condensed matter systems, 
since the crystal potential has a very small characteristic length scale.
For a single-band problem, however, 
we can manually find a ``position'' operator $\vb*{x}$ as the conjugate variable 
of the lattice momentum $\vb*{k}$,
which represents the center of the wave packet 
and in the coarse-grained macroscopic limit 
appears to be the commonly known position;
this definition of $\vb*{x}$ however depends on the 
shape of the band wave function,
and when there exist several bands, 
this procedure is not longer applicable
(\prettyref{sec:microscopic-bosonic-modes}).
The coverage of the quantum Boltzmann equation therefore 
faces a fundamental constraint in multi-band systems.

\section{Landau kinetic theory of neutral Fermi liquid and zero sound} 

The assumptions made in the last section 
do not impose any constraint on $\Re \Sigma$;
specifically, they do not dictate that $\Re \Sigma$
cannot have explicit dependence on $G(\vb*{r}, \vb*{r}', t, t')$
i.e. $f(\vb*{r}, \vb*{p}, t)$.
We can then insert the Fermi liquid effective energy
\begin{equation}
    \varepsilon_{\vb*{p} \sigma} = \varepsilon^0_{\vb*{p} \sigma} + 
    \frac{1}{V} \sum_{\vb*{p}', \sigma'} f_{\vb*{p} \vb*{p}' \sigma \sigma'} \var{n}_{\vb*{p}' \sigma'}
    \label{eq:landau-eq-assumption}
\end{equation}
into the quantum Boltzmann equation;
the resulting equation system is called \term{Landau equation}.
Since now $\var{n}$ has both $\vb*{p}$ and $\vb*{r}$ dependence,
$\varepsilon_{\vb*{p}}$ also has $\vb*{r}$ dependence.
No spatial dependence is added to $f_{\vb*{p} \vb*{p}'}$, however:
the fact that $n_{\vb*{k} \vb*{q}}$ has a non-zero value when $\vb*{q} \neq 0$
is due to external electromagnetic driving, 
which at the linear level does not change
the momentum conservation condition 
where an interaction line meets with two electron lines.

The inclusion of the Fermi liquid self-energy correction 
immediately leads to an important consequence 
of Fermi liquid: 
that when the temperature is zero and 
no collision is possible for quasiparticles, 
we still have density modes which resemble 
ordinary sound wave in some aspects. 
This mode is known as \emph{zero sound}.

When the temperature is non-zero,
$\tau \propto 1 / T^2$ is finite 
and zero sound faces strong damping 
when its frequency is too slow.
In the low frequency domain, 
where thermal equilibrium is almost always established,
we get ordinary sound or ``first sound''.
The first sound can be derived 
by calculating mechanical properties of the Fermi liquid in question 
and inserting the compressibility 
into $v = \sqrt{\pdv*{p}{\rho}}$ \cite{lifshitz2013statistical}.
This approach assumes the usual framework of hydrodynamics 
works for Fermi liquid; 
a direct verification can be found in \cite{belitz2022soft}.
The spectrum of first sound 
is connected to the spectrum of zero sound:
the zero sound and the first sound can be derived 
in a unified way \cite{khalatnikov1958dispersion},
and we may say first sound is 
merely zero sound with finite temperature correction, 
but this correction is so severe that the qualitative physical picture 
is radically changed.
First sound can be derived with macroscopic conservation equation 
usually used in fluid dynamics,
while this is no longer possible for zero sound;
the two types of sounds also have different dissipation mechanism:
zero sound is damped because of the non-zero collision integral,
while first sound is damped because the collision is not strong enough,
so as a first sound wave propagates,
it excites electron-hole pairs 
and loses energy
\cite{abel1966propagation,belitz2022soft}. 
The sound spectrum of Fermi liquid is therefore summarized in \prettyref{fig:sound-comparison}.

\begin{figure}
    \centering
    
% Gradient Info
  
\tikzset {_d56l9pf75/.code = {\pgfsetadditionalshadetransform{ \pgftransformshift{\pgfpoint{0 bp } { 0 bp }  }  \pgftransformrotate{-146 }  \pgftransformscale{2 }  }}}
\pgfdeclarehorizontalshading{_9togdt637}{150bp}{rgb(0bp)=(1,1,1);
rgb(56.84058598109654bp)=(1,1,1);
rgb(59.8930413382394bp)=(0.61,0.61,0.61);
rgb(62.5bp)=(0.29,0.29,0.29);
rgb(100bp)=(0.29,0.29,0.29)}

% Gradient Info
  
\tikzset {_jbj0scktb/.code = {\pgfsetadditionalshadetransform{ \pgftransformshift{\pgfpoint{0 bp } { 0 bp }  }  \pgftransformrotate{-129 }  \pgftransformscale{2 }  }}}
\pgfdeclarehorizontalshading{_gxn22itp6}{150bp}{rgb(0bp)=(0.29,0.29,0.29);
rgb(37.5bp)=(0.29,0.29,0.29);
rgb(43.7797600882394bp)=(1,1,1);
rgb(100bp)=(1,1,1)}
\tikzset{every picture/.style={line width=0.75pt}} %set default line width to 0.75pt        

\begin{tikzpicture}[x=0.75pt,y=0.75pt,yscale=-1,xscale=1]
%uncomment if require: \path (0,300); %set diagram left start at 0, and has height of 300

%Straight Lines [id:da6006990507896193] 
\draw    (257.92,94.17) -- (325.92,33.1) ;
%Straight Lines [id:da8463599501793553] 
\draw    (103,235) -- (344.17,235) ;
\draw [shift={(346.17,235)}, rotate = 180] [fill={rgb, 255:red, 0; green, 0; blue, 0 }  ][line width=0.08]  [draw opacity=0] (12,-3) -- (0,0) -- (12,3) -- cycle    ;
%Straight Lines [id:da2275445831732943] 
\draw    (103,235) -- (103,21.19) ;
\draw [shift={(103,19.19)}, rotate = 90] [fill={rgb, 255:red, 0; green, 0; blue, 0 }  ][line width=0.08]  [draw opacity=0] (12,-3) -- (0,0) -- (12,3) -- cycle    ;
%Straight Lines [id:da6099852885108679] 
\draw    (103,235) -- (184.26,201.02) ;
%Shape: Polygon Curved [id:ds3933722880021524] 
\draw  [draw opacity=0][shading=_9togdt637,_d56l9pf75] (294.51,202.92) .. controls (242.83,207.29) and (203.28,197.08) .. (165.26,208.96) .. controls (197.68,188.2) and (225.06,176.75) .. (250.35,126.15) .. controls (261.8,137.72) and (295.13,166.87) .. (294.51,202.92) -- cycle ;
%Straight Lines [id:da7618060721275304] 
\draw  [dash pattern={on 0.84pt off 2.51pt}]  (103,235) -- (325.17,141.52) ;
%Shape: Polygon Curved [id:ds9419710356804576] 
\draw  [draw opacity=0][shading=_gxn22itp6,_jbj0scktb] (171.32,89.99) .. controls (221.15,75.65) and (263.94,80.05) .. (296.92,58.92) .. controls (278.92,90.85) and (239.92,111.85) .. (229.57,156.7) .. controls (216.08,147.58) and (177.72,125.47) .. (171.32,89.99) -- cycle ;
%Straight Lines [id:da4615538031472426] 
\draw  [dash pattern={on 0.84pt off 2.51pt}]  (103,235) -- (326.17,32.85) ;
%Curve Lines [id:da7468246422953644] 
\draw  [dash pattern={on 4.5pt off 4.5pt}]  (184.26,201.02) .. controls (286.17,136.19) and (206.17,142.19) .. (291.92,63.64) ;
%Straight Lines [id:da9243960083758465] 
\draw  [dash pattern={on 0.84pt off 2.51pt}]  (104,127) -- (337.17,127) ;

% Text Node
\draw (328.17,35.85) node [anchor=north west][inner sep=0.75pt]   [align=left] {zero sound};
% Text Node
\draw (327.17,144.52) node [anchor=north west][inner sep=0.75pt]   [align=left] {first sound};
% Text Node
\draw (348.17,235) node [anchor=west] [inner sep=0.75pt]    {$q$};
% Text Node
\draw (101,19.19) node [anchor=east] [inner sep=0.75pt]    {$\omega $};
% Text Node
\draw (193,47) node [anchor=north west][inner sep=0.75pt]   [align=left] {collision};
% Text Node
\draw (233,210) node [anchor=north west][inner sep=0.75pt]   [align=left] {lack of collision};
% Text Node
\draw (101,127.09) node [anchor=east] [inner sep=0.75pt]    {$\sim \frac{1}{\tau }$};


\end{tikzpicture}

    \caption{Comparison between zero sound and first sound: 
    they can be seen as one continuous branch 
    on the spectrum, 
    where a large imaginary part is present when $\omega \tau \sim 1$,
    they have different physical pictures.
    Zero sound only exists in the $\omega \tau \gg 1$ region,
    while first sound only exists in the $\omega \tau \ll 1$ region;
    damping of zero sound is due to collision,
    while damping of first sound is due to the fact that
    collision is not strong enough to establish thermal equilibrium.
    First sound is a finite temperature effect.}
    \label{fig:sound-comparison}
\end{figure}

\section{Damping mechanisms}



\section{Charged Fermi liquid and the plasmon}

One more mechanism 

\section{Microscopic bosonic modes beyond the Landau equation}\label{sec:microscopic-bosonic-modes}

Not all bosonic modes can be obtained 
by observing oscillation modes of quantum Boltzmann equation,
since the latter only works for 
excitations with a large characteristic length scale 
in the $\vb*{x}_1 - \vb*{x}_2$ variable.
This constraint has two consequences.

First, it requires that $\vb*{q}$ is small enough.
When a uniform electric field is applied, 
it influences $\vb*{x}_1 - \vb*{x}_2$, not $(\vb*{x}_1 + \vb*{x}_2) / 2$;
accordingly, it influences $\vb*{k}$, 
which is now to be understood as the \emph{relative} momentum 
between the electron and the hole 
(absence of an electron with momentum $\vb*{k}$ 
is equivalent to existence of a hole with momentum $- \vb*{k}$).
If, however, the electric field has very strong spatial variance, 
the electron at $\vb*{r}_1$ feels a different force 
from that felt at $\vb*{r}_2$,
and the electron-hole pair gets driven as a whole,
giving a non-zero value to $\vb*{q}$.
Thus the small-$\vb*{q}$ condition is equivalent to 
the condition that the wave length of the driving electric field 
should be small compared with the atomic length scale.

Second, it rules out the possibility 
for the naive quantum Boltzmann equation to faithfully represent 
inter-band phenomena.
In the multi-band case the band Hamiltonians 
$H_1$ and $H_2$ (TODO: ref) should be compared directly,
both of which have atomic characteristic length scale 
and gradience expansion fails.
Another perspective to see the infeasibility to have 
a semi-classical picture for a multi-band problem 
is by directly observing the equation 
governing exciton energy,
which contains $\varepsilon^{\text{c}}_{\vb*{k} + \vb*{q}} - \varepsilon^{\text{v}}_{\vb*{k}}$.
Should the two energies be both from a single band, 
gradience expansion is possible when $\vb*{q}$ is small, 
which leads to the $\pdv{\varepsilon_{\vb*{k}}}{\vb*{k}} \pdv{n}{\vb*{r}}$ term;
for an exciton however this is not the case.
Thus the quantum Boltzmann equation outlined above 
is unable to describe an exciton mode.
This does not mean the quantum Boltzmann equation is useless for excitons:
excitons are usually modeled as independent particles themselves
\emph{besides} electrons,
and exciton formation is modeled as 
a three-particle vertex \cite{klimontovich1981quantum},
while 

\cite{pines2018theory}.
We may say zero sound/plasmon is a \emph{hydrodynamic} mode,


\section{Conclusion}

\printbibliography

\end{document}