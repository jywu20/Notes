\documentclass[hyperref, a4paper]{article}

\usepackage{geometry}
\usepackage{titling}
\usepackage{titlesec}
% No longer needed, since we will use enumitem package
% \usepackage{paralist}
\usepackage{enumitem}
\usepackage{footnote}
\usepackage{enumerate}
\usepackage{amsmath, amssymb, amsthm}
\usepackage{mathtools}
\usepackage{bbm}
\usepackage{graphicx}
\usepackage{subcaption}
\usepackage{physics}
\usepackage{tensor}
\usepackage{siunitx}
\usepackage[version=4]{mhchem}
\usepackage{tikz}
\usepackage{xcolor}
\usepackage{listings}
\usepackage{autobreak}
\usepackage[ruled, vlined, linesnumbered]{algorithm2e}
\usepackage{nameref,zref-xr}
\zxrsetup{toltxlabel}
\usepackage[sorting=none]{biblatex}
\addbibresource{theory.bib}
\usepackage[colorlinks,unicode]{hyperref} % , linkcolor=black, anchorcolor=black, citecolor=black, urlcolor=black, filecolor=black
\usepackage[most]{tcolorbox}
\usepackage{prettyref}

% Page style
\geometry{left=3.18cm,right=3.18cm,top=2.54cm,bottom=2.54cm}
\titlespacing{\paragraph}{0pt}{1pt}{10pt}[20pt]
\setlength{\droptitle}{-5em}

% More compact lists 
\setlist[itemize]{
    itemindent=17pt, 
    leftmargin=1pt,
    listparindent=\parindent,
    parsep=0pt,
}

% Math operators
\DeclareMathOperator{\timeorder}{\mathcal{T}}
\DeclareMathOperator{\diag}{diag}
\DeclareMathOperator{\legpoly}{P}
\DeclareMathOperator{\primevalue}{P}
\DeclareMathOperator{\sgn}{sgn}
\DeclareMathOperator{\res}{Res}
\newcommand*{\ii}{\mathrm{i}}
\newcommand*{\ee}{\mathrm{e}}
\newcommand*{\const}{\mathrm{const}}
\newcommand*{\suchthat}{\quad \text{s.t.} \quad}
\newcommand*{\argmin}{\arg\min}
\newcommand*{\argmax}{\arg\max}
\newcommand*{\normalorder}[1]{: #1 :}
\newcommand*{\pair}[1]{\langle #1 \rangle}
\newcommand*{\fd}[1]{\mathcal{D} #1}
\DeclareMathOperator{\bigO}{\mathcal{O}}

% TikZ setting
\usetikzlibrary{arrows,shapes,positioning}
\usetikzlibrary{arrows.meta}
\usetikzlibrary{decorations.markings}
\tikzstyle arrowstyle=[scale=1]
\tikzstyle directed=[postaction={decorate,decoration={markings,
    mark=at position .5 with {\arrow[arrowstyle]{stealth}}}}]
\tikzstyle ray=[directed, thick]
\tikzstyle dot=[anchor=base,fill,circle,inner sep=1pt]

% Algorithm setting
% Julia-style code
\SetKwIF{If}{ElseIf}{Else}{if}{}{elseif}{else}{end}
\SetKwFor{For}{for}{}{end}
\SetKwFor{While}{while}{}{end}
\SetKwProg{Function}{function}{}{end}
\SetArgSty{textnormal}

\newcommand*{\concept}[1]{{\textbf{#1}}}

% Embedded codes
\lstset{basicstyle=\ttfamily,
  showstringspaces=false,
  commentstyle=\color{gray},
  keywordstyle=\color{blue}
}

% Reference formatting
\newrefformat{fig}{Fig.~\ref{#1}}
\newcommand*{\term}[1]{\textit{#1}}

% Color boxes
\tcbuselibrary{skins, breakable, theorems}

\newtcbtheorem{infobox}{Box}{
    enhanced,
    boxrule=0pt,
    colback=blue!5,
    colframe=blue!5,
    coltitle=blue!50,
    borderline west={4pt}{0pt}{blue!65},
    sharp corners,
    fonttitle=\bfseries, 
    breakable,
    before upper={\parindent15pt\noindent}}{box}
\newtcbtheorem[use counter from=infobox]{theorybox}{Box}{
    enhanced,
    boxrule=0pt,
    colback=orange!5, 
    colframe=orange!5, 
    coltitle=orange!50,
    borderline west={4pt}{0pt}{orange!65},
    sharp corners,
    fonttitle=\bfseries, 
    breakable,
    before upper={\parindent15pt\noindent}}{box}
\newtcbtheorem[use counter from=infobox]{learnbox}{Box}{
    enhanced,
    boxrule=0pt,
    colback=green!5,
    colframe=green!5,
    coltitle=green!50,
    borderline west={4pt}{0pt}{green!65},
    sharp corners,
    fonttitle=\bfseries, 
    breakable,
    before upper={\parindent15pt\noindent}}{box}


\newenvironment{shelldisplay}{\begin{lstlisting}}{\end{lstlisting}}

\newcommand*{\kB}{k_{\text{B}}}
\newcommand*{\muB}{\mu_{\text{B}}}
\newcommand*{\efermi}{E_{\text{F}}}
\newcommand*{\pfermi}{p_{\text{F}}}
\newcommand*{\vfermi}{v_{\text{F}}}
\newcommand*{\sA}{\text{A}}
\newcommand*{\sB}{\text{B}}
\newcommand*{\Tc}{T_{\text{c}}}
\newcommand*{\hethree}{$^3$He}
\newcommand*{\hefour}{$^4$He}

\title{Bosonic modes in Fermi liquid}
\author{Jinyuan Wu}

\begin{document}

\maketitle

\section{Introduction}

The Fermi liquid theory can be justified by diagrammatic resummation: 

But interaction channels beside forward scattering 
that come from Coulomb interaction do not just disappear;
they are still a part of the Hamiltonian
and will contribute to the specific heat 
when the system is heated up.
Thus, it can be expected that a real condensed matter system 
that is said to be in a Fermi liquid phase 
contains more than electron-like quasiparticles. 

Characterization of the full spectrum of a system is 
generally only possible for exactly solvable systems. 
This report is constrained on bosonic modes in Fermi liquid,%
\footnote{
    There is a terminological confusion here: 
    the term \term{Fermi liquid} may refer to 
    a system whose Hamiltonian is exactly in the shape of 
    Fermi liquid energy functional,
    or it may refer to a system 
    in which the behavior of electron Green function 
    follows the Fermi liquid theory, 
    but may contain other excitations.
    This note uses the latter definition;
    thus the phrase ``a Fermi liquid'' 
    is a shorthand for ``a real-world condensed matter system 
    demonstrating Fermi liquid behaviors in its single-electron part''.
}
or to be specific, on excitations that are oscillation modes of 
operators with the shape of $c^\dagger_{\vb*{k} + \vb*{q} / 2} c_{\vb*{k} - \vb*{q} / 2}$.
Three-electron behaviors do exist TODO: trion 
but is beyond the scope of this report.

\section{Landau kinetic equation of neutral Fermi liquid}


Boltzmann equation can be derived using 

The most generalized derivation involves Keldysh field theory 
and is beyond the scope of this note; 
we only emphasize here that the derivation 
depends on gradience expansion 
and that the quasiparticle picture works well in the system 
so that the spectral function is approximately 
\begin{equation}
    A(\vb*{k}, \omega) = \delta(\omega - \Re \Sigma(\vb*{k}, \omega)),
\end{equation}
where the imaginary part of the self-energy is ignored 
in the spectral function,
but appears in the collision integral on the RHS.
The second assumption is by definition satisfied with a Fermi liquid;
the first assumption assumes everything in the system changes slowly 
in the $\vb*{x}_1 - \vb*{x}_2$ direction 
and imposes a fundamental constraint 
on the coverage of the quantum Boltzmann equation
(\prettyref{sec:microscopic-bosonic-modes}). 

It should be noted that the aforementioned procedure 
does not impose any constraint on $\Re \Sigma$;
specifically, it does not dictate that $\Re \Sigma$
cannot have explicit dependence on $G(\vb*{r}, \vb*{r}', t, t')$, 
and hence $f(\vb*{r}, \vb*{p}, t)$.
We can then insert 
\begin{equation}
    \varepsilon_{\vb*{p} \sigma} = \varepsilon^0_{\vb*{p} \sigma} + 
    \frac{1}{V} \sum_{\vb*{p}', \sigma'} f_{\vb*{p} \vb*{p}' \sigma \sigma'} \var{n}_{\vb*{p}' \sigma'}
    \label{eq:landau-eq-assumption}
\end{equation}
into the quantum Boltzmann equation;
the resulting equation system is called \term{Landau equation}.
Note that in order to get \eqref{eq:landau-eq-assumption},

In the 

When the temperature is non-zero,
$\tau \propto 1 / T^2$ is finite 
and zero sound faces strong damping 
when its frequency is too slow.
However, in the low frequency region another bosonic mode is possible:
ordinary sound or ``first sound'',
which is a density 

It should be noted that the spectrum of first sound 
is connected to the spectrum of zero sound:
In this sense we may say first sound is 
zero sound with finite temperature correction, 
but this correction is so severe that the qualitative physical picture 
is radically changed. 

\section{Damping mechanisms}



\section{Charged Fermi liquid and the plasmon}

\section{Microscopic bosonic modes beyond the Landau equation}\label{sec:microscopic-bosonic-modes}

Not all bosonic modes can be obtained 
by observing oscillation modes of quantum Boltzmann equation,
since the latter only works for 
excitations with a large characteristic length scale 
in the $\vb*{x}_1 - \vb*{x}_2$ variable.
This constraint has two indications.
First, it requires that $\vb*{q}$ is small enough.%
\footnote{
    When a uniform electric field is applied, 
    it influences $\vb*{x}_1 - \vb*{x}_2$, not $(\vb*{x}_1 + \vb*{x}_2) / 2$;
    accordingly, it influences $\vb*{k}$, 
    which is now to be understood as the \emph{relative} momentum 
    between the electron and the hole 
    (absence of an electron with momentum $\vb*{k}$ 
    is equivalent to existence of a hole with momentum $- \vb*{k}$).
    If, however, the electric field has very strong spatial variance, 
    the electron at $\vb*{r}_1$ feels a $\vb$
}
Second, it rules out the possibility to 
Now consider an exiton in a homogeneous electron liquid;
the structure of the exciton follows the same equations 
as those governing the hydrogen atom,
and the characteristic length scale between the electron and the hole 
is of atomic magnitude,
which breaks the ``slow variation in $\vb*{x}_1 - \vb*{x}_2$'' condition 
in the gradience expansion step used to derive the quantum Boltzmann equation,
and hence cannot be captured appropriately by the latter 
\cite{pines2018theory}.
We may say zero sound/plasmon is a \emph{hydrodynamic} mode,

The existence of 

Excitons, if considered in a kinetic theory, 
are usually modeled as independent particles themselves besides electrons,
and exciton formation is modeled as 
a three-particle vertex \cite{klimontovich1981quantum},
while 

\section{Conclusion}

\printbibliography

\end{document}