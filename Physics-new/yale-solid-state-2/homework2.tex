\documentclass[hyperref, a4paper]{article}

\usepackage{geometry}
\usepackage{titling}
\usepackage{titlesec}
% No longer needed, since we will use enumitem package
% \usepackage{paralist}
\usepackage{enumitem}
\usepackage{footnote}
\usepackage{amsmath, amssymb, amsthm}
\usepackage{mathtools}
\usepackage{bbm}
\usepackage{graphicx}
\usepackage{subcaption}
\usepackage{physics}
\usepackage{tensor}
\usepackage{siunitx}
\usepackage[version=4]{mhchem}
\usepackage{tikz}
\usepackage{xcolor}
\usepackage{listings}
\usepackage{autobreak}
\usepackage[ruled, vlined, linesnumbered]{algorithm2e}
\usepackage{nameref,zref-xr}
\zxrsetup{toltxlabel}
\usepackage[backend=bibtex]{biblatex}
\addbibresource{data.bib}
\addbibresource{experiments.bib}
\addbibresource{theory.bib}
\usepackage[colorlinks,unicode]{hyperref} % , linkcolor=black, anchorcolor=black, citecolor=black, urlcolor=black, filecolor=black
\usepackage[most]{tcolorbox}
\usepackage{prettyref}

% Page style
\geometry{left=3.18cm,right=3.18cm,top=2.54cm,bottom=2.54cm}
\titlespacing{\paragraph}{0pt}{1pt}{10pt}[20pt]
\setlength{\droptitle}{-5em}

% More compact lists 
\setlist[itemize]{
    itemindent=17pt, 
    leftmargin=1pt,
    listparindent=\parindent,
    parsep=0pt,
}

% Math operators
\DeclareMathOperator{\timeorder}{\mathcal{T}}
\DeclareMathOperator{\diag}{diag}
\DeclareMathOperator{\legpoly}{P}
\DeclareMathOperator{\primevalue}{P}
\DeclareMathOperator{\sgn}{sgn}
\DeclareMathOperator{\res}{Res}
\newcommand*{\ii}{\mathrm{i}}
\newcommand*{\ee}{\mathrm{e}}
\newcommand*{\const}{\mathrm{const}}
\newcommand*{\suchthat}{\quad \text{s.t.} \quad}
\newcommand*{\argmin}{\arg\min}
\newcommand*{\argmax}{\arg\max}
\newcommand*{\normalorder}[1]{: #1 :}
\newcommand*{\pair}[1]{\langle #1 \rangle}
\newcommand*{\fd}[1]{\mathcal{D} #1}
\DeclareMathOperator{\bigO}{\mathcal{O}}

% TikZ setting
\usetikzlibrary{arrows,shapes,positioning}
\usetikzlibrary{arrows.meta}
\usetikzlibrary{decorations.markings}
\tikzstyle arrowstyle=[scale=1]
\tikzstyle directed=[postaction={decorate,decoration={markings,
    mark=at position .5 with {\arrow[arrowstyle]{stealth}}}}]
\tikzstyle ray=[directed, thick]
\tikzstyle dot=[anchor=base,fill,circle,inner sep=1pt]

% Algorithm setting
% Julia-style code
\SetKwIF{If}{ElseIf}{Else}{if}{}{elseif}{else}{end}
\SetKwFor{For}{for}{}{end}
\SetKwFor{While}{while}{}{end}
\SetKwProg{Function}{function}{}{end}
\SetArgSty{textnormal}

\newcommand*{\concept}[1]{{\textbf{#1}}}

% Embedded codes
\lstset{basicstyle=\ttfamily,
  showstringspaces=false,
  commentstyle=\color{gray},
  keywordstyle=\color{blue}
}

% Reference formatting
\newrefformat{fig}{Fig.~\ref{#1}}
\newcommand*{\term}[1]{\textit{#1}}

% Color boxes
\tcbuselibrary{skins, breakable, theorems}

\newtcbtheorem[number within=section]{infobox}{Box}{
    enhanced,
    boxrule=0pt,
    colback=blue!5,
    colframe=blue!5,
    coltitle=blue!50,
    borderline west={4pt}{0pt}{blue!65},
    sharp corners,
    fonttitle=\bfseries, 
    breakable,
    before upper={\parindent15pt\noindent}}{box}
\newtcbtheorem[number within=section, use counter from=infobox]{theorybox}{Box}{
    enhanced,
    boxrule=0pt,
    colback=orange!5, 
    colframe=orange!5, 
    coltitle=orange!50,
    borderline west={4pt}{0pt}{orange!65},
    sharp corners,
    fonttitle=\bfseries, 
    breakable,
    before upper={\parindent15pt\noindent}}{box}
\newtcbtheorem[number within=section, use counter from=infobox]{learnbox}{Box}{
    enhanced,
    boxrule=0pt,
    colback=green!5,
    colframe=green!5,
    coltitle=green!50,
    borderline west={4pt}{0pt}{green!65},
    sharp corners,
    fonttitle=\bfseries, 
    breakable,
    before upper={\parindent15pt\noindent}}{box}

% Displaying texts in bookmarkers

\pdfstringdefDisableCommands{%
  \def\\{}%
  \def\ce#1{<#1>}%
}

\pdfstringdefDisableCommands{%
  \def\texttt#1{<#1>}%
  \def\mathbb#1{#1}%
}
\pdfstringdefDisableCommands{\def\eqref#1{(\ref{#1})}}

\makeatletter
\pdfstringdefDisableCommands{\let\HyPsd@CatcodeWarning\@gobble}
\makeatother

\newenvironment{shelldisplay}{\begin{lstlisting}}{\end{lstlisting}}

\newcommand*{\efermi}{E_{\text{F}}}
\newcommand*{\sA}{\text{A}}
\newcommand*{\sB}{\text{B}}
\newcommand*{\Tc}{T_{\text{c}}}
\newcommand*{\muB}{\mu_{\text{B}}}
\newcommand*{\kB}{k_{\text{B}}}

\title{Homework 2}
\author{Jinyuan Wu}

\begin{document}

\maketitle

\section{Quasiparticle weight in Landau Fermi liquid}

\subsection{Quasiparticle weight in \ce{Na}}

From Table 2.3 in A\&M, in \ce{Na}, 
$m^* / m = 1.3$, 
so $Z = m / m^* = 0.77$.
Here $m^*$ is obtained by thermodynamic measurement:
the Sommerfield model tells us 
\begin{equation}
    C_V = \frac{\pi^2}{2} \frac{\kB T}{\efermi} n \kB, \quad 
    \efermi = \frac{\hbar^2}{2m} \left(\frac{3 \pi^2 N}{V}\right)^{3/2},
\end{equation}
and therefore 
\begin{equation}
    \frac{C_{V, \text{measured}}}{C_{V, \text{free electron}}} = \frac{m^*}{m}.
\end{equation}

\subsection{Direct observation of occupation discontinuity}

From \cite{huotari2010momentum}, 
a direct measurement of the electron occupation 
tells us the experimental result for \ce{Na}: $Z = 0.58$.
This is much smaller than the value found in the thermodynamic measurement.


\section{Exotic phenomena in a Landau Fermi liquid}

\subsection{Heavy fermion systems}

Signatures of heavy fermion materials include 
a low-temperature heat capacity that is, say, 1000 times 
of the free electron heat capacity,
low conductivity, 
and flat bands that come across the Fermi energy.


\subsection{Zero sound}

Zero sound is a 

the interaction between particles has 
a short characteristic length scale in the real space,
which means the corresponding form in the momentum space 
doesn't show strong dependence on the exchanged momentum $\vb*{q}$.
Thus, after renormalization, the interaction between $\var{n}_{\vb*{p}}(\vb*{r})$
and $\var{n}_{\vb*{p}'} (\vb*{r}')$
-- essentially Wigner transforms of $c^\dagger_{\vb*{p} + \vb*{q}} c_{\vb*{p}}$ --
has no large differences with 
$f_{\vb*{p} \vb*{p}'} \var{n}_{\vb*{p}} \var{n}_{\vb*{p}'}$,
because $\vb*{q}$ dependence is not important when $\vb*{q}$ is small 
(or in other words, when the $\vb*{r}$-dependence of $\var{n_{\vb*{p}}}(\vb*{r})$
is smooth enough, which is always required if we want a reasonable Boltzmann equation).
This explains why when deriving the kinetic equation of Fermi liquid, 
we just insert $\var{n}_{\vb*{p}}(\vb*{r})$ in place of $\var{n}_{\vb*{p}}$:
this works only when the interaction is short-range
\cite{pines2018theory}.

\begin{figure}
    \centering
    \begin{tikzpicture}[x=0.75pt,y=0.75pt,yscale=-1,xscale=1]
    %uncomment if require: \path (0,365); %set diagram left start at 0, and has height of 365
    
    %Straight Lines [id:da6872150568051554] 
    \draw    (84.17,167) -- (100,167) ;
    %Shape: Square [id:dp5922961707036727] 
    \draw  [fill={rgb, 255:red, 200; green, 200; blue, 200 }  ,fill opacity=1 ] (100,122.19) -- (144.81,122.19) -- (144.81,167) -- (100,167) -- cycle ;
    %Straight Lines [id:da9892594970830582] 
    \draw    (144.81,167) -- (160.65,167) ;
    %Curve Lines [id:da38843132612978226] 
    \draw    (100,122.19) .. controls (109.83,99.19) and (132.21,96.94) .. (144.81,122.19) ;
    %Straight Lines [id:da18951246712734915] 
    \draw    (127.5,104.5) ;
    \draw [shift={(127.5,104.5)}, rotate = 180] [fill={rgb, 255:red, 0; green, 0; blue, 0 }  ][line width=0.08]  [draw opacity=0] (8.93,-4.29) -- (0,0) -- (8.93,4.29) -- (5.93,0) -- cycle    ;
    %Shape: Square [id:dp8684008123124931] 
    \draw  [fill={rgb, 255:red, 200; green, 200; blue, 200 }  ,fill opacity=1 ] (254,122.19) -- (298.81,122.19) -- (298.81,167) -- (254,167) -- cycle ;
    %Straight Lines [id:da6972654176001563] 
    \draw    (238.17,167) -- (254,167) ;
    %Straight Lines [id:da8746066105625017] 
    \draw    (298.81,167) -- (314.65,167) ;
    %Straight Lines [id:da8467825886053204] 
    \draw    (238.17,122.19) -- (254,122.19) ;
    %Straight Lines [id:da8409010978593827] 
    \draw    (298.81,122.19) -- (314.65,122.19) ;
    %Straight Lines [id:da04010074528608665] 
    \draw  [dash pattern={on 0.84pt off 2.51pt}]  (129,237.19) -- (129,282) ;
    %Straight Lines [id:da8075296669445466] 
    \draw    (113.17,282) -- (129,282) ;
    %Straight Lines [id:da4048862533630819] 
    \draw    (113.17,237.19) -- (129,237.19) ;
    %Straight Lines [id:da7548454697480842] 
    \draw    (129,237.19) -- (144.83,237.19) ;
    %Straight Lines [id:da9305669317129783] 
    \draw    (129,282) -- (144.83,282) ;
    %Straight Lines [id:da19648108066709624] 
    \draw    (241,249.45) -- (253.07,261.52) ;
    %Straight Lines [id:da43384207061841784] 
    \draw    (253.07,261.52) -- (241,273.59) ;
    %Straight Lines [id:da5593751281860175] 
    \draw    (302.98,249.45) -- (290.91,261.52) ;
    %Straight Lines [id:da7708264282935833] 
    \draw    (290.91,261.52) -- (302.98,273.59) ;
    %Straight Lines [id:da2679746724418741] 
    \draw  [dash pattern={on 0.84pt off 2.51pt}]  (253.07,261.52) -- (290.91,261.52) ;
    
    % Text Node
    \draw (82.17,167) node [anchor=east] [inner sep=0.75pt]    {$p,\sigma $};
    % Text Node
    \draw (162.65,167) node [anchor=west] [inner sep=0.75pt]    {$p,\sigma $};
    % Text Node
    \draw (127.5,99.5) node [anchor=south] [inner sep=0.75pt]    {$p',\sigma '$};
    % Text Node
    \draw (114,190.5) node [anchor=north west][inner sep=0.75pt]   [align=left] {(a)};
    % Text Node
    \draw (267,190.5) node [anchor=north west][inner sep=0.75pt]   [align=left] {(b)};
    % Text Node
    \draw (316.65,167) node [anchor=west] [inner sep=0.75pt]    {$p,\sigma $};
    % Text Node
    \draw (236.17,167) node [anchor=east] [inner sep=0.75pt]    {$p,\sigma $};
    % Text Node
    \draw (236.17,122.19) node [anchor=east] [inner sep=0.75pt]    {$p',\sigma '$};
    % Text Node
    \draw (316.65,122.19) node [anchor=west] [inner sep=0.75pt]    {$p',\sigma '$};
    % Text Node
    \draw (122.41,144.59) node    {$\Gamma $};
    % Text Node
    \draw (276.41,144.59) node    {$\Gamma $};
    % Text Node
    \draw (199,313.5) node [anchor=north west][inner sep=0.75pt]   [align=left] {(c)};
    % Text Node
    \draw (111.17,282) node [anchor=east] [inner sep=0.75pt]    {$p,\sigma $};
    % Text Node
    \draw (111.17,237.19) node [anchor=east] [inner sep=0.75pt]    {$p',\sigma '$};
    % Text Node
    \draw (146.83,237.19) node [anchor=west] [inner sep=0.75pt]    {$p',\sigma '$};
    % Text Node
    \draw (146.83,282) node [anchor=west] [inner sep=0.75pt]    {$p,\sigma $};
    % Text Node
    \draw (239,249.45) node [anchor=east] [inner sep=0.75pt]    {$p',\sigma '$};
    % Text Node
    \draw (239,273.59) node [anchor=east] [inner sep=0.75pt]    {$p,\sigma $};
    % Text Node
    \draw (304.98,249.45) node [anchor=west] [inner sep=0.75pt]    {$p',\sigma '$};
    % Text Node
    \draw (304.98,273.59) node [anchor=west] [inner sep=0.75pt]    {$p,\sigma $};
    
    
    \end{tikzpicture}
    
    \caption{Skeleton diagrams that decide Landau parameters 
    (a) The vertex function decides the single electron self-energy; 
    in other words, $f$ appears in $\varepsilon$.
    (b) The vertex function appears as the effective interaction channel; 
    in other words, $f$ appears in the energy functional i.e. the effective Hamiltonian.
    (c) Hartree-Fock approximation in the vertex function.
    The left one is the Hartree term ($\propto V(\vb*{q} = 0)$).
    The right one is the Fock term ($\propto V(\vb*{q} = \vb*{p} - \vb*{p}')$).}
    \label{fig:hf-vertex}
\end{figure}

It should be noted that kind of zero sound described by the original theory of Landau 
only exists in charge-neutral systems, 
such as $^3$\ce{He};
in electron systems,
the zero sound is essentially the plasmon, 
which receives a gap $\omega_{\text{p}}$.
This originates from the long-range property of Coulomb interaction,
which means at $\vb*{q} = 0$ we have a singularity, 
and this breaks the aforementioned condition 
that in the kinetic equation, 
the quasiparticle-interaction assumes no significant change when $\vb*{q}$ is changed. 
Essentially, this means the $f_{\vb*{p} \vb*{p}'}$ function is also not well-defined.
This can be explicitly checked by naively repeating the procedure 
to derive $f_{\vb*{p} \vb*{p}'}$ from the microscopic particle interaction potential:
if we use the Hartree-Fock approximation to find $f_{\vb*{p} \vb*{p}'}$, 
we get 
\begin{equation}
    f(\vb*{p}, \vb*{p}') = 
    V(\vb*{q} = 0) - \frac{1}{2} V(\abs*{\vb*{p} - \vb*{p}'}) (1 + \vb*{\sigma} \vb*{\sigma}'),
\end{equation}
with the first term being the Hartree term 
and the second term being the Fock term
(\prettyref{fig:hf-vertex}).
This expression gives an infinite result for Coulomb interaction, 
because $V(\vb*{q}) = 4 \pi e^2 / \vb*{q}^2$ now diverges at $\vb*{q} = 0$ \cite{silin1958theory}.%
\footnote{
    Note that we \emph{can't} correct the Hartree term with screened Coulomb potential!
    The $\var{n}_{\vb*{p} \sigma}(\vb*{r})$ variable used in the Hartree term 
    comes from the renormalized Green function, 
    which already contains, say, the ring diagrams that may appear 
    in the middle of a Coulomb interaction line.
    If we correct the Coulomb interaction line in the Hartree term, 
    double counting occurs. 
}

One way to work around this singularity 
is to analyze the Hartree term in the real space, 
while still attributing other corrections in \prettyref{fig:hf-vertex}
to $f_{\vb*{p} \vb*{p}'}$.
Thus, the kinetic equations for Fermi liquid in a metal 
now include three equations: 
\begin{itemize}
    \item A quantum Boltzmann equation coupled to a electrostatic field $\varphi$
    created by $\var{n}_{\vb*{p} \sigma}(\vb*{r})$.
    \item The effective single-electron energy equation 
    \begin{equation}
        \varepsilon_{\vb*{p} \sigma}(\vb*{r}) = \varepsilon^0_{\vb*{p} \sigma} 
        + \frac{1}{V} \sum_{\vb*{p}', \sigma'} 
        f_{\vb*{p} \vb*{p}' \sigma \sigma'} \var{n}_{\vb*{p}' \sigma'}(\vb*{r}).
    \end{equation}
    \item The Poisson equation 
    \begin{equation}
        \laplacian \varphi = \frac{e}{\epsilon_0} \sum_{\vb*{p}, \sigma} \var{n}_{\vb*{p} \sigma}(\vb*{r}).
    \end{equation}
\end{itemize}
The presence of the Poisson equation 
and the Hartree self-consistent field $\varphi$
means when $\vb*{q} \to 0$, 
we see plasma oscillations at frequency $\omega_{\text{p}}$.
The physical picture on the electron side 
is still the same: 
distortion of the Fermi surface propagating around,
creating a density mode in the momentum space instead of the real space.

\section{Particle number conservation}

\subsection{Proof of particle number conservation}

\printbibliography

\end{document}