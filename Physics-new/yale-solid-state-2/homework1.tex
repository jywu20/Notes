\documentclass[hyperref, a4paper]{article}

\usepackage{geometry}
\usepackage{titling}
\usepackage{titlesec}
% No longer needed, since we will use enumitem package
% \usepackage{paralist}
\usepackage{enumitem}
\usepackage{footnote}
\usepackage{amsmath, amssymb, amsthm}
\usepackage{mathtools}
\usepackage{bbm}
\usepackage{graphicx}
\usepackage{subcaption}
\usepackage{physics}
\usepackage{tensor}
\usepackage{siunitx}
\usepackage[version=4]{mhchem}
\usepackage{tikz}
\usepackage{xcolor}
\usepackage{listings}
\usepackage{autobreak}
\usepackage[ruled, vlined, linesnumbered]{algorithm2e}
\usepackage{nameref,zref-xr}
\zxrsetup{toltxlabel}
\usepackage[backend=bibtex]{biblatex}
\addbibresource{data.bib}
\addbibresource{experiments.bib}
\usepackage[colorlinks,unicode]{hyperref} % , linkcolor=black, anchorcolor=black, citecolor=black, urlcolor=black, filecolor=black
\usepackage[most]{tcolorbox}
\usepackage{prettyref}

% Page style
\geometry{left=3.18cm,right=3.18cm,top=2.54cm,bottom=2.54cm}
\titlespacing{\paragraph}{0pt}{1pt}{10pt}[20pt]
\setlength{\droptitle}{-5em}

% More compact lists 
\setlist[itemize]{
    itemindent=17pt, 
    leftmargin=1pt,
    listparindent=\parindent,
    parsep=0pt,
}

% Math operators
\DeclareMathOperator{\timeorder}{\mathcal{T}}
\DeclareMathOperator{\diag}{diag}
\DeclareMathOperator{\legpoly}{P}
\DeclareMathOperator{\primevalue}{P}
\DeclareMathOperator{\sgn}{sgn}
\DeclareMathOperator{\res}{Res}
\newcommand*{\ii}{\mathrm{i}}
\newcommand*{\ee}{\mathrm{e}}
\newcommand*{\const}{\mathrm{const}}
\newcommand*{\suchthat}{\quad \text{s.t.} \quad}
\newcommand*{\argmin}{\arg\min}
\newcommand*{\argmax}{\arg\max}
\newcommand*{\normalorder}[1]{: #1 :}
\newcommand*{\pair}[1]{\langle #1 \rangle}
\newcommand*{\fd}[1]{\mathcal{D} #1}
\DeclareMathOperator{\bigO}{\mathcal{O}}

% TikZ setting
\usetikzlibrary{arrows,shapes,positioning}
\usetikzlibrary{arrows.meta}
\usetikzlibrary{decorations.markings}
\tikzstyle arrowstyle=[scale=1]
\tikzstyle directed=[postaction={decorate,decoration={markings,
    mark=at position .5 with {\arrow[arrowstyle]{stealth}}}}]
\tikzstyle ray=[directed, thick]
\tikzstyle dot=[anchor=base,fill,circle,inner sep=1pt]

% Algorithm setting
% Julia-style code
\SetKwIF{If}{ElseIf}{Else}{if}{}{elseif}{else}{end}
\SetKwFor{For}{for}{}{end}
\SetKwFor{While}{while}{}{end}
\SetKwProg{Function}{function}{}{end}
\SetArgSty{textnormal}

\newcommand*{\concept}[1]{{\textbf{#1}}}

% Embedded codes
\lstset{basicstyle=\ttfamily,
  showstringspaces=false,
  commentstyle=\color{gray},
  keywordstyle=\color{blue}
}

% Reference formatting
\newrefformat{fig}{Fig.~\ref{#1}}
\newcommand*{\term}[1]{\textit{#1}}

% Color boxes
\tcbuselibrary{skins, breakable, theorems}

\newtcbtheorem[number within=section]{infobox}{Box}{
    enhanced,
    boxrule=0pt,
    colback=blue!5,
    colframe=blue!5,
    coltitle=blue!50,
    borderline west={4pt}{0pt}{blue!65},
    sharp corners,
    fonttitle=\bfseries, 
    breakable,
    before upper={\parindent15pt\noindent}}{box}
\newtcbtheorem[number within=section, use counter from=infobox]{theorybox}{Box}{
    enhanced,
    boxrule=0pt,
    colback=orange!5, 
    colframe=orange!5, 
    coltitle=orange!50,
    borderline west={4pt}{0pt}{orange!65},
    sharp corners,
    fonttitle=\bfseries, 
    breakable,
    before upper={\parindent15pt\noindent}}{box}
\newtcbtheorem[number within=section, use counter from=infobox]{learnbox}{Box}{
    enhanced,
    boxrule=0pt,
    colback=green!5,
    colframe=green!5,
    coltitle=green!50,
    borderline west={4pt}{0pt}{green!65},
    sharp corners,
    fonttitle=\bfseries, 
    breakable,
    before upper={\parindent15pt\noindent}}{box}

% Displaying texts in bookmarkers

\pdfstringdefDisableCommands{%
  \def\\{}%
  \def\ce#1{<#1>}%
}

\pdfstringdefDisableCommands{%
  \def\texttt#1{<#1>}%
  \def\mathbb#1{#1}%
}
\pdfstringdefDisableCommands{\def\eqref#1{(\ref{#1})}}

\makeatletter
\pdfstringdefDisableCommands{\let\HyPsd@CatcodeWarning\@gobble}
\makeatother

\newenvironment{shelldisplay}{\begin{lstlisting}}{\end{lstlisting}}

\newcommand*{\efermi}{E_{\text{F}}}
\newcommand*{\sA}{\text{A}}
\newcommand*{\sB}{\text{B}}
\newcommand*{\Tc}{T_{\text{c}}}
\newcommand*{\muB}{\mu_{\text{B}}}
\newcommand*{\kB}{k_{\text{B}}}

\title{Homework 1}
\author{Jinyuan Wu}

\begin{document}

\maketitle

\section{Hund’s rule and magnetic exchange as a result of Coulomb interactions}

\subsection{Why Coulomb interaction is stronger than magnetic dipole coupling?}\label{sec:hund.magnitude}

The magnetic dipole-dipole interaction Hamiltonian is 
\begin{equation}
    H=-\frac{\mu_0}{4 \pi|\vb*{r}|^3}\left[3\left(\vb*{m}_1 \cdot \hat{\vb*{r}}\right)\left(\vb*{m}_2 \cdot \hat{\vb*{r}}\right)-\vb*{m}_1 \cdot \vb*{m}_2\right]-\mu_0 \frac{2}{3} \vb*{m}_1 \cdot \vb*{m}_2 \delta(\vb*{r}).
\end{equation}
The relation between the magnetic moment and the spin is 
\begin{equation}
    \vb*{m} = - g \muB \vb*{S},
\end{equation}
where $\hbar \vb*{S}$ the spin angular momentum 
(and $\vb*{S}$ has no dimension in it).
So we have 
\begin{equation}
    \begin{aligned}
        H &= - \frac{\mu_0 \muB^2}{4 \pi \abs*{\vb*{r}}^3} 
        (3 (\vb*{S}_1 \cdot \vu*{r}) (\vb*{S}_2 \cdot \vu*{r}) - \vb*{S}_1 \cdot \vb*{S}_2) 
        - \frac{2}{3} \mu_0 \muB^2 \vb*{S}_1 \cdot \vb*{S}_2 \delta(\vb*{r}) \\
        &\sim - \frac{\mu_0 g_1 g_2 \muB^2}{4 \pi a^3} \sim - \frac{\mu_0 \muB^2}{4 \pi a^3},
    \end{aligned}
\end{equation}
where $a$ is the characteristic length scale 
about the distances between atoms,
and may be estimated by the lattice constant,
while $g_1, g_2$ are dimensionless and usually not large. 
Taking $a \sim \SI{1}{\angstrom}$,
we find $H_{\text{dipole-dipole}} \sim \SI{5e-5}{eV}$.

The exchange interaction depends more strongly on 
the distance between atoms, 
because it involves wave function overlapping.
We can derive $\Tc$ from the Weiss mean field theory using $J$ 
and then infer the magnitude of $J$ from experimentally observed $\Tc$.
Suppose we already find the relation between $\vb*{M}$ and $\vb*{H}$ when there is no interaction:
we label this ``free'' susceptibility as $\chi_0$.
We then 
\begin{equation}
    \vb*{M} = \chi_0 \vb*{H}_{\text{total}} = \chi_0 (\vb*{H} + \vb*{H}_M) 
    = \chi_0 (\vb*{H} + \lambda_M \vb*{M}).
    \label{eq:magnetic-scf}
\end{equation}
Note that when the electrons are completely localized,
the coupling between the external field and magnetic moments is linear,
and the interaction between spins 
is two-body ($\vb*{S}_{\vb*{i}} \cdot \vb*{S}_{\vb*{j}}$ 
or $\vb*{D} \cdot (\vb*{S}_{\vb*{i}} \times \vb*{S}_{\vb*{j}})$),
this formalism is \emph{exact}:
$\vb*{M}$ is essentially 
\[
    \frac{1}{V} \sum_{\vb*{i}, \sigma_1, \sigma_2} \frac{\hbar}{2} 
    \expval{
        c^\dagger_{\vb*{i} \sigma_1} 
        \vb*{\sigma}_{\sigma_1 \sigma_2}
        c_{\vb*{i} \sigma_2}
    },
\]
and can be obtained using 
Now since the interaction is highly localized,
the ring diagram approximation in the real space is exact
(other diagrams all vanish because of the $\delta_{\vb*{i} \vb*{j}}$ factor 
in the single electron propagator),
and each ring equals the electron occupation because $n_{\vb*{i}}^2 = n_{\vb*{i}}$,
and in this way,
letting $\lambda_M$ to be 
the ratio between the magnetic field generated by a ring diagram 
and the ring diagram itself,
we have
\[
    \vb*{M} = \chi_0 \vb*{H} + \lambda_M \chi_0 \vb*{H} + \lambda_M \chi_0 \lambda_M \chi_0 \vb*{H} + \cdots,
\]
which is just \eqref{eq:magnetic-scf}. 
Then we introduce the high-temperature approximation, 
and we get Curie's law
\begin{equation}
    \chi_0 = \frac{C}{T}, \quad C = \frac{\mu_0 n}{3 \kB} \muB^2 g^2 j(j+1),
    \label{eq:curie}
\end{equation}
where $n$ is the number of atoms per volume,
$g$ the Landé $g$ factor, 
$j$ the total angular momentum.

Until now no information concerning magnetic interaction is used. 
Now the next step is to use the magnetic interaction 
to find $\lambda_M$.
Here we introduce the mean-field approximation 
and assume that $\vb*{m}_j$ can be approximated by $\vb*{M} / n$, 
ignoring all thermal and quantum fluctuation of $\vb*{m}_j$.
Therefore, from the single-site Hamiltonian in the Heisenberg model
($z$ is the number of nearest neighbors of each site)
\begin{equation}
    H = - 2 z J \vb*{S}_{\vb*{i}} \cdot \vb*{S}_{\vb*{j}} 
    = - \frac{2 z J}{g^2 \muB^2} \vb*{m}_i \cdot \frac{\vb*{M}}{n},
    \label{eq:heisenberg.mf-ham}
\end{equation}
and therefore we get 
\begin{equation}
    \vb*{B}_M = \frac{2 z J}{n g^2 \muB^2} \vb*{M}, \quad 
    \vb*{H}_M = \underbrace{\frac{2 z J}{n \mu_0 g^2 \muB^2}}_{\lambda_M} \vb*{M}.
    \label{eq:weiss-mean-field-response}
\end{equation}
Putting \eqref{eq:weiss-mean-field-response} 
and \eqref{eq:curie} into \eqref{eq:magnetic-scf}, 
we get 
\begin{equation}
    \vb*{M} = \frac{C}{T - \Tc} \vb*{H}, \quad 
    \Tc = C \lambda_M = \frac{2 z J}{3 \kB} (j + 1) j.
\end{equation}
So now we can estimate $J$ from $\Tc$ by 
\begin{equation}
    J = \frac{3}{2} \frac{\kB \Tc}{z j (j + 1)}.
\end{equation}
For iron, the bcc structure means $z = 8$, 
and we can take $j = 1$,
and the experimentally measured $\Tc$ is \SI{1043}{K},
and we find $J = \SI{8.4e-3}{eV}$.

So we find $H_{\text{exchange}} / H_{\text{magnetic dipole-dipole}} \sim 160$,
and therefore the order of magnitude of the exchange interaction 
is much larger than the magnetic dipole-dipole interaction,
and that's why the exchange interaction is the dominant effect.

\subsection{The atomic ground state configuration of \ce{Cr^{3+}}, \ce{Fe^{3+}} and \ce{Gd^{3+}}}

The Hund's rules have their root in $L$-$S$ coupling. 
After the occupation of each electron shell is determined, 
we apply the following rules one by one: 
\begin{enumerate}
    \item For each electron shell, 
        make the total spin as large as possible. 
    \item After the total spin is determined, 
        make the total orbital angular momentum as large as possible. 
    \item When $S$ and $L$ are already determined, 
        if the shell is not yet half-filled, 
        take $J = \abs*{L - S}$;
        otherwise $J = L + S$. 
    \item Now $S$, $L$, and $J$ are all decided, 
        and the Landé g-factor is 
        \begin{equation}
            g_J = \frac{3}{2} + \frac{S(S+1) - L(L+1)}{2J(J+1)},
        \end{equation}
        and the total magnetic momentum is 
        \begin{equation}
            \vb*{\mu} = - g_J \muB \vb*{J},
        \end{equation}
        where $\hbar \vb*{J}$ is the total angular momentum.
\end{enumerate}

The electron configuration of \ce{Cr^{3+}} is [Ar]3d$^3$.
To maximize $L$,
we want the $l = 0, 1, 2$ orbitals to be filled,
so $L = 3$;
the maximal total spin momentum available is $S = 3/2$.
Since the d orbitals are not yet half-filled, 
we have $J = \abs*{L - S} = 3/2$,
and therefore $g = 2/5$,
and the total magnetic moment is 
$2/5 \cdot \muB \cdot 3/2 = 3/5 \muB$.
This deviates far from the experimentally measured value.
One explanation is here is angular momentum quenching happens,
and the total magnetic moment is essentially the total spin magnetic moment,
which is $3 \muB$,
exactly what is measured experimentally.

The electron configuration of \ce{Fe^{3+}} is [Ar]3d$^5$,
so the 3d shell is half-filled,
and each orbital therefore only contains one electron,
and the spins of all electrons are aligned to each other.
So in this shell we have $L = 0$, 
$S = 5/2$,
and therefore $J = 5/2$.
We have $g = 2$, 
so the magnitude of the magnetic moment is $2 \muB \cdot 5/2 = 5 \muB$,
which agrees with the experimental value.

The electron configuration of \ce{Gd^{3+}} is [Xe]4f$^7$, 
so the 4f shell is half-filled, 
and therefore $L = 0$, $S = 7/2$,
and therefore $J = 7 / 2$.
We have $g = 2$, so the magnitude of the magnetic moment is $2 \muB \cdot 7 / 2 = 7 \muB$,
which agrees with the experimental result.

\subsection{Other magnetic mechanism from Coulomb interaction}

Apart from the direct exchange interaction, 
we also have superexchange, double exchange, 
the Dzyaloshinskii-Moriya interaction,
and the RKKY interaction.

\section{Landau diamagnetism and Pauli paramagnetism}

\subsection{The  physical  origins  of  various  magnetic  responses  in  a  solid}

\concept{Pauli paramagnetism} comes from the spin-magnetic field coupling
of itinerant electrons: 
the external magnetic field splits 
the spin-$\uparrow$ band and the spin-$\downarrow$ band.
Since the Fermi energy in the two bands is the same, 
when, say, the magnetic field is upward, 
we have less spin-$\uparrow$ electrons than spin-$\downarrow$ electrons, 
and we have an overall downward spin 
and therefore an overall upward magnetic moment, 
so there is a paramagnetic response. 

\concept{Landau diamagnetism} comes from the coupling between orbital angular momentum
and the external magnetic field of itinerant electrons. 
(Although the electrons are ``constrained'' in circular motion in the $xy$ plane, 
the $\vb*{x}$ and $\vb*{p}$ degrees of freedom 
play an important role in Landau diamagnetism
instead of being totally frozen, 
so we still classify Landau diamagnetism 
as an itinerant mechanism.)
In an external magnetic field (say, in the $z$ direction),
the $k_x$ and $k_y$ variables are no longer good quantum numbers, 
and the Hamiltonian becomes 
an ``oscillator'' in the $xy$ plane 
plus a free-electron term in the $z$ direction
and reads
\begin{equation}
    E = \left(n + \frac{1}{2}\right) \hbar \omega_{\text{c}} + \frac{\hbar^2 k_z^2}{2m}, 
\end{equation}
where 
\begin{equation}
    \omega_{\text{c}} = \frac{e B}{m}.
\end{equation}
So we find 
\[
    E = - \underbrace{\left(- \frac{e}{m} \hbar \left( n + \frac{1}{2} \right)\right)}_{\mu} \cdot {B} 
    + \frac{\hbar^2 k_z^2}{2m},
\] 
and we find the magnetic moment is less than zero, 
and therefore we get a paramagnetic response.


\concept{Curie paramagnetism} comes from the coupling 
between the spin and the external magnetic field 
of localized electrons.
Here we no longer hand band structures and no $\vb*{k}$ variable, 
and the degrees of freedom are $n, l, m$ and the spin $m_s$,
and Curie's $1 / T$ law can be derived by putting the first-order perturbation of
\begin{equation}
    H = - \vb*{\mu} \cdot \vb*{B} 
    \label{eq:magnetic-dipole-to-external-field}
\end{equation}
into the partition function.

\concept{Van Vleck paramagnetism} comes from 
the second order perturbation of \eqref{eq:magnetic-dipole-to-external-field}.
It has the same order of magnitude with 
\concept{Larmor (or Langenvin) diamagnetism},
which comes from the $B^2$ term 
\begin{equation}
    H_{\text{Larmor}} = \mu_0^2 \frac{e^2}{8 m_{\mathrm{e}}} r_{\perp}^2 H^2
\end{equation}
in the electromagnetic coupling Hamiltonian.

Now we estimate the magnitudes of the mechanisms for copper.
From an existing QuantumESPRESSO calculation 
(see \href{http://lampx.tugraz.at/~hadley/ss1/materials/dos/fccCu_dos.html}{Electron density of states for fcc copper}),
we know that around the Fermi surface, 
the density of states is \SI{2.48e28}{eV^{-1} \cdot m^{-3}},
and we have 
\begin{equation}
    \chi_{\text {Pauli}}=2 \mu_0 \mu_{\mathrm{B}}^2 \rho_0\left(\varepsilon_F\right)
    = \num{3.2e-5} , \quad 
    \chi_{\text {Landau }}=-\frac{1}{3} \chi_{\text {Pauli }} = \num{-1.1e-5}.
\end{equation}
It should be noted that the second equation only works 
when the free electron approximation works well enough, 
and is just an estimation of magnitude. 
(And indeed, copper is diamagnetic,
while the estimation of magnitude above 
says it's paramagnetic.)
But anyway, Pauli paramagnetism 
and Landau diamagnetism 
have comparable orders of magnitude.

\cite{taylor1938magnetic} gives the paramagnetism of iron atoms in ferrohemoglobin
-- which can only be Curie paramagnetism.
The susceptibility is $\sim 10^{-2}$ in CGS units,
and therefore is $\sim \num{6e-2}$ in SI units.
Van-Vleck paramagnetism is shown in \ce{Eu2O3}. 
\cite{eu2o3} gives the CGS susceptibility $\sim \num{7e-3}$, 
which is $\sim \num{4e-2}$ in SI. 

It can be seen that localized magnetism is much stronger than itinerant magnetism.

\subsection{Strength of \ce{Cu}'s magnetic response}

\cite{cooper_magnetic} gives the magnetic susceptibility of cooper:
\begin{equation}
    \chi=\left(-0.083+\frac{0.023}{T}\right) \times 10^{-6} \unit{cgs\ unit}.
\end{equation}
Suppose the lab is in room temperature, 
and we set $T = \SI{300}{K}$.
(Although SQUID requires cooling of the superconducting probe, 
this doesn't mean the sample should also be cooled. 
Indeed, \cite{singh2014magnetoencephalography} shows 
some cases where SQUID is used to measure 
neural activities in the brain.)
Since $\chi$ doesn't have $H$ dependence,
the relation between $M$ and $H$ is linear,
and with $H = \SI{10}{Oe}$,
we have $M = \chi H = \SI{-8.29e-7}{cgs\ unit}$.
The expected magnetic moment is \SI{5e-8}{emu}, 
and therefore the volume needed is (note that we are working with cgs units)
\[
    \frac{\num{5e-8}}{\num{8.29e-7}} \unit{cm^3} = \SI{0.06}{cm^3},
\]
and the density of \ce{Cu} is \SI{8.96}{g/cm^3},
so we need \SI{0.54}{g} cooper.

\subsection{Itinerant magnetism in 2D}

Since the derivation of Pauli paramagnetism 
and Landau diamagnetism basically 
doesn't involve anything dependent to the dimension,
we still have 
\begin{equation}
    \chi_{\text{Pauli}} = 2 \mu_0 \muB^2 \rho_0(\efermi),
    \chi_{\text{Landau}} = - \frac{1}{3} \chi_{\text{Pauli}}
\end{equation}
for a free 2D electron gas. 
However, when the influence of the band structure is taken into account, 
from the derivation it can be seen that 
$\chi_{\text{Pauli}}$ relies on the effective mass $m^*$
while $\chi_{\text{Landau}}$ relies on the bare mass $m$
(the crystal periodic potential is never coupled to 
the electromagnetic field), 
and by adjusting $m^* / m$ 
we can let $\abs{\chi_{\text{Landau}}} > \abs{\chi_{\text{Pauli}}}$.
So both a paramagnetic total response and a diamagnetic total response are possible.


\section{Mean-field solution of 1D Ising model}

\subsection{Mean-field approximation}\label{sec:ising.mf}

The Ising model 
\begin{equation}
    H = - J \sum_{\pair{\vb*{i}, \vb*{j}}} s_{\vb*{i}} s_{\vb*{j}} - h \sum_{\vb*{i}} s_{\vb*{i}}
\end{equation}
is defined on a lattice,
each site of which has $q$ nearest neighbors.
Assuming the spins have minimal fluctuation, 
and we have $\expval{s_{\vb*{i}}} = m$, 
we have 
\[
    s_{\vb*{i}} s_{\vb*{j}} \approx 
    \expval{s_{\vb*{i}}} s_{\vb*{j}} + s_{\vb*{i}} \expval{s_{\vb*{j}}} 
    - \expval{s_{\vb*{i}}} \expval{s_{\vb*{j}}} = 
    m (s_{\vb*{i}} + s_{\vb*{j}}) - m^2,
\]
and therefore (the $1/2$ factor comes from the fact that a site is counted twice 
if we go over all bonds)
\begin{equation}
    \begin{aligned}
        H &= - J m q \sum_{\vb*{i}} s_{\vb*{i}} 
        + \frac{N q}{2} \cdot J m^2 - h \sum_{\vb*{i}} s_{\vb*{i}} \\
        &= - \underbrace{(h + Jmq)}_{h_{\text{eff}}} \sum_{\vb*{i}} s_{\vb*{j}} 
        + \frac{NqJm^2}{2} .
    \end{aligned}
    \label{eq:ising.mean-field-ham}
\end{equation}

This essentially is the Weiss molecular field approximation.
The $m^2$ term is absent in \eqref{eq:heisenberg.mf-ham}, 
but this doesn't matter: 
when calculating $\expval{s_{\vb*{i}}}$, 
$m$ is a constant and is not summed over, 
so the $m^2$ term can be seen as a constant 
and we have 
\[
    H \simeq - \underbrace{(h + Jmq)}_{h_{\text{eff}}} \sum_{\vb*{i}} s_{\vb*{j}} ,
\]
which is just \eqref{eq:heisenberg.mf-ham}.
The only difference is that in \prettyref{sec:hund.magnitude},
we have taken the high temperature limit, 
while here we do not.

(Another way to get this approximation is 
by integrate out microscopic $s_{\vb*{i}}$ degrees of freedom, 
and get a theory $f(m)$ about $\sum_{\vb*{i}} s_{\vb*{i}} / N$, 
and then find the $m$ that minimizes $f(m)$.
This last method can in principle be systematically improved 
because we can include more fluctuation around this minimum.) 

\subsection{The partition function}

As is said in the last section, 
we can just ignore the $m^2$ term, 
and the partition function is 
\begin{equation}
    \begin{aligned}
        Z &= \sum_{\{s_{\vb*{i}}\}} \ee^{ \beta h_{\text{eff}} \sum_{\vb*{i}} s_{\vb*{i}}} \\
        &= \prod_{\vb*{i}} \sum_{s_{\vb*{i}} = \pm 1} 
        \ee^{\beta h_{\text{eff}} s_{\vb*{i}}} = \prod_{\vb*{i}} 2 \cosh (\beta h_{\text{eff}}) \\
        &= 2^N \cosh^N (\beta h_{\text{eff}}). 
    \end{aligned} 
\end{equation}

\subsection{Find the total magnetization}

The total magnetic moment can be obtained using 
\begin{equation}
    \sum_{\vb*{i}} \expval{s_{\vb*{i}}} = \frac{1}{\beta} \frac{1}{Z} \pdv{Z}{h} , 
\end{equation}
which, in our case, is 
\[
    \begin{aligned}
        \sum_{\vb*{i}} \expval{s_{\vb*{i}}} &= \frac{1}{\beta} \frac{1}{Z} \pdv{Z}{h_{\text{eff}}} \\
        &= \frac{1}{\beta} \frac{1}{2^N \cosh^N (\beta h_{\text{eff}})} 
        2^N N \cosh^{N-1} (\beta h_{\text{eff}}) \cdot \beta \sinh(\beta h_{\text{eff}}) \\
        &= N \tanh (\beta h_{\text{eff}}), 
    \end{aligned}
\]
and therefore 
\[
    m = \frac{1}{N} \sum_{\vb*{i}} \expval{s_{\vb*{i}}}
    = \tanh(\beta h_{\text{eff}}),
\]
and we eventually get the self-consistent equation 
\begin{equation}
    m = \tanh(\frac{h + Jq m}{\kB T}).
    \label{eq:ising.scf}
\end{equation}

\subsection{The critical temperature}

In the $T \to \infty$ limit, we have the Curie-Weiss law 
\begin{equation}
    m = \frac{h + Jqm}{\kB T} \Rightarrow 
    m = \frac{h / \kB}{T - \Tc}, \quad 
    \Tc = \frac{Jq}{\kB}.
    \label{eq:ising.curie}
\end{equation}
This estimation of $\Tc$ is actually accurate in the level of \eqref{eq:ising.scf}: 
the condition to have non-zero solution of \eqref{eq:ising.scf} 
is that the slope of $\tanh$ is larger than 1 (the LHS of \eqref{eq:ising.scf}), 
so that when $T \to 0$, 
the RHS of \eqref{eq:ising.scf} goes to a constant, 
and a cross point occurs between the two parts of \eqref{eq:ising.scf}.
When $h = 0$, the slope of $\tanh$ is one, if and only if 
\[
    m = \frac{J q m}{\kB T} \Rightarrow T = \frac{J q}{\kB},
\]
which is just $\Tc$ in \eqref{eq:ising.curie}.

\subsection{Critical exponent of $m$}

When $h = 0$ and $T$ is near $\Tc$, 
$m$ is small,
but the first-order Taylor expansion in \eqref{eq:ising.curie} 
can't give the relation between $m$ and $T$ when $h = 0$, 
so we need to take the next term into consideration. 
We have 
\[
    m = \frac{J q m}{\kB T} - \frac{1}{3} \left(
        \frac{J q m}{\kB T}
    \right)^3 + \cdots, 
\]
or in other words 
\[
    m \approx \frac{\Tc}{T} m - \frac{1}{3} \left(
        \frac{\Tc}{T} m
    \right)^3,
\]
and we find 
\begin{equation}
    m \approx \pm \sqrt{
        \frac{3 T^2 (\Tc - T)}{\Tc^3} 
    } \approx \pm \sqrt{
        \frac{3 (\Tc - T)}{\Tc} 
    }.
    \label{eq:ising.mf.m}
\end{equation}
So we find $m \propto (\Tc - T)^{1/2}$, 
and therefore the critical exponent $\beta = 1/2$.

The mean-field level Landau theory gives exactly the same prediction, 
because as is said in \prettyref{sec:ising.mf}, 
finding the $m$ that minimizes the free energy 
is equivalent to the mean-field approximation done above, 
and the former is just the mean-field level Landau theory. 
But after more fluctuation is taken into account 
(i.e. we do a Ginzburg-Landau theory instead of a mean-field Landau theory), 
we find $\beta$ deviates from the mean-field value and 
has dimension dependence.
When $d = 2$, $\beta = 1/8$, 
and when $d = 3$, $\beta = 0.326$.

\subsection{Critical exponents for other quantities when $h = 0$}

For the free energy, we have 
\begin{equation}
    \begin{aligned}
        F &= - \kB T \ln Z = - \kB T N \ln 2 - N \kB T \ln \cosh (\beta h_{\text{eff}}) \\
        &= - \kB T N \ln 2 - N \kB T \ln \cosh (
            \frac{h + Jq m}{\kB T}
        ).
    \end{aligned}
\end{equation}
We still need to include the $m^2$ term in \eqref{eq:ising.mean-field-ham}
so the correct free energy is 
\begin{equation}
    F = - \kB T N \ln 2 - N \kB T \ln \cosh (
        \frac{h + Jq m}{\kB T}
    ) + \frac{1}{2} N q J m^2.
    \label{eq:ising.full-mf-free-energy}
\end{equation}

Here we consider the $h = 0$ case. 
When $T > \Tc$, $m = 0$, so 
\[
    F = - \kB T N \ln 2.
\]
When $T < \Tc$, 
Taylor expansion gives 
\[
    F = \frac{N \kB \Tc}{2} m^2 - \kB T N \ln 2 - \frac{N \kB \Tc^2}{2 T} m^2
    + \frac{N \kB \Tc^4}{12 T^3} m^4 + \cdots. 
\]
$m$ is given by \eqref{eq:ising.mf.m}, 
so near $\Tc$, we have (note that since $m \propto (\Tc - T)^{1/2}$, 
we can replace some occurrences $T$ by $\Tc$ in the coefficients 
if we only want the first several term in the expansion over $\Tc - T$).
The eventual result is 
\begin{equation}
    F = \begin{cases}
        - \kB T N \ln 2 &, T > \Tc, \\
        - \kB T N \ln 2 - \frac{3 \kB N (T - \Tc)^2}{4\Tc} &, T < \Tc. 
    \end{cases}
\end{equation}
It can be seen that the free energy is continuous 
at $T = \Tc$, 
but is not smooth at that point,
so we can expect that the specific heat capacity 
is not continuous at $T = \Tc$. 
This means the phase transition is a second-order phase transition.

The total energy, when $h = 0$, is (again we insert the $m^2$ term in \eqref{eq:ising.mean-field-ham})
\begin{equation}
    \begin{aligned}
        U &= - h_{\text{eff}} \sum_{\vb*{i}} \expval{s_{\vb*{i}}} + \frac{1}{2} N q J m^2
        = - Jqm \cdot N m + \frac{1}{2} N q J m^2 = - \frac{1}{2} N q J m^2  \\
        &= \begin{cases}
            - \frac{3 \kB N (\Tc - T)}{2} &, T < \Tc, \\
            0 &,  T > \Tc.
        \end{cases} 
    \end{aligned}
\end{equation}

The entropy is 
\begin{equation}
    S = \frac{U - F}{T} \approx \kB N \ln 2 - \begin{cases}
        \frac{3 \kB N (\Tc - T)}{2 \Tc } &, T < \Tc, \\
        0 &,  T > \Tc
    \end{cases}.
\end{equation}
Here we only keep the lowest order dependence of $\Tc - T$. 
This expression can also be obtained by calculating $- \pdv{F}{T}$.

The heat capacity per site can be found from $U$:
\begin{equation}
    c = - \frac{1}{N} T \pdv[2]{F}{T} = \begin{cases}
        0 &, T > \Tc, \\
        \frac{3}{2} \kB &, T < \Tc.
    \end{cases}
\end{equation}
So indeed this is a second-order phase transition.
It can also be found using $- T \pdv[2]{F}{T}$.

\subsection{Phase transition by tuning $h$}

For the sake of convenience let's assume $T$ is very close to zero.
In this way, Mathematica tells us \eqref{eq:ising.scf} has to solutions: $m = \pm 1$.
So when $h > 0$, 
the $- h \sum_{\vb*{i}} s_{\vb*{i}}$ term 
means $m = 1$, 
and when $h < 0$, 
$m = -1$.

Since now the order parameter 
(instead of its derivative with respect to one of the coordinates of the phase diagram) 
flips without a smooth transition, 
The phase transition is a first-order one.

\subsection{Fluctuation beyond the mean-field solution}

After fluctuation is added, 
the transition temperature should be lower than the mean-field temperature, 
because fluctuation tends to kill the low-temperature order, 
and therefore stronger fluctuation means lower $\Tc$.
The reason why low-dimension systems have stronger fluctuation 
is when the dimension goes higher, 
there will be more nearest neighbors of one site, 
so if the spin on the site is flipped, 
the energy change will be huge; 
this means fluctuation is weakened 
and the mean-field approximation gains higher accuracy.

\section{Spin waves}

\subsection{Spin wave with staggering Heisenberg coefficients}

So now we have two sublattices, denoted by A and B, 
and the Hamiltonian is 
\begin{equation}
    H = - 2 J_1 \sum_i \vb*{S}_{i \text{A}} \cdot \vb*{S}_{i \text{B}} 
    - 2 J_2 \sum_i \vb*{S}_{i \text{B}} \cdot \vb*{S}_{i+1, \text{A}}.
\end{equation}
We have 
\[
    \comm*{\vb*{S}_i}{\vb*{S}_i \cdot \vb*{S}_j} = - \ii \vb*{S}_i \times \vb*{S}_j,
\]
so 
\begin{equation}
    \begin{aligned}
        \dv{\vb*{S}_{i \text{A}}}{t} &= \frac{1}{\ii \hbar} \comm*{\vb*{S}_{i \text{A}}}{H}
        = \frac{1}{\ii \hbar} \comm*{\vb*{S}_{i \text{A}}}{
            - 2 J_2 \vb*{S}_{i-1, \text{B}} \cdot \vb*{S}_{i \text{A}} 
            - 2 J_1 \vb*{S}_{i \text{A}} \cdot \vb*{S}_{i \text{B}} 
        } \\
        &= \frac{2}{\hbar} \vb*{S}_{i \text{A}} \times (
            J_2 \vb*{S}_{i-1, \text{B}} + 
            J_1 \vb*{S}_{i \text{B}}
        ),
    \end{aligned}
    \label{eq:spin-wave.exact.1}
\end{equation}
and similarly
\begin{equation}
    \dv{\vb*{S}_{i \text{B}}}{t} = 
    \frac{2}{\hbar} \vb*{S}_{i \text{B}} \times (
        J_1 \vb*{S}_{i \text{A}}
        + J_2 \vb*{S}_{i+1, \text{A}}
    ).
    \label{eq:spin-wave.exact.2}
\end{equation}
The above equations are exact and nonlinear. 
Assuming each $\vb*{S}$ doesn't deviate much from its ground state value, i.e. $S \vu*{z}$,
we can reduce \eqref{eq:spin-wave.exact.1} and \eqref{eq:spin-wave.exact.2} 
into four equations about 
$S^{x/y}_{i, \text{A}/\text{B}}$,
ignoring the time evolution of $z$ components,
and replace all $S^z$ components in these equations by $S$,
and therefore we get 
\[
    \begin{aligned}
        \dv{S^x_{i \text{A}}}{t} &= 
        \frac{2}{\hbar} (
            S^y_{i \text{A}} (J_2 S^z_{i-1, \text{B}} + J_1 S^z_{i \text{B}}) 
            - S^z_{i \text{A}} (J_2 S^y_{i-1, \text{B}} + J_1 S^y_{i \text{B}}) 
        ) \\
        &= \frac{2 S}{\hbar} (
            (J_1 + J_2) S^y_{i \text{A}}
            - J_2 S^y_{i-1, \text{B}}
            - J_1 S^y_{i \text{B}}
        ),
    \end{aligned}
\]
and similarly
\[
    \begin{aligned}
        \dv{S^y_{i \text{A}}}{t} &= - \frac{2 S}{\hbar} 
        (
            (J_1 + J_2) S^x_{i \text{A}}
            - J_2 S^x_{i - 1, \text{B}}
            - J_1 S^x_{i \text{B}}
        ), \\
        \dv{S^x_{i \text{B}}}{t} &= \frac{2 S}{\hbar} (
            (J_1 + J_2) S^y_{i \text{B}}
            - J_2 S^y_{i + 1, \text{A}}
            - J_1 S^y_{i \text{A}}
        ), \\
        \dv{S^y_{i \text{B}}}{t} &= - \frac{2 S}{\hbar} 
        (
            (J_1 + J_2) S^x_{i \text{B}}
            - J_2 S^x_{i + 1, \text{A}}
            - J_1 S^x_{i \text{A}}
        ).
    \end{aligned}
\]
In the 4-momentum domain, this is equivalent to 
\begin{equation}
    \frac{2S}{\hbar} \pmqty{
        0 & J_1 + J_2 & 0 & - J_1 - J_2 \ee^{- \ii k a} \\
        - (J_1 + J_2) & 0 & J_1 + J_2 \ee^{- \ii k a} & 0 \\
        0 & - J_1 - J_2 \ee^{\ii k a} & 0 & J_1 + J_2 \\
        J_1 + J_2 \ee^{\ii k a} & 0 & - (J_1 + J_2) & 0
    }  \pmqty{ S_{i\text{A}}^x \\ S_{i\text{A}}^y \\ S_{i \text{B}}^x \\ S_{i \text{B}}^y } 
    = - \ii \omega 
    \pmqty{ S_{i\text{A}}^x \\ S_{i\text{A}}^y \\ S_{i \text{B}}^x \\ S_{i \text{B}}^y }. 
\end{equation}
The analytical form of the diagonalization of the above 
is too complex for human readers,
but we can always plot it.
The \href{plot/staggering-magnon-1.pdf}{attached file} contains 
Mathematica codes for plotting the relation between 
$\hbar \omega / 2S$ and $ka$ with different values of $J_1$ and $J_2$.
(Just ignore the negative frequencies:
since $\vb*{S}_i$ is a real field,
the modes with positive frequencies have already cover 
the whole spectrum.)
It can be seen that when $J_1 = J_2$,
we just get Brillouin zone folding, 
and when $J_1 \neq J_2$ a small gap opens 
at the boundary of the first Brillouin zone, 
quite similar to how the phonon spectrum splits into 
acoustic and optical branches;
the $\omega \propto k^2$ relation near the $\Gamma$ point 
is still preserved. 

\subsection{Distinguishing an AFM system from an FM one}

The difference of the dispersion relations of FM and AFM magnons 
is essentially the difference of their low energy effective Hamiltonians 
and therefore can be reflected by measuring the specific heat capacity.
For bosons, a $\omega \propto k$ dispersion relation means $C_V \propto T^3$,
while a $\omega \propto k^2$ dispersion relation means $C_V \propto T^{3/2}$,
so by measuring $C_V$ we are able to know whether the material is FM or AFM.

\subsection{Difference between a spin wave and a Bloch domain wall}

The spin orientations between a Bloch domain wall are opposite,
while this is not the case for a spin wave.

\printbibliography

\end{document}