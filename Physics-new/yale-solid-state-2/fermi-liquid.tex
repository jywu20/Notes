\documentclass[hyperref, a4paper]{article}

\usepackage{geometry}
\usepackage{titling}
\usepackage{titlesec}
% No longer needed, since we will use enumitem package
% \usepackage{paralist}
\usepackage{enumitem}
\usepackage{footnote}
\usepackage{enumerate}
\usepackage{amsmath, amssymb, amsthm}
\usepackage{mathtools}
\usepackage{bbm}
\usepackage{graphicx}
\usepackage{subcaption}
\usepackage{physics}
\usepackage{tensor}
\usepackage{siunitx}
\usepackage[version=4]{mhchem}
\usepackage{tikz}
\usepackage{xcolor}
\usepackage{listings}
\usepackage{autobreak}
\usepackage[ruled, vlined, linesnumbered]{algorithm2e}
\usepackage{nameref,zref-xr}
\zxrsetup{toltxlabel}
\usepackage[sorting=none]{biblatex}
\bibliography{squeezing}
\usepackage[colorlinks,unicode]{hyperref} % , linkcolor=black, anchorcolor=black, citecolor=black, urlcolor=black, filecolor=black
\usepackage[most]{tcolorbox}
\usepackage{prettyref}

% Page style
\geometry{left=3.18cm,right=3.18cm,top=2.54cm,bottom=2.54cm}
\titlespacing{\paragraph}{0pt}{1pt}{10pt}[20pt]
\setlength{\droptitle}{-5em}

% More compact lists 
\setlist[itemize]{
    itemindent=17pt, 
    leftmargin=1pt,
    listparindent=\parindent,
    parsep=0pt,
}

% Math operators
\DeclareMathOperator{\timeorder}{\mathcal{T}}
\DeclareMathOperator{\diag}{diag}
\DeclareMathOperator{\legpoly}{P}
\DeclareMathOperator{\primevalue}{P}
\DeclareMathOperator{\sgn}{sgn}
\DeclareMathOperator{\res}{Res}
\newcommand*{\ii}{\mathrm{i}}
\newcommand*{\ee}{\mathrm{e}}
\newcommand*{\const}{\mathrm{const}}
\newcommand*{\suchthat}{\quad \text{s.t.} \quad}
\newcommand*{\argmin}{\arg\min}
\newcommand*{\argmax}{\arg\max}
\newcommand*{\normalorder}[1]{: #1 :}
\newcommand*{\pair}[1]{\langle #1 \rangle}
\newcommand*{\fd}[1]{\mathcal{D} #1}
\DeclareMathOperator{\bigO}{\mathcal{O}}

% TikZ setting
\usetikzlibrary{arrows,shapes,positioning}
\usetikzlibrary{arrows.meta}
\usetikzlibrary{decorations.markings}
\tikzstyle arrowstyle=[scale=1]
\tikzstyle directed=[postaction={decorate,decoration={markings,
    mark=at position .5 with {\arrow[arrowstyle]{stealth}}}}]
\tikzstyle ray=[directed, thick]
\tikzstyle dot=[anchor=base,fill,circle,inner sep=1pt]

% Algorithm setting
% Julia-style code
\SetKwIF{If}{ElseIf}{Else}{if}{}{elseif}{else}{end}
\SetKwFor{For}{for}{}{end}
\SetKwFor{While}{while}{}{end}
\SetKwProg{Function}{function}{}{end}
\SetArgSty{textnormal}

\newcommand*{\concept}[1]{{\textbf{#1}}}

% Embedded codes
\lstset{basicstyle=\ttfamily,
  showstringspaces=false,
  commentstyle=\color{gray},
  keywordstyle=\color{blue}
}

% Reference formatting
\newrefformat{fig}{Fig.~\ref{#1}}
\newcommand*{\term}[1]{\textit{#1}}

% Color boxes
\tcbuselibrary{skins, breakable, theorems}
\newtcbtheorem[number within=section]{warning}{Warning}%
  {colback=orange!5,colframe=orange!65,fonttitle=\bfseries, breakable}{warn}
\newtcbtheorem[number within=section]{note}{Note}%
  {colback=green!5,colframe=green!65,fonttitle=\bfseries, breakable}{note}
\newtcbtheorem[number within=section]{info}{Info}%
  {colback=blue!5,colframe=blue!65,fonttitle=\bfseries, breakable}{info}

\newenvironment{shelldisplay}{\begin{lstlisting}}{\end{lstlisting}}

\newcommand*{\kB}{k_{\text{B}}}
\newcommand*{\muB}{\mu_{\text{B}}}
\newcommand*{\efermi}{E_{\text{F}}}
\newcommand*{\pfermi}{p_{\text{F}}}
\newcommand*{\sA}{\text{A}}
\newcommand*{\sB}{\text{B}}
\newcommand*{\Tc}{T_{\text{c}}}
\newcommand*{\hethree}{$^3$He}
\newcommand*{\hefour}{$^4$He}

\title{Fermi liquid}
\author{Jinyuan Wu}

\begin{document}

\maketitle

\section{Free electron gas}

A free electron gas is described by%
\footnote{
    For some reason, 
    when talking about Fermi liquid, 
    people like to use $\vb*{p}$ instead of $\vb*{k}$.
}
\begin{equation}
    H = \sum_i \frac{\vb*{p}_i^2}{2m},
\end{equation}
and we get a Fermi sphere in the momentum space 
containing all occupied states, 
and a Fermi momentum $\pfermi$.
After we add a periodic potential $V(\vb*{r})$ to the model, 
we get \concept{bands}: 
parabolic bands are displaced periodically 
according to the $\vb*{G}$-grid, 
and the crossing points of these curves 
undergo degeneracy breaking,
and the spectrum, eventually, 
becomes several bands confined in the first Brillouin zone. 
Despite this correction, 
the number of filled electrons never changes.

For a free electron gas, we have well-known observables like 
the density of states at $\efermi$ 
(here $\omega = \varepsilon - \efermi$)
\begin{equation}
    N(\omega = 0) = \frac{m}{\pi^2} \frac{\pfermi}{\hbar^3}, 
    \label{eq:free-electron.dos}
\end{equation}
the electronic specific heat 
\begin{equation}
    C_v = \underbrace{\frac{\pi^2 \kB^2}{3} N(\omega = 0)}_{\text{Sommerfield $\gamma$}} \cdot \ T, 
    \label{eq:free-electron.specific-heat}
\end{equation}
and the magnetic susceptibility 
\begin{equation}
    \chi = \mu_0  \muB^2 \cdot N(\omega = 0).
\end{equation}

\hethree{} atoms are fermions: 
we have two electrons, two protons, and only one neutron
in one \hethree{} atom, 
and therefore there are odd fermions inside 
and the composite state is a fermion. 
In a \hethree{} liquid, 
since the distance between the atoms is drastically reduced, 
we can expect the effect of repulsion is much stronger 
than the case in the gas phase. 
This is a cold atom model of electrons in solids: 
we can easily measure the specific heat, etc. 
of the liquid \hethree.

The $C_V$-$T$ curve measured looks like TODO: fig 1 
As is expected, 
when $T \to 0$, 
we see a $C_V \propto T$ behavior, 
qualitatively agreeing with \eqref{eq:free-electron.specific-heat}.
The coefficient however is not what is predicted by 
\eqref{eq:free-electron.dos} and
\eqref{eq:free-electron.specific-heat}.
Now if we apply pressure to liquid \hethree, 
$C_v / T$ clearly increases. 

For a fermion gas, 
we can measure the mass according to \eqref{eq:free-electron.specific-heat}
and \eqref{eq:free-electron.dos}. 
Now if we calculate the ``mass'' of \hethree according to the two equations, 
we get a pressure-dependent 
(quite surprising because the density of liquids almost stays the same when we press it)
``effective mass''
that is never the same as the bare mass of \hethree.

The next question, then, 
is whether liquid \hethree -- 
a system with known strong interaction -- 
can be equivalently described 
by something with the effective mass. 

\section{Assumptions of Fermi liquid theory}

\subsection{Motivation: scattering changes the distribution of electrons}

The general form of Hamiltonian of fermionic systems 
that is found in condensed matter physics looks like 
\begin{equation}
    H = \sum_i \frac{\vb*{p}_i^2}{2m} + 
    \sum_{i, j} V(\abs{\vb*{r}_i - \vb*{r}_j}).
\end{equation}
When we deal with a two-body problem with this dynamics, 
the collision behavior is completely described
by the scattering cross-section. 
If we extend this picture to a many-body system, 
we immediately see scattering changes 
the momentum distribution of particles, 
because the final states of scattering 
can obviously be over the Fermi surface, 
and therefore there is still some distribution above $\pfermi$ now. 

\subsection{Assumption: correspondence between interactive corrected states and uncorrected states}

\begin{figure}
    \centering
    \input{electron/fermi-liquid-distribution.tex}
    \caption{Two types of interaction correction to the electron distribution.
    (a) The Fermi liquid case: there is still a stepwise change of $n$ with respect to the energy.
    (b) The non-Fermi liquid case: the change of $n$ is smooth.}
    \label{fig:fermi-liquid-distribution}
\end{figure}

We assume that we still have ``states labeled by $\vb*{p}$, $n$, etc.'' 
in a Fermi liquid,
and the ground state and the first several excited states 
can be totally labeled by $\var{n(\vb*{p})}$,
which is $n(\vb*{p})$ -- some sort of ``filling'' on these states -- 
minus $n_0(\vb*{p})$ -- the Fermi gas filling. 
We further assume $\var{n}$ is small, so that what we get is \prettyref{fig:fermi-liquid-distribution}(a),
instead of \prettyref{fig:fermi-liquid-distribution}(b).
This agrees with the observed fact that the $C_v \propto T$ relation is still true,
so Fermi liquid should be somehow similar to the Fermi gas. 
Also, we assume the total number of electrons 
should change considerably:
\begin{equation}
    \var{N} = \sum_{\vb*{p}} \var{n(\vb*{p})} \ll \sum_{\vb*{p}} n(\vb*{p}). 
\end{equation}

\subsection{Energy change}

We define $E$ to be 
\begin{equation}
    \var{E} = \sum_{\vb*{p}} \varepsilon(\vb*{p}) \var{n(\vb*{p})} + 
    \bigO(\var{n}^2).
\end{equation}
The chemical potential can be immediately decided:
\begin{equation}
    \mu \coloneqq \eval{\fdv{E}{n(\vb*{p})}}_{\text{Fermi surface}} = \varepsilon(\pfermi).
\end{equation}

\begin{equation}
    \var{E} = \sum_{\vb*{p}} \varepsilon_{\vb*{p}} \var{n(\vb*{p})}
    + \frac{1}{2} \sum_{\vb*{p}, \vb*{p}'}
    f_{\vb*{p} \vb*{p}'} \var{n(\vb*{p})} \var{n(\vb*{p}')} + \bigO(\var{n}^3).
\end{equation}

\end{document}