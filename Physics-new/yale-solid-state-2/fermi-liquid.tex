\documentclass[hyperref, a4paper]{article}

\usepackage{geometry}
\usepackage{titling}
\usepackage{titlesec}
% No longer needed, since we will use enumitem package
% \usepackage{paralist}
\usepackage{enumitem}
\usepackage{footnote}
\usepackage{enumerate}
\usepackage{amsmath, amssymb, amsthm}
\usepackage{mathtools}
\usepackage{bbm}
\usepackage{graphicx}
\usepackage{subcaption}
\usepackage{physics}
\usepackage{tensor}
\usepackage{siunitx}
\usepackage[version=4]{mhchem}
\usepackage{tikz}
\usepackage{xcolor}
\usepackage{listings}
\usepackage{autobreak}
\usepackage[ruled, vlined, linesnumbered]{algorithm2e}
\usepackage{nameref,zref-xr}
\zxrsetup{toltxlabel}
\usepackage[sorting=none]{biblatex}
\bibliography{squeezing}
\usepackage[colorlinks,unicode]{hyperref} % , linkcolor=black, anchorcolor=black, citecolor=black, urlcolor=black, filecolor=black
\usepackage[most]{tcolorbox}
\usepackage{prettyref}

% Page style
\geometry{left=3.18cm,right=3.18cm,top=2.54cm,bottom=2.54cm}
\titlespacing{\paragraph}{0pt}{1pt}{10pt}[20pt]
\setlength{\droptitle}{-5em}

% More compact lists 
\setlist[itemize]{
    itemindent=17pt, 
    leftmargin=1pt,
    listparindent=\parindent,
    parsep=0pt,
}

% Math operators
\DeclareMathOperator{\timeorder}{\mathcal{T}}
\DeclareMathOperator{\diag}{diag}
\DeclareMathOperator{\legpoly}{P}
\DeclareMathOperator{\primevalue}{P}
\DeclareMathOperator{\sgn}{sgn}
\DeclareMathOperator{\res}{Res}
\newcommand*{\ii}{\mathrm{i}}
\newcommand*{\ee}{\mathrm{e}}
\newcommand*{\const}{\mathrm{const}}
\newcommand*{\suchthat}{\quad \text{s.t.} \quad}
\newcommand*{\argmin}{\arg\min}
\newcommand*{\argmax}{\arg\max}
\newcommand*{\normalorder}[1]{: #1 :}
\newcommand*{\pair}[1]{\langle #1 \rangle}
\newcommand*{\fd}[1]{\mathcal{D} #1}
\DeclareMathOperator{\bigO}{\mathcal{O}}

% TikZ setting
\usetikzlibrary{arrows,shapes,positioning}
\usetikzlibrary{arrows.meta}
\usetikzlibrary{decorations.markings}
\tikzstyle arrowstyle=[scale=1]
\tikzstyle directed=[postaction={decorate,decoration={markings,
    mark=at position .5 with {\arrow[arrowstyle]{stealth}}}}]
\tikzstyle ray=[directed, thick]
\tikzstyle dot=[anchor=base,fill,circle,inner sep=1pt]

% Algorithm setting
% Julia-style code
\SetKwIF{If}{ElseIf}{Else}{if}{}{elseif}{else}{end}
\SetKwFor{For}{for}{}{end}
\SetKwFor{While}{while}{}{end}
\SetKwProg{Function}{function}{}{end}
\SetArgSty{textnormal}

\newcommand*{\concept}[1]{{\textbf{#1}}}

% Embedded codes
\lstset{basicstyle=\ttfamily,
  showstringspaces=false,
  commentstyle=\color{gray},
  keywordstyle=\color{blue}
}

% Reference formatting
\newrefformat{fig}{Fig.~\ref{#1}}
\newcommand*{\term}[1]{\textit{#1}}

% Color boxes
\tcbuselibrary{skins, breakable, theorems}
\newtcbtheorem[number within=section]{warning}{Warning}%
  {colback=orange!5,colframe=orange!65,fonttitle=\bfseries, breakable}{warn}
\newtcbtheorem[number within=section]{note}{Note}%
  {colback=green!5,colframe=green!65,fonttitle=\bfseries, breakable}{note}
\newtcbtheorem[number within=section]{info}{Info}%
  {colback=blue!5,colframe=blue!65,fonttitle=\bfseries, breakable}{info}

\newenvironment{shelldisplay}{\begin{lstlisting}}{\end{lstlisting}}

\newcommand*{\kB}{k_{\text{B}}}
\newcommand*{\muB}{\mu_{\text{B}}}
\newcommand*{\efermi}{E_{\text{F}}}
\newcommand*{\pfermi}{p_{\text{F}}}
\newcommand*{\vfermi}{v_{\text{F}}}
\newcommand*{\sA}{\text{A}}
\newcommand*{\sB}{\text{B}}
\newcommand*{\Tc}{T_{\text{c}}}
\newcommand*{\hethree}{$^3$He}
\newcommand*{\hefour}{$^4$He}

\title{Fermi liquid}
\author{Jinyuan Wu}

\begin{document}

\maketitle

\section{Free electron gas}

A free electron gas is described by%
\footnote{
    For some reason, 
    when talking about Fermi liquid, 
    people like to use $\vb*{p}$ instead of $\vb*{k}$.
}
\begin{equation}
    H = \sum_i \frac{\vb*{p}_i^2}{2m},
\end{equation}
and we get a Fermi sphere in the momentum space 
containing all occupied states, 
and a Fermi momentum $\pfermi$.
After we add a periodic potential $V(\vb*{r})$ to the model, 
we get \concept{bands}: 
parabolic bands are displaced periodically 
according to the $\vb*{G}$-grid, 
and the crossing points of these curves 
undergo degeneracy breaking,
and the spectrum, eventually, 
becomes several bands confined in the first Brillouin zone. 
Despite this correction, 
the number of filled electrons never changes.

For a free electron gas, we have well-known observables like 
the density of states at $\efermi$ 
(here $\omega = \varepsilon - \efermi$)
\begin{equation}
    N(\omega = 0) = \frac{m}{\pi^2} \frac{\pfermi}{\hbar^3}, 
    \label{eq:free-electron.dos}
\end{equation}
the electronic specific heat 
\begin{equation}
    C_v = \underbrace{\frac{\pi^2 \kB^2}{3} N(\omega = 0)}_{\text{Sommerfield $\gamma$}} \cdot \ T, 
    \label{eq:free-electron.specific-heat}
\end{equation}
and the magnetic susceptibility 
\begin{equation}
    \chi = \mu_0  \muB^2 \cdot N(\omega = 0).
\end{equation}

\hethree{} atoms are fermions: 
we have two electrons, two protons, and only one neutron
in one \hethree{} atom, 
and therefore there are odd fermions inside 
and the composite state is a fermion. 
In a \hethree{} liquid, 
since the distance between the atoms is drastically reduced, 
we can expect the effect of repulsion is much stronger 
than the case in the gas phase. 
This is a cold atom model of electrons in solids: 
we can easily measure the specific heat, etc. 
of the liquid \hethree.

The $C_V$-$T$ curve measured looks like TODO: fig 1 
As is expected, 
when $T \to 0$, 
we see a $C_V \propto T$ behavior, 
qualitatively agreeing with \eqref{eq:free-electron.specific-heat}.
The coefficient however is not what is predicted by 
\eqref{eq:free-electron.dos} and
\eqref{eq:free-electron.specific-heat}.
Now if we apply pressure to liquid \hethree, 
$C_v / T$ clearly increases. 

For a fermion gas, 
we can measure the mass according to \eqref{eq:free-electron.specific-heat}
and \eqref{eq:free-electron.dos}. 
Now if we calculate the ``mass'' of \hethree according to the two equations, 
we get a pressure-dependent 
(quite surprising because the density of liquids almost stays the same when we press it)
``effective mass''
that is never the same as the bare mass of \hethree.

The next question, then, 
is whether liquid \hethree -- 
a system with known strong interaction -- 
can be equivalently described 
by something with the effective mass. 

\section{Assumptions of Fermi liquid theory}

\subsection{Motivation: scattering changes the distribution of electrons}

The general form of Hamiltonian of fermionic systems 
that is found in condensed matter physics looks like 
\begin{equation}
    H = \sum_i \frac{\vb*{p}_i^2}{2m} + 
    \sum_{i, j} V(\abs{\vb*{r}_i - \vb*{r}_j}).
\end{equation}
When we deal with a two-body problem with this dynamics, 
the collision behavior is completely described
by the scattering cross-section. 
If we extend this picture to a many-body system, 
we immediately see scattering changes 
the momentum distribution of particles, 
because the final states of scattering 
can obviously be over the Fermi surface, 
and therefore there is still some distribution above $\pfermi$ now. 

\subsection{Assumption: correspondence between interactive corrected states and uncorrected states}

\begin{figure}
    \centering
    \begin{tikzpicture}[x=0.75pt,y=0.75pt,yscale=-0.8,xscale=0.8]
    %uncomment if require: \path (0,300); %set diagram left start at 0, and has height of 300
    
    %Straight Lines [id:da009219527738367317] 
    \draw    (405,225) -- (617,225) ;
    \draw [shift={(619,225)}, rotate = 180] [fill={rgb, 255:red, 0; green, 0; blue, 0 }  ][line width=0.08]  [draw opacity=0] (12,-3) -- (0,0) -- (12,3) -- cycle    ;
    %Straight Lines [id:da2079961540240256] 
    \draw    (405,225) -- (405,84.5) ;
    \draw [shift={(405,82.5)}, rotate = 90] [fill={rgb, 255:red, 0; green, 0; blue, 0 }  ][line width=0.08]  [draw opacity=0] (12,-3) -- (0,0) -- (12,3) -- cycle    ;
    %Straight Lines [id:da25299815998590813] 
    \draw  [dash pattern={on 4.5pt off 4.5pt}]  (405,145.5) -- (512,145.5) ;
    %Straight Lines [id:da23611716817469208] 
    \draw  [dash pattern={on 4.5pt off 4.5pt}]  (512,145.5) -- (512,224.5) ;
    %Curve Lines [id:da9173257888464759] 
    \draw    (405,145.5) .. controls (500,143.5) and (519,226.5) .. (587,225) ;
    %Curve Lines [id:da006174472132643993] 
    \draw [draw opacity=0][fill={rgb, 255:red, 74; green, 144; blue, 226 }  ,fill opacity=0.5 ]   (405,146) .. controls (500,144) and (519,227) .. (587,225.5) .. controls (404,224) and (589,225) .. (405,225) .. controls (404,147) and (404,223) .. (405,146) -- cycle ;
    %Straight Lines [id:da4367238433084937] 
    \draw    (68,225) -- (280,225) ;
    \draw [shift={(282,225)}, rotate = 180] [fill={rgb, 255:red, 0; green, 0; blue, 0 }  ][line width=0.08]  [draw opacity=0] (12,-3) -- (0,0) -- (12,3) -- cycle    ;
    %Straight Lines [id:da3594146199392185] 
    \draw    (68,225) -- (68,84.5) ;
    \draw [shift={(68,82.5)}, rotate = 90] [fill={rgb, 255:red, 0; green, 0; blue, 0 }  ][line width=0.08]  [draw opacity=0] (12,-3) -- (0,0) -- (12,3) -- cycle    ;
    %Straight Lines [id:da3068774329734081] 
    \draw  [dash pattern={on 4.5pt off 4.5pt}]  (68,145.5) -- (175,145.5) ;
    %Straight Lines [id:da14782967759901622] 
    \draw  [dash pattern={on 4.5pt off 4.5pt}]  (175,145.5) -- (175,224.5) ;
    %Curve Lines [id:da7552878456427861] 
    \draw    (68,145.5) .. controls (110,144.5) and (137,146.5) .. (175,155.5) ;
    %Curve Lines [id:da40499842494527183] 
    \draw    (175,219.5) .. controls (195,222.5) and (217,224.5) .. (282,225) ;
    %Straight Lines [id:da31095032234024567] 
    \draw    (175,155.5) -- (175,219.5) ;
    %Curve Lines [id:da9323659415170968] 
    \draw [draw opacity=0][fill={rgb, 255:red, 74; green, 144; blue, 226 }  ,fill opacity=0.5 ]   (68,145.5) .. controls (131,145.5) and (120,144.5) .. (175,155.5) .. controls (176,220.5) and (174,159.5) .. (175,219.5) .. controls (227,224.5) and (229,225.5) .. (282,225) .. controls (83,223.5) and (262,226.5) .. (68,225) .. controls (69,151.5) and (67,226.5) .. (68,145.5) -- cycle ;
    %Straight Lines [id:da2726570766053802] 
    \draw [color={rgb, 255:red, 155; green, 155; blue, 155 }  ,draw opacity=1 ]   (175,155.5) -- (190,155.5) ;
    %Straight Lines [id:da794346269199613] 
    \draw [color={rgb, 255:red, 155; green, 155; blue, 155 }  ,draw opacity=1 ]   (175,219.5) -- (190,219.5) ;
    %Straight Lines [id:da8221957579922665] 
    \draw [color={rgb, 255:red, 155; green, 155; blue, 155 }  ,draw opacity=1 ]   (187,157.5) -- (187,217.5) ;
    \draw [shift={(187,219.5)}, rotate = 270] [fill={rgb, 255:red, 155; green, 155; blue, 155 }  ,fill opacity=1 ][line width=0.08]  [draw opacity=0] (12,-3) -- (0,0) -- (12,3) -- cycle    ;
    \draw [shift={(187,155.5)}, rotate = 90] [fill={rgb, 255:red, 155; green, 155; blue, 155 }  ,fill opacity=1 ][line width=0.08]  [draw opacity=0] (12,-3) -- (0,0) -- (12,3) -- cycle    ;
    
    % Text Node
    \draw (621,225) node [anchor=west] [inner sep=0.75pt]    {$k$};
    % Text Node
    \draw (405,79.5) node [anchor=south] [inner sep=0.75pt]    {$n_{k}$};
    % Text Node
    \draw (284,225) node [anchor=west] [inner sep=0.75pt]    {$k$};
    % Text Node
    \draw (68,79.5) node [anchor=south] [inner sep=0.75pt]    {$n_{k}$};
    % Text Node
    \draw (192,178) node [anchor=north west][inner sep=0.75pt]    {$Z$};
    % Text Node
    \draw (166,250) node [anchor=north west][inner sep=0.75pt]   [align=left] {(a)};
    % Text Node
    \draw (499,250) node [anchor=north west][inner sep=0.75pt]   [align=left] {(b)};
    
    
\end{tikzpicture}
    
    \caption{Two types of interaction correction to the electron distribution.
    (a) The Fermi liquid case: there is still a stepwise change of $n$ with respect to the energy.
    (b) The non-Fermi liquid case: the change of $n$ is smooth.}
    \label{fig:fermi-liquid-distribution}
\end{figure}

We assume that we still have ``states labeled by $\vb*{p}$, $n$, etc.'' 
in a Fermi liquid,
and the ground state and the first several excited states 
can be totally labeled by $\var{n(\vb*{p})}$,
which is $n(\vb*{p})$ -- some sort of ``filling'' on these states -- 
minus $n_0(\vb*{p})$ -- the Fermi gas filling. 
These states labeled by $\vb*{p}$, $n$, etc. 
are essentially the ``single-electron wave function''
in the single-electron Green function, 
and the ground state $n(\vb*{p})$ is essentially the spectral function.
We further assume $\var{n}$ is small, so that what we get is \prettyref{fig:fermi-liquid-distribution}(a),
instead of \prettyref{fig:fermi-liquid-distribution}(b).
This agrees with the observed fact that the $C_v \propto T$ relation is still true,
so Fermi liquid should be somehow similar to the Fermi gas. 
Also, we assume the total number of electrons 
should change considerably:
\begin{equation}
    \var{N} = \sum_{\vb*{p}} \var{n(\vb*{p})} \ll \sum_{\vb*{p}} n(\vb*{p}). 
\end{equation}

Note that although we are talking about ``partial occupation''
or even ``quasiparticle decaying'' (see the follows), 
Fermi liquid theory is still a \emph{pure state} theory; 
as is shown by Green function theory, 
partial occupation and even decaying can happening 
in the zero-temperature case, 
due to quantum fluctuation.
A single-electron state can evolve into a multiple-electron state, 
and therefore the ground state is a mixture of single-electron, double-electron, etc. states, 
and if we insist on a single-electron theory,
certain dissipation channels appear.
The Fermi liquid theory, essentially, 
is assuming that the single-particle picture still works well enough.

\subsection{The energy functional}

The energy of state $\{\var{n}_{\vb*{p}}\}$, $E$, reads 
\begin{equation}
    E = E_0 + \var{E}, \quad 
    \var{E}  = \sum_{\vb*{p}} \varepsilon_0(\vb*{p}) \var{n(\vb*{p})} + 
    \bigO(\var{n}^2).
    \label{eq:fermi-liquid.linear-energy}
\end{equation}
Here $E_0$ is the ground state energy of the free electron gas, 
\prettyref{eq:fermi-liquid.linear-energy} doesn't contain any many-body correction; 
a more accurate form of the energy of the system is 
\begin{equation}
    \var{E} = \sum_{\vb*{p}} \varepsilon_{\vb*{p}} \var{n(\vb*{p})}
    + \frac{1}{2 V} \sum_{\vb*{p}, \vb*{p}'}
    f_{\vb*{p} \vb*{p}'} \var{n(\vb*{p})} \var{n(\vb*{p}')} + \bigO(\var{n}^3).
\end{equation}
This means an electron in a Fermi liquid now receives a (almost trivial) self-energy correction 
\begin{equation}
    \var{E} = \sum_{\vb*{p}} \varepsilon(\vb*{p}) \var{n(\vb*{p})} , 
    \quad \varepsilon(\vb*{p}) = \varepsilon_0(\vb*{p}) 
    + \frac{1}{V} \sum_{\vb*{p}} f_{\vb*{p} \vb*{p}'} \var{n(\vb*{p}')}.
\end{equation}
The chemical potential can be immediately decided:
\begin{equation}
    \mu \coloneqq \eval{\fdv{E}{n(\vb*{p})}}_{\text{Fermi surface}} = \varepsilon(\pfermi).
\end{equation}

For condensed matter systems, 
we have to make a further modification to the energy functional shown above: 
we need to introduce spins. 
So the full theory is 
\begin{equation}
    \var{E} = \sum_{\vb*{p}, \sigma} \varepsilon_{0 \sigma} (\vb*{p}) \var{n_\sigma(\vb*{p})}
    + \frac{1}{2V} \sum_{\vb*{p}, \vb*{p}', \sigma, \sigma'}
    f_{\vb*{p} \vb*{p}' \sigma \sigma'} \var{n_{\sigma}(\vb*{p})} \var{n_{\sigma'} (\vb*{p}')}.
\end{equation}
When there is no outside spin polarizing factors 
like a magnetic field, 
we just have 
\begin{equation}
    f_{\uparrow \uparrow} = f_{\downarrow \downarrow}, \quad 
    f_{\uparrow \downarrow} = f_{\downarrow \uparrow}, \quad 
    \varepsilon_{0 \uparrow} = \varepsilon_{0 \downarrow} = \varepsilon_0,
\end{equation}
and therefore we have 
\[
    \begin{aligned}
        &\quad \sum_{\vb*{p}, \vb*{p}', \sigma, \sigma'} 
        f_{\vb*{p} \vb*{p}' \sigma \sigma'} \var{n_{\sigma}(\vb*{p})} \var{n_{\sigma'} (\vb*{p}')} \\
        &= \sum_{\vb*{p}, \vb*{p}'} \frac{1}{2} (f_{\uparrow \uparrow} + f_{\downarrow \uparrow})
        \var{n(\vb*{p})} \var{{n}(\vb*{p})}
        + \sum_{\vb*{p}, \vb*{p}'} \frac{1}{2} (f_{\uparrow \uparrow} - f_{\downarrow \uparrow})
        \var{\tilde{n}(\vb*{p})} \var{\tilde{n}(\vb*{p})},
    \end{aligned}
\]
where 
\begin{equation}
    \var{n(\vb*{p})} = \var{n_\uparrow(\vb*{p})} + \var{n_\downarrow(\vb*{p})}, \quad 
    \var{\tilde{n}(\vb*{p})} = \var{n_\uparrow(\vb*{p})} - \var{n_\downarrow(\vb*{p})}.
\end{equation}
We define 
\begin{equation}
    2f^{\text{S}} = f_{\uparrow \uparrow} + f_{\downarrow \uparrow}, \quad 
    2f^{\text{A}} = f_{\uparrow \uparrow} - f_{\downarrow \uparrow}, 
\end{equation}
and the interaction part of the energy functional then is 
\begin{equation}
    \sum_{\vb*{p}, \vb*{p}'} \left(
        f^{\text{S}}_{\vb*{p} \vb*{p}'} \var{n(\vb*{p})} \var{{n}(\vb*{p})}
    + f^{\text{A}}_{\vb*{p} \vb*{p}'} \var{\tilde{n}(\vb*{p})} \var{\tilde{n}(\vb*{p})}
    \right).
\end{equation}

Now we can make some symmetric arguments.
When we calculate physical quantities that only involves $\var{n}$, 
like the specific heat capacity, 
there is no need to deal with $\var{\tilde{n}}$;
when calculating the spin response the case is the opposite. 
We can do the angular momentum expansion 
\begin{equation}
    f^{\text{S/A}}_{\vb*{p} \vb*{p}'} = \sum_{l=0}^{\infty} f^{\text{S/A}}_l \legpoly_l(\cos \theta), 
    \label{eq:landau-liquid.angular-expansion}
\end{equation}
TODO: only the angle $\theta$ between $\vb*{p}$ and $\vb*{p}'$ matters. 
When we talk about an isotropic excitation mode, 
only the $l = 0$ term matters, 
while if we impose a magnetic field along one axis, 
then usually the $l = 1$ term has the strongest contribution. 

The expansion \prettyref{eq:landau-liquid.angular-expansion} 
also implies one way Landau Fermi liquid theory breaks: 
if high $l$ terms have very strong contribution to $\var{E}$, 
then even when we still have well-defined $\var{n(\vb*{p})}$, 
Landau Fermi liquid theory will be of limited use. 
Fortunately this is almost never the case. 

We usually define dimensionless Landau parameters 
\begin{equation}
    F^{\text{S/A}} = N(\omega = 0) f^{\text{S/A}},
\end{equation}
where $N(\omega = 0)$ is the density of states at $\efermi$.

\subsection{Some observables}

Following the procedure in electron gas, 
we define 
\begin{equation}
    \vfermi = \grad_{\vb*{p}} \varepsilon(\vb*{p}) |_{\pfermi}, 
\end{equation}
and 
\begin{equation}
    m^* = \frac{\pfermi}{\vfermi}.
\end{equation}
The density of states therefore is given by 
\begin{equation}
    N(\omega = 0) = m^* \frac{\pfermi}{\pi^2 \hbar^3}.
\end{equation}

This is of course related to the specific heat; 
let's derive the latter. 
Suppose we put a Fermi liquid under a finite temperature $T$.
The specific heat capacity can be found by 
\begin{equation}
    C_V = T \eval{\pdv{S}{T}}_V.
\end{equation}
The entropy is 
\begin{equation}
    S = - \kB \ln \Omega 
    = - \kB \frac{1}{V} \sum_{\vb*{p}, \sigma}
    (n_\sigma(\vb*{p}) \ln n_{\sigma}(\vb*{p})
    + (1 - n_\sigma(\vb*{p})) \ln (1 - n_{\sigma}(\vb*{p}))) ,
\end{equation} 
where now due to thermal fluctuation, 
we have 
\begin{equation}
    n_\sigma(\vb*{p}) = \frac{1}{
        1 + \ee^{(\varepsilon_\sigma(\vb*{p}) - \mu) / \kB T}
    },
\end{equation}
and we find 
\begin{equation}
    \var{S} = - \frac{1}{TV} \sum_{\vb*{p}, \sigma}
    (\varepsilon_{\sigma}(\vb*{p}) - \mu)
    \var{n_\sigma(\vb*{p})}
    = - \frac{1}{TV} \sum_{\vb*{p}, \sigma}
    (\varepsilon_{\sigma}(\vb*{p}) - \mu)
    \left(
        \pdv{n_\sigma(\vb*{p})}{\varepsilon_\sigma(\vb*{p})} \var{\epsilon_\sigma(\vb*{p})}
        + \pdv{n_\sigma(\vb*{p})}{T} \var{T}
    \right).
\end{equation}
TODO: eventually we find 
\begin{equation}
    \var{S} = - \frac{1}{TV} \int \dd{\varepsilon} N(\omega) 
    (\varepsilon - \mu) \frac{\varepsilon - \mu}{T} \dd{T},
\end{equation}
and 
\begin{equation}
    C_V = \frac{\pi^3}{3} N(\omega = 0) \kB^2 T.
\end{equation}

The next question is what's $m^*$.
There is actually a constraint imposed by 
the translational symmetry. 
Suppose we add a very small velocity $\vb*{v}$ globally. 
This means 
\begin{equation}
    \vb*{p} \to \vb*{p}' = \vb*{p} - m \vb*{v}, \quad 
    E \to E' = E - \vb*{p} \cdot \vb*{v} + \frac{1}{2} m v^2.
\end{equation}
Taylor expansion tells us
\begin{equation}
    \varepsilon(\vb*{p} - m \vb*{v}) = 
    \varepsilon(\vb*{p}) - \left(
        \frac{m}{m^*} - 1
    \right) \vb*{p} \cdot \vb*{v}, 
\end{equation}
and this has to agree with the 
TODO: 

\begin{equation}
    \frac{m^*}{m} = 1 + \frac{F^{\text{S}}_1}{3}.
\end{equation}
This means the effective mass only carries information about 
the symmetric part of $f_{\vb*{p} \vb*{p}' \sigma \sigma'}$.

\end{document}