\documentclass[hyperref, a4paper]{article}

\usepackage{geometry}
\usepackage{titling}
\usepackage{titlesec}
% No longer needed, since we will use enumitem package
% \usepackage{paralist}
\usepackage{enumitem}
\usepackage{footnote}
\usepackage{enumerate}
\usepackage{amsmath, amssymb, amsthm}
\usepackage{mathtools}
\usepackage{bbm}
\usepackage{graphicx}
\usepackage{subcaption}
\usepackage{physics}
\usepackage{tensor}
\usepackage{siunitx}
\usepackage[version=4]{mhchem}
\usepackage{tikz}
\usepackage{xcolor}
\usepackage{listings}
\usepackage{autobreak}
\usepackage[ruled, vlined, linesnumbered]{algorithm2e}
\usepackage{nameref,zref-xr}
\zxrsetup{toltxlabel}
\usepackage[sorting=none]{biblatex}
\bibliography{squeezing}
\usepackage[colorlinks,unicode]{hyperref} % , linkcolor=black, anchorcolor=black, citecolor=black, urlcolor=black, filecolor=black
\usepackage[most]{tcolorbox}
\usepackage{prettyref}

% Page style
\geometry{left=3.18cm,right=3.18cm,top=2.54cm,bottom=2.54cm}
\titlespacing{\paragraph}{0pt}{1pt}{10pt}[20pt]
\setlength{\droptitle}{-5em}

% More compact lists 
\setlist[itemize]{
    itemindent=17pt, 
    leftmargin=1pt,
    listparindent=\parindent,
    parsep=0pt,
}

% Math operators
\DeclareMathOperator{\timeorder}{\mathcal{T}}
\DeclareMathOperator{\diag}{diag}
\DeclareMathOperator{\legpoly}{P}
\DeclareMathOperator{\primevalue}{P}
\DeclareMathOperator{\sgn}{sgn}
\DeclareMathOperator{\res}{Res}
\newcommand*{\ii}{\mathrm{i}}
\newcommand*{\ee}{\mathrm{e}}
\newcommand*{\const}{\mathrm{const}}
\newcommand*{\suchthat}{\quad \text{s.t.} \quad}
\newcommand*{\argmin}{\arg\min}
\newcommand*{\argmax}{\arg\max}
\newcommand*{\normalorder}[1]{: #1 :}
\newcommand*{\pair}[1]{\langle #1 \rangle}
\newcommand*{\fd}[1]{\mathcal{D} #1}
\DeclareMathOperator{\bigO}{\mathcal{O}}

% TikZ setting
\usetikzlibrary{arrows,shapes,positioning}
\usetikzlibrary{arrows.meta}
\usetikzlibrary{decorations.markings}
\tikzstyle arrowstyle=[scale=1]
\tikzstyle directed=[postaction={decorate,decoration={markings,
    mark=at position .5 with {\arrow[arrowstyle]{stealth}}}}]
\tikzstyle ray=[directed, thick]
\tikzstyle dot=[anchor=base,fill,circle,inner sep=1pt]

% Algorithm setting
% Julia-style code
\SetKwIF{If}{ElseIf}{Else}{if}{}{elseif}{else}{end}
\SetKwFor{For}{for}{}{end}
\SetKwFor{While}{while}{}{end}
\SetKwProg{Function}{function}{}{end}
\SetArgSty{textnormal}

\newcommand*{\concept}[1]{{\textbf{#1}}}

% Embedded codes
\lstset{basicstyle=\ttfamily,
  showstringspaces=false,
  commentstyle=\color{gray},
  keywordstyle=\color{blue}
}

% Reference formatting
\newrefformat{fig}{Fig.~\ref{#1}}
\newcommand*{\term}[1]{\textit{#1}}

% Color boxes
\tcbuselibrary{skins, breakable, theorems}

\newtcbtheorem{infobox}{Box}{
    enhanced,
    boxrule=0pt,
    colback=blue!5,
    colframe=blue!5,
    coltitle=blue!50,
    borderline west={4pt}{0pt}{blue!65},
    sharp corners,
    fonttitle=\bfseries, 
    breakable,
    before upper={\parindent15pt\noindent}}{box}
\newtcbtheorem[use counter from=infobox]{theorybox}{Box}{
    enhanced,
    boxrule=0pt,
    colback=orange!5, 
    colframe=orange!5, 
    coltitle=orange!50,
    borderline west={4pt}{0pt}{orange!65},
    sharp corners,
    fonttitle=\bfseries, 
    breakable,
    before upper={\parindent15pt\noindent}}{box}
\newtcbtheorem[use counter from=infobox]{learnbox}{Box}{
    enhanced,
    boxrule=0pt,
    colback=green!5,
    colframe=green!5,
    coltitle=green!50,
    borderline west={4pt}{0pt}{green!65},
    sharp corners,
    fonttitle=\bfseries, 
    breakable,
    before upper={\parindent15pt\noindent}}{box}


\newenvironment{shelldisplay}{\begin{lstlisting}}{\end{lstlisting}}

\newcommand*{\kB}{k_{\text{B}}}
\newcommand*{\muB}{\mu_{\text{B}}}
\newcommand*{\efermi}{E_{\text{F}}}
\newcommand*{\pfermi}{p_{\text{F}}}
\newcommand*{\vfermi}{v_{\text{F}}}
\newcommand*{\sA}{\text{A}}
\newcommand*{\sB}{\text{B}}
\newcommand*{\Tc}{T_{\text{c}}}
\newcommand*{\hethree}{$^3$He}
\newcommand*{\hefour}{$^4$He}

\title{Phonon}
\author{Jinyuan Wu}


\begin{document}

\maketitle

\section{Lattice vibration is phonon}

\begin{equation}
    H = \frac{1}{2M} \sum_{j} {p}_j^2 + \frac{K}{2} \sum_{j} ({u}_{j+1} - {u}_j)^2.
\end{equation}
\begin{equation}
    {u}_j = \frac{1}{\sqrt{N}} \sum_q \ee^{\ii q j a} {u}_q,
\end{equation}
\begin{equation}
    {p}_j = \frac{1}{\sqrt{N}} \sum_q \ee^{\ii q j a} {p}_q,
\end{equation}
\begin{equation}
    u_j = \frac{1}{\sqrt{N}} \sum_q \sqrt{\frac{\hbar}{2 M \omega_q}} \ee^{\ii q j a}
    (a_k + a_{-k}^\dagger)
\end{equation}
\begin{equation}
    H = \sum_q \hbar \omega_q \left(
        a^\dagger_q a_q + \frac{1}{2}
    \right).
\end{equation}

\section{Electron-phonon coupling}

\subsection{First principles}

Recall that the total Hamiltonian of a condensed system is 
\begin{equation}
    H = \sum_{i} \frac{\vb*{p}_i^2}{2m} 
    + \sum_{i, j} \frac{e^2}{\abs*{\vb*{r}_i - \vb*{r}_j}}
    + \sum_{i, n} V(\vb*{r}_i - \vb*{R}_n)
    + \sum_{n} \frac{\vb*{p}_n^2}{2M_n}
    + \sum_{n, m} V(\vb*{R}_m - \vb*{R}_n),
\end{equation} 
where $i, j$ are indices of electrons 
and $m, n$ are the indices of atoms. 
The third part is the electron-lattice interaction energy; 
it's not the electron-phonon interaction energy yet, 
because it contains the energy of the interaction 
between electrons and the static lattice. 
The first and the second terms can be second-quantized.

In a crystal, we can replace $n$ by $n, \alpha$, 
where $n$ is the index of unit cells, 
and $\alpha$ is the index of atoms in one unit cell.
So the electron-lattice interaction Hamiltonian is now 
\begin{equation}
    H_{\text{electron-atom}} = \sum_{i, n, \alpha} V(\vb*{r}_i - \vb*{R}_{n \alpha})
    = \sum_{i, n, \alpha} V(\vb*{r}_i - \vb*{R}_{n \alpha}^0)
    + \sum_{i, n, \alpha} \vb*{u}_{n \alpha} \cdot \pdv{V(\vb*{r}_i - \vb*{R}_{n \alpha})}{\vb*{R}_{n \alpha}}
    + \cdots,
\end{equation}
where $\vb*{R}^0_{n \alpha}$ is the static position of the $n, \alpha$ atom, 
and $\vb*{u}_{n \alpha}$ is its displacement.

When the distortion of the lattice is small enough, 
we are in the \concept{linear expansion region}, 
and we can only keep the first two terms in the Taylor expansion.
The first term -- the static term -- 
gives rise to the band structure. 
The second term will be shown to be an electron-phonon coupling term.
Switching to the second quantization representation, 
this term is a single-body term for the electron degrees of freedom, 
and therefore we get 
\begin{equation}
    H_{\text{electron-atom}} = 
    \sum_{\vb*{k}, \vb*{k}'} \sum_{n, \alpha}
    \mel**{\vb*{k}}{\pdv{V(\vb*{r} - \vb*{R}_{n \alpha})}{\vb*{R}_{n \alpha}}}{\vb*{k}'} 
    \cdot \vb*{u}_{n \alpha} c^\dagger_{\vb*{k}} c_{\vb*{k}}.
\end{equation}
We can do expansion 
\begin{equation}
    V(\vb*{r} - \vb*{R}_{n \alpha}) 
    = \frac{1}{V} \sum_{\vb*{Q}} \ee^{\ii \vb*{Q} \cdot (\vb*{r} - \vb*{R}_{n \alpha})},
\end{equation}
and therefore 
\[
    \begin{aligned}
        &\quad \mel**{\vb*{k}}{\pdv{V(\vb*{r} - \vb*{R}_{n \alpha})}{\vb*{R}_{n \alpha}}}{\vb*{k}'} \\
        &= 
        \int\dd[3]{\vb*{r}} \frac{\ee^{- \ii \vb*{k} \cdot \vb*{r}} u^*_{\vb*{k}}(\vb*{r})}{\sqrt{V}} 
        \frac{1}{V} \sum_{\vb*{Q}}  (- \ii \vb*{Q} ) \ee^{\ii \vb*{Q} \cdot (\vb*{r} - \vb*{R}_{n \alpha})}
        \frac{\ee^{\ii \vb*{k}' \cdot \vb*{r}} u_{\vb*{k}'}(\vb*{r})}{\sqrt{V}} \\
        &= \sum_{\vb*{G}} \sum_{\vb*{Q}} \delta_{\vb*{G}, \vb*{Q} + \vb*{k}' - \vb*{k}}
        \sum_m \int_{\text{u.c.}} \dd[3]{\vb*{r}} 
        \frac{\ee^{\ii \vb*{r} \cdot (\vb*{Q} + \vb*{k}' - \vb*{k})}}{V} 
        u_{\vb*{k}}^* (\vb*{r}) u_{\vb*{k}'}(\vb*{r}) 
    \end{aligned}
\]
TODO: finish this 

\begin{equation}
    H_{\text{electron-phonon}} = \frac{1}{\sqrt{N}} \sum_{\vb*{k}, \vb*{q}}
    g_{\vb*{k} \vb*{q}} (b^\dagger_{- \vb*{q}} + b_{\vb*{q}}) c^\dagger_{\vb*{k} + \vb*{q}} c_{\vb*{k}}
\end{equation}
The dimension of $g_{\vb*{k} \vb*{q}}$ is energy; 
note that all the creation and annihilation operators
are dimensionless: 
this can be inferred from the fact that the commutation relation contains no dimension.
In Feynman diagram calculation, 
the $1 / \sqrt{N}$ factor should enter 
the mathematical expression corresponding to a diagram 
together with the vertex function $g_{\vb*{k} \vb*{q}}$;
since two vertices are connected together by a phonon propagator, 
the dependence of the sample size $V$
introduced by $1 / \sqrt{N}$ factor is canceled by 
the sum over $\vb*{q}$, 
and 
\[
    \sum_{\vb*{q}} \frac{1}{\sqrt{N}} g_{\vb*{k} \vb*{q}} \cdot (\cdots) \cdot \frac{1}{\sqrt{N}} g_{\vb*{q} \vb*{k}'} = \int \frac{\dd[3]{\vb*{q}}}{(2\pi)^3} g_{\vb*{k} \vb*{q}} g_{\vb*{q} \vb*{k}'} \cdot (\cdots).
\]

Inter-band electron-phonon scattering 
(i.e. a scattering in which after the scattering, 
the band index of the electron changes)
is usually not so common, 
because the magnitude of electron band gap is $\sim \SI{1}{eV}$,
which is much larger than the phonon energy.
But for semiconductors with small band gaps or metals, 
this condition breaks, 
and the full Hamiltonian involving inter-band scattering has to be used. 

\subsection{Nearly free electron and phonons}

Similar to the case of electron-electron Coulomb scattering, 
several interaction channels with clear physical pictures can be recognized,
and effective models have been proposed to capture the major behaviors of them.
The \concept{Frohlich Hamiltonian} models 
the interaction between nearly free electrons 
and longitude acoustic phonons
\begin{equation}
    H = \sum_{\vb*{k}} \varepsilon_{\vb*{k}} c^\dagger_{\vb*{k}} c_{\vb*{k}}
    + \sum_{\vb*{q}} \omega_{\vb*{q}} b^\dagger_{\vb*{q}} b_{\vb*{q}}
    + \frac{1}{\sqrt{N}} \sum_{\vb*{k}, \vb*{q}} 
    g_{\vb*{k} \vb*{q}} (b_{- \vb*{q}}^\dagger + b_{\vb*{q}}) c^\dagger_{\vb*{k} + \vb*{q}} c_{\vb*{k}},
\end{equation}
where 
\begin{equation} 
    g_{\vb*{k} \vb*{q}} = \sqrt{\frac{\hbar}{2 m \omega_{\vb*{q}}}} \vb*{q} \cdot \vu*{\epsilon} V_{\vb*{q}}, 
    \quad \omega_{\vb*{q}} = v_{\text{s}} \abs*{\vb*{q}}, 
\end{equation}
and $V_{\vb*{q}}$ is the screened Coulomb potential
(or the unscreened Coulomb potential 
when screening is not that strong).
This means as the system becomes metallic, 
the phonon line width should be smaller, 
because the potential is now screened 
and the scattering rate decreases.

Note that $\vb*{q}$ is the crystal momentum of phonons
plus a reciprocal lattice vector, 
and $\vu*{\epsilon}$ is the polarization direction;
this means if a phonon mode is transverse, 
then it has no interaction with the electrons 
in the linear expansion region
when $\vb*{G} = 0$.
When $\vb*{G} \neq 0$, 
we still have some interaction strength???
Without other mechanisms, like strong coupling between transverse and longitudinal phonons, 
interaction between transverse phonon modes and electrons 
is to be ignored.
This can be understood in the continuous limit,
where electrons interact with the charge density of the lattice,
and transverse phonon modes don't 
disrupt the density.

\subsection{Localized electrons and phonons}

If the electrons are not near-free (i.e. localized), 
we no longer work in the $\vb*{k}$-representation, 
and that leads to a change of the form of the effective model 
of electron-phonon interaction. 
\concept{Holstein Hamiltonian} captures a scenario in this case,
where the coupling between electrons and phonons is highly localized, 
and the phonons are approximately Einstein phonons (TODO: why?).
The Hamiltonian is
\begin{equation}
    H = - \sum_{\vb*{i}, \vb*{j}} t_{\vb*{i} \vb*{j}} c^\dagger_{\vb*{i}} c_{\vb*{j}}
    + \omega_0 \sum_{\vb*{i}} b^\dagger_{\vb*{i}} b_{\vb*{i}}
    + g \sum_{\vb*{i}} c^\dagger_{\vb*{i}} c_{\vb*{i}} (b^\dagger_{\vb*{i}} + b_{\vb*{i}}).
\end{equation}
The interaction term essentially is $g X_{\vb*{i}} n_{\vb*{i}}$, 
the physical of which is that 
atomic displacement at a particular point creates
a change in the density of positive charges,
and therefore an electrostatic field is established 
and is felt by the electron density.

The Holstein Hamiltonian is a toy model for structural distortion:
we can arrange the electron density and $X_{\vb*{i}}$
to change periodically so that at each site 
the electron-phonon interaction term is negative 
and by spontaneous reduction of symmetry,
we get a state with lower energy.

We can also generalize the Holstein model 
by considering interaction between $n_{\vb*{i}}$ and $X_{\vb*{j}}$.
This happens especially when Coulomb screening is weak, 
and long-range interaction is therefore possible.
When this condition -- that the Coulomb interaction between electrons and charge fluctuation of the lattice is long-range --
and the electron density is small, 
the Holstein model is just the Frohlich model.

\subsection{Structural distortion's influence on electron bands (and what interaction channel is needed to do this)}

Permanently displaced atoms due to electron-phonon coupling 
also changes the wave function overlap of band electrons,
and the latter can be captured by a mean-field Hamiltonian 
in which the fluctuation of atomic positions is ignored.
People may also call this \concept{Peierls Hamiltonian},
because the structural change is usually known as 
\concept{Peierls transition}.
It's originally developed for polyacetylene,
in which the alternating hopping coefficients 
is due to electronic structures instead of atomic positions,
but it works well for systems with Peierls transition.
The Hamiltonian is 
\begin{equation}
    H = - \sum_{\pair{\vb*{i}, \vb*{j}}} 
    t_{\vb*{i} \vb*{j}} c^\dagger_{\vb*{i}} c_{\vb*{j}},
\end{equation}
where by Taylor series, we have
\begin{equation}
    t_{\vb*{i} \vb*{j}} = \underbrace{t(\vb*{R}_{\vb*{i}}^{(0)} - \vb*{R}^{(0)}_{\vb*{j}})}_{t_0} 
    (1 \pm \alpha (\vb*{X}_{\vb*{i}} - \vb*{X}_{\vb*{j}}) + \cdots),
    \quad \vb*{R}_{\vb*{i}} = \vb*{R}^{(0)}_{\vb*{i}} + \vb*{X}_{\vb*{i}}.
\end{equation}

In 1D systems, a toy-model capturing this is the 
\concept{Su-Schrieffer-Heege (SSH) model}.
This model considers the simplest case of the influence of Peierls transition to electron band,
which only considers nearest-neighbor hopping.
If the material before Peierls transition only has nearest-neighbor hopping, 
then usually after Peierls transition, 
nearest-neighbor hopping is also enough,
except when some exotic interaction channels 
intermediate long-range hopping.
So the model now (note that we are in 1D) is 
\begin{equation}
    H = \sum_{\pair{m, n}} t_0 \alpha (X_{n} - X_m) c^\dagger_m c_n.
\end{equation}

When we do consider the fluctuation of atomic positions,
this Hamiltonian can be rewritten into the following electron-phonon interaction channel:
\begin{equation}
    H_{\text{electron-phonon coupling}}
    = - g \sum_{n} (c^\dagger_n c_{n+1} + c^\dagger_{n+1} c_n) (b^\dagger_{n} + b_n - b).
\end{equation}
This interaction channel is no longer 
a ``electron density responding to electrostatic field'' channel
(which is the underlying mechanism below both Frohlich and Holstein models).
The type of phonon involved in forming the SSH model 
is called a \concept{bond phonon},
because it changes the length of chemical bonds between atoms.
The interaction channel sketched above involving bond phonons 
is the third important type of electron-phonon coupling.
Note that it's not simply a density coupling.

\section{Corrections brought by electron-phonon coupling}

\subsection{Correction to single-electron spectrum}

Usually the influence of electron-phonon coupling can be treated as a perturbation,
because $g_{\vb*{k} \vb*{q}}$ is usually much smaller than both $\varepsilon_{\vb*{k}}$ and $\omega_{\vb*{q}}$
($\sim \SI{1}{meV}$ to $\SI{50}{meV}$),
and then $\varepsilon_{\vb*{k}}$ is much larger than $\omega_{\vb*{q}}$.

We first consider the correction to a single-electron state 
by the presence of phonons.
We also work under a very low temperature,
so that there is no thermal phonon in the system
(otherwise we need to consider thermally corrected -- and broadened -- 
single-electron spectrum, and things can be much more complicated). 
We do perturbative calculation up to second order.
Here we do an old-fashioned perturbation theory: 
\begin{equation}
    \tilde{\varepsilon}_{\vb*{k}} = \varepsilon_{\vb*{k}}^0 
    + \mel*{0}{c_{\vb*{k}} H_{\text{epc}} c^\dagger_{\vb*{k}}}{0}
    + \mel*{0}{c_{\vb*{k}} H_{\text{epc}} \frac{1 - c^\dagger_{\vb*{k}} \dyad*{0}{0} c_{\vb*{k}}}{\varepsilon^0_{\vb*{k}} - H_0} H_{\text{epc}} c^\dagger_{\vb*{k}} }{0}.
    \label{eq:phonon-perturbation-energy}
\end{equation}
The first-order term is proportional to $n_{\vb*{q} = 0}$,
and the non-existence of phonons in the ground state 
means this term vanishes.
The $\dyad{0}$ part in the second-order term vanishes
for the same reason the first-order term vanishes.
Then it's easy to see that 
\begin{equation}
    \tilde{\varepsilon}_{\vb*{k}} = \varepsilon_{\vb*{k}}^{0} - \frac{1}{N} \sum_{\vb*{q}} \frac{g_{\vb*{q}}^2}{\varepsilon^0_{\vb*{k} + \vb*{q}} + \omega_{-\vb*{q}} - \varepsilon_{\vb*{k}}^0}.
    \label{eq:electron-spectrum-phonon}
\end{equation}
This can also be obtained using self-energy correction 
(and this is expected because of the equivalence 
between Goldstone diagrams and Feynman diagrams).

As a rudimentary demonstration of what \eqref{eq:electron-spectrum-phonon} means, 
we assume the electron is nearly free 
and the phonon is Einstein phonon,
and this means 
\[
    \varepsilon_{\vb*{k}}^0 = \frac{\vb*{k}^2}{2m},
\]
and therefore
\begin{equation}
    \begin{aligned}
        \tilde{\varepsilon}_{\vb*{k}} &= \varepsilon^0_{\vb*{k}} - \frac{1}{N} \sum_{\vb*{q}} \frac{g_{\vb*{q}}^2}{\omega_0 + \frac{\vb*{q}^2}{2m} + \frac{\vb*{k} \cdot \vb*{q}}{m}} \\
        &= \varepsilon^0_{\vb*{k}} - \frac{1}{N} \sum_{\vb*{q}} \frac{g_{\vb*{q}}^2}{\omega_0 + \varepsilon^0_{\vb*{q}}} \left(
            1 + \frac{\vb*{k} \cdot \vb*{q}}{m (\omega_0 + \varepsilon_{\vb*{q}}^0)} + 
            \left(
                \frac{\vb*{k} \cdot \vb*{q}}{m (\omega_0 + \varepsilon_{\vb*{q}}^0)}
            \right)^2 + \cdots
        \right) ,
    \end{aligned}
\end{equation}
Here the $m$ parameter may be corrected by the crystal potential 
and/or electron-electron scattering (as in Fermi liquid).
Replacing the sum over the $\vb*{q}$ grid by an integral, 
we have 
\begin{equation}
    \begin{aligned}
        \tilde{\varepsilon}_{\vb*{k}} &= \varepsilon^0_{\vb*{k}}
        - V_{\text{u.c.}} \int \frac{\dd[3]{\vb*{q}}}{(2\pi)^3} \frac{g_{\vb*{q}}^2}{\omega_0 + \varepsilon^0_{\vb*{q}}} \left(
            1 + 
            \left(
                \frac{\vb*{k} \cdot \vb*{q}}{m (\omega_0 + \varepsilon_{\vb*{q}}^0)}
            \right)^2 + \cdots
        \right) \\
        &= \varepsilon^0_{\vb*{k}} - V_{\text{u.c.}} \int \frac{\dd[3]{\vb*{q}}}{(2\pi)^3} \frac{g_{\vb*{q}}^2}{\omega_0 + \varepsilon^0_{\vb*{q}}} 
        - V_{\text{u.c.}} \cdot \frac{\vb*{k}^2}{2m} \cdot \frac{2}{3m} \int \frac{\dd[3]{\vb*{q}}}{(2\pi)^3} \frac{g_{\vb*{q}}^2 }{\omega_0 + \varepsilon^0_{\vb*{q}}} .
    \end{aligned}
\end{equation}
Here we have used the fact that the $\vb*{k} \cdot \vb*{q}$ term vanishes in the integration, 
and that 
\begin{equation}
    \int \dd[3]{\vb*{q}} f(\abs{\vb*{q}}) (\vb*{k} \cdot \vb*{q})^2 = 
    \frac{1}{3} \vb*{k}^2 \cdot \int \dd[3]{\vb*{q}} f(\abs{\vb*{q}}) \vb*{q}^2.
\end{equation}
The corrected single-electron energy then becomes 
\begin{equation}
    \tilde{\varepsilon}_{\vb*{k}} = (1 - \lambda) \frac{\vb*{k}^2}{2m} - \Delta E,
\end{equation}
where $\Delta E, \lambda \sim g^2$.
Since $g$ is small, we find the dispersion relation can be rewritten as 
\begin{equation}
    \tilde{\epsilon}_{\vb*{k}} = -\Delta E + \frac{\vb*{k}^2}{2 m^*}, \quad m^* = (1 + \lambda) m.
\end{equation}

So we find the existence of phonons -- essentially \emph{virtual} phonons -- 
makes the mass of electrons heavier.
Or should we call the quasiparticle in question here ``electrons''?
The single-particle wave function is now 
\begin{equation}
    \begin{aligned}
        \ket*{\psi_{\vb*{k}}} &= c^\dagger_{\vb*{k}} \ket*{0}
        + \frac{1}{\sqrt{N}} \sum_{\vb*{q}} \frac{g_{\vb*{q}}}{\varepsilon^0_{\vb*{k} + \vb*{q}} + \omega_{-\vb*{q}} - \omega^0_{\vb*{k}}} (b^\dagger_{\vb*{-q}} + b_{\vb*{q}}) c^\dagger_{\vb*{k} + \vb*{q}} \ket*{0} \\
        &= c^\dagger_{\vb*{k}} \ket*{0}
        + \frac{1}{\sqrt{N}} \sum_{\vb*{q}} \frac{g_{\vb*{q}}}{\epsilon^0_{\vb*{k} + \vb*{q}} + \omega_0 - \epsilon^0_{\vb*{k}}} b^\dagger_{- \vb*{q}} c^\dagger_{\vb*{k} + \vb*{q}} \ket*{0}.
    \end{aligned}
    \label{eq:polaron-wave-function}
\end{equation}
So there is a small component in the wave function of this quasiparticle 
in which a phonon comes together with the electron.
This is like the picture in Fermi liquid:
the wave function of a quasiparticle 
have a predominant single-electron component, 
but other components in which there are an electron and several electron-hole pairs 
are also included.
If we display the components in the wave function as fluctuating configurations, 
then we can say when phonon coupling exists, 
an electron is surrounded by a cloud of virtual phonons.
The quasiparticle described by \eqref{eq:polaron-wave-function} is called \concept{polaron}.

\subsection{What happens in a Fermi liquid ground state}

Any real condensed matter system contains a soup of electrons.
The question then is what phonons do to the Fermi liquid ground state.
Now in \eqref{eq:phonon-perturbation-energy}, 
the second order term needs one modification:
the electron-phonon interaction vertex 
involves creation of another electron, 
which is only possible when the end state is outside of the Fermi sea.
The corrected single-electron energy then is 
\begin{equation}
    \tilde{\varepsilon}_{\vb*{k}} = \varepsilon_{\vb*{k}}^{0} - \frac{1}{N} \sum_{\vb*{q}} \frac{g_{\vb*{q}}^2}{\varepsilon^0_{\vb*{k} + \vb*{q}} + \omega_{-\vb*{q}} - \varepsilon_{\vb*{k}}^0} (1 - \expval{n_{\vb*{k} + \vb*{q}}}).
    \label{eq:polaron-in-fl}
\end{equation}
This means the phonon correction is only important when $\abs{\varepsilon_{\vb*{k}} - \varepsilon_{\text{F}}} \lesssim \omega_{\vb*{q}}$;
specifically, the imaginary part of \eqref{eq:polaron-in-fl} (obtained by adding an infinitesimal imaginary part of the denominator) 
is only non-zero when there exists a phonon mode such that $\abs{\varepsilon_{\vb*{k}} - \varepsilon_{\text{F}}} \leq \omega_{\vb*{q}}$.
A kink therefore can be observed in the electron spectral function:
above $\varepsilon_{\text{F}} - \omega_0$ (where $\omega_0$ is the maximal phonon frequency)
and below $\varepsilon_{\text{F}}$, 
we see a more broadened, less dispersive (because of the increased mass) band, 
which is the phonon-corrected electron band i.e. the polaron band;
below $\varepsilon_{\text{F}} - \omega_0$ however phonon correction is very weak,
and the electron band all of a sudden becomes more dispersive.
This prediction has been directly verified using ARPES.

\subsection{Effective attraction interaction and BCS}

In a superconductor, for some reason electrons in the Fermi sea attract each other, 
so the Fermi surface becomes instable 
and are completely destroyed,
and the low-energy excitations become electron pairs and not single electrons.

Phonon-induced effective Hamiltonian can be found in the same way we calculate electron energy renormalization.
It is 
\begin{equation}
    \begin{aligned}
        V_{\vb*{k} \vb*{k}' \vb*{q}} &= \mel*{0}{c_{\vb*{k}' - \vb*{q}} c_{\vb*{k} + \vb*{q}} H_{\text{epc}} \frac{1}{E_0 - H_0} H_{\text{epc}} c^\dagger_{\vb*{k}} c^\dagger_{\vb*{k}'}}{0} \\
        &= 2 \abs{g_{\vb*{q}}}^2 \left(
            \frac{1 - \expval{n_{\vb*{k} + \vb*{q}}}}{\varepsilon^0_{\vb*{k}} - \varepsilon^0_{\vb*{k} + \vb*{q}} - \omega_{- \vb*{q}}}
            + \frac{1 - \expval{n_{\vb*{k}' - \vb*{q}}}}{\varepsilon^0_{\vb*{k}'} - \varepsilon^0_{\vb*{k}' - \vb*{q}} - \omega_{\vb*{q}}}
        \right).
    \end{aligned}
\end{equation}
Again we pick up the approximation that $\omega_{\vb*{q}} = \omega_0$.
TODO: attraction condition

\subsection{Phonon softening and Kohn anomaly}

Similarly we can do self-energy correction to the phonons.
The frequency change is 
\begin{equation}
    \begin{aligned}
        \tilde{\omega}_{\vb*{q}} &= \omega^0_{\vb*{q}} + \frac{1}{N} \sum_{\vb*{k}} \abs{g_{\vb*{q}}}^2 \left(
            \frac{\expval{n_{\vb*{k}} (1 - n_{\vb*{k} + \vb*{q}})}}{\omega^0_{\vb*{q}} + \varepsilon^0_{\vb*{k}} - \varepsilon^0_{\vb*{k} + \vb*{q}}}
            + \frac{\expval{n_{\vb*{k}} (1 - n_{\vb*{k} - \vb*{q}})}}{\varepsilon^0_{\vb*{k}} - \varepsilon^0_{\vb*{k} - \vb*{q}} - \omega^0_{\vb*{q}}}
        \right) \\
        &= \omega^0_{\vb*{q}} + \frac{\abs{g_{\vb*{q}}}^2}{N} \sum_{\vb*{k}} 
        \frac{\expval{n_{\vb*{k}}} - \expval{n_{\vb*{k} + \vb*{q}}}}{\varepsilon^0_{\vb*{k}} - \varepsilon^0_{\vb*{k} + \vb*{q}} + \omega_{\vb*{q}}}.
    \end{aligned}
\end{equation}
Here we see something similar to the polaron-electron kink:
in order for the correction to be huge, 
we need to have $\abs*{\varepsilon^0_{\vb*{k}} - \varepsilon^0_{\vb*{k} + \vb*{q}}} \lesssim \omega^0_{\vb*{q}}$.
One way to do so is to let $\vb*{k}$ approach $k_\text{F}$ in one direction,
and let $\vb*{k} + \vb*{q}$ to approach $k_{\text{F}}$ in the opposite direction.
Then, we find $\abs{\vb*{q}} \sim 2 k_{\text{F}}$.
When this happens, we have a large negative correction to the phonon frequency 
(if $n_{\vb*{k}} = 1$, $n_{\vb*{k} + \vb*{q}} = 0$, then $\varepsilon^0_{\vb*{k}} < \varepsilon^0_{\vb*{k} + \vb*{q}}$).
This softening of phonon dispersion relation 
is called \concept{Kohn anomaly}.

If we manage to make $\tilde{\omega}_{\vb*{q}} = 0$ when $\abs{\vb*{q}} \sim 2 k_{\text{F}}$,
then it costs no energy to add one phonon, 
and at the ground state we have a phonon condensation phase at $q = 2 k_{\text{F}}$,
which means the lattice is distorted.
This is \concept{Peierls transition}.
After Peierls transition, a Brillouin zone folding happens, 
and the phonon mode that has Kohn anomaly 
splits into two phonon modes, 
and the part of the spectrum after $q = 2 k_{\text{F}}$
is moved back into the shrunk Brillouin zone 
and is mixed into the original acoustic mode,
because in the new Brillouin zone, 
around $\vb*{q} = 0$, the frequency is small.
(And correspondingly, a part of the original acoustic mode, 
due to crossing with the optical mode near $q = 2 k_{\text{F}}$,
is integrated into a new optical mode.)
Note that this phenomenon only happens in 1D, 
for in 2D or 3D, the structural susceptibility 
doesn't diverge near $2 k_{\text{F}}$.

\end{document}