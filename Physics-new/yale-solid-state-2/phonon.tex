\documentclass[hyperref, a4paper]{article}

\usepackage{geometry}
\usepackage{titling}
\usepackage{titlesec}
% No longer needed, since we will use enumitem package
% \usepackage{paralist}
\usepackage{enumitem}
\usepackage{footnote}
\usepackage{enumerate}
\usepackage{amsmath, amssymb, amsthm}
\usepackage{mathtools}
\usepackage{bbm}
\usepackage{graphicx}
\usepackage{subcaption}
\usepackage{physics}
\usepackage{tensor}
\usepackage{siunitx}
\usepackage[version=4]{mhchem}
\usepackage{tikz}
\usepackage{xcolor}
\usepackage{listings}
\usepackage{autobreak}
\usepackage[ruled, vlined, linesnumbered]{algorithm2e}
\usepackage{nameref,zref-xr}
\zxrsetup{toltxlabel}
\usepackage[sorting=none]{biblatex}
\bibliography{squeezing}
\usepackage[colorlinks,unicode]{hyperref} % , linkcolor=black, anchorcolor=black, citecolor=black, urlcolor=black, filecolor=black
\usepackage[most]{tcolorbox}
\usepackage{prettyref}

% Page style
\geometry{left=3.18cm,right=3.18cm,top=2.54cm,bottom=2.54cm}
\titlespacing{\paragraph}{0pt}{1pt}{10pt}[20pt]
\setlength{\droptitle}{-5em}

% More compact lists 
\setlist[itemize]{
    itemindent=17pt, 
    leftmargin=1pt,
    listparindent=\parindent,
    parsep=0pt,
}

% Math operators
\DeclareMathOperator{\timeorder}{\mathcal{T}}
\DeclareMathOperator{\diag}{diag}
\DeclareMathOperator{\legpoly}{P}
\DeclareMathOperator{\primevalue}{P}
\DeclareMathOperator{\sgn}{sgn}
\DeclareMathOperator{\res}{Res}
\newcommand*{\ii}{\mathrm{i}}
\newcommand*{\ee}{\mathrm{e}}
\newcommand*{\const}{\mathrm{const}}
\newcommand*{\suchthat}{\quad \text{s.t.} \quad}
\newcommand*{\argmin}{\arg\min}
\newcommand*{\argmax}{\arg\max}
\newcommand*{\normalorder}[1]{: #1 :}
\newcommand*{\pair}[1]{\langle #1 \rangle}
\newcommand*{\fd}[1]{\mathcal{D} #1}
\DeclareMathOperator{\bigO}{\mathcal{O}}

% TikZ setting
\usetikzlibrary{arrows,shapes,positioning}
\usetikzlibrary{arrows.meta}
\usetikzlibrary{decorations.markings}
\tikzstyle arrowstyle=[scale=1]
\tikzstyle directed=[postaction={decorate,decoration={markings,
    mark=at position .5 with {\arrow[arrowstyle]{stealth}}}}]
\tikzstyle ray=[directed, thick]
\tikzstyle dot=[anchor=base,fill,circle,inner sep=1pt]

% Algorithm setting
% Julia-style code
\SetKwIF{If}{ElseIf}{Else}{if}{}{elseif}{else}{end}
\SetKwFor{For}{for}{}{end}
\SetKwFor{While}{while}{}{end}
\SetKwProg{Function}{function}{}{end}
\SetArgSty{textnormal}

\newcommand*{\concept}[1]{{\textbf{#1}}}

% Embedded codes
\lstset{basicstyle=\ttfamily,
  showstringspaces=false,
  commentstyle=\color{gray},
  keywordstyle=\color{blue}
}

% Reference formatting
\newrefformat{fig}{Fig.~\ref{#1}}
\newcommand*{\term}[1]{\textit{#1}}

% Color boxes
\tcbuselibrary{skins, breakable, theorems}

\newtcbtheorem{infobox}{Box}{
    enhanced,
    boxrule=0pt,
    colback=blue!5,
    colframe=blue!5,
    coltitle=blue!50,
    borderline west={4pt}{0pt}{blue!65},
    sharp corners,
    fonttitle=\bfseries, 
    breakable,
    before upper={\parindent15pt\noindent}}{box}
\newtcbtheorem[use counter from=infobox]{theorybox}{Box}{
    enhanced,
    boxrule=0pt,
    colback=orange!5, 
    colframe=orange!5, 
    coltitle=orange!50,
    borderline west={4pt}{0pt}{orange!65},
    sharp corners,
    fonttitle=\bfseries, 
    breakable,
    before upper={\parindent15pt\noindent}}{box}
\newtcbtheorem[use counter from=infobox]{learnbox}{Box}{
    enhanced,
    boxrule=0pt,
    colback=green!5,
    colframe=green!5,
    coltitle=green!50,
    borderline west={4pt}{0pt}{green!65},
    sharp corners,
    fonttitle=\bfseries, 
    breakable,
    before upper={\parindent15pt\noindent}}{box}


\newenvironment{shelldisplay}{\begin{lstlisting}}{\end{lstlisting}}

\newcommand*{\kB}{k_{\text{B}}}
\newcommand*{\muB}{\mu_{\text{B}}}
\newcommand*{\efermi}{E_{\text{F}}}
\newcommand*{\pfermi}{p_{\text{F}}}
\newcommand*{\vfermi}{v_{\text{F}}}
\newcommand*{\sA}{\text{A}}
\newcommand*{\sB}{\text{B}}
\newcommand*{\Tc}{T_{\text{c}}}
\newcommand*{\hethree}{$^3$He}
\newcommand*{\hefour}{$^4$He}

\title{Phonon}
\author{Jinyuan Wu}


\begin{document}

\maketitle

\section{Lattice vibration is phonon}

\begin{equation}
    H = \frac{1}{2M} \sum_{j} {p}_j^2 + \frac{K}{2} \sum_{j} ({u}_{j+1} - {u}_j)^2.
\end{equation}
\begin{equation}
    {u}_j = \frac{1}{\sqrt{N}} \sum_q \ee^{\ii q j a} {u}_q,
\end{equation}
\begin{equation}
    {p}_j = \frac{1}{\sqrt{N}} \sum_q \ee^{\ii q j a} {p}_q,
\end{equation}
\begin{equation}
    u_j = \frac{1}{\sqrt{N}} \sum_q \sqrt{\frac{\hbar}{2 M \omega_q}} \ee^{\ii q j a}
    (a_k + a_{-k}^\dagger)
\end{equation}
\begin{equation}
    H = \sum_q \hbar \omega_q \left(
        a^\dagger_q a_q + \frac{1}{2}
    \right).
\end{equation}

\section{Electron-phonon coupling}

Recall that the total Hamiltonian of a condensed system is 
\begin{equation}
    H = \sum_{i} \frac{\vb*{p}_i^2}{2m} 
    + \sum_{i, j} \frac{e^2}{\abs*{\vb*{r}_i - \vb*{r}_j}}
    + \sum_{i, n} V(\vb*{r}_i - \vb*{R}_n)
    + \sum_{n} \frac{\vb*{p}_n^2}{2M_n}
    + \sum_{n, m} V(\vb*{R}_m - \vb*{R}_n),
\end{equation} 
where $i, j$ are indices of electrons 
and $m, n$ are the indices of atoms. 
The third part is the electron-lattice interaction energy; 
it's not the electron-phonon interaction energy yet, 
because it contains the energy of the interaction 
between electrons and the static lattice. 
The first and the second terms can be second-quantized.

In a crystal, we can replace $n$ by $n, \alpha$, 
where $n$ is the index of unit cells, 
and $\alpha$ is the index of atoms in one unit cell.
So the electron-lattice interaction Hamiltonian is now 
\begin{equation}
    H_{\text{electron-atom}} = \sum_{i, n, \alpha} V(\vb*{r}_i - \vb*{R}_{n \alpha})
    = \sum_{i, n, \alpha} V(\vb*{r}_i - \vb*{R}_{n \alpha}^0)
    + \sum_{i, n, \alpha} \vb*{u}_{n \alpha} \cdot \pdv{V(\vb*{r}_i - \vb*{R}_{n \alpha})}{\vb*{R}_{n \alpha}}
    + \cdots,
\end{equation}
where $\vb*{R}^0_{n \alpha}$ is the static position of the $n, \alpha$ atom, 
and $\vb*{u}_{n \alpha}$ is its displacement.

When the distortion of the lattice is small enough, 
we are in the \concept{linear expansion region}, 
and we can only keep the first two terms in the Taylor expansion.
The first term -- the static term -- 
gives rise to the band structure. 
The second term will be shown to be an electron-phonon coupling term.
Switching to the second quantization representation, 
this term is a single-body term for the electron degrees of freedom, 
and therefore we get 
\begin{equation}
    H_{\text{electron-atom}} = 
    \sum_{\vb*{k}, \vb*{k}'} \sum_{n, \alpha}
    \mel**{\vb*{k}}{\pdv{V(\vb*{r} - \vb*{R}_{n \alpha})}{\vb*{R}_{n \alpha}}}{\vb*{k}'} 
    \cdot \vb*{u}_{n \alpha} c^\dagger_{\vb*{k}} c_{\vb*{k}}.
\end{equation}
We can do expansion 
\begin{equation}
    V(\vb*{r} - \vb*{R}_{n \alpha}) 
    = \frac{1}{V} \sum_{\vb*{Q}} \ee^{\ii \vb*{Q} \cdot (\vb*{r} - \vb*{R}_{n \alpha})},
\end{equation}
and therefore 
\[
    \begin{aligned}
        &\quad \mel**{\vb*{k}}{\pdv{V(\vb*{r} - \vb*{R}_{n \alpha})}{\vb*{R}_{n \alpha}}}{\vb*{k}'} \\
        &= 
        \int\dd[3]{\vb*{r}} \frac{\ee^{- \ii \vb*{k} \cdot \vb*{r}} u^*_{\vb*{k}}(\vb*{r})}{\sqrt{V}} 
        \frac{1}{V} \sum_{\vb*{Q}}  (- \ii \vb*{Q} ) \ee^{\ii \vb*{Q} \cdot (\vb*{r} - \vb*{R}_{n \alpha})}
        \frac{\ee^{\ii \vb*{k}' \cdot \vb*{r}} u_{\vb*{k}'}(\vb*{r})}{\sqrt{V}} \\
        &= \sum_{\vb*{G}} \sum_{\vb*{Q}} \delta_{\vb*{G}, \vb*{Q} + \vb*{k}' - \vb*{k}}
        \sum_m \int_{\text{u.c.}} \dd[3]{\vb*{r}} 
        \frac{\ee^{\ii \vb*{r} \cdot (\vb*{Q} + \vb*{k}' - \vb*{k})}}{V} 
        u_{\vb*{k}}^* (\vb*{r}) u_{\vb*{k}'}(\vb*{r}) 
    \end{aligned}
\]
TODO: finish this 

\begin{equation}
    H_{\text{electron-phonon}} = \frac{1}{\sqrt{N}} \sum_{\vb*{k}, \vb*{q}}
    g_{\vb*{k} \vb*{q}} (b^\dagger_{- \vb*{q}} + b_{\vb*{q}}) c^\dagger_{\vb*{k} + \vb*{q}} c_{\vb*{k}}
\end{equation}
The dimension of $g_{\vb*{k} \vb*{q}}$ is energy; 
note that all the creation and annihilation operators
are dimensionless: 
this can be inferred from the fact that the commutation relation contains no dimension.
In Feynman diagram calculation, 
the $1 / \sqrt{N}$ factor should enter 
the mathematical expression corresponding to a diagram 
together with the vertex function $g_{\vb*{k} \vb*{q}}$;
since two vertices are connected together by a phonon propagator, 
the dependence of the sample size $V$
introduced by $1 / \sqrt{N}$ factor is canceled by 
the sum over $\vb*{q}$, 
and 
\[
    \sum_{\vb*{q}} \frac{1}{\sqrt{N}} g_{\vb*{k} \vb*{q}} \cdot (\cdots) \cdot \frac{1}{\sqrt{N}} g_{\vb*{q} \vb*{k}'} = \int \frac{\dd[3]{\vb*{q}}}{(2\pi)^3} g_{\vb*{k} \vb*{q}} g_{\vb*{q} \vb*{k}'} \cdot (\cdots).
\]

Inter-band electron-phonon scattering 
(i.e. a scattering in which after the scattering, 
the band index of the electron changes)
is usually not so common, 
because the magnitude of electron band gap is $\sim \SI{1}{eV}$,
which is much larger than the phonon energy.
But for semiconductors with small band gaps or metals, 
this condition breaks, 
and the full Hamiltonian involving inter-band scattering has to be used. 

Similar to the case of electron-electron Coulomb scattering, 
several interaction channels with clear physical pictures can be recognized,
and effective models have been proposed to capture the major behaviors of them.
The \concept{Frohlich Hamiltonian} models 
the interaction between nearly free electrons 
and longitude acoustic phonons
\begin{equation}
    H = \sum_{\vb*{k}} \varepsilon_{\vb*{k}} c^\dagger_{\vb*{k}} c_{\vb*{k}}
    + \sum_{\vb*{q}} \omega_{\vb*{q}} b^\dagger_{\vb*{q}} b_{\vb*{q}}
    + \frac{1}{\sqrt{N}} \sum_{\vb*{k}, \vb*{q}} 
    g_{\vb*{k} \vb*{q}} (b_{- \vb*{q}}^\dagger + b_{\vb*{q}}) c^\dagger_{\vb*{k} + \vb*{q}} c_{\vb*{k}},
\end{equation}
where 
\begin{equation} 
    g_{\vb*{k} \vb*{q}} = \sqrt{\frac{\hbar}{2 m \omega_{\vb*{q}}}} \vb*{q} \cdot \vu*{\epsilon} V_{\vb*{q}}, 
    \quad \omega_{\vb*{q}} = v_{\text{s}} \abs*{\vb*{q}}, 
\end{equation}
and $V_{\vb*{q}}$ is the screened Coulomb potential.
This means as the system becomes metallic, 
the phonon line width should be smaller, 
because the potential is now screened 
and the scattering rate decreases.
Note that if the electrons are not near-free (i.e. localized), 
we no longer work in the $\vb*{k}$-representation, 
and that leads to TODO

Note that $\vb*{q}$ is the crystal momentum of phonons
plus a reciprocal lattice vector, 
and $\vu*{\epsilon}$ is the polarization direction;
this means if a phonon mode is transverse, 
then it has no interaction with the electrons 
in the linear expansion region
when $\vb*{G} = 0$.
When $\vb*{G} \neq 0$, 
we still have some interaction strength???
Without other mechanisms, like strong coupling between transverse and longitudinal phonons, 
interaction between transverse phonon modes and electrons 
is to be ignored.
This can be understood in the continuous limit,
where electrons interact with the charge density of the lattice,
and transverse phonon modes don't 
disrupt the density.

\end{document}