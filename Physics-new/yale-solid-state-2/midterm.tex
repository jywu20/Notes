\documentclass[hyperref, a4paper]{article}

\usepackage{geometry}
\usepackage{titling}
\usepackage{titlesec}
% No longer needed, since we will use enumitem package
% \usepackage{paralist}
\usepackage{enumitem}
\usepackage{footnote}
\usepackage{amsmath, amssymb, amsthm}
\usepackage{mathtools}
\usepackage{bbm}
\usepackage{graphicx}
\usepackage{subcaption}
\usepackage{physics}
\usepackage{tensor}
\usepackage{siunitx}
\usepackage[version=4]{mhchem}
\usepackage{tikz}
\usepackage{xcolor}
\usepackage{listings}
\usepackage{autobreak}
\usepackage[ruled, vlined, linesnumbered]{algorithm2e}
\usepackage[backend=bibtex]{biblatex}
\addbibresource{data.bib}
\addbibresource{experiments.bib}
\addbibresource{theory.bib}
\usepackage[colorlinks,unicode]{hyperref} % , linkcolor=black, anchorcolor=black, citecolor=black, urlcolor=black, filecolor=black
\usepackage[most]{tcolorbox}
\usepackage{prettyref}

% Page style
\geometry{left=3.18cm,right=3.18cm,top=2.54cm,bottom=2.54cm}
\titlespacing{\paragraph}{0pt}{1pt}{10pt}[20pt]
\setlength{\droptitle}{-5em}

% More compact lists 
\setlist[itemize]{
    itemindent=17pt, 
    leftmargin=1pt,
    listparindent=\parindent,
    parsep=0pt,
}

% Math operators
\DeclareMathOperator{\timeorder}{\mathcal{T}}
\DeclareMathOperator{\diag}{diag}
\DeclareMathOperator{\legpoly}{P}
\DeclareMathOperator{\primevalue}{P}
\DeclareMathOperator{\sgn}{sgn}
\DeclareMathOperator{\res}{Res}
\newcommand*{\ii}{\mathrm{i}}
\newcommand*{\ee}{\mathrm{e}}
\newcommand*{\const}{\mathrm{const}}
\newcommand*{\suchthat}{\quad \text{s.t.} \quad}
\newcommand*{\argmin}{\arg\min}
\newcommand*{\argmax}{\arg\max}
\newcommand*{\normalorder}[1]{: #1 :}
\newcommand*{\pair}[1]{\langle #1 \rangle}
\newcommand*{\fd}[1]{\mathcal{D} #1}
\DeclareMathOperator{\bigO}{\mathcal{O}}

% TikZ setting
\usetikzlibrary{arrows,shapes,positioning}
\usetikzlibrary{arrows.meta}
\usetikzlibrary{decorations.markings}
\tikzstyle arrowstyle=[scale=1]
\tikzstyle directed=[postaction={decorate,decoration={markings,
    mark=at position .5 with {\arrow[arrowstyle]{stealth}}}}]
\tikzstyle ray=[directed, thick]
\tikzstyle dot=[anchor=base,fill,circle,inner sep=1pt]

% Algorithm setting
% Julia-style code
\SetKwIF{If}{ElseIf}{Else}{if}{}{elseif}{else}{end}
\SetKwFor{For}{for}{}{end}
\SetKwFor{While}{while}{}{end}
\SetKwProg{Function}{function}{}{end}
\SetArgSty{textnormal}

\newcommand*{\concept}[1]{{\textbf{#1}}}

% Embedded codes
\lstset{basicstyle=\ttfamily,
  showstringspaces=false,
  commentstyle=\color{gray},
  keywordstyle=\color{blue}
}

% Reference formatting
\newrefformat{fig}{Fig.~\ref{#1}}
\newcommand*{\term}[1]{\textit{#1}}

% Color boxes
\tcbuselibrary{skins, breakable, theorems}

\newtcbtheorem[number within=section]{infobox}{Box}{
    enhanced,
    boxrule=0pt,
    colback=blue!5,
    colframe=blue!5,
    coltitle=blue!50,
    borderline west={4pt}{0pt}{blue!65},
    sharp corners,
    fonttitle=\bfseries, 
    breakable,
    before upper={\parindent15pt\noindent}}{box}
\newtcbtheorem[number within=section, use counter from=infobox]{theorybox}{Box}{
    enhanced,
    boxrule=0pt,
    colback=orange!5, 
    colframe=orange!5, 
    coltitle=orange!50,
    borderline west={4pt}{0pt}{orange!65},
    sharp corners,
    fonttitle=\bfseries, 
    breakable,
    before upper={\parindent15pt\noindent}}{box}
\newtcbtheorem[number within=section, use counter from=infobox]{learnbox}{Box}{
    enhanced,
    boxrule=0pt,
    colback=green!5,
    colframe=green!5,
    coltitle=green!50,
    borderline west={4pt}{0pt}{green!65},
    sharp corners,
    fonttitle=\bfseries, 
    breakable,
    before upper={\parindent15pt\noindent}}{box}

% Displaying texts in bookmarkers

\pdfstringdefDisableCommands{%
  \def\\{}%
  \def\ce#1{<#1>}%
}

\pdfstringdefDisableCommands{%
  \def\texttt#1{<#1>}%
  \def\mathbb#1{#1}%
}
\pdfstringdefDisableCommands{\def\eqref#1{(\ref{#1})}}

\makeatletter
\pdfstringdefDisableCommands{\let\HyPsd@CatcodeWarning\@gobble}
\makeatother

\newenvironment{shelldisplay}{\begin{lstlisting}}{\end{lstlisting}}

\newcommand*{\efermi}{E_{\text{F}}}
\newcommand*{\sA}{\text{A}}
\newcommand*{\sB}{\text{B}}
\newcommand*{\Tc}{T_{\text{c}}}
\newcommand*{\muB}{\mu_{\text{B}}}
\newcommand*{\kB}{k_{\text{B}}}

\setlength{\fboxrule}{0.1pt}
\newcommand*\up{\fbox{$\mathord\upharpoonleft\phantom{\downharpoonright}$}}%
\newcommand*\dn{\fbox{$\mathord\downharpoonleft\phantom{\upharpoonright}$}}%
\newcommand*\updn{\fbox{$\upharpoonleft\downharpoonright$}}%
\newcommand*\emp{\fbox{$\phantom{\downharpoonright}\phantom{\downharpoonright}$}}%
\newcommand{\electron}[2]{{%
        \setlength\tabcolsep{0pt}% remove extra horizontal space from tabular
%       \setlength\fboxrule{0.2pt}% uncomment for original line width
        \begin{tabular}{c}
            \fboxsep=0pt\fbox{\fboxsep=3pt#2}\\[2pt]
            #1
        \end{tabular}%
}}

\title{Midterm}
\author{Jinyuan Wu}

\begin{document}

\maketitle

\section{Problem 1}

\subsection{}

The electron configuration of \ce{Fe} is [Ar]3d$^6$4s$^2$,
and according to Hund's rule, 
the spins are 
\begin{center}
    \electron{3d}{\updn \up \up \up \up} \electron{4s}{\updn}
\end{center}
The electron configuration of \ce{Fe^{2+}} is [Ar]3d$^6$,
and according to Hund's rule, 
the spins are 
\begin{center}  
    \electron{3d}{\updn \up \up \up \up}
\end{center}
\ce{Fe^{3+}} is obtained by reducing one electron and the spins are 
\begin{center}
    \electron{3d}{\up \up \up \up \up}
\end{center}

For iron atoms,
the total spin quantum number is $S = 2$,
and the total orbital angular momentum quantum number is $L = 2$.
Therefore 
\begin{equation}
    g_J = \frac{3}{2} + \frac{S(S+1) - L(L+1)}{2J(J+1)} = \frac{3}{2}.
\end{equation}
Since the 3d shell is more than half filled, 
we have $J = L + S = 4$,
and the total magnetic moment should be 
\begin{equation}
    \mu = \muB g_J J = 6 \muB.
\end{equation}

\subsection{}

The experimentally observed atomic magnetic moment is $2.22\muB$,
which doesn't agree with the aforementioned prediction.
If somehow the orbital angular momentum is quenched, 
then $S = J = 2$,
and $g_J = 2$, 
and 


\section{Problem 2}

\subsection{}

A material is \concept{metamagnetic} if 
when the external magnetic field passes a finite value $H_{\text{c}}$,
the magnetic configuration changes all of a sudden.
This is a phenomenological term 
and may be driven by various physical mechanisms.
The material \ce{Sr3Ru2O7} is metamagnetic, 
because experiments have observed that 
around $\mu_0 H = \SI{7.9}{T}$, 
a sharp peak can be seen in magnetic susceptibility
\cite{grigera2004disorder},
and therefore there is indeed a sudden change in the magnetization.

The low-field phase is paramagnetic, 
and the high-field phase is itinerantly ferromagnetic:
the material shows 
``a rapid change from a paramagnetic state at low fields to
a more highly polarized state'' \cite{perry2001metamagnetism}.
(On the other hand, some other metamagnetic materials 
undergo an antiferromagnetism-to-ferromagnetism transition;
this is not the case for \ce{Sr3Ru2O7}.)

The boundary between the two phases was once thought to be a quantum critical point:
above the phase boundary between the low-field phase and the high-field phase,
the resistance doesn't have typical Fermi-liquid behaviors
\cite{perry2001metamagnetism};
the phase boundary between the two phases 
is a first-order phrase transition line with a terminating end point, 
and this critical point is pushed to $T = 0$
when the external magnetic field is pointed towards the $c$ direction,
creating a quantum critical point \cite{grigera2003angular}.
Further investigations however have found 
that there are actually \emph{two} peaks in susceptibility 
near $\mu_0 H =\SI{7.9}{T}$, 
and this ``quantum critical point'' is surrounded by two first-order phase transitions
\cite{kitagawa2005metamagnetic,grigera2004disorder}.
The exotic temperature-resistance curve likely comes from 
an SDW order on top of the ferromagnetic moment
formed between the two aforementioned first-order phase transition,
which also gives rise to anisotropic resistance
(or in other words, electronic nematic)
which isn't induced by the crystal structure
and can't be seen away from $\mu_0 H =\SI{7.9}{T}$
\cite{lester2015field,borzi2007formation}.

\subsection{}

\ce{Sr3Ru2O7} is an itinerant magnetic material 
and the metamagnetic transition 
is likely due to Fermi surface instability.
This is explained in \cite{kee2005itinerant} with a toy model.
The effect or the coupling between the electron magnetic moment 
and the external magnetic field 
is modifying the chemical potential for 

This also explains why nematic electron fluid 
is only observed near the so-called quantum critical point 
$\mu_0 H = \SI{7.9}{T}$:
because when the external magnetic field is stronger, 
the 


\section{Problem 3}

\subsection{}



\printbibliography

\end{document}