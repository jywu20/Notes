\documentclass[hyperref, a4paper]{article}

\usepackage{geometry}
\usepackage{titling}
\usepackage{titlesec}
% No longer needed, since we will use enumitem package
% \usepackage{paralist}
\usepackage{enumitem}
\usepackage{footnote}
\usepackage{enumerate}
\usepackage{amsmath, amssymb, amsthm}
\usepackage{mathtools}
\usepackage{bbm}
\usepackage{graphicx}
\usepackage{subcaption}
\usepackage{physics}
\usepackage{tensor}
\usepackage{siunitx}
\usepackage[version=4]{mhchem}
\usepackage{tikz}
\usepackage{xcolor}
\usepackage{listings}
\usepackage{autobreak}
\usepackage[ruled, vlined, linesnumbered]{algorithm2e}
\usepackage{nameref,zref-xr}
\zxrsetup{toltxlabel}
\usepackage[sorting=none]{biblatex}
\bibliography{squeezing}
\usepackage[colorlinks,unicode]{hyperref} % , linkcolor=black, anchorcolor=black, citecolor=black, urlcolor=black, filecolor=black
\usepackage[most]{tcolorbox}
\usepackage{prettyref}

% Page style
\geometry{left=3.18cm,right=3.18cm,top=2.54cm,bottom=2.54cm}
\titlespacing{\paragraph}{0pt}{1pt}{10pt}[20pt]
\setlength{\droptitle}{-5em}

% More compact lists 
\setlist[itemize]{
    itemindent=17pt, 
    leftmargin=1pt,
    listparindent=\parindent,
    parsep=0pt,
}

% Math operators
\DeclareMathOperator{\timeorder}{\mathcal{T}}
\DeclareMathOperator{\diag}{diag}
\DeclareMathOperator{\legpoly}{P}
\DeclareMathOperator{\primevalue}{P}
\DeclareMathOperator{\sgn}{sgn}
\DeclareMathOperator{\res}{Res}
\newcommand*{\ii}{\mathrm{i}}
\newcommand*{\ee}{\mathrm{e}}
\newcommand*{\const}{\mathrm{const}}
\newcommand*{\suchthat}{\quad \text{s.t.} \quad}
\newcommand*{\argmin}{\arg\min}
\newcommand*{\argmax}{\arg\max}
\newcommand*{\normalorder}[1]{: #1 :}
\newcommand*{\pair}[1]{\langle #1 \rangle}
\newcommand*{\fd}[1]{\mathcal{D} #1}
\DeclareMathOperator{\bigO}{\mathcal{O}}

% TikZ setting
\usetikzlibrary{arrows,shapes,positioning}
\usetikzlibrary{arrows.meta}
\usetikzlibrary{decorations.markings}
\tikzstyle arrowstyle=[scale=1]
\tikzstyle directed=[postaction={decorate,decoration={markings,
    mark=at position .5 with {\arrow[arrowstyle]{stealth}}}}]
\tikzstyle ray=[directed, thick]
\tikzstyle dot=[anchor=base,fill,circle,inner sep=1pt]

% Algorithm setting
% Julia-style code
\SetKwIF{If}{ElseIf}{Else}{if}{}{elseif}{else}{end}
\SetKwFor{For}{for}{}{end}
\SetKwFor{While}{while}{}{end}
\SetKwProg{Function}{function}{}{end}
\SetArgSty{textnormal}

\newcommand*{\concept}[1]{{\textbf{#1}}}

% Embedded codes
\lstset{basicstyle=\ttfamily,
  showstringspaces=false,
  commentstyle=\color{gray},
  keywordstyle=\color{blue}
}

% Reference formatting
\newrefformat{fig}{Fig.~\ref{#1}}
\newcommand*{\term}[1]{\textit{#1}}

% Color boxes
\tcbuselibrary{skins, breakable, theorems}
\newtcbtheorem[number within=section]{warning}{Warning}%
  {colback=orange!5,colframe=orange!65,fonttitle=\bfseries, breakable}{warn}
\newtcbtheorem[number within=section]{note}{Note}%
  {colback=green!5,colframe=green!65,fonttitle=\bfseries, breakable}{note}
\newtcbtheorem[number within=section]{info}{Info}%
  {colback=blue!5,colframe=blue!65,fonttitle=\bfseries, breakable}{info}

\newenvironment{shelldisplay}{\begin{lstlisting}}{\end{lstlisting}}

\newcommand*{\efermi}{E_{\text{F}}}

\title{Magnetism}
\author{Jinyuan Wu}

\begin{document}

\maketitle

The word \term{magnetism} covers the response to a magnetic field 
(paramagnetism, diamagnetism)
and the magnetic ground state 
(ferromagnetism, antiferromagnetism).
In everyday language,
when we say a material is ``magnetic'',
we mean it has a magnetic ground state -- usually ferromagnetism -- 
but so-called non-magnetic materials 
can still have magnetic response 
to an external magnetic field.

The ground state is influenced by several mechanisms:
exchange interaction, 
itinerant magnetism,
topological magnetism.
Above the ground state, we have magnetic excitations like spin waves.

\section{Magnetic response of individual electrons}

For local electrons, 
we have Curie's law and Langevin diamagnetism (TODO: ref);
for itinerant electrons,
we have Pauli paramagnetism,
Landau diamagnetism, and more (TODO: ref). 

\subsection{The magnetic moment}

TODO: lecture on the Friday; Hund's rule 

\subsection{Electromagnetic effective theory}

TODO: $\chi, M$

\subsection{Paramagnetism of itinerant electrons}

\begin{equation}
    M = \mu_{B} (n_\uparrow - n_\downarrow) = \mu_B^2 B \rho(\efermi),
\end{equation}

\subsection{Landau diamagnetism}

We can see the magnitude of itinerant paramagnetism usually dominates that of Landau diamagnetism,
and therefore most metals are paramagnetic. 
Transitional metals on the right of the periodic table, like silver, cooper, and gold,
however demonstrate diamagnetism.
The explanation is nontrivial TODO

Generally all the magnetic responses are rather weak. 
We need very strong external magnetic field 
to make these phenomena mechanically significant.
With a $\sim \SI{16}{T}$ magnetic field,
the diamagnetism of water is obvious enough 
for us to trap a frog and let it levitate in the air.
For superconductors however, 
since they repulse magnetic field completely
(because electromagnetic modes inside a superconductor are gapped),
magnetic levitation can be easily observed.

\subsection{Van Vleck paramagnetism}

\section{Some easy-to-explain localized magnetic ground states}

In this section we discuss systems in which 
electrons that contribute most to the magnetic ground state 
are localized around atoms,
and therefore we can talk about magnetism of atoms.
In this section we put the problem of domain walls aside: 
we assume a magnetic order is formed in the thermodynamic limit. TODO: are domain walls classical enough?

A \concept{ferromagnetic} material is a material 
where the magnetic moments of all atoms are towards the same direction.
It breaks the spin rotational symmetry.
A \concept{antiferromagnetic} material is a material
where the material is broken into several sublattices,
each of which are in a ferromagnetic state,
but the magnetic momenta among the sublattices cancel each other.
It brakes both the spin rotational symmetry 
and the translational symmetry.
A \concept{ferrimagnetic} material is antiferromagnetic one with a non-zero overall magnetic moment.
The states all have \emph{long-range orders}:
$\expval*{\vb*{S}_{\vb*{i}} \cdot \vb*{S}_{\vb*{j}}}$ doesn't come to zero 
even when the distance between the two spins goes to $\infty$.

\subsection{The magnetic dipole-dipole interaction}

It can be proved that the dipole-dipole interaction energy is 
\begin{equation}
    E_{\text{dipole}} = 
    \frac{\mu_0}{4\pi}
    \left(
        \frac{\vb*{\mu}_1 \cdot \vb*{\mu}_2}{r^3}
        - 3 \frac{(\vb*{\mu}_1 \cdot \vb*{r}) \cdot (\vb*{\mu}_2 \cdot \vb*{r})}{r^5}
    \right).
    \label{eq:dipole-dipole}
\end{equation}
An order of magnitude estimation tells us the energy is $\sim \SI{1e-4}{eV}$.
We know the band energy is $\sim \SI{1}{eV}$,
and therefore \eqref{eq:dipole-dipole} is too weak.

\subsection{Spin-spin interaction caused by exchange interaction}

There however exists another interaction channel 
that gives us a $\vb*{S}_1 \cdot \vb*{S}_2$ Hamiltonian.
We know the eigenstates of the 2-electron system 
\begin{equation}
    H = \sum_{i = 1,2 } \left(
        \frac{\vb*{p}_i^2}{2m} - \frac{Z e^2}{r_i}
    \right)
    + \frac{e^2}{\abs{\vb*{r}_1 - \vb*{r}_2}}   
    \label{eq:two-electron-spatial}
\end{equation}
are the \concept{bonding wave function} and the \concept{anti-bonding wave function}:
the former is symmetric and the latter is antisymmetric.
Now \eqref{eq:two-electron-spatial} is only about the spatial part.
We know Pauli exclusion principle requires the many-body wave function 
to be antisymmetric,
and therefore the bonding wave function of the spatial part 
comes together with the antisymmetric spin wave function -- the \concept{singlet} subspace,
and the anti-bonding wave function of the spatial part 
comes with the symmetric spin wave function -- the \concept{triplet} subspace.
Thus, the energy difference between the bonding wave function and the anti-bonding wave function 
can be equivalently attributed to the spin orientation.
If $\psi^\text{B}$ is the ground state, 
then the spin configuration in the ground state is one-up-one-down.

In this mechanism, 
what's important is the Coulomb interaction between electrons localized around different atoms.
Thus, even if the magnetic momentum of a \emph{single} atom is strong,
if the exchange interaction isn't strong enough,
the ferromagnetism is still not stable enough.
This is (at least qualitatively) demonstrated by experimental results:
the Curie temperature seems to be largely 
independent of the magnetic momentum per atom.

Now we write an effective Hamiltonian between $\vb*{S}_1$ and $\vb*{S}_2$.
From $\vb*{S} = \vb*{S}_1 + \vb*{S}_2$, we know 
\begin{equation}
    2 \vb*{S}_1 \cdot \vb*{S}_2 = 
    s(s+1) - s_1 (s_1 + 1) - s_2 (s_2 + 1),
\end{equation}
and the RHS is $-3/2$ for a singlet and $1/2$ for a triplet,
and therefore we find 
\[
    \frac{1}{2} + 2 \vb*{S}_1 \cdot \vb*{S}_2 = \pm 1.
\]
Now the total energy is $C_{12} \pm J_{12}$,
and therefore we get 
\begin{equation}
    H = C_{12} \pm J_{12} 
    = C_{12} - \frac{1}{2} J_{12} - 2 J_{12} \vb*{S}_1 \cdot \vb*{S}_2.
\end{equation}
So we arrive at the \concept{Heisenberg Hamiltonian}.
Some people will replace $2 J_{12}$ by $J_{12}$:
this is a frequent notational difference.

\subsection{The Heisenberg model}

The Heisenberg model is then 
\begin{equation}
    H = - 2 \sum_{\pair{\vb*{i}, \vb*{j}}} J_{\vb*{i} \vb*{j}} \vb*{S}_{\vb*{i}} \vb*{S}_{\vb*{j}}.
    \label{eq:heisenberg}
\end{equation}
At this point we only place magnetic moments on atoms; 
we ignore all itinerant electrons which have the possibility to be magnetic.

\eqref{eq:heisenberg} is generally hard to solve.
One way to show that it models ferromagnetism is by mean-field theory.
We do \concept{Weiss} molecular field model:
we assume that \eqref{eq:heisenberg} is equivalent to a model with no spin-spin interaction,
where what is felt by each spin is a mean-field 
\begin{equation}
    \vb*{H}_{\text{total}} = \vb*{H} + \vb*{H}_M, \quad 
    \vb*{H}_M = \lambda_M \vb*{M},
\end{equation}
where $\vb*{H}$ is the external magnetic field 
(which should be introduced by adding a $\vb*{S} \cdot \vb*{H}$ term in \eqref{eq:heisenberg}),
and we then can obtain the relation between $\vb*{M}$ and $\vb*{H}_{\text{total}}$:  TODO 
\begin{equation}
    \vb*{M} = \left(
        \frac{N g^2 \mu_B^2 (j+1) j \vu*{\mu}}{3 k_{\text{B}} T} H_{\text{total}}
    \right),
\end{equation}
and we find 
\begin{equation}
    \vb*{M} = \chi \vb*{H}, \quad 
    \chi = \frac{C}{T - C \lambda_M}.
\end{equation}
Obviously, $\chi$ diverges when $T \to T_{\text{c}}^+$,
where even when we add a very small external $\vb*{H}$,
we get a very huge $\vb*{M}$ -- 
an indication that the system becomes ferromagnetic after $T$ passes $T_{\text{c}}$.

\end{document}