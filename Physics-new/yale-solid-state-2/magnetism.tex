\documentclass[hyperref, a4paper]{article}

\usepackage{geometry}
\usepackage{titling}
\usepackage{titlesec}
% No longer needed, since we will use enumitem package
% \usepackage{paralist}
\usepackage{enumitem}
\usepackage{footnote}
\usepackage{enumerate}
\usepackage{amsmath, amssymb, amsthm}
\usepackage{mathtools}
\usepackage{bbm}
\usepackage{graphicx}
\usepackage{subcaption}
\usepackage{physics}
\usepackage{tensor}
\usepackage{siunitx}
\usepackage[version=4]{mhchem}
\usepackage{tikz}
\usepackage{xcolor}
\usepackage{listings}
\usepackage{autobreak}
\usepackage[ruled, vlined, linesnumbered]{algorithm2e}
\usepackage{nameref,zref-xr}
\zxrsetup{toltxlabel}
\usepackage[sorting=none]{biblatex}
\bibliography{squeezing}
\usepackage[colorlinks,unicode]{hyperref} % , linkcolor=black, anchorcolor=black, citecolor=black, urlcolor=black, filecolor=black
\usepackage[most]{tcolorbox}
\usepackage{prettyref}

% Page style
\geometry{left=3.18cm,right=3.18cm,top=2.54cm,bottom=2.54cm}
\titlespacing{\paragraph}{0pt}{1pt}{10pt}[20pt]
\setlength{\droptitle}{-5em}

% More compact lists 
\setlist[itemize]{
    itemindent=17pt, 
    leftmargin=1pt,
    listparindent=\parindent,
    parsep=0pt,
}

% Math operators
\DeclareMathOperator{\timeorder}{\mathcal{T}}
\DeclareMathOperator{\diag}{diag}
\DeclareMathOperator{\legpoly}{P}
\DeclareMathOperator{\primevalue}{P}
\DeclareMathOperator{\sgn}{sgn}
\DeclareMathOperator{\res}{Res}
\newcommand*{\ii}{\mathrm{i}}
\newcommand*{\ee}{\mathrm{e}}
\newcommand*{\const}{\mathrm{const}}
\newcommand*{\suchthat}{\quad \text{s.t.} \quad}
\newcommand*{\argmin}{\arg\min}
\newcommand*{\argmax}{\arg\max}
\newcommand*{\normalorder}[1]{: #1 :}
\newcommand*{\pair}[1]{\langle #1 \rangle}
\newcommand*{\fd}[1]{\mathcal{D} #1}
\DeclareMathOperator{\bigO}{\mathcal{O}}

% TikZ setting
\usetikzlibrary{arrows,shapes,positioning}
\usetikzlibrary{arrows.meta}
\usetikzlibrary{decorations.markings}
\tikzstyle arrowstyle=[scale=1]
\tikzstyle directed=[postaction={decorate,decoration={markings,
    mark=at position .5 with {\arrow[arrowstyle]{stealth}}}}]
\tikzstyle ray=[directed, thick]
\tikzstyle dot=[anchor=base,fill,circle,inner sep=1pt]

% Algorithm setting
% Julia-style code
\SetKwIF{If}{ElseIf}{Else}{if}{}{elseif}{else}{end}
\SetKwFor{For}{for}{}{end}
\SetKwFor{While}{while}{}{end}
\SetKwProg{Function}{function}{}{end}
\SetArgSty{textnormal}

\newcommand*{\concept}[1]{{\textbf{#1}}}

% Embedded codes
\lstset{basicstyle=\ttfamily,
  showstringspaces=false,
  commentstyle=\color{gray},
  keywordstyle=\color{blue}
}

% Reference formatting
\newrefformat{fig}{Fig.~\ref{#1}}
\newcommand*{\term}[1]{\textit{#1}}

% Color boxes
\tcbuselibrary{skins, breakable, theorems}
\newtcbtheorem[number within=section]{warning}{Warning}%
  {colback=orange!5,colframe=orange!65,fonttitle=\bfseries, breakable}{warn}
\newtcbtheorem[number within=section]{note}{Note}%
  {colback=green!5,colframe=green!65,fonttitle=\bfseries, breakable}{note}
\newtcbtheorem[number within=section]{info}{Info}%
  {colback=blue!5,colframe=blue!65,fonttitle=\bfseries, breakable}{info}

\newenvironment{shelldisplay}{\begin{lstlisting}}{\end{lstlisting}}

\newcommand*{\efermi}{E_{\text{F}}}
\newcommand*{\sA}{\text{A}}
\newcommand*{\sB}{\text{B}}
\newcommand*{\Tc}{T_{\text{c}}}

\title{Magnetism}
\author{Jinyuan Wu}

\begin{document}

\maketitle

The word \term{magnetism} covers the response to a magnetic field 
(paramagnetism, diamagnetism)
and the magnetic ground state 
(ferromagnetism, antiferromagnetism).
In everyday language,
when we say a material is ``magnetic'',
we mean it has a magnetic ground state -- usually ferromagnetism -- 
but so-called non-magnetic materials 
can still have magnetic response 
to an external magnetic field.
The terms \term{paramagnetism} and \term{diamagnetism} 
are usually used for materials with non-magnetic ground states.

The ground state is influenced by several mechanisms:
exchange interaction, 
itinerant magnetism,
topological magnetism.
Above the ground state, we have magnetic excitations like spin waves.

\section{Magnetic response of individual electrons}

For local electrons, 
we have Curie's law and Langevin diamagnetism (TODO: ref);
for itinerant electrons,
we have Pauli paramagnetism,
Landau diamagnetism, and more (TODO: ref). 

\subsection{The magnetic moment}

TODO: lecture on the Friday; Hund's rule 

The total angular

\subsection{Electromagnetic effective theory}

TODO: $\chi, M$

\subsection{Paramagnetism of itinerant electrons}

\begin{equation}
    M = \mu_{B} (n_\uparrow - n_\downarrow) = \mu_B^2 B \rho(\efermi),
\end{equation}

\subsection{Landau diamagnetism}

We can see the magnitude of itinerant paramagnetism usually dominates that of Landau diamagnetism,
and therefore most metals are paramagnetic. 
Transitional metals on the right of the periodic table, like silver, cooper, and gold,
however demonstrate diamagnetism.
The explanation is nontrivial TODO

Generally all the magnetic responses are rather weak. 
We need very strong external magnetic field 
to make these phenomena mechanically significant.
With a $\sim \SI{16}{T}$ magnetic field,
the diamagnetism of water is obvious enough 
for us to trap a frog and let it levitate in the air.
For superconductors however, 
since they repulse magnetic field completely
(because electromagnetic modes inside a superconductor are gapped),
magnetic levitation can be easily observed.

\subsection{Van Vleck paramagnetism}

\section{Some easy-to-explain localized magnetic ground states}

In this section we discuss systems in which 
electrons that contribute most to the magnetic ground state 
are localized around atoms,
and therefore we can talk about magnetism of atoms.
In this section we put the problem of domain walls aside: 
we assume a magnetic order is formed in the thermodynamic limit. TODO: are domain walls classical enough?

A \concept{ferromagnetic} material is a material 
where the magnetic moments of all atoms are towards the same direction.
It breaks the spin rotational symmetry.
A \concept{antiferromagnetic} material is a material
where the material is broken into several sublattices,
each of which are in a ferromagnetic state,
but the magnetic momenta among the sublattices cancel each other.
It brakes both the spin rotational symmetry 
and the translational symmetry.
A \concept{ferrimagnetic} material is antiferromagnetic one with a non-zero overall magnetic moment.
The states all have \emph{long-range orders}:
$\expval*{\vb*{S}_{\vb*{i}} \cdot \vb*{S}_{\vb*{j}}}$ doesn't come to zero 
even when the distance between the two spins goes to $\infty$.

\subsection{Direct interaction between magnetic moments}

\subsubsection{The magnetic dipole-dipole interaction}\label{sec:localized.heisenberg.dipole}

It can be proved that the dipole-dipole interaction energy is 
\begin{equation}
    E_{\text{dipole}} = 
    \frac{\mu_0}{4\pi}
    \left(
        \frac{\vb*{\mu}_1 \cdot \vb*{\mu}_2}{r^3}
        - 3 \frac{(\vb*{\mu}_1 \cdot \vb*{r}) \cdot (\vb*{\mu}_2 \cdot \vb*{r})}{r^5}
    \right).
    \label{eq:dipole-dipole}
\end{equation}
An order of magnitude estimation tells us the energy is $\sim \SI{1e-4}{eV}$.
We know the band energy is $\sim \SI{1}{eV}$,
and therefore \eqref{eq:dipole-dipole} is too weak
for a microscopic description of magnetism -- 
and it can only leads to antiferromagnetism.

\subsubsection{Spin-spin interaction caused by exchange interaction}

There however exists another interaction channel 
that gives us a $\vb*{S}_1 \cdot \vb*{S}_2$ Hamiltonian.
We know the eigenstates of the 2-electron system 
\begin{equation}
    H = \sum_{i = 1,2 } \left(
        \frac{\vb*{p}_i^2}{2m} - \frac{Z e^2}{r_i}
    \right)
    + \frac{e^2}{\abs{\vb*{r}_1 - \vb*{r}_2}}   
    \label{eq:two-electron-spatial}
\end{equation}
are the \concept{bonding wave function} and the \concept{anti-bonding wave function}:
the former is symmetric and the latter is antisymmetric.
Now \eqref{eq:two-electron-spatial} is only about the spatial part.
We know Pauli exclusion principle requires the many-body wave function 
to be antisymmetric,
and therefore the bonding wave function of the spatial part 
comes together with the antisymmetric spin wave function -- the \concept{singlet} subspace,
and the anti-bonding wave function of the spatial part 
comes with the symmetric spin wave function -- the \concept{triplet} subspace.
Thus, the energy difference between the bonding wave function and the anti-bonding wave function 
can be equivalently attributed to the spin orientation.
If $\psi^\text{B}$ is the ground state, 
then the spin configuration in the ground state is one-up-one-down.

In this mechanism, 
what's important is the Coulomb interaction between electrons localized around different atoms.
Thus, even if the magnetic momentum of a \emph{single} atom is strong,
if the exchange interaction isn't strong enough,
the ferromagnetism is still not stable enough.
This is (at least qualitatively) demonstrated by experimental results:
the Curie temperature seems to be largely 
independent of the magnetic momentum per atom.

Now we write an effective Hamiltonian between $\vb*{S}_1$ and $\vb*{S}_2$.
From $\vb*{S} = \vb*{S}_1 + \vb*{S}_2$, we know 
\begin{equation}
    2 \vb*{S}_1 \cdot \vb*{S}_2 = 
    s(s+1) - s_1 (s_1 + 1) - s_2 (s_2 + 1),
\end{equation}
and the RHS is $-3/2$ for a singlet and $1/2$ for a triplet,
and therefore we find 
\[
    \frac{1}{2} + 2 \vb*{S}_1 \cdot \vb*{S}_2 = \pm 1.
\]
Now the total energy is $C_{12} \pm J_{12}$,
and therefore we get 
\begin{equation}
    H = C_{12} \pm J_{12} 
    = C_{12} - \frac{1}{2} J_{12} - 2 J_{12} \vb*{S}_1 \cdot \vb*{S}_2.
\end{equation}
So we arrive at the \concept{Heisenberg Hamiltonian}.
Some people will replace $2 J_{12}$ by $J_{12}$:
this is a frequent notational difference.

\subsubsection{The Heisenberg model}

The Heisenberg model is then 
\begin{equation}
    H = - 2 \sum_{\pair{\vb*{i}, \vb*{j}}} J_{\vb*{i} \vb*{j}} \vb*{S}_{\vb*{i}} \vb*{S}_{\vb*{j}}.
    \label{eq:heisenberg}
\end{equation}
At this point we only place magnetic moments on atoms; 
we ignore all itinerant electrons which have the possibility to be magnetic.

\subsubsection{The Weiss molecular field model}

\eqref{eq:heisenberg} is generally hard to solve.
One way to show that it models ferromagnetism is by mean-field theory.
We do \concept{Weiss molecular field model}:
we assume that \eqref{eq:heisenberg} is equivalent to a model with no spin-spin interaction,
where what is felt by each spin is a mean-field 
\begin{equation}
    \vb*{H}_{\text{total}} = \vb*{H} + \vb*{H}_M, \quad 
    \vb*{H}_M = \lambda_M \vb*{M},
\end{equation}
where $\vb*{H}$ is the external magnetic field 
(which should be introduced by adding a $\vb*{S} \cdot \vb*{H}$ term in \eqref{eq:heisenberg}),
and we have 
\begin{equation}
    \lambda_M = \frac{2 z J_{ij}}{N g^2 \mu_B^2 \hat{\mu}},
\end{equation}
where $z$ is the number of nearest neighbors of each site.
We then can obtain the relation between $\vb*{M}$ and $\vb*{H}_{\text{total}}$
from Curie's law (Dresselhaus p. 40-43?)
\begin{equation}
    \vb*{M} = \left(
        \frac{N g^2 \mu_B^2 (j+1) j \hat{\mu}}{3 k_{\text{B}} T} 
    \right) \vb*{H}_{\text{total}},
\end{equation}
and we find 
\begin{equation}
    \vb*{M} = \chi \vb*{H}, \quad 
    \chi = \frac{C}{T - C \lambda_M}.
\end{equation}
Obviously, $\chi$ diverges when $T \to T_{\text{c}}^+$,
where even when we add a very small external $\vb*{H}$,
we get a very huge $\vb*{M}$ -- 
an indication that the system becomes ferromagnetic after $T$ passes $T_{\text{c}}$.

It should be noted that since the response of $\vb*{M}$ to $\vb*{H}_{\text{total}}$
is assumed to follow the high-temperature pattern,
the mean-field solution shouldn't be expected to work \emph{quantitatively}
when $T$ really approaches $\Tc$.
Indeed, $\Tc$ can be seen as a temperature scale 
that tells us how high is ``high temperature'',
where the Curie law works quantitatively.

Another comment is the divergent behavior of $\chi$ doesn't rely on the 
exact form of the Hamiltonian:
the only things the above calculation takes from \eqref{eq:heisenberg} 
are $z$ and $J_{ij}$, which decide the value of $\lambda_M$.
It even isn't constrained to the field of magnetism.
The same Curie-Weiss law-like behavior can be found 
in nematic susceptibility (stress-deformation),
elastic moduli (stress-strain),
electric susceptibility (polarization-electric field),
charge susceptibility (charge number-chemical potential),
and more.

$\vb*{H}_{M}$ can be very large,
much larger than the strongest magnetic field created 
by deconstructive methods.
A magnet doesn't explode, 
because $\vb*{H}_M$ is highly localized.

\subsubsection{The origin of the anisotropy}

Since the crystal lattice structure is not isotropic, 
we shouldn't expect $J_{\vb*{i} \vb*{j}}$ to be isotropic.
The question is how the lattice influences magnetism.
The leading contribution is SOC. TODO: any model about this?

The anisotropic limit of the generalized Heisenberg model is the \concept{Ising model}
\begin{equation}
    H = - \sum_{\pair{\vb*{i}, \vb*{j}}} J_{\vb*{i} \vb*{j}} \sigma_{\vb*{i}} \sigma_{\vb*{j}}
    - \sum_{\vb*{j}} h_{\vb*{j}} \sigma_{\vb*{j}}.
\end{equation}
That's to say, only $z$-axis spins are actively engaging in magnetism.
Treatment of the Ising model is much easier,
because now at least we don't need to consider vector spin degrees of freedoms,
but still it's hard. 
Mean-field solutions are shown in textbooks, 
but they predict wrong $\Tc$'s 
and in the 1D case, falsely report that there is a phase transition 
while the exact solution of the model tells us 
it doesn't. 
The problem here is again universally seen in all mean-field methods:
fluctuation is not completely taken into account. 
We do have a phase transition in 2D,
because now a spin has more neighbors, 
and the energy cost to have a $\downarrow$-spin 
in a sea of $\uparrow$-spins is high, 
so neglecting thermal fluctuation is more acceptable. 

\subsection{Ferromagnetic systems}

\subsubsection{Magnetic domains}

In a ferromagnetic system 
we have magnetic domains, 
in each of which spins go in the same direction 
but the magnetic moments of two neighbor magnet domains
are opposite to each other.
The main reason of this is the competition between 
the long-range magnetic dipole interaction (\prettyref{sec:localized.heisenberg.dipole})
and the short-range exchange interaction:
the short-range interaction sorts the spins in a small region into one direction,
forming a large total magnetic moment,
while the long-range magnetic dipole interaction 
instructs these average magnetic moments to have staggering orientations.
The directions of magnetic domains around the boundary is usually parallel to the boundary,
TODO: easy axis, 
\begin{equation}
    H_{\text{cf}} = - \sum_{\vb*{i}} (\vb*{S}_{\vb*{i}} \cdot \vb*{n})^2.
\end{equation}
minimize $- \mu_0 \vb*{M} \cdot \vb*{H}$

If we break a magnet,
usually we will cut through many existing magnetic domains.
The broken domains immediately 
realign to minimize $- \mu_0 \vb*{M} \cdot \vb*{H}$
and now have orientations parallel to the boundary.
So now the attraction force across the cut 
is too weak to glue the two parts together into one magnet.

The energy of a domain wall is 
\begin{equation}
    \sigma_W = \underbrace{\frac{J S^2 \pi^2}{N a^2}}_{\text{exchange}} 
    + \underbrace{KNa}_\text{magneto-crystalline anisotropy ??? TODO}.
\end{equation}
Here the magneto-crystalline anisotropy term comes again from SOC: 
since there is anisotropy in the crystal structure, 
due to SOC, 
magnetic moments have preferred directions.

\subsubsection{M-H hysteresis}

Even when we already reach the transitional temperature,
it's still possible for a ferromagnetic system to have zero $\vb*{M}$,
because the directions of magnetic domains are staggering.

One thing to keep in mind is $\vb*{M}$ is not the only relevant degree of freedom 
for a macroscopic theory of a magnet:
the structure of internal magnetic domains is also important.
This means if we choose to use $\vb*{M}$ and $\vb*{H}$ only 
to describe a magnet,
the relation between the two depends on the history.  
TODO: more thought on this: is magnetic domains really necessary for the curve??
Since it seems Landau's theory is already enough

TODO: figure 1.30.1

According to the shape of the magnetic hysteresis,
we can classify magnets into \concept{soft} and \concept{hard} magnets. 
A hard magnet has very steep $\vb*{M}$-$\vb*{H}$ relation:
when the external field is not strong, 
no magnetization happens, 
and when $\vb*{H}$ exceeds a certain point, 
all of a sudden, 
a very strong magnetization occurs. 

TODO: what an exchange bias?

When we zoom in the $\vb*{M}$-$\vb*{H}$ curve, 
we can find small steps with random height.
This is called the \concept{Barkhausen effect}.
This is believed to be an effect involving defects.
There are magnetic defects in the material that strongly interact with the surrounding magnetic moments 
and create a mini ordered area around it. 
As is shown in \prettyref{fig:magnetic-defect-domain-wall-interplay}, 
when a magnetic domain wall moves across such a defect, 
a long tube connecting the ordered area around the defect 
and the magnetic domain that previously contained the defect will be formed (b);
this tube of course costs energy ($\propto L$), 
and when the energy cost is too high, 
eventually the tube will be broken, 
and the direction of the magnetic moments within the tube area 
flips all of a sudden. 
The contribution of these magnetic moments to the total magnetization 
decides the height of the small steps observed.
The Barkhausen effect is a main source of noise in the magnetic recording tape.

\begin{figure}
    \centering
    \tikzset{every picture/.style={line width=0.75pt}} %set default line width to 0.75pt        

\begin{tikzpicture}[x=0.75pt,y=0.75pt,yscale=-1,xscale=1]
%uncomment if require: \path (0,300); %set diagram left start at 0, and has height of 300

%Shape: Rectangle [id:dp23227140038029304] 
\draw  [draw opacity=0][fill={rgb, 255:red, 74; green, 144; blue, 226 }  ,fill opacity=0.1 ] (61.17,86.85) -- (104,86.85) -- (104,156.85) -- (61.17,156.85) -- cycle ;
%Straight Lines [id:da24044805644262146] 
\draw    (104,87) -- (104,156.85) ;
%Straight Lines [id:da6746174603382504] 
\draw    (83,122) ;
\draw [shift={(83,122)}, rotate = 45] [color={rgb, 255:red, 0; green, 0; blue, 0 }  ][line width=0.75]    (-5.59,0) -- (5.59,0)(0,5.59) -- (0,-5.59)   ;
%Straight Lines [id:da8992839725832915] 
\draw    (338.75,124) ;
\draw [shift={(338.75,124)}, rotate = 45] [color={rgb, 255:red, 0; green, 0; blue, 0 }  ][line width=0.75]    (-5.59,0) -- (5.59,0)(0,5.59) -- (0,-5.59)   ;
%Curve Lines [id:da20801752082073] 
\draw    (292,87) .. controls (291.79,108.44) and (294.37,120.1) .. (313.83,120.52) .. controls (333.29,120.94) and (329.79,116.44) .. (335.79,114.44) .. controls (341.79,112.44) and (347.79,117.69) .. (347.79,123.69) .. controls (347.79,129.69) and (340.79,134.94) .. (334.04,131.44) .. controls (327.29,127.94) and (332.79,124.69) .. (313.79,124.44) .. controls (294.79,124.19) and (292.29,137.44) .. (292,156.85) ;
%Shape: Rectangle [id:dp17248129737182882] 
\draw  [draw opacity=0][fill={rgb, 255:red, 208; green, 2; blue, 27 }  ,fill opacity=0.1 ] (104,86.85) -- (146.83,86.85) -- (146.83,156.85) -- (104,156.85) -- cycle ;
%Curve Lines [id:da39495239856656794] 
\draw [draw opacity=0][fill={rgb, 255:red, 74; green, 144; blue, 226 }  ,fill opacity=0.1 ]   (292,87) .. controls (291.79,108.44) and (294.37,120.1) .. (313.83,120.52) .. controls (333.29,120.94) and (329.79,116.44) .. (335.79,114.44) .. controls (341.79,112.44) and (347.79,117.69) .. (347.79,123.69) .. controls (347.79,129.69) and (340.79,134.94) .. (334.04,131.44) .. controls (327.29,127.94) and (332.79,124.69) .. (313.79,124.44) .. controls (294.79,124.19) and (292.29,137.44) .. (292,156.85) .. controls (280.54,157.02) and (278.54,156.77) .. (256,156.85) .. controls (256.29,94.1) and (256.33,136.46) .. (256.04,87.35) .. controls (276.04,87.6) and (258.29,87.1) .. (292,87) -- cycle ;
%Curve Lines [id:da6144255835160188] 
\draw [draw opacity=0][fill={rgb, 255:red, 208; green, 2; blue, 27 }  ,fill opacity=0.1 ]   (292,87) .. controls (291.79,108.44) and (294.37,120.1) .. (313.83,120.52) .. controls (333.29,120.94) and (329.79,116.44) .. (335.79,114.44) .. controls (341.79,112.44) and (347.79,117.69) .. (347.79,123.69) .. controls (347.79,129.69) and (340.79,134.94) .. (334.04,131.44) .. controls (327.29,127.94) and (332.79,124.69) .. (313.79,124.44) .. controls (294.79,124.19) and (292.29,137.44) .. (292,156.85) .. controls (280.54,157.02) and (383.54,156.77) .. (361,156.85) .. controls (361.29,94.1) and (361.33,136.46) .. (361.04,87.35) .. controls (381.04,87.6) and (258.29,87.1) .. (292,87) -- cycle ;
%Shape: Rectangle [id:dp7140929144714487] 
\draw  [draw opacity=0][fill={rgb, 255:red, 74; green, 144; blue, 226 }  ,fill opacity=0.1 ] (447.61,87) -- (467.28,87) -- (467.28,157) -- (447.61,157) -- cycle ;
%Straight Lines [id:da4713286386748359] 
\draw    (467.28,87) -- (467.28,156.85) ;
%Shape: Rectangle [id:dp9307193592961596] 
\draw  [draw opacity=0][fill={rgb, 255:red, 208; green, 2; blue, 27 }  ,fill opacity=0.1 ] (467.28,86.85) -- (533.83,86.85) -- (533.83,156.85) -- (467.28,156.85) -- cycle ;
%Shape: Circle [id:dp020567399671055364] 
\draw  [fill={rgb, 255:red, 255; green, 255; blue, 255 }  ,fill opacity=1 ] (504.74,121.85) .. controls (504.74,117.62) and (508.18,114.18) .. (512.42,114.18) .. controls (516.65,114.18) and (520.09,117.62) .. (520.09,121.85) .. controls (520.09,126.09) and (516.65,129.53) .. (512.42,129.53) .. controls (508.18,129.53) and (504.74,126.09) .. (504.74,121.85) -- cycle ;
%Shape: Circle [id:dp18933498134489657] 
\draw  [draw opacity=0][fill={rgb, 255:red, 74; green, 144; blue, 226 }  ,fill opacity=0.1 ] (504.74,121.85) .. controls (504.74,117.62) and (508.18,114.18) .. (512.42,114.18) .. controls (516.65,114.18) and (520.09,117.62) .. (520.09,121.85) .. controls (520.09,126.09) and (516.65,129.53) .. (512.42,129.53) .. controls (508.18,129.53) and (504.74,126.09) .. (504.74,121.85) -- cycle ;
%Straight Lines [id:da2894948145731697] 
\draw    (512.42,121.85) ;
\draw [shift={(512.42,121.85)}, rotate = 45] [color={rgb, 255:red, 0; green, 0; blue, 0 }  ][line width=0.75]    (-5.59,0) -- (5.59,0)(0,5.59) -- (0,-5.59)   ;
%Straight Lines [id:da05776141758085607] 
\draw  [dash pattern={on 0.84pt off 2.51pt}]  (504.74,119.85) -- (467.28,119.85) ;
%Straight Lines [id:da9078103279098249] 
\draw  [dash pattern={on 0.84pt off 2.51pt}]  (504.74,124.85) -- (467.28,124.85) ;

% Text Node
\draw (41,215.5) node [anchor=north west][inner sep=0.75pt]   [align=left] {(a)};
% Text Node
\draw (202,215.5) node [anchor=north west][inner sep=0.75pt]   [align=left] {(b)};
% Text Node
\draw (437,215.5) node [anchor=north west][inner sep=0.75pt]   [align=left] {(c)};


\end{tikzpicture}

    \caption{Interplay between a magnetic defect and a domain wall. 
    (a) A domain wall.
    (b) Domain wall stuck by defect.
    (c) Broken tube.}
    \label{fig:magnetic-defect-domain-wall-interplay}
\end{figure}

\subsection{Antiferromagnetic systems}

\subsubsection{Direction of $\vb*{H}$ and susceptibility}

\begin{figure}
    \centering
    \tikzset{every picture/.style={line width=0.75pt}} %set default line width to 0.75pt        

\begin{tikzpicture}[x=0.75pt,y=0.75pt,yscale=-0.8,xscale=0.8]
%uncomment if require: \path (0,300); %set diagram left start at 0, and has height of 300

%Straight Lines [id:da23700787016303826] 
\draw [color={rgb, 255:red, 208; green, 2; blue, 27 }  ,draw opacity=1 ]   (36,147) -- (75.83,147) ;
\draw [shift={(34,147)}, rotate = 0] [fill={rgb, 255:red, 208; green, 2; blue, 27 }  ,fill opacity=1 ][line width=0.08]  [draw opacity=0] (12,-3) -- (0,0) -- (12,3) -- cycle    ;
%Straight Lines [id:da9836829040434374] 
\draw [color={rgb, 255:red, 74; green, 144; blue, 226 }  ,draw opacity=1 ]   (85.83,147) -- (125.67,147) ;
\draw [shift={(127.67,147)}, rotate = 180] [fill={rgb, 255:red, 74; green, 144; blue, 226 }  ,fill opacity=1 ][line width=0.08]  [draw opacity=0] (12,-3) -- (0,0) -- (12,3) -- cycle    ;
%Straight Lines [id:da9282494645848378] 
\draw [color={rgb, 255:red, 208; green, 2; blue, 27 }  ,draw opacity=1 ]   (139.67,147) -- (179.5,147) ;
\draw [shift={(137.67,147)}, rotate = 0] [fill={rgb, 255:red, 208; green, 2; blue, 27 }  ,fill opacity=1 ][line width=0.08]  [draw opacity=0] (12,-3) -- (0,0) -- (12,3) -- cycle    ;
%Straight Lines [id:da5820259490762627] 
\draw [color={rgb, 255:red, 74; green, 144; blue, 226 }  ,draw opacity=1 ]   (189.5,147) -- (229.33,147) ;
\draw [shift={(231.33,147)}, rotate = 180] [fill={rgb, 255:red, 74; green, 144; blue, 226 }  ,fill opacity=1 ][line width=0.08]  [draw opacity=0] (12,-3) -- (0,0) -- (12,3) -- cycle    ;
%Straight Lines [id:da015198140066100319] 
\draw [color={rgb, 255:red, 208; green, 2; blue, 27 }  ,draw opacity=1 ]   (259.96,139.38) -- (299.83,147) ;
\draw [shift={(258,139)}, rotate = 10.83] [fill={rgb, 255:red, 208; green, 2; blue, 27 }  ,fill opacity=1 ][line width=0.08]  [draw opacity=0] (12,-3) -- (0,0) -- (12,3) -- cycle    ;
%Straight Lines [id:da44238795959698884] 
\draw [color={rgb, 255:red, 74; green, 144; blue, 226 }  ,draw opacity=1 ]   (309.83,147) -- (349.7,139.38) ;
\draw [shift={(351.67,139)}, rotate = 169.17] [fill={rgb, 255:red, 74; green, 144; blue, 226 }  ,fill opacity=1 ][line width=0.08]  [draw opacity=0] (12,-3) -- (0,0) -- (12,3) -- cycle    ;
%Straight Lines [id:da5681282150702647] 
\draw [color={rgb, 255:red, 208; green, 2; blue, 27 }  ,draw opacity=1 ]   (363.63,139.38) -- (403.5,147) ;
\draw [shift={(361.67,139)}, rotate = 10.83] [fill={rgb, 255:red, 208; green, 2; blue, 27 }  ,fill opacity=1 ][line width=0.08]  [draw opacity=0] (12,-3) -- (0,0) -- (12,3) -- cycle    ;
%Straight Lines [id:da6134912358779541] 
\draw [color={rgb, 255:red, 74; green, 144; blue, 226 }  ,draw opacity=1 ]   (413.5,147) -- (453.37,139.38) ;
\draw [shift={(455.33,139)}, rotate = 169.17] [fill={rgb, 255:red, 74; green, 144; blue, 226 }  ,fill opacity=1 ][line width=0.08]  [draw opacity=0] (12,-3) -- (0,0) -- (12,3) -- cycle    ;
%Straight Lines [id:da6965100693731445] 
\draw    (358,122) -- (358,42.19) ;
\draw [shift={(358,40.19)}, rotate = 90] [fill={rgb, 255:red, 0; green, 0; blue, 0 }  ][line width=0.08]  [draw opacity=0] (12,-3) -- (0,0) -- (12,3) -- cycle    ;

% Text Node
\draw (121,199.5) node [anchor=north west][inner sep=0.75pt]   [align=left] {(a)};
% Text Node
\draw (348,199.5) node [anchor=north west][inner sep=0.75pt]   [align=left] {(b)};
% Text Node
\draw (360,81.09) node [anchor=west] [inner sep=0.75pt]    {$H$};


\end{tikzpicture}

    \caption{Why a magnetic field perpendicular to the AFM order stimulates the strongest response}
    \label{fig:afm-response}
\end{figure}

To gain a response of $\vb*{M}$ -- instead of the AFM order parameter -- 
it's more convenient to apply an external field 
perpendicular to the molecular magnet moments. 
This has already been demonstrated experimentally:
below the antiferromagnetic transition temperature,
when $\vb*{H}$ is parallel to the direction of molecular magnetic moments,
as $T$ goes down, $\chi$ dramatically decreases, 
but when $\vb*{H}$ is perpendicular to molecular magnetic moments, 
$\chi$ almost doesn't change
(\prettyref{fig:afm-response}). 

\subsubsection{Domain walls}



\subsection{Exchange energy induced by itinerant electrons}

For any type of fermions, 
the larger the box is, 
the lower the energy can be.
That's to say, 
hopping can help to reduce the energy. TODO: quantitatively argument
This is the underlying mechanism for indirect exchange and 
super- and double-exchange.

\subsubsection{Superexchange}

\begin{figure}
    \centering
    \begin{tikzpicture}[x=0.75pt,y=0.75pt,yscale=-0.6,xscale=0.6]
    %uncomment if require: \path (0,300); %set diagram left start at 0, and has height of 300
    
    %Shape: Square [id:dp9015615478210528] 
    \draw   (33,110) -- (70.83,110) -- (70.83,147.83) -- (33,147.83) -- cycle ;
    %Shape: Square [id:dp311215149212267] 
    \draw   (70.83,110) -- (108.67,110) -- (108.67,147.83) -- (70.83,147.83) -- cycle ;
    %Shape: Square [id:dp020448992094490137] 
    \draw   (184.33,110) -- (222.17,110) -- (222.17,147.83) -- (184.33,147.83) -- cycle ;
    %Shape: Square [id:dp33283844067741963] 
    \draw   (108.67,110) -- (146.5,110) -- (146.5,147.83) -- (108.67,147.83) -- cycle ;
    %Shape: Square [id:dp7114859398363762] 
    \draw   (146.5,110) -- (184.33,110) -- (184.33,147.83) -- (146.5,147.83) -- cycle ;
    %Straight Lines [id:da14955803120703592] 
    \draw    (51.92,141.16) -- (51.92,118.68) ;
    \draw [shift={(51.92,116.68)}, rotate = 90] [fill={rgb, 255:red, 0; green, 0; blue, 0 }  ][line width=0.08]  [draw opacity=0] (12,-3) -- (0,0) -- (12,3) -- cycle    ;
    %Straight Lines [id:da1718789699337051] 
    \draw    (89.75,141.16) -- (89.75,118.68) ;
    \draw [shift={(89.75,116.68)}, rotate = 90] [fill={rgb, 255:red, 0; green, 0; blue, 0 }  ][line width=0.08]  [draw opacity=0] (12,-3) -- (0,0) -- (12,3) -- cycle    ;
    %Straight Lines [id:da5492296303313804] 
    \draw    (127.58,141.16) -- (127.58,118.68) ;
    \draw [shift={(127.58,116.68)}, rotate = 90] [fill={rgb, 255:red, 0; green, 0; blue, 0 }  ][line width=0.08]  [draw opacity=0] (12,-3) -- (0,0) -- (12,3) -- cycle    ;
    %Straight Lines [id:da7803820001216055] 
    \draw    (165.42,141.16) -- (165.42,118.68) ;
    \draw [shift={(165.42,116.68)}, rotate = 90] [fill={rgb, 255:red, 0; green, 0; blue, 0 }  ][line width=0.08]  [draw opacity=0] (12,-3) -- (0,0) -- (12,3) -- cycle    ;
    %Straight Lines [id:da08715111622312599] 
    \draw    (203.25,141.16) -- (203.25,118.68) ;
    \draw [shift={(203.25,116.68)}, rotate = 90] [fill={rgb, 255:red, 0; green, 0; blue, 0 }  ][line width=0.08]  [draw opacity=0] (12,-3) -- (0,0) -- (12,3) -- cycle    ;
    %Shape: Square [id:dp46528617160630903] 
    \draw   (358,110) -- (395.83,110) -- (395.83,147.83) -- (358,147.83) -- cycle ;
    %Shape: Square [id:dp9474006403088677] 
    \draw   (395.83,110) -- (433.67,110) -- (433.67,147.83) -- (395.83,147.83) -- cycle ;
    %Shape: Square [id:dp15356345607786182] 
    \draw   (509.33,110) -- (547.17,110) -- (547.17,147.83) -- (509.33,147.83) -- cycle ;
    %Shape: Square [id:dp6088605290734967] 
    \draw   (433.67,110) -- (471.5,110) -- (471.5,147.83) -- (433.67,147.83) -- cycle ;
    %Shape: Square [id:dp07077809356174858] 
    \draw   (471.5,110) -- (509.33,110) -- (509.33,147.83) -- (471.5,147.83) -- cycle ;
    %Straight Lines [id:da909759089884516] 
    \draw    (376.92,139.16) -- (376.92,116.68) ;
    \draw [shift={(376.92,141.16)}, rotate = 270] [fill={rgb, 255:red, 0; green, 0; blue, 0 }  ][line width=0.08]  [draw opacity=0] (12,-3) -- (0,0) -- (12,3) -- cycle    ;
    %Straight Lines [id:da5076104975483144] 
    \draw    (414.75,139.16) -- (414.75,116.68) ;
    \draw [shift={(414.75,141.16)}, rotate = 270] [fill={rgb, 255:red, 0; green, 0; blue, 0 }  ][line width=0.08]  [draw opacity=0] (12,-3) -- (0,0) -- (12,3) -- cycle    ;
    %Straight Lines [id:da09902208312732674] 
    \draw    (452.58,139.16) -- (452.58,116.68) ;
    \draw [shift={(452.58,141.16)}, rotate = 270] [fill={rgb, 255:red, 0; green, 0; blue, 0 }  ][line width=0.08]  [draw opacity=0] (12,-3) -- (0,0) -- (12,3) -- cycle    ;
    %Straight Lines [id:da09068138134518722] 
    \draw    (490.42,139.16) -- (490.42,116.68) ;
    \draw [shift={(490.42,141.16)}, rotate = 270] [fill={rgb, 255:red, 0; green, 0; blue, 0 }  ][line width=0.08]  [draw opacity=0] (12,-3) -- (0,0) -- (12,3) -- cycle    ;
    %Straight Lines [id:da06540931706293618] 
    \draw    (528.25,139.16) -- (528.25,116.68) ;
    \draw [shift={(528.25,141.16)}, rotate = 270] [fill={rgb, 255:red, 0; green, 0; blue, 0 }  ][line width=0.08]  [draw opacity=0] (12,-3) -- (0,0) -- (12,3) -- cycle    ;
    %Shape: Polygon Curved [id:ds1419759658889217] 
    \draw   (257.83,141.52) .. controls (232.83,135.52) and (234.83,121.52) .. (258.83,114.52) .. controls (282.83,107.52) and (300.83,148.52) .. (326.83,141.52) .. controls (352.83,134.52) and (346.83,121.52) .. (326.83,114.52) .. controls (306.83,107.52) and (282.83,147.52) .. (257.83,141.52) -- cycle ;
    %Straight Lines [id:da23962006126898316] 
    \draw    (264.58,139.16) -- (264.58,116.68) ;
    \draw [shift={(264.58,141.16)}, rotate = 270] [fill={rgb, 255:red, 0; green, 0; blue, 0 }  ][line width=0.08]  [draw opacity=0] (12,-3) -- (0,0) -- (12,3) -- cycle    ;
    %Straight Lines [id:da7394826196915838] 
    \draw    (321.58,140.16) -- (321.58,117.68) ;
    \draw [shift={(321.58,115.68)}, rotate = 90] [fill={rgb, 255:red, 0; green, 0; blue, 0 }  ][line width=0.08]  [draw opacity=0] (12,-3) -- (0,0) -- (12,3) -- cycle    ;
    %Curve Lines [id:da9922436246691269] 
    \draw [color={rgb, 255:red, 155; green, 155; blue, 155 }  ,draw opacity=1 ]   (205.26,106.87) .. controls (228.44,80.38) and (243.27,77.16) .. (264.58,108.68) ;
    \draw [shift={(203.83,108.52)}, rotate = 310.6] [fill={rgb, 255:red, 155; green, 155; blue, 155 }  ,fill opacity=1 ][line width=0.08]  [draw opacity=0] (12,-3) -- (0,0) -- (12,3) -- cycle    ;
    %Curve Lines [id:da9300905306472473] 
    \draw [color={rgb, 255:red, 155; green, 155; blue, 155 }  ,draw opacity=1 ]   (377.15,106.87) .. controls (353.98,80.38) and (339.15,77.16) .. (317.83,108.68) ;
    \draw [shift={(378.58,108.52)}, rotate = 229.4] [fill={rgb, 255:red, 155; green, 155; blue, 155 }  ,fill opacity=1 ][line width=0.08]  [draw opacity=0] (12,-3) -- (0,0) -- (12,3) -- cycle    ;
    
    % Text Node
    \draw (117,164.75) node [anchor=north west][inner sep=0.75pt]   [align=left] {3d$\displaystyle ^{5}$};
    % Text Node
    \draw (433,164.75) node [anchor=north west][inner sep=0.75pt]   [align=left] {3d$\displaystyle ^{5}$};
    % Text Node
    \draw (274,165.75) node [anchor=north west][inner sep=0.75pt]   [align=left] {oxygen};
    
    
    \end{tikzpicture}
    
    \caption{Superexchange interaction}
    \label{fig:superexchange}
\end{figure}

Antiferromagnetic in \ce{MnO} is confusing: 
the distance between \ce{Mn^{2+}} ions is too large for direct exchange.
The magnetic interaction therefore has to come from another mechanism.

The general idea is electrons tend to be delocalized to reduce Coulomb energy:
if they are localized, 
then clearly the density-density interaction channel will be very strong.
Of course, if somehow the crystal structure doesn't allow efficient hopping, 
then we are in a strongly correlated state. 
Now consider \prettyref{fig:superexchange}:
if the d orbitals from two different \ce{Mn} atoms are filled by 
electrons with their spins pointing to opposite directions, 
then hopping from the oxygen between the two \ce{Mn} atoms to the d orbitals
will be smooth; 
otherwise Pauli exclusion principle forbids one hopping channel. 

Superexchange is present in a simple system 
in which only nearest neighbor hopping is important 
and the role of Coulomb interaction 
is limited to on-site repulsion.
We know it's possible to integrate out hopping electrons in the half-filled Hubbard model 
to get a Heisenberg model.
This is a semi-quantitative demonstration of superexchange.

\subsubsection{Double exchange}

\begin{figure}
    \centering
    

\tikzset{every picture/.style={line width=0.75pt}} %set default line width to 0.75pt        

\begin{tikzpicture}[x=0.75pt,y=0.75pt,yscale=-0.6,xscale=0.6]
%uncomment if require: \path (0,407); %set diagram left start at 0, and has height of 407

%Straight Lines [id:da03130052291926644] 
\draw    (158,89) -- (211.83,89) ;
%Straight Lines [id:da14638729838051634] 
\draw    (241.83,89) -- (295.67,89) ;
%Straight Lines [id:da9531608725377501] 
\draw    (184.92,85) -- (184.92,56.85) ;
\draw [shift={(184.92,54.85)}, rotate = 90] [fill={rgb, 255:red, 0; green, 0; blue, 0 }  ][line width=0.08]  [draw opacity=0] (12,-3) -- (0,0) -- (12,3) -- cycle    ;
%Straight Lines [id:da523684010030097] 
\draw    (118,153) -- (171.83,153) ;
%Straight Lines [id:da7813668575753592] 
\draw    (201.83,153) -- (255.67,153) ;
%Straight Lines [id:da21544319074178375] 
\draw    (144.92,149) -- (144.92,120.85) ;
\draw [shift={(144.92,118.85)}, rotate = 90] [fill={rgb, 255:red, 0; green, 0; blue, 0 }  ][line width=0.08]  [draw opacity=0] (12,-3) -- (0,0) -- (12,3) -- cycle    ;
%Straight Lines [id:da21644789327584002] 
\draw    (284.83,153) -- (338.67,153) ;
%Straight Lines [id:da9264682078374162] 
\draw    (228.75,149) -- (228.75,120.85) ;
\draw [shift={(228.75,118.85)}, rotate = 90] [fill={rgb, 255:red, 0; green, 0; blue, 0 }  ][line width=0.08]  [draw opacity=0] (12,-3) -- (0,0) -- (12,3) -- cycle    ;
%Straight Lines [id:da8086560684186275] 
\draw    (311.75,149) -- (311.75,120.85) ;
\draw [shift={(311.75,118.85)}, rotate = 90] [fill={rgb, 255:red, 0; green, 0; blue, 0 }  ][line width=0.08]  [draw opacity=0] (12,-3) -- (0,0) -- (12,3) -- cycle    ;
%Straight Lines [id:da07010945752186815] 
\draw    (158,255) -- (211.83,255) ;
%Straight Lines [id:da860954015648226] 
\draw    (241.83,255) -- (295.67,255) ;
%Straight Lines [id:da8419847000317056] 
\draw    (118,319) -- (171.83,319) ;
%Straight Lines [id:da010896955388421725] 
\draw    (201.83,319) -- (255.67,319) ;
%Straight Lines [id:da7909484055082396] 
\draw    (144.92,315) -- (144.92,286.85) ;
\draw [shift={(144.92,284.85)}, rotate = 90] [fill={rgb, 255:red, 0; green, 0; blue, 0 }  ][line width=0.08]  [draw opacity=0] (12,-3) -- (0,0) -- (12,3) -- cycle    ;
%Straight Lines [id:da1178544566981643] 
\draw    (284.83,319) -- (338.67,319) ;
%Straight Lines [id:da9120813893067474] 
\draw    (228.75,315) -- (228.75,286.85) ;
\draw [shift={(228.75,284.85)}, rotate = 90] [fill={rgb, 255:red, 0; green, 0; blue, 0 }  ][line width=0.08]  [draw opacity=0] (12,-3) -- (0,0) -- (12,3) -- cycle    ;
%Straight Lines [id:da8667936523922042] 
\draw    (311.75,315) -- (311.75,286.85) ;
\draw [shift={(311.75,284.85)}, rotate = 90] [fill={rgb, 255:red, 0; green, 0; blue, 0 }  ][line width=0.08]  [draw opacity=0] (12,-3) -- (0,0) -- (12,3) -- cycle    ;
%Straight Lines [id:da553686933549888] 
\draw    (457,89) -- (510.83,89) ;
%Straight Lines [id:da08427689960431484] 
\draw    (540.83,89) -- (594.67,89) ;
%Straight Lines [id:da7793487755960506] 
\draw    (483.92,85) -- (483.92,56.85) ;
\draw [shift={(483.92,54.85)}, rotate = 90] [fill={rgb, 255:red, 0; green, 0; blue, 0 }  ][line width=0.08]  [draw opacity=0] (12,-3) -- (0,0) -- (12,3) -- cycle    ;
%Straight Lines [id:da12246367954119552] 
\draw    (417,153) -- (470.83,153) ;
%Straight Lines [id:da07271823765657959] 
\draw    (500.83,153) -- (554.67,153) ;
%Straight Lines [id:da30130198639796224] 
\draw    (443.92,149) -- (443.92,120.85) ;
\draw [shift={(443.92,118.85)}, rotate = 90] [fill={rgb, 255:red, 0; green, 0; blue, 0 }  ][line width=0.08]  [draw opacity=0] (12,-3) -- (0,0) -- (12,3) -- cycle    ;
%Straight Lines [id:da34817457507300853] 
\draw    (583.83,153) -- (637.67,153) ;
%Straight Lines [id:da9278658598401492] 
\draw    (527.75,149) -- (527.75,120.85) ;
\draw [shift={(527.75,118.85)}, rotate = 90] [fill={rgb, 255:red, 0; green, 0; blue, 0 }  ][line width=0.08]  [draw opacity=0] (12,-3) -- (0,0) -- (12,3) -- cycle    ;
%Straight Lines [id:da2806907833772132] 
\draw    (610.75,149) -- (610.75,120.85) ;
\draw [shift={(610.75,118.85)}, rotate = 90] [fill={rgb, 255:red, 0; green, 0; blue, 0 }  ][line width=0.08]  [draw opacity=0] (12,-3) -- (0,0) -- (12,3) -- cycle    ;
%Straight Lines [id:da5816016489121218] 
\draw    (457,255) -- (510.83,255) ;
%Straight Lines [id:da5598519278766829] 
\draw    (540.83,255) -- (594.67,255) ;
%Straight Lines [id:da9845231658402223] 
\draw    (417,319) -- (470.83,319) ;
%Straight Lines [id:da7638969052737217] 
\draw    (500.83,319) -- (554.67,319) ;
%Straight Lines [id:da4205402527869917] 
\draw    (443.92,313) -- (443.92,284.85) ;
\draw [shift={(443.92,315)}, rotate = 270] [fill={rgb, 255:red, 0; green, 0; blue, 0 }  ][line width=0.08]  [draw opacity=0] (12,-3) -- (0,0) -- (12,3) -- cycle    ;
%Straight Lines [id:da1214080709959593] 
\draw    (583.83,319) -- (637.67,319) ;
%Straight Lines [id:da4915803714613072] 
\draw    (527.75,313) -- (527.75,284.85) ;
\draw [shift={(527.75,315)}, rotate = 270] [fill={rgb, 255:red, 0; green, 0; blue, 0 }  ][line width=0.08]  [draw opacity=0] (12,-3) -- (0,0) -- (12,3) -- cycle    ;
%Straight Lines [id:da1329154077868715] 
\draw    (610.75,313) -- (610.75,284.85) ;
\draw [shift={(610.75,315)}, rotate = 270] [fill={rgb, 255:red, 0; green, 0; blue, 0 }  ][line width=0.08]  [draw opacity=0] (12,-3) -- (0,0) -- (12,3) -- cycle    ;
%Straight Lines [id:da618822820923101] 
\draw  [dash pattern={on 0.84pt off 2.51pt}]  (184.92,251) -- (184.92,222.85) ;
\draw [shift={(184.92,220.85)}, rotate = 90] [fill={rgb, 255:red, 0; green, 0; blue, 0 }  ][line width=0.08]  [draw opacity=0] (12,-3) -- (0,0) -- (12,3) -- cycle    ;
%Straight Lines [id:da6181635469013511] 
\draw [color={rgb, 255:red, 208; green, 2; blue, 27 }  ,draw opacity=1 ] [dash pattern={on 0.84pt off 2.51pt}]  (483.92,251) -- (483.92,222.85) ;
\draw [shift={(483.92,220.85)}, rotate = 90] [fill={rgb, 255:red, 208; green, 2; blue, 27 }  ,fill opacity=1 ][line width=0.08]  [draw opacity=0] (12,-3) -- (0,0) -- (12,3) -- cycle    ;
%Curve Lines [id:da9293459240386781] 
\draw [color={rgb, 255:red, 155; green, 155; blue, 155 }  ,draw opacity=1 ]   (173.51,243.71) .. controls (21.24,189.95) and (83.65,107.19) .. (172.49,74.02) ;
\draw [shift={(173.83,73.52)}, rotate = 159.86] [fill={rgb, 255:red, 155; green, 155; blue, 155 }  ,fill opacity=1 ][line width=0.08]  [draw opacity=0] (12,-3) -- (0,0) -- (12,3) -- cycle    ;
\draw [shift={(175.83,244.52)}, rotate = 199.09] [fill={rgb, 255:red, 155; green, 155; blue, 155 }  ,fill opacity=1 ][line width=0.08]  [draw opacity=0] (12,-3) -- (0,0) -- (12,3) -- cycle    ;
%Straight Lines [id:da9889807704577043] 
\draw [color={rgb, 255:red, 248; green, 231; blue, 28 }  ,draw opacity=1 ]   (483.92,235.93) .. controls (484.45,238.22) and (483.56,239.64) .. (481.27,240.17) .. controls (478.98,240.7) and (478.09,242.12) .. (478.62,244.41) .. controls (479.15,246.7) and (478.26,248.12) .. (475.97,248.65) .. controls (473.68,249.18) and (472.79,250.6) .. (473.32,252.89) .. controls (473.85,255.18) and (472.96,256.6) .. (470.67,257.13) .. controls (468.38,257.66) and (467.49,259.08) .. (468.02,261.37) .. controls (468.55,263.66) and (467.66,265.08) .. (465.37,265.61) .. controls (463.08,266.14) and (462.19,267.56) .. (462.72,269.85) .. controls (463.25,272.14) and (462.36,273.56) .. (460.07,274.09) .. controls (457.78,274.62) and (456.89,276.04) .. (457.42,278.33) .. controls (457.95,280.62) and (457.06,282.04) .. (454.77,282.57) .. controls (452.48,283.1) and (451.59,284.52) .. (452.12,286.81) .. controls (452.65,289.1) and (451.76,290.52) .. (449.47,291.05) .. controls (447.18,291.58) and (446.29,293) .. (446.82,295.29) .. controls (447.35,297.58) and (446.46,299) .. (444.17,299.53) -- (443.92,299.93) -- (443.92,299.93) ;
%Straight Lines [id:da04916643473945981] 
\draw [color={rgb, 255:red, 248; green, 231; blue, 28 }  ,draw opacity=1 ]   (483.92,235.93) .. controls (486.23,236.36) and (487.17,237.74) .. (486.74,240.05) .. controls (486.31,242.37) and (487.25,243.75) .. (489.57,244.18) .. controls (491.88,244.61) and (492.82,245.99) .. (492.39,248.3) .. controls (491.96,250.62) and (492.9,252) .. (495.22,252.43) .. controls (497.53,252.86) and (498.47,254.24) .. (498.04,256.55) .. controls (497.61,258.87) and (498.55,260.25) .. (500.87,260.68) .. controls (503.18,261.11) and (504.12,262.49) .. (503.69,264.8) .. controls (503.26,267.12) and (504.2,268.5) .. (506.52,268.93) .. controls (508.83,269.36) and (509.77,270.74) .. (509.34,273.05) .. controls (508.91,275.37) and (509.85,276.75) .. (512.17,277.18) .. controls (514.49,277.61) and (515.43,278.98) .. (515,281.3) .. controls (514.57,283.61) and (515.51,284.99) .. (517.82,285.43) .. controls (520.14,285.86) and (521.08,287.23) .. (520.65,289.55) .. controls (520.22,291.86) and (521.16,293.24) .. (523.47,293.68) .. controls (525.79,294.11) and (526.73,295.49) .. (526.3,297.81) -- (527.75,299.93) -- (527.75,299.93) ;
%Straight Lines [id:da4693533104050438] 
\draw [color={rgb, 255:red, 248; green, 231; blue, 28 }  ,draw opacity=1 ]   (483.92,235.93) .. controls (486.15,235.19) and (487.64,235.94) .. (488.38,238.18) .. controls (489.11,240.42) and (490.6,241.17) .. (492.84,240.43) .. controls (495.08,239.69) and (496.57,240.44) .. (497.31,242.68) .. controls (498.04,244.92) and (499.53,245.67) .. (501.77,244.94) .. controls (504.01,244.2) and (505.5,244.95) .. (506.24,247.19) .. controls (506.97,249.43) and (508.46,250.18) .. (510.7,249.44) .. controls (512.94,248.7) and (514.43,249.45) .. (515.16,251.69) .. controls (515.9,253.93) and (517.39,254.68) .. (519.63,253.95) .. controls (521.87,253.21) and (523.36,253.96) .. (524.09,256.2) .. controls (524.83,258.44) and (526.32,259.19) .. (528.56,258.45) .. controls (530.8,257.71) and (532.29,258.46) .. (533.02,260.7) .. controls (533.75,262.94) and (535.24,263.69) .. (537.48,262.96) .. controls (539.72,262.22) and (541.21,262.97) .. (541.95,265.21) .. controls (542.68,267.45) and (544.17,268.2) .. (546.41,267.46) .. controls (548.65,266.72) and (550.14,267.47) .. (550.88,269.71) .. controls (551.61,271.95) and (553.1,272.7) .. (555.34,271.97) .. controls (557.58,271.23) and (559.07,271.98) .. (559.8,274.22) .. controls (560.54,276.46) and (562.03,277.21) .. (564.27,276.47) .. controls (566.51,275.73) and (568,276.48) .. (568.73,278.72) .. controls (569.46,280.96) and (570.95,281.71) .. (573.19,280.98) .. controls (575.43,280.24) and (576.92,280.99) .. (577.66,283.23) .. controls (578.39,285.47) and (579.88,286.22) .. (582.12,285.48) .. controls (584.36,284.74) and (585.85,285.49) .. (586.59,287.73) .. controls (587.32,289.97) and (588.81,290.72) .. (591.05,289.99) .. controls (593.29,289.25) and (594.78,290) .. (595.51,292.24) .. controls (596.25,294.48) and (597.74,295.23) .. (599.98,294.49) .. controls (602.22,293.75) and (603.71,294.5) .. (604.44,296.74) .. controls (605.18,298.98) and (606.67,299.73) .. (608.91,299) -- (610.75,299.93) -- (610.75,299.93) ;

% Text Node
\draw (29,107) node [anchor=north west][inner sep=0.75pt]   [align=left] {\ce{Mn^{3+}}};
% Text Node
\draw (184.92,92) node [anchor=north] [inner sep=0.75pt]   [align=left] {$\displaystyle d_{z^{2}}$};
% Text Node
\draw (268.75,92) node [anchor=north] [inner sep=0.75pt]   [align=left] {$\displaystyle d_{x^{2} -y^{2}}$};
% Text Node
\draw (144.92,156) node [anchor=north] [inner sep=0.75pt]   [align=left] {$\displaystyle d_{xy}$};
% Text Node
\draw (228.75,156) node [anchor=north] [inner sep=0.75pt]   [align=left] {$\displaystyle d_{yz}$};
% Text Node
\draw (311.75,156) node [anchor=north] [inner sep=0.75pt]   [align=left] {$\displaystyle d_{zx}$};
% Text Node
\draw (184.92,258) node [anchor=north] [inner sep=0.75pt]   [align=left] {$\displaystyle d_{z^{2}}$};
% Text Node
\draw (268.75,258) node [anchor=north] [inner sep=0.75pt]   [align=left] {$\displaystyle d_{x^{2} -y^{2}}$};
% Text Node
\draw (144.92,322) node [anchor=north] [inner sep=0.75pt]   [align=left] {$\displaystyle d_{xy}$};
% Text Node
\draw (228.75,322) node [anchor=north] [inner sep=0.75pt]   [align=left] {$\displaystyle d_{yz}$};
% Text Node
\draw (311.75,322) node [anchor=north] [inner sep=0.75pt]   [align=left] {$\displaystyle d_{zx}$};
% Text Node
\draw (29,275) node [anchor=north west][inner sep=0.75pt]   [align=left] {\ce{Mn^{4+}}};
% Text Node
\draw (483.92,92) node [anchor=north] [inner sep=0.75pt]   [align=left] {$\displaystyle d_{z^{2}}$};
% Text Node
\draw (567.75,92) node [anchor=north] [inner sep=0.75pt]   [align=left] {$\displaystyle d_{x^{2} -y^{2}}$};
% Text Node
\draw (443.92,156) node [anchor=north] [inner sep=0.75pt]   [align=left] {$\displaystyle d_{xy}$};
% Text Node
\draw (527.75,156) node [anchor=north] [inner sep=0.75pt]   [align=left] {$\displaystyle d_{yz}$};
% Text Node
\draw (610.75,156) node [anchor=north] [inner sep=0.75pt]   [align=left] {$\displaystyle d_{zx}$};
% Text Node
\draw (483.92,258) node [anchor=north] [inner sep=0.75pt]   [align=left] {$\displaystyle d_{z^{2}}$};
% Text Node
\draw (567.75,258) node [anchor=north] [inner sep=0.75pt]   [align=left] {$\displaystyle d_{x^{2} -y^{2}}$};
% Text Node
\draw (443.92,322) node [anchor=north] [inner sep=0.75pt]   [align=left] {$\displaystyle d_{xy}$};
% Text Node
\draw (527.75,322) node [anchor=north] [inner sep=0.75pt]   [align=left] {$\displaystyle d_{yz}$};
% Text Node
\draw (610.75,322) node [anchor=north] [inner sep=0.75pt]   [align=left] {$\displaystyle d_{zx}$};


\end{tikzpicture}

    \caption{Double exchange}
    \label{fig:double-exchange}
\end{figure}

In a material we have \ce{Mn^{3+}} and \ce{Mn^{4+}} ions,
and because of crystal-field splitting, 
two of the d orbitals have higher energies than the rest of the three. 
Then to allow the electron filling the upper two orbitals in \ce{Mn^{3+}}
to hop to \ce{Mn^{4+}},
we have to make the magnetic moments of \ce{Mn^{3+}} and \ce{Mn^{4+}} ions the same, 
or otherwise after the hopping, 
a strong Hund energy penalty occurs
(\prettyref{fig:double-exchange}). 

\subsubsection{Dzyaloshinskii-Moriya interaction}\label{sec:localized.itinerant-media.dm}

The \concept{Dzyaloshinskii-Moriya interaction} 
\begin{equation}
    H = \vb*{D} \cdot (\vb*{S}_1 \times \vb*{S}_2)
\end{equation}
comes from 
the second order perturbation 
of Anderson superexchange and spin-orbital coupling
(the first step being SOC and the second step being superexchange,
or the inverse). 
The cross multiplication symbol comes 
when we sum $(\vb*{S}_1 \cdot \vb*{L}_1) (\vb*{S}_1 \cdot \vb*{S}_2)$ terms together.
The $\vb*{D}$ vector appears only when the inversion symmetry is broken. 

This interaction motivates 
spins to be \emph{not} parallel to each other, 
forming vortex configurations of spins. 
Since the inversion symmetry is by definition broken 
in interfacial magnets, 
it can be as large as \SI{20}{\percent} of direct exchange,
and therefore is strong enough to compete with the latter. 
It directly gives rise to skyrmions, 
and from skyrmions TODO: Wei et al. Rare Metals 40, 3076 (2021).

\subsubsection{The RKKY interaction}

\begin{figure}
    \centering
    \tikzset{every picture/.style={line width=0.75pt}} %set default line width to 0.75pt        

\begin{tikzpicture}[x=0.75pt,y=0.75pt,yscale=-0.7,xscale=0.7]
%uncomment if require: \path (0,300); %set diagram left start at 0, and has height of 300

%Shape: Rectangle [id:dp3897188190860059] 
\draw   (102,205.52) -- (296.83,205.52) -- (296.83,223.52) -- (102,223.52) -- cycle ;
%Shape: Rectangle [id:dp6888483467781361] 
\draw   (102,162.52) -- (296.83,162.52) -- (296.83,205.52) -- (102,205.52) -- cycle ;
%Straight Lines [id:da8737496131021489] 
\draw    (160.5,214.52) -- (236.33,214.52) ;
\draw [shift={(238.33,214.52)}, rotate = 180] [fill={rgb, 255:red, 0; green, 0; blue, 0 }  ][line width=0.08]  [draw opacity=0] (12,-3) -- (0,0) -- (12,3) -- cycle    ;
%Shape: Rectangle [id:dp7116079291498749] 
\draw   (102,144.52) -- (296.83,144.52) -- (296.83,162.52) -- (102,162.52) -- cycle ;
%Straight Lines [id:da09691248809748632] 
\draw    (181.96,153.52) -- (218.87,153.52) ;
\draw [shift={(179.96,153.52)}, rotate = 0] [fill={rgb, 255:red, 0; green, 0; blue, 0 }  ][line width=0.08]  [draw opacity=0] (12,-3) -- (0,0) -- (12,3) -- cycle    ;
%Shape: Rectangle [id:dp6421203027479565] 
\draw   (102,69.52) -- (296.83,69.52) -- (296.83,144.52) -- (102,144.52) -- cycle ;
%Shape: Rectangle [id:dp03151265514505708] 
\draw   (102,51.52) -- (296.83,51.52) -- (296.83,69.52) -- (102,69.52) -- cycle ;
%Straight Lines [id:da2218769473186446] 
\draw    (281.83,71.52) -- (281.83,142.52) ;
\draw [shift={(281.83,144.52)}, rotate = 270] [fill={rgb, 255:red, 0; green, 0; blue, 0 }  ][line width=0.08]  [draw opacity=0] (12,-3) -- (0,0) -- (12,3) -- cycle    ;
\draw [shift={(281.83,69.52)}, rotate = 90] [fill={rgb, 255:red, 0; green, 0; blue, 0 }  ][line width=0.08]  [draw opacity=0] (12,-3) -- (0,0) -- (12,3) -- cycle    ;

% Text Node
\draw (306.33,214.52) node [anchor=west] [inner sep=0.75pt]   [align=left] {magnetized layer 1};
% Text Node
\draw (199.42,184.02) node   [align=left] {spacer 1};
% Text Node
\draw (306.33,152.52) node [anchor=west] [inner sep=0.75pt]   [align=left] {magnetized layer 2};
% Text Node
\draw (199.42,107.02) node   [align=left] {spacer 2};
% Text Node
\draw (307,52) node [anchor=north west][inner sep=0.75pt]   [align=left] {test layer};
% Text Node
\draw (308,98) node [anchor=north west][inner sep=0.75pt]   [align=left] {variable height};


\end{tikzpicture}

    \caption{Experimental demonstration of RKKY interaction. 
    The relation between the height of the second spacer 
    and the magnetization of the uppermost layer is oscillatory.}
    \label{fig:rkky-experiment}
\end{figure}

A clear experimental demonstration of the ability for itinerant electrons 
to mediate the interaction between (very far) magnetic moments 
is shown in \prettyref{fig:rkky-experiment}:
the fact that the test layer still get magnetized 
even when the height of the second spacer layer is large 
means there must be some long-range spin coupling in the system,
and that the direction of the former's magnetization 
depends on the height of the spacer layer 
means this coupling has some interference nature. 

This oscillation of the magnetization can be attributed to the mechanism below.
We know there is direct exchange interaction 
between a localized magnetic moment and an itinerant electron, 
so the spin polarization around a local magnetic moment 
is regulated by it. 
We know electrons scattered by the same impurity can interfere with each other, 
with a spatial interference length of $\sim 1 / k_{\text{F}}$ -- 
this is exactly why we see interference rings in STM around impurities. 
Now we see the local magnetic moment as an impurity,
and if another local magnetic moment 
is placed in this surrounding of these spin-polarized electrons, 
due to its direct exchange interaction with the latter, 
a spatially oscillating magnetic coupling between the two local magnetic moments occurs. 
This is the \concept{Ruderman–Kittel–Kasuya–Yosida (RKKY)} interaction. 

\section{Itinerant magnetism}

\subsection{Stoner magnets}

Consider the density-density channel of electron-electron Coulomb repulsion $U n_\uparrow n_\downarrow$. 
To reduce the influence of this term in the total energy, 
electrons may choose to have a larger $u_\uparrow$ and therefore a smaller $n_{\downarrow}$
(or the inverse -- a spontaneous symmetry breaking is involved here),
and hence 
\begin{equation}
    \var{E} = U (n - \var{n}) (n + \var{n}) + \Delta \var{n} - U n^2,
\end{equation}
where $\Delta = \var{n} / \rho(\efermi)$.
If $\var{E} < 0$, which is equivalent to 
\begin{equation}
    \rho(\efermi) U > 1,
\end{equation}
then a magnetization order is formed. 
Magnets with this kind of magnetization mechanism 
are \concept{Stoner magnets}.

In real materials, 
what contributes to the Stoner mechanism 
are highly itinerant electrons;
f orbitals, since they are highly localized, 
have almost no contribution to the Fermi surface 
and therefore don't contribute to Stoner magnetism,
though they definitely contribute to localized magnetism.
But this doesn't mean electrons that do contribute to the Fermi surface 
to have very dispersive band structures,
or otherwise $\rho(\efermi)$ is very low. 
So the Stoner criterion works best for transitional metals, 
and it correctly predicts that \ce{Fe} and \ce{Ni} are magnetic.

When $\rho(\efermi) U < 1$, 
we don't have spontaneous magnetization, 
but still we find enhanced magnetic susceptibility under an external magnetic field. 
The energy variance under an external $B$ field is 
\begin{equation}
    \var{E} = - U (\var{n})^2 
    + \frac{1}{\rho(\efermi)} (\var{n})^2
    - M B, \quad 
    M = \mu_\text{B} \var{n}.
\end{equation}
So now we can minimize $\var{E}$ and find $M$, 
and $\chi$ can be found. 

\section{Topological 2D spin textures}

As is said in \prettyref{sec:localized.itinerant-media.dm}, 
certain excitations in which a clear pattern of spin texture 
can be induced by certain types of spin-spin interactions.
Many of them can be characterized by 
some topological charges. 
We have the \concept{Neel type skyrmion} 
(the direction of spins are all radial), 
the \concept{Bloch type skyrmion}
(the direction of spins are angular),
and hence we can define 
\concept{antiskyrmion} of a skyrmion.
A \concept{meron} is a half-skyrmion: 
if we extract the central part of a skyrmion 
and instruct the $r \to \infty$ spins to be perpendicular to the spin at $r=0$,
we get a meron.

All excitations in the skyrmion family can be characterized by 
\begin{equation}
    N = \frac{1}{4\pi} \int \vb*{n} \cdot \pdv{\vb*{n}}{x} \times \pdv{\vb*{n}}{y} \dd{x} \dd{y}.
\end{equation}
For skyrmions and antiskyrmions $N = \pm 1$; 
for merons and antimerons $N = \pm 1/2$.

Detection of these excitations usually involves neutrons; 
optical detection is not possible 
because photons can't see magnetic orders.
The imaging results in a periodic pattern 
and clear diffraction peaks. 

\section{Magnetic frustration}

Note that when deriving effective spin-spin interactions, 
we make no guarantee that 
a magnetic order can always be formed. 
The antiferromagnetic Heisenberg model, for example, 
doesn't lead to a stable AFM order on a triangular lattice. 
This leads to several possible consequences:
\begin{itemize}
    \item Slow spin dynamics: 
        relaxation may be slow enough,
        and we may get spin glass.
    \item Non-ergodicity.
    \item High residual entropy at $T \to 0$,
    \item Potential magnetic monopoles
     (TODO: Leon Balent). 
\end{itemize}

\section{Spin excitations}

\subsection{Magnons}

One thing to keep in mind is 
the magneto-crystalline anisotropy always leads to 
an energy gap of the magnon spectrum: 
this drastically changes the behavior of the magnons 
when $T \to 0$:
we get an exponentially decaying contribution to the specific heat,
quite similar to that of optical phonons. 
In a material with a strong magneto-crystalline anisotropy,
below a characteristic temperature,
the specific heat indeed drops exponentially.

In an AFM background, 
the magnon has linear dispersion relation when $\vb*{q} \to 0$ -- 
but the max energy can be as high as \SI{0.3}{eV},
much higher than the usual energy of acoustic phonons. 

\end{document}