\documentclass[hyperref, a4paper]{article}

\usepackage{geometry}
\usepackage{titling}
\usepackage{titlesec}
% No longer needed, since we will use enumitem package
% \usepackage{paralist}
\usepackage{enumitem}
\usepackage{footnote}
\usepackage{amsmath, amssymb, amsthm}
\usepackage{mathtools}
\usepackage{bbm}
\usepackage{graphicx}
\usepackage{subcaption}
\usepackage{physics}
\usepackage{tensor}
\usepackage{siunitx}
\usepackage[version=4]{mhchem}
\usepackage{tikz}
\usepackage{xcolor}
\usepackage{listings}
\usepackage{autobreak}
\usepackage[ruled, vlined, linesnumbered]{algorithm2e}
\usepackage{nameref,zref-xr}
\zxrsetup{toltxlabel}
\usepackage[backend=bibtex]{biblatex}
\addbibresource{data.bib}
\addbibresource{experiments.bib}
\addbibresource{theory.bib}
\usepackage[colorlinks,unicode]{hyperref} % , linkcolor=black, anchorcolor=black, citecolor=black, urlcolor=black, filecolor=black
\usepackage[most]{tcolorbox}
\usepackage{prettyref}

% Page style
\geometry{left=3.18cm,right=3.18cm,top=2.54cm,bottom=2.54cm}
\titlespacing{\paragraph}{0pt}{1pt}{10pt}[20pt]
\setlength{\droptitle}{-5em}

% More compact lists 
\setlist[itemize]{
    itemindent=17pt, 
    leftmargin=1pt,
    listparindent=\parindent,
    parsep=0pt,
}

% Math operators
\DeclareMathOperator{\timeorder}{\mathcal{T}}
\DeclareMathOperator{\diag}{diag}
\DeclareMathOperator{\legpoly}{P}
\DeclareMathOperator{\primevalue}{P}
\DeclareMathOperator{\sgn}{sgn}
\DeclareMathOperator{\res}{Res}
\newcommand*{\ii}{\mathrm{i}}
\newcommand*{\ee}{\mathrm{e}}
\newcommand*{\const}{\mathrm{const}}
\newcommand*{\suchthat}{\quad \text{s.t.} \quad}
\newcommand*{\argmin}{\arg\min}
\newcommand*{\argmax}{\arg\max}
\newcommand*{\normalorder}[1]{: #1 :}
\newcommand*{\pair}[1]{\langle #1 \rangle}
\newcommand*{\fd}[1]{\mathcal{D} #1}
\DeclareMathOperator{\bigO}{\mathcal{O}}

% TikZ setting
\usetikzlibrary{arrows,shapes,positioning}
\usetikzlibrary{arrows.meta}
\usetikzlibrary{decorations.markings}
\tikzstyle arrowstyle=[scale=1]
\tikzstyle directed=[postaction={decorate,decoration={markings,
    mark=at position .5 with {\arrow[arrowstyle]{stealth}}}}]
\tikzstyle ray=[directed, thick]
\tikzstyle dot=[anchor=base,fill,circle,inner sep=1pt]

% Algorithm setting
% Julia-style code
\SetKwIF{If}{ElseIf}{Else}{if}{}{elseif}{else}{end}
\SetKwFor{For}{for}{}{end}
\SetKwFor{While}{while}{}{end}
\SetKwProg{Function}{function}{}{end}
\SetArgSty{textnormal}

\newcommand*{\concept}[1]{{\textbf{#1}}}

% Embedded codes
\lstset{basicstyle=\ttfamily,
  showstringspaces=false,
  commentstyle=\color{gray},
  keywordstyle=\color{blue}
}

% Reference formatting
\newrefformat{fig}{Fig.~\ref{#1}}
\newcommand*{\term}[1]{\textit{#1}}

% Color boxes
\tcbuselibrary{skins, breakable, theorems}

\newtcbtheorem[number within=section]{infobox}{Box}{
    enhanced,
    boxrule=0pt,
    colback=blue!5,
    colframe=blue!5,
    coltitle=blue!50,
    borderline west={4pt}{0pt}{blue!65},
    sharp corners,
    fonttitle=\bfseries, 
    breakable,
    before upper={\parindent15pt\noindent}}{box}
\newtcbtheorem[number within=section, use counter from=infobox]{theorybox}{Box}{
    enhanced,
    boxrule=0pt,
    colback=orange!5, 
    colframe=orange!5, 
    coltitle=orange!50,
    borderline west={4pt}{0pt}{orange!65},
    sharp corners,
    fonttitle=\bfseries, 
    breakable,
    before upper={\parindent15pt\noindent}}{box}
\newtcbtheorem[number within=section, use counter from=infobox]{learnbox}{Box}{
    enhanced,
    boxrule=0pt,
    colback=green!5,
    colframe=green!5,
    coltitle=green!50,
    borderline west={4pt}{0pt}{green!65},
    sharp corners,
    fonttitle=\bfseries, 
    breakable,
    before upper={\parindent15pt\noindent}}{box}

% Displaying texts in bookmarkers

\pdfstringdefDisableCommands{%
  \def\\{}%
  \def\ce#1{<#1>}%
}

\pdfstringdefDisableCommands{%
  \def\texttt#1{<#1>}%
  \def\mathbb#1{#1}%
}
\pdfstringdefDisableCommands{\def\eqref#1{(\ref{#1})}}

\makeatletter
\pdfstringdefDisableCommands{\let\HyPsd@CatcodeWarning\@gobble}
\makeatother

\newenvironment{shelldisplay}{\begin{lstlisting}}{\end{lstlisting}}

\newcommand*{\efermi}{E_{\text{F}}}
\newcommand*{\sA}{\text{A}}
\newcommand*{\sB}{\text{B}}
\newcommand*{\Tc}{T_{\text{c}}}
\newcommand*{\muB}{\mu_{\text{B}}}
\newcommand*{\kB}{k_{\text{B}}}

\title{Homework 3}
\author{Jinyuan Wu}

\begin{document}

\maketitle

\section{}

\subsection{}

For the free electron gas, we have 
\begin{equation}
    \mu = \frac{1}{2m} \left(\frac{3 \pi^2 N}{V}\right)^{2/3},
\end{equation}
where we have considered the fact that electrons are spin-1/2 particles.
The free electron gas Green function is 
\begin{equation}
    G_{\sigma}(\vb*{k}, \omega) = \frac{1}{\omega - \frac{\vb*{k}^2}{2m} + \mu + \ii \sgn(\omega) 0^+}.
\end{equation}
The imaginary part is therefore 
\begin{equation}
    \begin{aligned}
        A_\sigma(\vb*{k}, \omega) &= - \frac{1}{\pi} \Im G_{\sigma}(\vb*{k}, \omega)
        = - \frac{1}{\pi} \cdot \Im (- \pi \ii) \sgn (\omega) \delta\left(
            \omega - \frac{\vb*{k}^2}{2m} + \mu
        \right) \\
        &= \sgn\left(
            \frac{\vb*{k}^2}{2m} - \mu
        \right) \delta\left(
            \omega - \frac{\vb*{k}^2}{2m} + \mu
        \right).
    \end{aligned}
\end{equation}
This function is not really analytical, 
since its peak has infinitesimal width 
and infinite height; 
but adding a small imaginary part to the energy of a single electron,
the spectral function should be analytical.

\subsection{}

When 
\begin{equation}
    \Sigma = \lambda \omega^2 + \ii (\alpha \omega^2 + \beta T^2),
\end{equation}
we have 
\begin{equation}
    \begin{aligned}
        G_\sigma(\vb*{k}, \omega)
        &= \frac{1}{\omega - \frac{\vb*{k}^2}{2m} - \Sigma + \ii \sgn(\omega) 0^+} \\
        &= \frac{1}{(1 - \lambda) \omega - \frac{\vb*{k}^2}{2m} + \mu - \ii (\alpha \omega^2 + \beta T^2)} \\
        &= \frac{(1 - \lambda) \omega - \frac{\vb*{k}^2}{2m} + \mu + \ii (\alpha \omega^2 + \beta T^2)}{
            \left(
                (1 - \lambda) \omega - \frac{\vb*{k}^2}{2m} + \mu
            \right)^2
            + (\alpha \omega^2 + \beta T^2)^2
        },
    \end{aligned}
\end{equation}
where the infinitesimal imaginary part can be ignored 
since we already have a finite imaginary part.
The imaginary part gives 
\begin{equation}
    A_\sigma(\vb*{k}, \omega)
    = - \frac{1}{\pi} \frac{ \alpha \omega^2 + \beta T^2}{
        \left(
            (1 - \lambda) \omega - \frac{\vb*{k}^2}{2m} + \mu
        \right)^2
        + (\alpha \omega^2 + \beta T^2)^2
    }.
\end{equation}
When $\omega$ is already given and is a real number, 
$\vb*{k}$ should satisfy 
\begin{equation}
    (1 - \lambda) \omega - \frac{\vb*{k}^2}{2m} + \mu = 0 \Rightarrow
    \omega = \frac{\vb*{k}^2}{2m^*} - \mu^*, \quad 
    m^* = (1 - \lambda) m 
\end{equation}
so that $A_\sigma(\vb*{k}, \omega)$ is maximized.
This is the effective dispersion relation measured by ARPES;
the effective mass is $(1 - \lambda) m$.
If $\omega$ is complex, 
the maximizing procedure will be much more tedious, 
but this isn't physical, 
since in ARPES the frequency of the input beam can't be complex.

\end{document}