\documentclass[hyperref, a4paper]{article}

\usepackage{geometry}
\usepackage{titling}
\usepackage{titlesec}
% No longer needed, since we will use enumitem package
% \usepackage{paralist}
\usepackage{enumitem}
\usepackage{footnote}
\usepackage{enumerate}
\usepackage{amsmath, amssymb, amsthm}
\usepackage{mathtools}
\usepackage{bbm}
\usepackage{graphicx}
\usepackage{subcaption}
\usepackage{physics}
\usepackage{tensor}
\usepackage{siunitx}
\usepackage[version=4]{mhchem}
\usepackage{tikz}
\usepackage{xcolor}
\usepackage{listings}
\usepackage{autobreak}
\usepackage[ruled, vlined, linesnumbered]{algorithm2e}
\usepackage{nameref,zref-xr}
\zxrsetup{toltxlabel}
\usepackage[sorting=none]{biblatex}
\addbibresource{experiments.bib}
\usepackage[colorlinks,unicode]{hyperref} % , linkcolor=black, anchorcolor=black, citecolor=black, urlcolor=black, filecolor=black
\usepackage[most]{tcolorbox}
\usepackage{prettyref}

% Page style
\geometry{left=3.18cm,right=3.18cm,top=2.54cm,bottom=2.54cm}
\titlespacing{\paragraph}{0pt}{1pt}{10pt}[20pt]
\setlength{\droptitle}{-5em}

% More compact lists 
\setlist[itemize]{
    itemindent=17pt, 
    leftmargin=1pt,
    listparindent=\parindent,
    parsep=0pt,
}

% Math operators
\DeclareMathOperator{\timeorder}{\mathcal{T}}
\DeclareMathOperator{\diag}{diag}
\DeclareMathOperator{\legpoly}{P}
\DeclareMathOperator{\primevalue}{P}
\DeclareMathOperator{\sgn}{sgn}
\DeclareMathOperator{\res}{Res}
\newcommand*{\ii}{\mathrm{i}}
\newcommand*{\ee}{\mathrm{e}}
\newcommand*{\const}{\mathrm{const}}
\newcommand*{\suchthat}{\quad \text{s.t.} \quad}
\newcommand*{\argmin}{\arg\min}
\newcommand*{\argmax}{\arg\max}
\newcommand*{\normalorder}[1]{: #1 :}
\newcommand*{\pair}[1]{\langle #1 \rangle}
\newcommand*{\fd}[1]{\mathcal{D} #1}
\DeclareMathOperator{\bigO}{\mathcal{O}}

% TikZ setting
\usetikzlibrary{arrows,shapes,positioning}
\usetikzlibrary{arrows.meta}
\usetikzlibrary{decorations.markings}
\tikzstyle arrowstyle=[scale=1]
\tikzstyle directed=[postaction={decorate,decoration={markings,
    mark=at position .5 with {\arrow[arrowstyle]{stealth}}}}]
\tikzstyle ray=[directed, thick]
\tikzstyle dot=[anchor=base,fill,circle,inner sep=1pt]

% Algorithm setting
% Julia-style code
\SetKwIF{If}{ElseIf}{Else}{if}{}{elseif}{else}{end}
\SetKwFor{For}{for}{}{end}
\SetKwFor{While}{while}{}{end}
\SetKwProg{Function}{function}{}{end}
\SetArgSty{textnormal}

\newcommand*{\concept}[1]{{\textbf{#1}}}

% Embedded codes
\lstset{basicstyle=\ttfamily,
  showstringspaces=false,
  commentstyle=\color{gray},
  keywordstyle=\color{blue}
}

% Reference formatting
\newrefformat{fig}{Fig.~\ref{#1}}
\newcommand*{\term}[1]{\textit{#1}}

% Color boxes
\tcbuselibrary{skins, breakable, theorems}

\newtcbtheorem{infobox}{Box}{
    enhanced,
    boxrule=0pt,
    colback=blue!5,
    colframe=blue!5,
    coltitle=blue!50,
    borderline west={4pt}{0pt}{blue!65},
    sharp corners,
    fonttitle=\bfseries, 
    breakable,
    before upper={\parindent15pt\noindent}}{box}
\newtcbtheorem[use counter from=infobox]{theorybox}{Box}{
    enhanced,
    boxrule=0pt,
    colback=orange!5, 
    colframe=orange!5, 
    coltitle=orange!50,
    borderline west={4pt}{0pt}{orange!65},
    sharp corners,
    fonttitle=\bfseries, 
    breakable,
    before upper={\parindent15pt\noindent}}{box}
\newtcbtheorem[use counter from=infobox]{learnbox}{Box}{
    enhanced,
    boxrule=0pt,
    colback=green!5,
    colframe=green!5,
    coltitle=green!50,
    borderline west={4pt}{0pt}{green!65},
    sharp corners,
    fonttitle=\bfseries, 
    breakable,
    before upper={\parindent15pt\noindent}}{box}


\newenvironment{shelldisplay}{\begin{lstlisting}}{\end{lstlisting}}

\newcommand*{\kB}{k_{\text{B}}}
\newcommand*{\muB}{\mu_{\text{B}}}
\newcommand*{\efermi}{E_{\text{F}}}
\newcommand*{\pfermi}{p_{\text{F}}}
\newcommand*{\vfermi}{v_{\text{F}}}
\newcommand*{\sA}{\text{A}}
\newcommand*{\sB}{\text{B}}
\newcommand*{\Tc}{T_{\text{c}}}
\newcommand*{\hethree}{$^3$He}
\newcommand*{\hefour}{$^4$He}

\title{Superconductivity}
\author{Jinyuan Wu}

\begin{document}

\maketitle

\section{Phenomenology}

Superconductivity is something discovered \emph{before} 
modern day solid state physics.
The motivation was like this:
if you cool down a metal towards $T \to 0$,
what will happen?
If the temperature is low enough, 
it seems the kinetic energy of electrons will be quenched 
and electrons will stall -- 
not necessarily, since we have a Fermi sea 
and there are always electrons with non-zero kinetic energy 
(but people at the end of 19th century of course couldn't have realize this).
But if the temperature is low enough, 
it seems scattering will also be weak, 
so then the resistivity will go to zero, 
although Ohm's law still works.

None of these claims proved to be true, however.
On 4/8/1911, mercury was cooled to a low temperature,
and resistivity \emph{suddenly} drops to zero.
Almost permanent current can be established,
which costs thousands of years to be damped.
But this permanent current results in no permanent magnet:
superconductors have perfect diamagnetism.
This is a strong indication that superconductors are not perfect metals:
if we first apply a magnetic field to a metal 
and then make it a perfect metal and then 
turn off the external magnetic field, 
there will be an internal magnetic field confined in the metal,
because permanent currents are established inside.
This doesn't happen for a superconductor:
it \emph{always} repels magnetic field.
This immediately leads to a phenomenon similar to the skin effect seen in AC current:
since we don't expect magnetic field within a superconducting cylinder,
we don't expect any current density within it.

Of course, if we increase the external magnetic field, 
it will try to sneak into the superconductor,
so when the magnetic field is strong enough 
we should expect the superconducting phase to be destroyed.

Another exotic phenomenon is magnetic vortices,
whose characteristic length scale 
ranging from several nanometers to several micrometers.
A magnetic vortex contains a small piece of normal electron liquid at its center 
and a superconducting current flowing around it.
Magnetic vortices appear when we apply a magnetic field that is strong enough 
to drive the superconductor away from the ideally diamagnetic Meissner phase, 
and not strong enough to kill the superconducting phase.
They can form vortex lattices,
and if we increase the temperature, 
they form vortex liquid, 
and finally when the temperature is too high, 
proliferation of magnetic vortices kills the superconducting phase.

In 1933, the first successful phenomenological model -- 
the Landau equations -- was proposed by the London brothers.
The basic idea is to reject the Ohm's law for superconducting current;
instead, 
\begin{equation}
    \pdv{\vb*{j}}{t} = \frac{n_\text{s} e^2}{m} \vb*{E}
\end{equation}
is used as the response of a superconductor.
This correctly predicts the absence of magnetic field in the superconductor.

In 1938 another exotic phenomenon was discovered: superfluidity.
A superfluid has no viscosity, 
and therefore it has unlimited capillary effect.
If we store it to a cup without a cover, 
it can climb up and flow out of the cup.
It can flow through very tiny holes that 
normal liquid can't go through.

In 1951 a huge discovery was made: the Isotope Effect.
It was found that the change to the critical temperature 
can be solely attributed to the mass of atoms.
This means superconductivity is strongly linked to phonon effects.
Froehlich and Huang showed that 
electron-phonon coupling results in an effective attracting force between two electrons.
Then in 1955, Bardeen and Pines showed that 
over-screening from phonons can overcome the much stronger Coulomb interaction.
Then Cooper, in 1966, demonstrated 
that electrons can be paired even with infinitesimal attraction between electrons.
In 1957, finally, Schrieffer gave an explicit many-body wave function ansatz 
of the superconducting ground state.
All of these derivations are unified using Green function methods by Nambu and Gorkov, 
and Abrikosov and Migdal extended the theory to intermediate coupling
(so the theory works for real-world materials);
actually before that, Bogoliubov had already developed 
essentially the same results but this discovery was not immediately known 
because of the Iron Curtain.
These discoveries are collectively named the BCS theory.

If we look at the specific heat of a superconductor 
and compare it with the specific heat of a normal metal, 
we will find the former decays exponentially when $T$ goes to zero.
This means superconductors are gapped: 
when the temperature goes to zero, 
all degrees of freedom are frozen.

We can build a superconductor-insulator-superconductor junction.
The contribution of single electrons to the current 
is the same as the insulator-insulator-insulator case, 
and the voltage above which there is current is $2 \Delta / e$,
where $\Delta$ is the aforementioned energy gap.
But at the middle of the $I$-$V$ curve, 
we have a contribution caused by tunneling of Cooper pairs, 
not gapped electrons and holes: 

After BCS people used to think that superconductivity was completely explained.
But then in 1986, \ce{BaLaCuO} -- 
an oxide that is essentially an \emph{insulator} in its normal state -- 
was found to be a superconductor.
Then a group of oxides with similar structures were found to be superconductors,
with $T_{\text{c}}$ being as high as \SI{120}{K}.
Unfortunately the expected large scale industrial usage of these materials didn't happen,
since these materials are not metals and are hard to shape, 
they are highly anisotropic 

\section{London equations}

We know the Ohm's law can be derived from the Drude model: 
we have 
\begin{equation}
    \expval{\vb*{v}} = \frac{e \vb*{E} \tau}{m}.
\end{equation}
The appearance of $\tau$ in this expression tells us 
it comes from electron collision;
and indeed the Ohm's law leads to dissipation in Maxwell equations.

The Drude model assumes constant equilibrium of electron motion.
In a superconductor we assume that there is no collision 
of whatever carries the charge current,
and now the current response is 
\begin{equation}
    \dv{\vb*{p}}{t} = - e \vb*{E} \Rightarrow 
    \pdv{\vb*{j}}{t} = \frac{n e^2}{m} \vb*{E}. 
    \label{eq:current-response-sc}
\end{equation}
Now the question is why the superconducting carriers 
can't be accelerated infinitely;
the answer is in an equilibrium superconductor,
there is no electric field.

To avoid ambiguity we use subscript $_{\text{s}}$ to represent 
quantities related to the superconducting current.
\eqref{eq:current-response-sc} -- \concept{London's first equation} -- means 
\begin{equation}
    \pdv{t} \curl{\vb*{j}} = \frac{n_\text{s} e^2}{m} \curl{\vb*{E}}
    = - \frac{n_\text{s} e^2}{m} \pdv{\vb*{B}}{t}
    \Rightarrow \partial_t(\curl{\vb*{j}} + \frac{n_\text{s} e^2}{m} \vb*{B}) = 0,
\end{equation}
and thus we have 
\begin{equation}
    \curl{\vb*{j}} + \frac{n_\text{s} e^2}{m} \vb*{B} = \text{something depending on $\vb*{x}$ only}.
\end{equation}
Using London's first equation that's all we can get. 
But then we can \emph{manually} impose a stronger equation: 
we assume \concept{London's second equation}
\begin{equation}
    \curl{\vb*{j}} + \frac{n_\text{s} e^2}{m} \vb*{B} = 0.
\end{equation}
This gives the spatial dependence of the superconducting current $\vb*{j}$.
Then, using 
\begin{equation}
    \curl{\vb*{B}} = \mu_0 \vb*{j},
\end{equation}
we find 
\[
    \curl{(\curl{\vb*{B}})} = \grad{(\div{\vb*{B}})} - \laplacian \vb*{B}
    = \mu_0 \curl{\vb*{j}} = - \mu_0 \frac{n_\text{s} e^2}{m} \vb*{B} ,
\]
\begin{equation}
    \laplacian \vb*{B} = \mu_0 \frac{n_\text{s} e^2}{m} \vb*{B}.
\end{equation}
Astonishing, we have already get rid of time dependence 
in an equation governing $\vb*{B}$.
(This doesn't mean $\vb*{B}$ can't change of course.)

We consider the case in which the superconductor has a rotational symmetry 
and the equation gives us 
\begin{equation}
    \vb*{B}(r) = \vb*{B}_0 \ee^{- r / \lambda_{\text{L}}}, \quad 
    \lambda_{\text{L}} = \sqrt{\frac{m}{\mu_0 n_{\text{s}} e^2}}.
\end{equation}
we find the magnetic field decays exponentially at the boundary. 
The length scale $\lambda_{\text{L}}$ is called \concept{London penetration depth}.
We can see the more superconducting carriers are, 
the shorter the penetration depth is, 
and the stronger the diamagnetic effect is.
Outside of the superconductor, 
the magnetic field is more concentrated, 
since magnetic field lines don't break 
and thus the superconductor squeezes magnetic field lines around it,
giving rise to a stronger magnetic field.
Thus, although the magnetic field's not going into the superconductor 
saves energy \emph{within} the superconductor, 
it increases the energy \emph{outside} the superconductor.
So if we have a very strong magnetic field 
or a very large but thin superconductor, 
the magnetic field will still go into the superconductor
to save energy (or otherwise the magnetic field concentration near the boundary is too strong);
the superconducting state is then killed in the equilibrium.

Another effect that can be found using London equations is
the skin current effect.
We can immediately find 
\begin{equation}
    \vb*{j} = \frac{1}{\mu_0} \curl{\vb*{B}}
    = - \frac{1}{\lambda_{\text{L}}} \abs{\vb*{B}_0} \vb*{B}_0 \times \vb*{r} \ee^{- r / \lambda_{\text{L}}}.
\end{equation}
Somehow strangely, we get equations that look like equations governing AC 
but we are working with DC. 

The superconducting carrier concentration $n_{\text{s}}$
has temperature dependence;
now we know superconducting carriers are just Cooper pairs, 
and when the temperature is high, 
they are broken into ordinary electrons,
and therefore the lower the temperature is, 
the stronger the skin effect is.

There is also an electromagnetic gauge called \concept{London gauge}.
We have 
\begin{equation}
    \partial_t \vb*{j} = \frac{1}{\lambda_{\text{L}}^2 \mu_0} \vb*{E}, \quad 
    \curl{\vb*{j}} = - \frac{1}{\lambda_{\text{L}}^2 \mu_0} \vb*{B}.
\end{equation}
Recalling that 
\begin{equation}
    \vb*{E} = - \grad{\phi} - \pdv{\vb*{A}}{t}, \quad 
    \vb*{B} = \curl{\vb*{B}},
\end{equation}
we have 
\begin{equation}
    \partial_t \left(
        \vb*{j} + \frac{1}{\lambda_{\text{L}}^2 \mu_0} \vb*{A}
    \right) = \curl{
        \left(
        \vb*{j} + \frac{1}{\lambda_{\text{L}}^2 \mu_0} \vb*{A}
        \right)
    } = 0.
\end{equation}
This tempts us to choose the following gauge 
\begin{equation}
    \vb*{j} + \frac{1}{\lambda_{\text{L}}^2 \mu_0} \vb*{A} = 0,
\end{equation}
from which we find 
\begin{equation}
    \div{\vb*{A}} = 0, \quad A_{\bot} = 0.
\end{equation}

\section{Ginzburg-Landau theory}

Ginzburg-Landau theory is a theory based on Landau's (general) theory of phase transition.
We can first look at a theory with no electromagnetic coupling,
which describes superfluid as well. 
we introduce a complex order parameter $\psi_{\text{s}}$ 
(sometimes called the macroscopic wave function 
since it can be used to find some macroscopic physical quantities),
and we have 
\begin{equation}
    n = \frac{1}{V} \int \dd[3]{\vb*{r}} \abs{\psi_{\text{s}}(\vb*{r})}^2,
\end{equation}
and to extract information about inhomogeneity of $n_{\text{s}}$, 
$\abs{\grad{\psi_{\text{s}}}}^2$ is also needed.
The free energy is therefore 
\begin{equation}
    F[\psi] = \int \dd[3]{\vb*{r}} \left(
        \alpha \abs{\psi}^2 + \frac{1}{2} \beta \abs*{\psi}^4 + \gamma \abs*{\grad{\psi}}^2
    \right).
\end{equation}
The minimum of this free energy can be obtained by taking the functional variation:
\begin{equation}
    \alpha \psi_0 + \beta \abs{\psi_0}^2 \psi_0 - \gamma \laplacian{\psi_0} = 0.
    \label{eq:gl-min}
\end{equation}
This equation is a non-linear Schrodinger equation.
We consider a boundary at $x = 0$;
in this case \eqref{eq:gl-min} is reduced to its one-dimensional version,
and the boundary conditions are 
\begin{equation}
    \psi(x = 0) = 0, \quad 
    \psi(x \to \infty) = \const.
\end{equation}
Since in the $x \to \infty$ limit the order parameter is a constant, 
from \eqref{eq:gl-min} we get 
\begin{equation}
    \psi(x \to \infty) = \sqrt{- \frac{\alpha}{\beta}}.
\end{equation}
The solution then is 
\begin{equation}
    \psi(x) = \sqrt{- \frac{\alpha}{\beta}} f(x), \quad 
    f(x) = \tanh \frac{x}{\sqrt{2} \xi}, \quad 
    \xi = \sqrt{- \frac{\gamma}{\alpha}}.
    \label{eq:gl-boundary}
\end{equation}
The length scale $\xi$ is named the \concept{coherent length}.

We can then evaluate the free energy of a boundary 
from \eqref{eq:gl-boundary};
if a boundary costs energy, 
then the length of all boundaries will be minimized.

Now we consider a charged liquid with electromagnetic coupling.
The coupling can be introduced easily using minimal coupling, 
in which we replace $\grad$ with $\grad - \ii q \vb*{A}$;
note that whether this is the \emph{correct} coupling 
can't be justified within the macroscopic theory;
it's from the fact that the free energy can be obtained by 
integrating out non-zero Matsubara frequency components 
in the Matsubara version of condensed matter field theory
where $\grad \to \grad - \ii q \vb*{A}$ 
is justified by taking the non-relativistic limit 
that we know the minimal coupling is still correct 
in Ginzburg-Landau free energy theory,
and, in the case of BCS superconductivity,
the fact that $q = - 2 e$ because $\psi$ is the field corresponding to Cooper pairs.
The new free energy in a static magnetic field is therefore 
\begin{equation}
    F[\psi, \vb*{A}] = \int \dd[3]{\vb*{r}}
    \left(
        \alpha \abs*{\psi}^2 + \frac{1}{2} \beta \abs*{\psi}^4 
        + \frac{1}{2 m^*} (\grad - \ii q \vb*{A})
        + \frac{\vb*{B}^2}{2 \mu_0}
    \right)^2.
\end{equation}
We have ignored the electric field part.
Repeating the variation calculation, we find 
\begin{equation}
    \frac{1}{2m^*} (\grad - \ii q \vb*{A})^2 \psi + \alpha \psi + \beta \abs*{\psi}^2 = 0, 
\end{equation}
and 
\begin{equation}
    \curl{\vb*{B}} = \underbrace{
        - \frac{q^2}{m^*} \abs*{\psi}^2 \vb*{A} 
        + \frac{\ii q}{2 m^*} (\psi \grad \psi^* - \psi^* \grad \psi)
    }_{\mu_0 \vb*{j}}.
    \label{eq:b-and-j}
\end{equation}
The second equation tells us an interesting fact of superconductivity:
we can have non-vanishing current with a uniform Cooper concentration.
If $\abs*{\psi}$ is uniform in space, 
since 
\begin{equation}
    \psi = \abs*{\psi} \ee^{\ii \chi},
\end{equation}
we have 
\begin{equation}
    \psi \grad{\psi^*} - \psi^* \grad{\psi}
    = - 2 \ii \abs*{\psi_0}^2 \grad{\chi},
\end{equation}
and thus 
\begin{equation}
    \mu_0 \vb*{j} = \frac{q}{m^*} \abs*{\psi_0}^2 \grad{\chi} - \frac{q^2}{m^*} \abs*{\psi_0}^2 \vb*{A}.
\end{equation}
If we assume phase homogeneity, 
then we just get 
\begin{equation}
    \mu_0 \vb*{j} = - \frac{q^2}{m^*} \underbrace{\abs*{\psi_0}^2}_{n_{\text{s}}} \vb*{A},
\end{equation}
the curl of which is London's second equation,
and the time derivative of which is London's first equation.

The phenomenological parameters $m^*$, $\alpha$ and $\beta$
can be decided by the two experimentally observed quantities, 
the Cooper pair concentration 
\begin{equation}
    \abs*{\psi_0}^2 = - \frac{\alpha}{\beta},
\end{equation}
the penetration depth 
\begin{equation}
    \lambda = \sqrt{\frac{- \beta m^*}{2 \alpha q^2}},
\end{equation}
and the coherent length 
\begin{equation}
    \xi = \sqrt{- \frac{1}{2 m^* \alpha}}.
\end{equation}
If we have 
\begin{equation}
    \alpha \propto T - \Tc,
\end{equation}
then $\lambda, \xi \propto (\Tc - T)^{-1/2}$.

The \concept{Ginzburg parameter} is defined as 
\begin{equation}
    \kappa = \frac{\lambda}{\xi} = \frac{m}{e} \sqrt{\beta}.
\end{equation}
When $\kappa$ is small -- the vanilla case -- 
the superconductor just expels magnetic field 
as is discussed above.
But it's possible to have a much larger $\kappa \gg 1$, 
and in this case, 
the magnetic concentration length is already comparable 
with the size of the superconductor region, 
and therefore although the magnetic field is still expelled 
within the superconducting phase, 
since the domain wall energy with the presence of a magnetic field is \emph{negative}, 
lots of magnetic vortices -- 
at the center of which is a beam of magnetic field lines 
and a little normal electron fluid,
surrounded by a superconducting carrier flow -- 
will be generated within the superconducting phase.

\section{Magnetic flux}

Recalling \eqref{eq:b-and-j}, we have 
\begin{equation}
    \vb*{A} = \frac{m^*}{q^2 \abs*{\psi_0}^2} \left(
        \frac{\ii q}{2 m^*} (\psi \grad{\psi^*} - \psi^* \grad{\psi})
        - \mu_0 \vb*{j}
    \right), 
\end{equation}
and the magnetic flux is 
\begin{equation}
    \Phi = \oint_{\partial{S}} \dd{\vb*{k}} \cdot \vb*{A}
    = \oint_{\partial S} \dd{\vb*{l}} \cdot \left(
        - \frac{m}{e^2 n_{\text{s}}} \vb*{j} + \frac{1}{-2e} \grad{\chi}
    \right),
\end{equation}
where we have already used the conditions that links $\psi$ and electron:
\begin{equation}
    m^* = 2 m, \quad q = - 2 e,
\end{equation}
ignoring interaction-induced mass renormalization.
The phase self-consistent condition dictates 
\begin{equation}
    \oint_{\partial S} \dd{\vb*{l}} \cdot \grad{\chi} = 2 \pi n,
\end{equation}
and if $n \neq 0$,
there is a singularity of $\chi$ inside $S$.
We thus have
\begin{equation}
    \Phi = \Phi' + \frac{m}{e} \oint_{\partial S} \dd{\vb*{l}} \cdot \vb*{v}_{\text{s}}, \quad 
    \Phi' = n \Phi_0 = n \cdot \frac{2 \pi}{2e},
\end{equation}
or if we have $\hbar$ back, 
\begin{equation}
    \Phi_0 = \frac{h}{2e}.
\end{equation}

\section{BCS}

\subsection{The two-body problem}

Consider two electrons from the Fermi sea.
At this stage, we don't bother to do a many-body calculation.
The two-body Hamiltonian is 
\begin{equation}
    H = - \frac{\hbar^2}{2m} \laplacian_1 - \frac{\hbar^2}{2m} \laplacian_2 
    + V(\vb*{r}_1 - \vb*{r}_2),
\end{equation}
and after switching to the COM coordinates, 
we have 
\begin{equation}
    - \frac{\hbar^2}{2m^*} \laplacian_{\vb*{r}} \psi(\vb*{r}) + V(\vb*{r}) \psi(\vb*{r}) = \tilde{E} \psi,
\end{equation}
where 
\begin{equation}
    E = \tilde{E} + \text{K.E. of center of mass}, \quad 
    \vb*{r} = \vb*{r}_1 - \vb*{r}_2, \quad
    m^* = \frac{m}{2}.
\end{equation}
In Fourier space we have
\begin{equation}
    \int \frac{\dd[3]{\vb*{k}}}{(2\pi)^3} V(\vb*{k} - \vb*{k}') \psi_{\vb*{k}'} = 
    (\tilde{E} - 2 \varepsilon_{\vb*{k}}) \psi_{\vb*{k}}.
    \label{eq:momentum-bcs-two-body}
\end{equation}
What we want to study is when bound states can be formed.
Cooper demonstrated that a good choice is 
\begin{equation}
    V(\vb*{k}, \vb*{k}') = \begin{cases}
        - V_0, &\quad
        \varepsilon_{\text{F}} < \frac{\hbar^2 \vb*{k}^2}{2m}, 
        \quad \frac{\hbar^2 \vb*{k}'^2}{2m} < \varepsilon_{\text{F}} + \hbar \omega_{\text{D}}, \\
        0, &\quad \text{otherwise}.
    \end{cases}
\end{equation}
This potential is motivated by the effective attraction force induced by phonons 
in the Frohlich model.
\eqref{eq:momentum-bcs-two-body}
can then be recast into (TODO)
\begin{equation}
    V_0 \int_{\varepsilon_{\text{F}}}^{\varepsilon_{\text{F}} + \hbar \omega_{\text{D}}}
    \dd{\varepsilon} \rho(\varepsilon) 
    \frac{1}{\tilde{E} - 2 \varepsilon} = 2,
\end{equation}
and when $\omega_{\text{D}}$ isn't large -- which is usually the case -- 
$g(\varepsilon)$ is approximately $g(\varepsilon_{\text{F}})$,
and we get 
\begin{equation}
    \tilde{E} = \varepsilon_{\text{F}} - 2 \hbar \omega_{\text{D}} \ee^{- \frac{2}{V_0 \rho(\varepsilon_{\text{F}})}}.
\end{equation}
This is a remarkable result: 
when $V_{0}$ is \emph{large}, 
we can perturbatively find $\tilde{E}$,
but for a \emph{weak} attraction,
the change of the ground state energy is \emph{not} 
perturbatively solvable.

The question, then, is where Coulomb interaction goes.
The short answer is it's screened.
In a typical metal, 
the length scale of the screened Coulomb interaction is small.
But then there is another doubt: 
if Coulomb interaction is screened so severely, 
that is electron-phonon interaction -- essentially a type of Coulomb interaction 
-- also screened?
Fortunately, if we evaluate everything rigorously, 
the \emph{total} interaction potential is something like the follow:
\begin{equation}
    V_{\vb*{q}}(\omega) = \frac{e^2}{\vb*{q}^2 + q_{\text{TF}}^2} \frac{1}{1 - \frac{\omega^2}{\omega_{\text{D}}^2}},
\end{equation}
where $\omega$ is the energy passing through the interaction line,
and therefore 
\begin{equation}
    \hbar \omega = \varepsilon_{\vb*{k}} - \epsilon_{\vb*{k}'}.
\end{equation} 
So we can see as long as 
\begin{equation}
    \abs*{\varepsilon_{\vb*{k}} - \varepsilon_{\vb*{k}'}}  < \hbar \omega_{\text{D}},
\end{equation}
we have an attractive potential.
This is known as \concept{overscreening}.

\subsection{The many-body ground state}

The next question is when electron pairing indeed happens,
what will happen to the many-body wave function.
An ansatz -- equivalent to the mean-field treatment of the attractive potential -- is 
introduced in this section.
First we consider the most likely pairs;
since we want the energy difference between two electrons to be as small as possible, 
a good idea is to just choose $\vb*{k} = - \vb*{k}'$
(we don't want $\vb*{k} = \vb*{k}'$ 
or otherwise Pauli exclusion may appear),
so the pair creation operator that is 
most salient in the ground state is 
\begin{equation}
    P^\dagger_{\vb*{k}} = c^\dagger_{\vb*{k} \uparrow} c^\dagger_{- \vb*{k} \downarrow}.
\end{equation}
Following the idea of superfluid, 
in the ground state the Cooper pairs should ``condense'',
so we write 
\begin{equation}
    \ket*{\Psi_{\text{BCS}}} = \frac{1}{\mathcal{N}} \prod_{\vb*{k}} (1 + c_{\vb*{k}} P_{\vb*{k}}^\dagger) \ket*{0}.
\end{equation}
To make normalization easier, 
we can distribute the normalization requirement into the following form:
\begin{equation}
    \ket*{\Psi_{\text{BCS}}} = \prod_{\vb*{k}}
    (u^*_{\vb*{k}} + v^*_{\vb*{k}}) P^\dagger_{\vb*{k}} \ket*{0},
\end{equation}
and 
\begin{equation}
    \abs*{u_{\vb*{k}}}^2 + \abs*{v_{\vb*{k}}}^2 = 1.
\end{equation}
When deriving the normalization condition,
we need to notice that 
\begin{equation}
    (P_{\vb*{k}}^\dagger)^2 = 0.
\end{equation}

Now we minimize the energy to get the (mean-field) ground state. 
The kinetic part is 
\begin{equation}
    \begin{aligned}
        \expval{\text{K.E.}} &= \sum_{\vb*{k}, \sigma} 
        \expval{\varepsilon_{\vb*{k}} c^\dagger_{\vb*{k} \sigma} c_{\vb*{k} \sigma}}{\Psi_{\text{BCS}}} \\
        &= 2 \sum_{\vb*{k}} \varepsilon_{\vb*{k}} \abs*{v_{\vb*{k}}}^2,
    \end{aligned}
    \label{eq:sc-ke}
\end{equation}
and the interaction potential part is 
\begin{equation}
    \begin{aligned}
        \expval{\text{P.E.}} &= \sum_{\vb*{k}, \vb*{k}'} 
        \expval{
            V_{\vb*{k} \vb*{k}'} c^\dagger_{\vb*{k} \uparrow}
            c^\dagger_{- \vb*{k} \downarrow}
            c_{- \vb*{k} \downarrow}
            c_{\vb*{k} \uparrow}
        }{\Psi_{\text{BCS}}} \\
        &= - \frac{1}{N} \sum_{\vb*{k}} V_0 v^*_{\vb*{k}} u_{\vb*{k}} v_{\vb*{k}'} u_{\vb*{k}'}^*,
    \end{aligned}
\end{equation}
so the optimization problem is 
\begin{equation}
    \min \expval{\text{K.E.}} + \expval{\text{P.E.}} - \mu \expval{N}, 
    \quad \text{s.t. $\abs*{u_{\vb*{k}}}^2 + \abs*{v_{\vb*{k}}}^2 = 1$}.
\end{equation}
By Lagrangian multiplier method, 
we get the following equation system: 
\begin{equation}
    \begin{aligned}
        &\Delta = V_0 \sum_{\vb*{k}} u_{\vb*{k}} v^*_{\vb*{k}}, \quad 
        \Delta^* = V_0 \sum_{\vb*{k}} u_{\vb*{k}}^* v_{\vb*{k}}, \quad \\
        &(\varepsilon_{\vb*{k}} - \mu) u^*_{\vb*{k}} + \Delta v^*_{\vb*{k}} = \lambda_{\vb*{k}} u_{\vb*{k}}, \quad 
        \Delta^* u^*_{\vb*{k}} - (\varepsilon_{\vb*{k}} - \mu) v^*_{\vb*{k}} = \lambda_{\vb*{k}} v^*_{\vb*{k}}.
    \end{aligned}
\end{equation}
From the equations we find 
\begin{equation}
    \lambda_{\vb*{k}} = \pm \sqrt{
        (\varepsilon_{\vb*{k}} - \mu)^2 + \abs*{\Delta}^2
    }, \quad 
    \abs*{u_{\vb*{k}}}^2, \abs*{v_{\vb*{k}}}^2 = 
    \frac{1}{2} \left(
        1 \pm \frac{\epsilon_{\vb*{k}} - \mu}{2}
    \right), \quad 
    u_{\vb*{k}} v^*_{\vb*{k}} = \frac{\Delta}{2 \lambda_{\vb*{k}}},
\end{equation}
and since 
\begin{equation}
    \expval{P^\dagger_{\vb*{k}}} = u_{\vb*{k}} v^*_{\vb*{k}},
\end{equation}
we are able to find two expressions of $\Delta$, and 
\begin{equation}
    \Delta = V_0 \sum_{\vb*{k}} \frac{\Delta}{2 \lambda_{\vb*{k}}}
    = V_0 \int_{\varepsilon_{\text{F}}}^{\varepsilon_{\text{F}} + \hbar \omega_{\text{D}}} 
    \rho(\varepsilon_{\text{F}}) \frac{\dd{\varepsilon}}{2 \sqrt{(\epsilon - \mu)^2 + \abs*{\Delta}^2} }.
\end{equation}
Assuming $\omega_{\text{D}}$ is very small compared with 
energy scales for electrons again, 
we get 
\begin{equation}
    \abs*{\Delta} = 2 \hbar \omega_{\text{D}} \ee^{- 1 / V_0 \rho(\varepsilon_{\text{F}})}.
\end{equation}
Of course we need to first get rid of the trivial solution $\Delta = 0$.

The next step is to investigate into the meaning of $\Delta$.
we need to directly compare the superconducting energy 
and the normal metal energy.
We can expect that the total kinetic energy in the superconducting phase 
to be higher than the normal metal energy,
since we can verify that in the superconducting phase, 
the momentum distribution of electrons is wider.
But on the other hand the attractive potential is negative.
The final result is 
\begin{equation}
    \Delta E = 2 \sum_{\abs*{\vb*{k}} < k_{\text{F}}} \left(
        \varepsilon_{\vb*{k}} - \frac{\varepsilon_{\vb*{k}}^2}{\sqrt{\varepsilon_{\vb*{k}}^2 + \Delta^2}}
    \right) - \frac{\Delta^2}{V_0}
    \approx - \frac{1}{2} \rho(\varepsilon_{\text{F}}) \Delta^2 ,
\end{equation}
and this superconducting band gap is also known as the condensation energy.

Now we already know one important fact about the critical field $H_{\text{c}}$:
it should be proportional to $\Delta$ at least when $T = 0$.

\subsection{Excited states}

Now we move to the excited states of the BCS superconductor.
We will be able to tell that $\Delta$ is the band gap
of fermionic modes in the superconductor,
which then will be the superconducting gap.
We do the mean-field approximation: 
we assume there is no fluctuation in $P_{\vb*{k}}$,
and the interaction part of the Hamiltonian then is approximately
\begin{equation}
    \sum_{\vb*{k}, \vb*{k}'} V_{\vb*{k} \vb*{k}'} P_{\vb*{k}'}^\dagger P_{\vb*{k}}
    \approx \sum_{\vb*{k}, \vb*{k}'} (
        \expval*{P_{\vb*{k}}} P^\dagger_{\vb*{k}'}
        + \expval*{P^\dagger_{\vb*{k}'}} P_{\vb*{k}}
        - \expval*{P_{\vb*{k}}} \expval*{P^\dagger_{\vb*{k}'}}
    ).
\end{equation}
We can then define 
\begin{equation}
    \Delta_{\vb*{k}} = - \sum_{\vb*{k}'} V_{\vb*{k} \vb*{k}'} \expval*{P_{\vb*{k}'}},
\end{equation}
and the mean-field Hamiltonian is 
\begin{equation}
    H_{\text{BCS-MF}} = \sum_{\vb*{k}, \sigma} \varepsilon_{\vb*{k}} c^\dagger_{\vb*{k} \sigma} c_{\vb*{k} \sigma}
    - \sum_{\vb*{k}} \Delta_{\vb*{k}} c^\dagger_{\vb*{k} \uparrow} c^\dagger_{- \vb*{k} \downarrow}
    - \text{h.c.} + \const,
\end{equation}
which then is diagonalized by 
\begin{equation}
    \pmqty{
        \gamma_{\vb*{k} \uparrow} \\ \gamma^\dagger_{\vb*{k} \downarrow}
    } = \pmqty{
        u^*_{\vb*{k}} & - v_{\vb*{k}} \\
        v^*_{\vb*{k}} & u_{\vb*{k}}
    } \pmqty{
        c_{\vb*{k} \uparrow} \\ c^\dagger_{- \vb*{k} \downarrow}
    }
\end{equation}
into 
\begin{equation}
    H_{\text{BCS-MF}} = \sum_{\vb*{k}, \sigma} 
    E_{\vb*{k}} \gamma^\dagger_{\vb*{k} \sigma} \gamma_{\vb*{k} \sigma}, \quad
    E_{\vb*{k}} = \pm \sqrt{\varepsilon_{\vb*{k}}^2 + \abs*{\Delta_{\vb*{k}}}^2}.
\end{equation}
The $\gamma_{\vb*{k} \sigma}$ operators follow fermionic anti-commutation relations.

The plus-minus symbol in the definition of $E_{\vb*{k}}$ 
needs some clarification,
since it gives us \emph{four} bands
-- two spin-up bands, two spin-down bands -- 
instead of the original two bands.
We can find that when $\vb*{k}$ is near the Fermi momentum, 
we have 
\begin{equation}
    \abs*{u_{\vb*{k}}}^2 = \abs*{v_{\vb*{k}}}^2 = \frac{1}{2},
\end{equation}
and when $\vb*{k}$ is far from the Fermi momentum, 
one of $\abs*{u_{\vb*{k}}}^2$ and $\abs*{v_{\vb*{k}}}^2$
is one and the other is zero.
This means electron and hole modes 
are severely mixed together only near the original Fermi surface.
The weight (in the single-electron Green function) of the $+$ branches 
fades away when we are below the original Fermi surface, 
and the weight of the $-$ branches fades away 
when we are above the original Fermi surface.
So when $\Delta$ is large and the superconducting phase is stable, 
there are still two Bogoliubov bands,
corresponding to the two spins, 
and they have discontinuities at the Fermi surface, 
going from $- \abs*{E_{\vb*{k}}}$ to $+ \abs*{E_{\vb*{k}}}$.
The superconducting band gap therefore is $2 \abs*{\Delta}$.
Fig. 1 in \cite{rinott2017tuning} gives a graphic illustration of this idea.

\subsection{Finite-temperature effects}

When the temperature is not zero, we need 
\begin{equation}
    \Delta_{\vb*{k}} = - \sum_{\vb*{k}'} V_{\vb*{k} \vb*{k}'} \expval*{P_{\vb*{k}}}
    = - \sum_{\vb*{k}'} V_{\vb*{k} \vb*{k}'} 
    u_{\vb*{k}'} v_{\vb*{k}'} (1 - 2 f(\varepsilon_{\vb*{k}'})),
\end{equation}
and we again assume the attractive interaction is $V_{\vb*{k} \vb*{k}'} = V_0$,
and now the self-consistent equation is 
\begin{equation}
    \frac{1}{V_0 \rho(\varepsilon_{\text{F}})}  
    = \int_{- \hbar \omega_{\text{D}}}^{ \hbar \omega_{\text{D}}} \dd{\epsilon}
    \frac{
        \tanh \beta \sqrt{\epsilon^2 + \abs*{\Delta}^2} / 2
    }{
        2 \sqrt{\epsilon^2 + \abs*{\Delta}^2}
    }, \quad \beta = \frac{1}{\kB T}.
\end{equation}
When the temperature is very high, 
the RHS will be very large,
but the LHS is a constant,
so there is no superconducting phase in the high-temperature limit.
When the temperature is low, 
we get to the case studied above 
and have a superconducting phase.
So there has to be a phase transition between the two.
We can use the small $\Delta$ limit to derive the critical temperature:
we will find 
\begin{equation}
    \frac{2 \beta_{\text{c}} \hbar \omega_{\text{D}}}{\pi} \ee^{\gamma}
    = \ee^{\frac{1}{\rho(\epsilon_{\text{F}}) V_0} },
\end{equation}
where $\gamma$ is Euler's constant.
So we will find 
\begin{equation}
    k_{\text{B}} T_\text{c} = 1.57 \Delta_0.
\end{equation}
Indeed this is just a refined version of the intuitive argument that when 
\begin{equation}
    \kB \Tc \sim 2 \Delta_0,
\end{equation}
the superconducting gap is not large enough.

\section{Superconductivity is more than BCS theory}

The flavor of BCS theory outlined above assumes 
homogeneous pairing and weak effective attractive potential.
Both assumptions fail for some materials.
The McMillan limit -- an estimation of the highest $\Tc$ available 
by phonon-induced superconductivity -- 
is derived with intermediate coupling, 
which goes beyond vanilla BCS.
Anisotropic pairing is also possible.
Finally, the above discussion is a mean-field theory;
it doesn't work when the superconducting order parameter fluctuates severely.

\section{Josephson junction}

\begin{equation}
    \psi_1 = \sqrt{n_1} \ee^{\ii \theta_1}, \quad 
    \psi_2 = \sqrt{n_2} \ee^{\ii \theta_2}.
\end{equation}
The EOMs, when the insulating layer is thick enough, are 
\begin{equation}
    \ii \hbar \partial_t \psi_1 =
    \mu_1 \psi_1, \quad 
    \ii \hbar \partial_t \psi_2 =
    \mu_2 \psi_2.
\end{equation}
Assuming that the insulating layer is thinner than the superconducting coherence length,
the superconducting order parameters on both sides have interaction and we get 
\begin{equation}
    \ii \hbar \partial_t \psi_1 =
    \mu_1 \psi_1 + \kappa \psi_2, \quad 
    \ii \hbar \partial_t \psi_2 =
    \mu_2 \psi_2 + \kappa \psi_1.
\end{equation}
We need to analyze the real and imaginary parts of the equations.
The amplitude part gives 
\begin{equation}
    \hbar \partial_t n_1 = - \hbar \partial_t n_2 = 2 \kappa \sqrt{n_1 n_2} \sin (\theta_1 - \theta_2),
\end{equation}
and 
\begin{equation}
    - \hbar \partial_t (\theta_2 - \theta_1) = \mu_2 - \mu_1.
\end{equation}
Note that the two equations have very clear physical meanings:
$\mu_2 - \mu_1$ is just the voltage across the insulating layer, 
and the first equation means the time evolution of $n_1$ and $n_2$ -- 
essentially the current -- 
are strongly influenced by the phase difference between the two superconducting sizes.
(A single phase has no meaning itself and is subject to gauge choice, 
but the phase difference has observable effects.)
We define 
\begin{equation}
    I = \pdv{n}{t}, \quad V = \frac{\mu_2 - \mu_1}{2e}, \quad 
    \delta = \theta_1 - \theta_2,
\end{equation}
and what we get is 
\begin{equation}
    I = I_0 \sin \delta, \quad 
    \partial_t \delta = 2 e \frac{V}{\hbar}
\end{equation}
The first equation -- the first Josephson relation -- 
gives the current-phase relation;
the second equation -- the second Josephson relation -- 
gives the phase-voltage relation
and means that the properties of Josephson junction 
is periodic ($\delta$ is equivalent to $\delta + 2\pi$)
even when the voltage across the insulating phase is stable;
this actually gives a definition of what a Volt is:
if we can measure $\delta$ with some interference setting 
and measure its period accurately enough, 
then after we measure $e$ and $\hbar$ accurately enough,
$V$ can be measured very accurately.

We can also measure $\delta$ from the current,
which then is determined by the magnetic field TODO 
This is called \concept{superconducting quantum inference device (SQUID)}.
\begin{equation}
    \delta_1 - \delta_2 = 2\pi \frac{\Phi}{\Phi_0}, \quad 
    \Phi_0 = \frac{h}{2e},
\end{equation}
\begin{equation}
    I_{\text{ext}} = I_{0} \sin \delta_1 + I_0 \sin \delta_2,
\end{equation}
and therefore 
\begin{equation}
    I_{\text{ext}} = 2 I_0 \sin(\delta_1 - \frac{\pi \Phi}{\Phi_0}) \cos \frac{\pi \Phi}{\Phi_0},
\end{equation}
and therefore when 
\begin{equation}
    \Phi = \frac{1}{2} \Phi_0 n ,
\end{equation}
the current measured is zero.
We can also use only one piece of superconductor to make a SQUID:
we want an anisotropic pairing superconductor,
so the 



\printbibliography

\end{document}