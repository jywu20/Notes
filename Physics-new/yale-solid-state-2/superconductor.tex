\documentclass[hyperref, a4paper]{article}

\usepackage{geometry}
\usepackage{titling}
\usepackage{titlesec}
% No longer needed, since we will use enumitem package
% \usepackage{paralist}
\usepackage{enumitem}
\usepackage{footnote}
\usepackage{enumerate}
\usepackage{amsmath, amssymb, amsthm}
\usepackage{mathtools}
\usepackage{bbm}
\usepackage{graphicx}
\usepackage{subcaption}
\usepackage{physics}
\usepackage{tensor}
\usepackage{siunitx}
\usepackage[version=4]{mhchem}
\usepackage{tikz}
\usepackage{xcolor}
\usepackage{listings}
\usepackage{autobreak}
\usepackage[ruled, vlined, linesnumbered]{algorithm2e}
\usepackage{nameref,zref-xr}
\zxrsetup{toltxlabel}
\usepackage[sorting=none]{biblatex}
\bibliography{squeezing}
\usepackage[colorlinks,unicode]{hyperref} % , linkcolor=black, anchorcolor=black, citecolor=black, urlcolor=black, filecolor=black
\usepackage[most]{tcolorbox}
\usepackage{prettyref}

% Page style
\geometry{left=3.18cm,right=3.18cm,top=2.54cm,bottom=2.54cm}
\titlespacing{\paragraph}{0pt}{1pt}{10pt}[20pt]
\setlength{\droptitle}{-5em}

% More compact lists 
\setlist[itemize]{
    itemindent=17pt, 
    leftmargin=1pt,
    listparindent=\parindent,
    parsep=0pt,
}

% Math operators
\DeclareMathOperator{\timeorder}{\mathcal{T}}
\DeclareMathOperator{\diag}{diag}
\DeclareMathOperator{\legpoly}{P}
\DeclareMathOperator{\primevalue}{P}
\DeclareMathOperator{\sgn}{sgn}
\DeclareMathOperator{\res}{Res}
\newcommand*{\ii}{\mathrm{i}}
\newcommand*{\ee}{\mathrm{e}}
\newcommand*{\const}{\mathrm{const}}
\newcommand*{\suchthat}{\quad \text{s.t.} \quad}
\newcommand*{\argmin}{\arg\min}
\newcommand*{\argmax}{\arg\max}
\newcommand*{\normalorder}[1]{: #1 :}
\newcommand*{\pair}[1]{\langle #1 \rangle}
\newcommand*{\fd}[1]{\mathcal{D} #1}
\DeclareMathOperator{\bigO}{\mathcal{O}}

% TikZ setting
\usetikzlibrary{arrows,shapes,positioning}
\usetikzlibrary{arrows.meta}
\usetikzlibrary{decorations.markings}
\tikzstyle arrowstyle=[scale=1]
\tikzstyle directed=[postaction={decorate,decoration={markings,
    mark=at position .5 with {\arrow[arrowstyle]{stealth}}}}]
\tikzstyle ray=[directed, thick]
\tikzstyle dot=[anchor=base,fill,circle,inner sep=1pt]

% Algorithm setting
% Julia-style code
\SetKwIF{If}{ElseIf}{Else}{if}{}{elseif}{else}{end}
\SetKwFor{For}{for}{}{end}
\SetKwFor{While}{while}{}{end}
\SetKwProg{Function}{function}{}{end}
\SetArgSty{textnormal}

\newcommand*{\concept}[1]{{\textbf{#1}}}

% Embedded codes
\lstset{basicstyle=\ttfamily,
  showstringspaces=false,
  commentstyle=\color{gray},
  keywordstyle=\color{blue}
}

% Reference formatting
\newrefformat{fig}{Fig.~\ref{#1}}
\newcommand*{\term}[1]{\textit{#1}}

% Color boxes
\tcbuselibrary{skins, breakable, theorems}

\newtcbtheorem{infobox}{Box}{
    enhanced,
    boxrule=0pt,
    colback=blue!5,
    colframe=blue!5,
    coltitle=blue!50,
    borderline west={4pt}{0pt}{blue!65},
    sharp corners,
    fonttitle=\bfseries, 
    breakable,
    before upper={\parindent15pt\noindent}}{box}
\newtcbtheorem[use counter from=infobox]{theorybox}{Box}{
    enhanced,
    boxrule=0pt,
    colback=orange!5, 
    colframe=orange!5, 
    coltitle=orange!50,
    borderline west={4pt}{0pt}{orange!65},
    sharp corners,
    fonttitle=\bfseries, 
    breakable,
    before upper={\parindent15pt\noindent}}{box}
\newtcbtheorem[use counter from=infobox]{learnbox}{Box}{
    enhanced,
    boxrule=0pt,
    colback=green!5,
    colframe=green!5,
    coltitle=green!50,
    borderline west={4pt}{0pt}{green!65},
    sharp corners,
    fonttitle=\bfseries, 
    breakable,
    before upper={\parindent15pt\noindent}}{box}


\newenvironment{shelldisplay}{\begin{lstlisting}}{\end{lstlisting}}

\newcommand*{\kB}{k_{\text{B}}}
\newcommand*{\muB}{\mu_{\text{B}}}
\newcommand*{\efermi}{E_{\text{F}}}
\newcommand*{\pfermi}{p_{\text{F}}}
\newcommand*{\vfermi}{v_{\text{F}}}
\newcommand*{\sA}{\text{A}}
\newcommand*{\sB}{\text{B}}
\newcommand*{\Tc}{T_{\text{c}}}
\newcommand*{\hethree}{$^3$He}
\newcommand*{\hefour}{$^4$He}

\title{Superconductivity}
\author{Jinyuan Wu}

\begin{document}

\maketitle

\section{Phenomenology}

Superconductivity is something discovered \emph{before} 
modern day solid state physics.
The motivation was like this:
if you cool down a metal towards $T \to 0$,
what will happen?
If the temperature is low enough, 
it seems the kinetic energy of electrons will be quenched 
and electrons will stall -- 
not necessarily, since we have a Fermi sea 
and there are always electrons with non-zero kinetic energy 
(but people at the end of 19th century of course couldn't have realize this).
But if the temperature is low enough, 
it seems scattering will also be weak, 
so then the resistivity will go to zero, 
although Ohm's law still works.

None of these claims proved to be true, however.
On 4/8/1911, mercury was cooled to a low temperature,
and resistivity \emph{suddenly} drops to zero.
Almost permanent current can be established,
which costs thousands of years to be damped.
But this permanent current results in no permanent magnet:
superconductors have perfect diamagnetism.
This is a strong indication that superconductors are not perfect metals:
if we first apply a magnetic field to a metal 
and then make it a perfect metal and then 
turn off the external magnetic field, 
there will be an internal magnetic field confined in the metal,
because permanent currents are established inside.
This doesn't happen for a superconductor:
it \emph{always} repels magnetic field.
This immediately leads to a phenomenon similar to the skin effect seen in AC current:
since we don't expect magnetic field within a superconducting cylinder,
we don't expect any current density within it.

Of course, if we increase the external magnetic field, 
it will try to sneak into the superconductor,
so when the magnetic field is strong enough 
we should expect the superconducting phase to be destroyed.

Another exotic phenomenon is magnetic vortices,
whose characteristic length scale 
ranging from several nanometers to several micrometers.
A magnetic vortex contains a small piece of normal electron liquid at its center 
and a superconducting current flowing around it.
Magnetic vortices appear when we apply a magnetic field that is strong enough 
to drive the superconductor away from the ideally diamagnetic Meissner phase, 
and not strong enough to kill the superconducting phase.
They can form vortex lattices,
and if we increase the temperature, 
they form vortex liquid, 
and finally when the temperature is too high, 
proliferation of magnetic vortices kills the superconducting phase.

In 1933, the first successful phenomenological model -- 
the Landau equations -- was proposed by the London brothers.
The basic idea is to reject the Ohm's law for superconducting current;
instead, 
\begin{equation}
    \pdv{\vb*{j}}{t} = \frac{n_\text{s} e^2}{m} \vb*{E}
\end{equation}
is used as the response of a superconductor.
This equation then means 
\begin{equation}
    \pdv{t} \curl{\vb*{j}} = \frac{n_\text{s} e^2}{m} \curl{\vb*{E}}
    = - \frac{n_\text{s} e^2}{m} \pdv{\vb*{B}}{t}
    \Rightarrow \curl{\vb*{j}} + \frac{n_\text{s} e^2}{m} \vb*{B} = 0,
\end{equation}
and since 
\begin{equation}
    \curl{\vb*{B}} = \mu_0 \vb*{j},
\end{equation}
we find the magnetic field decays exponentially at the boundary. 

In 1938 another exotic phenomenon was discovered: superfluidity.
A superfluid has no viscosity, 
and therefore it has unlimited capillary effect.
If we store it to a cup without a cover, 
it can climb up and flow out of the cup.
It can flow through very tiny holes that 
normal liquid can't go through.

In 1951 a huge discovery was made: the Isotope Effect.
It was found that the change to the critical temperature 
can be solely attributed to the mass of atoms.
This means superconductivity is strongly linked to phonon effects.
Froehlich and Huang showed that 
electron-phonon coupling results in an effective attracting force between two electrons.
Then in 1955, Bardeen and Pines showed that 
over-screening from phonons can overcome the much stronger Coulomb interaction.
Then Cooper, in 1966, demonstrated 
that electrons can be paired even with infinitesimal attraction between electrons.
In 1957, finally, Schrieffer gave an explicit many-body wave function ansatz 
of the superconducting ground state.
All of these derivations are unified using Green function methods by Nambu and Gorkov, 
and Abrikosov and Migdal extended the theory to intermediate coupling
(so the theory works for real-world materials);
actually before that, Bogoliubov had already developed 
essentially the same results but this discovery was not immediately known 
because of the Iron Curtain.
These discoveries are collectively named the BCS theory.

If we look at the specific heat of a superconductor 
and compare it with the specific heat of a normal metal, 
we will find the former decays exponentially when $T$ goes to zero.
This means superconductors are gapped: 
when the temperature goes to zero, 
all degrees of freedom are frozen.

We can build a superconductor-insulator-superconductor junction.
The contribution of single electrons to the current 
is the same as the insulator-insulator-insulator case, 
and the voltage above which there is current is $2 \Delta / e$,
where $\Delta$ is the aforementioned energy gap.
But at the middle of the $I$-$V$ curve, 
we have a contribution caused by tunneling of Cooper pairs, 
not gapped electrons and holes: 

\subsection{High-$T_{\text{c}}$ superconducting} 

After BCS people used to think that superconductivity was completely explained.
But then in 1986, \ce{BaLaCuO} -- 
an oxide that is essentially an \emph{insulator} in its normal state -- 
was found to be a superconductor.
Then a group of oxides with similar structures were found to be superconductors,
with $T_{\text{c}}$ being as high as \SI{120}{K}.
Unfortunately the expected large scale industrial usage of these materials didn't happen,
since these materials are not metals and are hard to shape, 
they are highly anisotropic 

\end{document}