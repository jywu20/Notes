\documentclass[hyperref, a4paper]{article}

\usepackage{geometry}
\usepackage{titling}
\usepackage{titlesec}
% No longer needed, since we will use enumitem package
% \usepackage{paralist}
\usepackage{enumitem}
\usepackage{footnote}
\usepackage{enumerate}
\usepackage{amsmath, amssymb, amsthm}
\usepackage{mathtools}
\usepackage{bbm}
\usepackage{graphicx}
\usepackage{subcaption}
\usepackage{physics}
\usepackage{tensor}
\usepackage{siunitx}
\usepackage[version=4]{mhchem}
\usepackage{tikz}
\usepackage{xcolor}
\usepackage{listings}
\usepackage{autobreak}
\usepackage[ruled, vlined, linesnumbered]{algorithm2e}
\usepackage{nameref,zref-xr}
\zxrsetup{toltxlabel}
\usepackage[sorting=none]{biblatex}
\bibliography{squeezing}
\usepackage[colorlinks,unicode]{hyperref} % , linkcolor=black, anchorcolor=black, citecolor=black, urlcolor=black, filecolor=black
\usepackage[most]{tcolorbox}
\usepackage{prettyref}

% Page style
\geometry{left=3.18cm,right=3.18cm,top=2.54cm,bottom=2.54cm}
\titlespacing{\paragraph}{0pt}{1pt}{10pt}[20pt]
\setlength{\droptitle}{-5em}

% More compact lists 
\setlist[itemize]{
    itemindent=17pt, 
    leftmargin=1pt,
    listparindent=\parindent,
    parsep=0pt,
}

% Math operators
\DeclareMathOperator{\timeorder}{\mathcal{T}}
\DeclareMathOperator{\diag}{diag}
\DeclareMathOperator{\legpoly}{P}
\DeclareMathOperator{\primevalue}{P}
\DeclareMathOperator{\sgn}{sgn}
\DeclareMathOperator{\res}{Res}
\newcommand*{\ii}{\mathrm{i}}
\newcommand*{\ee}{\mathrm{e}}
\newcommand*{\const}{\mathrm{const}}
\newcommand*{\suchthat}{\quad \text{s.t.} \quad}
\newcommand*{\argmin}{\arg\min}
\newcommand*{\argmax}{\arg\max}
\newcommand*{\normalorder}[1]{: #1 :}
\newcommand*{\pair}[1]{\langle #1 \rangle}
\newcommand*{\fd}[1]{\mathcal{D} #1}
\DeclareMathOperator{\bigO}{\mathcal{O}}

% TikZ setting
\usetikzlibrary{arrows,shapes,positioning}
\usetikzlibrary{arrows.meta}
\usetikzlibrary{decorations.markings}
\tikzstyle arrowstyle=[scale=1]
\tikzstyle directed=[postaction={decorate,decoration={markings,
    mark=at position .5 with {\arrow[arrowstyle]{stealth}}}}]
\tikzstyle ray=[directed, thick]
\tikzstyle dot=[anchor=base,fill,circle,inner sep=1pt]

% Algorithm setting
% Julia-style code
\SetKwIF{If}{ElseIf}{Else}{if}{}{elseif}{else}{end}
\SetKwFor{For}{for}{}{end}
\SetKwFor{While}{while}{}{end}
\SetKwProg{Function}{function}{}{end}
\SetArgSty{textnormal}

\newcommand*{\concept}[1]{{\textbf{#1}}}

% Embedded codes
\lstset{basicstyle=\ttfamily,
  showstringspaces=false,
  commentstyle=\color{gray},
  keywordstyle=\color{blue}
}

% Reference formatting
\newrefformat{fig}{Fig.~\ref{#1}}
\newcommand*{\term}[1]{\textit{#1}}

% Color boxes
\tcbuselibrary{skins, breakable, theorems}

\newtcbtheorem{infobox}{Box}{
    enhanced,
    boxrule=0pt,
    colback=blue!5,
    colframe=blue!5,
    coltitle=blue!50,
    borderline west={4pt}{0pt}{blue!65},
    sharp corners,
    fonttitle=\bfseries, 
    breakable,
    before upper={\parindent15pt\noindent}}{box}
\newtcbtheorem[use counter from=infobox]{theorybox}{Box}{
    enhanced,
    boxrule=0pt,
    colback=orange!5, 
    colframe=orange!5, 
    coltitle=orange!50,
    borderline west={4pt}{0pt}{orange!65},
    sharp corners,
    fonttitle=\bfseries, 
    breakable,
    before upper={\parindent15pt\noindent}}{box}
\newtcbtheorem[use counter from=infobox]{learnbox}{Box}{
    enhanced,
    boxrule=0pt,
    colback=green!5,
    colframe=green!5,
    coltitle=green!50,
    borderline west={4pt}{0pt}{green!65},
    sharp corners,
    fonttitle=\bfseries, 
    breakable,
    before upper={\parindent15pt\noindent}}{box}


\newenvironment{shelldisplay}{\begin{lstlisting}}{\end{lstlisting}}

\newcommand*{\kB}{k_{\text{B}}}
\newcommand*{\muB}{\mu_{\text{B}}}
\newcommand*{\efermi}{E_{\text{F}}}
\newcommand*{\pfermi}{p_{\text{F}}}
\newcommand*{\vfermi}{v_{\text{F}}}
\newcommand*{\sA}{\text{A}}
\newcommand*{\sB}{\text{B}}
\newcommand*{\Tc}{T_{\text{c}}}
\newcommand*{\hethree}{$^3$He}
\newcommand*{\hefour}{$^4$He}

\title{Superconductivity}
\author{Jinyuan Wu}

\begin{document}

\maketitle

\section{Phenomenology}

Superconductivity is something discovered \emph{before} 
modern day solid state physics.
The motivation was like this:
if you cool down a metal towards $T \to 0$,
what will happen?
If the temperature is low enough, 
it seems the kinetic energy of electrons will be quenched 
and electrons will stall -- 
not necessarily, since we have a Fermi sea 
and there are always electrons with non-zero kinetic energy 
(but people at the end of 19th century of course couldn't have realize this).
But if the temperature is low enough, 
it seems scattering will also be weak, 
so then the resistivity will go to zero, 
although Ohm's law still works.

None of these claims proved to be true, however.
On 4/8/1911, mercury was cooled to a low temperature,
and resistivity \emph{suddenly} drops to zero.
Almost permanent current can be established,
which costs thousands of years to be damped.
But this permanent current results in no permanent magnet:
superconductors have perfect diamagnetism.
This is a strong indication that superconductors are not perfect metals:
if we first apply a magnetic field to a metal 
and then make it a perfect metal and then 
turn off the external magnetic field, 
there will be an internal magnetic field confined in the metal,
because permanent currents are established inside.
This doesn't happen for a superconductor:
it \emph{always} repels magnetic field.
This immediately leads to a phenomenon similar to the skin effect seen in AC current:
since we don't expect magnetic field within a superconducting cylinder,
we don't expect any current density within it.

Of course, if we increase the external magnetic field, 
it will try to sneak into the superconductor,
so when the magnetic field is strong enough 
we should expect the superconducting phase to be destroyed.

Another exotic phenomenon is magnetic vortices,
whose characteristic length scale 
ranging from several nanometers to several micrometers.
A magnetic vortex contains a small piece of normal electron liquid at its center 
and a superconducting current flowing around it.
Magnetic vortices appear when we apply a magnetic field that is strong enough 
to drive the superconductor away from the ideally diamagnetic Meissner phase, 
and not strong enough to kill the superconducting phase.
They can form vortex lattices,
and if we increase the temperature, 
they form vortex liquid, 
and finally when the temperature is too high, 
proliferation of magnetic vortices kills the superconducting phase.

In 1933, the first successful phenomenological model -- 
the Landau equations -- was proposed by the London brothers.
The basic idea is to reject the Ohm's law for superconducting current;
instead, 
\begin{equation}
    \pdv{\vb*{j}}{t} = \frac{n_\text{s} e^2}{m} \vb*{E}
\end{equation}
is used as the response of a superconductor.
This correctly predicts the absence of magnetic field in the superconductor.

In 1938 another exotic phenomenon was discovered: superfluidity.
A superfluid has no viscosity, 
and therefore it has unlimited capillary effect.
If we store it to a cup without a cover, 
it can climb up and flow out of the cup.
It can flow through very tiny holes that 
normal liquid can't go through.

In 1951 a huge discovery was made: the Isotope Effect.
It was found that the change to the critical temperature 
can be solely attributed to the mass of atoms.
This means superconductivity is strongly linked to phonon effects.
Froehlich and Huang showed that 
electron-phonon coupling results in an effective attracting force between two electrons.
Then in 1955, Bardeen and Pines showed that 
over-screening from phonons can overcome the much stronger Coulomb interaction.
Then Cooper, in 1966, demonstrated 
that electrons can be paired even with infinitesimal attraction between electrons.
In 1957, finally, Schrieffer gave an explicit many-body wave function ansatz 
of the superconducting ground state.
All of these derivations are unified using Green function methods by Nambu and Gorkov, 
and Abrikosov and Migdal extended the theory to intermediate coupling
(so the theory works for real-world materials);
actually before that, Bogoliubov had already developed 
essentially the same results but this discovery was not immediately known 
because of the Iron Curtain.
These discoveries are collectively named the BCS theory.

If we look at the specific heat of a superconductor 
and compare it with the specific heat of a normal metal, 
we will find the former decays exponentially when $T$ goes to zero.
This means superconductors are gapped: 
when the temperature goes to zero, 
all degrees of freedom are frozen.

We can build a superconductor-insulator-superconductor junction.
The contribution of single electrons to the current 
is the same as the insulator-insulator-insulator case, 
and the voltage above which there is current is $2 \Delta / e$,
where $\Delta$ is the aforementioned energy gap.
But at the middle of the $I$-$V$ curve, 
we have a contribution caused by tunneling of Cooper pairs, 
not gapped electrons and holes: 

After BCS people used to think that superconductivity was completely explained.
But then in 1986, \ce{BaLaCuO} -- 
an oxide that is essentially an \emph{insulator} in its normal state -- 
was found to be a superconductor.
Then a group of oxides with similar structures were found to be superconductors,
with $T_{\text{c}}$ being as high as \SI{120}{K}.
Unfortunately the expected large scale industrial usage of these materials didn't happen,
since these materials are not metals and are hard to shape, 
they are highly anisotropic 

\section{London equations}

We know the Ohm's law can be derived from the Drude model: 
we have 
\begin{equation}
    \expval{\vb*{v}} = \frac{e \vb*{E} \tau}{m}.
\end{equation}
The appearance of $\tau$ in this expression tells us 
it comes from electron collision;
and indeed the Ohm's law leads to dissipation in Maxwell equations.

The Drude model assumes constant equilibrium of electron motion.
In a superconductor we assume that there is no collision 
of whatever carries the charge current,
and now the current response is 
\begin{equation}
    \dv{\vb*{p}}{t} = - e \vb*{E} \Rightarrow 
    \pdv{\vb*{j}}{t} = \frac{n e^2}{m} \vb*{E}. 
    \label{eq:current-response-sc}
\end{equation}
Now the question is why the superconducting carriers 
can't be accelerated infinitely;
the answer is in an equilibrium superconductor,
there is no electric field.

To avoid ambiguity we use subscript $_{\text{s}}$ to represent 
quantities related to the superconducting current.
\eqref{eq:current-response-sc} -- \concept{London's first equation} -- means 
\begin{equation}
    \pdv{t} \curl{\vb*{j}} = \frac{n_\text{s} e^2}{m} \curl{\vb*{E}}
    = - \frac{n_\text{s} e^2}{m} \pdv{\vb*{B}}{t}
    \Rightarrow \partial_t(\curl{\vb*{j}} + \frac{n_\text{s} e^2}{m} \vb*{B}) = 0,
\end{equation}
and thus we have 
\begin{equation}
    \curl{\vb*{j}} + \frac{n_\text{s} e^2}{m} \vb*{B} = \text{something depending on $\vb*{x}$ only}.
\end{equation}
Using London's first equation that's all we can get. 
But then we can \emph{manually} impose a stronger equation: 
we assume \concept{London's second equation}
\begin{equation}
    \curl{\vb*{j}} + \frac{n_\text{s} e^2}{m} \vb*{B} = 0.
\end{equation}
This gives the spatial dependence of the superconducting current $\vb*{j}$.
Then, using 
\begin{equation}
    \curl{\vb*{B}} = \mu_0 \vb*{j},
\end{equation}
we find 
\[
    \curl{(\curl{\vb*{B}})} = \grad{(\div{\vb*{B}})} - \laplacian \vb*{B}
    = \mu_0 \curl{\vb*{j}} = - \mu_0 \frac{n_\text{s} e^2}{m} \vb*{B} ,
\]
\begin{equation}
    \laplacian \vb*{B} = \mu_0 \frac{n_\text{s} e^2}{m} \vb*{B}.
\end{equation}
Astonishing, we have already get rid of time dependence 
in an equation governing $\vb*{B}$.
(This doesn't mean $\vb*{B}$ can't change of course.)

We consider the case in which the superconductor has a rotational symmetry 
and the equation gives us 
\begin{equation}
    \vb*{B}(r) = \vb*{B}_0 \ee^{- r / \lambda_{\text{L}}}, \quad 
    \lambda_{\text{L}} = \sqrt{\frac{m}{\mu_0 n_{\text{s}} e^2}}.
\end{equation}
we find the magnetic field decays exponentially at the boundary. 
The length scale $\lambda_{\text{L}}$ is called \concept{London penetration depth}.
We can see the more superconducting carriers are, 
the shorter the penetration depth is, 
and the stronger the diamagnetic effect is.
Outside of the superconductor, 
the magnetic field is more concentrated, 
since magnetic field lines don't break 
and thus the superconductor squeezes magnetic field lines around it,
giving rise to a stronger magnetic field.
Thus, although the magnetic field's not going into the superconductor 
saves energy \emph{within} the superconductor, 
it increases the energy \emph{outside} the superconductor.
So if we have a very strong magnetic field 
or a very large but thin superconductor, 
the magnetic field will still go into the superconductor
to save energy (or otherwise the magnetic field concentration near the boundary is too strong);
the superconducting state is then killed in the equilibrium.

Another effect that can be found using London equations is
the skin current effect.
We can immediately find 
\begin{equation}
    \vb*{j} = \frac{1}{\mu_0} \curl{\vb*{B}}
    = - \frac{1}{\lambda_{\text{L}}} \abs{\vb*{B}_0} \vb*{B}_0 \times \vb*{r} \ee^{- r / \lambda_{\text{L}}}.
\end{equation}
Somehow strangely, we get equations that look like equations governing AC 
but we are working with DC. 

The superconducting carrier concentration $n_{\text{s}}$
has temperature dependence;
now we know superconducting carriers are just Cooper pairs, 
and when the temperature is high, 
they are broken into ordinary electrons,
and therefore the lower the temperature is, 
the stronger the skin effect is.

There is also an electromagnetic gauge called \concept{London gauge}.
We have 
\begin{equation}
    \partial_t \vb*{j} = \frac{1}{\lambda_{\text{L}}^2 \mu_0} \vb*{E}, \quad 
    \curl{\vb*{j}} = - \frac{1}{\lambda_{\text{L}}^2 \mu_0} \vb*{B}.
\end{equation}
Recalling that 
\begin{equation}
    \vb*{E} = - \grad{\phi} - \pdv{\vb*{A}}{t}, \quad 
    \vb*{B} = \curl{\vb*{B}},
\end{equation}
we have 
\begin{equation}
    \partial_t \left(
        \vb*{j} + \frac{1}{\lambda_{\text{L}}^2 \mu_0} \vb*{A}
    \right) = \curl{
        \left(
        \vb*{j} + \frac{1}{\lambda_{\text{L}}^2 \mu_0} \vb*{A}
        \right)
    } = 0.
\end{equation}
This tempts us to choose the following gauge 
\begin{equation}
    \vb*{j} + \frac{1}{\lambda_{\text{L}}^2 \mu_0} \vb*{A} = 0,
\end{equation}
from which we find 
\begin{equation}
    \div{\vb*{A}} = 0, \quad A_{\bot} = 0.
\end{equation}

\section{Ginzburg-Landau theory}

Ginzburg-Landau theory is a theory based on Landau's (general) theory of phase transition.
We can first look at a theory with no electromagnetic coupling,
which describes superfluid as well. 
we introduce a complex order parameter $\psi_{\text{s}}$ 
(sometimes called the macroscopic wave function 
since it can be used to find some macroscopic physical quantities),
and we have 
\begin{equation}
    n = \frac{1}{V} \int \dd[3]{\vb*{r}} \abs{\psi_{\text{s}}(\vb*{r})}^2,
\end{equation}
and to extract information about inhomogeneity of $n_{\text{s}}$, 
$\abs{\grad{\psi_{\text{s}}}}^2$ is also needed.
The free energy is therefore 
\begin{equation}
    F[\psi] = \int \dd[3]{\vb*{r}} \left(
        \alpha \abs{\psi}^2 + \frac{1}{2} \beta \abs*{\psi}^4 + \gamma \abs*{\grad{\psi}}^2
    \right).
\end{equation}
The minimum of this free energy can be obtained by taking the functional variation:
\begin{equation}
    \alpha \psi_0 + \beta \abs{\psi_0}^2 \psi_0 - \gamma \laplacian{\psi_0} = 0.
    \label{eq:gl-min}
\end{equation}
This equation is a non-linear Schrodinger equation.
We consider a boundary at $x = 0$;
in this case \eqref{eq:gl-min} is reduced to its one-dimensional version,
and the boundary conditions are 
\begin{equation}
    \psi(x = 0) = 0, \quad 
    \psi(x \to \infty) = \const.
\end{equation}
Since in the $x \to \infty$ limit the order parameter is a constant, 
from \eqref{eq:gl-min} we get 
\begin{equation}
    \psi(x \to \infty) = \sqrt{- \frac{\alpha}{\beta}}.
\end{equation}
The solution then is 
\begin{equation}
    \psi(x) = \sqrt{- \frac{\alpha}{\beta}} f(x), \quad 
    f(x) = \tanh \frac{x}{\sqrt{2} \xi}, \quad 
    \xi = \sqrt{- \frac{\gamma}{\alpha}}.
    \label{eq:gl-boundary}
\end{equation}
The length scale $\xi$ is named the \concept{coherent length}.

We can then evaluate the free energy of a boundary 
from \eqref{eq:gl-boundary};
if a boundary costs energy, 
then the length of all boundaries will be minimized.

Now we consider a charged liquid with electromagnetic coupling.
The coupling can be introduced easily using minimal coupling, 
in which we replace $\grad$ with $\grad - \ii q \vb*{A}$;
note that whether this is the \emph{correct} coupling 
can't be justified within the macroscopic theory;
it's from the fact that the free energy can be obtained by 
integrating out non-zero Matsubara frequency components 
in the Matsubara version of condensed matter field theory
where $\grad \to \grad - \ii q \vb*{A}$ 
is justified by taking the non-relativistic limit 
that we know the minimal coupling is still correct 
in Ginzburg-Landau free energy theory,
and, in the case of BCS superconductivity,
the fact that $q = - 2 e$ because $\psi$ is the field corresponding to Cooper pairs.
The new free energy in a static magnetic field is therefore 
\begin{equation}
    F[\psi, \vb*{A}] = \int \dd[3]{\vb*{r}}
    \left(
        \alpha \abs*{\psi}^2 + \frac{1}{2} \beta \abs*{\psi}^4 
        + \frac{1}{2 m^*} (\grad - \ii q \vb*{A})
        + \frac{\vb*{B}^2}{2 \mu_0}
    \right)^2.
\end{equation}
We have ignored the electric field part.
Repeating the variation calculation, we find 
\begin{equation}
    \frac{1}{2m^*} (\grad - \ii q \vb*{A})^2 \psi + \alpha \psi + \beta \abs*{\psi}^2 = 0, 
\end{equation}
and 
\begin{equation}
    \curl{\vb*{B}} = \underbrace{
        - \frac{q^2}{m^*} \abs*{\psi}^2 \vb*{A} 
        + \frac{\ii q}{2 m^*} (\psi \grad \psi^* - \psi^* \grad \psi)
    }_{\mu_0 \vb*{j}}.
    \label{eq:b-and-j}
\end{equation}
The second equation tells us an interesting fact of superconductivity:
we can have non-vanishing current with a uniform Cooper concentration.
If $\abs*{\psi}$ is uniform in space, 
since 
\begin{equation}
    \psi = \abs*{\psi} \ee^{\ii \chi},
\end{equation}
we have 
\begin{equation}
    \psi \grad{\psi^*} - \psi^* \grad{\psi}
    = - 2 \ii \abs*{\psi_0}^2 \grad{\chi},
\end{equation}
and thus 
\begin{equation}
    \mu_0 \vb*{j} = \frac{q}{m^*} \abs*{\psi_0}^2 \grad{\chi} - \frac{q^2}{m^*} \abs*{\psi_0}^2 \vb*{A}.
\end{equation}
If we assume phase homogeneity, 
then we just get 
\begin{equation}
    \mu_0 \vb*{j} = - \frac{q^2}{m^*} \underbrace{\abs*{\psi_0}^2}_{n_{\text{s}}} \vb*{A},
\end{equation}
the curl of which is London's second equation,
and the time derivative of which is London's first equation.

The phenomenological parameters $m^*$, $\alpha$ and $\beta$
can be decided by the two experimentally observed quantities, 
the Cooper pair concentration 
\begin{equation}
    \abs*{\psi_0}^2 = - \frac{\alpha}{\beta},
\end{equation}
the penetration depth 
\begin{equation}
    \lambda = \sqrt{\frac{- \beta m^*}{2 \alpha q^2}},
\end{equation}
and the coherent length 
\begin{equation}
    \xi = \sqrt{- \frac{1}{2 m^* \alpha}}.
\end{equation}
If we have 
\begin{equation}
    \alpha \propto T - \Tc,
\end{equation}
then $\lambda, \xi \propto (\Tc - T)^{-1/2}$.

The \concept{Ginzburg parameter} is defined as 
\begin{equation}
    \kappa = \frac{\lambda}{\xi} = \frac{m}{e} \sqrt{\beta}.
\end{equation}
When $\kappa$ is small -- the vanilla case -- 
the superconductor just expels magnetic field 
as is discussed above.
But it's possible to have a much larger $\kappa \gg 1$, 
and in this case, 
the magnetic concentration length is already comparable 
with the size of the superconductor region, 
and therefore although the magnetic field is still expelled 
within the superconducting phase, 
since the domain wall energy with the presence of a magnetic field is \emph{negative}, 
lots of magnetic vortices -- 
at the center of which is a beam of magnetic field lines 
and a little normal electron fluid,
surrounded by a superconducting carrier flow -- 
will be generated within the superconducting phase.

\section{Magnetic flux}

Recalling \eqref{eq:b-and-j}, we have 
\begin{equation}
    \vb*{A} = \frac{m^*}{q^2 \abs*{\psi_0}^2} \left(
        \frac{\ii q}{2 m^*} (\psi \grad{\psi^*} - \psi^* \grad{\psi})
        - \mu_0 \vb*{j}
    \right), 
\end{equation}
and the magnetic flux is 
\begin{equation}
    \Phi = \oint_{\partial{S}} \dd{\vb*{k}} \cdot \vb*{A}
    = \oint_{\partial S} \dd{\vb*{l}} \cdot \left(
        - \frac{m}{e^2 n_{\text{s}}} \vb*{j} + \frac{1}{-2e} \grad{\chi}
    \right),
\end{equation}
where we have already used the conditions that links $\psi$ and electron:
\begin{equation}
    m^* = 2 m, \quad q = - 2 e,
\end{equation}
ignoring interaction-induced mass renormalization.
The phase self-consistent condition dictates 
\begin{equation}
    \oint_{\partial S} \dd{\vb*{l}} \cdot \grad{\chi} = 2 \pi n,
\end{equation}
and if $n \neq 0$,
there is a singularity of $\chi$ inside $S$.
We thus have
\begin{equation}
    \Phi = \Phi' + \frac{m}{e} \oint_{\partial S} \dd{\vb*{l}} \cdot \vb*{v}_{\text{s}}, \quad 
    \Phi' = n \Phi_0 = n \cdot \frac{2 \pi}{2e},
\end{equation}
or if we have $\hbar$ back, 
\begin{equation}
    \Phi_0 = \frac{h}{2e}.
\end{equation}

\end{document}