\documentclass[hyperref, a4paper]{article}

\usepackage{float}
\usepackage{geometry}
\usepackage{titling}
\usepackage{titlesec}
% No longer needed, since we will use enumitem package
% \usepackage{paralist}
\usepackage{enumitem}
\usepackage{footnote}
\usepackage{amsmath, amssymb, amsthm}
\usepackage{mathtools}
\usepackage{bbm}
\usepackage{graphicx}
\usepackage{subcaption}
\usepackage{soulutf8}
\usepackage{physics}
\usepackage{tensor}
\usepackage{siunitx}
\usepackage[version=4]{mhchem}
\usepackage{tikz}
\usepackage{xcolor}
\usepackage{listings}
\usepackage{autobreak}
\usepackage[ruled, vlined, linesnumbered]{algorithm2e}
\usepackage{nameref,zref-xr}
\zxrsetup{toltxlabel}
\usepackage[backend=bibtex]{biblatex}
\addbibresource{elasticity.bib}
\usepackage[colorlinks,unicode]{hyperref} % , linkcolor=black, anchorcolor=black, citecolor=black, urlcolor=black, filecolor=black
\usepackage[most]{tcolorbox}
\usepackage{prettyref}

% Page style
\geometry{left=3.18cm,right=3.18cm,top=2.54cm,bottom=2.54cm}
\titlespacing{\paragraph}{0pt}{1pt}{10pt}[20pt]
\setlength{\droptitle}{-5em}

% More compact lists 
\setlist[itemize]{
    itemindent=17pt, 
    leftmargin=1pt,
    listparindent=\parindent,
    parsep=0pt,
}

% Math operators
\DeclareMathOperator{\timeorder}{\mathcal{T}}
\DeclareMathOperator{\diag}{diag}
\DeclareMathOperator{\legpoly}{P}
\DeclareMathOperator{\primevalue}{P}
\DeclareMathOperator{\sgn}{sgn}
\DeclareMathOperator{\res}{Res}
\newcommand*{\ii}{\mathrm{i}}
\newcommand*{\ee}{\mathrm{e}}
\newcommand*{\const}{\mathrm{const}}
\newcommand*{\suchthat}{\quad \text{s.t.} \quad}
\newcommand*{\argmin}{\arg\min}
\newcommand*{\argmax}{\arg\max}
\newcommand*{\normalorder}[1]{: #1 :}
\newcommand*{\pair}[1]{\langle #1 \rangle}
\newcommand*{\fd}[1]{\mathcal{D} #1}
\DeclareMathOperator{\bigO}{\mathcal{O}}

% TikZ setting
\usetikzlibrary{arrows,shapes,positioning}
\usetikzlibrary{arrows.meta}
\usetikzlibrary{decorations.markings}
\tikzstyle arrowstyle=[scale=1]
\tikzstyle directed=[postaction={decorate,decoration={markings,
    mark=at position .5 with {\arrow[arrowstyle]{stealth}}}}]
\tikzstyle ray=[directed, thick]
\tikzstyle dot=[anchor=base,fill,circle,inner sep=1pt]

% Algorithm setting
% Julia-style code
\SetKwIF{If}{ElseIf}{Else}{if}{}{elseif}{else}{end}
\SetKwFor{For}{for}{}{end}
\SetKwFor{While}{while}{}{end}
\SetKwProg{Function}{function}{}{end}
\SetArgSty{textnormal}

\newcommand*{\concept}[1]{{\textbf{#1}}}

% Embedded codes
\lstset{basicstyle=\ttfamily,
  showstringspaces=false,
  commentstyle=\color{gray},
  keywordstyle=\color{blue}
}

% Reference formatting
\newcommand*{\citesec}[1]{\S~{#1}}
\newcommand*{\citechap}[1]{chap.~{#1}}
\newcommand*{\citefig}[1]{Fig.~{#1}}
\newcommand*{\citetable}[1]{Table~{#1}}
\newcommand*{\citepage}[1]{pp.~{#1}}
\newrefformat{fig}{Fig.~\ref{#1}}
\newcommand*{\term}[1]{\textit{#1}}

% Color boxes
\tcbuselibrary{skins, breakable, theorems}

\newtcbtheorem{infobox}{Box}{
    enhanced,
    boxrule=0pt,
    colback=blue!5,
    colframe=blue!5,
    coltitle=blue!50,
    borderline west={4pt}{0pt}{blue!65},
    sharp corners,
    fonttitle=\bfseries, 
    breakable,
    before upper={\parindent15pt\noindent}}{box}
\newtcbtheorem[use counter from=infobox]{theorybox}{Box}{
    enhanced,
    boxrule=0pt,
    colback=orange!5, 
    colframe=orange!5, 
    coltitle=orange!50,
    borderline west={4pt}{0pt}{orange!65},
    sharp corners,
    fonttitle=\bfseries, 
    breakable,
    before upper={\parindent15pt\noindent}}{box}
\newtcbtheorem[use counter from=infobox]{learnbox}{Box}{
    enhanced,
    boxrule=0pt,
    colback=green!5,
    colframe=green!5,
    coltitle=green!50,
    borderline west={4pt}{0pt}{green!65},
    sharp corners,
    fonttitle=\bfseries, 
    breakable,
    before upper={\parindent15pt\noindent}}{box}


\newenvironment{shelldisplay}{\begin{lstlisting}}{\end{lstlisting}}

\newcommand*{\kB}{k_{\text{B}}}
\newcommand*{\muB}{\mu_{\text{B}}}
\newcommand*{\efermi}{E_{\text{F}}}
\newcommand*{\pfermi}{p_{\text{F}}}
\newcommand*{\vfermi}{v_{\text{F}}}
\newcommand*{\sA}{\text{A}}
\newcommand*{\sB}{\text{B}}
\newcommand*{\Tc}{T_{\text{c}}}
\newcommand*{\hethree}{$^3$He}
\newcommand*{\hefour}{$^4$He}

\title{Elasticity in structural mechanics}
\author{Jinyuan Wu}

\begin{document}

\maketitle

\section{Rigid body analysis}

\section{Elastic medium}

\paragraph*{Definition} The deformation $\vb*{u}(t)$ of the system 
is completely decided by the external loading at $t$.
Notable counterparts:
\begin{itemize}
    \item \emph{Fluid}. $\vb*{u}$ $\Leftarrow$ $\vb*{v}$ $\Leftarrow$ $\vb*{F}$: not elastic.
    \item \emph{Plastic}. $\vb*{u}$ depends on history: not elastic.
\end{itemize}

\paragraph*{Degrees of freedom, with infinitesimal deformation} We deal with two sets of variables:
\begin{itemize}
    \item \emph{Stress $\sigma_{ij}$}. $\dd{F}_i = \sigma_{ij} \dd{A}_j$.
    Moment is not needed for bulk equation of equilibrium; 
    but it's needed to capture the spatially fast-varying internal force 
    in low-dimensional systems.
    \item \emph{Strain $u_{ij}$}. For small deformation
    \begin{equation}
        u_{ij} = \frac{1}{2} \left(
            \pdv{u_i}{x_j} + \pdv{u_j}{x_i}
        \right).
    \end{equation}
    \item \emph{Constitutive relations}. $\sigma_{ij} = \sigma_{ij}[u_{ij}]$.
\end{itemize}

\paragraph*{Uniform isotropic linear medium} Constitutive relation  
\begin{equation}
    \sigma_{ik} = K u_{ll} \delta_{ik}
    + 2 \mu \left(
        u_{ik} - \frac{1}{3} \delta_{ik} u_{ll}
    \right).
    \label{eq:constitutive-relation}
\end{equation}

\paragraph*{Temperature expansion} The strain induced by temperature change:
\begin{equation}
    \dv{u}{x} = \alpha (T - T_0),
\end{equation}
where $T_0$ is the ``overall'' temperature.

\section{Uniform isotropic linear medium, in experiments} 

\paragraph*{Two modes of strain}
\begin{itemize}
    \item \emph{Compression/tension}. Along one direction (for example $z$):
    \begin{equation}
        \epsilon = \frac{\delta}{L} = u_{zz}.
    \end{equation}
    \item \emph{Shear}. On the $xy$ plane:
    \begin{equation}
        \gamma = \theta_{xx'} + \theta_{yy'} = 2 u_{xy}.
    \end{equation}
\end{itemize} 

\paragraph*{Young's modulus} Relation between tension and force:
\begin{equation}
    E = \frac{P}{\epsilon} = \frac{P L}{\delta} \Rightarrow
    F = PA = \frac{\delta}{L} \cdot EA.
\end{equation}

\paragraph*{Poisson's ratio} Relation between transverse strain and axial strain 
(in Young's modulus experiment):
\begin{equation}
    \sigma = \nu = - \dv{\epsilon_{\text{transverse}}}{\epsilon_{\text{axial}}}.
\end{equation}
This is how the material becomes thinner when stretched.

\paragraph*{Volume modulus} Relation between pressure and volume:
\begin{equation}
    K = - V \dv{P}{V}. 
\end{equation}
Here $K$ is that parameter in \eqref{eq:constitutive-relation}.

\paragraph*{Shear modulus} Relation between shear stress and shear strain:
\begin{equation}
    \mu = G = \frac{\tau}{\gamma}.
\end{equation}
Here $\tau$ is $\sigma_{xy}$ (or $yz$ or $zx$); $\gamma$ is the shear strain.


\paragraph*{How many independent parameters?} In isothermal process:  
\begin{equation}
    E = \frac{9 K \mu}{3 K + \mu}, \quad 
    \sigma = \frac{1}{2} \frac{3K - 2 \mu}{3 K + \mu}.
\end{equation}

\paragraph*{When is the linear elasticity condition broken?} 
\begin{enumerate}
    \item Linear region.
    \item Proportional limit.
    \item Elastic limit.
    \item Yield point.
    \item Ultimate tensile point.
    \item Breaking point.
\end{enumerate}


\section{Low dimension system: torsion of cylinder-like rod} 

\paragraph*{Reaction of $\varphi$ to torque} Here $T$ is the torque:
\begin{equation}
    \dv{\varphi}{z} = \frac{T(z)}{JG}, \quad 
    T(z) = \int_{0}^{z} \dd{z'} \dv{\ \mathrm{torque}}{z'}.
\end{equation}

\paragraph*{Relation between torque and stress}
\begin{equation}
    \gamma = \gamma_{xz} = \dv{\varphi}{z} r, \quad \tau = G \gamma, 
\end{equation}
\begin{equation}
    \tau_\text{max} = G \dv{\varphi}{z} R 
    = \frac{T R}{J} .
\end{equation}
Here $R$ may also be written as $c$.

\paragraph*{Note on $J$} It's actually not moment of inertia! 

\section{Low dimension system: beam, or rod predominantly bended}

\begin{figure}
    \centering
    \begin{tikzpicture}[x=0.75pt,y=0.75pt,yscale=-1,xscale=1]
        %uncomment if require: \path (0,300); %set diagram left start at 0, and has height of 300
        
        %Shape: Rectangle [id:dp7091983334625076] 
        \draw  [color={rgb, 255:red, 155; green, 155; blue, 155 }  ,draw opacity=1 ] (48,82.94) -- (305.6,82.94) -- (305.6,100) -- (48,100) -- cycle ;
        %Shape: Rectangle [id:dp0667722272978708] 
        \draw  [draw opacity=0][fill={rgb, 255:red, 155; green, 155; blue, 155 }  ,fill opacity=0.2 ] (94.5,82.44) -- (132.8,82.44) -- (132.8,99.5) -- (94.5,99.5) -- cycle ;
        %Straight Lines [id:da004859031102435907] 
        \draw [color={rgb, 255:red, 74; green, 144; blue, 226 }  ,draw opacity=1 ] [dash pattern={on 4.5pt off 4.5pt}]  (41.6,90.97) -- (324.2,90.97) ;
        \draw [shift={(327.2,90.97)}, rotate = 180] [fill={rgb, 255:red, 74; green, 144; blue, 226 }  ,fill opacity=1 ][line width=0.08]  [draw opacity=0] (7.14,-3.43) -- (0,0) -- (7.14,3.43) -- (4.74,0) -- cycle    ;
        %Straight Lines [id:da6875279376729626] 
        \draw [color={rgb, 255:red, 74; green, 144; blue, 226 }  ,draw opacity=1 ] [dash pattern={on 4.5pt off 4.5pt}]  (169.65,90.97) -- (169.65,120.67) ;
        \draw [shift={(169.65,123.67)}, rotate = 270] [fill={rgb, 255:red, 74; green, 144; blue, 226 }  ,fill opacity=1 ][line width=0.08]  [draw opacity=0] (7.14,-3.43) -- (0,0) -- (7.14,3.43) -- (4.74,0) -- cycle    ;
        %Curve Lines [id:da3616568436051619] 
        \draw [color={rgb, 255:red, 74; green, 144; blue, 226 }  ,draw opacity=1 ]   (48,90) .. controls (68,75) and (116.8,58.07) .. (176,101.27) .. controls (235.2,144.47) and (291.2,91.67) .. (305.6,91.67) ;
        %Straight Lines [id:da8788339880180727] 
        \draw [color={rgb, 255:red, 74; green, 144; blue, 226 }  ,draw opacity=1 ]   (257.65,90.97) -- (257.65,109.47) ;
        \draw [shift={(257.65,112.47)}, rotate = 270] [fill={rgb, 255:red, 74; green, 144; blue, 226 }  ,fill opacity=1 ][line width=0.08]  [draw opacity=0] (7.14,-3.43) -- (0,0) -- (7.14,3.43) -- (4.74,0) -- cycle    ;
        %Shape: Rectangle [id:dp9072433812710614] 
        \draw  [draw opacity=0][fill={rgb, 255:red, 155; green, 155; blue, 155 }  ,fill opacity=0.2 ] (94.5,193.64) -- (164,193.64) -- (164,234.87) -- (94.5,234.87) -- cycle ;
        %Straight Lines [id:da19103095051672914] 
        \draw [color={rgb, 255:red, 126; green, 211; blue, 33 }  ,draw opacity=1 ]   (164,213.44) -- (164,251.71) ;
        \draw [shift={(164,254.71)}, rotate = 270] [fill={rgb, 255:red, 126; green, 211; blue, 33 }  ,fill opacity=1 ][line width=0.08]  [draw opacity=0] (7.14,-3.43) -- (0,0) -- (7.14,3.43) -- (4.74,0) -- cycle    ;
        %Curve Lines [id:da7679114487230241] 
        \draw [color={rgb, 255:red, 126; green, 211; blue, 33 }  ,draw opacity=1 ]   (181.4,239.94) .. controls (198.6,222.98) and (191.33,202.3) .. (183.12,190.31) ;
        \draw [shift={(181.4,187.94)}, rotate = 52.5] [fill={rgb, 255:red, 126; green, 211; blue, 33 }  ,fill opacity=1 ][line width=0.08]  [draw opacity=0] (7.14,-3.43) -- (0,0) -- (7.14,3.43) -- (4.74,0) -- cycle    ;
        %Curve Lines [id:da9862032299860455] 
        \draw [color={rgb, 255:red, 126; green, 211; blue, 33 }  ,draw opacity=1 ]   (85.4,239.94) .. controls (68.2,222.98) and (75.47,202.3) .. (83.68,190.31) ;
        \draw [shift={(85.4,187.94)}, rotate = 127.5] [fill={rgb, 255:red, 126; green, 211; blue, 33 }  ,fill opacity=1 ][line width=0.08]  [draw opacity=0] (7.14,-3.43) -- (0,0) -- (7.14,3.43) -- (4.74,0) -- cycle    ;
        %Straight Lines [id:da6540035568159142] 
        \draw [color={rgb, 255:red, 126; green, 211; blue, 33 }  ,draw opacity=1 ]   (128.8,166.87) -- (128.8,191.8) ;
        \draw [shift={(128.8,194.8)}, rotate = 270] [fill={rgb, 255:red, 126; green, 211; blue, 33 }  ,fill opacity=1 ][line width=0.08]  [draw opacity=0] (7.14,-3.43) -- (0,0) -- (7.14,3.43) -- (4.74,0) -- cycle    ;
        %Straight Lines [id:da6513847673071842] 
        \draw [color={rgb, 255:red, 155; green, 155; blue, 155 }  ,draw opacity=1 ] [dash pattern={on 0.84pt off 2.51pt}]  (94.5,99.5) -- (94.5,193.64) ;
        %Straight Lines [id:da8977621367535318] 
        \draw [color={rgb, 255:red, 155; green, 155; blue, 155 }  ,draw opacity=1 ] [dash pattern={on 0.84pt off 2.51pt}]  (132.8,99.5) -- (164,193.64) ;
        %Shape: Polygon Curved [id:ds8786065012864805] 
        \draw  [color={rgb, 255:red, 155; green, 155; blue, 155 }  ,draw opacity=1 ][fill={rgb, 255:red, 155; green, 155; blue, 155 }  ,fill opacity=0.2 ] (414.21,82.6) .. controls (424.57,85.1) and (436.63,86.6) .. (447.85,83.44) .. controls (452.21,100.94) and (447.71,83.44) .. (452.46,101.44) .. controls (437.32,107.38) and (427.96,107.19) .. (410.85,101.5) .. controls (413.96,83.85) and (411.21,101.85) .. (414.21,82.6) -- cycle ;
        %Straight Lines [id:da5359999304107681] 
        \draw [color={rgb, 255:red, 74; green, 144; blue, 226 }  ,draw opacity=1 ] [dash pattern={on 4.5pt off 4.5pt}]  (410.85,101.5) -- (427.46,0.6) ;
        %Straight Lines [id:da3771255472520405] 
        \draw [color={rgb, 255:red, 74; green, 144; blue, 226 }  ,draw opacity=1 ] [dash pattern={on 4.5pt off 4.5pt}]  (452.46,101.44) -- (427.46,0.6) ;
        %Shape: Rectangle [id:dp15144920564462083] 
        \draw  [draw opacity=0][fill={rgb, 255:red, 155; green, 155; blue, 155 }  ,fill opacity=0.2 ] (400,189.84) -- (469.5,189.84) -- (469.5,231.07) -- (400,231.07) -- cycle ;
        %Straight Lines [id:da9946258092408058] 
        \draw [color={rgb, 255:red, 74; green, 144; blue, 226 }  ,draw opacity=1 ] [dash pattern={on 4.5pt off 4.5pt}]  (434.75,210.45) -- (434.75,240.16) ;
        \draw [shift={(434.75,243.16)}, rotate = 270] [fill={rgb, 255:red, 74; green, 144; blue, 226 }  ,fill opacity=1 ][line width=0.08]  [draw opacity=0] (7.14,-3.43) -- (0,0) -- (7.14,3.43) -- (4.74,0) -- cycle    ;
        %Straight Lines [id:da6008369177101607] 
        \draw [color={rgb, 255:red, 74; green, 144; blue, 226 }  ,draw opacity=1 ] [dash pattern={on 4.5pt off 4.5pt}]  (432.83,210.45) -- (388.83,210.45) ;
        \draw [shift={(385.83,210.45)}, rotate = 360] [fill={rgb, 255:red, 74; green, 144; blue, 226 }  ,fill opacity=1 ][line width=0.08]  [draw opacity=0] (7.14,-3.43) -- (0,0) -- (7.14,3.43) -- (4.74,0) -- cycle    ;
        %Straight Lines [id:da5422371619455366] 
        \draw [color={rgb, 255:red, 155; green, 155; blue, 155 }  ,draw opacity=1 ]   (365.83,210.45) -- (332.83,210.45) ;
        \draw [shift={(329.83,210.45)}, rotate = 360] [fill={rgb, 255:red, 155; green, 155; blue, 155 }  ,fill opacity=1 ][line width=0.08]  [draw opacity=0] (7.14,-3.43) -- (0,0) -- (7.14,3.43) -- (4.74,0) -- cycle    ;
        
        % Text Node
        \draw (113.65,90.97) node  [color={rgb, 255:red, 155; green, 155; blue, 155 }  ,opacity=1 ]  {$\mathrm{d} x$};
        % Text Node
        \draw (329.2,90.97) node [anchor=west] [inner sep=0.75pt]  [color={rgb, 255:red, 74; green, 144; blue, 226 }  ,opacity=1 ]  {$x$};
        % Text Node
        \draw (169.65,126.67) node [anchor=north] [inner sep=0.75pt]  [color={rgb, 255:red, 74; green, 144; blue, 226 }  ,opacity=1 ]  {$z$};
        % Text Node
        \draw (257.65,115.47) node [anchor=north] [inner sep=0.75pt]  [color={rgb, 255:red, 74; green, 144; blue, 226 }  ,opacity=1 ]  {$w$};
        % Text Node
        \draw (166,254.71) node [anchor=west] [inner sep=0.75pt]  [color={rgb, 255:red, 126; green, 211; blue, 33 }  ,opacity=1 ] [align=left] {$\displaystyle V$};
        % Text Node
        \draw (197.8,213.94) node [anchor=west] [inner sep=0.75pt]  [color={rgb, 255:red, 126; green, 211; blue, 33 }  ,opacity=1 ]  {$M+\mathrm{d} M$};
        % Text Node
        \draw (50,211.54) node [anchor=west] [inner sep=0.75pt]  [color={rgb, 255:red, 126; green, 211; blue, 33 }  ,opacity=1 ]  {$M$};
        % Text Node
        \draw (130.8,180.84) node [anchor=west] [inner sep=0.75pt]  [color={rgb, 255:red, 126; green, 211; blue, 33 }  ,opacity=1 ]  {$q\mathrm{d} x$};
        % Text Node
        \draw (129.25,214.25) node  [color={rgb, 255:red, 155; green, 155; blue, 155 }  ,opacity=1 ]  {$\mathrm{d} x$};
        % Text Node
        \draw (431.5,94.5) node  [color={rgb, 255:red, 155; green, 155; blue, 155 }  ,opacity=1 ]  {$\mathrm{d} x’$};
        % Text Node
        \draw (398.13,37.35) node [anchor=north west][inner sep=0.75pt]  [color={rgb, 255:red, 74; green, 144; blue, 226 }  ,opacity=1 ]  {$R$};
        % Text Node
        \draw (434.75,246.16) node [anchor=north] [inner sep=0.75pt]  [color={rgb, 255:red, 74; green, 144; blue, 226 }  ,opacity=1 ]  {$z$};
        % Text Node
        \draw (383.83,210.45) node [anchor=east] [inner sep=0.75pt]  [color={rgb, 255:red, 74; green, 144; blue, 226 }  ,opacity=1 ]  {$y$};
        % Text Node
        \draw (347.83,207.45) node [anchor=south] [inner sep=0.75pt]  [color={rgb, 255:red, 155; green, 155; blue, 155 }  ,opacity=1 ]  {$I$};
        
        
    \end{tikzpicture}        
    \caption{Analysis of a beam}
    \label{fig:beam}
\end{figure}

\paragraph*{Important degrees of freedom} Suppose the beam is in direction $z$.
Sign convention: for force, deflection: downwards = $+$; 
for moment: counterclockwise = $+$.
\begin{itemize}
    \item \emph{Deflection}. Referred to as $w$.
    \item \emph{Shear force}. The internal force, averaged: 
    \begin{equation}
        \vb*{F}_{\bot} = \vb*{\sigma} \cdot \vb*{A} 
        = \sigma_{xz} A \vu*{x} + \sigma_{yz} A \vu*{y},
    \end{equation}
    and usually we only consider one direction (say $x$), 
    and $\vb*{F}_\bot = V \vu*{x}$.
    Below we change $z$ to $x$.
    \item \emph{Moment}. The ``first-order moment'' of internal force:
    \begin{equation}
        M + \dd{M} = M + V \dd{x} \Rightarrow V = \pdv{M}{x}.
    \end{equation}
\end{itemize}

\paragraph*{Equation of equilibrium} For determining $V$: 
\begin{equation}
    \pdv{V}{x} + q = 0,
\end{equation}
where $q$ is force per unit length. 
The relation between moment and $w$:
\begin{equation}
    M = - EI \pdv[2]{w}{x}.
\end{equation}
Here the axis of $I$ is the same as the direction of $M$.

\paragraph*{Details in bending stress} Assuming $R$ being large,  
each beam element can be seen as a beam element feeling stretching only, 
and thus 
\begin{equation}
    \dv{x'}{x} = \frac{R+z}{R} , \quad \sigma \coloneqq \sigma_{xx} = E u_{xx} = \frac{z}{R} E,
\end{equation}
while 
\begin{equation}
    M = \int \dd{z} \dd{y} \sigma \cdot z 
    = \frac{E}{R} \underbrace{\int \dd{z} \dd{y} z^2}_{\eqqcolon I}.
\end{equation}    
So after $M$ is found from one of the equations above, 
\begin{equation}
    \sigma = \frac{z}{I} M,
\end{equation}
and thus at a given point, 
\begin{equation}
    \sigma_{\text{max}} = \frac{z_{\text{max}}}{I} M,
\end{equation}
which is needed to determine whether the beam fails.

\paragraph*{Boundary condition} The boundary condition of concentrated load can be determined
if the following rules are followed: 
\begin{itemize}
    \item $V$ is a downward force applied to the right end of a beam element 
    by the beam element following it. 
    \item No $V$ is applied at the left boundary of the first beam element: 
    thus if $F(x = 0)$ is upward, 
    then $V(x = 0)$ is downward and is positive.
    \item No $V$ is applied at the right boundary of the last beam element:
    thus if $F(x = L)$ is upward, 
    then the shear force the $L - \dd{x}$ beam element applies to the beam element at $L$ 
    is downward, and therefore the shear force 
    applied to the beam element at $L - \dd{x}$ is upward, 
    and $V(x = L)$ is negative.
\end{itemize}

\paragraph*{Details in shear stress}

\begin{figure}
    \centering
    \begin{tikzpicture}[x=0.75pt,y=0.75pt,yscale=-1,xscale=1]
        %uncomment if require: \path (0,300); %set diagram left start at 0, and has height of 300
        
        %Shape: Parallelogram [id:dp8814763041593383] 
        \draw  [color={rgb, 255:red, 155; green, 155; blue, 155 }  ,draw opacity=1 ][fill={rgb, 255:red, 155; green, 155; blue, 155 }  ,fill opacity=0.2 ] (154.9,91.59) -- (214.99,77.55) -- (215.74,143.93) -- (155.65,157.97) -- cycle ;
        %Shape: Parallelogram [id:dp5215193013721346] 
        \draw  [color={rgb, 255:red, 155; green, 155; blue, 155 }  ,draw opacity=1 ][fill={rgb, 255:red, 155; green, 155; blue, 155 }  ,fill opacity=0.2 ] (172.9,101.59) -- (232.99,87.55) -- (233.74,153.93) -- (173.65,167.97) -- cycle ;
        %Straight Lines [id:da08380864345813333] 
        \draw    (281.32,95.01) -- (299.13,104.9) ;
        \draw [shift={(301.75,106.35)}, rotate = 209.04] [fill={rgb, 255:red, 0; green, 0; blue, 0 }  ][line width=0.08]  [draw opacity=0] (7.14,-3.43) -- (0,0) -- (7.14,3.43) -- (4.74,0) -- cycle    ;
        %Straight Lines [id:da9429214629058593] 
        \draw    (281.32,95.01) -- (281.32,115.85) ;
        \draw [shift={(281.32,118.85)}, rotate = 270] [fill={rgb, 255:red, 0; green, 0; blue, 0 }  ][line width=0.08]  [draw opacity=0] (7.14,-3.43) -- (0,0) -- (7.14,3.43) -- (4.74,0) -- cycle    ;
        %Straight Lines [id:da8385887473966083] 
        \draw    (281.32,95.01) -- (264.62,100) ;
        \draw [shift={(261.75,100.85)}, rotate = 343.38] [fill={rgb, 255:red, 0; green, 0; blue, 0 }  ][line width=0.08]  [draw opacity=0] (7.14,-3.43) -- (0,0) -- (7.14,3.43) -- (4.74,0) -- cycle    ;
        %Straight Lines [id:da6485090842982482] 
        \draw [color={rgb, 255:red, 155; green, 155; blue, 155 }  ,draw opacity=1 ]   (154.9,91.59) -- (172.9,101.59) ;
        %Straight Lines [id:da40933827059496863] 
        \draw [color={rgb, 255:red, 155; green, 155; blue, 155 }  ,draw opacity=1 ]   (214.99,77.55) -- (232.99,87.55) ;
        %Straight Lines [id:da7844700040601715] 
        \draw [color={rgb, 255:red, 155; green, 155; blue, 155 }  ,draw opacity=1 ]   (215.74,143.93) -- (233.74,153.93) ;
        %Straight Lines [id:da7318855463507685] 
        \draw [color={rgb, 255:red, 155; green, 155; blue, 155 }  ,draw opacity=1 ]   (155.65,157.97) -- (173.65,167.97) ;
        %Curve Lines [id:da7635515132188251] 
        \draw [color={rgb, 255:red, 126; green, 211; blue, 33 }  ,draw opacity=1 ] [dash pattern={on 0.84pt off 2.51pt}]  (245.83,53.6) .. controls (235.22,68.08) and (230.19,55.55) .. (210.98,72.63) ;
        \draw [shift={(208.83,74.6)}, rotate = 316.4] [fill={rgb, 255:red, 126; green, 211; blue, 33 }  ,fill opacity=1 ][line width=0.08]  [draw opacity=0] (7.14,-3.43) -- (0,0) -- (7.14,3.43) -- (4.74,0) -- cycle    ;
        %Shape: Parallelogram [id:dp7525120896038273] 
        \draw  [draw opacity=0][fill={rgb, 255:red, 126; green, 211; blue, 33 }  ,fill opacity=0.2 ] (173.3,101.5) -- (232.99,87.55) -- (233.19,128.95) -- (173.49,142.89) -- cycle ;
        %Straight Lines [id:da30693318891597166] 
        \draw [color={rgb, 255:red, 126; green, 211; blue, 33 }  ,draw opacity=1 ]   (176.12,136.9) -- (229.88,123.96) ;
        \draw [shift={(232.79,123.26)}, rotate = 166.46] [fill={rgb, 255:red, 126; green, 211; blue, 33 }  ,fill opacity=1 ][line width=0.08]  [draw opacity=0] (7.14,-3.43) -- (0,0) -- (7.14,3.43) -- (4.74,0) -- cycle    ;
        \draw [shift={(173.21,137.6)}, rotate = 346.46] [fill={rgb, 255:red, 126; green, 211; blue, 33 }  ,fill opacity=1 ][line width=0.08]  [draw opacity=0] (7.14,-3.43) -- (0,0) -- (7.14,3.43) -- (4.74,0) -- cycle    ;
        
        % Text Node
        \draw (303.75,106.35) node [anchor=west] [inner sep=0.75pt]    {$x$};
        % Text Node
        \draw (281.32,121.85) node [anchor=north] [inner sep=0.75pt]    {$z$};
        % Text Node
        \draw (259.75,100.85) node [anchor=east] [inner sep=0.75pt]    {$y$};
        % Text Node
        \draw (245.83,50.6) node [anchor=south] [inner sep=0.75pt]  [color={rgb, 255:red, 126; green, 211; blue, 33 }  ,opacity=1 ]  {$\sigma _{xz} =0$};
        % Text Node
        \draw (199.13,115.52) node [anchor=north west][inner sep=0.75pt]  [color={rgb, 255:red, 0; green, 0; blue, 0 }  ,opacity=1 ]  {$t$};
        
        
    \end{tikzpicture}
    \caption{Analysis of shear stress}     
    \label{fig:shear-stress-beam}
\end{figure}

Due to $V$ we also have $\sigma_{zx}$, and 
\begin{equation}
    0 = \partial_x \sigma_{xx} + \partial_z \sigma_{zx} \Rightarrow
    0 = \frac{1}{I} \pdv{M}{x} \int \dd{z} \dd{y} z + \int \dd{y} \sigma_{zx},
\end{equation}
and the average shear stress is 
($t$ is the width in $y$ coordinate at $z$)
\begin{equation}
    \tau \coloneqq \bar{\sigma}_{zx} = 
    \frac{1}{I t} \pdv{M}{x} \underbrace{
        \int_{\text{area above or below $z$}} z \dd{y} \dd{z}
    }_{Q}
    = \frac{Q}{It} V .
\end{equation}
The integration range used in calculating $Q$ is the Green region in 
\prettyref{fig:shear-stress-beam}.

\paragraph*{Procedure} 
\begin{enumerate}
    \item Finding all reaction forces from the loading.
    \item Finding $V$. 
    \item Finding $M$.
    \item Finding $\sigma$ and $\sigma_{\text{max}}$.
    \item Finding $w$ if necessary.
\end{enumerate}

\section{Problems}

\begin{itemize}
    \item Using deformation to decide forces (that otherwise can't be determined).
    \item How fast a shaft can rotate: $P = T \omega$. Then $\tau_{\text{max}}$ can be found.
    \item Beam analysis, and whether it fails because $\sigma_{\text{max}}$ is too large.
\end{itemize}

\end{document}