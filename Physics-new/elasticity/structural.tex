\documentclass[hyperref, a4paper]{article}

\usepackage{geometry}
\usepackage{titling}
\usepackage{titlesec}
% No longer needed, since we will use enumitem package
% \usepackage{paralist}
\usepackage{enumitem}
\usepackage{footnote}
\usepackage{amsmath, amssymb, amsthm}
\usepackage{mathtools}
\usepackage{bbm}
\usepackage{graphicx}
\usepackage{subcaption}
\usepackage{soulutf8}
\usepackage{physics}
\usepackage{tensor}
\usepackage{siunitx}
\usepackage[version=4]{mhchem}
\usepackage{tikz}
\usepackage{xcolor}
\usepackage{listings}
\usepackage{autobreak}
\usepackage[ruled, vlined, linesnumbered]{algorithm2e}
\usepackage{nameref,zref-xr}
\zxrsetup{toltxlabel}
\usepackage[backend=bibtex]{biblatex}
\addbibresource{elasticity.bib}
\usepackage[colorlinks,unicode]{hyperref} % , linkcolor=black, anchorcolor=black, citecolor=black, urlcolor=black, filecolor=black
\usepackage[most]{tcolorbox}
\usepackage{prettyref}

% Page style
\geometry{left=3.18cm,right=3.18cm,top=2.54cm,bottom=2.54cm}
\titlespacing{\paragraph}{0pt}{1pt}{10pt}[20pt]
\setlength{\droptitle}{-5em}

% More compact lists 
\setlist[itemize]{
    itemindent=17pt, 
    leftmargin=1pt,
    listparindent=\parindent,
    parsep=0pt,
}

% Math operators
\DeclareMathOperator{\timeorder}{\mathcal{T}}
\DeclareMathOperator{\diag}{diag}
\DeclareMathOperator{\legpoly}{P}
\DeclareMathOperator{\primevalue}{P}
\DeclareMathOperator{\sgn}{sgn}
\DeclareMathOperator{\res}{Res}
\newcommand*{\ii}{\mathrm{i}}
\newcommand*{\ee}{\mathrm{e}}
\newcommand*{\const}{\mathrm{const}}
\newcommand*{\suchthat}{\quad \text{s.t.} \quad}
\newcommand*{\argmin}{\arg\min}
\newcommand*{\argmax}{\arg\max}
\newcommand*{\normalorder}[1]{: #1 :}
\newcommand*{\pair}[1]{\langle #1 \rangle}
\newcommand*{\fd}[1]{\mathcal{D} #1}
\DeclareMathOperator{\bigO}{\mathcal{O}}

% TikZ setting
\usetikzlibrary{arrows,shapes,positioning}
\usetikzlibrary{arrows.meta}
\usetikzlibrary{decorations.markings}
\tikzstyle arrowstyle=[scale=1]
\tikzstyle directed=[postaction={decorate,decoration={markings,
    mark=at position .5 with {\arrow[arrowstyle]{stealth}}}}]
\tikzstyle ray=[directed, thick]
\tikzstyle dot=[anchor=base,fill,circle,inner sep=1pt]

% Algorithm setting
% Julia-style code
\SetKwIF{If}{ElseIf}{Else}{if}{}{elseif}{else}{end}
\SetKwFor{For}{for}{}{end}
\SetKwFor{While}{while}{}{end}
\SetKwProg{Function}{function}{}{end}
\SetArgSty{textnormal}

\newcommand*{\concept}[1]{{\textbf{#1}}}

% Embedded codes
\lstset{basicstyle=\ttfamily,
  showstringspaces=false,
  commentstyle=\color{gray},
  keywordstyle=\color{blue}
}

% Reference formatting
\newcommand*{\citesec}[1]{\S~{#1}}
\newcommand*{\citechap}[1]{chap.~{#1}}
\newcommand*{\citefig}[1]{Fig.~{#1}}
\newcommand*{\citetable}[1]{Table~{#1}}
\newcommand*{\citepage}[1]{pp.~{#1}}
\newrefformat{fig}{Fig.~\ref{#1}}
\newcommand*{\term}[1]{\textit{#1}}

% Color boxes
\tcbuselibrary{skins, breakable, theorems}

\newtcbtheorem{infobox}{Box}{
    enhanced,
    boxrule=0pt,
    colback=blue!5,
    colframe=blue!5,
    coltitle=blue!50,
    borderline west={4pt}{0pt}{blue!65},
    sharp corners,
    fonttitle=\bfseries, 
    breakable,
    before upper={\parindent15pt\noindent}}{box}
\newtcbtheorem[use counter from=infobox]{theorybox}{Box}{
    enhanced,
    boxrule=0pt,
    colback=orange!5, 
    colframe=orange!5, 
    coltitle=orange!50,
    borderline west={4pt}{0pt}{orange!65},
    sharp corners,
    fonttitle=\bfseries, 
    breakable,
    before upper={\parindent15pt\noindent}}{box}
\newtcbtheorem[use counter from=infobox]{learnbox}{Box}{
    enhanced,
    boxrule=0pt,
    colback=green!5,
    colframe=green!5,
    coltitle=green!50,
    borderline west={4pt}{0pt}{green!65},
    sharp corners,
    fonttitle=\bfseries, 
    breakable,
    before upper={\parindent15pt\noindent}}{box}


\newenvironment{shelldisplay}{\begin{lstlisting}}{\end{lstlisting}}

\newcommand*{\kB}{k_{\text{B}}}
\newcommand*{\muB}{\mu_{\text{B}}}
\newcommand*{\efermi}{E_{\text{F}}}
\newcommand*{\pfermi}{p_{\text{F}}}
\newcommand*{\vfermi}{v_{\text{F}}}
\newcommand*{\sA}{\text{A}}
\newcommand*{\sB}{\text{B}}
\newcommand*{\Tc}{T_{\text{c}}}
\newcommand*{\hethree}{$^3$He}
\newcommand*{\hefour}{$^4$He}

\title{Elasticity in structural mechanics}
\author{Jinyuan Wu}

\begin{document}

\maketitle

\section{Rigid body analysis}

\section{Elastic medium}

\paragraph*{Definition} The deformation $\vb*{u}(t)$ of the system 
is completely decided by the external loading at $t$.
Notable counterparts:
\begin{itemize}
    \item \emph{Fluid}. $\vb*{u}$ $\Leftarrow$ $\vb*{v}$ $\Leftarrow$ $\vb*{F}$: not elastic.
    \item \emph{Plastic}. $\vb*{u}$ depends on history: not elastic.
\end{itemize}

\paragraph*{Degrees of freedom, with infinitesimal deformation} We deal with two sets of variables:
\begin{itemize}
    \item \emph{Stress $\sigma_{ij}$}. $\dd{F}_i = \sigma_{ij} \dd{A}_j$.
    \item \emph{Strain $u_{ij}$}. For small deformation
    \begin{equation}
        u_{ij} = \frac{1}{2} \left(
            \pdv{u_i}{x_j} + \pdv{u_j}{x_i}
        \right).
    \end{equation}
    \item \emph{Constitutive relations}. $\sigma_{ij} = \sigma_{ij}[u_{ij}]$.
\end{itemize}

\paragraph*{Uniform isotropic linear medium} Constitutive relation  
\begin{equation}
    \sigma_{ik} = K u_{ll} \delta_{ik}
    + 2 \mu \left(
        u_{ik} - \frac{1}{3} \delta_{ik} u_{ll}
    \right).
    \label{eq:constitutive-relation}
\end{equation}

\paragraph*{Temperature expansion} The strain induced by temperature change:
\begin{equation}
    \dv{u}{x} = \alpha (T - T_0),
\end{equation}
where $T_0$ is the ``overall'' temperature.

\section{Uniform isotropic linear medium, in experiments} 

\paragraph*{Two modes of strain}
\begin{itemize}
    \item \emph{Compression/tension}. Along one direction (for example $z$):
    \begin{equation}
        \epsilon = \frac{\delta}{L} = u_{zz}.
    \end{equation}
    \item \emph{Shear}. On the $xy$ plane:
    \begin{equation}
        \gamma = \theta_{xx'} + \theta_{yy'} = 2 u_{xy}.
    \end{equation}
\end{itemize} 

\paragraph*{Young's modulus} Relation between tension and force:
\begin{equation}
    E = \frac{P}{\epsilon} = \frac{P L}{\delta} \Rightarrow
    F = PA = \frac{\delta}{L} \cdot EA.
\end{equation}

\paragraph*{Poisson's ratio} Relation between transverse strain and axial strain 
(in Young's modulus experiment):
\begin{equation}
    \sigma = \nu = - \dv{\epsilon_{\text{transverse}}}{\epsilon_{\text{axial}}}.
\end{equation}
This is how the material becomes thinner when stretched.

\paragraph*{Volume modulus} Relation between pressure and volume:
\begin{equation}
    K = - V \dv{P}{V}. 
\end{equation}
Here $K$ is that parameter in \eqref{eq:constitutive-relation}.

\paragraph*{Shear modulus} Relation between shear stress and shear strain:
\begin{equation}
    \mu = G = \frac{\tau}{\gamma}.
\end{equation}
Here $\tau$ is $\sigma_{xy}$ (or $yz$ or $zx$); $\gamma$ is the shear strain.


\paragraph*{How many independent parameters?} In isothermal process:  
\begin{equation}
    E = \frac{9 K \mu}{3 K + \mu}, \quad 
    \sigma = \frac{1}{2} \frac{3K - 2 \mu}{3 K + \mu}.
\end{equation}

\paragraph*{When is the linear elasticity condition broken?} 
\begin{enumerate}
    \item Linear region.
    \item Proportional limit.
    \item Elastic limit.
    \item Yield point.
    \item Ultimate tensile point.
    \item Breaking point.
\end{enumerate}


\section{Low dimension system: torsion of cylinder-like rod} 

\paragraph*{Reaction of $\varphi$ to torque} Here $T$ is the torque:
\begin{equation}
    \dv{\varphi}{z} = \frac{T(z)}{JG}, \quad 
    T(z) = \int_{0}^{z} \dd{z'} \dv{\ \mathrm{torque}}{z'}.
\end{equation}

\paragraph*{Relation between torque and stress}
\begin{equation}
    \gamma = \gamma_{xz} = \dv{\varphi}{z} r, \quad \tau = G \gamma, 
\end{equation}
\begin{equation}
    \tau_\text{max} = G \dv{\varphi}{z} R 
    = \frac{T R}{J} .
\end{equation}
Here $R$ may also be written as $c$.

\paragraph*{Note on $J$} It's actually not moment of inertia! 

\section{Low dimension system: beam, or rod predominantly bended}



\section{Problems}

\begin{itemize}
    \item Using deformation to decide forces (that otherwise can't be determined).
    \item How fast a shaft can rotate: $P = T \omega$. Then $\tau_{\text{max}}$ can be found.
\end{itemize}

\end{document}