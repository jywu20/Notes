\documentclass[hyperref, a4paper]{article}

\usepackage{geometry}
\usepackage{titling}
\usepackage{titlesec}
% No longer needed, since we will use enumitem package
% \usepackage{paralist}
\usepackage{enumitem}
\usepackage{footnote}
\usepackage{enumerate}
\usepackage{amsmath, amssymb, amsthm}
\usepackage{mathtools}
\usepackage{bbm}
\usepackage{graphicx}
\usepackage{subcaption}
\usepackage{soulutf8}
\usepackage{physics}
\usepackage{tensor}
\usepackage{siunitx}
\usepackage[version=4]{mhchem}
\usepackage{tikz}
\usepackage{xcolor}
\usepackage{listings}
\usepackage{autobreak}
\usepackage[ruled, vlined, linesnumbered]{algorithm2e}
\usepackage{nameref,zref-xr}
\zxrsetup{toltxlabel}
\usepackage[backend=bibtex]{biblatex}
\addbibresource{elasticity.bib}
\usepackage[colorlinks,unicode]{hyperref} % , linkcolor=black, anchorcolor=black, citecolor=black, urlcolor=black, filecolor=black
\usepackage[most]{tcolorbox}
\usepackage{prettyref}

% Page style
\geometry{left=3.18cm,right=3.18cm,top=2.54cm,bottom=2.54cm}
\titlespacing{\paragraph}{0pt}{1pt}{10pt}[20pt]
\setlength{\droptitle}{-5em}

% More compact lists 
\setlist[itemize]{
    itemindent=17pt, 
    leftmargin=1pt,
    listparindent=\parindent,
    parsep=0pt,
}

% Math operators
\DeclareMathOperator{\timeorder}{\mathcal{T}}
\DeclareMathOperator{\diag}{diag}
\DeclareMathOperator{\legpoly}{P}
\DeclareMathOperator{\primevalue}{P}
\DeclareMathOperator{\sgn}{sgn}
\DeclareMathOperator{\res}{Res}
\newcommand*{\ii}{\mathrm{i}}
\newcommand*{\ee}{\mathrm{e}}
\newcommand*{\const}{\mathrm{const}}
\newcommand*{\suchthat}{\quad \text{s.t.} \quad}
\newcommand*{\argmin}{\arg\min}
\newcommand*{\argmax}{\arg\max}
\newcommand*{\normalorder}[1]{: #1 :}
\newcommand*{\pair}[1]{\langle #1 \rangle}
\newcommand*{\fd}[1]{\mathcal{D} #1}
\DeclareMathOperator{\bigO}{\mathcal{O}}

% TikZ setting
\usetikzlibrary{arrows,shapes,positioning}
\usetikzlibrary{arrows.meta}
\usetikzlibrary{decorations.markings}
\tikzstyle arrowstyle=[scale=1]
\tikzstyle directed=[postaction={decorate,decoration={markings,
    mark=at position .5 with {\arrow[arrowstyle]{stealth}}}}]
\tikzstyle ray=[directed, thick]
\tikzstyle dot=[anchor=base,fill,circle,inner sep=1pt]

% Algorithm setting
% Julia-style code
\SetKwIF{If}{ElseIf}{Else}{if}{}{elseif}{else}{end}
\SetKwFor{For}{for}{}{end}
\SetKwFor{While}{while}{}{end}
\SetKwProg{Function}{function}{}{end}
\SetArgSty{textnormal}

\newcommand*{\concept}[1]{{\textbf{#1}}}

% Embedded codes
\lstset{basicstyle=\ttfamily,
  showstringspaces=false,
  commentstyle=\color{gray},
  keywordstyle=\color{blue}
}

% Reference formatting
\newcommand*{\citesec}[1]{\S~{#1}}
\newcommand*{\citechap}[1]{chap.~{#1}}
\newcommand*{\citefig}[1]{Fig.~{#1}}
\newcommand*{\citetable}[1]{Table~{#1}}
\newcommand*{\citepage}[1]{pp.~{#1}}
\newrefformat{fig}{Fig.~\ref{#1}}
\newcommand*{\term}[1]{\textit{#1}}

% Color boxes
\tcbuselibrary{skins, breakable, theorems}

\newtcbtheorem{infobox}{Box}{
    enhanced,
    boxrule=0pt,
    colback=blue!5,
    colframe=blue!5,
    coltitle=blue!50,
    borderline west={4pt}{0pt}{blue!65},
    sharp corners,
    fonttitle=\bfseries, 
    breakable,
    before upper={\parindent15pt\noindent}}{box}
\newtcbtheorem[use counter from=infobox]{theorybox}{Box}{
    enhanced,
    boxrule=0pt,
    colback=orange!5, 
    colframe=orange!5, 
    coltitle=orange!50,
    borderline west={4pt}{0pt}{orange!65},
    sharp corners,
    fonttitle=\bfseries, 
    breakable,
    before upper={\parindent15pt\noindent}}{box}
\newtcbtheorem[use counter from=infobox]{learnbox}{Box}{
    enhanced,
    boxrule=0pt,
    colback=green!5,
    colframe=green!5,
    coltitle=green!50,
    borderline west={4pt}{0pt}{green!65},
    sharp corners,
    fonttitle=\bfseries, 
    breakable,
    before upper={\parindent15pt\noindent}}{box}


\newenvironment{shelldisplay}{\begin{lstlisting}}{\end{lstlisting}}

\newcommand*{\kB}{k_{\text{B}}}
\newcommand*{\muB}{\mu_{\text{B}}}
\newcommand*{\efermi}{E_{\text{F}}}
\newcommand*{\pfermi}{p_{\text{F}}}
\newcommand*{\vfermi}{v_{\text{F}}}
\newcommand*{\sA}{\text{A}}
\newcommand*{\sB}{\text{B}}
\newcommand*{\Tc}{T_{\text{c}}}
\newcommand*{\hethree}{$^3$He}
\newcommand*{\hefour}{$^4$He}

\title{Elasticity}
\author{Jinyuan Wu}

\begin{document}

\maketitle

\section{The theoretical framework}

\subsection{The displacement field}

For now, we use $\vb*{r}$ to refer to the position vector of a position in a continuum,
and $\vb*{r}'(\vb*{r}, t)$ its corresponding position at time $t$ -- 
but note that in this note sometimes
we will need to use the symbol $\vb*{r}$ to refer to $\vb*{r}'$ 
(\prettyref{sec:lagrangian-and-eulerian}).
The displacement field is therefore 
\begin{equation}
    \vb*{u}(\vb*{r}, t) = \vb*{r}'(\vb*{r}, t) - \vb*{r}.
\end{equation}

A rough estimation of the number of degrees of freedom 
implies that all information about the material -- the position of each atom -- 
has already been stored in $\vb*{u}(\vb*{r}, t)$.
To see why, note that we can do Fourier transform to $\vb*{u}(\vb*{r}, t)$ 
in variable $\vb*{r}$.
Suppose the size of the system is $\sim L^d$,
where $d$ is the dimension of the system,
and the microscopic length scale of the system is $\sim a$.
The wave vector components of $k_{x, y, z}$ therefore are
confined to the sequence that starts with $0$ and ends with $2\pi / a$ 
(the microscopic cutoff),
with a step of $2\pi / L$;
the total length of this sequence is $\simeq L / a$,
and thus the total number of possible wave vectors is $(L / a)^d$.
Thus the number of real number variables included in the field variable $\vb*{u}(\vb*{r}, t)$ 
at a given time step 
is $d \cdot (L / a)^d$:
the prefactor $d$ comes from the $d$ components of $\vb*{u}$.

On the other hand, there are $\simeq (L / a)^d$ atoms in the system,
and each of them has $d$ directions of motion,
and therefore, we find that number of real number variables included in $\vb*{u}(\vb*{r}, t)$
is the same as the number of real number variables of the atoms,
and thus $\vb*{u}(\vb*{r}, t)$ contains all the information contained in the system.
This should not be surprising:
that we have a wave vector cutoff $\simeq 2\pi /a$ 
means we have a real space resolution of $\simeq a$,
so at each time step $t$, $\vb*{u}(\vb*{r}, t)$ 
can be completely described by a real space grid with 
the separation between the sample points being $\simeq a$ -- 
and the points in the grid is just equivalent to the initial positions of the atoms.

There are however some subtleties in the above argument.
If the material is not a crystal,
when $k \simeq 2\pi / a$,
the translational symmetry is already broken,
and therefore the wave vector is not well-defined any more;
thus the wave vector cutoff should be set to a much lower value.
If the system is an insulator, 
we can take $a$ to be the length scale of the primitive unit cell,
and $2\pi / a$ is the magnitude of the width of the first Brillouin zone.
If there is only one kind of atom,
then indeed $\vb*{u}(\vb*{r}, t)$ contains all information in the system;
but often we have more than one kind of atoms 
(two atoms that are not connected by any symmetry operations 
should be considered to be two kinds of atoms,
even when they are of the same species),
and then $\vb*{u}(\vb*{r}, t)$ only contains the acoustic phonon modes.

So in conclusion, 
the displacement field $\vb*{u}(\vb*{r}, t)$ only contains 
the acoustic modes,
both in crystals and non-crystals;
in both cases there may be additional hidden microscopic degrees of freedom.
These hidden degrees of freedom, 
like dislocation,
can be treated by introducing various discontinuities to $\vb*{u}(\vb*{r})$.

Below we often use $u_i$ to represent $\vb*{u}$, 
since we are going to deal with tensors frequently 
and it's a good idea to keep the notation the same
as that of higher order tensors.

\subsection{General guidelines of elastic dynamics}

Let's now move to the following question
of what the dynamics of an elastic solid system looks like.
In this section we progressively introduce assumptions on this topic.

\subsubsection{Dynamic variables: $u_i$ only}

The first assumption we introduce 
is \ul{\emph{the dynamics of the system should be able to be cast into a form that is only about 
$u_i$ and its conjugate momentum}}. 
Note that even when more dynamic variables are involved, 
we still can use, say, Mori-Zwanzig formalism 
to get a theory about $u_i$ only,
and thus in order for this approximation 
to be truly restrictive, 
we need to impose some implicit constraints on 
the formalism used when modeling the dynamics.
If we constrain the formalism to be Hamiltonian dynamics 
plus some dissipation terms,
then essentially we are assuming that the material is elastic (\prettyref{sec:elasticity-def}).

This assumption means the system we are investigating into 
doesn't have typical fluid behavior:
in Navier-Stokes equation, 
the density $\rho(\vb*{r}, t)$ is also a dynamic variable 
which can be used to decide $p$ on the RHS of the equation;%
\footnote{
    When the fluid is incompressible, 
    $p$ needs to be used in place of $\rho$ as a dynamic variable.
}
this isn't necessary in solid mechanics: 
even when we do have $\rho$ dependence in the EOMs, 
it should be able to be decided explicitly by, say, $\div{\vb*{u}}$.%
\footnote{
    Note that solid-like dynamics is not the logical opposition of fluid-like dynamics:
    it's possible to have systems with both solid-like and fluid-like behaviors.
    These materials are studied in rheology.
}

In other words, when $T = 0$, the Hamiltonian of the system is basically a Hamiltonian of phonons.
The quantum Hamiltonian of fluid,
on the other hand,
involves $\rho$ and $\vb*{v}$ (and a strange commutation relation between them) \cite{wiegmann2005}.

Sometimes we do need corrections to this assumption:
it's possible that long-range electromagnetic fields are created,
and in this case $\vb*{E}$ and $\vb*{B}$ should also be included into 
the dynamic variables (or otherwise we have memory effects, retardation effects, etc.).
The approximation therefore may be loosened to 
\ul{\emph{no degree of freedom other that $u_i$ that is about the atomic positions in the system 
is needed in the dynamics}}.

\subsubsection{Elasticity: statelessness and immediate response}\label{sec:elasticity-def}

Elasticity is sometimes defined as the follows:
\ul{\emph{when we stop applying force to the system,
it always goes back to one equilibrium shape}}.
In other words, 
if we push the system and then stop,
the system bounces back.

This definition is too weak:
it may be possible that after a loading-unloading process,
although the system goes back to its original shape,
its inner structure has changed 
and its reaction to another round of loading and unloading 
is different from that of the previous round.
So we need to impose a stronger formulation of elasticity:
\ul{\emph{the deformation of the system has nothing to do 
with its history 
and is completely decided by the force applied to it}}.
This formulation implies the weaker version:
when no load is present,
the system has one and only one configuration. 

The above formulation however has some ambiguity.
In experiments usually the force applied to the system 
is changed very slowly,
and the system is quasi-static.
In dynamics, however, we may be interested in oscillation,
where the force applied changes quickly,
and now it's possible that the response of the system is retarded;%
\footnote{
    Note that as is known in effective field theory,
    the response of the system to external driving force 
    and internal force should be the same,
    as long as the external driving force is coupled to the 
    deformation of the system in the same way 
    internal forces are coupled to the deformation system,
    and since we call external forces ``force'' 
    this should be true.
}
if we go to frequency space,
we may find that the response of the system has frequency dependence.
This is of course possible,
and we may still say that 
the deformation of the system is decided by the force applied to it.
Often, we make a even stronger assumption:
\ul{\emph{the deformation of the system in a particular moment 
is complete decided by the force applied to it in that very moment}}.
This means the system has immediate response to the force applied to it.

\subsubsection{Dissipation and finite temperature effects}

When $T = 0$, we may want to use a Hamiltonian formalism 
(a quantum many-body theory for phonons, actually)
to describe the elastic system;
this however is usually not sufficient 
since we have various dissipations:
it's inevitable that some weight 
flows into the hidden degrees of freedom
that we lose track of.%
\footnote{
    It's still possible that even the system does have a 
    dissipation-less Hamiltonian description,
    there is still damping behavior for a single mode,
    when the mixing between modes is too severe 
    (as in, say, Landau damping).
    Since all degrees of freedom are well kept track of,
    this is ``dissipation-less damping''.
}
An accurate formalism for dissipation is hard 
since it involves quantum master equation of the density matrix 
and dissipative quantum jump channels;
but in ordinary elastic theory,
usually we don't care about  what happens to the quantum many-body wave function,
and all we want to know is $\expval{u_i}$,
ignoring its higher order correlations.
Thus usually dissipation is just modeled 
by adding a term like $\partial_t u_i$ to the EOM.

Another issue is finite temperature effects. 
Here we do \emph{local equilibrium approximation}:
we assume that when $T \neq 0$,
there can be $T$ gradience in the system,
but \ul{\emph{at each given point, 
we have local equilibrium with the local temperature, the force and the deformation
being thermodynamic coordinates,
and the relaxation time is ignored}}.
Local equilibrium approximation is equivalent to the 
condition that \ul{\emph{$\omega \tau \ll 1$,
where $\omega$ is the characteristic frequency scale of the dynamic
and $\tau$ the relaxation time}}.
This corresponds to the $\omega \tau \ll 1$ limit in Fermi liquid kinetics,
where we have ordinary hydrodynamics and ordinary sound mode \cite{belitz2022soft}.
Again, the most accurate treatment of finite temperature effects involves 
quantum kinetic theory,
but in the $\omega \tau \ll 1$ limit 
the quantum kinetic theory is reduced to local thermodynamics.

\subsubsection{The form of EOM with local equilibrium}

Local equilibrium gives us well-defined 
local thermodynamic variables like the entropy density or the temperature;
still, we need to find some way to incorporate them 
into a dynamic theory.
The variables $u_i$ and the internal force density $f_i$ are not thermalized;
adding them into the thermodynamic potential 
in the local equilibrium state, and taking its variance, we get 
\begin{equation}
    \rho \pdv[2]{u_i}{t} = - \fdv{\Phi[\vb*{u}]}{u_i} + f^{\text{dissipation}}_i(\partial_t \vb*{u}, \partial_t^2 \vb*{u}, \ldots),
    \label{eq:eom-general-1}
\end{equation}
where $\Phi$ is the thermodynamic potential,
and the functional derivative is taken by 
assuming the conditions that make $\Phi$ the appropriate thermodynamic potential
(like, when $\Phi$ is $U$, 
we need to assume an adiabatic process, 
while when $\Phi$ is $F$,
we are working with an isothermal process).
Choosing the appropriate thermodynamic potential is important:
for statics, we usually assume local isothermal evolution,
while for things like sound propagation,
we assume local adiabatic evolution.
When $T \to 0$, $F$ becomes the total energy of the system 
and we go back to Hamiltonian dynamics.

Formally, here we insert the free energy into the Lagrangian
as the potential energy;
of course this raises the doubt about 
mixing mixed state and pure state formalisms,
but the formalism turns out to be correct
if we reason like above.

From a microscopic perspective, 
\eqref{eq:eom-general-1} can also be derived from a kinetic theory; 
as \cite{belitz2022soft} says, for example, 
conservation laws can be obtained from the quantum Boltzmann equation of Fermi liquid,
from which we get Navier-Stokes equations
or similar equations,
where the internal forces of the system 
can be obtained by taking the derivatives of certain thermodynamic potentials. 

It's kind of silly to spend so much time 
to get \eqref{eq:eom-general-1},
but we have seen that for many frequently seen materials,
we need to go beyond \eqref{eq:eom-general-1}:
when the system has strong fluid-like behavior,
when we have plastic deformation,
and when there is retardation.

\subsection{Stress}

\subsubsection{There are only surface forces}

For real elastic systems, 
the form of the EOM often can be further constrained.
First, \emph{$F[\vb*{u}]$ is usually local},
and thus we have 
\begin{equation}
    \fdv{F[\vb*{u}]}{u_i} = \pdv{f}{u_i}, \quad F = \int \dd[d]{\vb*{r}} f,
\end{equation}
where $f$ is the volume density of free energy;
in the rest of this note we simply use $F$ to refer to 
the volume density of free energy,
because $f$ may be confused with ``force''.
Moreover, we can assume that the length scale of the interaction 
between atoms 
is very small -- 
if we have an interaction channel with a large length scale 
then we just introduce the relevant part of electromagnetic field 
into dynamic variables -- 
and since the spatial resolution of $\vb*{u}$ is taken to be much larger than $a$,
effectively we can assume that 
\ul{\emph{the internal forces are all surface forces}},
and therefore we write down%
\footnote{
    The meaning of $\partial_t$ on LHS has a subtlety
    discussed immediately below.
}
\begin{equation}
    \rho \pdv[2]{u_i}{t} = \partial_{j} \sigma_{ji} + f_i^{\text{dissipation}},
    \label{eq:stress-eom-1}
\end{equation}
where we have replaced the internal force with 
the divergence of a second order tensor $\sigma_{ji}$,
so that
\begin{equation}
    \int_V \dd[d]{\vb*{r}} \partial_j \sigma_{ji}
    = \int_{\partial V} \dd[d-1]{s_j} \sigma_{ji}.
    \label{eq:surface-force}
\end{equation}
We can alternatively get $\sigma_{ji}$ by 
writing down the conservation equation of the momentum
and then $\sigma_{ji}$ is the ``current of momentum''.
We call $\sigma_{ij}$ the \concept{Cauchy stress tensor}.
From \eqref{eq:surface-force} we immediately get 
the boundary condition about force passing:
if the force acting on a surface element $\dd[d-1]{s_i}$ is 
$p_{i} \dd[d-1]{s_i}$,
then at the position of the surface element, we have 
\begin{equation}
    p_i = n_j \sigma_{ji},
    \label{eq:force-boundary-cond}
\end{equation}
where $n_j$ is the $j$th component of the normal vector $\vb*{n}$ of 
the surface element $\dd[d-1]{s_i}$,
and we have 
\begin{equation}
    \dd[d-1]{s_i} = n_i \dd{s}.
\end{equation}

There is however an important subtlety in \eqref{eq:stress-eom-1}.
When we take the time derivative on the LHS, 
we are doing so with $\vb*{u} = \vb*{u}(\vb*{r}, t)$,
and $\vb*{r}$ is kept unchanged when we change $t$
(recall that $\vb*{r}$ is actually the continuous version 
of the index of atoms in the system).
On the RHS, however, what I call $\vb*{r}$ is actually $\vb*{r}'$!
This can be easily found by looking at \eqref{eq:force-boundary-cond}:
definitely the direction of $\vb*{u}$ is decided 
by the surface \emph{after} deformation,
not before deformation.
Now since $\vb*{n}$ is defined in terms of $\vb*{r}'$,
then so is $\dd[d-1]{s_j}$ and hence $\dd[d]{\vb*{r}}$:
the last is just $\dd[d]{\vb*{r}'}$.

In order to keep the meaning of $\partial_t$ and $\partial_{\vb*{r}}$ the same 
on both sides of \eqref{eq:eom-general-1},
some remarks on the relation 
between $\vb*{u}(\vb*{r}, t)$ and $\vb*{u}(\vb*{r}', t)$ is needed.
This is done in the next section.

\subsubsection{Interlude: Lagrangian and Eulerian description}\label{sec:lagrangian-and-eulerian}

We say we are working with \concept{Lagrangian}% 
\footnote{
    Not to be confused with Lagrangian $\mathcal{L}$.
}
description of the system if we work with $\vb*{r}$
(i.e. the coordinates that are attached to the 
unstrained system) and $t$,
and \concept{Eulerian} description of the system 
if we work with $\vb*{r}'$ and $t$.

The time derivative in the Lagrangian framework 
(it appears on the LHS of \eqref{eq:eom-general-1} and \eqref{eq:stress-eom-1})
can be written in terms of Eulerian derivatives as follows:
\begin{equation}
    \begin{aligned}
        \left(\pdv{t}\right)_{\vb*{r}} &= \left(\pdv{t}\right)_{\vb*{r}'} +
        \left(\pdv{\vb*{r}'}{t}\right)_{\vb*{r}} \cdot \left(\pdv{\vb*{r}'}\right)_t \\
        &= \left(\pdv{t}\right)_{\vb*{r}'} +
        \underbrace{\left(\pdv{\vb*{u}}{t}\right)_{\vb*{r}}}_{\eqqcolon \vb*{v}} \cdot \left(\pdv{\vb*{r}'}\right)_t,
    \end{aligned}
\end{equation}
where $\vb*{v}$ can be evaluated directly from $\vb*{u}$ in the Lagrangian description
but is better regarded as a dynamic variable in the Eulerian description 
(as in Navier-Stokes equation).
People sometimes call $(\partial_t)_{\vb*{r}}$ 
the \concept{material derivative} 
and use a distinct symbol to represent it,
sometimes $\mathrm{D}/\mathrm{D}t$;
I will just use $\dv*{t}$,
the meaning of which is self-evident if we regard $\vb*{r}$ as a discrete index.

So the correct form of \eqref{eq:stress-eom-1},
in the Eulerian framework, is 
\begin{equation}
    \rho \left(
        \pdv{\vb*{v}}{t} + \vb*{v} \cdot \grad' \vb*{v}
    \right) = \pdv{\sigma_{ji}}{r_j'} + f_i^{\text{dissipation}},
\end{equation}
or, if we want to consistently work in the Eulerian framework,
we can just rename $\vb*{r}'$ as $\vb*{r}$
and ignore $\vb*{r}$,
and now the equation is 
\begin{equation}
    \rho \left(
        \pdv{\vb*{v}}{t} + \vb*{v} \cdot \grad \vb*{v}
    \right) = \pdv{\sigma_{ji}}{r_j} + f_i^{\text{dissipation}},
    \label{eq:eom-stress-2}
\end{equation}
and we also have the additional equation about the definition of $\vb*{v}$:
\begin{equation}
    \vb*{v} = \pdv{\vb*{u}}{t} + \vb*{v} \cdot \grad \vb*{u}.
    \label{eq:from-u-to-v}
\end{equation}
Note that the density is always the density changed by possible deformation
and not the original density:
we need the third equation -- just the mass conservation equation 
\begin{equation}
    \pdv{\rho}{t} + \div{\rho \vb*{v}} = 0
    \label{eq:rho-conserve}
\end{equation}
to complete the equation system about $\vb*{u}, \vb*{v}$ and $\rho$.

Note that the equation system also works for fluids;
we still need to assume that $\sigma_{ij}$ and $f_i^\text{dissipation}$
are only dependent to $\vb*{u}$ and its derivatives
to confine the equation system to the regime of solids. 

\eqref{eq:eom-stress-2}, \eqref{eq:from-u-to-v} and \eqref{eq:rho-conserve}
contain an additional variable $\vb*{v}$.
We can alternatively describe everything within the Lagrangian framework
and get rid of the $\vb*{v}$ variable,
but this means we need another stress tensor defined in the Lagrangian framework
that's different from the Cauchy stress tensor
(which is defined in the Eulerian framework -- 
see the comments at the end of the last section)
which is hard to conceptualize intuitively,
so I just skip this part.
In practice, however, the above formalisms are almost useless for elastic theory,
because $\vb*{u}$ almost will always be small;
when $\vb*{u}$ is not that small, 
usually we are just dealing with very complicated static problem 
and minimizing the free energy should be enough.

From now on, most of the time we will work in the Eulerian framework.

\subsubsection{Is the Cauchy stress tensor symmetric?}

An additional constraint on $\sigma_{ij}$ that is not necessary 
but works for most elastic bodies is that 
\ul{\emph{$\sigma_{ij}$ is symmetric}}.
This constraint is equivalent to the condition 
that \ul{\emph{there is no body torsion}}:
the torsion on a volume is 
\begin{equation}
    \begin{aligned}
        M_{ij} &= \int \dd[d]{\vb*{r}} 
        \left(
            r_i \partial_k \sigma_{kj} - \partial_k \sigma_{ki} r_j
        \right) \\
        &= \int \dd[d]{\vb*{r}} \left(
            \partial_k(r_i \sigma_{kj}) - \sigma_{kj} \delta_{ik}
            - \partial_k (r_j \sigma_{ki}) + \sigma_{ki} \delta_{jk}
        \right) \\
        &= \int \dd[d-1]{s_k} (r_i \sigma_{kj} - r_j \sigma_{ki})
        - \int \dd[d]{\vb*{r}} (\sigma_{ij} - \sigma_{ji}),
    \end{aligned}
\end{equation}
and it can be seen that the first term is just the torsion caused 
by the forces applied to the surface of this volume;
if we assume that there is no body torsion,
then the second term should always be zero regardless 
of the exact shape and position of the volume,
and this condition is equivalent to 
\begin{equation}
    \sigma_{ij} = \sigma_{ji}.
    \label{eq:sigma-sym}
\end{equation}
Of course, we don't have body torsion in an equilibrium system,
since if we have, 
then a part of the system will begin to rotate.
However we do have body torsion sometimes: 
in a liquid crystal system, for example, 
if a group of rod-like molecules get an angle difference 
from the rods near them,
they will be pulled back, 
and in this \emph{transient} process a torsion is created.
Another case is when macroscopic electromagnetic field is present: 
it may be able to apply a torque to a volume element, 
and in order to have equilibrium, 
$\sigma_{ij}$ has to be different with $\sigma_{ji}$.

Essentially, whether we have \eqref{eq:sigma-sym} 
is equivalent to whether a volume element in the system 
can be seen as a point mass, 
i.e. a degree of freedom with only $\vb*{r}, \vb*{p}$ variables 
and without any internal state.
In the cases above where \eqref{eq:sigma-sym} breaks down, 
a volume element has internal angular momentum and 
can no longer be treated as a point mass; 
we can of course go to smaller length scales 
so that the internal degrees of freedom are reduced, 
but then it's possible that we can no longer treat 
interaction between volume elements as surface forces. 
Of course, the internal angular momentum always exists in reality,
so by saying that the internal angular momentum of a volume element is not important,
we are hinting at decoupling between 
macroscopic dynamics and the dynamics of the internal angular momentum:
the latter always reaches equilibrium quickly enough.

Since this note is about quasi-static behaviors of ordinary elastic materials,
we will just adopt the assumption \eqref{eq:sigma-sym}.
It should be noted that the absence of \emph{local (or inherent) angular momentum} 
doesn't mean \emph{orbital angular momentum} is not considered: 
we know in a multiple point mass system, 
we can divide $\sum_{\text{all points}} \vb*{r} \times \vb*{p}$
into inherent angular momenta of its subsystems 
and orbital angular momenta of them; 
by ignoring the internal angular momentum, 
we are still able to treat the orbital angular momentum.

\subsubsection{Stress and free energy}

Now we explicitly derive the relation between the free energy density $F$
and the stress $\sigma_{ij}$.
In order to ``mold'' the configuration of the elastic body 
(without causing any real plastic deformation, of course),
we need to introduce a static external force $f_i$ such that 
\begin{equation}
    f_i + \partial_k \sigma_{ki} = 0.
\end{equation}
Essentially $f_i$ is the Lagrange multiplier 
to fix the deformation configuration of the system to what we want.
Thus when we change the deformation of the system,
work is done by $f_i$,
and we have (below we use $\var$ to refer to infinitesimal change 
caused by variance of the deformation,
and $\dd$ to refer to things like volume element 
that come from the coordinate system background)
\begin{equation}
    \var{U} = T \var{S} + f_i \var{u}_i, \quad 
    \var{F} = \var{U} - \var{TS} = - S \var{T} + f_i \var{u}_i.
\end{equation}
The total free energy changes as 
\begin{equation}
    \begin{aligned}
        \var \int F \dd[d]{\vb*{r}} &= - \int S \var{T} \dd[d]{\vb*{r}} 
        + \int \dd[d]{\vb*{r}} f_i \var{u_i}  \\
        &= - \int S \var{T} \dd[d]{\vb*{r}} 
        - \int \dd[d]{\vb*{r}} \partial_k \sigma_{ki} \var{u_i} \\
        &= - \int S \var{T} \dd[d]{\vb*{r}}
        - \oint \dd[d-1]{s_k} \sigma_{ki} \var{u_i}
        + \int \dd[d]{\vb*{r}} \sigma_{ki} \var (\partial_k u_i).
    \end{aligned}
\end{equation}
The three variables appearing after $\var$ -- the temperature in the bulk,
the displacement on the boundary,
and the gradient of the displacement in the bulk -- 
are all independent variables.
Thus, we find 
\begin{equation}
    \sigma_{ki} = \fdv{\partial_k u_i} \int F \dd[d]{\vb*{r}} = \pdv{F}{(\partial_k u_i)}.
\end{equation}
This is a very important finding:
it seems we need to focus on the gradient of $\vb*{u}$,
instead of $\vb*{u}$ itself;
and indeed this is true,
since we can trivially add a global translation to the system 
and have a very large $\vb*{u}$;
besides the global translation,
the global rotation component of $\partial_k u_i$, if any, is also trivial;
to get rid of these we can simply require that $\vb*{u} = 0$ at $\infty$.

\subsection{Strain}

\subsubsection{Decomposition of deformation gradient}

\begin{equation}
    E_{ij} =
\end{equation}

It should be noted that there is no guarantee 
that the antisymmetric component of $\partial_i u_j$ is not involved 
in the expression of $F$.
For example, if we analyze the deformation by 
analyzing the change of the metric,
we may want to define another type of stress: 
\begin{equation}
    u_{ij} = \pdv{u_i}{x_j} + \pdv{u_j}{x_i} + \pdv{u_l}{x_i} \pdv{u_l}{x_j}
\end{equation}
and its third term can only be obtained by using $a_{il} a_{jl}$.
What we can say is the symmetric component of the deformation gradient,
when appearing in $F$,
is able to have an order that is lower than 
any other components.

TODO: transport theorem, etc.

TODO: from microscopic theory to this

\subsubsection{}

TODO: in $\vb*{r}$ or $\vb*{r}'$

\subsection{Summary}

Now I list what we get in the discussion above, 
and their conditions.
\begin{itemize}
    \item In a \ul{solid} the dynamics of the system is mainly about the 
        displacement field $\vb*{u}$ and its time derivative,
        possibly with other variables like the electromagnetic field.
    \item When \ul{the deformation of a system at a moment is completely decided by the force applied to 
    it at that very moment, 
    without any retardation or memory effect},
    we say the system is elastic.
    \item When \ul{the thermodynamic relaxation time is much smaller 
    than $1 / \omega$, with $\omega$ being the frequency scale we are interested in},
        we can assume local thermodynamic equilibrium
        and use the derivative of the free energy to find 
        the relation between deformation and internal force.
    \item When \ul{the interaction length scale is small enough}, 
    the internal interaction in a system is described by the Cauchy stress tensor $\sigma_{ij}$.
    Furthermore, when \ul{we don't consider any transient internal torsion}, 
    we have $\sigma_{ij} = \sigma_{ji}$.
    The Cauchy stress tensor can be obtained from the free energy 
    by taking its derivative with respect to the deformation gradient.
    \item The Cauchy stress tensor $\sigma_{ij}$ guides the time evolution of the system 
    according to \eqref{eq:eom-stress-2}, \eqref{eq:from-u-to-v} and \eqref{eq:rho-conserve}.
    Note that the form of the equation system,
    without constraints over what $\sigma_{ij}$ depends to,
    also works for fluids.
    When \ul{the deformation is small},
    terms like $\vb*{v}$ can be removed in the equations.
\end{itemize}

All underlined sentences above have well-known counterexamples.
\begin{itemize}
    \item In a liquid the density is a very important dynamic variable,
    so it's quite common that $\vb*{u}$ isn't the only dynamic variable.
    \item Plastic deformation is common: 
    the deformation history changes the internal degrees of freedom of the system. 
    \item Non-equilibrium behaviors can be seen in, say, 
    zero sound modes of Fermi liquid.
    \item We may work in a very small length scale 
    where the internal forces in a system can no longer be described 
    as surface forces.
    \item Large deformation does exist.
\end{itemize}

\section{Linear elasticity for bulk system}

Three-dimensional elastic systems 
generally have neither large $\vb*{u}$ (with no global translation or rotation, of course)
nor large strain.
This sometimes may not be the case for 2D or 1D systems.
Therefore, the elastic theory of three-dimensional systems 
is usually linear and therefore can be decided completely by symmetry.


\section{Statics of boards}

One subtlety is we've never used the concept of moment of force
when establishing a theory about the bulk state; 
the appearance of the moment in a board seems strange. 
The reason is, the force density $f$ along the edge 
is not enough to characterize the small-scale force distribution at the edge: 
besides $f$, which should be multiplied by a $\delta$ function 
when understood as a \emph{bulk} force density, 
we can also have terms proportional to $\delta'(\vb*{r} \cdot \vu*{n})$.
The latter case just gives us the boundary condition about the moment.
We can derive the boundary condition about the moment 
by putting the force density proportional to $\delta'$ 
into the bulk elastic equations; 
alternatively we can insert the $\delta'$ term into the free energy change, 
where we transfer the derivative to $\var{\zeta}$,
and therefore get a $\var{\pdv*{\zeta}{n}}$ term 
in the variance of the free energy, 
which again gives the boundary condition about the moment.

\subsection{Föppl–von Kármán equations for  large deflections of thin flat plates}

A brief overview of Föppl–von Kármán equations can be found 
in \citesec{14} in \cite{landau1986theory}.
The theory is based on the following assumptions,
many of which are faced with questions:
\begin{itemize}
    \item \ul{The von Kármán strain is linearly linked to the stress tensor.}
    One may expect this to be true since 
    the von Kármán strain is a natural generalization 
    of the engineering strain;
    but then we can also say that 
    since a high order term appears in $F$,
    similarly a high order term of the engineering strain can appear in $F$ as well.
    The linear relation between the von Kármán strain 
    is therefore not theoretically justified,
    and experimental demonstration of this constitutive relation 
    is hard to carry out.
\end{itemize}

\printbibliography

\end{document}