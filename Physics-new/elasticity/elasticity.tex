\documentclass[hyperref, a4paper]{article}

\usepackage{geometry}
\usepackage{titling}
\usepackage{titlesec}
% No longer needed, since we will use enumitem package
% \usepackage{paralist}
\usepackage{enumitem}
\usepackage{footnote}
\usepackage{enumerate}
\usepackage{amsmath, amssymb, amsthm}
\usepackage{mathtools}
\usepackage{bbm}
\usepackage{graphicx}
\usepackage{subcaption}
\usepackage{physics}
\usepackage{tensor}
\usepackage{siunitx}
\usepackage[version=4]{mhchem}
\usepackage{tikz}
\usepackage{xcolor}
\usepackage{listings}
\usepackage{autobreak}
\usepackage[ruled, vlined, linesnumbered]{algorithm2e}
\usepackage{nameref,zref-xr}
\zxrsetup{toltxlabel}
\usepackage[sorting=none]{biblatex}
\usepackage[colorlinks,unicode]{hyperref} % , linkcolor=black, anchorcolor=black, citecolor=black, urlcolor=black, filecolor=black
\usepackage[most]{tcolorbox}
\usepackage{prettyref}

% Page style
\geometry{left=3.18cm,right=3.18cm,top=2.54cm,bottom=2.54cm}
\titlespacing{\paragraph}{0pt}{1pt}{10pt}[20pt]
\setlength{\droptitle}{-5em}

% More compact lists 
\setlist[itemize]{
    itemindent=17pt, 
    leftmargin=1pt,
    listparindent=\parindent,
    parsep=0pt,
}

% Math operators
\DeclareMathOperator{\timeorder}{\mathcal{T}}
\DeclareMathOperator{\diag}{diag}
\DeclareMathOperator{\legpoly}{P}
\DeclareMathOperator{\primevalue}{P}
\DeclareMathOperator{\sgn}{sgn}
\DeclareMathOperator{\res}{Res}
\newcommand*{\ii}{\mathrm{i}}
\newcommand*{\ee}{\mathrm{e}}
\newcommand*{\const}{\mathrm{const}}
\newcommand*{\suchthat}{\quad \text{s.t.} \quad}
\newcommand*{\argmin}{\arg\min}
\newcommand*{\argmax}{\arg\max}
\newcommand*{\normalorder}[1]{: #1 :}
\newcommand*{\pair}[1]{\langle #1 \rangle}
\newcommand*{\fd}[1]{\mathcal{D} #1}
\DeclareMathOperator{\bigO}{\mathcal{O}}

% TikZ setting
\usetikzlibrary{arrows,shapes,positioning}
\usetikzlibrary{arrows.meta}
\usetikzlibrary{decorations.markings}
\tikzstyle arrowstyle=[scale=1]
\tikzstyle directed=[postaction={decorate,decoration={markings,
    mark=at position .5 with {\arrow[arrowstyle]{stealth}}}}]
\tikzstyle ray=[directed, thick]
\tikzstyle dot=[anchor=base,fill,circle,inner sep=1pt]

% Algorithm setting
% Julia-style code
\SetKwIF{If}{ElseIf}{Else}{if}{}{elseif}{else}{end}
\SetKwFor{For}{for}{}{end}
\SetKwFor{While}{while}{}{end}
\SetKwProg{Function}{function}{}{end}
\SetArgSty{textnormal}

\newcommand*{\concept}[1]{{\textbf{#1}}}

% Embedded codes
\lstset{basicstyle=\ttfamily,
  showstringspaces=false,
  commentstyle=\color{gray},
  keywordstyle=\color{blue}
}

% Reference formatting
\newrefformat{fig}{Fig.~\ref{#1}}
\newcommand*{\term}[1]{\textit{#1}}

% Color boxes
\tcbuselibrary{skins, breakable, theorems}

\newtcbtheorem{infobox}{Box}{
    enhanced,
    boxrule=0pt,
    colback=blue!5,
    colframe=blue!5,
    coltitle=blue!50,
    borderline west={4pt}{0pt}{blue!65},
    sharp corners,
    fonttitle=\bfseries, 
    breakable,
    before upper={\parindent15pt\noindent}}{box}
\newtcbtheorem[use counter from=infobox]{theorybox}{Box}{
    enhanced,
    boxrule=0pt,
    colback=orange!5, 
    colframe=orange!5, 
    coltitle=orange!50,
    borderline west={4pt}{0pt}{orange!65},
    sharp corners,
    fonttitle=\bfseries, 
    breakable,
    before upper={\parindent15pt\noindent}}{box}
\newtcbtheorem[use counter from=infobox]{learnbox}{Box}{
    enhanced,
    boxrule=0pt,
    colback=green!5,
    colframe=green!5,
    coltitle=green!50,
    borderline west={4pt}{0pt}{green!65},
    sharp corners,
    fonttitle=\bfseries, 
    breakable,
    before upper={\parindent15pt\noindent}}{box}


\newenvironment{shelldisplay}{\begin{lstlisting}}{\end{lstlisting}}

\newcommand*{\kB}{k_{\text{B}}}
\newcommand*{\muB}{\mu_{\text{B}}}
\newcommand*{\efermi}{E_{\text{F}}}
\newcommand*{\pfermi}{p_{\text{F}}}
\newcommand*{\vfermi}{v_{\text{F}}}
\newcommand*{\sA}{\text{A}}
\newcommand*{\sB}{\text{B}}
\newcommand*{\Tc}{T_{\text{c}}}
\newcommand*{\hethree}{$^3$He}
\newcommand*{\hefour}{$^4$He}

\title{Elasticity}
\author{Jinyuan Wu}

\begin{document}

\maketitle

\section{The theoretical framework}

\subsection{The displacement field}

We use $\vb*{r}$ to refer to the position vector of a position in a continuum,
and $\vb*{r}'(\vb*{r}, t)$ its corresponding position at time $t$.
The displacement field is therefore 
\begin{equation}
    \vb*{u}(\vb*{r}, t) = \vb*{r}'(\vb*{r}, t) - \vb*{r}.
\end{equation}

A rough estimation of the number of degrees of freedom 
implies that all information about the material -- the position of each atom -- 
has already been stored in $\vb*{u}(\vb*{r}, t)$.
To see why, note that we can do Fourier transform to $\vb*{u}(\vb*{r}, t)$ 
in variable $\vb*{r}$.
Suppose the size of the system is $\sim L^d$,
where $d$ is the dimension of the system,
and the microscopic length scale of the system is $\sim a$.
The wave vector components of $k_{x, y, z}$ therefore are
confined to the sequence that starts with $0$ and ends with $2\pi / a$ 
(the microscopic cutoff),
with a step of $2\pi / L$;
the total length of this sequence is $\simeq L / a$,
and thus the total number of possible wave vectors is $(L / a)^d$.
Thus the number of real number variables included in the field variable $\vb*{u}(\vb*{r}, t)$ 
at a given time step 
is $d \cdot (L / a)^d$:
the prefactor $d$ comes from the $d$ components of $\vb*{u}$.

On the other hand, there are $\simeq (L / a)^d$ atoms in the system,
and each of them has $d$ directions of motion,
and therefore, we find that number of real number variables included in $\vb*{u}(\vb*{r}, t)$
is the same as the number of real number variables of the atoms,
and thus $\vb*{u}(\vb*{r}, t)$ contains all the information contained in the system.
This should not be surprising:
that we have a wave vector cutoff $\simeq 2\pi /a$ 
means we have a real space resolution of $\simeq a$,
so at each time step $t$, $\vb*{u}(\vb*{r}, t)$ 
can be completely described by a real space grid with 
the separation between the sample points being $\simeq a$ -- 
and the points in the grid is just equivalent to the initial positions of the atoms.

There are however some subtleties in the above argument.
If the material is not a crystal,
when $k \simeq 2\pi / a$,
the translational symmetry is already broken,
and therefore the wave vector is not well-defined any more;
thus the wave vector cutoff should be set to a much lower value.
If the system is an insulator, 
we can take $a$ to be the length scale of the primitive unit cell,
and $2\pi / a$ is the magnitude of the width of the first Brillouin zone.
If there is only one kind of atom,
then indeed $\vb*{u}(\vb*{r}, t)$ contains all information in the system;
but often we have more than one kind of atoms 
(two atoms that are not connected by any symmetry operations 
should be considered to be two kinds of atoms,
even when they are of the same species),
and then $\vb*{u}(\vb*{r}, t)$ only contains the acoustic phonon modes.

So in conclusion, 
the displacement field $\vb*{u}(\vb*{r}, t)$ only contains 
the acoustic modes,
both in crystals and non-crystals;
in both cases there may be additional hidden microscopic degrees of freedom.

Below we often use $u_i$ to represent $\vb*{u}$, 
since we are going to deal with tensors frequently 
and it's a good idea to keep the notation the same
as that of higher order tensors.

\subsection{The kinetics of deformation}

Laws governing the time evolution of $\vb*{u}(\vb*{r}, t)$
are to be written in terms of $\partial_{\vb*{r}} \vb*{u}$,
$\partial_t \vb*{u}$, etc.
But there is a problem: 


TODO: transport theorem, etc.

TODO: from microscopic theory to this

\subsection{Dynamics of perfect elasticity}

We assume $u_i$ and its conjugate momentum are the only dynamic variables;
thus, once 

TODO: degree of freedom counting: does $u_i$ include optical phonons?

\subsection{Dissipation}

Dissipation comes from two sources.
The first kind of dissipation involves the aforementioned hidden degrees of freedom 
that we lose track of.


\subsection{Finite temperature}

The theory of elasticity at $T = 0$ is just 
the low-energy effective field theory about phonons.

When 

\end{document}