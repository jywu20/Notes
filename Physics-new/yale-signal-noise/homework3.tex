\documentclass[hyperref, a4paper]{article}

\usepackage{geometry}
\usepackage{titling}
\usepackage{titlesec}
% No longer needed, since we will use enumitem package
% \usepackage{paralist}
\usepackage{enumitem}
\usepackage{footnote}
\usepackage{enumerate}
\usepackage{amsmath, amssymb, amsthm}
\usepackage{mathtools}
\usepackage{bbm}
\usepackage{cite}
\usepackage{graphicx}
\usepackage{subfigure}
\usepackage{physics}
\usepackage{tensor}
\usepackage{siunitx}
\usepackage[version=4]{mhchem}
\usepackage{tikz}
\usepackage{xcolor}
\usepackage{listings}
\usepackage{autobreak}
\usepackage[ruled, vlined, linesnumbered]{algorithm2e}
\usepackage{nameref,zref-xr}
\zxrsetup{toltxlabel}
\usepackage[colorlinks,unicode]{hyperref} % , linkcolor=black, anchorcolor=black, citecolor=black, urlcolor=black, filecolor=black
\usepackage[most]{tcolorbox}
\usepackage{prettyref}

% Page style
\geometry{left=3.18cm,right=3.18cm,top=2.54cm,bottom=2.54cm}
\titlespacing{\paragraph}{0pt}{1pt}{10pt}[20pt]
\setlength{\droptitle}{-5em}

% More compact lists 
\setlist[itemize]{
    itemindent=17pt, 
    leftmargin=1pt,
    listparindent=\parindent,
    parsep=0pt,
}

% Math operators
\DeclareMathOperator{\timeorder}{\mathcal{T}}
\DeclareMathOperator{\diag}{diag}
\DeclareMathOperator{\legpoly}{P}
\DeclareMathOperator{\primevalue}{P}
\DeclareMathOperator{\sgn}{sgn}
\newcommand*{\ii}{\mathrm{i}}
\newcommand*{\ee}{\mathrm{e}}
\newcommand*{\const}{\mathrm{const}}
\newcommand*{\suchthat}{\quad \text{s.t.} \quad}
\newcommand*{\argmin}{\arg\min}
\newcommand*{\argmax}{\arg\max}
\newcommand*{\normalorder}[1]{: #1 :}
\newcommand*{\pair}[1]{\langle #1 \rangle}
\newcommand*{\fd}[1]{\mathcal{D} #1}
\DeclareMathOperator{\bigO}{\mathcal{O}}

% TikZ setting
\usetikzlibrary{arrows,shapes,positioning}
\usetikzlibrary{arrows.meta}
\usetikzlibrary{decorations.markings}
\tikzstyle arrowstyle=[scale=1]
\tikzstyle directed=[postaction={decorate,decoration={markings,
    mark=at position .5 with {\arrow[arrowstyle]{stealth}}}}]
\tikzstyle ray=[directed, thick]
\tikzstyle dot=[anchor=base,fill,circle,inner sep=1pt]

% Algorithm setting
% Julia-style code
\SetKwIF{If}{ElseIf}{Else}{if}{}{elseif}{else}{end}
\SetKwFor{For}{for}{}{end}
\SetKwFor{While}{while}{}{end}
\SetKwProg{Function}{function}{}{end}
\SetArgSty{textnormal}

\newcommand*{\concept}[1]{{\textbf{#1}}}

% Embedded codes
\lstset{basicstyle=\ttfamily,
  showstringspaces=false,
  commentstyle=\color{gray},
  keywordstyle=\color{blue}
}

% Reference formatting
\newrefformat{fig}{Figure~\ref{#1}}

% Color boxes
\tcbuselibrary{skins, breakable, theorems}
\newtcbtheorem[number within=section]{warning}{Warning}%
  {colback=orange!5,colframe=orange!65,fonttitle=\bfseries, breakable}{warn}
\newtcbtheorem[number within=section]{note}{Note}%
  {colback=green!5,colframe=green!65,fonttitle=\bfseries, breakable}{note}
\newtcbtheorem[number within=section]{info}{Info}%
  {colback=blue!5,colframe=blue!65,fonttitle=\bfseries, breakable}{info}

\newenvironment{shelldisplay}{\begin{lstlisting}}{\end{lstlisting}}

\title{Homework 3}
\author{Jinyuan Wu}

\begin{document}

\maketitle

\paragraph{Lecture 8, Exercise 3}

\paragraph{Solution} We have 
\[
    U = I_1 R = \frac{1}{C} \int \dd{t} I_2 = L \dv{I_3}{t},
\]
and 
\[
    I = I_1 + I_2 + I_3.
\]
In the frequency domain, we have 
\[
    U[\omega] = I_1[\omega] R = \frac{1}{C} \frac{1}{- \ii \omega} I_2 = L (- \ii \omega) I_3,
\]
and therefore 
\begin{equation}
    Z[\omega] = \frac{U[\omega]}{I_1[\omega] + I_2[\omega] + I_3[\omega]} 
    = \frac{1}{\frac{1}{R} - \ii \omega C - \frac{1}{\ii \omega L}}.
\end{equation}
So we have 
\begin{equation}
    \Re Z[\omega] = \frac{1}{R} \frac{1}{\frac{1}{R^2} + \left( \frac{1}{\omega L} - \omega C \right)^2},
\end{equation}
and 
\begin{equation}
    \Im Z[\omega] = - \left( \frac{1}{\omega L} - \omega C \right) \frac{1}{\frac{1}{R^2} + \left( \frac{1}{\omega L} - \omega C \right)^2}.
\end{equation}
The zero points of the denominator are given by 
\begin{equation}
    \omega = \omega_{1, 2} \coloneqq \frac{- \frac{\ii}{R} \pm \sqrt{- \frac{1}{R^2} + \frac{4C}{L}}}{2 C}.
\end{equation}

\paragraph{Problem} 

\begin{figure}
    \centering
    \begin{tikzpicture}[x=0.75pt,y=0.75pt,yscale=-1,xscale=1]
    %uncomment if require: \path (0,300); %set diagram left start at 0, and has height of 300
    
    %Shape: Arc [id:dp5331133978803468] 
    \draw  [draw opacity=0] (179.13,164.25) .. controls (175.44,163.61) and (172.51,155.1) .. (172.51,144.7) .. controls (172.51,134.08) and (175.56,125.44) .. (179.36,125.12) -- (179.58,144.7) -- cycle ; \draw  [color={rgb, 255:red, 155; green, 155; blue, 155 }  ,draw opacity=1 ] (179.13,164.25) .. controls (175.44,163.61) and (172.51,155.1) .. (172.51,144.7) .. controls (172.51,134.08) and (175.56,125.44) .. (179.36,125.12) ;  
    %Straight Lines [id:da12247143299984486] 
    \draw    (300.46,109.33) -- (319.83,109.33) ;
    %Straight Lines [id:da6109342480428939] 
    \draw    (319.83,109.33) -- (389.83,109.33) ;
    %Shape: Capacitor [id:dp8125842692023104] 
    \draw   (319.83,109.33) -- (319.83,143.9) (332.58,151.58) -- (307.08,151.58) (332.58,143.9) -- (307.08,143.9) (319.83,151.58) -- (319.83,186.15) ;
    %Shape: Capacitor [id:dp3820330659104405] 
    \draw   (389.83,109.33) -- (389.83,143.9) (402.58,151.58) -- (377.08,151.58) (402.58,143.9) -- (377.08,143.9) (389.83,151.58) -- (389.83,186.15) ;
    %Shape: Inductor (Air Core) [id:dp21252438201509394] 
    \draw   (389.83,186.15) -- (377.23,186.15) .. controls (377.23,191.11) and (374.72,195.15) .. (371.63,195.15) .. controls (368.54,195.15) and (366.03,191.11) .. (366.03,186.15) .. controls (366.03,191.11) and (363.52,195.15) .. (360.43,195.15) .. controls (357.34,195.15) and (354.83,191.11) .. (354.83,186.15) .. controls (354.83,191.11) and (352.32,195.15) .. (349.23,195.15) .. controls (346.14,195.15) and (343.63,191.11) .. (343.63,186.15) .. controls (343.63,191.11) and (341.12,195.15) .. (338.03,195.15) .. controls (334.94,195.15) and (332.43,191.11) .. (332.43,186.15) -- (319.83,186.15) ;
    %Straight Lines [id:da21200291272484972] 
    \draw    (389.83,109.33) -- (402.33,109.33) ;
    %Straight Lines [id:da041781966587811414] 
    \draw    (300.21,186.15) -- (319.83,186.15) ;
    %Straight Lines [id:da4699416797302651] 
    \draw    (389.83,186.15) -- (402.33,186.15) ;
    %Shape: Can [id:dp23647903681372484] 
    \draw   (115,125.1) -- (179.13,125.1) .. controls (182.37,125.1) and (185,133.87) .. (185,144.68) .. controls (185,155.49) and (182.37,164.25) .. (179.13,164.25) -- (115,164.25) .. controls (111.75,164.25) and (109.13,155.49) .. (109.13,144.68) .. controls (109.13,133.87) and (111.75,125.1) .. (115,125.1) .. controls (118.24,125.1) and (120.87,133.87) .. (120.87,144.68) .. controls (120.87,155.49) and (118.24,164.25) .. (115,164.25) ;
    %Straight Lines [id:da8686341750411] 
    \draw    (105.88,145.1) -- (196,145.1) ;
    %Straight Lines [id:da6849184867595193] 
    \draw [color={rgb, 255:red, 74; green, 144; blue, 226 }  ,draw opacity=1 ]   (131.63,145.1) -- (131.63,162.1) ;
    \draw [shift={(131.63,164.1)}, rotate = 270] [fill={rgb, 255:red, 74; green, 144; blue, 226 }  ,fill opacity=1 ][line width=0.08]  [draw opacity=0] (8.4,-2.1) -- (0,0) -- (8.4,2.1) -- cycle    ;
    %Shape: Arc [id:dp3677584369492668] 
    \draw  [draw opacity=0] (185.67,168.56) .. controls (183.44,173.05) and (180.54,175.77) .. (177.38,175.77) .. controls (170.33,175.77) and (164.63,162.34) .. (164.63,145.77) .. controls (164.63,129.2) and (170.33,115.77) .. (177.38,115.77) .. controls (180.54,115.77) and (183.44,118.49) .. (185.67,122.98) -- (177.38,145.77) -- cycle ; \draw  [color={rgb, 255:red, 80; green, 227; blue, 194 }  ,draw opacity=1 ] (185.67,168.56) .. controls (183.44,173.05) and (180.54,175.77) .. (177.38,175.77) .. controls (170.33,175.77) and (164.63,162.34) .. (164.63,145.77) .. controls (164.63,129.2) and (170.33,115.77) .. (177.38,115.77) .. controls (180.54,115.77) and (183.44,118.49) .. (185.67,122.98) ;  
    %Straight Lines [id:da5720848890832353] 
    \draw [color={rgb, 255:red, 80; green, 227; blue, 194 }  ,draw opacity=1 ]   (185.67,168.56) -- (188.49,164.92) ;
    \draw [shift={(189.71,163.34)}, rotate = 127.8] [fill={rgb, 255:red, 80; green, 227; blue, 194 }  ,fill opacity=1 ][line width=0.08]  [draw opacity=0] (12,-3) -- (0,0) -- (12,3) -- cycle    ;
    %Straight Lines [id:da08490537832946576] 
    \draw [color={rgb, 255:red, 74; green, 144; blue, 226 }  ,draw opacity=1 ]   (174.88,145.1) -- (203.75,145.1) ;
    \draw [shift={(205.75,145.1)}, rotate = 180] [fill={rgb, 255:red, 74; green, 144; blue, 226 }  ,fill opacity=1 ][line width=0.08]  [draw opacity=0] (12,-3) -- (0,0) -- (12,3) -- cycle    ;
    %Straight Lines [id:da5107008145824161] 
    \draw [color={rgb, 255:red, 74; green, 144; blue, 226 }  ,draw opacity=1 ]   (142.63,145.1) -- (142.63,162.1) ;
    \draw [shift={(142.63,164.1)}, rotate = 270] [fill={rgb, 255:red, 74; green, 144; blue, 226 }  ,fill opacity=1 ][line width=0.08]  [draw opacity=0] (8.4,-2.1) -- (0,0) -- (8.4,2.1) -- cycle    ;
    %Straight Lines [id:da5804823414578775] 
    \draw [color={rgb, 255:red, 74; green, 144; blue, 226 }  ,draw opacity=1 ]   (153.63,145.1) -- (153.63,162.1) ;
    \draw [shift={(153.63,164.1)}, rotate = 270] [fill={rgb, 255:red, 74; green, 144; blue, 226 }  ,fill opacity=1 ][line width=0.08]  [draw opacity=0] (8.4,-2.1) -- (0,0) -- (8.4,2.1) -- cycle    ;
    %Straight Lines [id:da2696125138845171] 
    \draw [color={rgb, 255:red, 208; green, 2; blue, 27 }  ,draw opacity=1 ][fill={rgb, 255:red, 208; green, 2; blue, 27 }  ,fill opacity=1 ]   (127.38,145.1) -- (153.63,145.1) ;
    \draw [shift={(155.63,145.1)}, rotate = 180] [fill={rgb, 255:red, 208; green, 2; blue, 27 }  ,fill opacity=1 ][line width=0.08]  [draw opacity=0] (12,-3) -- (0,0) -- (12,3) -- cycle    ;
    %Straight Lines [id:da2708166657474578] 
    \draw [color={rgb, 255:red, 208; green, 2; blue, 27 }  ,draw opacity=1 ][fill={rgb, 255:red, 208; green, 2; blue, 27 }  ,fill opacity=1 ]   (125.38,125.1) -- (151.63,125.1) ;
    \draw [shift={(153.63,125.1)}, rotate = 180] [fill={rgb, 255:red, 208; green, 2; blue, 27 }  ,fill opacity=1 ][line width=0.08]  [draw opacity=0] (12,-3) -- (0,0) -- (12,3) -- cycle    ;
    %Shape: Path Data [id:dp48367091522979666] 
    \draw  [draw opacity=0][fill={rgb, 255:red, 155; green, 155; blue, 155 }  ,fill opacity=0.2 ] (335.46,101.52) -- (335.46,173.44) -- (386.71,173.44) -- (386.71,198.27) -- (303.71,198.27) -- (303.71,101.52) -- (335.46,101.52) -- cycle ;
    
    % Text Node
    \draw (305.08,143.9) node [anchor=east] [inner sep=0.75pt]    {$V_{n}$};
    % Text Node
    \draw (404.58,143.9) node [anchor=west] [inner sep=0.75pt]    {$V_{n+1}$};
    % Text Node
    \draw (148.63,134.94) node [anchor=west] [inner sep=0.75pt]  [color={rgb, 255:red, 208; green, 2; blue, 27 }  ,opacity=1 ]  {$I_{1}$};
    % Text Node
    \draw (142.63,167.5) node [anchor=north] [inner sep=0.75pt]  [color={rgb, 255:red, 74; green, 144; blue, 226 }  ,opacity=1 ]  {$E_{\text{static}}$};
    % Text Node
    \draw (191.71,163.34) node [anchor=west] [inner sep=0.75pt]  [color={rgb, 255:red, 74; green, 144; blue, 226 }  ,opacity=1 ]  {$\textcolor[rgb]{0.31,0.89,0.76}{B}$};
    % Text Node
    \draw (217.88,144.68) node [anchor=west] [inner sep=0.75pt]  [color={rgb, 255:red, 74; green, 144; blue, 226 }  ,opacity=1 ]  {$E_{\text{ind}}$};
    % Text Node
    \draw (140.67,218.19) node [anchor=north west][inner sep=0.75pt]   [align=left] {(a)};
    % Text Node
    \draw (346,218.19) node [anchor=north west][inner sep=0.75pt]   [align=left] {(b)};
    % Text Node
    \draw (126.88,112.44) node [anchor=west] [inner sep=0.75pt]  [color={rgb, 255:red, 208; green, 2; blue, 27 }  ,opacity=1 ]  {$I_{2}$};
    % Text Node
    \draw (404.33,109.33) node [anchor=west] [inner sep=0.75pt]    {$I_{2}$};
    % Text Node
    \draw (404.33,186.15) node [anchor=west] [inner sep=0.75pt]    {$I_{1}$};
    
    
    \end{tikzpicture}
    
    \caption{Transmission line}
    \label{fig:transmission}
\end{figure}

\paragraph{Solution} 
\begin{itemize}
\item[(a)] The transmission line model is still a quasi-magnetostatic model.
Ignoring all resistance, and assuming that 
the only conducting objects in the system 
are the line at $r = 0$ and the boundary of the transmission line
(or otherwise we need several ``vertical layers'' in the effective circuit model),
that the inner structures of which can be ignored
(or otherwise there may be multiple modes of the system),
and that the boundary is grounded everywhere so there is no spatial variation of voltage on it
(or otherwise there has to be an inductor on the top line in \prettyref{fig:transmission}(b)),
the physics of a line element is shown in \prettyref{fig:transmission}(a):
The current along the $x$ direction creates a magnetic field, 
which induces an electric field component along the $x$ axis 
because $\curl{\vb*{E}} = - \pdv*{\vb*{B}}{t}$.
We also have a radial component of $\vb*{E}$ 
because of the electrostatic field established between the surface and the inner conducting line.

Thus, in \prettyref{fig:transmission}(a),
we have two electric current degrees of freedom (one at $r=0$, the other on the boundary),
one electric field degrees of freedom between the central line and the boundary,
one magnetic field degree of freedom 
which gives $\vb*{E}_{\text{ind}}$ along the $x$ direction.
The distribution of the magnetic field is completely decided by the currents 
so we only need one scalar -- the magnetic flux -- to represent the magnetic field,
and similarly we only need two scalars -- the voltage on the central conducting line 
and the voltage between the $r=0$ line and the surface -- 
to represent $\vb*{E}_{\text{ind}}$ and $\vb*{E}_{\text{static}}$.
The resulting theory for the line element in \prettyref{fig:transmission}(a) 
has the same equations of motion with the shaded area in \prettyref{fig:transmission}(b).

Suppose the parameters of the capacitor and the inductor in \prettyref{fig:transmission}(b)
are $C$ and $L$, respectively.
Note that $\varphi(x, t)$ is defined on the capacitors, not the inductors,
and the variable doesn't represent the magnetic flux.

By basic circuit analysis we know 
the electromotive force on the inductor in \prettyref{fig:transmission}(b) is
\[
    \mathcal{E} = V_{n+1} - V_{n},
\]
and according to Faraday's law of induction, the electromotive force on the inductor is 
(here the ``inductor'' is single-turned, which can be found by looking at \prettyref{fig:transmission}(a))
\[
    \mathcal{E} = - \dv{\Phi}{t},
\]
so the current is 
\[
    I = \frac{1}{L} \Phi = - \frac{1}{L} \int \dd{t} \mathcal{E} = \frac{1}{L} (\varphi_n - \varphi_{n+1}) ,
\]
so the energy is 
\begin{equation}
    \frac{1}{2} L I^2 = \frac{1}{2L} (\varphi_n - \varphi_{n+1})^2.
    \label{eq:magnetic-energy}
\end{equation}
The energy of $C_n$ is by definition
\begin{equation}
    \frac{1}{2} C V_n^2 = \frac{1}{2} C \dot{\varphi_n}^2.
    \label{eq:electric-energy}
\end{equation}
Tracing the origin of \eqref{eq:magnetic-energy} and \eqref{eq:electric-energy},
we find they eventually come from the magnetic part and the electric part of electromagnetic energy,
and since in the electromagnetism Lagrangian,
they are recognized as the kinetic energy part and the potential energy part, respectively,
in our effective model the same should be done, 
so the Lagrangian of the shaded part in \prettyref{fig:transmission}(b) is
\begin{equation}
    L_{n} = \frac{1}{2} C \dot{\varphi}_n^2 - \frac{1}{2 L} (\varphi_n - \varphi_{n+1})^2.
\end{equation}
So the total Lagrangian is 
\begin{equation}
    \begin{aligned}
        L &= \sum_n L_n = 
        \sum_n \left( \frac{1}{2} C \dot{\varphi}_n^2 - \frac{1}{2 L} (\varphi_n - \varphi_{n+1})^2 \right) 
        = \sum \frac{\Delta{x}}{a} 
        \left( \frac{1}{2} C \dot{\varphi}_n^2 - \frac{1}{2 L} (\varphi_n - \varphi_{n+1})^2 \right) \\
        &= \sum \Delta{x} \left(
            \frac{1}{2} \frac{C}{a} \dot{\varphi_n}^2 
            - \frac{1}{2 L / a} 
            \frac{(\varphi_n - \varphi_{n+1})^2}{a^2}  
        \right) \\
        &= \int \dd{x} \left(
            \frac{1}{2} c (\partial_t \varphi)^2 - \frac{1}{2 l} (\grad{\varphi})^2
        \right).
    \end{aligned}
\end{equation}
The dynamic equation is 
\[
    \partial_t \pdv{\mathcal{L}}{\partial_t \varphi} + \div \pdv{\mathcal{L}}{\grad{\varphi}} = 0,
\]
\begin{equation}
    c \pdv[2]{\varphi}{t} = \frac{1}{l} \laplacian \varphi.
    \label{eq:telegraph-eq}
\end{equation}

\item[(b)] By definition, 
\begin{equation}
    \pi(x, t) = \pdv{\mathcal{L}}{\dot{\varphi}} = c \dot{\varphi},
\end{equation}
and the Hamiltonian is 
\begin{equation}
    \begin{aligned}
        H &= \int \dd{x} (\pi \dot{\varphi} - \mathcal{L}) 
        = \int \dd{x} \left( \frac{1}{2} c \dot{\varphi}^2 + \frac{1}{2l} (\grad{\varphi})^2 \right) \\
        &= \int \dd{x} \left( \frac{1}{2c} \pi^2 + \frac{1}{2l} (\grad{\varphi})^2 \right) .
    \end{aligned}
\end{equation}
The EOM in Hamiltonian language is 
\begin{equation}
    \dot{\varphi} = \frac{1}{c} \pi, \quad 
    \dot{\pi} = - \fdv{H}{\varphi} = \frac{1}{l} \laplacian \varphi.
\end{equation}

\item[(c)] Suppose (I will readjust the definitions at the end of this subsection)
\begin{equation}
    \varphi(x, t) = \frac{1}{\sqrt{L}} \sum_k 
    (\underbrace{A_k \ee^{- \ii \omega_k t}}_{A_k(t)} \ee^{\ii k x}
    + \underbrace{A_k^* \ee^{\ii \omega_k t}}_{A^*_k(t)} \ee^{- \ii k x}).
\end{equation}
The momentum variable is 
\begin{equation}
    \pi(x, t) = c \dot{\varphi} 
    = \frac{c}{\sqrt{L}} \sum_k 
    (- \ii \omega_k \underbrace{A_k \ee^{- \ii \omega_k t}}_{A_k(t)} \ee^{\ii k x}
    + \ii \omega_k \underbrace{A_k^* \ee^{\ii \omega_k t}}_{A^*_k(t)} \ee^{- \ii k x}).
\end{equation}
So we have 
\[
    \begin{aligned}
        &\underbrace{\frac{1}{\sqrt{L}} \int \ee^{- \ii k x} \varphi(x, t)}_{\varphi_k(t)}
        = A_k(t) + A^*_{-k} , \\
        &\frac{1}{c} \underbrace{\frac{1}{\sqrt{L}} \int \ee^{- \ii k x} \pi(x, t)}_{\pi_k(t)}
        = - \ii \omega_k A_k(t) + \ii \omega_{-k} A_{-k}^*(t),
    \end{aligned}
\] 
where from \eqref{eq:telegraph-eq}, we have 
\begin{equation}
    \omega_k = \frac{\abs{k}}{\sqrt{lc}} = \omega_{-k}.
\end{equation}
So now $A_k(t)$ can be found from the above equation system:
\begin{equation}
    A_k(t) = \frac{1}{2} \left( \varphi_k - \frac{1}{\ii c \omega_k} \pi_k \right)
    = \frac{1}{2} \left( \varphi_k - \frac{Z}{\ii \abs{k}} \pi_k \right),
\end{equation}
and 
\begin{equation}
    A_k^*(t) = \frac{1}{2} \left( \varphi_{-k} + \frac{1}{\ii c \omega_k} \pi_{-k} \right)
    = \frac{1}{2} \left( \varphi_{-k} + \frac{Z}{\ii \abs{k}} \pi_{-k} \right).
\end{equation}
Now we compare this with $A^\leftarrow$ and $A^\rightarrow$.
We have 
\begin{equation}
    A^\leftarrow(x, t) = \frac{1}{2 \sqrt{Z}} (V(x, t) - Z I(x, t)).
\end{equation}
Circuit analysis tells us
\begin{equation}
    I_{n-1} - I_n = C \dv{V}{t},
\end{equation}
and its continuous version is 
\begin{equation}
    - \pdv{I}{x} = c \pdv{V}{t} = c \pdv[2]{\varphi}{t} = \pdv{\pi}{t}, 
\end{equation}
So we have 
\begin{equation}
    A^\leftarrow(x, t) = \frac{1}{2 \sqrt{Z}} \left( \pdv{\varphi}{t} + Z \int \dd{x'} \pdv{\pi(x', t)}{t} \right).
\end{equation}
After Fourier transformation, we have 
\begin{equation}
    A_k^\leftarrow(t) = \frac{1}{2 \sqrt{Z}} \pdv{t} \left( \varphi_k(t) + Z \frac{1}{\ii k} \pi_k(t) \right)
    = \frac{- \ii \omega_k}{2 \sqrt{Z}} \left( \varphi_k(t) + Z \frac{1}{\ii k} \pi_k(t) \right).
\end{equation}
Similarly, we have 
\begin{equation}
    A_k^\rightarrow(t) = \frac{1}{2 \sqrt{Z}} \pdv{t} \left( \varphi_k(t) + Z \frac{1}{\ii k} \pi_k(t) \right)
    = \frac{- \ii \omega_k}{2 \sqrt{Z}} \left( \varphi_k(t) - Z \frac{1}{\ii k} \pi_k(t) \right).
\end{equation}
So when $k > 0$, we have 
\begin{equation}
    \begin{aligned}
        A_k^\rightarrow(t) &= \frac{- \ii \omega_k}{\sqrt{Z}} A_k(t), \\
        A_k^\leftarrow(t) &= \frac{- \ii \omega_k}{\sqrt{Z}} A_{-k}^*(t),
    \end{aligned}
\end{equation}
and when $k < 0$, we have 
\begin{equation}
    \begin{aligned}
        A_k^\leftarrow(t) &= \frac{- \ii \omega_k}{\sqrt{Z}} A_k(t), \\
        A_k^\rightarrow(t) &= \frac{- \ii \omega_k}{\sqrt{Z}} A_{-k}^*(t).
    \end{aligned}
\end{equation}
From the definition, it can be seen that in the Fourier expansion of $A^{\rightarrow}(x, t)$,
$k$ and $\omega$ have the same sign,
while for $A^{\leftrightarrow}(x, t)$, $k$ and $x$ have the opposite sign,
so when $k>0$, $A_k^\rightarrow(t)$ has to be proportion to the $A_k(t)$ components
while $A_k^\leftarrow(t)$ has to be proportion to the $A_k^*(t)$ components.
So the above results are expected.

To make the definition of $A_k(t)$ closer to the definition of $A^{\leftrightarrow, \rightarrow}_k(t)$
so that the relation between 

\item[(d)] 
\item[(f)] The equation is 
(here $C_{\text{ext}}$ and $L_{\text{ext}}$ are used to avoid confusion with $C$ and $L$ in the transmission line;
note that if we regard the inductor as a power source, 
then its \emph{electromotive force} is $- L \dot{I}$,
while if we regard its as an ordinary component,
then its \emph{voltage} is $L \dot{I}$)
\begin{equation}
    \frac{q_{\text{ext}}}{C_{\text{ext}}} + L_{\text{ext}} \dv{I_{\text{ext}}}{t} = V_0(t) = V(0, t),
\end{equation}
and (following the notation in the lecture notes,
here $I_0$ is the current on the first inductor in the transmission line)
\begin{equation}
    \underbrace{I_0}_{I(0, t)} + \underbrace{\dv{q_0}{t}}_{C \dot{V}(0, t)} + I_{\text{ext}} = 0.
\end{equation}
Since $C = ca$, in the continuous limit, the second term in the equation can be thrown away.
(Which is equivalent to directly removing $C_0$ and starting the transmission line with 
the inductor instead of the capacitor).
So the EOM of the external circuit is 
\begin{equation}
    \frac{1}{C_{\text{ext}}} q_{\text{ext}} + L_{\text{ext}} \dv{I_{\text{ext}}}{t} = V(0, t),
    \quad I(0, t) + I_{\text{ext}} = 0.
\end{equation}
Following the same procedure in the lecture note,



\end{itemize}

\begin{figure}
    \centering
    \begin{tikzpicture}[x=0.75pt,y=0.75pt,yscale=-1,xscale=1]
    %uncomment if require: \path (0,322); %set diagram left start at 0, and has height of 322
    
    %Straight Lines [id:da11749834712937979] 
    \draw    (339.83,276.15) -- (409.83,276.15) ;
    %Shape: Capacitor [id:dp09598906024538234] 
    \draw   (339.83,160.04) -- (339.83,194.61) (352.58,202.29) -- (327.08,202.29) (352.58,194.61) -- (327.08,194.61) (339.83,202.29) -- (339.83,236.85) ;
    %Shape: Inductor (Air Core) [id:dp8899610885698173] 
    \draw   (409.83,129.33) -- (397.23,129.33) .. controls (397.23,134.3) and (394.72,138.33) .. (391.63,138.33) .. controls (388.54,138.33) and (386.03,134.3) .. (386.03,129.33) .. controls (386.03,134.3) and (383.52,138.33) .. (380.43,138.33) .. controls (377.34,138.33) and (374.83,134.3) .. (374.83,129.33) .. controls (374.83,134.3) and (372.32,138.33) .. (369.23,138.33) .. controls (366.14,138.33) and (363.63,134.3) .. (363.63,129.33) .. controls (363.63,134.3) and (361.12,138.33) .. (358.03,138.33) .. controls (354.94,138.33) and (352.43,134.3) .. (352.43,129.33) -- (339.83,129.33) ;
    %Straight Lines [id:da4427013233239925] 
    \draw    (409.83,129.33) -- (422.33,129.33) ;
    %Straight Lines [id:da8544967510982182] 
    \draw    (280.5,276.15) -- (339.83,276.15) ;
    %Straight Lines [id:da3980805643302381] 
    \draw    (409.83,276.15) -- (422.33,276.15) ;
    %Straight Lines [id:da021438736433621264] 
    \draw    (280.5,129.33) -- (339.83,129.33) ;
    %Shape: Capacitor [id:dp20286796618144542] 
    \draw   (280.5,129.33) -- (280.5,163.9) (293.25,171.58) -- (267.75,171.58) (293.25,163.9) -- (267.75,163.9) (280.5,171.58) -- (280.5,206.15) ;
    %Shape: Inductor (Air Core) [id:dp461246695780545] 
    \draw   (280.5,206.15) -- (280.5,218.75) .. controls (285.47,218.75) and (289.5,221.26) .. (289.5,224.35) .. controls (289.5,227.44) and (285.47,229.95) .. (280.5,229.95) .. controls (285.47,229.95) and (289.5,232.46) .. (289.5,235.55) .. controls (289.5,238.64) and (285.47,241.15) .. (280.5,241.15) .. controls (285.47,241.15) and (289.5,243.66) .. (289.5,246.75) .. controls (289.5,249.84) and (285.47,252.35) .. (280.5,252.35) .. controls (285.47,252.35) and (289.5,254.86) .. (289.5,257.95) .. controls (289.5,261.04) and (285.47,263.55) .. (280.5,263.55) -- (280.5,276.15) ;
    %Straight Lines [id:da3713429559344812] 
    \draw    (339.83,236.85) -- (339.83,276.15) ;
    %Straight Lines [id:da19554953394992203] 
    \draw    (339.83,129.33) -- (339.83,160.04) ;
    %Shape: Capacitor [id:dp008034011040416456] 
    \draw   (409.83,160.04) -- (409.83,194.61) (422.58,202.29) -- (397.08,202.29) (422.58,194.61) -- (397.08,194.61) (409.83,202.29) -- (409.83,236.85) ;
    %Straight Lines [id:da07456917566674814] 
    \draw    (409.83,236.85) -- (409.83,276.15) ;
    %Straight Lines [id:da1286310190144886] 
    \draw    (409.83,129.33) -- (409.83,160.04) ;
    %Straight Lines [id:da6895241562197532] 
    \draw    (249,147) -- (249,199.85) ;
    \draw [shift={(249,201.85)}, rotate = 270] [fill={rgb, 255:red, 0; green, 0; blue, 0 }  ][line width=0.08]  [draw opacity=0] (12,-3) -- (0,0) -- (12,3) -- cycle    ;
    %Straight Lines [id:da6652369709419201] 
    \draw    (348.43,116.48) -- (399.5,116.48) ;
    \draw [shift={(401.5,116.48)}, rotate = 180] [fill={rgb, 255:red, 0; green, 0; blue, 0 }  ][line width=0.08]  [draw opacity=0] (12,-3) -- (0,0) -- (12,3) -- cycle    ;
    %Rounded Rect [id:dp3218082028954583] 
    \draw  [draw opacity=0][fill={rgb, 255:red, 80; green, 227; blue, 194 }  ,fill opacity=0.2 ] (222.5,129.85) .. controls (222.5,123.78) and (227.42,118.85) .. (233.5,118.85) -- (296.5,118.85) .. controls (302.58,118.85) and (307.5,123.78) .. (307.5,129.85) -- (307.5,275.85) .. controls (307.5,281.93) and (302.58,286.85) .. (296.5,286.85) -- (233.5,286.85) .. controls (227.42,286.85) and (222.5,281.93) .. (222.5,275.85) -- cycle ;
    
    % Text Node
    \draw (456,190.4) node [anchor=north west][inner sep=0.75pt]    {$\cdots $};
    % Text Node
    \draw (327.08,191.61) node [anchor=south] [inner sep=0.75pt]   [align=left] {+};
    % Text Node
    \draw (327.08,205.29) node [anchor=north] [inner sep=0.75pt]   [align=left] {\mbox{-}};
    % Text Node
    \draw (267.75,160.9) node [anchor=south] [inner sep=0.75pt]   [align=left] {+};
    % Text Node
    \draw (268.5,227.75) node [anchor=south] [inner sep=0.75pt]   [align=left] {+};
    % Text Node
    \draw (267.75,174.58) node [anchor=north] [inner sep=0.75pt]   [align=left] {\mbox{-}};
    % Text Node
    \draw (271.08,252.29) node [anchor=north] [inner sep=0.75pt]   [align=left] {\mbox{-}};
    % Text Node
    \draw (247,174.43) node [anchor=east] [inner sep=0.75pt]    {$I$};
    % Text Node
    \draw (374.97,113.08) node [anchor=south] [inner sep=0.75pt]    {$I_{0}$};
    % Text Node
    \draw (354.58,194.61) node [anchor=west] [inner sep=0.75pt]    {$C_{0}$};
    % Text Node
    \draw (424.58,194.61) node [anchor=west] [inner sep=0.75pt]    {$C_{1}$};
    % Text Node
    \draw (397.08,191.61) node [anchor=south] [inner sep=0.75pt]   [align=left] {+};
    % Text Node
    \draw (397.08,205.29) node [anchor=north] [inner sep=0.75pt]   [align=left] {\mbox{-}};
    % Text Node
    \draw (352.43,137.33) node [anchor=north] [inner sep=0.75pt]   [align=left] {+};
    % Text Node
    \draw (397.23,137.33) node [anchor=north] [inner sep=0.75pt]   [align=left] {\mbox{-}};
    % Text Node
    \draw (262.47,111.85) node [anchor=south] [inner sep=0.75pt]  [color={rgb, 255:red, 80; green, 227; blue, 194 }  ,opacity=1 ] [align=left] {resonator};
    
    
    \end{tikzpicture}
    
    \caption{A serial $LC$ resonator and a transmission line}
    \label{fig:transmission-load}
\end{figure}

\end{document}