\documentclass[hyperref, a4paper]{article}

\usepackage{geometry}
\usepackage{titling}
\usepackage{titlesec}
% No longer needed, since we will use enumitem package
% \usepackage{paralist}
\usepackage{enumitem}
\usepackage{footnote}
\usepackage{enumerate}
\usepackage{amsmath, amssymb, amsthm}
\usepackage{mathtools}
\usepackage{bbm}
\usepackage{cite}
\usepackage{graphicx}
\usepackage{subfigure}
\usepackage{physics}
\usepackage{tensor}
\usepackage{siunitx}
\usepackage[version=4]{mhchem}
\usepackage{tikz}
\usepackage{xcolor}
\usepackage{listings}
\usepackage{autobreak}
\usepackage[ruled, vlined, linesnumbered]{algorithm2e}
\usepackage{nameref,zref-xr}
\zxrsetup{toltxlabel}
\usepackage[colorlinks,unicode]{hyperref} % , linkcolor=black, anchorcolor=black, citecolor=black, urlcolor=black, filecolor=black
\usepackage[most]{tcolorbox}
\usepackage{prettyref}

% Page style
\geometry{left=3.18cm,right=3.18cm,top=2.54cm,bottom=2.54cm}
\titlespacing{\paragraph}{0pt}{1pt}{10pt}[20pt]
\setlength{\droptitle}{-5em}

% More compact lists 
\setlist[itemize]{
    itemindent=17pt, 
    leftmargin=1pt,
    listparindent=\parindent,
    parsep=0pt,
}

% Math operators
\DeclareMathOperator{\timeorder}{\mathcal{T}}
\DeclareMathOperator{\diag}{diag}
\DeclareMathOperator{\legpoly}{P}
\DeclareMathOperator{\primevalue}{P}
\DeclareMathOperator{\sgn}{sgn}
\DeclareMathOperator{\res}{Res}
\newcommand*{\ii}{\mathrm{i}}
\newcommand*{\ee}{\mathrm{e}}
\newcommand*{\const}{\mathrm{const}}
\newcommand*{\suchthat}{\quad \text{s.t.} \quad}
\newcommand*{\argmin}{\arg\min}
\newcommand*{\argmax}{\arg\max}
\newcommand*{\normalorder}[1]{: #1 :}
\newcommand*{\pair}[1]{\langle #1 \rangle}
\newcommand*{\fd}[1]{\mathcal{D} #1}
\DeclareMathOperator{\bigO}{\mathcal{O}}

% TikZ setting
\usetikzlibrary{arrows,shapes,positioning}
\usetikzlibrary{arrows.meta}
\usetikzlibrary{decorations.markings}
\tikzstyle arrowstyle=[scale=1]
\tikzstyle directed=[postaction={decorate,decoration={markings,
    mark=at position .5 with {\arrow[arrowstyle]{stealth}}}}]
\tikzstyle ray=[directed, thick]
\tikzstyle dot=[anchor=base,fill,circle,inner sep=1pt]

% Algorithm setting
% Julia-style code
\SetKwIF{If}{ElseIf}{Else}{if}{}{elseif}{else}{end}
\SetKwFor{For}{for}{}{end}
\SetKwFor{While}{while}{}{end}
\SetKwProg{Function}{function}{}{end}
\SetArgSty{textnormal}

\newcommand*{\concept}[1]{{\textbf{#1}}}

% Embedded codes
\lstset{basicstyle=\ttfamily,
  showstringspaces=false,
  commentstyle=\color{gray},
  keywordstyle=\color{blue}
}

% Reference formatting
\newrefformat{fig}{Figure~\ref{#1}}

% Color boxes
\tcbuselibrary{skins, breakable, theorems}
\newtcbtheorem[number within=section]{warning}{Warning}%
  {colback=orange!5,colframe=orange!65,fonttitle=\bfseries, breakable}{warn}
\newtcbtheorem[number within=section]{note}{Note}%
  {colback=green!5,colframe=green!65,fonttitle=\bfseries, breakable}{note}
\newtcbtheorem[number within=section]{info}{Info}%
  {colback=blue!5,colframe=blue!65,fonttitle=\bfseries, breakable}{info}

\newenvironment{shelldisplay}{\begin{lstlisting}}{\end{lstlisting}}

\title{Squeezing of quantum noise}
\author{Jinyuan Wu}

\begin{document}
    
\maketitle

\section{Introduction}

Noise usually arises from coupling with an unknown external object:
this creates joint distribution of two objects 
(which in the quantum context means entanglement),
and when we keep our eyes only on one of them,
we need to average over the state of the other,
which corrects the theory with noise and dissipation \cite{zwanzig_nonequilibrium_2001}.
When the full quantum treatment is needed,
however, another type of noise appears:
the state itself does not have a definite value of the observable in question, 
and any experimental setting 
-- no matter how carefully the system is isolated from the environment -- 
has noisy results.
This kind of noise is called \concept{quantum noise}.

In systems that can be well described by 
the harmonic oscillator picture,
in which for each oscillation mode, 
we have two variables $X$ and $P$, and $[X, P] = \ii$,
and the Hamiltonian is $H \simeq c_1 X^2 + c_2 P^2$.%
\footnote{
    The Planck system of units is used in this report, 
    so we take $\hbar = 1$ if there is no special mention of the units.
}
After diagonalization, we get $H \simeq \sum_{\text{modes}} \omega (a^\dagger a + 1/2)$ 
plus possible interaction terms,
and this zero-point energy arises from the non-commutative nature of $X$ and $P$.
Another way to make sense of the $1/2$ term is to notice that at the ground state,
though $\expval{X} = \expval{P} = 0$,
and everything seems to be very definite,
$\expval{X^2} \simeq H / 2 \neq 0$.
Most of the time, the observable actually measured in such systems 
is $\simeq n = a^\dagger a \sim X^2$,
and quantum noise appears in the measurement.
People therefore sometimes say that 
quantum noise comes from the zero point-energy.

% We may also consult this:
% Quantum noise and quantum measurement
% https://clerkgroup.uchicago.edu/PDFfiles/LesHouchesNotesAC.pdf

This report focuses on quantum noise in systems mentioned above.
This includes linear optics and micromechanics.
\prettyref{sec:overview-rep} discusses more quantitative ways to represent and characterize quantum noise.
\prettyref{sec:interferometer} calculates quantum noise 
in a prevalent phase measurement scheme using interferometer.
\prettyref{sec:squeezing} shows how to squeeze the quantum noise.

\section{Overview of representation of states in quantum optics}\label{sec:overview-rep}

\section{Quantum noise in the Mach-Zehnder interferometer}\label{sec:interferometer}



To have a concrete example of quantum noise, 
let us move to the Mach-Zehnder interferometer.

We are going to work in the Heisenberg picture. 
Since this section is just to exemplify the overall idea of quantum noise,
for the sake of simplicity we assume the time evolution operator of the beam splitter is 
\begin{equation}
    S_{\text{beam splitter}} = \frac{1}{\sqrt{2}} \pmqty{ 1 & -1 \\ 1 & 1 } .
\end{equation}
The time evolution operator of the whole system is therefore
\begin{equation}
    \begin{aligned}
        S_\text{total}(\varphi) &= \frac{1}{\sqrt{2}} \pmqty{ 1 & 1 \\ 1 & -1 } \cdot \pmqty{\dmat{\ee^{\ii \varphi}, \ee^{- \ii \varphi}}} \cdot \frac{1}{\sqrt{2}} \pmqty{ 1 & -1 \\ 1 & 1 } \\
        &= \pmqty{ \cos \varphi & - \ii \sin \varphi \\ \ii \sin \varphi & - \cos \varphi },
    \end{aligned}
    \label{eq:total-matrix}
\end{equation} 
and therefore 
\[
    \pmqty{b_1^\dagger \\ b_2^\dagger} = \pmqty{ \cos \varphi & - \ii \sin \varphi \\ \ii \sin \varphi & - \cos \varphi } \pmqty{a_1^\dagger \\ a_2^\dagger},
\]
from which we find 
\begin{equation}
    \pmqty{a_1^\dagger \\ a_2^\dagger} = \pmqty{ \cos \varphi & - \ii \sin \varphi \\ \ii \sin \varphi & - \cos \varphi } \pmqty{b_1^\dagger \\ b_2^\dagger}.
\end{equation}
In actual experiment settings,
usually a beam of laser is injected into input port 1,
so the state at input port 1 is a coherent state.
Then, the wave function is 
\begin{equation}
    \begin{aligned}
        \ket{\psi} &= \ee^{\alpha a_1 - \alpha^* a_1^\dagger} \\
        &= \ee^{\alpha (\cos \varphi b_1^\dagger - \ii \sin \varphi b_2^\dagger) -\alpha^{*} (\cos \varphi b_1 + \ii \sin \varphi b_2)}|0\rangle \\
        &= \ket{b_1 = \alpha \cos \varphi, b_2 = - \ii \alpha \sin \varphi} .
    \end{aligned}
    \label{eq:out-state}
\end{equation}

The expectation value of 

\section{Squeezing the quantum noise}\label{sec:squeezing}



\bibliographystyle{plain}
\bibliography{squeezing}

\end{document}