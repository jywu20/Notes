\documentclass[hyperref, a4paper]{article}

\usepackage{geometry}
\usepackage{titling}
\usepackage{titlesec}
% No longer needed, since we will use enumitem package
% \usepackage{paralist}
\usepackage{enumitem}
\usepackage{footnote}
\usepackage{enumerate}
\usepackage{amsmath, amssymb, amsthm}
\usepackage{mathtools}
\usepackage{bbm}
\usepackage{cite}
\usepackage{graphicx}
\usepackage{subfigure}
\usepackage{physics}
\usepackage{tensor}
\usepackage{siunitx}
\usepackage[version=4]{mhchem}
\usepackage{tikz}
\usepackage{xcolor}
\usepackage{listings}
\usepackage{autobreak}
\usepackage[ruled, vlined, linesnumbered]{algorithm2e}
\usepackage{nameref,zref-xr}
\zxrsetup{toltxlabel}
\usepackage[colorlinks,unicode]{hyperref} % , linkcolor=black, anchorcolor=black, citecolor=black, urlcolor=black, filecolor=black
\usepackage[most]{tcolorbox}
\usepackage{prettyref}

% Page style
\geometry{left=3.18cm,right=3.18cm,top=2.54cm,bottom=2.54cm}
\titlespacing{\paragraph}{0pt}{1pt}{10pt}[20pt]
\setlength{\droptitle}{-5em}

% More compact lists 
\setlist[itemize]{
    itemindent=17pt, 
    leftmargin=1pt,
    listparindent=\parindent,
    parsep=0pt,
}

% Math operators
\DeclareMathOperator{\timeorder}{\mathcal{T}}
\DeclareMathOperator{\diag}{diag}
\DeclareMathOperator{\legpoly}{P}
\DeclareMathOperator{\primevalue}{P}
\DeclareMathOperator{\sgn}{sgn}
\DeclareMathOperator{\res}{Res}
\newcommand*{\ii}{\mathrm{i}}
\newcommand*{\ee}{\mathrm{e}}
\newcommand*{\const}{\mathrm{const}}
\newcommand*{\suchthat}{\quad \text{s.t.} \quad}
\newcommand*{\argmin}{\arg\min}
\newcommand*{\argmax}{\arg\max}
\newcommand*{\normalorder}[1]{: #1 :}
\newcommand*{\pair}[1]{\langle #1 \rangle}
\newcommand*{\fd}[1]{\mathcal{D} #1}
\DeclareMathOperator{\bigO}{\mathcal{O}}

% TikZ setting
\usetikzlibrary{arrows,shapes,positioning}
\usetikzlibrary{arrows.meta}
\usetikzlibrary{decorations.markings}
\tikzstyle arrowstyle=[scale=1]
\tikzstyle directed=[postaction={decorate,decoration={markings,
    mark=at position .5 with {\arrow[arrowstyle]{stealth}}}}]
\tikzstyle ray=[directed, thick]
\tikzstyle dot=[anchor=base,fill,circle,inner sep=1pt]

% Algorithm setting
% Julia-style code
\SetKwIF{If}{ElseIf}{Else}{if}{}{elseif}{else}{end}
\SetKwFor{For}{for}{}{end}
\SetKwFor{While}{while}{}{end}
\SetKwProg{Function}{function}{}{end}
\SetArgSty{textnormal}

\newcommand*{\concept}[1]{{\textbf{#1}}}

% Embedded codes
\lstset{basicstyle=\ttfamily,
  showstringspaces=false,
  commentstyle=\color{gray},
  keywordstyle=\color{blue}
}

% Reference formatting
\newrefformat{fig}{Figure~\ref{#1}}

% Color boxes
\tcbuselibrary{skins, breakable, theorems}
\newtcbtheorem[number within=section]{warning}{Warning}%
  {colback=orange!5,colframe=orange!65,fonttitle=\bfseries, breakable}{warn}
\newtcbtheorem[number within=section]{note}{Note}%
  {colback=green!5,colframe=green!65,fonttitle=\bfseries, breakable}{note}
\newtcbtheorem[number within=section]{info}{Info}%
  {colback=blue!5,colframe=blue!65,fonttitle=\bfseries, breakable}{info}

\newenvironment{shelldisplay}{\begin{lstlisting}}{\end{lstlisting}}

\title{Squeezing of quantum noise}
\author{Jinyuan Wu}

\begin{document}
    
\maketitle

\section{Introduction}

Noise usually arises from coupling with an unknown external object:
when we keep our eyes only on one of the objects (the ``system''),
we need to average over the state of the other (``the environment''),
which corrects the theory with noise and dissipation \cite{zwanzig_nonequilibrium_2001}.
When the full quantum treatment is needed,
however, another type of noise appears:
the state itself does not have a definite value of the observable in question, 
and any experimental setting 
-- no matter how carefully the system is isolated from the environment -- 
has noisy results.
This kind of noise is called \concept{quantum noise}.

In systems that can be well described by 
the harmonic oscillator picture,
in which for each oscillation mode, 
we have two variables $X$ and $P$, and $[X, P] = \ii$,
and the Hamiltonian is $H \simeq c_1 X^2 + c_2 P^2$.%
\footnote{
    The Planck system of units is used in this report, 
    so we take $\hbar = 1$ if there is no special mention of the units.
}
After diagonalization, we get $H \simeq \sum_{\text{modes}} \omega (a^\dagger a + 1/2)$ 
plus possible interaction terms,
and this zero-point energy arises from the non-commutative nature of $X$ and $P$.
Another way to make sense of the $1/2$ term is to notice that at the ground state,
though $\expval{X} = \expval{P} = 0$,
and everything seems to be very definite,
since we have the zero-point energy,
$\expval*{X^2} \simeq H / 2 \neq 0$.
Note that most of the time, the observable actually measured in such systems 
is $\simeq n = a^\dagger a \sim X^2$,
and quantum noise appears in the measurement.
People therefore sometimes say that 
quantum noise comes from the zero point-energy.

% We may also consult this:
% Quantum noise and quantum measurement
% https://clerkgroup.uchicago.edu/PDFfiles/LesHouchesNotesAC.pdf

This report focuses on quantum noise in systems mentioned above.
This includes linear optics and micromechanics.
\prettyref{sec:overview-rep} discusses more quantitative ways to represent and characterize quantum noise.
\prettyref{sec:interferometer} calculates quantum noise 
in a prevalent phase measurement scheme using interferometer.
\prettyref{sec:squeezing} shows how to squeeze the quantum noise.

\section{Overview of representation of states in quantum optics}\label{sec:overview-rep}



\section{Quantum noise in the Mach-Zehnder interferometer}\label{sec:interferometer}

\begin{figure}
    \centering
    \begin{tikzpicture}[x=0.75pt,y=0.75pt,yscale=-1,xscale=1]
    %uncomment if require: \path (0,300); %set diagram left start at 0, and has height of 300
    
    %Shape: Square [id:dp5396904315538396] 
    \draw   (153,108) -- (127,108) -- (127,134) -- (153,134) -- cycle ;
    %Straight Lines [id:da8471914401201053] 
    \draw    (153,134) -- (127,108) ;
    
    %Straight Lines [id:da7935401918038376] 
    \draw    (60,121) -- (140,121) ;
    \draw [shift={(100,121)}, rotate = 180] [fill={rgb, 255:red, 0; green, 0; blue, 0 }  ][line width=0.08]  [draw opacity=0] (12,-3) -- (0,0) -- (12,3) -- cycle    ;
    %Straight Lines [id:da12011802802769855] 
    \draw    (140,121) -- (140,188.67) ;
    \draw [shift={(140,154.83)}, rotate = 270] [fill={rgb, 255:red, 0; green, 0; blue, 0 }  ][line width=0.08]  [draw opacity=0] (12,-3) -- (0,0) -- (12,3) -- cycle    ;
    %Straight Lines [id:da45241792789009994] 
    \draw    (140,66.62) -- (140,121) ;
    \draw [shift={(140,93.81)}, rotate = 270] [fill={rgb, 255:red, 0; green, 0; blue, 0 }  ][line width=0.08]  [draw opacity=0] (12,-3) -- (0,0) -- (12,3) -- cycle    ;
    %Straight Lines [id:da5153136174085808] 
    \draw    (140,121) -- (220,121) ;
    \draw [shift={(180,121)}, rotate = 180] [fill={rgb, 255:red, 0; green, 0; blue, 0 }  ][line width=0.08]  [draw opacity=0] (12,-3) -- (0,0) -- (12,3) -- cycle    ;
    %Shape: Square [id:dp7233820176028007] 
    \draw   (384,177) -- (358,177) -- (358,203) -- (384,203) -- cycle ;
    %Straight Lines [id:da4232871242423555] 
    \draw    (384,203) -- (358,177) ;
    
    %Shape: Rectangle [id:dp8767012605022662] 
    \draw   (219.71,105.12) -- (289.71,105.12) -- (289.71,136) -- (219.71,136) -- cycle ;
    %Straight Lines [id:da4420821653329212] 
    \draw    (140,189) -- (371,189) ;
    \draw [shift={(255.5,189)}, rotate = 180] [fill={rgb, 255:red, 0; green, 0; blue, 0 }  ][line width=0.08]  [draw opacity=0] (12,-3) -- (0,0) -- (12,3) -- cycle    ;
    %Straight Lines [id:da5818295505336346] 
    \draw    (290,121) -- (370,121) ;
    \draw [shift={(330,121)}, rotate = 180] [fill={rgb, 255:red, 0; green, 0; blue, 0 }  ][line width=0.08]  [draw opacity=0] (12,-3) -- (0,0) -- (12,3) -- cycle    ;
    %Straight Lines [id:da06983065057757765] 
    \draw    (353.98,104.98) -- (386.02,137.02) ;
    %Straight Lines [id:da7059684030986733] 
    \draw    (123.98,172.65) -- (156.02,204.69) ;
    %Straight Lines [id:da18624562535781375] 
    \draw    (371,122.33) -- (371,190) ;
    \draw [shift={(371,156.17)}, rotate = 270] [fill={rgb, 255:red, 0; green, 0; blue, 0 }  ][line width=0.08]  [draw opacity=0] (12,-3) -- (0,0) -- (12,3) -- cycle    ;
    %Straight Lines [id:da06098229649677367] 
    \draw    (371,189) -- (451,189) ;
    \draw [shift={(411,189)}, rotate = 180] [fill={rgb, 255:red, 0; green, 0; blue, 0 }  ][line width=0.08]  [draw opacity=0] (12,-3) -- (0,0) -- (12,3) -- cycle    ;
    %Straight Lines [id:da7157358634488349] 
    \draw    (371,190) -- (371,253.73) ;
    \draw [shift={(371,221.86)}, rotate = 270] [fill={rgb, 255:red, 0; green, 0; blue, 0 }  ][line width=0.08]  [draw opacity=0] (12,-3) -- (0,0) -- (12,3) -- cycle    ;
    
    % Text Node
    \draw (254.71,120.56) node    {$\mathrm{e}^{\mathrm{i} 2\varphi }$};
    % Text Node
    \draw (108,86.4) node [anchor=north west][inner sep=0.75pt]    {$S_{1}$};
    % Text Node
    \draw (386,206.4) node [anchor=north west][inner sep=0.75pt]    {$S_{2}$};
    % Text Node
    \draw (42,109.4) node [anchor=north west][inner sep=0.75pt]    {$a_{1}^{\dagger }$};
    % Text Node
    \draw (133,43.4) node [anchor=north west][inner sep=0.75pt]    {$a_{2}^{\dagger }$};
    % Text Node
    \draw (458,176.4) node [anchor=north west][inner sep=0.75pt]    {$b_{2}^{\dagger }$};
    % Text Node
    \draw (366,262.4) node [anchor=north west][inner sep=0.75pt]    {$b_{1}^{\dagger }$};
    
    
    \end{tikzpicture}
    
    \caption{The schematic structure of the Mach-Zehnder interferometer}
    \label{fig:mz-inter}
\end{figure}

To have a concrete example of quantum noise, 
let us move to the Mach-Zehnder interferometer,
illustrated in \prettyref{fig:mz-inter}.
The interferometer contains two beam splitters, 
two ideal mirrors, 
and one sample that introduces a $2\varphi$ phase shift to the light beam going through it.
The two beams created by the first beam splitter gain a phase difference caused by the sample,
and are then remixed together by the second beam splitter,
so there is interference,
and $\varphi$ can be found by comparing the output signal with the injected laser beam.

We are going to work in the Heisenberg picture,
because the whole system is linear 
and time evolution can be easily described by a linear transformation on the operators. 
Since this section is just to exemplify the overall idea of quantum noise,
for the sake of simplicity we assume the time evolution operator of the beam splitter is 
\begin{equation}
    S_{\text{beam splitter}} = \frac{1}{\sqrt{2}} \pmqty{ 1 & -1 \\ 1 & 1 } .
\end{equation}
The time evolution operator of the whole system is therefore
\begin{equation}
    \begin{aligned}
        S_\text{total}(\varphi) &= \frac{1}{\sqrt{2}} \pmqty{ 1 & 1 \\ 1 & -1 } \cdot \pmqty{\dmat{\ee^{\ii \varphi}, \ee^{- \ii \varphi}}} \cdot \frac{1}{\sqrt{2}} \pmqty{ 1 & -1 \\ 1 & 1 } \\
        &= \pmqty{ \cos \varphi & - \ii \sin \varphi \\ \ii \sin \varphi & - \cos \varphi },
    \end{aligned}
    \label{eq:total-matrix}
\end{equation} 
and therefore 
\begin{equation}
    \pmqty{b_1^\dagger \\ b_2^\dagger} = \pmqty{ \cos \varphi & - \ii \sin \varphi \\ \ii \sin \varphi & - \cos \varphi } \pmqty{a_1^\dagger \\ a_2^\dagger},
    \label{eq:b-def-by-a}
\end{equation}
from which we find 
\begin{equation}
    \pmqty{a_1^\dagger \\ a_2^\dagger} = \pmqty{ \cos \varphi & - \ii \sin \varphi \\ \ii \sin \varphi & - \cos \varphi } \pmqty{b_1^\dagger \\ b_2^\dagger}.
\end{equation}
In actual experiment settings,
usually a beam of laser is injected into input port 1,
so the state at input port 1 is a coherent state.
Then, the wave function is 
\begin{equation}
    \begin{aligned}
        \ket{\psi} &= \underbrace{\ee^{\alpha a_1 - \alpha^* a_1^\dagger}}_{D_{a_1}(\alpha)} \ket{0} \\
        &= \ee^{\alpha (\cos \varphi b_1^\dagger - \ii \sin \varphi b_2^\dagger) -\alpha^{*} (\cos \varphi b_1 + \ii \sin \varphi b_2)}|0\rangle \\
        &= \ket{b_1 = \alpha \cos \varphi, b_2 = - \ii \alpha \sin \varphi} .
    \end{aligned}
    \label{eq:out-state}
\end{equation}

The next step is measurement.
Usually, the influence of the sample to the light beam is small,
and therefore the signal in the $b_2$ mode is $\sim \varphi$,
while the signal in the $b_1$ mode is $\sim \varphi^2$
(the signal in the $b_1$ mode should be defined as $\alpha - \alpha \cos \varphi$,
because when $\varphi = 0$, the output state is $\ket{b_1 = \alpha, b_2 = 0}$).
So we usually measure the intensity of the $b_2$ mode.
The expectation of photon number is 
\begin{equation}
    \expval{n_{b_2}} = \abs{\alpha}^2 \sin^2 \varphi.
\end{equation}
This output comes with a quantum uncertainty, which is given by 
\begin{equation}
    \begin{aligned}
        \Delta n_{b_2} &= \sqrt{ \expval*{n_{b_2}^2} - \expval*{n_{b_2}}^2 } \\
        &= \sqrt{ \expval*{ b_2^\dagger (b_2^\dagger b_2 + 1) b_2} - \expval*{n_{b_2}}^2 } \\
        &= \sqrt{ \abs{\alpha}^4 \sin^4 \varphi + \abs{\alpha}^2 \sin^2 \varphi - \abs{\alpha}^4 \sin^4 \varphi } \\
        &= \abs{\alpha} \sin \varphi.
    \end{aligned}
    \label{eq:stdvar-n2}
\end{equation}
Here we have used the property of coherent states as eigenstates of the annihilation operator.

So it can be seen that the relative uncertainty of the measurement is 
\begin{equation}
    \frac{\Delta n_{b_2}}{\expval{n_{b_2}}} = \frac{1}{\abs{\alpha} \sin \varphi} = \frac{1}{\sqrt{n_{b_2}}}.
    \label{eq:relvar-n2}
\end{equation}
It can be seen from \eqref{eq:stdvar-n2} that this standard error comes from the commutation relation 
\begin{equation}
    \comm*{b_2}{b_2^\dagger} = 1,
\end{equation}
which, eventually, comes from the fact that $\vb*{E}$ and $\vb*{A}$ in electromagnetism don't commute.
This is therefore a quantum noise.

Of course, \eqref{eq:relvar-n2} can be systematically reduced by using stronger and stronger laser beams,
but then another problem occurs:
in a real measurement setting,
the detecting laser beam of course perturbs the sample,
introducing another source of error.
(For example, if the sample is an additional optical path,
which is the case in gravitational wave detection,
then the mirrors used will be heated
and begin to have thermal vibration.)
In a classical theory,
we can always use weaker and weaker laser beams to do the measurement,
and do an extrapolation for the measured results
to systematically reduce the perturbation of measurement to the system
and get results as accurate as we want,
but in the quantum theory,
this results in stronger quantum noise.

This leads to an astonishing fact:
when quantum noise is present, 
even in principle, 
we still can't find a way to reduce the error as much as we want.
There is a non-zero minimum error,
which is met when $\abs{\alpha}$ strikes a balance 
between the thermal fluctuation caused by large $\abs{\alpha}$
and quantum noise caused by small $\abs{\alpha}$.
This actually makes sense,
or otherwise we are faced with the problem 
that an infinitely accurate continuous degree of freedom can store infinite bits of information.

\section{Squeezing the quantum noise}\label{sec:squeezing}

One thing to keep in mind is \eqref{eq:relvar-n2} 
is about the quantum noise of the photon number, not others.
This quantity is not the only observable in $b_2$ mode,
but it is the only thing actually measured.
Thus it is possible that we reduce the quantum noise of $n_{b_2}$
and in exchange, get a larger quantum noise of other variables,
which we do not care.
This is called ``squeezing'' the quantum noise
-- the term will be visualized in the following discussion.
To do so, it is necessary to trace the origin of $\Delta{n_{b_2}}$ 
in terms of quantum noises of $a_1$ and $a_2$. 
For sake of simplicity,
here we keep the input state in the $a_1$ mode $\ket{\alpha}$,
without modifying anything,
and we also exclude all cross terms between $a_1$ and $a_2$ in the wave function, 
so we have 
\begin{equation}
    \ket{\psi} = D_{a_1}(\alpha) f(a_2, a_2^\dagger) \ket{0},
    \label{eq:coherent-plus-squeeze}
\end{equation}
and squeezing $\Delta{n_{b_2}}$ therefore reduces to squeezing the quantum error of some operator in mode $a_2$.

From \eqref{eq:b-def-by-a}, we have 
\begin{equation}
    n_{b_2} = b_2^\dagger b_2 = 
    \sin^2 \varphi a_1^\dagger a_1 + \cos^2 \varphi a_2^\dagger a_2
    - \ii \sin \varphi \cos \varphi (a_1^\dagger a_2 - a_2^\dagger a_1),
    \label{eq:nb2-a}
\end{equation}
and therefore 
\begin{equation}
    \begin{aligned}
        n_{b2}^2 &= \sin^4 \varphi (a_1^\dagger a_1)^2 + \cos^4 \varphi (a_2^\dagger a_2)^2 - \sin^2 \varphi \cos^2 \varphi (a_1^\dagger a_2 - a_2^\dagger a_1)^2 + 2 \sin^2 \varphi \cos^2 \varphi a_1^\dagger a_1 a_2^\dagger a_2 \\
        &\quad - \ii \sin \varphi \cos^3 \varphi \acomm*{a_2^\dagger a_2}{(a_1^\dagger a_2 - a_2^\dagger a_1)}
        - \ii \sin^3 \varphi \cos \varphi \acomm*{a_1^\dagger a_1}{(a_1^\dagger a_2 - a_2^\dagger a_1)}.
    \end{aligned}
    \label{eq:nb2-sq-def}
\end{equation}
If we are sure the wave function takes the form of \eqref{eq:coherent-plus-squeeze},
then all $a_1$ can be replaced by $\alpha$ in the above two equations.
Strictly speaking, before that we should first complete normal ordering of $a_1$ and $a_1^\dagger$,
because otherwise the non-trivial commutation relation of $a_1$ is ignored,
and thus the quantum fluctuation of the $a_1$ mode is ignored.
This however involves highly complicated calculation,
because we need to do normal ordering for \eqref{eq:nb2-sq-def}.

A huge simplification 
(and hence a neat explanation of the nature of the quantum noise in the interferometer) 
however can be made when the following conditions are satisfied.
First, we assume $\varphi$ is small enough, 
so the fluctuation of the first term in \eqref{eq:nb2-a} is suppressed by the $\sin^2 \varphi$ factor.
Then we assume that in \eqref{eq:coherent-plus-squeeze},
in the state of $a_2$ mode, the expected number of photon 
is very small compared with $\abs{\alpha}^2$.
Then the fluctuation of the second term in \eqref{eq:nb2-a} is also neglected,
because 
\begin{equation}
    \Delta (a_2^\dagger a_2) = \expval*{ (a_2^\dagger a_2)^2 } - \expval*{ a_2^\dagger a_2 }^2 = 
    \expval*{a_2^\dagger a_2^\dagger a_2 a_2} + \expval*{a_2^\dagger a_2} - \expval*{ a_2^\dagger a_2 }^2 \simeq 0,
\end{equation}
compared with the absolute magnitude of 
\begin{equation}
    \Delta (a_1^\dagger a_1) = 
    \expval*{a_1^\dagger a_1^\dagger a_1 a_1} + \expval*{a_1^\dagger a_1} - \expval*{ a_1^\dagger a_1 }^2 
    = \abs{\alpha}^2.
\end{equation}
Under the above two conditions, approximately we have 
\begin{equation}
    \Delta n_{b_2} = \sin \varphi \cos \varphi \Delta (a_1^\dagger a_2 - a_2^\dagger a_1 ) ,
\end{equation}
and again by the argument above that 
the absolute magnitude of the fluctuation of $n_{a_1}$ 
is much larger than that of $n_{a_2}$,
we find the final expression of $\Delta n_{b_2}$: 
it is 
\begin{equation}
    \Delta n_{b_2} = \sin \varphi \cos \varphi \Delta (\alpha^* a_2 - \alpha a_2^\dagger)
    \approx \varphi \Delta (\alpha^* a_2 - \alpha a_2^\dagger).
    \label{eq:nb2-single-mode}
\end{equation}

We can do a sanity check:
if the $a_2$ state is vacuum,
then $\expval*{a_2} = \expval*{a_2^\dagger} = 0$, 
and again due to the non-trivial commutation relation, we have 
\begin{equation}
    \begin{aligned}
        \Delta (\alpha^* a_2 - \alpha a_2^\dagger)
        &= \sqrt{ \abs*{\expval*{ (\alpha^* a_2 - \alpha a_2^\dagger)^2 } 
        - \expval*{(\alpha^* a_2 - \alpha a_2^\dagger)}^2} } \\
        &= \sqrt{
            \abs{\alpha}^2 \expval*{a_2 a_2^\dagger}
        } \\
        &= \abs{\alpha} \sqrt{ 1 + a^\dagger_2 a_2 } = \abs{\alpha},
    \end{aligned}
\end{equation}
so 
\begin{equation}
    \Delta n_{b_2} = \varphi \abs{\alpha},
\end{equation}
which is the small-angle approximation of \eqref{eq:stdvar-n2}.

Note that \eqref{eq:nb2-single-mode} only concerns one mode now,
and the quantum noise can also be read from the quasi-probabilistic distribution,
and indeed this leads us to design an appropriate quantum noise squeezing scheme.
Redefining 

\bibliographystyle{plain}
\bibliography{squeezing}

\end{document}