\documentclass[hyperref, a4paper]{article}

\usepackage{geometry}
\usepackage{titling}
\usepackage{titlesec}
% No longer needed, since we will use enumitem package
% \usepackage{paralist}
\usepackage{enumitem}
\usepackage{footnote}
\usepackage{enumerate}
\usepackage{amsmath, amssymb, amsthm}
\usepackage{mathtools}
\usepackage{bbm}
\usepackage{cite}
\usepackage{graphicx}
\usepackage{subfigure}
\usepackage{physics}
\usepackage{tensor}
\usepackage{siunitx}
\usepackage[version=4]{mhchem}
\usepackage{tikz}
\usepackage{xcolor}
\usepackage{listings}
\usepackage{autobreak}
\usepackage[ruled, vlined, linesnumbered]{algorithm2e}
\usepackage{nameref,zref-xr}
\zxrsetup{toltxlabel}
\usepackage[colorlinks,unicode]{hyperref} % , linkcolor=black, anchorcolor=black, citecolor=black, urlcolor=black, filecolor=black
\usepackage[most]{tcolorbox}
\usepackage{prettyref}

% Page style
\geometry{left=3.18cm,right=3.18cm,top=2.54cm,bottom=2.54cm}
\titlespacing{\paragraph}{0pt}{1pt}{10pt}[20pt]
\setlength{\droptitle}{-5em}

% More compact lists 
\setlist[itemize]{
    itemindent=17pt, 
    leftmargin=1pt,
    listparindent=\parindent,
    parsep=0pt,
}

% Math operators
\DeclareMathOperator{\timeorder}{\mathcal{T}}
\DeclareMathOperator{\diag}{diag}
\DeclareMathOperator{\legpoly}{P}
\DeclareMathOperator{\primevalue}{P}
\DeclareMathOperator{\sgn}{sgn}
\newcommand*{\ii}{\mathrm{i}}
\newcommand*{\ee}{\mathrm{e}}
\newcommand*{\const}{\mathrm{const}}
\newcommand*{\suchthat}{\quad \text{s.t.} \quad}
\newcommand*{\argmin}{\arg\min}
\newcommand*{\argmax}{\arg\max}
\newcommand*{\normalorder}[1]{: #1 :}
\newcommand*{\pair}[1]{\langle #1 \rangle}
\newcommand*{\fd}[1]{\mathcal{D} #1}
\DeclareMathOperator{\bigO}{\mathcal{O}}

% TikZ setting
\usetikzlibrary{arrows,shapes,positioning}
\usetikzlibrary{arrows.meta}
\usetikzlibrary{decorations.markings}
\tikzstyle arrowstyle=[scale=1]
\tikzstyle directed=[postaction={decorate,decoration={markings,
    mark=at position .5 with {\arrow[arrowstyle]{stealth}}}}]
\tikzstyle ray=[directed, thick]
\tikzstyle dot=[anchor=base,fill,circle,inner sep=1pt]

% Algorithm setting
% Julia-style code
\SetKwIF{If}{ElseIf}{Else}{if}{}{elseif}{else}{end}
\SetKwFor{For}{for}{}{end}
\SetKwFor{While}{while}{}{end}
\SetKwProg{Function}{function}{}{end}
\SetArgSty{textnormal}

\newcommand*{\concept}[1]{{\textbf{#1}}}

% Embedded codes
\lstset{basicstyle=\ttfamily,
  showstringspaces=false,
  commentstyle=\color{gray},
  keywordstyle=\color{blue}
}

% Reference formatting
\newrefformat{fig}{Figure~\ref{#1}}

% Color boxes
\tcbuselibrary{skins, breakable, theorems}
\newtcbtheorem[number within=section]{warning}{Warning}%
  {colback=orange!5,colframe=orange!65,fonttitle=\bfseries, breakable}{warn}
\newtcbtheorem[number within=section]{note}{Note}%
  {colback=green!5,colframe=green!65,fonttitle=\bfseries, breakable}{note}
\newtcbtheorem[number within=section]{info}{Info}%
  {colback=blue!5,colframe=blue!65,fonttitle=\bfseries, breakable}{info}

\newenvironment{shelldisplay}{\begin{lstlisting}}{\end{lstlisting}}


\title{Summary of concepts in near-equilibrium physics}
\author{Jinyuan Wu}

\begin{document}

\maketitle

It should be noted that the fluctuation-dissipation theorem 
is derived \emph{without} mentioning any thermal fluctuation of the system:
We just assume the initial state is fully thermalized,
and then the system evolves on its own.
When we invoke the fluctuation-dissipation theorem to derive a constraint on 
the damping term and the fluctuating term in a Langevin equation,
we are not invoking the theorem considering the \emph{particle} as the system.
Here we have an onion-like structure 
for any system described by a Langevin-like equation:
The system is surrounded by an environment
(in the case of Brownian motion, it's the water surrounding the particle,
and in the case of atomic damping, 
it's the light field in the cavity) that is large, 
and usually has linear response to any external fluctuation,
and the environment is surrounded by a larger environment.
The damping and fluctuating terms in the Langevin equation 
are the properties of the water, the light field, etc.,
that is, the smaller environment directly surrounding the system described by the Langevin equation.
We can safely apply the fluctuation-dissipation theorem to the smaller environment,
which is now the \emph{system}, 
and the larger environment is now the \emph{environment}.
The smaller system is still large enough 
so that its time evolution is basically decided by its own Hamiltonian,
not by its interaction with the larger environment.

It's kind of misleading to show the basic techniques of linear response theory 
by calculating a harmonic oscillator, therefore:
A single oscillator is too small compared with the usual environment,
and its behaviors are better described by a Langevin-like equation.
Note that even when $T = 0$,
there can still be damping and fluctuation in the Langevin equation,
simply because of there is still interaction between the system and the environment
(which is described by fluctuation-dissipation theorem and linear response theory,
while the system is described the Langevin equation and can have highly nonlinear response).

It's still possible to write a Langevin-like equation for a large system,
like the water in Brownian motion.
Here the system in the Langevin equation is the low-energy degrees of freedom,
and the smaller environment is the hidden degrees of freedom 
(like small scale flows),
and the larger environment is the environment surrounding the many-body system.
This is covered in the (88.9) and (88.10) of the volume on condensed matter physics in 
The fluctuation in a classical liquid is assumed to come from 
``local spontaneous stress and heat flow'',
and a Langevin equation (88.8) -- this time on a \emph{field} instead of a coordinate --
is given.

\end{document}