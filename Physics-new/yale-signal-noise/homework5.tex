\documentclass[hyperref, a4paper]{article}

\usepackage{geometry}
\usepackage{titling}
\usepackage{titlesec}
\usepackage{booktabs}
\usepackage{multirow}
\usepackage{enumitem}
\usepackage{footnote}
\usepackage{enumerate}
\usepackage{amsmath, amssymb, amsthm}
\usepackage{mathtools}
\usepackage{bbm}
\usepackage{cite}
\usepackage{graphicx}
\usepackage{subfigure}
\usepackage{physics}
\usepackage{tensor}
\usepackage{siunitx}
\usepackage[version=4]{mhchem}
\usepackage{tikz}
\usepackage{xcolor}
\usepackage{listings}
\usepackage{autobreak}
\usepackage[ruled, vlined, linesnumbered]{algorithm2e}
\usepackage{nameref,zref-xr}
\zxrsetup{toltxlabel}
\usepackage[colorlinks,unicode]{hyperref} % , linkcolor=black, anchorcolor=black, citecolor=black, urlcolor=black, filecolor=black
\usepackage[most]{tcolorbox}
\usepackage{prettyref}

% Page style
\geometry{left=3.18cm,right=3.18cm,top=2.54cm,bottom=2.54cm}
\titlespacing{\paragraph}{0pt}{1pt}{10pt}[20pt]
\setlength{\droptitle}{-5em}

% More compact lists 
\setlist[itemize]{
    itemindent=17pt, 
    leftmargin=1pt,
    listparindent=\parindent,
    parsep=0pt,
}

% Math operators
\DeclareMathOperator{\timeorder}{\mathcal{T}}
\DeclareMathOperator{\diag}{diag}
\DeclareMathOperator{\legpoly}{P}
\DeclareMathOperator{\primevalue}{P}
\DeclareMathOperator{\sgn}{sgn}
\DeclareMathOperator{\res}{Res}
\newcommand*{\ii}{\mathrm{i}}
\newcommand*{\ee}{\mathrm{e}}
\newcommand*{\const}{\mathrm{const}}
\newcommand*{\suchthat}{\quad \text{s.t.} \quad}
\newcommand*{\argmin}{\arg\min}
\newcommand*{\argmax}{\arg\max}
\newcommand*{\normalorder}[1]{: #1 :}
\newcommand*{\pair}[1]{\langle #1 \rangle}
\newcommand*{\fd}[1]{\mathcal{D} #1}
\DeclareMathOperator{\bigO}{\mathcal{O}}

% TikZ setting
\usetikzlibrary{arrows,shapes,positioning}
\usetikzlibrary{arrows.meta}
\usetikzlibrary{decorations.markings}
\tikzstyle arrowstyle=[scale=1]
\tikzstyle directed=[postaction={decorate,decoration={markings,
    mark=at position .5 with {\arrow[arrowstyle]{stealth}}}}]
\tikzstyle ray=[directed, thick]
\tikzstyle dot=[anchor=base,fill,circle,inner sep=1pt]

% Algorithm setting
% Julia-style code
\SetKwIF{If}{ElseIf}{Else}{if}{}{elseif}{else}{end}
\SetKwFor{For}{for}{}{end}
\SetKwFor{While}{while}{}{end}
\SetKwProg{Function}{function}{}{end}
\SetArgSty{textnormal}

\newcommand*{\concept}[1]{{\textbf{#1}}}

% Embedded codes
\lstset{basicstyle=\ttfamily,
  showstringspaces=false,
  commentstyle=\color{gray},
  keywordstyle=\color{blue}
}

% Reference formatting
\newrefformat{fig}{Figure~\ref{#1}}
\newrefformat{tbl}{Table~\ref{#1}}

% Color boxes
\tcbuselibrary{skins, breakable, theorems}
\newtcbtheorem[number within=section]{warning}{Warning}%
  {colback=orange!5,colframe=orange!65,fonttitle=\bfseries, breakable}{warn}
\newtcbtheorem[number within=section]{note}{Note}%
  {colback=green!5,colframe=green!65,fonttitle=\bfseries, breakable}{note}
\newtcbtheorem[number within=section]{info}{Info}%
  {colback=blue!5,colframe=blue!65,fonttitle=\bfseries, breakable}{info}

\newenvironment{shelldisplay}{\begin{lstlisting}}{\end{lstlisting}}

\title{Homework 5}
\author{Jinyuan Wu}

\begin{document}
    
\maketitle

\paragraph{Exercise 1 in Lecture 15}  In the classical limit
\[
    \begin{aligned}
\hbar & \rightarrow 0 \\
\expval*{[\hat{B}(t), \hat{A}]}_\text{eq} & \rightarrow 0
\end{aligned}
\]
and the ratio becomes ill-defined. It acquires meaning when we replace the commutator $\ii [\hat{B}(t), \hat{A}]$ divided by $\hbar$ by the corresponding Poisson bracket. We then recover a classical expression for the susceptibility, soluble analytically for certain models like the Caldeira-Leggett model (see corresponding lecture).

\paragraph{Solution} Suppose 
\begin{equation}
    \hat{A} = \sum_{m, n \geq 0} A_{mn} \hat{x}^m \hat{p}^n, 
\end{equation}
\begin{equation}
    \hat{B} = \sum_{m, n \geq 0} B_{mn} \hat{x}^m \hat{p}^n,
\end{equation}
and we have (below we omit the hat over operators)
\begin{equation}
    \begin{aligned}
        \comm{A}{B} &= \sum_{m, n, p, q \geq 0} A_{mn} B_{pq} 
        ( x^m \comm*{p^n}{x^p} p^q + 
          x^p \comm*{x^m}{p^q} p^n ) \\
        &=\sum_{m, n, p, q \geq 0} A_{mn} B_{pq} 
        (- x^m \comm*{x^p}{p^n} p^q + 
          x^p \comm*{x^m}{p^q} p^n ).
    \end{aligned}
\end{equation}
Here 
\begin{equation}
    \begin{aligned}
        \comm{x^m}{p^q} &= x^{m-1} \comm*{x}{p^q} + \comm*{x^{m-1}}{p^q} x \\
        &= x^{m-1} \comm*{x}{p^q} + ( x^{m-2} \comm*{x}{p^q} + \comm*{x^{m-2}}{p^q} x ) x \\
        &= x^{m-1} \comm*{x}{p^q} + x^{m-2} \comm*{x}{p^q} x +  \comm*{x^{m-2}}{p^q} x^2 \\
        &= \cdots \\
        &= \underbrace{
            x^{m-1} \comm*{x}{p^q} + x^{m-2} \comm*{x}{p^q} x + x^{m-3} \comm*{x}{p^q} x^2 + \cdots + 
            \comm*{x}{p^q} x^{m-1}
        }_{\text{$m$ terms}},
    \end{aligned}
\end{equation}
and further we have 
\begin{equation}
    \begin{aligned}
        \comm*{x}{p^q} &= p \comm*{x}{p^{q-1}} + \underbrace{\comm*{x}{p}}_{\ii \hbar} p^{q-1} \\
        &= p ( \underbrace{\comm*{x}{p}}_{\ii \hbar} p^{q-2} + p \comm*{x}{p^{q-2}} ) + \ii \hbar p^{q-1} \\
        &= 2 \ii \hbar p^{q-1} + p^2 \comm*{x}{p^{q-2}} \\
        &= \cdots = \ii \hbar q p^{q-1},
    \end{aligned}
\end{equation}
so 
\begin{equation}
    \comm{x^m}{p^q} = \ii \hbar q ( \underbrace{x^{m-1} p^{q-1} + x p^{q-1} x^{m-2} + \cdots + p^{q-1} x^{m-1}}_{\text{$m$ terms}} ).
\end{equation}
In the classical limit we have $xp \approx px$, so 
\[
    \comm{x^m}{p^q} \approx \ii \hbar q m \cdot x^{m-1} p^{q-1},
\]
and 
\begin{equation}
    \lim_{\hbar \to 0} \frac{1}{\ii \hbar} \comm*{x^m}{p^q} = qm \cdot x^{m-1} p^{q-1}.
\end{equation}
So we find 
\begin{equation}
    \begin{aligned}
        \lim_{\hbar \to 0} \frac{1}{\ii \hbar} \comm*{A}{B} 
        &= \sum_{m, n, p, q \geq 0} A_{mn} B_{pq} 
        ( - pn \cdot x^{m} x^{p-1} p^{n-1} p^q + mq \cdot x^p x^{m-1} p^{q-1} p^n ) \\
        &= \sum_{m, n, p, q \geq 0} \left(
            A_{mn} m x^{m-1} p^n \cdot B_{pq} q x^p p^{q-1}
            - A_{mn} n x^m p^{n-1} \cdot B_{pq} p x^{p-1} p^{q}
        \right) \\
        &= \pdv{A}{x} \pdv{B}{p} - \pdv{A}{p} \pdv{B}{x} \eqqcolon \{ A, B \}.
    \end{aligned}
\end{equation}
So we recover the classical expression.

\paragraph{Exercise 2 in Lecture 15} When is 
\begin{equation}
    \chi_{B A}[z]=\frac{1}{\hbar} \sum_{n, q}\left(p_q-p_n\right) B_{n q} A_{q n} \frac{1}{z-\omega_{q n}}
\end{equation}
independent to the temperature?
Here 
\begin{equation}
    \omega_{q n}=\left(E_q-E_n\right) / \hbar,
\end{equation}
and 
\begin{equation}
    \chi_{B A}[\omega]=\int_{-\infty}^{+\infty} \dd t \chi_{B A}(t) \mathrm{e}^{\ii \omega t}, \quad 
    \chi_{B A}(t)=\frac{\ii}{\hbar}\langle[\hat{B}(t), \hat{A}]\rangle_\text{eq} \Theta(t).
\end{equation}

\paragraph{Solution} When $\comm{\hat{B}(t)}{\hat{A}}$ is a number (and not an operator),
the susceptibility is always independent to the temperature.
This can be seen for example in the case of harmonic oscillator
in the next exercise.

Another example of temperature-independent susceptibility is diamagnetism,
but this originates from $B^2$ dependence of the Hamiltonian 
and is beyond the linear response theory.

\paragraph{Exercise 4 in Lecture 16} Show that the quantum susceptibility $\chi_{B A}$ is temperatureinvariant for a reservoir consisting of harmonic oscillators (note: here we must take $\chi_{B A}$ in its most primitive sense, where the operators $A$ and $B$ are linear in the position and momentum variables of the oscillators).

\paragraph{Solution} Here we have 
\begin{equation}
    H = \frac{p^2}{2m} + \frac{1}{2} m \omega^2 x^2,
\end{equation}
and the EOMs of operators are 
\begin{equation}
    \dot{x} = \frac{1}{\ii \hbar} \comm{x}{H} = \frac{p}{m}, \quad 
    \dot{p} = \frac{1}{\ii \hbar} \comm{p}{H} = - m \omega^2 x,
\end{equation}
so we have 
\begin{equation}
    x = x(0) \cos \omega t + \frac{p(0)}{m} \sin \omega t, \quad 
    p = p(0) \cos \omega t - m \omega^2 x(0) \sin \omega t.
\end{equation}
So $\comm{x(t)}{x(0)}$, $\comm{x(t)}{p(0)}$, $\comm{p(t)}{x(0)}$, and $\comm{p(t)}{p(0)}$
are all pure numbers and contain no operator.
Now $A$ and $B$ are all linear combinations of $x_i$ and $p_i$,
with $i$ indexing the harmonic oscillators in the reservoir,
so $\comm{A(t)}{B(0)}$ or the inverse is also a number and contains no operator,
so by definition $\chi_{BA} \propto \expval*{\comm{B(t)}{A(0)}}$ 
is just that number times $\ii \Theta(t) / \hbar$, 
without any temperature dependence.

\paragraph{Exercise 3 in Lecture 17}

\paragraph{Solution} The matrix element $\mel{1}{\Phi}{0}$ corresponds to the 
dipole matrix element $\mel{f}{\vb*{d}}{i}$,
and $\Re Y$ corresponds to the density of states of photons
(which is the imaginary part of the photon Green function).

\end{document}