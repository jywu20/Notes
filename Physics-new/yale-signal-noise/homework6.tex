\documentclass[hyperref, a4paper]{article}

\usepackage{geometry}
\usepackage{titling}
\usepackage{titlesec}
\usepackage{booktabs}
\usepackage{multirow}
\usepackage{enumitem}
\usepackage{footnote}
\usepackage{enumerate}
\usepackage{amsmath, amssymb, amsthm}
\usepackage{mathtools}
\usepackage{bbm}
\usepackage{cite}
\usepackage{graphicx}
\usepackage{subfigure}
\usepackage{physics}
\usepackage{tensor}
\usepackage{siunitx}
\usepackage[version=4]{mhchem}
\usepackage{tikz}
\usepackage{xcolor}
\usepackage{listings}
\usepackage{autobreak}
\usepackage[ruled, vlined, linesnumbered]{algorithm2e}
\usepackage{nameref,zref-xr}
\zxrsetup{toltxlabel}
\usepackage[colorlinks,unicode]{hyperref} % , linkcolor=black, anchorcolor=black, citecolor=black, urlcolor=black, filecolor=black
\usepackage[most]{tcolorbox}
\usepackage{prettyref}

% Page style
\geometry{left=3.18cm,right=3.18cm,top=2.54cm,bottom=2.54cm}
\titlespacing{\paragraph}{0pt}{1pt}{10pt}[20pt]
\setlength{\droptitle}{-5em}

% More compact lists 
\setlist[itemize]{
    itemindent=17pt, 
    leftmargin=1pt,
    listparindent=\parindent,
    parsep=0pt,
}

% Math operators
\DeclareMathOperator{\timeorder}{\mathcal{T}}
\DeclareMathOperator{\diag}{diag}
\DeclareMathOperator{\legpoly}{P}
\DeclareMathOperator{\primevalue}{P}
\DeclareMathOperator{\sgn}{sgn}
\DeclareMathOperator{\res}{Res}
\newcommand*{\ii}{\mathrm{i}}
\newcommand*{\ee}{\mathrm{e}}
\newcommand*{\const}{\mathrm{const}}
\newcommand*{\suchthat}{\quad \text{s.t.} \quad}
\newcommand*{\argmin}{\arg\min}
\newcommand*{\argmax}{\arg\max}
\newcommand*{\normalorder}[1]{: #1 :}
\newcommand*{\pair}[1]{\langle #1 \rangle}
\newcommand*{\fd}[1]{\mathcal{D} #1}
\DeclareMathOperator{\bigO}{\mathcal{O}}

% TikZ setting
\usetikzlibrary{arrows,shapes,positioning}
\usetikzlibrary{arrows.meta}
\usetikzlibrary{decorations.markings}
\tikzstyle arrowstyle=[scale=1]
\tikzstyle directed=[postaction={decorate,decoration={markings,
    mark=at position .5 with {\arrow[arrowstyle]{stealth}}}}]
\tikzstyle ray=[directed, thick]
\tikzstyle dot=[anchor=base,fill,circle,inner sep=1pt]

% Algorithm setting
% Julia-style code
\SetKwIF{If}{ElseIf}{Else}{if}{}{elseif}{else}{end}
\SetKwFor{For}{for}{}{end}
\SetKwFor{While}{while}{}{end}
\SetKwProg{Function}{function}{}{end}
\SetArgSty{textnormal}

\newcommand*{\concept}[1]{{\textbf{#1}}}

% Embedded codes
\lstset{basicstyle=\ttfamily,
  showstringspaces=false,
  commentstyle=\color{gray},
  keywordstyle=\color{blue}
}

% Reference formatting
\newrefformat{fig}{Figure~\ref{#1}}
\newrefformat{tbl}{Table~\ref{#1}}

% Color boxes
\tcbuselibrary{skins, breakable, theorems}
\newtcbtheorem[number within=section]{warning}{Warning}%
  {colback=orange!5,colframe=orange!65,fonttitle=\bfseries, breakable}{warn}
\newtcbtheorem[number within=section]{note}{Note}%
  {colback=green!5,colframe=green!65,fonttitle=\bfseries, breakable}{note}
\newtcbtheorem[number within=section]{info}{Info}%
  {colback=blue!5,colframe=blue!65,fonttitle=\bfseries, breakable}{info}

\newenvironment{shelldisplay}{\begin{lstlisting}}{\end{lstlisting}}

\title{Homework 6}
\author{Jinyuan Wu}

\begin{document}
    
\maketitle

\paragraph{Lecture 21 Exercise 1}

\paragraph{Solution} The cumulants of te Poisson distribution can be found by 
using the characteristic function.
It is 
\begin{equation}
    \begin{aligned}
        \varphi(t) = \expval{\ee^{\ii t n}} &= 
        \sum_{n \geq 0} \ee^{\ii t n} \exp(-\expval{n}) \frac{\expval{n}^n}{n!}  \\
        &= \exp(-\expval{n}) \sum_{n \geq 0} \frac{1}{n!} (\expval{n} \ee^{\ii t})^n  \\
        &= \exp(-\expval{n}) \exp(\ee^{\ii t} \expval{n}) 
        = \exp((\ee^{\ii t} - 1) \expval{n}),
    \end{aligned}
\end{equation}
and therefore we have 
\begin{equation}
    \log \expval{\ee^{\ii t n}} = (\ee^{\ii t} - 1) \expval{n} = 
    \expval{n} \ii t  + \expval{n} \frac{(\ii t)^2}{2!} + \expval{n} \frac{(\ii t)^3}{3!} + \cdots,
\end{equation}
and the $n$-th cumulant is the factor before $(\ii t)^n / n!$,
so we find every cumulant of the Poisson distribution is $\expval{n}$.
Specifically, we have $C_3 = C_4 = \expval{n}$.

\paragraph{Lecture 21 Exercise 4} Isn't it a paradox that shot noise is not Gaussian noise despite the fact the distribution of shots in a long time interval tends towards a Gaussian distribution?

\paragraph{Solution} This is not a paradox.
What makes the construction of a Poisson distribution is 
it can be seen as the sum of the number of photon between $t=0$ and $t = \Delta t$,
the number of photon between $t = \Delta t$ and $t = 2 \Delta t$, \dots,
and each term has a binominal distribution,
so it seems we are summing up lots of independent variables 
and should get a a Gaussian result.
The problem here is the distribution of each term involves $\Delta t$:
as $\Delta t = \text{total time} / N$ goes smaller,
the probability that there is a photon occurring between $t = n \Delta t$ and $t = (n+1) \Delta t$ 
also goes smaller.
This breaks the condition of the central limit theorem,
because the central limit theorem is about the sum 
of ``fully encapsulated'' random variables which do not vary as we take the $N \to \infty$ limit.

\paragraph{Lecture 22 Exercise 3}

\paragraph{Solution} For method 2: 
now by writing down the scattering matrix, we have
\begin{equation}
    \pmqty{A_1 \\ A_2} = \pmqty{\epsilon & - \ii  \\ - \ii  & \epsilon} 
    \pmqty{ \dmat{\ee^{\ii \phi}, 1} } 
    \pmqty{ \epsilon & \ii  \\ \ii  & \epsilon } \pmqty{A \\ B},
\end{equation}
and we have 
\begin{equation}
    A_1 = A + \ii \epsilon (\ee^{\ii \phi} - 1) B, \quad 
    A_2 = - \ii \epsilon (\ee^{\ii \phi} - 1) A + \ee^{\ii \phi} B.
\end{equation}
% and therefore
% \begin{equation}
%     A_1^\dagger A_1 = A^\dagger A + \ii \epsilon (\ee^{\ii \phi} - 1) A^\dagger B + \text{h.c.},
% \end{equation}
% \begin{equation}
%     A_2^\dagger A_2 = \ii \epsilon (1 - \ee^{\ii \phi}) A^\dagger B + \text{h.c.} + {B^\dagger} B,
% \end{equation}
% \begin{equation}
%     D = A_1^\dagger A_1 - A_2^\dagger A_2 = A^\dagger A + 2 \ii \epsilon (\ee^{\ii \phi} - 1) A^\dagger B + \text{h.c.} - B^\dagger B,
% \end{equation}
% and 
% \begin{equation}
%     A_1^\dagger A_1 + A_2^\dagger A_2 = A^\dagger A + B^\dagger B.
% \end{equation}
Since $B$ mode is in vacuum, 
the only operator that carries enough information about $\phi$
is the amplitude of $A_2$.
So there is no loss of information.
The phase of $A_2$ is mainly determined by the $\ee^{\ii \phi} B$ part,
and since $B$ is in vacuum,
measuring it would be impossible.
The amplitude and phase of $A_1$ are mainly determined by the $A$ part, 
which tell nothing about $\phi$.

For method 3: here the wave function of the system is just 
\begin{equation}
    \ket{\psi} = A^\dagger \ket{0} = \ii \ee^{\ii \phi / 2} 
    \left( \sin \frac{\phi}{2} A_1^\dagger + \cos \frac{\phi}{2} A_2^\dagger \right)  \ket{0}.
\end{equation}
In the lecture we have already seen that measuring $I_1 + I_2$ tells nothing about $\phi$,
and since a Fock state is ``squeezed in the particle number direction'',
two phase-related operators in $A_1$ and $A_2$ modes also tell nothing about $\phi$.
So there is no loss of information.

\paragraph{Lecture 23 Exercise 1} 

\paragraph{Solution} The relation between output and input quadratures is 
\begin{equation}
    \begin{aligned}
        \frac{1}{\sqrt{2}} (a_{\text{out}} + a^\dagger_{\text{out}}) 
        &= \frac{1}{\sqrt{2}} (a_{\text{in}} + a^\dagger_{\text{in}}) \cdot G, \\
        \frac{1}{\sqrt{2} \ii} (a_{\text{out}} - a^\dagger_{\text{out}}) 
        &= \frac{1}{\sqrt{2} \ii} (a_{\text{in}} - a^\dagger_{\text{in}}) \cdot \frac{1}{G}, \\
    \end{aligned}
\end{equation}
and therefore 
\begin{equation}
    a_{\text{out}} = \frac{1}{2} \left( G + \frac{1}{G} \right) a_{\text{in}}
    + \frac{1}{2} \left( G - \frac{1}{G} \right) a_{\text{in}}^\dagger,
\end{equation}
and therefore 
\begin{equation}
    \begin{aligned}
        \comm*{a_{\text{out}}}{a^\dagger_{\text{out}}} 
        &= \frac{1}{4} \left( G + \frac{1}{G} \right)^2 \comm{a_{\text{in}}}{a^\dagger_{\text{in}}} 
        + \frac{1}{4} \left( G - \frac{1}{G} \right)^2 \comm{a_{\text{in}}^\dagger}{a_{\text{in}}} \\
        &= \frac{1}{4} \cdot 4 = 1,
    \end{aligned}
\end{equation}
and therefore an amplifier that amplifies one quadrature
and desamplifies another is at least logically consistent,
without the need to introduce another mode -- the noise -- coupled to $a$.
So in principle such an amplifier can be noiseless.

\end{document}