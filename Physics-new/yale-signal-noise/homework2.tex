\documentclass[hyperref, a4paper]{article}

\usepackage{geometry}
\usepackage{titling}
\usepackage{titlesec}
% No longer needed, since we will use enumitem package
% \usepackage{paralist}
\usepackage{enumitem}
\usepackage{footnote}
\usepackage{enumerate}
\usepackage{amsmath, amssymb, amsthm}
\usepackage{mathtools}
\usepackage{bbm}
\usepackage{cite}
\usepackage{graphicx}
\usepackage{subfigure}
\usepackage{physics}
\usepackage{tensor}
\usepackage{siunitx}
\usepackage[version=4]{mhchem}
\usepackage{tikz}
\usepackage{xcolor}
\usepackage{listings}
\usepackage{autobreak}
\usepackage[ruled, vlined, linesnumbered]{algorithm2e}
\usepackage{nameref,zref-xr}
\zxrsetup{toltxlabel}
\usepackage[colorlinks,unicode]{hyperref} % , linkcolor=black, anchorcolor=black, citecolor=black, urlcolor=black, filecolor=black
\usepackage[most]{tcolorbox}
\usepackage{prettyref}

% Page style
\geometry{left=3.18cm,right=3.18cm,top=2.54cm,bottom=2.54cm}
\titlespacing{\paragraph}{0pt}{1pt}{10pt}[20pt]
\setlength{\droptitle}{-5em}

% More compact lists 
\setlist[itemize]{
    itemindent=17pt, 
    leftmargin=1pt,
    listparindent=\parindent,
    parsep=0pt,
}

% Math operators
\DeclareMathOperator{\timeorder}{\mathcal{T}}
\DeclareMathOperator{\diag}{diag}
\DeclareMathOperator{\legpoly}{P}
\DeclareMathOperator{\primevalue}{P}
\DeclareMathOperator{\sgn}{sgn}
\newcommand*{\ii}{\mathrm{i}}
\newcommand*{\ee}{\mathrm{e}}
\newcommand*{\const}{\mathrm{const}}
\newcommand*{\suchthat}{\quad \text{s.t.} \quad}
\newcommand*{\argmin}{\arg\min}
\newcommand*{\argmax}{\arg\max}
\newcommand*{\normalorder}[1]{: #1 :}
\newcommand*{\pair}[1]{\langle #1 \rangle}
\newcommand*{\fd}[1]{\mathcal{D} #1}
\DeclareMathOperator{\bigO}{\mathcal{O}}

% TikZ setting
\usetikzlibrary{arrows,shapes,positioning}
\usetikzlibrary{arrows.meta}
\usetikzlibrary{decorations.markings}
\tikzstyle arrowstyle=[scale=1]
\tikzstyle directed=[postaction={decorate,decoration={markings,
    mark=at position .5 with {\arrow[arrowstyle]{stealth}}}}]
\tikzstyle ray=[directed, thick]
\tikzstyle dot=[anchor=base,fill,circle,inner sep=1pt]

% Algorithm setting
% Julia-style code
\SetKwIF{If}{ElseIf}{Else}{if}{}{elseif}{else}{end}
\SetKwFor{For}{for}{}{end}
\SetKwFor{While}{while}{}{end}
\SetKwProg{Function}{function}{}{end}
\SetArgSty{textnormal}

\newcommand*{\concept}[1]{{\textbf{#1}}}

% Embedded codes
\lstset{basicstyle=\ttfamily,
  showstringspaces=false,
  commentstyle=\color{gray},
  keywordstyle=\color{blue}
}

% Reference formatting
\newrefformat{fig}{Figure~\ref{#1}}

% Color boxes
\tcbuselibrary{skins, breakable, theorems}
\newtcbtheorem[number within=section]{warning}{Warning}%
  {colback=orange!5,colframe=orange!65,fonttitle=\bfseries, breakable}{warn}
\newtcbtheorem[number within=section]{note}{Note}%
  {colback=green!5,colframe=green!65,fonttitle=\bfseries, breakable}{note}
\newtcbtheorem[number within=section]{info}{Info}%
  {colback=blue!5,colframe=blue!65,fonttitle=\bfseries, breakable}{info}

\newenvironment{shelldisplay}{\begin{lstlisting}}{\end{lstlisting}}

\title{Homework 2}
\author{Jinyuan Wu}

\begin{document}

\maketitle

\paragraph{Problem 1} 

\paragraph{Solution} \begin{itemize}
\item[(a)] The dissipation term pulls the system back to the equilibrium.
Since $\gamma^{-1} \ll t$, the region around $x_0$ must be thermalized enough.
Since $t \ll \Gamma_{\text{esc}}^{-1}$,
the particle doesn't have time to ``explore'' the $x > x_b$ region.
So the $x < x_b$ region is described by the stationary equilibrium Maxwell–Boltzmann distribution,
while the $x > x_b$ region isn't.

When $t \gg \Gamma_{\text{esc}}^{-1}$
the probabilistic distribution of the particle on the whole axis is described 
by the stationary equilibrium Maxwell–Boltzmann distribution.
Since the global minimum is much lower than the local minimum at $x_0$,
the particle is most likely to appear near the global minimum on the right of the barrier;
there is still some possibility to find the particle near $x_0$.

\item[(b)] I basically will follow the derivation in \cite{kramers1940284}. 
The Fokker-Planck equation 
\begin{equation}
    \frac{\partial}{\partial t} W=\left\{-\frac{\partial}{\partial x} v+\frac{\partial}{\partial v}[\gamma v-f(x)]+\frac{\partial^2}{\partial v^2} \gamma^2 D\right\} W
\end{equation}
can be recast into 
\[
    \pdv{W}{t} = \gamma \left( \pdv{v} - \frac{1}{\gamma} \pdv{x} \right)
    \left( vW + \gamma D \pdv{W}{v} - \frac{1}{\gamma} f W + D \pdv{W}{x} \right) 
    + \pdv{x} \left( - \frac{1}{\gamma} f W + D \pdv{W}{x} \right).
\]
Now we perform the integral $\int \dd{v}$ on the straight line $v + \gamma x = \const$.
We have 
\[
    \int_{v + \gamma x = \const} \dd{v} \left( \pdv{v} - \frac{1}{\gamma} \pdv{x} \right) (\cdots) 
    \propto \int \dd{\vb*{l}} \cdot \left( \pdv{v}, \pdv{x} \right) (\cdots),
\]
and since when $\abs{v} \to \infty$, $W \to 0$,
this term vanishes.
By definition we have 
\[
    \frac{f}{\gamma} = \mu M f = \mu F.
\]
So we get 
\[
    \pdv{t} \int_{v + \gamma x = C} \dd{v} W(x, v, t) = 
    \int_{v + \gamma x = C} \dd{v} \left( - \mu \pdv{x} (F W) + D \pdv[2]{W}{x} \right).
\]
Until now the derivation is strictly true.
Now if $\gamma$ is somehow ``large'' enough, 
the equation $v + \gamma x = C$ is just a specification of the value of $x$,
and hence we get 
\begin{equation}
    \pdv{t} \int \dd{v} W(x, v, t) = \left( - \mu \pdv{x} F + \pdv[2]{x} D \right) \int \dd{v} W(x, v, t),
    \label{eq:proto-smoluchowsky}
\end{equation}
and by defining 
\begin{equation}
    W(x, t) = \int \dd{v} W(x, v, t),
\end{equation}
we get the Smoluchowsky equation
\begin{equation}
    \pdv{W}{t} = \left( - \mu \pdv{x} F + \pdv[2]{x} D \right) W.
\end{equation}

The question of validity is now how large should $\gamma$ be. 
When $\gamma$ is large, the deviation of the error on the LHS of \eqref{eq:proto-smoluchowsky} is always small,
but if the spatial variation of $W$ is large, then the RHS will introduce a large error.
So to prevent this from happening, $U$ shouldn't have large variation, i.e. $F$ shouldn't be too strong,
and thus we have $\gamma \ll \omega_0$,
because $\omega_0$ is the only constant that has the dimension of $[T]^{-1}$
and also measures the strength of $F$.

\item[(c)] Since  
\begin{equation}
    \pdv{W}{t} = - \pdv{j_{\text{d}}}{x},
\end{equation}
we can just take 
\begin{equation}
    j_{\text{d}} = \mu F(x) W(x, t) - \pdv{x} (D W).
\end{equation}

\item[(d)] Since the barrier is high,
we can assume in the time scale of our observation,
the state of the whole system doesn't change much.
So first, $j_{\text{d}}$ doesn't change as time goes by,
and second, since $W$ also doesn't change much,
$j_{\text{d}}$ is a constant under spatial translation.
So $j_{\text{d}}$ is a constant anywhere and at any time.
In this way the trick of (13) in \cite{kramers1940284} can be used to find $j_{\text{d}}$.
We have 
\[
    \begin{aligned}
        j_{\text{d}} &= - \mu W \pdv{U}{x} - \mu k_{\text{B}} T \pdv{W}{x} \\
        &= - \mu k_{\text{B}} T \ee^{- U / k_{\text{B}} T} 
        \pdv{x} \left( \ee^{U / k_{\text{B}} T} W \right) ,
    \end{aligned}
\]
and therefore 
\[
    j_{\text{d}} \ee^{U / k_{\text{B}} T} = - \mu k_{\text{B}} T \pdv{x} \left( \ee^{U / k_{\text{B}} T} W \right),
\]
and since $j_{\text{d}}$ is a constant, 
we can integrate the above equation and get 
\begin{equation}
    j_{\text{d}} \int_a^b \ee^{U(x) / k_{\text{B}} T} \dd{x} 
    = D \left(\ee^{U(x_a) / k_{\text{B}} T} W(x_a) - \ee^{U(x_b) / k_{\text{B}} T} W(x_b) \right).
\end{equation}
In our case, let $a$ be $x_0$ and $b$ be $x_{\text{b}}$.
Since $\Delta U$ is large, at first it's almost impossible for any particle to appear there,
and thus we have $W(x_{\text{b}}) = 0$.
So we have 
\[
    j_{\text{d}} = \frac{
        D \ee^{U(x_0) / k_{\text{B}} T}
    }{
        \int_{x_0}^{x_{\text{b}}} \ee^{U(x) / k_{\text{B}} T} \dd{x}
    } W(x_0).
\]
We don't know exactly what $W(x_0)$ is, but since $\Delta U$ is large,
we can expect in the region where $W(x)$ isn't nearly zero, approximately 
\begin{equation}
    U(x) - U(x_0) = \frac{1}{2} M \omega_0^2 (x - x_0)^2,
\end{equation}
so we have 
\[
    \begin{aligned}
        1 &= \int \dd{x} W(x) \\
        &= \int \dd{x} W(x_0) \ee^{- (U(x) - U(x_0) / k_{\text{B}} T} \\
        &\approx W(x_0) \int_{-\infty}^\infty \dd{x} \ee^{- \frac{M \omega_0^2}{k_{\text{B}} T} (x - x_0)^2} \\
        &= W(x_0) \sqrt{\frac{2\pi k_{\text{B}} T}{M \omega_0^2}}.
    \end{aligned}
\]
So 
\begin{equation}
    j_{\text{d}} = \frac{D \ee^{U(x_0) / k_{\text{B}} T} }%
    {\int_{x_0}^{x_b} \ee^{U(x) / k_{\text{B}} T} \dd{x}}
    \sqrt{\frac{M \omega_0^2}{2 \pi k_{\text{B}} T}} .
    \label{eq:j-d-half-finished}
\end{equation}

\item[(e)] Similar to the above procedure we normalize $W$,
the main contribution to the denominator of \eqref{eq:j-d-half-finished} is the quadratic part,
and we have 
\[
    \int_{x_0}^{x_{\text{b}}} \ee^{U(x) / k_{\text{B}} T} \dd{x}
    \approx \int_{x_0}^{x_{\text{b}}} 
    \ee^{ \frac{ U(x_{\text{b}}) - \frac{1}{2} M \omega_\text{b}^2 (x - x_{\text{b}})^2 }{k_{\text{B}} T} } \dd{x} 
    = \ee^{U(x_{\text{b}}) / k_{\text{B}} T} \sqrt{\frac{2\pi k_{\text{B}} T}{M \omega_{\text{b}}^2}}.
\]
Thus 
\begin{equation}
    j_{\text{d}} = D \ee^{- \Delta U / k_{\text{B}} T} \cdot \frac{M}{2\pi k_{\text{B}} T} \omega_0 \omega_{\text{b}} 
    = \frac{\mu M \omega_0 \omega_{\text{b}}}{2\pi} \ee^{- \Delta U / k_{\text{B}} T} = 
    \frac{\omega_0 \omega_{\text{b}}}{2\pi \gamma} \ee^{- \Delta U / k_{\text{B}} T}.
\end{equation}
Since we are in 1D, this is just the escaping rate.

\item[(f)] Assuming the dissipation is small enough (but is still sufficient to drive the system to equilibrium).
Now the equilibrium distribution function is 
\begin{equation}
    W(x, v) = \frac{1}{\mathcal{N}} \ee^{- \frac{M v^2 / 2 + U(x)}{k_{\text{B}} T}}.
\end{equation}
If a particle at $x_{\text{b}}$ with velocity $v$ is moving rightward,
it goes to the $x > x_{\text{b}}$ region.
So the transition rate is 
\[
    \begin{aligned}
        \Gamma &= \int_{v \geq 0} v \dd{v} W(x_{\text{b}}, v) \\
        &= \frac{1}{\mathcal{N}} \ee^{- U(x_0) / k_{\text{B}} T}
        \ee^{- \Delta U / k_{\text{B}} T} 
        \int_{v \geq 0} v \dd{v} \ee^{- \frac{M v^2 / 2}{k_{\text{B}} T}} \\
        &= \frac{1}{\mathcal{N}} \ee^{- U(x_0) / k_{\text{B}} T}
        \ee^{- \Delta U / k_{\text{B}} T}
        \frac{k_{\text{B}} T}{M}.
    \end{aligned}
\]
Now we find the normalization constant.
Similar to the above steps, we have 
\[
    \begin{aligned}
        1 &= \frac{1}{\mathcal{N}} \int \dd{x} \int \dd{v} \ee^{- \frac{M v^2 / 2 + U(x)}{k_{\text{B}} T}} \\
        &\approx  \frac{1}{\mathcal{N}}  \int \dd{x} \int \dd{v}
        \ee^{- \frac{M v^2 + M \omega_0^2 (x - x_0)^2}{k_{\text{B}} T}} 
        \ee^{- U(x_0) / k_{\text{B}} T }\\
        &= \frac{1}{\mathcal{N}} \sqrt{\frac{2\pi k_{\text{B}} T}{M}} 
        \sqrt{\frac{2\pi k_{\text{B}} T}{M \omega_0^2}} \ee^{- U(x_0) / k_{\text{B}} T }.
    \end{aligned}
\]
So we get 
\begin{equation}
    \Gamma = \left(\sqrt{\frac{2\pi k_{\text{B}} T}{M}} 
    \sqrt{\frac{2\pi k_{\text{B}} T}{M \omega_0^2}} \right)^{-1}
    \ee^{- \Delta U / k_{\text{B}} T}
    \frac{k_{\text{B}} T}{M}
    = \frac{\omega_0}{2\pi} \ee^{- \Delta U / k_{\text{B}} T}.
\end{equation}

\item[(g)] The two approximate conditions, $\gamma \to 0$ and 
\begin{equation}
    \gamma^{-1} \ll t \ll \Gamma_{\text{esc}}^{-1},
\end{equation}
are actually not contradictory:
the first condition is actually $\gamma \ll \omega_0$,
while the second is 
\[
    \Gamma_{\text{esc}} \ll \gamma,
\]
\[
    \frac{\omega_0}{2\pi} \ee^{- \Delta U / k_{\text{B}} T} \ll \gamma.
\]
It's of course possible to have 
\begin{equation}
    \frac{\omega_0}{2\pi} \ee^{- \Delta U / k_{\text{B}} T} \ll \gamma \ll \omega_0
\end{equation}
because $\Delta U$ is large, so there is no contradictory here.

The trajectories ignored are ``particles going to the $x > x_{\text{b}}$ region and then going back''.
In (d) and (e), these trajectories are automatically taken into consideration 
because $v$ has been integrated over.
When $\gamma$ is small enough, we are certain that even for a small region on the right of $x_{\text{b}}$,
the distribution isn't established,
so it's a good idea to ignore these trajectories.

\end{itemize}

\paragraph{Exercise 1 in lecture 4} 

\paragraph{Solution} The EOMs can be straightforwardly given as  
\begin{equation}
    \begin{aligned}
        M   \ddot{X}   &= K_1 (q_1 - X) + K_2 (q_2 - X), \\
        m_1 \ddot{q_1} &= K_1 (X - q_1) + F(q_1 - q_2), \\
        m_2 \ddot{q_2} &= K_2 (X - q_2) - F(q_1 - q_2). 
    \end{aligned}
\end{equation}
Here $F$ is the force between the two molecules.

\paragraph{Exercise 2 in lecture 5}

\paragraph{Solution} We first simply the original definition of $S_{XX}[\omega]$. We have 
\begin{equation}
    \begin{aligned}
        S_{XX}[\omega] &= \lim_{T_a \to \infty} \frac{1}{T_a} 
        \int_{-T_a / 2}^{T_a / 2} \dd{t_1} \int_{- T_a / 2}^{T_a / 2} \dd{t_2} 
        \ee^{\ii \omega (t_1 - t_2)} 
        \expval*{X(t + t_1) X(t + t_2)} \\
        &= \lim_{T_a \to \infty} \frac{1}{T_a} 
        \int_{- T_a / 2}^{T_a / 2} \dd{T} \int_{- T_a + 2T}^{T_a - 2T} \dd{\tau}
        \ee^{\ii \omega \tau} \expval*{X(t + T + \tau / 2) X(t + T - \tau / 2)} \\
        &= \lim_{T_a \to \infty} \frac{1}{T_a} 
        \int_{- T_a / 2}^{T_a / 2} \dd{T} \int_{- T_a + 2T}^{T_a - 2T} \dd{\tau}
        \ee^{\ii \omega \tau} \expval*{X(0) X(\tau)}.
    \end{aligned}
    \label{eq:def-1-eq}
\end{equation}
Here we define 
\begin{equation}
    t_1 = \frac{\tau}{2} + T, \quad t_2 = - \frac{\tau}{2} + T.
    \label{eq:another-time}
\end{equation}
The time translational symmetry is used,
which is always correct, or otherwise we need an additional time variable in the spectral density function.

On the other hand, the second definition 
\begin{equation}
    \left\langle X\left[\omega_1\right] X\left[\omega_2\right]\right\rangle=2 \pi S_{X X}\left[\omega_1\right] \delta\left(\omega_1+\omega_2\right)
\end{equation}
is equivalent to 
\[
    \begin{aligned}
        \expval*{X(t_1) X(t_2)} &= 
        \int_{-\infty}^\infty \frac{\dd{\omega_1}}{2\pi} \ee^{- \ii \omega_1 t_1} 
        \int_{-\infty}^\infty \frac{\dd{\omega_2}}{2\pi} \ee^{- \ii \omega_2 t_2} 
        2 \pi \delta(\omega_1 + \omega_2) S_{XX}[\omega_1] \\
        &= \int_{-\infty}^\infty \frac{\dd{\omega_1}}{2\pi} S_{XX}[\omega_1] \ee^{- \ii \omega_1 (t_1 - t_2)},
    \end{aligned}
\]
which is in turn equivalent to 
\begin{equation}
    \begin{aligned}
        S_{XX}[\omega] &= \int_{-\infty}^\infty \dd{t_1} \ee^{\ii \omega (t_1 - t_2)} \expval*{X(t_1) X(t_2)} \\
        &= \int_{-\infty}^\infty \dd{\tau} \ee^{\ii \omega \tau} 
        \expval*{X(T - \tau /2) X(T + \tau / 2)}  \\
        &= \int_{-\infty}^\infty \dd{\tau} \ee^{\ii \omega \tau} \expval*{X(0) X(\tau)} .
    \end{aligned}
    \label{eq:def-2-eq}
\end{equation}
Here the definition of $\tau$ and $T$ is the same as \eqref{eq:another-time}.

Now the two definitions are equivalent, 
if and only if \eqref{eq:def-1-eq} and \eqref{eq:def-2-eq} are equivalent.
As long as we can change the region of the integral in the RHS of \eqref{eq:def-1-eq}
into $[- T_a, T_a]$, we have
\[
    \begin{aligned}
        \text{RHS of \eqref{eq:def-1-eq}} &= 
        \lim_{T_a \to \infty} \frac{1}{T_a} \int_{- T_a / 2}^{T_a / 2} \dd{T} 
        \int_{- T_a}^{T_a} \dd{\tau} \ee^{\ii \omega \tau}
        \expval*{X(t - \tau / 2) X(t + \tau / 2)} \\
        &= \lim_{T_a \to \infty} \int_{- T_a}^{T_a} \dd{\tau} \ee^{\ii \omega \tau}
        \expval*{X(t - \tau / 2) X(t + \tau / 2)} = \text{RHS of \eqref{eq:def-2-eq}},
    \end{aligned}
\]
and thus the two definitions are equivalent.
We find as long as the correlation time scale of $\expval*{X(0) X(\tau)}$ is finite,
changing $[- T_a + 2T, T_a - 2T]$ in the RHS of \eqref{eq:def-1-eq} is acceptable,
because as \prettyref{fig:spectral-density-approx} shows,
we can always choose a very large $T_a$ compared with the correlation time scale of $\expval*{X(0) X(\tau)}$,
and in this case for most $T$'s,
integration with regard to $\tau$ over $[- T_a + 2T, T_a - 2T]$ 
has nothing different with integral over $[-T_a, T_a]$ or $(-\infty, \infty)$ 
(see the green line in \prettyref{fig:spectral-density-approx}),
because only the part in which $\abs{\tau}$ is smaller than the correlation time scale 
is important.

\begin{figure}
    \centering
    \begin{tikzpicture}[x=0.75pt,y=0.75pt,yscale=-1,xscale=1]
    %uncomment if require: \path (0,315); %set diagram left start at 0, and has height of 315
    
    %Straight Lines [id:da29302887512641407] 
    \draw [color={rgb, 255:red, 74; green, 144; blue, 226 }  ,draw opacity=0.6 ][line width=6]    (84.01,260.82) -- (234.82,110.01) ;
    %Shape: Square [id:dp3965860496444462] 
    \draw   (100,126) -- (218.83,126) -- (218.83,244.83) -- (100,244.83) -- cycle ;
    %Straight Lines [id:da10692747482044185] 
    \draw    (67.3,277.54) -- (250.12,94.71) ;
    \draw [shift={(251.54,93.3)}, rotate = 135] [fill={rgb, 255:red, 0; green, 0; blue, 0 }  ][line width=0.08]  [draw opacity=0] (12,-3) -- (0,0) -- (12,3) -- cycle    ;
    %Straight Lines [id:da4651029928641668] 
    \draw    (251.54,277.54) -- (68.71,94.71) ;
    \draw [shift={(67.3,93.3)}, rotate = 45] [fill={rgb, 255:red, 0; green, 0; blue, 0 }  ][line width=0.08]  [draw opacity=0] (12,-3) -- (0,0) -- (12,3) -- cycle    ;
    %Straight Lines [id:da12570083559205503] 
    \draw [color={rgb, 255:red, 74; green, 144; blue, 226 }  ,draw opacity=1 ]   (81.33,258.33) -- (67.61,272.06) ;
    %Straight Lines [id:da8653694800743532] 
    \draw [color={rgb, 255:red, 74; green, 144; blue, 226 }  ,draw opacity=1 ]   (87,262.94) -- (73.28,276.67) ;
    %Straight Lines [id:da2534623004469161] 
    \draw [color={rgb, 255:red, 74; green, 144; blue, 226 }  ,draw opacity=1 ]   (58.15,235.15) -- (79.92,256.92) ;
    \draw [shift={(81.33,258.33)}, rotate = 225] [fill={rgb, 255:red, 74; green, 144; blue, 226 }  ,fill opacity=1 ][line width=0.08]  [draw opacity=0] (12,-3) -- (0,0) -- (12,3) -- cycle    ;
    %Straight Lines [id:da3424301928975353] 
    \draw [color={rgb, 255:red, 74; green, 144; blue, 226 }  ,draw opacity=1 ]   (88.41,264.36) -- (110.19,286.13) ;
    \draw [shift={(87,262.94)}, rotate = 45] [fill={rgb, 255:red, 74; green, 144; blue, 226 }  ,fill opacity=1 ][line width=0.08]  [draw opacity=0] (12,-3) -- (0,0) -- (12,3) -- cycle    ;
    %Straight Lines [id:da9266881802278928] 
    \draw [color={rgb, 255:red, 80; green, 227; blue, 194 }  ,draw opacity=1 ]   (100.33,196.08) -- (148.61,244.35) ;
    %Curve Lines [id:da5722891533132757] 
    \draw [color={rgb, 255:red, 80; green, 227; blue, 194 }  ,draw opacity=1 ]   (66.94,173.52) .. controls (86.08,184.08) and (69.34,238.42) .. (107.82,210.4) ;
    \draw [shift={(109,209.52)}, rotate = 143.13] [fill={rgb, 255:red, 80; green, 227; blue, 194 }  ,fill opacity=1 ][line width=0.08]  [draw opacity=0] (12,-3) -- (0,0) -- (12,3) -- cycle    ;
    
    % Text Node
    \draw (253.54,89.9) node [anchor=south west] [inner sep=0.75pt]    {$T=\frac{t_{1} +t_{2}}{2}$};
    % Text Node
    \draw (67.3,89.9) node [anchor=south] [inner sep=0.75pt]    {$\tau =t_{1} -t_{2}$};
    % Text Node
    \draw (161.19,250.48) node [anchor=north] [inner sep=0.75pt]    {$T_{a}$};
    % Text Node
    \draw (42.33,284.67) node [anchor=north west][inner sep=0.75pt]  [color={rgb, 255:red, 74; green, 144; blue, 226 }  ,opacity=1 ] [align=left] {correlation time scale};
    % Text Node
    \draw (-0.33,143.97) node [anchor=north west][inner sep=0.75pt]  [color={rgb, 255:red, 80; green, 227; blue, 194 }  ,opacity=1 ] [align=left] {One integraion \\path of $\displaystyle \tau $};
    
    
    \end{tikzpicture}
    
    \caption{The integral in \eqref{eq:def-1-eq}. 
    The blue region labels the region in which $\expval*{X(0) X(\tau)}$ has 
    considerable absolute values.
    Its width is the scale of the time correlation.} 
    \label{fig:spectral-density-approx}
\end{figure}

So as long as $X(t)$ is stationary and $\expval*{X(0) X(\tau)}$ has a finite correlation time,
the two definitions are equivalent.

\paragraph{Exercise 3 in lecture 6}

\paragraph{Solution} We have 
\[
    \begin{aligned}
        \left\langle\delta V\left(t_1\right) \delta V\left(t_2\right)\right\rangle &= q \int_0^{t_1} \int_0^{t_2} \mathrm{e}^{-\gamma\left(t_1-t_1^{\prime}\right)} \mathrm{e}^{-\gamma\left(t_2-t_2^{\prime}\right)} \delta\left(t_1^{\prime}-t_2^{\prime}\right) \dd t_1^{\prime} \dd t_2^{\prime} \\
        &= q \ee^{- \gamma (t_1 + t_2)} \int_0^{t_1} \dd{t_1'} \int_0^{t_2} \dd{t_2'}
        \ee^{\gamma (t_1' + t_2')} \delta (t'_1 - t'_2). 
    \end{aligned}
\]
When $t_1 > t_2$, in the $t_1' > t_2$ region, 
it's impossible to have $t_1' = t_2'$, and the $\delta$-function is always zero, so we have 
\[
    \begin{aligned}
        \expval*{\delta V(t_1) \delta V(t_2)} &= q \ee^{- \gamma (t_1 + t_2)} 
        \int_{0}^{t_2} \dd{t_1'} \int_0^{t_2} \dd{t_2'} \ee^{\gamma (t_1' + t_2')} \delta (t_1' - t_2')  \\
        &= q \ee^{- \gamma (t_1 + t_2)} \int_0^{t_2} \dd{t_2'} \ee^{2 \gamma t_1'} \\
        &= q \ee^{- \gamma (t_1 + t_2)} \frac{1}{2 \gamma} (\ee^{2 \gamma t_2} - 1) \\
        &= \frac{q}{2 \gamma} (\ee^{- \gamma (t_1 - t_2)} - \ee^{- \gamma (t_1 + t_2)}).
    \end{aligned}
\] 
Due to the symmetry between $t_1$ and $t_2$,
when $t_1 < t_2$, we can just switch the two variables and we have 
\[
    \expval*{\delta V(t_1) \delta V(t_2)} = 
    \frac{q}{2 \gamma} (\ee^{- \gamma (t_2 - t_1)} - \ee^{- \gamma (t_1 + t_2)}).
\]
So putting the two cases together, we get 
\begin{equation}
    \expval*{\delta V(t_1) \delta V(t_2)} = 
    \frac{q}{2 \gamma} (\ee^{- \gamma \abs{t_1 - t_2}} - \ee^{- \gamma (t_1 + t_2)}).
\end{equation}

\paragraph{Exercise 2 in lecture 7}

\paragraph{Solution} What we need to show is when the Fokker-Planck-like equation is not derived 
from the Langevin equation,
\begin{equation}
    W = \frac{1}{\mathcal{N}} \ee^{- \frac{M v^2 / 2 + U}{k_{\text{B}} T}}
    \label{eq:boltzmann}
\end{equation}
is not a solution of the equation.
To see if it is so, we check the condition to have both \eqref{eq:boltzmann} and 
\begin{equation}
    \left(-\frac{\partial}{\partial x} v+\frac{\partial}{\partial v}[\gamma v-f(x)]+\frac{\partial^2}{\partial v^2} \gamma^2 D\right) W = 0.
    \label{eq:stationary-fp}
\end{equation}
We have 
\[
    v \pdv{W}{x} = \frac{v}{k_{\text{B}} T} \left( - \pdv{U}{x} \right) W = \frac{v}{k_{\text{B}} T} F W, 
\]
\[
    f \pdv{W}{v} = - f \frac{M v}{k_{\text{B}} T} W = - \frac{v}{k_{\text{B}} T} F W,
\]
so the first and the third term in \eqref{eq:stationary-fp} cancel with each other.
We also have 
\[
    \pdv{v} (\gamma v W) = \gamma W - \gamma v \cdot \frac{M v}{k_{\text{B}} T} W,
\]
and 
\[
    \pdv[2]{W}{v} = \pdv{v} \left( - \frac{M v}{k_{\text{B}} T} W \right) 
    = - \frac{M}{k_{\text{B}} T} W + \left( \frac{M v}{k_{\text{B}} T} \right)^2.
\]
So after extracting out common factors, we get 
\begin{equation}
    \text{LHS of \eqref{eq:stationary-fp}} = \left( \gamma - \gamma^2 D \frac{M}{k_{\text{B}} T} \right) 
    \left(1 - \frac{M v^2}{k_{\text{B}} T}\right) W.
\end{equation}
To make the RHS of the equation above constantly zero,
we can't rely on the second factor containing $v$ -- our only choice is 
\[
    \gamma = \underbrace{\gamma^2 D}_{\mathcal{D}} \frac{M}{k_{\text{B}} T},
\]
which is just 
\begin{equation}
    \mathcal{D} = \frac{\gamma}{M} k_{\text{B}} T,
\end{equation}
which appears in the Langevin equation.
So when this condition is broken, \eqref{eq:boltzmann} is not a solution of the Fokker-Planck-like equation.

\bibliographystyle{plain}
\bibliography{equations}

\end{document}