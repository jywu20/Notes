\documentclass[hyperref, a4paper]{article}

\usepackage{geometry}
\usepackage{titling}
\usepackage{titlesec}
% No longer needed, since we will use enumitem package
% \usepackage{paralist}
\usepackage{enumitem}
\usepackage{footnote}
\usepackage[colorinlistoftodos]{todonotes}
\usepackage{amsmath, amssymb, amsthm}
\usepackage{mathtools}
\usepackage{bbm}
\usepackage{graphicx}
\usepackage{subcaption}
\usepackage{soulutf8}
\usepackage{physics}
\usepackage{tensor}
\usepackage{siunitx}
\usepackage[version=4]{mhchem}
\usepackage{tikz}
\usepackage{xcolor}
\usepackage{listings}
\usepackage{autobreak}
\usepackage[ruled, vlined, linesnumbered]{algorithm2e}
\usepackage{nameref,zref-xr}
\zxrsetup{toltxlabel}
\usepackage[backend=bibtex]{biblatex}
\addbibresource{nonequilibrium.bib}
\usepackage[colorlinks,unicode]{hyperref} % , linkcolor=black, anchorcolor=black, citecolor=black, urlcolor=black, filecolor=black
\usepackage[most]{tcolorbox}
\usepackage{prettyref}

% Page style
\geometry{left=3.18cm,right=3.18cm,top=2.54cm,bottom=2.54cm}
\titlespacing{\paragraph}{0pt}{1pt}{10pt}[20pt]
\setlength{\droptitle}{-5em}

% More compact lists 
\setlist[itemize]{
    itemindent=17pt, 
    leftmargin=1pt,
    listparindent=\parindent,
    parsep=0pt,
}

% Math operators
\DeclareMathOperator{\timeorder}{\mathcal{T}}
\DeclareMathOperator{\diag}{diag}
\DeclareMathOperator{\legpoly}{P}
\DeclareMathOperator{\primevalue}{P}
\DeclareMathOperator{\sgn}{sgn}
\DeclareMathOperator{\res}{Res}
\newcommand*{\ii}{\mathrm{i}}
\newcommand*{\ee}{\mathrm{e}}
\newcommand*{\const}{\mathrm{const}}
\newcommand*{\suchthat}{\quad \text{s.t.} \quad}
\newcommand*{\argmin}{\arg\min}
\newcommand*{\argmax}{\arg\max}
\newcommand*{\normalorder}[1]{: #1 :}
\newcommand*{\pair}[1]{\langle #1 \rangle}
\newcommand*{\fd}[1]{\mathcal{D} #1}
\DeclareMathOperator{\bigO}{\mathcal{O}}

% TikZ setting
\usetikzlibrary{arrows,shapes,positioning}
\usetikzlibrary{arrows.meta}
\usetikzlibrary{decorations.markings}
\tikzstyle arrowstyle=[scale=1]
\tikzstyle directed=[postaction={decorate,decoration={markings,
    mark=at position .5 with {\arrow[arrowstyle]{stealth}}}}]
\tikzstyle ray=[directed, thick]
\tikzstyle dot=[anchor=base,fill,circle,inner sep=1pt]

% Algorithm setting
% Julia-style code
\SetKwIF{If}{ElseIf}{Else}{if}{}{elseif}{else}{end}
\SetKwFor{For}{for}{}{end}
\SetKwFor{While}{while}{}{end}
\SetKwProg{Function}{function}{}{end}
\SetArgSty{textnormal}

\newcommand*{\concept}[1]{{\textbf{#1}}}

% Embedded codes
\lstset{basicstyle=\ttfamily,
  showstringspaces=false,
  commentstyle=\color{gray},
  keywordstyle=\color{blue}
}

% Reference formatting
\newcommand*{\citesec}[1]{\S~{#1}}
\newcommand*{\citechap}[1]{chap.~{#1}}
\newcommand*{\citefig}[1]{Fig.~{#1}}
\newcommand*{\citetable}[1]{Table~{#1}}
\newcommand*{\citepage}[1]{pp.~{#1}}
\newrefformat{fig}{Fig.~\ref{#1}}
\newcommand*{\term}[1]{\textit{#1}}

% Color boxes
\tcbuselibrary{skins, breakable, theorems}

\newtcbtheorem{infobox}{Box}{
    enhanced,
    boxrule=0pt,
    colback=blue!5,
    colframe=blue!5,
    coltitle=blue!50,
    borderline west={4pt}{0pt}{blue!65},
    sharp corners,
    fonttitle=\bfseries, 
    breakable,
    before upper={\parindent15pt\noindent}}{box}
\newtcbtheorem[use counter from=infobox]{theorybox}{Box}{
    enhanced,
    boxrule=0pt,
    colback=orange!5, 
    colframe=orange!5, 
    coltitle=orange!50,
    borderline west={4pt}{0pt}{orange!65},
    sharp corners,
    fonttitle=\bfseries, 
    breakable,
    before upper={\parindent15pt\noindent}}{box}
\newtcbtheorem[use counter from=infobox]{learnbox}{Box}{
    enhanced,
    boxrule=0pt,
    colback=green!5,
    colframe=green!5,
    coltitle=green!50,
    borderline west={4pt}{0pt}{green!65},
    sharp corners,
    fonttitle=\bfseries, 
    breakable,
    before upper={\parindent15pt\noindent}}{box}


\newenvironment{shelldisplay}{\begin{lstlisting}}{\end{lstlisting}}

\newcommand*{\kB}{k_{\text{B}}}
\newcommand*{\muB}{\mu_{\text{B}}}
\newcommand*{\efermi}{E_{\text{F}}}
\newcommand*{\pfermi}{p_{\text{F}}}
\newcommand*{\vfermi}{v_{\text{F}}}
\newcommand*{\sA}{\text{A}}
\newcommand*{\sB}{\text{B}}
\newcommand*{\Tc}{T_{\text{c}}}
\newcommand*{\hethree}{$^3$He}
\newcommand*{\hefour}{$^4$He}
\newcommand{\epsr}{\epsilon_{\text{r}}}
\newcommand{\chie}{\chi_{\text{e}}}
\newcommand{\cf}{c_{\text{F}}}
\newcommand{\fn}{F_{\text{N}}}
\newcommand{\ff}{F_{\text{f}}}

\title{Fluctuation}
\author{Jinyuan Wu}


\begin{document}

\maketitle

\section{Motivation}

\subsection{Brownian motion in a fluid}

The experimental motivation to study fluctuation is 
probably the Brownian motion: 
in this specific case, 
large particles in fluid feel the fluctuation of the fluid,
and undergo random motion.
There are three length scales and three time scales 
in a typical experimental setting:
\begin{itemize}
    \item The molecule scales: \SI{1e-10}{m}, \SI{1e-12}{s};
    \item The particle scales: typically \SI{1e-6}{m}, \SI{1e-3}{s};
    \item The enclosure (i.e. container of the fluid) scales: 
    at least \SI{1e-2}{m}, \SI{1e2}{s}.
\end{itemize}
The separation between the molecule scales and the particle scales 
justifies treating the fluid as a continuum,
while the separation between the particle scales 
and the enclosure scales 
justifies treating the configuration space of the particles 
as an infinite one.
There actually should be another scale: 
the mean distance between particles;
if this is large enough, 
then particles are independent to each other; 
but we may want to consider this as the enclosure scales.

When the aforementioned conditions of separation of scales are all true, 
the behavior of a particle independent to others is described 
by the \concept{Langevin equation} 
\begin{equation}
    M \dv[2]{X}{t} + \pdv{U}{X} = - c_{\text{F}} \dot{X} + F_{\text{N}}(t),
    \label{eq:langevin}
\end{equation}
the terms on the LHS being effects about the particle itself,
the terms on the RHS being the deterministic damping term 
and the stochastic Langevin force, accordingly.

Solving \eqref{eq:langevin}, therefore, involves two steps:
finding reliable descriptions of $c_{\text{F}}$ and $F_{\text{N}}$,
and solving \eqref{eq:langevin} as a stochastic differential equation.
The first problem is essentially about establishing a theory 
of fluctuation ($F_{\text{N}}$) and response ($\gamma$) 
in the fluid surrounding the particle,
and the second problem is about the mathematical treatment 
of stochastic processes.

The two problems are actually two specific cases 
of the underlying quantum non-equilibrium theory.
\cite{sieberer2016keldysh} contains a neat review of it; 
the most generic formalism however is not necessary 
in most of the cases, 
and this note would rather focus on more concrete cases.
Also, phenomena described by the Langevin equation 
are also described by various master equations,
classical or quantum;
the two formalisms are also equivalent to each other
\cite{van1997langevin,coppola2023lindblad}.
The details however are also emitted.

\subsection{Quantum circuit}

The formalism of Langevin equation is not limited to soft condensed matter systems:
since formally Newton's second law 
is analogous to circuit theory,
fluctuation and damping in a circuit immersed in 
a sea of smaller circuits can also be described in Langevin's formalism.
Indeed this is how resistance comes into being \todo{Ref.}
In circuit analysis we have another problem:
now the Langevin equation should be \emph{quantum},
since although no one would perform a Brownian motion experiment 
in a low temperature quantum liquid,
we indeed can perform quantum circuit experiments 
at a very low temperature.

\todo{Commutation relations?}

\section{Correlation, response, and fluctuation-dissipation theorem} 

In this section we briefly review the linear response theory.
In general, a driven Hamiltonian assumes the following form: 
\begin{equation}
    H = H_0(x_1, \ldots, x_n, p_1, \ldots, p_n) - A(x_1, \ldots, x_n, p_1, \ldots, p_n) F_A(t),
\end{equation}
where we have assumed that degrees of freedom in the system 
follow the canonical commutation relation;
this assumption of course is frequently broken
(as in, say, a magnetic spin model),
but in this note let's focus on the most familiar case. 
We can readily have two examples: 
the first is 
\begin{equation}
    H = \frac{p^2}{2m} + \frac{1}{2} k x^2 - x F(t),
\end{equation}
of which the EOM is 
\begin{equation}
    \dot{p} = - kx + F(t), \quad 
    \dot{x} = \frac{p}{m} ,
\end{equation}
as is expected.
The second is 
\begin{equation}
    H = \frac{\Phi^2}{2L} + \frac{Q^2}{2L} - Q V,
\end{equation}
\todo{Commutation relations: cQED?}
Here we intentionally leave the discussion of damping;
it can't be directly treated in the Hamiltonian formalism 
and will be modeled by an infinite bath of other degrees of freedom.
Suppose we have another physical quantity $B$ 
which is also a polynomial of $x_1, \ldots, x_n$
and $p_1, \ldots, p_n$.
When the external driving force $F(t)$ is applied, 
we get 
\begin{equation}
    \var{B}(t) \coloneqq B(t) - B_0 
    = \int_{-\infty}^\infty \dd{t_1} \chi_{BA}(t, t_1) F_A(t_1) 
    + B_{\text{noise}} + \bigO{(F_A^2)}.
\end{equation}
The first term equals to the linear term in $\expval{\var{B}(t)}$;
the noise term contains information about correlation of $B$
with other variables -- or maybe itself.
It can be statistical noise (noise in classical probability theory)
when there is a bath (see below);
but even if we are dealing with a pure state theory,
$B_{\text{noise}} = B - \expval{B}$ will still bring some ``quantum noise''.
The response function is 
\begin{equation}
    \chi_{BA}(t, t_1) = \chi_{BA}(t - t_1)
\end{equation}
when $H_0$ is time-independent.

Now we turn to another aspect: correlation.
We define 
\begin{equation}
    S_{BA}(\omega) = \int \ee^{\ii \omega t} \expval{B(0) A(t)} \dd{t}.
\end{equation}
This is known as the greater Green function or the lesser Green function 
when $A$ and $B$ are field operators.
We now state the following \concept{fluctuation-dissipation theorem},
which links the near-equilibrium lesser Green function 
and the response function (i.e. the retarded function):
for a classical system, we have 
\begin{equation}
    \expval*{\dot{A}(0) B(t)} \theta(t) = \kB T \chi_{BA}(t),
\end{equation}
and in the frequency domain, 
if we assume that the response function 
is even under time reversal operation, we get 
\begin{equation}
    S_{\dot{A} B} (\omega) = 2 \kB T \Re \chi_{BA}(\omega).
\end{equation}
and the equation becomes 
\begin{equation}
    S_{AB}(\omega) = \frac{2 \kB T}{\omega} \Im \chi_{AB}(\omega).
\end{equation}
\todo{Details about time reversal; the point is how to get rid of $\theta(t)$}
In the quantum case we have 
\begin{equation}
    S_{AB}(\omega) = \frac{2\hbar}{1 - \ee^{- \hbar \omega / \kB T}} \Im \chi_{AB}(\omega).
\end{equation}
To save space the proof of the theorem is not shown here; 
a straightforward derivation can be found in section 9.4 in \cite{coleman2015introduction}.
Note that in the $T \to 0$ limit, 
classically $S_{AB}$ vanishes,
but in the quantum case we still have a non-vanishing remaining correlation
as long as we see a response of $A$ to $B$
which is known as quantum noise:
if $A$ responds to the $- B f(t)$ term in the Hamiltonian,
$A$ and $B$ don't commute (or otherwise we don't have non-trivial time evolution),
and thus $S_{AB}$ is definitely not zero when $T \to 0$
i.e. when in the pure state theory.

\section{Analysis of common noises}

\todo{Topic: what does noise do to a signal?}

\section{Langevin equation revisited}

\subsection{Correlation function of the noise}

In the light of fluctuation-dissipation theorem, 
we see that although the details of $\fn(t)$ and $\cf$
are still largely unknown,
if $\fn(t)$ and $\cf$ ``naturally'' appear 
by coupling with the bosonic bath, 
the properties of $\fn(t)$ and $\cf$ in 
\begin{equation}
    M \ddot{X} + \cf \dot{X} + \pdv{U}{X} = \fn(t)
\end{equation}
have to follow the fluctuation-dissipation theorem.
We know the force applied by the fluid to the particle is 
\begin{equation}
    \ff = - k v + \fn(t),
\end{equation} 
and the coupling Hamiltonian between the particle and the fluid is 
\begin{equation}
    H_{\text{int}} = - x \ff(t).
\end{equation}
Thus we choose $A$ and $B$ to be both $\ff$,
and fluctuation-dissipation theorem tells us 
\begin{equation}
    \expval*{\dot{F}(t') F(t)} \theta(t - t') = \kB T \chi_{\ff(t') \to \ff(t)} (t - t').
\end{equation}
Here we have gone back to the double-time formalism 
since we want to eventually get rid of the time derivative acting 
on the first $F$ in the correlation function 
and writing the time of the first $F$ 
makes the derivation less confusing. 
Also, note that we are applying the fluctuation-dissipation theorem 
\emph{to the fluid} (the particle doesn't have linear response anyway)
and therefore now $x$ should be regarded as the external field.
Given the form of the interaction Hamiltonian, 
we find 
\begin{equation}
    \chi_{\ff(t') \to \ff(t)} (t - t') = 
    \fdv{\ff(t)}{x(t')} ,
\end{equation}
and therefore we want to rephrase the proportional relation 
between $\ff$ and $v$
as a functional between $\ff(t)$ and $x(t')$.
This is done by the follows:
\begin{equation}
    \ff(t) = - \cf v = - \int \dd{t'} \cf \delta(t - t') \dv{x}{t'}
    = \int \dd{t'} x(t') \cf \dv{t'} \delta(t - t') .
\end{equation}
The definition of the $\delta(t - t')$ function however involves a subtlety:
in reality, $t$ should always be slightly larger than $t'$ 
or otherwise causality is broken,
and $\delta(t - t')$ therefore should be understood as 
$\delta(t - t' - 0^+)$.
The fluctuation-dissipation theorem now reads 
\begin{equation}
    \expval*{\dot{F}(t') F(t)} \theta(t - t') = \kB T \chi_{\ff(t') \to \ff(t)} (t - t')
    = \kB T \cdot \cf \dv{t'} \delta(t - t' - 0^+) .
\end{equation}
In other words, we have 
(note that in the equation below, 
there should be no infinitesimal time shift in $\delta$ and stepwise functions)
\begin{equation}
    \begin{aligned}
        \dv{t'} (
            \expval*{F(t') F(t)} \theta(t - t')
        ) &= \expval*{\dot{F}(t') F(t)} \theta(t - t')
        - \expval*{F(t') F(t)} \delta(t - t') \\
        &= \kB T \cf \dv{t'} \delta(t - t') 
        - \expval*{F(t') F(t)} \delta(t - t'),
    \end{aligned}
\end{equation}
and we can then integrate over the equation on the both sides over $t'$ and get 
\begin{equation}
    \expval*{F(t') F(t)} \theta(t - t') = \kB T \cf \delta(t - t'),
\end{equation}
where since the $\delta$ function in $\expval*{F(t') F(t)}$
contains an infinitesimal positive shift, 
when $t$ is \emph{exactly} $t'$ it should be zero.
(Or, to be more rigorous, we can always replace 
$\delta(t - t' - 0^+)$ by a smooth function that 
vanishes when $t < t'$, 
and we can check that the equations above are all correct.)
The correlation function in fluctuation-dissipation theorem 
is evaluated when the fluid is at rest,
and therefore $\ff(t) = \fn(t)$, and we find 
\begin{equation}
    \expval*{\fn(t') \fn(t)} \theta(t - t') = \kB T \cf \delta(t - t').
    \label{eq:proto-langevin-correlation}
\end{equation}
The $\delta(t - t')$ function in \eqref{eq:proto-langevin-correlation}
is double-sided: it looks like a Gaussian peak,
and therefore 
\begin{equation}
    \theta(t - t') \delta(t - t') = \frac{1}{2} \delta(t - t'),
\end{equation}
and eventually, the correlation function of $\fn$ reads 
\begin{equation}
    \expval{\fn(0) \fn(t)} = 2 \kB T \cf \delta(t).
\end{equation}

In conclusion, if we are sure that the damping force from the fluid 
is in the form of $F = - k v$,
then the EOM of a particle in the fluid always takes the form of 
\begin{equation}
    M \ddot{X} + \cf \dot{X} + \pdv{U}{X} = \fn(t), \quad 
    \expval{\fn(0) \fn(t)} = 2 \kB T \cf \delta(t).
    \label{eq:langevin-mechanic}
\end{equation} 
Thanks to the fluctuation-dissipation theorem,
we now find that the statistical properties of the noise 
is related to the damping coefficient.

\subsection{An implementation: fluid as a number of harmonic oscillators}

In principle higher order non-trivial correlations are possible; 
but in most of the systems that are experimentally feasible,
the environment can be very effectively modeled as 
a heat bath containing lots of harmonic oscillators.
In this case $\fn(t)$ should be a Gaussian noise.
Thus, the smooth, realistic form of the double-sided $\delta(t)$ is something like 
\begin{equation}
    \frac{1}{\tau} \ee^{- \abs*{t} / \tau}.
\end{equation}

In this section we explicitly show how this can be done.
If the fluid can be modeled as a pool of harmonic oscillators, 
the EOMs are 
\begin{equation}
    M \ddot{X} = - \pdv{U}{X} + \sum_{i} k_i (x_i - X), \quad 
    m_i \ddot{x}_i = - k_i (x_i - X) \Rightarrow 
    \ddot{x}_i + \omega^2 x_i = \omega^2 X ,
\end{equation}
where 
\begin{equation}
    \omega_i = \sqrt{\frac{k_i}{m_i}}.
\end{equation}
We can directly verify that 
\begin{equation}
    x_i(t) = a_i \sin(\omega_i t + \phi_i) 
    + \int_{-\infty}^{t} \dd{\tau} \cos(\omega_i (t - \tau)) \dot{X}(\tau)
    + X(\tau),
\end{equation}
and therefore the effective EOM for the particle is 
\begin{equation}
    M \ddot{X} = - \pdv{U}{X} + 
    \int_{-\infty}^{t} \underbrace{
        \sum_i k_i \cos(\omega_i (t - \tau))
    }_{K(t - \tau)} \dot{X}(\tau)
    + \underbrace{
        \sum_i k_i a_i \sin(\omega_i t + \phi_i)
    }_{\fn(t)} .
\end{equation}
Now we assume $\{x_i\}$ to be in thermal equilibrium;
this is somehow a self-conflicting assumption 
since we have assumed no interaction between the oscillators 
and they simply won't thermalize; 
but we can always assume that the oscillators are coupled 
with some external reservoir and hence are thermalized,
or that there is weak interaction among them,
which is weak enough to have no strong correction to the energy 
but strong enough to lead to thermalization.
The energy of each oscillator is 
\begin{equation}
    E = \frac{p_i^2}{2m_i} + \frac{1}{2} m_i \omega_i^2 x^2_i
    = m_i \omega_i^2 a_i^2,
\end{equation}
and from the equipartition theorem we find 
\begin{equation}
    \expval{E} = \kB T \Rightarrow \expval*{a_i^2} = \frac{2\kB T}{m_i \omega_i^2},
\end{equation}
while 
\begin{equation}
    \expval*{a_i a_j} = 0 \quad i \neq j.
\end{equation}
We thus find 
\begin{equation}
    \begin{aligned}
        \expval*{\fn(0) \fn(t)} &= 
        \sum_{i, j} \expval*{a_i a_j} k_i k_j \sin \phi_i \sin(\omega_j t + \phi_j) \\
        &= \sum_i \frac{2 \kB T}{m_i \omega_i^2} k_i^2 \sin \phi_i \sin (\omega_i t + \phi_i) \\
        &= \kB T \sum_{i} 2 k_i \sin \phi_i \sin(\omega_i t + \phi_i) \\
        &= \kB T \sum_i k_i (\cos(\omega_i t) - \cos(\omega_i t + 2 \phi_i)) \\
        &\approx \kB T \sum_i k_i \cos (\omega_i t),
    \end{aligned}
\end{equation}
where in the last line we have assumed that 
there are so many oscillators and their phases are random,
so the second term in the last second line vanishes.
Thus we find 
\begin{equation}
    \expval*{\fn(0) \fn(t)} = \kB T K(t).
\end{equation}
Now we get 

\subsection{Some notes on mathematics of stochastic processes}

Most of the time in this note,
the noise term does not depend on the coordinates,
and that's why Langevin's equation is well-defined mathematically.
When we are dealing with something like 
\begin{equation}
    \dot{X} = A(X) + B(X) F(t),
\end{equation}
where $F(t)$ is a random variable, the interpretation of the equation becomes problematic:
it is unclear whether the $X$ in $B(X)$ is $X(t)$ or $X(t + \dd{t})$ or whatever else.
This does not influence the time evolution in deterministic calculus,
but in stochastic calculus, the non-trivial correction function of the noise term means that 
$\dd{X}^2$ can have a term proportional to $\dd{t}$,
and hence the difference between $B(X(t))$ and $B(X(t + \dd{t}))$ accumulates.
For instance, if we consider \eqref{eq:langevin-mechanic}, we find 
\begin{equation}
    \expval{(F_{\text{N}}(t) \dd{t})^2} = 2 \kB T c_{\text{F}} \dd{t}.
    \label{eq:fdt-sq}
\end{equation}
This can also be expected by considering an infinitesimal small random walk process.

Two frequently used conventions are the Ito scheme and the Stratonovich scheme.
The first regards $B(X)$ as $B(X(t))$,
while the second regards $B(X)$ as 
\[
    \frac{B(X(t + \dd{t})) + B(X(t))}{2}.
\]
This means transforming between the two conventions alternates the definition of $A(X)$ if we're to keep the form of the $B(X) F(t)$ term formally invariant.
Which, in turn, means the heuristic derivation of stochastic equations of motion is not
completely conceptually sound,
as the form of $A(X)$ now changes depending on how we interpret $B(X)$.
This is to be expected, as to be honest,
we do not really have any guarantee that $A(X)$ should have the identical form that it has should there be no fluctuations.

Journal of Statistical Physics, Vol. 24, No. 1, 1981 is a thorough discussion 
on the comparison between the two conventions.
The Stratonovich scheme is arguably preferable as it can be derived from 
a non-Markovian process by making the correlation time of the noise shorter and shorter.
But this does not prevent us from modifying $A(X)$ and getting an equivalent Ito description.

When the fluctuations are independent enough to the system,
we can argue that $A(X)$ should not change and write an equation of motion with
$A(X)$ and a deterministic version of $F(t)$,
and then assume randomness and calculate the correlation spectrum of $F(t)$
and take the short correlation time limit.
This naturally gives rise to a Stratonovich stochastic differential equation
in which $A(X)$ is identical to its form when the fluctuation is off.
When the fluctuation is less independent,
strictly speaking, the whole system should be modeled as a whole by a master equation,
and generally the Langevin-like equation does not correctly capture
higher order correlation functions in the system.
$A(X)$ can still be defined in terms of expectation values,
but how it is defined is more a notational thing.
(What is essentially the same question:
can you have a universal definition ``renormalized frequency'' of a state 
suitable to all spectroscopic techniques
when its self-energy correction has a large imaginary part and very complicated frequency dependence?)
The naive idea that there always exists a well-defined $A(X)$
is simply not reliable.

The Ito-vs-Stratonovich distinction also matters in interpreting any expression containing $X$.
When a function of $X$ and the time $t$, $f(X, t)$ can be expanded as
(here any $Y \dd{X}$ form is to be interpreted in the Ito scheme)
\begin{equation}
    \dd{f} = \pdv{f}{t} \dd{t} + \pdv{f}{X} \dd{X} + \frac{1}{2} \pdv[2]{f}{X} \dd{X}^2 ,
    \label{eq:taylor}
\end{equation}
where we have included the $\dd{X}^2$ term because of the aforementioned problem that 
$\dd{X}^2$ can have a component proportional to $\dd{t}$.
As 
\begin{equation}
    \dd{X} = A(X) \dd{t} + B(X(t)) \underbrace{F(t) \dd{t}}_{\dd{W}},
\end{equation}
where $\dd{W}$ is the `'`infinitesimal noise''.
From \eqref{eq:fdt-sq} we know the only non-zero contribution to $\dd{X}^2$ term
is $B(X)^2$ times the variance of $\dd{W}$.
Thus the final form of $\dd{f}$ is 
\begin{equation}
    \dd{f} = \left(
        \pdv{f}{t} + \frac{1}{2} \pdv[2]{f}{X} B(X)^2 \sigma_F^2
    \right) \dd{t} + \pdv{f}{X} \dd{X},
\end{equation}
which is known as the Ito lemma.
We emphasize here that Ito lemma is only correct when the noise is a Markovian, Gaussian one.

In Stratonovich calculus, we interpret forms like $Y \dd{X}$ in a different way,
and thus 
\begin{equation}
    \begin{aligned}
        \eval{\pdv{f}{X} \dd{X}}_{\text{Stratonovich}} &= 
        \frac{1}{2} \left(
            \eval{\pdv{f}{X} \dd{X}}_{t} + 
            \eval{\pdv{f}{X} \dd{X}}_{t + \dd{t}}
        \right) \\
        &= \pdv{f}{X} \dd{X} + \frac{1}{2} \pdv[2]{f}{X} \dd{X}^2.
    \end{aligned}
\end{equation}
Thus \eqref{eq:taylor}, in the Stratonovich scheme, becomes
\begin{equation}
    \dd{f} = \pdv{f}{X} \dd{X} + \pdv{f}{t} \dd{t},
\end{equation}
which looks identical to the chain rule in deterministic calculus,
although the reason is that $\pdv{f}{X} \dd{X}$ is defined in a clever way.

\section{Fokker-Planck equation}

The Langevin equation in the last section is a specific case of the following SDE 
\begin{equation}
    \dot{X} = A W + F, \quad 
    \expval{F_i(0) F_j(t)} = B_{ij} \delta(t),
\end{equation}
the time evolution of the probabilistic distribution $W(X_1, X_2, \ldots, X_n)$ 
can be rewritten into  
\begin{equation}
    \pdv{W}{t} = \underbrace{
        - \div{A W}
    }_{\text{drift}} + 
    \underbrace{
        \frac{1}{2} \div \div (B W)
    }_{\text{diffusion}} . 
\end{equation}
This equation is known as the \concept{Fokker-Planck equation}.

The classical tunneling rate is \todo{derivation}
\begin{equation}
    \Gamma = \frac{\omega_0}{2\pi} \ee^{- \Delta U / \kB T}
\end{equation}

\section{Transmission line and resistance}

We can also study the infinite transmission line,
the counterpart of the infinite harmonic oscillator array 
attached to the particle,
from the perspective of linear response -- 
and show why it looks like a resistor.
This time we proceed in the frequency domain,
and we will see it makes many things simpler. 

\begin{equation}
    Y_m(\omega) = \left(
        - \ii \omega L_m + \frac{1}{- \ii \omega C}
    \right)^{-1} 
    = \ii \frac{y_m \omega_m \omega}{
        \omega^2 - \omega_m^2
    }. 
\end{equation}
where 
\begin{equation}
    \omega_m = \frac{1}{\sqrt{L_m C_m}}, \quad 
    y_m = \sqrt{\frac{C_m}{L_m}}
\end{equation}
This is supposed to be a response function,
so let's add a little imaginary part to $\omega$ 
so that all poles are on the lower plane,
and we get 
\begin{equation}
    Y_m(\omega) = \left(
        - \ii \omega L_m + \frac{1}{- \ii \omega C}
    \right)^{-1} 
    = \ii y_m \omega_m \frac{ \omega}{
        (\omega + \ii 0^+)^2 - \omega_m^2
    } = \chi_{I Q}(\omega). 
\end{equation}
The real part of 

On the other hand, the current going into the main circuit is 
\begin{equation}
    I(t) = \sum_m I_m(t),
\end{equation}
and the 
\begin{equation}
    S_{II}(\omega) = 2 \kB T \Re \chi_{I Q} (\omega),
\end{equation}
where $A = Q$ and $B = I$.

\printbibliography

\end{document}