\documentclass[hyperref, a4paper]{article}

\usepackage{geometry}
\usepackage{titling}
\usepackage{titlesec}
% No longer needed, since we will use enumitem package
% \usepackage{paralist}
\usepackage{enumitem}
\usepackage{footnote}
\usepackage[colorinlistoftodos]{todonotes}
\usepackage{amsmath, amssymb, amsthm}
\usepackage{mathtools}
\usepackage{bbm}
\usepackage{graphicx}
\usepackage{subcaption}
\usepackage{soulutf8}
\usepackage{physics}
\usepackage{tensor}
\usepackage{siunitx}
\usepackage[version=4]{mhchem}
\usepackage{tikz}
\usepackage{xcolor}
\usepackage{listings}
\usepackage{autobreak}
\usepackage[ruled, vlined, linesnumbered]{algorithm2e}
\usepackage{nameref,zref-xr}
\zxrsetup{toltxlabel}
\usepackage[backend=bibtex]{biblatex}
\addbibresource{nonequilibrium.bib}
\usepackage[colorlinks,unicode]{hyperref} % , linkcolor=black, anchorcolor=black, citecolor=black, urlcolor=black, filecolor=black
\usepackage[most]{tcolorbox}
\usepackage{prettyref}

% Page style
\geometry{left=3.18cm,right=3.18cm,top=2.54cm,bottom=2.54cm}
\titlespacing{\paragraph}{0pt}{1pt}{10pt}[20pt]
\setlength{\droptitle}{-5em}

% More compact lists 
\setlist[itemize]{
    itemindent=17pt, 
    leftmargin=1pt,
    listparindent=\parindent,
    parsep=0pt,
}

% Math operators
\DeclareMathOperator{\timeorder}{\mathcal{T}}
\DeclareMathOperator{\diag}{diag}
\DeclareMathOperator{\legpoly}{P}
\DeclareMathOperator{\primevalue}{P}
\DeclareMathOperator{\sgn}{sgn}
\DeclareMathOperator{\res}{Res}
\newcommand*{\ii}{\mathrm{i}}
\newcommand*{\ee}{\mathrm{e}}
\newcommand*{\const}{\mathrm{const}}
\newcommand*{\suchthat}{\quad \text{s.t.} \quad}
\newcommand*{\argmin}{\arg\min}
\newcommand*{\argmax}{\arg\max}
\newcommand*{\normalorder}[1]{: #1 :}
\newcommand*{\pair}[1]{\langle #1 \rangle}
\newcommand*{\fd}[1]{\mathcal{D} #1}
\DeclareMathOperator{\bigO}{\mathcal{O}}

% TikZ setting
\usetikzlibrary{arrows,shapes,positioning}
\usetikzlibrary{arrows.meta}
\usetikzlibrary{decorations.markings}
\tikzstyle arrowstyle=[scale=1]
\tikzstyle directed=[postaction={decorate,decoration={markings,
    mark=at position .5 with {\arrow[arrowstyle]{stealth}}}}]
\tikzstyle ray=[directed, thick]
\tikzstyle dot=[anchor=base,fill,circle,inner sep=1pt]

% Algorithm setting
% Julia-style code
\SetKwIF{If}{ElseIf}{Else}{if}{}{elseif}{else}{end}
\SetKwFor{For}{for}{}{end}
\SetKwFor{While}{while}{}{end}
\SetKwProg{Function}{function}{}{end}
\SetArgSty{textnormal}

\newcommand*{\concept}[1]{{\textbf{#1}}}

% Embedded codes
\lstset{basicstyle=\ttfamily,
  showstringspaces=false,
  commentstyle=\color{gray},
  keywordstyle=\color{blue}
}

% Reference formatting
\newcommand*{\citesec}[1]{\S~{#1}}
\newcommand*{\citechap}[1]{chap.~{#1}}
\newcommand*{\citefig}[1]{Fig.~{#1}}
\newcommand*{\citetable}[1]{Table~{#1}}
\newcommand*{\citepage}[1]{pp.~{#1}}
\newrefformat{fig}{Fig.~\ref{#1}}
\newcommand*{\term}[1]{\textit{#1}}

% Color boxes
\tcbuselibrary{skins, breakable, theorems}

\newtcbtheorem{infobox}{Box}{
    enhanced,
    boxrule=0pt,
    colback=blue!5,
    colframe=blue!5,
    coltitle=blue!50,
    borderline west={4pt}{0pt}{blue!65},
    sharp corners,
    fonttitle=\bfseries, 
    breakable,
    before upper={\parindent15pt\noindent}}{box}
\newtcbtheorem[use counter from=infobox]{theorybox}{Box}{
    enhanced,
    boxrule=0pt,
    colback=orange!5, 
    colframe=orange!5, 
    coltitle=orange!50,
    borderline west={4pt}{0pt}{orange!65},
    sharp corners,
    fonttitle=\bfseries, 
    breakable,
    before upper={\parindent15pt\noindent}}{box}
\newtcbtheorem[use counter from=infobox]{learnbox}{Box}{
    enhanced,
    boxrule=0pt,
    colback=green!5,
    colframe=green!5,
    coltitle=green!50,
    borderline west={4pt}{0pt}{green!65},
    sharp corners,
    fonttitle=\bfseries, 
    breakable,
    before upper={\parindent15pt\noindent}}{box}


\newenvironment{shelldisplay}{\begin{lstlisting}}{\end{lstlisting}}

\newcommand*{\kB}{k_{\text{B}}}
\newcommand*{\muB}{\mu_{\text{B}}}
\newcommand*{\efermi}{E_{\text{F}}}
\newcommand*{\pfermi}{p_{\text{F}}}
\newcommand*{\vfermi}{v_{\text{F}}}
\newcommand*{\sA}{\text{A}}
\newcommand*{\sB}{\text{B}}
\newcommand*{\Tc}{T_{\text{c}}}
\newcommand*{\hethree}{$^3$He}
\newcommand*{\hefour}{$^4$He}
\newcommand{\epsr}{\epsilon_{\text{r}}}
\newcommand{\chie}{\chi_{\text{e}}}
\newcommand{\cf}{c_{\text{F}}}
\newcommand{\fn}{F_{\text{N}}}

\title{Fluctuation}
\author{Jinyuan Wu}


\begin{document}

\maketitle

\section{Motivation: Langevin equation}

The experimental motivation to study fluctuation is 
probably the Brownian motion: 
in this specific case, 
large particles in fluid feel the fluctuation of the fluid,
and undergo random motion.
There are three length scales and three time scales 
in a typical experimental setting:
\begin{itemize}
    \item The molecule scales: \SI{1e-10}{m}, \SI{1e-12}{s};
    \item The particle scales: typically \SI{1e-6}{m}, \SI{1e-3}{s};
    \item The enclosure (i.e. container of the fluid) scales: 
    at least \SI{1e-2}{m}, \SI{1e2}{s}.
\end{itemize}
The separation between the molecule scales and the particle scales 
justifies treating the fluid as a continuum,
while the separation between the particle scales 
and the enclosure scales 
justifies treating the configuration space of the particles 
as an infinite one.
There actually should be another scale: 
the mean distance between particles;
if this is large enough, 
then particles are independent to each other; 
but we may want to consider this as the enclosure scales.

When the aforementioned conditions of separation of scales are all true, 
the behavior of a particle independent to others is described 
by the \concept{Langevin equation} 
\begin{equation}
    M \dv[2]{X}{t} + \pdv{U}{X} = - c_{\text{F}} \dot{X} + F_{\text{N}}(t),
    \label{eq:langevin}
\end{equation}
the terms on the LHS being effects about the particle itself,
the terms on the RHS being the deterministic damping term 
and the stochastic Langevin force, accordingly.

Solving \eqref{eq:langevin}, therefore, involves two steps:
finding reliable descriptions of $c_{\text{F}}$ and $F_{\text{N}}$,
and solving \eqref{eq:langevin} as a stochastic differential equation.
The first problem is essentially about establishing a theory 
of fluctuation ($F_{\text{N}}$) and response ($\gamma$) 
in the fluid surrounding the particle,
and the second problem is about the mathematical treatment 
of stochastic processes.

\section{Quantum circuit}

The formalism of Langevin equation is not limited to soft condensed matter systems:
since formally Newton's second law 
is analogous to circuit theory,
fluctuation and damping in a circuit immersed in 
a sea of smaller circuits can also be described in Langevin's formalism.
Indeed this is how resistance comes into being \todo{Ref.}
In circuit analysis we have another problem:
now the Langevin equation should be \emph{quantum},
since although no one would perform a Brownian motion experiment 
in a low temperature quantum liquid,
we indeed can perform quantum circuit experiments 
at a very low temperature.

\todo{Commutation relations?}

\section{Correlation, response, and fluctuation-dissipation theorem} 

In this section we briefly review the linear response theory.
In general, a driven Hamiltonian assumes the following form: 
\begin{equation}
    H = H_0(x_1, \ldots, x_n, p_1, \ldots, p_n) - A(x_1, \ldots, x_n, p_1, \ldots, p_n) F_A(t),
\end{equation}
where we have assumed that degrees of freedom in the system 
follow the canonical commutation relation;
this assumption of course is frequently broken
(as in, say, a magnetic spin model),
but in this note let's focus on the most familiar case. 
We can readily have two examples: 
the first is 
\begin{equation}
    H = \frac{p^2}{2m} + \frac{1}{2} k x^2 - x F(t),
\end{equation}
of which the EOM is 
\begin{equation}
    \dot{p} = - kx + F(t), \quad 
    \dot{x} = \frac{p}{m} ,
\end{equation}
as is expected.
The second is 
\begin{equation}
    H = \frac{\Phi^2}{2L} + \frac{Q^2}{2L} - Q V,
\end{equation}
\todo{Commutation relations: cQED?}
Here we intentionally leave the discussion of damping;
it can't be directly treated in the Hamiltonian formalism 
and will be modeled by an infinite bath of other degrees of freedom.
Suppose we have another physical quantity $B$ 
which is also a polynomial of $x_1, \ldots, x_n$
and $p_1, \ldots, p_n$.
When the external driving force $F(t)$ is applied, 
we get 
\begin{equation}
    \var{B}(t) \coloneqq B(t) - B_0 
    = \int_{-\infty}^\infty \dd{t_1} \chi_{BA}(t, t_1) F_A(t_1) 
    + B_{\text{noise}} + \bigO{(F_A^2)}.
\end{equation}
The first term equals to the linear term in $\expval{\var{B}(t)}$;
the noise term contains information about correlation of $B$
with other variables -- or maybe itself.
It can be statistical noise (noise in classical probability theory)
when there is a bath (see below);
but even if we are dealing with a pure state theory,
$B_{\text{noise}} = B - \expval{B}$ will still bring some ``quantum noise''.
The response function is 
\begin{equation}
    \chi_{BA}(t, t_1) = \chi_{BA}(t - t_1)
\end{equation}
when $H_0$ is time-independent.

Now we turn to another aspect: correlation.
We define 
\begin{equation}
    S_{BA}(\omega) = \int \ee^{\ii \omega t} \expval{B(0) A(t)} \dd{t}.
\end{equation}
\todo{Heisenberg description of the system: from density matrix to??? What is $A(t)$ when we have noise in the system?}
This is known as the greater Green function or the lesser Green function 
when $A$ and $B$ are field operators.
We now state the following \concept{fluctuation-dissipation theorem}:
for a classical system, we have 
\begin{equation}
    \expval*{\dot{A}(0) B(t)} \theta(t) = \kB T \chi_{BA}(t),
\end{equation}
and in the frequency domain, 
\begin{equation}
    S_{\dot{A} B} (\omega) = 2 \kB T \Re \chi(\omega).
\end{equation}
Replacing $\dot{A}$ by $A$, we get a $1 / (- \ii \omega)$ factor,
and the equation becomes 
\begin{equation}
    S_{AB}(\omega) = \frac{2 \kB T}{\omega} \Im \chi_{AB}(\omega).
\end{equation}
In the quantum case we have 
\begin{equation}
    S_{AB}(\omega) = \frac{2\hbar}{1 - \ee^{- \hbar \omega / \kB T}} \Im \chi_{AB}(\omega).
\end{equation}
To save space the proof of the theorem is not shown here; 
a straightforward derivation can be found in section 9.4 in \cite{coleman2015introduction}.
Note that in the $T \to 0$ limit, 
classically $S_{AB}$ vanishes,
but in the quantum case we still have a non-vanishing remaining correlation
as long as we see a response of $A$ to $B$
which is known as quantum noise:
if $A$ responds to the $- B f(t)$ term in the Hamiltonian,
$A$ and $B$ don't commute (or otherwise we don't have non-trivial time evolution),
and thus $S_{AB}$ is definitely not zero when $T \to 0$
i.e. when in the pure state theory.

\section{Analysis of common noises}

\section{Langevin equation revisited}

Now we can see that although the details of $\fn(t)$ and $\cf$
are still largely unknown,
if $\fn(t)$ and $\cf$ ``naturally'' appear 
by coupling with the bosonic bath, \todo{Derive $\fn$ as correlation function, $\cf$ as response}
the properties of $\fn(t)$ and $\cf$ in 
\begin{equation}
    M \ddot{X} + \cf \dot{X} + \pdv{U}{X} = \fn(t)
\end{equation}
follow the relation 
\begin{equation}
    \expval{\fn(0) \fn(t)} = 2 \kB T \cf \delta(t),
\end{equation}
or, to be accurate, $\delta(t)$ should be replaced by something like
\begin{equation}
    \frac{1}{\tau} \ee^{- \abs*{t} / \tau}
\end{equation}
if the environment degrees of freedom are ``free'' \todo{Being more clear here}
and therefore are Gaussian variables.

\section{Fokker-Planck equation}

The Langevin equation in the last section is a specific case of the following SDE 
\begin{equation}
    \dot{X} = A W + F, \quad 
    \expval{F_i(0) F_j(t)} = B_{ij} \delta(t),
\end{equation}
the time evolution of the probabilistic distribution $W(X_1, X_2, \ldots, X_n)$ 
can be rewritten into  
\begin{equation}
    \pdv{W}{t} = \underbrace{
        - \div{A W}
    }_{\text{drift}} + 
    \underbrace{
        \frac{1}{2} \div \div (B W)
    }_{\text{diffusion}} . 
\end{equation}
This equation is known as the \concept{Fokker-Planck equation}.

The classical tunneling rate is \todo{derivation}
\begin{equation}
    \Gamma = \frac{\omega_0}{2\pi} \ee^{- \Delta U / \kB T}
\end{equation}

\section{Transmission line and resistance}

We can also check the fluctuation-dissipation theorem on the infinite transmission line.

\begin{equation}
    Y_m(\omega) = \left(
        - \ii \omega L_m + \frac{1}{- \ii \omega C}
    \right)^{-1} 
    = - 
\end{equation}

\printbibliography

\end{document}