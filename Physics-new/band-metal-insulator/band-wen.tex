\documentclass[hyperref, a4paper]{article}

\usepackage{geometry}
\usepackage{titling}
\usepackage{titlesec}
% No longer needed, since we will use enumitem package
% \usepackage{paralist}
\usepackage{enumitem}
\usepackage{footnote}
\usepackage{marginnote}
\usepackage{enumerate}
\usepackage{amsmath, amssymb, amsthm}
\usepackage{mathtools}
\usepackage{bbm}
\usepackage{cite}
\usepackage{graphicx}
%\usepackage{subfigure}
\usepackage{subcaption}
\usepackage{physics}
\usepackage{tensor}
\usepackage{siunitx}
\usepackage[version=4]{mhchem}
\usepackage{tikz}
\usepackage{xcolor}
\usepackage{listings}
\usepackage{autobreak}
\usepackage[ruled, vlined, linesnumbered]{algorithm2e}
\usepackage{nameref,zref-xr}
\zxrsetup{toltxlabel}
\zexternaldocument*[solid-]{../solid/solid}[solid.pdf]
\zexternaldocument*[optics-]{../optics/optics}[optics.pdf]
\usepackage[colorlinks,unicode]{hyperref} % , linkcolor=black, anchorcolor=black, citecolor=black, urlcolor=black, filecolor=black
\usepackage[most]{tcolorbox}
\usepackage{prettyref}

% Page style
\geometry{left=3.18cm,right=3.18cm,top=2.54cm,bottom=2.54cm}
\titlespacing{\paragraph}{0pt}{1pt}{10pt}[20pt]
\setlength{\droptitle}{-5em}
\preauthor{\vspace{-10pt}\begin{center}}
\postauthor{\par\end{center}}

% More compact lists 
\setlist[itemize]{
    itemindent=17pt, 
    leftmargin=1pt,
    listparindent=\parindent,
    parsep=0pt,
}

% Math operators
\DeclareMathOperator{\timeorder}{\mathcal{T}}
\DeclareMathOperator{\diag}{diag}
\DeclareMathOperator{\legpoly}{P}
\DeclareMathOperator{\primevalue}{P}
\DeclareMathOperator{\sgn}{sgn}
\newcommand*{\ii}{\mathrm{i}}
\newcommand*{\ee}{\mathrm{e}}
\newcommand*{\const}{\mathrm{const}}
\newcommand*{\suchthat}{\quad \text{s.t.} \quad}
\newcommand*{\argmin}{\arg\min}
\newcommand*{\argmax}{\arg\max}
\newcommand*{\normalorder}[1]{: #1 :}
\newcommand*{\pair}[1]{\langle #1 \rangle}
\newcommand*{\fd}[1]{\mathcal{D} #1}
\DeclareMathOperator{\bigO}{\mathcal{O}}

% TikZ setting
\usetikzlibrary{arrows,shapes,positioning}
\usetikzlibrary{arrows.meta}
\usetikzlibrary{decorations.markings}
\tikzstyle arrowstyle=[scale=1]
\tikzstyle directed=[postaction={decorate,decoration={markings,
    mark=at position .5 with {\arrow[arrowstyle]{stealth}}}}]
\tikzstyle ray=[directed, thick]
\tikzstyle dot=[anchor=base,fill,circle,inner sep=1pt]

% Algorithm setting
% Julia-style code
\SetKwIF{If}{ElseIf}{Else}{if}{}{elseif}{else}{end}
\SetKwFor{For}{for}{}{end}
\SetKwFor{While}{while}{}{end}
\SetKwProg{Function}{function}{}{end}
\SetArgSty{textnormal}

\newcommand*{\concept}[1]{{\textbf{#1}}}

% Embedded codes
\lstset{basicstyle=\ttfamily,
  showstringspaces=false,
  commentstyle=\color{gray},
  keywordstyle=\color{blue}
}

% Reference formatting
\newrefformat{fig}{Figure~\ref{#1}}

% Color boxes
\tcbuselibrary{skins, breakable, theorems}
\newtcbtheorem[number within=section]{warning}{Warning}%
  {colback=orange!5,colframe=orange!65,fonttitle=\bfseries, breakable}{warn}
\newtcbtheorem[number within=section]{note}{Note}%
  {colback=green!5,colframe=green!65,fonttitle=\bfseries, breakable}{note}
\newtcbtheorem[number within=section]{info}{Info}%
  {colback=blue!5,colframe=blue!65,fonttitle=\bfseries, breakable}{info}

\newcommand{\soliddoc}{\href{../solid/solid.pdf}{this solid state physics note}}
\newcommand{\opticsdoc}{\href{../optics/optics.pdf}{this optics note}}

\title{Phenomena That Can Be Explained Solely by Band Theory}
\author{Jinyuan Wu}

\begin{document}

\maketitle

This article is a reading note of Xiaogang Wen's Quantum Field Theories of Many-body Systems, Chapter~4.

\section{The shape of the Fermi surface and algebraic long-range orders}

\begin{figure}
    \centering
    

\tikzset{every picture/.style={line width=0.75pt}} %set default line width to 0.75pt        

\begin{tikzpicture}[x=0.75pt,y=0.75pt,yscale=-1,xscale=1]
%uncomment if require: \path (0,369); %set diagram left start at 0, and has height of 369

%Shape: Rectangle [id:dp7890850960485711] 
\draw  [draw opacity=0][fill={rgb, 255:red, 155; green, 155; blue, 155 }  ,fill opacity=0.08 ] (150.46,216.46) -- (273,41) -- (371.39,109.71) -- (248.85,285.17) -- cycle ;
%Shape: Polygon Curved [id:ds16933255681931936] 
\draw  [fill={rgb, 255:red, 80; green, 227; blue, 194 }  ,fill opacity=0.26 ] (228,112) .. controls (248,102) and (279,79) .. (318,112) .. controls (357,145) and (298,142) .. (318,172) .. controls (338,202) and (248,202) .. (228,172) .. controls (208,142) and (208,122) .. (228,112) -- cycle ;
%Straight Lines [id:da8172830438322201] 
\draw [color={rgb, 255:red, 155; green, 155; blue, 155 }  ,draw opacity=1 ]   (137,145) -- (439,145) ;
%Straight Lines [id:da8464582519121175] 
\draw [color={rgb, 255:red, 155; green, 155; blue, 155 }  ,draw opacity=1 ]   (288,251.25) -- (288,43) ;
%Straight Lines [id:da6926385458902968] 
\draw  [dash pattern={on 4.5pt off 4.5pt}]  (440,259) -- (139,35) ;
%Straight Lines [id:da01759015854493362] 
\draw [color={rgb, 255:red, 74; green, 144; blue, 226 }  ,draw opacity=1 ][line width=1.5]    (273,41) -- (167,192) ;
%Straight Lines [id:da14995718249679646] 
\draw [color={rgb, 255:red, 74; green, 144; blue, 226 }  ,draw opacity=1 ]   (302.5,63.5) -- (196.5,214.5) ;
%Straight Lines [id:da4192220598532561] 
\draw [color={rgb, 255:red, 74; green, 144; blue, 226 }  ,draw opacity=1 ]   (330.5,85.5) -- (224.5,236.5) ;
%Straight Lines [id:da6663112276637151] 
\draw [color={rgb, 255:red, 74; green, 144; blue, 226 }  ,draw opacity=1 ][line width=1.5]    (372.5,109.5) -- (266.5,260.5) ;
%Straight Lines [id:da48007887551929507] 
\draw    (537,43) -- (558,43) ;
%Straight Lines [id:da8390776098943569] 
\draw    (537,43) -- (537,21) ;
%Straight Lines [id:da41530809059346585] 
\draw    (288,145) -- (318.26,183.43) ;
\draw [shift={(319.5,185)}, rotate = 231.78] [fill={rgb, 255:red, 0; green, 0; blue, 0 }  ][line width=0.08]  [draw opacity=0] (12,-3) -- (0,0) -- (12,3) -- cycle    ;
%Straight Lines [id:da7596590644475403] 
\draw [color={rgb, 255:red, 248; green, 231; blue, 28 }  ,draw opacity=1 ][line width=1.5]    (227,171) -- (280,95) ;
%Straight Lines [id:da33448048650418194] 
\draw    (288,145) -- (249.64,118.15) ;
\draw [shift={(248,117)}, rotate = 34.99] [fill={rgb, 255:red, 0; green, 0; blue, 0 }  ][line width=0.08]  [draw opacity=0] (12,-3) -- (0,0) -- (12,3) -- cycle    ;
%Straight Lines [id:da08297341295296312] 
\draw    (288,145) -- (216.91,123.58) ;
\draw [shift={(215,123)}, rotate = 16.77] [fill={rgb, 255:red, 0; green, 0; blue, 0 }  ][line width=0.08]  [draw opacity=0] (12,-3) -- (0,0) -- (12,3) -- cycle    ;
%Straight Lines [id:da2776423007341342] 
\draw [color={rgb, 255:red, 248; green, 231; blue, 28 }  ,draw opacity=1 ][line width=1.5]    (258,189) -- (315,108) ;

% Text Node
\draw (250,113.6) node [anchor=south west] [inner sep=0.75pt]    {$\hat{\boldsymbol{x}}$};
% Text Node
\draw (211,122) node [anchor=east] [inner sep=0.75pt]    {$\boldsymbol{k}_{\text{F}}(\hat{\boldsymbol{x}})$};
% Text Node
\draw (254,17.4) node [anchor=north west][inner sep=0.75pt]    {$k=\boldsymbol{k}_{\text{F}}(\hat{\boldsymbol{x}}) \cdot \hat{\boldsymbol{x}}$};
% Text Node
\draw (539,39.6) node [anchor=south west] [inner sep=0.75pt]    {$\boldsymbol{k}$};
% Text Node
\draw (380,83.4) node [anchor=north west][inner sep=0.75pt]    {$k=\boldsymbol{k}_{\text{F}}( -\hat{\boldsymbol{x}}) \cdot \hat{\boldsymbol{x}}$};
% Text Node
\draw (319.5,188.4) node [anchor=north] [inner sep=0.75pt]    {$\boldsymbol{k}_{\text{F}}( -\hat{\boldsymbol{x}})$};
% Text Node
\draw (87,285.4) node [anchor=north west][inner sep=0.75pt]    {$\boldsymbol{k}_{\text{F}}( -\hat{\boldsymbol{x}}) \cdot \hat{\boldsymbol{x}} < k< \boldsymbol{k}_{\text{F}}(\hat{\boldsymbol{x}}) \cdot \hat{\boldsymbol{x}}$};


\end{tikzpicture}

    \caption{The shape of the Fermi surface and \eqref{eq:n-tilde}. The green shadow is the Fermi sea. The slim blue lines are $k = \vb*{k} \cdot \vu*{x}$ planes. The yellow lines represents the intersection between the Fermi sea and $k = \vb*{k} \cdot \vu*{x}$ planes. The two thick blue lines are the $\vb*{k} \cdot \vu*{x} = \vb*{k}_\text{F}(\vu*{x}) \cdot \vu*{x}$ plane and the $\vb*{k} \cdot \vu*{x} = \vb*{k}_\text{F}(-\vu*{x}) \cdot \vu*{x}$ plane. $\tilde{N}$ is only non-zero in the grey area where $\boldsymbol{k}_{\text{F}}( -\hat{\boldsymbol{x}}) \cdot \hat{\boldsymbol{x}} < k< \boldsymbol{k}_{\text{F}}(\hat{\boldsymbol{x}}) \cdot \hat{\boldsymbol{x}}$.}
    \label{fig:n-tilde}
\end{figure}

In this section we explicitly evaluate the equal-time Green function.  \marginnote{Sec.~4.2.4}
An important fact is that it is highly affected by the shape of the Fermi surface. When $T=0$, we have 
(when not explicitly mentioned, when there is no spin polarization mentioned, we are working with only 
one spin polarization)
\begin{equation} 
    \begin{aligned}
        \ii G(-0^+, \vb*{x}) &= \timeorder \expval*{c(\vb*{x}, -0^+) c^\dagger(0, 0)} = - \expval*{c^\dagger(0, 0) c(\vb*{x}, 0)} \\
        &= - \int \frac{\dd[d]{\vb*{k}}}{(2\pi)^d} n_\text{F}(\xi_{\vb*{k}}) \ee^{\ii \vb*{k} \cdot \vb*{x}}
        = - \int \frac{\dd[d]{\vb*{k}}}{(2\pi)^d} \Theta(-\xi_{\vb*{k}}) \ee^{\ii \vb*{k} \cdot \vb*{x}}.
    \end{aligned}
\end{equation}
We define 
\begin{equation}
    \begin{aligned}
        \tilde{N}(k, \vu*{x}) &\coloneqq \int \frac{\dd[d]{\vb*{k}}}{(2\pi)^d} \Theta(-\xi_{\vb*{k}}) \delta(k - \vb*{k} \cdot \vu*{x}) \\
        &= \frac{1}{(2\pi)^d} \times \text{intersection area between the Fermi sea and the $k = \vb*{k} \cdot \vu*{x}$ plane} ,
    \end{aligned}
    \label{eq:n-tilde}
\end{equation}
where the second line is since the $\delta$-function is non-zero on the plane $\vb*{k} \cdot \vu*{x} = k$ in 
the momentum space, and we have 
\[
    \int \dd[d]{\vb*{k}} \delta(k - \vb*{k} \cdot \vu*{x}) \cdots = \int \dd[d-1]{{S}} \frac{1}{\abs*{\grad_{\vb*{k}} (\vb*{k} \cdot \vu*{x})}} \cdots = \int \dd[d-1]{{S}} \cdots.
\] 
Since when $k = \vb*{k} \cdot \vu*{x}$, we have $k \abs*{\vb*{x}} = \vb*{k} \cdot \vb*{x}$, we have 
\begin{equation}
    \ii G(-0^+, \vb*{x}) = - \int_{-\infty}^\infty \dd{k} \tilde{N}(k, \vu*{x}) \ee^{\ii k \abs*{\vb*{x}}}.
    \label{eq:n-tilde-to-green}
\end{equation}

Now the most important task is to evaluate \eqref{eq:n-tilde}. Since it strongly depends on the shape of 
the Fermi surface, we are not going to give a generalized form of \eqref{eq:n-tilde}. What we want to focus 
on is the fact that Fermionic systems usually have \emph{algebraic long-range orders}. The Fourier transformation
relation in the first equation of \eqref{eq:n-tilde} means that a long-range component 
in $G(- 0^+, \vb*{x})$ is caused by a highly localized feature in $\tilde{N}(k, \vu*{x})$, and 
smooth, continuous components in $N(k, \vu*{x})$ with regard to $k$ contribute to details with small 
characteristic length scales in $G(-0^+, \vb*{x})$. Therefore, to investigate possible long-range orders 
in $G(-0^+, \vb*{x})$, we need to look for something that is kind of \emph{singular} in $\tilde{N}(k, \vu*{x})$.

From \prettyref{fig:n-tilde} find that when $\vu*{x}$ is given, $\tilde{N}(k, \vu*{x})$ is only non-zero 
in the interval 
\begin{equation}
    \vb*{k}_\text{F}(-\vu*{x}) \cdot \vu*{x} < k < \vb*{k}_\text{F}(\vu*{x}) \cdot \vu*{x},
    \label{eq:non-zero-n}
\end{equation}
where we define $\vb*{k}_\text{F}(\vu*{x})$ to be the point of tangency of a tangent plane of the Fermi
surface that is perpendicular to $\vb*{x}$.
Therefore, there are two singularities in the derivative of $\tilde{N}(k, \vu*{x})$ with respect to $k$,
which are the upper and lower limits of the interval. Therefore, the terms that contribute to a long-range 
order are 
\begin{equation}
    \begin{aligned}
        \tilde{N}(k, \hat{\boldsymbol{x}}) \sim &\  c_{+} \Theta\left(\boldsymbol{k}_{F}(\hat{\boldsymbol{x}}) \cdot \hat{\boldsymbol{x}}-k\right)\left|\boldsymbol{k}_{F}(\hat{\boldsymbol{x}}) \cdot \hat{\boldsymbol{x}}-k\right|^{(d-1) / 2} \\
        &+c_{-} \Theta\left(-\boldsymbol{k}_{F}(-\hat{\boldsymbol{x}}) \cdot \hat{\boldsymbol{x}}+k\right)\left|-\boldsymbol{k}_{F}(-\hat{\boldsymbol{x}}) \cdot \hat{\boldsymbol{x}}+k\right|^{(d-1) / 2},
        \end{aligned}
        \label{eq:n-tilde-singularity}
\end{equation}
where the exponent $(d-1)/2$ can be explained by \prettyref{fig:radius}(a). Now we can evaluate \eqref{eq:n-tilde-to-green} and get 
\begin{equation}
    \left.\mathrm{i} G\left(-0^{+}, \boldsymbol{x}\right)\right|_{\boldsymbol{x} \rightarrow \infty} \sim \const \times \left(c_{+} \mathrm{e}^{\mathrm{i} \boldsymbol{k}_\text{F}(\hat{\boldsymbol{x}}) \cdot \boldsymbol{x}} \mathrm{c}^{-\mathrm{i} \frac{\pi(d+1)}{4}}+c_{-} \mathrm{e}^{\mathrm{i} \boldsymbol{k}_\text{F}(-\hat{\boldsymbol{x}}) \cdot \boldsymbol{x}} \mathrm{e}^{\mathrm{i} \frac{\pi(d+1)}{4}}\right) \frac{1}{|x|^{(d+1) / 2}}.
    \label{eq:g-two-point-order}
\end{equation}
The first term comes from 
\[
    \begin{aligned}
        &\quad \int_{-\infty}^\infty \ee^{\ii k \abs*{\vb*{x}}} \dd{k} \Theta(\vb*{k}_\text{F}(\vu*{x}) \cdot \vu*{x} - k) \abs*{\vb*{k}_\text{F}(\vu*{x}) \cdot \vu*{x} - k)}^{(d-1)/2} \\
        &= \int_{-\infty}^\infty \ee^{\ii (\vb*{k}_\text{F}(\vu*{x}) \cdot \vu*{x} - k') \abs*{\vb*{x}}} \Theta(k') \abs*{k'}^{(d-1)/2} \\
        &= \ee^{ \ii \vb*{k}_\text{F}(\vu*{x}) \cdot \vb*{x}} \int_0^\infty k'^{(d-1)/2} \ee^{- \ii k' x} \dd{k'},
    \end{aligned}
\]
and we have 
\[
    \int_0^\infty k^D \ee^{ \ii k x} \dd{k} = ( \ii x )^{-1-D} \Gamma(1+D), \quad \Re D > -1, \ \Im x > 0.
\]
The second term can be obtained in a similar manner.
Note that \eqref{eq:n-tilde-singularity} is not zero outside \eqref{eq:non-zero-n}, and when integrating 
over $k$, we need to set a cutoff for both terms in \eqref{eq:n-tilde-singularity}, but this does not 
change the form of \eqref{eq:g-two-point-order}, since we just have 
\[
    \int_0^\Lambda k^D \ee^{-\ii k x } \dd{k} = (\ii x)^{-1-D} (\Gamma(1+D) - \Gamma(1+D, \ii x \Lambda)).
\]
Actually, the form of \eqref{eq:g-two-point-order} can be observed by a dimensional analysis and some common
sense about the step function -- the dimensional analysis of $k^{(d-1)/2} \dd{k}$ means we have a 
$\abs*{\vb*{x}}^{-((d-1)/2+1)}$ factor in $(-0^+, \vb*{x})$, and the step function brings in the oscillation.
We see that \eqref{eq:g-two-point-order} oscillates as $\abs*{\vb*{x}} \to \infty$ and damps in an algebraic way.
The algebraic damping eventually comes from the mere existence of a sharp boundary of $n_\text{F}(\xi_{\vb*{k}})$,
or in other words, comes from the existence of the Fermi surface. 

\begin{figure}
    \centering
    

\tikzset{every picture/.style={line width=0.75pt}} %set default line width to 0.75pt        

\begin{tikzpicture}[x=0.75pt,y=0.75pt,yscale=-0.7,xscale=0.7]
%uncomment if require: \path (0,300); %set diagram left start at 0, and has height of 300

%Straight Lines [id:da918640877893085] 
\draw [color={rgb, 255:red, 0; green, 0; blue, 0 }  ,draw opacity=0.2 ]   (485,113) -- (546,113) ;
%Shape: Polygon Curved [id:ds7664500750896397] 
\draw  [draw opacity=0][fill={rgb, 255:red, 80; green, 227; blue, 194 }  ,fill opacity=0.26 ] (525,95) .. controls (562,93.33) and (605,95.33) .. (663,94) .. controls (692,94.33) and (690,118.33) .. (689,131.33) .. controls (580,133.33) and (636,130.33) .. (513,131.33) .. controls (512,115.33) and (509,104.33) .. (525,95) -- cycle ;
%Straight Lines [id:da13905621281512137] 
\draw    (217,96) -- (217,225) ;
\draw [shift={(217,94)}, rotate = 90] [fill={rgb, 255:red, 0; green, 0; blue, 0 }  ][line width=0.08]  [draw opacity=0] (12,-3) -- (0,0) -- (12,3) -- cycle    ;
%Straight Lines [id:da5797227150605533] 
\draw    (305.75,114.56) -- (217,225) ;
\draw [shift={(307,113)}, rotate = 128.78] [fill={rgb, 255:red, 0; green, 0; blue, 0 }  ][line width=0.08]  [draw opacity=0] (12,-3) -- (0,0) -- (12,3) -- cycle    ;
%Straight Lines [id:da16317094389877962] 
\draw    (167,113) -- (307,113) ;
%Straight Lines [id:da11010905773487067] 
\draw  [dash pattern={on 4.5pt off 4.5pt}]  (217,94) -- (217,48) ;
%Straight Lines [id:da5866673345272755] 
\draw [color={rgb, 255:red, 248; green, 231; blue, 28 }  ,draw opacity=1 ][line width=1.5]    (167,113) -- (305,113) ;
%Straight Lines [id:da20682805869926613] 
\draw [color={rgb, 255:red, 0; green, 0; blue, 0 }  ,draw opacity=0.2 ]   (305,113) -- (366,113) ;
%Straight Lines [id:da5804485238189274] 
\draw [color={rgb, 255:red, 0; green, 0; blue, 0 }  ,draw opacity=0.2 ]   (217,94) -- (365,94) ;
%Straight Lines [id:da5258300587121219] 
\draw [color={rgb, 255:red, 0; green, 0; blue, 0 }  ,draw opacity=0.2 ]   (365,94) -- (365,113) ;
%Straight Lines [id:da0012951429125565017] 
\draw [color={rgb, 255:red, 0; green, 0; blue, 0 }  ,draw opacity=0.2 ]   (365,75) -- (365,92) ;
\draw [shift={(365,94)}, rotate = 270] [fill={rgb, 255:red, 0; green, 0; blue, 0 }  ,fill opacity=0.2 ][line width=0.08]  [draw opacity=0] (12,-3) -- (0,0) -- (12,3) -- cycle    ;
%Straight Lines [id:da4484021127908355] 
\draw [color={rgb, 255:red, 0; green, 0; blue, 0 }  ,draw opacity=0.2 ]   (365,115) -- (365,132) ;
\draw [shift={(365,113)}, rotate = 90] [fill={rgb, 255:red, 0; green, 0; blue, 0 }  ,fill opacity=0.2 ][line width=0.08]  [draw opacity=0] (12,-3) -- (0,0) -- (12,3) -- cycle    ;
%Straight Lines [id:da38084988502300243] 
\draw [color={rgb, 255:red, 0; green, 0; blue, 0 }  ,draw opacity=0.2 ]   (167,113) -- (167,64.33) ;
%Straight Lines [id:da3622174351254601] 
\draw [color={rgb, 255:red, 0; green, 0; blue, 0 }  ,draw opacity=0.2 ]   (305,113) -- (305,64.33) ;
%Straight Lines [id:da46269600902222074] 
\draw [color={rgb, 255:red, 0; green, 0; blue, 0 }  ,draw opacity=0.2 ][line width=0.75]    (168,73.33) -- (304,73.33) ;
\draw [shift={(306,73.33)}, rotate = 180] [fill={rgb, 255:red, 0; green, 0; blue, 0 }  ,fill opacity=0.2 ][line width=0.08]  [draw opacity=0] (12,-3) -- (0,0) -- (12,3) -- cycle    ;
\draw [shift={(166,73.33)}, rotate = 0] [fill={rgb, 255:red, 0; green, 0; blue, 0 }  ,fill opacity=0.2 ][line width=0.08]  [draw opacity=0] (12,-3) -- (0,0) -- (12,3) -- cycle    ;
%Shape: Rectangle [id:dp9029391185795892] 
\draw  [draw opacity=0][fill={rgb, 255:red, 80; green, 227; blue, 194 }  ,fill opacity=0.26 ] (145,128) -- (327,128) -- (327,252.33) -- (145,252.33) -- cycle ;
%Straight Lines [id:da7567604850412664] 
\draw    (535,94) -- (663,94) ;
%Straight Lines [id:da5213543435102419] 
\draw [color={rgb, 255:red, 0; green, 0; blue, 0 }  ,draw opacity=0.2 ]   (485,94) -- (546,94) ;
%Straight Lines [id:da4088065176465008] 
\draw [color={rgb, 255:red, 248; green, 231; blue, 28 }  ,draw opacity=1 ][line width=1.5]    (514,113) -- (690,113) ;
%Curve Lines [id:da11116595072683566] 
\draw [fill={rgb, 255:red, 80; green, 227; blue, 194 }  ,fill opacity=0.26 ]   (145,128) .. controls (185,98) and (258,67) .. (327,128) ;
%Curve Lines [id:da9329285452626859] 
\draw    (513,131.33) .. controls (512,121.33) and (508,93.33) .. (535,94) ;
%Curve Lines [id:da04442669114161091] 
\draw    (663,94) .. controls (688,94.33) and (693,111.33) .. (690,129.33) ;
%Shape: Rectangle [id:dp7639687953438465] 
\draw  [draw opacity=0][fill={rgb, 255:red, 80; green, 227; blue, 194 }  ,fill opacity=0.26 ] (514,131.33) -- (689,131.33) -- (689,255.67) -- (514,255.67) -- cycle ;
%Straight Lines [id:da07975102143157087] 
\draw [color={rgb, 255:red, 0; green, 0; blue, 0 }  ,draw opacity=0.2 ]   (485,94) -- (485,113) ;
%Straight Lines [id:da06258113523514597] 
\draw [color={rgb, 255:red, 0; green, 0; blue, 0 }  ,draw opacity=0.2 ]   (485,75) -- (485,92) ;
\draw [shift={(485,94)}, rotate = 270] [fill={rgb, 255:red, 0; green, 0; blue, 0 }  ,fill opacity=0.2 ][line width=0.08]  [draw opacity=0] (12,-3) -- (0,0) -- (12,3) -- cycle    ;
%Straight Lines [id:da7363046817146814] 
\draw [color={rgb, 255:red, 0; green, 0; blue, 0 }  ,draw opacity=0.2 ]   (485,115) -- (485,132) ;
\draw [shift={(485,113)}, rotate = 90] [fill={rgb, 255:red, 0; green, 0; blue, 0 }  ,fill opacity=0.2 ][line width=0.08]  [draw opacity=0] (12,-3) -- (0,0) -- (12,3) -- cycle    ;

% Text Node
\draw (174,162.4) node [anchor=north west][inner sep=0.75pt]    {$\sim R$};
% Text Node
\draw (397,92.4) node [anchor=north west][inner sep=0.75pt]    {$k-\boldsymbol{k} \cdot \hat{\boldsymbol{x}}$};
% Text Node
\draw (264,172.4) node [anchor=north west][inner sep=0.75pt]    {$\sim R$};
% Text Node
\draw (158.47,69.2) node [anchor=east] [inner sep=0.75pt]    {$\sim 2\sqrt{2R( k-\boldsymbol{k} \cdot \hat{\boldsymbol{x}})}$};
% Text Node
\draw (231,265.5) node [anchor=north west][inner sep=0.75pt]   [align=left] {(a)};
% Text Node
\draw (595,265.5) node [anchor=north west][inner sep=0.75pt]   [align=left] {(b)};


\end{tikzpicture}

    \caption{Estimation of $\tilde{N}(k, \vu*{x})$. (a) The case when the Fermi surface is not flat. The intersection in \eqref{eq:n-tilde} is a $d-1$ hypersurface, the ``radius'' of which is about $2 \sqrt{2R(k - \vb*{k} \cdot \vu*{x})} \sim \sqrt{k - \vb*{k} \cdot \vu*{x}}$, and therefore the ``area'' of the intersection is proportion to $(k - \vb*{k} \cdot \vu*{x})^{(d-1)/2}$. (b) The Fermi surface is flat on some directions. In this case there is no well-defined $R$, and the ``area'' of the intersection is almost constant when $k$ enters \eqref{eq:non-zero-n}.}
    \label{fig:radius}
\end{figure}

Our derivation above does not include the case in which there are uncountably infinite $\vb*{k}_\text{F}(\vu*{x})$'s.
This happens when a part of the Fermi surface is flat. This means we should not approximate the behavior 
of $\tilde{N}(k, \vu*{x})$ near the boundary points of \eqref{eq:non-zero-n} with \prettyref{fig:n-tilde}(b).
In this case, we have 
\begin{equation}
    \tilde{N}(k, \vu*{x}) \sim c_+ \Theta\left(\boldsymbol{k}_\text{F}(\hat{\boldsymbol{x}}) \cdot \hat{\boldsymbol{x}}-k\right) + c_- \Theta\left(- \boldsymbol{k}_\text{F}(-\hat{\boldsymbol{x}}) \cdot \hat{\boldsymbol{x}}+ k\right) \left|-\boldsymbol{k}_\text{F}(-\hat{\boldsymbol{x}}) \cdot \hat{\boldsymbol{x}}+k\right|^{(d-1) / 2}.
    \label{eq:one-side-flat-n-tilde}
\end{equation}
Again we evaluate \eqref{eq:n-tilde-to-green} and get 
\begin{equation}
    \left.\ii G(-0^+, \vb*{x})\right|_{\vb*{x} \to \infty} \sim - \ii \ee^{\ii \vb*{k}_\text{F}(\vu*{x})} \frac{1}{\abs*{\vb*{x}}}.
\end{equation}
Note that here we throw away the term contributed by the second term of \eqref{eq:one-side-flat-n-tilde}
because it decays more quickly ($\sim \abs*{\vb*{x}}^{-(d+1)}$) than the contribution of the first term 
($\sim \abs*{\vb*{x}}$). In other words, a flat Fermi surface induces an algebraic long-range order that decays slower.
If the Fermi surface is flat on both side -- i.e. $\vb*{k}_\text{F}(\vu*{x})$ and $\vb*{k}_\text{F}(-\vu*{x})$ --
then we have 
\begin{equation}
    \left.\ii G(-0^+, \vb*{x})\right|_{\vb*{x} \to \infty} \sim - \ii \ee^{\ii \vb*{k}_\text{F}(\vu*{x})} \frac{1}{\abs*{\vb*{x}}} + \ii \ee^{\ii \vb*{k}_\text{F}(- \vu*{x}) \cdot \vb*{x}} \frac{1}{\abs*{\vb*{x}}}.
\end{equation}

\section{Density-density correlation function}

Now we discuss the simplest two-body correlation function. \marginnote{Sec.~4.3.1} This topic is important for two reasons. 
First, it gives the electrostatics of a band material. Second, if a \concept{charge density wave (CDW)}
order forms, we can see a periodic pattern in the real space density-density correlation function,
and if we are able to find something like the algebraic long-range order discussed in the past section,
it is a good hint of what kind of Fermi surface tends to induce a CDW. Note that in the jellium model 
(see \href{electron-gas.pdf}{here}), the only strong correlation effect is the Wigner crystal. 
We can conclude that the rich strong correlation effects in condensed matter physics are mostly 
controlled by the \emph{lattice}, and the lattice shapes the bands and thus the Fermi surface.



\section{Linear response and effective theory}

Chern-Simons

\end{document}