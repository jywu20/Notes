\documentclass[hyperref, a4paper]{article}

\usepackage{geometry}
\usepackage{titling}
\usepackage{titlesec}
% No longer needed, since we will use enumitem package
% \usepackage{paralist}
\usepackage{enumitem}
\usepackage{footnote}
\usepackage{enumerate}
\usepackage{amsmath, amssymb, amsthm}
\usepackage{mathtools}
\usepackage{bbm}
\usepackage{cite}
\usepackage{graphicx}
\usepackage{subfigure}
\usepackage{physics}
\usepackage{tensor}
\usepackage{siunitx}
\usepackage[version=4]{mhchem}
\usepackage{tikz}
\usepackage{xcolor}
\usepackage{listings}
\usepackage{autobreak}
\usepackage[ruled, vlined, linesnumbered]{algorithm2e}
\usepackage{nameref,zref-xr}
\zxrsetup{toltxlabel}
\usepackage[colorlinks,unicode]{hyperref} % , linkcolor=black, anchorcolor=black, citecolor=black, urlcolor=black, filecolor=black
\usepackage[most]{tcolorbox}
\usepackage{prettyref}

% Page style
\geometry{left=3.18cm,right=3.18cm,top=2.54cm,bottom=2.54cm}
\titlespacing{\paragraph}{0pt}{1pt}{10pt}[20pt]
\setlength{\droptitle}{-5em}
\preauthor{\vspace{-10pt}\begin{center}}
\postauthor{\par\end{center}}

% More compact lists 
\setlist[itemize]{
    itemindent=17pt, 
    leftmargin=1pt,
    listparindent=\parindent,
    parsep=0pt,
}

% Math operators
\DeclareMathOperator{\timeorder}{\mathcal{T}}
\DeclareMathOperator{\diag}{diag}
\DeclareMathOperator{\legpoly}{P}
\DeclareMathOperator{\primevalue}{P}
\DeclareMathOperator{\sgn}{sgn}
\newcommand*{\ii}{\mathrm{i}}
\newcommand*{\ee}{\mathrm{e}}
\newcommand*{\const}{\mathrm{const}}
\newcommand*{\suchthat}{\quad \text{s.t.} \quad}
\newcommand*{\argmin}{\arg\min}
\newcommand*{\argmax}{\arg\max}
\newcommand*{\normalorder}[1]{: #1 :}
\newcommand*{\pair}[1]{\langle #1 \rangle}
\newcommand*{\fd}[1]{\mathcal{D} #1}
\DeclareMathOperator{\bigO}{\mathcal{O}}

% TikZ setting
\usetikzlibrary{arrows,shapes,positioning}
\usetikzlibrary{arrows.meta}
\usetikzlibrary{decorations.markings}
\tikzstyle arrowstyle=[scale=1]
\tikzstyle directed=[postaction={decorate,decoration={markings,
    mark=at position .5 with {\arrow[arrowstyle]{stealth}}}}]
\tikzstyle ray=[directed, thick]
\tikzstyle dot=[anchor=base,fill,circle,inner sep=1pt]

% Algorithm setting
% Julia-style code
\SetKwIF{If}{ElseIf}{Else}{if}{}{elseif}{else}{end}
\SetKwFor{For}{for}{}{end}
\SetKwFor{While}{while}{}{end}
\SetKwProg{Function}{function}{}{end}
\SetArgSty{textnormal}

\newcommand*{\concept}[1]{{\textbf{#1}}}

% Embedded codes
\lstset{basicstyle=\ttfamily,
  showstringspaces=false,
  commentstyle=\color{gray},
  keywordstyle=\color{blue}
}

% Reference formatting
\newrefformat{fig}{Figure~\ref{#1} on page~\pageref{#1}}

% Color boxes
\tcbuselibrary{skins, breakable, theorems}
\newtcbtheorem[number within=section]{warning}{Warning}%
  {colback=orange!5,colframe=orange!65,fonttitle=\bfseries, breakable}{warn}

\title{Excitons in Band Metal and Insulators}
\author{Jinyuan Wu}

\begin{document}

\maketitle

\section{Underlying particles}

In the first several chapters in \href{../solid/solid.pdf}{the solid state physics note} we discussed 
the underlying particles in condensed matters and how to treat quasiparticles in condensed matters 
as irreducible representations of groups. Here we briefly review the underlying degrees of freedom,
or in other words, review the first principles.

In QED, by computing the low energy scattering amplitude we find the Coulomb interaction 
between electrons, which is an \emph{effective} interaction channel in high energy physics, but we 
usually just accept it as the first principle interaction between electrons in condensed matter physics. 

Electrons in condensed matter physics are also \emph{not} the Dirac fermion in the sense of QED. 
Actually, we need to integrate out the anti-particle modes of the Dirac field to get 
a $\vb*{p}^2 / (\ii \partial_t - e \varphi + m)$ term in the Hamiltonian, separate a $\ee^{\ii m t}$
factor from the electron field and replace $\ii \partial_t$ with $m$, and throw away the $e \varphi$  
to get a non-relativistic approximation, in order to get \concept{condensed matter physics electrons}.
The \emph{electrons} in condensed matter physics are highly dressed compared to electrons in QED.
After the whole procedure, the quadratic Hamiltonian for electrons is 
\begin{equation}
    H = \frac{(\vb*{p} - e \vb*{A})^2}{2m} + \vb*{\mu} \cdot \vb*{B},
\end{equation}
where $e = - \abs*{e}$.

So in the end, in condensed matter physics, electrons are non-relativistic massive particles and are 
minimally coupled to an external electromagnetic field $\vb*{A}$, and their spins are also coupled 
to an external magnetic field, and electrons scatter with each other according to Coulomb law.

We also have atomic nuclei, which may be viewed as point particles. Light nuclei's quantum fluctuation
may be significant. Heavy nuclei's low orbitals may have relativistic effects, but in condensed matters 
we are usually only interested in valence electrons, which are almost always non-relativistic.

In solid state physics, the role of atoms can be concluded into several aspects. First they regulate 
the motion of electrons, and free electrons (in the sense of condensed matter physics) are regulated 
into \concept{band electrons}. The vibration of atoms gives rise to \concept{phonons}. Nonlinearity of 
atom motion is equivalent to \concept{phonon-phonon scattering}, and the presence of phonons may cause 
charge density change, which leads to \concept{electron-phonon scattering}. Deviation from ideal lattice
is taken into account as \concept{defects} or \concept{disorder} (defects with certain random distribution).

\section{Excitation in a free electron gas}

\href{electron-gas.pdf}{This article} discusses the jellium model, where we find screening 

Plasmon, Fermi liquid

\section{Electronic states in insulators}

exciton

\section{Polarization}

Polaron, polariton

\end{document}