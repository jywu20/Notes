\documentclass[hyperref, a4paper]{article}

\usepackage{geometry}
\usepackage{titling}
\usepackage{titlesec}
% No longer needed, since we will use enumitem package
% \usepackage{paralist}
\usepackage{enumitem}
\usepackage{footnote}
\usepackage{marginnote}
\usepackage{enumerate}
\usepackage{amsmath, amssymb, amsthm}
\usepackage{mathtools}
\usepackage{bbm}
\usepackage{cite}
\usepackage{graphicx}
\usepackage{subfigure}
\usepackage{physics}
\usepackage{tensor}
\usepackage{siunitx}
\usepackage[version=4]{mhchem}
\usepackage{tikz}
\usepackage{xcolor}
\usepackage{listings}
\usepackage{autobreak}
\usepackage[ruled, vlined, linesnumbered]{algorithm2e}
\usepackage{nameref,zref-xr}
\zxrsetup{toltxlabel}
\zexternaldocument*[solid-]{../solid/solid}[solid.pdf]
\usepackage[colorlinks,unicode]{hyperref} % , linkcolor=black, anchorcolor=black, citecolor=black, urlcolor=black, filecolor=black
\usepackage[most]{tcolorbox}
\usepackage{prettyref}

% Page style
\geometry{left=3.18cm,right=3.18cm,top=2.54cm,bottom=2.54cm}
\titlespacing{\paragraph}{0pt}{1pt}{10pt}[20pt]
\setlength{\droptitle}{-5em}
\preauthor{\vspace{-10pt}\begin{center}}
\postauthor{\par\end{center}}

% More compact lists 
\setlist[itemize]{
    itemindent=17pt, 
    leftmargin=1pt,
    listparindent=\parindent,
    parsep=0pt,
}

% Math operators
\DeclareMathOperator{\timeorder}{\mathcal{T}}
\DeclareMathOperator{\diag}{diag}
\DeclareMathOperator{\legpoly}{P}
\DeclareMathOperator{\primevalue}{P}
\DeclareMathOperator{\sgn}{sgn}
\newcommand*{\ii}{\mathrm{i}}
\newcommand*{\ee}{\mathrm{e}}
\newcommand*{\const}{\mathrm{const}}
\newcommand*{\suchthat}{\quad \text{s.t.} \quad}
\newcommand*{\argmin}{\arg\min}
\newcommand*{\argmax}{\arg\max}
\newcommand*{\normalorder}[1]{: #1 :}
\newcommand*{\pair}[1]{\langle #1 \rangle}
\newcommand*{\fd}[1]{\mathcal{D} #1}
\DeclareMathOperator{\bigO}{\mathcal{O}}

% TikZ setting
\usetikzlibrary{arrows,shapes,positioning}
\usetikzlibrary{arrows.meta}
\usetikzlibrary{decorations.markings}
\tikzstyle arrowstyle=[scale=1]
\tikzstyle directed=[postaction={decorate,decoration={markings,
    mark=at position .5 with {\arrow[arrowstyle]{stealth}}}}]
\tikzstyle ray=[directed, thick]
\tikzstyle dot=[anchor=base,fill,circle,inner sep=1pt]

% Algorithm setting
% Julia-style code
\SetKwIF{If}{ElseIf}{Else}{if}{}{elseif}{else}{end}
\SetKwFor{For}{for}{}{end}
\SetKwFor{While}{while}{}{end}
\SetKwProg{Function}{function}{}{end}
\SetArgSty{textnormal}

\newcommand*{\concept}[1]{{\textbf{#1}}}

% Embedded codes
\lstset{basicstyle=\ttfamily,
  showstringspaces=false,
  commentstyle=\color{gray},
  keywordstyle=\color{blue}
}

% Reference formatting
\newrefformat{fig}{Figure~\ref{#1} on page~\pageref{#1}}

% Color boxes
\tcbuselibrary{skins, breakable, theorems}
\newtcbtheorem[number within=section]{warning}{Warning}%
  {colback=orange!5,colframe=orange!65,fonttitle=\bfseries, breakable}{warn}
\newtcbtheorem[number within=section]{note}{Note}%
  {colback=green!5,colframe=green!65,fonttitle=\bfseries, breakable}{note}
\newtcbtheorem[number within=section]{info}{Info}%
  {colback=blue!5,colframe=blue!65,fonttitle=\bfseries, breakable}{info}

\newcommand{\soliddoc}{\href{../solid/solid.pdf}{this solid state physics note}}

\title{Electron Gas with Coulomb Interaction}
\author{Jinyuan Wu}

\begin{document}

\maketitle

It is said that for a simple demonstration of what happens in a metal, we can only work with the jellium model,
ignoring the details of the lattice, which means we can just work with a non-relativistic electron gas with Coulomb
interaction. \marginnote{Zhengzhong Li, Sec.~4.2} This article is an extension of Section~\ref{solid-sec:high-density} in \soliddoc. We will discuss some early development of RPA.

\section{Notations and basic facts about the jellium model} 

We define 
\begin{equation}
    \rho(\vb*{r}) = \sum_\sigma \psi^\dagger_\sigma(\vb*{r}) \psi_\sigma(\vb*{r}) 
    = \frac{1}{V} \sum_{\vb*{k}, \vb*{k}', \sigma} c^\dagger_{\vb*{k} \sigma} c_{\vb*{k}' \sigma} \ee^{- \ii (\vb*{k} - \vb*{k}') \cdot \vb*{r}} 
    = \frac{1}{V} \sum_{\vb*{q}} \sum_{\vb*{k}, \sigma} c^\dagger_{\vb*{k} \sigma} c_{\vb*{k} + \vb*{q}, \sigma} \ee^{\ii \vb*{q} \cdot \vb*{r}},
\end{equation}
and the commutation relations are 
\begin{equation}
    \comm*{c_{\vb*{k}}}{c^\dagger_{\vb*{k}'}} = \delta_{\vb*{k} \vb*{k}'}.
\end{equation}
We can also write down $\rho(\vb*{r})$ in the first quantization formulation, i.e. 
\begin{equation}
    \rho(\vb*{r}) = \sum_i \delta(\vb*{r} - \vb*{r}_i) = \sum_i \frac{1}{V} \sum_{\vb*{q}} \ee^{\ii \vb*{q} \cdot (\vb*{r} - \vb*{r}_i)} = \frac{1}{V} \sum_{\vb*{q}} \sum_i \ee^{\ii \vb*{q} \cdot (\vb*{r} - \vb*{r}_i)}.
\end{equation}
We have 
\begin{equation}
    \rho_{\vb*{q}} \coloneqq \sum_i \ee^{- \ii \vb*{q} \cdot \vb*{r}_i} = \sum_{\vb*{k}, \sigma} c^\dagger_{\vb*{k} \sigma} c_{\vb*{k} + \vb*{q}, \sigma} , \quad 
    \rho(\vb*{r}) = \frac{1}{V} \sum_{\vb*{q}} \rho_{\vb*{q}} \ee^{\ii \vb*{q} \cdot \vb*{r}}.
    \label{eq:density}
\end{equation}
Easily, we find 
\begin{equation}
    \rho_{\vb*{q} = 0} = \int \dd[3]{\vb*{r}} \rho(\vb*{r}) = N.
\end{equation}
The Hamiltonian of electrons is 
\[
    H_\text{electron} = \sum_{\vb*{k}, \sigma} \epsilon_{\vb*{k}} c^\dagger_{\vb*{k} \sigma} c_{\vb*{k} \sigma} 
    + \frac{1}{2V} \sum_{\vb*{k}, \vb*{k}', \vb*{q} , \sigma, \sigma'} 
    c^\dagger_{\vb*{k} + \vb*{q}, \sigma}  c^\dagger_{\vb*{k}' - \vb*{q}, \sigma'} V(q)
    c_{\vb*{k}' \sigma'} c_{\vb*{k} \sigma},
\]
where 
\begin{equation}
    V(q)= \int \ee^{- \ii \vb*{q} \cdot \vb*{r}} \frac{e^2}{r} = \frac{4 \pi e^2}{\vb*{q}^2}.
\end{equation}
The Hamiltonian of the lattice and the electron-lattice coupling is 
\[
    H_\text{lattice} = \frac{1}{2V} \sum_{\vb*{q}} \rho_\text{ion,$-\vb*{q}$} V(q)\rho_\text{ion,$\vb*{q}$} -
    \frac{1}{V} \sum_{\vb*{q}} \rho_\text{ion,$\vb*{q}$} V(q) \rho_{\vb*{q}}.
\]
When $\abs*{\vb*{q}} = 0$, $V(q)$ is divergent, but this does not matter. 
In the jellium model, the only non-zero Fourier component of $\rho_\text{ion}(\vb*{r})$ is 
$\rho_\text{ion,$\vb*{q}=0$} = N$ (the solid is neutral so the number of positive charges must be equal to 
the number of negative charges), and we soon find 
\[
    H_\text{lattice} + \frac{1}{2V} \sum_{\vb*{q} = 0, \vb*{k}, \vb*{k}', \sigma, \sigma'} c^\dagger_{\vb*{k} + \vb*{q}, \sigma}  c^\dagger_{\vb*{k}' - \vb*{q}, \sigma'} V(q)
    c_{\vb*{k}' \sigma'} c_{\vb*{k} \sigma} = 0.
\]
So all divergences cancel with each other, and in the end, the total Hamiltonian is 
\begin{equation}
    \begin{aligned}
        H &= \sum_{\vb*{k}, \sigma} \epsilon_{\vb*{k}} c^\dagger_{\vb*{k} \sigma} c_{\vb*{k} \sigma} 
        + \frac{1}{2V} \sum_{\vb*{q} \neq 0} \sum_{\vb*{k}, \vb*{k}', \sigma, \sigma'} c^\dagger_{\vb*{k} + \vb*{q}, \sigma}  c^\dagger_{\vb*{k}' - \vb*{q}, \sigma'} V(q)
        c_{\vb*{k}' \sigma'} c_{\vb*{k} \sigma} \\
        &= \sum_{\vb*{k}, \sigma} \epsilon_{\vb*{k}} c^\dagger_{\vb*{k} \sigma} c_{\vb*{k} \sigma} 
        + \frac{1}{2V} \sum_{\vb*{q} \neq 0} \rho^\dagger(\vb*{q}) V(q) \rho(\vb*{q}).
    \end{aligned}
    \label{eq:whole-ham}
\end{equation}
The contribution of the positive ion ``jell'' both constrains the electrons in the solid (or in other words,
give a chemical potential) and regularize the singularity of $V(\vb*{q} = 0)$. 

Note that \eqref{eq:whole-ham} differs with 
\begin{equation}
    H = \sum_{i} \frac{\vb*{p}_i^2}{2m} + \frac{1}{2} \sum_{i \neq j} \frac{e^2}{\abs*{\vb*{r}_i - \vb*{r}_j}}
\end{equation} 
with the $i=j$ term. This is an infinite constant and does not matter. 

\section{Classical (or first quantized) theory of the collective oscillation of electrons} 

From \eqref{eq:density}, we have \marginnote{Zhengzhong Li, Sec.~4.3}
\[
    \dot{\rho}_{\vb*{q}} = \sum_j (- \ii \vb*{q}) \cdot \vb*{v}_j \ee^{- \ii \vb*{q} \cdot \vb*{r}_j},
\]
\begin{equation}
    \ddot{\rho}_{\vb*{q}} = \sum_j (- \ii \vb*{q} \cdot \dot{\vb*{v}}_j - (\vb*{q} \cdot \vb*{v}_j)^2) \ee^{- \ii \vb*{q} \cdot \vb*{r}_j}.
    \label{eq:rho-q-eom}
\end{equation}
Now $\dot{\vb*{v}}_j$ can be derived from \eqref{eq:whole-ham}:
\[
    \begin{aligned}
        m \dot{\vb*{v}}_j &= - \grad_j \frac{1}{2V} \sum_{\vb*{q} \neq 0} V(q) \rho^\dagger_{\vb*{q}} \rho_{\vb*{q}} \\
        &= - \frac{1}{2V} \grad_j \sum_{\vb*{q} \neq 0} V(q) \sum_{i, k} \ee^{\ii \vb*{q} \cdot (\vb*{r}_i - \vb*{r}_k)} \\
        &= - \frac{1}{2V} \sum_{\vb*{q} \neq 0} V(q) \sum_k (\ii \vb*{q}) \ee^{\ii \vb*{q} \cdot (\vb*{r}_j - \vb*{r}_k)} + \sum_i (- \ii \vb*{q}) \ee^{\ii \vb*{q} \cdot (\vb*{r}_j - \vb*{r}_i)} \\
        &= - \frac{1}{V} \sum_{\vb*{q} \neq 0} \ii \vb*{q} V(q) \sum_i \ee^{\ii \vb*{q} \cdot (\vb*{r}_j - \vb*{r}_i)} \\
        &= - \frac{4 \pi e^2}{V} \sum_{\vb*{q} \neq 0} \frac{\ii \vb*{q}}{q^2} \ee^{\ii \vb*{q} \cdot \vb*{r}_j} \rho_{\vb*{q}},
    \end{aligned}
\]
and from \eqref{eq:rho-q-eom}, we have 
\[
    \begin{aligned}
        \ddot{\rho}_{\vb*{q}} &= - \sum_j \frac{4 \pi e^2}{m V} \sum_{\vb*{q}' \neq 0} \frac{\vb*{q} \cdot \vb*{q}'}{q'^2} \ee^{\ii \vb*{q}' \cdot \vb*{r}_j} \rho_{\vb*{q}'} \ee^{- \ii \vb*{q} \cdot \vb*{r}_j} - \sum_j (\vb*{q} \cdot \vb*{v}_j)^2 \ee^{- \ii \vb*{q} \cdot \vb*{r}_j} \\
        &= - \frac{4 \pi e^2}{m V} \sum_{\vb*{q}' \neq 0} \rho_{\vb*{q}'} \rho_{\vb*{q} - \vb*{q}'} - \sum_j (\vb*{q} \cdot \vb*{v}_j)^2 \ee^{- \ii \vb*{q} \cdot \vb*{r}_j}.
    \end{aligned}
\]
Now we make the \concept{random phase approximation (RPA)} in its original form: We assume that only the 
$\vb*{q} = \vb*{q}'$ term in the first term is important, because in the high density limit, there are no 
position preference of electrons (when the density is low, there might be a Wigner crystal, and RPA fails), 
and when $\vb*{q} \neq 0$, both $\rho_{\vb*{q}'}$ and $\rho_{\vb*{q} - \vb*{q}'}$ are sums of almost random 
phase factors $\ee^{- \ii \vb*{q} \cdot \vb*{r}_j}$, and therefore are both small enough. So we get the EOM 
after RPA:
\begin{equation}
    \begin{aligned}
        \ddot{\rho}_{\vb*{q}} &= - \frac{4 \pi e^2}{mV} \rho_{\vb*{q}} \rho_{0} - \sum_j (\vb*{q} \cdot \vb*{v}_j)^2 \ee^{- \ii \vb*{q} \cdot \vb*{r}_j} \\
        &= - \frac{4 \pi e^2 n}{m} \rho_{\vb*{q}} - \sum_j (\vb*{q} \cdot \vb*{v}_j)^2 \ee^{- \ii \vb*{q} \cdot \vb*{r}_j} ,
    \end{aligned}
\end{equation}  
where $n = N / V$ is the jellium density. Section~\eqref{solid-sec:linear-response-energy-band} in \soliddoc{} 
tells us that the electron-hole pair excitations are gapless, but from the EOM above we soon find that in the 
$\vb*{q} \to 0$ case there is a finite $\omega$ solution, which is given by 
\begin{equation}
    \ddot{\rho}_{\vb*{q}} + \omega_\text{p}^2 \rho_{\vb*{q}} = 0, \quad \omega_\text{p}^2 = \frac{4 \pi e^2 n}{m}.
\end{equation}
We see that this term arises from the Coulomb interaction. In other words, long range interaction induces a 
collective modes in the metal, which is now known as \concept{plasmon}. We know it is plasmon, or quantized 
plasma oscillation, because our derivation above also works for the oscillation of negative charges around 
positive charges in a plasma.

\section{Second quantized EOM of density modes}

Now we try to derive plasmon in the second quantization EOM framework. Generally speaking, a generalized 
hydrodynamic mode in an electron gas is made of a linear combination of 
\begin{equation}
    \rho_{\vb*{k} \vb*{q}}^\dagger \eqqcolon c^\dagger_{\vb*{k} + \vb*{q}} c_{\vb*{k}}
\end{equation}
with different 

\marginnote{Zhengzhong Li, Sec.~4.4}

\section{The Green function theory}

\section{The electric susceptibility and bosonic modes}\label{sec:boson-modes}

Now it is time to evaluate \marginnote{Zhengzhong Li, Sec.~4.6}

\end{document}