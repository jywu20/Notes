\documentclass[hyperref, a4paper]{article}

\usepackage{geometry}
\usepackage{titling}
\usepackage{titlesec}
% No longer needed, since we will use enumitem package
% \usepackage{paralist}
\usepackage{enumitem}
\usepackage{footnote}
\usepackage{marginnote}
\usepackage{enumerate}
\usepackage{amsmath, amssymb, amsthm}
\usepackage{mathtools}
\usepackage{bbm}
\usepackage{cite}
\usepackage{graphicx}
\usepackage{subfigure}
\usepackage{physics}
\usepackage{tensor}
\usepackage{siunitx}
\usepackage[version=4]{mhchem}
\usepackage{tikz}
\usepackage{xcolor}
\usepackage{listings}
\usepackage{autobreak}
\usepackage[ruled, vlined, linesnumbered]{algorithm2e}
\usepackage{nameref,zref-xr}
\zxrsetup{toltxlabel}
\zexternaldocument*[solid-]{../solid/solid}[solid.pdf]
\zexternaldocument*[optics-]{../optics/optics}[optics.pdf]
\zexternaldocument*[early-]{../advanced-electrodynamics/early-lectures-1}[early-lectures-1.pdf]
\usepackage[colorlinks,unicode]{hyperref} % , linkcolor=black, anchorcolor=black, citecolor=black, urlcolor=black, filecolor=black
\usepackage[most]{tcolorbox}
\usepackage{prettyref}

% Page style
\geometry{left=3.18cm,right=3.18cm,top=2.54cm,bottom=2.54cm}
\titlespacing{\paragraph}{0pt}{1pt}{10pt}[20pt]
\setlength{\droptitle}{-5em}
\preauthor{\vspace{-10pt}\begin{center}}
\postauthor{\par\end{center}}

% More compact lists 
\setlist[itemize]{
    itemindent=17pt, 
    leftmargin=1pt,
    listparindent=\parindent,
    parsep=0pt,
}

% Math operators
\DeclareMathOperator{\timeorder}{\mathcal{T}}
\DeclareMathOperator{\diag}{diag}
\DeclareMathOperator{\legpoly}{P}
\DeclareMathOperator{\primevalue}{P}
\DeclareMathOperator{\sgn}{sgn}
\newcommand*{\ii}{\mathrm{i}}
\newcommand*{\ee}{\mathrm{e}}
\newcommand*{\const}{\mathrm{const}}
\newcommand*{\suchthat}{\quad \text{s.t.} \quad}
\newcommand*{\argmin}{\arg\min}
\newcommand*{\argmax}{\arg\max}
\newcommand*{\normalorder}[1]{: #1 :}
\newcommand*{\pair}[1]{\langle #1 \rangle}
\newcommand*{\fd}[1]{\mathcal{D} #1}
\DeclareMathOperator{\bigO}{\mathcal{O}}

% TikZ setting
\usetikzlibrary{arrows,shapes,positioning}
\usetikzlibrary{arrows.meta}
\usetikzlibrary{decorations.markings}
\tikzstyle arrowstyle=[scale=1]
\tikzstyle directed=[postaction={decorate,decoration={markings,
    mark=at position .5 with {\arrow[arrowstyle]{stealth}}}}]
\tikzstyle ray=[directed, thick]
\tikzstyle dot=[anchor=base,fill,circle,inner sep=1pt]

% Algorithm setting
% Julia-style code
\SetKwIF{If}{ElseIf}{Else}{if}{}{elseif}{else}{end}
\SetKwFor{For}{for}{}{end}
\SetKwFor{While}{while}{}{end}
\SetKwProg{Function}{function}{}{end}
\SetArgSty{textnormal}

\newcommand*{\concept}[1]{{\textbf{#1}}}

% Embedded codes
\lstset{basicstyle=\ttfamily,
  showstringspaces=false,
  commentstyle=\color{gray},
  keywordstyle=\color{blue}
}

% Reference formatting
\newrefformat{fig}{Figure~\ref{#1} on page~\pageref{#1}}

% Color boxes
\tcbuselibrary{skins, breakable, theorems}
\newtcbtheorem[number within=section]{warning}{Warning}%
  {colback=orange!5,colframe=orange!65,fonttitle=\bfseries, breakable}{warn}
\newtcbtheorem[number within=section]{note}{Note}%
  {colback=green!5,colframe=green!65,fonttitle=\bfseries, breakable}{note}
\newtcbtheorem[number within=section]{info}{Info}%
  {colback=blue!5,colframe=blue!65,fonttitle=\bfseries, breakable}{info}

\newcommand{\soliddoc}{\href{../solid/solid.pdf}{this solid state physics note}}
\newcommand{\opticsdoc}{\href{../optics/optics.pdf}{this optics note}}
\newcommand{\earlydoc}{\href{../advanced-electrodynamics/early-lectures-1.pdf}{this note}}

\title{High Density Electron Gas with Coulomb Interaction}
\author{Jinyuan Wu}

\begin{document}

\maketitle

It is said that for a simple demonstration of what happens in a metal, we can only work with the \concept{jellium model},
ignoring the details of the lattice, which means we can just work with a non-relativistic electron gas with Coulomb
interaction. \marginnote{Zhengzhong Li, Sec.~4.2} This article is an extension of Section~\ref{solid-sec:high-density} in \soliddoc. We will discuss some early development of RPA.

\section{Notations and basic facts about the jellium model} 

We define 
\begin{equation}
    \rho(\vb*{r}) = \sum_\sigma \psi^\dagger_\sigma(\vb*{r}) \psi_\sigma(\vb*{r}) 
    = \frac{1}{V} \sum_{\vb*{k}, \vb*{k}', \sigma} c^\dagger_{\vb*{k} \sigma} c_{\vb*{k}' \sigma} \ee^{- \ii (\vb*{k} - \vb*{k}') \cdot \vb*{r}} 
    = \frac{1}{V} \sum_{\vb*{q}} \sum_{\vb*{k}, \sigma} c^\dagger_{\vb*{k} \sigma} c_{\vb*{k} + \vb*{q}, \sigma} \ee^{\ii \vb*{q} \cdot \vb*{r}},
\end{equation}
and the commutation relations are 
\begin{equation}
    \comm*{c_{\vb*{k}}}{c^\dagger_{\vb*{k}'}} = \delta_{\vb*{k} \vb*{k}'}.
\end{equation}
We can also write down $\rho(\vb*{r})$ in the first quantization formulation, i.e. 
\begin{equation}
    \rho(\vb*{r}) = \sum_i \delta(\vb*{r} - \vb*{r}_i) = \sum_i \frac{1}{V} \sum_{\vb*{q}} \ee^{\ii \vb*{q} \cdot (\vb*{r} - \vb*{r}_i)} = \frac{1}{V} \sum_{\vb*{q}} \sum_i \ee^{\ii \vb*{q} \cdot (\vb*{r} - \vb*{r}_i)}.
\end{equation}
We have 
\begin{equation}
    \rho_{\vb*{q}} \coloneqq \sum_i \ee^{- \ii \vb*{q} \cdot \vb*{r}_i} = \sum_{\vb*{k}, \sigma} c^\dagger_{\vb*{k} \sigma} c_{\vb*{k} + \vb*{q}, \sigma} , \quad 
    \rho(\vb*{r}) = \frac{1}{V} \sum_{\vb*{q}} \rho_{\vb*{q}} \ee^{\ii \vb*{q} \cdot \vb*{r}}.
    \label{eq:density}
\end{equation}
Easily, we find 
\begin{equation}
    \rho_{\vb*{q} = 0} = \int \dd[3]{\vb*{r}} \rho(\vb*{r}) = N.
\end{equation}
The Hamiltonian of electrons is 
\[
    H_\text{electron} = \sum_{\vb*{k}, \sigma} \epsilon_{\vb*{k}} c^\dagger_{\vb*{k} \sigma} c_{\vb*{k} \sigma} 
    + \frac{1}{2V} \sum_{\vb*{k}, \vb*{k}', \vb*{q} , \sigma, \sigma'} 
    c^\dagger_{\vb*{k} + \vb*{q}, \sigma}  c^\dagger_{\vb*{k}' - \vb*{q}, \sigma'} V(q)
    c_{\vb*{k}' \sigma'} c_{\vb*{k} \sigma},
\]
where 
\begin{equation}
    V(q)= \int \ee^{- \ii \vb*{q} \cdot \vb*{r}} \frac{e^2}{r} = \frac{4 \pi e^2}{\vb*{q}^2}.
\end{equation}
The Hamiltonian of the lattice and the electron-lattice coupling is 
\[
    H_\text{lattice} = \frac{1}{2V} \sum_{\vb*{q}} \rho_\text{ion,$-\vb*{q}$} V(q)\rho_\text{ion,$\vb*{q}$} -
    \frac{1}{V} \sum_{\vb*{q}} \rho_\text{ion,$\vb*{q}$} V(q) \rho_{\vb*{q}}.
\]
When $\abs*{\vb*{q}} = 0$, $V(q)$ is divergent, but this does not matter. 
In the jellium model, the only non-zero Fourier component of $\rho_\text{ion}(\vb*{r})$ is 
$\rho_\text{ion,$\vb*{q}=0$} = N$ (the solid is neutral so the number of positive charges must be equal to 
the number of negative charges), and we soon find 
\[
    H_\text{lattice} + \frac{1}{2V} \sum_{\vb*{q} = 0, \vb*{k}, \vb*{k}', \sigma, \sigma'} c^\dagger_{\vb*{k} + \vb*{q}, \sigma}  c^\dagger_{\vb*{k}' - \vb*{q}, \sigma'} V(q)
    c_{\vb*{k}' \sigma'} c_{\vb*{k} \sigma} = 0.
\]
So all divergences cancel with each other, and in the end, the total Hamiltonian is 
\begin{equation}
    \begin{aligned}
        H &= \sum_{\vb*{k}, \sigma} \epsilon_{\vb*{k}} c^\dagger_{\vb*{k} \sigma} c_{\vb*{k} \sigma} 
        + \frac{1}{2V} \sum_{\vb*{q} \neq 0} \sum_{\vb*{k}, \vb*{k}', \sigma, \sigma'} c^\dagger_{\vb*{k} + \vb*{q}, \sigma}  c^\dagger_{\vb*{k}' - \vb*{q}, \sigma'} V(q)
        c_{\vb*{k}' \sigma'} c_{\vb*{k} \sigma} \\
        &= \sum_{\vb*{k}, \sigma} \epsilon_{\vb*{k}} c^\dagger_{\vb*{k} \sigma} c_{\vb*{k} \sigma} 
        + \frac{1}{2V} \sum_{\vb*{q} \neq 0} \rho^\dagger_{\vb*{q}} V(q) \rho_{\vb*{q}} 
        = \sum_{\vb*{k}, \sigma} \epsilon_{\vb*{k}} c^\dagger_{\vb*{k} \sigma} c_{\vb*{k} \sigma} 
        + \frac{1}{2V} \sum_{\vb*{q} \neq 0} \rho_{-\vb*{q}} V(q) \rho_{\vb*{q}}.
    \end{aligned}
    \label{eq:whole-ham}
\end{equation}
The contribution of the positive ion ``jell'' both constrains the electrons in the solid (or in other words,
give a chemical potential) and regularize the singularity of $V(\vb*{q} = 0)$. 

Note that \eqref{eq:whole-ham} differs with 
\begin{equation}
    H = \sum_{i} \frac{\vb*{p}_i^2}{2m} + \frac{1}{2} \sum_{i \neq j} \frac{e^2}{\abs*{\vb*{r}_i - \vb*{r}_j}}
\end{equation} 
with the $i=j$ term. This is an infinite constant and does not matter. 

\section{Classical (or first quantized) theory of the collective oscillation of electrons} 

From \eqref{eq:density}, we have \marginnote{Zhengzhong Li, Sec.~4.3}
\[
    \dot{\rho}_{\vb*{q}} = \sum_j (- \ii \vb*{q}) \cdot \vb*{v}_j \ee^{- \ii \vb*{q} \cdot \vb*{r}_j},
\]
\begin{equation}
    \ddot{\rho}_{\vb*{q}} = \sum_j (- \ii \vb*{q} \cdot \dot{\vb*{v}}_j - (\vb*{q} \cdot \vb*{v}_j)^2) \ee^{- \ii \vb*{q} \cdot \vb*{r}_j}.
    \label{eq:rho-q-eom}
\end{equation}
Now $\dot{\vb*{v}}_j$ can be derived from \eqref{eq:whole-ham}:
\[
    \begin{aligned}
        m \dot{\vb*{v}}_j &= - \grad_j \frac{1}{2V} \sum_{\vb*{q} \neq 0} V(q) \rho^\dagger_{\vb*{q}} \rho_{\vb*{q}} \\
        &= - \frac{1}{2V} \grad_j \sum_{\vb*{q} \neq 0} V(q) \sum_{i, k} \ee^{\ii \vb*{q} \cdot (\vb*{r}_i - \vb*{r}_k)} \\
        &= - \frac{1}{2V} \sum_{\vb*{q} \neq 0} V(q) \sum_k (\ii \vb*{q}) \ee^{\ii \vb*{q} \cdot (\vb*{r}_j - \vb*{r}_k)} + \sum_i (- \ii \vb*{q}) \ee^{\ii \vb*{q} \cdot (\vb*{r}_j - \vb*{r}_i)} \\
        &= - \frac{1}{V} \sum_{\vb*{q} \neq 0} \ii \vb*{q} V(q) \sum_i \ee^{\ii \vb*{q} \cdot (\vb*{r}_j - \vb*{r}_i)} \\
        &= - \frac{4 \pi e^2}{V} \sum_{\vb*{q} \neq 0} \frac{\ii \vb*{q}}{q^2} \ee^{\ii \vb*{q} \cdot \vb*{r}_j} \rho_{\vb*{q}},
    \end{aligned}
\]
and from \eqref{eq:rho-q-eom}, we have 
\[
    \begin{aligned}
        \ddot{\rho}_{\vb*{q}} &= - \sum_j \frac{4 \pi e^2}{m V} \sum_{\vb*{q}' \neq 0} \frac{\vb*{q} \cdot \vb*{q}'}{q'^2} \ee^{\ii \vb*{q}' \cdot \vb*{r}_j} \rho_{\vb*{q}'} \ee^{- \ii \vb*{q} \cdot \vb*{r}_j} - \sum_j (\vb*{q} \cdot \vb*{v}_j)^2 \ee^{- \ii \vb*{q} \cdot \vb*{r}_j} \\
        &= - \frac{4 \pi e^2}{m V} \sum_{\vb*{q}' \neq 0} \rho_{\vb*{q}'} \rho_{\vb*{q} - \vb*{q}'} - \sum_j (\vb*{q} \cdot \vb*{v}_j)^2 \ee^{- \ii \vb*{q} \cdot \vb*{r}_j}.
    \end{aligned}
\]
Now we make the \concept{random phase approximation (RPA)} in its original form: We assume that only the 
$\vb*{q} = \vb*{q}'$ term in the first term is important, because in the high density limit, there are no 
position preference of electrons (when the density is low, there might be a Wigner crystal, and RPA fails), 
and when $\vb*{q} \neq 0$, both $\rho_{\vb*{q}'}$ and $\rho_{\vb*{q} - \vb*{q}'}$ are sums of almost random 
phase factors $\ee^{- \ii \vb*{q} \cdot \vb*{r}_j}$, and therefore are both small enough. So we get the EOM 
after RPA:
\begin{equation}
    \begin{aligned}
        \ddot{\rho}_{\vb*{q}} &= - \frac{4 \pi e^2}{mV} \rho_{\vb*{q}} \rho_{0} - \sum_j (\vb*{q} \cdot \vb*{v}_j)^2 \ee^{- \ii \vb*{q} \cdot \vb*{r}_j} \\
        &= - \frac{4 \pi e^2 n}{m} \rho_{\vb*{q}} - \sum_j (\vb*{q} \cdot \vb*{v}_j)^2 \ee^{- \ii \vb*{q} \cdot \vb*{r}_j} ,
    \end{aligned}
\end{equation}  
where $n = N / V$ is the jellium density. Section~\eqref{solid-sec:linear-response-energy-band} in \soliddoc{} 
tells us that the electron-hole pair excitations are gapless, but from the EOM above we soon find that in the 
$\vb*{q} \to 0$ case there is a finite $\omega$ solution, which is given by 
\begin{equation}
    \ddot{\rho}_{\vb*{q}} + \omega_\text{p}^2 \rho_{\vb*{q}} = 0, \quad \omega_\text{p}^2 = \frac{4 \pi e^2 n}{m}.
\end{equation}
We see that this term arises from the Coulomb interaction. In other words, long range interaction induces a 
collective modes in the metal, which is now known as \concept{plasmon}. We know it is plasmon, or quantized 
plasma oscillation, because our derivation above also works for the oscillation of negative charges around 
positive charges in a plasma.

\section{Second quantized EOM of density modes}\label{sec:second-quantization}

Now we try to derive plasmon in the second quantization EOM framework. Generally speaking, a generalized \marginnote{Zhengzhong Li, Sec.~4.4}
hydrodynamic mode in an electron gas is made of a linear combination of 
\begin{equation}
    \rho_{\vb*{k} \vb*{q}}^\dagger \eqqcolon \sum_\sigma c^\dagger_{\vb*{k} + \vb*{q}, \sigma} c_{\vb*{k} \sigma}.
\end{equation}
By evaluating the EOM, we may be able to identify stable density modes.

We start from the free theory. It can be verified that 
\begin{equation}
    \comm*{c_1^\dagger c_2}{c_3^\dagger c_4} = \delta_{23} c_1^\dagger c_4 - \delta_{14} c_3^\dagger c_2,
\end{equation} 
and therefore we have 
\[
    \begin{aligned}
        \ii \dot{\rho}_{\vb*{k} \vb*{q}}^\dagger &= \comm*{{\rho}_{\vb*{k} \vb*{q}}^\dagger}{H_0} 
        = \sum_{\vb*{p}, \alpha} \frac{\vb*{p}^2}{2m} \sum_\sigma \comm*{c^\dagger_{\vb*{k} + \vb*{q}, \sigma} c_{\vb*{k} \sigma}}{c^\dagger_{\vb*{p} \alpha} c_{\vb*{p} \alpha}} \\
        &= \sum_{\vb*{p}, \alpha, \sigma} \frac{\vb*{p}^2}{2m} (c^\dagger_{\vb*{k} + \vb*{q}, \sigma} c_{\vb*{p} \alpha} \delta_{\vb*{k} \vb*{p}} \delta_{\sigma \alpha} - c^\dagger_{\vb*{p} \alpha} c_{\vb*{k} \sigma} \delta_{\vb*{k}+\vb*{q}, \vb*{p}} \delta_{\sigma \alpha}) \\
        &= \sum_{\sigma} \left( \frac{\vb*{k}^2}{2m} c^\dagger_{\vb*{k}+\vb*{q}, \sigma} c_{\vb*{k} \sigma} - \frac{(\vb*{q} + \vb*{k})^2}{2m} c^\dagger_{\vb*{k}+\vb*{q}, \sigma} c_{\vb*{k} \sigma} \right),
    \end{aligned}
\]
so 
\begin{equation}
    \ii \dot{\rho}^\dagger_{\vb*{k} \vb*{q}} = \comm*{\rho_{\vb*{k} \vb*{q}}^\dagger}{H_0} = - \omega^\text{pair}_{\vb*{k} \vb*{q}} {\rho}^\dagger_{\vb*{k} \vb*{q}}, 
    \quad \omega^\text{pair}_{\vb*{k} \vb*{q}} = \frac{(\vb*{q} + \vb*{k})^2}{2m} - \frac{\vb*{k}^2}{2m}.
\end{equation}
Therefore, we find there are electron-hole pairs in the free theory. In the last section we find that in a plasmon
mode, we have $\rho_{\vb*{k}} = \sum_{\vb*{q}} \rho_{\vb*{k} \vb*{q}} \sim \ee^{- \ii \omega_{\vb*{p}} t}$, 
and $\omega_{\vb*{q} = 0} \neq 0$, but here when $\vb*{q} = 0$, $\omega^\text{pair}
_{\vb*{k} \vb*{q}}$ vanishes, so there is no plasmon. This is expected, since in the last section we find 
the plasmon comes from the Coulomb interaction.

Now we move on to discuss the jellium model. We have 
\[
    \begin{aligned}
        \comm*{\rho^\dagger_{\vb*{k} \vb*{q}}}{H} &= \comm*{\rho^\dagger_{\vb*{k} \vb*{q}}}{H_0 + H_\text{Coulomb}} = 
        - \omega^\text{pair}_{\vb*{k} \vb*{q}} {\rho}^\dagger_{\vb*{k} \vb*{q}} + \frac{1}{2V} \sum_{\vb*{q}' \neq 0} V(q') \comm*{\rho_{\vb*{k} \vb*{q}}^\dagger}{\rho_{-\vb*{q}'} \rho_{\vb*{q}'}} \\
        &= - \omega^\text{pair}_{\vb*{k} \vb*{q}} {\rho}^\dagger_{\vb*{k} \vb*{q}} + \frac{1}{2V} \sum_{\vb*{q}' \neq 0} V(q') (\rho_{-\vb*{q}'} \comm*{\rho_{\vb*{k} \vb*{q}}^\dagger}{ \rho_{\vb*{q}'}} + \comm*{\rho_{\vb*{k} \vb*{q}}^\dagger}{ \rho_{-\vb*{q}'}} \rho_{\vb*{q}'} ) \\
        &= - \omega^\text{pair}_{\vb*{k} \vb*{q}} {\rho}^\dagger_{\vb*{k} \vb*{q}} + \frac{1}{2V} \sum_{\vb*{q}' \neq 0} V(q') (\rho_{-\vb*{q}'} \comm*{\rho_{\vb*{k} \vb*{q}}^\dagger}{ \rho_{\vb*{q}'}} + \comm*{\rho_{\vb*{k} \vb*{q}}^\dagger}{ \rho_{\vb*{q}'}} \rho_{-\vb*{q}'} ) ,
    \end{aligned}
\] 
where 
\[
    \begin{aligned}
        \comm*{\rho_{\vb*{k} \vb*{q}}^\dagger}{ \rho_{\vb*{q}'}} &= 
        \sum_{\alpha} \sum_{\vb*{k}', \beta} \comm*{c_{\vb*{k}+\vb*{q}, \alpha}^\dagger c_{\vb*{k} \alpha}}{c^\dagger_{\vb*{k}' \beta} c_{\vb*{k}' + \vb*{q}', \beta}} \\
        &= \sum_{\alpha} ( c^\dagger_{\vb*{k} + \vb*{q}, \alpha} c_{\vb*{k} + \vb*{q}', \alpha} - c^\dagger_{\vb*{k} + \vb*{q} - \vb*{q}' , \alpha} c_{\vb*{k} \alpha} ).
    \end{aligned}
\]
So the EOM of $\rho_{\vb*{k} \vb*{q}}^\dagger$ is finally  \marginnote{Zhengzhong Li (4.4.9)}
\begin{equation}
    \ii \dot{\rho}_{\vb*{k} \vb*{q}}^\dagger = - \omega^\text{pair}_{\vb*{k} \vb*{q}} {\rho}^\dagger_{\vb*{k} \vb*{q}} - \frac{1}{2V} \sum_{\vb*{q}' \neq 0, \alpha} V(q') \acomm*{\rho_{-\vb*{q}'}}{c^\dagger_{\vb*{k} + \vb*{q} - \vb*{q}' , \alpha} c_{\vb*{k} \alpha} - c^\dagger_{\vb*{k} + \vb*{q}, \alpha} c_{\vb*{k} + \vb*{q}', \alpha}}.
    \label{eq:bosonic-eq}
\end{equation}

By taking all possible momenta in \eqref{eq:bosonic-eq}, we get a closed group of equations about density 
modes. In principle, this gives the (generalized) hydrodynamic modes in the jellium model completely, 
though we all know it is highly difficult to solve EOM of operators. This can also be seen as a successful 
\emph{bosonization} the theory, though unlike the case in a Luttinger liquid, this does not help us understand
the behavior of the jellium model. A possible way to simplify 
\eqref{eq:bosonic-eq} is to assume that $c^\dagger_{\vb*{k} + \vb*{q} - \vb*{q}' , \alpha} c_{\vb*{k} \alpha} 
- c^\dagger_{\vb*{k} + \vb*{q}, \alpha} c_{\vb*{k} + \vb*{q}', \alpha}$ does not fluctuate much, and we 
can replace it by the expectation value. if under this approximation we find a stable mode, 
then we have 
\[
    \begin{aligned}
        (\omega - \omega_{\vb*{k} \vb*{q}}^\text{pair}) \rho^\dagger_{\vb*{k} \vb*{q}} &= \frac{1}{2V} \sum_{\vb*{q}' \neq 0, \alpha} V(q') \acomm*{\rho_{-\vb*{q}'}}{\expval*{c^\dagger_{\vb*{k} + \vb*{q} - \vb*{q}' , \alpha} c_{\vb*{k} \alpha} - c^\dagger_{\vb*{k} + \vb*{q}, \alpha} c_{\vb*{k} + \vb*{q}', \alpha}}} \\
        &= \frac{1}{2V} \sum_{\vb*{q}' \neq 0, \alpha} 2 V(q') \rho_{\vb*{q}'}^\dagger (n_{\vb*{k}} \delta_{\vb*{q} \vb*{q}'} - n_{\vb*{k} + \vb*{q}} \delta_{\vb*{q} \vb*{q}'}) \\
        &= \frac{2}{V} \rho^\dagger_{\vb*{q}} (n_{\vb*{k}} - n_{\vb*{k} + \vb*{q}}) V(q),
    \end{aligned}
\] 
where 
\begin{equation}
    n_{\vb*{k}} \coloneqq \expval*{c^\dagger_{\vb*{k} \sigma} c_{\vb*{k} \sigma}}. \quad \quad (\text{no summation over $\sigma$})
\end{equation}
From the equation above we find 
\begin{equation}
    \rho_{\vb*{k} \vb*{q}}^\dagger = \frac{2 V(q)}{V} \rho^\dagger_{\vb*{q}} \frac{n_{\vb*{k}} - n_{\vb*{k} + \vb*{q}}}{\omega - \omega_{\vb*{k} \vb*{q}}^\text{pair}},
    \label{eq:rpa-operator-result}
\end{equation}
and summing over $\vb*{k}$, we have 
\begin{equation}
    1 = \frac{2 V(q)}{V} \sum_{\vb*{k}} \frac{n_{\vb*{k}} - n_{\vb*{k} + \vb*{q}}}{\omega - \omega_{\vb*{k} \vb*{q}}^\text{pair}}.
    \label{eq:freq-eq}
\end{equation}

What we are doing here is actually a mean field approximation. 
It is natural to guess that the approximation corresponds to one Green function resummation strategy, 
which we will discuss in detail in \prettyref{sec:green}.
An intuitive way to see what is going on is to note that \eqref{eq:rpa-operator-result} is actually
\begin{equation}
    \begin{gathered}
        \begin{tikzpicture}[x=0.75pt,y=0.75pt,yscale=-1,xscale=1]
            %uncomment if require: \path (0,300); %set diagram left start at 0, and has height of 300
            
            %Shape: Circle [id:dp9547132007370212] 
            \draw  [fill={rgb, 255:red, 155; green, 155; blue, 155 }  ,fill opacity=1 ] (56,136) .. controls (56,122.19) and (67.19,111) .. (81,111) .. controls (94.81,111) and (106,122.19) .. (106,136) .. controls (106,149.81) and (94.81,161) .. (81,161) .. controls (67.19,161) and (56,149.81) .. (56,136) -- cycle ;
            %Curve Lines [id:da4224241216982587] 
            \draw    (101,121) .. controls (123,96) and (149,106) .. (162,133) ;
            %Curve Lines [id:da07143179889536055] 
            \draw    (103,147) .. controls (125,172) and (149,160) .. (162,133) ;
            %Straight Lines [id:da4398750229808517] 
            \draw    (135,107) ;
            \draw [shift={(135,107)}, rotate = 180] [fill={rgb, 255:red, 0; green, 0; blue, 0 }  ][line width=0.08]  [draw opacity=0] (12,-3) -- (0,0) -- (12,3) -- cycle    ;
            %Straight Lines [id:da25703305779579977] 
            \draw    (129,160) -- (122,160) ;
            \draw [shift={(120,160)}, rotate = 360] [fill={rgb, 255:red, 0; green, 0; blue, 0 }  ][line width=0.08]  [draw opacity=0] (12,-3) -- (0,0) -- (12,3) -- cycle    ;
            \end{tikzpicture}            
    \end{gathered} \quad = \quad  \begin{gathered}
        \begin{tikzpicture}[x=0.75pt,y=0.75pt,yscale=-1,xscale=1]
            %uncomment if require: \path (0,300); %set diagram left start at 0, and has height of 300
            
            %Straight Lines [id:da5267990075099112] 
            \draw    (100,121) -- (139,160) ;
            %Straight Lines [id:da17272928206943328] 
            \draw    (139,160) -- (100,199) ;
            %Straight Lines [id:da957803192854821] 
            \draw    (119.5,140.5) -- (123.59,144.59) ;
            \draw [shift={(125,146)}, rotate = 225] [fill={rgb, 255:red, 0; green, 0; blue, 0 }  ][line width=0.08]  [draw opacity=0] (12,-3) -- (0,0) -- (12,3) -- cycle    ;
            %Straight Lines [id:da3141174634629429] 
            \draw    (119.5,179.5) -- (114.91,184.09) ;
            \draw [shift={(113.5,185.5)}, rotate = 315] [fill={rgb, 255:red, 0; green, 0; blue, 0 }  ][line width=0.08]  [draw opacity=0] (12,-3) -- (0,0) -- (12,3) -- cycle    ;
            \end{tikzpicture}            
    \end{gathered} \quad + \quad \begin{gathered}
        \begin{tikzpicture}[x=0.75pt,y=0.75pt,yscale=-1,xscale=1]
            %uncomment if require: \path (0,300); %set diagram left start at 0, and has height of 300
            
            %Shape: Circle [id:dp9547132007370212] 
            \draw  [fill={rgb, 255:red, 155; green, 155; blue, 155 }  ,fill opacity=1 ] (56,136) .. controls (56,122.19) and (67.19,111) .. (81,111) .. controls (94.81,111) and (106,122.19) .. (106,136) .. controls (106,149.81) and (94.81,161) .. (81,161) .. controls (67.19,161) and (56,149.81) .. (56,136) -- cycle ;
            %Curve Lines [id:da4224241216982587] 
            \draw    (101,121) .. controls (123,96) and (149,106) .. (162,133) ;
            %Curve Lines [id:da07143179889536055] 
            \draw    (103,147) .. controls (125,172) and (149,160) .. (162,133) ;
            %Straight Lines [id:da4398750229808517] 
            \draw    (135,107) ;
            \draw [shift={(135,107)}, rotate = 180] [fill={rgb, 255:red, 0; green, 0; blue, 0 }  ][line width=0.08]  [draw opacity=0] (12,-3) -- (0,0) -- (12,3) -- cycle    ;
            %Straight Lines [id:da25703305779579977] 
            \draw    (129,160) -- (122,160) ;
            \draw [shift={(120,160)}, rotate = 360] [fill={rgb, 255:red, 0; green, 0; blue, 0 }  ][line width=0.08]  [draw opacity=0] (12,-3) -- (0,0) -- (12,3) -- cycle    ;
            %Curve Lines [id:da7048949301311622] 
            \draw    (209,133) .. controls (230,87) and (272,97) .. (283,132) ;
            %Curve Lines [id:da9048001439980424] 
            \draw    (209,133) .. controls (233,177) and (271,165) .. (283,132) ;
            %Straight Lines [id:da32267260611388426] 
            \draw    (254,102) ;
            \draw [shift={(254,102)}, rotate = 180] [fill={rgb, 255:red, 0; green, 0; blue, 0 }  ][line width=0.08]  [draw opacity=0] (12,-3) -- (0,0) -- (12,3) -- cycle    ;
            %Straight Lines [id:da310042974722734] 
            \draw    (250,161) -- (243,161) ;
            \draw [shift={(241,161)}, rotate = 360] [fill={rgb, 255:red, 0; green, 0; blue, 0 }  ][line width=0.08]  [draw opacity=0] (12,-3) -- (0,0) -- (12,3) -- cycle    ;
            %Straight Lines [id:da6796744681944953] 
            \draw    (162,133) .. controls (163.67,131.33) and (165.33,131.33) .. (167,133) .. controls (168.67,134.67) and (170.33,134.67) .. (172,133) .. controls (173.67,131.33) and (175.33,131.33) .. (177,133) .. controls (178.67,134.67) and (180.33,134.67) .. (182,133) .. controls (183.67,131.33) and (185.33,131.33) .. (187,133) .. controls (188.67,134.67) and (190.33,134.67) .. (192,133) .. controls (193.67,131.33) and (195.33,131.33) .. (197,133) .. controls (198.67,134.67) and (200.33,134.67) .. (202,133) .. controls (203.67,131.33) and (205.33,131.33) .. (207,133) -- (209,133) -- (209,133) ;
            \end{tikzpicture}            
    \end{gathered},
\end{equation}
where the $(n_{\vb*{k}} - n_{\vb*{k} + \vb*{q}}) / (\omega - \omega_{\vb*{k} \vb*{q}}^\text{pair})$
factor is ``the propagator of an electron-hole pair''. The approximation we have made is also 
a \emph{random phase approximation}, because after the approximation, all $\vb*{q}' \neq \vb*{q}$
terms in \eqref{eq:bosonic-eq} disappear. 

Now we can easily find the frequencies and the states of all density modes. This also shows a benefit 
of our operator EOM based approach: in a Green function based approach, we can only easily find the 
spectrum of the density modes by checking the singularities of $\expval*{nn}$, but it is not that 
easy to identify their actual ``shape''. We, however, will postpone the solution of \eqref{eq:freq-eq} \marginnote{}
to \prettyref{sec:boson-modes}, since \eqref{eq:bosonic-eq} is actually just about the \emph{real} part of 
the Green function: we can recognize factors like $1 / (\omega - \omega^\text{pair}_{\vb*{k} \vb*{q}})$,
which misses the important infinitesimal $\ii 0^+$ imaginary part in the denominator, which can be used to 
determine the damping rate.  

\section{Feynman diagrams for the jellium model}\label{sec:green}

In this section we will discuss the jellium model as a field theory based on Huaiyu Wang's book.%
\footnote{Which is accused of plagiarism of Eleftherios' book about Green functions -- but since I have only read the former, I will go on with the section index of the former.} \marginnote{Huaiyu Wang Sec.~13.3.1}%
The ultimate purpose for any condensed matter field theoretic calculation is to derive the response 
of the material to any external stimulation. We will discuss the linear response in \prettyref{sec:response},
and in this section we just make some formal development about what to consider in a jellium model.

First we estimate the magnitude of the kinetic energy and the Coulomb repulsion energy. By definition we have 
\[
    \frac{V}{N} = \frac{4}{3} \pi r_0^3, 
\]
and 
\[
    N = \frac{V}{(2\pi)^3} \cdot \frac{4}{3} \pi k_\text{F}^3,
\]
and therefore we have 
\begin{equation}
    r_\text{s} \eqqcolon \frac{r_0}{a_0} = \left(\frac{9}{4} \pi\right)^{1/3} \frac{1}{k_\text{F} a_0},
\end{equation}
where
\begin{equation}
    a_0 = \frac{1}{2 m e^2}
\end{equation}
is the Bohr radius. The dimensionless $r_\text{s}$ gives the ratio between the average distance 
between electrons and the Bohr radius. The high density condition $r_\text{s} \ll 1$ is therefore equivalent to 
\begin{equation} \marginnote{Huaiyu Wang Eq. (13.3.8)}
    k_\text{F} a_0 \gg 1.
\end{equation}
The kinetic energy of a single electron is about 
\begin{equation}
    E_\text{K} \sim \frac{k_\text{F}^2}{2m},
    \label{eq:kinetic-magnitude}
\end{equation}
while the potential energy of a single electron is about 
\begin{equation}
    E_\text{V} \sim \frac{1}{r_\text{s} a_0},
    \label{eq:potential-magnitude}
\end{equation}
and the ratio between the two is 
\begin{equation}
    \frac{E_\text{V}}{E_\text{K}} \sim \frac{e^2}{r_\text{s} a_0} \frac{2m}{k_\text{F}^2} = \frac{1}{(k_\text{F} a_0)^2 r_\text{s}} \ll 1.
\end{equation}
We reach a quite counter-intuitive result: in a \emph{high} density electron gas, where the Coulomb repulsion 
is \emph{strong}, the Coulomb energy is just a small perturbation! The explanation can be found in 
Section~\ref{solid-sec:hartree-fock-approximation} in \soliddoc. 

\begin{note*}{}
    We need to be cautious when comparing energies, since we can add an arbitrary large or small constant to 
    any term in the Hamiltonian. What really matters is the \emph{range} in which an energy term can vary.
    For example, an electron can take any of the momenta in the Fermi sea, and therefore its kinetic 
    energy ranges from $0$ to $k_\text{F}^2 / 2m$, and therefore we have \eqref{eq:kinetic-magnitude}.
\end{note*}

Now we conduct loop diagram corrections. The first-order \marginnote{Huaiyu Wang Eq.~(13.3.10)}
self-energy correction is done in \eqref{solid-eq:example-electron-hf-self-energy} in \soliddoc, and we find 
it is finite until we go close to the Fermi surface. This is not really concerning, because if we replace 
the bare interaction line (``photon line'') by the \emph{screened} version (which we are going to discuss soon), 
the divergence vanishes. Here we follow the approach in Section~13.3.2 and use a \marginnote{Huaiyu Wang Sec.~13.3.2}
power counting method to estimate the divergence of corrections. It is often the case that a certain type of 
diagrams all diverges severely, but resummation of them removes the divergence. In our current study on the 
jellium model, divergence analysis is especially important. Suppose $q$ is the four-momentum of an interaction 
line (i.e. a ``virtual photon line'' that carries the Coulomb repulsion), which of course we need to integrate
over (since an interaction line cannot be an external line), we will show that only diagrams that 
are severely divergent in $\abs*{\vb*{q}}$ are important with the argument in the 
discussion around Eq.~(5.25) in Altland. \marginnote{Altland Eq.~(5.25)}
We know an interaction line introduces the following component into the diagram:
\[
    \begin{gathered}
        \begin{tikzpicture}[x=0.75pt,y=0.75pt,yscale=-1,xscale=1]
            %uncomment if require: \path (0,300); %set diagram left start at 0, and has height of 300
            
            %Shape: Arc [id:dp6714149015701996] 
            \draw  [draw opacity=0] (294.02,193.98) .. controls (270.13,193.19) and (251,173.58) .. (251,149.5) .. controls (251,124.92) and (270.92,105) .. (295.5,105) .. controls (296.01,105) and (296.52,105.01) .. (297.03,105.03) -- (295.5,149.5) -- cycle ; \draw   (294.02,193.98) .. controls (270.13,193.19) and (251,173.58) .. (251,149.5) .. controls (251,124.92) and (270.92,105) .. (295.5,105) .. controls (296.01,105) and (296.52,105.01) .. (297.03,105.03) ;
            %Straight Lines [id:da829269451996034] 
            \draw  [dash pattern={on 4.5pt off 4.5pt}]  (179,149) -- (251,149) ;
            %Straight Lines [id:da8285742970491377] 
            \draw    (266,183) -- (259.33,175.49) ;
            \draw [shift={(258,174)}, rotate = 48.37] [fill={rgb, 255:red, 0; green, 0; blue, 0 }  ][line width=0.08]  [draw opacity=0] (12,-3) -- (0,0) -- (12,3) -- cycle    ;
            %Straight Lines [id:da5376592952493859] 
            \draw    (260,123) -- (267.61,115.39) ;
            \draw [shift={(269.02,113.98)}, rotate = 135] [fill={rgb, 255:red, 0; green, 0; blue, 0 }  ][line width=0.08]  [draw opacity=0] (12,-3) -- (0,0) -- (12,3) -- cycle    ;
            
            % Text Node
            \draw (205,125.4) node [anchor=north west][inner sep=0.75pt]    {$q$};
            % Text Node
            \draw (245,183.4) node [anchor=north west][inner sep=0.75pt]    {$k$};
            % Text Node
            \draw (239,82.4) node [anchor=north west][inner sep=0.75pt]    {$k+q$};
            \end{tikzpicture}  
    \end{gathered}    \quad ,
\] 
and the two electron propagators result in something like \eqref{solid-eq:ext-electron-retarded-green-function} 
in \soliddoc. We find that when $\vb*{q} \to 0$, $\vb*{k}$ is near the Fermi surface, and $q_0$ is also approaching
zero, $\sum_{k} G_k G_{q+k}$ reaches a peak because there is a step between $f(\xi_{\vb*{k}})$ and 
$f(\xi_{\vb*{k} + \vb*{q}})$, just as is the case in \eqref{solid-eq:back-to-thomas-fermi} 
in \soliddoc. Generally speaking, $q_0$ will not be zero because for example, we can assume that the two 
electron lines in the above diagram are external ones, and in this case $q_0$ is usually non-zero. However, 
if the diagram has algebraic divergence in $\vb*{q}$, again we have something like 
\[
    \frac{1}{\abs*{\vb*{q}}} \sum_k G_k G_{q+k} = \frac{2}{V} \sum_{\vb*{k}} \frac{1}{\omega + \xi_{\vb*{k}} - \xi_{\vb*{k} + \vb*{q}}} \frac{f(\xi_{\vb*{k}}) - f(\xi_{\vb*{k} + \vb*{q}})}{\abs*{\vb*{q}}},
\]
which evaluates to something proportion to the density of states at the Fermi surface. The denser the electron 
gas is, the larger this value is. We therefore conclude that in a high density electron system, 
the main contribution comes from Feynman diagrams with algebraic divergence in $\abs*{\vb*{q}}$. 

\begin{note*}{Four momentum in condensed matter physics}
    In condensed matter physics, though we do not have Lorentz invariance, the four-momentum notation is 
    still used to avoiding using too many letters and to connect the frequency and the momentum of a particle.
\end{note*}

Therefore, for a high-density electron system, we only need to sum all diagrams which are divergent in the 
momentum of the interaction lines. We should also note that since each interaction line brings in a $- \ii e^2$
factor, we need to first classify the diagrams according to their order in $e$ and then pick out the most 
divergent diagrams in $q$ with the same order in $e$. In this way, among diagrams with the same degree of
divergence, only those with the lowest order in $e$ are evaluated, agreeing with our previous conclusion that 
the Coulomb repulsion is indeed a perturbation. 

The problem, than, is how to find the most divergent diagrams. All interaction lines in the most divergent diagram 
should have as many interaction lines with the same momentum $q$ as possible so that $q$ has a large negative
order. We find that this can only be achieved if the Feynman diagram is in the following form:
\[
    \begin{gathered}
        \begin{tikzpicture}[x=0.75pt,y=0.75pt,yscale=-0.7,xscale=0.7]
            %uncomment if require: \path (0,300); %set diagram left start at 0, and has height of 300
            
            %Straight Lines [id:da7194672619784075] 
            \draw  [dash pattern={on 4.5pt off 4.5pt}]  (86,140) -- (156.71,140) ;
            %Straight Lines [id:da0961032152097947] 
            \draw  [dash pattern={on 4.5pt off 4.5pt}]  (228.71,140) -- (299.41,140) ;
            %Straight Lines [id:da6612121231382311] 
            \draw    (110,131) -- (137,131) ;
            \draw [shift={(139,131)}, rotate = 180] [fill={rgb, 255:red, 0; green, 0; blue, 0 }  ][line width=0.08]  [draw opacity=0] (12,-3) -- (0,0) -- (12,3) -- cycle    ;
            %Straight Lines [id:da41879029014489766] 
            \draw    (249,131) -- (276,131) ;
            \draw [shift={(278,131)}, rotate = 180] [fill={rgb, 255:red, 0; green, 0; blue, 0 }  ][line width=0.08]  [draw opacity=0] (12,-3) -- (0,0) -- (12,3) -- cycle    ;
            %Shape: Circle [id:dp6319678962588287] 
            \draw  [fill={rgb, 255:red, 155; green, 155; blue, 155 }  ,fill opacity=0.63 ] (156.71,140) .. controls (156.71,120.12) and (172.82,104) .. (192.71,104) .. controls (212.59,104) and (228.71,120.12) .. (228.71,140) .. controls (228.71,159.88) and (212.59,176) .. (192.71,176) .. controls (172.82,176) and (156.71,159.88) .. (156.71,140) -- cycle ;
            %Shape: Circle [id:dp6185393492226732] 
            \draw  [fill={rgb, 255:red, 155; green, 155; blue, 155 }  ,fill opacity=0.63 ] (299.41,140) .. controls (299.41,120.12) and (315.53,104) .. (335.41,104) .. controls (355.29,104) and (371.41,120.12) .. (371.41,140) .. controls (371.41,159.88) and (355.29,176) .. (335.41,176) .. controls (315.53,176) and (299.41,159.88) .. (299.41,140) -- cycle ;
            %Straight Lines [id:da4438497340369987] 
            \draw  [dash pattern={on 4.5pt off 4.5pt}]  (371.71,141) -- (442.41,141) ;
            %Straight Lines [id:da1721593165285087] 
            \draw    (393,132) -- (420,132) ;
            \draw [shift={(422,132)}, rotate = 180] [fill={rgb, 255:red, 0; green, 0; blue, 0 }  ][line width=0.08]  [draw opacity=0] (12,-3) -- (0,0) -- (12,3) -- cycle    ;
            %Shape: Circle [id:dp34243982393371275] 
            \draw  [fill={rgb, 255:red, 155; green, 155; blue, 155 }  ,fill opacity=0.63 ] (545.41,141) .. controls (545.41,121.12) and (561.53,105) .. (581.41,105) .. controls (601.29,105) and (617.41,121.12) .. (617.41,141) .. controls (617.41,160.88) and (601.29,177) .. (581.41,177) .. controls (561.53,177) and (545.41,160.88) .. (545.41,141) -- cycle ;
            %Straight Lines [id:da2504622647260659] 
            \draw  [dash pattern={on 4.5pt off 4.5pt}]  (474.71,141) -- (545.41,141) ;
            %Straight Lines [id:da9559326020798105] 
            \draw    (496,132) -- (523,132) ;
            \draw [shift={(525,132)}, rotate = 180] [fill={rgb, 255:red, 0; green, 0; blue, 0 }  ][line width=0.08]  [draw opacity=0] (12,-3) -- (0,0) -- (12,3) -- cycle    ;
            %Shape: Circle [id:dp06286415718885618] 
            \draw  [fill={rgb, 255:red, 155; green, 155; blue, 155 }  ,fill opacity=0.63 ] (14,140) .. controls (14,120.12) and (30.12,104) .. (50,104) .. controls (69.88,104) and (86,120.12) .. (86,140) .. controls (86,159.88) and (69.88,176) .. (50,176) .. controls (30.12,176) and (14,159.88) .. (14,140) -- cycle ;
            %Straight Lines [id:da10218710059909841] 
            \draw    (22,11.92) -- (40,105.92) ;
            \draw [shift={(31,58.92)}, rotate = 259.16] [fill={rgb, 255:red, 0; green, 0; blue, 0 }  ][line width=0.08]  [draw opacity=0] (12,-3) -- (0,0) -- (12,3) -- cycle    ;
            %Straight Lines [id:da06418788038001799] 
            \draw    (40,174.92) -- (18,254.92) ;
            \draw [shift={(29,214.92)}, rotate = 285.38] [fill={rgb, 255:red, 0; green, 0; blue, 0 }  ][line width=0.08]  [draw opacity=0] (12,-3) -- (0,0) -- (12,3) -- cycle    ;
            %Straight Lines [id:da18365482031569647] 
            \draw    (274,52.92) -- (216.71,112) ;
            \draw [shift={(245.35,82.46)}, rotate = 314.12] [fill={rgb, 255:red, 0; green, 0; blue, 0 }  ][line width=0.08]  [draw opacity=0] (12,-3) -- (0,0) -- (12,3) -- cycle    ;
            %Straight Lines [id:da5315887681597193] 
            \draw    (136,33.92) -- (181.71,106) ;
            \draw [shift={(158.85,69.96)}, rotate = 237.62] [fill={rgb, 255:red, 0; green, 0; blue, 0 }  ][line width=0.08]  [draw opacity=0] (12,-3) -- (0,0) -- (12,3) -- cycle    ;
            %Straight Lines [id:da2918531451861728] 
            \draw    (225,32.92) -- (202.71,106) ;
            \draw [shift={(213.85,69.46)}, rotate = 286.96] [fill={rgb, 255:red, 0; green, 0; blue, 0 }  ][line width=0.08]  [draw opacity=0] (12,-3) -- (0,0) -- (12,3) -- cycle    ;
            %Straight Lines [id:da7704804563326548] 
            \draw    (327,174.92) -- (304.71,248) ;
            \draw [shift={(315.85,211.46)}, rotate = 286.96] [fill={rgb, 255:red, 0; green, 0; blue, 0 }  ][line width=0.08]  [draw opacity=0] (12,-3) -- (0,0) -- (12,3) -- cycle    ;
            %Straight Lines [id:da456021544531211] 
            \draw    (595,173.92) -- (642,246.92) ;
            \draw [shift={(618.5,210.42)}, rotate = 237.23] [fill={rgb, 255:red, 0; green, 0; blue, 0 }  ][line width=0.08]  [draw opacity=0] (12,-3) -- (0,0) -- (12,3) -- cycle    ;
            %Straight Lines [id:da11316731421295656] 
            \draw    (357,168.92) -- (404,241.92) ;
            \draw [shift={(380.5,205.42)}, rotate = 237.23] [fill={rgb, 255:red, 0; green, 0; blue, 0 }  ][line width=0.08]  [draw opacity=0] (12,-3) -- (0,0) -- (12,3) -- cycle    ;
            %Straight Lines [id:da11311938374029396] 
            \draw    (589,105.92) -- (610,22.92) ;
            \draw [shift={(599.5,64.42)}, rotate = 104.2] [fill={rgb, 255:red, 0; green, 0; blue, 0 }  ][line width=0.08]  [draw opacity=0] (12,-3) -- (0,0) -- (12,3) -- cycle    ;
            
            % Text Node
            \draw (124.5,124.6) node [anchor=south] [inner sep=0.75pt]    {$q$};
            % Text Node
            \draw (263.5,124.6) node [anchor=south] [inner sep=0.75pt]    {$q$};
            % Text Node
            \draw (407.5,125.6) node [anchor=south] [inner sep=0.75pt]    {$q$};
            % Text Node
            \draw (444.41,139) node [anchor=west] [inner sep=0.75pt]    {$\cdots $};
            % Text Node
            \draw (510.5,125.6) node [anchor=south] [inner sep=0.75pt]    {$q$};
            % Text Node
            \draw (247.35,85.86) node [anchor=north west][inner sep=0.75pt]    {$\cdots $};
            \end{tikzpicture}            
    \end{gathered},
\]
and there should be as less interaction lines in the grey blobs as possible. If a grey blob is not connected 
to an external line, we find it can only be in the following form: 
\[
    \begin{gathered}
        \begin{tikzpicture}[x=0.75pt,y=0.75pt,yscale=-0.7,xscale=0.7]
            %uncomment if require: \path (0,300); %set diagram left start at 0, and has height of 300
            
            %Straight Lines [id:da574782386048565] 
            \draw  [dash pattern={on 4.5pt off 4.5pt}]  (100,120) -- (170.71,120) ;
            %Shape: Circle [id:dp6848705791533556] 
            \draw   (170.71,120) .. controls (170.71,101.77) and (185.48,87) .. (203.71,87) .. controls (221.93,87) and (236.71,101.77) .. (236.71,120) .. controls (236.71,138.23) and (221.93,153) .. (203.71,153) .. controls (185.48,153) and (170.71,138.23) .. (170.71,120) -- cycle ;
            %Straight Lines [id:da9035029099809071] 
            \draw  [dash pattern={on 4.5pt off 4.5pt}]  (236.71,120) -- (307.41,120) ;
            %Straight Lines [id:da7914879483026394] 
            \draw    (209.5,87.25) ;
            \draw [shift={(209.5,87.25)}, rotate = 180] [fill={rgb, 255:red, 0; green, 0; blue, 0 }  ][line width=0.08]  [draw opacity=0] (12,-3) -- (0,0) -- (12,3) -- cycle    ;
            %Straight Lines [id:da750355443917712] 
            \draw    (202.33,152.5) -- (198.33,152.5) ;
            \draw [shift={(196.33,152.5)}, rotate = 360] [fill={rgb, 255:red, 0; green, 0; blue, 0 }  ][line width=0.08]  [draw opacity=0] (12,-3) -- (0,0) -- (12,3) -- cycle    ;
            %Straight Lines [id:da538651807911716] 
            \draw    (118,111) -- (145,111) ;
            \draw [shift={(147,111)}, rotate = 180] [fill={rgb, 255:red, 0; green, 0; blue, 0 }  ][line width=0.08]  [draw opacity=0] (12,-3) -- (0,0) -- (12,3) -- cycle    ;
            %Straight Lines [id:da9843725337404243] 
            \draw    (257,111) -- (284,111) ;
            \draw [shift={(286,111)}, rotate = 180] [fill={rgb, 255:red, 0; green, 0; blue, 0 }  ][line width=0.08]  [draw opacity=0] (12,-3) -- (0,0) -- (12,3) -- cycle    ;
            
            % Text Node
            \draw (132.5,104.6) node [anchor=south] [inner sep=0.75pt]    {$q$};
            % Text Node
            \draw (202.33,155.9) node [anchor=north] [inner sep=0.75pt]    {$k$};
            % Text Node
            \draw (203.71,83.6) node [anchor=south] [inner sep=0.75pt]    {$k+q$};
            % Text Node
            \draw (271.5,104.6) node [anchor=south] [inner sep=0.75pt]    {$q$};
            \end{tikzpicture}            
    \end{gathered}.
\]
The conclusion is that in a highly dense electron system, we only need to make the following loop diagram
correction: replacing the \emph{bare} interaction line with the following:
\begin{equation}
    \begin{aligned}
        \begin{gathered}
            \begin{tikzpicture}[x=0.75pt,y=0.75pt,yscale=-0.7,xscale=0.7]
                %uncomment if require: \path (0,300); %set diagram left start at 0, and has height of 300
                
                %Straight Lines [id:da9158108405032224] 
                \draw  [dash pattern={on 4.5pt off 4.5pt}]  (226.71,140) -- (290,140)(226.71,143) -- (290,143) ;
                \end{tikzpicture}
        \end{gathered} \quad &= \quad \begin{gathered}
            \begin{tikzpicture}[x=0.75pt,y=0.75pt,yscale=-0.7,xscale=0.7]
                %uncomment if require: \path (0,300); %set diagram left start at 0, and has height of 300
                
                %Straight Lines [id:da9158108405032224] 
                \draw  [dash pattern={on 4.5pt off 4.5pt}]  (213.71,159) -- (290,159) ;
            \end{tikzpicture} 
        \end{gathered} \quad + \quad \begin{gathered}
            \begin{tikzpicture}[x=0.75pt,y=0.75pt,yscale=-0.7,xscale=0.7]
                %uncomment if require: \path (0,300); %set diagram left start at 0, and has height of 300
                
                %Straight Lines [id:da0014168747947034266] 
                \draw  [dash pattern={on 4.5pt off 4.5pt}]  (140.71,142) -- (180.71,142) ;
                %Shape: Circle [id:dp6136073910163897] 
                \draw   (180.71,142) .. controls (180.71,123.77) and (195.48,109) .. (213.71,109) .. controls (231.93,109) and (246.71,123.77) .. (246.71,142) .. controls (246.71,160.23) and (231.93,175) .. (213.71,175) .. controls (195.48,175) and (180.71,160.23) .. (180.71,142) -- cycle ;
                %Straight Lines [id:da24276323775928632] 
                \draw    (219.5,109.25) ;
                \draw [shift={(219.5,109.25)}, rotate = 180] [fill={rgb, 255:red, 0; green, 0; blue, 0 }  ][line width=0.08]  [draw opacity=0] (12,-3) -- (0,0) -- (12,3) -- cycle    ;
                %Straight Lines [id:da5824966682961659] 
                \draw    (212.33,174.5) -- (208.33,174.5) ;
                \draw [shift={(206.33,174.5)}, rotate = 360] [fill={rgb, 255:red, 0; green, 0; blue, 0 }  ][line width=0.08]  [draw opacity=0] (12,-3) -- (0,0) -- (12,3) -- cycle    ;
                %Straight Lines [id:da5514620719913015] 
                \draw  [dash pattern={on 4.5pt off 4.5pt}]  (246.71,142) -- (292,142) ;
                \end{tikzpicture}                
        \end{gathered} \quad + \cdots  \\
        & \quad \quad + \quad \begin{gathered}
            \begin{tikzpicture}[x=0.75pt,y=0.75pt,yscale=-0.7,xscale=0.7]
                %uncomment if require: \path (0,300); %set diagram left start at 0, and has height of 300
                
                %Straight Lines [id:da3963818408569353] 
                \draw  [dash pattern={on 4.5pt off 4.5pt}]  (100.71,140) -- (140.71,140) ;
                %Shape: Circle [id:dp49286917160300825] 
                \draw   (140.71,140) .. controls (140.71,121.77) and (155.48,107) .. (173.71,107) .. controls (191.93,107) and (206.71,121.77) .. (206.71,140) .. controls (206.71,158.23) and (191.93,173) .. (173.71,173) .. controls (155.48,173) and (140.71,158.23) .. (140.71,140) -- cycle ;
                %Straight Lines [id:da12799067746596915] 
                \draw    (179.5,107.25) ;
                \draw [shift={(179.5,107.25)}, rotate = 180] [fill={rgb, 255:red, 0; green, 0; blue, 0 }  ][line width=0.08]  [draw opacity=0] (12,-3) -- (0,0) -- (12,3) -- cycle    ;
                %Straight Lines [id:da840312664371196] 
                \draw    (172.33,172.5) -- (168.33,172.5) ;
                \draw [shift={(166.33,172.5)}, rotate = 360] [fill={rgb, 255:red, 0; green, 0; blue, 0 }  ][line width=0.08]  [draw opacity=0] (12,-3) -- (0,0) -- (12,3) -- cycle    ;
                %Straight Lines [id:da7934152347348731] 
                \draw  [dash pattern={on 4.5pt off 4.5pt}]  (206.71,140) -- (257,140) ;
                %Shape: Circle [id:dp8590618643032912] 
                \draw   (257,140) .. controls (257,121.77) and (271.77,107) .. (290,107) .. controls (308.23,107) and (323,121.77) .. (323,140) .. controls (323,158.23) and (308.23,173) .. (290,173) .. controls (271.77,173) and (257,158.23) .. (257,140) -- cycle ;
                %Straight Lines [id:da45688158211198093] 
                \draw    (295.5,107.25) ;
                \draw [shift={(295.5,107.25)}, rotate = 180] [fill={rgb, 255:red, 0; green, 0; blue, 0 }  ][line width=0.08]  [draw opacity=0] (12,-3) -- (0,0) -- (12,3) -- cycle    ;
                %Straight Lines [id:da45633743508297653] 
                \draw    (290.33,172.5) -- (286.33,172.5) ;
                \draw [shift={(284.33,172.5)}, rotate = 360] [fill={rgb, 255:red, 0; green, 0; blue, 0 }  ][line width=0.08]  [draw opacity=0] (12,-3) -- (0,0) -- (12,3) -- cycle    ;
                %Straight Lines [id:da3751639063231236] 
                \draw  [dash pattern={on 4.5pt off 4.5pt}]  (323,140) -- (373.29,140) ;
                %Straight Lines [id:da1040340540636353] 
                \draw  [dash pattern={on 4.5pt off 4.5pt}]  (406.29,140) -- (456.59,140) ;
                %Shape: Circle [id:dp20071411307184106] 
                \draw   (456,140) .. controls (456,121.77) and (470.77,107) .. (489,107) .. controls (507.23,107) and (522,121.77) .. (522,140) .. controls (522,158.23) and (507.23,173) .. (489,173) .. controls (470.77,173) and (456,158.23) .. (456,140) -- cycle ;
                %Straight Lines [id:da20812325261772835] 
                \draw    (494.5,107.25) ;
                \draw [shift={(494.5,107.25)}, rotate = 180] [fill={rgb, 255:red, 0; green, 0; blue, 0 }  ][line width=0.08]  [draw opacity=0] (12,-3) -- (0,0) -- (12,3) -- cycle    ;
                %Straight Lines [id:da8266206782881969] 
                \draw    (489.33,172.5) -- (485.33,172.5) ;
                \draw [shift={(483.33,172.5)}, rotate = 360] [fill={rgb, 255:red, 0; green, 0; blue, 0 }  ][line width=0.08]  [draw opacity=0] (12,-3) -- (0,0) -- (12,3) -- cycle    ;
                %Straight Lines [id:da7168426085353536] 
                \draw  [dash pattern={on 4.5pt off 4.5pt}]  (522,140) -- (562,140) ;
                
                % Text Node
                \draw (380,140) node [anchor=west] [inner sep=0.75pt]    {$\cdots $};
                \end{tikzpicture}            
        \end{gathered} \quad + \cdots .
    \end{aligned}
    \label{eq:rpa-diagram}
\end{equation}
No other corrections are needed. 

Diagrammatically, \eqref{eq:rpa-diagram} can be justified by the intuition of ``high density'': the Coulomb field 
generated by an electron immediately excites an electron-hole pair in the dense electron gas nearby, and the 
process repeats over and over again to screen the bare Coulomb interaction.

\eqref{eq:rpa-diagram} is called the \concept{ring diagram approximation} for obvious reasons. It is also 
called the RPA, since it only works in the high-density circumstance, and agree with \eqref{eq:eom-rpa-diagram}
derived from the real random phase approximation. In modern days, \eqref{eq:rpa-diagram} appears much more 
frequent than the original RPA, though they are equivalent. The advantage of the field theoretic version of RPA 
is that we know what we ignore and can improve the accuracy of our theory systematically.

We can check \eqref{eq:rpa-diagram} for the electron self energy. In the jellium model since $V(0) = 0$, \marginnote{Huaiyu Wang Eq.~(13.3.12) to Eq.~(13.3.14)}
the Hartree term vanishes, and the first order self-energy diagram is just the Fock term: 
\begin{equation}
    \tikzset{every picture/.style={line width=0.75pt}} %set default line width to 0.75pt        
\begin{tikzpicture}[x=0.75pt,y=0.75pt,yscale=-1,xscale=1, baseline=(XXXX.south) ]
\path (0,100);\path (72,0);\draw    ($(current bounding box.center)+(0,0.3em)$) node [anchor=south] (XXXX) {};
%Straight Lines [id:da5223929287505769] 
\draw    (20,12.38) -- (20,28.69) ;
%Straight Lines [id:da7302689907086644] 
\draw    (20,28.69) -- (20,37.85) ;
%Straight Lines [id:da01308896954413119] 
\draw    (20,71.02) -- (20,87.33) ;
%Straight Lines [id:da06691797475458383] 
\draw    (20,61.85) -- (20,71.02) ;
%Rounded Rect [id:dp14046432895193872] 
\draw  [fill={rgb, 255:red, 155; green, 155; blue, 155 }  ,fill opacity=0.5 ] (8,49.85) .. controls (8,43.23) and (13.37,37.85) .. (20,37.85) -- (50.56,37.85) .. controls (57.19,37.85) and (62.56,43.23) .. (62.56,49.85) -- (62.56,49.85) .. controls (62.56,56.48) and (57.19,61.85) .. (50.56,61.85) -- (20,61.85) .. controls (13.37,61.85) and (8,56.48) .. (8,49.85) -- cycle ;
%Curve Lines [id:da028420830877608827] 
\draw  [dash pattern={on 0.84pt off 2.51pt}]  (20,28.69) .. controls (27.56,28.85) and (38.56,25.85) .. (50.56,37.85) ;
%Curve Lines [id:da30022339914384943] 
\draw  [dash pattern={on 0.84pt off 2.51pt}]  (20,71.02) .. controls (27.56,71.19) and (39.56,71.85) .. (50.56,61.85) ;
\end{tikzpicture}
=\tikzset{every picture/.style={line width=0.75pt}} %set default line width to 0.75pt        
\begin{tikzpicture}[x=0.75pt,y=0.75pt,yscale=-1,xscale=1, baseline=(XXXX.south) ]
\path (0,126);\path (100,0);\draw    ($(current bounding box.center)+(0,0.3em)$) node [anchor=south] (XXXX) {};
%Straight Lines [id:da48837091194077753] 
\draw    (38,11.38) -- (38,27.69) ;
%Straight Lines [id:da6245477936225734] 
\draw    (38,27.69) -- (38,36.63) ;
%Shape: Circle [id:dp9156357526719092] 
\draw  [fill={rgb, 255:red, 155; green, 155; blue, 155 }  ,fill opacity=0.5 ] (29.71,44.92) .. controls (29.71,40.34) and (33.42,36.63) .. (38,36.63) .. controls (42.58,36.63) and (46.29,40.34) .. (46.29,44.92) .. controls (46.29,49.5) and (42.58,53.21) .. (38,53.21) .. controls (33.42,53.21) and (29.71,49.5) .. (29.71,44.92) -- cycle ;
%Straight Lines [id:da8175996896472917] 
\draw    (38,53.21) -- (38,73.15) ;
%Curve Lines [id:da5601518258972764] 
\draw  [dash pattern={on 0.84pt off 2.51pt}]  (38,27.69) .. controls (45.56,27.85) and (53,24.63) .. (65,36.63) ;
%Shape: Circle [id:dp9387607240705733] 
\draw  [fill={rgb, 255:red, 155; green, 155; blue, 155 }  ,fill opacity=0.5 ] (56.71,44.92) .. controls (56.71,40.34) and (60.42,36.63) .. (65,36.63) .. controls (69.58,36.63) and (73.29,40.34) .. (73.29,44.92) .. controls (73.29,49.5) and (69.58,53.21) .. (65,53.21) .. controls (60.42,53.21) and (56.71,49.5) .. (56.71,44.92) -- cycle ;
%Curve Lines [id:da06768156246299317] 
\draw  [dash pattern={on 0.84pt off 2.51pt}]  (70.58,50.63) .. controls (78.15,50.79) and (72.58,77.63) .. (56.34,73.15) ;
%Rounded Rect [id:dp36099962391992424] 
\draw  [fill={rgb, 255:red, 155; green, 155; blue, 155 }  ,fill opacity=0.5 ] (26,81.39) .. controls (26,76.83) and (29.69,73.15) .. (34.24,73.15) -- (56.34,73.15) .. controls (60.89,73.15) and (64.58,76.83) .. (64.58,81.39) -- (64.58,81.39) .. controls (64.58,85.94) and (60.89,89.63) .. (56.34,89.63) -- (34.24,89.63) .. controls (29.69,89.63) and (26,85.94) .. (26,81.39) -- cycle ;
%Straight Lines [id:da4074365389253567] 
\draw    (38,99.31) -- (38,115.62) ;
%Straight Lines [id:da20749065455619364] 
\draw    (38,90.15) -- (38,99.31) ;
%Curve Lines [id:da8459534075874513] 
\draw  [dash pattern={on 0.84pt off 2.51pt}]  (38,99.31) .. controls (45.56,99.48) and (45.34,99.63) .. (56.34,89.63) ;
% Text Node
\draw (38,44.92) node   [align=left] {1};
% Text Node
\draw (65,44.92) node   [align=left] {2};
% Text Node
\draw (45.29,81.39) node   [align=left] {3};
\end{tikzpicture}
=\tikzset{every picture/.style={line width=0.75pt}} %set default line width to 0.75pt        
\begin{tikzpicture}[x=0.75pt,y=0.75pt,yscale=-1,xscale=1, baseline=(XXXX.south) ]
\path (0,100);\path (100,0);\draw    ($(current bounding box.center)+(0,0.3em)$) node [anchor=south] (XXXX) {};
%Straight Lines [id:da5935374757180278] 
\draw    (26,10.04) -- (26,26.35) ;
%Straight Lines [id:da9161140391853602] 
\draw    (28.5,26.35) -- (28.5,60.29)(25.5,26.35) -- (25.5,60.29) ;
%Curve Lines [id:da2664666235328814] 
\draw  [dash pattern={on 0.84pt off 2.51pt}]  (27.3,24.88) .. controls (36.8,26.84) and (43.93,33.44) .. (46.71,41.46) .. controls (47.55,43.87) and (47.99,46.42) .. (47.99,49) .. controls (47.99,50.29) and (47.88,51.59) .. (47.66,52.89) .. controls (46.6,58.89) and (43.05,64.92) .. (36.23,69.86)(26.7,27.82) .. controls (35.08,29.55) and (41.42,35.35) .. (43.88,42.44) .. controls (44.61,44.54) and (44.99,46.75) .. (44.99,49) .. controls (44.99,50.12) and (44.9,51.24) .. (44.7,52.37) .. controls (43.76,57.72) and (40.53,63.05) .. (34.47,67.43) ;
%Shape: Circle [id:dp2138731016466453] 
\draw   (18.65,68.65) .. controls (18.65,64.03) and (22.39,60.29) .. (27,60.29) .. controls (31.61,60.29) and (35.35,64.03) .. (35.35,68.65) .. controls (35.35,73.26) and (31.61,77) .. (27,77) .. controls (22.39,77) and (18.65,73.26) .. (18.65,68.65) -- cycle ;
%Straight Lines [id:da42871257292968323] 
\draw    (27,77) -- (27,93.31) ;
\end{tikzpicture},
\end{equation}
with no divergence in $\vb*{q}$. The second order self-energy diagrams are 
\begin{equation}
    \begin{gathered}
        \begin{tikzpicture}[x=0.75pt,y=0.75pt,yscale=-0.7,xscale=0.7]
            %uncomment if require: \path (0,300); %set diagram left start at 0, and has height of 300
            
            %Straight Lines [id:da3041484358604427] 
            \draw    (76,155) -- (199.71,155) ;
            %Straight Lines [id:da07378190318685784] 
            \draw    (148,155.12) ;
            \draw [shift={(148,155.12)}, rotate = 180] [fill={rgb, 255:red, 0; green, 0; blue, 0 }  ][line width=0.08]  [draw opacity=0] (12,-3) -- (0,0) -- (12,3) -- cycle    ;
            %Shape: Circle [id:dp47981680317475917] 
            \draw   (112.17,89.33) .. controls (112.17,74.88) and (123.88,63.17) .. (138.33,63.17) .. controls (152.78,63.17) and (164.5,74.88) .. (164.5,89.33) .. controls (164.5,103.78) and (152.78,115.5) .. (138.33,115.5) .. controls (123.88,115.5) and (112.17,103.78) .. (112.17,89.33) -- cycle ;
            %Straight Lines [id:da9057585470587182] 
            \draw    (145.5,63.25) ;
            \draw [shift={(145.5,63.25)}, rotate = 180] [fill={rgb, 255:red, 0; green, 0; blue, 0 }  ][line width=0.08]  [draw opacity=0] (12,-3) -- (0,0) -- (12,3) -- cycle    ;
            %Straight Lines [id:da6868030069609872] 
            \draw    (138.33,115.5) -- (134.33,115.5) ;
            \draw [shift={(132.33,115.5)}, rotate = 360] [fill={rgb, 255:red, 0; green, 0; blue, 0 }  ][line width=0.08]  [draw opacity=0] (12,-3) -- (0,0) -- (12,3) -- cycle    ;
            %Curve Lines [id:da7139348254177498] 
            \draw  [dash pattern={on 4.5pt off 4.5pt}]  (76,155) .. controls (80,128.67) and (94,100.67) .. (112.17,89.33) ;
            %Curve Lines [id:da945710836678433] 
            \draw  [dash pattern={on 4.5pt off 4.5pt}]  (200.67,155) .. controls (196.67,128.67) and (182.67,100.67) .. (164.5,89.33) ;
            
            % Text Node
            \draw (120,163.08) node [anchor=north west][inner sep=0.75pt]    {$k-q$};
            % Text Node
            \draw (138.33,118.9) node [anchor=north] [inner sep=0.75pt]    {$k'$};
            % Text Node
            \draw (138.33,59.77) node [anchor=south] [inner sep=0.75pt]    {$k'+q$};
            % Text Node
            \draw (76,93.4) node [anchor=north west][inner sep=0.75pt]    {$q$};
            % Text Node
            \draw (190,96.4) node [anchor=north west][inner sep=0.75pt]    {$q$};
            \end{tikzpicture}            
    \end{gathered} \sim e^4 \int \dd[3]{\vb*{q}} \frac{1}{q^2} \frac{1}{q^2} \sim e^4 \frac{1}{q},
    \label{eq:most-divergent-1}
\end{equation}
\begin{equation}
    \begin{gathered}
        \begin{tikzpicture}[x=0.75pt,y=0.75pt,yscale=-0.7,xscale=0.7]
            %uncomment if require: \path (0,300); %set diagram left start at 0, and has height of 300
            
            %Straight Lines [id:da10067367186322329] 
            \draw    (96,175) -- (276,175) ;
            %Straight Lines [id:da263219676634183] 
            \draw    (135,175.12) ;
            \draw [shift={(135,175.12)}, rotate = 180] [fill={rgb, 255:red, 0; green, 0; blue, 0 }  ][line width=0.08]  [draw opacity=0] (12,-3) -- (0,0) -- (12,3) -- cycle    ;
            %Shape: Arc [id:dp7293618732725813] 
            \draw  [draw opacity=0][dash pattern={on 4.5pt off 4.5pt}] (96.01,175.62) .. controls (96,175.22) and (95.99,174.82) .. (95.99,174.42) .. controls (95.99,149.15) and (123.53,128.67) .. (157.5,128.67) .. controls (191.11,128.67) and (218.42,148.72) .. (218.99,173.63) -- (157.5,174.42) -- cycle ; \draw  [dash pattern={on 4.5pt off 4.5pt}] (96.01,175.62) .. controls (96,175.22) and (95.99,174.82) .. (95.99,174.42) .. controls (95.99,149.15) and (123.53,128.67) .. (157.5,128.67) .. controls (191.11,128.67) and (218.42,148.72) .. (218.99,173.63) ;
            %Shape: Arc [id:dp3248985164010143] 
            \draw  [draw opacity=0][dash pattern={on 4.5pt off 4.5pt}] (275.98,175.72) .. controls (274.91,201.11) and (247.66,221.19) .. (214.46,220.8) .. controls (181.88,220.41) and (155.49,200.45) .. (154.08,175.69) -- (215.02,173.88) -- cycle ; \draw  [dash pattern={on 4.5pt off 4.5pt}] (275.98,175.72) .. controls (274.91,201.11) and (247.66,221.19) .. (214.46,220.8) .. controls (181.88,220.41) and (155.49,200.45) .. (154.08,175.69) ;
            %Straight Lines [id:da9543303217791992] 
            \draw    (195,175) ;
            \draw [shift={(195,175)}, rotate = 180] [fill={rgb, 255:red, 0; green, 0; blue, 0 }  ][line width=0.08]  [draw opacity=0] (12,-3) -- (0,0) -- (12,3) -- cycle    ;
            %Straight Lines [id:da6776482937810475] 
            \draw    (252,175) ;
            \draw [shift={(252,175)}, rotate = 180] [fill={rgb, 255:red, 0; green, 0; blue, 0 }  ][line width=0.08]  [draw opacity=0] (12,-3) -- (0,0) -- (12,3) -- cycle    ;
            
            % Text Node
            \draw (105,179.08) node [anchor=north west][inner sep=0.75pt]    {$k-q$};
            % Text Node
            \draw (146,105.4) node [anchor=north west][inner sep=0.75pt]    {$q$};
            % Text Node
            \draw (202,227.4) node [anchor=north west][inner sep=0.75pt]    {$p$};
            % Text Node
            \draw (158.08,179.09) node [anchor=north west][inner sep=0.75pt]    {$k-q-p$};
            % Text Node
            \draw (233.08,152.09) node [anchor=north west][inner sep=0.75pt]    {$k-p$};
            \end{tikzpicture}            
    \end{gathered} \quad  \sim e^4 \int \dd[3]{\vb*{q}} \frac{1}{q^2} \int \dd[3]{\vb*{p}} \frac{1}{p^2} \sim \text{finite value},
\end{equation}
\begin{equation}
    \begin{gathered}
        \begin{tikzpicture}[x=0.75pt,y=0.75pt,yscale=-0.7,xscale=0.7]
            %uncomment if require: \path (0,300); %set diagram left start at 0, and has height of 300

%Straight Lines [id:da5247039683752392] 
\draw    (116,113) -- (296,113) ;
%Straight Lines [id:da04264830886220938] 
\draw    (155,113.12) ;
\draw [shift={(155,113.12)}, rotate = 180] [fill={rgb, 255:red, 0; green, 0; blue, 0 }  ][line width=0.08]  [draw opacity=0] (12,-3) -- (0,0) -- (12,3) -- cycle    ;
%Shape: Arc [id:dp3998429676946509] 
\draw  [draw opacity=0][dash pattern={on 4.5pt off 4.5pt}] (116.14,109.67) .. controls (119.18,84.54) and (158.02,64.67) .. (205.47,64.67) .. controls (253.5,64.67) and (292.69,85.02) .. (294.9,110.57) -- (205.47,112.85) -- cycle ; \draw  [dash pattern={on 4.5pt off 4.5pt}] (116.14,109.67) .. controls (119.18,84.54) and (158.02,64.67) .. (205.47,64.67) .. controls (253.5,64.67) and (292.69,85.02) .. (294.9,110.57) ;
%Shape: Arc [id:dp4738986594161243] 
\draw  [draw opacity=0][dash pattern={on 4.5pt off 4.5pt}] (173.99,111.39) .. controls (174.45,95.46) and (190.84,82.67) .. (210.99,82.67) .. controls (230.78,82.67) and (246.94,95.01) .. (247.95,110.53) -- (210.99,112.07) -- cycle ; \draw  [dash pattern={on 4.5pt off 4.5pt}] (173.99,111.39) .. controls (174.45,95.46) and (190.84,82.67) .. (210.99,82.67) .. controls (230.78,82.67) and (246.94,95.01) .. (247.95,110.53) ;
%Straight Lines [id:da753046974443065] 
\draw    (215,113) ;
\draw [shift={(215,113)}, rotate = 180] [fill={rgb, 255:red, 0; green, 0; blue, 0 }  ][line width=0.08]  [draw opacity=0] (12,-3) -- (0,0) -- (12,3) -- cycle    ;
%Straight Lines [id:da8214240576458545] 
\draw    (272,113) ;
\draw [shift={(272,113)}, rotate = 180] [fill={rgb, 255:red, 0; green, 0; blue, 0 }  ][line width=0.08]  [draw opacity=0] (12,-3) -- (0,0) -- (12,3) -- cycle    ;

% Text Node
\draw (112,117.08) node [anchor=north west][inner sep=0.75pt]    {$k-q$};
% Text Node
\draw (198,38.4) node [anchor=north west][inner sep=0.75pt]    {$q$};
% Text Node
\draw (204,82.4) node [anchor=north west][inner sep=0.75pt]    {$p$};
% Text Node
\draw (178.08,117.09) node [anchor=north west][inner sep=0.75pt]    {$k-q-p$};
% Text Node
\draw (262.08,118.09) node [anchor=north west][inner sep=0.75pt]    {$k-q$};
            \end{tikzpicture}            
    \end{gathered} \quad \sim e^4 \int \dd[3]{\vb*{q}} \frac{1}{q^2} \int \dd[3]{\vb*{p}} \frac{1}{p^2} \sim \text{finite value}.
\end{equation}
So we find only \eqref{eq:most-divergent-1} is worth noticing. The following diagram 
\[
    \begin{gathered}
        \begin{tikzpicture}[x=0.75pt,y=0.75pt,yscale=-0.7,xscale=0.7]
            %uncomment if require: \path (0,300); %set diagram left start at 0, and has height of 300
            
            %Straight Lines [id:da19683421857457994] 
            \draw    (96,175) -- (219.71,175) ;
            %Straight Lines [id:da2027210917931932] 
            \draw    (168,175.12) ;
            \draw [shift={(168,175.12)}, rotate = 180] [fill={rgb, 255:red, 0; green, 0; blue, 0 }  ][line width=0.08]  [draw opacity=0] (12,-3) -- (0,0) -- (12,3) -- cycle    ;
            %Shape: Circle [id:dp8177746314896752] 
            \draw   (132.17,109.33) .. controls (132.17,94.88) and (143.88,83.17) .. (158.33,83.17) .. controls (172.78,83.17) and (184.5,94.88) .. (184.5,109.33) .. controls (184.5,123.78) and (172.78,135.5) .. (158.33,135.5) .. controls (143.88,135.5) and (132.17,123.78) .. (132.17,109.33) -- cycle ;
            %Curve Lines [id:da8337458763239065] 
            \draw  [dash pattern={on 4.5pt off 4.5pt}]  (96,175) .. controls (100,148.67) and (114,120.67) .. (132.17,109.33) ;
            %Curve Lines [id:da4927523430855625] 
            \draw  [dash pattern={on 4.5pt off 4.5pt}]  (220.67,175) .. controls (216.67,148.67) and (202.67,120.67) .. (184.5,109.33) ;
            %Straight Lines [id:da8574857978112103] 
            \draw  [dash pattern={on 4.5pt off 4.5pt}]  (158.33,83.17) -- (158.33,136.17) ;
            
            % Text Node
            \draw (140,183.08) node [anchor=north west][inner sep=0.75pt]    {$k-q$};
            % Text Node
            \draw (96,113.4) node [anchor=north west][inner sep=0.75pt]    {$q$};
            % Text Node
            \draw (210,116.4) node [anchor=north west][inner sep=0.75pt]    {$q$};
            % Text Node
            \draw (163,103.4) node [anchor=north west][inner sep=0.75pt]    {$p$};          
            \end{tikzpicture}            
    \end{gathered} \quad \sim e^6 \int \dd[3]{\vb*{q}} \frac{1}{q^2} \frac{1}{q^2} \int \dd[3]{\vb*{p}} \frac{1}{p^2} \sim \frac{e^6}{q}
\]
has the same divergence degree with \eqref{eq:most-divergent-1} but with a higher order $e^n$ factor,
so it is thrown away as well. Therefore, we have verified \eqref{eq:rpa-diagram} to the second order.

\section{The effective potential, linear response and the electric susceptibility}\label{sec:response}

Now we go and see what \eqref{eq:rpa-diagram} means to the electric susceptibility. We have already discussed this and the density-density Green function in the free theory in \eqref{solid-eq:ext-electron-retarded-green-function} in \soliddoc. % TODO: wording

TODO: linear response theory

The LHS of \eqref{eq:rpa-diagram} is often defined as the \concept{effective (or screened) potential} $V^\text{eff}_{q}$.
We can easily find that if we put an external charge into the electron gas, the potential actually felt 
by the electrons is also $V^\text{eff}_q$, because we can repeat the argument of the field theoretic version 
of RPA to the external potential field and conclude that what we need to calculate is just diagrams like this:
\begin{equation}
    \begin{gathered}
        \begin{tikzpicture}[x=0.75pt,y=0.75pt,yscale=-0.7,xscale=0.7]
            %uncomment if require: \path (0,300); %set diagram left start at 0, and has height of 300
            
            %Straight Lines [id:da310724551214032] 
            \draw    (64.71,147) .. controls (66.38,145.33) and (68.04,145.33) .. (69.71,147) .. controls (71.38,148.67) and (73.04,148.67) .. (74.71,147) .. controls (76.38,145.33) and (78.04,145.33) .. (79.71,147) .. controls (81.38,148.67) and (83.04,148.67) .. (84.71,147) .. controls (86.38,145.33) and (88.04,145.33) .. (89.71,147) .. controls (91.38,148.67) and (93.04,148.67) .. (94.71,147) .. controls (96.38,145.33) and (98.04,145.33) .. (99.71,147) .. controls (101.38,148.67) and (103.04,148.67) .. (104.71,147) -- (104.71,147) ;
            %Shape: Circle [id:dp02531381623404605] 
            \draw   (104.71,147) .. controls (104.71,128.77) and (119.48,114) .. (137.71,114) .. controls (155.93,114) and (170.71,128.77) .. (170.71,147) .. controls (170.71,165.23) and (155.93,180) .. (137.71,180) .. controls (119.48,180) and (104.71,165.23) .. (104.71,147) -- cycle ;
            %Straight Lines [id:da2766460250771936] 
            \draw    (143.5,114.25) ;
            \draw [shift={(143.5,114.25)}, rotate = 180] [fill={rgb, 255:red, 0; green, 0; blue, 0 }  ][line width=0.08]  [draw opacity=0] (12,-3) -- (0,0) -- (12,3) -- cycle    ;
            %Straight Lines [id:da10629574750535431] 
            \draw    (136.33,179.5) -- (132.33,179.5) ;
            \draw [shift={(130.33,179.5)}, rotate = 360] [fill={rgb, 255:red, 0; green, 0; blue, 0 }  ][line width=0.08]  [draw opacity=0] (12,-3) -- (0,0) -- (12,3) -- cycle    ;
            %Straight Lines [id:da6019760109470824] 
            \draw  [dash pattern={on 4.5pt off 4.5pt}]  (170.71,147) -- (221,147) ;
            %Shape: Circle [id:dp9479643644310827] 
            \draw   (221,147) .. controls (221,128.77) and (235.77,114) .. (254,114) .. controls (272.23,114) and (287,128.77) .. (287,147) .. controls (287,165.23) and (272.23,180) .. (254,180) .. controls (235.77,180) and (221,165.23) .. (221,147) -- cycle ;
            %Straight Lines [id:da9823188277948145] 
            \draw    (259.5,114.25) ;
            \draw [shift={(259.5,114.25)}, rotate = 180] [fill={rgb, 255:red, 0; green, 0; blue, 0 }  ][line width=0.08]  [draw opacity=0] (12,-3) -- (0,0) -- (12,3) -- cycle    ;
            %Straight Lines [id:da332327528288457] 
            \draw    (254.33,179.5) -- (250.33,179.5) ;
            \draw [shift={(248.33,179.5)}, rotate = 360] [fill={rgb, 255:red, 0; green, 0; blue, 0 }  ][line width=0.08]  [draw opacity=0] (12,-3) -- (0,0) -- (12,3) -- cycle    ;
            %Straight Lines [id:da3242151721310709] 
            \draw  [dash pattern={on 4.5pt off 4.5pt}]  (287,147) -- (337.29,147) ;
            %Straight Lines [id:da3359280014770576] 
            \draw  [dash pattern={on 4.5pt off 4.5pt}]  (370.29,147) -- (420.59,147) ;
            %Shape: Circle [id:dp8034356584200093] 
            \draw   (420,147) .. controls (420,128.77) and (434.77,114) .. (453,114) .. controls (471.23,114) and (486,128.77) .. (486,147) .. controls (486,165.23) and (471.23,180) .. (453,180) .. controls (434.77,180) and (420,165.23) .. (420,147) -- cycle ;
            %Straight Lines [id:da587535030314791] 
            \draw    (458.5,114.25) ;
            \draw [shift={(458.5,114.25)}, rotate = 180] [fill={rgb, 255:red, 0; green, 0; blue, 0 }  ][line width=0.08]  [draw opacity=0] (12,-3) -- (0,0) -- (12,3) -- cycle    ;
            %Straight Lines [id:da6566134634535647] 
            \draw    (453.33,179.5) -- (449.33,179.5) ;
            \draw [shift={(447.33,179.5)}, rotate = 360] [fill={rgb, 255:red, 0; green, 0; blue, 0 }  ][line width=0.08]  [draw opacity=0] (12,-3) -- (0,0) -- (12,3) -- cycle    ;
            %Straight Lines [id:da8325413369422081] 
            \draw  [dash pattern={on 4.5pt off 4.5pt}]  (486,147) -- (526,147) ;
            %Flowchart: Summing Junction [id:dp15994467566403903] 
            \draw   (43.37,147) .. controls (43.37,141.11) and (48.15,136.33) .. (54.04,136.33) .. controls (59.93,136.33) and (64.71,141.11) .. (64.71,147) .. controls (64.71,152.89) and (59.93,157.67) .. (54.04,157.67) .. controls (48.15,157.67) and (43.37,152.89) .. (43.37,147) -- cycle ; \draw   (46.5,139.46) -- (61.58,154.54) ; \draw   (61.58,139.46) -- (46.5,154.54) ;
            %Straight Lines [id:da701612358462329] 
            \draw    (526,147) -- (570,222) ;
            %Straight Lines [id:da10662427746086744] 
            \draw    (545.05,179.7) -- (548,184.5) ;
            \draw [shift={(544,178)}, rotate = 58.39] [fill={rgb, 255:red, 0; green, 0; blue, 0 }  ][line width=0.08]  [draw opacity=0] (12,-3) -- (0,0) -- (12,3) -- cycle    ;
            %Straight Lines [id:da49775314062862286] 
            \draw    (570,72) -- (526,147) ;
            %Straight Lines [id:da2798959215281991] 
            \draw    (550.95,104.7) -- (548,109.5) ;
            \draw [shift={(552,103)}, rotate = 121.61] [fill={rgb, 255:red, 0; green, 0; blue, 0 }  ][line width=0.08]  [draw opacity=0] (12,-3) -- (0,0) -- (12,3) -- cycle    ;
            
            % Text Node
            \draw (339.29,147) node [anchor=west] [inner sep=0.75pt]    {$\cdots $};
            \end{tikzpicture}            
    \end{gathered}.
    \label{eq:external-screened}
\end{equation}
We define 
\begin{equation}
    \begin{gathered}
        \begin{tikzpicture}[x=0.75pt,y=0.75pt,yscale=-0.7,xscale=0.7]
            %uncomment if require: \path (0,300); %set diagram left start at 0, and has height of 300
            
            %Shape: Circle [id:dp27687562605272853] 
            \draw   (124.71,167) .. controls (124.71,148.77) and (139.48,134) .. (157.71,134) .. controls (175.93,134) and (190.71,148.77) .. (190.71,167) .. controls (190.71,185.23) and (175.93,200) .. (157.71,200) .. controls (139.48,200) and (124.71,185.23) .. (124.71,167) -- cycle ;
            %Straight Lines [id:da44287508502417894] 
            \draw    (163.5,134.25) ;
            \draw [shift={(163.5,134.25)}, rotate = 180] [fill={rgb, 255:red, 0; green, 0; blue, 0 }  ][line width=0.08]  [draw opacity=0] (12,-3) -- (0,0) -- (12,3) -- cycle    ;
            %Straight Lines [id:da6995765588189331] 
            \draw    (156.33,199.5) -- (152.33,199.5) ;
            \draw [shift={(150.33,199.5)}, rotate = 360] [fill={rgb, 255:red, 0; green, 0; blue, 0 }  ][line width=0.08]  [draw opacity=0] (12,-3) -- (0,0) -- (12,3) -- cycle    ;
            %Straight Lines [id:da2952513301761772] 
            \draw    (124.71,167) -- (111,167) ;
            %Straight Lines [id:da19550468559149015] 
            \draw    (204.41,167) -- (190.71,167) ;
            
            % Text Node
            \draw (109,167) node [anchor=east] [inner sep=0.75pt]    {$q$};
            % Text Node
            \draw (177,117) node [anchor=east] [inner sep=0.75pt]    {$k+q$};
            % Text Node
            \draw (168,216) node [anchor=east] [inner sep=0.75pt]    {$k$};            
            \end{tikzpicture}                      
    \end{gathered} = 2 \times (-1) \times  \int \frac{\dd[4]{k}}{(2\pi)^4} \ii G^0_{k} \times \ii G^0_{k+q} \eqqcolon \ii \Pi^0_q,
    \label{eq:normal-interaction-correction}
\end{equation}
where the factor $2$ comes from the spin-$1/2$ nature of the electron, $(-1)$ comes from the closed fermionic line,
and $G^0$ is the free electron Green function. In this way, \eqref{eq:external-screened} is just 
\[
    - \ii V^\text{eff}_q = - \ii V^\text{ext}_q + (- \ii V^\text{ext}_q) (\ii \Pi^0_q) (- \ii V_q) + 
    (- \ii V^\text{ext}_q) (\ii \Pi^0_q) (- \ii V_q) (\ii \Pi^0_q) (- \ii V_q) + \cdots,
\]
and the resummation gives 
\begin{equation} \marginnote{Wen Sec.~5.4.3}
    V^\text{eff}_q = V^\text{ext}_q \frac{1}{1 - V_q \Pi^0_q}.
\end{equation}
For \eqref{eq:rpa-diagram}, we find that when $\vb*{q} \to 0$, $V^\text{eff}_q$ is no longer divergent. 
TODO: This also eliminates the diverging \eqref{solid-eq:group-velocity-fermi-surface-hf} in \soliddoc.
Also, now since we have 
\[
    \vb*{E} \sim V^\text{eff}, \quad \vb*{D} \sim \epsilon_0 V^\text{ext},
\]
it seem that  
\begin{equation} \marginnote{Zhengzhong Li Eq.~(4.6.14)}
    \epsilon_\text{r} = \frac{V^\text{ext}_{q}}{V^\text{eff}_q} = 1 - V_q \Pi^0_q.
    \label{eq:epsilon-r}
\end{equation}

\begin{figure}
    \centering
    

\tikzset{every picture/.style={line width=0.75pt}} %set default line width to 0.75pt        

\begin{tikzpicture}[x=0.75pt,y=0.75pt,yscale=-0.7,xscale=0.7]
%uncomment if require: \path (0,300); %set diagram left start at 0, and has height of 300

%Straight Lines [id:da5359044413830067] 
\draw    (100,66) .. controls (101.67,64.33) and (103.33,64.33) .. (105,66) .. controls (106.67,67.67) and (108.33,67.67) .. (110,66) .. controls (111.67,64.33) and (113.33,64.33) .. (115,66) .. controls (116.67,67.67) and (118.33,67.67) .. (120,66) .. controls (121.67,64.33) and (123.33,64.33) .. (125,66) .. controls (126.67,67.67) and (128.33,67.67) .. (130,66) .. controls (131.67,64.33) and (133.33,64.33) .. (135,66) .. controls (136.67,67.67) and (138.33,67.67) .. (140,66) .. controls (141.67,64.33) and (143.33,64.33) .. (145,66) .. controls (146.67,67.67) and (148.33,67.67) .. (150,66) .. controls (151.67,64.33) and (153.33,64.33) .. (155,66) .. controls (156.67,67.67) and (158.33,67.67) .. (160,66) -- (160,66) ;
%Shape: Circle [id:dp9519210306756258] 
\draw  [fill={rgb, 255:red, 155; green, 155; blue, 155 }  ,fill opacity=0.63 ] (160,66) .. controls (160,52.19) and (171.19,41) .. (185,41) .. controls (198.81,41) and (210,52.19) .. (210,66) .. controls (210,79.81) and (198.81,91) .. (185,91) .. controls (171.19,91) and (160,79.81) .. (160,66) -- cycle ;
%Straight Lines [id:da9845404378135585] 
\draw    (210,66) .. controls (211.67,64.33) and (213.33,64.33) .. (215,66) .. controls (216.67,67.67) and (218.33,67.67) .. (220,66) .. controls (221.67,64.33) and (223.33,64.33) .. (225,66) .. controls (226.67,67.67) and (228.33,67.67) .. (230,66) .. controls (231.67,64.33) and (233.33,64.33) .. (235,66) .. controls (236.67,67.67) and (238.33,67.67) .. (240,66) .. controls (241.67,64.33) and (243.33,64.33) .. (245,66) -- (246,66) -- (246,66) ;
%Straight Lines [id:da9218097539710475] 
\draw    (281,66) .. controls (282.67,64.33) and (284.33,64.33) .. (286,66) .. controls (287.67,67.67) and (289.33,67.67) .. (291,66) .. controls (292.67,64.33) and (294.33,64.33) .. (296,66) .. controls (297.67,67.67) and (299.33,67.67) .. (301,66) .. controls (302.67,64.33) and (304.33,64.33) .. (306,66) .. controls (307.67,67.67) and (309.33,67.67) .. (311,66) .. controls (312.67,64.33) and (314.33,64.33) .. (316,66) -- (317,66) -- (317,66) ;
%Shape: Circle [id:dp2967808715516478] 
\draw  [fill={rgb, 255:red, 155; green, 155; blue, 155 }  ,fill opacity=0.63 ] (317,66) .. controls (317,52.19) and (328.19,41) .. (342,41) .. controls (355.81,41) and (367,52.19) .. (367,66) .. controls (367,79.81) and (355.81,91) .. (342,91) .. controls (328.19,91) and (317,79.81) .. (317,66) -- cycle ;
%Straight Lines [id:da18142547031292544] 
\draw    (367,66) .. controls (368.67,64.33) and (370.33,64.33) .. (372,66) .. controls (373.67,67.67) and (375.33,67.67) .. (377,66) .. controls (378.67,64.33) and (380.33,64.33) .. (382,66) .. controls (383.67,67.67) and (385.33,67.67) .. (387,66) .. controls (388.67,64.33) and (390.33,64.33) .. (392,66) .. controls (393.67,67.67) and (395.33,67.67) .. (397,66) .. controls (398.67,64.33) and (400.33,64.33) .. (402,66) .. controls (403.67,67.67) and (405.33,67.67) .. (407,66) .. controls (408.67,64.33) and (410.33,64.33) .. (412,66) .. controls (413.67,67.67) and (415.33,67.67) .. (417,66) .. controls (418.67,64.33) and (420.33,64.33) .. (422,66) .. controls (423.67,67.67) and (425.33,67.67) .. (427,66) -- (427,66) ;
%Straight Lines [id:da7407262669447372] 
\draw [color={rgb, 255:red, 0; green, 0; blue, 0 }  ,draw opacity=0.17 ]   (160,66) -- (126,215) ;
%Straight Lines [id:da004521370029740401] 
\draw [color={rgb, 255:red, 0; green, 0; blue, 0 }  ,draw opacity=0.17 ]   (203,48.67) -- (336,210.67) ;
%Shape: Circle [id:dp3823864123193821] 
\draw   (121.71,246.78) .. controls (121.71,233.96) and (132.1,223.56) .. (144.93,223.56) .. controls (157.75,223.56) and (168.15,233.96) .. (168.15,246.78) .. controls (168.15,259.6) and (157.75,270) .. (144.93,270) .. controls (132.1,270) and (121.71,259.6) .. (121.71,246.78) -- cycle ;
%Straight Lines [id:da7505964025545913] 
\draw    (149,223.74) ;
\draw [shift={(149,223.74)}, rotate = 180] [fill={rgb, 255:red, 0; green, 0; blue, 0 }  ][line width=0.08]  [draw opacity=0] (12,-3) -- (0,0) -- (12,3) -- cycle    ;
%Straight Lines [id:da5246344988939031] 
\draw    (143.96,269.65) -- (141.74,269.65) ;
\draw [shift={(139.74,269.65)}, rotate = 360] [fill={rgb, 255:red, 0; green, 0; blue, 0 }  ][line width=0.08]  [draw opacity=0] (12,-3) -- (0,0) -- (12,3) -- cycle    ;
%Straight Lines [id:da6736609833829976] 
\draw  [dash pattern={on 4.5pt off 4.5pt}]  (168.15,246.78) -- (203.54,246.78) ;
%Shape: Ellipse [id:dp8105133647452574] 
\draw   (203.54,246.78) .. controls (203.54,233.96) and (213.93,223.56) .. (226.76,223.56) .. controls (239.58,223.56) and (249.98,233.96) .. (249.98,246.78) .. controls (249.98,259.6) and (239.58,270) .. (226.76,270) .. controls (213.93,270) and (203.54,259.6) .. (203.54,246.78) -- cycle ;
%Straight Lines [id:da2547575394842021] 
\draw    (230.63,223.74) ;
\draw [shift={(230.63,223.74)}, rotate = 180] [fill={rgb, 255:red, 0; green, 0; blue, 0 }  ][line width=0.08]  [draw opacity=0] (12,-3) -- (0,0) -- (12,3) -- cycle    ;
%Straight Lines [id:da25448296192833997] 
\draw    (226.99,269.65) -- (224.77,269.65) ;
\draw [shift={(222.77,269.65)}, rotate = 360] [fill={rgb, 255:red, 0; green, 0; blue, 0 }  ][line width=0.08]  [draw opacity=0] (12,-3) -- (0,0) -- (12,3) -- cycle    ;
%Straight Lines [id:da2539557521545195] 
\draw  [dash pattern={on 4.5pt off 4.5pt}]  (249.98,246.78) -- (285.36,246.78) ;
%Straight Lines [id:da6768780561998557] 
\draw  [dash pattern={on 4.5pt off 4.5pt}]  (308.58,246.78) -- (343.97,246.78) ;
%Shape: Circle [id:dp6868444995809804] 
\draw   (343.56,246.78) .. controls (343.56,233.96) and (353.96,223.56) .. (366.78,223.56) .. controls (379.6,223.56) and (390,233.96) .. (390,246.78) .. controls (390,259.6) and (379.6,270) .. (366.78,270) .. controls (353.96,270) and (343.56,259.6) .. (343.56,246.78) -- cycle ;
%Straight Lines [id:da0009943502085687506] 
\draw    (370.65,223.74) ;
\draw [shift={(370.65,223.74)}, rotate = 180] [fill={rgb, 255:red, 0; green, 0; blue, 0 }  ][line width=0.08]  [draw opacity=0] (12,-3) -- (0,0) -- (12,3) -- cycle    ;
%Straight Lines [id:da5826619710533885] 
\draw    (367.01,269.65) -- (364.79,269.65) ;
\draw [shift={(362.79,269.65)}, rotate = 360] [fill={rgb, 255:red, 0; green, 0; blue, 0 }  ][line width=0.08]  [draw opacity=0] (12,-3) -- (0,0) -- (12,3) -- cycle    ;

% Text Node
\draw (248,66) node [anchor=west] [inner sep=0.75pt]    {$\cdots $};
% Text Node
\draw (283.07,246.78) node [anchor=west] [inner sep=0.75pt]    {$\cdots $};


\end{tikzpicture}

    \caption{After obtaining \eqref{eq:epsilon-r} we need to take the self-energy correction of photons as well}
    \label{fig:another-self-energy}
\end{figure}

\begin{note*}{So many Green functions}
    If we are working in the zero temperature real time field theory, $G^0$ in 
    \eqref{eq:normal-interaction-correction} is not the retarded Green function 
    \begin{equation} \marginnote{Huaiyu Wu Eq.~(9.4.21)}
        G^{0, \text{ret}}_k = \frac{1}{k^0 - \xi_{\vb*{k}} + \ii 0^+} 
    \end{equation}  
    but the Feynman Green function 
    \begin{equation} \marginnote{Huaiyu Wu Eq.~(9.4.23)}
        G^{0}_k = \frac{1}{k^0 - \xi_{\vb*{k}} + \ii 0^+ \sgn(\omega)} = \frac{\theta(\xi_{\vb*{k}})}{k^0 - \xi_{\vb*{k}} + \ii 0^+} + \frac{\theta(- \xi_{\vb*{k}})}{k^0 - \xi_{\vb*{k}} - \ii 0^+}.
    \end{equation}
    The Feynman Green function includes the contribution of holes (the antiparticle of the electron in condensed
    matter field theory). We can see that it is actually much easier to work in the Matsubara formalism.
\end{note*}

Note that the definition of $\epsilon$ itself is ambiguous. For example, we may also consider the 
self-energy correction of \emph{photons} (beside photons passing the Coulomb interaction) themselves, as is shown in \prettyref{fig:another-self-energy}.
This gives rise to \concept{local field enhancement} (see Section~\ref{early-sec:clausius} in \earlydoc, 
for example). 

\section{Evaluating the electric susceptibility and bosonic modes}\label{sec:boson-modes}

Now it is time to evaluate the analytic properties of the susceptibility. \marginnote{Zhengzhong Li, Sec.~4.6}
The dispersion relations of bosonic modes are given by $\epsilon = 0$ (not $\epsilon \to \infty$ -- see 
Section~\ref{optics-sec:green-and-linear-response} in \opticsdoc). 
Note that the imaginary part of $\epsilon$ indicates damping (see Section~\ref{optics-sec:in-interior-uniform}
in \opticsdoc), and therefore if $\Im \epsilon \neq 0$, even if on an $(\omega, \vb*{k})$ point we have 
$\Re \epsilon = 0$, this is not an eigenmode. However, if $\Im \epsilon$ is not large, we can still say 
that ``there is a mode at $(\omega, \vb*{k})$ with a finite lifetime''. So what we are going to do is 
to solve the equation $\Re \epsilon(\omega, \vb*{k}) = 0$ and check the lifetime of each mode.



\begin{note*}{}
    The occurrence of the plasmon is not accidental. What we are doing is to actually treat some of the 
    interaction lines as actually (quasi)particles. The energy of the plasmon comes from the Coulomb 
    repulsion energy. This is an example of the Hubbard-Stratonovich transformation, where a $\psi \psi 
    \bar{\psi} \bar{\psi}$ interaction term is assumed to be carried by a kind of boson. RPA for electrons 
    is just the self energy correction for the plasmon (or more generally, the bosonic field introduced 
    by the Hubbard-Stratonovich transformation).
\end{note*}

\end{document}