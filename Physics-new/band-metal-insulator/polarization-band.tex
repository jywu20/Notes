\documentclass[hyperref, a4paper]{article}

\usepackage{geometry}
\usepackage{titling}
\usepackage{titlesec}
% No longer needed, since we will use enumitem package
% \usepackage{paralist}
\usepackage{enumitem}
\usepackage{footnote}
\usepackage{enumerate}
\usepackage{amsmath, amssymb, amsthm}
\usepackage{mathtools}
\usepackage{bbm}
\usepackage{graphicx}
\usepackage{subcaption}
\usepackage{soulutf8}
\usepackage{physics}
\usepackage{tensor}
\usepackage{siunitx}
\usepackage[version=4]{mhchem}
\usepackage{tikz}
\usepackage{xcolor}
\usepackage{listings}
\usepackage{autobreak}
\usepackage[ruled, vlined, linesnumbered]{algorithm2e}
\usepackage{nameref,zref-xr}
\zxrsetup{toltxlabel}
\usepackage[backend=bibtex]{biblatex}
\addbibresource{elasticity.bib}
\usepackage[colorlinks,unicode]{hyperref} % , linkcolor=black, anchorcolor=black, citecolor=black, urlcolor=black, filecolor=black
\usepackage[most]{tcolorbox}
\usepackage{prettyref}

% Page style
\geometry{left=3.18cm,right=3.18cm,top=2.54cm,bottom=2.54cm}
\titlespacing{\paragraph}{0pt}{1pt}{10pt}[20pt]
\setlength{\droptitle}{-5em}

% More compact lists 
\setlist[itemize]{
    itemindent=17pt, 
    leftmargin=1pt,
    listparindent=\parindent,
    parsep=0pt,
}

% Math operators
\DeclareMathOperator{\timeorder}{\mathcal{T}}
\DeclareMathOperator{\diag}{diag}
\DeclareMathOperator{\legpoly}{P}
\DeclareMathOperator{\primevalue}{P}
\DeclareMathOperator{\sgn}{sgn}
\DeclareMathOperator{\res}{Res}
\newcommand*{\ii}{\mathrm{i}}
\newcommand*{\ee}{\mathrm{e}}
\newcommand*{\const}{\mathrm{const}}
\newcommand*{\suchthat}{\quad \text{s.t.} \quad}
\newcommand*{\argmin}{\arg\min}
\newcommand*{\argmax}{\arg\max}
\newcommand*{\normalorder}[1]{: #1 :}
\newcommand*{\pair}[1]{\langle #1 \rangle}
\newcommand*{\fd}[1]{\mathcal{D} #1}
\DeclareMathOperator{\bigO}{\mathcal{O}}

% TikZ setting
\usetikzlibrary{arrows,shapes,positioning}
\usetikzlibrary{arrows.meta}
\usetikzlibrary{decorations.markings}
\tikzstyle arrowstyle=[scale=1]
\tikzstyle directed=[postaction={decorate,decoration={markings,
    mark=at position .5 with {\arrow[arrowstyle]{stealth}}}}]
\tikzstyle ray=[directed, thick]
\tikzstyle dot=[anchor=base,fill,circle,inner sep=1pt]

% Algorithm setting
% Julia-style code
\SetKwIF{If}{ElseIf}{Else}{if}{}{elseif}{else}{end}
\SetKwFor{For}{for}{}{end}
\SetKwFor{While}{while}{}{end}
\SetKwProg{Function}{function}{}{end}
\SetArgSty{textnormal}

\newcommand*{\concept}[1]{{\textbf{#1}}}

% Embedded codes
\lstset{basicstyle=\ttfamily,
  showstringspaces=false,
  commentstyle=\color{gray},
  keywordstyle=\color{blue}
}

% Reference formatting
\newcommand*{\citesec}[1]{\S~{#1}}
\newcommand*{\citechap}[1]{chap.~{#1}}
\newcommand*{\citefig}[1]{Fig.~{#1}}
\newcommand*{\citetable}[1]{Table~{#1}}
\newcommand*{\citepage}[1]{pp.~{#1}}
\newrefformat{fig}{Fig.~\ref{#1}}
\newcommand*{\term}[1]{\textit{#1}}

% Color boxes
\tcbuselibrary{skins, breakable, theorems}

\newtcbtheorem{infobox}{Box}{
    enhanced,
    boxrule=0pt,
    colback=blue!5,
    colframe=blue!5,
    coltitle=blue!50,
    borderline west={4pt}{0pt}{blue!65},
    sharp corners,
    fonttitle=\bfseries, 
    breakable,
    before upper={\parindent15pt\noindent}}{box}
\newtcbtheorem[use counter from=infobox]{theorybox}{Box}{
    enhanced,
    boxrule=0pt,
    colback=orange!5, 
    colframe=orange!5, 
    coltitle=orange!50,
    borderline west={4pt}{0pt}{orange!65},
    sharp corners,
    fonttitle=\bfseries, 
    breakable,
    before upper={\parindent15pt\noindent}}{box}
\newtcbtheorem[use counter from=infobox]{learnbox}{Box}{
    enhanced,
    boxrule=0pt,
    colback=green!5,
    colframe=green!5,
    coltitle=green!50,
    borderline west={4pt}{0pt}{green!65},
    sharp corners,
    fonttitle=\bfseries, 
    breakable,
    before upper={\parindent15pt\noindent}}{box}


\newenvironment{shelldisplay}{\begin{lstlisting}}{\end{lstlisting}}

\newcommand*{\kB}{k_{\text{B}}}
\newcommand*{\muB}{\mu_{\text{B}}}
\newcommand*{\efermi}{E_{\text{F}}}
\newcommand*{\pfermi}{p_{\text{F}}}
\newcommand*{\vfermi}{v_{\text{F}}}
\newcommand*{\sA}{\text{A}}
\newcommand*{\sB}{\text{B}}
\newcommand*{\Tc}{T_{\text{c}}}
\newcommand*{\hethree}{$^3$He}
\newcommand*{\hefour}{$^4$He}

\title{Polarization in band theory}
\author{Jinyuan Wu}

\begin{document}

\maketitle

\section{Ill-definedness of $\vb*{P}$ using $\rho(\vb*{r})$}

A tempting definition of the polarization vector is 
\begin{equation}
    \vb*{P}_{\text{dip}} = \frac{1}{V_{\text{u.c.}}} \int_{\text{u.c.}} \vb*{r} \rho(\vb*{r}) \dd[3]{\vb*{r}},
\end{equation}
where $\rho(\vb*{r})$ includes the charge distribution 
of both electrons and atoms.
Despite its clear physical meaning of ``total dipole moment'',
this definition depends on the choice of the primitive unit cell. 
To see why, note that if we do an infinitesimal change of $\vb*{P}$,
then according to the charge conservation equation, 
which may be seen as a constraint on the variance, 
we have 
\begin{equation}
    \begin{aligned}
        \var{\vb*{P}}_{\text{dip}} &= 
        \frac{1}{V_{\text{u.c.}}} \int_{\text{u.c.}} \vb*{r} \var{\rho}(\vb*{r}) \dd[3]{\vb*{r}} \\
        &= \frac{1}{V_{\text{u.c.}}} \int_{\text{u.c.}} \vb*{r} (- \var{t} \div{\vb*{j}}) \dd[3]{\vb*{r}} \\
        &= - \frac{\var{t}}{V_{\text{u.c.}}} \int_{\partial(\text{u.c.})} \vb*{r} (\dd{\vb*{S}} \cdot \vb*{j})
        + \frac{\var{t}}{V_{\text{u.c.}}} \int_{\text{u.c.}} \vb*{j} \dd[3]{\vb*{r}}.
    \end{aligned}
\end{equation}
The last term is the average current,
which does have a clear definition: 
\begin{equation}
    \vb*{J}(t) = \frac{1}{V_{\text{u.c.}}} \int_{\text{u.c.}} \vb*{j}(\vb*{r}, t) \dd[3]{\vb*{r}},
\end{equation}
which is \emph{not} dependent to the spatial position of the primitive unit cell, 
since at the ground state, 
$\rho(\vb*{r})$ is strictly periodic:
the atomic part is of course periodic,
and the electronic part $- e \sum_{n, \vb*{k}} \abs*{\psi_{n \vb*{k}}}^2$,
since $\abs*{\psi_{n \vb*{k}}}^2 = \abs*{u_{n \vb*{k}}}^2$,
the latter being periodic, 
is also periodic.
So we find the variance of $\vb*{P}_{\text{dip}}$ 
equals to one term that is not dependent to the primitive unit cell 
and another term that is highly dependent to the surface 
of the primitive unit cell. 
On the other hand, from physical intuition we expect to see 
\begin{equation}
    \var{\vb*{P}(t)} = \vb*{J} \var{t}.
    \label{eq:j-is-derivative-of-p}
\end{equation}
Thus $\vb*{P}_{\text{dip}}$ can't be a good definition of $\vb*{P}$.

One way to make up for the loss is to average $\vb*{P}_{\text{dip}}$ 
over possible positions of the primitive unit cell.
This however gives a constantly vanishing value, 
since now 
\[
    \bar{\vb*{P}}_{\text{dip}} 
    \propto \int_{\text{u.c.}} \dd[3]{\vb*{r}'} \int_{\text{u.c.}} \vb*{r} \rho(\vb*{r} - \vb*{r}') \dd[3]{\vb*{r}}  ,
\]
where instead of moving the primitive unit cell,
we equivalently move the charge distribution.
But then due to charge neutrality, 
we find $\int \dd[3]{\vb*{r}'} \rho(\vb*{r} - \vb*{r}') = 0$,
so the whole equation vanishes. 

We may also want to try to define $\vb*{P}$ for a supercell, 
possibly the crystal itself,
and if the surface-dependent term vanishes, 
we now at least have a macroscopic definition of $\vb*{P}$.
But then note that $\vb*{r} \sim L$ 
while $\dd{\vb*{S}} \sim L^2$,
and $1 / V \sim 1 / L^3$,
so this doesn't seem to be possible for a generic crystal.

\section{A dynamic definition} 

An alternative way to define $\vb*{P}$
is to integrate \eqref{eq:j-is-derivative-of-p}.
That's to say, we first need an expression of $\partial_\lambda \vb*{P}$.

\subsection{Review of some quantum mechanic results}

Before doing that let's show some general formalism in quantum mechanics. 
Suppose we have Hamiltonian $H(\lambda)$,
and thus 
\begin{equation}
    \partial_\lambda \expval{O}_n = 
    \mel{\partial_\lambda n}{O}{n} 
    + \mel{n}{O}{\partial_\lambda n}
    = 2 \Re \mel{n}{O}{\partial_\lambda n}
    = 2 \Re \mel{\partial_\lambda n}{O}{n},
\end{equation}
and it can be proved that we have the following Sternheimer equation
\begin{equation}
    \ket{\partial_\lambda n} = - \ii A_n \ket*{n} 
    + \sum_{m \neq n} \frac{\dyad{m}}{E_n - E_m} \partial_\lambda H \ket*{n},
    \label{eq:sternheimer}
\end{equation}
where 
\begin{equation}
    A_n = \ii \braket{n}{\partial_\lambda n}
    \label{eq:berry-phase}
\end{equation}
is the Berry connection, 
the integral of which is the Berry phase.
Note that the only $\ket*{n}$ component of $\ket*{\partial_\lambda n}$ 
is the $- \ii A_n$ term, 
and therefore from the norm conservation condition, 
we find that $A_n$ is always real. 

We define 
\begin{equation}
    T_n = \sum_{m \neq n} \frac{\dyad{m}}{E_n - E_m}.
\end{equation}
Again, since the only $\ket*{n}$ component in $\ket*{\partial_\lambda n}$ is the Berry phase term, 
we have 
\begin{equation}
    Q_n \ket*{\partial_\lambda n} = T_n \partial_\lambda H \ket*{n},
    \label{eq:q-t-lambda}
\end{equation}
where 
\begin{equation}
    Q_n = \sum_{m \neq n} \dyad{m},
\end{equation}
which removes the $\ket*{n}$ component from a wave function.
\eqref{eq:sternheimer} then becomes 
\begin{equation}
    Q_n \ket*{\partial_\lambda n} = T_n \partial_\lambda H \ket*{n}.
\end{equation}
Since in 
\[
    \mel{n}{O}{\partial_n n} = 
    - \ii A_n \mel{n}{O }{n}
    + \mel{n}{O Q_n}{n},
\]
the first term is always imaginary, we have 
\begin{equation}
    \partial_\lambda \expval{O}_n = 2 \Re \mel{n}{O Q_n}{\partial_\lambda n}
    = 2 \Re \mel{\partial_\lambda n}{Q_n O}{n}.
\end{equation}

Now if we are in a many-body fermionic system with very weak interaction between particles,
we have 
\begin{equation}
    \partial_\lambda \expval{O} = \sum_{n}^{\text{occ}} \partial_\lambda \expval{O}_n
    = 2 \Re \sum_{n}^{\text{occ}} \mel{n}{O Q_n}{\partial_\lambda n}
    = 2 \Re \sum_{n}^{\text{occ}} \mel{\partial_\lambda n}{Q_n O}{n}.
    \label{eq:pdv-O-lambda-1}
\end{equation}
This expression can be further simplified:
after defining 
\begin{equation}
    Q = \sum_{n}^{\text{unocc}} \dyad{n} = 1 - \sum_{n}^{\text{occ}} \dyad{n},
\end{equation}
we have
\[
    \begin{aligned}
        2\Re \sum_{n}^{\text{occ}} \mel{n}{O (Q_n - Q)}{\partial_\lambda n}
        &= 2\Re \sum_{n}^{\text{occ}}  \left(
            \sum_{m \neq n} 
            - \sum_{m}^{\text{unocc}}
        \right)
        \mel{n}{O}{m} \braket{m}{\partial_\lambda n}  \\
        &= 2\Re \sum_{n}^{\text{occ}} \sum_{m \neq n}^{\text{occ}}
        \mel{n}{O}{m} \braket{m}{\partial_\lambda n} \\
        &= \sum_{n}^{\text{occ}} \sum_{m \neq n}^{\text{occ}}
        (
            \mel{n}{O}{m} \braket{m}{\partial_\lambda n}
            + \mel{m}{O}{n} \braket{\partial_\lambda n}{m}
        ).
    \end{aligned}
\]
Since $\braket{n}{m}$ doesn't change no matter how you adjust $\lambda$,
we have 
\begin{equation}
    0 = \partial_\lambda \braket{n}{m} 
    = \braket*{\partial_\lambda n}{m} 
    + \braket*{n}{\partial_\lambda m},
    \label{eq:switch-partial}
\end{equation}
and thus 
\[
    \begin{aligned}
        2\Re \sum_{n}^{\text{occ}} \mel{n}{O (Q_n - Q)}{\partial_\lambda n}
        &= \sum_{n}^{\text{occ}} \sum_{m \neq n}^{\text{occ}}
        (
            \mel{n}{O}{m} \braket{m}{\partial_\lambda n}
            - \mel{m}{O}{n} \braket*{n}{\partial_\lambda m}
        ) \\
        &= 0,
    \end{aligned}
\]
and thus \eqref{eq:pdv-O-lambda-1} becomes 
\begin{equation}
    \partial_\lambda \expval{O} 
    = 2 \Re \sum_{n}^{\text{occ}} \mel{\partial_\lambda n}{Q O}{n}
    = 2 \Re \sum_{n}^{\text{occ}} \mel{n}{O Q}{\partial_\lambda n}.
    \label{eq:pdv-O-lambda-2}
\end{equation}
This is the formula that's going to be used 
to evaluate $\partial_\lambda \vb*{P}$.

\subsection{The change of polarization}

In band theory, different $\vb*{k}$ points don't 
influence each other, 
and therefore $\vb*{k}$ can be thought of 
as an instance of $\lambda$.
Suppose $H = T + V$ is the single electron Hamiltonian,
which acts on Bloch states 
\begin{equation}
    \psi_{n \vb*{k}}(\vb*{r}) = \ee^{\ii \vb*{k} \cdot \vb*{r}} u_{n \vb*{k}}(\vb*{r}),
\end{equation}
and we find the Hamiltonian acting on $u_{n \vb*{k}}$ is 
\begin{equation}
    H_{\vb*{k}} = \ee^{- \ii \vb*{k} \cdot \vb*{r}} H \ee^{\ii \vb*{k} \cdot \vb*{r}},
\end{equation}
and we have 
\begin{equation}
    H_{\vb*{k}} u_{n \vb*{k}} = E_{n \vb*{k}} u_{n \vb*{k}}.
\end{equation}
Since we are going to work with $u_{n \vb*{k}}$, 
we need to modify the normalization convention.
Usually, the inner product is defined as 
\begin{equation}
    \int \dd[d]{\vb*{r}} \psi^*_{n' \vb*{k}'}(\vb*{r}) \psi_{n \vb*{k}}(\vb*{r})
    = \delta_{n n'} \delta_{\vb*{k} \vb*{k}'},
\end{equation}
in which we integrate over the whole sample $V$.
Thus, in empty lattice, we have 
\begin{equation}
    \psi_{n \vb*{k}}(\vb*{r}) = \ee^{\ii \vb*{k} \cdot \vb*{r}}
    \underbrace{
        \frac{1}{\sqrt{V}} \ee^{\ii \vb*{G}_n \cdot \vb*{r}}
    }_{u_{n \vb*{k}}}.
\end{equation}
However, for TODO 

Formally, the polarization is 
\begin{equation}
    \vb*{P} = \frac{1}{V} \sum_{\vb*{k}} \sum_{n}^{\text{occ}} \expval{- e \vb*{r}}{\psi_{n \vb*{k}}},
\end{equation}
and thus 
\begin{equation}
    \begin{aligned}
        \partial_\lambda \expval{\vb*{P}} &= 
        - e \int \frac{\dd[d]{\vb*{k}}}{(2\pi)^d} 
        \partial_\lambda \expval{\vb*{r}}_{\vb*{k}} \\
        &= - e \int \frac{\dd[d]{\vb*{k}}}{(2\pi)^d} 
        \cdot 2 \Re \sum_{n}^{\text{occ}} \mel{\partial_\lambda u_{n \vb*{k}}}{Q_{\vb*{k}} \vb*{r}}{u_{n \vb*{k}}}.
    \end{aligned}
    \label{eq:change-of-p-1}
\end{equation}
Note that here $Q_{\vb*{k}}$ is not $Q_{n}$ in \eqref{eq:pdv-O-lambda-1},
but $Q$ in \eqref{eq:pdv-O-lambda-2}: 
here $\vb*{k}$ is considered an ``external'' parameter.
What we want to do is to eliminate any $\vb*{r}$ in the expression 
to avoid ambiguity mentioned above. 
We define 
\begin{equation}
    \vb*{v}_{\vb*{k}} = \frac{1}{\ii \hbar} \comm*{\vb*{r}}{H_{\vb*{k}}}
    = \ee^{- \ii \vb*{k} \cdot \vb*{r}} v \ee^{\ii \vb*{k} \cdot \vb*{r}},
    \label{eq:v-k-def}
\end{equation}
which is the counterpart of $\vb*{v}$ 
in the quantum mechanics about $\{u_{n \vb*{k}}\}$,
and we can then find 
\begin{equation}
    \vb*{v}_{\vb*{k}} = \frac{1}{\hbar} \pdv{H_{\vb*{k}}}{\vb*{k}},
    \label{eq:v-is-pdv-h-k}
\end{equation}
since 
\[
    \pdv{H_{\vb*{k}}}{\vb*{k}}
    = (- \ii \vb*{r}) \ee^{- \ii \vb*{k} \cdot \vb*{r}}  H \ee^{\ii \vb*{k} \cdot \vb*{r}}
    + \ee^{- \ii \vb*{k} \cdot \vb*{r}} H \ee^{\ii \vb*{k} \cdot \vb*{r}} \cdot \ii \vb*{r}
    = - \ii \comm*{\vb*{r}}{H_{\vb*{k}}} .
\]
Then from \eqref{eq:v-k-def}, we find 
\[
    \mel{u_{m \vb*{k}}}{\vb*{v}_{\vb*{k}}}{u_{n \vb*{k}}}
    = \frac{1}{\ii \hbar} \mel{u_{m \vb*{k}}}{\vb*{r} H_{\vb*{k}} - H_{\vb*{k}} \vb*{r}}{u_{n \vb*{k}}}
    = \frac{1}{\ii \hbar} (E_{n \vb*{k}} - E_{m \vb*{k}}) 
    \mel{u_{m \vb*{k}}}{\vb*{r}}{u_{n \vb*{k}}},
\]
\[
    \mel{u_{m \vb*{k}}}{\vb*{r}}{u_{n \vb*{k}}} = \frac{\ii \hbar}{E_{n \vb*{k}} - E_{m \vb*{k}}}
    \mel{u_{m \vb*{k}}}{\vb*{v}_{\vb*{k}}}{u_{n \vb*{k}}},
\]
and therefore 
\[
    \underbrace{\sum_{m \neq n} \dyad{u_{m \vb*{k}}}}_{Q_{n \vb*{k}}} \vb*{r} \ket{u_{n \vb*{k}}}
    = \ii \hbar \underbrace{
        \sum_{m \neq n} \frac{\dyad{u_{m \vb*{k}}}}{E_{n \vb*{k}} - E_{m \vb*{k}}} 
    }_{T_{n \vb*{k}}}
    \vb*{v}_{\vb*{k}} \ket{u_{n \vb*{k}}},
\]
and the RHS of \eqref{eq:change-of-p-1} becomes 
(note that according to \eqref{eq:pdv-O-lambda-1} and \eqref{eq:pdv-O-lambda-2},
$Q_{n \vb*{k}}$ and $Q_{\vb*{k}}$ 
can be used in place of each other in the expression)
\begin{equation}
    \begin{aligned}
        \partial_\lambda \expval{\vb*{P}}
        &= - e \int \frac{\dd[d]{\vb*{k}}}{(2\pi)^d} 
        \cdot 2 \Re \sum_{n}^{\text{occ}} 
        \mel{\partial_\lambda u_{n \vb*{k}}}{\ii \hbar T_{n \vb*{k}} \vb*{v}_{\vb*{k}}}{u_{n \vb*{k}}} \\
        &= e \hbar \int \frac{\dd[d]{\vb*{k}}}{(2\pi)^d} 
        \cdot 2 \Im \sum_{n}^{\text{occ}} 
        \mel{\partial_\lambda u_{n \vb*{k}}}{T_{n \vb*{k}} \vb*{v}_{\vb*{k}}}{u_{n \vb*{k}}}.
    \end{aligned}
\end{equation}
Further, using \eqref{eq:v-is-pdv-h-k} and \eqref{eq:q-t-lambda}, we have 
\[
    T_{n \vb*{k}} \vb*{v}_{\vb*{k}} \ket*{u_{n \vb*{k}}}
    = \frac{1}{\hbar} T_{n \vb*{k}} \pdv{H_{\vb*{k}}}{\vb*{k}} \ket*{u_{n \vb*{k}}}
    = \frac{1}{\hbar} Q_{n \vb*{k}} \ket*{\partial_{\vb*{k}} u_{n \vb*{k}}},
\]
and thus 
\begin{equation}
    \begin{aligned}
        \partial_{\lambda} \expval{\vb*{P}}
        &= e \int \frac{\dd[d]{\vb*{k}}}{(2\pi)^d} 
        \cdot 2 \Im \sum_{n}^{\text{occ}} 
        \mel{\partial_\lambda u_{n \vb*{k}}}{Q_{n \vb*{k}}}{\partial_{\vb*{k}} u_{n \vb*{k}}} \\
        &= e \int \frac{\dd[d]{\vb*{k}}}{(2\pi)^d} 
        \cdot 2 \Im \sum_{n}^{\text{occ}} 
        \mel{\partial_\lambda u_{n \vb*{k}}}{Q_{\vb*{k}}}{\partial_{\vb*{k}} u_{n \vb*{k}}}.
    \end{aligned}
    \label{eq:change-of-p-2}
\end{equation}
The derivation from \eqref{eq:change-of-p-1} to \eqref{eq:change-of-p-2}
can be heuristically done by replacing $\vb*{r}$ by $\ii \partial_{\vb*{k}}$,
where we identify $\vb*{k}$ and real momentum $\vb*{p}$;
this however is not always true, 
since the commutation relation between $\vb*{r}$ and 
the operator corresponding to $\vb*{k}$ is not clear.
The last step is to get rid of $Q_{\vb*{k}}$.
To see why, note that 
\[
    \sum_{n}^{\text{occ}} \mel{\partial_\lambda u_{n \vb*{k}}}{Q_{\vb*{k}}}{u_{n \vb*{k}}}
    = \sum_{n}^{\text{occ}} \braket{\partial_\lambda u_{n \vb*{k}}}{u_{n \vb*{k}}}
    - \sum_{n}^{\text{occ}} \sum_{m}^{\text{occ}}
    \braket{\partial_{\lambda} u_{n \vb*{k}}}{u_{m \vb*{k}}}
    \braket{u_{m \vb*{k}}}{\partial_{\vb*{k}} u_{n \vb*{k}}},
\]
and since we have \eqref{eq:switch-partial}, the complex conjugate of the second term is 
\[
    \sum_{n}^{\text{occ}} \sum_{m}^{\text{occ}}
    \braket{u_{m \vb*{k}}}{\partial_{\lambda} u_{n \vb*{k}}}
    \braket{\partial_{\vb*{k}} u_{n \vb*{k}}}{u_{m \vb*{k}}}
    = \sum_{n}^{\text{occ}} \sum_{m}^{\text{occ}}
    (- \braket{\partial_{\lambda} u_{n \vb*{k}}}{u_{m \vb*{k}}})
    (- \braket{u_{m \vb*{k}}}{\partial_{\vb*{k}} u_{n \vb*{k}}})
\]
and is the same as the original form, 
and thus we find that the second term is always real, 
and thus 
\begin{equation}
    \partial_{\lambda} \expval{\vb*{P}} = e \int \frac{\dd[d]{\vb*{k}}}{(2\pi)^d} 
    \cdot 2 \Im \sum_{n}^{\text{occ}} 
    \braket{\partial_\lambda u_{n \vb*{k}}}{\partial_{\vb*{k}} u_{n \vb*{k}}}.
    \label{eq:change-of-p-3}
\end{equation}
Thus, the change of the polarization vector 
is rewritten into a rather elegant form.

\subsection{Berry phase and polarization}

The next problem is what's the RHS of \eqref{eq:change-of-p-3}.
We will find it's Berry curvature. 
Here again we first review the general formalism of Berry phase in quantum mechanics.
Recall that we have \eqref{eq:sternheimer} and \eqref{eq:berry-phase}, 
and from the equations we find in adiabatic evolution, 
we have 
\begin{equation}
    \ket*{n(\lambda)} = \ee^{\text{some other dynamic factor}} \ee^{- \ii \int_{0}^{\lambda} A_n \dd{\lambda} } \ket*{n(\lambda = 0)}.
\end{equation}
The notation $A_n \dd{\lambda}$ actually is 
$A_{n, \mu} \dd{\lambda^\mu}$, 
since there may be more than one turning parameter in the Hamiltonian.
So $A_n$ is just a \emph{connection} in differential geometry,
where the underlying manifold is the range of $\lambda$,
and at each point in the manifold we have a vector space,
the index of which is $n$,
and the parallel transport is adiabatic evolution.
Thus, the status of $A_n$ is comparable to 
the vector potential in electromagnetism,
or Christoffel symbols in general relativity.
Then we may ask whether the Berry connection is ``inherently curved'',
i.e. whether if we slowly change $\lambda$ 
and ultimately turn it back, 
we get a non-zero phase factor before $\ket*{n}$.
In electromagnetism this means we need to calculate $\vb*{B} = \curl{\vb*{A}}$;
in general relativity this means we need to calculate the Riemann curvature tensor. 
In the general formalism of quantum mechanics 
this means we need to calculate \emph{Berry curvature};
note that in an experimental setting 
where particles are confined in a very small range compared with 
the length scale of the variance of the magnetic field, 
we can change $\lambda$ to $\vb*{r}$,
and the Berry phase is just the phase factor induced by the vector potential.
The Berry curvature is then 
\begin{equation}
    \Omega_{n, \mu \nu} = \partial_{\mu} A_{n, \nu} - \partial_{\nu} A_{n, \mu}.
\end{equation}
Here $\mu, \nu$, etc. are indices of $\lambda$.
Since 
\begin{equation}
    A_{n, \mu} = \ii \braket{n}{\partial_\mu n}, 
\end{equation}
we have 
\begin{equation}
    \begin{aligned}
        \Omega_{n, \mu \nu} &= \partial_{\mu} A_{n, \nu} - \partial_{\nu} A_{n, \mu} \\
        &= \ii \braket{\partial_{\mu} n}{\partial_{\nu} n}
        - \ii \braket{\partial_{\nu} n}{\partial_{\mu} n} \\
        &= - 2 \Im \braket{\partial_{\mu} n}{\partial_{\nu} n}.
    \end{aligned}
\end{equation}
Suppose $C$ is a closed routine in the space of possible $\lambda$, 
and the Berry phase accumulated 
by moving $\lambda$ around this routine is equal to 
\begin{equation}
    \ee^{- \ii \oint_C A_{n, \mu} \dd{\lambda^\mu}}
    = \ee^{- \ii \int_S \Omega_{n, \mu \nu} \dd{x^\mu} \dd{x^\nu}},
\end{equation}
where $S$ is a surface in the parameter space whose boundary is $C$.
$A_{n, \mu}$ changes when we add an arbitrary phase factor before wave functions; 
but $\Omega_{n, \mu \nu}$ is invariant under such a gauge transformation.
There, of course, is a subtlety about whether $\Omega$ is well-defined within $S$:
it's possible that $\Omega$ has some kind of spikes within $C$,
which may be due to gauge choice in the wave function phase 
and is reflected by the $2\pi n$ degree of freedom on the LHS.

Now we find the RHS of \eqref{eq:change-of-p-3} is just the Berry curvature:
we then have 
\begin{equation}
    \partial_\lambda \expval{\vb*{P}}
    = - e \int \frac{\dd[d]{\vb*{k}}}{(2\pi)^d} \sum_{n}^{\text{occ}} \Omega_{n, \lambda \vb*{k}}.
\end{equation}

\section{Wannier function and charge center}



\end{document}