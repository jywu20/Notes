\documentclass[hyperref, a4paper]{article}

\usepackage{geometry}
\usepackage{titling}
\usepackage{titlesec}
% No longer needed, since we will use enumitem package
% \usepackage{paralist}
\usepackage{enumitem}
\usepackage{footnote}
\usepackage{marginnote}
\usepackage{enumerate}
\usepackage{amsmath, amssymb, amsthm}
\usepackage{mathtools}
\usepackage{bbm}
\usepackage{cite}
\usepackage{graphicx}
\usepackage{subfigure}
\usepackage{physics}
\usepackage{tensor}
\usepackage{siunitx}
\usepackage[version=4]{mhchem}
\usepackage{tikz}
\usepackage{xcolor}
\usepackage{listings}
\usepackage{autobreak}
\usepackage[ruled, vlined, linesnumbered]{algorithm2e}
\usepackage{nameref,zref-xr}
\zxrsetup{toltxlabel}
\zexternaldocument*[solid-]{../solid/solid}[solid.pdf]
\zexternaldocument*[optics-]{../optics/optics}[optics.pdf]
\zexternaldocument*[rpa-]{./electron-gas}[electron-gas.pdf]
\zexternaldocument*[1027-]{../advanced-electrodynamics/lecture-10-27}[lecture-10-27.pdf]
\usepackage[colorlinks,unicode]{hyperref} % , linkcolor=black, anchorcolor=black, citecolor=black, urlcolor=black, filecolor=black
\usepackage[most]{tcolorbox}
\usepackage{prettyref}

% Page style
\geometry{left=3.18cm,right=3.18cm,top=2.54cm,bottom=2.54cm}
\titlespacing{\paragraph}{0pt}{1pt}{10pt}[20pt]
\setlength{\droptitle}{-5em}
\preauthor{\vspace{-10pt}\begin{center}}
\postauthor{\par\end{center}}

% More compact lists 
\setlist[itemize]{
    itemindent=17pt, 
    leftmargin=1pt,
    listparindent=\parindent,
    parsep=0pt,
}

% Math operators
\DeclareMathOperator{\timeorder}{\mathcal{T}}
\DeclareMathOperator{\diag}{diag}
\DeclareMathOperator{\legpoly}{P}
\DeclareMathOperator{\primevalue}{P}
\DeclareMathOperator{\sgn}{sgn}
\newcommand*{\ii}{\mathrm{i}}
\newcommand*{\ee}{\mathrm{e}}
\newcommand*{\const}{\mathrm{const}}
\newcommand*{\suchthat}{\quad \text{s.t.} \quad}
\newcommand*{\argmin}{\arg\min}
\newcommand*{\argmax}{\arg\max}
\newcommand*{\normalorder}[1]{: #1 :}
\newcommand*{\pair}[1]{\langle #1 \rangle}
\newcommand*{\fd}[1]{\mathcal{D} #1}
\DeclareMathOperator{\bigO}{\mathcal{O}}

% TikZ setting
\usetikzlibrary{arrows,shapes,positioning}
\usetikzlibrary{arrows.meta}
\usetikzlibrary{decorations.markings}
\tikzstyle arrowstyle=[scale=1]
\tikzstyle directed=[postaction={decorate,decoration={markings,
    mark=at position .5 with {\arrow[arrowstyle]{stealth}}}}]
\tikzstyle ray=[directed, thick]
\tikzstyle dot=[anchor=base,fill,circle,inner sep=1pt]

% Algorithm setting
% Julia-style code
\SetKwIF{If}{ElseIf}{Else}{if}{}{elseif}{else}{end}
\SetKwFor{For}{for}{}{end}
\SetKwFor{While}{while}{}{end}
\SetKwProg{Function}{function}{}{end}
\SetArgSty{textnormal}

\newcommand*{\concept}[1]{{\textbf{#1}}}

% Embedded codes
\lstset{basicstyle=\ttfamily,
  showstringspaces=false,
  commentstyle=\color{gray},
  keywordstyle=\color{blue}
}

% Reference formatting
\newrefformat{fig}{Figure~\ref{#1} on page~\pageref{#1}}

% Color boxes
\tcbuselibrary{skins, breakable, theorems}
\newtcbtheorem[number within=section]{warning}{Warning}%
  {colback=orange!5,colframe=orange!65,fonttitle=\bfseries, breakable}{warn}
\newtcbtheorem[number within=section]{note}{Note}%
  {colback=green!5,colframe=green!65,fonttitle=\bfseries, breakable}{note}

\newcommand{\soliddoc}{\href{../solid/solid.pdf}{this solid state physics note}}
\newcommand{\opticsdoc}{\href{../optics/optics.pdf}{this optics note}}
\newcommand{\rpanote}{\href{./electron-gas.pdf}{this note}}
\newcommand{\advancedelenote}{\href{../advanced-electrodynamics/lecture-10-27.pdf}{this note}}

\title{Phonons in Band Metal and Insulators}
\author{Jinyuan Wu}

\begin{document}

\maketitle

The general theory of phonons can be found in \soliddoc. In this note we just discuss some phenomenon induced by 
phonons. We have \concept{longitude acoustic (LA) phonons}, \concept{transverse acoustic (TA) phonons},
\concept{longitude optical (LO) phonons}, and \concept{transverse optical (TO) phonons}.

\section{Self-energy correction of LA phonons and the Bohm-Staver formula}

Phonons are usually derived by considering the displacement of each atom in the lattice as a bosonic field and 
then doing canonical quantization. This approach is based on effective potentials between atoms, which takes the 
effective interaction induced by electrons into account. In this section, we consider the case of alkali metals
and show how can we derive the spectrum of phonons from just Coulomb interaction.

Ions in a alkali metal can be thought as point charges with charge $+\abs{e}$, and they repulse each other. 
If we view the electrons in the metal as a uniform jellium, we can just apply \eqref{rpa-eq:uniform-oscillation} in \rpanote
to \emph{atoms}, and find that 
\begin{equation}
    \ddot{Q}_{\vb*{q}} + \Omega^2_{\vb*{q}} Q_{\vb*{q}} = 0, \quad \Omega^2_{\vb*{q}} = \frac{4 \pi n e^2}{M},
\end{equation}
where $n$ is the density of number of electron and $M$ is the mass of an atom. 
This phonon branch is a longitude mode, since it is related to density fluctuations and therefore we must have 
$\div{\vb*{u}} \sim \vb*{k} \cdot \vb*{u} \neq 0$. It is also acoustic, since in a plasmon mode, 
all particles vibrate synchronously, so we can see a long range density fluctuation.
Note that $\Omega_{\vb*{q}}$ has nothing to do with $\vb*{q}$, and we do not see the regular linear dispersion. 

\begin{note*}{}
    Note that since we have a lattice structure, there are collective modes other than long-range density
    fluctuations, and that is how TA phonons come into existence. For an ideal \emph{fluid}, with just 
    longitude interaction between particles (here the longitude interaction force is the Coulomb force),
    it is impossible to have transverse modes, but in a crystal it is possible. Suppose that the interaction
    potential is 
    \[
        V(r) = \frac{1}{2} k r^2,
    \]
    and therefore the force is 
    \[
        \vb*{F} = k \vb*{r}, \quad \curl{\vb*{F}} = 0.
    \]
    Now the EOM of an atom is 
    \[
        m \ddot{\vb*{u}_{\vb*{i}}} = k (\vb*{u}_{\vb*{i} + \vu*{x}} + \vb*{u}_{\vb*{i} - \vu*{x}} + \vb*{u}_{\vb*{i} + \vu*{y}} + \vb*{u}_{\vb*{i} - \vu*{y}} + \vb*{u}_{\vb*{i} + \vu*{z}} + \vb*{u}_{\vb*{i} - \vu*{z}} - 6 \vb*{u}_{\vb*{i}}),
    \]
    and we find that the spectrum of phonon modes is independent of the amplitude -- and therefore, 
    the polarization -- of the modes. 

    In this section things are a little bit more complicated, since the interaction force is the Coulomb force 
    and not linear to $\vb*{k}$. By invoking the jellium model, what we are doing here is to find the longitude
    (and therefore concerning density fluctuations) collective modes of the atom lattice with Coulomb repulsion 
    in an easier way. It does not mean that it is the \emph{only} way to have atomic collective modes.
\end{note*}

Now we introduce the self-energy correction of phonons. 
\begin{equation}
    \begin{gathered}
        \begin{tikzpicture}[x=0.75pt,y=0.75pt,yscale=-0.7,xscale=0.7]
            %uncomment if require: \path (0,300); %set diagram left start at 0, and has height of 300
            
            %Straight Lines [id:da05028659912954647] 
            \draw    (100,104.5) .. controls (101.67,102.83) and (103.33,102.83) .. (105,104.5) .. controls (106.67,106.17) and (108.33,106.17) .. (110,104.5) .. controls (111.67,102.83) and (113.33,102.83) .. (115,104.5) .. controls (116.67,106.17) and (118.33,106.17) .. (120,104.5) .. controls (121.67,102.83) and (123.33,102.83) .. (125,104.5) .. controls (126.67,106.17) and (128.33,106.17) .. (130,104.5) .. controls (131.67,102.83) and (133.33,102.83) .. (135,104.5) .. controls (136.67,106.17) and (138.33,106.17) .. (140,104.5) .. controls (141.67,102.83) and (143.33,102.83) .. (145,104.5) .. controls (146.67,106.17) and (148.33,106.17) .. (150,104.5) .. controls (151.67,102.83) and (153.33,102.83) .. (155,104.5) .. controls (156.67,106.17) and (158.33,106.17) .. (160,104.5) .. controls (161.67,102.83) and (163.33,102.83) .. (165,104.5) .. controls (166.67,106.17) and (168.33,106.17) .. (170,104.5) .. controls (171.67,102.83) and (173.33,102.83) .. (175,104.5) .. controls (176.67,106.17) and (178.33,106.17) .. (180,104.5) -- (180,104.5)(100,107.5) .. controls (101.67,105.83) and (103.33,105.83) .. (105,107.5) .. controls (106.67,109.17) and (108.33,109.17) .. (110,107.5) .. controls (111.67,105.83) and (113.33,105.83) .. (115,107.5) .. controls (116.67,109.17) and (118.33,109.17) .. (120,107.5) .. controls (121.67,105.83) and (123.33,105.83) .. (125,107.5) .. controls (126.67,109.17) and (128.33,109.17) .. (130,107.5) .. controls (131.67,105.83) and (133.33,105.83) .. (135,107.5) .. controls (136.67,109.17) and (138.33,109.17) .. (140,107.5) .. controls (141.67,105.83) and (143.33,105.83) .. (145,107.5) .. controls (146.67,109.17) and (148.33,109.17) .. (150,107.5) .. controls (151.67,105.83) and (153.33,105.83) .. (155,107.5) .. controls (156.67,109.17) and (158.33,109.17) .. (160,107.5) .. controls (161.67,105.83) and (163.33,105.83) .. (165,107.5) .. controls (166.67,109.17) and (168.33,109.17) .. (170,107.5) .. controls (171.67,105.83) and (173.33,105.83) .. (175,107.5) .. controls (176.67,109.17) and (178.33,109.17) .. (180,107.5) -- (180,107.5) ;
            \end{tikzpicture}            
    \end{gathered} \quad = \quad \begin{gathered}
        \begin{tikzpicture}[x=0.75pt,y=0.75pt,yscale=-0.7,xscale=0.7]
            %uncomment if require: \path (0,300); %set diagram left start at 0, and has height of 300
            
            %Straight Lines [id:da05028659912954647] 
            \draw    (100,106) .. controls (101.67,104.33) and (103.33,104.33) .. (105,106) .. controls (106.67,107.67) and (108.33,107.67) .. (110,106) .. controls (111.67,104.33) and (113.33,104.33) .. (115,106) .. controls (116.67,107.67) and (118.33,107.67) .. (120,106) .. controls (121.67,104.33) and (123.33,104.33) .. (125,106) .. controls (126.67,107.67) and (128.33,107.67) .. (130,106) .. controls (131.67,104.33) and (133.33,104.33) .. (135,106) .. controls (136.67,107.67) and (138.33,107.67) .. (140,106) .. controls (141.67,104.33) and (143.33,104.33) .. (145,106) .. controls (146.67,107.67) and (148.33,107.67) .. (150,106) .. controls (151.67,104.33) and (153.33,104.33) .. (155,106) .. controls (156.67,107.67) and (158.33,107.67) .. (160,106) .. controls (161.67,104.33) and (163.33,104.33) .. (165,106) .. controls (166.67,107.67) and (168.33,107.67) .. (170,106) .. controls (171.67,104.33) and (173.33,104.33) .. (175,106) .. controls (176.67,107.67) and (178.33,107.67) .. (180,106) -- (180,106) ;
            \end{tikzpicture}            
    \end{gathered} \quad + \quad \begin{gathered}
        \begin{tikzpicture}[x=0.75pt,y=0.75pt,yscale=-0.7,xscale=0.7]
            %uncomment if require: \path (0,300); %set diagram left start at 0, and has height of 300
            
            %Straight Lines [id:da05028659912954647] 
            \draw    (67,120) .. controls (68.67,118.33) and (70.33,118.33) .. (72,120) .. controls (73.67,121.67) and (75.33,121.67) .. (77,120) .. controls (78.67,118.33) and (80.33,118.33) .. (82,120) .. controls (83.67,121.67) and (85.33,121.67) .. (87,120) .. controls (88.67,118.33) and (90.33,118.33) .. (92,120) .. controls (93.67,121.67) and (95.33,121.67) .. (97,120) .. controls (98.67,118.33) and (100.33,118.33) .. (102,120) .. controls (103.67,121.67) and (105.33,121.67) .. (107,120) .. controls (108.67,118.33) and (110.33,118.33) .. (112,120) -- (114,120) -- (114,120) ;
            %Shape: Circle [id:dp25295678135186517] 
            \draw  [fill={rgb, 255:red, 155; green, 155; blue, 155 }  ,fill opacity=0.5 ] (114,120) .. controls (114,106.19) and (125.19,95) .. (139,95) .. controls (152.81,95) and (164,106.19) .. (164,120) .. controls (164,133.81) and (152.81,145) .. (139,145) .. controls (125.19,145) and (114,133.81) .. (114,120) -- cycle ;
            %Straight Lines [id:da27340363918966326] 
            \draw    (164,118.5) .. controls (165.67,116.83) and (167.33,116.83) .. (169,118.5) .. controls (170.67,120.17) and (172.33,120.17) .. (174,118.5) .. controls (175.67,116.83) and (177.33,116.83) .. (179,118.5) .. controls (180.67,120.17) and (182.33,120.17) .. (184,118.5) .. controls (185.67,116.83) and (187.33,116.83) .. (189,118.5) .. controls (190.67,120.17) and (192.33,120.17) .. (194,118.5) .. controls (195.67,116.83) and (197.33,116.83) .. (199,118.5) .. controls (200.67,120.17) and (202.33,120.17) .. (204,118.5) .. controls (205.67,116.83) and (207.33,116.83) .. (209,118.5) -- (211,118.5) -- (211,118.5)(164,121.5) .. controls (165.67,119.83) and (167.33,119.83) .. (169,121.5) .. controls (170.67,123.17) and (172.33,123.17) .. (174,121.5) .. controls (175.67,119.83) and (177.33,119.83) .. (179,121.5) .. controls (180.67,123.17) and (182.33,123.17) .. (184,121.5) .. controls (185.67,119.83) and (187.33,119.83) .. (189,121.5) .. controls (190.67,123.17) and (192.33,123.17) .. (194,121.5) .. controls (195.67,119.83) and (197.33,119.83) .. (199,121.5) .. controls (200.67,123.17) and (202.33,123.17) .. (204,121.5) .. controls (205.67,119.83) and (207.33,119.83) .. (209,121.5) -- (211,121.5) -- (211,121.5) ;
            \end{tikzpicture}
    \end{gathered} ,
\end{equation}
where under the RPA, we have 
\begin{equation}
    \begin{aligned}
        \begin{gathered}
            \begin{tikzpicture}[x=0.75pt,y=0.75pt,yscale=-0.7,xscale=0.7]
                %uncomment if require: \path (0,300); %set diagram left start at 0, and has height of 300
                
                %Shape: Circle [id:dp25295678135186517] 
                \draw  [fill={rgb, 255:red, 155; green, 155; blue, 155 }  ,fill opacity=0.5 ] (114,120) .. controls (114,106.19) and (125.19,95) .. (139,95) .. controls (152.81,95) and (164,106.19) .. (164,120) .. controls (164,133.81) and (152.81,145) .. (139,145) .. controls (125.19,145) and (114,133.81) .. (114,120) -- cycle ;
                \end{tikzpicture}            
        \end{gathered} \quad &= \quad \begin{gathered}
            \begin{tikzpicture}[x=0.75pt,y=0.75pt,yscale=-0.7,xscale=0.7]
                %uncomment if require: \path (0,300); %set diagram left start at 0, and has height of 300
                
                %Shape: Circle [id:dp25295678135186517] 
                \draw  [fill={rgb, 255:red, 255; green, 255; blue, 255 }  ,fill opacity=1 ] (114,120) .. controls (114,106.19) and (125.19,95) .. (139,95) .. controls (152.81,95) and (164,106.19) .. (164,120) .. controls (164,133.81) and (152.81,145) .. (139,145) .. controls (125.19,145) and (114,133.81) .. (114,120) -- cycle ;
                %Straight Lines [id:da3302260003559434] 
                \draw    (146,96) ;
                \draw [shift={(146,96)}, rotate = 180] [fill={rgb, 255:red, 0; green, 0; blue, 0 }  ][line width=0.08]  [draw opacity=0] (12,-3) -- (0,0) -- (12,3) -- cycle    ;
                %Straight Lines [id:da744246828754719] 
                \draw    (140,145) -- (134.71,145) ;
                \draw [shift={(132.71,145)}, rotate = 360] [fill={rgb, 255:red, 0; green, 0; blue, 0 }  ][line width=0.08]  [draw opacity=0] (12,-3) -- (0,0) -- (12,3) -- cycle    ;
                \end{tikzpicture}        
        \end{gathered} \quad + \quad \begin{gathered}
            \begin{tikzpicture}[x=0.75pt,y=0.75pt,yscale=-0.7,xscale=0.7]
                %uncomment if require: \path (0,300); %set diagram left start at 0, and has height of 300
                
                %Shape: Circle [id:dp25295678135186517] 
                \draw  [fill={rgb, 255:red, 255; green, 255; blue, 255 }  ,fill opacity=1 ] (114,120) .. controls (114,106.19) and (125.19,95) .. (139,95) .. controls (152.81,95) and (164,106.19) .. (164,120) .. controls (164,133.81) and (152.81,145) .. (139,145) .. controls (125.19,145) and (114,133.81) .. (114,120) -- cycle ;
                %Straight Lines [id:da3302260003559434] 
                \draw    (146,96) ;
                \draw [shift={(146,96)}, rotate = 180] [fill={rgb, 255:red, 0; green, 0; blue, 0 }  ][line width=0.08]  [draw opacity=0] (12,-3) -- (0,0) -- (12,3) -- cycle    ;
                %Straight Lines [id:da744246828754719] 
                \draw    (140,145) -- (134.71,145) ;
                \draw [shift={(132.71,145)}, rotate = 360] [fill={rgb, 255:red, 0; green, 0; blue, 0 }  ][line width=0.08]  [draw opacity=0] (12,-3) -- (0,0) -- (12,3) -- cycle    ;
                %Shape: Circle [id:dp04752317906905801] 
                \draw  [fill={rgb, 255:red, 255; green, 255; blue, 255 }  ,fill opacity=1 ] (213,120) .. controls (213,106.19) and (224.19,95) .. (238,95) .. controls (251.81,95) and (263,106.19) .. (263,120) .. controls (263,133.81) and (251.81,145) .. (238,145) .. controls (224.19,145) and (213,133.81) .. (213,120) -- cycle ;
                %Straight Lines [id:da4128638497812116] 
                \draw    (245,96) ;
                \draw [shift={(245,96)}, rotate = 180] [fill={rgb, 255:red, 0; green, 0; blue, 0 }  ][line width=0.08]  [draw opacity=0] (12,-3) -- (0,0) -- (12,3) -- cycle    ;
                %Straight Lines [id:da9001038688985299] 
                \draw    (239,145) -- (233.71,145) ;
                \draw [shift={(231.71,145)}, rotate = 360] [fill={rgb, 255:red, 0; green, 0; blue, 0 }  ][line width=0.08]  [draw opacity=0] (12,-3) -- (0,0) -- (12,3) -- cycle    ;
                %Straight Lines [id:da6548078805333186] 
                \draw  [dash pattern={on 4.5pt off 4.5pt}]  (164,118.5) -- (213,118.5)(164,121.5) -- (213,121.5) ;
                \end{tikzpicture}            
        \end{gathered} \\
        &= \ii M_{\vb*{q}} \ii \Pi^0_q \ii M_{- \vb*{q}} + \ii M_{\vb*{q}} \ii \Pi^0_q (- \ii V^\text{eff}_q) \ii \Pi^0_q \ii M_{- \vb*{q}} \\ 
        &= - \ii \Pi^0_q \abs*{M_{\vb*{q}}}^2 - \ii \abs*{M_{\vb*{q}}}^2 (\Pi^0_q)^2 V^\text{eff}_q,
    \end{aligned}
\end{equation}
where $M_{\vb*{q}}$ is the phonon-electron interaction vertex and $\Pi^0_q$ is defined in \eqref{rpa-eq:normal-interaction-correction} in \rpanote. 
The propagator of the phonon is 
\begin{equation}
    \ii D^0_q = \ii \frac{2 \Omega_{\vb*{q}}}{(q^0)^2 - \Omega_{\vb*{q}}^2 + \ii 0^+},
\end{equation}
and we have 
\[
    \begin{aligned}
        &\quad \ii D_q = \frac{\ii D^0_q}{1 - \ii D^0_q \times \text{normal self-energy}} \\
        &= \frac{\ii 2 \Omega_{\vb*{q}}}{\omega^2 - \Omega^2_{\vb*{q}} + \ii 0^+ - \ii 2 \Omega_{\vb*{q}} \times (- \ii \Pi^0_q \abs*{M_{\vb*{q}}}^2 - \ii \abs*{M_{\vb*{q}}}^2 (\Pi^0_q)^2 V^\text{eff}_q )} \\
        &= \ii \frac{2 \Omega_{\vb*{q}}}{\omega^2 - \Omega^2_{\vb*{q}} - 2 \Omega_{\vb*{q}} \abs*{M_{\vb*{q}}}^2 \Pi^0_q (1 + \Pi^0_q V^\text{eff}_q)}. 
    \end{aligned}
\]
The corrected pole is no longer $\omega = \pm \Omega_{\vb*{q}}$, but 
\[
    \omega^2 = \Omega^2_{\vb*{q}} + 2 \Omega_{\vb*{q}} \abs*{M_{\vb*{q}}}^2 \Pi^0_q (1 + \Pi^0_q V^\text{eff}_q).
\]
Since 
\[
    V_q^\text{eff} = \frac{V_q}{1 - V_q \Pi^0_q},
\]
we have 
\begin{equation}
    \omega^2 = \Omega^2_{\vb*{q}} + 2 \Omega_{\vb*{q}} \abs*{M_{\vb*{q}}}^2 \frac{\Pi^0_q}{1 - \Pi^0_q V_q} .
    \label{eq:renormalized-phonon-freq}
\end{equation}

In this simple case we are able to find $M_{\vb*{q}}$. We have 
\[
    \begin{aligned}
        V_\text{e-i} &= - \sum_\sigma \int \dd[3]{\vb*{r}} \sum_{\vb*{i}} \frac{e^2 \psi^\dagger_\sigma(\vb*{r}) \psi_\sigma(\vb*{r})}{\abs*{\vb*{r} - \vb*{R}_{\vb*{i}}^0 - \vb*{X}_{\vb*{i}}}} \\
        &\approx - \sum_\sigma \int \dd[3]{\vb*{r}} \sum_{\vb*{i}} \vb*{X}_{\vb*{i}} \cdot \grad_{\vb*{X}_{\vb*{i}}} \frac{e^2 \psi^\dagger_\sigma(\vb*{r}) \psi_\sigma(\vb*{r})}{\abs*{\vb*{r} - \vb*{R}_{\vb*{i}}^0 - \vb*{X}_{\vb*{i}}}} \\
        &= \sum_\sigma \int \dd[3]{\vb*{r}} \sum_{\vb*{i}} \vb*{X}_{\vb*{i}} \cdot \grad_{\vb*{r}} \frac{e^2 \psi^\dagger_\sigma(\vb*{r}) \psi_\sigma(\vb*{r})}{\abs*{\vb*{r} - \vb*{R}_{\vb*{i}}^0 }} ,
    \end{aligned}
\]
and by Fourier transformation, we have 
\[
    \begin{aligned}
        V_\text{e-i} &= \sum_\sigma \int \dd[3]{\vb*{r}} \sum_{\vb*{i}} \sum_{\vb*{k}, \vb*{k}'} \frac{1}{\sqrt{V}} \ee^{- \ii \vb*{k} \cdot \vb*{r}} c^\dagger_{\vb*{k} \sigma} \frac{1}{\sqrt{V}} \ee^{\ii \vb*{k}' \cdot \vb*{r}} c_{\vb*{k}' \sigma} \vb*{X}_{\vb*{i}} \cdot \grad_{\vb*{r}} \frac{ e^2}{\abs*{\vb*{r} - \vb*{R}_{\vb*{i}}^0 }} \\
        &= \frac{1}{V} \sum_{\sigma} \sum_{\vb*{i}} \sum_{\vb*{k}, \vb*{k}'} c^\dagger_{\vb*{k} \sigma} c_{\vb*{k}' \sigma} \int \dd[3]{\vb*{r}} \vb*{X}_{\vb*{i}} \cdot \grad_{\vb*{r}} \frac{e^2}{\abs*{\vb*{r} - \vb*{R}^0_{\vb*{i}} }} \ee^{\ii (\vb*{k}' - \vb*{k}) \cdot \vb*{r}} \\
        &= - \frac{1}{V} \sum_{\sigma} \sum_{\vb*{i}} \sum_{\vb*{k}, \vb*{k}'} c^\dagger_{\vb*{k} \sigma} c_{\vb*{k}' \sigma} \int \dd[3]{\vb*{r}} \vb*{X}_{\vb*{i}} \cdot \frac{e^2}{\abs*{\vb*{r} - \vb*{R}^0_{\vb*{i}} }} \grad_{\vb*{r}} \ee^{\ii (\vb*{k}' - \vb*{k}) \cdot (\vb*{r} - \vb*{R}_{\vb*{i}}^0 )} \ee^{\ii (\vb*{k}' - \vb*{k}) \cdot \vb*{R}^0_{\vb*{i}}} \\
        &=  - \frac{1}{V} \sum_{\sigma} \sum_{\vb*{i}} \sum_{\vb*{k}, \vb*{k}'} c^\dagger_{\vb*{k} \sigma} c_{\vb*{k}' \sigma} \vb*{X}_{\vb*{i}} \cdot \ii (\vb*{k}' - \vb*{k}) \ee^{\ii (\vb*{k}' - \vb*{k}) \cdot \vb*{R}^0_{\vb*{i}}} \int \dd[3]{\vb*{r}} \frac{e^2}{\abs*{\vb*{r} - \vb*{R}^0_{\vb*{i}} }}  \ee^{\ii (\vb*{k}' - \vb*{k}) \cdot (\vb*{r} - \vb*{R}_{\vb*{i}}^0 )} \\
        &= \frac{1}{V} \sum_{\sigma} \sum_{\vb*{i}} \sum_{\vb*{k}, \vb*{k}'} c^\dagger_{\vb*{k} \sigma} c_{\vb*{k}' \sigma} \vb*{X}_{\vb*{i}} \cdot \ii (\vb*{k} - \vb*{k}') \ee^{\ii (\vb*{k}' - \vb*{k}) \cdot \vb*{R}^0_{\vb*{i}}} \frac{4 \pi e^2}{\abs*{\vb*{k} - \vb*{k}'}}.
    \end{aligned}
\]
Now we use the expansion \eqref{solid-eq:phonon-displacement-general} in \soliddoc, and we have 
\[
    \begin{aligned}
        V_\text{e-i} &= \frac{1}{V} \sum_{\sigma} \sum_{\vb*{i}} \sum_{\vb*{k}, \vb*{k}'} \sum_{\vb*{q}, \lambda} c^\dagger_{\vb*{k} \sigma} c_{\vb*{k}' \sigma} \frac{1}{\sqrt{2 M N \Omega_{\vb*{q} \lambda}}} (\vb*{\lambda} b_{\vb*{q} \lambda} \ee^{\ii \vb*{q} \cdot \vb*{R}^0_{\vb*{i}}} + \text{h.c.}) \cdot \ii (\vb*{k} - \vb*{k}') \ee^{\ii (\vb*{k}' - \vb*{k}) \cdot \vb*{R}^0_{\vb*{i}}} \frac{4 \pi e^2}{\abs*{\vb*{k} - \vb*{k}'}} \\
        &= \frac{N}{V} \sum_{\sigma} \sum_{\vb*{k}, \vb*{k}'} \sum_{\vb*{q}, \lambda} c^\dagger_{\vb*{k} \sigma} c_{\vb*{k}' \sigma} \frac{1}{\sqrt{2 M N \Omega_{\vb*{q} \lambda}}} \ii (\vb*{k} - \vb*{k}') \cdot \vb*{\lambda} (  b_{\vb*{q} \lambda} \delta_{\vb*{q} + \vb*{k}', \vb*{k}} + b^\dagger_{\vb*{q} \lambda} \delta_{- \vb*{q} + \vb*{k}' , \vb*{k})}) \frac{4 \pi e^2}{\abs*{\vb*{k} - \vb*{k}'}} \\
        &= \frac{1}{V} \sum_{\sigma} \sum_{\vb*{k}, \vb*{k}'} \sum_{\vb*{q}, \lambda} c^\dagger_{\vb*{k} \sigma} c_{\vb*{k}' \sigma} \sqrt{\frac{N}{2 M \Omega_{\vb*{q} \lambda}}} ( \ii \vb*{q} \cdot \vb*{\lambda} b_{\vb*{q} \lambda} \delta_{\vb*{q} + \vb*{k}', \vb*{k}} - \ii \vb*{q} \cdot \vb*{\lambda} b^\dagger_{\vb*{q} \lambda} \delta_{- \vb*{q} + \vb*{k}' , \vb*{k})}) \frac{4 \pi e^2}{\vb*{q}^2}.
    \end{aligned}
\]
Note that this interaction channel only involves longitude phonons, which is expected, since it comes from Coulomb
interaction. From symmetry analysis, we have 
\[
    V_\text{e-i} = \frac{1}{V} \sum_{\vb*{k}, \vb*{q}, \sigma, \lambda} M_{\vb*{q}} a^\dagger_{\vb*{k} \sigma} a_{\vb*{k} - \vb*{q}, \sigma} b_{\vb*{q} \lambda}  + \text{h.c.},
\]
and therefore we have 
\begin{equation}
    M_{\vb*{q}} = \ii \vb*{q} \cdot \vb*{\lambda} \sqrt{\frac{N}{2 M \Omega_{\vb*{q} \lambda}}} \frac{4 \pi e^2}{\vb*{q}^2} = \frac{\ii 4 \pi e^2}{\abs*{\vb*{q}} } \sqrt{\frac{N}{2 M \Omega_{\vb*{q} \lambda}}}.
\end{equation}

Now we can evaluate \eqref{eq:renormalized-phonon-freq} explicitly. The renormalized pole of the phonon propagator is 
\[
    \begin{aligned}
        \omega^2 &= \Omega^2_{\vb*{q}} + 2 \Omega_{\vb*{q}} \abs*{M_{\vb*{q}}}^2 \frac{\Pi^0_q}{1 - \Pi^0_q V_q} \\
        &= \Omega^2_{\vb*{q}} + 2 \Omega_{\vb*{q}} \frac{4 \pi e^2 N}{2M \Omega_{\vb*{q} \lambda}} \frac{4 \pi e^2}{\vb*{q}^2} \frac{\Pi^0_q}{1 - \Pi^0_q V_q}  \\
        &= \Omega^2_{\vb*{q}} + \Omega^2_{\vb*{q}} V_q \frac{\Pi^0_q}{1 - \Pi^0_q V_q} = \frac{\Omega^2_{\vb*{q}}}{1 - \Pi^0_q V_q},
    \end{aligned}
\]
and therefore we get the renormalized LA phonon spectrum:
\begin{equation}
    \omega_{\vb*{q}}^2 = \frac{\Omega_{\vb*{q}}^2}{\epsilon(\omega_{\vb*{q}}, \vb*{q})} ,
\end{equation}
or in the language of atom motion, 
\begin{equation}
    \ddot{Q}_{\vb*{q}} + \frac{\Omega_{\vb*{q}}^2}{\epsilon(\vb*{q})} Q_{\vb*{q}} = 0.
    \label{eq:oscillating-eq-renormalized}
\end{equation}
Note that if we only have the $\bar{\psi} \psi b$ vertex,
by analyzing the structure of diagrams,
\eqref{eq:oscillating-eq-renormalized} is good even beyond RPA.

Since usually LA phonon frequencies are much lower than the characteristic frequency of electrons, we can take 
the Thomas-Fermi approximation (see \eqref{1027-eq:thomas-fermi-approx} in \advancedelenote, and here $\epsilon_0 = 1 / 4 \pi$; this essentially means the Fock term is ignored)
\begin{equation}
    \epsilon(\omega=0, \vb*{q}) = 1 + \frac{k_\text{TF}^2}{\vb*{q}^2}, \quad k_\text{TF}^2 = \frac{6 \pi e^2 n}{\epsilon_\text{F}},
\end{equation}
and the dispersion relation of the LA phonon is 
\[
    \omega^2 \approx \frac{\Omega^2_{\vb*{q}}}{1 + \frac{k_\text{TF}^2}{\vb*{q}^2}} = \frac{\Omega^2_{\vb*{q}}}{\vb*{q}^2 + k_\text{TF}^2} \vb*{q}^2,
\] 
and in the long wave length approximation, i.e. $\vb*{q} \to 0$, we have 
\begin{equation}
    \omega_{\vb*{q}}^2 \approx \frac{\Omega^2_{\vb*{q}}}{k_\text{TF}^2} \vb*{q}^2 = \frac{2 \epsilon_\text{F}}{3M} \vb*{q}^2 = \frac{m}{3 M} v_\text{F}^2 \vb*{q}^2, \quad \omega_{\vb*{q}} = \sqrt{\frac{m}{3M}} v_\text{F} \abs*{\vb*{q}}. 
\end{equation}
This is called the \concept{Bohm-Staver formula} of sound speed.

The above derivation can be simplified if we notice the fact that what electrons do is to screen the Coulomb 
interaction between ions in the metal, which is just Coulomb repulsion exactly the same as the mutual 
interaction between electrons, and the screening is therefore the same as \eqref{rpa-eq:external-screened} 
and \marginnote{Zhengzhong Li Sec.~5.3} \eqref{rpa-eq:v-eff-resum} in \rpanote. 
With this insight, since $\Omega_{\vb*{q}}^2$ is just proportion to 
the Coulomb repulsion force, we can directly write down \eqref{eq:oscillating-eq-renormalized}. 

\end{document}