\documentclass[hyperref, a4paper]{article}

\usepackage{geometry}
\usepackage{titling}
\usepackage{titlesec}
% No longer needed, since we will use enumitem package
% \usepackage{paralist}
\usepackage{enumitem}
\usepackage{footnote}
\usepackage{amsmath, amssymb, amsthm}
\usepackage{mathtools}
\usepackage{bbm}
\usepackage{cite}
\usepackage{graphicx}
\usepackage{subfigure}
\usepackage{physics}
\usepackage{tensor}
\usepackage{siunitx}
\usepackage[version=4]{mhchem}
\usepackage{tikz}
\usepackage{xcolor}
\usepackage{listings}
\usepackage{underscore}
\usepackage{autobreak}
\usepackage[ruled, vlined, linesnumbered]{algorithm2e}
\usepackage{nameref,zref-xr}
\zxrsetup{toltxlabel}
\usepackage[colorlinks,unicode]{hyperref} % , linkcolor=black, anchorcolor=black, citecolor=black, urlcolor=black, filecolor=black
\usepackage[most]{tcolorbox}
\usepackage{prettyref}

% Page style
\geometry{left=3.18cm,right=3.18cm,top=2.54cm,bottom=2.54cm}
\titlespacing{\paragraph}{0pt}{1pt}{10pt}[20pt]
\setlength{\droptitle}{-5em}

% More compact lists 
\setlist[itemize]{
    %itemindent=17pt, 
    %leftmargin=1pt,
    listparindent=\parindent,
    parsep=0pt,
}

\setlist[enumerate]{
    %itemindent=17pt, 
    %leftmargin=1pt,
    listparindent=\parindent,
    parsep=0pt,
}

% Math operators
\DeclareMathOperator{\timeorder}{\mathcal{T}}
\DeclareMathOperator{\diag}{diag}
\DeclareMathOperator{\legpoly}{P}
\DeclareMathOperator{\primevalue}{P}
\DeclareMathOperator{\sgn}{sgn}
\DeclareMathOperator{\res}{Res}
\newcommand*{\ii}{\mathrm{i}}
\newcommand*{\ee}{\mathrm{e}}
\newcommand*{\const}{\mathrm{const}}
\newcommand*{\suchthat}{\quad \text{s.t.} \quad}
\newcommand*{\argmin}{\arg\min}
\newcommand*{\argmax}{\arg\max}
\newcommand*{\normalorder}[1]{: #1 :}
\newcommand*{\pair}[1]{\langle #1 \rangle}
\newcommand*{\fd}[1]{\mathcal{D} #1}
\DeclareMathOperator{\bigO}{\mathcal{O}}

% TikZ setting
\usetikzlibrary{arrows,shapes,positioning}
\usetikzlibrary{arrows.meta}
\usetikzlibrary{decorations.markings}
\usetikzlibrary{calc}
\tikzstyle arrowstyle=[scale=1]
\tikzstyle directed=[postaction={decorate,decoration={markings,
    mark=at position .5 with {\arrow[arrowstyle]{stealth}}}}]
\tikzstyle ray=[directed, thick]
\tikzstyle dot=[anchor=base,fill,circle,inner sep=1pt]

% Algorithm setting
% Julia-style code
\SetKwIF{If}{ElseIf}{Else}{if}{}{elseif}{else}{end}
\SetKwFor{For}{for}{}{end}
\SetKwFor{While}{while}{}{end}
\SetKwProg{Function}{function}{}{end}
\SetArgSty{textnormal}

\newcommand*{\concept}[1]{{\textbf{#1}}}

% Embedded codes
\lstset{basicstyle=\ttfamily,
  showstringspaces=false,
  commentstyle=\color{gray},
  keywordstyle=\color{blue}
}

\lstdefinestyle{console}{
    basicstyle=\footnotesize\ttfamily,
    breaklines=true,
    postbreak=\mbox{\textcolor{red}{$\hookrightarrow$}\space}
}

% Reference formatting
\newrefformat{fig}{Figure~\ref{#1}}

% Color boxes
\tcbuselibrary{skins, breakable, theorems}
\newtcbtheorem[number within=section]{warning}{Warning}%
  {colback=orange!5,colframe=orange!65,fonttitle=\bfseries, breakable}{warn}
\newtcbtheorem[number within=section]{note}{Note}%
  {colback=green!5,colframe=green!65,fonttitle=\bfseries, breakable}{note}
\newtcbtheorem[number within=section]{info}{Info}%
  {colback=blue!5,colframe=blue!65,fonttitle=\bfseries, breakable}{info}

% Displaying texts in bookmarkers

\pdfstringdefDisableCommands{%
  \def\\{}%
  \def\ce#1{<#1>}%
}

\pdfstringdefDisableCommands{%
  \def\texttt#1{<#1>}%
  \def\mathbb#1{#1}%
}
\pdfstringdefDisableCommands{\def\eqref#1{(\ref{#1})}}

\makeatletter
\pdfstringdefDisableCommands{\let\HyPsd@CatcodeWarning\@gobble}
\makeatother

\newenvironment{shelldisplay}{\begin{lstlisting}}{\end{lstlisting}}

\newcommand{\shortcode}[1]{\texttt{#1}}

\lstset{style = console}

% Make subsubsection labeled
\setcounter{secnumdepth}{4}
\setcounter{tocdepth}{4}

\title{Homework 1}
\author{Jinyuan Wu}

\begin{document}

\maketitle

\section{}

\paragraph*{Problem} \begin{equation}
    1+e^{y / x}-\frac{y}{x} e^{y / x}+e^{y / x} y^{\prime}=0 , \quad  y(1)=-5.
\end{equation}

\paragraph*{Solution} The equation is equivalent to 
\[
    \ee^{y/x} \dd{y} + 
    \left(
        1 + \ee^{y/x} - \frac{y}{x} \ee^{y/x} 
    \right) \dd{x} = 0,
\]
and we have 
\[
    \pdv{x} \ee^{y/x} = \pdv{y} \left(
        1 + \ee^{y/x} - \frac{y}{x} \ee^{y/x} 
    \right) = - \frac{y}{x^2} \ee^{y/x},
\]
so the equation is exact. 
Suppose the solution is $\phi(x, y) = C$, and we have 
\[
    \phi(x, y) = \int \dd{y} \ee^{y/x} = x \ee^{y/x} + f(x),
\] 
and therefore 
\[
    1 + \ee^{y/x} - \frac{y}{x} \ee^{y/x}  = \pdv{\phi}{x} 
    = \ee^{y/x} + x \cdot \frac{-y}{x^2} \ee^{y/x} + \pdv{f}{x}
    \Rightarrow f(x) = x + \const.  
\]
So the solution is 
\begin{equation}
    \phi(x, y) = x \ee^{y/x} + x = C. 
\end{equation}
When $x = 1$, $y = -5$, 
so we find $C = 1 + \ee^{-5}$,
and the final solution is 
\begin{equation}
    y = x \ln \left(
        \frac{1 + \ee^{-5}}{x} - 1
    \right).
\end{equation}

\section{}

\paragraph*{Problem} \begin{equation}
    y^{\prime}=\frac{1}{2 x} y^2-\frac{1}{x} y-\frac{4}{x}.
\end{equation}

\paragraph*{Solution} We have 
\[
    \dv{y}{x} = \frac{y^2 - 2y - 8}{2 x}, 
\]
and therefore
\[
    \int \frac{\dd{x}}{2x} = \int \frac{\dd{y}}{y^2 - 2y - 8} 
    = \frac{1}{6} \int \dd{y} \left(
        \frac{1}{y-4} - \frac{1}{y+2}
    \right) , 
\]
\[
    \ln x = \frac{1}{3} (\ln (y - 4) - \ln (y + 2)) + C.
\]
Solving out $y$, we get 
\begin{equation}
    y = -2 + \frac{6}{1 - C x^3}.
\end{equation}

\section{}

\paragraph*{Problem} \begin{equation}
    y^{\prime \prime}(x)-6 y^{\prime}(x)+13 y=-e^x , \quad y(0)=-1, \quad y^{\prime}(0)=1.
\end{equation}


\paragraph*{Solution} The solutions of the homogeneous equation is given by
\[
    y = \ee^{\lambda x}, 
\]
\[
    \lambda^2 - 6 \lambda + 13 = 0 \Rightarrow
    \lambda = 3 \pm 2 \ii,
\]
and therefore a linear combination of the solutions corresponding to the two $\lambda$'s gives 
\begin{equation}
    y_1 = \ee^{3x} \cos(2x), \quad 
    y_2 = \ee^{3x} \sin(2x).
\end{equation}
Now we need to find a particular solution. 
Using the ansatz $y = A \ee^{x}$, 
we have 
\[
    A - 6 A + 13 A= - 1 \Rightarrow A = -\frac{1}{8}.
\]
So the general solution is 
\begin{equation}
    y = A \ee^{3x} \cos(2x) + B \ee^{3x} \sin(2x) - \frac{1}{8} \ee^{x}.
\end{equation}
The initial conditions mean
\[
    A - \frac{1}{8} = -1, \quad 
    3 A + 2 B - \frac{1}{8} = 1
    \Rightarrow A = - \frac{7}{8}, \quad B = \frac{15}{8},
\]
so 
\begin{equation}
    y = - \frac{7}{8} \ee^{3x} \cos(2x) + \frac{15}{8} \ee^{3x} \sin(2x) - \frac{1}{8} \ee^{x}.
\end{equation}

\section{}

\paragraph*{Problem} \begin{equation}
    y^{\prime \prime}+2 y^{\prime}-3 y=0 , \quad y(0)=6, \quad y^{\prime}(0)=-2.
\end{equation}

\paragraph*{Solution} Following the same recipe above, we try to solve 
\[
    \lambda^2 + 2 \lambda - 3 = 0,
\]
and we find $\lambda = 1, -3$, so 
\begin{equation}
    y = A \ee^{x} + B \ee^{-3x},
\end{equation}
and 
\[
    A + B = 6, \quad A - 3 B = -2 \Rightarrow
    A = 4, \quad B = 2.
\]
So we get 
\begin{equation}
    y = 4 \ee^{x} + 2 \ee^{-3x}.
\end{equation}

\section{}

\paragraph*{Problem} \begin{equation}
    y^{\prime \prime}-y^{\prime}-6 y=8 e^{2 x}.
\end{equation}

\paragraph*{Solution} Again following the same procedure: 
the general solutions of the homogeneous equation is 
\[
    y = A \ee^{3x} + B \ee^{-2x},
\]
and taking the ansatz $y = A \ee^{2x}$ to find a particular solution, we have 
\[
    A - A - 6 A = 8 \Rightarrow A = - \frac{4}{3},
\]
so the general solution is 
\begin{equation}
    y = A \ee^{3x} + B \ee^{-2x} - \frac{4}{3} \ee^{2x}.
\end{equation}

\section{}

\paragraph*{Problem} \begin{equation}
    x^2 y^{\prime \prime}+x y^{\prime}-4 y=0.
\end{equation}

\paragraph*{Solution} This is a Euler equation. 
We use the ansatz $y = x^a$ to find a particular solution:
we have 
\[
    a(a-1) + a - 4 = 0 \Rightarrow a = \pm 2.
\]
So the general solution is 
\begin{equation}
    y = A x^2 + \frac{B}{x^2}.
\end{equation}

\section{}

\paragraph*{Problem} \begin{equation}
    y^{\prime \prime}+x y^{\prime}+x y=0.
\end{equation}

\paragraph*{Solution} Since the polynomials before $y, y', y''$ are all analytic, 
the general form of a solution is 
\[
    y = \sum_{n=0}^\infty c_n x^n,
    \quad xy = \sum_{n=1}^{\infty} c_{n-1} x^n,
\]
and therefore 
\[
    y' = \sum_{n=1}^{\infty} c_n n x^{n-1} , \quad 
    x y' = \sum_{n=1}^{\infty} c_n n x^n,
\]
and 
\[
    y'' = \sum_{n=2}^{\infty} c_n n (n-1) x^{n-2}
    = \sum_{n=0}^{\infty} c_{n+2} (n+2) (n+1) x^n.
\]
So the equation becomes 
\[
    \begin{aligned}
        0 &= \sum_{n=0}^{\infty} c_{n+2} (n+2) (n+1) x^n 
        + \sum_{n=1}^{\infty} c_n n x^n
        + \sum_{n=1}^{\infty} c_{n-1} x^n \\
        &= c_2 + \sum_{n=1}^\infty x^n \left(
            c_{n+2} (n+2) (n+1) + c_n n + c_{n-1}
        \right)
    \end{aligned},
\]
and we get 
\begin{equation}
    c_2 = 0, \quad (n+2) (n+1) c_{n+2} + n c_n + c_{n-1} = 0.
\end{equation}

To pick up two particular solutions, we can set $c_0 = 1, c_1 = 0$, and vice versa. 
When $c_1 = 0, c_0 = 1$, we have 
\[
    6 c_3 + c_1 + c_0 = 0 \Rightarrow c_3 = - \frac{1}{6},
\]
and 
\[
    12 c_4 + 2 c_2 + c_1 = 0 \Rightarrow c_4 = 0.
\]
So 

\section{}

\paragraph*{Problem} \begin{equation}
    4 x y^{\prime \prime}+2 y^{\prime}+2 y=0.
\end{equation}

\paragraph*{Solution} 

\end{document}