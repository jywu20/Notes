\documentclass[hyperref, a4paper]{article}

\usepackage{geometry}
\usepackage{titling}
\usepackage{titlesec}
% No longer needed, since we will use enumitem package
% \usepackage{paralist}
\usepackage{enumitem}
\usepackage{footnote}
\usepackage{amsmath, amssymb, amsthm}
\usepackage{mathtools}
\usepackage{bbm}
\usepackage{cite}
\usepackage{graphicx}
\usepackage{subfigure}
\usepackage{physics}
\usepackage{tensor}
\usepackage{siunitx}
\usepackage[version=4]{mhchem}
\usepackage{tikz}
\usepackage{xcolor}
\usepackage{listings}
\usepackage{underscore}
\usepackage{autobreak}
\usepackage[ruled, vlined, linesnumbered]{algorithm2e}
\usepackage{nameref,zref-xr}
\zxrsetup{toltxlabel}
\usepackage[colorlinks,unicode]{hyperref} % , linkcolor=black, anchorcolor=black, citecolor=black, urlcolor=black, filecolor=black
\usepackage[most]{tcolorbox}
\usepackage{prettyref}

% Page style
\geometry{left=3.18cm,right=3.18cm,top=2.54cm,bottom=2.54cm}
\titlespacing{\paragraph}{0pt}{1pt}{10pt}[20pt]
\setlength{\droptitle}{-5em}

% More compact lists 
\setlist[itemize]{
    %itemindent=17pt, 
    %leftmargin=1pt,
    listparindent=\parindent,
    parsep=0pt,
}

\setlist[enumerate]{
    %itemindent=17pt, 
    %leftmargin=1pt,
    listparindent=\parindent,
    parsep=0pt,
}

% Math operators
\DeclareMathOperator{\timeorder}{\mathcal{T}}
\DeclareMathOperator{\diag}{diag}
\DeclareMathOperator{\legpoly}{P}
\DeclareMathOperator{\primevalue}{P}
\DeclareMathOperator{\sgn}{sgn}
\DeclareMathOperator{\res}{Res}
\newcommand*{\ii}{\mathrm{i}}
\newcommand*{\ee}{\mathrm{e}}
\newcommand*{\const}{\mathrm{const}}
\newcommand*{\suchthat}{\quad \text{s.t.} \quad}
\newcommand*{\argmin}{\arg\min}
\newcommand*{\argmax}{\arg\max}
\newcommand*{\normalorder}[1]{: #1 :}
\newcommand*{\pair}[1]{\langle #1 \rangle}
\newcommand*{\fd}[1]{\mathcal{D} #1}
\DeclareMathOperator{\bigO}{\mathcal{O}}

% TikZ setting
\usetikzlibrary{arrows,shapes,positioning}
\usetikzlibrary{arrows.meta}
\usetikzlibrary{decorations.markings}
\usetikzlibrary{calc}
\tikzstyle arrowstyle=[scale=1]
\tikzstyle directed=[postaction={decorate,decoration={markings,
    mark=at position .5 with {\arrow[arrowstyle]{stealth}}}}]
\tikzstyle ray=[directed, thick]
\tikzstyle dot=[anchor=base,fill,circle,inner sep=1pt]

% Algorithm setting
% Julia-style code
\SetKwIF{If}{ElseIf}{Else}{if}{}{elseif}{else}{end}
\SetKwFor{For}{for}{}{end}
\SetKwFor{While}{while}{}{end}
\SetKwProg{Function}{function}{}{end}
\SetArgSty{textnormal}

\newcommand*{\concept}[1]{{\textbf{#1}}}

% Embedded codes
\lstset{basicstyle=\ttfamily,
  showstringspaces=false,
  commentstyle=\color{gray},
  keywordstyle=\color{blue}
}

\lstdefinestyle{console}{
    basicstyle=\footnotesize\ttfamily,
    breaklines=true,
    postbreak=\mbox{\textcolor{red}{$\hookrightarrow$}\space}
}

% Reference formatting
\newrefformat{fig}{Figure~\ref{#1}}

% Color boxes
\tcbuselibrary{skins, breakable, theorems}
\newtcbtheorem[number within=section]{warning}{Warning}%
  {colback=orange!5,colframe=orange!65,fonttitle=\bfseries, breakable}{warn}
\newtcbtheorem[number within=section]{note}{Note}%
  {colback=green!5,colframe=green!65,fonttitle=\bfseries, breakable}{note}
\newtcbtheorem[number within=section]{info}{Info}%
  {colback=blue!5,colframe=blue!65,fonttitle=\bfseries, breakable}{info}

% Displaying texts in bookmarkers

\pdfstringdefDisableCommands{%
  \def\\{}%
  \def\ce#1{<#1>}%
}

\pdfstringdefDisableCommands{%
  \def\texttt#1{<#1>}%
  \def\mathbb#1{#1}%
}
\pdfstringdefDisableCommands{\def\eqref#1{(\ref{#1})}}

\makeatletter
\pdfstringdefDisableCommands{\let\HyPsd@CatcodeWarning\@gobble}
\makeatother

\newenvironment{shelldisplay}{\begin{lstlisting}}{\end{lstlisting}}

\newcommand{\shortcode}[1]{\texttt{#1}}

\lstset{style = console}

% Make subsubsection labeled
\setcounter{secnumdepth}{4}
\setcounter{tocdepth}{4}

\title{ODEs}
\author{Jinyuan Wu}

\begin{document}

\maketitle

\section{First order ODEs}

\subsection{Linear ODEs}

An ODE in the form of 
\begin{equation}
    y'(x) + p(x) y(x) = q(x) 
\end{equation}
is considered \concept{linear}.
All linear ODEs can be solved by the following procedure.
First we have 
\begin{equation}
    (y' + py) \ee^{\int p \dd{x}} 
    = q \ee^{\int p \dd{x}},
\end{equation}
and now the LHS is a derivative:
\begin{equation}
    \dv{x} \left(
        y \ee^{\int p \dd{x}}
    \right) = q \ee^{\int p \dd{x}},
\end{equation}
and now we can integrate over $x$ and get 
\begin{equation}
    y \ee^{\int p \dd{x}} = 
    \int q \ee^{\int p \dd{x}} \dd{x},
\end{equation}
\begin{equation}
    y  = \ee^{- \int p \dd{x}}
    \int q \ee^{\int p \dd{x}} \dd{x} .
\end{equation}

\subsection{``Energy-conservation lines'' and exact equations}

Another way to represent the solution of an ODE is 
the form $\phi(x, y) = \const$.
Note that the RHS contains no variables, and we have 
\begin{equation}
    0 = \dv{\phi}{x} = \pdv{\phi}{x} + \pdv{\phi}{y} \dv{y}{x},
    \label{eq:implicit-derivative}
\end{equation}
and thus if 
\begin{equation}
    y' = f(x, y)
\end{equation}
is algebraically equivalent to \eqref{eq:implicit-derivative},
the equation is already solved:
We should find $M, N$ such that 
\begin{equation}
    y' = - \frac{M}{N}, \quad
    M = \pdv{\phi}{x}, \quad
    N = \pdv{\phi}{y},
\end{equation}
and then $\phi(x, y)$ solves the equation.
In this case we say $y' = - M / N$ is \concept{exact}.

To test for exactness, we only have to test whether
\begin{equation}
    \pdv{M}{y} = \pdv{N}{x},
\end{equation}
and if so, the existence of $\phi$ is guaranteed.
(Since we work on a topological trivial space, 
things like cohomology group will not bother us.)
We can now use ``partial integral'' to find $\phi$.

Example: suppose in a calculation we find  
\begin{equation}
    \pdv{\phi}{x} = 2 y^2 + y \ee^{xy}, \quad 
    \pdv{\phi}{y} = 4xy + x \ee^{xy} + 2y.
\end{equation}
After partial integration, we find 
\begin{equation}
    \phi(x, y) = \underbrace{2 xy^2 + \ee^{xy} + h(y)}_{
        \int \pdv{\phi}{x} \dd{x}
    }
    = \underbrace{2 xy^2 + \ee^{xy} + y^2 + g(x)}_{
        \int \pdv{\phi}{y} \dd{y}
    },
\end{equation}
and we have to choose 
\begin{equation}
    h(y) = y^2, \quad g(x) = \const,
\end{equation}
and the solution is 
\begin{equation}
    \phi(x, y) = 2 xy^2 + \ee^{xy} + y^2 + \const.
\end{equation}

Note that even when the decomposition $f = - M / N$
doesn't give an exact equation for us, 
we can still use the method of exact equations:
we can multiply a factor $\mu$ to both $M$ and $N$,
and try to guess the form of $\mu$ so that 
\begin{equation}
    \pdv{(\mu M)}{y} = 
    \pdv{(\mu N)}{x}.
    \label{eq:generalized-exact-condition}
\end{equation}

An example can be found in solving 
\begin{equation}
    y' = - \frac{1}{3x - \ee^{-2y}}.
\end{equation}
We have 
\[
    \pdv{1}{y} = 0, \quad 
    \pdv{(3x- \ee^{-2y})}{x} = 3,
\]
so the equation is not exact if we choose $M = 1$ and 
$N = 3x - \ee^{-2y}$.
However, \eqref{eq:generalized-exact-condition} 
can be fulfilled now: 
it's now 
\[
    \pdv{\mu}{y} = 3 \mu + \left(
        3x - \ee^{-2y}
    \right) \pdv{\mu}{x},
\]
and the most convenient way to solve it 
(we \emph{don't} need to find all solutions of this equation!)
is to let $\mu$ contain $y$ only,
so the tricky term on the RHS disappears,
and thus we choose $\mu = \ee^{3y}$,
and we get 
\[
    \phi(x, y) = \int \mu M \dd{x} 
    = \int \ee^{3y} \dd{x} = x \ee^{3y} + u(y),
\]
\[
    \phi(x, y) = \int \mu N \dd{y}
    = \int (3x \ee^{3y} - \ee^{y} ) \dd{y} 
    = x \ee^{3y} - \ee^{y} + v(x),
\]
so 
\begin{equation}
    \phi(x, y) = x \ee^{3y} - \ee^{y} + \const.
\end{equation}

\end{document}