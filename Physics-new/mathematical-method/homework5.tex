\documentclass[hyperref, a4paper]{article}

\usepackage{geometry}
\usepackage{titling}
\usepackage{titlesec}
% No longer needed, since we will use enumitem package
% \usepackage{paralist}
\usepackage{enumitem}
\usepackage{footnote}
\usepackage{amsmath, esint, amssymb, amsthm}
\usepackage{mathtools}
\usepackage{bbm}
\usepackage{cite}
\usepackage{graphicx}
\usepackage{subcaption}
\usepackage{physics}
\usepackage{tensor}
\usepackage{siunitx}
\usepackage[version=4]{mhchem}
\usepackage{tikz}
\usepackage{xcolor}
\usepackage{listings}
\usepackage{underscore}
\usepackage{autobreak}
\usepackage[ruled, vlined, linesnumbered]{algorithm2e}
\usepackage{nameref,zref-xr}
\zxrsetup{toltxlabel}
\usepackage[colorlinks,unicode]{hyperref} % , linkcolor=black, anchorcolor=black, citecolor=black, urlcolor=black, filecolor=black
\usepackage[most]{tcolorbox}
\usepackage{prettyref}

% Page style
\geometry{left=3.18cm,right=3.18cm,top=2.54cm,bottom=2.54cm}
\titlespacing{\paragraph}{0pt}{1pt}{10pt}[20pt]
\setlength{\droptitle}{-5em}

% More compact lists 
\setlist[itemize]{
    %itemindent=17pt, 
    %leftmargin=1pt,
    listparindent=\parindent,
    parsep=0pt,
}

\setlist[enumerate]{
    %itemindent=17pt, 
    %leftmargin=1pt,
    listparindent=\parindent,
    parsep=0pt,
}

% Math operators
\DeclareMathOperator{\timeorder}{\mathcal{T}}
\DeclareMathOperator{\diag}{diag}
\DeclareMathOperator{\legpoly}{P}
\DeclareMathOperator{\primevalue}{P}
\DeclareMathOperator{\sgn}{sgn}
\DeclareMathOperator{\res}{Res}
\newcommand*{\ii}{\mathrm{i}}
\newcommand*{\ee}{\mathrm{e}}
\newcommand*{\const}{\mathrm{const}}
\newcommand*{\suchthat}{\quad \text{s.t.} \quad}
\newcommand*{\argmin}{\arg\min}
\newcommand*{\argmax}{\arg\max}
\newcommand*{\normalorder}[1]{: #1 :}
\newcommand*{\pair}[1]{\langle #1 \rangle}
\newcommand*{\fd}[1]{\mathcal{D} #1}
\DeclareMathOperator{\bigO}{\mathcal{O}}

% TikZ setting
\usetikzlibrary{arrows,shapes,positioning}
\usetikzlibrary{arrows.meta}
\usetikzlibrary{decorations.markings}
\usetikzlibrary{calc}
\tikzstyle arrowstyle=[scale=1]
\tikzstyle directed=[postaction={decorate,decoration={markings,
    mark=at position .5 with {\arrow[arrowstyle]{stealth}}}}]
\tikzstyle ray=[directed, thick]
\tikzstyle dot=[anchor=base,fill,circle,inner sep=1pt]

% Algorithm setting
% Julia-style code
\SetKwIF{If}{ElseIf}{Else}{if}{}{elseif}{else}{end}
\SetKwFor{For}{for}{}{end}
\SetKwFor{While}{while}{}{end}
\SetKwProg{Function}{function}{}{end}
\SetArgSty{textnormal}

\newcommand*{\concept}[1]{{\textbf{#1}}}

% Embedded codes
\lstset{basicstyle=\ttfamily,
  showstringspaces=false,
  commentstyle=\color{gray},
  keywordstyle=\color{blue}
}

\lstdefinestyle{console}{
    basicstyle=\footnotesize\ttfamily,
    breaklines=true,
    postbreak=\mbox{\textcolor{red}{$\hookrightarrow$}\space}
}

% Reference formatting
\newrefformat{fig}{Figure~\ref{#1}}

% Color boxes
\tcbuselibrary{skins, breakable, theorems}
\newtcbtheorem[number within=section]{warning}{Warning}%
  {colback=orange!5,colframe=orange!65,fonttitle=\bfseries, breakable}{warn}
\newtcbtheorem[number within=section]{note}{Note}%
  {colback=green!5,colframe=green!65,fonttitle=\bfseries, breakable}{note}
\newtcbtheorem[number within=section]{info}{Info}%
  {colback=blue!5,colframe=blue!65,fonttitle=\bfseries, breakable}{info}

% Displaying texts in bookmarkers

\pdfstringdefDisableCommands{%
  \def\\{}%
  \def\ce#1{<#1>}%
}

\pdfstringdefDisableCommands{%
  \def\texttt#1{<#1>}%
  \def\mathbb#1{#1}%
}
\pdfstringdefDisableCommands{\def\eqref#1{(\ref{#1})}}

\makeatletter
\pdfstringdefDisableCommands{\let\HyPsd@CatcodeWarning\@gobble}
\makeatother

\newenvironment{shelldisplay}{\begin{lstlisting}}{\end{lstlisting}}

\newcommand{\shortcode}[1]{\texttt{#1}}

\lstset{style = console}

% Make subsubsection labeled
\setcounter{secnumdepth}{4}
\setcounter{tocdepth}{4}

\newcommand*{\laplace}{\mathcal{L}}
\newcommand*{\fourier}{\mathcal{F}}
\newcommand*{\zerotoinf}{\int_{0}^{\infty}}
\newcommand*{\inftoinf}{\int_{-\infty}^{\infty}}
\newcommand*{\mat}[1]{\vb{#1}}

\title{Homework 5}
\author{Jinyuan Wu}

\begin{document}

\maketitle

\section{}

For the surface 
\begin{equation}
    z - x^2 - y^2 = 0,
\end{equation}
the direction of normal vectors is given by 
\begin{equation}
    \grad(z - x^2 - y^2)
    = (-2x, -2y, 1),
\end{equation} 
and a normal vector at $(-1, 1, 2)$ is $(2, -2, 1)$.
The corresponding normal line equation is therefore 
\begin{equation}
    \frac{x + 1}{2} = \frac{y - 1}{-2} = \frac{z - 2}{1},
\end{equation}
and the tangent plane is given by 
\begin{equation}
    2 (x + 1) - 2 (y - 1) + (z - 2) = 0
\end{equation}
or in other words 
\begin{equation}
    2 x - 2y + z = -2.
\end{equation}

\section{}

The equation governing streamlines is 
\begin{equation}
    \dv{x}{t} = F_x, \quad \dv{y}{t} = F_y, \quad \dv{z}{t} = F_z,
\end{equation}
and therefore 
\begin{equation}
    \dv{x}{t} = \cos y, \quad \dv{y}{t} = \sin x, \quad \dv{z}{t} = 0
    \Rightarrow z = \const.
\end{equation}
The equation about $x, y$ is 
\begin{equation}
    \frac{\dd{x}}{\cos y} = \frac{\dd{y}}{\sin x} \Rightarrow
    \sin x \dd{x} = \cos y \dd{y} \Rightarrow
    - \cos x = \sin y + \const.
\end{equation}
So the streamline equation is 
\begin{equation}
    \cos x + \sin y = C_1, \quad z = C_2.
\end{equation}

\section{}

Since $\mathbf{F}=\cos x \mathbf{i}-y \mathbf{j}+x z \mathbf{k}$ 
and $\mathbf{R}=t \mathbf{i}-t^2 \mathbf{j}+\mathbf{k}$,
and $0 \leq t \leq 3$, we have 
\begin{equation}
    \int_C \vb*{F} \cdot \dd{\vb*{R}}
    = \int_{0}^{3} (\cos x + y \cdot 2 t )  \dd{t}
    = \int_{0}^{3} (\cos t - 2 t^3) \dd{t}
    = \sin 3 - \frac{81}{2}.
\end{equation}

\section{}

$C$ is the circle of radius 4 about $(1, 3)$. 
$\mathbf{F}=2 y \mathbf{i}-x \mathbf{j}$ so 
\begin{equation}
    \pdv{F_y}{x} - \pdv{F_x}{y} = -3,
\end{equation}
and 
\begin{equation}
    \oint_{\partial C} \vb*{F} \cdot \dd{\vb*{R}} = \int_C \left(
        \pdv{F_y}{x} - \pdv{F_x}{y}
    \right) \dd{x} \dd{y}
    = - 3 \cdot \int_C \dd{x} \dd{y} = - 3 \cdot \pi \cdot 4^2 = - 48 \pi.
\end{equation}

\section{}

The equation of the surface is equivalent to 
\begin{equation}
    z = \frac{1}{10} (25 - 4x - 8y).
\end{equation}
The integral is therefore 
\begin{equation}
    \begin{aligned}
        \iint_\Sigma (x + y) \dd{\sigma} &= 
        \iint_{\Sigma'} (x + y) \sqrt{
            1 + \left(\pdv{z}{x}\right)^2 + \left(\pdv{z}{y}\right)^2
        } \dd{x} \dd{y} \\
        &= \iint_{\Sigma'} \frac{3}{5} \sqrt{5} (x + y) \dd{x} \dd{y},
    \end{aligned}
\end{equation}
where $\Sigma'$ is the triangle between $(0, 0)$, $(1, 0)$ and $(1, 1)$.
Then 
\begin{equation}
    \iint_{\Sigma'} (x + y) \dd{x} \dd{y}
    = \int_{0}^{1} \dd{x} \int_{0}^{x} \dd{y} (x + y)
    = \int_{0}^{1} \left(x^2 + \frac{x^2}{2}\right) \dd{x}
    = \frac{1}{2},
\end{equation}
so the final result is 
\begin{equation}
    \iint_\Sigma (x + y) \dd{\sigma} = \frac{3}{10}\sqrt{5}.
\end{equation}

\section{}

$\mathbf{F}=x^3 \mathbf{i}+y^3 \mathbf{j}+z^3 \mathbf{k}$ and $\Sigma$ is the sphere of radius 1 about the origin.
So 
\begin{equation}
    \oiint_\Sigma\vb*{F} \cdot \vb*{n} \dd{\sigma}
    = \iiint \div{\vb*{F}} \dd{V}
    = \iiint 3 (x^2 + y^2 + z^2) \dd{V}
    = 3 \cdot 4 \pi \int_{0}^{1} r^2 \cdot r^2 \dd{r}
    = \frac{12}{5} \pi.
\end{equation}

\section{}

We have 
\begin{equation}
    \log z = \log \abs*{z} \ee^{\ii \theta}
    = \log \abs*{z} + \ii \arg(z),
\end{equation}
where $\arg(z)$ is multi-valued.
When $z = 1 + 5 \ii$, we have 
\begin{equation}
    \log z = \log \sqrt{26} + \ii \arg(z)
    = \log \sqrt{26} + \ii \arccos \frac{1}{\sqrt{26}} + 2 \pi n \ii, \quad 
    n \in \mathbb{Z}.
\end{equation}

\section{}

No singularities are present and we have 
\begin{equation}
    \int_\gamma \ii z^2 \dd{z} = \ii \eval{\frac{z^3}{3} }^{3 + \ii}_{1 + 3 \ii}
    = \ii (44 + 44 \ii) = - 44 + 44 \ii.
\end{equation}

\section{}

Suppose $z = r \ee^{\ii \theta}$. 
Since $\gamma$ is a circle of radius 5 about the origin,
we have 
\begin{equation}
    \oint_\gamma \frac{1}{\bar{z}} \dd{z} 
    = \oint_\gamma \frac{1}{r \ee^{- \ii \theta}} \dd{(r \ee^{\ii \theta})}
    = \oint_\gamma \frac{1}{r \ee^{- \ii \theta}} r \dd{ \ee^{\ii \theta}}
    = \oint_\gamma \frac{1}{r \ee^{- \ii \theta}} r  \ee^{\ii \theta} \cdot \ii \dd{\theta}
    = \ii \int_{0}^{2\pi} \ee^{2 \ii \theta} \dd{\theta}
    = 0.
\end{equation}

\section{}

The function
\begin{equation}
    f(z) = \frac{\cos (z - \ii)}{(z + 2 \ii)^3}
\end{equation}
has a third-order pole at $z = - 2 \ii$,
and therefore the residue is 
\begin{equation}
   \begin{aligned}
    \res_{z = - 2 \ii} f(z) &= \frac{1}{2!} \lim_{z \to - 2 \ii} \dv[2]{z}
    (z + 2 \ii)^3 f(z) \\
    &= \frac{1}{2} \eval{\dv[2]{\cos(z - \ii)}{z}}_{z = - 2 \ii} 
    = - \frac{1}{2} \cos (- 3 \ii) = - \frac{1}{4} (\ee^{3} + \ee^{-3}).
   \end{aligned}
\end{equation}
The integral is 
\begin{equation}
    \oint_{\text{near $- 2 \ii$}} f(z) \dd{z} = 2 \pi \ii \cdot - \frac{1}{4} (\ee^{3} + \ee^{-3})
    = - \frac{\ii}{2}  (\ee^{3} + \ee^{-3}).
\end{equation}

\section{}

We apply the ratio test to $\sum_{n=0}^{\infty}\left(\frac{2 i}{5+i}\right)^n(z+3-4 i)^n$:
\begin{equation}
    \abs{\left(\frac{2 i}{5+i}\right)(z+3-4 i)} < 1 \Rightarrow
    \abs{z + 3 - 4 \ii} < \frac{\abs{5 + \ii}}{\abs{2 \ii}} = \frac{\sqrt{26}}{2}.
\end{equation}
The convergence radius is $\sqrt{26} / 2$; 
the open disk of convergence is a circle with radius $\sqrt{26} / 2$ 
about $- 3 + 4 \ii$.

\section{}

$\gamma$ is the square of side length 3 and sides parallel to the axes, centered at $-2 \mathrm{i}$,
and therefore it doesn't contain the $2 \ii$ pole.
Therefore 
\begin{equation}
    \oint_\gamma \frac{\cos z}{4+z^2} \dd{z}
    = 2 \pi \ii \res_{z = - 2 \ii} \frac{\cos z}{4+z^2} 
    = 2 \pi \ii \lim_{z \to - 2 \ii} (z + 2 \ii) \frac{\cos z}{4+z^2} 
    = 2 \pi \ii \frac{\cos(-2 \ii)}{- 4 \ii}
    = - \frac{\pi}{4} (\ee^{2} + \ee^{-2}).
\end{equation}

\section{}

Only the pole $2$ is in $\gamma$ the circle of radius 2 about 2.
Thus 
\begin{equation}
    \oint_\gamma \frac{(1-z)^2}{z^3-8} \dd z = 
    2 \pi \ii \lim_{z \to 2} \frac{(1-z)^2}{z^3-8} (z - 2)
    = \frac{\pi \ii}{6}.
\end{equation}

\end{document}