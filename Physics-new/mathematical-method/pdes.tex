\documentclass[hyperref, a4paper]{article}

\usepackage{geometry}
\usepackage{titling}
\usepackage{titlesec}
% No longer needed, since we will use enumitem package
% \usepackage{paralist}
\usepackage{enumitem}
\usepackage{footnote}
\usepackage{amsmath, amssymb, amsthm}
\usepackage{mathtools}
\usepackage{bbm}
\usepackage{cite}
\usepackage{graphicx}
\usepackage{subfigure}
\usepackage{physics}
\usepackage{tensor}
\usepackage{siunitx}
\usepackage[version=4]{mhchem}
\usepackage{tikz}
\usepackage{xcolor}
\usepackage{listings}
\usepackage{underscore}
\usepackage{autobreak}
\usepackage[ruled, vlined, linesnumbered]{algorithm2e}
\usepackage{nameref,zref-xr}
\zxrsetup{toltxlabel}
\usepackage[colorlinks,unicode]{hyperref} % , linkcolor=black, anchorcolor=black, citecolor=black, urlcolor=black, filecolor=black
\usepackage[most]{tcolorbox}
\usepackage{prettyref}

% Page style
\geometry{left=3.18cm,right=3.18cm,top=2.54cm,bottom=2.54cm}
\titlespacing{\paragraph}{0pt}{1pt}{10pt}[20pt]
\setlength{\droptitle}{-5em}

% More compact lists 
\setlist[itemize]{
    %itemindent=17pt, 
    %leftmargin=1pt,
    listparindent=\parindent,
    parsep=0pt,
}

\setlist[enumerate]{
    %itemindent=17pt, 
    %leftmargin=1pt,
    listparindent=\parindent,
    parsep=0pt,
}

% Math operators
\DeclareMathOperator{\timeorder}{\mathcal{T}}
\DeclareMathOperator{\diag}{diag}
\DeclareMathOperator{\legpoly}{P}
\DeclareMathOperator{\primevalue}{P}
\DeclareMathOperator{\sgn}{sgn}
\DeclareMathOperator{\res}{Res}
\newcommand*{\ii}{\mathrm{i}}
\newcommand*{\ee}{\mathrm{e}}
\newcommand*{\const}{\mathrm{const}}
\newcommand*{\suchthat}{\quad \text{s.t.} \quad}
\newcommand*{\argmin}{\arg\min}
\newcommand*{\argmax}{\arg\max}
\newcommand*{\normalorder}[1]{: #1 :}
\newcommand*{\pair}[1]{\langle #1 \rangle}
\newcommand*{\fd}[1]{\mathcal{D} #1}
\DeclareMathOperator{\bigO}{\mathcal{O}}

% TikZ setting
\usetikzlibrary{arrows,shapes,positioning}
\usetikzlibrary{arrows.meta}
\usetikzlibrary{decorations.markings}
\usetikzlibrary{calc}
\tikzstyle arrowstyle=[scale=1]
\tikzstyle directed=[postaction={decorate,decoration={markings,
    mark=at position .5 with {\arrow[arrowstyle]{stealth}}}}]
\tikzstyle ray=[directed, thick]
\tikzstyle dot=[anchor=base,fill,circle,inner sep=1pt]

% Algorithm setting
% Julia-style code
\SetKwIF{If}{ElseIf}{Else}{if}{}{elseif}{else}{end}
\SetKwFor{For}{for}{}{end}
\SetKwFor{While}{while}{}{end}
\SetKwProg{Function}{function}{}{end}
\SetArgSty{textnormal}

\newcommand*{\concept}[1]{{\textbf{#1}}}

% Embedded codes
\lstset{basicstyle=\ttfamily,
  showstringspaces=false,
  commentstyle=\color{gray},
  keywordstyle=\color{blue}
}

\lstdefinestyle{console}{
    basicstyle=\footnotesize\ttfamily,
    breaklines=true,
    postbreak=\mbox{\textcolor{red}{$\hookrightarrow$}\space}
}

% Reference formatting
\newrefformat{fig}{Figure~\ref{#1}}

% Color boxes
\tcbuselibrary{skins, breakable, theorems}
\newtcbtheorem[number within=section]{warning}{Warning}%
  {colback=orange!5,colframe=orange!65,fonttitle=\bfseries, breakable}{warn}
\newtcbtheorem[number within=section]{note}{Note}%
  {colback=green!5,colframe=green!65,fonttitle=\bfseries, breakable}{note}
\newtcbtheorem[number within=section]{info}{Info}%
  {colback=blue!5,colframe=blue!65,fonttitle=\bfseries, breakable}{info}

% Displaying texts in bookmarkers

\pdfstringdefDisableCommands{%
  \def\\{}%
  \def\ce#1{<#1>}%
}

\pdfstringdefDisableCommands{%
  \def\texttt#1{<#1>}%
  \def\mathbb#1{#1}%
}
\pdfstringdefDisableCommands{\def\eqref#1{(\ref{#1})}}

\makeatletter
\pdfstringdefDisableCommands{\let\HyPsd@CatcodeWarning\@gobble}
\makeatother

\newenvironment{shelldisplay}{\begin{lstlisting}}{\end{lstlisting}}

\newcommand{\shortcode}[1]{\texttt{#1}}

\lstset{style = console}

% Make subsubsection labeled
\setcounter{secnumdepth}{4}
\setcounter{tocdepth}{4}

\newcommand*{\laplace}{\mathcal{L}}
\newcommand*{\fourier}{\mathcal{F}}
\newcommand*{\zerotoinf}{\int_{0}^{\infty}}
\newcommand*{\inftoinf}{\int_{-\infty}^{\infty}}

\title{PDEs}
\author{Jinyuan Wu}

\begin{document}

\maketitle

\section{The linear heat equation}

\subsection{Homogeneous boundary condition}

Consider one-dimensional heat equation 
\begin{equation}
    \pdv{u}{t} = K \pdv[2]{u}{x}.
    \label{eq:heat.homogeneous.bulk}
\end{equation}
As an example, we consider the case where 
the boundaries of the string in question are kept to zero temperature, 
and the boundary and initial conditions are 
\begin{equation}
    u(0, t) = u(L, t) = 0, \quad 
    u(x, 0) = f(x).
    \label{eq:heat.homogeneous.boundary}
\end{equation}
By separation of variables
\begin{equation}
    u(x, t) = X(x) T(t), 
\end{equation}
we find 
\[
    \frac{T'}{K T} = \frac{X''}{X},
\]
which can therefore only be a constant, 
because otherwise it's impossible for something that only depends on $t$ 
and something that only depends on $x$
to be equal to each other constantly.
So we have 
\[
    X'' = \lambda X, \quad T' = K \lambda T.
\]
When $\lambda > 0$, we find 
\[
    X(x) = A \ee^{\sqrt{\lambda} x} + B \ee^{- \sqrt{\lambda} x}, 
\] 
and therefore the boundary conditions mean 
\[
    X(0) = 0 \Rightarrow A + B = 0, 
\]
\[
    X(L) = 0 \Rightarrow A (\ee^{\sqrt{\lambda} x} - \ee^{- \sqrt{\lambda} x}) = 0 \Rightarrow A = B = 0,
\]
which gives a trivial solution.
Similarly $\lambda = 0$ gives a trivial solution.
So we find we should only consider $\lambda < 0$.
So now we replace $\lambda$ by $- \lambda$, 
and from 
\[
    X'' = - \lambda X
\]
we find 
\[
    X(x) = A \cos(\sqrt{\lambda} x) + B \sin(\sqrt{\lambda} x).
\]
The boundary condition $X(0) = 0$ means $A = 0$, 
and we should then keep $B$ to be non-zero.
Then $X(L) = 0$ means 
\[
    \sqrt{\lambda} L = n \pi, \quad n = 1, 2, 3, \dots
\]
So now $\lambda$ is completely determined, 
and the next step is to find $T$, 
which is trivial: 
\[
    T(t) \propto \ee^{- \lambda t}.
\]
So the final solution is 
\begin{equation}
    u(x, t) = \sum_{n=1}^{\infty} 
        c_n T_n(t) X_n(x), 
\end{equation}
where 
\begin{equation}
    T_n(t) = \ee^{- \frac{n^2 \pi^2 K t}{L^2}}, 
\end{equation}
and 
\begin{equation}
    X_n(x) = \sin(\frac{n \pi x}{L}).
\end{equation}
The constants $\{c_n\}$ then can be solved from the initial condition: 
\[
    f(x) = \sum_{n=1}^{\infty} c_n \sin (\frac{n \pi x}{L}) ,
\]
which is just a Fourier series, so 
\begin{equation}
    c_n = \frac{2}{L} \int_{0}^{L} f(x) \sin (\frac{n \pi x}{L}) \dd{x}.
\end{equation}

\subsection{Inhomogeneous boundary condition}

We can consider another problem: 
now the boundaries are still isothermal, 
but the temperatures there are no longer zero.
The conditions are 
\begin{equation}
    u(0, t) = T_1, \quad 
    u(L, t) = T_2.
\end{equation}
Note that the temperatures can be different, 
and when $t \to \infty$, the stable solution may still be non-zero. 
Linearity guarantees the validity of the following decomposition:
\begin{equation}
    u(x, t) = u_0(x, t) + \psi(x),
\end{equation}
where $\psi(x)$ satisfies 
\begin{equation}
    \psi''(x) = 0, \quad 
    \psi(0, t) = T_1, \quad 
    \psi(L, t) = T_2,
\end{equation}
so that $u_0(x, t)$ satisfies 
the problem \eqref{eq:heat.homogeneous.bulk} plus \eqref{eq:heat.homogeneous.boundary} 
just solved above -- 
but note that $f$ in \eqref{eq:heat.homogeneous.boundary}
should be replaced by $f(x) - \psi(x)$.
Now $\psi(x)$ can be found easily: 
it's just 
\begin{equation}
    \psi(x) = \frac{T_2 - T_1}{L} x + T_1.
\end{equation}

\subsection{Heat conduction in an infinite medium}

Now we consider 
\begin{equation}
    \pdv{u}{t} = K \pdv[2]{u}{x}, \quad -\infty < x < \infty, \quad u(x, 0) = f(x).
\end{equation}
The problem can be solved by Laplace transform as well as Fourier transform; 
or we can do Fourier transform in $x$ 
and Laplace transform in $t$.
We have (here we are using $\omega$ to refer to the frequency of $x$, not $t$)
\[
    \fourier \left[\pdv{u}{t}\right] = \pdv{t} \hat{u}(\omega, t),
\]
and 
\[
    \fourier\left[\pdv[2]{u}{x}\right] 
    = (\ii \omega)^2 \hat{u}(\omega, t).
\]
The bulk equation now is 
\[
    \pdv{\hat{u}(\omega, t)}{t} = - K \omega^2 \hat{u}(\omega, t) ,
\]
and we have 
\[
    \hat{u}(\omega , t) = \ee^{- K \omega^2 t} \underbrace{\hat{u}(\omega, 0)}_{\hat{f}(\omega)}.
\] 
So we find 
\begin{equation}
    u(x, t) = \frac{1}{2\pi} \inftoinf \hat{u}(\omega, t) \ee^{\ii \omega x} \dd{\omega}
    = \frac{1}{2\pi} \inftoinf \hat{f}(\omega) \ee^{- K \omega^2 t} \ee^{\ii \omega x} \dd{\omega}.
    \label{eq:heat.infinite.general}
\end{equation}

A common initial condition is 
\begin{equation}
    u(x, 0) = \delta(x),
\end{equation}
which means when $t = 0$, 
all heat is concentrated in a rather small region.
Then 
\[
    \hat{f}(\omega) = 1,
\]
and \eqref{eq:heat.infinite.general} tells us 
\begin{equation}
    u(x, t) = \frac{1}{2\pi} \inftoinf \ee^{- K \omega^2 t + \ii \omega x} \dd{\omega} 
    = \frac{1}{2 \sqrt{\pi K t}}  \ee^{- \frac{x^2}{4 K t}} .
\end{equation}
Here the integral can be calculated as 
\[
    \begin{aligned}
        \inftoinf \ee^{- K \omega^2 t + \ii \omega x} \dd{\omega} 
        &= \ee^{- \frac{x^2}{4 K t}} \inftoinf 
            \ee^{- K t \left(
                \omega - \frac{\ii x}{2 K t}
            \right)^2 } \dd{\omega} \\
        &= \sqrt{\frac{\pi}{Kt}} \ee^{- \frac{x^2}{4 K t}} .
    \end{aligned}
\]
The solution is always Gaussian, 
but as time goes by, 
it becomes wider and wider.

\subsection{Heat conduction on a semi-infinite domain}

Let's then consider the following boundary and initial conditions:
\begin{equation}
    u(x, 0) = T, \quad 
    u(0, t) = 0.
\end{equation}
This means we first heat the material 
and establish a homogeneous temperature field inside it, 
and then touch it with a colder point.
Since this is a half-infinite problem, 
we can use Laplace transform on the time $t$.
We have 
\[
    \laplace\left[\pdv{u}{t}\right]
    = s U(x, s) - u(x, 0) = s U(x, s) - T, 
\]
and 
\[
    \laplace\left[\pdv[2]{u}{x}\right] = \pdv[2]{U(x, s)}{x}.
\]
The bulk equation then becomes 
\[
    s U(x, s) - T = K \pdv[2]{U(x, s)}{x}, 
\]
\[
    \pdv[2]{U(x, s)}{x} - \frac{s}{K} U = - \frac{T}{K}.
\]
The homogeneous solution of this equation is just (note that $A$ and $B$ may have $s$ dependence)
\[
    U = A \ee^{\sqrt{s / K} x} + B \ee^{- \sqrt{s / K} x} .
\] 
A specific solution is 
\[
    U = \frac{T}{s}.
\]
$A$ has to be zero, 
because $u(x, t)$ should be finite when $x \to \infty$.
So we find 
\[
    U = \frac{T}{s} + B(s) \ee^{- \sqrt{s / K} x}.
\]
We still need to use the condition $u(0 , t) = 0$, 
which, after Laplace transform, is $U(0, s) = 0$,
and we find 
\[
    U = \frac{T}{s} (1 - \ee^{- \sqrt{s / K} x}).
\]
So 
\begin{equation}
    u(x, t) = \laplace^{-1} \left[
        \frac{T}{s} (1 - \ee^{- \sqrt{s / K} x})
    \right] = T \mathrm{erf} \left(
        \frac{x}{2 \sqrt{Kt}}
    \right).
\end{equation}

\end{document}