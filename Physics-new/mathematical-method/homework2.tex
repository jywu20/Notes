\documentclass[hyperref, a4paper]{article}

\usepackage{geometry}
\usepackage{titling}
\usepackage{titlesec}
% No longer needed, since we will use enumitem package
% \usepackage{paralist}
\usepackage{enumitem}
\usepackage{footnote}
\usepackage{amsmath, amssymb, amsthm}
\usepackage{mathtools}
\usepackage{bbm}
\usepackage{cite}
\usepackage{graphicx}
\usepackage{subfigure}
\usepackage{physics}
\usepackage{tensor}
\usepackage{siunitx}
\usepackage[version=4]{mhchem}
\usepackage{tikz}
\usepackage{xcolor}
\usepackage{listings}
\usepackage{underscore}
\usepackage{autobreak}
\usepackage[ruled, vlined, linesnumbered]{algorithm2e}
\usepackage{nameref,zref-xr}
\zxrsetup{toltxlabel}
\usepackage[colorlinks,unicode]{hyperref} % , linkcolor=black, anchorcolor=black, citecolor=black, urlcolor=black, filecolor=black
\usepackage[most]{tcolorbox}
\usepackage{prettyref}

% Page style
\geometry{left=3.18cm,right=3.18cm,top=2.54cm,bottom=2.54cm}
\titlespacing{\paragraph}{0pt}{1pt}{10pt}[20pt]
\setlength{\droptitle}{-5em}

% More compact lists 
\setlist[itemize]{
    %itemindent=17pt, 
    %leftmargin=1pt,
    listparindent=\parindent,
    parsep=0pt,
}

\setlist[enumerate]{
    %itemindent=17pt, 
    %leftmargin=1pt,
    listparindent=\parindent,
    parsep=0pt,
}

% Math operators
\DeclareMathOperator{\timeorder}{\mathcal{T}}
\DeclareMathOperator{\diag}{diag}
\DeclareMathOperator{\legpoly}{P}
\DeclareMathOperator{\primevalue}{P}
\DeclareMathOperator{\sgn}{sgn}
\DeclareMathOperator{\res}{Res}
\newcommand*{\ii}{\mathrm{i}}
\newcommand*{\ee}{\mathrm{e}}
\newcommand*{\const}{\mathrm{const}}
\newcommand*{\suchthat}{\quad \text{s.t.} \quad}
\newcommand*{\argmin}{\arg\min}
\newcommand*{\argmax}{\arg\max}
\newcommand*{\normalorder}[1]{: #1 :}
\newcommand*{\pair}[1]{\langle #1 \rangle}
\newcommand*{\fd}[1]{\mathcal{D} #1}
\DeclareMathOperator{\bigO}{\mathcal{O}}

% TikZ setting
\usetikzlibrary{arrows,shapes,positioning}
\usetikzlibrary{arrows.meta}
\usetikzlibrary{decorations.markings}
\usetikzlibrary{calc}
\tikzstyle arrowstyle=[scale=1]
\tikzstyle directed=[postaction={decorate,decoration={markings,
    mark=at position .5 with {\arrow[arrowstyle]{stealth}}}}]
\tikzstyle ray=[directed, thick]
\tikzstyle dot=[anchor=base,fill,circle,inner sep=1pt]

% Algorithm setting
% Julia-style code
\SetKwIF{If}{ElseIf}{Else}{if}{}{elseif}{else}{end}
\SetKwFor{For}{for}{}{end}
\SetKwFor{While}{while}{}{end}
\SetKwProg{Function}{function}{}{end}
\SetArgSty{textnormal}

\newcommand*{\concept}[1]{{\textbf{#1}}}

% Embedded codes
\lstset{basicstyle=\ttfamily,
  showstringspaces=false,
  commentstyle=\color{gray},
  keywordstyle=\color{blue}
}

\lstdefinestyle{console}{
    basicstyle=\footnotesize\ttfamily,
    breaklines=true,
    postbreak=\mbox{\textcolor{red}{$\hookrightarrow$}\space}
}

% Reference formatting
\newrefformat{fig}{Figure~\ref{#1}}

% Color boxes
\tcbuselibrary{skins, breakable, theorems}
\newtcbtheorem[number within=section]{warning}{Warning}%
  {colback=orange!5,colframe=orange!65,fonttitle=\bfseries, breakable}{warn}
\newtcbtheorem[number within=section]{note}{Note}%
  {colback=green!5,colframe=green!65,fonttitle=\bfseries, breakable}{note}
\newtcbtheorem[number within=section]{info}{Info}%
  {colback=blue!5,colframe=blue!65,fonttitle=\bfseries, breakable}{info}

% Displaying texts in bookmarkers

\pdfstringdefDisableCommands{%
  \def\\{}%
  \def\ce#1{<#1>}%
}

\pdfstringdefDisableCommands{%
  \def\texttt#1{<#1>}%
  \def\mathbb#1{#1}%
}
\pdfstringdefDisableCommands{\def\eqref#1{(\ref{#1})}}

\makeatletter
\pdfstringdefDisableCommands{\let\HyPsd@CatcodeWarning\@gobble}
\makeatother

\newenvironment{shelldisplay}{\begin{lstlisting}}{\end{lstlisting}}

\newcommand{\shortcode}[1]{\texttt{#1}}

\lstset{style = console}

% Make subsubsection labeled
\setcounter{secnumdepth}{4}
\setcounter{tocdepth}{4}

\title{Homework 2}
\author{Jinyuan Wu}

\begin{document}

\maketitle

\section{}

\paragraph*{Problem} 
\begin{equation}
    y' - 9y = t, \quad y(0) = 5.
\end{equation}

\paragraph*{Solution} After Laplace transformation we get 
\[
    s Y(s) - 5 - 9 Y(s) = \frac{1}{s^2}, 
\]
\[
    Y(s) = \frac{5}{s - 9} + \frac{1}{s^2 (s - 9)}.
\]
The second term can be decomposed 
(by multiplying $x$, $x^2$ or $(x - 9)$ and taking the limit $x \to 0$ and $x \to 9$) as
\[
    \frac{1}{s^2 (s - 9)} = \frac{1}{81} \frac{1}{s - 9} 
    - \frac{1}{81} \frac{1}{s} 
    + \frac{1}{9}  \frac{1}{s^2}, 
\]
and 
\[
    Y(s) = \frac{406}{81} \frac{1}{s - 9} 
    - \frac{1}{81} \frac{1}{s} 
    - \frac{1}{9}  \frac{1}{s^2}. 
\]
The inverse Laplace transform is 
\begin{equation}
    y(t) = \frac{406}{81} \ee^{9t} - \frac{1}{81} + \frac{1}{9} t. 
\end{equation}

\section{}

\paragraph*{Problem} 
\[
    \begin{gathered}
        y^{\prime \prime}-4 y^{\prime}+4 y=f(t) ; y(0)=-2, y^{\prime}(0)=1 \text { with } \\
        f(t)= \begin{cases}t & \text { for } 0 \leq t<3 \\
        t+2 & \text { for } t \geq 3\end{cases}
    \end{gathered}
\]

\paragraph*{Solution} The Laplace transform of the LHS is 
\[
    s^2 Y(s) - s y(0) - y'(0) 
    - 4 (s Y(s) - y(0)) 
    + 4 Y(s) = 
    (s-2)^2 Y(s) + 2s - 9.
\]
The RHS is 
\[
    f(t) = t (H(t) - H(t-3)) + (t + 2) H(t - 3) 
    = t H(t) + 2 H(t - 3) \stackrel{\mathcal{L}}{\longrightarrow}
    \frac{1}{s^2} + 2 \cdot \ee^{-3s} \cdot \frac{1}{s}.
\]
So the equation is equivalent to 
\[
    (s - 2)^2 Y(s) + 2s - 9 = \frac{1}{s^2} + \frac{2 \ee^{- 3s}}{s},
\]
\begin{equation}
    Y(s) = - \frac{2}{s - 2} + \frac{5}{(s - 2)^2}
    + \frac{1}{s^2 (s - 2)^2}
    + \frac{2}{s (s - 2)^2} \ee^{-3 s}.
\end{equation}


\end{document}