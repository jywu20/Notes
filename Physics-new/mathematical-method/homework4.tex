\documentclass[hyperref, a4paper]{article}

\usepackage{geometry}
\usepackage{titling}
\usepackage{titlesec}
% No longer needed, since we will use enumitem package
% \usepackage{paralist}
\usepackage{enumitem}
\usepackage{footnote}
\usepackage{amsmath, amssymb, amsthm}
\usepackage{mathtools}
\usepackage{bbm}
\usepackage{cite}
\usepackage{graphicx}
\usepackage{subcaption}
\usepackage{physics}
\usepackage{tensor}
\usepackage{siunitx}
\usepackage[version=4]{mhchem}
\usepackage{tikz}
\usepackage{xcolor}
\usepackage{listings}
\usepackage{underscore}
\usepackage{autobreak}
\usepackage[ruled, vlined, linesnumbered]{algorithm2e}
\usepackage{nameref,zref-xr}
\zxrsetup{toltxlabel}
\usepackage[colorlinks,unicode]{hyperref} % , linkcolor=black, anchorcolor=black, citecolor=black, urlcolor=black, filecolor=black
\usepackage[most]{tcolorbox}
\usepackage{prettyref}

% Page style
\geometry{left=3.18cm,right=3.18cm,top=2.54cm,bottom=2.54cm}
\titlespacing{\paragraph}{0pt}{1pt}{10pt}[20pt]
\setlength{\droptitle}{-5em}

% More compact lists 
\setlist[itemize]{
    %itemindent=17pt, 
    %leftmargin=1pt,
    listparindent=\parindent,
    parsep=0pt,
}

\setlist[enumerate]{
    %itemindent=17pt, 
    %leftmargin=1pt,
    listparindent=\parindent,
    parsep=0pt,
}

% Math operators
\DeclareMathOperator{\timeorder}{\mathcal{T}}
\DeclareMathOperator{\diag}{diag}
\DeclareMathOperator{\legpoly}{P}
\DeclareMathOperator{\primevalue}{P}
\DeclareMathOperator{\sgn}{sgn}
\DeclareMathOperator{\res}{Res}
\newcommand*{\ii}{\mathrm{i}}
\newcommand*{\ee}{\mathrm{e}}
\newcommand*{\const}{\mathrm{const}}
\newcommand*{\suchthat}{\quad \text{s.t.} \quad}
\newcommand*{\argmin}{\arg\min}
\newcommand*{\argmax}{\arg\max}
\newcommand*{\normalorder}[1]{: #1 :}
\newcommand*{\pair}[1]{\langle #1 \rangle}
\newcommand*{\fd}[1]{\mathcal{D} #1}
\DeclareMathOperator{\bigO}{\mathcal{O}}

% TikZ setting
\usetikzlibrary{arrows,shapes,positioning}
\usetikzlibrary{arrows.meta}
\usetikzlibrary{decorations.markings}
\usetikzlibrary{calc}
\tikzstyle arrowstyle=[scale=1]
\tikzstyle directed=[postaction={decorate,decoration={markings,
    mark=at position .5 with {\arrow[arrowstyle]{stealth}}}}]
\tikzstyle ray=[directed, thick]
\tikzstyle dot=[anchor=base,fill,circle,inner sep=1pt]

% Algorithm setting
% Julia-style code
\SetKwIF{If}{ElseIf}{Else}{if}{}{elseif}{else}{end}
\SetKwFor{For}{for}{}{end}
\SetKwFor{While}{while}{}{end}
\SetKwProg{Function}{function}{}{end}
\SetArgSty{textnormal}

\newcommand*{\concept}[1]{{\textbf{#1}}}

% Embedded codes
\lstset{basicstyle=\ttfamily,
  showstringspaces=false,
  commentstyle=\color{gray},
  keywordstyle=\color{blue}
}

\lstdefinestyle{console}{
    basicstyle=\footnotesize\ttfamily,
    breaklines=true,
    postbreak=\mbox{\textcolor{red}{$\hookrightarrow$}\space}
}

% Reference formatting
\newrefformat{fig}{Figure~\ref{#1}}

% Color boxes
\tcbuselibrary{skins, breakable, theorems}
\newtcbtheorem[number within=section]{warning}{Warning}%
  {colback=orange!5,colframe=orange!65,fonttitle=\bfseries, breakable}{warn}
\newtcbtheorem[number within=section]{note}{Note}%
  {colback=green!5,colframe=green!65,fonttitle=\bfseries, breakable}{note}
\newtcbtheorem[number within=section]{info}{Info}%
  {colback=blue!5,colframe=blue!65,fonttitle=\bfseries, breakable}{info}

% Displaying texts in bookmarkers

\pdfstringdefDisableCommands{%
  \def\\{}%
  \def\ce#1{<#1>}%
}

\pdfstringdefDisableCommands{%
  \def\texttt#1{<#1>}%
  \def\mathbb#1{#1}%
}
\pdfstringdefDisableCommands{\def\eqref#1{(\ref{#1})}}

\makeatletter
\pdfstringdefDisableCommands{\let\HyPsd@CatcodeWarning\@gobble}
\makeatother

\newenvironment{shelldisplay}{\begin{lstlisting}}{\end{lstlisting}}

\newcommand{\shortcode}[1]{\texttt{#1}}

\lstset{style = console}

% Make subsubsection labeled
\setcounter{secnumdepth}{4}
\setcounter{tocdepth}{4}

\newcommand*{\laplace}{\mathcal{L}}
\newcommand*{\fourier}{\mathcal{F}}
\newcommand*{\zerotoinf}{\int_{0}^{\infty}}
\newcommand*{\inftoinf}{\int_{-\infty}^{\infty}}
\newcommand*{\mat}[1]{\vb{#1}}

\title{Homework 4}
\author{Jinyuan Wu}

\begin{document}

\maketitle

\section{}

The original matrix is 
\begin{equation}
    \mat{A} = \pmqty{
        0 & 1 \\ 0 & 0 \\ 1 & 3 \\ 0 & 1
    }.
\end{equation}
Following these steps:
\begin{enumerate}
    \item Move the third line to the top, and
    \item Subtract the second line from the fourth line,
\end{enumerate}
we get the row reduced form 
\begin{equation}
    \mat{A}_{\text{R}} = \pmqty{
        1 & 3 \\ 0 & 1 \\ 0 & 0 \\ 0 & 0
    },
\end{equation}
and applying the same procedure to $\mat{I}_{4 \times 4}$ we get 
\begin{equation}
    \mat{\Omega}_{\text{R}} = \pmqty{
        0 & 0 & 1 & 0 \\
        1 & 0 & 0 & 0 \\
        0 & 1 & 0 & 0 \\ 
        -1 & 0 & 0 & 1
    },
\end{equation}
such that $\mat{\Omega}_{\text{R}} \mat{A} = \mat{A}_{\text{R}}$.

\section{}

The equations are 
\begin{equation}
    \begin{gathered}
        6 x_1-x_2+x_3=0 \\
        x_1-x_4+2 x_5=0 \\
        x_1-2 x_5=0
        \end{gathered},
\end{equation}
which is equivalent to 
\begin{equation}
    \pmqty{
        6 & -1 & 1 & 0 & 0 \\
        1 & 0 & 0 & -1 & 2 \\
        1 & 0 & 0 & 0 & -2
    } \mat{x} = \mat{0}.
\end{equation}
The row reduced form of the matrix in the LHS is 
\[
    \pmqty{
        1 & 0 & 0 & 0 & -2 \\
        0 & 1 & -1 & 0 & -12 \\
        0 & 0 & 0 & 1 & -4
    }
\]
If we switch the third and the fourth coulomb, we immediate find two independent solutions 
\[
    \pmqty{
        0 \\ -1 \\ 0 \\ -1 \\ 0
    }, \quad 
    \pmqty{
        -2 \\ -12 \\ -4 \\ 0 \\ -1
    },
\]
and if we switch the coulombs back we find a basis of the solution space is 
\begin{equation}
    \pmqty{
        0 \\ -1 \\ -1 \\ 0 \\ 0
    }, \quad 
    \pmqty{
        -2 \\ -12 \\ 0 \\ -4 \\ -1
    },
\end{equation}
and the solution space is 2-dimensional, 
and the general solution looks like 
\begin{equation}
    x_1 = 2 t_2, \quad 
    x_2 = t_1 + 12 t_2, \quad 
    x_3 = t_1, \quad 
    x_4 = 4 t_2, \quad 
    x_5 = t_2.
\end{equation}

\section{}

The equation system 
\begin{equation}
    \begin{gathered}
        2 x_1-3 x_2+x_4=1 \\
        3 x_1+x_3-x_4=0 \\
        2 x_1-3 x_2+10 x_3=0
        \end{gathered}
\end{equation}
is equivalent to 
\begin{equation}
    \pmqty{
        2 & -3 & 0 & 1 \\
        3 & 0 & 1 & -1 \\
        2 & -3 & 10 & 0
    } \pmqty{
        x_1 \\ x_2 \\ x_3 \\x_4
    } = \pmqty{
        1 \\ 0 \\ 0
    }.
\end{equation}
The reduced matrix of the LHS is 
\[
    \pmqty{
        1 & 0 & 0 & - \frac{3}{10} \\
        0 & 1 & 0 & - \frac{8}{15} \\
        0 & 0 & 1 & - \frac{1}{10} 
    },
\]
and the general solution of the homogeneous version of the equation is therefore 
\[
    t \pmqty{
        \frac{3}{10} \\ \frac{8}{15} \\ \frac{1}{10} \\ 1
    }.
\]
A specific solution of the equation system can be easily found by setting $x_4 = 0$:
\[
    \pmqty{
        \frac{1}{30} \\ - \frac{14}{45} \\ - \frac{1}{10}
    }.
\]
So the general solution is 
\begin{equation}
    x_1 = \frac{3}{10} t + \frac{1}{30}, \quad 
    x_2 = \frac{8}{15} t - \frac{14}{45}, \quad 
    x_3 = \frac{1}{10} t - \frac{1}{10}, \quad 
    x_4 = t.
\end{equation}

\section{}

Since 
\begin{equation}
    \mathbf{A}=\left(\begin{array}{cc}
        -1 & 0 \\
        4 & 4
        \end{array}\right),
\end{equation}
we have $\det \mat{A} = - 4$, and therefore it's not singular. 
The inverse is given by 
\begin{equation}
    \mat{A}^{-1} = \frac{1}{\det \mat{A}} \pmqty{
        4 & -4 \\ 
        0 & -1
    }^{\top} = \pmqty{
        -1 & 0 \\ 
        1 & 1/4
    }.
\end{equation}

\section{}

The equation system 
\begin{equation}
    \begin{gathered}
        8 x_1-4 x_2+3 x_3=0 \\
        x_1+5 x_2-x_3=-5 \\
        -2 x_1+6 x_2+x_3=-4
        \end{gathered}
\end{equation}
is equivalent to 
\begin{equation}
    \underbrace{\pmqty{
        8 & - 4 & 3 \\
        1 & 5 & -1 \\
        -2 & 6 & 1
    }}_{\mat{A}} \pmqty{x_1 \\ x_2 \\ x_3} 
    = \pmqty{0 \\ -5 \\ -4}.
\end{equation}
We have 
\begin{equation}
    \det \mat{A} = 132, 
\end{equation}
and by Cramer's rule we have 
\begin{equation}
    x_1 = \frac{1}{132} \cdot -66 = - \frac{1}{2}, \quad 
    x_2 = \frac{1}{132} \cdot -114 = - \frac{19}{22}, \quad 
    x_3 = \frac{1}{132} \cdot 24 = \frac{2}{11}.
\end{equation}

\section{}

\begin{equation}
    \mathbf{A}=\left(\begin{array}{lll}
        0 & 0 & 0 \\
        1 & 0 & 2 \\
        0 & 1 & 3
        \end{array}\right),
\end{equation}
so 
\begin{equation}
    \det (\mat{A} - \lambda \mat{I}) = 0 \Rightarrow
    \lambda = \frac{3 \pm \sqrt{17}}{2}, 0.
\end{equation}
$\lambda = 0$ corresponds to 
\[
    \pmqty{
        0 & 0 & 0 \\
        1 & 0 & 2 \\
        0 & 1 & 3
    } \vb{v} = 0, 
\]
a solution of which is 
\[
    \pmqty{2 \\ 3 \\ -1}.
\]
$\lambda = (3 - \sqrt{17}) / 2$ corresponds to 
\[
    \pmqty{
        \frac{\sqrt{17} - 3}{2} & 0 & 0 \\
        1 & \frac{\sqrt{17} - 3}{2} & 2 \\
        0 & 1 & \frac{3 + \sqrt{17}}{2}
    } \vb{v} = \vb{0},
\]
a solution of which is 
\[
    \pmqty{
        0 \\ \frac{\sqrt{17} + 3}{2} \\ -1
    }.
\]
$\lambda = (3 + \sqrt{17}) / 2$ corresponds to 
\[
    \pmqty{
        -\frac{\sqrt{17} + 3}{2} & 0 & 0 \\
        1 & -\frac{\sqrt{17} + 3}{2} & 2 \\
        0 & 1 & \frac{3 - \sqrt{17}}{2}
    } \vb{v} = \vb{0},
\]
a solution of which is 
\[
    \pmqty{
        0 \\ \frac{3 - \sqrt{17}}{2} \\ -1
    }.
\]
So we have 
\begin{equation}
    \mat{P} = \pmqty{
         0 & 0  & 2\\
        \frac{3 - \sqrt{17}}{2} & \frac{3 + \sqrt{17}}{2} & 3 \\
        -1 & -1 & -1
    },
\end{equation}
and 
\begin{equation}
    \mat{A} = \mat{P} \pmqty{\dmat{\frac{3 + \sqrt{17}}{2}, \frac{3 - \sqrt{17}}{2}, 0}} \mat{P}^{-1}.
\end{equation}

\section{}

The procedure is the same as the last problem, 
and we find the eigenvalues are $\pm \sqrt{17}$,
and $\sqrt{17}$ corresponds to $(\sqrt{17} - 4, 1)$, 
and $- \sqrt{17}$ corresponds to $(- \sqrt{17} - 4, 1)$.
Then the dot product of the two eigenvectors is 
\[
    (\sqrt{17} - 4)(-\sqrt{17} - 4) + 1 = 0,
\]
so the eigenvectors are orthogonal to each other.
After normalization we get 
\begin{equation}
    \mat{P} = \left(
        \begin{array}{cc}
         \frac{\sqrt{17}-4}{\sqrt{34-8 \sqrt{17}}} & -\frac{4+\sqrt{17}}{\sqrt{34+8 \sqrt{17}}} \\
         \frac{1}{\sqrt{34-8 \sqrt{17}}} & \frac{1}{\sqrt{34+8 \sqrt{17}}} \\
        \end{array}
        \right),
\end{equation}
and 
\begin{equation}
    \mat{A} = \mat{P} \pmqty{\dmat{\sqrt{17}, - \sqrt{17}}} \mat{P}^{-1}.
\end{equation}

\section{}

\section{}

Following the same procedure used above, 
we do eigenvalue decomposition and find 
\begin{equation}
    \mat{A} = \left(\begin{array}{cc}
        4 & -2 \\
        -2 & 1
        \end{array}\right) = \left(
            \begin{array}{cc}
             -2 & 1 \\
             1 & 2 \\
            \end{array}
        \right) 
        \pmqty{\dmat{5, 0}}
        \left(
        \begin{array}{cc}
        -2 & 1 \\
        1 & 2 \\
        \end{array}
        \right)^{-1}.
\end{equation}
The transition matrix is therefore
\begin{equation}
    \ee^{\mat{A} t} = \pmqty{
        -2 & 1 \\
        1 & 2
    } 
    \pmqty{\dmat{\ee^{5t}, 1}}
    \pmqty{
        -2 & 1 \\
        1 & 2
    }^{-1} = \pmqty{
        \frac{4 e^{5 t}}{5}+\frac{1}{5} & \frac{2}{5}-\frac{2 e^{5 t}}{5} \\
        \frac{2}{5}-\frac{2 e^{5 t}}{5} & \frac{e^{5 t}}{5}+\frac{4}{5} \\
    }.
\end{equation}

\section{}

We do eigenvalue decomposition 
\begin{equation}
    \mat{A} = \left(\begin{array}{ll}
        1 & 1 \\
        1 & 1
        \end{array}\right) = \underbrace{\frac{1}{\sqrt{2}} \pmqty{1 & -1 \\ 1 & 1} }_{\mat{P}}
        \pmqty{\dmat{2, 0}}
        \frac{1}{\sqrt{2}} \pmqty{1 & 1 \\ -1 & 1},
\end{equation}
and the equation 
\begin{equation}
    \vb{X}' = \mat{A} \vb{X} + \vb{G}
\end{equation}
is then equivalent to 
\begin{equation}
    (\mat{P}^{-1} \vb{X} )' = \pmqty{\dmat{2, 0}} (\mat{P}^{-1} \vb{X} ) 
    + \mat{P}^{-1} \vb{G} ,
\end{equation}
where 
\begin{equation}
    \mat{P}^{-1} \vb{G}  = \frac{1}{\sqrt{2}} \pmqty{1 & 1 \\ -1 & 1} \left(\begin{array}{c}
        6 e^{3 t} \\
        4
        \end{array}\right) 
        = \frac{1}{\sqrt{2}} \pmqty{
            6 \ee^{3t} + 4 \\ 
            - 6 \ee^{3t} + 4
        }.
\end{equation}
The general solution of 
\begin{equation}
    y' = 2 y + \frac{1}{\sqrt{2}} ( 6 \ee^{3t} + 4)
\end{equation}
is 
\begin{equation}
    y = c_1 \ee^{2t} + \frac{1}{\sqrt{2}} (6 \ee^{3t} - 2),
\end{equation}
and the general solution of 
\begin{equation}
    y' = \frac{1}{\sqrt{2}} (- 6 \ee^{3t} + 4)
\end{equation}
is 
\begin{equation}
    y = c_2 + \frac{1}{\sqrt{2}} (- 2 \ee^{3t} + 4 t)
\end{equation}
The general solution of the original problem then is 
\begin{equation}
    \vb{X} = \mat{P} \pmqty{
        c_1 \ee^{2t} + \frac{1}{\sqrt{2}} (6 \ee^{3t} - 2) \\
        c_2 + \frac{1}{\sqrt{2}} (- 2 \ee^{3t} + 4 t)
    } = 
    \pmqty{
        C_1 \ee^{2t} - C_2 + 4 \ee^{3t} - 1 - 2 t \\
        C_1 \ee^{2t} + C_2 + 2 \ee^{3t} - 1 + 2t
    }.
\end{equation}

\end{document}