\documentclass[hyperref, a4paper]{article}

\usepackage{geometry}
\usepackage{titling}
\usepackage{titlesec}
% No longer needed, since we will use enumitem package
% \usepackage{paralist}
\usepackage{enumitem}
\usepackage{footnote}
\usepackage{marginnote}
\usepackage{enumerate}
\usepackage{amsmath, amssymb, amsthm}
\usepackage{mathtools}
\usepackage{bbm}
\usepackage{cite}
\usepackage{graphicx}
\usepackage{subfigure}
\usepackage{physics}
\usepackage{tensor}
\usepackage{siunitx}
\usepackage[version=4]{mhchem}
\usepackage{tikz}
\usepackage{xcolor}
\usepackage{listings}
\usepackage{autobreak}
\usepackage[ruled, vlined, linesnumbered]{algorithm2e}
\usepackage{nameref,zref-xr}
\zxrsetup{toltxlabel}
\zexternaldocument*[optics-]{../optics/optics}[optics.pdf]
\zexternaldocument*[solid-]{../solid/solid}[solid.pdf]
\zexternaldocument*[jellium-]{../band-metal-insulator/electron-gas}[electron-gas.pdf]
\usepackage[colorlinks,unicode]{hyperref} % , linkcolor=black, anchorcolor=black, citecolor=black, urlcolor=black, filecolor=black
\usepackage[most]{tcolorbox}
\usepackage{prettyref}

% Page style
\geometry{left=3.18cm,right=3.18cm,top=2.54cm,bottom=2.54cm}
\titlespacing{\paragraph}{0pt}{1pt}{10pt}[20pt]
\setlength{\droptitle}{-5em}
\preauthor{\vspace{-10pt}\begin{center}}
\postauthor{\par\end{center}}

% More compact lists 
\setlist[itemize]{
    itemindent=17pt, 
    leftmargin=1pt,
    listparindent=\parindent,
    parsep=0pt,
}

% Math operators
\DeclareMathOperator{\timeorder}{\mathcal{T}}
\DeclareMathOperator{\diag}{diag}
\DeclareMathOperator{\legpoly}{P}
\DeclareMathOperator{\primevalue}{P}
\DeclareMathOperator{\sgn}{sgn}
\newcommand*{\ii}{\mathrm{i}}
\newcommand*{\ee}{\mathrm{e}}
\newcommand*{\const}{\mathrm{const}}
\newcommand*{\suchthat}{\quad \text{s.t.} \quad}
\newcommand*{\argmin}{\arg\min}
\newcommand*{\argmax}{\arg\max}
\newcommand*{\normalorder}[1]{: #1 :}
\newcommand*{\pair}[1]{\langle #1 \rangle}
\newcommand*{\fd}[1]{\mathcal{D} #1}
\DeclareMathOperator{\bigO}{\mathcal{O}}

% TikZ setting
\usetikzlibrary{arrows,shapes,positioning}
\usetikzlibrary{arrows.meta}
\usetikzlibrary{decorations.markings}
\tikzstyle arrowstyle=[scale=1]
\tikzstyle directed=[postaction={decorate,decoration={markings,
    mark=at position .5 with {\arrow[arrowstyle]{stealth}}}}]
\tikzstyle ray=[directed, thick]
\tikzstyle dot=[anchor=base,fill,circle,inner sep=1pt]

% Algorithm setting
% Julia-style code
\SetKwIF{If}{ElseIf}{Else}{if}{}{elseif}{else}{end}
\SetKwFor{For}{for}{}{end}
\SetKwFor{While}{while}{}{end}
\SetKwProg{Function}{function}{}{end}
\SetArgSty{textnormal}

\newcommand*{\concept}[1]{{\textbf{#1}}}

% Embedded codes
\lstset{basicstyle=\ttfamily,
  showstringspaces=false,
  commentstyle=\color{gray},
  keywordstyle=\color{blue}
}

% Reference formatting
\newrefformat{fig}{Figure~\ref{#1} on page~\pageref{#1}}

% Color boxes
\tcbuselibrary{skins, breakable, theorems}
\newtcbtheorem[number within=section]{warning}{Warning}%
  {colback=orange!5,colframe=orange!65,fonttitle=\bfseries, breakable}{warn}
\newtcbtheorem[number within=section]{note}{Note}%
  {colback=green!5,colframe=green!65,fonttitle=\bfseries, breakable}{note}

\newcommand{\jellium}{\href{../band-metal-insulator/electron-gas.pdf}{this note}}

\title{Basic Concepts in Conformal Field Theories}
\author{Jinyuan Wu}

\begin{document}

\maketitle

This is a reading note of \cite{francesco2012conformal}, covering Chapter 4, 5, and \cite{blumenhagen2009introduction}. %TODO
When not mentioned, all equation, section and chapter indexes are in \cite{francesco2012conformal}.

\section{Some details about 2D conformal group}

The derivation from (5.2) and (5.3) to (5.4) and (5.5) is purely algebraic. \marginnote{Sec.~5.1.1}

From 
\[
    \partial_0 = \pdv{z}{z^0} \partial_z + \pdv{\bar{z}}{z^0} \partial_{\bar{z}} = \partial_z + \partial_{\bar{z}},
\]
and 
\[
    \partial_1 = \pdv{z}{z^1} \partial_z + \pdv{\bar{z}}{z^1} \partial_{\bar{z}} = \ii \partial_z - \ii \partial_{\bar{z}},
\]
we can derive the equations about derivative operators in (5.6).

The metrics $g_{\mu \nu}$ in (5.7) can be obtained by verifying that 
\[
    \begin{aligned}
        \frac{1}{2} \dd{z} \dd{\bar{z}} + \frac{1}{2} \dd{\bar{z}} \dd{z} &= 
        \frac{1}{2} (\dd{z^0} + \ii \dd{z^1}) (\dd{z^0} - \ii \dd{z^1})
        +  \frac{1}{2} (\dd{z^0} - \ii \dd{z^1}) (\dd{z^0} + \ii \dd{z^1}) \\
        &= (\dd{z^0})^2 + (\dd{z^1})^2.
    \end{aligned}
\]
$g^{\mu \nu}$ can be calculated by taking the inverse.

(5.8) can also be derived by noticing that 
\[
    \epsilon^{z \bar{z}} = \epsilon^{01} \pdv{z}{x^0} \pdv{\bar{z}}{x^1} + \epsilon^{10} \pdv{z}{x^1} \pdv{\bar{z}}{x^0} = 1 \cdot 1 \cdot (-\ii) + (-1) \cdot 1 \cdot \ii = - 2 \ii,
\]
etc. Note that % TODO: how to derive these by definition??

Any function holomorphic on a subset of the complex plane may serve as a local 2D conformal transformation. \marginnote{Sec.~5.1.2}
The amplitude of $\dv*{w}{z}$ is the dilation factor, while its phase is the rotation.
A \emph{global} 2D conformal transformation can only have one pole (the ``rotation center'') $z_0$, and it must 
map $\mathbb{C} \backslash \{z_0\}$ into itself. To guarantee this, we can only take a transformation in the form of 
(5.12). 

For \emph{local} conformal transformations, the relation between $w$ and $z$ can be expanded into a Laurent 
series, and without loss of generality, we only consider a local conformal transformation defined around 
$z=0$ (local conformal transformations defined around other points can be obtained by a translation, which is 
already covered in the Lorentz group), and get (5.17), so we can identify (5.18), and then from (5.18) 
we get (5.19), the \concept{Witt algebra}.  \marginnote{Discussion from (5.15) to (5.19)}
Note that when working with (5.16), it is most convenient to work with the pushforward of $\epsilon(z)$ (i.e. $\epsilon(z')$). 

Actually, (5.18) contains \emph{all} generators of the 2D conformal group ($n, m$ can take negative values). 
We have 
\begin{equation} \marginnote{Discussion around (5.20)}
    l_{-1} = - \partial_z, \quad \bar{l}_{-1} = - \partial_{\bar{z}},
\end{equation}
and the generators along the $x$ and $y$ axises are linear combinations of these two generators.
Similarly, it is easy to see that 
\begin{equation}
    l_0 = - z \partial_z, \quad \bar{l}_0 = - \bar{z} \partial_{\bar{z}}
    \label{eq:l0-operator}
\end{equation}
are just two independent linear combinations of the rotation and dilation generators in (4.18).
Since there is only one generator for rotation and one generator for dilation in the global 2D conformal group, 
\eqref{eq:l0-operator} covers all global rotation and dilation operations.
Also,  
\begin{equation}
    l_1 = - z^2 \partial_z , \quad \bar{l}_1 = - \bar{z}^2 \partial_{\bar{z}}
    \label{eq:l1-generator}
\end{equation}
are just two independent linear combination of the special conformal transformation in (4.18), and since there 
are two special conformal transformation generators, again we find that \eqref{eq:l1-generator} covers all 
special conformal transformations. 

\section{Fields, correlation functions and the energy-momentum tensor}

Here we repeat the definition in \cite{blumenhagen2009introduction}. \marginnote{Chapter 2 in \cite{blumenhagen2009introduction}, Definition 3-5}
Fields depending only on $z$ are called \concept{chiral fields}, and fields depending only on $\bar{z}$ are 
called \concept{anti-chiral fields}. Since an CFT can be regarded as a RG fixed point, it can be expected that 
at least some fields may transform under a scaling $z \to \lambda z$ according to 
\begin{equation}
    \phi(z, \bar{z}) \to \phi'(z, \bar{z}) = \lambda^h \bar{\lambda}^{\bar{h}} \phi(\lambda z, \bar{\lambda} \bar{z}).
    \label{eq:conformal-dimension-1}
\end{equation}
In this case, we say the field has \concept{conformal dimensions} $(h, \bar{h})$. Note that \eqref{eq:conformal-dimension-1} can be rewritten into 
\[
    \phi'(z, \bar{z}) = \lambda^{-h} \bar{\lambda}^{-\bar{h}} \phi(\lambda^{-1} z, \bar{\lambda}^{-1} \bar{z}),
\]
or in other words
\begin{equation}
    \phi'(z', \bar{z}') = \lambda^{-h} \bar{\lambda}^{- \bar{h}} \phi(z, \bar{z}) , \quad z' = \lambda z.
    \label{eq:conformal-dimension-2}
\end{equation}
Why $h$ and $\bar{h}$ are different can be seen in the discussion around \eqref{eq:h-hbar-delta-s}.
Here we can see the notation is consistent with the usual convention in ordinary RG calculations, because we have 
\[
    \int \dd[d]{x'} \mathcal{L}[\phi'] \simeq \int \dd[d]{x} \mathcal{L}[\phi].
\]
Since we are working with conformal invariance, we can impose a stricter restriction than 
\eqref{eq:conformal-dimension-1}: if a field transforms under a conformal transformation $z \to f(z)$
according to 
\begin{equation}
    \phi(z, \bar{z}) \to \phi'(z, \bar{z}) = \left(\pdv{f}{z}\right)^{h} \left(\pdv{\bar{f}}{\bar{z}}\right)^{\bar{h}} \phi(f(z), \bar{f}(\bar{z})),
    \label{eq:primary-field-1}
\end{equation}
or in other words 
\begin{equation}
    \begin{aligned}
        \phi(z, \bar{z}) \to \phi'(z', \bar{z}') &= \left(\pdv{z}{z'}\right)^{h} \left(\pdv{\bar{z}}{\bar{z}'}\right)^{\bar{h}} \phi(z, \bar{z}) \\
        &= \left(\pdv{z'}{z}\right)^{- h} \left(\pdv{\bar{z}'}{\bar{z}}\right)^{- \bar{h}} \phi(z, \bar{z}) ,
    \end{aligned}
    \label{eq:primary-field-def}
\end{equation}
we say it is a \concept{primary field} with conformal dimensions $(h, \bar{h})$. If \eqref{eq:primary-field-1}
only holds for \emph{global} conformal transformations, we say $\phi$ is a \concept{quasi-primary field}.

For a scalar field, we have 
\begin{equation} \marginnote{(4.32) and (4.33)}
    \phi(x) \to \phi'(x') = \abs{\pdv{x'}{x}}^{- \Delta / d} \phi(x) = \abs*{\lambda}^{- \Delta} \phi(x), \quad x' = \lambda x, 
\end{equation}
where the denominator $d$ is used to counter the $d$ factor arising from the Jacobian. Therefore we have 
\begin{equation}
    h = \bar{h} = \frac{1}{2} \Delta.
\end{equation}

Since we are working on $\mathbb{R}^2$, it is possible for a field to carry a representation of $SU(2)$.
$SU(2) \simeq U(1)$ is an Abelian group, and its irreducible representations are all 1-dimensional, and 
a field carrying an $s$-representation of $SU(2)$ (i.e. a field with \concept{planar spin} $s$) transforms as 
\begin{equation}
    \phi(x) \to \phi'(x) = \ee^{- \ii s \phi} \phi(\Lambda_{- \phi} x) ,
\end{equation} % TODO: sign in front of $\ii s \phi$
where $\phi$ is the angle of the rotation. In this case, if a field with planar spin $s$ has well-defined conformal 
dimensions, under a generic transformation, we have 
\begin{equation}
    \phi(z, \bar{z}) \to \phi'(z', \bar{z}') = \abs*{\lambda}^{- \Delta} \ee^{- \ii \phi l} \phi(z, \bar{z}), \quad z' = \underbrace{\abs*{\lambda} \ee^{\ii \phi}}_\lambda z,
\end{equation}
and we can rewrite this equation into the form of \eqref{eq:conformal-dimension-2}, where 
\begin{equation} \marginnote{(5.21)}
    h = \frac{1}{2} (\Delta + s), \quad \bar{h} = \frac{1}{2}(\Delta - s).
    \label{eq:h-hbar-delta-s}
\end{equation}
So here we see the meaning of $h$ and $\bar{h}$. 

Now we need to see how we should define a \emph{primary field} with certain conformal dimensions. 
The reasonable choice seems to just generalize the definition \eqref{eq:h-hbar-delta-s} of $h$ and $\bar{h}$ 
into \eqref{eq:primary-field-def}. % TODO
So a spin-$s$ primary field is a field transforms like \eqref{eq:primary-field-def}, where \eqref{eq:h-hbar-delta-s} holds.
We can see that despite having only one component, a primary field with conformal dimensions $(h, \bar{h})$
transforms like a tensor defined in \emph{two} coordinate systems, with $h$ $z$-indices and $\bar{h}$ $\bar{z}$-indices. 
This is kind of strange, since in ordinary QFT spins are strongly related to the component structure of fields.

Now we can determine the structure of correlation functions. First, rotational and translational invariance require 
\[ \marginnote{(4.50)}
    \expval{\phi_1(x_1) \phi_2(x_2)} = f(\abs*{x_1 - x_2}),
\]
and then scale invariance requires 
\[
    \expval{\phi_1(x_1) \phi_2(x_2)} = \frac{C_{12}}{\abs*{x_1 - x_2}^\Delta}.
\]
Considering \eqref{eq:primary-field-def}, we find 

\bibliographystyle{plain}
\bibliography{../cft-fudan/cft}

\end{document}