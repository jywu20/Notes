\documentclass[hyperref, a4paper]{article}

\usepackage{geometry}
\usepackage{float}
\usepackage{titling}
\usepackage{titlesec}
% No longer needed, since we will use enumitem package
% \usepackage{paralist}
\usepackage{enumitem}
\usepackage{footnote}
\usepackage{enumerate}
\usepackage{amsmath, amssymb, amsthm}
\usepackage{mathtools}
\usepackage{bbm}
\usepackage{cite}
\usepackage{graphicx}
\usepackage{subfigure}
\usepackage{physics}
\usepackage{tensor}
\usepackage{siunitx}
\usepackage{booktabs}
\usepackage[version=4]{mhchem}
\usepackage{tikz}
\usepackage{xcolor}
\usepackage{listings}
\usepackage{autobreak}
\usepackage[ruled, vlined, linesnumbered]{algorithm2e}
\usepackage{nameref,zref-xr}
\zxrsetup{toltxlabel}
\zexternaldocument*[solid-]{../solid/solid}[solid.pdf]
\zexternaldocument*[last-]{./2022-3-15}[2022-3-15.pdf]
\usepackage[colorlinks,unicode]{hyperref} % , linkcolor=black, anchorcolor=black, citecolor=black, urlcolor=black, filecolor=black
\usepackage[most]{tcolorbox}
\usepackage{prettyref}

% Page style
\geometry{left=3.18cm,right=3.18cm,top=2.54cm,bottom=2.54cm}
\titlespacing{\paragraph}{0pt}{1pt}{10pt}[20pt]
\setlength{\droptitle}{-5em}

% More compact lists 
% \setlist[itemize]{itemindent=17pt, leftmargin=1pt}

% Math operators
\DeclareMathOperator{\timeorder}{T}
\DeclareMathOperator{\diag}{diag}
\DeclareMathOperator{\legpoly}{P}
\DeclareMathOperator{\primevalue}{P}
\DeclareMathOperator{\sgn}{sgn}
\newcommand*{\ii}{\mathrm{i}}
\newcommand*{\ee}{\mathrm{e}}
\newcommand*{\const}{\mathrm{const}}
\newcommand*{\suchthat}{\quad \text{s.t.} \quad}
\newcommand*{\argmin}{\arg\min}
\newcommand*{\argmax}{\arg\max}
\newcommand*{\normalorder}[1]{: #1 :}
\newcommand*{\pair}[1]{\langle #1 \rangle}
\newcommand*{\fd}[1]{\mathcal{D} #1}
\DeclareMathOperator{\bigO}{\mathcal{O}}
\DeclareMathOperator{\id}{id}

% TikZ setting
\usetikzlibrary{decorations.text}
\usetikzlibrary{arrows,shapes,positioning}
\usetikzlibrary{arrows.meta}
\usetikzlibrary{decorations.markings}
\tikzstyle arrowstyle=[scale=1]
\tikzstyle directed=[postaction={decorate,decoration={markings,
    mark=at position .5 with {\arrow[arrowstyle]{stealth}}}}]
\tikzstyle ray=[directed, thick]
\tikzstyle dot=[anchor=base,fill,circle,inner sep=1pt]

% Algorithm setting
% Julia-style code
\SetKwIF{If}{ElseIf}{Else}{if}{}{elseif}{else}{end}
\SetKwFor{For}{for}{}{end}
\SetKwFor{While}{while}{}{end}
\SetKwProg{Function}{function}{}{end}
\SetArgSty{textnormal}

\newcommand*{\concept}[1]{{\textbf{#1}}}

\DeclareMathOperator{\im}{im}

% Embedded codes
\lstset{basicstyle=\ttfamily,
  showstringspaces=false,
  commentstyle=\color{gray},
  keywordstyle=\color{blue}
}

\newcommand{\soliddoc}{\href{../solid/solid.pdf}{this note}}
\newcommand{\lastlec}{\href{./2022-3-15.pdf}{the last lecture}}

% Color boxes
\tcbuselibrary{skins, breakable, theorems}
\newtcbtheorem[number within=section]{warning}{Warning}%
  {colback=orange!5,colframe=orange!65,fonttitle=\bfseries, breakable}{warn}
\newtcbtheorem[number within=section]{note}{Note}%
  {colback=green!5,colframe=green!65,fonttitle=\bfseries, breakable}{note}
\newtcbtheorem[number within=section]{info}{Info}%
  {colback=blue!5,colframe=blue!65,fonttitle=\bfseries, breakable}{info}

% Disable unsupported commands in bookmark titles 
\pdfstringdefDisableCommands{%
  \def\\{}%
  \def\texttt#1{<#1>}%
  \def\mathbb#1{#1}%
}
\pdfstringdefDisableCommands{\def\eqref#1{(\ref{#1})}}

\makeatletter
\pdfstringdefDisableCommands{\let\HyPsd@CatcodeWarning\@gobble}
\makeatother

\title{Prof. Yang Qi on topological classification of gapped free fermion models}
\author{Jinyuan Wu}

\begin{document}

\maketitle

Gapped systems satisfy 
the adiabatic theorem, so with small perturbation on the Hamiltonian, we can transform the $n$th eigenstate 
of one Hamiltonian into the $n$th eigenstate of another Hamiltonian, and then go back, so in this sense, 
these states are \emph{equivalent}. This gives a natural criterion of Hamiltonian classification, each 
equivalence class of which can be said as a homotopy equivalence class, and in this sense, 
we achieve a \emph{topological classification} of the Hamiltonians.

For free fermions, perturbation of the Hamiltonian is just perturbation of the band structure, so 
topological classification of Hamiltonians reduces to topological classification of band structures:
two band structures that can be connected by a smooth deformation which doesn't cross the Fermi surface
(or otherwise when a band crosses the Fermi surface, the system becomes a metal and is gapless). 

In the following lectures we discuss topological classification of gapped free fermions.
These systems include insulators and the electron-like excitations in superconductors (as well as other 
systems with certain kind of electron condensation).

It's possible that a topological equivalence class involves several separate subclasses, each of which 
has a different symmetry group, and a smooth transformation from one class into another inevitably 
breaks or adds some symmetry. In this case, the subclasses are said to be \emph{symmetry protected} --
as long as the symmetry is present, we are sure that perturbation on the Hamiltonian won't get 
the system out of its subclass.

\section{Antiunitary symmetry of free fermions}

Consider the following second quantized Hamiltonian:
\begin{equation}
    \hat{H} = \sum_{\alpha, \beta} c^\dagger_\alpha H_{\alpha \beta} c_\beta,
    \label{eq:second-quantization}
\end{equation}
where operators with $\hat{}$ are second quantized operators. In this section, we consider how 
antiunitary symmetries acts on \eqref{eq:second-quantization}. The action of a unitary group element can be 
written as 
\begin{equation}
    g \cdot \ket{\alpha} = \ket{\beta} \varphi(g)_{\beta \alpha}, \quad 
    g \cdot \bra{\alpha} = \varphi(g)^\dagger_{\alpha \beta} \bra{\beta}, 
\end{equation}
and 
\begin{equation}
    g \cdot c^\dagger_\alpha = g c^\dagger_\alpha g^{-1} = c^\dagger_\beta \varphi(g)_{\beta \alpha}, \quad 
    g \cdot c_\alpha = g c_\alpha g^{-1} = \varphi(g)^\dagger_{\alpha \beta} c_\beta.
\end{equation}
So the Hamiltonian transforms as 
\begin{equation}
    g \cdot \hat{H} = c^\dagger_{\alpha'} \varphi(g)_{\alpha' \alpha} H_{\alpha \beta} \varphi(g)^\dagger_{\beta \beta'} c_{\beta'}.
\end{equation}
Of course, this is just in the form of basis transition. If the Hamiltonian has the symmetry of $g$, we have 
\begin{equation}
    H_{\alpha' \beta'} = \varphi(g)_{\alpha' \alpha} H_{\alpha \beta} \varphi(g)^\dagger_{\beta \beta'}.
\end{equation}
Suppose $\{\ket{\alpha}\}$ is real with regard of the time reversal operation $\mathcal{T} = T \mathcal{K}$.
In this case, the time reversal symmetry doesn't act on the basis, but $\mathcal{K}$ acts on $H_{\alpha \beta}$
and adds a star, so if the system has time reversal symmetry, we have 
\begin{equation}
    T H T^{-1} = {H}^*.
    \label{eq:time-reversal-first-quantization}
\end{equation}
Note that here $H$ is the \emph{first quantized} Hamiltonian, i.e. the coefficient matrix 
in \eqref{eq:second-quantization}. The operation $T$ is also the \emph{unitary} transformation associated 
to $\mathcal{K}$, instead of the full \emph{second quantized} $\mathcal{T}$. We work with first quantization
because it's easier: we don't need to actually work with complex conjugate of creation and annihilation operators.

Now we move to the ``real'' particle-hole symmetry in an insulator. Consider 
\begin{equation}
    \hat{H} = - \sum_{\pair{\vb*{i}, \vb*{j}}} (t_{\vb*{i} \vb*{j}} c^\dagger_{\vb*{i}} c_{\vb*{j}} + \text{h.c.}).
\end{equation}
We use $\mathcal{C}$ to denote the particle-hole transformation from $c$ to $c^\dagger$, i.e.
\begin{equation}
    \mathcal{C} c_{\vb*{i}} \mathcal{C}^{-1} = c_{\vb*{i}}^\dagger, \quad \mathcal{C} c_{\vb*{i}}^\dagger \mathcal{C}^{-1} = c_{\vb*{i}},
\end{equation}
then we have 
\begin{equation}
    \mathcal{C} \hat{H} \mathcal{C}^{-1} = - \sum_{\pair{\vb*{i}, \vb*{j}}} (t_{\vb*{i} \vb*{j}} c_{\vb*{i}} c^\dagger_{\vb*{j}} + \text{h.c.}) = \sum_{\pair{\vb*{i}, \vb*{j}}} (t_{\vb*{j} \vb*{i}}^* c^\dagger_{\vb*{j}} c_{\vb*{i}} + \text{h.c.}) = - \hat{H}^* = - \hat{H}^\top.
\end{equation}
We know the spectrum of the tight-binding model is a cosine curve, so the minus sign comes as is expected, meaning flipping the spectrum with the Fermi surface as a mirror, while the transpose 
operation comes from flipping $\ket{\vb*{i}}$ to $\bra{\vb*{i}}$. Here we need to keep in mind a strange property 
of particle-hole transformation. $\mathcal{C}$ should be \emph{unitary} in the second quantization formalism, because 
we want 
\begin{equation}
    c_{\vb*{k}}^\dagger \stackrel{\mathcal{C}}{\longrightarrow} c_{- \vb*{k}}
\end{equation}
to keep momentum conservation, and if $\mathcal{C}$ is unitary, it doesn't act on a complex factor, so we have 
\[
    \mathcal{C} \cdot c_{\vb*{k}}^\dagger = \mathcal{C} \cdot \frac{1}{\sqrt{N}} \sum_{\vb*{i}} \ee^{- \ii \vb*{k} \cdot \vb*{r}_{\vb*{i}}} c^\dagger_{\vb*{i}} = \sum_{\vb*{i}} \ee^{- \ii \vb*{k} \cdot \vb*{r}_{\vb*{i}}} c_{\vb*{i}} = c_{- \vb*{k}},
\]
which is exactly what we want. However, the first quantization version of $\mathcal{C}$ maps a single-electron 
wave function in the single electron Hilbert space to the \emph{dual} space, and a map $\mathcal{H} \to \mathcal{H}^*$
should be \emph{antiunitary} to keep naturalness.

The action of $\mathcal{C}$ on an arbitrary basis is 
\begin{equation}
    \mathcal{C} c_{\alpha} \mathcal{C}^{-1} = c_\beta^\dagger C_{\beta \alpha}, \quad 
    \mathcal{C} c^\dagger_{\alpha} \mathcal{C}^{-1} = C^{-1}_{\alpha \beta} c_\beta,
    \label{eq:arbitrary-particle-hole}
\end{equation}
and the second quantized Hamiltonian under \eqref{eq:arbitrary-particle-hole} transforms as 
\begin{equation}
    \begin{aligned}
        \mathcal{C} \hat{H} \mathcal{C}^{-1} &= \mathcal{C} c_\alpha^\dagger \mathcal{C}^{-1} H_{\alpha \beta} \mathcal{C} c_\beta \mathcal{C}^{-1} = C^{-1}_{\alpha \alpha'} c_{\alpha'} \underbrace{H_{\alpha \beta}}_{H_{\beta \alpha}^*} c^\dagger_{\beta'} C_{\beta' \beta} \\
        &= - c^\dagger_{\beta'} C_{\beta' \beta} H^*_{\beta \alpha} C^{-1}_{\alpha \alpha'} c_{\alpha'} = - c^\dagger_{\beta'} (CH^* C^{-1})_{\beta' \alpha'} c_{\alpha'}.
    \end{aligned}
\end{equation}
So the particle-hole symmetry condition for the first quantized Hamiltonian is 
\begin{equation}
    H = - C H^* C^{-1}.
    \label{eq:charge-symmetry-first-quantization}
\end{equation}

Despite their weirdness, $\mathcal{T}$ and $\mathcal{C}$ can be fully demonstrated by constraints on 
the coefficient matrix $H$ (i.e. first quantized Hamiltonian) in \eqref{eq:second-quantization}.
Classification of free fermion systems is therefore classification of these matrices. 

$\mathcal{T}$ and $\mathcal{C}$ have non-trivial multiplication relation: since $\mathcal{T}$ is antiunitary
in both first and second quantization formulation, and $\mathcal{C}$ is unitary in second quantization but 
antiunitary in first quantization, 
\begin{equation}
    \mathcal{S} = \mathcal{T} \mathcal{C}
\end{equation} 
is unitary in first quantization but antiunitary in second quantization. We call it \concept{chiral symmetry}
to keep consistent with the definition in particle physics.
The condition that a second quantized Hamiltonian has chiral symmetry can be verified similar to \eqref{eq:time-reversal-first-quantization} and \eqref{eq:charge-symmetry-first-quantization}, and we have 
\begin{equation}
    S H S^{-1} =  - H.
    \label{eq:chiral-first-quantization}
\end{equation}

In the momentum representation, $\mathcal{C}$ is bond to change the momentum in the $\vb*{k} \to - \vb*{k}$ 
way, because of momentum conservation. The rest of labels (band index, spin, etc.) are mixed together with a finite dimensional matrix.
So suppose $H(\vb*{k})$ is the block in $H$ with momentum $\vb*{k}$, and we have 
\begin{equation}
    T H(\vb*{k}) T^{-1} = H(- \vb*{k})^*.
\end{equation}
where $C$ is the block in the $C$ matrix in \eqref{eq:time-reversal-first-quantization} that mixes labels other than $\vb*{k}$.
Similarly from \eqref{eq:charge-symmetry-first-quantization}, we have 
\begin{equation}
    C H(- \vb*{k})^* C^{-1} = - H(\vb*{k}).
\end{equation}
The chiral symmetry doesn't change the momentum, since it's the multiplication of $\mathcal{T}$ and $\mathcal{C}$,
so from \eqref{eq:chiral-first-quantization} we have 
\begin{equation}
    S H(\vb*{k}) S^{-1} =  - H(\vb*{k}).
\end{equation}

Now in a model with multiple bands, it's possible that 
there are several different particle-hole symmetry: it may be possible that band A's particles are equivalent 
to \emph{both} band B and band C's holes. But in this case, band B and band C must coincide and there is 
a symmetry switching B and C, so in the end we only need to define one particle-hole symmetry as the generator:
the rest of the particle-hole symmetries can be automatically generated using composition.

\section{Particle-hole symmetry in superconductors}

The following BCS model of superconductivity:
\begin{equation}
    \hat{H} = \sum_{\vb*{k}, \alpha} \xi_{\vb*{k}} c^\dagger_{\vb*{k} \alpha} c_{\vb*{k} \alpha} - V \sum_{\vb*{k}, \vb*{k}', \vb*{q}, \alpha, \beta} c^\dagger_{\vb*{k}' - \vb*{q}, \alpha} c^\dagger_{\vb*{k} + \vb*{q}, \beta} c_{\vb*{k} \beta} c_{\vb*{k}' \alpha},
    \label{eq:bcs}
\end{equation}
can be turned into 
\begin{equation}
    \hat{H} = \sum_{\vb*{k}, \alpha} \xi_{\vb*{k}} c^\dagger_{\vb*{k} \alpha} c_{\vb*{k} \alpha} 
    + \sum_{\vb*{k}} (\Delta c^\dagger_{- \vb*{k} \downarrow} c^\dagger_{\vb*{k} \uparrow} + \Delta^* c_{\vb*{k} \uparrow} c_{- \vb*{k} \downarrow}),
    \label{eq:bdg}
\end{equation}
where $\Delta$ is the BCS order parameter. \eqref{eq:bdg} is called the \concept{BdG Hamiltonian}, 
which describes the spectrum of (electron-like) quasiparticles in a BCS superconductor. Note that 
the order parameter in \eqref{eq:bdg} has quantum fluctuation, but what we are discussing here is 
the \emph{topological band behavior} of fermions, so ignoring the fluctuation of $\Delta$ makes sense.
It's possible that the fluctuation of $\Delta$ destroys the ordinary BCS order, but it's not the case 
in 3D. Note also that our current approach -- ignoring the fluctuation of $\Delta$, i.e. ignoring the 
many-body effect introduced by electron interaction -- is a non-interaction limit of the general theory 
of topological classification with interaction. 

In 1D and 2D, Mermin–Wagner theorem means the 
effective theory about fermionic quasiparticles of \eqref{eq:bcs} is not \eqref{eq:bdg}, because 
$U(1)$ symmetry -- a continuous symmetry -- can't be broken here. But we can always use a 
3D bulk state to ``induce'' a low-dimensional superconducting phase, which has electron pairing anyway
and can be described by \eqref{eq:bdg}, though this time \eqref{eq:bdg} has nothing to do with BCS mechanism.
So henceforth we will work with a free-fermionic model like \eqref{eq:bdg} and ignore what induces superconductivity
pairing. 

In such a Hamiltonian, the existence of a pairing channel means we have terms like $c_\alpha c_\beta$,
and therefore the Hamiltonian can't be recast into \eqref{eq:second-quantization} from which we can extract 
the first quantized Hamiltonian $H$. The solution is a classical procedure in BCS theory: \eqref{eq:bdg} can be rephrased into 
\begin{equation}
    \hat{H} = \sum_{\vb*{k}} \Psi_{\vb*{k}}^\dagger \pmqty{ \xi_{\vb*{k}} & - \Delta \\ - \Delta^* & -\xi_{\vb*{k}} } \Psi_{\vb*{k}}, \quad \Psi_{\vb*{k}} = \pmqty{c_{\vb*{k} \uparrow} \\ c^\dagger_{- \vb*{k}, \downarrow}}.
\end{equation}
Often, we call this form of Hamiltonian as the \concept{BdG Hamiltonian} rather than \eqref{eq:bdg}.

BdG Hamiltonians often have \emph{redundant} particle-hole symmetry. The point here is that suppose there are $d$ kind of fermions (i.e. 
possible values of labels other than the momentum), then the existence of $c_\alpha c_\beta$ terms means 
$\Psi_{\vb*{k}}$ should be $2d$ dimensional to include all possible terms. So the resulting $H$ matrix 
is $2d \times 2d$ dimensional, the same as the first quantized Hamiltonian of $2d$ kind of fermions,
but it describes $d$ kind of fermions anyway.
Since $d$ kinds of fermions are actually just holes for the rest $d$ kinds of fermions, 
there is possibly a particle-hole symmetry to make sure the spectrum of the first quantized Hamiltonian 
reflects this fact. Consider, for example, the following simplest case:
\begin{equation}
    \hat{H} = \sum_{\vb*{k}} \xi_{\vb*{k}} c^\dagger_{\vb*{k}} c = \frac{1}{2} \sum_{\vb*{k}} \pmqty{c^\dagger_{\vb*{k}} & c_{\vb*{k}} } \pmqty{\dmat{\xi_{\vb*{k}}, - \xi_{\vb*{k}}}} \pmqty{c_{\vb*{k}} \\ c^\dagger_{\vb*{k}}} + \text{const},
\end{equation}
the first quantized Hamiltonian of which obviously has a particle-hole symmetry, which is based on the simple 
fact that (here the complex conjugate operation only applies to the elements of a matrix)
\[
    \pmqty{0 & 1 \\ 1 & 0} \pmqty{c_{\vb*{k}} \\ c^\dagger_{\vb*{k}}}^\dagger = \pmqty{c_{\vb*{k}} \\ c^\dagger_{\vb*{k}}}.
\]

The BCS BdG Hamiltonian \eqref{eq:bdg} also has a particle-hole symmetry. Because $\xi_{\vb*{k}} = \xi_{- \vb*{k}}$, we have 
\begin{equation}
    \pmqty{0 & 1 \\ -1 & 0} \pmqty{\xi_{\vb*{k}} & - \Delta \\ - \Delta^* & - \xi_{\vb*{k}}} \pmqty{0 & -1 \\ 1 & 0} = - \pmqty{\xi_{- \vb*{k}} & - \Delta \\ - \Delta^* & - \xi_{- \vb*{k}}}^*.
\end{equation}
In the BCS case there are two kind of fermions and $\Psi$ is 2-dimensional, so the particle-hole symmetry is 
not a byproduct of double counting of degrees of freedom. 
Sometimes the particle-hole symmetry comes from existing symmetries of the Hamiltonian. In the 
BCS case, it's easy to find that it's just the spin flipping symmetry.

The particle-hole symmetry is not always here. For example, the following superconductor Hamiltonian with the spin rotational symmetry on the $z$ direction:
\begin{equation}
    \hat{H} = \sum_{\vb*{k}} \pmqty{c^\dagger_{\vb*{k} \uparrow} & c_{\vb*{k} \downarrow}} \pmqty{\epsilon_{\vb*{k} \uparrow} & \Delta \\ \Delta & - \epsilon_{- \vb*{k} \downarrow}} \pmqty{c_{\vb*{k} \uparrow} \\ c^\dagger_{\vb*{k} \downarrow}}.
    \label{eq:no-particle-hole-sc-example}
\end{equation}
Its first quantized Hamiltonian don't have any $\mathcal{T}, \mathcal{C}$ or $\mathcal{S}$ symmetry and therefore 
can't be distinguished from, say, the first quantized Hamiltonian of IQHE. 

A superconductor has \emph{more} symmetry than a similar insulator 
because its first quantized Hamiltonian has a particle-hole symmetry, while it has \emph{fewer} actual symmetry than a similar 
insulator because it breaks the $U(1)$ symmetry. 

\section{The ten-fold way}

So there are several possibility of these three weird symmetries. $\mathcal{T}$ and $\mathcal{C}$ symmetries 
may simply not exist, which we denote as ${T} = 0$ or ${C} = 0$. When $\mathcal{T}$ is present, for fermions 
we have $\mathcal{T}^2 = -1$ (fermion parity) and for bosons $\mathcal{T}^2 = 1$. It's also possible to 
redefine $\mathcal{T}$ in a fermionic system so that $\mathcal{T}'^2 = 1$ (see discussion around \eqref{last-eq:so-called-time-reversal} in \lastlec). We denote the $\mathcal{T}^2 = 1$ case as $T=1$, and the 
$\mathcal{T}^2 = -1$ case $T=-1$. For both fermionic and bosonic systems it can be verified that 
$\mathcal{C}^2 = 1$ (for multiple band systems, it may be the case that several bands A, B, C, etc. overlap
and $\mathcal{C}$ turn A into B, B into C, etc., and in this case $\mathcal{C}^2 = 1$ still works for the 
specific definition of $\mathcal{C}$ which only exchange ban labels in pairs), but for certain types of 
$SU(2)$ superconductors, it's useful to define $\mathcal{C}$ such that $\mathcal{C}^2 = -1$. 
So for $\mathcal{C}$ we also have $C = 0, \pm 1$. If both $\mathcal{C}$ and $\mathcal{T}$ are present, 
$\mathcal{S}$ is bond to be a symmetry, while if only one of them is present, $\mathcal{S}$ can't be 
a symmetry. This leaves us a final choice for systems where $T = 0, C = 0$ and $\mathcal{S}$ may be present 
or absent. So in the end, the symmetry of a free fermionic or bosonic system concerning $\mathcal{C}$, 
$\mathcal{T}$ and $\mathcal{S}$ has $3 \times 3 - 1 + 2 = 10$ possibilities, called \concept{the ten-fold way}. Here we list the notation for these symmetry classes and some examples \cite{Ryu_2010}. Note that 
models in a class may have \emph{richer} symmetries than the symmetry labels of the class. For example, 
most models have certain degrees of space group symmetries, which are not considered in the ten-fold way (but 
may also protect certain topological properties). Some models -- say, the SSH model -- have more symmetries
regarding $T, C$ and $S$ than its class, but it doesn't use these symmetries to protect any topological features.
If the additional symmetries are broken, the topological properties are still there.

\begin{itemize}
    \item Class A: $T = 0, \  C = 0, \ S = 0$. Without the particle-hole symmetry,
    typically this is an insulator, and therefore charge conservation holds. Example: Chern-insulators, IQHE.
    But a superconductor like \eqref{eq:no-particle-hole-sc-example} can also be a realization of this symmetry class.
    \item Class AIII: $T = 0, \  C = 0, \ S = 1$. Again, typically an insulator, while superconductors are also possible. Example: SSH model. 
    \item Class D: $T = 0,\  C = 1, \ S = 0$. This class is typically realized as a superconductor,
    which actually may not have any physical symmetry other than the fermion parity symmetry.
    Example: 1D Kitaev chain, 2D $(p + \ii p)$ superconductor.
    Of course, this class can also be implemented by an insulator with a real particle-hole symmetry.
    \item Class DIII: $T=-1, C = 1, S = 1$. Usual superconductors belong to this class, in which 
    $\mathcal{T}^2$ is the fermion parity symmetry. 1D topological superconductors are in this class.
    \item Class AII: $T = -1, C = 0, S = 0$. 
    Topological insulators in the ordinary sense are in this class.
    \item Class CII: $T = -1, C = -1, S = 1$. This is a strange symmetry class. $C = -1$ means the 
    symmetry class can be realized by a $SU(2)$ superconductor, but typically in this case we define $T = 1$.
    Another way to think of this symmetry class is to realize it as an insulator with charge-hole symmetry.
    \item Class C: $T = 0, C = -1, S = 0$. This can be realized by a $SU(2)$ superconductor.
    \item Class CI: $T = 1, C = -1, S = 0$. This can also be realized by a $SU(2)$ superconductor with 
    time reversal symmetry, in which we redefine $\mathcal{T}$ by attaching an $\ii$ factor to it so that $T = 1$.
    \item Class AI: $T = 1, C = 0, S = 0$. A spinless insulator can realize this symmetry class. For electronic
    systems without spin-orbital coupling, by considering states with the following form 
    \[
        \frac{1}{\sqrt{2}}(\ket{\vb*{k}, \uparrow} + \ket{\vb*{k}, \downarrow})
    \]
    we can construct a ``spinless'' subspace.
    \item Class BDI: $T = 1, C = 1, S = 0$. A typical realization is a spinless superconductor, like a $p$-wave 
    superconductor in which the pairing order parameter $\Delta$ is real.
\end{itemize}

\section{The SSH model}

\subsection{The Hamiltonian, topological phases and the winding number}

Band structures in class A and class AIII may be classified by their Chern numbers. It can be proved that 
the possible Chern classes of class A and class AIII gapped free fermion are periodic with respect of the 
dimension with period 2, while the period of other classes is %TODO

\begin{figure}
    \centering
    

\tikzset{every picture/.style={line width=0.75pt}} %set default line width to 0.75pt        

\begin{tikzpicture}[x=0.75pt,y=0.75pt,yscale=-1,xscale=1]
%uncomment if require: \path (0,300); %set diagram left start at 0, and has height of 300

%Straight Lines [id:da9956235879402113] 
\draw    (100,101) -- (134,84) ;
%Straight Lines [id:da7247828675996284] 
\draw    (202,84) -- (168,101) ;
%Straight Lines [id:da5773329935549767] 
\draw    (202,84) -- (236,101) ;
%Straight Lines [id:da3206012026384706] 
\draw    (168,101) -- (144.4,89.2) -- (134,84) ;
%Straight Lines [id:da6366740347566855] 
\draw    (236,101) -- (270,84) ;
%Straight Lines [id:da9111075840879808] 
\draw    (107,104) -- (132,91) ;
%Straight Lines [id:da5940492749723987] 
\draw    (177,103) -- (202,90) ;
%Straight Lines [id:da21751025441510952] 
\draw    (245,103) -- (270,90) ;
%Straight Lines [id:da7004634811978112] 
\draw    (90,146) -- (269,146) ;
%Straight Lines [id:da7336134345483176] 
\draw [color={rgb, 255:red, 208; green, 2; blue, 27 }  ,draw opacity=1 ]   (102,146) ;
\draw [shift={(102,146)}, rotate = 0] [color={rgb, 255:red, 208; green, 2; blue, 27 }  ,draw opacity=1 ][fill={rgb, 255:red, 208; green, 2; blue, 27 }  ,fill opacity=1 ][line width=0.75]      (0, 0) circle [x radius= 3.35, y radius= 3.35]   ;
%Straight Lines [id:da5304413655428435] 
\draw [color={rgb, 255:red, 74; green, 144; blue, 226 }  ,draw opacity=1 ]   (134,146) ;
\draw [shift={(134,146)}, rotate = 0] [color={rgb, 255:red, 74; green, 144; blue, 226 }  ,draw opacity=1 ][fill={rgb, 255:red, 74; green, 144; blue, 226 }  ,fill opacity=1 ][line width=0.75]      (0, 0) circle [x radius= 3.35, y radius= 3.35]   ;
%Straight Lines [id:da016904275323671447] 
\draw [color={rgb, 255:red, 208; green, 2; blue, 27 }  ,draw opacity=1 ]   (168,146) ;
\draw [shift={(168,146)}, rotate = 0] [color={rgb, 255:red, 208; green, 2; blue, 27 }  ,draw opacity=1 ][fill={rgb, 255:red, 208; green, 2; blue, 27 }  ,fill opacity=1 ][line width=0.75]      (0, 0) circle [x radius= 3.35, y radius= 3.35]   ;
%Straight Lines [id:da27782244408746837] 
\draw [color={rgb, 255:red, 74; green, 144; blue, 226 }  ,draw opacity=1 ]   (202,146) ;
\draw [shift={(202,146)}, rotate = 0] [color={rgb, 255:red, 74; green, 144; blue, 226 }  ,draw opacity=1 ][fill={rgb, 255:red, 74; green, 144; blue, 226 }  ,fill opacity=1 ][line width=0.75]      (0, 0) circle [x radius= 3.35, y radius= 3.35]   ;
%Straight Lines [id:da6527169495554346] 
\draw [color={rgb, 255:red, 208; green, 2; blue, 27 }  ,draw opacity=1 ]   (236,146) ;
\draw [shift={(236,146)}, rotate = 0] [color={rgb, 255:red, 208; green, 2; blue, 27 }  ,draw opacity=1 ][fill={rgb, 255:red, 208; green, 2; blue, 27 }  ,fill opacity=1 ][line width=0.75]      (0, 0) circle [x radius= 3.35, y radius= 3.35]   ;

% Text Node
\draw (115,151.4) node [anchor=north west][inner sep=0.75pt]    {$t$};
% Text Node
\draw (145,151.4) node [anchor=north west][inner sep=0.75pt]    {$t'$};
% Text Node
\draw (181,151.4) node [anchor=north west][inner sep=0.75pt]    {$t$};
% Text Node
\draw (213,151.4) node [anchor=north west][inner sep=0.75pt]    {$t'$};
% Text Node
\draw (96,116) node [anchor=north west][inner sep=0.75pt]  [color={rgb, 255:red, 208; green, 2; blue, 27 }  ,opacity=1 ] [align=left] {A};
% Text Node
\draw (162,116) node [anchor=north west][inner sep=0.75pt]  [color={rgb, 255:red, 208; green, 2; blue, 27 }  ,opacity=1 ] [align=left] {A};
% Text Node
\draw (128,116) node [anchor=north west][inner sep=0.75pt]  [color={rgb, 255:red, 74; green, 144; blue, 226 }  ,opacity=1 ] [align=left] {B};
% Text Node
\draw (197,116) node [anchor=north west][inner sep=0.75pt]  [color={rgb, 255:red, 74; green, 144; blue, 226 }  ,opacity=1 ] [align=left] {B};


\end{tikzpicture}

    \caption{The lattice of the SSH model}
    \label{fig:ssh-lattice}
\end{figure}

The \concept{SSH model} is first a tight-binding model on a polyacetyene chain. From the lattice structure 
\prettyref{fig:ssh-lattice}, we can write down the second quantized Hamiltonian
\begin{equation}
    \begin{aligned}
        \hat{H} &= - t \sum_i (a_i^\dagger b_i + \text{h.c.}) - t' \sum_i (b^\dagger_i a_{i+1} + \text{h.c.}) \\
        &= \sum_k \pmqty{a^\dagger_k & b^\dagger_k} \pmqty{0 & - t - t' \ee^{- \ii k} \\ - t - t' \ee^{\ii k} & 0} \pmqty{a_k \\ b_k}.
    \end{aligned}
\end{equation}
The first quantized Hamiltonian is 
\begin{equation}
    H(k) = \pmqty{0 & - t - t' \ee^{- \ii k} \\ - t - t' \ee^{\ii k} & 0} \eqqcolon \pmqty{0 & D(k) \\ D^*(k) & 0}.
    \label{eq:ssh-first-quant}
\end{equation}
Now we check the chiral symmetry. Consider the following definition:
\begin{equation}
    \mathcal{S} : a_i \mapsto a_i^\dagger, \quad b_i \mapsto - b_i^\dagger, \quad \ii \mapsto - \ii.
\end{equation}
This is a typical definition of chiral operation, which involves both flipping $c$ and $c^\dagger$ 
($\mathcal{C}$) and the complex conjugate ($\mathcal{T}$). Under this transformation,
\begin{equation}
    a_k = \sum_i a_i \ee^{- \ii k r_i} \mapsto \sum_i a_i^\dagger \ee^{\ii k r_i} = a_k^\dagger,
\end{equation}
\begin{equation}
    b_k = \sum_i b_i \ee^{- \ii k r_i} \mapsto - \sum_i b_i^\dagger \ee^{\ii k r_i} = - b_k^\dagger,
\end{equation}
so the first quantization transformation is 
\begin{equation}
    S = \tau^z = \pmqty{\dmat{1, -1}},
\end{equation}
and we find 
\begin{equation}
    S H(k) S^{-1} = - H(k),
\end{equation}
so indeed the SSH model has a chiral symmetry. 

All information of a SSH band structure is stored in $D(k)$ in \eqref{eq:ssh-first-quant}. We see $D(k)$
is a circle on the complex plane, and it can't go over the origin (which means the gap is closed).
It's easy to check that when $t = t'$, the gap is closed. This is the critical point which separate the 
$t > t'$ topological phase and the $t < t'$ topological phase. 

We consider the \emph{flattened} Hamiltonian, in which the energy of occupied eigenstate is assigned as $-1$,
and the energy of unoccupied states is assigned as $1$. This means $H^2 = 1$, and all bands of this Hamiltonian
is flatten. We usually denote it as $Q$ to distinct it from the real Hamiltonian $H$. 
For the SSH model, it's 
\begin{equation}
    Q(k) = \pmqty{0 & g(k) \\ g^*(k) & 0}, \quad g(k) = \frac{D(k)}{\abs{D(k)}} \eqqcolon \ee^{\ii \theta(k)},
\end{equation}
and we have 
\begin{equation}
    Q(k)^2 = \pmqty{\dmat{g(k) g^*(k), g(k) g^*(k)}} = 1.
\end{equation}
So now a winding number can be defined for $\theta$, which can also be calculated by 
\begin{equation}
    W = \ii \int \frac{\dd{k}}{2\pi} g^*(k) \partial_k g(k) ,
\end{equation}
because 
\begin{equation}
    W= \ii \int \frac{\dd{k}}{2\pi} g^*(k) \partial_k g(k)  = \ii \int \frac{\dd{k}}{2\pi} \ii \partial_k \theta = - \int_0^{2\pi} \dd{\theta} = - \theta(k)|^{2\pi}_{k=0}.
\end{equation}
Note that $\theta(k)$ can be multivalued, as long as $g(2\pi) = g(0)$, which means
\[
    \theta(2\pi) - \theta(0) = 2\pi n , \quad n \in \mathbb{Z},
\]
so we find $W \in \mathbb{Z}$.

If we can lift the $\theta$

Here a possible paradox occurs: we can decide the topological property of $Q$ (and hence $H$) by calculating 
the winding number, while the shape of bands of $Q$ and a trivial flat band model created by highly localized 
Wannier functions are the same. So what's the information loss after we diagonalize the Hamiltonian?
Actually the topological information is stored in the \emph{basis} after diagonalization. 

\subsection{Polarization}

Now we turn to a concept first developed in \emph{ab initio} calculation: the microscopic definition of 
charge polarization. Now we use $n$ (instead of $i$) to denote the site index and $\alpha$ the sublattice label
(and hence the band index). We have 
\begin{equation}
    \vb*{P} = \expval{\vb*{r}}{n \alpha}, \quad P_x = \expval{x}{n \alpha}.
\end{equation}
Now we are going to show that $P_x$ is actually given by the Berry phase on the $x$ direction.

A Bloch state is in the form of 
\begin{equation}
    \psi_{k \alpha} (x) = \ee^{\ii k x} u_{n \alpha} (x),
\end{equation}
which in the bra-ket notation is 
\begin{equation}
    \ket{\psi_{k \alpha}} = \ee^{\ii k \hat{x}} \ket{u_{k n}}.
\end{equation}
The Wannier function is 
\begin{equation}
    \ket{n \alpha} = \frac{1}{\sqrt{N}} \sum_k \ee^{\ii k (\hat{x} - x_n)} \ket{u_{k n}}
\end{equation}
When the polarization is uniform enough, we have
\begin{equation}
    \begin{aligned}
        P_x &= \sum_{\alpha < 0} \expval{\hat{x}}{n=0, \alpha} = \frac{1}{N} \sum_{k, k', \alpha < 0} \mel{u_{k' \alpha}}{\ee^{- \ii k' \hat{x}} \hat{x} \ee^{\ii k \hat{x}} }{u_{k\alpha}} \\
        &= % TODO
        \frac{\ii}{N} \sum_{k, \alpha < 0} \mel{u_{k \alpha}}{\pdv{k}}{u_{k \alpha}} = \sum_{\alpha < 0} \int \frac{\dd{k}}{2\pi} \ii \mel{u_{k \alpha}}{\pdv{k}}{u_{k \alpha}} .
    \end{aligned}
\end{equation}
This is just the integral 
\begin{equation}
    P_x = \int \frac{\dd{k}}{2\pi} A_x
\end{equation}
of the Berry phase
\begin{equation}
    A_x = \ii \sum_{\alpha < 0} \mel{u_{k \alpha}}{\pdv{k}}{u_{k \alpha}}.
\end{equation}
The polarization $P_x$ is only well defined mod 1, because we can do a gauge transformation
\begin{equation}
    \ket{u_{k \alpha}} \to \ee^{\ii \theta(k)} \ket{u_{k \alpha}},
\end{equation}
and 
\begin{equation}
    P_x \to P_x + \frac{\theta(2\pi) - \theta(0)}{2\pi}.
\end{equation}

Now we will find the chiral symmetry in class AIII requires 
\begin{equation}
    P_x = 0, \  \frac{1}{2} \text{ mod 1}
\end{equation}
for the SSH model. The eigenstate of $Q(k)$ can be found as 
\begin{equation}
    Q \pmqty{u_k \\ \pm q^*(k) u_k } = \pmqty{\pm q(k) q^*(k) u_k \\ q^*(k) u_k} = \pm \pmqty{u_k \\ \pm q^*(k) u_k }.
\end{equation}
Note that by selecting these states as the basis, we have already chosen a \emph{gauge} for $A_x$. 
Since $(u_k, - q^*(k) u_k)$ is the occupied state, we have 
\begin{equation}
    \begin{aligned}
        A_x(k) &= \ii \pmqty{u^*_k & - q(k) u_k^*} \pdv{k} \pmqty{u_k \\ - q^*(k) u_k}  \\
        &= \frac{\ii}{2} \trace (q(k) \partial_k q^*(k)) 
    \end{aligned}
\end{equation}

We find that though $Q$ is as flat as a trivial flat band model constructed by localized Wannier functions, 
by extracting information from the \emph{basis}, we are able to recover the topological classification obtained 
previously by the winding number.

\section{Class A insulators when $d=2$}

Now we consider the $d=2$ class A insulators. Examples of this class include IQHE and Chern insulators.
This time the physical quantity bearing the topological index is the \concept{Chern number}
\begin{equation}
    C_\text{L} = \int_\text{B.Z.} \frac{\dd{k_1} \dd{k_2}}{2 \pi} F_{ij}, \quad F_{ij} = \partial_i A_j - \partial_j A_i,
\end{equation}
which, according to the Kubo formula, is actually the transverse (i.e. Hall) conductivity. 
Here we define 
\begin{equation}
    \vb*{A} = \ii \sum_\alpha \mel{\vb*{k} \alpha}{\partial_{\vb*{k}}}{\vb*{k} \alpha} = \trace \vb*{A}_{\alpha \beta},
\end{equation}
where 
\begin{equation}
    \vb*{A}_{\alpha \beta} = \ii \mel{\vb*{k} \alpha}{\partial_{\vb*{k}}}{\vb*{k} \beta},
\end{equation}
which may be seen as a non-Abelian gauge field defined in the momentum space. 
The Chern-number, then, can be seen as 
\begin{equation}
    C_\text{L} = P_x(k_y = 2\pi) - P_x(k_y = 0).
\end{equation}

Now we consider the following Hamiltonian:
\begin{equation}
    H(\vb*{k}) = k_x \sigma^x + k_y \sigma^y + m \sigma^z.
    \label{eq:starting-dirac-model}
\end{equation}
The spectrum is obviously 
\begin{equation}
    E_{\vb*{k}} = \pm \sqrt{\vb*{k}^2 + m^2},
\end{equation}
giving us a Dirac cone at the $\vb*{k} = 0$ point. To keep the Dirac cone while make the $\vb*{k} \to \infty$ limit of $H$ a homogeneous one, we modify \eqref{eq:starting-dirac-model} to
\begin{equation}
    H(\vb*{k}) = k_x \sigma^x + k_y \sigma^y + \left(m - \frac{1}{2} k_x^2 - \frac{1}{2} k_y^2 \right) \sigma^z
    \label{eq:uniform-single-dirac-cone}
\end{equation}
Moreover, since gap opening and closing is only possible around the Dirac cone, the band structure away from 
the Dirac cone can be distorted and we can work with
\begin{equation}
    H(\vb*{k}) = \sin k_x \sigma^x + \sin k_y \sigma^y + (m + 2 - \cos k_x - \cos k_y) \sigma^z.
    \label{eq:sin-k-model}
\end{equation}
This is clearly a tight-binding model.

\eqref{eq:sin-k-model} has four Dirac cones in one Brillouin zone. Note that since we are considering models 
on a periodic lattice, so \eqref{eq:starting-dirac-model} is not a well-defined model. If we decide to ``copy''
Dirac cones of \eqref{eq:starting-dirac-model} to make a square lattice, what we get is just \eqref{eq:sin-k-model}.

The $\abs{m} \to \infty$ limits are obviously the trivial phases. On the other hand, it can be verified that 
when $m=4$, the $k_x = k_y = \pi$ gap is closed; when $m$ is reduced to 2, the $(0, \pi)$ and $(\pi 0)$ gaps 
are closed, and when $m$ is reduced to 0, the $(0, 0)$ gap is closed.

\begin{figure}
    \centering
    

\tikzset{every picture/.style={line width=0.75pt}} %set default line width to 0.75pt        

\begin{tikzpicture}[x=0.75pt,y=0.75pt,yscale=-1,xscale=1]
%uncomment if require: \path (0,300); %set diagram left start at 0, and has height of 300

%Shape: Ellipse [id:dp6127707106898874] 
\draw  [color={rgb, 255:red, 0; green, 0; blue, 0 }  ,draw opacity=0.2 ] (100,183.56) .. controls (100,144.04) and (158.2,112) .. (230,112) .. controls (301.8,112) and (360,144.04) .. (360,183.56) .. controls (360,223.08) and (301.8,255.11) .. (230,255.11) .. controls (158.2,255.11) and (100,223.08) .. (100,183.56) -- cycle ;
%Straight Lines [id:da5498048117239558] 
\draw [color={rgb, 255:red, 208; green, 2; blue, 27 }  ,draw opacity=1 ]   (230,172.56) -- (230,198.56) ;
\draw [shift={(230,200.56)}, rotate = 270] [fill={rgb, 255:red, 208; green, 2; blue, 27 }  ,fill opacity=1 ][line width=0.08]  [draw opacity=0] (12,-3) -- (0,0) -- (12,3) -- cycle    ;
%Straight Lines [id:da8202921841687136] 
\draw [color={rgb, 255:red, 74; green, 144; blue, 226 }  ,draw opacity=1 ]   (360,197.56) -- (360,171.56) ;
\draw [shift={(360,169.56)}, rotate = 90] [fill={rgb, 255:red, 74; green, 144; blue, 226 }  ,fill opacity=1 ][line width=0.08]  [draw opacity=0] (12,-3) -- (0,0) -- (12,3) -- cycle    ;
%Straight Lines [id:da9640331300182798] 
\draw [color={rgb, 255:red, 74; green, 144; blue, 226 }  ,draw opacity=1 ]   (343,160.56) -- (343,134.56) ;
\draw [shift={(343,132.56)}, rotate = 90] [fill={rgb, 255:red, 74; green, 144; blue, 226 }  ,fill opacity=1 ][line width=0.08]  [draw opacity=0] (12,-3) -- (0,0) -- (12,3) -- cycle    ;
%Straight Lines [id:da03242442582564298] 
\draw [color={rgb, 255:red, 74; green, 144; blue, 226 }  ,draw opacity=1 ]   (305,136.56) -- (305,110.56) ;
\draw [shift={(305,108.56)}, rotate = 90] [fill={rgb, 255:red, 74; green, 144; blue, 226 }  ,fill opacity=1 ][line width=0.08]  [draw opacity=0] (12,-3) -- (0,0) -- (12,3) -- cycle    ;
%Straight Lines [id:da22373788973980524] 
\draw [color={rgb, 255:red, 74; green, 144; blue, 226 }  ,draw opacity=1 ]   (262,127.56) -- (262,101.56) ;
\draw [shift={(262,99.56)}, rotate = 90] [fill={rgb, 255:red, 74; green, 144; blue, 226 }  ,fill opacity=1 ][line width=0.08]  [draw opacity=0] (12,-3) -- (0,0) -- (12,3) -- cycle    ;
%Straight Lines [id:da7619994864927513] 
\draw [color={rgb, 255:red, 155; green, 155; blue, 155 }  ,draw opacity=1 ] [dash pattern={on 4.5pt off 4.5pt}]  (230,183.56) -- (389.22,101.03) ;
\draw [shift={(391,100.11)}, rotate = 152.6] [fill={rgb, 255:red, 155; green, 155; blue, 155 }  ,fill opacity=1 ][line width=0.08]  [draw opacity=0] (12,-3) -- (0,0) -- (12,3) -- cycle    ;
%Shape: Ellipse [id:dp3388918869189159] 
\draw  [color={rgb, 255:red, 0; green, 0; blue, 0 }  ,draw opacity=0.2 ][dash pattern={on 4.5pt off 4.5pt}] (174.25,183.56) .. controls (174.25,166.61) and (199.21,152.87) .. (230,152.87) .. controls (260.79,152.87) and (285.75,166.61) .. (285.75,183.56) .. controls (285.75,200.51) and (260.79,214.24) .. (230,214.24) .. controls (199.21,214.24) and (174.25,200.51) .. (174.25,183.56) -- cycle ;
%Straight Lines [id:da5105977529379926] 
\draw [color={rgb, 255:red, 0; green, 0; blue, 0 }  ,draw opacity=1 ]   (300,183.56) -- (267,183.56) ;
\draw [shift={(265,183.56)}, rotate = 360] [fill={rgb, 255:red, 0; green, 0; blue, 0 }  ,fill opacity=1 ][line width=0.08]  [draw opacity=0] (12,-3) -- (0,0) -- (12,3) -- cycle    ;
%Straight Lines [id:da00855087001959598] 
\draw [color={rgb, 255:red, 0; green, 0; blue, 0 }  ,draw opacity=1 ]   (163,183.56) -- (194,183.56) ;
\draw [shift={(196,183.56)}, rotate = 180] [fill={rgb, 255:red, 0; green, 0; blue, 0 }  ,fill opacity=1 ][line width=0.08]  [draw opacity=0] (12,-3) -- (0,0) -- (12,3) -- cycle    ;
%Straight Lines [id:da24612971832884312] 
\draw [color={rgb, 255:red, 0; green, 0; blue, 0 }  ,draw opacity=1 ]   (294,156.11) -- (264.78,171.2) ;
\draw [shift={(263,172.11)}, rotate = 332.7] [fill={rgb, 255:red, 0; green, 0; blue, 0 }  ,fill opacity=1 ][line width=0.08]  [draw opacity=0] (12,-3) -- (0,0) -- (12,3) -- cycle    ;
%Straight Lines [id:da6500494018973344] 
\draw [color={rgb, 255:red, 0; green, 0; blue, 0 }  ,draw opacity=1 ]   (177,215.11) -- (204.25,200.08) ;
\draw [shift={(206,199.11)}, rotate = 151.11] [fill={rgb, 255:red, 0; green, 0; blue, 0 }  ,fill opacity=1 ][line width=0.08]  [draw opacity=0] (12,-3) -- (0,0) -- (12,3) -- cycle    ;
%Straight Lines [id:da5631657523776881] 
\draw [color={rgb, 255:red, 0; green, 0; blue, 0 }  ,draw opacity=1 ]   (279,214.11) -- (253.66,197.22) ;
\draw [shift={(252,196.11)}, rotate = 33.69] [fill={rgb, 255:red, 0; green, 0; blue, 0 }  ,fill opacity=1 ][line width=0.08]  [draw opacity=0] (12,-3) -- (0,0) -- (12,3) -- cycle    ;
%Straight Lines [id:da5663091296229177] 
\draw [color={rgb, 255:red, 0; green, 0; blue, 0 }  ,draw opacity=1 ]   (241,227.11) -- (230.77,202.41) ;
\draw [shift={(230,200.56)}, rotate = 67.5] [fill={rgb, 255:red, 0; green, 0; blue, 0 }  ,fill opacity=1 ][line width=0.08]  [draw opacity=0] (12,-3) -- (0,0) -- (12,3) -- cycle    ;

% Text Node
\draw (362,121.4) node [anchor=north west][inner sep=0.75pt]  [color={rgb, 255:red, 74; green, 144; blue, 226 }  ,opacity=1 ]  {$\boldsymbol{k}\rightarrow \infty $};
% Text Node
\draw (206,157.4) node [anchor=north west][inner sep=0.75pt]  [color={rgb, 255:red, 208; green, 2; blue, 27 }  ,opacity=1 ]  {$\boldsymbol{k} =0$};
% Text Node
\draw (100,156.4) node [anchor=north west][inner sep=0.75pt]    {$1\gg |\boldsymbol{k} |\gg m$};


\end{tikzpicture}

    \caption{The ground state Bloch vector of \eqref{eq:uniform-single-dirac-cone}}
    \label{fig:ground-state-dirac-cone}
\end{figure}

Now we calculate the Chern number. First we consider the $0 < m \ll 1$ case. At each $\vb*{k}$ point, 
we can diagonalize $H(\vb*{k})$ and plot the corresponding states on the Bloch sphere. 
For simplicity, we go back to the single Dirac cone model \eqref{eq:uniform-single-dirac-cone}. 
The Bloch vector plot can be found in \prettyref{fig:ground-state-dirac-cone}.
When $\vb*{k} = 0$, we have $H \approx m \sigma^z$, and with half filling, the Bloch vector is $- \vu{e}_z$.
On the other hand 
%TODO
So we find for each Dirac cone, when $m$ changes its sign, $\Delta C_\text{L} = 1$.

We then go back to \eqref{eq:sin-k-model}.

The first proposed Chern insulator is the \concept{Haldane model}. 

\section{2D topological insulators}

Topological insulators are actually the first type of topological band systems. The topic originated from 
\concept{quantum spin Hall effect (QSH)}. Standard IQHE need a magnetic field to break the time reversal 
symmetry to gain non-trivial Chern-number, which is not convenient experimentally. One possible solution 
is to use a $T=-1$ time reversal symmetry, 
% TODO: the relation between QSH and TI

\section{Realization of topological band models in each class in the ten-fold way}

In this section we discuss the \concept{ABS construction}, which gives at least one topological non-trivial 
examples in each symmetry class in the ten-fold way \cite{stone2010symmetries}. Mathematically, the energy 
band structure of an insulator/superconductor is a vector bundle, which may be classified by the \emph{K-theory},
which can further be constructed by the ABS construction. 

\subsection{Classification of class A}

A \concept{vector bundle} consists of a topological space $X$ (the \concept{base space}) and a total space $E$
% TODO

$K(X)$ is a topological invariant of $X$. In condensed matter physics, the $X$ in a $d$-dimensional is 
$\mathbb{T}^d$, but usually people work with $K(S^d)$, not $K(\mathbb{T}^d)$, because the latter contains 
too much information. For example, in 3D topological insulators, we have 
\begin{equation}
    K(\mathbb{T}^3) = 3 \mathbb{Z},
\end{equation}
which actually includes so-called topological non-trivial states which have non-trivial topological 
indices on low-dimensional panes or lines, but no 3-dimensional topological indices. By topological
deformation, we can drive away these low-dimensional states to the boundary of the Brillouin zone, 
and since these states can be studied as lower dimensional topological systems, by requiring 
the boundary of the Brillouin zone to be trivial, we can get rid of these so-called 
\concept{weak topological states}, and requiring the boundary of the Brillouin zone to be trivial
just means to consider the boundary as a single point which can't house any topological non-trivial
band features, so hence we work with $S^d$.

Now it can be proved that each representative of $K^{-d}(\text{point})$ may take the following form:
\begin{equation}
    H(\vb*{k}) = \sum_{i=1}^d k_i \Gamma_i + \Gamma_0 \left(1 - \frac{1}{2} \vb*{k}^2 \right),
\end{equation}
where 
\begin{equation}
    \acomm{\Gamma_i}{\Gamma_j} = 2 \delta_{ij}, \quad \Gamma_i = \Gamma_i^\dagger , \quad \Gamma_i^2 = 1,
\end{equation}
so $\{\Gamma_\mu \}$ is a representation of the $(d+1)$-dimensional complex Clifford algebra. Classification of 
possible topological band models is just classification of these representations.

\subsection{Classification of class AIII}

For class AIII, we have an additional requirement: $\comm{H}{S} = 0$, and we immediately find that 
if we add $S$ into $\{ \Gamma_\mu \}$, we get a $(d+2)$-dimensional complex Clifford algebra.

\subsection{Classification of class D}

In this symmetry class, the particle-hole symmetry means the $\{\Gamma_i\}$ matrices are antisymmetric real ones,

% TODO
This also opens door for non-trivial topological behavior \emph{in the real space}: it's possible to define 
a \emph{local} band structure for a material of which the Hamiltonian has microscopically very slow spatial change,
and in this way, just like the model we use when deriving the Bloch oscillation, we can talk about 
$H(\vb*{k}, \vb*{r})$. In principle, the interplay between the real space and the momentum space can 

why class A and class AIII don't have real space topological behavior?

\section{Classification of topological non-trivial Hamiltonians}

In this section we consider how to classify Hamiltonians into symmetry classes that protect the topological 
properties. This is of course much more complicated than the previous section, and still lacks a unified theory. 
For reference, see \cite{RevModPhys.88.035005}.

We calculate the corresponding field strength tensor of the Berry connection:
\begin{equation}
    \mathcal{F} = \dd{\mathcal{A}} + \mathcal{A} \wedge \mathcal{A},
\end{equation}
and we can define two topological indices from it. The first is the Chern-number. For example, when $d + D = 2$, 
we have 
\begin{equation}
    \text{Ch}_2 = \frac{\ii}{2\pi} \int \dd[d]{\vb*{k}} \dd[D]{\vb*{r}} \trace \mathcal{F}. 
\end{equation}
The berry connection $\mathcal{A}$ can be seen as a 1-form, and therefore $\mathcal{F}$ is a 2-form.

\bibliographystyle{plain}
\bibliography{symmetry-topological-band}

\end{document}