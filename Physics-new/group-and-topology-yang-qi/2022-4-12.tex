\documentclass[hyperref, a4paper]{article}

\usepackage{geometry}
\usepackage{float}
\usepackage{titling}
\usepackage{titlesec}
% No longer needed, since we will use enumitem package
% \usepackage{paralist}
\usepackage{enumitem}
\usepackage{footnote}
\usepackage{enumerate}
\usepackage{amsmath, amssymb, amsthm}
\usepackage{mathtools}
\usepackage{bbm}
\usepackage{cite}
\usepackage{graphicx}
\usepackage{subfigure}
\usepackage{physics}
\usepackage{tensor}
\usepackage{siunitx}
\usepackage{booktabs}
\usepackage[version=4]{mhchem}
\usepackage{tikz}
\usepackage{xcolor}
\usepackage{listings}
\usepackage{autobreak}
\usepackage[ruled, vlined, linesnumbered]{algorithm2e}
\usepackage{nameref,zref-xr}
\zxrsetup{toltxlabel}
\zexternaldocument*[solid-]{../solid/solid}[solid.pdf]
\zexternaldocument*[last-]{./2022-3-15}[2022-3-15.pdf]
\usepackage[colorlinks,unicode]{hyperref} % , linkcolor=black, anchorcolor=black, citecolor=black, urlcolor=black, filecolor=black
\usepackage[most]{tcolorbox}
\usepackage{prettyref}

% Page style
\geometry{left=3.18cm,right=3.18cm,top=2.54cm,bottom=2.54cm}
\titlespacing{\paragraph}{0pt}{1pt}{10pt}[20pt]
\setlength{\droptitle}{-5em}

% More compact lists 
% \setlist[itemize]{itemindent=17pt, leftmargin=1pt}

% Math operators
\DeclareMathOperator{\timeorder}{T}
\DeclareMathOperator{\diag}{diag}
\DeclareMathOperator{\legpoly}{P}
\DeclareMathOperator{\primevalue}{P}
\DeclareMathOperator{\sgn}{sgn}
\newcommand*{\ii}{\mathrm{i}}
\newcommand*{\ee}{\mathrm{e}}
\newcommand*{\const}{\mathrm{const}}
\newcommand*{\suchthat}{\quad \text{s.t.} \quad}
\newcommand*{\argmin}{\arg\min}
\newcommand*{\argmax}{\arg\max}
\newcommand*{\normalorder}[1]{: #1 :}
\newcommand*{\pair}[1]{\langle #1 \rangle}
\newcommand*{\fd}[1]{\mathcal{D} #1}
\DeclareMathOperator{\bigO}{\mathcal{O}}
\DeclareMathOperator{\id}{id}

% TikZ setting
\usetikzlibrary{decorations.text}
\usetikzlibrary{arrows,shapes,positioning}
\usetikzlibrary{arrows.meta}
\usetikzlibrary{decorations.markings}
\tikzstyle arrowstyle=[scale=1]
\tikzstyle directed=[postaction={decorate,decoration={markings,
    mark=at position .5 with {\arrow[arrowstyle]{stealth}}}}]
\tikzstyle ray=[directed, thick]
\tikzstyle dot=[anchor=base,fill,circle,inner sep=1pt]

% Algorithm setting
% Julia-style code
\SetKwIF{If}{ElseIf}{Else}{if}{}{elseif}{else}{end}
\SetKwFor{For}{for}{}{end}
\SetKwFor{While}{while}{}{end}
\SetKwProg{Function}{function}{}{end}
\SetArgSty{textnormal}

\newcommand*{\concept}[1]{{\textbf{#1}}}

\DeclareMathOperator{\im}{im}

% Embedded codes
\lstset{basicstyle=\ttfamily,
  showstringspaces=false,
  commentstyle=\color{gray},
  keywordstyle=\color{blue}
}

\newcommand{\soliddoc}{\href{../solid/solid.pdf}{this note}}
\newcommand{\lastlec}{\href{./2022-3-15.pdf}{the last lecture}}

% Color boxes
\tcbuselibrary{skins, breakable, theorems}
\newtcbtheorem[number within=section]{warning}{Warning}%
  {colback=orange!5,colframe=orange!65,fonttitle=\bfseries, breakable}{warn}
\newtcbtheorem[number within=section]{note}{Note}%
  {colback=green!5,colframe=green!65,fonttitle=\bfseries, breakable}{note}
\newtcbtheorem[number within=section]{info}{Info}%
  {colback=blue!5,colframe=blue!65,fonttitle=\bfseries, breakable}{info}

\title{Prof. Yang Qi on topological classification of gapped free fermion models}
\author{Jinyuan Wu}

\begin{document}

\maketitle

Gapped systems satisfy 
the adiabatic theorem, so with small perturbation on the Hamiltonian, we can transform the $n$th eigenstate 
of one Hamiltonian into the $n$th eigenstate of another Hamiltonian, and then go back, so in this sense, 
these states are \emph{equivalent}. This gives a natural criterion of Hamiltonian classification, each 
equivalence class of which can be said as a homotopy equivalence class, and in this sense, 
we achieve a \emph{topological classification} of the Hamiltonians.

For free fermions, perturbation of the Hamiltonian is just perturbation of the band structure, so 
topological classification of Hamiltonians reduces to topological classification of band structures:
two band structures that can be connected by a smooth deformation which doesn't cross the Fermi surface
(or otherwise when a band crosses the Fermi surface, the system becomes a metal and is gapless). 

In the following lectures we discuss topological classification of gapped free fermions.
These systems include insulators and the electron-like excitations in superconductors (as well as other 
systems with certain kind of electron condensation).

It's possible that a topological equivalence class involves several separate subclasses, each of which 
has a different symmetry group, and a smooth transformation from one class into another inevitably 
breaks or adds some symmetry. In this case, the subclasses are said to be \emph{symmetry protected} --
as long as the symmetry is present, we are sure that perturbation on the Hamiltonian won't get 
the system out of its subclass.

\section{Antiunitary symmetry of free fermions}

Consider the following second quantized Hamiltonian:
\begin{equation}
    \hat{H} = \sum_{\alpha, \beta} c^\dagger_\alpha H_{\alpha \beta} c_\beta,
    \label{eq:second-quantization}
\end{equation}
where operators with $\hat{}$ are second quantized operators. In this section, we consider how 
antiunitary symmetries acts on \eqref{eq:second-quantization}. The action of a unitary group element can be 
written as 
\begin{equation}
    g \cdot \ket{\alpha} = \ket{\beta} \varphi(g)_{\beta \alpha}, \quad 
    g \cdot \bra{\alpha} = \varphi(g)^\dagger_{\alpha \beta} \bra{\beta}, 
\end{equation}
and 
\begin{equation}
    g \cdot c^\dagger_\alpha = g c^\dagger_\alpha g^{-1} = c^\dagger_\beta \varphi(g)_{\beta \alpha}, \quad 
    g \cdot c_\alpha = g c_\alpha g^{-1} = \varphi(g)^\dagger_{\alpha \beta} c_\beta.
\end{equation}
So the Hamiltonian transforms as 
\begin{equation}
    g \cdot \hat{H} = c^\dagger_{\alpha'} \varphi(g)_{\alpha' \alpha} H_{\alpha \beta} \varphi(g)^\dagger_{\beta \beta'} c_{\beta'}.
\end{equation}
Of course, this is just in the form of basis transition. If the Hamiltonian has the symmetry of $g$, we have 
\begin{equation}
    H_{\alpha' \beta'} = \varphi(g)_{\alpha' \alpha} H_{\alpha \beta} \varphi(g)^\dagger_{\beta \beta'}.
\end{equation}
Suppose $\{\ket{\alpha}\}$ is real with regard of the time reversal operation $\mathcal{T} = T \mathcal{K}$.
In this case, the time reversal symmetry doesn't act on the basis, but $\mathcal{K}$ acts on $H_{\alpha \beta}$
and adds a star, so if the system has time reversal symmetry, we have 
\begin{equation}
    T H T^{-1} = {H}^*.
    \label{eq:time-reversal-first-quantization}
\end{equation}
Note that here $H$ is the \emph{first quantized} Hamiltonian, i.e. the coefficient matrix 
in \eqref{eq:second-quantization}. The operation $T$ is also the \emph{unitary} transformation associated 
to $\mathcal{K}$, instead of the full \emph{second quantized} $\mathcal{T}$. We work with first quantization
because it's easier: we don't need to actually work with complex conjugate of creation and annihilation operators.

Now we move to the ``real'' particle-hole symmetry in an insulator. Consider 
\begin{equation}
    \hat{H} = - \sum_{\pair{\vb*{i}, \vb*{j}}} (t_{\vb*{i} \vb*{j}} c^\dagger_{\vb*{i}} c_{\vb*{j}} + \text{h.c.}).
\end{equation}
We use $\mathcal{C}$ to denote the particle-hole transformation from $c$ to $c^\dagger$, i.e.
\begin{equation}
    \mathcal{C} c_{\vb*{i}} \mathcal{C}^{-1} = c_{\vb*{i}}^\dagger, \quad \mathcal{C} c_{\vb*{i}}^\dagger \mathcal{C}^{-1} = c_{\vb*{i}},
\end{equation}
then we have 
\begin{equation}
    \mathcal{C} \hat{H} \mathcal{C}^{-1} = - \sum_{\pair{\vb*{i}, \vb*{j}}} (t_{\vb*{i} \vb*{j}} c_{\vb*{i}} c^\dagger_{\vb*{j}} + \text{h.c.}) = \sum_{\pair{\vb*{i}, \vb*{j}}} (t_{\vb*{j} \vb*{i}}^* c^\dagger_{\vb*{j}} c_{\vb*{i}} + \text{h.c.}) = - \hat{H}^* = - \hat{H}^\top.
\end{equation}
We know the spectrum of the tight-binding model is a cosine curve, so the minus sign comes as is expected, meaning flipping the spectrum with the Fermi surface as a mirror, while the transpose 
operation comes from flipping $\ket{\vb*{i}}$ to $\bra{\vb*{i}}$. Here we need to keep in mind a strange property 
of particle-hole transformation. $\mathcal{C}$ should be \emph{unitary} in the second quantization formalism, because 
we want 
\begin{equation}
    c_{\vb*{k}}^\dagger \stackrel{\mathcal{C}}{\longrightarrow} c_{- \vb*{k}}
\end{equation}
to keep momentum conservation, and if $\mathcal{C}$ is unitary, it doesn't act on a complex factor, so we have 
\[
    \mathcal{C} \cdot c_{\vb*{k}}^\dagger = \mathcal{C} \cdot \frac{1}{\sqrt{N}} \sum_{\vb*{i}} \ee^{- \ii \vb*{k} \cdot \vb*{r}_{\vb*{i}}} c^\dagger_{\vb*{i}} = \sum_{\vb*{i}} \ee^{- \ii \vb*{k} \cdot \vb*{r}_{\vb*{i}}} c_{\vb*{i}} = c_{- \vb*{k}},
\]
which is exactly what we want. However, the first quantization version of $\mathcal{C}$ maps a single-electron 
wave function in the single electron Hilbert space to the \emph{dual} space, and a map $\mathcal{H} \to \mathcal{H}^*$
should be \emph{antiunitary} to keep naturalness.

The action of $\mathcal{C}$ on an arbitrary basis is 
\begin{equation}
    \mathcal{C} c_{\alpha} \mathcal{C}^{-1} = c_\beta^\dagger C_{\beta \alpha}, \quad 
    \mathcal{C} c^\dagger_{\alpha} \mathcal{C}^{-1} = C^{-1}_{\alpha \beta} c_\beta,
    \label{eq:arbitrary-particle-hole}
\end{equation}
and the second quantized Hamiltonian under \eqref{eq:arbitrary-particle-hole} transforms as 
\begin{equation}
    \begin{aligned}
        \mathcal{C} \hat{H} \mathcal{C}^{-1} &= \mathcal{C} c_\alpha^\dagger \mathcal{C}^{-1} H_{\alpha \beta} \mathcal{C} c_\beta \mathcal{C}^{-1} = C^{-1}_{\alpha \alpha'} c_{\alpha'} \underbrace{H_{\alpha \beta}}_{H_{\beta \alpha}^*} c^\dagger_{\beta'} C_{\beta' \beta} \\
        &= - c^\dagger_{\beta'} C_{\beta' \beta} H^*_{\beta \alpha} C^{-1}_{\alpha \alpha'} c_{\alpha'} = - c^\dagger_{\beta'} (CH^* C^{-1})_{\beta' \alpha'} c_{\alpha'}.
    \end{aligned}
\end{equation}
So the particle-hole symmetry condition for the first quantized Hamiltonian is 
\begin{equation}
    H = - C H^* C^{-1}.
    \label{eq:charge-symmetry-first-quantization}
\end{equation}

Despite their weirdness, $\mathcal{T}$ and $\mathcal{C}$ can be fully demonstrated by constraints on 
the coefficient matrix $H$ (i.e. first quantized Hamiltonian) in \eqref{eq:second-quantization}.
Classification of free fermion systems is therefore classification of these matrices. 

$\mathcal{T}$ and $\mathcal{C}$ have non-trivial multiplication relation: since $\mathcal{T}$ is antiunitary
in both first and second quantization formulation, and $\mathcal{C}$ is unitary in second quantization but 
antiunitary in first quantization, 
\begin{equation}
    \mathcal{S} = \mathcal{T} \mathcal{C}
\end{equation} 
is unitary in first quantization but antiunitary in second quantization. We call it \concept{chiral symmetry}
to keep consistent with the definition in particle physics.
The condition that a second quantized Hamiltonian has chiral symmetry can be verified similar to \eqref{eq:time-reversal-first-quantization} and \eqref{eq:charge-symmetry-first-quantization}, and we have 
\begin{equation}
    S H S^{-1} =  - H.
    \label{eq:chiral-first-quantization}
\end{equation}

In the momentum representation, $\mathcal{C}$ is bond to change the momentum in the $\vb*{k} \to - \vb*{k}$ 
way, because of momentum conservation. The rest of labels (band index, spin, etc.) are mixed together with a finite dimensional matrix.
So suppose $H(\vb*{k})$ is the block in $H$ with momentum $\vb*{k}$, and we have 
\begin{equation}
    T H(\vb*{k}) T^{-1} = H(- \vb*{k})^*.
\end{equation}
where $C$ is the block in the $C$ matrix in \eqref{eq:time-reversal-first-quantization} that mixes labels other than $\vb*{k}$.
Similarly from \eqref{eq:charge-symmetry-first-quantization}, we have 
\begin{equation}
    C H(- \vb*{k})^* C^{-1} = - H(\vb*{k}).
\end{equation}
The chiral symmetry doesn't change the momentum, since it's the multiplication of $\mathcal{T}$ and $\mathcal{C}$,
so from \eqref{eq:chiral-first-quantization} we have 
\begin{equation}
    S H(\vb*{k}) S^{-1} =  - H(\vb*{k}).
\end{equation}

Now in a model with multiple bands, it's possible that 
there are several different particle-hole symmetry: it may be possible that band A's particles are equivalent 
to \emph{both} band B and band C's holes. But in this case, band B and band C must coincide and there is 
a symmetry switching B and C, so in the end we only need to define one particle-hole symmetry as the generator:
the rest of the particle-hole symmetries can be automatically generated using composition.

\section{Particle-hole symmetry in superconductors}

The following BCS model of superconductivity:
\begin{equation}
    H = \sum_{\vb*{k}, \alpha} \xi_{\vb*{k}} c^\dagger_{\vb*{k} \alpha} c_{\vb*{k} \alpha} - V \sum_{\vb*{k}, \vb*{k}', \vb*{q}, \alpha, \beta} c^\dagger_{\vb*{k}' - \vb*{q}, \alpha} c^\dagger_{\vb*{k} + \vb*{q}, \beta} c_{\vb*{k} \beta} c_{\vb*{k}' \alpha},
    \label{eq:bcs}
\end{equation}
can be turned into 
\begin{equation}
    H = \sum_{\vb*{k}, \alpha} \xi_{\vb*{k}} c^\dagger_{\vb*{k} \alpha} c_{\vb*{k} \alpha} 
    + \sum_{\vb*{k}} (\Delta c^\dagger_{- \vb*{k} \downarrow} c^\dagger_{\vb*{k} \uparrow} + \Delta^* c_{\vb*{k} \uparrow} c_{- \vb*{k} \downarrow}),
    \label{eq:bdg}
\end{equation}
where $\Delta$ is the BCS order parameter. \eqref{eq:bdg} is called the \concept{BdG Hamiltonian}, 
which describes the spectrum of (electron-like) quasiparticles in a BCS superconductor. Note that 
the order parameter in \eqref{eq:bdg} has quantum fluctuation, but what we are discussing here is 
the \emph{topological band behavior} of fermions, so ignoring the fluctuation of $\Delta$ makes sense.
It's possible that the fluctuation of $\Delta$ destroys the ordinary BCS order, but it's not the case 
in 3D. Note also that our current approach -- ignoring the fluctuation of $\Delta$, i.e. ignoring the 
many-body effect introduced by electron interaction -- is a non-interaction limit of the general theory 
of topological classification with interaction. 

In 1D and 2D, Mermin–Wagner theorem means the 
effective theory about fermionic quasiparticles of \eqref{eq:bcs} is not \eqref{eq:bdg}, because 
$U(1)$ symmetry -- a continuous symmetry -- can't be broken here. But we can always use a 
3D bulk state to ``induce'' a low-dimensional superconducting phase, which has electron pairing anyway
and can be described by \eqref{eq:bdg}, though this time \eqref{eq:bdg} has nothing to do with BCS mechanism.
So henceforth we will work with a free-fermionic model like \eqref{eq:bdg} and ignore what induces superconductivity
pairing. 

In such a Hamiltonian, the existence of a pairing channel means we have terms like $c_\alpha c_\beta$,
and therefore the Hamiltonian can't be recast into \eqref{eq:second-quantization} from which we can extract 
the first quantized Hamiltonian $H$. The solution is a classical procedure in BCS theory: \eqref{eq:bdg} can be rephrased into 
\begin{equation}
    H = \sum_{\vb*{k}} \Psi_{\vb*{k}}^\dagger \pmqty{ \xi_{\vb*{k}} & - \Delta \\ - \Delta^* & -\xi_{\vb*{k}} } \Psi_{\vb*{k}}, \quad \Psi_{\vb*{k}} = \pmqty{c_{\vb*{k} \uparrow} \\ c^\dagger_{- \vb*{k}, \downarrow}}.
\end{equation}
This Hamiltonian has a particle-hole symmetry. Because $\xi_{\vb*{k}} = \xi_{- \vb*{k}}$, we have 
\begin{equation}
    \pmqty{0 & 1 \\ -1 & 0} \pmqty{\xi_{\vb*{k}} & - \Delta \\ - \Delta^* & - \xi_{\vb*{k}}} \pmqty{0 & -1 \\ 1 & 0} = - \pmqty{\xi_{- \vb*{k}} & - \Delta \\ - \Delta^* & - \xi_{- \vb*{k}}}^*.
\end{equation}
But the particle-hole symmetry is actually \emph{redundant}. BdG Hamiltonians without this symmetry aren't 
qualified theories of superconductors. The point here is that suppose there are $d$ kind of fermions (i.e. 
possible values of labels other than the momentum), then the existence of $c_\alpha c_\beta$ terms means 
$\Psi_{\vb*{k}}$ should be $2d$ dimensional to include all possible terms. (In the BCS case there are two 
kind of fermions and $\Psi$ is 2-dimensional, but we can always introduce $c_{\vb*{k} \downarrow}$ and 
$c^\dagger_{- \vb*{k}, \uparrow}$ components and assign zero matrix elements for them, and the resulting 
first-quantized Hamiltonian still has a particle-hole symmetry.) So the resulting $H$ matrix 
is $2d \times 2d$ dimensional, the same as the first quantized Hamiltonian of $2d$ kind of fermions,
but it describes $d$ kind of fermions anyway.
Since $d$ kinds of fermions are actually just holes for the rest $d$ kinds of fermions, 
there must be a particle-hole symmetry to make sure the spectrum of the first quantized Hamiltonian 
reflects this fact.

Sometimes the particle-hole symmetry comes from existing symmetries of the Hamiltonian. In the 
BCS case, it's easy to find that it's just the spin flipping symmetry.
Sometimes it's just a redundancy: the $C$ matrix acting on a state gives exactly the state itself.
Consider, for example, the following simplest case:
\begin{equation}
    H = \sum_{\vb*{k}} \xi_{\vb*{k}} c^\dagger_{\vb*{k}} c = \frac{1}{2} \sum_{\vb*{k}} \pmqty{c^\dagger_{\vb*{k}} & c_{\vb*{k}} } \pmqty{\dmat{\xi_{\vb*{k}}, - \xi_{\vb*{k}}}} \pmqty{c_{\vb*{k}} \\ c^\dagger_{\vb*{k}}} + \text{const},
\end{equation}
the first quantized Hamiltonian of which obviously has a particle-hole symmetry, which is based on the simple 
fact that (here the complex conjugate operation only applies to the elements of a matrix)
\[
    \pmqty{0 & 1 \\ 1 & 0} \pmqty{c_{\vb*{k}} \\ c^\dagger_{\vb*{k}}}^\dagger = \pmqty{c_{\vb*{k}} \\ c^\dagger_{\vb*{k}}}.
\]
A superconductor has \emph{more} symmetry than a similar insulator 
because its first quantized Hamiltonian has a particle-hole symmetry, while it has \emph{fewer} actual symmetry than a similar 
insulator because it breaks the $U(1)$ symmetry. 

\section{The ten-fold way}

So there are several possibility of these three weird symmetries. $\mathcal{T}$ and $\mathcal{C}$ symmetries 
may simply not exist, which we denote as ${T} = 0$ or ${C} = 0$. When $\mathcal{T}$ is present, for fermions 
we have $\mathcal{T}^2 = -1$ (fermion parity) and for bosons $\mathcal{T}^2 = 1$. It's also possible to 
redefine $\mathcal{T}$ in a fermionic system so that $\mathcal{T}'^2 = 1$ (see discussion around \eqref{last-eq:so-called-time-reversal} in \lastlec). We denote the $\mathcal{T}^2 = 1$ case as $T=1$, and the 
$\mathcal{T}^2 = -1$ case $T=-1$. For both fermionic and bosonic systems it can be verified that 
$\mathcal{C}^2 = 1$ (for multiple band systems, it may be the case that several bands A, B, C, etc. overlap
and $\mathcal{C}$ turn A into B, B into C, etc., and in this case $\mathcal{C}^2 = 1$ still works for the 
specific definition of $\mathcal{C}$ which only exchange ban labels in pairs), but for certain types of 
$SU(2)$ superconductors, it's useful to define $\mathcal{C}$ such that $\mathcal{C}^2 = -1$. 
So for $\mathcal{C}$ we also have $C = 0, \pm 1$. If both $\mathcal{C}$ and $\mathcal{T}$ are present, 
$\mathcal{S}$ is bond to be a symmetry, while if only one of them is present, $\mathcal{S}$ can't be 
a symmetry. This leaves us a final choice for systems where $T = 0, C = 0$ and $\mathcal{S}$ may be present 
or absent. So in the end, the symmetry of a free fermionic or bosonic system concerning $\mathcal{C}$, 
$\mathcal{T}$ and $\mathcal{S}$ has $3 \times 3 - 1 + 2 = 10$ possibilities, called \concept{the ten-fold way}. Here we list the notation for these symmetry classes and some examples \cite{Ryu_2010}.

\begin{itemize}
    \item Class A: $T = 0, \  C = 0, \ S = 0$. Without the particle-hole symmetry this can't be a superconductor,
     so it has to be an insulator, and therefore charge conservation holds. Example: Chern-insulators, IQHE.
    \item Class AIII: $T = 0, \  C = 0, \ S = 1$. Again, insulator only. Example: SSH model. 
    \item Class D: $T = 0,\  C = 1, \ S = 0$. This class is typically realized as a superconductor,
    which actually may not have any physical symmetry other than the fermion parity symmetry.
    Example: 1D Kitaev chain, 2D $(p + \ii p)$ superconductor.
    Of course, this class can also be implemented by an insulator with a real particle-hole symmetry.
    \item Class DIII: $T=-1, C = 1, S = 1$. Usual superconductors belong to this class, in which 
    $\mathcal{T}^2$ is the fermion parity symmetry. 1D topological superconductors are in this class.
    \item Class AII: $T = -1, C = 0, S = 0$. Without the particle-hole symmetry it has to be an insulator.
    Topological insulators in the ordinary sense are in this class.
    \item Class CII: $T = -1, C = -1, S = 1$. This is a strange symmetry class. $C = -1$ means the 
    symmetry class can be realized by a $SU(2)$ superconductor, but typically in this case we define $T = 1$.
    Another way to think of this symmetry class is to realize it as an insulator with charge-hole symmetry.
    \item Class C: $T = 0, C = -1, S = 0$. This can be realized by a $SU(2)$ superconductor.
    \item Class CI: $T = 1, C = -1, S = 0$. This can also be realized by a $SU(2)$ superconductor with 
    time reversal symmetry, in which we redefine $\mathcal{T}$ by attaching an $\ii$ factor to it so that $T = 1$.
    \item Class AI: $T = 1, C = 0, S = 0$. A spinless insulator can realize this symmetry class. For electronic
    systems without spin-orbital coupling, by considering states with the following form 
    \[
        \frac{1}{\sqrt{2}}(\ket{\vb*{k}, \uparrow} + \ket{\vb*{k}, \downarrow})
    \]
    we can construct a ``spinless'' subspace.
    \item Class BDI: $T = 1, C = 1, S = 0$. A typical realization is a spinless superconductor, like a $p$-wave 
    superconductor in which the pairing order parameter $\Delta$ is real.
\end{itemize}

\bibliographystyle{plain}
\bibliography{symmetry-topological-band}

\end{document}