\documentclass[hyperref, a4paper]{article}

\usepackage{geometry}
\usepackage{float}
\usepackage{titling}
\usepackage{titlesec}
% No longer needed, since we will use enumitem package
% \usepackage{paralist}
\usepackage{enumitem}
\usepackage{footnote}
\usepackage{enumerate}
\usepackage{amsmath, amssymb, amsthm}
\usepackage{mathtools}
\usepackage{bbm}
\usepackage{cite}
\usepackage{graphicx}
\usepackage{subfigure}
\usepackage{physics}
\usepackage{tensor}
\usepackage{siunitx}
\usepackage{booktabs}
\usepackage[version=4]{mhchem}
\usepackage{tikz}
\usepackage{xcolor}
\usepackage{listings}
\usepackage{autobreak}
\usepackage[ruled, vlined, linesnumbered]{algorithm2e}
\usepackage{nameref,zref-xr}
\zxrsetup{toltxlabel}
\zexternaldocument*[solid-]{../solid/solid}[solid.pdf]
\usepackage[colorlinks,unicode]{hyperref} % , linkcolor=black, anchorcolor=black, citecolor=black, urlcolor=black, filecolor=black
\usepackage[most]{tcolorbox}
\usepackage{prettyref}

% Page style
\geometry{left=3.18cm,right=3.18cm,top=2.54cm,bottom=2.54cm}
\titlespacing{\paragraph}{0pt}{1pt}{10pt}[20pt]
\setlength{\droptitle}{-5em}
\preauthor{\vspace{-10pt}\begin{center}}
\postauthor{\par\end{center}}

% More compact lists 
% \setlist[itemize]{itemindent=17pt, leftmargin=1pt}

% Math operators
\DeclareMathOperator{\timeorder}{T}
\DeclareMathOperator{\diag}{diag}
\DeclareMathOperator{\legpoly}{P}
\DeclareMathOperator{\primevalue}{P}
\DeclareMathOperator{\sgn}{sgn}
\newcommand*{\ii}{\mathrm{i}}
\newcommand*{\ee}{\mathrm{e}}
\newcommand*{\const}{\mathrm{const}}
\newcommand*{\suchthat}{\quad \text{s.t.} \quad}
\newcommand*{\argmin}{\arg\min}
\newcommand*{\argmax}{\arg\max}
\newcommand*{\normalorder}[1]{: #1 :}
\newcommand*{\pair}[1]{\langle #1 \rangle}
\newcommand*{\fd}[1]{\mathcal{D} #1}
\DeclareMathOperator{\bigO}{\mathcal{O}}
\DeclareMathOperator{\id}{id}

% TikZ setting
\usetikzlibrary{arrows,shapes,positioning}
\usetikzlibrary{arrows.meta}
\usetikzlibrary{decorations.markings}
\tikzstyle arrowstyle=[scale=1]
\tikzstyle directed=[postaction={decorate,decoration={markings,
    mark=at position .5 with {\arrow[arrowstyle]{stealth}}}}]
\tikzstyle ray=[directed, thick]
\tikzstyle dot=[anchor=base,fill,circle,inner sep=1pt]

% Algorithm setting
% Julia-style code
\SetKwIF{If}{ElseIf}{Else}{if}{}{elseif}{else}{end}
\SetKwFor{For}{for}{}{end}
\SetKwFor{While}{while}{}{end}
\SetKwProg{Function}{function}{}{end}
\SetArgSty{textnormal}

\newcommand*{\concept}[1]{{\textbf{#1}}}

\DeclareMathOperator{\im}{im}

% Embedded codes
\lstset{basicstyle=\ttfamily,
  showstringspaces=false,
  commentstyle=\color{gray},
  keywordstyle=\color{blue}
}

\newcommand{\soliddoc}{\href{../solid/solid.pdf}{this note}}

% Color boxes
\tcbuselibrary{skins, breakable, theorems}
\newtcbtheorem[number within=section]{warning}{Warning}%
  {colback=orange!5,colframe=orange!65,fonttitle=\bfseries, breakable}{warn}
\newtcbtheorem[number within=section]{note}{Note}%
  {colback=green!5,colframe=green!65,fonttitle=\bfseries, breakable}{note}
\newtcbtheorem[number within=section]{info}{Info}%
  {colback=blue!5,colframe=blue!65,fonttitle=\bfseries, breakable}{info}

\title{Prof. Yang Qi on topological classification of free fermion models}
\author{Jinyuan Wu}

\begin{document}

\maketitle

\section{Antiunitary symmetry of second quantized Hamiltonian of free fermions}

Consider the following second quantized Hamiltonian:
\begin{equation}
    \hat{H} = \sum_{\alpha, \beta} c^\dagger_\alpha H_{\alpha \beta} c_\beta,
    \label{eq:second-quantization}
\end{equation}
where operators with $\hat{}$ are second quantized operators. In this section, we consider how 
antiunitary symmetries acts on \eqref{eq:second-quantization}. The action of a unitary group element can be 
written as 
\begin{equation}
    g \cdot \ket{\alpha} = \ket{\beta} \varphi(g)_{\beta \alpha}, \quad 
    g \cdot \bra{\alpha} = \varphi(g)^\dagger_{\alpha \beta} \bra{\beta}, 
\end{equation}
and 
\begin{equation}
    g \cdot c^\dagger_\alpha = g c^\dagger_\alpha g^{-1} = c^\dagger_\beta \varphi(g)_{\beta \alpha}, \quad 
    g \cdot c_\alpha = g c_\alpha g^{-1} = \varphi(g)^\dagger_{\alpha \beta} c_\beta.
\end{equation}
So the Hamiltonian transforms as 
\begin{equation}
    g \cdot \hat{H} = c^\dagger_{\alpha'} \varphi(g)_{\alpha' \alpha} H_{\alpha \beta} \varphi(g)^\dagger_{\beta \beta'} c_{\beta'}.
\end{equation}
Of course, this is just in the form of basis transition. If the Hamiltonian has the symmetry of $g$, we have 
\begin{equation}
    H_{\alpha' \beta'} = \varphi(g)_{\alpha' \alpha} H_{\alpha \beta} \varphi(g)^\dagger_{\beta \beta'}.
\end{equation}
Suppose $\{\ket{\alpha}\}$ is real with regard of the time reversal operation $\mathcal{T} = T \mathcal{K}$.
In this case, the time reversal symmetry doesn't act on the basis, but $\mathcal{K}$ acts on $H_{\alpha \beta}$
and adds a star, so if the system has time reversal symmetry, we have 
\begin{equation}
    T H T^{-1} = {H}^*.
\end{equation}

Now we move to the ``real'' particle-hole symmetry in an insulator. Consider 
\begin{equation}
    \hat{H} = - \sum_{\pair{\vb*{i}, \vb*{j}}} (t_{\vb*{i} \vb*{j}} c^\dagger_{\vb*{i}} c_{\vb*{j}} + \text{h.c.}).
\end{equation}
We use $\mathcal{C}$ to denote the particle-hole transformation from $c$ to $c^\dagger$, i.e.
\begin{equation}
    \mathcal{C} c_{\vb*{i}} \mathcal{C}^{-1} = c_{\vb*{i}}^\dagger, \quad \mathcal{C} c_{\vb*{i}}^\dagger \mathcal{C}^{-1} = c_{\vb*{i}},
\end{equation}
then we have 
\begin{equation}
    \mathcal{C} \hat{H} \mathcal{C}^{-1} = - \sum_{\pair{\vb*{i}, \vb*{j}}} (t_{\vb*{i} \vb*{j}} c_{\vb*{i}} c^\dagger_{\vb*{j}} + \text{h.c.}) = \sum_{\pair{\vb*{i}, \vb*{j}}} (t_{\vb*{j} \vb*{i}}^* c^\dagger_{\vb*{j}} c_{\vb*{i}} + \text{h.c.}) = - \hat{H}^* = - \hat{H}^\top.
\end{equation}
The minus sign, physically, means flipping the spectrum with the Fermi surface as a mirror, while the transpose 
operation comes from flipping $\ket{\vb*{i}}$ to $\bra{\vb*{i}}$. Here we need to keep in mind a strange property 
of particle-hole transformation. $\mathcal{C}$ should be unitary in the second quantization formalism, because 
we want 
\begin{equation}
    c_{\vb*{k}}^\dagger \stackrel{\mathcal{C}}{\longrightarrow} c_{- \vb*{k}}
\end{equation}
to keep momentum conservation, and if $\mathcal{C}$ is unitary, we have 
\[
    \mathcal{C} c_{\vb*{k}}^\dagger = \mathcal{C} \frac{1}{\sqrt{N}} \sum_{\vb*{i}} \ee^{- \ii \vb*{k} \cdot \vb*{r}_{\vb*{i}}} c^\dagger_{\vb*{i}} = \sum_{\vb*{i}} \ee^{- \ii \vb*{k} \cdot \vb*{r}_{\vb*{i}}} c_{\vb*{i}} = c_{- \vb*{k}},
\]
which is exactly what we want. However, the first quantization version of $\mathcal{C}$ maps a single-electron 
wave function in the single electron Hilbert space to the \emph{dual} space, and a map $\mathcal{H} \to \mathcal{H}^*$
should be antiunitary to keep naturalness.

The action of $\mathcal{C}$ on an arbitrary state is 
\begin{equation}
    \mathcal{C} c^\dagger_{\alpha} \mathcal{C}^{-1} = c_\beta C_{\beta \alpha}, \quad 
    \mathcal{C} c_{\alpha} \mathcal{C}^{-1} = C^{-1}_{\alpha \beta} c_\beta,
\end{equation}
% TODO
\begin{equation}
    \mathcal{C} H \mathcal{C}^{-1} = - C H^* C^{-1}.
\end{equation}

Despite their weirdness, $\mathcal{T}$ and $\mathcal{C}$ can be fully demonstrated by constraints on 
the coefficient matrix $H$ (i.e. first quantized Hamiltonian) in \eqref{eq:second-quantization}.
Classification of free fermion systems is therefore classification of these matrices.

\section{Particle-hole symmetry in superconductors}

Particle-hole symmetry is another antiunitary symmetry that may concern us for fermionic systems. 
Consider, for example, the following BCS model of superconductivity:
\begin{equation}
    H = \sum_{\vb*{k}, \alpha} \xi_{\vb*{k}} c^\dagger_{\vb*{k} \alpha} c_{\vb*{k} \alpha} - V \sum_{\vb*{k}, \vb*{k}', \vb*{q}, \alpha, \beta} c^\dagger_{\vb*{k}' - \vb*{q}, \alpha} c^\dagger_{\vb*{k} + \vb*{q}, \beta} c_{\vb*{k} \beta} c_{\vb*{k}' \alpha},
    \label{eq:bcs}
\end{equation}
which turns into 
\begin{equation}
    H = \sum_{\vb*{k}, \alpha} \xi_{\vb*{k}} c^\dagger_{\vb*{k} \alpha} c_{\vb*{k} \alpha} 
    + \sum_{\vb*{k}} (\Delta c^\dagger_{- \vb*{k} \downarrow} c^\dagger_{\vb*{k} \uparrow} + \Delta^* c_{\vb*{k} \uparrow} c_{- \vb*{k} \downarrow}),
    \label{eq:bdg}
\end{equation}
where $\Delta$ is the BCS order parameter. \eqref{eq:bdg} is called the \concept{BdG Hamiltonian}, 
which describes the spectrum of (electron-like) quasiparticles in a BCS superconductor. Note that 
the order parameter in \eqref{eq:bdg} has quantum fluctuation, but what we are discussing here is 
the \emph{topological band behavior} of fermions, so ignoring the fluctuation of $\Delta$ makes sense.
It's possible that the fluctuation of $\Delta$ destroys the ordinary BCS order, but it's not the case 
in 3D. Note also that our current approach -- ignoring the fluctuation of $\Delta$, i.e. ignoring the 
many-body effect introduced by electron interaction -- is a non-interaction limit of the general theory 
of topological classification with interaction. 

In 1D and 2D, Mermin–Wagner theorem means the 
effective theory about fermionic quasiparticles of \eqref{eq:bcs} is not \eqref{eq:bdg}, because 
$U(1)$ symmetry -- a continuous symmetry -- can't be broken here. But we can always use a 
3D bulk state to ``induce'' a low-dimensional superconducting phase, which has electron pairing anyway
and can be described by \eqref{eq:bdg}, though this time \eqref{eq:bdg} has nothing to do with BCS mechanism.
So henceforth we will work with the free-fermionic model \eqref{eq:bdg} and ignore what induces superconductivity
pairing.

Now we repeat the classical procedure in BCS theory: \eqref{eq:bdg} can be rephrased into 
\begin{equation}
    H = \sum_{\vb*{k}} \Psi_{\vb*{k}}^\dagger \pmqty{ \xi_{\vb*{k}} & - \Delta \\ - \Delta^* & -\xi_{\vb*{k}} } \Psi_{\vb*{k}}, \quad \Psi_{\vb*{k}} = 
\end{equation}

The particle-hole symmetry is actually a \emph{redundancy}: BdG Hamiltonians without this symmetry aren't 
qualified theories of superconductors. A superconductor has \emph{more} symmetry than a similar insulator 
because it has a particle-hole symmetry, while it has \emph{fewer} actual symmetry than a similar 
insulator because it breaks the $U(1)$ symmetry. 

\section{The 10-fold way}

\begin{itemize}
    \item Class A: no time reversal symmetry, no particle-hole symmetry, no chiral symmetry. An insulator with
    charge conservation. Example: Chern-insulators, IQHE.
    \item Class AIII: no time reversal symmetry, no particle-hole symmetry, but there is a chiral symmetry. 
    Example: SSH model. 
    \item Class D: no time reversal, but there is a particle-hole symmetry. This is a superconductor.
    It actually may not have any physical symmetry other than the fermion parity symmetry.
    Example: 1D Kitaev chain, 2D $p + \ii $ superconductor.
    \item Class DIII: $T=-1, $
\end{itemize}

\end{document}