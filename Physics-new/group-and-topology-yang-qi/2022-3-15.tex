\documentclass[hyperref, a4paper]{article}

\usepackage{geometry}
\usepackage{float}
\usepackage{titling}
\usepackage{titlesec}
% No longer needed, since we will use enumitem package
% \usepackage{paralist}
\usepackage{enumitem}
\usepackage{footnote}
\usepackage{enumerate}
\usepackage{amsmath, amssymb, amsthm}
\usepackage{mathtools}
\usepackage{bbm}
\usepackage{cite}
\usepackage{graphicx}
\usepackage{subfigure}
\usepackage{physics}
\usepackage{tensor}
\usepackage{siunitx}
\usepackage{booktabs}
\usepackage[version=4]{mhchem}
\usepackage{tikz}
\usepackage{xcolor}
\usepackage{listings}
\usepackage{autobreak}
\usepackage[ruled, vlined, linesnumbered]{algorithm2e}
\usepackage{nameref,zref-xr}
\zxrsetup{toltxlabel}
\zexternaldocument*[solid-]{../solid/solid}[solid.pdf]
\usepackage[colorlinks,unicode]{hyperref} % , linkcolor=black, anchorcolor=black, citecolor=black, urlcolor=black, filecolor=black
\usepackage[most]{tcolorbox}
\usepackage{prettyref}

% Page style
\geometry{left=3.18cm,right=3.18cm,top=2.54cm,bottom=2.54cm}
\titlespacing{\paragraph}{0pt}{1pt}{10pt}[20pt]
\setlength{\droptitle}{-5em}
\preauthor{\vspace{-10pt}\begin{center}}
\postauthor{\par\end{center}}

% More compact lists 
\setlist[itemize]{itemindent=17pt, leftmargin=1pt}

% Math operators
\DeclareMathOperator{\timeorder}{T}
\DeclareMathOperator{\diag}{diag}
\DeclareMathOperator{\legpoly}{P}
\DeclareMathOperator{\primevalue}{P}
\DeclareMathOperator{\sgn}{sgn}
\newcommand*{\ii}{\mathrm{i}}
\newcommand*{\ee}{\mathrm{e}}
\newcommand*{\const}{\mathrm{const}}
\newcommand*{\suchthat}{\quad \text{s.t.} \quad}
\newcommand*{\argmin}{\arg\min}
\newcommand*{\argmax}{\arg\max}
\newcommand*{\normalorder}[1]{: #1 :}
\newcommand*{\pair}[1]{\langle #1 \rangle}
\newcommand*{\fd}[1]{\mathcal{D} #1}
\DeclareMathOperator{\bigO}{\mathcal{O}}

% TikZ setting
\usetikzlibrary{arrows,shapes,positioning}
\usetikzlibrary{arrows.meta}
\usetikzlibrary{decorations.markings}
\tikzstyle arrowstyle=[scale=1]
\tikzstyle directed=[postaction={decorate,decoration={markings,
    mark=at position .5 with {\arrow[arrowstyle]{stealth}}}}]
\tikzstyle ray=[directed, thick]
\tikzstyle dot=[anchor=base,fill,circle,inner sep=1pt]

% Algorithm setting
% Julia-style code
\SetKwIF{If}{ElseIf}{Else}{if}{}{elseif}{else}{end}
\SetKwFor{For}{for}{}{end}
\SetKwFor{While}{while}{}{end}
\SetKwProg{Function}{function}{}{end}
\SetArgSty{textnormal}

\newcommand*{\concept}[1]{{\textbf{#1}}}

\DeclareMathOperator{\im}{im}

% Embedded codes
\lstset{basicstyle=\ttfamily,
  showstringspaces=false,
  commentstyle=\color{gray},
  keywordstyle=\color{blue}
}

\newcommand{\soliddoc}{\href{../solid/solid.pdf}{this note}}

% Color boxes
\tcbuselibrary{skins, breakable, theorems}
\newtcbtheorem[number within=section]{warning}{Warning}%
  {colback=orange!5,colframe=orange!65,fonttitle=\bfseries, breakable}{warn}
\newtcbtheorem[number within=section]{note}{Note}%
  {colback=green!5,colframe=green!65,fonttitle=\bfseries, breakable}{note}
\newtcbtheorem[number within=section]{info}{Info}%
  {colback=blue!5,colframe=blue!65,fonttitle=\bfseries, breakable}{info}

\title{Space Groups and Their Representations by Prof. Yang Qi}
\author{Jinyuan Wu}

\begin{document}

\maketitle

\section{Group theory in condensed matter physics}

Let's first briefly review the role of group theory in quantum mechanics.
We are usually interested in Lie group symmetries and discrete symmetries, the latter being more important in 
condensed matter physics because in this field we have to deal with lattices with point group symmetries, 
and many symmetries that are important for topological states of matter are associated with discrete symmetries 
like parity symmetry and a special symmetry for fermions named the \concept{fermion-parity symmetry}, which 
exists in all fermionic systems and arises from the fact that each term in the Hamiltonian always contains 
even fermionic operators.

So we will just work with discrete groups hereafter. Actually we have the continuous spin rotational symmetry,
but since spin-orbital coupling exists in all realistic materials, operations on the spin degrees of freedom 
are just a part of \emph{doubled} space groups.

A \concept{linear representation} is a group homomorphism from a symmetry group $G$ to $\mathrm{GL}(\mathcal{H})$,
where $\mathcal{H}$ is the Hilbert space. Since the wave function of the system can differ a phase factor 
after a symmetric operation, we can also have \concept{projective representation}. 
Representation theory gives very generic results for studying symmetries.

Let $\varphi(\cdot)$ be the mapping from $G$ to $\mathrm{GL}(\mathcal{H})$. If $\varphi$ is a linear representation,
then it's easy to see 
\begin{equation}
    \comm*{H}{\varphi(g)} = 0,
\end{equation}
for any group element $g$. We often say that $\varphi(g)$'s \emph{label} the energy eigenstates. 
For Abelian symmetric group $G$, the labels of $\ket*{n}$ are simply given by $\varphi(g_i) \ket*{n}$, where $\{g_i\}$ 
are generators of $G$. If $G$ is non-Abelian, we need to work with its irreducible representations, and 
energy eigenstates are classified into several irreducible representations of $G$, and group elements turn one 
state into another state in the same irreducible representation. The degeneracy of energy eigenstates is 
always the dimension of \emph{one} representation. Since a reducible representation can always be split into 
two irreducible representations by a perturbation, from then on we only investigate energy degeneracy protected 
by irreducible representations.

\section{Space groups}

In this section we are not going to show that there are 230 space groups in $\mathbb{R}^3$. 
Actually we usually call 2D space groups as \concept{wallpaper groups}. A group element of a space group 
is in the form of the following affine transformation
\begin{equation}
    (R| \vb*{t}): \vb*{v} \mapsto R \vb*{v} + \vb*{t},
\end{equation}
or in the form of 
\begin{equation}
    (R | \vb*{t}) : \tilde{\vb*{v}} \coloneqq \pmqty{\vb*{v} \\ 1} \mapsto \underbrace{\pmqty{R & \vb*{t} \\ 0 & 1 }}_{\coloneqq \tilde{R}} \tilde{\vb*{v}},
\end{equation}
which is the preferred form of space group operation in numerical calculations.

Detailed discussion on the structure of space groups can be found in 
Section.~\ref{solid-sec:space-group-structure-classification} in \soliddoc.
Here we list some of the elements. First we discuss point groups. 
We have $C_n$ axises, the generator of which is labeled as $n$: 
we have 1, 2, 3, 4, 6 axises, the first being the identity. We also have rotation-reflection axises
$\bar{1}, \bar{2}, \bar{3}, \bar{4}, \bar{6}$, the first being the inversion operation and the second 
being the mirror reflection. $\bar{3}$ is not an independent generator because we have 
\[
    (\bar{3})^3 = \bar{1}, \quad \bar{3} \bar{1} = 3.
\]
Similarly, $\bar{6}$ is also not an independent generator.

There are also \concept{nonsymmorphic} operations, which involve fraction translation. There are two 
types of nonsymmorphic generators: \concept{screw axis} $(c_n | n \vb*{R} / m)$, where $m = 1, 2, 3, \ldots$,
and \concept{glide plane} $(\sigma | \vb*{R} / 2)$.

The point group of a space group $G$ is $K = G / \mathbb{T}$, where $\mathbb{T}$ is the translation group.
Note that if $G$ is nonsymmorphic, $K$ is \emph{not} a subgroup of $G$. $G$ is actually \emph{semidirect}
product of $K$ and $\mathbb{T}$. We have 
\begin{equation}
    G = \mathbb{T} \rtimes K.
\end{equation}

\begin{info*}{Short exact sequence and group extension}
    Consider the following short exact sequence
    \begin{equation}
        1 \to N \stackrel{f}{\to} G \stackrel{\pi}{\to} Q \to 1.
    \end{equation}
    The definition of exact sequence means 
    \begin{itemize}
        \item $f$ is injective, since $\ker f = \im (\text{the monomorphism $1 \to N$}) = 1$.
        \item $g$ is surjective, since $\im \pi = \ker (\text{the monomorphism $Q \to 1$}) = Q$.
        \item $Q \simeq G / \im{f}$, because $G / \ker \pi = \im \pi$, and by definition we have $\ker \pi = \im f$,
        and $\im \pi = Q$ (from last line).
    \end{itemize}
    $G$ is defined said to be an \concept{extension} of $Q$ over $N$. 
    
    We consider the case where $N$ is Abelian. The simplest kind of group extension 
    is just direct product. We also have \concept{semidirect product} or \concept{split extension} $N \rtimes Q$
    (for an abstract definition, see \href{https://en.wikipedia.org/wiki/Semidirect\_product\#Inner\_semidirect\_product\_definitions}{this Wikipedia page}), 
    the multiplication rule of which is defined as 
    \begin{equation}
        (a, g) \times (b, h) \coloneqq 
    \end{equation}
    We also say $G$ \concept{splits} over $N$.
    The third kind of group extension is \concept{central extension}. 
\end{info*}

Realizing this, we can find the ``deep'' reason of the emergence of nonsymmorphic operations. 

\end{document}