\documentclass[hyperref, a4paper]{article}

\usepackage{geometry}
\usepackage{float}
\usepackage{titling}
\usepackage{titlesec}
% No longer needed, since we will use enumitem package
% \usepackage{paralist}
\usepackage{enumitem}
\usepackage{footnote}
\usepackage{enumerate}
\usepackage{amsmath, amssymb, amsthm}
\usepackage{mathtools}
\usepackage{bbm}
\usepackage{cite}
\usepackage{graphicx}
\usepackage{subfigure}
\usepackage{physics}
\usepackage{tensor}
\usepackage{siunitx}
\usepackage{booktabs}
\usepackage[version=4]{mhchem}
\usepackage{tikz}
\usepackage{xcolor}
\usepackage{listings}
\usepackage{autobreak}
\usepackage[ruled, vlined, linesnumbered]{algorithm2e}
\usepackage{nameref,zref-xr}
\zxrsetup{toltxlabel}
\zexternaldocument*[solid-]{../solid/solid}[solid.pdf]
\usepackage[colorlinks,unicode]{hyperref} % , linkcolor=black, anchorcolor=black, citecolor=black, urlcolor=black, filecolor=black
\usepackage[most]{tcolorbox}
\usepackage{prettyref}

% Page style
\geometry{left=3.18cm,right=3.18cm,top=2.54cm,bottom=2.54cm}
\titlespacing{\paragraph}{0pt}{1pt}{10pt}[20pt]
\setlength{\droptitle}{-5em}
\preauthor{\vspace{-10pt}\begin{center}}
\postauthor{\par\end{center}}

% More compact lists 
\setlist[itemize]{itemindent=17pt, leftmargin=1pt}

% Math operators
\DeclareMathOperator{\timeorder}{T}
\DeclareMathOperator{\diag}{diag}
\DeclareMathOperator{\legpoly}{P}
\DeclareMathOperator{\primevalue}{P}
\DeclareMathOperator{\sgn}{sgn}
\newcommand*{\ii}{\mathrm{i}}
\newcommand*{\ee}{\mathrm{e}}
\newcommand*{\const}{\mathrm{const}}
\newcommand*{\suchthat}{\quad \text{s.t.} \quad}
\newcommand*{\argmin}{\arg\min}
\newcommand*{\argmax}{\arg\max}
\newcommand*{\normalorder}[1]{: #1 :}
\newcommand*{\pair}[1]{\langle #1 \rangle}
\newcommand*{\fd}[1]{\mathcal{D} #1}
\DeclareMathOperator{\bigO}{\mathcal{O}}
\DeclareMathOperator{\id}{id}

% TikZ setting
\usetikzlibrary{arrows,shapes,positioning}
\usetikzlibrary{arrows.meta}
\usetikzlibrary{decorations.markings}
\tikzstyle arrowstyle=[scale=1]
\tikzstyle directed=[postaction={decorate,decoration={markings,
    mark=at position .5 with {\arrow[arrowstyle]{stealth}}}}]
\tikzstyle ray=[directed, thick]
\tikzstyle dot=[anchor=base,fill,circle,inner sep=1pt]

% Algorithm setting
% Julia-style code
\SetKwIF{If}{ElseIf}{Else}{if}{}{elseif}{else}{end}
\SetKwFor{For}{for}{}{end}
\SetKwFor{While}{while}{}{end}
\SetKwProg{Function}{function}{}{end}
\SetArgSty{textnormal}

\newcommand*{\concept}[1]{{\textbf{#1}}}

\DeclareMathOperator{\im}{im}

% Embedded codes
\lstset{basicstyle=\ttfamily,
  showstringspaces=false,
  commentstyle=\color{gray},
  keywordstyle=\color{blue}
}

\newcommand{\soliddoc}{\href{../solid/solid.pdf}{this note}}

% Color boxes
\tcbuselibrary{skins, breakable, theorems}
\newtcbtheorem[number within=section]{warning}{Warning}%
  {colback=orange!5,colframe=orange!65,fonttitle=\bfseries, breakable}{warn}
\newtcbtheorem[number within=section]{note}{Note}%
  {colback=green!5,colframe=green!65,fonttitle=\bfseries, breakable}{note}
\newtcbtheorem[number within=section]{info}{Info}%
  {colback=blue!5,colframe=blue!65,fonttitle=\bfseries, breakable}{info}

\title{Space Groups and Their Representations by Prof. Yang Qi}
\author{Jinyuan Wu}

\begin{document}

\maketitle

\section{Group theory in condensed matter physics}

Let's first briefly review the role of group theory in quantum mechanics.
We are usually interested in Lie group symmetries and discrete symmetries, the latter being more important in 
condensed matter physics because in this field we have to deal with lattices with point group symmetries, 
and many symmetries that are important for topological states of matter are associated with discrete symmetries 
like parity symmetry and a special symmetry for fermions named the \concept{fermion-parity symmetry}, which 
exists in all fermionic systems and arises from the fact that each term in the Hamiltonian always contains 
even fermionic operators.

So we will just work with discrete groups hereafter. Actually we have the continuous spin rotational symmetry,
but since spin-orbital coupling exists in all realistic materials, operations on the spin degrees of freedom 
are just a part of \emph{doubled} space groups.

A \concept{linear representation} is a group homomorphism from a symmetry group $G$ to $\mathrm{GL}(\mathcal{H})$,
where $\mathcal{H}$ is the Hilbert space. Since the wave function of the system can differ a phase factor 
after a symmetric operation, we can also have \concept{projective representation}. 
Representation theory gives very generic results for studying symmetries.

Let $\varphi(\cdot)$ be the mapping from $G$ to $\mathrm{GL}(\mathcal{H})$. If $\varphi$ is a linear representation,
then it's easy to see 
\begin{equation}
    \comm*{H}{\varphi(g)} = 0,
\end{equation}
for any group element $g$. We often say that $\varphi(g)$'s \emph{label} the energy eigenstates. 
For Abelian symmetric group $G$, the labels of $\ket*{n}$ are simply given by $\varphi(g_i) \ket*{n}$, where $\{g_i\}$ 
are generators of $G$. If $G$ is non-Abelian, we need to work with its irreducible representations, and 
energy eigenstates are classified into several irreducible representations of $G$, and group elements turn one 
state into another state in the same irreducible representation. The degeneracy of energy eigenstates is 
always the dimension of \emph{one} representation. Since a reducible representation can always be split into 
two irreducible representations by a perturbation, from then on we only investigate energy degeneracy protected 
by irreducible representations.

\section{Group extension and group cohomology}

\subsection{Group extension}

Consider the following short exact sequence
\begin{equation}
    1 \to N \stackrel{f}{\to} G \stackrel{\pi}{\to} Q \to 1.
    \label{eq:short-exact-sequence-group}
\end{equation}
The definition of exact sequence means 
\begin{itemize}
    \item $f$ is injective, since $\ker f = \im (\text{the monomorphism $1 \to N$}) = 1$.
    \item $\pi$ is surjective, since $\im \pi = \ker (\text{the monomorphism $Q \to 1$}) = Q$.
    \item $Q \simeq G / \im{f}$, because $G / \ker \pi = \im \pi$, and by definition we have $\ker \pi = \im f$,
    and $\im \pi = Q$ (from last line). Note that since $f$ is injective, $\im f$ is a perfect copy of $N$,
    and can be seen as an inclusion map from $N$ into $G$.
\end{itemize}
$G$ is defined said to be an \concept{extension} of $Q$ over $N$. 

We consider the case where $N$ is Abelian. What we want to do is to find and classify all group extensions
-- under the assumption that $N$ is Abelian, we are actually working on \emph{central extensions}. 
Two group extensions are said to be \concept{equivalent} if and only if the following diagram 
\begin{equation}
    \begin{tikzpicture}
        \node (A) at (-2,0) {$1$};
        \node (B) at (-1,0) {$N$};
        \draw[->] (A) -- (B);
        \node (C) at (0, 0.6) {$G$};
        \node (D) at (0, -0.6) {$G'$};
        \draw [<->] (C) -- (D);
        \draw [->] (B) -- (C);
        \draw [->] (B) -- (D);
        \node (E) at (1, 0) {$Q$};
        \draw [->] (C) -- (E);
        \draw [->] (D) -- (E);
        \node (F) at (2, 0) {$1$};
        \draw [->] (E) -- (F);
        \node at (-0.6, 0.6) {$f$};
        \node at (-0.7, -0.6) {$f'$};
        \node at (0.6, 0.6) {$\pi$};
        \node at (0.6, -0.6) {$\pi'$};
    \end{tikzpicture}
    \label{eq:equivalent-commuting-diagram}
\end{equation} commutes, and we need to find equivalent classes of 
this equivalence relation.

\subsection{Direct product and semidirect product}

The simplest kind of group extension is just direct product: we define $G$ to be 
\begin{equation}
    G = N \otimes Q,
\end{equation}
and the multiplication operation is defined as 
\begin{equation}
    (a, g) \circ (b, h) \coloneqq (a + b, g h).
    \label{eq:direct-product-def}
\end{equation}
We also have \concept{semidirect product} or \concept{split extension} $N \rtimes Q$, 
the multiplication rule of which is defined as 
\begin{equation}
    (a, g) \circ (b, h) \coloneqq (a + g b g^{-1}, gh).
    \label{eq:semidirect-product-def-original}
\end{equation}
We also say $G$ \concept{splits} over $N$. Note that the notation $g b g^{-1}$ is kind of ambiguous,
because up to now we haven't introduced any multiplication relation between $N$ and $Q$.
A more accurate notation is to introduce a group homomorphism $\varphi: Q \to \mathrm{Aut}(N)$
and interpret $\varphi(g)(a)$ as $\tilde{g} a \tilde{g}^{-1}$, where $\pi(\tilde{g}) = g$. 
According to the definition, $\pi$ is surjective and not injective, because after group extension, 
the same group element $g$ in $Q$ is ``split'' into several copies in $G$, and we use $\tilde{g}$
to denote a certain ``copy'' of $g$. \eqref{eq:semidirect-product-def-original} is therefore corrected into 
\begin{equation}
    (a, g) \circ (b, h) \coloneqq (a + \tilde{g} b \tilde{g}^{-1}, gh).
    \label{eq:semidirect-product-def}
\end{equation}

We need to check whether \eqref{eq:direct-product-def} and 
\eqref{eq:semidirect-product-def} can really make $G$ a group, i.e. we need to check whether the 
associative law holds for $\{(a, g)\}$. For direct product, this is obviously true.
For semidirect product, we have 
\[
    \begin{aligned}
        ((a, g) (b, h)) (c, i) &= (a + \tilde{g} b \tilde{g}^{-1}, gh) (c, i) \\
        &= (a + \tilde{g} b \tilde{g}^{-1} + (\widetilde{g h} ) c (\widetilde{g h})^{-1}, ghi),
    \end{aligned}
\]
and 
\[
    \begin{aligned}
        (a, g) ((b, h) (c, i)) &= (a, g) (b + \tilde{h} c \tilde{h}^{-1}, hi) \\
        &= (a + \tilde{g}(b + \tilde{h} c \tilde{h}^{-1}) \tilde{g}^{-1}, ghi).
    \end{aligned}
\]
Usually, we define multiplication rules under the definition of $(a, h)$-pairs as is in 
\eqref{eq:pair-element-def}, i.e. $\tilde{g} = s(g)$ and doesn't depend on $a, b$, and under this assumption, we have \eqref{eq:s-inv-pi-return-n}.
So we have
\[
    \tilde{g} \tilde{h} c \tilde{h}^{-1} \tilde{g}^{-1} = \widetilde{gh} c \widetilde{gh}^{-1},
\]
because 
\[
    \begin{aligned}
        &\quad \widetilde{gh}^{-1} \tilde{g} \tilde{h} c \tilde{h}^{-1} \tilde{g}^{-1} \widetilde{gh}  \\
        &= \underbrace{\widetilde{gh}^{-1} \tilde{g} \tilde{h}}_{\in \im f} + c + \underbrace{(\widetilde{gh}^{-1} \tilde{g} \tilde{h})^{-1}}_{\in \im f} \\
        &= c.
    \end{aligned}
\]
and therefore \eqref{eq:semidirect-product-def} is always self-consistent. 

The motivation of semidirect product is to imitate \emph{inner} semidirect product. 
An abstract definition of inner semidirect product can be found \href{https://en.wikipedia.org/wiki/Semidirect\_product\#Inner\_semidirect\_product\_definitions}{this Wikipedia page}.
In short, if $G$ has a subgroup $H$ and a normal subgroup $N$ with no nontrivial intersection with 
$Q$, and $G = NQ$, then we say $G$ is the \concept{inner semidirect product} of $N$ and $Q$.
\eqref{eq:semidirect-product-def}, on the other hand, is \concept{outer semidirect product}, and after we 
construct $G$ according to it, we will find $G$ is the inner semidirect product of $N \times e_Q$ and
$0_N \times Q$. 

Similarly, we have inner direct product. But it's often the case that we can only guarantee that $N$ is normal 
and there is nontrivial multiplication relation between $N$ and $Q$. In this case, $G$ simply can't be 
written as the (outer) direct product of $N$ and $Q$, and if we still want to \emph{factor} $G$ into 
something like $N \times Q$, inner \emph{semidirect} product will be handy. This explains the motivation
to introduce \eqref{eq:semidirect-product-def}. (We can define $N \times Q$, but the result doesn't have 
the same structure with $G$.)

\subsection{Generic group extensions and their equivalence relations}

Now we want to investigate a generic central extension. The multiplication rule is 
\begin{equation}
    (a, g) \circ (b, h) \coloneqq (a + \tilde{g} b \tilde{g}^{-1} + \omega_2(g, h), gh),
    \label{eq:generic-central-extension}
\end{equation}
where
\begin{equation}
    \omega_2 : Q \times Q \to N.
\end{equation}
Here we assume that $\omega_2$ is only dependent on $g$ and $h$ and doesn't include $a$ and $b$, 
because of the argument below. Since $\pi$ is surjective, we can find a map $s$ from $Q$ to $G$ such that 
$\pi \circ s = \id_Q$. Note that $s$ is not necessarily a homomorphism, or otherwise 
\eqref{eq:short-exact-sequence-group} is split. % TODO
Usually we define 
\begin{equation}
    (a, g) = f(a) s(g)
    \label{eq:pair-element-def}
\end{equation}
to derive multiplication rules. A little calculation gives
\begin{equation}
    \begin{aligned}
        (a, g) \circ (b, h) &= f(a) s(g) f(b) s(h) \\
        &= f(a) s(g) f(b) s(g)^{-1} s(g) s(h) s(gh)^{-1} s(gh).
    \end{aligned}
    \label{eq:mul-by-def}
\end{equation}
Note that since $\pi$ is a homomorphism, we have 
\[
    \begin{aligned}
        \pi (s(g) s(h) s(gh)^{-1}) &= \pi(s(g)) \pi(s(h)) \pi(s(gh))^{-1} \\
        &= g h (gh)^{-1} = e,
    \end{aligned}
\]
and the derivation also works for $s(gh)^{-1} s(g) s(h)$, so we have 
\begin{equation}
    s(gh)^{-1} s(g) s(h), s(g) s(h) s(gh)^{-1} \in \ker \pi = \im i,
    \label{eq:s-inv-pi-return-n}
\end{equation}
So \eqref{eq:mul-by-def} is now 
\[
    (a, g) \circ (b, h) = (f(a) + s(g) f(b) s(g)^{-1} + \underbrace{s(g) s(h) s(gh)^{-1}}_{\coloneqq \omega_2(g,h)}) s(gh),
\]
and we get \eqref{eq:generic-central-extension}.

The next step is to check the constraint on $\omega_2$ to ensure that $G$ is really a group. 
From 
\[
    ((a, g) \circ (b, h)) \circ (c, i) = (a, g) \circ ((b, h) \circ (c, i)),
\]
we have 
\begin{equation}
    \omega_2(g, h) + \omega_2(gh, i) = \underbrace{\tilde{g} \omega_2(h, i) \tilde{g}^{-1}}_{g \cdot \omega_2(h, i)} + \omega_2(g, hi).
    \label{eq:omega-consistent-condition}
\end{equation}
This constraint on $\omega_2$ seems strange, but actually it's a part of a larger structure: 
\concept{group cohomology}. 

\subsection{Group cohomology}

A cohomology structure can be illustrated as 
\begin{equation}
    \cdots \to C^{n-1} \stackrel{\dd_{n-1}}{\longrightarrow} C^n \stackrel{\dd_n}{\longrightarrow} C^{n+1} \to \cdots ,
    \label{eq:cohomology-chain}
\end{equation}
where 
\begin{equation}
    \dd_{n+1} \circ \dd_n \eqqcolon \dd^2 = 0,
\end{equation} 
and 
\begin{equation}
    \im \dd_{n-1} = \ker \dd_n.
\end{equation}
An element $\omega \in C^{n}$ is an \concept{$n$-cochain}. An $n$-cochain $\omega$ is said to be an 
\concept{$n$-cocycle} if and only if $\dd{\omega} = 0$, the set of which is denoted as
\begin{equation}
    Z^n = \ker \dd_n.
\end{equation}
and an $n$-cochain $\omega$ is said to be an 
\concept{$n$-coboundary} if and only if $\omega = \dd{\mu}$, the set of which is denoted as 
\begin{equation}
    B^n = \im \dd_{n-1}.
\end{equation}
Of course, $n$-coboundaries are all 
$n$-cocycles, but not inverse. A natural equivalence relation between $n$-cocycles is that 
\begin{equation}
    \omega \sim \omega' \quad \text{iff} \quad \omega - \omega' \in B^n,
\end{equation}
and the set of equivalence classes is 
\begin{equation}
    H^n = Z^n / B^n.
\end{equation}
If \eqref{eq:cohomology-chain} is exact, then $H^n$'s are trivial. Cohomology is a tool to measure how 
\emph{inexact} \eqref{eq:cohomology-chain} is.

For groups, one possible way to obtain cohomology is to define 
\begin{equation}
    C^n(G, N) \coloneqq C^n \to N,
\end{equation}
and 
\begin{equation}
    \begin{aligned}
        (\dd_n \omega)(g_1, \ldots, g_{n+1}) &\coloneqq g_1 \cdot \omega(g_2, \ldots, g_{n+1}) + \sum_{i=1}^{n} (-1)^i \omega(g_1, \ldots, g_i g_{i+1}, \ldots, g_{n+1})  \\
        &\quad \quad - (-1)^n \omega(g_1, \ldots, g_n).
    \end{aligned}
\end{equation}
We also assume that $\dd$ and $g \cdot$ commute.
Now we verify the $\dd^2 = 0$ condition. We have 
\[
    \begin{aligned}
        &\quad\dd_{n+1} (\{g_i\}_{i=1}^{n+1} \mapsto g_1 \cdot \omega(g_2, \ldots, g_{n+1})) (g_1', \ldots, g_{n+2}') \\
        &= g_1' g_2' \cdot \omega(g_3', \ldots, g_{n+2}') + \sum_{i=1}^{n+1} (-1)^i g_1' \cdot \omega(g_2', \ldots, g_i' g_{i+1}', \ldots, g_{n+2}') - (-1)^{n+1} g_1' \cdot \omega(g_2', \ldots, g_{n+1}'),
    \end{aligned}
\]
\[
    \begin{aligned}
        \dd_{n+1} (\{g_i\}_{i=1}^{n+1} \mapsto \omega(g_1, \ldots, g_n)) = 
    \end{aligned}
\]
\begin{equation}
    \cdots \to C^{n-1}(G, N) \stackrel{\dd^{n-1}}{\longrightarrow} C^n(G, N) \stackrel{\dd^n}{\longrightarrow} C^{n+1}(G, N) \to \cdots ,
\end{equation}
Specifically, we have 
\begin{equation}
    \dd{\omega}(g_1, g_2, g_3) = g_1 \cdot \omega(g_2, g_3) - \omega(g_1 g_2, g_3) + \omega(g_1, g_2 g_3) - \omega(g_2, g_3),
\end{equation}
and this means the self-consistent constraint \eqref{eq:omega-consistent-condition} on $\omega_2$ 
appearing in \eqref{eq:generic-central-extension} is just equivalent to 
\begin{equation}
    \dd{\omega_2} = 0 \quad \text{or in other words} \quad \omega_2 \in Z^2(G, N).
\end{equation}
The fact $\omega_2$ in \eqref{eq:generic-central-extension} being a 2-cocycle hints us that 
classification of possible $\omega_2$'s can also be done in terms of group cohomology.
We need to check the condition whether $\omega_2$ is equivalent to $\omega_2'$. 

\section{Space groups}

In this section we are not going to show that there are 230 space groups in $\mathbb{R}^3$. 
Actually we usually call 2D space groups as \concept{wallpaper groups}. A group element of a space group 
is in the form of the following affine transformation
\begin{equation}
    (R| \vb*{t}): \vb*{v} \mapsto R \vb*{v} + \vb*{t},
\end{equation}
or in the form of 
\begin{equation}
    (R | \vb*{t}) : \tilde{\vb*{v}} \coloneqq \pmqty{\vb*{v} \\ 1} \mapsto \underbrace{\pmqty{R & \vb*{t} \\ 0 & 1 }}_{\coloneqq \tilde{R}} \tilde{\vb*{v}},
\end{equation}
which is the preferred form of space group operation in numerical calculations.

Detailed discussion on the structure of space groups can be found in 
Section.~\ref{solid-sec:space-group-structure-classification} in \soliddoc.
Here we list some of the elements. First we discuss point groups. 
We have $C_n$ axises, the generator of which is labeled as $n$: 
we have 1, 2, 3, 4, 6 axises, the first being the identity. We also have rotation-reflection axises
$\bar{1}, \bar{2}, \bar{3}, \bar{4}, \bar{6}$, the first being the inversion operation and the second 
being the mirror reflection. $\bar{3}$ is not an independent generator because we have 
\[
    (\bar{3})^3 = \bar{1}, \quad \bar{3} \bar{1} = 3.
\]
Similarly, $\bar{6}$ is also not an independent generator.

There are also \concept{nonsymmorphic} operations, which involve fraction translation. There are two 
types of nonsymmorphic generators: \concept{screw axis} $(c_n | n \vb*{R} / m)$, where $m = 1, 2, 3, \ldots$,
and \concept{glide plane} $(\sigma | \vb*{R} / 2)$.

We can find the ``deep'' reason of the emergence of nonsymmorphic operations . 
The point group of a space group $G$ is $K = G / \mathbb{T}$, where $\mathbb{T}$ is the translation group.
Note that if $G$ is nonsymmorphic, $K$ is \emph{not} a subgroup of $G$. For symmorphic space groups, $G$ is \emph{semidirect}
product of $K$ and $\mathbb{T}$. We have 
\begin{equation}
    G = \mathbb{T} \rtimes K.
\end{equation}

\section{Wave vector star and little group}

We define the \concept{little group} of $G$ to be the subgroup of $G$ where every group element keeps 
the wave vector invariant, i.e. 
\begin{equation}
    LG_{\vb*{k}} = \{g \in S | g \vb*{k} \simeq \vb*{k} \},
\end{equation}
where $\simeq$ means two wave vectors being equivalent, i.e. differ with a $\vb*{R}_m$.
We define $\vb*{k}^*$ to be the set of wave vectors that can be obtained by acting $G$ on $\vb*{k}$, and we have 
\begin{equation}
    \abs*{\vb*{k}^*} = \frac{\abs*{G}}{\abs*{LG}}.
\end{equation}
Since $T$ is a normal subgroup of $LG_{\vb*{k}}$, we define the \concept{little cogroup} 
\begin{equation}
    P G_{\vb*{k}} = \{ g \in G / \mathbb{T} | g \vb*{k} \simeq \vb*{k} \} = LG_{\vb*{k}} / \mathbb{T}.
\end{equation}
This gives a short exact sequence: 
\begin{equation}
    1 \to \mathbb{T} \to LG_{\vb*{k}} \to PG_{\vb*{k}} \to 1.
\end{equation}

Now if we know all irreducible representations of $LG_{\vb*{k}}$, then we can easily construct 

Representations of space groups can be found in \href{https://www.cryst.ehu.es/}{Bilbao Server}. 

\end{document}