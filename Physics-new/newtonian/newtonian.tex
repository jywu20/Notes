\documentclass[hyperref, UTF8, a4paper, oneside]{ctexbook}

\usepackage{geometry}
\usepackage{titling}
\usepackage{titlesec}
\usepackage{paralist}
\usepackage{footnote}
\usepackage{enumerate}
\usepackage{autobreak}
\usepackage{amsmath, amssymb, amsthm}
\usepackage{mathtools}
\usepackage{bbm}
\usepackage{cite}
\usepackage{graphicx}
\usepackage{subfigure}
\usepackage{physics}
\usepackage{siunitx}
\usepackage{tikz}
\usepackage[compat=1.1.0]{tikz-feynhand}
\usepackage[ruled, vlined, linesnumbered, noend]{algorithm2e}
\usepackage{xr-hyper}
\usepackage[colorlinks, linkcolor=black, anchorcolor=black, citecolor=black, filecolor=black]{hyperref}
\usepackage[most]{tcolorbox}
\usepackage{caption}
\usepackage{prettyref}

\externaldocument[optics-]{../optics/optics}[optics.pdf]
\externaldocument[vasp-]{../cond-comp/vasp/vasp}[vasp.pdf]
\externaldocument[qft-]{../relativistic-qft/relativistic-qft}[relativistic-qft.pdf]
\externaldocument[soft-]{../soft/soft}[soft.pdf]

\geometry{left=3.18cm,right=3.18cm,top=2.54cm,bottom=2.54cm}
\titlespacing{\paragraph}{0pt}{1pt}{10pt}[20pt]
\setlength{\droptitle}{-5em}
\preauthor{\vspace{-10pt}\begin{center}}
\postauthor{\par\end{center}}

\DeclareMathOperator{\timeorder}{\mathcal{T}}
\DeclareMathOperator{\diag}{diag}
\DeclareMathOperator{\legpoly}{P}
\DeclareMathOperator{\primevalue}{P}
\DeclareMathOperator{\sgn}{sgn}
\newcommand*{\ii}{\mathrm{i}}
\newcommand*{\ee}{\mathrm{e}}
\newcommand*{\const}{\mathrm{const}}
\newcommand*{\suchthat}{\quad \text{s.t.} \quad}
\newcommand*{\argmin}{\arg\min}
\newcommand*{\argmax}{\arg\max}
\newcommand*{\normalorder}[1]{: #1 :}
\newcommand*{\pair}[1]{\langle #1 \rangle}
\newcommand*{\fd}[1]{\mathcal{D} #1}

\newcommand*{\st}{\quad \text{s.t.} \quad}
\newcommand*{\natnums}{\mathbb{N}}
\newcommand*{\reals}{\mathbb{R}}
\newcommand*{\complexes}{\mathbb{C}}
\newcommand*{\ogroup}[1]{\mathrm{O}(#1)}
\newcommand*{\sogroup}[1]{\mathrm{SO}(#1)}\DeclareMathOperator{\laguerre}{L}
\newcommand*{\lsterm}[3]{$^{#1}${#2}$_{#3}$}
\newcommand*{\nuclear}[3]{^{#2}_{#3}\text{#1}}
\newcommand*{\hankelone}{\mathrm{h}^{(1)}}
\newcommand*{\hankeltwo}{\mathrm{h}^{(2)}}

\newenvironment{bigcase}{\left\{\quad\begin{aligned}}{\end{aligned}\right.}

\newrefformat{chap}{第\ref{#1}章}
\newrefformat{sec}{第\ref{#1}节}
\newrefformat{note}{注\ref{#1}}
\newrefformat{fig}{图\ref{#1}}
\newrefformat{alg}{算法\ref{#1}}
\newrefformat{back}{背景知识\ref{#1}}
\newrefformat{info}{资料框\ref{#1}}
\newrefformat{warn}{注意事项\ref{#1}}
\renewcommand{\autoref}{\prettyref}

\usetikzlibrary{arrows,shapes,positioning}
\usetikzlibrary{arrows.meta}
\usetikzlibrary{decorations.markings}
\tikzstyle arrowstyle=[scale=1]
\tikzstyle directed=[postaction={decorate,decoration={markings,
    mark=at position .5 with {\arrow[arrowstyle]{stealth}}}}]
\tikzstyle ray=[directed, thick]
\tikzstyle dot=[anchor=base,fill,circle,inner sep=1pt]

% Algorithm setting
\renewcommand{\algorithmcfname}{算法}
% Python-style code
\SetKwIF{If}{ElseIf}{Else}{if}{:}{elif:}{else:}{}
\SetKwFor{For}{for}{:}{}
\SetKwFor{While}{while}{:}{}
\SetKwInput{KwData}{输入}
\SetKwInput{KwResult}{输出}
\SetArgSty{textnormal}

\tcbuselibrary{skins, breakable, theorems}

\newtcbtheorem[number within=chapter]{back}{背景知识}%
  {colback=blue!5,colframe=blue!65,fonttitle=\bfseries, breakable}{back}
\newtcbtheorem[number within=chapter]{info}{资料框}%
  {colback=blue!5,colframe=blue!65,fonttitle=\bfseries, breakable}{info}
\newtcbtheorem[number within=chapter]{warning}{注意事项}%
  {colback=orange!5,colframe=orange!65,fonttitle=\bfseries, breakable}{warn}

\renewcommand{\emph}[1]{\textbf{#1}}
\newcommand*{\concept}[1]{\underline{\textbf{#1}}}

\numberwithin{equation}{chapter}

\newcommand{\hmn}[1]{% Hermann-Maguin notation
  \ensuremath{\begingroup\setupHMN #1\endgroup}%
}

\newcommand{\setupHMN}{%
  \doHMN{-}{\HMNoverline}%
  \doHMN{*}{\HMNminverse}%
  \doHMN{i}{\infty}
}

\newcommand{\doHMN}[2]{%
  \begingroup\lccode`~=`#1
  \lowercase{\endgroup\let~}#2%
  \mathcode`#1="8000
}

\newcommand{\HMNminverse}[1]{\frac{#1}{m}}
\newcommand{\HMNoverline}[1]{\mkern1mu\overline{\mkern-1mu#1\mkern-1mu}\mkern1mu}

\newcommand{\Ztwo}{$\mathbb{Z}_2$}

\newcommand{\bigO}[1]{\mathcal{O}(#1)}

\newcommand{\vaspdoc}{\href{../computational/vasp/vasp.pdf}{VASP笔记}}
\newcommand{\opticsdoc}{\href{../optics/optics}{光学笔记}}
\newcommand{\softdoc}{\href{../soft/soft}{流体和软物质物理笔记}}
\newcommand{\qftdoc}{\href{../relativistic-qft/relativistic-qft}{相对论性量子场论笔记}}

\title{牛顿力学和非相对论量子力学}
\author{吴晋渊}

\begin{document}

\maketitle

\part{牛顿力学}

\documentclass[UTF8, a4paper, oneside, scheme=plain, 12pt]{ctexrep}

\usepackage[T1]{fontenc}
\usepackage{libertinus}
\usepackage{geometry}
\usepackage{float}
\usepackage{titling}
\usepackage{titlesec}
\usepackage{paralist}
\usepackage{footnote}
\usepackage{enumerate}
\usepackage{amsmath, amsthm}
\usepackage{gb4e}
\noautomath
\usepackage{bbm}
\usepackage{textcomp}
\usepackage{soul}
\usepackage{graphicx}
\usepackage{siunitx}
\usepackage[table,xcdraw]{xcolor}
\usepackage{tikz}
\usepackage[ruled, vlined, linesnumbered, noend]{algorithm2e}
\usepackage{xr-hyper}
\usepackage[colorlinks, citecolor = purple]{hyperref} % linkcolor=black, anchorcolor=black, citecolor=black, filecolor=black
\usepackage[most]{tcolorbox}
\usepackage{caption}
\usepackage{subcaption}
\usepackage{booktabs}
\usepackage{multirow}
\usepackage[figuresright]{rotating}
\usepackage{acro}
\usepackage[citestyle=authoryear,backend=bibtex,natbib=true,doi=false,isbn=false,url=false]{biblatex}
\addbibresource{references/classical-grammars.bib}
\addbibresource{references/classical-lexicon.bib}
\addbibresource{references/historical-phonology.bib}
\addbibresource{references/general-typology.bib}
\addbibresource{references/other-languages.bib}
\usepackage{prettyref}

\geometry{left=3.18cm,right=3.18cm,top=2.54cm,bottom=2.54cm}
\titlespacing{\paragraph}{0pt}{1pt}{10pt}[20pt]
\setlength{\droptitle}{-5em}

\DeclareMathOperator{\timeorder}{\mathcal{T}}
\DeclareMathOperator{\diag}{diag}
\DeclareMathOperator{\legpoly}{P}
\DeclareMathOperator{\primevalue}{P}
\DeclareMathOperator{\sgn}{sgn}
\newcommand*{\ii}{\mathrm{i}}
\newcommand*{\ee}{\mathrm{e}}
\newcommand*{\const}{\mathrm{const}}
\newcommand*{\suchthat}{\quad \text{s.t.} \quad}
\newcommand*{\argmin}{\arg\min}
\newcommand*{\argmax}{\arg\max}
\newcommand*{\normalorder}[1]{: #1 :}
\newcommand*{\pair}[1]{\langle #1 \rangle}
\newcommand*{\fd}[1]{\mathcal{D} #1}
\newcommand*{\textto}{$\to$}
\newcommand*{\textgt}{$>$ }
\newcommand{\focus}[1]{\textbf{#1}}

\newcommand*{\citesec}[1]{\S~{#1}}
\newcommand*{\citechap}[1]{Ch.~{#1}}
\newcommand*{\citechaps}[1]{Chs.~{#1}}
\newcommand*{\citefig}[1]{Fig.~{#1}}
\newcommand*{\citetable}[1]{Table~{#1}}
\newcommand*{\citepage}[1]{p.~{#1}}
\newcommand*{\citepages}[1]{pp.~{#1}}
\newcommand*{\citefootnote}[1]{fn.~{#1}}

\newrefformat{sec}{\citesec{\ref{#1}}}
\newrefformat{fig}{\citefig{\ref{#1}}}
\newrefformat{tbl}{\citetable{\ref{#1}}}
\newrefformat{chap}{\citechap{\ref{#1}}}
\newrefformat{fn}{\citefootnote{\ref{#1}}}
\newrefformat{box}{Box~\ref{#1}}
\newrefformat{ex}{\ref{#1}}

% color boxes

\tcbuselibrary{skins, breakable, theorems}

\newtcbtheorem[number within=chapter]{infobox}{Box}{
    enhanced,
    boxrule=0pt,
    %colback=blue!5,
    %colframe=blue!5,
    colback=white,
    colframe=white,
    coltitle=blue!60,
    borderline west={4pt}{0pt}{blue!65},
    sharp corners,
    fonttitle=\bfseries, 
    breakable,
    before upper={\parindent15pt\noindent}}{box}
\newtcbtheorem[number within=chapter, use counter from=infobox]{theorybox}{Box}{
    enhanced,
    boxrule=0pt,
    %colback=orange!5, 
    %colframe=orange!5, 
    colback=white,
    colframe=white,
    coltitle=orange!65,
    borderline west={4pt}{0pt}{orange!65},
    sharp corners,
    fonttitle=\bfseries, 
    breakable,
    before upper={\parindent15pt\noindent}}{box}
\newtcbtheorem[number within=chapter, use counter from=infobox]{learnbox}{Box}{
    enhanced,
    boxrule=0pt,
    colback=green!5,
    colframe=green!5,
    coltitle=green!50,
    borderline west={4pt}{0pt}{green!65},
    sharp corners,
    fonttitle=\bfseries, 
    breakable,
    before upper={\parindent15pt\noindent}}{box}
\newtcbtheorem[number within=chapter, use counter from=infobox]{todobox}{Box}{
    enhanced,
    boxrule=0pt,
    colback=red!5,
    colframe=red!5,
    coltitle=red!50,
    borderline west={4pt}{0pt}{red!65},
    sharp corners,
    fonttitle=\bfseries, 
    breakable,
    before upper={\parindent15pt\noindent}}{box}
\newtcbtheorem[number within=chapter, use counter from=infobox]{perspectivebox}{Box}{
    enhanced,
    boxrule=0pt,
    %colback=red!5,
    %colframe=red!5,
    colback=white,
    colframe=white,
    coltitle=red!50,
    borderline west={4pt}{0pt}{red!65},
    sharp corners,
    fonttitle=\bfseries, 
    breakable,
    before upper={\parindent15pt\noindent}}{box}


\AtBeginEnvironment{infobox}{\small}
\AtBeginEnvironment{todobox}{\small}
\AtBeginEnvironment{theorybox}{\small}

\newcommand*{\concept}[1]{\textbf{#1}}
\newcommand*{\term}[1]{\emph{#1}}
\newcommand{\form}[1]{\emph{#1}}
\newcommand{\work}[1]{\textit{#1}}
\newcommand{\species}[1]{\textit{#1}}

\newcommand{\redp}{\textasciitilde}

\DeclareAcronym{blt}{short = BLT, long = Basic Linguistic Theory}
\DeclareAcronym{cgel}{short = CGEL, long = The Cambridge Grammar of the English Language}
\DeclareAcronym{dm}{short = DM, long = Distributed Morphology}
\DeclareAcronym{tag}{long = Tree-adjoining grammar, short = TAG}
\DeclareAcronym{sfp}{long = sentence-final particle, short = \textsc{sfp}}
\DeclareAcronym{np}{long = noun phrase, short = NP}
\DeclareAcronym{vp}{long = verb phrase, short = VP}
\DeclareAcronym{pp}{long = preposition phrase, short = PP}
\DeclareAcronym{cls}{long = classifier, short = CLS}
\DeclareAcronym{dist}{long = distal, short = DIST}
\DeclareAcronym{prox}{long = proximate, short = PROX}
\DeclareAcronym{dem}{long = demonstrative, short = DEM}
\DeclareAcronym{classify}{long = classifier, short = \textsc{cl}}
\DeclareAcronym{dur}{long = durative, short = DUR}
\DeclareAcronym{neg}{long = negative, short = \textsc{neg}}
\DeclareAcronym{cc}{long = copular complement, short = CC}
\DeclareAcronym{cs}{long = copular subject, short = CS}
\DeclareAcronym{tam}{long = {tense, aspect, mood}, short = TAM}
\DeclareAcronym{past}{long = past, short = PST}
\DeclareAcronym{nonpast}{long = non-past, short = NPST}
\DeclareAcronym{present}{long = present, short = PRES}
\DeclareAcronym{progressive}{long = progressive, short = \textsc{poss}}
\DeclareAcronym{perfect}{long = perfect, short = \textsc{perf}}
\DeclareAcronym{passive}{long = passive, short = \textsc{pass}}
\DeclareAcronym{copula}{long = copula, short = COP}
\DeclareAcronym{possessive}{long = possessive, short = \textsc{poss}}

\newcommand{\asis}[1]{\textsc{#1}}
\newcommand{\oneof}[1]{{#1}}
\newcommand*{\homo}[2]{#1$_{\text{#2}}$}

\newcommand{\cgel}{\href{../English/cambridge.pdf}{my notes about CGEL}}
\newcommand{\latin}{\href{../Latin/latin-notes.pdf}{my notes about Latin}}
\newcommand{\alignment}{\href{../alignment/alignment.pdf}{my notes about alignment}}
\newcommand{\exerciseone}{\href{../Exercise/2021-3.pdf}{this exercise}}
\newcommand{\method}{\href{../methodology/glossing.pdf}{this note about my understanding of descriptive grammars}}

\newcommand{\ala}{à la}
\newcommand{\translate}[1]{`#1'}
\newcommand{\vP}{\textit{v}P}
\newcommand*{\category}[1]{\textsc{#1}}
\newcommand*{\specialunit}[1]{$<$\textit{#1}$>$}
\newcommand{\before}{$> \ $}

% Make subsubsection labeled
\setcounter{secnumdepth}{4}
\setcounter{tocdepth}{4}
% reset example counter every chapter (but do not include the chapter number to the label)
\counterwithin{exx}{chapter} 

% Reference formats
\renewcommand*{\nameyeardelim}{\space} % No comma between year and name
\DeclareNameAlias{sortname}{family-given} % Putting the family name before the given name
\DeclareNameAlias{default}{family-given} 
\DeclareFieldFormat{labelnumberwidth}{} % No number label like [12] in the reference list
\setlength{\biblabelsep}{0pt} % No space for these labels

\makeindex

\title{Notes about Classical Chinese}
\author{Jinyuan Wu}

\begin{document}

\automath

\maketitle

\chapter{Introduction}

\section{The name of the language}

This note is about Classical Chinese,
the high variety of more than two millennia of diglossia in China.
The language is known natively (in Mandarin Chinese) as 文言 (\translate{lit. cultured speech})
or sometimes 古文 (\translate{lit. ancient articles})
or 古汉语 (\translate{lit. ancient Chinese}).
Note that there were several stages of the development of Chinese
and Classical Chinese is mostly (but not completely) based on Old Chinese
(\prettyref{sec:introduction.history}).

The language is sometimes known as \form{Wen-li} by Western missionaries,
especially in Bible translation.
This seems to be a misunderstanding of the word 文理,
which is a nominal compound and means rhetorics (i.e. 文) and meanings (i.e. 理) of literature works.
An educated person therefore would be described as ``通文理'' (\translate{fluent in rhetorics and meanings}).
Such a person of course would have decent understanding of Classical Chinese,
and hence 文理 was probably mistranslated as ``Classical Chinese'',
although the word was not natively used to refer to the latter.

\section{Historical background}\label{sec:introduction.history}

Since there was no attempt at explicit and systematic grammatical standardization
(\prettyref{sec:introduction.previous.tradition}),
prescriptive authority of Classical Chinese was a collection of canonical literature works
consensually regarded as classical (\prettyref{sec:introduction.text}).
The whole canon was finished before the collapse of Han
and therefore falls under the term Old Chinese.
Both temporal and regional variances can be observed in Old Chinese texts, though,
and not all varieties contribute to Classical Chinese equally.
In this section, we briefly overview the history of Sinitic language(s)
and analyze how they shape Classical Chinese.

\subsection{Pre-classical period}

The earliest attested Sinitic texts were oracle bone inscriptions,
a 20th century archeological re-discovery not known to Classical Chinese authors.
For them, the earliest available texts are 
documents preserved in 《尚书》 (lit. \translate{venerated documents}),
often known as the \work{Book of Documents} in English.
Since these texts are from ancient kings 
whose deeds were romanticized by Confucian scholars,
these texts were highly venerated and yet deemed as
诘屈聱牙 (\translate{twisted, hard to pronounce})
by post-Classical authors.%
\footnote{
    For example by Han Yu in 《进学解》 (\work{Analysis of academic advancement}).
}
They were something that had to be read with commentaries,
the latter written in easier Classical Chinese.
These documents therefore should be regarded as pre-Classical,
although they did contribute sporadic phrases
and grammatical words (e.g. the copula 惟 or the pronoun 厥)
that were occasionally used in Classical Chinese works as a way to polish an article.

One thing worth mentioning is that 
the language of the \work{Book of Documents} and the language of oracle bone inscriptions are not identical.
The most notable fact on this aspect is that
the aforementioned pronoun 厥 appears frequently in the \work{Documents},
but it appears neither in oracle bone inscriptions nor in Spring and Autumn works. 
Possibly, \work{Book of Documents} contains predominantly early Zhou dynasty texts,
while oracle bones dates back to Shang,
and the differences we are observing reflect dialectal differences between the ruling classes of the two dynasties.

Another fairly early source is 《诗经》 (lit. \translate{poem classics}),
also known as the \work{Book of Odes},
which contains poems dates back to as early as early Zhou.
We note that the \work{Odes} is usually considered Classical,
and yet given its poetic nature, its influences to Classical proses are not direct.

\subsection{Spring and Autumn and Warring States}

The majority of texts that shaped Classical Chinese proses
were written in a time when Zhou was already substantially weakened.
This period that witnessed prolificacy of Old Chinese works
can be divided into two periods:
the Spring and Autumn period which was filled with chaotic (but not intense) wars between numerous dukedoms,
and the Warring States period which observed intense wars between seven major states,
resulting in a unified Qin empire,
which soon broke down because of resistances to its barbaric policies 
and eventually was superseded by Han dynasty (\prettyref{sec:introduction.history.han}).
The language of this period diverges tremendously from the pre-Classical period.
For example, the copula 惟 had died out of use 
and the copula construction had been largely replaced by the nominal predication construction
(\prettyref{sec:grammatical.clause.nominal}).
The conjunction 而 is never seen in pre-Sprint and Autumn texts
but had already made its way into the \work{Analects}.
The lexicon also underwent huge changes.

\begin{todobox}{Lexicon change}{lexicon-change-oc}
    List some lexicon changes.
\end{todobox}

There are clues suggesting regional variances.
Students of Confucius noticed that when he recited Classical texts and presided rituals,
he used 雅言 or \translate{elegant speech} (\work{Analects} 7:18).
This suggests a possible diglossia at as early as Confucius's own age,
with the ``elegant speech'' conceivably being the language of intellectuals of Zhou Dynasty.
Comparison between the language of Classical proses and the language(s) of poetry
shows the relative homogeneity of the former,
while the latter both demonstrate divergence from the language of the proses
and regional differences.

\begin{todobox}{Peotry and prose}{peotry-and-prose-before-han}
    This is presumably due to how the texts were transmitted.
    It is likely that they were passed by recitation,
    and regularization happened to proses when there was a predominant dialect,
    while the prosody and rhyme structures of poems
    efficiently locked them to their original forms.
\end{todobox}

The language of 楚辞 (\work{Verses of Chu}), for example,
has the following differences with the language of the proses. 
The first is a Kra–Dai substrate.

\begin{todobox}{Chu dialect}{chu-dialect}
    Find references.
\end{todobox}

The language of the \work{Odes} also seems to slightly deviates from 

Dialectal differences have also been observed within the \work{Odes} \citep{list2017vowel}.



\subsection{Han dynasty}\label{sec:introduction.history.han}

The last batch of uncontroversially classical works were composed during Han dynasty,
among them the most important being \work{Records of the Grand Historian}.
The language of \form{Records of the Grand Historian} shows notable but largely qualitative differences
compared with earlier historical works,
the most important one being 《左传》.
Notable changes include more pre-verbal adverbials,
reduction of prepositional verbs,
regularization of constituent orders,
and also proliferation of disyllable words
It is therefore suggested that Han dynasty texts and pre-Qin texts 
reflect two stages of post-Zhou developments of Chinese,
although the change was definitely not as radical as the change 
from the \form{Documents} to Spring and Autumn texts 
\citep[\citepages{260-264}]{he2005shiji}.

\subsection{Post-Classical periods}

The end of Old Chinese -- and hence the end of the classical period --
is marked by the collapse of the case inflection in the personal pronoun system,
the emergence of 是 as a copula (and not just a demonstrative),
the appearance of the disposal construction (i.e. the 把 construction)
and the so-called long passive construction.

\begin{todobox}{References for Middle Chinese and modern Mandarin}{middle-chinese-ref}
    \begin{itemize}
        \item James Huang
        \item etc.
    \end{itemize}
\end{todobox}

Expectedly, despite purification attempts,
vernacular elements made their ways into not only administrative documents
but also pure literature and scholar works.
Classical Chinese or 文言, in the broadest sense,
is a term that covers all genres whose grammars are roughly based on the Old Chinese canon
but may have a varieties of innovations.

\begin{todobox}{Late regularization attempts}{hanyu-etc-regularization}
    韩愈、因明学
\end{todobox}

\section{About this work}

\subsection{Theoretical framework}\label{sec:intro.theory}

{\small
The theoretical framework of this work is essentially Distributed Morphology plus Cartographic Syntax,
although I intentionally choose to reuse the terminology in descriptive grammars (see below).

The architecture of grammar is assumed to be in line with the basic assumptions of Distributed Morphology (\prettyref{sec:grammatical.intro}),
where we have a list of roots (List A),
each of which is only compatible with certain syntactic positions in post-syntactic phonological realization (List B),
and grammatical objects -- bundle of roots and functional heads, or even larger objects -- can be lexicalized with custom meanings (List C).
Lexicalization or in other words \emph{semantic} fossilization is important for certain aspects of Classical Chinese grammar (e.g. \prettyref{sec:grammatical.clause.verbal.argument-structure.causative.fossilization}),
which however can be well captured within Distributed Morphology,
without lexicalist assumptions (cf. \citealt{bruening2018lexicalist}).
We note that lexicalization of a complex structure may eventually lead to the collapse of its internal makeup, causing \emph{syntactic} fossilization.
An example is the collapse of verbs taking complement clauses into compound verbs.
The end point of fossilization is a synchronic \emph{root},
on which only diachronic analysis is possible.
Syntactic fossilization is important in Classical Chinese due to its long history.

The analyses in \prettyref{sec:grammatical.clause} and \prettyref{sec:grammatical.noun-phrase} are inspired by the extended CP and DP structures proposed in Cartography.
To avoid confusion caused by technical terms in generative syntax,
I intentionally use terms like \term{sentence}, \term{nucleus clause} and \term{argument structure} in place of CP, TP and vP.
Further, the notion of functional heads should be avoided, and concepts like SpecTP and SpecvP have to be replaced by concepts like \term{clausal subject/pivot} and \term{subject in the argument structure} (\prettyref{sec:grammatical.verbal.subject}),
and we should talk about \term{noun phrases} instead of DPs.
Similarly we cannot talk about \category{do} or \category{cause} light verbs;
I replace these concepts by concepts like \term{\category{do} clause} or \term{\category{cause} clause} (\prettyref{sec:grammatical.clause.verbal.argument}).
The ``core'' of CPs and DPs (i.e. the roots at their centers plus the categorizers) -- instead of the grammatical markers -- should be known as heads. 
After doing so, we rediscover the good old subject-predicate and verb-object relations,
subordination and coordination structures, and other \term{constructions}, as well as \term{grammatical relations} within them.%
\footnote{
    Here the term is used \emph{without} the implication 
    that a construction is somehow understood as a whole 
    and its internal structures should not be further analyzed,
    contrary to the fundamental hypothesis of e.g. various Construction Grammars.
}
This procedure has been demonstrated in \citet{deng2010},
which shows that Minimalist generative syntax and the constituency-based analysis strongly inspired by American structuralism and is exemplified in \citet{cgel} and works outlined in \prettyref{sec:introduction.previous.modern} are compatible to each other,
with the former being a more concise form of the latter 
and the latter being a logical consequence of the former under certain assumptions.%
\footnote{
    Note that \citet{deng2010}, \citet{cgel} and works in \prettyref{sec:introduction.previous.modern} are all lexicalist, which we have argued is not necessary to account for phenomena purportedly supporting the lexicalist hypothesis.
} 

Grammars can be written in terms of \emph{dependency relations}, instead of \emph{constituency relations}.
The two however are largely equivalent \citep{boston2009dependency}.
In his Basic Linguistic Theory, \citet{dixon2009basic} fervently argues against constituency analysis (and also other aspects of generative syntax) and advocates for a ``flat'' constituency structure, possibly with the levels of clause and noun phrase only,
where the rest of the grammatical information is represented by dependency relations 
(e.g. \citealt[\citepage{49}]{dixon2009basic}).
Yet the primacy of noun phrases and clauses in syntax is indeed acknowledged by generative syntax (for instance they are \term{phases}), and the binary constituency relations in generative syntax and in American structuralism have consequences.
For instance, when applied to the argument structure,
they are related to extractional properties of arguments in valency alternation (\prettyref{sec:grammatical.verbal.subject.argument-structure}),
and the subject-predicate binary division is directly related to the clausal pivotal status of the subject.
These phenomena of course have to be taken into account by Basic Linguistic Theory,
and labels like ``clausal pivot'' and ``surface S, deep O'' have to be attached to dependency arcs between the verb and the arguments,
essentially labeling the ``distance'' between the two. 
As a parallel, the ancient India grammarian Pāṇini initially proposes a grammatical framework  in which arguments are all equal, but later commentators still effectively set up a pivot position in the argument structure (\prettyref{box:panini-difference}).
Other phenomena that can be directly include
ordering of tense and aspect auxiliary and adverbs à la Cartography, which is attested in languages completely unrelated to Germanic or Romance languages that have been thoroughly investigated in Cartography (e.g. \citealt[\citepages{166-167}]{grimm2021grammar}),
and also available clause sizes: it is rare that in a language with tense, aspect and modality marking,
there exists a clause type with discourse devices but no tense, aspect and modality marking,
which can be explained simply by the fact that CP is built on TP.

Besides the theoretical problems outlined above, there are also some minor, largely notational inconsistencies between the grammatical theories mentioned here.
One such inconsistency is the definition of the \term{phrase}.
\citet{dixon2009basic} calls the main verb plus auxiliaries -- without any argument -- as the \term{verb phrase}.
This actually makes sense in generative syntax because arguments are phases themselves,
and the tense, aspect and modality categories marked by the auxiliaries are in some senses closer to the main verb.
Yet the term \term{verb phrase} generally means the verb plus internal arguments in constituency-based analyses.
Another problem is the definition of \term{word}.
Since we reject the lexicalist hypothesis,
we need to distinguish between phonological wordhood,
morphological wordhood (the boundary of the morphological template is the boundary of the morphological word),
and syntactic wordhood.
Syntactic wordhood in turn has several definitions.
We can define a word to be a very small constituent:
if it is impossible to infer the argument structure of a compound verb in a given language,
then we conclude that the two branches of the compound are not categorized,
and therefore the compound is a rather small constituent and hence a word.
But in this way \form{sinned} is \emph{not} a syntactic word as it involves a clausal category (i.e. the past tense).
For \form{sinned} to be a syntactic word, syntactic wordhood can be based on Dixon's verb phrase,
and inevitably \form{have been eating} is a syntactic word.

In conclusion, we maintain that Chomskyan generative syntax,
American structuralism-styled constituency analysis as in \citet{cgel},
and Basic Linguistic Theory (the de facto unified framework in modern linguistic description of underdocumented languages) are coherent and can be seen as three ``representations'' of the same grammatical complexity class,
and their differences are mostly notational.%
\footnote{
    A more important controversy is the mental status of grammar.
    The position of this work treats grammar as a semi-autonomous component of human's cognitive abilities.
    But it has been argued that grammatical constructions originate from domain-general cognitive abilities, and there is actually no such thing as an autonomous mental grammar.
    Detailed discussions on this topic are far beyond the scope of this work,
    and can only be finally settled down with the assistance of neurological studies.
    Here we just note that currently no comprehensive description of a language has been successfully attempted under this line of thinking.
    On the contrary, in physicists' terms, the grammatical framework adopted here is at least a good \term{effective theory}.

    Another topic is whether natural language-like ``grammars'' appear in systems that are demonstrably independent to the human neurological linguistic capacity,
    which may also challenge the domain-specific status of the latter.
    This is partly discussed in \prettyref{sec:writing-system.theoretical}.
    Note however that such discussions are largely irrelevant to linguistic description,
    for reasons mentioned above.
}
Which framework to use is to be determined by the properties of the language.
For instance, although the definition of the verb phrase does not alternate the grammar system substantially,
Dixon's definition works more smoothly for a language with a lot of auxiliaries but rather infrequent subject-sharing coordinations.
It turns out that for most constructions, the method in \citet{cgel} is a good choice for Classical Chinese.


}

\subsection{Coverage}\label{sec:introduction.theory.coverage}

Contemporary grammatical description typically starts in a topic-by-topic way:
the first grammar of a language likely outlines how a head noun is modified by an adjective or another noun,
while the relative order of different modifiers in the noun phrase and their scopes are typically skipped.
To some degree, such a strategy may be described as an \emph{ethno-philological} approach:
it prepares the reader to comprehend \emph{natural texts} in the language being described,
as complex structures, after all, tend to appear less frequently than simpler ones.
Classical Chinese is a classical language, and expectedly, philology-oriented works have dominated the field of grammatical research on Classical Chinese.

This work, on the other hand, is an attempt to study the grammar of Classical Chinese as a machine that takes lexical items and produces \emph{arbitrarily} complex utterances,
and to understand the structure of this machine.
The approach is admittedly inherently problematic for a dead language,
which has no native speaker with acceptability intuitions.
Further, as is outlined in \prettyref{sec:introduction.history},
texts that are considered to be define Classical Chinese have great internal diversity.
The grammatical system presented in this work therefore resembles what \citet{cgel} calls \term{International Standard English},
i.e. the shared core of all important contemporary varieties of English.

It should be noted that the existence of such a shared Classical grammatical core should not be taken for granted in all Classical-looking texts.
During the Republican period of modern Chinese history, for example,
some official documents were written in a pseudo-Classical style:
the abstract syntax behind these documents can well be captured within the framework of modern Mandarin,
but the possessive marker 的 is replaced by 之,
and the relative marker 的 in headless relative clauses is replaced by 者 (不服管教的 \translate{who does not conform to instructions} is replaced by 不服管教者).
What is Classical in these documents is primarily the superficial forms.
Still, because of the highly formulaic nature of these documents (for instance, usually no sentence final particle appears, and valency alternation besides the passive is discouraged),
it is also likely that these documents can be considered as Classical in syntax as well.
More details can be found in \prettyref{sec:genres.pseudo}.

\subsection{Notations}

In \prettyref{sec:intro.theory}, we have mentioned that the framework of \citet{cgel} seems to be a good starting point of the description of Classical Chinese.
An example of the tree diagram representation of the constituency analysis (e.g. \citealt[\citepage{954} {[9]}]{cgel}) is shown in \prettyref{fig:tree-example}.
This representation does have one problem:
it does not clearly distinguish function words from content words,
while the distinction has syntactic consequences 
(see e.g. \prettyref{sec:grammatical.clause.verbal.tam} and \prettyref{sec:grammatical.clause.argument.verbal-complementation.direct-quotation}).

\begin{figure}[H]
    \caption{Tree diagrammatic analysis of 季氏將伐顓臾 \translate{Jishi (a nobel family which controls the politics of the Kingdom of Lu) is going to invade Zhuanyu (a vassal state of Lu).} (\work{Analects})}
    \label{fig:tree-example}
    \centering
    \tikzset{every picture/.style={line width=0.75pt}} %set default line width to 0.75pt        

    \begin{tikzpicture}[x=0.75pt,y=0.75pt,yscale=-1,xscale=1]

    %Straight Lines [id:da11048781322003709] 
    \draw    (142.83,60.72) -- (42.83,81.72) ;
    %Straight Lines [id:da299417566562584] 
    \draw    (142.83,60.72) -- (245.83,81.72) ;
    %Straight Lines [id:da054868595215823435] 
    \draw    (245.83,126.72) -- (166.83,146.72) ;
    %Straight Lines [id:da8396665320183578] 
    \draw    (245.83,126.72) -- (346.83,146.72) ;
    %Straight Lines [id:da7685535117555358] 
    \draw    (341.83,193.72) -- (262.83,213.72) ;
    %Straight Lines [id:da8037729579653405] 
    \draw    (341.83,193.72) -- (442.83,213.72) ;
    %Straight Lines [id:da9049141182356701] 
    \draw    (442.83,260.72) -- (428.83,404.72) ;
    %Straight Lines [id:da5026244353726979] 
    \draw    (442.83,260.72) -- (464.83,404.72) ;
    %Straight Lines [id:da12606692518025153] 
    \draw    (464.83,404.72) -- (428.83,404.72) ;
    %Straight Lines [id:da010652791992527] 
    \draw    (262.83,258.72) -- (262.83,402.72) ;
    %Straight Lines [id:da28218202395457315] 
    \draw    (166.83,171.72) -- (166.83,402.72) ;
    %Straight Lines [id:da7946233083294161] 
    \draw    (42.83,129.72) -- (27.83,401.72) ;
    %Straight Lines [id:da07047783638293326] 
    \draw    (42.83,129.72) -- (57.83,402.72) ;
    %Straight Lines [id:da5251636947941096] 
    \draw    (57.83,401.72) -- (27.83,401.72) ;

    % Text Node
    \draw (123,37) node [anchor=north west][inner sep=0.75pt]   [align=left] {Clause};
    % Text Node
    \draw (42.83,84.72) node [anchor=north] [inner sep=0.75pt]   [align=left] {\begin{minipage}[lt]{39.23pt}\setlength\topsep{0pt}
    \begin{center}
    Subject:\\NP
    \end{center}

    \end{minipage}};
    % Text Node
    \draw (245.83,78) node [anchor=north] [inner sep=0.75pt]   [align=left] {\begin{minipage}[lt]{47.6pt}\setlength\topsep{0pt}
    \begin{center}
    Predicate:\\VP
    \end{center}

    \end{minipage}};
    % Text Node
    \draw (166.83,149.72) node [anchor=north] [inner sep=0.75pt]   [align=left] {\begin{minipage}[lt]{21.41pt}\setlength\topsep{0pt}
    \begin{center}
    Aux
    \end{center}

    \end{minipage}};
    % Text Node
    \draw (346.83,149.72) node [anchor=north] [inner sep=0.75pt]   [align=left] {\begin{minipage}[lt]{38.85pt}\setlength\topsep{0pt}
    \begin{center}
    Head:\\core VP
    \end{center}

    \end{minipage}};
    % Text Node
    \draw (262.83,210) node [anchor=north] [inner sep=0.75pt]   [align=left] {\begin{minipage}[lt]{52.16pt}\setlength\topsep{0pt}
    \begin{center}
    Predicator:\\V
    \end{center}

    \end{minipage}};
    % Text Node
    \draw (442.83,216.72) node [anchor=north] [inner sep=0.75pt]   [align=left] {\begin{minipage}[lt]{35.83pt}\setlength\topsep{0pt}
    \begin{center}
    Object:\\NP
    \end{center}

    \end{minipage}};
    % Text Node
    \draw (446.83,407.72) node [anchor=north] [inner sep=0.75pt]   [align=left] {\begin{minipage}[lt]{240pt}\setlength\topsep{0pt}
    \begin{center}
    顓臾
    \end{center}

    \end{minipage}};
    % Text Node
    \draw (262.83,405.72) node [anchor=north] [inner sep=0.75pt]   [align=left] {\begin{minipage}[lt]{12.92pt}\setlength\topsep{0pt}
    \begin{center}
    伐
    \end{center}

    \end{minipage}};
    % Text Node
    \draw (166.83,405.72) node [anchor=north] [inner sep=0.75pt]   [align=left] {\begin{minipage}[lt]{12.92pt}\setlength\topsep{0pt}
    \begin{center}
    將
    \end{center}

    \end{minipage}};
    % Text Node
    \draw (42.83,404.72) node [anchor=north] [inner sep=0.75pt]   [align=left] {\begin{minipage}[lt]{40pt}\setlength\topsep{0pt}
    \begin{center}
    季氏
    \end{center}

    \end{minipage}};


    \end{tikzpicture}

\end{figure}

Researchers in modern China conventionally draw lines below constituents with labels to represent constituency relations.
\prettyref{fig:underline-example} contains exactly the same information in \prettyref{fig:tree-example} and is more compact.
It also highlights the lexical/functional distinction by refraining from giving a label to the  auxiliary 將,
implying that it is a grammatical marker and not a (content) constituent of the predicate.

\begin{figure}[H]
    \caption{More compact version of \prettyref{fig:tree-example}}
    \label{fig:underline-example}
    \centering
    \tikzset{every picture/.style={line width=0.75pt}} %set default line width to 0.75pt        

    \begin{tikzpicture}[x=0.75pt,y=0.75pt,yscale=-1,xscale=1]
    %uncomment if require: \path (0,300); %set diagram left start at 0, and has height of 300
    
    %Straight Lines [id:da2012441962165148] 
    \draw    (112,78) -- (164.83,78) ;
    %Straight Lines [id:da23301561594500164] 
    \draw    (186.83,78) -- (353.83,78) ;
    %Straight Lines [id:da4829535964487486] 
    \draw    (230.83,106) -- (281.83,106) ;
    %Straight Lines [id:da7895684119531474] 
    \draw    (301.83,106) -- (352.83,106) ;
    
    % Text Node
    \draw (321.83,55) node [anchor=north] [inner sep=0.75pt]   [align=left] {\begin{minipage}[lt]{40pt}\setlength\topsep{0pt}
    \begin{center}
    顓臾
    \end{center}
    
    \end{minipage}};
    % Text Node
    \draw (258.83,55) node [anchor=north] [inner sep=0.75pt]   [align=left] {\begin{minipage}[lt]{12.92pt}\setlength\topsep{0pt}
    \begin{center}
    伐
    \end{center}
    
    \end{minipage}};
    % Text Node
    \draw (195.83,55) node [anchor=north] [inner sep=0.75pt]   [align=left] {\begin{minipage}[lt]{12.92pt}\setlength\topsep{0pt}
    \begin{center}
    將
    \end{center}
    
    \end{minipage}};
    % Text Node
    \draw (138.42,71) node [anchor=south] [inner sep=0.75pt]   [align=left] {\begin{minipage}[lt]{40pt}\setlength\topsep{0pt}
    \begin{center}
    季氏
    \end{center}
    
    \end{minipage}};
    % Text Node
    \draw (138.42,81) node [anchor=north] [inner sep=0.75pt]   [align=left] {subject};
    % Text Node
    \draw (270.33,81) node [anchor=north] [inner sep=0.75pt]   [align=left] {predicate};
    % Text Node
    \draw (256.33,109) node [anchor=north] [inner sep=0.75pt]   [align=left] {predicator};
    % Text Node
    \draw (327.33,109) node [anchor=north] [inner sep=0.75pt]   [align=left] {object};
    
    
    \end{tikzpicture}
    
\end{figure}

The problem of \prettyref{fig:underline-example} is also its compactness:
the label of a constituent cannot be too long, or otherwise the whole diagram cannot be fit within one line.
In this work, we represent examples in a way that is more aligned to modern description grammars:
we place constituents in brackets, and then label these brackets by subscripts.
(\ref{ex:introduction.theory.format.1}) is an example.
Here we do not put grammatical markers (e.g. the auxiliary 將) in brackets to highlight the lexical/functional distinction,
consistent with the suggestion of \citet[\citepage{49}]{dixon2009basic}.

\begin{exe}
    \ex\label{ex:introduction.theory.format.1} 
    \gll [[季-氏]_{\text{subject: NP}} [將_{\text{\category{tam}}} [伐 顓臾]_{\text{core VP}}]_{\text{predicate: VP}}]_{\text{verbal clause}} \\
    Ji-family.branch.lineage.name will  subdue Zhuanyu \\
    \glt\translate{Jishi is going to invade Zhuanyu.} (\work{Analects})
\end{exe}

Note that in (\ref{ex:introduction.theory.format.1}) we do not transcribe texts written in Chinese characters into any alphabetic form.
This is a practice \emph{against} contemporary common practices in descriptive linguistics.
Just like Latin, historically, Classical Chinese was read by different readers with different phonological systems.
The texts that set standards for Classical Chinese were excerpts from several closely related varieties of Old Chinese, which were likely spoken languages when the texts were written
(\prettyref{sec:introduction.history}),
but reconstruction of the phonology of Old Chinese remains a highly debated topic,
and no mutually accepted lossless romanization that reflects the historical phonology exists currently.
Alternatively we may choose to use the pronunciation of modern Standard Mandarin,
but this does not give us much insights into the phonological reality of these texts.
Therefore, when doing interlinear glossing, we keep the Chinese characters.
When translating texts into English, modern Standard Mandarin \form{pinyin} (i.e. the standard romanization) is used for proper names.
This is however just for convenience, as the \form{pinyin} representations of these names also appear in the gloss.

\subsection{Organization of the book}

We adopt a top-down order in writing this book.
The starting point is the ``philological'', surface-oriented enumeration of the linear templates of attested clauses and noun phrases in \prettyref{sec:grammatical.clause.nominal}, \prettyref{sec:grammatical.clause.verbal}, and \prettyref{sec:grammatical.clause.verbal}.
Grammatical systems identified when analyzing and comparing surface forms
are then introduced in the rest of the sections of \prettyref{sec:grammatical.clause} and \prettyref{sec:grammatical.noun-phrase},
based on which a lexical parts of speech distinction is made.
If a section in \prettyref{chap:grammatical} is too long,
it is moved to chapters following \prettyref{chap:grammatical}.

This organization is in contrast with many descriptive grammars,
which start with e.g. the nominal and verbal morphology,
and then discuss grammatical relations like possession, modification, and argument structure,
and then outline the structure of the simple clause and complex clauses.
This is not the approach taken here,
partly because of what is discussed in \prettyref{sec:introduction.theory.coverage},
partly because Classical Chinese is rather analytic in its grammar,
and a well organized study of its grammar has to have the whole syntactic environment in mind:
otherwise it will likely collapse into a dictionary of the various usages of function words.
This is exactly what happened in the native grammatical tradition of Classical Chinese
(\prettyref{sec:introduction.previous.tradition}).

\section{Previous studies}

\subsection{Native grammatical traditions}\label{sec:introduction.previous.tradition}

Classical Chinese authors had conversations about grammaticality 
and uses of grammatical particles
reminiscent of how English native speakers
with some exposure to the study of English grammar but no formal training:
``delete the \form{the} here and your sentence looks more concise''.
No attempts were made to establish intermediate concepts between words and utterances,
like structural templates of phrases or grammatical relations, 
and to organize the grammar as a machine producing acceptable utterances:
discussions on grammatical topics were either for education or for rhetorics.

The grammatical awareness of Classical Chinese authors was somehow comparable to 
what an ancient Roman grammarian or \form{grammaticus} did,
who set his main role as a secondary educator,
refrained from analyzing some sort of ``underlying'' or ``internalized'' system behind the surface forms
and was satisfied by mostly surface-oriented patterns,
and would not set up any intermediate concepts between the word and the utterance
\citep[\citepages{7,35,47-48}]{matthews2019graeco}.
On the other hand, this approach is contrary to the practice
of the Paninian Sanskrit grammatical tradition,
which, in today's terminology, starts with dependency relations  and abstract features (\prettyref{sec:intro.theory}; \prettyref{box:panini-difference})
and uses a set of morphophonological rewriting rules to produce the corresponding surface forms
\citep{kiparsky2009architecture}.

\begin{theorybox}{Where does Pāṇini disagree with modern linguists?}{panini-difference}
    The main difference between Pāṇini's treatment of Sanskrit and modern linguistic theories
    is that Pāṇini apparently treats all dependency relations equally
    and there is, for example, no concept of the pivot or the ``external argument'' of a clause.
    This is however modified in the commentaries of his \work{Aṣṭādhyāyī},
    which explicitly allows an argument being promoted to the agent position 
    because of the intentions of the speaker \citep{keidan2017subjecthood}.
    The agent position thus becomes a subject position at least in the argument structure, consistent with modern practices (\prettyref{sec:grammatical.verbal.subject.argument-structure}).

    The Paninian tradition therefore is extremely close to modern linguistic description practice;
    the most important difference probably is that
    modern linguistic description, practically, may even be less rigorous than \work{Aṣṭādhyāyī},
    because of possible competing ``mind grammars'' among speakers with mutual intelligibility
    or even within the mind of one speaker,
    and also the fact that a description as detailed as \work{Aṣṭādhyāyī}
    requires corpus data whose quality and quantity exceed the capacity of most field linguists.
\end{theorybox}

The Classical Chinese grammatical tradition appears even looser compared with the Roman tradition
in that the former did not even attempt to recognize parts of speech;
this however was deeply rooted in the structure of Classical Chinese
in that 

\begin{todobox}{Ancient Chinese grammatical tradition and Roman tradition}{china-rome-compare}
    Is the situation somehow close to what a Roman grammarian (\form{grammaticus}) would do?
    It seems that Roman grammarians also didn't care about abstract structures.
    See:
    \begin{itemize}
        \item Use and Function of Grammatical Examples in Roman Grammarians
        \item Quintilian’s ‘Grammar’ (Inst.1.4-8) and its Importance for the History of Roman Grammar
        \item What Graeco-Roman Grammar was about
    \end{itemize}
\end{todobox}

On the other hand, phonology was an active topic in ancient China.
This was probably due to the influence of 

\subsection{Perspectives of European missionaries}

Large-scale, systematic and reliable grammatical description of Classical Chinese had unfortunately been largely lacking for quite a while, and until recently, when Classical Chinese was studied linguistically, it was the historical phonology that was studied \citep[\citepage{xiii}]{pulleyblank1995outline}.
Several missionary grammars however were still produced 

\begin{todobox}{List of notable works}{missionary-work}
    马若瑟 汉语札记

    1652年,卫匡国完成了《中国文法》一书。他將汉语划分出十大词类,并在动词部分对汉语主动语态和被动语态有所介绍。1703年万济国的《华语官话语法》,依旧沿袭拉丁语法,进行词类划分,并对其形态和范畴进行描写。除此之外,对汉语的一些造句规则(主要是主动句型和被动句型)也进行了归纳说明。1728年,马若瑟的《汉语札记》首次区分了汉语口语和书面语的语法规则,指出了汉语中“实字”与“虚字”的区别。不过马若瑟并没有依照拉丁语法构造汉语语法规则,而是从语言事实出发对汉语进行描写,这样的做法更符合汉语言的实际情况。

    进入19世纪,随着欧洲汉学的兴起,越来越多的传教士和专业学者投身于汉语的语法研究中。法国汉学家雷慕莎的《汉文启蒙》在词类划分的基础上,探讨了一些句法规则,并指出汉语的语法不能用西方传统的拉丁语法去生搬硬套。在传教士方面,马礼逊的《通用汉言之法》其系统性比之前的语法专著更强,但主要内容还是集中在词法上,句法部分只归纳了九条规则,仅占五页篇幅。马士曼的《中国言法》在汉语语法本体研究上,也是由八大词类入手着重讨论词法,对句法的讨论较为简单,甚至有时句法与词法的界限并不清晰,將许多构词法放入了句法部分。艾约瑟的《中国官话语法》在前人基础上有所进步。除词法外,他对汉语句法规则的考察也颇为详细,这是此前西方人研究中所少见的。
\end{todobox}

\subsection{Modern descriptions}\label{sec:introduction.previous.modern}

\section{Texts}\label{sec:introduction.text}

The great historical work 《史记》 (\translate{lit. historical records}),
often known as \form{Records of the Grand Historian} in English
(a translation of 太史公记, the earliest known title of the work),
laid the paradigm of official historiography of all Chinese dynasties after Han.
The author 司马迁 \form{Sima Qian} is known as the \form{Lord Grand Historian} or 太史公.
太史 \translate{grand historian} was the title of

\section{Remarkable features}

Classical Chinese has several notable typological features.

\begin{todobox}{Remarkable features}{remarkable-features}
    \begin{itemize}
        \item Part of speech
        \item Topic-comment
        \item ``Coverb'', or is there real preposition
        \item Prosody (and hence a chapter on phonology and writing system)
        \item The chapter on phonology and writing system can be very hard:
            lots of historical facts
        \item Passivization and so on
        \item Higher tolerance of ad-hoc recategorization: 名作动, 使动意动, etc.
    \end{itemize}
\end{todobox}

\chapter{Grammatical overview}\label{chap:grammatical}

\section{General principles}\label{sec:grammatical.intro}

Like all natural languages, the syntax of Classical Chinese can be divided into 
the syntax of the clause (\prettyref{sec:grammatical.clause}) and the syntax of the noun phrase (\prettyref{sec:grammatical.noun-phrase}),
both of which contain a hierarchy of grammatical systems.
Concepts like noun-hood and verb-hood can then be defined according to the syntactic environment:
a noun is what appears at the center of a \ac{np},
and a verb is what appears at the center of a clause
(e.g. the distinction between nominal predication and verbal predication 
in \prettyref{sec:grammatical.clause.nominal.distinction}).
In this sense, noun-hood and verb-hood in Classical Chinese have nothing inherently different from their counterparts in other languages.

Besides the syntactic constructions, 
a language also has a \emph{lexicon} that dictates 
the details of whether and how a root or derived stem or a larger construction appears
in certain syntactic environments and be phonologically realized.
Here, we have a slightly different definition of 
parts of speech tags like \term{noun} or \term{verb}:
they are defined \emph{lexical} labels 
representing the structure of the \emph{lexicon}, not the \emph{grammar}.
The \term{noun} class now represents a group of lexical items with shared grammatical properties,
like frequently appearing as heads of noun phrases and having certain morphological properties.
Demarcation of \emph{this} definition of parts of speech often shows considerable cross-linguistic variance as well as language-internal irregularities (for instance, the English adjective \form{worth} takes an object, a structure not otherwise seen for other adjectives) and is discussed in \prettyref{sec:grammatical.pos} for Classical Chinese.

\begin{theorybox}{Comparison between Latin and English nouns}{latin-english-noun-comparison}
    For example, to say ``the Latin word \form{canis} is a noun'' 
    means to say that the form \form{canis} usually appear as the head of an \ac{np},
    that it carries an inherent gender feature and a number feature,
    and that its inflection pattern follows one of Latin nominal declensions.
    Modern English does not have rich inflectional morphology
    but does have nominal modification constructions (e.g. \form{a [dog]_{\text{nominal (not \ac{np})}} tag}),
    so saying that \form{dog} is a noun means something different with 
    saying that \form{canis} is a noun.

    We note that \form{canis} can be further analyzed as a root plus an ending.
    The Latin lexeme \form{canis} is actually a bundle of the root 
    \form{cane-}, the masculine gender, a case feature (here nominative),
    a number feature (here singular),
    and the fact that it is the head of some complete \ac{np}.
    On the other hand, the root \form{cane-} appearing as the main verb of a clause is impossible,
    because a bundle of the root \form{cane-} plus some verbal features
    is \emph{not} in the mental dictionary of a Roman.
    Nominal attributes are not possible in Latin,
    again because the mental dictionary of Romans does not contain anything like
    the root \form{cane-} without the head status of a \ac{np}.

    The Latin form class \term{noun}, then,
    means the bundle ``a gender feature, a case feature, a number feature, and the head-of-\ac{np} status''
    plus how it is morphophonologically realized (i.e. the five declensions).
    The English concept of \term{noun} is quite different from that.
    Indeed, if we accept the hypothesis that abstract principles of language structures 
    are more or less the same cross-linguistically,
    then the lexicon \emph{has to} be highly diverse across languages
    because it is exactly the locus of language variance, besides morphophonology.
    
    Turning back to Classical Chinese, when we say Classical Chinese has a noun-verb distinction in the \emph{lexical} sense,
    we need to demonstrate that the lexicon of Classical Chinese has two largely non-overlapping groups,
    the elements of which regularly head noun phrases and clauses and have distinct properties in other morphosyntactic processes, respectively.
    In \prettyref{sec:grammatical.pos} we demonstrate that this is indeed the case.
\end{theorybox}




\section{The overall clausal structure}\label{sec:grammatical.clause}

Like all other languages, a Classical Chinese clause can be a simple clause
or a complex one constructed from subordination (\prettyref{sec:grammatical.clause.linking})
and coordination (\prettyref{sec:grammatical.clause.coordination}).
A simple Classical Chinese clause can be divided into a nucleus clause (\prettyref{sec:grammatical.clause.nominal}, \prettyref{sec:grammatical.clause.verbal})
plus discourse-related devices,
including its speech act (\prettyref{sec:grammatical.clause.force}) marked by sentence final particles (\prettyref{sec:grammatical.clause.sfp}),
and topicalization or focalization (\prettyref{sec:grammatical.clause.information}).
Topicalization can also happen for a complex clause (\prettyref{sec:grammatical.clause.coordination.topic-chain}).

It appears that only sentences -- clauses that appear as utterance units -- have the aforementioned discourse-related devices like topicalization and sentence final particles.
All embedded clauses (apart from direct quotations; \prettyref{sec:grammatical.clause.argument.verbal-complementation.direct-quotation}) in Classical Chinese do have these devices.

The nucleus clause may be either a nominal predicate clause (\prettyref{sec:grammatical.clause.nominal})
or a verbal clause (\prettyref{sec:grammatical.clause.verbal}).
Both constructions seem to have a well-defined subject position (which is not the same as the topic),
which however is often left blank (\prettyref{sec:grammatical.verbal.subject}).
Classical Chinese lacks obligatory \ac{tam} marking but does have a \ac{tam} adverb system, and certain semantic subtleties displayed by sentence final particles also suggests grammaticalized \ac{tam} information (\prettyref{sec:grammatical.clause.verbal.tam}).

\begin{todobox}{Clause types}{clause-types}
    In \citet[\citepage{131}]{meiguang2018},
    he classifies clauses into 说明句, 描写句, and 叙事句.
    The classification is comparable to that given in  http://area.hcjh.tn.edu.tw/noise/hcjh-ca/4-b.htm\#0303 .
    Mei doesn't mention on which basis he makes this distinction.
    In the latter source, it seems the distinction is made based on the type of the predicate.
    Thus a 描写句 is a stative (adjectival?) clause,
    and a 判断句 is a nominal predicate construction,
    and a 叙事句 is a verbal predicate construction that is not a 描写句.
    So what does Mei mean by 说明句?
    The term appears in \citet{li2004grammar} as well.


    We can go to places where he mentions the term.
    \citepage{445}: 矣 is for 叙事, and 也 is for 说明.
    \citepages{264-265}: 事件句 (叙事句) 和非事件句 (描写句和说明句)
    The distinction is also mentioned in http://paper.wenweipo.com/2018/02/14/ED1802140024.htm
    
    So it's related to the event structure.
    We need to know where the event structure resides in the vP-TP-CP hierarchy.
    Particularly, we need to identify \emph{where} the category of this distinction lies.
    I think probably that's related to the aspect:
    consider the distinction between a habitual clause and a prototypical ``event'' clause.
    
    The distinction has syntactic significances.
    We note that certain topicalization constructions seem to be only compatible with 说明句
    (\prettyref{box:a-zhiyu-b}).
\end{todobox}

\subsection{Nominal predication}\label{sec:grammatical.clause.nominal}

\subsubsection{Real nominal predicates}\label{sec:grammatical.clause.nominal.real}

The top-level structure of a Classical Chinese clause may contain 
a (optional) subject and a \ac{np} acting as the predicate
(\ref{ex:grammatical.clause.nominal.isa.1},
\ref{ex:grammatical.clause.nominal.havea.1}).
A nominal predicate may express an ``is-a'' relation between the subject (see \prettyref{sec:grammatical.verbal.subject.clause-pivot} for discussions on the meaning of the term) and the predicate,
which is the case of (\ref{ex:grammatical.clause.nominal.isa.1}).
Some nominal clauses however express a possessive relation between the two 
(\ref{ex:grammatical.clause.nominal.havea.1}).

\begin{exe}
    \ex\label{ex:grammatical.clause.nominal.isa.1} 
    \gll [秦]_{\text{subject}}, [虎 狼 之 国]_{\text{predicate}} \\
    Qin tiger wolf \category{gen} country \\
    \glt\translate{Qin is a country of tigers and wolves (i.e. cruel and not reliable).} 

    \ex\label{ex:grammatical.clause.nominal.havea.1} 
    \gll [蟹]_{\text{subject}} [六 跪 而 二 螯]_{\text{predicate}} \\
    crab six leg \category{conj} two claw \\
    \glt\translate{A crab has six legs and two claws.}
\end{exe}

\begin{todobox}{The possessive nominal predicate construction}{possessive-nominal}
    It seems the predicate in the possessive nominal predicate construction
    can never be a bare noun without any modification.
    The modification can be a numeral or an attributive.

    \begin{exe}
        \ex 王六军,大国三军
        \ex 秦王[为人]_{\text{\prettyref{box:a-zhiyu-b}}},蜂准,长目,挚鸟膺,豺声,少恩而虎狼心
    \end{exe}
    
    Another problem is that the 者-也 construction seems to be incompatible with the possessive nominal predicate.
\end{todobox}

Negation in Classical Chinese nominal clauses is usually expressed by 非,
placed before the nominal predicate (\ref{ex:grammatical.clause.nominal.is-not.1}).

\begin{exe}
    \ex\label{ex:grammatical.clause.nominal.is-not.1} 
    凡群臣之言事秦者,皆奸人,非忠臣也
\end{exe}

The nominal predicate is compatible with \ac{tam} adverbs

\begin{exe}
    \ex 且蔺相如素贱人
\end{exe}

\subsubsection{Topicalization of nominal predicate construction}
\label{sec:grammatical.clause.nominal.real.judgement}

(\ref{ex:grammatical.clause.nominal.isa.1}) is much less frequent than the 者…也 construction,
often known as 判断句 in Modern Chinese or the \translate{judgemental clause}.
A judgemental clause usually contains a particle 也 (\prettyref{sec:grammatical.clause.sfp})
at its end (\ref{ex:grammatical.clause.nominal.isa-topic.1}),
or a particle 者 after the subject (\ref{ex:grammatical.clause.nominal.isa-topic.2}), or both.
It seems that the judgemental clause is better analyzed as a topic-comment construction
(\prettyref{sec:grammatical.clause.topic}).

\begin{exe}
    \ex\label{ex:grammatical.clause.nominal.isa-topic.1} 
    \gll [城 北 徐-公]_{\text{topic: \ac{np}}}, [齐-国 之 美-丽 者]_{\text{comment: \prettyref{sec:grammatical.noun-phrase.determinative-relative}}} 也 \\
    city north \category{name}-\category{gong} Qi-country \category{gen} beautiful-beautiful \category{rel} \category{sfp} \\
    \glt\translate{Mr. Xu from the north of the city is a handsome guy in the country Qi.}

    \ex\label{ex:grammatical.clause.nominal.isa-topic.2} 
    \gll [兵]_{\text{topic: \ac{np}}} 者, [不 祥 之 器]_{\text{comment, predicate}} \\
    weapon \category{topic} \category{neg} fortunate \category{gen} instrument \\
    \glt\translate{Weapons are not auspicious.}
\end{exe}


\subsubsection{Distinction between a nominal clause and a verbal clause}
\label{sec:grammatical.clause.nominal.distinction}

Note that the term \term{nominal} in \term{nominal predication} or \term{nominal clause}
refers to the fact that the predicate is structurally a \ac{np},
not whether the head of the predicate usually appears like a noun or a verb in a dictionary
(\prettyref{sec:grammatical.intro}).
In some sentences although the predicate of a clause mostly appears as the head of a \ac{np} 
and therefore may be referred to as a noun in dictionaries,
the clause is clearly a verbal clause
because it expresses a dynamic event and not just a state,
the possibility of \ac{tam} markers, etc.,
as in (\ref{ex:grammatical.clause.nominal.noun-to-verb.1}).
Here 水 \translate{water} is used as a verb, meaning \translate{swim},
which is also modified by the modality auxiliary 能 \translate{can}.

\begin{exe}
    \ex\label{ex:grammatical.clause.nominal.noun-to-verb.1} 
    \gll [假 舟 楫 者]_{\text{subject}}, [非 能 水 也]_{\text{predicate: VP}}…… \\
    draw.help boat paddle \category{rel} \category{neg} can swim \category{sfp} \\
    \glt\translate{Those who draw help from boats and paddles cannot swim, \dots} 
\end{exe}

There are cases where the meaning of the predicate is comparable to that of a real nominal predicate.
We still classify them as verbal clauses,
because of their similarity with prototypical verbal clauses
with respect to negation, \ac{tam} modification, TODO

\begin{exe}
    \ex 大楚兴,陈胜王
    \ex 然而不王者,未之有也
\end{exe}

On the other hand, there is one thing a nominal predicate can do
while a verbal predicate \emph{cannot} do:
a nominal predicate can be topicalized 
(\prettyref{sec:grammatical.clause.topic}, \ref{ex:grammatical.clause.topic.predicate.1}).

\subsubsection{Copula constructions} 

All the constructions mentioned above are without a copula.
In the pre-Classical copula age there is a copula 惟,
which however had largely died out of use in Classical texts.
Meanwhile, grammaticalization had added several copulas to Classical Chinese
\citep[\citepages{20-22}]{pulleyblank1995outline}.

\subsection{Verbal predication}\label{sec:grammatical.clause.verbal}

The structure of clauses with verbal predicates is much more complicated,
and the details can only be described in the following sections.
In this section we overview grammatical systems within verbal clauses.


\subsubsection{Constituents and ordering}\label{sec:grammatical.clause.verbal.linear}

In clauses with verbal predicates,
the constituent order of core constituents of transitive clauses is almost always SVO
(\ref{ex:grammatical.clause.svo.declarative.1}, \ref{ex:grammatical.clause.svo.interrogative.1}).
Intransitive clauses have a SV constituent order
(\ref{ex:grammatical.clause.svo.intransitive.1}).
The usage of the term \term{subject} is justified in \prettyref{sec:grammatical.verbal.subject},
and the contents of a verbal clause besides the subject
is often defined as the \ac{vp}.
\Acp{vp} can be coordinated (\prettyref{sec:grammatical.clause.coordination}).
Prepositional complements are also placed after the verb
(\ref{ex:grammatical.clause.svo.pp-declarative.1}).
The term \term{object}, without specification, means any argument in the \ac{vp} that is not marked by a preposition (\prettyref{sec:grammatical.verbal.argument.prepositional}).

\begin{exe}
    \ex\label{ex:grammatical.clause.svo.declarative.1} 
    [子张]_{\text{subject (\prettyref{sec:grammatical.verbal.subject})}} [[学]_{\text{verb}} [干禄]_{\text{object}}]_{\text{predicate: VP}}

    \ex\label{ex:grammatical.clause.svo.interrogative.1} 
    [子]_{\text{subject}} [奚]_{\text{reason (\prettyref{sec:grammatical.clause.argument.high-level})}} 不 [为]_{\text{verb}} [政]_{\text{object}}

    \ex\label{ex:grammatical.clause.svo.intransitive.1}
    君子不器

    \ex\label{ex:grammatical.clause.svo.pp-declarative.1}
    君子博学于文
\end{exe}

Object pronouns however can be extracted before the verb in negative clauses (\ref{ex:grammatical.clause.sov.neg.1}), leading to a SOV order.
An interrogative object pronoun can also be fronted (\ref{ex:grammatical.clause.sov.interrogative.1}).

\begin{exe}
    \ex\label{ex:grammatical.clause.sov.neg.1}
    恐 [年岁 之 [不吾与]_{\text{VP: Neg-OV}}]_{\text{complement clause}}
    
    \ex\label{ex:grammatical.clause.sov.interrogative.1} 
    以五十步笑百步,则 [何如]_{\text{SOV interrogative clause}}
\end{exe}

\subsubsection{The structure of the verb}
It is possible that the main verb of a verbal clause contains more than one root.
Such a verb is known as a complex predicate.

\begin{todobox}{Classical Chinese complex predicate}{cp}
    Directional complement and resultative complement
\end{todobox}

\subsubsection{Positions of modifiers}
Adverbial constituents in the nucleus can be divided into \ac{tam} ones 
and so-called peripheral arguments, including location, manner, instrument, etc.
The peripheral arguments can be post-verbal
(\ref{ex:grammatical.clause.peripheral.postverbal.1},
\ref{ex:grammatical.clause.peripheral.postverbal.2},
\ref{ex:grammatical.clause.peripheral.postverbal.3})
or pre-verbal
(\ref{ex:grammatical.clause.peripheral.preverbal.1},
\ref{ex:grammatical.clause.peripheral.preverbal.2},
\ref{ex:grammatical.clause.peripheral.preverbal.3}),
with the pre-verbal order gaining popularity as time went by.
The linear order of peripheral arguments is similar to that in Mandarin
\citep[\citepages{286-287}]{he2005shiji}.

\begin{exe}
    \ex\label{ex:grammatical.clause.peripheral.postverbal.1} 侍饮于长者
    \ex\label{ex:grammatical.clause.peripheral.postverbal.2} 孟孙问孝于我
    \ex\label{ex:grammatical.clause.peripheral.postverbal.3} 祷尔于上下神祇
    \ex\label{ex:grammatical.clause.peripheral.preverbal.1} 韩生南向坐
    \ex\label{ex:grammatical.clause.peripheral.preverbal.2} 於人之罪无所忘
    \ex\label{ex:grammatical.clause.peripheral.preverbal.3} 为人谋而不忠乎
\end{exe}

The \ac{tam} adverbials are almost always preverbal.

\begin{exe}
    \ex 文王既没,文不在兹乎
    \ex 孔子既得合葬于防
    \ex 我未之能易也
\end{exe}

When \ac{tam} adverbs and peripheral arguments both appear before the verb,
the order is always \ac{tam} \before peripheral argument.
The reverse order is never attested.
The whole \ac{vp} therefore can be analyzed as a core \ac{vp}
plus peripheral arguments surrounding it,
plus \ac{tam} adverbs preceding the pre-verbal peripheral arguments.
The clause then is the complete \ac{vp} plus the subject.

\begin{exe}
    \ex 三王 [既]_{\text{\ac{tam}}} [以]_{\text{instrument}} [定法度]_{\text{VO}}
\end{exe}

\begin{todobox}{Adverbials combination}{adverbial-combine}
    Is it possible to use multiple pre-verbal peripheral adverbials?
    What's the relevant order constraint?
\end{todobox}

\begin{todobox}{Position of adverbials in SOV case}{adverbial-sov}
    Where to place adverbials in SOV case?
\end{todobox}

\begin{todobox}{Position of negator}{negative-template}
    Where is the position of the negator?
\end{todobox}



\subsubsection{Sentence final particles}\label{sec:grammatical.clause.sfp}

Classical sentence final particles have a variety of functions.
It may mark the interrogative force (\ref{ex:grammatical.clause.sfp.interrogative.1}), 
a judgemental meaning (\ref{ex:grammatical.clause.sfp.judgement.1}),
and aspectual values (\ref{ex:grammatical.clause.sfp.aspectual.1}).

\begin{exe}
    \ex\label{ex:grammatical.clause.sfp.interrogative.1} 大车无輗,小车无軏,其何以行之哉
    \ex\label{ex:grammatical.clause.sfp.judgement.1} 人而无信,不知其可也
    \ex\label{ex:grammatical.clause.sfp.aspectual.1} 温故而知新,可以为师矣
\end{exe}

It seems a sentence final particle can be shared by two conjuncts.

\begin{exe}
    \ex\label{ex:grammatical.clause.sfp.judgement.2} 虎者,戾虫;人者,甘饵也
\end{exe}

In early texts, a sentence final particle can be inserted after the main verb.

\begin{exe}
    \ex 巧言令色,鲜矣仁
\end{exe}


\subsubsection{The gerundive construction}\label{sec:grammatical.clause.verbal.gerundive}

Classical Chinese has one gerundive construction, which may appear as the object
(\ref{ex:grammatical.clause.verbal.gerundive.1})
or as a subordinated clause, like a conditional or temporal clause
(\ref{ex:grammatical.clause.verbal.gerundive.2}; \prettyref{sec:grammatical.clause.linking}).
The construction is sometimes known as nominalization.
We reject this analysis, because it seems the construction does not admit adjectival modification,
while there is nothing looking like an object (e.g. 子 in \ref{ex:grammatical.clause.verbal.gerundive.2}) in prototypical Classical Chinese noun phrases.
The construction therefore has a structure comparable to the English gerundive non-finite clause \form{his playing national anthem}
and is structurally different from the noun phrase.

\begin{exe}
    \ex\label{ex:grammatical.clause.verbal.gerundive.1} 
    \gll 王 如 知 此, 则 无 望 [民 之 多 于 邻 国]_{\text{object: gerundive}} 也 \\
    king if know this then \category{neg} hope people \category{gen} more than neighbor country \category{sfp} \\
    \glt\translate{If Your Majesty knows this, then don't expect your people to be more plentiful than your neighboring countries' people.}

    \ex\label{ex:grammatical.clause.verbal.gerundive.2}
    [父母之爱子]_{\text{condition: gerundive}},则为之计深远
\end{exe}

\subsection{Argument structures}\label{sec:grammatical.clause.verbal.argument}

\subsubsection{Core argument structures}

\paragraph{\category{do}, \category{be} and \category{become}}
Consistent with cross-linguistic generalizations, in Classical Chinese,
a verbal clause can be about an intentionally initiated event (\category{do}; \prettyref{sec:valency.simple.do}),
a state (\category{be}) or a change of the state (\category{become}; \prettyref{sec:valency.simple.state-and-change}).
The \category{do} type can further be divided into the transitive and intransitive classes.
\category{be} and \category{become} clauses are intransitive by definition.
The distinction between the three classes has consequences for
animacy and volition of the subject (see the sections referred above)
as well as the viability of certain grammatical processes (\prettyref{sec:valency.simple.do.properties}).

\category{become} and \category{be} clauses are often inputs to the synthetic causative construction,
resulting in \category{cause}-\category{become}/\category{be} clauses
(\prettyref{sec:grammatical.clause.argument.alternation.increase}),
which often develop lexicalized usages (\prettyref{sec:grammatical.clause.argument.lexical}).

\paragraph{Prepositional arguments and applicative constructions}\label{sec:grammatical.verbal.argument.prepositional}
Prepositional arguments can also be observed in Classical Chinese.
In (\ref{ex:grammatical.clause.verbal.argument.prepositional.1}),
for example, the prepositional phrase 于车 is the \term{source} of the event.
A prepositional argument can also appear in a transitive construction,
coding a wide varieties of semantic roles,
like the recipient (\ref{ex:grammatical.clause.verbal.argument.prepositional.give.1}, \ref{ex:grammatical.clause.verbal.argument.prepositional.give.2}),
or the target of a question (\ref{ex:grammatical.clause.verbal.argument.prepositional.ask.1}).
It is not possible for a prepositional argument to appear before the object.
Classical Chinese also does not have quirky subjects:
it is not possible for a prepositional phrase to appear in the subject position.

\begin{exe}
    \ex\label{ex:grammatical.clause.verbal.argument.prepositional.1} 
    \gll 公 惧, 队 于 车 \\
    king afraid fall from carriage \\
    \glt\translate{The king was afraid and fell from the carriage.}

    \ex\label{ex:grammatical.clause.verbal.argument.prepositional.give.1}  成王、康王……故赐之以重祭

    \ex\label{ex:grammatical.clause.verbal.argument.prepositional.give.2} 秦复予我河外及封陵为和

    \ex\label{ex:grammatical.clause.verbal.argument.prepositional.ask.1} 季康子问政于孔子
\end{exe}

Classical Chinese has applicative constructions that
turn an argument structure containing a prepositional argument into a double object construction
(\ref{ex:grammatical.clause.verbal.argument.prepositional.ask.double-object.2}).
In this case, the argument corresponding to the prepositional argument in (\ref{ex:grammatical.clause.verbal.argument.prepositional.ask.1})
behaves like the monotransitive object in constituent orders and in valency decreasing
(\prettyref{sec:grammatical.verbal.subject.argument-structure.alternation}, \prettyref{sec:grammatical.clause.verbal.argument-structure.pseudo-passive.multiple-argument}),
which means it is somehow more ``external'' or ``subject-like'' (\prettyref{sec:grammatical.verbal.subject.argument-structure.alternation}).

\begin{exe}
    \ex\label{ex:grammatical.clause.verbal.argument.prepositional.ask.double-object.2} 上问上林尉诸禽兽簿
\end{exe}

It should be noted that there exists another type of double object construction
derived from verbs with a prepositional argument:
the preposition of the prepositional argument may be omitted after the verb:
compare the prepositional (\ref{ex:grammatical.clause.verbal.argument.prepositional.give.3})
and the double-object example (\ref{ex:grammatical.clause.verbal.argument.prepositional.give.double-object.3}).
The omission seems to be in line with the omission of the preposition in post-verbal locative arguments (cf. \ref{ex:grammatical.clause.verbal.argument.peripheral.omission.1}).

\begin{exe}
    \ex\label{ex:grammatical.clause.verbal.argument.prepositional.give.3} 有献不死之药于荆王者
    \ex\label{ex:grammatical.clause.verbal.argument.prepositional.give.double-object.3}  请献盆缶秦王 
\end{exe}

In some languages, the argument structure of verbs meaning giving and receiving
seems to contain a small clause.
In English, for instance, we have \form{give this to him and that to her},
and in Latin we even have standalone small clauses like \form{Deo gratias}.
No trace of such possessive or directional small clauses is found in Classical Chinese:
double object clauses with giving or receiving meanings seem to be analyzable as applicative clauses
\citep[\citepages{416-421}]{meiguang2018}.

\subsubsection{Verbal complementation}

Verbal complementation in Classical Chinese includes various complement clause constructions,
and clauses with sub-clausal complements.


\begin{exe}
    \ex 子使漆雕開仕
    \ex 雍也可使南面
\end{exe}

\paragraph{Direct quotations}\label{sec:grammatical.clause.argument.verbal-complementation.direct-quotation}

Direct quotations in Classical Chinese may appear within the \ac{vp} just like an ordinary object (\ref{ex:grammatical.clause.argument.verbal-complementation.direct-quotation.say.1}).
Note that the quoted content is a \emph{sentence}, which includes a sentence final particle,
which is usually not allowed in other embedded clauses (TODO: ref).
We however note that the verb in the V-quotation construction illustrated in (\ref{ex:grammatical.clause.argument.verbal-complementation.direct-quotation.say.1}) is limited to 曰.
For other ``speaking'' verbs, the direct quotation is introduced in the way of (\ref{ex:grammatical.clause.argument.verbal-complementation.direct-quotation.v-say.1}).

(\ref{ex:grammatical.clause.argument.verbal-complementation.direct-quotation.v-say.1}) can be analyzed as an object sharing construction,
where 问于子贡 and 曰 are coordinated at the argument structure level (\prettyref{sec:grammatical.clause.coordination.vp}),
and the direction quotation is an argument of both 问 (cf. \ref{ex:grammatical.clause.verbal.argument.prepositional.ask.1}, where the direct quotation in \ref{ex:grammatical.clause.argument.verbal-complementation.direct-quotation.v-say.1} is replaced by the \ac{np} 政 \translate{politics}).
We however note that 曰 has possibly grammaticalized in Classical texts.
In (\ref{ex:grammatical.clause.argument.verbal-complementation.direct-quotation.v-say.2}),
the constituent introduced by 曰 is a proper name,
and clauses with 曰 as the main verb where the subject is a person and the object is a proper name are rare if not impossible.
If 曰 in (\ref{ex:grammatical.clause.argument.verbal-complementation.direct-quotation.v-say.2})
is understood as a marker of a direct quotation, however, the sentence makes sense:
the second argument has a \emph{metalinguistic} usage,
whose semantic interpretation is the quoted 灵台 itself,
without referring to anything in the real world.

\begin{exe}
    \ex\label{ex:grammatical.clause.argument.verbal-complementation.direct-quotation.say.1} 	
    子曰:“不患人之不己知,患不知人也。”

    \ex\label{ex:grammatical.clause.argument.verbal-complementation.direct-quotation.v-say.1} 
    \gll [子禽]_{\text{subject: NP}} [问]_{\text{verb}} [于 子贡]_{\text{target: PP}} [曰:“…”]_{\text{direct quotation}} \\
    \category{name} ask at \category{name} say \\
    \glt\translate{Ziqin asked Zigong: \dots}

    \ex\label{ex:grammatical.clause.argument.verbal-complementation.direct-quotation.v-say.2} 
    [谓]_{\text{verb}} [其 台]_{\text{object: NP}} [曰 [灵 台]_{\text{proper name}}]_{\text{??}}  
\end{exe}

Therefore, in Classical Chinese, 曰 is both used as a lexical verb (\ref{ex:grammatical.clause.argument.verbal-complementation.direct-quotation.say.1})
and a grammaticalized marker of direct quotations.
What is quoted can be a sentence (\ref{ex:grammatical.clause.argument.verbal-complementation.direct-quotation.v-say.1})
or a noun phrase (\ref{ex:grammatical.clause.argument.verbal-complementation.direct-quotation.v-say.2}).
Under this analysis, examples like (\ref{ex:grammatical.clause.argument.verbal-complementation.direct-quotation.nominal-say.1}) are probably clauses with nominal predicates,
with the predicate being a direct quotation.

\begin{exe}
    \ex\label{ex:grammatical.clause.argument.verbal-complementation.direct-quotation.nominal-say.1} 其名曰觙
\end{exe}

\paragraph{``Prototypical'' complement clause constructions}

\begin{exe}
    \ex 臣窃以为 [不便於君]
\end{exe}

\begin{exe}
    \ex {} [知和而和,不以礼节之],亦不可行也
\end{exe}

\paragraph{Pivot constructions or argument sharing}

Certain verbs have two internal arguments,
the first is the object of the main clause,
and the second is a complement clause, whose subject is the aforementioned first object,
i.e. the object of the matrix clause.
This construction is sometimes known as the \term{pivot construction} \citep[\citepage{40}]{pulleyblank1995outline} or 兼语式 in Chinese \citep[\citepage{375}]{meiguang2018}.

A clear instance of the pivot construction is the analytic causative construction (\prettyref{sec:valency.causative.analytic}).
(\ref{ex:grammatical.clause.argument.verbal-complementation.raising.1}) is an example.
Uncontroversial pivot constructions are limited in number in early texts,
because many of them can also be analyzed as argument structure-level coordination
(\citealt[\citepage{376}]{meiguang2018}; \prettyref{sec:grammatical.clause.coordination.vp}).

\begin{exe}
    \ex\label{ex:grammatical.clause.argument.verbal-complementation.raising.1} 令军勿敢犯
\end{exe}

\begin{todobox}{Other verb frames}{verb-frame}
    Control construction, etc.
\end{todobox}

\subsubsection{Possession in argument structure}

\begin{exe}
    \ex 陈胜者,阳城人也,字涉
\end{exe}

\subsubsection{Valency alternation}

\paragraph{Valency increasing}\label{sec:grammatical.clause.argument.alternation.increase}

Various valency increasing constructions exist in Classical Chinese,
which all append a subject, i.e. an external argument (\prettyref{sec:grammatical.verbal.subject.argument-structure}) to an existing argument structure.

The most productive valency increasing construction is probably the synthetic causative,
whose outputs are ``transitive'' \category{cause} clauses
(\prettyref{sec:grammatical.clause.verbal.argument-structure.causative})
that are similar to but subtly different from transitive \category{do} ones.

Classical Chinese seems to already have a prototype of what is later known as the disposal construction or the 把 construction in later Sinitic languages.

\begin{exe}
    \ex 尽以其宝器赂献于周厘王
\end{exe}

\paragraph{Valency decreasing}

Classical Chinese also has valency decreasing constructions, which \emph{suppress} the subject 
and promote an internal argument (\prettyref{sec:grammatical.verbal.subject.argument-structure.alternation}) to the subject position (\prettyref{sec:grammatical.clause.verbal.argument-structure.passive}).
Valency increasing after valency decreasing is also possible:
(\ref{ex:grammatical.clause.verbal.argument.valency-alternation.1})
is an example of a causative clause based on the pseudo-passive construction.

\begin{exe}
    \ex\label{ex:grammatical.clause.verbal.argument.valency-alternation.1} …… 杀御叔 (=\ref{ex:grammatical.clause.verbal.argument.causative.synthetic.1} in \prettyref{sec:grammatical.clause.verbal.argument-structure.causative.synthetic})
\end{exe}

\paragraph{Applicatives}

In \prettyref{sec:grammatical.verbal.argument.prepositional},
we see that Classical Chinese also has productive applicative constructions,
and the resulting argument structure is subject to valency decreasing.



\subsubsection{Argument structure and verb classes}\label{sec:grammatical.clause.argument.lexical}

The only fundamental constraint to whether a stem appears in a transitive \category{do} or intransitive \category{do} or \category{become} construction is its semantics.
In world languages, however, whether a root or a stem is compatible with a certain verb frame
is dictated by the lexicon of the language,
and a group of root-environment complexes with shared properties 
is known as a part of speech (\prettyref{sec:grammatical.intro}).
Thus we can say if a \emph{verb} (and not the clause it heads) is transitive or intransitive,
or whether it is an action verb (i.e. a \category{do} verb),
a stative verb (i.e. a \category{be} verb),
a internally caused change of state verb (i.e. a \category{become} verb)
or an externally caused change of state verb (i.e. a \category{cause} verb,
which may be \category{cause}-\category{become} or \category{cause}-\category{be}
or even without an intransitive counterpart).
Analysis of argument and event structures is closely related to verb classification.

\subsubsection{High-level categories of the \ac{vp}}\label{sec:grammatical.clause.argument.high-level}

Certain constituents within the \ac{vp} can appear before the main verb.
In (\ref{ex:grammatical.clause.verbal.argument.focus.1}),
the pre-verbal constituent 以其妹 seems to be a fronted prepositional argument.
It cannot be the sentential focus,
because the subject 季康子 seems to stay in-situ with no pause after it.
It is likely that a \ac{vp}-internal focus position exists in Classical Chinese,
which, cross-linguistically, is not rare (e.g. see \citet{danckaert2011left}).
A further piece of evidence suggesting the in-\ac{vp} analysis of (\ref{ex:grammatical.clause.verbal.argument.focus.1}) is that the fronted prepositional phrase may further undergo preposition-object inversion (\ref{ex:grammatical.clause.verbal.argument.focus.2}),
an operation that is otherwise not observed in Classical texts and also not motivated:
the best explanation of this inversion seems to be focalization
\citep[\citepage{323}]{meiguang2018}.

\begin{exe}
    \ex\label{ex:grammatical.clause.verbal.argument.focus.1} 季康子 [以其妹]_{\text{VP-focus: prepositional phrase}_i} 妻 之 ---_{i}
    \ex\label{ex:grammatical.clause.verbal.argument.focus.2} 室於怒而市於色
\end{exe}

It is possible to omit the preposition of certain post-verbal 

\begin{exe}
    \ex\label{ex:grammatical.clause.verbal.argument.peripheral.omission.1} 
    \gll [大-王]_{\text{subject: \ac{np}}} 见 [臣]_{\text{object: \ac{np}}} [列 观]_{\text{locative: \ac{np}}} \\
    great-king see servant/minister regular palace \\
    \glt\translate{Your Majesty see (your) servant (i.e. me) at a regular palace.}
\end{exe}

\subsection{Tense, aspect, modality, and things like that}\label{sec:grammatical.clause.verbal.tam}

The \emph{meaning} of tense, aspect, and modality can be expressed by various syntactic devices,
not all of which should be considered \emph{syntactic} \ac{tam} categories comparable to, say, English \form{should have been doing}.
\emph{Syntactic} \ac{tam} categories are within a single clause
and form a system with implications for e.g. linear orders of auxiliaries and adverbs.
Thus \form{should}, \form{have}, and \form{been} in \form{should have been doing}
are considered real auxiliaries and their existence demonstrate that English has a syntactic \ac{tam} system,
as their relative orders cannot be shifted even with semantic motivations.
English adverbs like \form{now} or \form{frankly} are also likely a part of the \ac{tam} system
because they have to follow the same kind of order constraints \citep{cinque1999adverbs}.
On the other hand, English \form{be able to} is not a grammaticalized \ac{tam} marker (yet),
as \form{many of us would have been able to have made this kind of bomb} is attested,
where the \category{perfect} marking appears both before and after \form{able},
modifying both the fact many of us being able to do something
and the thing being done (i.e. making this kind of bomb),
strongly suggesting that we are looking at a biclausal construction.

An analysis of time in Classical Chinese along this line however proves hard,
because we do not have native speakers and certain grammatical details may never be known for certain:
it is not possible to know that a certain construction is not possible,
and therefore there is always room to argue that a certain word is not fully grammaticalized.
The analyses below therefore are all tentative to a certain degree.

\begin{todobox}{In search of auxiliaries?}{aux-search}
    See e.g. \href{http://www.ziyexing.com/files-5/guhanyu/guhanyu_3_1.htm}{here}.
    \begin{itemize}
        \item 克”“能”“堪”“可”“可以”“可得”“得”“足”“足以”等
        \item 欲”“肯”“將”“宁”“敢”“忍”“愿”“屑”“憗(yìn)”等,其中“敢”“忍”“屑”通常用于否定句,“肯”“(不)敢”“(不)忍”“(不)屑”沿用至今。“欲”“將”“愿”“憗”表示主观的希望或打算,可以翻译成“希望”或“打算”。例如:
        \item 当”“如”“宜”“任”“合”“应”,其中“合”“应”
        \item 
    \end{itemize}
    
    能 does not seem to be a prototypical auxiliary.
    The main argument against its status as an authentic auxiliary is the prevalence of sequences like 不能不 (as in 人主不能不有遊觀安燕之時).
    Here the two negations means \translate{not doing \dots is impossible},
    and the structure is routinely used to emphasize the core \ac{vp}.
    The structure of these examples seems to be better analyzed as a complement clause construction,
    as nowhere else in Classical Chinese can we find two same negators in a single clause.
    (非不 is possible, but not 不不)

    cf. Negation in Cartography
\end{todobox}

Whether a clause is a \category{do} clause or a \category{become} clause or a \category{be} clause
is also related to the lexical aspect of the clause,
which in turn may have non-trivial interactions with \ac{tam} categories.

\begin{todobox}{Aspects of TAM marking}{tam-aspect}
    Are TAM markers allowed in the pivot construction?
    This is related to whether Classical Chinese has infinitives \citep[\citepage{375}]{meiguang2018}.
\end{todobox}

The appearance of the sentence final particle 矣 seems to have a tense effect:
it introduces a \emph{specific} time point.
The event time can have an arbitrary relation with the time point introduced,
and interestingly, there seems to be no distinction between the speech time and the reference time, meaning that in Classical Chinese, there is no past tense \citep[\citepages{437-446}]{meiguang2018}.

Other sentence final particles, like 也, do not have this effect:
hence an alternation between narrations about events 
and descriptions on what these events mean on the battlefield
\citep[\citepages{444-445}]{meiguang2018}.


\subsection{Subjecthood}\label{sec:grammatical.verbal.subject}

In \prettyref{sec:grammatical.clause.nominal.real} and in \prettyref{sec:grammatical.clause.verbal.linear},
we both mention the concept of \term{subject},
which needs justification.
Further, the bipartite division of a verbal nucleus into a subject and a \ac{vp} in \prettyref{sec:grammatical.clause.verbal.linear} means that
the subject is in some senses \emph{external},
while other arguments within the \ac{vp} are \emph{internal}.
Cross-linguistically, it is not impossible for a language to demonstrate two types of \term{externality},
one based on argument structure properties like obligatory argument omission in control constructions
or binding of reflexive pronouns
(\prettyref{sec:grammatical.verbal.subject.argument-structure}),
another based on clausal pivot properties like subject sharing in coordination and relativization
(\prettyref{sec:grammatical.verbal.subject.clause-pivot}).
The two being different means syntactic ergativity,
which is rare among attested world languages \citep{aldridge2008generative},
and is absent in Classical Chinese.

Another problem is the distinction between subject and topic.
Since both the topic and the subject appear at the beginning of a clause,
the distinction between the two seems unclear.
We can even go as far as claiming that 
Classical Chinese has only information structure and no argument structure in its syntax
\citep[\citepage{122}]{meiguang2018}.
The matter is further complicated by the fact that Classical Chinese has no native speakers now
and detailed grammaticality tests are not available,
and that Classical Chinese is a pro-drop language 
so obligatoriness is not a viable criterion.
Still, we believe that the existing evidence is sufficient to justify postulating a subject grammatical function in Classical Chinese
besides the topic position, which is for information structure marking
(\prettyref{sec:grammatical.verbal.subject.topic-comparison}).

\subsubsection{Subjecthood in argument structure}\label{sec:grammatical.verbal.subject.argument-structure}

\paragraph{Obligatory relation between semantic roles and linear order}

We observe that 
the semantic relation between some clause-initial \ac{np}s and the verb 
is fixed by properties of the verb,
while the semantic relation between some clause-initial \ac{np}s and the verb is more flexible,
and these \acp{np} are related to some internal positions of the nucleus clause.
We therefore rightfully call the first type of clause-initial \acp{np} subjects,
and the second type of clause-initial \acp{np} topics.

(\ref{ex:grammatical.clause.subject.causative.1}), for example, contains three constituents,
and therefore can only be a verbal clause, with the last constituent being the object.
The verb 客 is a derivation from the noun 客.
Such a derivation, according to our experiences with other Classical Chinese texts,
can only be causative or tropative or benefactive.
By considering the context we will know the clause is a tropative one
and the right translation is \translate{Lord Mengchang considers me as a guest.}
Therefore, by virtue of being at the initial of the clause, 
the \ac{np} 孟尝君 \emph{has to} be understood as what initiates the event,
can be neither the patient nor peripheral roles in the event (e.g. an instrument).
The \emph{obligatory} relation between the verb and the \ac{np} 孟尝君 
clearly shows the latter is a subject, and not a topic.

\begin{exe}
    \ex\label{ex:grammatical.clause.subject.causative.1} 孟尝君客我 
\end{exe}

It should be noted that the subject can be the patient,
as in (\ref{ex:grammatical.clause.subject.passive.1}).
This however again is an \emph{obligatory} semantic relation
between the verb 定 and the \ac{np}s 国 and 天下.
By virtue of being the only argument of 定 and appearing before the verb,
国 and 天下 are obligatorily understood as the patients.
They cannot be understood as, say, the location of the event
(\translate{*Someone causes peace (i.e. 定) to something else in the state (国) and the universe (天下)}).
Clauses like (\ref{ex:grammatical.clause.subject.passive.1})
are therefore better analyzed as valency alternation constructions.

\begin{exe}
    \ex\label{ex:grammatical.clause.subject.passive.1} 国定而天下定
\end{exe}

\paragraph{Subjecthood in argument structure and valency alternation}
\label{sec:grammatical.verbal.subject.argument-structure.alternation}
The existence of a subject on the level of argument structure 
is relevant in valency alternation, as in
e.g. \prettyref{sec:grammatical.clause.verbal.argument.simple.non-conventional-state}
and \prettyref{sec:grammatical.clause.verbal.argument-structure.causative.synthetic}:
a structure with a subject is too ``big'' for certain operations.

An internal argument -- always an object, not a prepositional argument -- in a clause
may be promoted to the external subject position in another clause
(e.g. \prettyref{sec:valency.simple.state-and-change}).
It is possible for multiple objects to co-exist,
and as all double object constructions in Classical Chinese
seem to be related to one applicative construction or another
(\prettyref{sec:grammatical.verbal.argument.prepositional}),
the rule is that the object introduced by the applicative gets promoted to the subject position.
We may say that the object created by the applicative is the ``second most external'' argument.
This hierarchy of externality of arguments is not uncommon in world languages
(\prettyref{box:multiple-external-arguments}).

\begin{infobox}{Multiple external arguments?}{multiple-external-arguments}
    In Japhug, it is possible to have a \term{causer}\textto\term{instrument}\textto\term{agent}\textto\term{patient} argument structure,
    and the personal indexation marker seems to be decided by first taking the two most internal arguments and decide which is more salient on the empathy hierarchy,
    and then compare the result with the third most internal argument and decide which is more salient,
    and finally compare the result of the last step with the most external argument;
    hence 1\textto 3\textto 2\textto 3 is morphologically equivalent to 1\textto 2
    \citep[\citepage{310};\citepage{584},(116);\citepage{848},(67)]{jacques2021grammar}.
    
    In Classical Chinese, the synthetic causative construction 
    generally cannot be applied to an argument structure already with a subject (\prettyref{sec:grammatical.clause.verbal.argument-structure.causative.synthetic}),
    so structures like this are not possible.
    Yet as is seen above, a similar hierarchy can be built by the applicative.
\end{infobox}

\subsubsection{Subject as clausal pivot}\label{sec:grammatical.verbal.subject.clause-pivot}

In \prettyref{sec:grammatical.verbal.subject.argument-structure},
we see that subjecthood can be defined in the argument structure in verbal clauses.
Yet properties commonly attributed to subjecthood are not just about the argument structure.
For instance, in the nominal predicate construction (\prettyref{sec:grammatical.clause.nominal}),
we call the first \ac{np} the subject,
and there is no such thing as the argument structure there.
What we want to know is whether both verbal and nominal clauses in Classical Chinese
have a pivotal position in it which everything else ``revolves around''
which could be called the \term{subject}.

The most clear criterion that defines clausal pivotal subjecthood is probably coordination:
if when clauses are coordinated,
one constituent seems to be shared by all of them,
then this constituent is probably the clausal pivot.
It turns out that what is defined as the subject according to its behaviors in the argument structure
indeed is also the pivot in coordination
(\prettyref{sec:grammatical.clause.coordination},
\ref{ex:grammatical.clause.coordination.subject-vp.1}).
Note that two coordinated clauses can also share a topic,
but there are signs which tell us that what is shared is the topic and not the subject
(e.g. \prettyref{sec:grammatical.clause.coordination}, \ref{ex:grammatical.clause.coordination.topic.1}, where the topic is the object of the first clause and the subject of the second clause).

\begin{todobox}{Definition of VP}{nominative} 
    Can a verbal and a nominal predicate be coordinated? 
\end{todobox}

\begin{todobox}{Subject and \ac{tam}}{subject-tam}
    Subject and \ac{tam} in English are closely related. (e.g. control construction)
    What about Classical Chinese?
\end{todobox}

The observation that the argument structure subject in verbal clause
turns out to be the clausal pivot 
and that the TODO: nominal clause 
justify the usage of the term \term{subject} outlined in the beginning of this section.

\subsubsection{Comparison with topic}\label{sec:grammatical.verbal.subject.topic-comparison}

We have already argued that in every Classical Chinese clause,
there is a (possibly empty) subject position,
which is largely \emph{independent} to information structure factors
and therefore is not a topic.
On the other hand, authentic, information structure-related topics
are marked by devices not always available for subjects
(like the particle 者 or a pause; \prettyref{sec:grammatical.clause.topic}).
So indeed subject and topic are two distinct concepts in Classical Chinese.

This does not mean that there are no blurry cases. 
This probably leads to some scholars to treat any constituent that seems to be ``external'' as a subject
(e.g. \citealt[\citepage{41}]{li2004grammar}),
and hence the topic is a ``big subject''
\citep[\citepage{42}]{li2004grammar}.

\subsection{Information packaging}\label{sec:grammatical.clause.information}

\subsubsection{Topicalization}\label{sec:grammatical.clause.topic}

Topicalization in Classical Chinese is usually marked by adding the particle 者 after the topic.
In the reading tradition, a pause is often inserted after 者,
which cross-linguistically suggests topicalization
(\ref{ex:grammatical.clause.topic.1}, \ref{ex:grammatical.clause.topic.2}).
In these examples, the subject is topicalized.
We note that the structure of these two examples is comparable to that of the ``judgemental clause''
(\prettyref{sec:grammatical.clause.nominal}),
which obliges us to analyze the judgemental clause as topicalization of the nominal predicate construction.

\begin{exe}
    \ex\label{ex:grammatical.clause.topic.1} 此二人者,实弑寡君
    \ex\label{ex:grammatical.clause.topic.2} 单父人吕公……吕公者,好相人,……
\end{exe}

What is topicalized is of course not restricted to the subject.
This fact is a piece of evidence supporting the distinction between subject and topic in Classical Chinese.
In (\ref{ex:grammatical.clause.topic.predicate.1}),
the comment clearly has a nominal predicate.
What is promoted to the topic position however is not the subject of the nominal predicate construction,
but the predicate. The subject is likely the \emph{focus} and not the topic
\citep[\citepage{138}]{meiguang2018}.
Topicalization of other clausal constituents is also possible
(e.g. \prettyref{sec:grammatical.clause.coordination}, \ref{ex:grammatical.clause.coordination.topic.1}).

\begin{exe}
    \ex\label{ex:grammatical.clause.topic.predicate.1} 
    [仁之实]_{\text{topic: NP_i}},[[事亲]_{\text{subject: ?}} [是]_{\text{predicate: pronoun_i}}]_{\text{comment}} 也 
\end{exe}

We also note that the 者…也… framework is not limited to topicalization of nominal predicate clauses
(i.e. ``judgemental clauses'').
For instance, we have (\ref{ex:grammatical.clause.topic.existential.1}),
in which the sole argument in a existential clause is topicalized,
and the comment receives 也 as its sentence final particle.

\begin{exe}
    \ex\label{ex:grammatical.clause.topic.existential.1} 然而不王者,未之有也
\end{exe}

We also note that topicalization can happen multiple times
(\ref{ex:grammatical.clause.topic.chain.1}).

\begin{exe}
    \ex\label{ex:grammatical.clause.topic.chain.1} 万乘之国,弑其君者,必千乘之家
\end{exe}

\begin{todobox}{Dangling topic}{dangling-topic}
    Are there dangling topics in Classical Chinese?
    If not, it's a another piece of evidence supporting the distinction between subject and topic.
\end{todobox}

\begin{todobox}{A之于B也}{a-zhiyu-b}
    \begin{exe}
        \ex 寡人之于国也,尽心焉耳矣
    \end{exe}

    \begin{exe}
        \ex 其为人也,发愤忘食,乐以忘忧,不知老之將至云尔
    \end{exe}

    A之于B也 (or A之为B也),predicate, or A为B也, predicate.
    The structure seems to be parallel to the English
    \form{I, as a concerned citizen, want to emphasize that \dots},
    where \form{as a concerned citizen} obligatorily modifies the subject \form{I}.
    其为人也 here seems to be a \emph{frame}, somehow comparable to the ``global'' temporal or locational phrase.
    Another issue is that the sentence seems to be unable to represent a specific event:
    *昨日,孔子为人也,发愤忘食, while we have \form{yesterday, as a concerned citizen, I \dots}
    
    \begin{exe}
        \ex 水之积也不厚,则其负大舟也无力。
    \end{exe}

\end{todobox}

\subsubsection{Focalization}

Topicalization is marked by fronting and a pause,
but what is fronted and before a pause is not necessarily a topic.
In (\ref{ex:grammatical.clause.focus.vp-fronted.1}),
for instance, the verbal predicate is fronted,
which likely is not topical.
Note that the sentence final particle is fronted as well.

\begin{todobox}{Fronted SFP}{sfp-fronting}
    The phenomenon can be analyzed in multiple ways.
    We may assume that (\ref{ex:grammatical.clause.focus.vp-fronted.1}) is essentially some sort of cleft construction,
    in which the subject is first separated from the rest of the clause
    and then the rest of the clause is focalized.
    Or we can analyze 矣 as a \ac{tam} marker, and not a marker from the CP layer.
    Or maybe we can argue that markers from CP layers are morphologically verbal
    and have to be attached to either the main verb or the verb phrase.
    Which analysis works best depends on whether they are consistent with other phenomena.
    \begin{itemize}
        \item Can we prove that the SFPs are very ``high-level'' and are above the topic layer? For example, can two clauses with different subjects share a SFP?
        \item Semantically do 矣 carry \ac{tam} meanings?
    \end{itemize}
\end{todobox}

\begin{exe}
    \ex\label{ex:grammatical.clause.focus.vp-fronted.1} 
    \gll [[甚]_{\text{VP_i}} 矣]_{\text{focus}}, [[汝 之 不 惠]_{\text{subject: gerundive (\prettyref{sec:grammatical.clause.verbal.gerundive})}} ---_{\text{predicate_i}}]_{\text{nucleus clause}} \\
    extreme \category{sfp} 2 \category{gen} \category{neg} smart \\
    \glt\translate{You are so stupid! (lit. So extreme is your being unintelligent!)}
\end{exe}

\begin{todobox}{A complete overview of the left periphery}{left-periphery}
    See https://referenceworks.brill.com/display/entries/ECLO/COM-000248.xml
\end{todobox}

\subsection{Speech acts}\label{sec:grammatical.clause.force}

\subsubsection{Sentential aspect}

At the first glance, the sentence final particle 矣 marks the perfect aspect
(\ref{ex:grammatical.clause.force.sentential-aspect.yi.1}),
while 也 is for clauses describing something happening regularly, not a single concrete event (\ref{ex:grammatical.clause.force.sentential-aspect.ye.1})
\citep[\citepages{443-445}]{meiguang2018}.

\begin{exe}
    \ex\label{ex:grammatical.clause.force.sentential-aspect.yi.1} 余助苗长矣
    \ex\label{ex:grammatical.clause.force.sentential-aspect.ye.1} 將发命也
\end{exe}

矣 and 也, however, are different from prototypical aspect markers in several aspects.
First, 矣 appears predominantly in direct quotations in Classical texts,
which suggests that it has conversational functions.
也 frequently appears in narratives as a part of the judgemental construction
(\prettyref{sec:grammatical.clause.nominal.real.judgement}),
which lacks \ac{tam} marking, and 也 cannot be a prototypical aspect marker in that context.
Second, it seems that 矣 and 也 can be shared by two coordinated conjuncts with different subject (\prettyref{sec:grammatical.clause.sfp}),
which is rather unusual for an aspect marker.

Therefore, the grammatical category corresponding to 矣 and 也, whatever it is,
is not a typical \ac{tam} category,
and hence we disagree with \citet{meiguang2018}'s analysis.
Its scope is wider than \ac{tam} categories:
a \ac{tam} category is in relation with a nucleus clause,
while 矣 and 也 are in relation with a sentence,
i.e. an arbitrarily complex clause that is one utterance in a conversational context.
This is consistent with the usual analysis of sentence final particles 
in modern Standard Mandarin
\citep{paul2014particles,pan2021sentence}.

\begin{exe}
    \ex 亦各言其志也已矣
\end{exe}

\begin{infobox}{Alternative analysis}{sfp-alternative-analysis}
    \citet[\citepage{233}]{zhudexigrammar} acknowledges the wide spread of the analysis that sentence final particles are in relation with the whole sentence, not the nucleus clause,
    but insists that certain sentence final particles are a part of the predicate.
    Yet no convincing argument is provided.
    Among the three distributional classes he recognizes in Mandarin,
    two (marking the interrogative/imperative force, and attitude of the speaker)
    are uncontroversially attached to \emph{sentences} and not nucleus clauses.
    The remaining class, which is called the ``tense'' class by \citet{zhudexigrammar}
    and is structurally the innermost,
    resembles the class of 矣 and 也 discussed here,
    seems to be forbidden in most embedded clauses \citep{deng2010},
    just like the other two class do.
    Therefore all the three classes of Mandarin sentence final particles described in \citet{zhudexigrammar}
    are indeed in relation with the sentence and not the nucleus clause,
    which is consistent with the structural status of sentence final particles in Classical Chinese.
    
    We also note that it is not completely impossible for two independent nucleus clauses to share one \ac{tam} marker.
    In Japhug, for example, a series of nucleus clauses with different subjects can be coordinated with the \ac{tam} categories being marked at the end of the compound clause \citep[\citepages{1090-1091}]{jacques2021grammar}.
    However, Japhug lacks clear clause-level subject
    (\citealt[\citesec{2.5.3}]{jacques2021grammar}; although subjecthood is well-defined at the level of argument structure (\prettyref{box:multiple-external-arguments})),
    and therefore coordinated nucleus clauses with a shared \ac{tam} marker but different ``subjects'' is less strange in Japhug than it is in Classical Chinese:
    in the latter, we have a well-defined clausal pivot grammatical relation (\prettyref{sec:grammatical.verbal.subject.clause-pivot}) whose scope is over all \ac{tam} categories,
    making clauses sharing the \ac{tam} marker but not subjects rather unusual,
    but in the former this is probably not the case.

    \citet[\citepages{443-445}]{meiguang2018} relies solely on semantic criteria.
    Although we do not believe his analysis of 也 and 矣 as tense marker is completely correct,
    the two clearly have non-trivial interaction with \ac{tam} categories,
    a phenomenon also observed in Mandarin,
    where the lowest sentence final particle has access to \ac{tam} categories of the nucleus clause \citep[\citepage{258}]{paul2014new}.
\end{infobox}

\subsubsection{Interrogative, exclamative, and imperative}

The interrogative speech act, for example, is marked by 乎 and other particles
(\ref{ex:grammatical.clause.force.interrogative.1}).
The exclamative speech act is similarly marked by sentence final particles
(\ref{ex:grammatical.clause.force.exclamative.1}).
We note that in (\ref{ex:grammatical.clause.force.exclamative.1}),
the sentential aspect marker 矣 appears before the exclamative 夫,
which means that the two systems of particles can coexist.
The reverse order *夫矣 is not possible.

\begin{exe}
    \ex\label{ex:grammatical.clause.force.interrogative.1} 其能久乎?
    \ex\label{ex:grammatical.clause.force.exclamative.1} 吾死矣夫!
\end{exe}

\subsection{Subordination}\label{sec:grammatical.clause.linking}

The term \term{subordination} sometimes means all kinds of clause embedding.
In this section we primarily focus on bipartite clauses
with the structure and meaning of \translate{if \dots then \dots} or \translate{when ...},
and leave relative clauses and complement clauses to TODO: ref

An overview of subordination constructions in Classical Chinese can be found in \citet[\citechap{3}]{meiguang2018}.
In all Classical Chinese conditional constructions,
the condition usually appears before the consequence
(\prettyref{ex:grammatical.clause.linking.conditional.1},
\prettyref{ex:grammatical.clause.linking.conditional.2},
\prettyref{ex:grammatical.clause.linking.conditional.3}).
The consequence can be marked by 则
(\prettyref{ex:grammatical.clause.linking.conditional.1}).
Sometimes the marker 则 is dropped
(\prettyref{ex:grammatical.clause.linking.conditional.2})
but putting it back should never render a sentence ungrammatical
\citep[\citepage{86}]{meiguang2018}.
The marker 乃 is also available as a marker of the consequence clause
\citep[\citepage{87}]{meiguang2018}.

The condition clause can also be marked.
Classical Chinese distinguishes between realis and irrealis conditional constructions:
the former are marked by e.g. 既 (\prettyref{ex:grammatical.clause.linking.conditional.1}),
while the latter are marked by e.g. 若 (\prettyref{ex:grammatical.clause.linking.conditional.3}).
This distinction is relevant to the licensing of \ac{tam} markers
\citep[\citepage{81}]{meiguang2018}.
Other markers for the condition clause are also available \citep[\citechap{3}]{meiguang2018}.
We note that the marker 若 is able to appear \emph{after} the subject of the condition clause
\citep[\citepage{94}]{meiguang2018}.

\begin{exe}
    \ex\label{ex:grammatical.clause.linking.conditional.1} [既 来之]_{\text{condition}},[则安之]_{\text{consequence}}
    \ex\label{ex:grammatical.clause.linking.conditional.2} [杀女]_{\text{condition}},[我伐之]_{\text{consequence}}
    \ex\label{ex:grammatical.clause.linking.conditional.3} [若已食] 则退
\end{exe}

\begin{todobox}{Position of condition marker}{condition-marker-position}
    When the subjects of the two clauses are shared,
    it seems 若 obligatorily appears after the subject of the first clause.
    A possible analysis is to assume that the subordination construction is working at the level of VPs.
\end{todobox}

An interesting phenomenon is that the condition (\ref{ex:grammatical.clause.linking.gerundive.condition.1})
or temporal clause (\ref{ex:grammatical.clause.linking.gerundive.condition.1})
can be a gerundive one 
(\prettyref{sec:grammatical.clause.verbal.gerundive}).
This is not surprising cross-linguistically,
as the condition clause or the temporal clause is usually the ``subordinate'' clause,
while the consequence clause is the ``main'' clause,
and it is not uncommon for the subordinate clause in a clause subordination construction
to have a non-finite structure.
This is observed in for example Japanese and Turkish.
Note that the marker 若 can be attached to the gerundive condition clause as well
\citep[\citepage{98}]{meiguang2018}.
In some condition clauses,
the marker 而, instead of the otherwise genitive marker 之, appears between the subject and the predicate,
forming a clause type that is not gerundive and only appears as an irrealis condition clause
\citep[\citepages{100-102}]{meiguang2018}.

\begin{exe}
    \ex\label{ex:grammatical.clause.linking.gerundive.condition.1} 我之不德,民將弃我
    \ex\label{ex:grammatical.clause.linking.gerundive.temporal.1} 臣之壮也,犹不如人
\end{exe}

\subsection{Coordination}\label{sec:grammatical.clause.coordination}

Explicit marking of coordination is primarily done by the marker 而.
When used as a conjunction marker, 而 can be used to link two clauses 
or two verb phrases with a shared subject (\ref{ex:grammatical.clause.coordination.subject-vp.1}),
but not two nominal constituents.
Note that the functionalities of 而 is not restricted to conjunction \citep[\citepage{183}]{meiguang2018}.

\begin{exe}
    \ex\label{ex:grammatical.clause.coordination.subject-vp.1} 
    声伯四日不食以待之,食使者,而后食
\end{exe}

\subsubsection{Coordination of \acp{vp}}\label{sec:grammatical.clause.coordination.vp}

Recall that a \ac{vp} contains an argument structure (\prettyref{sec:grammatical.clause.verbal.argument}) and a set of \ac{tam} markers (\prettyref{sec:grammatical.clause.verbal.tam}).
Therefore, coordination of two \acp{vp} actually has two structural possibilities:
coordination of two argument structures, resulting in a \emph{single} situation
(and the clause is \emph{not} a prototypical compound clause),
or coordination of two full \acp{vp}
\citep[\citepages{192-201}]{meiguang2018}.
In languages with \ac{tam} inflections, in the first scenario,
it is likely that the two verbs have one \ac{tam} marker in total,
or obligatorily have two identical \ac{tam} markers.
The distinction may also influence relativization \citep[\citepage{207}]{meiguang2018}.

Classical Chinese does not have \ac{tam}-based verbal inflection,
but the distinction between the two can still be told 

\subsubsection{Topic chains as syntactic coordination}\label{sec:grammatical.clause.coordination.topic-chain}

An interesting question is the interaction between topicalization and coordination.
\citet[\citepage{217}]{meiguang2018} contends that ``topic chains'',
i.e. several clauses with a shared topic \citep[\citechap{4} \citesec{3.3}]{meiguang2018},
are discourse structures and not syntactic structures.
Therefore topicalization happens first, and coordination happens then:
after that no further topicalization is possible.
He further argues that clauses in a topic chain cannot be linked together by 而.
(\ref{ex:grammatical.clause.coordination.topic.1}) however seems to be a counterexample.
This example clearly contains two coordinated clauses.
In the first clause 取之于蓝, 之, appearing after the verb, 
can only be a pronoun, and the only sensible reading of the clause
is that 之 (the object) is coreferential with 青 at the initial of the sentence,
and 取之于蓝 then means \translate{(people) extract it (i.e. indigo dye) from \species{Indigofera}.}
Therefore, 青 at the initial of the sentence is the object of the first clause
and the subject of the second clause,
meaning it cannot be the shared subject.
This, together with the traditional pause after the first 青,
means the first 青 likely is a topic,
which means here topicalization happens \emph{after} coordination.

\begin{exe}
    \ex\label{ex:grammatical.clause.coordination.topic.1} 
    \gll [青]_{\text{topic: NP_i}},---_i 取 [之]_{\text{object: Pronoun_i}} 于 蓝 而 ---_i 青 于 蓝 \\
    indigo.dye pick it from \species{Indigofera} \category{conj} {} blue than \species{Indigofera} \\
    \glt\translate{Indigo dye, people extract it from \species{Indigofera}, but it's bluer than \species{Indigofera}.}
\end{exe}

\section{The noun phrase}\label{sec:grammatical.noun-phrase}

The Classical Chinese \ac{np} can be roughly divided into 
the determiner region and the ``core'' region,
the latter known in \citet{cgel} as the \term{nominal}.%
\footnote{
    In this note, when the term \term{nominal} is used as a noun,
    it refers to the determined region in \ac{np}s,
    while when it is used as an adjective,
    it refers to the status of being the head of a \ac{np}. 
}
The latter is just the head noun plus possible complements and modifications,
and the first can be left empty or be a demonstrative, or a ``possessor'',
the role of the latter being not confined to a semantic possessor
\citep[\citepage{61}]{pulleyblank1995outline}.
When the ``possessor'' is present, the particle 之 appears between the possessor and the nominal region
(\ref{ex:grammatical.np.template.gen.1}, \ref{ex:grammatical.np.template.gen.2}).
When only the demonstrative is present, no marking is present
(\ref{ex:grammatical.np.template.dem.1}).

\begin{todobox}{Determiner region}{determiner}
    Give a comprehensive list of determiners.
\end{todobox}

\begin{exe}
    \ex\label{ex:grammatical.np.template.gen.1} 王之诸臣
    \ex\label{ex:grammatical.np.template.gen.2} 马之死者
    \ex\label{ex:grammatical.np.template.dem.1} [此心] 之所以合于王者
\end{exe}

\subsection{Structural template}\label{sec:grammatical.noun-phrase.linear}

\subsection{The nominal region} 

\paragraph*{Pre-head attributives} 

\begin{todobox}{Pre-head attributive}{pre-head-attributive}
Is the following paragraph right?

An interesting feature of Classical Chinese is 
that adjectives before the head noun seem strongly discouraged. 
The meaning of, say, \translate{an ugly big old bear},
is canonically expressed by several strategies.
One is the 者 construction introduced below, 
which can be described as a relative clause construction (but with caveats)
and seems to have no complexity constraints
(\ref{ex:grammatical.np.nominal.relative.long-1}).
Semantically non-restrictive attributives can always replaced by clausal coordination.

Multiple adjectives are indeed possible.


\end{todobox}

\paragraph*{The marker 者 and the relative clause construction} 
\label{sec:grammatical.noun-phrase.determinative-relative}
The marker 者 looks like a relativizer.
It is different from relativizers in many other languages in that
further structural add-ons can be applied to the fused relative clause formed by it,
while the fused relative clause constructions in many other languages 
are unable to undergo further modification.
This seems to be the only productive way to form complex nominals 
(\ref{ex:grammatical.np.nominal.relative.long-1}).

\begin{exe}
    \ex 马之千里者
    \ex\label{ex:grammatical.np.nominal.relative.long-1} 若[至力农畜,工虞商贾,为权利以成富,大者倾郡,中者倾县,下者倾乡里者],不可胜数 
\end{exe}

\begin{todobox}{Relative clause complexity}{relative-clause-complexity}
    Can a relative clause contain a NP that in turn contains a relative clause?
\end{todobox}

\begin{todobox}{\form{zhi}-\form{zhe} construction}{zhi-zhe}
    The structure of the 之-者 construction may cause some debates.
    It can be analyzed as a possessive construction on top of a fused relative clause construction
    and translated word-to-word into English as 
    \translate{[those who go one thousand miles] of horses}.
    An interesting question then is whether we have any other appearances of the N 之 V 者 construction
    where the relation between N and [V 者] is prototypically possessive.
    It seems this is indeed possible: 城北徐公,齐国之美丽者也.
    
    Under this analysis, 楚人有吹箫于市者 is composed by applying the external possessive construction
    to 楚人之吹箫于市者

    One fact (or is it really a fact?) supporting the determinative analysis of 之-者
    is the construction seems to be unable to receive a further determiner:
    *此马之千里者.
    The sequence 此马之千里者 does appear but it is almost always a nominal predication construction.
\end{todobox}

\begin{todobox}{What can be relativized, and possible external possession}{external-possession-or-relative-clause}
    若至[力农畜,工虞商贾,为权利以成富,大者倾郡,中者倾县,下者倾乡里者],不可胜数
    
    It seems what is relativized here is the subject of the bracketed clause.
    But then what's the role of 大者倾郡?
    If we consider it to be a coordinated clause,
    then it seems an argument is moved from only one branch of a coordination construction:
    a clear violation of the coordinate structure constraint of extraction!
    
    If we consider it to be a coordinated VP,
    then Classical Chinese should have a external possession construction:
    [商人]_{\text{subject}} [大者 倾郡]_{\text{predicate}},
    in which 大者 is a part of 商人.
    
    Or maybe this is a clausal pseudo-coordination:
    \form{what did Alex go to the store and buy}.
\end{todobox}

\subsection{The determiner system}

\subsection{Prepositions}

In Old Chinese, there are only two prepositions: 于 and 於.
The exact usages of the two prepositions are not clear.
In \work{Zuo Zhuan}, 于 is reserved for prepositional complements (\prettyref{sec:grammatical.verbal.argument.prepositional}),
while 於 is for inter-predicate focalization (TODO).
Other Old Chinese works have different conventions. 

It is possible to omit the object of a preposition.

\begin{exe}
    \ex 孔子 [因 ---]_{\text{reason}} 叹
\end{exe}

\section{Parts of speech division}\label{sec:grammatical.pos}

Having had an overview of grammatical constructions in Classical Chinese,
we turn to analyze the structure of its lexicon.
That is to say, we now study the parts of speech division in Classical Chinese
in the second sense in \prettyref{sec:grammatical.intro}.

Classical Chinese has no inflectional morphology for content words 
so it is not possible to define parts of speech based on inflections.
Content words show much flexibility 
in their distributions in various syntactic environments,
sometimes without any formal indications.
These facts lead some to claim that Classical Chinese 
is a language without clear part of speech distinctions,
so although we can talk about the nominal or verbal usage of a root or a compound,
strictly speaking we cannot talk about nouns or verbs,
as there are no inherent lexical properties attached to roots
that dictate their nominal or verbal usages. 
A more careful analysis, though, seems to reveal that
at least some part of speech distinctions 
can be maintained in Classical Chinese,
although Classical Chinese is much more tolerant to ad hoc re-categorization of roots than, say, English.

\subsection{Nouns and verbs}

A noun-verb distinction is supported by carefully examining traditionally called noun-used-as-verb phenomena
(\prettyref{sec:pos.verb.noun-to-verb}).
If the lexicon of Classical Chinese contains \emph{only} non-categorized roots,
the interpretation of verbal usages of a word that usually appears in nominal environments
should vary rather freely.
What is actually attested however is not different from similar phenomena in other languages.
In some cases, it seems a root is first categorized as a noun 
and then undergoes something similar to English \form{-ize} (albeit without any explicit marking),
so only the nominal usage needs to be recorded as a lexical entry,
but the lexicon controls whether a derivation step is viable.
In other cases, 
both the nominal and verbal usages are to be recorded in the lexicon,
as they cannot be inferred regularly from each other.
In both cases, how a root is possibly categorized is stored in the lexicon,
meaning that calling the nominal use of a root a \term{noun} and the verbal use of a root a \term{verb}
is not problematic at all even in Classical Chinese.
Sporadic ad hoc re-categorization of roots does exist,
but this does not support the idea that part of speech division does not exist at all in the lexicon.

A terminological caveat is what appears as an argument is not necessarily a \ac{np}:
it can be a complement clause.
The main verb of a complement clause is not in a nominal position.
Some may call complement clauses ``nominal clauses'',
but this is misleading as the internal structure of a complement clause is not the same as that of a \ac{np}.

\subsection{The adjective class}

An adjective class can also be established in Classical Chinese,
although its behavior is strongly verbal. 

A caveat, similar to the caveat that an argument is not necessarily a \ac{np},
is that an attributive phrase is not always an adjective phrase.
In existing modern studies, statements like ``a verb used as an adjective'' is usually avoided:
wordings like ``something is used as an attributive'' are adopted instead.

\begin{todobox}{Traditional grammars}{traditional-grammar-list}
    List some Classical Chinese grammars in which 动词作形容词 etc. never appear.
\end{todobox}

\begin{todobox}{A comprehensive list of Classical Chinese parts of speech}{pos-list}
    Noun, verb, adjective: any other content words?
\end{todobox}

\chapter{Phonology and the writing system}

\begin{todobox}{On the writing system}{writing}
    \begin{itemize}
        \item 隶定和简化: one keeps the structure of a character and only alters the components, another messes up the structure
        \item 谐声
        \item Syntax within the character?
    \end{itemize}
\end{todobox}

\section{Theoretical consequences}\label{sec:writing-system.theoretical}

\small{
    Now we discuss the cognitive status of the ``grammar of characters'' sketched above.
    Questions relevant to this topic include whether grammar-like rules governing the structure of Chinese characters are synchronic or historical,
    and if they are synchronic, whether they derive from human's language capacity or from some other cognitive capacities.
    We note that the latter question is ultimately related to the big questions in theoretical linguistics and cognitive science (see also discussions at the end of \prettyref{sec:intro.theory}):
    if the grammar of Chinese characters mimics the grammar of spoken natural languages but the network in the brain processing Chinese characters is completely independent of the language network,
    then what we thought were unique to languages probably are not domain-specific to languages at all.
    
    Regarding the first question, neurolinguistic experiments suggest that both holistic and sub-lexical processing exist in human brain.
    The existence of holistic processing is supported by the fact that Chinese readers find it easier to tell completely different characters apart than to tell characters with shared components apart,
    while the existence of sub-lexical processing is supported by the fact that characters with valid semantic or phonetic components are possessed more quickly
    \citep[\citesec{2.2}]{duan2024chinese}.
    So literate Chinese speakers do have the components (subconsciously) in mind when reading Chinese characters.
    Yet the same can be said for all orthographic systems \citep[\citepages{23-25}]{myers2019grammar}.
    English orthography, for instance, has phonology-like rules like \form{-y} + suffix \textto{}  \form{-i}-suffix \citep[\citepage{26}]{myers2019grammar}.
    
    What makes Chinese characters special is that we have non-regular structures?
    
    Regarding the neurolinguistic properties of Chinese characters,
    we note that the brain region primarily responsible for recognizing Chinese characters
    is the Visual Word Form Area, which is not a part of the language network in the brain,
    and this suffices to be an argument against the assumption that the ``grammar'' of Chinese characters and the grammars of spoken languages have the same neurological origin \citep[\citepages{209-210}]{myers2019grammar}.
    Still, 
}



\chapter{Parts of speech}

In this chapter we discuss parts of speech in Classical Chinese in detail.
In principle, function words can be introduced together with their grammatical functions,
but since the correct analyses of some constructions are still controversial
and it may well be possible that the controversies reflect
real historical linguistic divergence among speakers,
function words are also included in this section for easier reference.

\begin{todobox}{Parts of speech, a chapter}{chap-pos}
    This chapter depends on a list of POS (\prettyref{box:pos-list}).
    The content:
    \begin{itemize}
        \item Noun; 
        \item Verbs; the details about noun-used-as-verb can be placed here, as a source of verbs.
        \item Look-up tables for particles
    \end{itemize}
\end{todobox}

\section{Nouns}


The verbs 出 (\translate{go out}), 入 (\translate{enter}), 亡 (\translate{die, decay})
are regularly derived to 出 (\translate{what goes out}), 入 (\translate{what comes in})
and 亡 (\translate{what dies}).
This derivation pattern however is not 

\begin{todobox}{Deverbalization derivation}{deverbal}
    Summarize deverbal derivations.  
\end{todobox}

\section{Verbs}

\subsection{``Nouns used as verbs''}\label{sec:pos.verb.noun-to-verb}

The conventional term in Mandarin Chinese 名词作动词 \translate{nouns used as verbs} covers two phenomena,
corresponding to multiple functions and zero derivation \citep[\citesec{11.3}]{dixon2010basic2},
and also the rare case of ad hoc re-categorization of a root.

\paragraph*{Multiple functions}
Some roots have both nominal and verbal uses,
and there is usually some semantic connection between the interpretations of the two uses,
but this is not regularly inferrable. 
Here we consider some examples in \citet{yang1991dict}:
\begin{itemize}
    \item 楚 may mean \translate{the Chu state} or \translate{do what Chu people do}.
    \item 床 may mean \translate{bed} or \translate{settle down your bed or sleep on a bed}.
    \item 城 may mean \translate{city, castle} or \translate{build a city}.
\end{itemize}
The interpretation of the verbal usage is usually \emph{not} decided
from the meaning of the root and that the root is used in a verbal environment;
rather, it is instructed by the lexicon.
Therefore, the verbal usage of 城市 only means \translate{build a city}
although the \translate{do city-related things} reading in principle could make sense. 

Therefore, roots like 城, 楚 and 床 have double functions: nominal and verbal,
but the two functions are likely not related to each other by regular grammatical rules.
This corresponds to the ``multiple function'' case in \citet[\citesec{11.3}]{dixon2010basic2}.
Moreover, what is stored in the lexicon is not the bare, non-categorized root 城,
but one noun lexeme 城 \translate{city} that specifies its nominal usage 
and one verb lexeme \translate{build a city} that specifies its verbal usage,
and other seemingly possible ways to categorize the root, although attested elsewhere,
are ruled out by their absence in the lexicon.

The boundary between roots with double functions and roots undergoing zero derivation (see below) 
is somehow blurry,
as the nominal and verbal uses of 城 and 床 still seem to show a common pattern
and may be understood as a rare derivation.
This blurriness leads many grammatical works on Classical Chinese 
to simply refer to the two phenomena uniformly as ``nouns used as verbs''.

\paragraph*{Zero derivation}
In other cases the meaning of the verbal use of a root usually appearing in a nominal context
is regularly derived from the nominal meaning.
This is because although tropative or causative derivations in Classical Chinese
are mainly verb-to-verb,
they can also be applied to nouns.
In this way from 臣 \translate{servant, official, minister}
we have the causative verbal usage \translate{make sb. dependent to},
and from 客 \translate{guest} we have the tropative usage \translate{consider sb. as a guest}.
These verbal usages are nothing different from noun-to-verb derivation observed in other languages,
so we regard the relevant phenomena as zero derivation as in \citet[\citesec{11.3}]{dixon2010basic2}.

In zero derivation, the meaning of the nominal usage has to be recorded in the lexicon,
the meaning of the verbal usage can be automatically decided from the derivation rule.
These derivations are however not completely regular and not for every word:
the lexicon also controls whether a derivational rule applies.

\paragraph*{Ad hoc re-categorization}
There are sporadic verbal usages of nouns that are almost never attested elsewhere,
like 军 in 沛公军霸上.
This means that ad hoc re-categorization of roots is possible in Classical Chinese,
and the meaning is to be decided from the context.
This is also possible in English
(as in \form{I might [guinea pig] it for you.}) 
but usually not accepted in formal texts.
Alleged ad hoc categorized Classical Chinese roots are indeed a possibility, after all,
although their frequency is not high enough and cannot be exaggerated to be the norm rather than the exception.

\section{Pronouns}

\begin{todobox}{Third person pronouns}{third-person-pronoun}
    之 seems to be the accusative pronoun in Old Chinese.
    其 seems to be the genitive pronoun,
    and may be a phonological fusion of 之 and a possessive marker.
    
    See Mei, Guang.
\end{todobox}

\section{Particles}

Grammatical particles are not content words
and in principle can be introduced together with the grammatical categories and relations they express.
The long and complicated history evolution of Classical Chinese
however means a particle may have multiple quite different uses
possibly due to grammaticalization,
so a surface form-to-function discussion on particles is of great descriptive value.

\begin{todobox}{Classification of particles}{particle-classification}
    Do I need to classify particles?
\end{todobox}

\paragraph*{者} The particle 者 most frequently appears as a relativizer, a complementizer,
or in the \form{zhe}-\form{ye} construction.
The three functions can be uniformly analyzed as the function of a low-level determiner \citep{aldridge2009old}. 

\paragraph*{之} This 

\chapter{Verb valency}

\begin{todobox}{More topics on argument structure}{argument-structure-topics}
    \begin{itemize}
        \item Morphology?
    \end{itemize}
\end{todobox}

\section{Simple argument structures}
\label{sec:grammatical.clause.verbal.argument.simple}

\subsection{Prototypical \category{do}}
\label{sec:valency.simple.do}

The subject of a \category{do} clause usually has to be animate,
because it voluntarily initiates the event described by the clause
(\ref{ex:valency.simple.do.1}).
The subject is an \term{agent}, as opposed to a \term{causer} (\prettyref{sec:grammatical.clause.verbal.argument-structure.causative})
or a \term{theme} (\prettyref{sec:valency.simple.state-and-change}).

\begin{exe}
    \ex\label{ex:valency.simple.do.1} 桓公杀公子纠
\end{exe}

\subsubsection{Unique properties of \category{do} verbs}\label{sec:valency.simple.do.properties}

\citet[\citepage{272}]{meiguang2018} lists some criteria
to distinguish a transitive \category{do} verb 
from a transitive \category{cause} verb
(\prettyref{sec:grammatical.clause.verbal.argument-structure.causative.synthetic}).

We note that certain \category{cause} verbs may gradually develop a lexicalized meaning
and eventually get reanalyzed as a \category{do} verb
\citep[\citepages{269-271}]{meiguang2018}.

\subsection{Prototypical \category{become} and \category{be} verbs}
\label{sec:valency.simple.state-and-change}

\subsubsection{The intransitive usage}

A \category{be} verb describes a state;
a \category{become} verb describes the change of a state.
In both types of argument structures,
the sole argument is a \term{theme}:
the situation happening to it just happens,
and usually it does not have much control over it nor any volition to trigger it
(\citealt[\citepage{345}]{li2004grammar}; \citealt[\citepage{275}]{meiguang2018}).

In Classical Chinese, just like in other languages,
\category{become}/\category{be} verbs often have established causative usages,
forming \category{cause}-\category{become}/\category{be} argument structures
with the \term{causer} argument being the subject and the \term{theme} argument being internal
(\prettyref{sec:grammatical.clause.verbal.argument-structure.causative.synthetic}).
When a causer is absent, the structure of the clause 
is comparable to what sometimes is known as the middle voice in English
(e.g. \form{the door opened}; cf. the transitive \category{cause}-\category{become} \form{I opened the door}).
(\ref{ex:grammatical.clause.verbal.stative.1}) is an instance:
in its \category{cause}-\category{be} usage (\ref{ex:grammatical.clause.verbal.argument.causative.synthetic.2}),
the argument that is described as weak is an internal argument appears after the verb,
but in (\ref{ex:grammatical.clause.verbal.stative.1}),
the argument that is described as weak is the \emph{subject}:
the internal theme argument gets promoted to the subject position.

\begin{exe}
    \ex\label{ex:grammatical.clause.verbal.stative.1} 秦强而赵弱
\end{exe}

The ``middle voice'' construction exemplified in (\ref{ex:grammatical.clause.verbal.stative.1})
(known as 内动 in \citet{meiguang2018}) has a subject,
which corresponds to the argument that is the object in the \category{cause}-\category{become}/\category{be} construction
(i.e. the internal argument).
It is however possible (although rare) for the subject position to be unfilled,
and the internal argument remains in-situ.
For instance, the verb 鸣 \translate{chirp} appears in ``middle voice'' clauses (\ref{ex:grammatical.clause.verbal.stative.2}),
but its sole argument can also stay \emph{after} the verb (\ref{ex:grammatical.clause.verbal.stative.3}).
The structure of (\ref{ex:grammatical.clause.verbal.stative.3}) can only be reasonably conceived
if we assume that 鸣 is a \category{be} verb,
denoting a state where bugs continue to make noise,
and that the sole argument 蜩 remains in-situ and is not promoted to the subject position.
No other analysis is available: for instance a \category{do} verb can never have such a behavior
\citep[\citepage{351}]{meiguang2018}.

\begin{exe}
    \ex\label{ex:grammatical.clause.verbal.stative.2}
    \gll 蝼蝈 鸣 \\
    \species{?} chirp \\
    \glt\translate{??? chirp.} (礼记·月令)

    \ex\label{ex:grammatical.clause.verbal.stative.3}
    \gll [五 月]_{\text{temporal}} [鸣]_{\text{predicate}} [蜩]_{\text{internal argument}} \\
    five month chirp cicada \\
    \glt\translate{In the fifth (lunar) month, cicadas chirp.}
\end{exe}

\subsubsection{The alternation between \category{be} and \category{become}}

Alternation between \category{be} and \category{become} verb frames is natural.
Some \category{become} verbs however do not have \category{be} counterparts.

\subsection{Non-conventional \category{be}/\category{become} clauses}
\label{sec:grammatical.clause.verbal.argument.simple.non-conventional-state}

Some verbs license subjects that look like arguments of prototypical \category{become} or \category{be} verbs:
the subject may be animate but it does not volitionally trigger the event.
The situation ``just happens to be the case'', and the subject can be described as a \term{theme} and not an \term{agent}.
What sets them apart from prototypical \category{become} or \category{be} verbs 
in \prettyref{sec:valency.simple.state-and-change}
is the fact that the subject seems quite unlike an internal argument.
In (\ref{ex:grammatical.clause.verbal.argument.non-conventional-theme.1}),
the subject 火 \translate{fire} is definitely a theme and not an agent:
the fire does not get to \emph{decide} if it burns the flag
\citep[\citepage{276}]{meiguang2018}.
Still the clause is not a prototypical \category{become} one
as there is an internal argument 其旗 in it,
and the theme 火 is an \emph{external} theme
\citep[\citepage{353}]{meiguang2018}.
These verbs therefore have difficulties participating in synthetic causativization
(\prettyref{sec:grammatical.clause.verbal.argument-structure.causative.synthetic}).

\begin{exe}
    \ex\label{ex:grammatical.clause.verbal.argument.non-conventional-theme.1} 火焚其旗
\end{exe}

\subsection{Experience verbs}
\label{sec:grammatical.clause.verbal.argument.simple.experience}
Some experience verbs, mostly verbs about emotions, behave like \category{become} verbs
\citep[\citepage{273}]{meiguang2018}:
when used as transitive verbs,
the subject do not look quite agentative and the clause is likely causative
(\ref{ex:grammatical.clause.verbal.argument.simple.experience.2}),
and when used as intransitive verbs,
there is a clear internal change-of-state meaning
(\ref{ex:grammatical.clause.verbal.argument.simple.experience.1}).

\begin{exe}
    \ex\label{ex:grammatical.clause.verbal.argument.simple.experience.1} 孔子成春秋,而乱臣贼子惧 (孟子·滕文公章句下)
    \ex\label{ex:grammatical.clause.verbal.argument.simple.experience.2} 惧之以怒 (左传·昭公十三年)
\end{exe}

On the other hand, perception verbs (e.g. 见 \translate{look}) and cognition verbs (e.g. 知 \translate{know})
are often transitive,
and therefore are not compatible with the synthetic causative construction 
(\prettyref{sec:grammatical.clause.verbal.argument-structure.causative.synthetic};
\citealt[\citepage{274}]{meiguang2018}).
Intuitively, these verbs are \category{do}-like according to the criteria listed in \prettyref{sec:valency.simple.do}.
For instance, they can appear in 所 construction
(\ref{ex:grammatical.clause.verbal.argument.simple.experience.do.1}).

\begin{exe}
    \ex\label{ex:grammatical.clause.verbal.argument.simple.experience.do.1} 異乎吾所聞
\end{exe}

Certain perception verbs however have developed a figurative, fossilized meaning,
and when intransitivized, can participate in synthetic causativization
(\prettyref{sec:grammatical.clause.verbal.argument-structure.causative.synthetic},
\ref{ex:grammatical.clause.verbal.argument.causative.synthetic.3}).
This possibility indicates that these fossilized usages are \category{become}- or \category{be}-like:
见 \translate{meet formally} therefore means \translate{in the state of regularly meeting an important figure}.

\section{Various causative constructions}\label{sec:grammatical.clause.verbal.argument-structure.causative}

A \term{causer} makes a situation to be the case,
but does not always do so intentionally.
It can therefore be inanimate,
as opposed to how an \term{agent} behaves
(\prettyref{sec:valency.simple.do}).

We can divide causative constructions in Classical Chinese into
synthetic and analytic ones.
In the synthetic causative construction,
there is only one verb in the surface form:
the causative valency alternation is supposedly marked by a prefix \form{*s-},
which is invisible in the written texts but is reflected by tonal changes of the verb.
If a root develops a lexicalized usage in the synthetic causative construction,
then \category{cause} verb is formed.

\subsection{Synthetic causative}\label{sec:grammatical.clause.verbal.argument-structure.causative.synthetic}

The synthetic causative construction applies to existing argument structures, or sometimes bare roots.
The synthetic causative construction cannot be applied to a \category{do} construction:
the reason is probably because a \category{do} construction is too ``big'',
already having a full-fledged wannabe subject \citep[\citepage{363-364}]{meiguang2018}.
On the other hand, the syntactic causative construction can be applied to 
``passive'' (\ref{ex:grammatical.clause.verbal.argument.causative.synthetic.1})
and \category{become} or \category{be} (\ref{ex:grammatical.clause.verbal.argument.causative.synthetic.2}) argument structures.
Certain intransitivized experience verbs,
possibly having an argument structure comparable to a \category{become}/\category{do} verb (\prettyref{sec:grammatical.clause.verbal.argument.simple.experience}) with a wannabe subject also have causative usages
(\ref{ex:grammatical.clause.verbal.argument.causative.synthetic.3}),
but their transitive counterparts are never compatible with the synthetic causative construction
\citep[\citepage{274}]{meiguang2018}.
Finally, the synthetic causative construction can be directly applied to a root (\ref{ex:grammatical.clause.verbal.argument.causative.synthetic.4}):
the word 妻 \translate{wife} is sometimes used as a verb,
meaning \translate{to marry daughter to \dots},
inconsistent with the meaning of (\ref{ex:grammatical.clause.verbal.argument.causative.synthetic.4}).
Therefore, in (\ref{ex:grammatical.clause.verbal.argument.causative.synthetic.4}),
妻 is ad hoc categorized into a \category{cause} verb,
its usual verbal usage being irrelevant here.

\begin{exe}
    \ex\label{ex:grammatical.clause.verbal.argument.causative.synthetic.1}
    \gll 是 夭 子蛮, 杀 御叔…… \\
    this die.young \category{name} kill \category{name} \\
    \glt\translate{This woman made Ziman die at a young age, and got Yushu killed\dots}
    
    \ex\label{ex:grammatical.clause.verbal.argument.causative.synthetic.2}
    \gll 以 弱 天下 之 民 \\
    \category{purpose} weak world \category{gen} people \\
    \glt\translate{\dots to weaken the people.}

    \ex\label{ex:grammatical.clause.verbal.argument.causative.synthetic.3}
    \gll 子尾 见 疆 \\
    \category{name} formally.visit \category{name} \\
    \glt\translate{Ziwei let Jiang formally visit (with Xuanzi).} (左传·昭公二年)

    \ex\label{ex:grammatical.clause.verbal.argument.causative.synthetic.4} 妻帝之二女
\end{exe}

The labile S/O alternation between the \category{be}/\category{become} usage
and the \category{cause}-\category{be}/\category{become} usage
is quite regular in Classical Chinese;
verbs allowing this alternation are sometimes known as \term{ergative verbs}
\citep[\citepage{378}]{meiguang2018},
although the phenomenon is about the core argument structure and has nothing to do with ergativity in alignment.
It should be noted that not all \category{become}/\category{be} verbs are compatible
with the synthetic causative construction.
For instance, 鸣 \translate{chirp} in (\ref{ex:grammatical.clause.verbal.stative.3})
does not have a transitive \category{cause}-\category{be} usage.
More examples are given in \citet[\citepage{276}]{meiguang2018}.
On the other hand, some clauses that look like \category{cause}-\category{be}/\category{become} clauses
actually do not have \category{be} or \category{become} counterparts 
(\prettyref{sec:grammatical.clause.verbal.argument-structure.causative.fossilization}).

\subsection{Fossilization of synthetic causative construction}
\label{sec:grammatical.clause.verbal.argument-structure.causative.fossilization}

Some \category{cause} verbs are fossilized,
and do not have clear intransitive counterparts.
For instance, 伤 \translate{hurt} typically is a state transition verb meaning body, etc. being hurt,
and it also has a causative (i.e. \category{cause}-\category{become}) meaning
(\translate{make \dots hurt}).
The \category{cause}-\category{become} verb frame of 伤 however has gained a separate lexicalized specific that can't be transparently inferred from the meaning of the \category{become} usage:
it can mean \translate{let \dots be demaged},
in which the object is not necessarily body or a person.
This usage of 伤 has no \category{become} or other intransitive counterpart.
The absence of a \category{become} counterpart can be proven by 
the ability for this fossilized figurative usage of 伤 to undergo ``passivization'' 
(\ref{ex:grammatical.clause.argument.causative.fossilize.1}),
which is otherwise not possible (\prettyref{sec:grammatical.clause.verbal.argument-structure.pseudo-passive}).

\begin{exe}
    \ex\label{ex:grammatical.clause.argument.causative.fossilize.1} 女红伤则寒之原也
\end{exe}

Verbs like 伤 in like (\ref{ex:grammatical.clause.argument.causative.fossilize.1}) can easily be reanalyzed as \category{do} verbs.
This is likely a diachronic path of the creation of \category{do} verbs.
The verb 败 for example seems to be originally a \category{become} verb (\translate{to get corrupted})
and have later gained a specific meaning of \translate{to defeat} in its \category{cause}-\category{become} usage,
which had eventually evolved into a \category{do} usage
\citep[\citepage{285}]{meiguang2018}.

\subsection{Analytic causative}\label{sec:valency.causative.analytic}

Classical Chinese has an analytic construction to express the causative meaning.
In (\ref{ex:valency.causative.analytic.1}),
the word 使 is applied to the stative structure 渚者居中原 \translate{people living on small lanbds reside inland},
meaning \translate{let people living on small lands in water live inland}.

\begin{exe}
    \ex\label{ex:valency.causative.analytic.1}
    \gll 不 使 [渚 者]_{\text{shared object: \ac{np}}} [居]_{\text{\category{be}}} [中-原]_{\text{locative object: \ac{np}}} \\
    \category{neg} let small.land.in.water \category{nmlz} reside middle-land \\
    \glt\translate{\dots do not let people living on small lands in water live inland}
\end{exe}

The word 使 can be replaced by 令 (\ref{ex:valency.causative.analytic.2}) or 俾 (\ref{ex:valency.causative.analytic.3}) \citep[\citepage{376}]{meiguang2018}.
Note that in (\ref{ex:valency.causative.analytic.2}), 令 seems to applied to a \category{do} argument structure.

\begin{exe}
    \ex\label{ex:valency.causative.analytic.2} 令军勿敢犯
    \ex\label{ex:valency.causative.analytic.3} 俾民不迷
\end{exe}

A question is whether 使 is a lexical verb, or just a non-incorporated causative marker.
The fact that (\ref{ex:valency.causative.analytic.2}) involves a \category{do} verb
deviates from the behavior of the synthetic causative construction,
which is not compatible with \category{do} verbs (\prettyref{sec:grammatical.clause.verbal.argument-structure.causative.synthetic}).
Further, the fact that we can choose among 使, 令 and 俾 is rather unusual for a grammatical marker:
we also note that 使 retains the meaning of \translate{send sb. to do sth.}
and 令 retains the meaning of \translate{command sb. to do sth.}
Therefore, we consider 使, 令 and 俾 to be \emph{lexical verbs},
and not grammaticalized causative markers.
What is grammaticalized is the causative clause with the form of V_1 NP VP,
with V_1 being one of these verbs.


\section{``Passive'' constructions}\label{sec:grammatical.clause.verbal.argument-structure.passive}

What is often known as the passive in Classical Chinese is not really a passive construction
comparable to the English or Latin passive.
The main problem is the lack of a grammaticalized way to say the agent:
in a ``true'' passive construction, the original subject is somehow demoted
(represented by the appearance of \form{by} in English or the ablative case in Latin)
and sometimes omitted, and an internal argument is promoted to the subject position,
while the so-called ``passive'' constructions in Classical Chinese
are \emph{obligatorily removed} \citep[\citepage{287-289}]{meiguang2018}.

\subsection{The agent-less ``passive''}\label{sec:grammatical.clause.verbal.argument-structure.pseudo-passive}

It is rare to apply the ``passive'' construction to a causative construction
\citep[\citepage{283}]{meiguang2018}.
Suppose we have a bivalence causative construction,
and we want to suppress the external argument and let the internal argument to be the subject.
But such a bivalence causative construction usually has a \category{cause}-\category{become} structure,
and removing the causer leaves us a clause that looks just like a \category{become} clause.
So there are two competing analyses,
and since usually if a verb root is lexically licensed to head a \category{cause}-\category{become} clause,
then its usage in a \category{become} clause is also in the lexicon,
the simpler \category{become} analysis is preferred.
However, where this preference is eliminated, ``passivization'' of a causative construction is possible 
(\citealt[\citepages{284,370-372}]{meiguang2018}; \prettyref{sec:grammatical.clause.verbal.argument-structure.causative.fossilization}).

\subsubsection{The case with multiple internal arguments}\label{sec:grammatical.clause.verbal.argument-structure.pseudo-passive.multiple-argument}

When the pseudo-passive construction is applied to double object clauses,
it is the \emph{recipient} that is promoted to the subject position 
(\ref{ex:valency.decrease.pseudo-passive.multi-valent.1}) \citep[\citepage{421}]{meiguang2018}.
On the other hand, when the pseudo-passive construction is applied to the corresponding prepositional argument construction,
it is the \emph{theme} that is promoted to the subject position
(\ref{ex:valency.decrease.pseudo-passive.multi-valent.2}).
These phenomena establish a hierarchy of \term{externality} of arguments (\prettyref{sec:grammatical.verbal.subject.argument-structure.alternation}).

\begin{exe}
    \ex\label{ex:valency.decrease.pseudo-passive.multi-valent.1} 诸侯,赐弓矢然后征
    \ex\label{ex:valency.decrease.pseudo-passive.multi-valent.2} 药言先献于贵,然后闻于卑
\end{exe}

\section{Experiential valency increasing}

\subsection{The affective constructions}

Classical Chinese has two affective constructions,
in which the subject is an experiencer suffering something bad from the situation described by the latter
\citep[\citepages{354-358}]{meiguang2018}.

The first affective construction simply attaches an experiencer to an argument structure
(\ref{ex:grammatical.clause.verbal.argument.affective.1}).
In (\ref{ex:grammatical.clause.verbal.argument.affective.1.become}),
亡 appears as a \category{become} verb (\prettyref{sec:valency.simple.state-and-change}):
it is intransitive and its subject, the \work{Odes}, did not have control over its being ignored.
In (\ref{ex:grammatical.clause.verbal.argument.affective.1.affective-become}),
a new argument -- the experiencer subject -- is introduced to the argument structure of 亡:
the meaning of the sentence is \translate{the shepherds suffered from the sheeps getting lost.}

\begin{exe}
    \ex\label{ex:grammatical.clause.verbal.argument.affective.1} 
    \begin{xlist}
        \ex\label{ex:grammatical.clause.verbal.argument.affective.1.become} 亡 as a \category{become} verb
        \gll [\focus{诗}]_{\text{subject,theme: NP}} [\focus{亡}]_{\text{\category{become}}} 然后 春秋 作 \\
        \focus{poem} \focus{get.lost} then Spring-Autumn compose \\
        
        \glt\translate{The \work{Odes} got lost, and then the \work{Spring and Autumn Annals} was composed.}
        \ex\label{ex:grammatical.clause.verbal.argument.affective.1.affective-become}
        亡 as a \category{affective}-\category{become} verb
        \gll [\focus{二} \focus{人}]_{\text{subject: NP}} [相 与 牧 羊, 而 俱 [\focus{亡}]_{\text{\category{affective}-\category{become}}} \focus{其} \focus{羊}]_{\text{coordinated VP}} \\
        \focus{two} \focus{person} mutually go.together herd sheep \category{conj} all \focus{get.lost} \focus{\category{poss}} \focus{sheep} \\
        \glt\translate{Two people herded their sheep together, and they both lost their sheep (lit. suffer from their sheep's missing).}
    \end{xlist}
\end{exe}

The second affective construction \emph{obligatorily} has an object,
which is in possession of the subject.
The meaning of (\ref{ex:valency.affective.type-2.1}), for example,
is that Confucius was frustrated by the fact that his tree was cut in Song.

The fact that the object (树 and 胁 here) should not contain any possessive markers
and has to be interpreted as something being possessed by the subject
suggests that the second affective construction is an external possession construction.

\begin{exe}
    \ex\label{ex:valency.affective.type-2.1} \focus{吾} 再 逐 於鲁, 伐 树 於 宋
    \ex\label{ex:valency.affective.type-2.2} 范睢折胁於魏
\end{exe}

\begin{todobox}{External possession}{external-possession}
    External possession as subject
\end{todobox}

\subsection{Tropative}

Tropative is a construction which attaches an experiencer to a \category{be} argument structure,
with the meaning being \translate{$A$ consider $B$ to be \dots}
\citep[\citepages{413-414}]{meiguang2018}
The Classical Chinese tropative is actually not limited to stative verbs:
it also applies to nouns.
It is however not likely that this construction comes from transformation of the nominal predicate construction
(\prettyref{sec:grammatical.clause.nominal}).
The main difference is that in the nominal predicate construction,
the predicate is a noun \emph{phrase},
but the tropative construction never takes a nominal predicate as input.

\section{Applicative constructions}

\begin{todobox}{Applicative constructions}{applicative}
    benefactive; the claim that double object constructions are similar to benefactive constructions;
    what object gets passivized.
    See \citet[\citepage{421}]{meiguang2018}.
\end{todobox}

\chapter{Tense, aspect, modality}

\section{Semi-auxiliaries}\label{sec:tam.semi-auxiliaries}



\chapter{Negation}

\chapter{Genres and formulae}

\section{}

\section{Classical-like formal documents}

\subsection{Background}

\begin{todobox}{Background}{formal-documents-background}
    \begin{itemize}
        \item 各代公文
        \item 晚清民国
    \end{itemize}
\end{todobox}

\subsection{Pseudo-Classical documents}\label{sec:genres.pseudo}

The text shown in (\ref{ex:pseudo.text.1}) is an article from the Chinese version of Geneva Conventions, an example of the pseudo-Classical Chinese used in official documents.
The text is Classical-like when we inspect the grammatical markers appearing in it:
之 in place of 的 as a relative clause marker marker and also a possessive marker,
是 in place of 这 as a demonstrative,
and 彼等 in place of 他们 as a third person pronoun.
Yet several features can be immediately noticed that deviate from the Classical standard.

\begin{exe}
    \ex\label{ex:pseudo.text.1} 战俘不得放弃本公约或上条所述之特别协定――如其订有是项协定――所赋予彼等权利之一部或全部
\end{exe}

First, polysyllabic content words are prevalent.
The verb 放弃 \translate{to give up} is not unheard of in Classical texts (cf. \ref{ex:pseudo.compound-verb.1}),
but it is usually analyzable as a verb-level coordination of 放 \translate{to put down} and 弃 \translate{to abandon}.
放弃 as a whole with the meaning of giving up something abstract is a feature of Modern Standard Mandarin,
where 弃 is no longer acceptable when used alone as a verb,
which is indeed the case in the Chinese version of Geneva Conventions.

Nouns like 战俘 or 协定 are never attested in common Classical texts.
Their structures can be explained with the Classical grammar.
Still, the fact that roots within them do not seem to appear freely suggests that the translator likely had the lexicon of Mandarin in their mind.

\begin{exe}
    \ex\label{ex:pseudo.compound-verb.1} 放棄詩書,極意聲色
\end{exe}

Second, 

It is not impossible for Classical grammar to appear in a pseudo-Classical document, though.

\begin{exe}
    \ex 有组织之抵抗运动人员之在其本国领土内外活动者
\end{exe}

\subsubsection{Formulae in letters}

\begin{todobox}{Formulae}{official-documents}
    \begin{itemize}
        \item 此令
        \item 钧鉴
        \item 谨呈
        \item 母亲大人膝下,敬禀者,日前寄上海婴照片一张,想已收到
    \end{itemize}
\end{todobox}

\chapter{Discussions on quirky examples}

\begin{exe}
    \ex {} [良人者]_{\text{subject}} [所仰望而终身]_{\text{predicate}} 也
\end{exe}

It seems subjects of ordinary verbal clauses cannot be topicalized
(\ref{ex:grammatical.clause.subject.no-topic-1}),
but if the \ac{vp} is emphasized, topicalization is possible.

\begin{exe}
    \ex\label{ex:grammatical.clause.subject.no-topic-1} \begin{xlist}    
        \ex 三王既以定法度
        \ex *三王,既以定法度
    \end{xlist}
\end{exe}

\begin{exe}
    \ex 秦,虎狼之国,不可信,不如毋行
\end{exe}

What's the structure of the sentence below?

\begin{exe}
    \ex 將者,欲伐而未成,见其臣尚可以谏,而季氏尚可以止也
\end{exe}

In the example below, the subject of the second clause is \translate{other people},
and yet it seems the three clauses share a topic (i.e. 人).

\begin{exe}
    \ex 人而不仁,疾之已甚,亂也
\end{exe}

What's the structure of the sentence below? \translate{But as for Qiu, is it about state affairs?}

\begin{exe}
    \ex 唯求則非邦也與
\end{exe}

The sentence below seems to be due to omission: 宗庙会同,非诸侯而何为之?

\begin{exe}
    \ex\label{ex:quirky.topic.1} {} [宗庙会同]_i, 非诸侯而何 ---_{\text{main verb}} ---_i
\end{exe}

\begin{exe}
    \ex 赤也为之小,孰能为之大
\end{exe}



\begin{exe}
    \ex 其为人也孝弟,而好犯上者
    \ex 且夫水之积也不厚,则其负大舟也无力
\end{exe}

\begin{exe}
    \ex (人)如礼何
    \ex 如之奈何
\end{exe}

\begin{exe}
    \ex 父母唯其疾之忧
\end{exe}

\printbibliography[title=References]

\end{document}

\part{单粒子量子力学}

\documentclass[UTF8, a4paper]{ctexart}

\usepackage{geometry}
\usepackage{titling}
\usepackage{titlesec}
\usepackage{paralist}
\usepackage{footnote}
\usepackage{enumerate}
\usepackage{amsmath, amssymb, amsthm}
\usepackage{cite}
\usepackage{graphicx}
\usepackage{subfigure}
\usepackage{physics}
\usepackage{tikz}
\usepackage[colorlinks, linkcolor=black, anchorcolor=black, citecolor=black]{hyperref}

\geometry{left=3.18cm,right=3.18cm,top=2.54cm,bottom=2.54cm}
\titlespacing{\paragraph}{0pt}{1pt}{10pt}[20pt]
\setlength{\droptitle}{-5em}
\preauthor{\vspace{-10pt}\begin{center}}
\postauthor{\par\end{center}}

\DeclareMathOperator{\timeorder}{T}
\DeclareMathOperator{\diag}{diag}
\newcommand*{\ii}{\mathrm{i}}
\newcommand*{\ee}{\mathrm{e}}
\newcommand*{\diff}{\mathop{}\!\mathrm{d}}
\newcommand*{\st}{\quad \text{s.t.} \quad}
\newcommand*{\const}{\mathrm{const}}
\newcommand*{\comment}{\paragraph{注记}}
\newcommand*{\scheq}{Schr\"odinger's Equation}
\newcommand*{\reals}{\mathbb{R}}

\newenvironment{bigcase}{\left\{\quad\begin{aligned}}{\end{aligned}\right.}

\title{量子物理基本概念}
\author{wujinq}

\begin{document}

\maketitle

% 总之我在这几篇文章中把“可观察量”一词玩坏了,它可以表示厄米算符也可以表示其期望……
% 类似的还有“多粒子态”……

\section{抽象代数}

\subsection{指标升降}

符号约定:$\vb*{A}^2$代表$\vb*{A}$的模长平方,$A^2$则表示分量。

\subsection{算符和态}

\[
    \comm{\hat{A}}{\hat{B}}^\dagger = \comm{\hat{B}}{\hat{A}},
\]
因此两个算符对易当且仅当它们的共轭转置对易。

% TODO:明确地点出动力学变量这个概念

% TODO:分析三种绘景下的态
% 两个态表示了同样的物理状态,当且仅当,
% $\ket{\psi}$和$\hat{A}$组成的系统和$\hat{Q}\ket{\psi}$和$\hat{Q} \hat{A} \hat{Q}^\dagger$组成的系统等价,其中$\hat{Q}$是一个幺正算符;反之,如果两个长度等价的向量描述等价的系统,
% TODO: 设算符$\hat{A}$是CSCO,且它在幺正变换$\hat{P}$下不变,那么对任何一个本征值$A_i$,有一个单位复数,使得$\hat{P} \ket{A_i} = c \ket{A_i}$.
% TODO:虽然描写一个态空间可以需要不止一个算符(或者说这个空间的CSCO的大小不为1),但往往可以将这些CSCO拼凑成一个:
% \hat{A} \ket{a_1 a_2 \cdots} = \pmqty{a_1 & a_2 & \cdots} \ket{a_1 a_2 \cdots}
% 只要推导中不涉及本征值的乘除,这样做就没有任何问题。
% 因此下文中将常常这么写。
% 谱结构和对易关系之间有什么联系?

\textbf{表象}指的是态空间的一组正交完备基。由于通常这样一组基是某个CSCO$\hat{M}$的本征态,我们通常使用对应的$CSCO$来标记表象。例如,我们有坐标表象、动量表象,等等。
表象变换公式
\begin{equation}
    \braket{A_i}{\psi} = \sum_j \braket{A_i}{B_j} \braket{B_j}{\psi}
\end{equation}
是基的完备性的推论。

变换
\[
    \braket{A_i}{\psi} \longrightarrow \mel{A_i}{\hat{B}}{\psi}
\]
称为算符$\hat{B}$在$A$表象下的表示。显然,$\hat{A}$在$A$表象下的表示就是
\[
    \braket{A_i}{\psi} \longrightarrow A_i \braket{A_i}{\psi}.
\]

在离散谱的情况下,归一化条件相当简单:
\[
    \braket{A_i}{A_j} = \delta_{ij}
\]
在连续谱的情况下,需要使用积分代替求和,使用$\delta$函数代替$\delta$符号。
多分量算符的本征值有可能不是按照$\reals^n$的方式分布的,而是分布在一个弯曲的空间上(例如,分布在一个球面上)。此时通常需要使用类似于
\[
    \int \dd[n]{x} \delta(f(x))
\]
这样的测度,其中$f(x)$为描述弯曲空间的方程。其结果是,即使两个表象中的态矢量能够做到一一对应,由于使用的测度不同,它们仍然可以差一个模长不为1的系数。
换而言之,相同的态在不同的表象中会以不同的内积被归一化。

\subsection{李群和李代数,以及它们的表示}
% TODO:形如$\exp(\phi_1 G_1 + \phi_2 G_2 + \ldots)$的映射是不是一定可以写成$\exp (\phi_1' G_1) \exp (\phi_2' G_2) \ldots$?
在讨论对称性和守恒量的联系的时候

李群$U(t,t_0)$对应的无穷小生成元$H(t)$定义为
\begin{equation}
    H(t) = \lim_{\Delta t \to 0} \frac{U(t+\Delta t, t)}{\Delta t}
\end{equation}
使用编时算符$\timeorder$可以写出形式解
\begin{eqnarray}
    U()
\end{eqnarray}

如果$U(a+b)=U(a)U(b)$则生成元是常量。

有必要分析一下升降算符的东西:作用在一个本征态上得到另一个本征态??
可以看成一种平移:
\[
    U(\epsilon) \ket{x} = \ket{x + \epsilon}
\]

李代数中的$XY$和$YX$都未必是李代数的元素,但是$[X,Y]$一定是。

需要注意的是即使一个群的生成元是
\[
    \det \ee^{A} = \ee^{\trace A}
\]
只要知道了李括号就完全确定了整个李代数的结构。

生成元会变的情况??编时算符。

特别的,如果一个变换不改变哈密顿量,或者说“不改变物理规律”,且这个变换的生成元不显含时间,那么其生成元就是守恒量。(也就是说哈密顿量是这个群的卡西米尔算符??)

不可约表示中的卡西米尔算符的表示一定是恒等矩阵的倍数。例如,$\laplacian$是空间平移群的卡西米尔算符,而空间平移群在形如$A\exp (\ii \vb*{k} \cdot \vb*{r})$这样的平面波组成的线性空间上的表示是不可约表示(它自己就是一个不变空间,没有更小的不变子空间),那么$\laplacian u = - k^2 u$,可见确实是恒等变换的倍数。

幺正的李群按照$\exp(\theta J)$的形式得到的生成元是反厄米的,按照$\exp(\ii \theta J)$的形式得到的生成元是厄米的。

交换左右手坐标系的坐标变换行列式是负的,否则是正的。

\subsection{常用公式罗列}

使用算符代数的时候需要特别小心,因为不对易性很容易让我们习以为常的公式失效。


我们有
\[
    \dv{t} \ee^{t A} = A \ee^{t A} = \ee^{t A} A
\]
然而,
\[
    \dv{t} \ee^{A(t)} = \dv{A}{t} \ee^{A(t)} = \ee^{A(t)} \dv{A}{t}
\]
并不一般成立。

回顾经典物理,我们会发现对任何一种系统我们都尝试使用一系列固定的物理量描述它,例如一个粒子有位置、动量,一个场有各点的场量,等等。
有时也可以使用另一些物理量描述它,例如我们可以在速度和动量之间切换,可以使用不同的坐标系,等等。
因此,虽然实际计算中常常使用由一系列物理量的值组成的列表描述物理系统(举例:“粒子质量多少多少、位于$x$坐标多少多少、$y$坐标多少多少的位置”),
但观念上我们使用了两种对象:其一是系统的状态,它是某种流形上的一个点(例如在经典哈密顿力学中它是辛流形上的点),
其二是物理量,它是从这个流形到实数、复数、矢量、张量等“量”的映射。

在经典体系中“态”能够做的运算无非是从一个态转移到另一个态;态和态之间是完全孤立的。
然而,无论是理论上的推广(如将经典力学看成某个波动方程的程函方程)还是实验上的发现(如双缝干涉实验)都表明,这并不是完美描述自然界的正确方式。
实际上,态是可以像矢量一样叠加的(例如,干涉条纹意味着电子在空间中的分布可以看成某种场,
这种场满足叠加原理,那么电子在空间中的分布就是$\delta$函数为基底张成的向量空间中的元素)。
因此,一个\textbf{量子系统}指的是其状态可以使用某个希尔伯特空间$\mathcal{H}$中的向量$\ket{\psi(t)}$来描述、并且可以做叠加、数乘等运算的系统。

我们还需要一个额外的假设:一个态矢量数乘上一个复数得到的态矢量和原态矢量代表了同一个态。因此,我们将使用归一化的态矢量$\ket{\psi}$,并简称它们为“态”。
长度为零的态矢量不能归一化,我们认为它是非物理的,仅仅用于保证正确的代数结构,而不起多大作用。
同时我们也假定态矢量如果随时间发生演化,那么它一直是归一化的;进一步,任何作用于态之上的可逆变换都应该保持态的归一化。
或者说,任何作用于态上的可逆变换都应该是\textbf{幺正}的。

下一个问题是,我们怎样“诊断”或者说“读取”这系统的状态,也就是说怎样构造量子体系中的物理量。
实验上的观察(如双缝干涉实验中如果输入电子束密度足够低,是能够捕捉到单个电子的,但是其位置不固定)表明,
一个态$\ket{\psi}$并不对应着一个固定的物理量取值(刚才的例子表明一个态通常并不对应一个固定的位置)。
但是注意到一旦物理量的定义给定了(比如,假定我们接下来要测定位置),的确有\textbf{一些}态能够毫无疑义地确定物理量的取值
(例如,$\delta(x-x_0)$当然就对应一个位于$x_0$处的尖峰)。
因此量子物理中的物理量应该是这种“能够确定物理量取值”的态连同对应的物理量打包而成的结构。
至于那些不能明确确定物理量取值的态,可以把它写成能够明确确定物理量取值的态的线性组合来判断它对应哪些可能的物理量取值,这些物理量取值占比多少。
什么样的结构可以用来做这件事?一个自然的想法是\textbf{算符}:设诸$A_i$为可能的物理量$A$的取值,$\ket{A_i}$为这些取值对应的一组非零态,定义
\begin{equation}
    \hat{A} = \sum_i A_i \dyad{A_i}
\end{equation}
为该物理量对应的算符%
\footnote{一个细节:如果同一个$A_i$对应多个可能的$\ket{A_i}$,则容易看出这个$A_i$对应的所有态矢量对应一个向量空间。
此时需要写出这个向量空间的一组基矢量$\ket*{A_i^{(1)}}, \ket*{A_i^{(2)}}, \ldots$,然后用
\[
    A_i \left(\ket{A_i^{(1)}} + \ket{A_i^{(2)}} + \cdots\right)
\]
代替$A_i \ket{A_i}$。}
。这样一来,“物理量的取值能够确定”的态就是算符$\hat{A}$的本征态,于是我们就可以使用线性代数来处理有关的问题。
物理量随着时间演化就意味着我们有
\begin{equation}
    \hat{A}(t) = \sum_i A_i(t) \dyad{A_i(t)},
\end{equation}
也就是说每一个时间点都对应一个算符。
由于量子物理中大部分有意义的物理量都对应算符,接下来我们将经常混用物理量和算符这两个词%
\footnote{当然的确有一些物理量和算符关系不大,比如质量等,但因为它们总是被当成常数处理因此无大碍。}
。

需要注意的是如果一个态和另一个态的内积$\braket{\psi}{\phi}$不为零,那
%TODO:正交性的意义
因此我们之后均认定对应不同$A_i$的$\ket{A_i}$彼此正交,从而可以毫无顾虑地使用bra-ket记号。
另一方面有意义的物理量值都是实数(有时引入复数单纯是为了方便计算,如表示相位,等等),这就意味着物理量对应的算符都是\textbf{厄米算符}。

\subsection{对系统的等价描述}

设两个希尔伯特空间$\mathcal{H}$和$\mathcal{H}'$,它们使用一个可逆算符$\hat{A}$相关联,当然$\hat{A}$是幺正的。也就是说
\[
    \ket{\psi'} = \hat{A} \ket{\psi}, \quad \ket{\psi} \in \mathcal{H},  \ket{\psi}' \in \mathcal{H}'
\]
$\hat{A}$是一个同构。容易看出,若$\hat{O}$是$\mathcal{H}$中的一个算符,
那么
\begin{equation}
    \hat{O}' = \hat{A} \hat{O} \hat{A}^{-1} = \hat{A} \hat{O} \hat{A}^\dagger
\end{equation}
就是对应的$\mathcal{H}'$中保持代数结构不变的算符,这是下面几个式子的结果:
\[
    \begin{split}
        \ket{\psi'} = \hat{A} \ket{\psi}, \quad \ket{\phi'} = \hat{A} \ket{\phi}, \\
        \ket{\phi} = \hat{O} \ket{\psi}, \quad \ket{\phi'} = \hat{O}' \ket{\psi'}
    \end{split}
\]
一种常见的情况是,$\mathcal{H}$与$\mathcal{H}'$实际上是同一个空间,
则$\hat{O}$在变换$\hat{A}$下不变的充要条件是$\hat{O}'=\hat{O}$,也就是说$\hat{O}$与$\hat{A}$对易。
进一步,如果算符$\hat{O}$在一个李群作用下不变,那么它和每个群元都对易,这又等价于它和这个李群的所有生成元都对易。

上面我们讨论了对希尔伯特空间做一个变换会导致其上的算符做对应的变换。
现在我们讨论反过来的问题:如果两个算符的代数结构彼此对应,那么它们作用的希尔伯特空间之间会有什么样的关系。

% TODO:写串词
设有态矢量$\ket{\psi}$、算符$\hat{O}$,以及$\ket{\psi'}$和$\hat{O}'$,
若$\hat{O}$和$\hat{O}'$的谱结构相同(不变子空间同构,对应的本征值相同),且两个态矢量中含有的可观察量的各本征态的占比一致
则认为两系统等价,因为它们的代数结构不可区分。这时可以证明
\begin{equation}
    \mel{\psi}{A}{\psi} = \mel{\psi'}{A'}{\psi'}
\end{equation}
实际上,像这样的等价系统能够且只能够使用下面的方式产生:
\begin{equation}
    A' = U A U^\dagger, \quad \ket{\psi'} = U \ket{\psi}
\end{equation}
其中$U$为酉算符。要求$U$是酉算符是为了确保变换之后的$A'$的本征态的正交性,从而确保它确实是可观察量。
(由此也可以看出,要求使用复希尔伯特空间来描述系统而又一定要求可观察量的取值为实数实际上是很强的条件)

\subsubsection{从李群到李代数}

本文中我们将不对李群的流形结构进行正式的分析,而仅仅满足于使用一定的群参数把一个李群完整地表示出来。
一个李群中的成员可以一般地写成
\begin{equation}
    g = \exp(\ii \theta_i \sigma_i) \equiv \exp (\ii \theta^i \sigma_i) = \exp (\ii \vb*{\theta} \vb*{\sigma}),
    \label{eq:lie-group-element}
\end{equation}
其中$\theta_i$指的是群参数,而$\sigma_i$指的是生成元。
通常要求群参数为实数。
$\ii$是一个无关紧要的系数,加上它和不加上它唯一的区别就是$\sigma$需不需要乘上一个$\ii$。
为了方便,常常将诸$\theta$排成行向量,$\sigma$排成列向量。由于没有度规,无需区分上下指标。
对应的,设$\theta$是一个群参数,对应的生成元为
\begin{equation}
    \sigma = \frac{1}{\ii} \dv{g}{\theta}.
\end{equation}
需注意\eqref{eq:lie-group-element}假定了
\[
    g(\theta_1) g(\theta_2) = g(\theta_1 + \theta_2),
\]
这又等价于,无论$\theta$取什么值,$g$对$\theta$求导都会得到完全相同的结果。
在大多数情况下可以不失一般性地要求这个性质成立,因为群参数到底是什么并不重要
——我们总是可以巧妙地定义$\theta$使得$g$对$\theta$求导的结果与$\theta$无关%
\footnote{这是来自常微分方程的基本结论:设$X$是一个生成元,那么必定可以找到李群的一个单参数子群$c(t)$,使得
\[
    \dv{t} c(t) = c(t) \cdot X,
\]
从而可以定义指数映射。这是解析映射,因此可以使用诸如求导等运算。},
% 但是真的如此吗?时间演化一定构成李群吗?
% 一种可能的质疑是:在球面上随意画一条闭合轨迹,它显然描述了起点位于球心,终点位于球上面的矢量的一个连续变换,
% 然而它却不能使用$\exp (\alpha G)$的形式表示出来。
% 但这个质疑本身不成立,因为通常的李群总是可以作用在线性空间上的,然而上述变换显然没有线性性。
% 感觉还是很奇怪。
但是有一个重要的例外:时间演化。
我们关注的是“正常人眼中的时间”,而不能随意定义时间流逝的速率,
因此并没有什么能够保证不同$t$处时间演化算符对$t$求导的结果都是$t=0$(也就是恒等映射附近)时间演化算符对$t$求导的结果。
记$U(t, t_0)$为从$t_0$演化到$t$的算符,也即
\[
    U(t, t_0) U(t_0) = U(t),
\]
由于$t$不再能够任意选取,我们不能够写出\eqref{eq:lie-group-element}这样的指数映射,但是可以证明,一定存在一个$H(t)$使得
\begin{equation}
    U(t, t_0) = T \exp \left( \int_{t_0}^t \dd{t} H(t) \right).
    \label{eq:time-dependent-lie-group}
\end{equation}
这里我们略去了\autoref{sec:time-evolution}中的公式中的因子$- \ii /\hbar$,不过这无关紧要。$T$为编时算符。
在不同时刻的$H(t)$彼此对易的情况下可以把$T$去掉,因为此时重排各算符顺序不会产生任何影响。

\eqref{eq:lie-group-element}和\eqref{eq:time-dependent-lie-group}的区别体现在很多地方。
\eqref{eq:lie-group-element}意味着
\[
    g^{-1}(\theta) = g(-\theta),
\]
或者说
\[
    \left( \exp(\theta \sigma) \right)^{-1} = \exp(- \theta \sigma),
\]
但是在不同时刻的$H(t)$彼此不对易时,
\[
    T \exp(\int \dd{t} H(t))^{-1} \neq T \exp(- \int \dd{t} H(t)).
\]
相应的,
\[
    \dv{t} \left(T \exp(\int \dd{t} H(t))^{-1}\right) \neq -H.
\]
这就是\autoref{sec:time-evolution}中做绘景变换时不同绘景下的哈密顿算符不相等的根本原因。

李代数是李群在单位元附近的切空间,也就是说,是$g$在$\theta=0$附近沿着任意方向对$\theta$求导之后得到的结果组成的代数。
接下来我们将讨论\eqref{eq:lie-group-element}的李群,因为“不同点处求导结果不同”基本上只会在处理时间演化时用到,
而此时只有一个生成元(就是哈密顿量),没有必要讨论李代数。
由于李代数的封闭性,设$g_1, g_2, \ldots$是一组相互独立的生成元,它们中任意两个的李括号$\comm*{g_1}{g_2}$一定也是一个生成元,
这意味着它可以使用$g_1, g_2, \ldots$线性表示。
从而我们有
\begin{equation}
    \comm*{g_i}{g_j} = f_{ij}^k g_k.
    \label{eq:structure-of-lie-algebra}
\end{equation}
如果我们只讨论抽象的李代数的性质而不考虑它作用在某些对象上产生的结果,那么\eqref{eq:structure-of-lie-algebra}就完全刻画了一个李代数的结构。
因此,称$f_{ij}^k$为\textbf{结构常数}。

\subsubsection{李代数的具体计算}

% TODO:把前面用到这一节的内容的部分写得更加简洁一些
若
\[
    \comm*{\hat{q}}{\hat{p}} = c,
\]
则
\[
    \comm*{\hat{q}}{\hat{p}^n} = n c \hat{p}^{n-1}.
\]

\subsubsection{表示论}\label{sec:rep-th}

接下来需要讨论李群和李代数的表示。
通常考虑两种表示,其一是李群和李代数在向量空间上的作用,
也就是说,我们在李群、李代数和向量空间上的算符组成的群(以算符的复合为乘法)之间建立一个同态,
一旦建立起这个同态,我们实际上就得到了李群或李代数的一个表示。
比较方便的做法是,先讨论李代数在特定向量空间上的表示,然后使用指数映射获得对应的李群的表示。
第二种表示是,李群和李代数在向量空间上的算符构成的向量空间上的作用。
这种表示和第一种表示是紧密相关的。
设李群$G$在向量空间$V$上的表示为$G_V$,则$G_V \subset GL(V)$。这就自然地诱导出了李群在$GL(V)$上的表示。
算符$\hat{B} \in GL(V)$建立起了这样的关系:
\[
    \phi = \hat{B} \psi,
\]
现在我们把$\hat{A} \in G_V$作用在$\phi$和$\psi$上面,就得到
\[
    \phi' = \hat{A} \phi, \quad \psi' = \hat{A} \psi,
\]
如果我们还是希望在$\phi'$和$\psi'$之间建立关系
\[
    \phi' = \hat{B}' \psi',
\]
应该怎么选取$\hat{B}$?
考虑到$\phi$和$\psi$的任意性,容易看出,
\[
    \hat{B}' = \hat{A} \hat{B} \hat{A}^{-1}.
\]
我们没有规定$\hat{B}$是什么——它是完全任意选取的。这样一来,$G_V$中的每一个元素$\hat{A}$都对应到下面的映射:
\begin{equation}
    \hat{B} \longrightarrow \hat{A} \hat{B} \hat{A}^{-1},
    \label{eq:group-action-on-operators}
\end{equation}
\eqref{eq:group-action-on-operators}是一个从$GL(V)$到$GL(V)$的映射,也就是满足封闭性。
请注意该映射是$GL(GL(V))$的成员,而不是$GL(V)$的成员——它作用在$V$上的算符上而不是$V$中的向量上。
因此,我们通常只讨论简单的向量空间上的群表示,因为这些向量空间上的算符组成的向量空间上的群表示可以使用前者按照\eqref{eq:group-action-on-operators}写出。
另外注意,不同的$\hat{A}$可能对应着同一个\eqref{eq:group-action-on-operators}型的从算符到算符的映射。
这一点在处理旋转群时体现得很明显。

李群和李代数通常被作用在几类向量空间上。
首先是有有限个分量的向量空间。李群在其上的作用形如
\[
    v \longrightarrow v', \quad (v')^\mu = R_{\nu}^\mu (\Lambda) v^\nu.
\]
其中$\Lambda$指抽象的李群。
在有限维向量空间$V$上的表示可能有不变子空间,也就是说,存在$V$的一个子空间$V'$,使得李群中的任何一个成员作用在$v \in V'$上之后得到的结果都还是在$V'$中。当然,$V$以及$\{0\}$一定是不变子空间。
如果一个表示有不是这两个空间的不变子空间,那么这就是一个\textbf{可约表示},反之则为\textbf{不可约表示}。
可以证明,任何一个可约表示都可以写成一系列不可约表示的直和。因此对有限维表示而言,只需要讨论不可约表示就可以了,因为可约表示可以使用不可约表示组装出来。
现在讨论不可约有限维表示。
首先可以证明,任何李群的生成元至少有一个(当然也可以有很多个)可以相似变换为对角矩阵。
% TODO:是不是每一个生成元都可以?
这些被对角化的生成元的集合称为Cartan子代数,它是对应的李群的李代数的表示的子代数。
Cartan子代数中的诸算符共享一组可以张成整个$V$的本征矢量,对应的各生成元的本征值——也就是对角矩阵的对角元——可以用来标记这个不可约表示。
要找到一组Cartan子代数并不难:只需要从李群中找到一个交换子代数,然后尝试对角化这个交换子代数中的某一个成员就可以了。
% TODO:李代数在怎样的程度上决定了对应的算符的谱结构?
非奇异矩阵表示一定可以通过相似变换而变成幺正表示(就是所有矩阵都是幺正的表示)。
这也就是我们频繁地讨论幺正表示的原因。但有许多重要的群——例如洛伦兹群——都不是紧致的(或者说群对应的流形无界),因此它们实际上并没有有限维的幺正表示。就洛伦兹群而言,我们将会看到,其推动生成元的有限维表示不是厄米的,因此整个群也没有幺正的有限维表示。

容易验证,设$\hat{X}$是厄米算符,且
\begin{equation}
    \comm*{\hat{a}^\dagger}{\hat{X}} = c \hat{a}^\dagger,
    \label{eq:raising-operator}
\end{equation}
那么
\[
    \hat{a}^\dagger \ket{X} \propto \ket{X+c},
\]
相应的,
\[
    \hat{a} \ket{X} \propto \ket{X-c}.
\]
因此称$\hat{a}^\dagger$为$\hat{X}$的\textbf{升算符},$\hat{a}$为$\hat{X}$的\textbf{降算符}。
数学上可以证明,在李代数的有限维表示上可以定义内积
\begin{equation}
    \langle \hat{A}, \hat{B} \rangle = \trace \hat{A} \hat{B},
\end{equation}
通过合适的线性组合,能够写出一组正交归一化的生成元。
此时非Cartan子代数的生成元中的每一个都是Cartan子代数中的每一个成员的升降算符,
并且任意两个非Cartan子代数的生成元的对易子都可以使用Cartan子代数的成员线性表示。
% TODO:看起来Cartan子代数似乎构成它的不可约表示空间的一个CSCO
% Symmetry and the Standard Model, p108
因此对一个一般的、没有正交归一化的李代数的有限维表示,我们总是可以从李代数的成员构造出一个升算符。设$\hat{X}$为$g_i$,且
\[
    \hat{a}^\dagger = \lambda^j g_j,
\]
则\eqref{eq:raising-operator}等价于
\[
    \comm*{\lambda^j g_j}{g_i} = c \lambda^j g_j,
\]
代入\eqref{eq:structure-of-lie-algebra},上式又等价于
\begin{equation}
    \left( f^k_{ji} - c \delta_j^k \right) \lambda^j = 0,
    \label{eq:determine-ladder-operators}
\end{equation}
于是通过求解
\begin{equation}
    \det \left( f^k_{ji} - c \delta_j^k \right) = 0
    \label{eq:possible-c}
\end{equation}
就可以得到所有可能的$c$,然后将它们代入\eqref{eq:determine-ladder-operators}就能够得到所有能够被非Cartan子代数表示出来的升降算符。
最后,由于是有限维表示,通过以上手法得到的升降算符实际上就是全部可能的升降算符,因此从一个本征态出发,通过它们可以构造出所有的本征态。
有限维表示还意味着,设$\hat{a}^\dagger$是某个升算符,那么对充分大的$N$,$(\hat{a}^\dagger)^N = 0$,$\hat{a}^N=0$,因为本征态的个数有限。
这些条件可用于确定什么样的不可约表示是被允许的。
% TODO:数学证明,不过多半鸽了
这些操作的一个典型的例子见对旋转群的处理。

现在我们分析一种比较特殊的情况。以上我们都是在“李代数可以分解成一个Cartan子代数和非Cartan元素,后者构成前者的升降算符”的框架下分析问题,那么如果李代数中所有元素都对易,那此时它会有怎样的表示?
由于没有非Cartan元素,这样的一个李代数——从而它的李群——不会有有限维的不可约表示。
通常这样的李群对应着某种空间平移操作。

% TODO:连续谱的情况
以上讨论的不可约表示都是有限维的。无限维表示——这里指的是函数空间的表示——则需要一套不同的框架。设$\hat{q}$具有连续谱,且
\begin{equation}
    \comm*{\hat{q}}{\hat{p}} = \ii,
\end{equation}
则
\begin{equation}
    \exp \left( \ii \lambda \hat{p} \right) \ket{q} = \ket{q + \lambda}.
\end{equation}
也就是说$\exp (\ii \lambda \hat{p})$是让$\hat{q}$的本征矢对应的本征值上升$\lambda$的升算符。

由于空间坐标无非是一种向量,李群和李代数也可以被作用在坐标上。
作用在坐标上的有限维表示又诱导出了作用在函数上的无限维表示%
\footnote{在有限维表示中,上下标$\mu$标记向量的诸分量;在函数空间中,坐标$x^\mu$标记“向量”——也就是函数——的诸“分量”——也就是函数在这一点的值。
李群在有限维向量空间上的表示通常是某个矩阵群,它将不同分量混合在一起,即
\[
    \psi^\mu \longrightarrow R^\mu_\nu \psi^\nu.    
\]
李群在无限维向量空间上的表示通常是“改变坐标$x^\mu$”。
}%
。设$f=f(x)$,若李群在坐标上的表示为
\[
    x \longrightarrow x', \quad (x')^\mu = R_\nu^\mu (\Lambda) x^\nu,
\]
则它在关于坐标的函数——也就是“场”——组成的无限维向量空间上的表示就是
\[
    f \longrightarrow f', \quad f(x) = f'(x') = f'(R(\Lambda) x),
\]
或者等价的,
\begin{equation}
    (x \mapsto f(x)) \longrightarrow (x \mapsto f'(x) = f(R(\Lambda)^{-1} x)).
    \label{eq:infinite-dim-rep}
\end{equation}
换而言之,坐标变动“牵引”了从坐标到场值的映射。
考虑到$f$可能是某个多分量对象(比如矢量、矢量的张量积,或者接下来要看到的旋量)的分量,
李群在此多分量场上的作用还包括通常的有限维表示,也就是
\[
    \psi^a \longrightarrow M(\Lambda)^a_b \psi^b.
\]
需注意此处我们使用了另外一个表示$M^a_b$而不是$R^\mu_\nu$,因为不能够保证$\Lambda$在多分量场$\psi$上的作用和它在坐标向量上的作用来自同一个有限维表示。
由于大部分情况下我们都是从一个群在通常意义上的矢量的作用出发讨论其结构的,可以将$R(\Lambda) x$简记为$\Lambda x$,也就是群元$\Lambda$在$x$上的作用。
这样上式就可以简洁地写成
\begin{equation}
    \psi^a(x) \longrightarrow {\psi'}^a (x) = M^a_b (\Lambda) \psi^b (\Lambda^{-1} x).
    \label{eq:wigner-transform}
\end{equation}
这种同时考虑了多分量场在李群作用下各分量重新混合(这是一个有限维表示)和李群作用下坐标拖曳而改变场(这对坐标而言是另一个有限维表示,对场而言是一个无限维表示)的李群的表示就是\textbf{场表示}。
需要注意的是,不同的$\Lambda$作用到坐标上可能会得出同样的结果,而它们对应的$M$作用到场上却有不同的结果,正如$SU(2)$和$SO(3)$的关系告诉我们的那样。

\eqref{eq:wigner-transform}给出的是李群的场表示的一般形式,但此时我们还只有形式上的变换而没有显式的表达式。
我们来分析其李代数。取%
\footnote{虽然可以任意地调整群参数,从而让生成元前面的系数随意变动,但是通常对有限维表示和无限维表示我们总是采用同样的群参数。这就意味着,在有限维表示确定之后不能随意调节无限维表示的生成元前面的系数,不能随意加一个$\ii$或者改变正负号。这也就是我们在场表示中一并处理有限维表示和无限维表示的原因,因为此时两者的群参数自动地就是相同的。

下式中的$g$的定义可以是\[
    g = \frac{1}{\ii} \pdv{G}{g},
\]
但也可以是像我们定义旋转生成元时的那样,取
\[
    g = \ii \pdv{G}{g},
\]
只需要将$\epsilon$取为负值就可以了。无论$g$是怎么定义的,下式都是成立的。}%
\[
    \Lambda = I + \ii \epsilon g,
\]
其中$g$是一个生成元,我们就有
\[
    \begin{aligned}
        \psi^a \longrightarrow {\psi'}^a &= M^a_b (\Lambda) \psi^b (\Lambda^{-1} x) \\
        &= (I + \ii \epsilon M^a_b(g)) \psi^b (x - \ii \epsilon g x) \\
        &= (I + \ii \epsilon M^a_b(g)) (\psi^b - \ii \epsilon g x \cdot \grad{\psi^b}) \\
        &= \psi^b + \ii \epsilon M^a_b(g) \psi^b - \ii \epsilon g x \cdot \grad{\psi^b},
    \end{aligned}
\]
于是
\[
    {\psi'}^a = (I + \ii \epsilon  (M^a_b(g) - g x \cdot \grad)) \psi^a,
\]
于是场表示的生成元可以写成
\begin{equation}
    M_\text{field} = M_\text{fin} + M_\text{inf}, \quad M_\text{fin} = M^a_b(g), \quad M_\text{inf} = - (g x) \cdot \grad.
    \label{eq:fin-and-inf-rep}
\end{equation}
其中$M_\text{fin}$就是我们所熟悉的李群在有限维向量空间上的矩阵表示,而$M_\text{inf}$则是李群作用在坐标上,拖曳坐标而对场产生的影响。
显然,它们和$g$之间能够建立同态关系。$gx$和$\Lambda x$一样,都是“$g$在坐标空间上的有限维矩阵表示作用于$x$”的简写。
与通常物理中的记号不同,此处的梯度算符作用在所有坐标上,不仅仅是空间坐标,还包括时间坐标。

在以上讨论的基础上我们讨论态矢量。我们总是使用李群在希尔伯特空间上的幺正表示,因为需要保证变换前后的态矢量都是物理的,也就是说,都是正交归一化的。
我们刚才讨论了李群的场表示,这个场表示当然可以被作用在算符场上。但是注意到算符场是态空间上的算符,因此按照\eqref{eq:group-action-on-operators},李群的场表示自然地如下导出了李群在希尔伯特空间上的表示:
\begin{equation}
    \hat{U}(\Lambda) \hat{\psi}^b(\vb*{x}) \hat{U}^{-1}(\Lambda) = M^a_b (\Lambda) \psi^b (\Lambda^{-1} x).
    \label{eq:field-rep-and-state-rep-lie-group}
\end{equation}
由于对$\hat{\phi}$的变换等价于对其本征值做变换,这又等价于保持本征值不变而重新安排本征态,按照上式诱导出的在希尔伯特空间上的李群表示$\hat{U}$也是幺正的。

相应的,\eqref{eq:field-rep-and-state-rep-lie-group}也导致了对应的李代数在希尔伯特空间上的表示。对\eqref{eq:field-rep-and-state-rep-lie-group}取微元,得到
\[
    (1 + \ii \epsilon M_\text{state}) \hat{\psi}^b (1 - \ii \epsilon M_\text{state}) = \ii \epsilon M_\text{field} \hat{\psi},
\]
从而
\begin{equation}
    \comm*{M_\text{state}}{\psi} = M_\text{field} \psi.
    \label{eq:field-rep-and-state-rep-gen}
\end{equation}
实际上,时间演化方程\eqref{eq:quantum-evolution}就是一个例子:时间平移群在希尔伯特空间上的表示是哈密顿算符$\hat{H}$,在场——这里是任何一种物理量——上的表示是$\frac{1}{\ii} \dv{t}$,那么
\[
    \comm*{\hat{H}}{\hat{A}} = \frac{1}{\ii} \dv{t},
\]
这就是时间演化方程。

考虑一个简单的单粒子量子力学的例子:$\hat{x} + a$是将大小为$a$的平移作用在$\hat{x}$上的结果,而考虑被$\hat{x}$完全描述的一个希尔伯特空间,在其上有
\[
    \hat{x} + a = \int \dd{x} x \dyad{x} + a \int \dd{x} \dyad{x} 
    = \int \dd{x} (x + a) \dyad{x} = \int \dd{x'} x' \dyad{x'-a},
\]
因此作用在$\hat{x}$上的大小为$a$的平移就等价于作用在态空间基矢量上的大小为$-a$的平移。
更一般的,将某一个李群$Q(a)$作用在某一算符上就相当于将这一李群的群参数倒转过来得到新的李群$Q'$,
也就是定义$Q'(a) = Q(a)^{-1}$(由于是群,$Q'$和$Q$同构),然后将$Q'(a)$作用在态空间的基矢量上。
由于$Q'$和$Q$同构,两者的区别仅仅是重新规定了群参数,因此它们对应着同样的对称性。
% TODO:以上说法的推广
总之,我们既可以直接从某种李群的场表示出发,推导它允许的算符场有哪些,然后使用二次量子化的有关知识导出其对应的单粒子态,%
\footnote{关于何为“粒子”需要说明:一般把能够使用一个不很复杂的CSCO描述的量子系统称为粒子,例如可以使用$\hat{\vb*{x}}$或$\hat{\vb*{p}}$描述一个粒子。但按照这种定义,原子能级也可以算粒子了——实际上这并不是胡思乱想,在处理量子光学等领域的一些问题时确实可以将能级看成一种粒子,定义其产生湮灭算符,得到费米场,等等——因此,何为粒子更多的是一种约定的说法。实际上任何一个哈密顿量都可以对角化,写出能级之后将不同能级看成不同粒子,然后使用二次量子化的语言描述它。}%
也可以从李群在希尔伯特空间上的表示出发,直接得到单粒子态然后构造算符场。
两种方法是完全一致的。舒尔引理告诉我们,卡西米尔算符(和所有生成元都对易)在不可约表示中一定是恒等算符的常数倍。这个常数可以用来标记相应的不可约表示;事实上这一类常数往往会出现在相应的表示描写的场/粒子的运动方程中,因为运动方程中会出现卡西米尔算符的场表示。
相对而言,在推导运动方程的时候,使用场的观点更加方便,因为相对论情况下粒子数通常是不确定的,因此使用单粒子态难以写出哈密顿量。

概括以下我们至今得到的结果:李群和李代数的表示有下面几种,它们彼此之间有非常密切的关系。
首先,李群和李代数在有限维向量空间上的表示是矩阵,它们或者是可约表示,或者不可约,前者可以通过直和运算由后者组装出来。
不可约有限维表示的结构可以通过使用李代数中的非Cartan元素构造Cartan子代数的升降算符来确定。
通过将有限维表示作用在坐标上,我们得到了作用在关于坐标的函数组成的向量空间上的无限维表示。
将作用在多分量对象上的有限维表示和作用在坐标函数上的无限维表示结合起来,就得到了场表示。
李群在向量空间上的表示很自然地就诱导出了李群在作用在向量空间上的算符上的表示。

\section{动力学}

% 似乎拉格朗日动力学中含有虚部的场要看成两个场,而哈密顿动力学中含有虚部的场只需要看成一个场。
在进一步展开下面的叙述之前,我们先回顾现代物理的数学框架。总的来说,有两套可用的框架,
其一是拉格朗日动力学,路径积分方法是它的量子版本;其二是哈密顿动力学,正则量子化是它的量子版本。
尽管这两个框架在数学上是独立的,我们仍然可以找到它们之间非常深厚的联系。

本节首先从经典拉氏量出发,然后得到经典哈密顿量,然后再过渡到量子形式。
常见的物理问题涉及$3+1$维闵可夫斯基时空中或$0+1$维时空,而后者可以看成前者的一个退化情况,
于是我们将局限在$3+1$维闵可夫斯基时空中,
虽然无论是拉格朗日动力学还是哈密顿力学都适用于比这广得多的体系。
所谓闵可夫斯基时空指的是度规可以化为
\begin{equation}
    \eta_{\mu\nu} = \diag (1, -1, -1, -1)
\end{equation}
的四维几何。通常使用$t, x, y, z$或者$x^0, x^1, x^2, x^3$来依次标记这4个坐标。
容易看出$x, y, z$或者说$x^1, x^2, x^3$就构成一个三维欧氏几何,它们是\textbf{空间维}。
$x^0$则是\textbf{时间维}。

我们还将假设,所有场量在无穷远处的值都是零。
我们将要分析的对象是时空中的场,它是从闵可夫斯基时空到某一线性空间的光滑映射。

\subsection{拉格朗日动力学}

所谓\textbf{拉氏量密度}$\mathcal{L}$——在场论中简称为\textbf{拉氏量}——是这样一个量,它是场的局域泛函,
这就是说,它可以写成$\phi, \partial_\mu \phi, \ldots$以及时空坐标的函数。
本文假定所有的拉氏量仅含有一阶导数,这是为了避免含有高阶导数的拉氏量产生“可以无穷下降的能量”等反直觉现象,并且简化计算。
幸运的是,已有的实验数据并不要求我们考虑更高阶的拉氏量。
我们还假定物理规律在时空上是均匀的,因此我们不认为拉氏量中显含时空坐标。%
\footnote{
    需要注意的是在系统中有相互作用且其中一部分的运动状态已知的情况下,另一部分的等效拉氏量中是有可能出现时空坐标的,
    例如粒子在势场中的运动就是一个典型例子,在那里由于产生势场的物理机制远远比粒子本身要强,因此势场可以看成是给定的,
    于是粒子具有的等效拉氏量就显含了空间坐标。}%
从而我们有
\begin{equation}
    \mathcal{L} = \mathcal{L}(\phi, \partial_\mu \phi).
    \label{eq:lagrangian}
\end{equation}
需要注意的是\eqref{eq:lagrangian}中的$\phi$可以代表任何一个“从时空坐标到数量”的映射,
它可能是一个标量场也可能是一个矢量场的分量,或者是别的什么东西。
\textbf{作用量}是拉氏量在整个闵可夫斯基时空上的积分。

现在我们将一个任意的无穷小变换作用在泛函$S$上,观察其无穷小变动。
需要注意的是无穷小变换同时作用在$\phi$的场值和坐标上,从而$\phi$完整的变化%
\footnote{在实际计算时往往更加容易求出$\var{\phi}$,因为一旦把$\phi'(x')$完全写出,只需要计算$\phi'(x')-\phi(x)$ 即可。}%
同时包含两部分:
\begin{equation}
    \var{\phi} = \bar{\var} \phi + \partial_\mu \phi \var{x}^\mu,
    \label{eq:variance-of-phi}
\end{equation}
其中第一项指的是场值本身的变化%
\footnote{这个变化又有可能来自两个方面。
其一是“场的平移”,也就是我们手动把场$\phi$加减特定值;
其二是“场的旋转”,当$\phi$实际上是某个更大的对象(如矢量)的某个分量时,基矢量的旋转会导致不同的分量混在一起。
通常我们使用一样的基矢量来书写场的分量和坐标的分量,因此除了坐标平移外,坐标变换也伴随着非零的$\bar{\var}{\phi}$。}%
,第二项指的是坐标变换的“拖曳”作用。
坐标的变化还会导致导数算符和积分测度发生变化。这两个几何效应的具体表达式为
\begin{equation}
    \begin{bigcase}
        \partial_{\mu'} = \partial_\mu - \partial_\mu \var{x^\nu} \partial_\nu, \\
        \dd[4]{x'} = (1 + \partial_\mu \var{x^\mu}) \dd[4]{x}.
    \end{bigcase}
\end{equation}
由于$\partial_\mu$算符随着坐标变换会发生变化,我们发现$\partial_\mu \phi$的变化量的形式和$\phi$不完全一致:
\begin{equation}
    \var{\partial_\mu \phi} = \partial_\mu \bar{\var}{\phi} + \partial_\mu \partial_\nu \phi \var{x^\nu}.
\end{equation}
这样一来我们可以计算出
\begin{equation}
    \var{S} = \int \dd[4]{x} \left(
        \left( \pdv{\mathcal{L}}{\phi} - \partial_\mu \pdv{\mathcal{L}}{\partial_\mu \phi} \right) \bar{\var}{\phi} + 
        \partial_\mu \left( \mathcal{L} \var{x^\mu} + \pdv{\mathcal{L}}{\partial_\mu \phi} \bar{\var}{\phi} \right)
    \right).
    \label{eq:variance-of-s}
\end{equation}
在推导\eqref{eq:variance-of-s}时我们没有使用任何关于$\var{\phi}$和$\var{x}$的假设,因此它给出的是最一般的$\var{S}$形式。

实际的场的动力学由保持时空坐标$x$不变且$\phi$在无穷远处固定为零(从而无穷远处$\bar{\var}{\phi}$为零)的情况下的泛函极值问题
\begin{equation}
    \var{S} = \var{\int \dd[4]x \mathcal{L}(\phi, \partial_\mu \phi)}
    \label{eq:min-action}
\end{equation}
给出。
显然这个泛函极值问题的解就是
\begin{equation}
    \pdv{\mathcal{L}}{\phi} - \partial_\mu \pdv{\mathcal{L}}{\partial_\mu \phi} = 0.
    \label{eq:el-eq}
\end{equation}
这就是欧拉-拉格朗日方程。
由于推导欧拉-拉格朗日方程时用到了$\var{\phi}$的任意性,这意味着$\phi$被假定是一个实的场。
如果某些场有虚部,那么在使用\eqref{eq:el-eq}以及相关结论的时候需要把它的实部和虚部分开,当成两个场来处理。
并且,容易证明,设复场$\phi$的实部和虚部分别是$\phi_1$和$\phi_2$,且
\[
    \pmqty{\psi_1 \\ \psi_2} = \pmqty{a & b \\ c & d} \pmqty{\phi_1 \\ \phi_2},
\]
其中$a,b,c,d$为复常数,则$\psi_1$和$\psi_2$的运动方程也可以从\eqref{eq:el-eq}得出。
常见的选择包括取
\[
    \psi_1 = \phi, \psi_2 = \phi^\dagger,
\]
或者如果$\phi$是多分量场,设有一系列复矩阵(不必都是复矩阵,有一个是复的就可以)$\gamma^\mu$,取
\[
    \psi_1 = \phi, \psi_2 = \gamma^\mu \phi_\mu.
\]

需要注意如果两个拉氏量的形式不同,这并不意味着它们描述了不同的物理过程。
实际上容易看出,两个拉氏量描述了相同的物理过程,
当且仅当,它们给出的作用量$S$只相差一个相对于$\dd[4]{x}$的零测集上的积分(这样的积分不影响泛函极值问题的求解,因为它“太小”),
这又等价于这两个拉氏量相差一个散度项,即存在一个$\Lambda^\mu$使得
\begin{equation}
\mathcal{L}' = \mathcal{L} + \partial_\mu \Lambda^\mu.
\end{equation}

当场量$\phi$是物理解的时候,将$\phi$代入到$S$中,然后再做一个无穷小变换,此时\eqref{eq:variance-of-s}中的第一项为零,
于是我们有
\[
    \var{S} = \int \dd[4]{x} \partial_\mu \left( \mathcal{L} \var{x}^\mu + \pdv{\mathcal{L}}{\partial_\mu \phi} \bar{\var}\phi \right).
\]
如果这个无穷小变换实际上不改变系统的动力学,也就是说系统在这个无穷小变化下是对称的,
那么$\var{S}$就应该能够写成一个表面积分,于是我们得到
\begin{equation}
    \partial_\mu \left(\pdv{\mathcal{L}}{\partial_\mu \phi} \bar{\var}\phi + \mathcal{L} \var{x^\mu} + \Lambda^\mu\right) = 0.
    \label{eq:noether}
\end{equation}
当然,如果无穷小变换更进一步不改变拉氏量,那么$\Lambda=0$。

如果无穷小变换是一个李群的李代数的表示,那么$\bar{\var}{\phi},\var{x^\mu}, \Lambda^\mu$都是完全确定的。可以使用小量近似将$\bar{\var}{\phi}$写成小量$ \ii \epsilon$乘以李代数的场表示\eqref{eq:fin-and-inf-rep},$\var{x^\mu}$写成小量$\ii \epsilon$乘以李代数的四维矢量表示,
于是我们在\eqref{eq:noether}中除去一个$\epsilon$,就得到了一个守恒流。
于是\eqref{eq:noether}的括号中的内容能够完全写成坐标的函数。
这就是\textbf{诺特定理}:系统的无穷小对称性诱导出一个守恒流。
由于是四维闵可夫斯基时空,四维的一个守恒流
\begin{equation}
    \partial_\mu j^\mu = 0
\end{equation}
就意味着三维的一个输运方程
\begin{equation}
    \partial_t j^0 + \partial_a j^a = 0.
\end{equation}
从而,
\begin{equation}
    Q = \int \dd[3]x j^0
\end{equation}
就是一个\textbf{守恒荷}。如果其积分范围是一个有限的区域,那么它就是一个局域守恒量,也就是
\[
    \dv{t} Q = - \int \dd{\vb*{S}} \cdot \vb*{j},
\]
而如果其积分范围是全空间,那么它就是守恒的。

我们来检查一下常见的对称性导致的守恒量。%
\footnote{表面上看,下面的讨论在体系并不非常对称的情况下并无意义,而不非常对称的体系占了多数。
不对称性带来的后果是,我们不再有完美的守恒流方程,取而代之的是一个有源的输运方程
\[
    \partial_\mu j^\mu = \text{something},
\]
由于对称性分析无助于找到源的具体形式,使用对称性诱导出特定的物理量似乎并没有什么意义。
然而,我们相信,最基本的物理定律总应该是对称的,因此大部分体系的不对称性可以归结为我们人为地将它从环境中隔离出来进行研究,从而导致类似下面的方程:
\[
    \partial_\mu (j^\mu_\text{sys} + j^\mu_\text{env}), \quad \partial j^\mu_\text{sys} = - j^\mu_\text{env}
\]
第二个方程给出了我们想要的含源的输运方程。因此在分析基本的物理框架时我们可以不讨论“不对称”的情况,
而是导出了基本的方程之后再通过“隔离出一部分系统”来引入不对称性。
}%
假定拉氏量在变换下不变。下面处理的问题都只含有一个场,不过由拉氏量的叠加性,在拉氏量含有多个场的时候只需要把各部分加起来即可。
首先是最简单的平移。处理平移时假定场是标量场,这无损一般性,因为平移没有有限维表示,因此不会导致场分量发生混合。
平移变换作用于场上得到的结果是:
\[
    \begin{split}
        x^\mu \longrightarrow x^{\mu'} = x^\mu + a^\mu, \\
        \var{\phi} = \phi'(x') - \phi(x) = 0.
    \end{split}
\]
% TODO:群作用怎么取
按照\eqref{eq:variance-of-phi},可以计算出
\[
    \bar{\var}{\phi} = - \partial_\mu \phi \var{a^\mu},
\]
或者,由于场在坐标拖曳下的变动实际上就是平移变换的无限维表示,可以直接使用平移变换的无限维表示
\[
    P_\mu = - \ii \partial_\mu
\]
得到上式。
于是对应的守恒流为
\[
    0 = \partial_\mu \left( - \pdv{\mathcal{L}}{\partial_\mu \phi} \partial_\nu \phi \var{a^\nu} + \mathcal{L} \var{a^\mu} \right) 
    = \partial_\mu \left( - \pdv{\mathcal{L}}{\partial_\mu \phi} \partial_\nu \phi + \mathcal{L} \delta^\mu_\nu \right) \var{a^\nu},
\]
考虑到$\var{a^\mu}$的任意性,我们有
\begin{equation}
    T_\mu^\nu = \pdv{\mathcal{L}}{\partial_\nu \phi} \partial_\mu \phi - \mathcal{L} \delta^\nu_\mu, \quad \partial_\nu T_\mu^\nu = 0.
\end{equation}
我们称$T^\nu_\mu$为\textbf{能动张量}。它给出了4个守恒荷,其中一个是来自时间平移不变性的\textbf{能量}
\begin{equation}
    E = \int \dd[3]{x} T^0_0 = \int \dd[3]{x} \left( \pdv{\mathcal{L}}{\partial_0 \phi} \partial_0 \phi - \mathcal{L} \right) ,
    \label{eq:field-energy}
\end{equation}
另外三个是来自空间平移不变性的\textbf{动量}
\begin{equation}
    P_i = \int \dd[3]{x} T^0_i = \int \dd[3]{x} \pdv{\mathcal{L}}{\partial_0 \phi} \partial_i \phi .
    \label{eq:field-momentum}
\end{equation}
能动张量的纯空间部分是能量和动量的输运流,因此就是\textbf{应力张量}。%
\footnote{在非相对论连续介质力学中这些结果也是成立的,因为时间和空间平移同时出现在伽利略群和庞加莱群中。}
相应的,
\begin{equation}
    \mathcal{P}_\mu = \pdv{\mathcal{L}}{\partial_0 \phi} \partial_\mu \phi - g_\mu^0 \mathcal{L}
\end{equation}
为四维动量$(E, \vb*{p})$的密度。
在计算场的三维动量时要注意一点:由于闵可夫斯基度规为$(+, -, -, -)$,闵可夫斯基时空中空间部分的基矢量实际上是指向空间坐标减少的方向的。从而,
\[
    \begin{aligned}
        \vb*{P} &= \int \dd[3]{x} \pdv{\mathcal{L}}{\partial_0 \phi} \partial_i \phi \vb*{g}^i \\
        &= - \int \dd[3]{x} \pdv{\mathcal{L}}{\partial_0 \phi} \partial_i \phi \vb*{g}^i_{\text{3dim}},
    \end{aligned}
\]
也即
\begin{equation}
    \vb*{P} = - \int \dd[3]{\vb*{x}} \pi \grad{\phi}.
\end{equation}

接下来是旋转对称性。%
\footnote{同样,这个对称性无论是在相对论性场论还是非相对论性场论中都是成立的。}%
旋转对称性不涉及时间维,于是我们有
\[
    \var{x^i} = \epsilon^i_{\ jk}  x^j \theta^k,
\]
相应的
\[
    \bar{\var}{\phi^a} = \ii (J_i)^a_{\ b} \theta^i \phi^b - \epsilon^i_{\ jk}  x^j \theta^k \partial_i \phi^a.
\]
其中指标$a,b$跑遍$\phi$的所有分量,$J$指的是旋转生成元在$\phi$所属的向量空间上的表示。

然后我们分析场的内禀对称性带来的守恒量。容易看出场的平移,也就是
\[
    \bar{\var}{\phi} = a, \; \var{x} = 0
\]
对应着守恒流
\[
    \partial_\mu \pdv{\mathcal{L}}{\partial_\mu \phi } a = 0,
\]
其守恒荷为
\begin{equation}
    \Pi = \int \dd[3]x \pdv{\mathcal{L}}{\partial_0 \phi}.
\end{equation}
这称为$\phi$的\textbf{共轭动量},相应的其密度
\begin{equation}
    \pi = \pdv{\mathcal{L}}{\partial_0 \phi}
    \label{eq:def-pi}
\end{equation}
就是\textbf{共轭动量密度}。
需注意此“动量”的名称只是类比而得,它未必和$P_i$有特别紧密的联系。

\subsection{哈密顿动力学}

\subsubsection{经典哈密顿动力学}
原本可以直接从拉氏量通过一个勒让德变换得到哈密顿动力学,但当底流形有多个坐标时我们需要选择合适的一个或几个坐标来充当“时间”,也就是哈密顿系统的参数。
共轭动量使我们有了一个很好的选择。本文取$t=x^0$为哈密顿系统的单参数。接下来我们要观察共轭动量的变化情况,从而凑出一个哈密顿系统。

容易看出$\Pi$的运动方程为%
\footnote{本节的结果也不仅仅适用于相对论性场论。任何能够良定义场的平移并且保证场平移不改变拉氏量的拉格朗日动力学场论都可以使用本节的方法构造对应的哈密顿表述,因为本节只用到了场的内禀平移不变性诱导出的结构。}
\[
    \dv{\Pi}{t} = \int \dd[3]x \partial_0 \pdv{\mathcal{L}}{\partial_0 \phi} = \int \dd[3]x \left( \pdv{\mathcal{L}}{\phi} - \partial_i \pdv{\mathcal{L}}{\partial_i \phi} \right).
\]
被积函数是$\int \dd[3]x \mathcal{L}$在将$x^0$当成常数后对$\phi$泛函求导的结果。于是定义%
\footnote{在$\phi$是多分量场的时候,我们把它看成列向量,记号$\partial \mathcal{L} / \partial \phi$定义为一个行向量,从而所有公式形式上仍然成立。例如,
\[
    \pdv{\vb*{a} \cdot \vb*{x}}{\vb*{x}} = \vb*{a}^\top.
\]
% TODO:需要使用度规吗?
}%
\begin{equation}
    H = \int \dd[3]x \mathcal{H} 
    = \int \dd[3]x \eval{\left( \pdv{\mathcal{L}}{\partial_0 \phi} \partial_0 \phi - \mathcal{L} \right)}_{\partial_0 \phi \to \pi} 
    = \int \dd[3]x \eval{\left( \pi \partial_0 \phi - \mathcal{L} \right)}_{\partial_0 \phi \to \pi}.
    \label{eq:lagrangian-to-hamitonian}
\end{equation}
我们通过将$\partial_0 \phi$用$\pi$表示使得任何对$H$的泛函求导都不会将$\partial_0 \phi$当成变量。
看出,$H$对$\phi$泛函求导就是$-\int \dd[3]x \mathcal{L}$对$\phi$泛函求导,于是我们有
\[
    \dv{\Pi}{t} = - \int \dd[3]x \fdv{H}{\phi}.
\]
另一方面由于$H$不显含任何$\pi$的导数,我们有
\[
    \begin{aligned}
        \fdv{H}{\pi} &= \pdv{\pi} \eval{\left( \pi \partial_0 \phi - \mathcal{L} \right)}_{\partial_0 \phi \to \pi} 
        = \partial_0 \phi + \pi \pdv{\partial_0 \phi}{\pi} - \pdv{\pi} \eval{\mathcal{L}}_{\partial_0 \to \pi} \\
        &= \partial_0 \phi + \pi \pdv{\partial_0 \phi}{\pi} - \pdv{\mathcal{L}}{\partial_0} \pdv{\partial_0}{\pi} = \partial_0 \phi.
    \end{aligned}
\]
于是就得到了3+1维场论的哈密顿表述:
\begin{equation}
    \dv{\pi}{t} = - \fdv{H}{\phi}, \quad \dv{\phi}{t} = \fdv{H}{\pi}.
    \label{eq:hamitonian-eq}
\end{equation}
其中$H$仅仅是$\phi, \partial_i \phi$和$\pi$的函数。
方程中的全导数也可以写成偏导数,我们把它写成全导数是因为我们通常只在一个固定的空间点观察场的变化,也就是说在\eqref{eq:hamitonian-eq}中我们只把时间看成变量而将空间坐标看成“标签”(见\autoref{note:spacial-label})。
由于我们讨论的基本上是场论问题,常常使用下面的记号:%
\footnote{在一些上下文中,场的时间全导数常常被定义为某个位置会随时间发生变化的场点处的场的导数,也就是
\[
    \dot{\phi} = \dv{t} \phi(\vb*{x}, t) = \pdv{\phi}{t} + \dot{\vb*{x}} \cdot \grad{\phi} = \partial_0 A = \partial^0 A.
\]
本文不涉及这样的问题,因此不使用这个记号。
}%
\[
    \dot{A} = \dv{A}{t} = \pdv{A(\vb*{x}, t)}{t}.
\]
在同一个场有多个分量的情况下,我们记各场为$\phi^i$,如果还是希望维持形式上的指标升降关系,$\pi$就可以写成$\pi_i$。

总之,使用拉氏量描述的3+1维经典场也能够使用一个哈密顿动力学描述,这个哈密顿动力学的演化参数为$x^0$也就是时间维,而使用空间维作为连续的“正则坐标”的标记。%
\footnote{也就是说,空间坐标$x^1, x^2, x^3$对应离散情况下的场量标签,
如$\phi^1(x, y, z)$指的是以$1, x, y, z$为标签的一个正则坐标,正如离散时的$q^{1}$代表以$1$为标签的一个正则坐标。
注意到这种哈密顿表述并没有以统一的方式对待时间和空间。\label{note:spacial-label}}%
任何物理量都是$\phi$和$\partial_\mu \phi$的函数,因此它们能够写成$\phi, \partial_i \phi$和$\pi$的函数,从而它们的演化都可以使用\eqref{eq:hamitonian-eq}确定,因为
\begin{equation}
    \dv{A}{t} = \pdv{A}{\phi} \dv{\phi}{t} + \pdv{A}{\partial_i \phi} \partial_i \dv{\phi}{t} + \pdv{A}{\pi} \dv{\pi}{t}.
    \label{eq:evolution-of-any-quantity}
\end{equation}

哈密顿动力学(无论是经典哈密顿动力学还是下一节讨论的正则量子化)中如果场是复的,仍然可以使用\eqref{eq:lagrangian-to-hamitonian}从拉氏量得到哈氏量,但此时不能够保证$\phi$、$\phi^\dagger$、$\pi$、$\pi^\dagger$彼此独立。
% TODO

\subsubsection{正则量子化}\label{sec:canonical-quantization}

下面我们转而讨论量子情况下的哈密顿动力学。这种使用哈密顿动力学建立量子理论的方法称为\textbf{正则量子化}。
完整地描述一个量子系统的状态和演化情况需要一个三元组:
首先是一个希尔伯特空间,称为\textbf{态空间},其中的矢量称为\textbf{态矢量},它们表示了系统的状态,
并且我们认为只差了一个倍数的态矢量等价,从而我们可以仅使用单位长度的态矢量描述任何的系统;
其次是一组\textbf{可观察量},它们是希尔伯特空间上的厄米算符,这意味着它们可以被幺正对角化,并且本征值都是实数%
\footnote{后面会提到,如果一个理论在正则量子化时必须选择反对易子的量子化方案,那么实际上它描写的场算符的本征值是格拉斯曼数。
但是如果我们在正则量子化的框架下工作,就从来不关注这种理论对应的场算符的本征值到底是多少,因此没有必要特意讨论它们。
在路径积分量子化的框架中,由于需要讨论费米子的经典场,格拉斯曼数是比较重要的。}%
;
最后是一个\textbf{哈密顿量}或者说\textbf{哈密顿算符},它自身也是一个可观察量(在经典极限下就是经典哈密顿量),且它指示了系统的演化方式。
经典哈密顿理论中同样有对应的三元组,
只不过态空间并不是一个可以做线性叠加的向量空间,从而可观察量也只是从态到实数的映射而不是希尔伯特空间上的算符。
由于所谓的场量$\phi$需要使用算符$\hat{\phi}$代替,因此不再能够良好地定义$\mathcal{H}$对各场量的偏导数,
从而我们也不能良好地定义$\var{H}/\var{\phi}$,等等。
现在动力学方程由
\begin{equation}
    \dv{\hat{A}}{t} = \frac{1}{\ii \hbar} [\hat{A}, \hat{H}]
    \label{eq:quantum-evolution}
\end{equation}
确定。%
\footnote{我们不讨论其定义显含时间的算符,因为它们不会出现在基本的物理规律中。}%
此时有意义的物理量虽然是算符,但在正则量子化之下仍然能够写成场算符$\phi, \partial_i \phi$和$\pi$的函数,
因此一旦$\phi$和$\pi$的演化确定了,\eqref{eq:evolution-of-any-quantity}就以一种和经典情况完全一致的方式确定了所有物理量的演化。
换而言之,\eqref{eq:hamitonian-eq-quantum}完全等价于
\begin{equation}
    \dv{\hat{\phi}}{t} = \frac{1}{\ii \hbar} [\hat{\phi}, \hat{H}], 
    \quad \dv{\hat{\pi}}{t} = \frac{1}{\ii \hbar} [\hat{\pi}, \hat{H}].
    \label{eq:hamitonian-eq-quantum}
\end{equation}

运动方程\eqref{eq:quantum-evolution}意味着,哈密顿量就是与时间平移不变性对应的守恒量。我们称其本征值为\textbf{能量},相应的,其本征态就是\textbf{能量本征态}。

要确定系统的动力学,只需要讨论$[\hat{\phi}, \hat{H}]$和$[\hat{\pi}, \hat{H}]$就可以,
而要讨论这两者又只需要讨论所有有关的场之间的对易关系即可,因为我们总是可以把$H$写成这些场的多项式。
(下文中讨论量子化方案时有对这一点的形象说明)
因此,取对易子为李括号,一个理论中涉及的所有算符就构成了一个李代数,而基本的场之间的对易关系又完全确定了这个李代数的结构。

仅仅有一个抽象的李代数并不能获得完整的理论。
例如,单粒子体系中$\hat{\vb*{x}}$和$\hat{\vb*{p}}$之间的李代数和多粒子体系中每一个粒子的$\hat{\vb*{x}}$和$\hat{\vb*{p}}$之间的李代数具有完全一样的结构,但是显然单粒子体系不是多粒子体系。
例如单粒子体系中$\vb*{x}$的谱没有简并而多粒子体系中$\vb*{x}$的谱有简并。
要获得完整的理论,我们还需要讨论态空间的结构。
我们将不讨论完整的数学,而只是对物理上常用的一些操作做一些说明。

当我们选定一个希尔伯特空间并且将(抽象的李代数中的)算符作用于其上时,实际上是对这个算符做了一个幺正表示。
进一步,当我们说一个系统的希尔伯特空间$H$能够被一组相互对易的算符$S$完全描述时,
我们实际上是说,算符集合$S$在$H$上的(幺正)表示组成了$H$上的一个完备相容算符集合,也即,$S$中各个算符在$H$上的表示共享的本征矢量构成$H$的一组基。
可以证明,如果$S_1,S_2$是$S$的一个划分,且$S_1$完全描述了$H_1$而$S_2$完全描述了$H_2$,那么就有$H$和$H_1 \otimes H_2$同构。
因此我们把$H$完全分解成了若干空间的直积,这些空间中的每一个都由完整描述系统需要的算符中的其中一个完全刻画。

一旦同时知道了各算符的对易关系(从而建立起它们的李代数),以及完整描述系统需要的完备相容算符集合,
我们就可以完整地推导出这个系统每一时刻的状态以及其演化方式了。
实际上,我们真正关注的是完备相容算符集合中各算符的谱结构。

根据上下文,我们可以容易地分辨作为抽象的李代数成员的算符,以及它们在各个希尔伯特空间上的表示,
因此为方便陈述,以下不再对这些略有不同的对象做详细的区分。
对算符而言这样做是合理的,因为从某个表示中得到的代数关系只要不涉及具体的表示的细节,就在抽象的李代数中也成立。
例如如果我们在某个表示中推导出$[\hat{x}, \hat{p}] = \ii \hbar \hat{I}$,那么在抽象的李代数中必定也有这个式子成立,
因为其中只牵扯到算符而没有牵扯到态矢量。
同样,可以比较容易地分辨各个希尔伯特空间中的态矢量,因此在不引起混淆的情况下我们也不刻意区分它们。

现在我们要做的是,分析$\phi$和$\pi$之间要具有什么样的代数关系%
\footnote{实际上只需要分析同一个时间$t$下$\phi(\vb*{x}, t)$和$\pi(\vb*{y}, t)$之间的关系就可以了,
因为\eqref{eq:hamitonian-eq-quantum}中从来不会出现不同时间的量之间的对易子。
这是量子版本的哈密顿动力学不适宜用于分析洛伦兹协变性的一个例子。
另外请注意这套理论并不能原封不动地适用于广义相对论时空,因为那里会需要讨论“不同时间处的量之间的关系”。},
才能够让\eqref{eq:hamitonian-eq-quantum}在$\hbar \to 0$时退化到经典情况\eqref{eq:hamitonian-eq}。
选定这样一个代数关系就称为选取一种\textbf{量子化方案},因为一旦给定了这样的代数关系,我们就把\eqref{eq:hamitonian-eq}推广到了量子理论中。
我们将不试图穷举所有可以的量子化方案,而只是举两个行之有效的例子——也就是说,实验数据要求使用这样的量子化方案。%
\footnote{我们说“穷举所有情况”意味着,面对同一个经典哈密顿量密度$\mathcal{H}$,
有不止一种指定$\hat{\phi}$和$\hat{\pi}$的方式,
使得我们能够得到一个量子动力学\eqref{eq:hamitonian-eq-quantum},
并且在$\hbar \to 0$的极限情况下回退到经典动力学\eqref{eq:hamitonian-eq}。
这是可以预期的,因为“取$\hbar\to 0$的极限”这个操作显然不是一一对应的,或者说量子化方案可以不止一种。
}

第一个方案是指定对易子为
\begin{equation}
    [\hat{\phi}^i(\vb*{x}, t), \hat{\pi}_j(\vb*{y}, t)] = \ii \hbar \delta^i_j \delta^3(\vb*{x} - \vb*{y}), 
    \quad [\hat{\phi}^i(\vb*{x}, t), \hat{\phi}^j(\vb*{y}, t)] = [\hat{\pi}_i(\vb*{x}, t), \hat{\pi}_j(\vb*{y}, t)] = 0.
    \label{eq:symmetry-commutator}
\end{equation}
从中可以容易地看出
\begin{equation}
    [\partial_\mu \hat{\phi}^i(\vb*{x}, t), \hat{\phi}^j(\vb*{y}, t)] = 0, 
    \quad [\grad{\hat{\phi}^i(\vb*{x}, t)}, \hat{\pi}_j(\vb*{y}, t)] = \ii \hbar \delta^i_j \grad{\delta^3(\vb*{x} - \vb*{y})}.
    \label{eq:symmetry-partial-mu-commutator}
\end{equation}
只需要将导数看成非常接近的两个量之差,然后利用对易子的线性性即可导出上式。

现在我们来推导$\hbar\to 0$时\eqref{eq:hamitonian-eq-quantum}的极限。
为方便起见,推导过程中假定$\hat{\phi}$是标量场。这并不会有损一般性,因为推导过程中没有用到任何关于坐标变换导致场变化的知识,从而我们可以把同一个多分量场的不同分量看成不同的标量场。
使用\eqref{eq:symmetry-commutator},已知$\hat{\phi} \hat{\pi}$就能够写出$\hat{\pi} \hat{\phi}$。
将$\hat{\mathcal{H}}$写成关于$\hat{\phi}, \partial_i \hat{\phi}$和$\pi$的多项式,
我们可以使用\eqref{eq:symmetry-commutator}和\eqref{eq:symmetry-partial-mu-commutator}
将形如$\hat{\pi}\hat{\phi}$、$\hat{\pi}\partial_i \hat{\phi}$这样的式子改写为形如$\hat{\phi}\hat{\pi}$、$\hat{\partial_i \hat{\phi}}\hat{\pi}$这样的式子。这称为取\textbf{正规积序}:我们总是可以使用对易关系把一个算符多项式转化为一个与之恒等的多项式,后者中的每一项中的算符排列顺序都满足一定的要求。
因此不失一般性地,我们认为$\hat{\mathcal{H}}$的多项式表达式中的每一项都形如$\hat{\phi}^l (\partial_i \hat{\phi})^m \pi^n$。
我们有(因为时间$t$都一样,以下略去$t$变量)%
\footnote{注意下面的$\delta(\vb*{x} - \vb*{x}')$是系数,因此可以自由地移动;写在它们左边的算符不会作用在它们上面!}
\[
    \begin{aligned}
        &\quad\comm{\hat{\phi}(\vb*{x}')}{\int \dd[3]{x} \hat{\phi}^l(\vb*{x}) (\partial_i \hat{\phi})^m (\vb*{x}) \pi^n(\vb*{x})} \\
        &= \int \dd[3]{x} \comm{\hat{\phi}(\vb*{x}')}{\hat{\phi}^l (\vb*{x}) (\partial_i \hat{\phi})^m (\vb*{x}) \pi^n (\vb*{x})} \\
        &= \int \dd[3]{x} \left( \hat{\phi}^l (\vb*{x}) \comm{\hat{\phi}(\vb*{x}')}{(\partial_i \hat{\phi})^m (\vb*{x}) \pi^n (\vb*{x})} + [\hat{\phi}(\vb*{x}'), \hat{\phi}^l (\vb*{x})] \partial_i \hat{\phi}^m (\vb*{x}) \pi^n(\vb*{x}) \right) \\
        &= \int \dd[3]{x} \hat{\phi}^l (\vb*{x}) \left( (\partial_i \hat{\phi})^m (\vb*{x}) \comm{\hat{\phi} (\vb*{x}')}{\hat{\pi}^n (\vb*{x})} + \comm{\hat{\phi} (\vb*{x}')}{(\partial_i \hat{\phi})^m (\vb*{x})} \pi^n (\vb*{x}) \right) \\
        &= \int \dd[3]{x} \hat{\phi}^l (\vb*{x}) (\partial_i \hat{\phi})^m (\vb*{x}) \comm{\hat{\phi} (\vb*{x}')}{\hat{\pi}^n (\vb*{x})} \\
        &= \int \dd[3]{x} \hat{\phi}^l (\vb*{x}) (\partial_i \hat{\phi})^m (\vb*{x}) \ii \hbar \delta^3(\vb*{x'} - \vb*{x}) n \hat{\pi}^{n-1}(\vb*{x}) \\
        &= \ii \hbar n \hat{\phi}^l (\vb*{x}') (\partial_i \hat{\phi})^m (\vb*{x}') \hat{\pi}^{n-1}(\vb*{x}'), 
    \end{aligned}
\]
这正是求导公式。当$\hbar$接近零的时候$\phi$和$\pi$可以交换,于是$\mathcal{H}$可以写成普通的、字母顺序无关紧要的函数,此时
我们有
\[
    \begin{aligned}
        \dv{\hat{\phi}}{t} = \frac{1}{\ii \hbar} [\hat{\phi}(\vb*{x}), \hat{H}] &= \frac{1}{\ii \hbar} \sum_\text{terms} \comm{\hat{\phi}(\vb*{x}')}{\int \dd[3]{x} \hat{\phi}^l(\vb*{x}) (\partial_i \hat{\phi})^m (\vb*{x}) \pi^n(\vb*{x})} \\
        &= \frac{1}{\ii \hbar} \sum_\text{terms} \ii \hbar n \hat{\phi}^l (\vb*{x}') (\partial_i \hat{\phi})^m (\vb*{x}') \hat{\pi}^{n-1}(\vb*{x}') \\
        &= \pdv{\hat{\mathcal{H}}}{\hat{\pi}} = \fdv{\hat{H}}{\hat{\pi}},
    \end{aligned}
\]
这就意味着$\hbar\to 0$时关于$\hat{\phi}$的方程能够回退到经典版本。
同样也有
\[
    \begin{aligned}
        &\quad \comm{\hat{\pi}(\vb*{x}')}{\int \dd[3]{x} \hat{\phi}^l(\vb*{x}) (\partial_i \hat{\phi})^m (\vb*{x}) \pi^n(\vb*{x})} \\ 
        &= \int \dd[3]{x} \comm{\hat{\pi}(\vb*{x}')}{\hat{\phi}^l(\vb*{x}) (\partial_i \hat{\phi})^m (\vb*{x}) \pi^n(\vb*{x})} \\
        &= \int \dd[3]{x} \left( \hat{\phi}^l (\vb*{x}) (\partial_i \hat{\phi} )^m (\vb*{x}) \comm{\hat{\pi}(\vb*{x}')}{\hat{\pi}^n(\vb*{x})} + \hat{\phi}^l (\vb*{x}) \comm{\hat{\pi}(\vb*{x}')}{(\partial_i \hat{\phi})^m (\vb*{x})} \hat{\pi}^n (\vb*{x}) + \comm{\hat{\pi}(\vb*{x}')}{\hat{\phi}^l (\vb*{x})} (\partial_i \hat{\phi})^m (\vb*{x}) \hat{\pi}^n (\vb*{x}) \right) \\
        &= \int \dd[3]{x} \left( - \hat{\phi}^l (\vb*{x}) \ii \hbar \grad{\delta^3 (\vb*{x} - \vb*{x}')} m (\partial_i \hat{\phi})^{m-1} (\vb*{x}) \hat{\pi}^n (\vb*{x}) - \ii \hbar \delta^3(\vb*{x} - \vb*{x}') l \hat{\phi}^{l-1} (\vb*{x}) (\partial_i \hat{\phi})^m (\vb*{x}) \hat{\pi}^n (\vb*{x}) \right) \\
        &= \ii \hbar \partial_i \left(m (\partial_i \hat{\phi})^{m-1} (\vb*{x}') \hat{\pi}^n (\vb*{x}')\right) - \ii \hbar l \hat{\phi}^{l-1} (\vb*{x}') (\partial_i \hat{\phi})^m (\vb*{x}') \hat{\pi}^n (\vb*{x}').
    \end{aligned}
\]
当$\hbar \to 0$时
\[
    \begin{aligned}
        \dv{\hat{\pi}}{t} &= \frac{1}{\ii \hbar} [\hat{\pi}(\vb*{x}), H] \\
        &= \frac{1}{\ii \hbar} \sum_\text{terms} \left(\ii \hbar \partial_i \left(m (\partial_i \hat{\phi})^{m-1} (\vb*{x}') \hat{\pi}^n (\vb*{x}')\right) - \ii \hbar l \hat{\phi}^{l-1} (\vb*{x}') (\partial_i \hat{\phi})^m (\vb*{x}') \hat{\pi}^n (\vb*{x}')\right) \\
        &= - \sum_\text{terms} \left(l \hat{\phi}^{l-1} (\vb*{x}') (\partial_i \hat{\phi})^m (\vb*{x}') \hat{\pi}^n (\vb*{x}') - \partial_i \left(m (\partial_i \hat{\phi})^{m-1} (\vb*{x}') \hat{\pi}^n (\vb*{x}')\right) \right) \\
        &= - \left( \pdv{\hat{\mathcal{H}}}{\hat{\phi}} - \partial_i \pdv{\hat{\mathcal{H}}}{\partial_i \hat{\phi}} \right) = - \fdv{\hat{H}}{\hat{\phi}},
    \end{aligned}
\]
因此关于$\pi$的方程也回退到经典情况。
这表明\eqref{eq:symmetry-commutator}是一个可行的量子化方案。

第二个方案是指定反对易子——而不是对易子——为%
\footnote{彼此无关的场,无论它们自己服从\eqref{eq:symmetry-commutator}还是\eqref{eq:antisymmetry-commutator},相互之间总是对易的。在\eqref{eq:symmetry-commutator}中这是显然的,因为可以将无关的场看成某个多分量场的分量,然后因为它们是不同的分量,它们自然对易。但在\eqref{eq:antisymmetry-commutator}方案下需要额外增加一个规定:
\[
    \comm*{\hat{\phi}(\vb*{x}, t)}{\hat{\psi}(\vb*{y}, t)} = 0.
\]
}
\begin{equation}
    \{\hat{\phi}(\vb*{x}, t), \hat{\pi}(\vb*{y}, t)\} = \ii \hbar \delta^3(\vb*{x} - \vb*{y}), \quad \{\hat{\phi}(\vb*{x}, t), \hat{\phi}(\vb*{y}, t)\} = \{\hat{\pi}(\vb*{x}, t), \hat{\pi}(\vb*{y}, t)\} = 0.
    \label{eq:antisymmetry-commutator}
\end{equation}
同样,我们可以将哈密顿量写成若干个$\hat{\phi}^l (\partial_i \hat{\phi})^m \pi^n$形式的项的和。
需要注意的是\eqref{eq:antisymmetry-commutator}直接导出
\[
    \hat{\phi}(\vb*{x})^2 = 0, \quad \hat{\pi}(\vb*{x})^2 = 0, \quad (\partial_i \hat{\phi})^2(\vb*{x}) = 0,
\]
因此哈密顿量中$l, m, n \leq 1$。
这意味着这个量子化方案并不适用于所有的场,而是只适用于能够保证在任何情况下哈密顿量中的每一项都满足$l, m, n \leq 1$的场。
对于正常的实数/复数值场,这是一个不可能的事情。
事实上,设$\hat{\phi}$在$\vb*{x}$处的值为$\int \dd \phi(\vb*{x}) \dyad{\phi}$,
在$\vb*{y}$处的值为$\int \dd \phi(\vb*{y}) \dyad{\phi}$,则通过反对易关系能够得到
\[
    \phi(\vb*{x}) \phi(\vb*{y}) = - \phi(\vb*{y}) \phi(\vb*{x}).
\]
因此,反对易子意味着对应的场算符的本征值——也就是其经典极限——实际上并不是实数,甚至也不是复数,而是格拉斯曼数。
在复数域中满足反对易关系的场算符不能被对角化。
在路径积分量子化中,格拉斯曼数非常重要,因为路径积分量子化会分析经典场值的演化路径。
在正则量子化中只需要把这些格拉斯曼数看成算符(准确地说,是产生算符)就可以了——我们并不会用到它的微积分,因此也无需将它们看成数。

为了看出反对易方案的不同寻常,我们指出如下事实:一个通过\eqref{eq:antisymmetry-commutator}量子化的场不可能是厄米的。
我们有
\[
    \hat{\phi}(\vb*{x}, t) \hat{\pi} (\vb*{y}, t) + \hat{\pi} (\vb*{y}, t) \hat{\phi} (\vb*{x}, t) = \ii \hbar \delta(\vb*{x} - \vb*{y}),
\]
于是
\[
    \left(\hat{\phi}(\vb*{x}, t) \hat{\pi} (\vb*{y}, t) + \hat{\pi} (\vb*{y}, t) \hat{\phi} (\vb*{x}, t)\right)^\dagger = - \ii \hbar \delta(\vb*{x} - \vb*{y}),
\]
如果场是厄米的,那么就有
\[
    \hat{\pi} (\vb*{y}, t) \hat{\phi} (\vb*{x}, t) + \hat{\phi}(\vb*{x}, t) \hat{\pi} (\vb*{y}, t) = - \ii \hbar \delta(\vb*{x} - \vb*{y}).
\]
于是我们得到了一个矛盾。因此,使用\eqref{eq:antisymmetry-commutator}量子化的场应该分解成非零的厄米和反厄米部分,即
\begin{equation}
    \hat{\phi} = \hat{\phi}_1 + \ii \hat{\phi}_2,
\end{equation}
其中$\hat{\phi}_1$和$\hat{\phi}_2$分别是两个厄米算符。
对应的,描述它的拉氏量当中的场有实部和虚部,需要把它们——或者它们的线性组合——看成两个独立的场来列写\eqref{eq:el-eq}。

此外,$\mathcal{H}$中各项阶数的限制还意味着由此导出的运动方程在时间上只能是一阶的。
从而,$\pi$和$\phi$不是彼此独立的。这样,在哈密顿量只关于$\phi$和$\pi$时,我们总是可以适当地调节拉氏量和哈密顿量,或者对$\phi$和$\pi$做一些线性变换,使得$\pi$和$\phi$之间有线性关系。
这个关系显然不能是“乘以某个倍数”,否则将不能够区分这两个变量。
因此两者之间的关系涉及复共轭。通常取
\begin{equation}
    \pi = \ii \phi^\dagger.
\end{equation}
这也表明了取$\phi$为复场的重要性——否则将不能够区分$\phi$和$\pi$,从而难以建立哈密顿动力学。
% TODO:这段还是有问题

需要注意的是,无论是\eqref{eq:symmetry-commutator}还是\eqref{eq:antisymmetry-commutator},实际上都假定了$\phi^i$和$\pi_i$在时空变换下是协变的。
在场具有某些附加结构——例如,有某些外加约束以消除非物理的自由度——的时候,如果我们直接把独立的自由度拿出来写成$\phi^i$,就不能保证它们的协变性(虽然把原来的场恢复出来之后它仍是协变的),此时不能直接套用\eqref{eq:symmetry-commutator}或\eqref{eq:antisymmetry-commutator},而需要使用带约束的场论的有关知识。

此外,虽然本节通过表明指定对易子或者反对易子能够得到经典哈密顿动力学来论证量子化方案\eqref{eq:symmetry-commutator}和\eqref{eq:antisymmetry-commutator}的合理性,但是实际上这两个方案在本文展示的经典哈密顿动力学以外仍然适用。例如,如果哈氏量中出现了广义动量的导数,那么\eqref{eq:hamitonian-eq}需要做出修正,但是\eqref{eq:hamitonian-eq-quantum}仍然适用。换而言之,本节展示的量子动力学实际上才是最根本的理论。
% TODO:真的吗?

从正则量子化得到的算符运动方程就是经典的场运动方程算符化的结果,而后者又等价于通过最小作用量原理求出的运动方程。
这就产生了一个问题:路径积分量子化告诉我们,最小作用量原理只是路径积分的最速下降近似而已,
为什么在正则量子化中精确的运动方程却可以从最小作用量原理求出?
其原因在于,算符在演化过程中不同的本征态会混在一起(一个经典情况下不可能出现的现象),正是这一点构成了量子和经典的区别,
正则量子化中的本征态混合正好对应于路径积分量子化中非经典的路径。

\subsubsection{时间演化和绘景}\label{sec:time-evolution}
% TODO: 表征了相同的物理状态的态矢量之间只差了一个复数常数。好量子数就是守恒量(这是一个算符!)的值,它可以用来标记态。
在\autoref{sec:canonical-quantization}中我们仅仅将态矢量当成一个可以让场算符作用上去的对象。
但实际上如果我们想要的话,也可以让态矢量动起来而对算符做对应的修改,使得算符的谱结构始终不变(本征矢量重数一一对应、彼此对应的本征矢的内积相同),并且本征值不变。
只要算符的谱结构不变、对应的各个本征值不变,算符就正确地描述了系统。
两个算符的谱结构一致、对应的本征值相同的充要条件是它们酉相似(相似矩阵可以随时间变化)。
需要注意的是两个描述了同一个系统的算符会给出不同的基矢量,所以切换绘景的时候还需要改变态矢量。
综上,绘景变换公式为
\begin{equation}
    \hat{A}' = \hat{Q} \hat{A} \hat{Q}^\dagger, \quad \ket{\psi'} = \hat{Q} \ket{\psi},
    \label{eq:picture-trans}
\end{equation}
其中$\hat{Q}$为一个幺正算符,它可以显含时间。
对易子在绘景变换之下会发生下面的改变:
\begin{equation}
    \comm{\hat{A}}{\hat{B}} \longrightarrow \hat{Q} \comm{A}{B} \hat{Q}^\dagger = \comm{\hat{A}'}{\hat{B}'}.
\end{equation}

在\autoref{sec:canonical-quantization}中我们已经讨论了态矢量固定不动时怎么确定系统的动力学。
这种让态矢量固定、算符变动的方案称为\textbf{海森堡绘景}。以下我们使用上标$H$代表海森堡绘景下的量。
我们要证明的第一件事是,不同时间点上的同一个可观察量的值彼此酉相似。
要看清楚这是为什么,我们将酉相似的方程
\begin{equation}
    \hat{A}^H (t) = \hat{U}^H(t, t_0) \hat{A}^H (t_0) (\hat{U}^H)^\dagger(t, t_0)
    \label{eq:quantum-evolution-hes-u-operator}
\end{equation}
做一个等价变换,看看它等价于什么。%
\eqref{eq:quantum-evolution-hes-u-operator}中的$U$在$t=t_0$时必定为恒等变换,因为此时$\hat{A}^H (t) = \hat{A}^H (t_0)$;同时容易看出$\hat{U}^H(\tau)$实际上构成一个李群。这样我们就能够写出其生成元,记之为$\hat{G}(t)$:
\[
    \hat{U}^H(t+\dd{t}, t) = \hat{I} + \frac{\ii}{\hbar} \hat{G}(t) \dd{t}.
\]
$\hat{U}^H$是幺正的等价于$\hat{G}$是厄米的。
于是就能够写出\eqref{eq:quantum-evolution-hes-u-operator}的无穷小等价形式:
\[
    \hat{A}^H (t_0) + \dd{\hat{A}^H}(t_0) = \left( \hat{I} + \frac{\ii}{\hbar} \hat{G}(t) \dd{t} \right) \hat{A}^H (t_0) \left( \hat{I} - \frac{\ii}{\hbar} \hat{G}(t) \right) = \hat{A}^H (t_0) + \frac{\dd{t}}{\ii \hbar} \comm{\hat{A}^H}{\hat{G}(t)}.
\]
我们发现这就是\eqref{eq:quantum-evolution},只需要把$\hat{G}(t)$换成$\hat{H}(t)$;并且正则量子化的时候已经要求$\hat{H}$是厄米的了,因此$\hat{G}$的确是厄米的,从而$\hat{U}^H$是幺正的。
于是我们得出结论:海森堡绘景中的算符演化实际上是在做幺正变换,或者等价地说,海森堡绘景中各算符的本征态在做幺正变换。算符的变换式为\eqref{eq:quantum-evolution-hes-u-operator},相应的,本征态的变换式为
\begin{equation}
    \ket{a(t)} = \hat{U}^H(t, t_0) \ket{a(t_0)}.
\end{equation}
于是我们称$\hat{U}^H$为海森堡绘景下的时间演化算符。
$\hat{U}^H$可以写出显式表达式
\begin{equation}
    \hat{U}^H(t, t_0) = T \exp \left( \frac{\ii}{\hbar} \int_{t_0}^t \dd{t} \hat{H}^H (t) \right).
\end{equation}
注意\eqref{eq:quantum-evolution-hes-u-operator}保证了,一个可观察量在经过时间演化之后仍然是可观察量。

现在我们尝试使用\eqref{eq:picture-trans}来把时间演化完全转移到态矢量上面。
因此,我们希望在新的绘景中,$\hat{A}$始终不变。我们称这新的绘景为\textbf{薛定谔绘景}。
按照\eqref{eq:quantum-evolution-hes-u-operator},有
\[
    \hat{A}^H(t) = \hat{U}^H(t, t_0) \hat{A}^H (t_0) (\hat{U}^H)^\dagger(t, t_0) = \hat{U}^H(t, t_0) \hat{A}^S( \hat{U}^H)^\dagger(t, t_0),
\]
不失一般性地我们取$t=0$时的$\hat{A}^H$为$\hat{A}^S$,那么我们有
\[
    \hat{A}^H (t) = \hat{U}^H(t, 0) \hat{A}^S( \hat{U}^H)^\dagger(t, 0).
\]
将这个方程和\eqref{eq:picture-trans}对比可以看出
\[
    \hat{Q} = (\hat{U}^H)^\dagger(t, 0),
\]
于是得到薛定谔绘景下的态矢量演化公式
\[
    \ket{\psi^S(t)} = \hat{Q} \ket{\psi^H} = (\hat{U}^H)^\dagger (t, 0) \ket{\text{a constant}},
\]
考虑到$t=0$时$\hat{U}^H (t, 0)$就是恒等算符,上式又等价于
\[
    \ket{\psi^S(t)} = (\hat{U}^H)^\dagger (t, 0) \ket{\psi^S (0)},
\]
也即,薛定谔绘景下的时间演化算符和海森堡绘景下的时间演化算符互为逆。
这个方程还告诉我们,
\[
    \ket{\psi^H} = \ket{\psi^S(t_0)}.
\]
现在推导时间演化方程的微分形式。我们有
\[
    \begin{aligned}
        \ket{\psi^S (t + \dd{t})} &= \left( \hat{U}^H (t + \dd{t}, t) \hat{U}^H (t, 0)  \right)^\dagger \ket{\psi^S(0)} \\
        &= \left( (\hat{I} + \frac{\ii}{\hbar} \hat{H}(t) \dd{t})   \hat{U}^H (t, 0) \right)^\dagger \ket{\psi^S (0)} \\
        &= (\hat{U}^H)^\dagger (t, 0) \ket{\psi^S (0)} + \frac{\dd{t}}{\ii \hbar} (\hat{U}^H)^\dagger (t, 0) \hat{H}(t) \ket{\psi^S (0)} \\
        &= \ket{\psi^S (t)} + \frac{\dd{t}}{\ii \hbar} (\hat{U}^H)^\dagger (t, 0) \hat{H}(t) \hat{U}^H (t, 0) \ket{\psi^S (t)},
    \end{aligned}
\]
从而
\[
    \ii \hbar \dv{t} \ket{\psi^S (t)} = (\hat{U}^H)^\dagger (t, 0) \hat{H}(t) \hat{U}^H (t, 0) \ket{\psi^S (t)}.
\]
为了方便区分,我们将海森堡绘景中的$\hat{H}$记作$\hat{H}^H$,则它对应的薛定谔绘景中的算符为
\[
    \hat{H}^S = \hat{Q} \hat{H}^H \hat{Q}^\dagger = (\hat{U}^H)^\dagger (t, 0) \hat{H}^H(t) \hat{U}^H (t, 0), 
\]
这正是薛定谔绘景中态矢量的运动方程中出现的那个量,因此就获得了薛定谔绘景中的运动方程:
\begin{equation}
    \ii \hbar \dv{t} \ket{\psi^S(t)} = \hat{H}^S (t) \ket{\psi^S(t)}.
\end{equation}
% TODO:证明$\hat{H}^I_i$确实是$\hat{H}_i^H$在相互作用绘景下的
设$\hat{U}^S(t, t_0)$是薛定谔绘景下的时间演化算符,则容易证明$\hat{H}^S$是它的生成元,既然$\hat{H}^H$是厄米的,$\hat{H}^S$也是厄米的,从而$\hat{U}^S$是幺正的。%
\footnote{注意$\hat{H}^H$是$\hat{U}^H$的生成元而$\hat{H}^S$是$(\hat{U}^H)^\dagger$的生成元;由于$\hat{H}^H$可能含时,一般情况下
\[
    T \exp(\int \hat{H}^H (t) \dd{t})^\dagger \neq T \exp(- \int \hat{H}^H (t) \dd{t}),
\]
也就是说$\hat{H}^H$和$\hat{H}^S$之间没有简单的关系,而必须使用绘景变换公式联系两者。
}
因此薛定谔绘景中时间演化始终保持态矢量的幺正性。
时间演化算符的显式表达式为
\begin{equation}
    \hat{U}^S(t, t_0) = T \exp \left( - \frac{\ii}{\hbar} \int_{t_0}^t \dd{t} \hat{H}^S(t) \right),
\end{equation}
其中$T$为编时算符。

为了明显起见,我们将薛定谔绘景和海森堡绘景中哈密顿量相互换算的关系重复如下:
\begin{equation}
    \begin{aligned}
        \hat{H}^H(t) = T \exp \left( \frac{\ii}{\hbar} \int_{t_0}^t \dd{t} \hat{H}^H(t) \right) \hat{H}^S(t_0) \left(T \exp \left( \frac{\ii}{\hbar} \int_{t_0}^t \dd{t} \hat{H}^H(t) \right)\right)^\dagger, \\
        \hat{H}^S(t_0) = T \exp \left( - \frac{\ii}{\hbar} \int_{t_0}^t \dd{t} \hat{H}^S \right) \hat{H}^H(t) \left( T \exp \left( - \frac{\ii}{\hbar} \int_{t_0}^t \dd{t} \hat{H}^S \right)\right)^\dagger,
    \end{aligned}
\end{equation}
在$\hat{H}^H$在各个时间点的值彼此对易时,$\hat{U}^H$无非是$\hat{H}^H$的级数,因此它们对易,从而$\hat{H}^S$和$\hat{H}^H$相等。
这也等价于$\hat{H}^S$在各个时间点的值彼此对易。

事实上,虽然我们是从海森堡绘景出发建立我们的理论框架的,但\autoref{sec:back-to-classical}告诉我们,和经典力学中的系统状态直接对应的实际上就是态矢量,而不是算符,因此很多文献是从薛定谔绘景出发建立理论的。

% TODO:这里有些地方写得是有问题的。标准的相互作用绘景应该是薛定谔绘景的推论。但这个也奇怪得很:量子场论中的微扰论难道是使用薛定谔绘景的吗??
现在我们已经讨论了“让可观察量变动”和让基矢量变动“两种方案的不同了。我们还可以把哈密顿算符分解成一个比较简单的不含时部分和一个含时的部分,并要求这两者均为厄米算符,然后分别用两者让算符和态矢量都动起来。这样的方案称为\textbf{相互作用绘景}。
为方便起见,考虑从薛定谔绘景到相互作用绘景的变换。当然也可以从海森堡绘景出发推导相互作用绘景,但实际上这样会很不自然。对薛定谔绘景下的哈密顿量做分解
\begin{equation}
    \hat{H}^S = \hat{H}_0^S + \hat{H}_i^S,
\end{equation}
称前者为\textbf{自由哈密顿量}(通常我们要求它不显含时间),后者为\textbf{相互作用哈密顿量},并指定
\begin{equation}
    \ket{\psi^I(t)} = \hat{U}_0^\dagger(t,t_0) \ket{\psi^S(t)},
\end{equation}
其中
\begin{equation}
    \hat{U}_0 = T \exp \left( - \frac{\ii}{\hbar} \int_{t_0}^t \dd{t} \hat{H}_0^S(t) \right).
\end{equation}
于是可观察量的绘景变换为
\begin{equation}
    \hat{A}^I(t) = \hat{U}_0^\dagger(t,t_0) \hat{A}^S(t) \hat{U}_0(t,t_0).
    \label{eq:operator-from-schodinger-to-interaction}
\end{equation}
通过求导,分别可以计算出态矢量和可观察量的时间演化方程为
\begin{equation}
    \ii \hbar \dv{t} \ket{\psi^I(t)} = \hat{H}^I_i(t) \ket{\psi^I(t)},
    \label{eq:time-evolution-in-interation-picture}
\end{equation}
以及
\begin{equation}
    \dv{t} \hat{A}^I(t) = \frac{1}{\ii \hbar} \comm*{\hat{A}^I(t)}{\hat{H}_0^I}.
\end{equation}
其中$\hat{H}_0^I$和$\hat{H}_i^I$正是对$\hat{H}_0^S$和$\hat{H}_i^S$做绘景变换\eqref{eq:operator-from-schodinger-to-interaction}得到的结果。
这样我们就成功地让时间演化分别由态矢量和可观察量各自承担一部分。

如果我们在海森堡绘景中工作,要怎么样切换到相互作用绘景中呢?最一般的公式非常复杂。
但是,实际上,如果哈密顿量含时,通常直接在薛定谔绘景中工作;如果哈密顿量不含时,那么薛定谔绘景和海森堡绘景下的哈密顿量是一样的,那么只需要选择一个较简单的可观察量$\hat{H}_0$,指定它为$\hat{H}_0^S$,就可以切换到相互作用绘景。需注意整个过程并没有用到$\hat{H}_0^H$,一般来说,它和$\hat{H}_0^S$可能会有区别,但是我们从来不关注这个区别。

相互作用绘景在微扰量子场论计算中起到了非常重要的作用,因为通过对称性分析可以直接得到自由场的哈密顿量密度和演化方程,因此我们可以将相互作用项——也就是不同场之间的耦合——独立考虑,从而大大简化计算。
更加重要的是,此时相互作用绘景可以为我们提供有关量子场的态空间的结构的信息。如果假定态空间中有一个唯一的真空态——也就是所有场都是零的态——那么量子场的态空间就是多粒子态福克空间,在此基础上我们可以很自然地处理粒子创生和湮灭的过程。这就为我们展示了量子理论的另一面:波动看起来就像粒子一样。%
\footnote{需要注意的是,在处理相对论性量子场论的时候其实并不能完全放心地使用相互作用绘景。如果我们取$\hat{H}_i=0$,那么相互作用绘景就退化为了自由场的海森堡绘景;这样我们就看到了$\hat{H}_i$项的作用:它把带相互作用的场的态(也就是$\ket{\psi^I(t)}$)和自由场的态($\ket{\psi^I(0)}$,因为如果$\hat{H}_i=0$那么态就不会变化)使用一个幺正算符联系了起来,而且这个幺正算符是唯一的。然而Haag定理说,含相互作用的场有无数个不等价的幺正表示,因此我们并不能唯一地将带相互作用的场的态和自由场的态使用一个唯一的幺正算符联系起来。特别的,由于我们要求自由场和相互作用场的态空间都满足一定的物理条件(如有稳定的真空态,等等),自由场的态空间和相互作用场的满足这些条件的态空间一般来说并不幺正等价。这意味着类似于$\int \dd{t} \hat{H}^H_i$之类的表达式实际上并不收敛,于是相互作用绘景就失效了。但是有很多手段可以绕过这个定理的限制——例如因为我们从来只讨论一定能标下的物理现象而不把相对论性量子场论当成终极理论,实际上我们可以把空间格点化,这样量子场论就变成了有限自由度的量子力学,于是就可以使用相互作用绘景了。}

此外容易验证,各种形式的时间演化算符都满足以下公式:
\begin{equation}
    \hat{U}(t_3,t_2) \hat{U}(t_2,t_1) = \hat{U}(t_3,t_1),
\end{equation}
以及
\begin{equation}
    \hat{U}^\dagger (t_2, t_1) = \hat{U} (t_1, t_2).
\end{equation}

\subsubsection{测量}\label{sec:measure}

\textbf{测量}指的是这样一个过程:两个系统(分别称为\textbf{待测系统}和\textbf{仪器})发生相对剧烈而时间短促的相互作用,相互作用后待测系统的态发生很大改变,而仪器的态则体现了相互作用前待测系统的某些信息。
采用相互作用绘景,设$\hat{q}$完全描述了仪器的态空间,$\hat{a}$是关于待测系统的某个算符,它和另一个算符$\hat{b}$共同描述了待测系统的态空间。(被测量的量$\hat{a}$未必能够完整描述待测系统。下文中需要将待测系统的态做展开,因此引入$\hat{b}$)
由于相互作用非常剧烈而时间短促,仪器和待测系统的相互作用哈密顿量可以写成
\begin{equation}
    H_\text{int} = - \gamma(t-t_0) \hat{a} \otimes \hat{p},
\end{equation}
% 为什么偏偏就是这个形式?为什么所有量都是一次项?
其中$\hat{p}$是$\hat{q}$对应的共轭动量,也就是说
\[
    \comm*{\hat{q}}{\hat{p}} = \ii \hbar,
\]
$\gamma$是一个函数,它是一个$t_0$附近的尖峰。
极限情况下,$\gamma(t) = g \delta(t)$,这称为\textbf{冯诺依曼测量}或者\textbf{标准量子测量},
我们在相互作用绘景下分析问题。系统初态为
\[
    \ket{i} = \ket{\psi_i} \ket{D} = \int \dd{q} \sum_{k, n} \braket{q}{D} \braket{a_k, b_n}{\psi_i} \ket{a_k} \ket{b_n} \ket{q},
\]
其中$\ket{\psi_i}$和$\ket{D}$分别为待测系统和仪器的初态,本征态$\ket{a}_k$,$\ket{b}_n$和$\ket{q}$是$t_0$时刻对应算符的本征态(下同)。%
\footnote{提醒:算符本征态反映的是算符的代数结构,它们的时间演化是由自由哈密顿量而不是相互作用哈密顿量指导的。}%
我们要求$\hat{q}$是连续谱,而$\hat{a}$和$\hat{b}$可以是离散谱也可以是连续谱。要求$\hat{q}$是连续谱的原因很快就可以看到。
系统的末态为
\[
    \begin{aligned}
        \ket{f} &= T \exp \left( - \frac{\ii}{\hbar} \int \dd{t} H_\text{int} \right) \ket{i} \\
        &= T \exp \left( \frac{\ii}{\hbar} g \int \dd{t} \delta(t-t_0) \hat{a}(t) \otimes \hat{p}(t) \right) \ket{i} \\
        &= \exp \left( \frac{\ii}{\hbar} g \hat{a}(t_0) \otimes \hat{p}(t_0) \right) \ket{i} \\
        &= \sum_{n=0}^\infty \frac{1}{n!} \left(\frac{\ii}{\hbar} g\right)^n \hat{a}(t_0)^n \hat{p}(t_0)^n \int \dd{q} \sum_{k, l} \braket{q}{D} \braket{a_k, b_l}{\psi_i} \ket{a_k} \ket{b_l} \ket{q} \\
        &= \int \dd{q} \sum_{k, l} \braket{q}{D} \braket{a_k, b_l}{\psi_i} \sum_{n=0}^\infty \frac{1}{n!} \left(\frac{\ii}{\hbar} g\right)^n \hat{a}(t_0)^n \ket{a_k} \ket{b_l} \hat{p}(t_0)^n \ket{q} \\
        &= \int \dd{q} \sum_{k, l} \braket{q}{D} \braket{a_k, b_l}{\psi_i} \sum_{n=0}^\infty \frac{1}{n!} \left(\frac{\ii}{\hbar} g\right)^n a_k^n \ket{a_k} \ket{b_l} \hat{p}(t_0)^n \ket{q} \\
        &= \int \dd{q} \sum_{k, l} \braket{q}{D} \braket{a_k, b_l}{\psi_i} \ket{a_k} \ket{b_l} \sum_{n=0}^\infty \frac{1}{n!} \left(\frac{\ii}{\hbar} g a_k \hat{p}(t_0) \right)^n \ket{q} \\
        &= \int \dd{q} \sum_{k, l} \braket{q}{D} \braket{a_k, b_l}{\psi_i} \ket{a_k} \ket{b_l} \exp \left( \frac{\ii}{\hbar} g a_k \hat{p}(t_0) \right) \ket{q} \\
        &= \int \dd{q} \sum_{k, l} \braket{q}{D} \braket{a_k, b_l}{\psi_i} \ket{a_k} \ket{b_l} \ket{q + g a_k} \\
        &= \int \dd{q} \sum_{k, l} \braket{q - g a_k}{D} \braket{a_k, b_l}{\psi_i} \ket{a_k} \ket{b_l} \ket{q} .
    \end{aligned}
\]
总之我们得到经典测量前后态的变化公式
\begin{equation}
    \ket{f} = \int \dd{q} \sum_{k, l} \braket{q - g a_k}{D} \braket{a_k, b_l}{\psi_i} \ket{a_k} \ket{b_l} \ket{q}.
    \label{eq:standard-measurement}
\end{equation}
需要注意的是由于我们采取的是相互作用绘景,算符$\hat{a}$和$\hat{b}$一直会发生变化。
然而,由于自由哈密顿量不显含时间,\eqref{eq:standard-measurement}中$\ket{a_k, b_l}$的时间演化和$\bra{a_k, b_l}$的时间演化抵消了,等等,从而$\ket{f}$在相互作用结束后没有时间演化——正如我们预期的那样,因为相互作用结束之后相互作用哈密顿量就是零。

\eqref{eq:standard-measurement}看起来仍然十分复杂。
然而,在很多情况下(具体是什么情况我们很快会看到)仪器的初始态非常接近$\hat{q}$的本征态,也就是说$\braket{q}{D}$只有在$q$和某一个$q_0$非常接近的时候才有较大的值,其余时候都接近零,因此实际上是一个$\delta$函数。
这样的情况称为\textbf{理想测量}。我们现在可以看到为什么要求$\hat{q}$具有连续谱了,因为要实施一次理想测量必须允许仪器有连续分布的状态。此时\eqref{eq:standard-measurement}近似为
\begin{equation}
    \ket{f} = \sum_{k, l} \braket{a_k, b_l}{\psi_i} \ket{a_k} \ket{b_l} \ket{q = q_0 + g a_k}.
    \label{eq:ideal-measurement}
\end{equation}
我们这样就得到了一个典型的纠缠态,其中每一个分量中,仪器和待测系统在测量之后都处于完全对应的状态。
总之,如果待测系统和仪器组成的系统和外界毫无相互作用,那么测量就是如下所示的过程:
\[
    \ket{i} = \left(\sum_{k, l} \braket{a_k, b_l}{\psi_i} \ket{a_k} \ket{b_l} \right) \ket{D} \longrightarrow \ket{f} = \sum_{k, l} \braket{a_k, b_l}{\psi_i} \ket{a_k} \ket{b_l} \ket{q = q_0 + g a_k},
\]
也就是待测系统将其信息复制到了仪器当中。
然而,假如仪器足够大,那么待测系统和仪器组成的系统和外界将会有大量的相互作用。
例如,仪器可能被放置在灯光下来方便我们读取其示数,这就意味着它要不停地受到四面八方的光子的轰击。
这就意味着\eqref{eq:ideal-measurement}会很快发生退相干,最后终结于$\hat{a} \otimes \hat{b} \otimes \hat{q}$的某个本征态上,因此最后仪器停留在某个$q=q_0 + g a_k$附近,且待测系统的态也转化为$\ket{a_k}$。
将待测系统和仪器组成的系统以及所有可能的环境变量放在一起就得到了一个系综;系综中,待测系统和仪器组成的系统在退相干之后停留在本征态$\ket{a_k} \ket{b_l} \ket{q = q_0 + g a_k}$的概率正是$\abs{\braket{a_k, b_l}{\psi_i}}^2$,
也就是说,在时刻$t$测量$\hat{a}$得到$a_k$(同时将待测系统的态转化为$\ket{a_k}$)的概率就是
\begin{equation}
    P_t(a_k) = \sum_l \abs{\braket{a_k, b_l(t)}{\psi_i}}^2,
    \label{eq:probablity-of-measurement}
\end{equation}
由\eqref{eq:probablity-of-measurement}出发容易证明,待测系统为$\ket{\psi_i}$态时做测量,测量值的期望为
\begin{equation}
    \expval{\hat{a}}(t) = \mel{\psi_i}{\hat{a}(t)}{\psi_i}.
\end{equation}

实际上,我们可以把四面八方的光子或者空气分子或者这一类的干扰看成是一个巨型仪器:它和待测系统的相互作用使待测系统和它的态按照\eqref{eq:ideal-measurement}纠缠在一起,而由于这是开放体系,退相干快速发生,这就意味着在充满干扰的环境中实际上很难真的展示出待测系统的量子特性:待测系统几乎总是出现在其偏好本征态附近,因为它没完没了地受到测量。
这也是理想测量很容易就能够实现的原因:真的会用来做测量的仪器总是被做得很大,因此它们自身可以看成不停地被空气、杂散光或者别的什么东西不断测量的系统,因此它们的态总是出现在其偏好本征态附近。

在$\hat{a}$本身是待测系统的一个CSCO,从而不需要$\hat{b}$的情况下,测量$\hat{a}$得到$a_k$的概率为
\begin{equation}
    P(a_k) = \abs{\braket{a_k}{\psi_i}}^2.
\end{equation}
这表明,假如我们有一个正交归一化基$\{\ket{a_k}\}_k$,就可以使用一组不同的实数$a_k$构造算符
\[
    \hat{a} = \sum_k a_k \dyad{a_k},
\]
使用这个算符对系统做测量,则测量结束之后系统位于态$\ket{a_k}$的概率就是
\begin{equation}
    P(\ket{a_k}) = \abs{\braket{a_k}{\psi_i}}^2.
\end{equation}
注意到这个表达式只和$\ket{a_k}$有关。因此,对态矢量为$\ket{\psi}$的系统做一次测量,发现系统测量后处于态$\ket{\phi}$的概率为
\begin{equation}
    P(\ket{\phi}) = \abs{\braket{\phi}{\psi}}^2,
\end{equation}
于是我们称$\braket{\phi}{\psi}$为\textbf{概率振幅}。

需注意以上讨论建立在几个关键假设上:其一,仪器和待测系统的相互作用非常强而短促;其二,仪器和环境有杂乱无章的相互作用。
这意味着合理地构造不怎么受外界干扰而又不会严重地扰动待测系统的仪器,我们就能够得到关于待测系统状态的不完整信息而与此同时不让待测系统的态塌缩到某个本征态上。
这称为\textbf{弱测量}。

\subsubsection{有效哈密顿量}

有时,一个物理系统的哈密顿量涉及大量复杂的过程,而特定的初始条件意味着这个系统的态基本上只会出现在态空间的一小部分当中。
但这并不意味着态空间的其它部分就不会对系统的动力学造成影响。
例如,设想一个三能级系统,其中一个能级的能量远远高于另外两个能级,这意味着系统基本上不可能出现在这个能级上,但如果其余两个能级和这个高能量能级有耦合,那么这个高能量能级就可能成为另外两个能级相互转换的渠道。

设投影算符$\hat{P}$选择出了我们关注的那部分态空间,而且这部分态空间的定义不随时间变化而变化;设$\hat{Q}$是与之互补的投影算符,则
\[
    \hat{P} + \hat{Q} = 1, \quad \hat{P}^2 = \hat{P}, \quad \hat{Q}^2 = \hat{Q}.
\]
考虑薛定谔绘景,运动方程为
\[
    \hat{H} \ket{\psi} = \ii \hbar \dv{t} \ket{\psi},
\]
将投影算符作用于其上得到
\[
    \begin{aligned}
        \hat{P} \hat{H}(\hat{P}+\hat{Q}) \ket{\psi} = \ii \hbar \dv{t} \hat{P} \ket{\psi}, \\
        \hat{Q} \hat{H}(\hat{P}+\hat{Q}) \ket{\psi} = \ii \hbar \dv{t} \hat{Q} \ket{\psi}.
    \end{aligned}
\]
哈密顿量可以分成四部分,一部分完全位于$\hat{P}$筛选出来的空间中,一部分完全位于$\hat{Q}$筛选出来的空间中,另外两部分从其中一个空间跳跃到另一个空间,这四部分分别是
\[
    \hat{H}_{PP} = \hat{P} \hat{H} \hat{P}, \quad \hat{H}_{QQ} = \hat{Q} \hat{H} \hat{Q}, \quad \hat{H}_{PQ} = \hat{P} \hat{H} \hat{Q}, \quad \hat{H}_{QP} = \hat{Q} \hat{H} \hat{P}.
\]
使用投影算符的性质可以写出
\[
    \begin{aligned}
        \hat{H}_{PP} \hat{P} \ket{\psi} + \hat{H}_{PQ} \hat{Q} \ket{\psi} = \ii \hbar \dv{t} \hat{P} \ket{\psi}, \\
        \hat{H}_{QP} \hat{P} \ket{\psi} + \hat{H}_{QQ} \hat{Q} \ket{\psi} = \ii \hbar \dv{t} \hat{Q} \ket{\psi},
    \end{aligned}
\]
从后一个方程可以解出
\[
    \hat{Q} \ket{\psi} = \frac{1}{\ii \hbar \dv{t} - \hat{H}_{QQ}} \hat{H}_{QP} \hat{P} \ket{\psi},
\]
代入前一个方程就得到
\[
    \ii \hbar \dv{t} \hat{P} \ket{\psi} = \left(\hat{H}_{PP} + \hat{H}_{PQ} \frac{1}{\ii \hbar \dv{t} - \hat{H}_{QQ}} \hat{H}_{QP}\right) \hat{P} \ket{\psi}.
\]
因此我们发现,我们关注的那一部分态的时间演化由等效哈密顿量
\begin{equation}
    \hat{H}_\text{eff} = \hat{H}_{PP} + \hat{H}_{PQ} \frac{1}{\ii \hbar \dv{t} - \hat{H}_{QQ}} \hat{H}_{QP}
    \label{eq:effective-hamiltonian-original}
\end{equation}
指导,而且由$\hat{H}_{PP}, \hat{H}_{PQ}, \hat{H}_{QP}$的定义,该等效哈密顿量是$\hat{P}$筛选出的空间中的算符。
\eqref{eq:effective-hamiltonian-original}非常符合我们的直觉:时间演化可以仅仅涉及$\hat{H}_{PP}$,也可以以$\hat{H}_{QQ}$为中介。

一种特殊的情况是,态空间可以写成两个空间(记为$\mathcal{H}_1$和$\mathcal{H}_2$)的直积,系统的初始条件决定了大部分有意义的过程都发生在$\mathcal{H}_1$中,但由于耦合,不能简单地将$\mathcal{H}_2$排除掉。
这时可以构造算符$\hat{P}$使之筛选出只在$\mathcal{H}_1$中有显著活动的态,计算出有效哈密顿量;$\hat{P}$筛选出的态均形如$\ket{\psi}_1 \otimes \ket{0}_2$,由于有效哈密顿量仅涉及$\ket{\psi}_1$,不会出现两个空间之间的耦合,于是可以直接将$\mathcal{H}_2$去掉,使用$\mathcal{H}_1$和$\hat{H}_\text{eff}$来描述系统。

然而,\eqref{eq:effective-hamiltonian-original}显含一个时间求导算符的倒数,这意味着$\hat{H}_\text{eff}$实际上显含时间,而且还显含关于时间的算符,也即我们实际上是手动把关于$\mathcal{H}_2$的时间演化放进了有效哈密顿量当中,这是不便计算的。
为了让有效哈密顿量看起来像一个正常的哈密顿量,设我们考虑的过程的能量近似在$E_r$水平上,则对$\mathcal{H}_1$空间中的态,近似有
\[
    \ii \hbar \dv{t} \sim E_r,
\]
于是
\[
    \hat{H}_\text{eff} \sim \hat{H}_{PP} + \hat{H}_{PQ} \frac{1}{E_r - \hat{H}_{QQ}} \hat{H}_{QP}.
\]
对$\mathcal{H}_1$中$\hat{H}$的本征态而言,上式严格成立,我们得到自洽方程
\begin{equation}
    \left( \hat{H}_{PP} + \hat{H}_{PQ} \frac{1}{E - \hat{H}_{QQ}} \hat{H}_{QP} \right) \ket{\psi} = E \ket{\psi}.
\end{equation}
从这个方程求解出$E$,我们就得到了$\hat{H}_\text{eff}$在$\mathcal{H}_1$的一组基上的作用结果,于是也就完全确定下了$\hat{H}_\text{eff}$。
换而言之,完全精确求解的有效哈密顿量保留了原哈密顿量在我们关注的空间上的全部能谱。

然而,即使上述自洽方程也难以求解。为此通常使用微扰展开的方法。
设原哈密顿量中$\mathcal{H}_1$与$\mathcal{H}_2$没有耦合的部分为$\hat{H}_0$,其余部分为$\hat{H}'$,也即,以$\mathcal{H}_1$和$\mathcal{H}_2$为子空间将算符做分块,则$\hat{H}_0$包含对角部分,$\hat{H}'$包含非对角部分,则
\[
    \hat{H}_\text{eff} \sim \hat{H}_{0} + \hat{H}'_{PQ} \frac{1}{E_r - \hat{H}_{QQ}} \hat{H}'_{QP}.
\]
如果$\hat{H}'$让

在高能自由度和低能自由度的耦合并不明显时,高能自由度的存在与否对$\mathcal{H}_1$中的能量本征态只有不大的影响,这时可以以原哈密顿量中仅包含低能自由度的部分的本征值和本征态为起点,以低能自由度和高能自由度的耦合以及高能自由度的哈密顿量为微扰,求解出$\hat{H}$在$\mathcal{H}_1$中的本征态$\{\ket{n}\}$和本征值$\{E_n\}$。由于是本征态,它们和高能自由度没有耦合,于是低能自由度的运动完全由
\begin{equation}
    \hat{H}_\text{eff} = \sum_{\ket{n} \in \mathcal{H}_1} E_n \dyad{n}
\end{equation}
确定,我们也就得到了有效哈密顿量。

\begin{equation}
    \mel{m}{\hat{H}_\text{eff}}{n} = E_m \delta_{mn} + \mel{m}{\hat{H}'}{n} + \frac{1}{2} \sum_{\text{$l$ in $\mathcal{H}_2$}} \left( \frac{\mel{m}{\hat{H}'}{l} \mel{l}{\hat{H}'}{n}}{E_m - E_l} + \frac{\mel{m}{\hat{H}'}{l} \mel{l}{\hat{H}'}{n}}{E_n - E_l} \right) + \cdots.
\end{equation}

另一种做实际计算的方法是,考虑$\hat{H}$的本征态$\ket{\psi_n}$,其能量为$E_n$,则我们有
\[
    \hat{H} \ket{\psi_n} = E_n \ket{\psi_n}.
\]
有效哈密顿量只需要指导$\mathcal{H}_1$中的态的运行即可,因此它需要满足
\[
    \hat{H}_\text{eff} \ket{\psi_n} = E_n \ket{\psi_n}, \quad \text{for $\ket{\psi_n} \in \mathcal{H}_1$}.
\]
由于$\hat{H}_\text{eff}$是$\mathcal{H}_1$中的算符,以上方程的展开式为
\begin{equation}
    \sum_{\text{span}\{\ket{m}\} = \mathcal{H}_1} (\mel{l}{\hat{H}_\text{eff}}{m} - E_n \delta_{lm}) \braket{m}{\psi_n} = 0.
    \label{eq:effective-hamiltonian-eq-unfolded}
\end{equation}
这里我们使用一组任意的正交归一化基底$\{\ket{m}\}$,它们未必就是哈密顿量的本征态。相应的,$\hat{H}$满足
\begin{equation}
    \sum_m (\mel{l}{\hat{H}}{m} - E_n \delta_{lm}) \braket{m}{\psi_n} = 0, \quad \text{for $\ket{l} \in \mathcal{H}_1$}.
    \label{eq:hamiltonian-eq-unfolded}
\end{equation}
显然,\eqref{eq:hamiltonian-eq-unfolded}必须能够推导出\eqref{eq:effective-hamiltonian-eq-unfolded},线性代数上的结论告诉我们,记那些张成$\mathcal{H}_1$的基矢量的编号组成的集合为$W$,其余基矢量的编号组成的集合为$U$,则有
\[
    \mel{l}{\hat{H}_\text{eff}}{m} = \mel{l}{\hat{H}}{m} - \sum_{\alpha \in U} \frac{\mel{l}{H}{\alpha} \mel{\alpha}{H}{m}} {D_\alpha^W} + \sum_{\alpha \neq \beta \in U} \frac{\mel{l}{H}{\alpha} \mel{\alpha}{H}{\beta} \mel{\beta}{H}{m}}{D^W_{\alpha \beta}} + \cdots,
\]
其中,记$S$为一系列编号组成的集合,单脚标的$D$函数定义为
\[
    D_\alpha^S = H_{\alpha \alpha} - E_n - \sum_{\beta \in U, \beta \notin S} \frac{\mel{\alpha}{H}{\beta} \mel{\beta}{H}{\alpha}}{D_\alpha^S} + \sum_{\beta \neq \gamma \in U, \; \beta, \gamma \notin S} \frac{\mel{l}{H}{\alpha} \mel{\alpha}{H}{\beta} \mel{\beta}{H}{m}}{D_{\alpha \beta}^S} + \cdots,
\]
多脚标的$D$函数递归定义为
\[
    D_{\alpha \beta}^S = D_\alpha^S D_{\beta}^{S, \alpha}, \quad D_{\alpha, \beta, \gamma}^S = D_{\alpha}^S D_{\beta}^{S, \alpha} D_{\gamma}^{S, \alpha, \beta}, \ldots
\]
% TODO:和通常使用的微扰论有何关系???

\subsection{关于单位制的注记}

到现在为止我们的理论还带有一些常数。用以标记我们的理论多大程度上偏离了经典情况的$\hbar$是一个重要的常数,同时标记了时间和空间的换算关系的光速$c$是另外一个。
通过做变换
\[
    t \longrightarrow t' = ct,
\]
我们可以让光速$c$从所有的公式中消失。相应的,时间导数算符发生了
\[
    \partial_t \longrightarrow \partial_{t'} = \frac{1}{c} \partial_t
\]
的变换。
$\hbar$在计算对易子的时候出现。做变换
\[
    \pi \longrightarrow \pi' = \frac{\pi}{\hbar}
\]
也可以完全消去这个常数。由于$\pi$是通过$\mathcal{L}$对$\partial_0 \phi$求导计算出来的,这个变换实际上就是对拉氏量做了变换
\[
    \mathcal{L} \longrightarrow \mathcal{L}' = \frac{\mathcal{L}}{\hbar},
\]
而这当然不影响实际的物理。事实上它改变的是能量和动量的单位。

从本节开始,在本文的剩余部分我们将使用自然单位制,那就是说,取消时间和空间的单位差异,并且取$\hbar = 1$。
从自然单位制恢复到国际单位制就是把上面的变换反过来,也就是做变换
\[
    \begin{aligned}
        \mathcal{L}_\text{nat} &\longrightarrow \mathcal{L}_\text{int} = \hbar \mathcal{L}_\text{nat}, \\
        E_\text{nat} &\longrightarrow E_\text{int} = \hbar E_\text{nat} , \\
        \vb*{p}_\text{nat} &\longrightarrow \vb*{p}_\text{int} = \hbar \vb*{p}_\text{nat}, \\
        t_\text{nat} &\longrightarrow t_\text{int} = c t_\text{nat}.
    \end{aligned}
\]
与此同时保持各个公式的形式不变。

\subsection{单粒子情况}

在已经知道了3+1维场论的理论之后,单粒子情况实际上就是一个退化情况,因为它实际上是0+1维场论。
在单粒子情况下底流形就是时间轴,其上定义有各种物理量$\hat{A}(t)$。单粒子情况下几乎不需要使用反对易量子化方案\eqref{eq:antisymmetry-commutator},物理量和它的共轭动量之间的关系可以全部取
\begin{equation}
    \comm*{\hat{x}}{\hat{p}} = \ii.
\end{equation}
下面推导$\hat{x}, \hat{p}$和任意物理量的对易关系。
设能够将物理量$\hat{F}$展开为$\hat{x}, \hat{p}$的多项式$\hat{F} = F(\hat{x}, \hat{p})$。
对其中的每一项,都可以使用对易关系
\[
    \comm*{\hat{x}}{\hat{p}} = \ii
\]
把$\hat{x}$挪到最前面而把$\hat{p}$挪到后面,
因此展开式最后就可以写成若干个$a \hat{x}^m \hat{p}^n$形式的项之和。
现在分析其中的一项:
\[
    [\hat{x}, \hat{x}^m \hat{p}^n] = \hat{x}^m [\hat{x}, \hat{p}^n] + [\hat{x}, \hat{x}^m] \hat{p}^n = \hat{x}^m [\hat{x}, \hat{p}^n],
\]
而
\[
    [\hat{x}, \hat{p}^n] = [\hat{x}, \hat{p} \hat{p}^{n-1}] = 
    \hat{p} [\hat{x}, \hat{p}^{n-1}] + [\hat{x}, \hat{p}] \hat{p}^{n-1} = \hat{p} [\hat{x}, \hat{p}^{n-1}] + \ii \hat{p}^{n-1}
\]
于是递推得到
\[
    [\hat{x}, \hat{p}^n] = \ii n \hat{p}^{n-1},
\]
因此
\[
    [\hat{x}, \hat{x}^m \hat{p}^n] = \ii n \hat{x}^m \hat{p}^{n-1}.
\]
这样就可以写出
\begin{equation}
    [\hat{x}, \hat{F}(\hat{x}, \hat{p})] = \ii \pdv{p} \hat{F}(\hat{x}, \hat{p}),
\end{equation}
在作用偏微分符号之前需要先把$F$中的每一项都变形成$\hat{x}$在前$\hat{p}$在后的形式。
使用同样的方法还可以导出
\begin{equation}
    [\hat{p}, \hat{F}(\hat{x}, \hat{p})] = - \ii \pdv{x} \hat{F}(\hat{x}, \hat{p}),
\end{equation}
同样,作用偏微分符号之前需要先把$F$中的每一项都变形成$\hat{x}$在前$\hat{p}$在后的形式。

在海森堡绘景下
\[
    \dv{\hat{A}}{t} = \frac{1}{\ii} [\hat{A}, H] + \pdv{\hat{A}}{t},
\]
于是
\[
    \dv{\hat{x}}{t} = \frac{1}{\ii} [\hat{x}, H] = \pdv{p} \hat{H}(\hat{x}, \hat{p}), \quad
    \dv{\hat{p}}{t} = \frac{1}{\ii} [\hat{p}, H] = -\pdv{x} \hat{H}(\hat{x}, \hat{p})
\]
当$\hbar \to 0$时,上式仍然成立,而此时$\hat{x}$和$\hat{p}$已经是对易的了,因此它们退化为了可以直接使用实数表示的情况,我们也就过渡到了经典力学。

\section{准经典理论}

\subsection{退化到经典情况}\label{sec:back-to-classical}

前面提到,$\hbar \to 0$时,海森堡绘景下的量子时间演化方程\eqref{eq:quantum-evolution}退化为经典的时间演化方程\eqref{eq:evolution-of-any-quantity}。
但需要注意的是,在$\hbar\to 0$时由\eqref{eq:quantum-evolution}退化得到的方程仍然是一个算符方程。
要获得通常的关于物理量的方程,还需要做一些操作。$\hbar\to 0$时得到的演化方程是
\[
    \dv{\hat{A}}{t} = \pdv{\hat{A}}{\hat{\phi}} \dv{\hat{\phi}}{t} + \pdv{\hat{A}}{\partial_i \hat{\phi}} \partial_i \dv{\hat{\phi}}{t} + \pdv{\hat{A}}{\hat{\pi}} \dv{\hat{\pi}}{t},
\]
这个方程仅在海森堡绘景下成立。记系统的态矢量为$\ket{\psi}$,我们就得到
\[
    \dv{t} \mel{\psi}{\hat{A}}{\psi} =  \mel{\psi}{\pdv{\hat{A}}{\hat{\phi}} \dv{\hat{\phi}}{t}}{\psi} + \mel{\psi}{\pdv{\hat{A}}{\partial_i \hat{\phi}} \partial_i \dv{\hat{\phi}}{t}}{\psi} + \mel{\psi}{\pdv{\hat{A}}{\hat{\pi}} \dv{\hat{\pi}}{t}}{\psi}.
\]
在$\hbar\to 0$时,所有算符都近似是对易的,从而它们全部可以在同一组基下对角化。设这一组基为$\{\ket{n}\}$,则
% TODO:似乎$\ket{\psi}$总是几乎是这组基中的一个,为什么?
\[
    \begin{aligned}
        \mel{\psi}{\hat{A}\hat{B}}{\psi} &= \sum_{m,n} \braket{\psi}{m} \mel{m}{\hat{A}\hat{B}}{n} \braket{n}{\psi} \\
        &= \sum_{n} \braket{\psi}{n} \mel{n}{\hat{A}\hat{B}}{n} \braket{n}{\psi} \\
        &= 
    \end{aligned}
\]
% TODO
% 总之核心思想是,算符在$\hbar\to 0$时也不是实数物理量,真正的实数物理量的表达式必定会牵扯到态矢量。这也就是场算符的傅里叶分量看起来似乎是固定的值一样的原因,因为场算符本身包含了所有可能的经典场的取值,在$\hbar\to 0$时经典场的取值是多少不是场算符决定的而是态矢量决定的。

正则对易关系
\[
    [q_i, p_j] = \delta_{ij}
\]
实际上是非常自然的,因为使用这个关系推导出来的方程和使用对应的拉氏量和E-L方程推导出来的运动方程是一样的。
任何两个物理量的对易子$[A,B]$最后都可以写成一系列形如$\gamma_1 \gamma_2 \cdots [\gamma, \gamma] \cdots$这样的项的
叠加,其中每一个$\gamma$都是一个基本算符(坐标、动量、自旋等等),如果我们已知$[p, q] = \ii \hbar \cdot \text{something}$
而运动方程为
\[
    \dv{A}{t} = \frac{1}{\ii \hbar} [A, H]
\]
那么在最后得到的运动方程中$\ii \hbar$就被消去了。
现在让$\hbar \to 0$,我们会发现运动方程的形式没有发生变化(因为它根本就和$\hbar$无关),但是此时所有的物理量都是对易的了。
重新定义
\[
    \{A, B\} = \frac{1}{\ii \hbar}[A, B],
\]
它$\hbar \to 0$时仍然收敛于有限值。然后使用对易关系可以推导出它就是所谓的泊松括号。

使用不随时间变化的态矢量$\ket{\psi}$表述系统。
可观察量算符$A$随着时间的演化为
\begin{equation}
    \dv{A}{t} = \frac{1}{\ii \hbar} [A, H] + \pdv{A}{t}
    \label{eq:canonical-time-evolution}
\end{equation}

形式上这个式子和经典力学中的式子差了一个系数$\ii \hbar$。表面上看这正是量子力学和经典力学不同的地方(引入了常数$\hbar$),但实际上并非如此,因为在量子力学中有
\begin{equation}
    [x_i, p_j] = \ii \hbar \delta_{ij}
\end{equation}
一来一去,系数$\ii \hbar$就约掉了,实际上,完全可以定义
\[
    [x_i, p_j] = \delta_{ij}
\]
而此时的演化方程就变成
\[
    \dv{A}{t} = [A, H] + \pdv{A}{t}
\]
形式上和经典情况完全一致。那么量子力学和经典力学到底相差在哪里?
最关键的差别实际上是,量子力学中的$x, p$等量都是算符,因此有可能
\[
    AB - BA \neq 0
\]
而经典情况下上式恒为零。并且,这个不对易性直接和$[\cdot, \cdot]$的定义有关:
\[
    [A, B] = AB - BA
\]
在经典力学中$AB-BA$也是一个反对称的运算,但是它恒为零,因此和系统的演化无关——经典力学中和系统演化有关的那种$[\cdot, \cdot]$完全由
\[
    [A, B] = \sum_i \left( \pdv{A}{q_i} \pdv{B}{p_i} - \pdv{A}{p_i} \pdv{B}{q_i} \right)
\]
定义,上式又等价于两个假设:乘法交换律,以及
\[
    [q_i, q_j] = 0, [p_i, p_j] = 0, [q_i, p_j] = \delta_{ij}
\];
而在量子力学中,我们假定$xp-px=\ii \hbar$,并且认为
\[
    [A, B] = AB - BA
\]

总之,在括号$[\cdot, \cdot]$的性质、坐标和动量之间的括号的取值上,经典力学和量子力学之间完全没有差异。两者的差异在于,经典力学假定所有物理量都是可交换的实数,此时我们可以推导出泊松括号的表达式;量子力学假定$[\cdot, \cdot]$就代表两个物理量(现在是算符了!)的交换子。

因此在经典力学中使用“对易”一词可能引起误解:它可能指“两个量的乘积是不是可以交换”,此时的回答一概是“是”;它也可能指“两个量的泊松括号是不是零”。这两种理解之间完全没有联系。而在量子力学中这两种理解实际上是等价的。

可观察量经过时间演化之后还应该是可观察量。但是这个怎么证明呢?

\[
    \left(\dv{A}{t}\right)^\dagger = - \frac{1}{\ii \hbar} [A, H]^\dagger + \left(\pdv{A}{t}\right)^\dagger = \frac{1}{\ii \hbar} [A^\dagger, H^\dagger] + \left(\pdv{A}{t}\right)^\dagger
\]
如果在某一时刻$A$是观察算符,则下一刻它仍然是观察算符的充要条件就是
\[
    \frac{1}{\ii \hbar} [A, H^\dagger] + \left(\pdv{A}{t}\right)^\dagger = \frac{1}{\ii \hbar} [A, H] + \pdv{A}{t}
\]
所以什么情况下确凿无疑的有$H$是厄米算符呢?

直和:一个参数本来只能取这些值,现在可以取另一些值了(加入了基矢量),则两个空间要做直和。

直积:本来只需要考虑这个参数,现在需要考虑别的参数了。

TODO:正则对易关系与运动方程。好像如果不使用正则对易关系,那么算符演化方程就和通过对应的拉氏量写出的运动方程不一致。

可以使用傅里叶变换把哈密顿量中的$\nabla \phi$之类的项弄掉。
然后得到的哈密顿量做对角化(大部分情况下已经对角化好了),就得到了一系列谐振子哈密顿量的叠加:
\[
    H = \int \dd x^3 a a^\dagger + \text{something}
\]

拉氏量的耦合对应着态空间的耦合?混合态、直积还有一系列神奇的东西。直和其实是增加了基矢量。
也就是说一个算符的各个不变子空间的直和构成全空间。
\[
    \delta(\vb*{r} - \vb*{r}_0) = \delta(x - x_0) \delta(y - y_0) \delta(z - z_0)
\]
所以三维态矢量其实是一维态矢量的直积。

把问题规范一下:现在我们已知系统的演化可以完全由一组算符$\hat{O}_1, \hat{O}_2, \ldots, \hat{O}_n$描述,也就是说能够写出哈密顿算符$\hat{H}$来描述它们的演化。此外,这些算符的对易关系全部给定,从而$[\hat{O}_i, \hat{H}]$也确定了。
现在的问题是,态矢量应该怎么取?或者说,对应的希尔伯特空间应该是怎样的结构?
实际上在量子场论中这似乎并不是一个问题,因为很少用到态矢量。这是因为只有算符是重要的,态矢量实际上只是算符对应的李代数的幺正表示而已。

设算符$\hat{O}_1, \hat{O}_2$分别是希尔伯特空间$H_1$、$H_2$的CSCO,且它们组成的集合是$H$的CSCO,那么$H = H_1 \otimes H_2$,并且$\ket{x_1, x_2} = \ket{x_1} \otimes \ket{x_2}$,其中$\ket{x_1} \in H_1, \ket{x_2} \in H_2$,$\ket{x_1, x_2} \in H$。
顺便抨击一下常见的量子力学教材:一上来就讲态矢量在概念上真的很不清楚!

我好像有点反应过来了。CSCO就是用来做这个的!
设$\hat{O}_1, \hat{O}_2, \ldots, \hat{O}_n$组成了一个空间$H$上的CSCO,且与$\hat{O}_1$对应的

作用在一个算符上的元算符如果不改变它所作用的那个算符的定义域,那么将这个元算符作用在另一个算符上就相当于将第三个算符和第二个算符复合。

一些问题:是不是任何一个幺正算符都对应着某个物理过程?(从初态到末态的映射)

粒子数表象。

\section{二次量子化}

\section{微扰散射理论}
% TODO:T矩阵,S矩阵,似乎前者是关于“跃迁率”的,即每秒事件发生的可能性,后者是关于无限长时间的散射

任何一个幺正过程都对应一个散射过程。

关于升降算符:设算符$\hat{x}$组成希尔伯特空间$\mathcal{H}$上的CSCO,其本征态为$\ket{x_1}, \ket{x_2}, \ldots$。
由于需要且只需要给定基矢量的像就能够确定一个算符,必定存在这样一个算符$\hat{a}$,它能够将$\ket{x_1}$映射为$\ket{x_2}$的某个非零倍数,将$\ket{x_2}$映射为$\ket{x_3}$的某个非零倍数,等等。这个算符称为升算符;升算符的逆就是降算符。显然,只需要一个本征态和升降算符就能够完全把态空间确定下来。另:如果本征值有上界,那么升算符作用在最大的本征值对应的本征态后得到$0$;同理,如果本征值有下界,那么降算符作用在最小的本征值对应的本征态之后得到$0$。
将$\ket{x}$提升到$\ket{x+c}$(可能差一个常数)的算符$\hat{a}$满足
\[
    [\hat{x}, \hat{a}] = c \hat{a}.
\]
特别的,若$\hat{x}$是厄米算符,且$\hat{a}$让本征态提升了$c$,那么$\hat{a}^\dagger$就会让本征态下降$c$,也就是说升降算符互为共轭转置。

现在的问题是怎样构造出升降算符。当然,任何情况下升降算符都应该满足对易关系$[\hat{x}, \hat{a}] = c \hat{a}$,但是这是不是足够了?
实际上还是能够构造出反例的,但是这些反例都是基于具体的分析构造,而物理上应该仅仅关心有关的代数结构。

能量、能级:能级实际上只对二次型哈密顿量比较好处理,此时总是可以把哈密顿量写成
\[
    \hat{H} = \sum_i E_i \hat{a}^\dagger_i \hat{a}_i,
\]
然后可以讨论某个能级上有几个粒子,等等。

有相互作用不方便处理能级。

\subsection{跃迁}

\subsubsection{费米黄金法则}

设系统的自由哈密顿量为$\hat{H}_0$,它的一组基矢量为$\{\ket{m}\}$,本征值记为$\{E_m\}$,相互作用哈密顿量为$\hat{H}'$。
假定相互作用哈密顿量不显含时间。设系统初态为$\ket{m}$,态随时间的演化为
\[
    \ket{\psi(t)} = \sum_n a_n(t) \ket{n} \ee^{- \ii E_n t / \hbar},
\]
显然$t=0$时除了$a_m=1$以外其它$a$均为零。使用Dyson级数并只计算到一阶,有
\[
    \ii \hbar a^{(1)}_k(t) = \sum_n \int \dd{t'} \mel{k}{\hat{H}'}{n} \ee^{\ii \omega_{kn} t'} a_n^{(0)},
\]
其中
\[
    \omega_{mn} = \frac{E_m - E_n}{\hbar}.
\]
$a_n^{(0)}$只在$n=m$时有非零值,且时间演化是从$t'=0$演化到$t'=t$,于是
\[
    \begin{aligned}
        \ii \hbar a^{(1)}_k(t) &= \sum_n \int \dd{t'} \mel{k}{\hat{H}'}{n} \ee^{\ii \omega_{kn} t'} a_n^{(0)} \\
        &= \mel{k}{\hat{H}'}{m} \frac{\sin \omega_{km} t / 2}{\omega_{km} / 2} \ee^{\ii \omega_{km} t / 2}. 
    \end{aligned}
\]
注意到$m \neq k$时$a_k = a_k^{(1)}$,而$a_k(t)$的模长平方正是$t'=0$时系统状态为$\ket{m}$而$t'=t$时系统经过观测状态为$\ket{k}$的概率,此概率就是所谓的\textbf{跃迁概率},于是跃迁概率的表达式就是
\begin{equation}
    P_k(t) = \frac{4 \abs*{\mel{k}{\hat{H}'}{m}}^2}{\hbar^2} \frac{\sin^2 \omega_{km} t / 2}{\omega_{km}^2}.
\end{equation}
现在假如系统实际上是一个开放系统,且诸$\ket{m}$构成一组偏好基,则可以使用一个经典马尔可夫过程来描述系统的演化,而使用经典的态(各个态出现的概率就是$\ket{m}$的振幅的模长平方)描述系统的状态。
系统每个时刻都有一定概率跃迁,也有一定概率不跃迁而等待下一时刻,此时跃迁速率为% TODO:从量子转向经典地一般描述,还有类似于“经典电磁场和量子粒子耦合”这样有经典有量子的系统
\begin{equation}
    \Gamma_k(t) = \dv{P_k}{t} = \frac{2 \abs*{\mel{k}{\hat{H}'}{m}^2}}{\hbar^2} \frac{\sin \omega_{km} t}{\omega_{km}}.
\end{equation}
如果我们要计算系统跃迁到一系列态上的概率,那么总跃迁速率为
\[
    \Gamma(t) = \sum_k \Gamma_k(t) = \sum_{E_k} \frac{2 \abs*{\mel{k}{\hat{H}'}{m}^2}}{\hbar^2} \frac{\sin \omega_{km} t}{\omega_{km}},
\]
让$\omega$连续取值,并引入态密度来表示哪些态是允许的,有
\begin{equation}
    \begin{aligned}
        \Gamma(t) &= \int \dd{E} \rho(E) \frac{2 \abs*{\mel{k}{\hat{H}'}{m}^2}}{\hbar^2} \frac{\sin \omega t}{\omega} \\
        &= \int \dd{\omega} \rho(E) \frac{2 \abs*{\mel{k}{\hat{H}'}{m}^2}}{\hbar} \frac{\sin \omega t}{\omega}.
    \end{aligned}
\end{equation}
其中$E = \hbar \omega + E_m$。如果能够保证能量守恒,并且可以持续很长时间都有跃迁,则$t$很大,于是$\sin \omega t / \omega$是高度振荡的函数,则可以把态密度$\rho(E)$提到积分号外面,而积分号变成跃迁矩阵元的平均乘以$\sin \omega t / \omega$的积分值,最后计算得到
\begin{equation}
    \Gamma(t) = \frac{2\pi}{\hbar} \rho(E) \expval{\abs*{\mel{k}{\hat{H}'}{m}^2}}_k.
\end{equation}
这样,如果我们有数目巨大的一系列完全相同的系统,总数为$N$,它们之间相互影响很小,那么在$\dd{t}$时间内,发生跃迁的系统的数目几乎确定为$N \Gamma(t) \dd{t}$。
因此可以列出某个状态的系统的数目服从的微分方程。

\subsubsection{周期性策动}

再考虑一个略有不同的情况——外界作用周期性地施加在
% TODO

\section{自由度解耦}

设有哈密顿量$\hat{H}$,如果某个物理量$\hat{A}$满足
\[
    \comm*{\hat{H}}{\hat{A}} = \comm*{\hat{H}'(A)}{\hat{A}},
\]
其中$\hat{H}'$是只和$\hat{A}$有关的物理量,那么在只关心$\hat{A}$时可以以$\hat{H}'$为等效哈密顿量。

其实也可以通过路径积分的观点看这个问题:与$\hat{A}$对易的那些自由度可以直接积掉而只留下一个因子。

% TODO:什么时候可以将算符用其期望替代,因为这会改变对易关系。。

\section{浸染近似和几何相位}

% TODO:希尔伯特空间分裂成几个分支
设总的希尔伯特空间为$H$,由于守恒量、拓扑性质等,满足某个特定约束条件的态组成的空间为$H'$,则$H / H'$标记了不同的守恒量、拓扑性质等的参数。
例如对称性自发破缺意味着有序参量出现。
% 衡量哈密顿量的大小需要通过其造成的能级分裂的大小,而不是绝对大小,因为绝对大小可以通过加上或者减去一个任意的常数来轻易改变。
有些时候,一个理论中的两个不同的态可以看成是等价的态,这当然是因为系统中有冗余的、非物理的自由度。
这种情况下通常不说这是“对称性”,而是说这是“冗余的自由度”。这种对称性只是拉氏量的对称;态空间中的这种“对称”已经被模掉了。
% TODO:拉氏量的对称性和态空间的对称性

\end{document}

\part{原子物理}

\documentclass[UTF8, a4paper]{ctexart}

\usepackage{geometry}
\usepackage{titling}
\usepackage{titlesec}
\usepackage{paralist}
\usepackage{footnote}
\usepackage{enumerate}
\usepackage{amsmath, amssymb, amsthm}
\usepackage{cite}
\usepackage{graphicx}
\usepackage{subfigure}
\usepackage{physics}
\usepackage[colorlinks, linkcolor=black, anchorcolor=black, citecolor=black]{hyperref}

\geometry{left=3.28cm,right=3.28cm,top=2.54cm,bottom=2.54cm}
\titlespacing{\paragraph}{0pt}{1pt}{10pt}[20pt]
\setlength{\droptitle}{-5em}
\preauthor{\vspace{-10pt}\begin{center}}
\postauthor{\par\end{center}}

\newcommand*{\ee}{\mathrm{e}}
\newcommand*{\ii}{\mathrm{i}}
\renewcommand*{\dd}{\mathop{}\!\mathrm{d}}
\newcommand*{\st}{\quad \text{s.t.} \quad}
\newcommand*{\const}{\mathrm{const}}
\newcommand*{\natnums}{\mathbb{N}}
\newcommand*{\reals}{\mathbb{R}}
\newcommand*{\complexes}{\mathbb{C}}
\DeclareMathOperator{\timeorder}{T}
\newcommand*{\ogroup}[1]{\mathrm{O}(#1)}
\newcommand*{\sogroup}[1]{\mathrm{SO}(#1)}

\title{原子物理}
\author{wujinq}

\begin{document}

\maketitle

\section{氢原子}

\end{document}

\end{document}