\documentclass[hyperref, UTF8, a4paper, oneside]{ctexbook}

\usepackage{geometry}
\usepackage{titling}
\usepackage{titlesec}
\usepackage{paralist}
\usepackage{footnote}
\usepackage{enumerate}
\usepackage{autobreak}
\usepackage{amsmath, amssymb, amsthm}
\usepackage{mathtools}
\usepackage{bbm}
\usepackage{cite}
\usepackage{graphicx}
\usepackage{subfigure}
\usepackage{physics}
\usepackage{siunitx}
\usepackage{tikz}
\usepackage[compat=1.1.0]{tikz-feynhand}
\usepackage[ruled, vlined, linesnumbered, noend]{algorithm2e}
\usepackage{xr-hyper}
\usepackage[colorlinks, linkcolor=black, anchorcolor=black, citecolor=black, filecolor=black]{hyperref}
\usepackage[most]{tcolorbox}
\usepackage{caption}
\usepackage{prettyref}

\externaldocument[optics-]{../optics/optics}[optics.pdf]
\externaldocument[vasp-]{../cond-comp/vasp/vasp}[vasp.pdf]
\externaldocument[qft-]{../relativistic-qft/relativistic-qft}[relativistic-qft.pdf]
\externaldocument[soft-]{../soft/soft}[soft.pdf]

\geometry{left=3.18cm,right=3.18cm,top=2.54cm,bottom=2.54cm}
\titlespacing{\paragraph}{0pt}{1pt}{10pt}[20pt]
\setlength{\droptitle}{-5em}
\preauthor{\vspace{-10pt}\begin{center}}
\postauthor{\par\end{center}}

\DeclareMathOperator{\timeorder}{\mathcal{T}}
\DeclareMathOperator{\diag}{diag}
\DeclareMathOperator{\legpoly}{P}
\DeclareMathOperator{\primevalue}{P}
\DeclareMathOperator{\sgn}{sgn}
\newcommand*{\ii}{\mathrm{i}}
\newcommand*{\ee}{\mathrm{e}}
\newcommand*{\const}{\mathrm{const}}
\newcommand*{\suchthat}{\quad \text{s.t.} \quad}
\newcommand*{\argmin}{\arg\min}
\newcommand*{\argmax}{\arg\max}
\newcommand*{\normalorder}[1]{: #1 :}
\newcommand*{\pair}[1]{\langle #1 \rangle}
\newcommand*{\fd}[1]{\mathcal{D} #1}

\newcommand*{\st}{\quad \text{s.t.} \quad}
\newcommand*{\natnums}{\mathbb{N}}
\newcommand*{\reals}{\mathbb{R}}
\newcommand*{\complexes}{\mathbb{C}}
\newcommand*{\ogroup}[1]{\mathrm{O}(#1)}
\newcommand*{\sogroup}[1]{\mathrm{SO}(#1)}\DeclareMathOperator{\laguerre}{L}
\newcommand*{\lsterm}[3]{$^{#1}${#2}$_{#3}$}
\newcommand*{\nuclear}[3]{^{#2}_{#3}\text{#1}}
\newcommand*{\hankelone}{\mathrm{h}^{(1)}}
\newcommand*{\hankeltwo}{\mathrm{h}^{(2)}}

\newenvironment{bigcase}{\left\{\quad\begin{aligned}}{\end{aligned}\right.}

\newrefformat{chap}{第\ref{#1}章}
\newrefformat{sec}{第\ref{#1}节}
\newrefformat{note}{注\ref{#1}}
\newrefformat{fig}{图\ref{#1}}
\newrefformat{alg}{算法\ref{#1}}
\newrefformat{back}{背景知识\ref{#1}}
\newrefformat{info}{资料框\ref{#1}}
\newrefformat{warn}{注意事项\ref{#1}}
\renewcommand{\autoref}{\prettyref}

\usetikzlibrary{arrows,shapes,positioning}
\usetikzlibrary{arrows.meta}
\usetikzlibrary{decorations.markings}
\tikzstyle arrowstyle=[scale=1]
\tikzstyle directed=[postaction={decorate,decoration={markings,
    mark=at position .5 with {\arrow[arrowstyle]{stealth}}}}]
\tikzstyle ray=[directed, thick]
\tikzstyle dot=[anchor=base,fill,circle,inner sep=1pt]

% Algorithm setting
\renewcommand{\algorithmcfname}{算法}
% Python-style code
\SetKwIF{If}{ElseIf}{Else}{if}{:}{elif:}{else:}{}
\SetKwFor{For}{for}{:}{}
\SetKwFor{While}{while}{:}{}
\SetKwInput{KwData}{输入}
\SetKwInput{KwResult}{输出}
\SetArgSty{textnormal}

\tcbuselibrary{skins, breakable, theorems}

\newtcbtheorem[number within=chapter]{back}{背景知识}%
  {colback=blue!5,colframe=blue!65,fonttitle=\bfseries, breakable}{back}
\newtcbtheorem[number within=chapter]{info}{资料框}%
  {colback=blue!5,colframe=blue!65,fonttitle=\bfseries, breakable}{info}
\newtcbtheorem[number within=chapter]{warning}{注意事项}%
  {colback=orange!5,colframe=orange!65,fonttitle=\bfseries, breakable}{warn}

\renewcommand{\emph}[1]{\textbf{#1}}
\newcommand*{\concept}[1]{\underline{\textbf{#1}}}

\numberwithin{equation}{chapter}

\newcommand{\hmn}[1]{% Hermann-Maguin notation
  \ensuremath{\begingroup\setupHMN #1\endgroup}%
}

\newcommand{\setupHMN}{%
  \doHMN{-}{\HMNoverline}%
  \doHMN{*}{\HMNminverse}%
  \doHMN{i}{\infty}
}

\newcommand{\doHMN}[2]{%
  \begingroup\lccode`~=`#1
  \lowercase{\endgroup\let~}#2%
  \mathcode`#1="8000
}

\newcommand{\HMNminverse}[1]{\frac{#1}{m}}
\newcommand{\HMNoverline}[1]{\mkern1mu\overline{\mkern-1mu#1\mkern-1mu}\mkern1mu}

\newcommand{\Ztwo}{$\mathbb{Z}_2$}

\newcommand{\bigO}[1]{\mathcal{O}(#1)}

\newcommand{\vaspdoc}{\href{../computational/vasp/vasp.pdf}{VASP笔记}}
\newcommand{\opticsdoc}{\href{../optics/optics}{光学笔记}}
\newcommand{\softdoc}{\href{../soft/soft}{流体和软物质物理笔记}}
\newcommand{\qftdoc}{\href{../relativistic-qft/relativistic-qft}{相对论性量子场论笔记}}

\title{牛顿力学和非相对论量子力学}
\author{吴晋渊}

\begin{document}

\maketitle

\part{牛顿力学}

\documentclass[UTF8, a4paper, oneside, scheme=plain, 12pt]{ctexrep}

\usepackage[T1]{fontenc}
\usepackage{libertinus}
\usepackage{geometry}
\usepackage{float}
\usepackage{titling}
\usepackage{titlesec}
\usepackage{paralist}
\usepackage{footnote}
\usepackage{enumerate}
\usepackage{amsmath, amsthm}
\usepackage{gb4e}
\noautomath
\usepackage{bbm}
\usepackage{textcomp}
\usepackage{soul}
\usepackage{graphicx}
\usepackage{siunitx}
\usepackage[table,xcdraw]{xcolor}
\usepackage{tikz}
\usepackage[ruled, vlined, linesnumbered, noend]{algorithm2e}
\usepackage{xr-hyper}
\usepackage[colorlinks, citecolor = purple]{hyperref} % linkcolor=black, anchorcolor=black, citecolor=black, filecolor=black
\usepackage[most]{tcolorbox}
\usepackage{caption}
\usepackage{subcaption}
\usepackage{booktabs}
\usepackage{multirow}
\usepackage[figuresright]{rotating}
\usepackage{acro}
\usepackage[citestyle=authoryear,backend=bibtex,natbib=true,doi=false,isbn=false,url=false]{biblatex}
\addbibresource{references/classical-grammars.bib}
\addbibresource{references/classical-lexicon.bib}
\addbibresource{references/historical-phonology.bib}
\addbibresource{references/general-typology.bib}
\addbibresource{references/other-languages.bib}
\usepackage{prettyref}

\geometry{left=3.18cm,right=3.18cm,top=2.54cm,bottom=2.54cm}
\titlespacing{\paragraph}{0pt}{1pt}{10pt}[20pt]
\setlength{\droptitle}{-5em}

\DeclareMathOperator{\timeorder}{\mathcal{T}}
\DeclareMathOperator{\diag}{diag}
\DeclareMathOperator{\legpoly}{P}
\DeclareMathOperator{\primevalue}{P}
\DeclareMathOperator{\sgn}{sgn}
\newcommand*{\ii}{\mathrm{i}}
\newcommand*{\ee}{\mathrm{e}}
\newcommand*{\const}{\mathrm{const}}
\newcommand*{\suchthat}{\quad \text{s.t.} \quad}
\newcommand*{\argmin}{\arg\min}
\newcommand*{\argmax}{\arg\max}
\newcommand*{\normalorder}[1]{: #1 :}
\newcommand*{\pair}[1]{\langle #1 \rangle}
\newcommand*{\fd}[1]{\mathcal{D} #1}
\newcommand*{\textto}{$\to$}
\newcommand*{\textgt}{$>$ }
\newcommand{\focus}[1]{\textbf{#1}}

\newcommand*{\citesec}[1]{\S~{#1}}
\newcommand*{\citechap}[1]{Ch.~{#1}}
\newcommand*{\citechaps}[1]{Chs.~{#1}}
\newcommand*{\citefig}[1]{Fig.~{#1}}
\newcommand*{\citetable}[1]{Table~{#1}}
\newcommand*{\citepage}[1]{p.~{#1}}
\newcommand*{\citepages}[1]{pp.~{#1}}
\newcommand*{\citefootnote}[1]{fn.~{#1}}

\newrefformat{sec}{\citesec{\ref{#1}}}
\newrefformat{fig}{\citefig{\ref{#1}}}
\newrefformat{tbl}{\citetable{\ref{#1}}}
\newrefformat{chap}{\citechap{\ref{#1}}}
\newrefformat{fn}{\citefootnote{\ref{#1}}}
\newrefformat{box}{Box~\ref{#1}}
\newrefformat{ex}{\ref{#1}}

% color boxes

\tcbuselibrary{skins, breakable, theorems}

\newtcbtheorem[number within=chapter]{infobox}{Box}{
    enhanced,
    boxrule=0pt,
    %colback=blue!5,
    %colframe=blue!5,
    colback=white,
    colframe=white,
    coltitle=blue!50,
    borderline west={4pt}{0pt}{blue!65},
    sharp corners,
    fonttitle=\bfseries, 
    breakable,
    before upper={\parindent15pt\noindent}}{box}
\newtcbtheorem[number within=chapter, use counter from=infobox]{theorybox}{Box}{
    enhanced,
    boxrule=0pt,
    colback=orange!5, 
    colframe=orange!5, 
    coltitle=orange!50,
    borderline west={4pt}{0pt}{orange!65},
    sharp corners,
    fonttitle=\bfseries, 
    breakable,
    before upper={\parindent15pt\noindent}}{box}
\newtcbtheorem[number within=chapter, use counter from=infobox]{learnbox}{Box}{
    enhanced,
    boxrule=0pt,
    colback=green!5,
    colframe=green!5,
    coltitle=green!50,
    borderline west={4pt}{0pt}{green!65},
    sharp corners,
    fonttitle=\bfseries, 
    breakable,
    before upper={\parindent15pt\noindent}}{box}
\newtcbtheorem[number within=chapter, use counter from=infobox]{todobox}{Box}{
    enhanced,
    boxrule=0pt,
    colback=red!5,
    colframe=red!5,
    coltitle=red!50,
    borderline west={4pt}{0pt}{red!65},
    sharp corners,
    fonttitle=\bfseries, 
    breakable,
    before upper={\parindent15pt\noindent}}{box}

\AtBeginEnvironment{infobox}{\small}
\AtBeginEnvironment{todobox}{\small}

\newcommand*{\concept}[1]{\textbf{#1}}
\newcommand*{\term}[1]{\emph{#1}}
\newcommand{\form}[1]{\emph{#1}}
\newcommand{\work}[1]{\textit{#1}}
\newcommand{\species}[1]{\textit{#1}}

\newcommand{\redp}{\textasciitilde}

\DeclareAcronym{blt}{short = BLT, long = Basic Linguistic Theory}
\DeclareAcronym{cgel}{short = CGEL, long = The Cambridge Grammar of the English Language}
\DeclareAcronym{dm}{short = DM, long = Distributed Morphology}
\DeclareAcronym{tag}{long = Tree-adjoining grammar, short = TAG}
\DeclareAcronym{sfp}{long = sentence-final particle, short = \textsc{sfp}}
\DeclareAcronym{np}{long = noun phrase, short = NP}
\DeclareAcronym{vp}{long = verb phrase, short = VP}
\DeclareAcronym{pp}{long = preposition phrase, short = PP}
\DeclareAcronym{cls}{long = classifier, short = CLS}
\DeclareAcronym{dist}{long = distal, short = DIST}
\DeclareAcronym{prox}{long = proximate, short = PROX}
\DeclareAcronym{dem}{long = demonstrative, short = DEM}
\DeclareAcronym{classify}{long = classifier, short = \textsc{cl}}
\DeclareAcronym{dur}{long = durative, short = DUR}
\DeclareAcronym{neg}{long = negative, short = \textsc{neg}}
\DeclareAcronym{cc}{long = copular complement, short = CC}
\DeclareAcronym{cs}{long = copular subject, short = CS}
\DeclareAcronym{tam}{long = {tense, aspect, mood}, short = TAM}
\DeclareAcronym{past}{long = past, short = PST}
\DeclareAcronym{nonpast}{long = non-past, short = NPST}
\DeclareAcronym{present}{long = present, short = PRES}
\DeclareAcronym{progressive}{long = progressive, short = \textsc{poss}}
\DeclareAcronym{perfect}{long = perfect, short = \textsc{perf}}
\DeclareAcronym{passive}{long = passive, short = \textsc{pass}}
\DeclareAcronym{copula}{long = copula, short = COP}
\DeclareAcronym{possessive}{long = possessive, short = \textsc{poss}}

\newcommand{\asis}[1]{\textsc{#1}}
\newcommand{\oneof}[1]{{#1}}
\newcommand*{\homo}[2]{#1$_{\text{#2}}$}

\newcommand{\cgel}{\href{../English/cambridge.pdf}{my notes about CGEL}}
\newcommand{\latin}{\href{../Latin/latin-notes.pdf}{my notes about Latin}}
\newcommand{\alignment}{\href{../alignment/alignment.pdf}{my notes about alignment}}
\newcommand{\exerciseone}{\href{../Exercise/2021-3.pdf}{this exercise}}
\newcommand{\method}{\href{../methodology/glossing.pdf}{this note about my understanding of descriptive grammars}}

\newcommand{\ala}{à la}
\newcommand{\translate}[1]{`#1'}
\newcommand{\vP}{\textit{v}P}
\newcommand*{\category}[1]{\textsc{#1}}
\newcommand*{\specialunit}[1]{$<$\textit{#1}$>$}
\newcommand{\before}{$> \ $}

% Make subsubsection labeled
\setcounter{secnumdepth}{4}
\setcounter{tocdepth}{4}
% reset example counter every chapter (but do not include the chapter number to the label)
\counterwithin{exx}{chapter} 

% Reference formats
\renewcommand*{\nameyeardelim}{\space} % No comma between year and name
\DeclareNameAlias{sortname}{family-given} % Putting the family name before the given name
\DeclareNameAlias{default}{family-given} 
\DeclareFieldFormat{labelnumberwidth}{} % No number label like [12] in the reference list
\setlength{\biblabelsep}{0pt} % No space for these labels

\makeindex

\title{Notes about Classical Chinese}
\author{Jinyuan Wu}

\begin{document}

\automath

\maketitle

\chapter{Introduction}

\section{The name of the language}

This note is about Classical Chinese,
the high variety of more than two millennia of diglossia in China.
The language is known natively (in Mandarin Chinese) as 文言 (\translate{lit. cultured speech})
or sometimes 古文 (\translate{lit. ancient articles})
or 古汉语 (\translate{lit. ancient Chinese}).
Note that there were several stages of the development of Chinese
and Classical Chinese is mostly (but not completely) based on Old Chinese
(\prettyref{sec:introduction.history}).

The language is sometimes known as \form{Wen-li} by Western missionaries,
especially in Bible translation.
This seems to be a misunderstanding of the word 文理,
which is a nominal compound and means rhetorics (i.e. 文) and meanings (i.e. 理) of literature works.
An educated person therefore would be described as ``通文理'' (\translate{fluent in rhetorics and meanings}).
Such a person of course would have decent understanding of Classical Chinese,
and hence 文理 was probably mistranslated as ``Classical Chinese'',
although the word was not natively used to refer to the latter.

\section{Historical background}\label{sec:introduction.history}

Since there was no attempt at explicit and systematic grammatical standardization
(\prettyref{sec:introduction.previous.tradition}),
prescriptive authority of Classical Chinese was a collection of canonical literature works
consensually regarded as classical (\prettyref{sec:introduction.text}).
The whole canon was finished before the collapse of Han
and therefore falls under the term Old Chinese.
Both temporal and regional variances can be observed in Old Chinese texts, though,
and not all varieties contribute to Classical Chinese equally.
In this section, we briefly overview the history of Sinitic language(s)
and analyze how they shape Classical Chinese.

\subsection{Pre-classical period}

The earliest attested Sinitic texts were oracle bone inscriptions,
a 20th century archeological re-discovery not known to Classical Chinese authors.
For them, the earliest available texts are 
documents preserved in 《尚书》 (lit. \translate{venerated documents}),
often known as \work{Book of Documents} in English.
Since these texts are from ancient kings 
whose deeds were romanticized by Confucian scholars,
these texts were highly venerated and yet deemed as
诘屈聱牙 (\translate{twisted, hard to pronounce})
by post-Classical authors.%
\footnote{
    For example by Han Yu in 《进学解》 (\work{Analysis of academic advancement}).
}
They were something that had to be read with commentaries,
the latter written in easier Classical Chinese.
These documents therefore should be regarded as pre-Classical,
although they did contribute sporadic phrases
and grammatical words (e.g. the copula 惟 or the pronoun 厥)
that were occasionally used in Classical Chinese works as a way to polish an article.

One thing worth mentioning is that 
the language of the \work{Book of Documents} and the language of oracle bone inscriptions are not identical.
The most notable fact on this aspect is that
the aforementioned pronoun 厥 appears frequently in the \work{Documents},
but it appears neither in oracle bone inscriptions nor in Spring and Autumn works. 
Possibly, \work{Book of Documents} contains predominantly early Zhou dynasty texts,
while oracle bones dates back to Shang,
and the differences we are observing reflect dialectal differences between the ruling classes of the two dynasties.

Another fairly early source is 《诗经》 (lit. \translate{poem classics}),
also known as \work{Book of Odes},
which contains poems dates back to as early as early Zhou.

\subsection{Spring and Autumn and Warring States}

The majority of texts that shaped Classical Chinese prose
were written in a time when Zhou was already substantially weakened.
This period that witnessed prolificacy of Old Chinese works
can be divided into two periods:
the Spring and Autumn period which was filled with chaotic (but not intense) wars between numerous dukedoms,
and the Warring States period which observed intense wars between seven major states,
resulting in a unified Qin empire,
which soon broke down because of resistances to its barbaric policies 
and eventually was superseded by Han dynasty (\prettyref{sec:introduction.history.han}).
The language of this period diverges tremendously from the pre-Classical period.
For example, the copula 惟 had died out of use 
and the copula construction had been largely replaced by the nominal predication construction
(\prettyref{sec:grammatical.clause.nominal}).
The conjunction 而 is never seen in pre-Sprint and Autumn texts
but had already made its way into the \work{Analects}.
The lexicon also underwent huge changes.

\begin{todobox}{Lexicon change}{lexicon-change-oc}
    List some lexicon changes.
\end{todobox}

There are clues suggesting regional variances.
Students of Confucius noticed that when he recited Classical texts and presided rituals,
he used 雅言 or \translate{elegant speech} (\work{Analects} 7:18).
This suggests a possible diglossia at as early as Confucius's own age,
with the ``elegant speech'' conceivably being the language of intellectuals of Zhou Dynasty.
Comparison between the language of Classical proses and the language(s) of poetry
shows the relative homogeneity of the former,
while the latter both demonstrate divergence from the language of the proses
and regional differences.

\begin{todobox}{Peotry and prose}{peotry-and-prose-before-han}
    This is presumably due to how the texts were transmitted.
    It is likely that they were passed by recitation,
    and regularization happened to proses when there was a predominant dialect,
    while the prosody and rhyme structures of poems
    efficiently locked them to their original forms.
\end{todobox}

The language of 楚辞 (\work{Verses of Chu}), for example,
has the following differences with the language of the proses. 
The first is a Kra–Dai substrate.

\begin{todobox}{Chu dialect}{chu-dialect}
    Find references.
\end{todobox}

The language of the \work{Odes} also seems to slightly deviates from 

Dialectal differences have also been observed within the \work{Odes} \citep{list2017vowel}.



\subsection{Han dynasty}\label{sec:introduction.history.han}

The last batch of uncontroversially classical works were composed during Han dynasty,
among them the most important being \work{Records of the Grand Historian}.
The language of \form{Records of the Grand Historian} shows notable but largely qualitative differences
compared with earlier historical works,
the most important one being 《左传》.
Notable changes include more pre-verbal adverbials,
reduction of prepositional verbs,
regularization of constituent orders,
and also proliferation of disyllable words
It is therefore suggested that Han dynasty texts and pre-Qin texts 
reflect two stages of post-Zhou developments of Chinese,
although the change was definitely not as radical as the change 
from the \form{Documents} to Spring and Autumn texts 
\citep[\citepages{260-264}]{he2005shiji}.

\subsection{Post-Classical periods}

The end of Old Chinese -- and hence the end of the classical period --
is marked by the collapse of the personal pronoun system,
the emergence of 是 as a copula (and not just a demonstrative),
the appearance of the disposal construction (i.e. the 把 construction)
and the so-called long passive construction.

\begin{todobox}{References for Middle Chinese and modern Mandarin}{middle-chinese-ref}
    \begin{itemize}
        \item James Huang
        \item etc.
    \end{itemize}
\end{todobox}

Expectedly, despite purification attempts,
vernacular elements made their ways into not only administrative documents
but also pure literature and scholar works.
Classical Chinese or 文言, in the broadest sense,
is a term that covers all genres whose grammars are roughly based on the Old Chinese canon
but may have a varieties of innovations.

\begin{todobox}{Late regularization attempts}{hanyu-etc-regularization}
    韩愈、因明学
\end{todobox}

\section{Texts}\label{sec:introduction.text}

The great historical work 《史记》 (\translate{lit. historical records}),
often known as \form{Records of the Grand Historian} in English
(a translation of 太史公记, the earliest known title of the work),
laid the paradigm of official historiography of all Chinese dynasties after Han.
The author 司马迁 \form{Sima Qian} is known as the \form{Lord Grand Historian} or 太史公.
太史 \translate{grand historian} was the title of



\section{Previous studies}

\paragraph*{Grammatical traditions}\label{sec:introduction.previous.tradition}
Classical Chinese authors had conversations about grammaticality 
and uses of grammatical particles
reminiscent of how English native speakers
with some exposure to the study of English grammar but no formal training:
``delete the \form{the} here and your sentence looks more concise''.
No attempts were made to establish intermediate concepts between words and utterances,
like structural templates of phrases or grammatical relations, 
and to organize the grammar as a machine producing acceptable utterances:
discussions on grammatical topics were either for education or for rhetorics.

The grammatical awareness of Classical Chinese authors was somehow comparable to 
what an ancient Roman grammarian or \form{grammaticus} did,
who set his main role as a secondary educator,
refrained from analyzing some sort of ``underlying'' or ``internalized'' system behind the surface forms
and was satisfied by mostly surface-oriented patterns,
and would not set up any intermediate concepts between the word and the utterance
\citep[\citepages{7,35,47-48}]{matthews2019graeco}.
On the other hand, this approach is contrary to the practice
of the Paninian Sanskrit grammatical tradition,
which, in today's terminology, starts with dependency relations and abstract features
and uses a set of morphophonological rewriting rules to produce the corresponding surface forms
\citep{kiparsky2009architecture}.%
\footnote{
    The main difference between Pāṇini's treatment of Sanskrit and modern linguistic theories
    is that Pāṇini apparently treats all dependency relations equally
    and there is, for example, no concept of the pivot or the ``external argument'' of a clause.
    This is however modified in the commentaries of his \work{Aṣṭādhyāyī},
    which explicitly allows an argument being promoted to the agent position 
    because of the intentions of the speaker \citep{keidan2017subjecthood}.
    The Paninian tradition therefore is extremely close to modern linguistic description practice;
    the most important difference probably is that
    modern linguistic description, practically, may even be less rigorous than \work{Aṣṭādhyāyī},
    because of possible competing ``mind grammars'' among speakers with mutual intelligibility
    or even within the mind of one speaker,
    and also the fact that a description as detailed as \work{Aṣṭādhyāyī}
    requires corpus data whose quality and quantity exceed the capacity of most field linguists.
}

The Classical Chinese grammatical tradition appears even looser compared with the Roman tradition
in that the former did not even attempt to recognize parts of speech;
this however was deeply rooted in the structure of Classical Chinese
in that 

\begin{todobox}{Ancient Chinese grammatical tradition and Roman tradition}{china-rome-compare}
    Is the situation somehow close to what a Roman grammarian (\form{grammaticus}) would do?
    It seems that Roman grammarians also didn't care about abstract structures.
    See:
    \begin{itemize}
        \item Use and Function of Grammatical Examples in Roman Grammarians
        \item Quintilian’s ‘Grammar’ (Inst.1.4-8) and its Importance for the History of Roman Grammar
        \item What Graeco-Roman Grammar was about
    \end{itemize}
\end{todobox}

On the other hand, phonology was an active topic in ancient China.
This was probably due to the influence of 

\paragraph*{Perspectives of European missionaries}

\paragraph*{Modern descriptions}

\section{Remarkable features}

Classical Chinese has several notable typological features.

\begin{todobox}{Remarkable features}{remarkable-features}
    \begin{itemize}
        \item Part of speech
        \item Topic-comment
        \item ``Coverb'', or is there real preposition
        \item Prosody (and hence a chapter on phonology and writing system)
        \item The chapter on phonology and writing system can be very hard:
            lots of historical facts
        \item Passivization and so on
        \item Higher tolerance of ad-hoc recategorization: 名作动, 使动意动, etc.
    \end{itemize}
\end{todobox}

\chapter{Grammatical overview}

\section{Parts of speech}\label{sec:grammatical.pos}

Concepts like noun-hood and verb-hood are clearly definable in Classical Chinese
if the two are understood as bundles of grammatical properties.
Therefore a noun is what appears at the center of a \ac{np},
and a verb is what appears at the center of a clause
(e.g. the distinction between nominal predication and verbal predication 
in \prettyref{sec:grammatical.clause.nominal.distinction}),
and they are not quite different from those of other natural languages.

Besides the syntactic constructions, 
a language also has a \emph{lexicon} that dictates 
the details of whether and how a root or derived stem or a larger construction appears
in certain syntactic environments and be phonologically realized.
In linguistic descriptions, therefore,
part of speech tags like \term{noun} or \term{verb} are \emph{lexical} labels 
representing the structure of the lexicon,
and how parts of speech are demarcated often shows considerable cross-linguistic variance.%
\footnote{
    For example, to say ``the Latin word \form{canis} is a noun'' 
    means to say that the form \form{canis} usually appear as the head of an \ac{np},
    that it carries an inherent gender feature and a number feature,
    and that its inflection pattern follows one of Latin nominal declensions.
    Modern English does not have rich inflectional morphology
    but does have nominal modification constructions (e.g. \form{a [dog]_{\text{nominal (not \ac{np})}} tag}),
    so saying that \form{dog} is a noun means something different with 
    saying that \form{canis} is a noun.

    We note that \form{canis} can be further analyzed as a root plus an ending.
    The Latin lexeme \form{canis} is actually a bundle of the root 
    \form{cane-}, the masculine gender, a case feature (here nominative),
    a number feature (here singular),
    and the fact that it is the head of some complete \ac{np}.
    On the other hand, the root \form{cane-} appearing as the main verb of a clause is impossible,
    because a bundle of the root \form{cane-} plus some verbal features
    is \emph{not} in the mental dictionary of a Roman.
    Nominal attributes are not possible in Latin,
    again because the mental dictionary of Romans does not contain anything like
    the root \form{cane-} without the head status of a \ac{np}.

    The Latin form class \term{noun}, then,
    means the bundle ``a gender feature, a case feature, a number feature, and the head-of-\ac{np} status''
    plus how it is morphophonologically realized (i.e. the five declensions).
    The English concept of \term{noun} is much different.
    Indeed, if we accept the hypothesis that abstract principles of language structures 
    are more or less the same cross-linguistically,
    then the lexicon \emph{has to} be highly diverse across languages
    because it is exactly the locus of language variance, besides morphophonology.
}

Classical Chinese has no inflectional morphology for content words 
so the morphophonological part is moot. 
The content words also show much more flexibility 
in their distributions in various syntactic environments,
sometimes without any formal indications.
These facts lead some to claim that Classical Chinese 
is a language without clear part of speech distinctions,
so although we can talk about the nominal or verbal usage of a root or a compound,
strictly speaking we cannot talk about nouns or verbs,
as there are no inherent lexical properties attached to roots
that dictate their nominal or verbal usages. 
A more careful analysis, though, seems to reveal that
at least some part of speech distinctions 
can be maintained in Classical Chinese.

\subsection{Nouns and verbs}

A noun-verb distinction is supported by carefully examining traditionally called noun-used-as-verb phenomena
(\prettyref{sec:pos.verb.noun-to-verb}).
If the lexicon of Classical Chinese contains \emph{only} non-categorized roots,
the interpretation of verbal usages of a word that usually appears in nominal environments
should vary rather freely.
What is actually attested however is not different from similar phenomena in other languages.
In some cases, it seems a root is first categorized as a noun 
and then undergoes something similar to English \form{-ize} (albeit without any explicit marking),
so only the nominal usage needs to be recorded as a lexical entry,
but the lexicon controls whether a derivation step is viable.
In other cases, 
both the nominal and verbal usages are to be recorded in the lexicon,
as they cannot be inferred regularly from each other.
In both cases, how a root is possibly categorized is stored in the lexicon,
meaning that calling the nominal use of a root a \term{noun} and the verbal use of a root a \term{verb}
is not problematic at all even in Classical Chinese.
Sporadic ad hoc re-categorization of roots does exist,
but this does not support the idea that part of speech division does not exist at all in the lexicon.

A terminological caveat is what appears as an argument is not necessarily a \ac{np}:
it can be a complement clause.
The main verb of a complement clause is not in a nominal position.
Some may call complement clauses ``nominal clauses'',
but this is misleading as the internal structure of a complement clause is not the same as that of a \ac{np}.

\subsection{The adjective class}

An adjective class can also be established in Classical Chinese,
although its behavior is strongly verbal. 

A caveat, similar to the caveat that an argument is not necessarily a \ac{np},
is that an attributive phrase is not always an adjective phrase.
In existing modern studies, statements like ``a verb used as an adjective'' is usually avoided:
wordings like ``something is used as an attributive'' are adopted instead.

\begin{todobox}{Traditional grammars}{traditional-grammar-list}
    List some Classical Chinese grammars in which 动词作形容词 etc. never appear.
\end{todobox}

\begin{todobox}{A comprehensive list of Classical Chinese parts of speech}{pos-list}
    Noun, verb, adjective: any other content words?
\end{todobox}

\section{The overall clausal structure}\label{sec:grammatical.clause}

Like all other languages, a Classical Chinese clause can be a simple clause
or a complex one constructed from subordination (\prettyref{sec:grammatical.clause.linking})
and coordination (\prettyref{sec:grammatical.clause.coordination}).
A simple Classical Chinese clause can be divided into a nucleus clause (\prettyref{sec:grammatical.clause.nominal}, \prettyref{sec:grammatical.clause.verbal})
plus discourse-related devices,
including its speech act (\prettyref{sec:grammatical.clause.force}) marked by sentence final particles (\prettyref{sec:grammatical.clause.sfp}),
and topicalization or focalization (\prettyref{sec:grammatical.clause.information}).
Topicalization can also happen for a complex clause (\prettyref{sec:grammatical.clause.coordination.topic-chain}).
It appears that all embedded clauses in Classical Chinese cannot have discourse-related devices 
like topicalization and sentence final particles.

The nucleus clause may be either a nominal predicate clause (\prettyref{sec:grammatical.clause.nominal})
or a verbal clause (\prettyref{sec:grammatical.clause.verbal}).
Both constructions seem to have a well-defined subject position (which is not the same as the topic),
which however is often left blank (\prettyref{sec:grammatical.verbal.subject}).
\Ac{tam} modifications (TODO: ref) seem to be only available for the verbal clause.

\begin{todobox}{Clause types}{clause-types}
    In \citet[\citepage{131}]{meiguang2018},
    he classifies clauses into 说明句, 描写句, and 叙事句.
    The classification is comparable to that given in  http://area.hcjh.tn.edu.tw/noise/hcjh-ca/4-b.htm\#0303 .
    Mei doesn't mention on which basis he makes this distinction.
    In the latter source, it seems the distinction is made based on the type of the predicate.
    Thus a 描写句 is a stative (adjectival?) clause,
    and a 判断句 is a nominal predicate construction,
    and a 叙事句 is a verbal predicate construction that is not a 描写句.
    So what does Mei mean by 说明句?
    The term appears in \citet{li2004grammar} as well.


    We can go to places where he mentions the term.
    \citepage{445}: 矣 is for 叙事, and 也 is for 说明.
    \citepages{264-265}: 事件句 (叙事句) 和非事件句 (描写句和说明句)
    The distinction is also mentioned in http://paper.wenweipo.com/2018/02/14/ED1802140024.htm
    
    So it's related to the event structure.
    We need to know where the event structure resides in the vP-TP-CP hierarchy.
    Particularly, we need to identify \emph{where} the category of this distinction lies.
    I think probably that's related to the aspect:
    consider the distinction between a habitual clause and a prototypical ``event'' clause.
    
    The distinction has syntactic significances.
    We note that certain topicalization constructions seem to be only compatible with 说明句
    (\prettyref{box:a-zhiyu-b}).
\end{todobox}

\subsection{Nominal predication}\label{sec:grammatical.clause.nominal}

\subsubsection{Real nominal predicates}\label{sec:grammatical.clause.nominal.real}

The top-level structure of a Classical Chinese clause may contain 
a (optional) subject and a \ac{np} acting as the predicate
(\ref{ex:grammatical.clause.nominal.isa.1},
\ref{ex:grammatical.clause.nominal.havea.1}).
A nominal predicate may express an ``is-a'' relation between the subject (see \prettyref{sec:grammatical.verbal.subject.clause-pivot} for discussions on the meaning of the term) and the predicate,
which is the case of (\ref{ex:grammatical.clause.nominal.isa.1}).
Some nominal clauses however express a possessive relation between the two 
(\ref{ex:grammatical.clause.nominal.havea.1}).

\begin{exe}
    \ex\label{ex:grammatical.clause.nominal.isa.1} 
    \gll [秦]_{\text{subject}}, [虎 狼 之 国]_{\text{predicate}} \\
    Qin tiger wolf \category{gen} country \\
    \glt\translate{Qin is a country of tigers and wolves (i.e. cruel and not reliable).} 

    \ex\label{ex:grammatical.clause.nominal.havea.1} 
    \gll [蟹]_{\text{subject}} [六 跪 而 二 螯]_{\text{predicate}} \\
    crab six leg \category{conj} two claw \\
    \glt\translate{A crab has six legs and two claws.}
\end{exe}

\begin{todobox}{The possessive nominal predicate construction}{possessive-nominal}
    It seems the predicate in the possessive nominal predicate construction
    can never be a bare noun without any modification.
    The modification can be a numeral or an attributive.

    \begin{exe}
        \ex 王六军,大国三军
        \ex 秦王[为人]_{\text{\prettyref{box:a-zhiyu-b}}},蜂准,长目,挚鸟膺,豺声,少恩而虎狼心
    \end{exe}
    
    Another problem is that the 者-也 construction seems to be incompatible with the possessive nominal predicate.
\end{todobox}

Negation in Classical Chinese nominal clauses is usually expressed by 非,
placed before the nominal predicate (\ref{ex:grammatical.clause.nominal.is-not.1}).

\begin{exe}
    \ex\label{ex:grammatical.clause.nominal.is-not.1} 
    凡群臣之言事秦者,皆奸人,非忠臣也
\end{exe}

It seems besides the negation marker,
no other constituents are allowed to appear in the nominal predicate construction.

\paragraph*{Topicalization of nominal predicate construction}
\label{sec:grammatical.clause.nominal.real.judgement}

(\ref{ex:grammatical.clause.nominal.isa.1}) is much less frequent than the 者…也 construction,
often known as 判断句 in Modern Chinese or the \translate{judgemental clause}.
A judgemental clause usually contains a particle 也 (\prettyref{sec:grammatical.clause.sfp})
at its end (\ref{ex:grammatical.clause.nominal.isa-topic.1}),
or a particle 者 after the subject (\ref{ex:grammatical.clause.nominal.isa-topic.2}), or both.
It seems that the judgemental clause is better analyzed as a topic-comment construction
(\prettyref{sec:grammatical.clause.topic}).

\begin{exe}
    \ex\label{ex:grammatical.clause.nominal.isa-topic.1} 
    \gll [城 北 徐-公]_{\text{topic: \ac{np}}}, [齐-国 之 美-丽 者]_{\text{comment: \prettyref{sec:grammatical.noun-phrase.determinative-relative}}} 也 \\
    city north \category{name}-\category{gong} Qi-country \category{gen} beautiful-beautiful \category{rel} \category{sfp} \\
    \glt\translate{Mr. Xu from the north of the city is a handsome guy in the country Qi.}

    \ex\label{ex:grammatical.clause.nominal.isa-topic.2} 
    \gll [兵]_{\text{topic: \ac{np}}} 者, [不 祥 之 器]_{\text{comment, predicate}} \\
    weapon \category{topic} \category{neg} fortunate \category{gen} instrument \\
    \glt\translate{Weapons are not auspicious.}
\end{exe}


\subsubsection{Distinction between a nominal clause and a verbal clause}
\label{sec:grammatical.clause.nominal.distinction}

Note that the term \term{nominal} in \term{nominal predication} or \term{nominal clause}
refers to the fact that the predicate is structurally a \ac{np},
not whether the head of the predicate usually appears like a noun or a verb in a dictionary
(\prettyref{sec:grammatical.pos}).
In some sentences although the predicate of a clause mostly appears as the head of a \ac{np} 
and therefore may be referred to as a noun in dictionaries,
the clause is clearly a verbal clause
because it expresses a dynamic event and not just a state,
the possibility of \ac{tam} markers, etc.,
as in (\ref{ex:grammatical.clause.nominal.noun-to-verb.1}).
Here 水 \translate{water} is used as a verb, meaning \translate{swim},
which is also modified by the modality auxiliary 能 \translate{can}.

\begin{exe}
    \ex\label{ex:grammatical.clause.nominal.noun-to-verb.1} 
    \gll [假 舟 楫 者]_{\text{subject}}, [非 能 水 也]_{\text{predicate: VP}}…… \\
    draw.help boat paddle \category{rel} \category{neg} can swim \category{sfp} \\
    \glt\translate{Those who draw help from boats and paddles cannot swim, \dots} 
\end{exe}

There are cases where the meaning of the predicate is comparable to that of a real nominal predicate.
We still classify them as verbal clauses,
because of their similarity with prototypical verbal clauses
with respect to negation, \ac{tam} modification, TODO

\begin{exe}
    \ex 大楚兴,陈胜王
    \ex 然而不王者,未之有也
\end{exe}

On the other hand, there is one thing a nominal predicate can do
while a verbal predicate \emph{cannot} do:
a nominal predicate can be topicalized 
(\prettyref{sec:grammatical.clause.topic}, \ref{ex:grammatical.clause.topic.predicate.1}).

\subsubsection{Copula constructions} 

All the constructions mentioned above are without a copula.
In the pre-Classical copula age there is a copula 惟,
which however had largely died out of use in Classical texts.
Meanwhile, grammaticalization had added several copulas to Classical Chinese
\citep[\citepages{20-22}]{pulleyblank1995outline}.

\subsection{Verbal predication}\label{sec:grammatical.clause.verbal}

The structure of clauses with verbal predicates is much more complicated,
and the details can only be described in the following sections.
In this section we overview grammatical systems within verbal clauses.

\subsubsection{Linear template}\label{sec:grammatical.clause.verbal.linear}

\paragraph{Constituents and ordering}

In clauses with verbal predicates,
the constituent order of core constituents of transitive clauses is almost always SVO
(\ref{ex:grammatical.clause.svo.declarative.1}, \ref{ex:grammatical.clause.svo.interrogative.1}).
Intransitive clauses have a SV constituent order
(\ref{ex:grammatical.clause.svo.intransitive.1}).
The usage of the term \term{subject} is justified in \prettyref{sec:grammatical.verbal.subject},
and the contents of a verbal clause besides the subject
is often defined as the \ac{vp}.
\Acp{vp} can be coordinated (\prettyref{sec:grammatical.clause.coordination}).
Prepositional complements are also placed after the verb
(\ref{ex:grammatical.clause.svo.pp-declarative.1}).
The term \term{object}, without specification, means any argument in the \ac{vp} that is not marked by a preposition (\prettyref{sec:grammatical.verbal.argument.prepositional}).

\begin{exe}
    \ex\label{ex:grammatical.clause.svo.declarative.1} 
    [子张]_{\text{subject}} [[学]_{\text{verb}} [干禄]_{\text{object}}]_{\text{predicate: VP}}

    \ex\label{ex:grammatical.clause.svo.interrogative.1} 
    [子]_{\text{subject}} [奚]_{\text{reason}} 不 [为]_{\text{verb}} [政]_{\text{object}}

    \ex\label{ex:grammatical.clause.svo.intransitive.1}
    君子不器

    \ex\label{ex:grammatical.clause.svo.pp-declarative.1}
    君子博学于文
\end{exe}

SOV is however attested in negative (\ref{ex:grammatical.clause.sov.neg.1})
or interrogative situations (\ref{ex:grammatical.clause.sov.interrogative.1}).

\begin{exe}
    \ex\label{ex:grammatical.clause.sov.neg.1}
    恐 [年岁 之 [不吾与]_{\text{VP: Neg-OV}}]_{\text{complement clause}}
    
    \ex\label{ex:grammatical.clause.sov.interrogative.1} 
    以五十步笑百步,则 [何如]_{\text{SOV clause}}
\end{exe}

\paragraph{The structure of the verb}
It is possible that the main verb of a verbal clause contains more than one root.
Such a verb is known as a complex predicate.

\begin{todobox}{Classical Chinese complex predicate}{cp}
    Directional complement and resultative complement
\end{todobox}

\paragraph{Positions of modifiers}
Adverbial constituents in the nucleus can be divided into \ac{tam} ones 
and so-called peripheral arguments, including location, manner, instrument, etc.
The peripheral arguments can be post-verbal
(\ref{ex:grammatical.clause.peripheral.postverbal.1},
\ref{ex:grammatical.clause.peripheral.postverbal.2},
\ref{ex:grammatical.clause.peripheral.postverbal.3})
or pre-verbal
(\ref{ex:grammatical.clause.peripheral.preverbal.1},
\ref{ex:grammatical.clause.peripheral.preverbal.2},
\ref{ex:grammatical.clause.peripheral.preverbal.3}),
with the pre-verbal order gaining popularity as time went by.
The linear order of peripheral arguments is similar to that in Mandarin
\citep[\citepages{286-287}]{he2005shiji}.

\begin{exe}
    \ex\label{ex:grammatical.clause.peripheral.postverbal.1} 侍饮于长者
    \ex\label{ex:grammatical.clause.peripheral.postverbal.2} 孟孙问孝于我
    \ex\label{ex:grammatical.clause.peripheral.postverbal.3} 祷尔于上下神祇
    \ex\label{ex:grammatical.clause.peripheral.preverbal.1} 韩生南向坐
    \ex\label{ex:grammatical.clause.peripheral.preverbal.2} 於人之罪无所忘
    \ex\label{ex:grammatical.clause.peripheral.preverbal.3} 为人谋而不忠乎
\end{exe}

The \ac{tam} adverbials are almost always preverbal.

\begin{exe}
    \ex 文王既没,文不在兹乎
    \ex 孔子既得合葬于防
    \ex 我未之能易也
\end{exe}

When \ac{tam} adverbs and peripheral arguments both appear before the verb,
the order is always \ac{tam} \before peripheral argument.
The reverse order is never attested.
The whole \ac{vp} therefore can be analyzed as a core \ac{vp}
plus peripheral arguments surrounding it,
plus \ac{tam} adverbs preceding the pre-verbal peripheral arguments.
The clause then is the complete \ac{vp} plus the subject.

\begin{exe}
    \ex 三王 [既]_{\text{\ac{tam}}} [以]_{\text{instrument}} [定法度]_{\text{VO}}
\end{exe}

\begin{todobox}{Adverbials combination}{adverbial-combine}
    Is it possible to use multiple pre-verbal peripheral adverbials?
    What's the relevant order constraint?
\end{todobox}

\begin{todobox}{Position of adverbials in SOV case}{adverbial-sov}
    Where to place adverbials in SOV case?
\end{todobox}

\begin{todobox}{Position of negator}{negative-template}
    Where is the position of the negator?
\end{todobox}



\paragraph{Sentence final particles}\label{sec:grammatical.clause.sfp}

Classical sentence final particles have a variety of functions.
It may mark the interrogative force (\ref{ex:grammatical.clause.sfp.interrogative.1}), 
a judgemental meaning (\ref{ex:grammatical.clause.sfp.judgement.1}),
and aspectual values (\ref{ex:grammatical.clause.sfp.aspectual.1}).

\begin{exe}
    \ex\label{ex:grammatical.clause.sfp.interrogative.1} 大车无輗,小车无軏,其何以行之哉
    \ex\label{ex:grammatical.clause.sfp.judgement.1} 人而无信,不知其可也
    \ex\label{ex:grammatical.clause.sfp.aspectual.1} 温故而知新,可以为师矣
\end{exe}

It seems a sentence final particle can be shared by two conjuncts.

\begin{exe}
    \ex\label{ex:grammatical.clause.sfp.judgement.2} 虎者,戾虫;人者,甘饵也
\end{exe}

\subsubsection{Argument structures}\label{sec:grammatical.clause.verbal.argument}

\paragraph{\category{do}, \category{be} and \category{become}}
Consistent with cross-linguistic generalizations, in Classical Chinese,
a verbal clause can be about an intentionally initiated event (\category{do}; \prettyref{sec:valency.simple.do}),
a state (\category{be}) or a change of the state (\category{become}; \prettyref{sec:valency.simple.state-and-change}).
The \category{do} type can further be divided into the transitive and intransitive classes.
\category{be} and \category{become} clauses are intransitive by definition.
The distinction between the three classes has consequences for
animacy and volition of the subject (see the sections referred above)
as well as the viability of certain grammatical processes (\prettyref{sec:valency.simple.do.properties}).



\paragraph{Prepositional arguments and applicative constructions}\label{sec:grammatical.verbal.argument.prepositional}
Prepositional arguments can also be observed in Classical Chinese.
In (\ref{ex:grammatical.clause.verbal.argument.prepositional.1}),
for example, the prepositional phrase 于车 is the \term{source} of the event.
A prepositional argument can also appear in a transitive construction,
coding a wide varieties of semantic roles,
like the recipient (\ref{ex:grammatical.clause.verbal.argument.prepositional.give.1}, \ref{ex:grammatical.clause.verbal.argument.prepositional.give.2}),
or the target of a question (\ref{ex:grammatical.clause.verbal.argument.prepositional.ask.1}).
It is not possible for a prepositional argument to appear before the object.
Classical Chinese also does not have quirky subjects:
it is not possible for a prepositional phrase to appear in the subject position.

\begin{exe}
    \ex\label{ex:grammatical.clause.verbal.argument.prepositional.1} 
    \gll 公 惧, 队 于 车 \\
    king afraid fall from carriage \\
    \glt\translate{The king was afraid and fell from the carriage.}

    \ex\label{ex:grammatical.clause.verbal.argument.prepositional.give.1}  成王、康王……故赐之以重祭

    \ex\label{ex:grammatical.clause.verbal.argument.prepositional.give.2} 秦复予我河外及封陵为和

    \ex\label{ex:grammatical.clause.verbal.argument.prepositional.ask.1} 季康子问政于孔子
\end{exe}

Classical Chinese has applicative constructions that
turn an argument structure containing a prepositional argument into a double object construction
(\ref{ex:grammatical.clause.verbal.argument.prepositional.ask.double-object.2}).
In this case, the argument corresponding to the prepositional argument in (\ref{ex:grammatical.clause.verbal.argument.prepositional.ask.1})
behaves like the monotransitive object in constituent orders and in valency decreasing
(\prettyref{sec:grammatical.verbal.subject.argument-structure.alternation}, \prettyref{sec:grammatical.clause.verbal.argument-structure.pseudo-passive.multiple-argument}).

\begin{exe}
    \ex\label{ex:grammatical.clause.verbal.argument.prepositional.ask.double-object.2} 上问上林尉诸禽兽簿
\end{exe}

In some languages, the argument structure of verbs meaning giving and receiving
seems to contain a small clause.
In English, for instance, we have \form{give this to him and that to her},
and in Latin we even have standalone small clauses like \form{Deo gratias}.
No trace of small clauses is found in Classical Chinese:
double object clauses with giving or receiving meanings seem to be analyzable as applicative clauses
\citep[\citepages{416-421}]{meiguang2018}.

\paragraph{Valency alternation}
Various valency increasing constructions, 
for example causative constructions, can be applied to them,
and create ``transitive'' constructions
(e.g. various \category{cause} constructions in \prettyref{sec:grammatical.clause.verbal.argument-structure.causative})
that are similar to but subtly different from transitive \category{do} ones.
And then we have valency decreasing constructions, which \emph{suppress} the subject 
and promote an internal argument (\prettyref{sec:grammatical.verbal.subject.argument-structure.alternation}) to the subject position (\prettyref{sec:grammatical.clause.verbal.argument-structure.passive}).
Valency increasing after valency decreasing is also possible:
(\ref{ex:grammatical.clause.verbal.argument.valency-alternation.1})
is an example of a causative clause based on the pseudo-passive construction.

\begin{exe}
    \ex\label{ex:grammatical.clause.verbal.argument.valency-alternation.1} …… 杀御叔 (=\ref{ex:grammatical.clause.verbal.argument.causative.synthetic.1} in \prettyref{sec:grammatical.clause.verbal.argument-structure.causative.synthetic})
\end{exe}

Classical Chinese seems to already have a prototype of what is later known as the disposal construction or the 把 construction in later Sinitic languages.

\begin{exe}
    \ex 尽以其宝器赂献于周厘王
\end{exe}

\paragraph{Complement clause constructions}

\begin{todobox}{Other verb frames}{verb-frame}
    Control construction, etc.
\end{todobox}

\paragraph{Argument structure and verb classes}
The only fundamental constraint to whether a stem appears in a transitive \category{do} or intransitive \category{do} or \category{become} construction is its semantics.
In world languages, however, whether a root or a stem is compatible with a certain verb frame
is dictated by the lexicon of the language,
and a group of root-environment complexes with shared properties 
is known as a part of speech (\prettyref{sec:grammatical.pos}).
Thus we can say if a \emph{verb} (and not the clause it heads) is transitive or intransitive,
or whether it is an action verb (i.e. a \category{do} verb),
a stative verb (i.e. a \category{be} verb),
a internally caused change of state verb (i.e. a \category{become} verb)
or an externally caused change of state verb (i.e. a \category{cause} verb,
which may be \category{cause}-\category{become} or \category{cause}-\category{be}
or even without an intransitive counterpart).
Analysis of argument and event structures is closely related to verb classification.

\paragraph{High-level categories of the \ac{vp}}

Certain constituents within the \ac{vp} can appear before the main verb.
In (\ref{ex:grammatical.clause.verbal.argument.focus.1}),
the pre-verbal constituent 以其妹 seems to be a fronted prepositional argument.
It cannot be the sentential focus,
because the subject 季康子 seems to stay in-situ with no pause after it.
It is likely that a \ac{vp}-internal focus position exists in Classical Chinese,
which, cross-linguistically, is not rare (e.g. see \citet{danckaert2011left}).
A further piece of evidence suggesting the in-\ac{vp} analysis of (\ref{ex:grammatical.clause.verbal.argument.focus.1}) is that the fronted prepositional phrase may further undergo preposition-object inversion (\ref{ex:grammatical.clause.verbal.argument.focus.2}),
an operation that is otherwise not observed in Classical texts and also not motivated:
the best explanation of this inversion seems to be focalization
\citep[\citepage{323}]{meiguang2018}.

\begin{exe}
    \ex\label{ex:grammatical.clause.verbal.argument.focus.1} 季康子 [以其妹]_{\text{VP-focus: prepositional phrase}_i} 妻 之 ---_{i}
    \ex\label{ex:grammatical.clause.verbal.argument.focus.2} 室於怒而市於色
\end{exe}

Certain linear 

\begin{todobox}{VP internal structure}{vp-internal-structure}
    In-VP focalization, objecthood (why 问政于孔子, not 问于孔子政, considering we have 问孔子政?)
\end{todobox}

\subsubsection{Tense, aspect, modality, and things like that}\label{sec:grammatical.clause.verbal.tam}

Whether a clause is a \category{do} clause or a \category{become} clause or a \category{be} clause
is also related to the lexical aspect of the clause,
which in turn may have non-trivial interactions with \ac{tam} categories.



\subsubsection{The gerundive construction}\label{sec:grammatical.clause.verbal.gerundive}

Classical Chinese has one gerundive construction 
with a structure comparable to the English gerundive non-finite clause \form{his playing national anthem}.
This construction may appear as the object
(\ref{ex:grammatical.clause.verbal.gerundive.1})
or as a subordinated clause, like a conditional or temporal clause
(\ref{ex:grammatical.clause.verbal.gerundive.2}; \prettyref{sec:grammatical.clause.linking}).

\begin{exe}
    \ex\label{ex:grammatical.clause.verbal.gerundive.1} 
    \gll 王 如 知 此, 则 无 望 [民 之 多 于 邻 国]_{\text{object: gerundive}} 也 \\
    king if know this then \category{neg} hope people \category{gen} more than neighbor country \category{sfp} \\
    \glt\translate{If Your Majesty knows this, then don't expect your people to be more plentiful than your neighboring countries' people.}

    \ex\label{ex:grammatical.clause.verbal.gerundive.2}
    [父母之爱子]_{\text{condition: gerundive}},则为之计深远
\end{exe}





\subsection{Subjecthood}\label{sec:grammatical.verbal.subject}

In \prettyref{sec:grammatical.clause.nominal.real} and in \prettyref{sec:grammatical.clause.verbal.linear},
we both mention the concept of \term{subject},
which needs justification.
Further, the bipartite division of a verbal nucleus into a subject and a \ac{vp} in \prettyref{sec:grammatical.clause.verbal.linear} means that
the subject is in some senses \emph{external},
while other arguments within the \ac{vp} are \emph{internal}.
Cross-linguistically, it is not impossible for a language to demonstrate two types of \term{externality},
one based on argument structure properties like obligatory argument omission in control constructions
or binding of reflexive pronouns
(\prettyref{sec:grammatical.verbal.subject.argument-structure}),
another based on clausal pivot properties like subject sharing in coordination and relativization
(\prettyref{sec:grammatical.verbal.subject.clause-pivot}).
The two being different means syntactic ergativity,
which is rare among attested world languages \citep{aldridge2008generative},
and is absent in Classical Chinese.

Another problem is the distinction between subject and topic.
Since both the topic and the subject appear at the beginning of a clause,
the distinction between the two seems unclear.
We can even go as far as claiming that 
Classical Chinese has only information structure and no argument structure in its syntax
\citep[\citepage{122}]{meiguang2018}.
The matter is further complicated by the fact that Classical Chinese has no native speakers now
and detailed grammaticality tests are not available,
and that Classical Chinese is a pro-drop language 
so obligatoriness is not a viable criterion.
Still, we believe that the existing evidence is sufficient to justify postulating a subject grammatical function in Classical Chinese
besides the topic position, which is for information structure marking
(\prettyref{sec:grammatical.verbal.subject.topic-comparison}).

\subsubsection{Subjecthood in argument structure}\label{sec:grammatical.verbal.subject.argument-structure}

\paragraph{Obligatory relation between semantic roles and linear order}

We observe that 
the semantic relation between some clause-initial \ac{np}s and the verb 
is fixed by properties of the verb,
while the semantic relation between some clause-initial \ac{np}s and the verb is more flexible,
and these \acp{np} are related to some internal positions of the nucleus clause.
We therefore rightfully call the first type of clause-initial \acp{np} subjects,
and the second type of clause-initial \acp{np} topics.

(\ref{ex:grammatical.clause.subject.causative.1}), for example, contains three constituents,
and therefore can only be a verbal clause, with the last constituent being the object.
The verb 客 is a derivation from the noun 客.
Such a derivation, according to our experiences with other Classical Chinese texts,
can only be causative or tropative or benefactive.
By considering the context we will know the clause is a tropative one
and the right translation is \translate{Lord Mengchang considers me as a guest.}
Therefore, by virtue of being at the initial of the clause, 
the \ac{np} 孟尝君 \emph{has to} be understood as what initiates the event,
can be neither the patient nor peripheral roles in the event (e.g. an instrument).
The \emph{obligatory} relation between the verb and the \ac{np} 孟尝君 
clearly shows the latter is a subject, and not a topic.

\begin{exe}
    \ex\label{ex:grammatical.clause.subject.causative.1} 孟尝君客我 
\end{exe}

It should be noted that the subject can be the patient,
as in (\ref{ex:grammatical.clause.subject.passive.1}).
This however again is an \emph{obligatory} semantic relation
between the verb 定 and the \ac{np}s 国 and 天下.
By virtue of being the only argument of 定 and appearing before the verb,
国 and 天下 are obligatorily understood as the patients.
They cannot be understood as, say, the location of the event
(\translate{*Someone causes peace (i.e. 定) to something else in the state (国) and the universe (天下)}).
Clauses like (\ref{ex:grammatical.clause.subject.passive.1})
are therefore better analyzed as valency alternation constructions.

\begin{exe}
    \ex\label{ex:grammatical.clause.subject.passive.1} 国定而天下定
\end{exe}

\paragraph{Subjecthood in argument structure and valency alternation}
\label{sec:grammatical.verbal.subject.argument-structure.alternation}
The existence of a subject on the level of argument structure 
is relevant in valency alternation, as in
e.g. \prettyref{sec:grammatical.clause.verbal.argument.simple.non-conventional-state}
and \prettyref{sec:grammatical.clause.verbal.argument-structure.causative.synthetic}:
a structure with a subject is too ``big'' for certain operations.

An internal argument -- always an object, not a prepositional argument -- in a clause
may be promoted to the external subject position in another clause
(e.g. \prettyref{sec:valency.simple.state-and-change}).
It is possible for multiple objects to co-exist,
and as all double object constructions in Classical Chinese
seem to be related to one applicative construction or another
(\prettyref{sec:grammatical.verbal.argument.prepositional}),
the rule is that the object introduced by the applicative gets promoted to the subject position.
We may say that the object created by the applicative is the ``second most external'' argument.
This hierarchy of externality of arguments is not uncommon in world languages
(\prettyref{box:multiple-external-arguments}).

\begin{infobox}{Multiple external arguments?}{multiple-external-arguments}
    In Japhug, it is possible to have a \term{causer}\textto\term{instrument}\textto{agent}\textto{patient} argument structure,
    and the personal indexation marker seems to be decided by first taking the two most internal arguments and decide which is more salient on the empathy hierarchy,
    and then compare the result with the third most internal argument and decide which is more salient,
    and finally compare the result of the last step with the most external argument;
    hence 1\textto 3\textto 2\textto 3 is morphologically equivalent to 1\textto 2
    \citep[\citepage{310};\citepage{584},(116);\citepage{848},(67)]{jacques2021grammar}.
    
    In Classical Chinese, the synthetic causative construction 
    generally cannot be applied to an argument structure already with a subject (\prettyref{sec:grammatical.clause.verbal.argument-structure.causative.synthetic}),
    so structures like this are not possible.
    Yet as is seen above, a similar hierarchy can be built by the applicative.
\end{infobox}

\subsubsection{Subject as clausal pivot}\label{sec:grammatical.verbal.subject.clause-pivot}

In \prettyref{sec:grammatical.verbal.subject.argument-structure},
we see that subjecthood can be defined in the argument structure in verbal clauses.
Yet properties commonly attributed to subjecthood are not just about the argument structure.
For instance, in the nominal predicate construction (\prettyref{sec:grammatical.clause.nominal}),
we call the first \ac{np} the subject,
and there is no such thing as the argument structure there.
What we want to know is whether both verbal and nominal clauses in Classical Chinese
have a pivotal position in it which everything else ``revolves around''
which could be called the \term{subject}.

The most clear criterion that defines clausal pivotal subjecthood is probably coordination:
if when clauses are coordinated,
one constituent seems to be shared by all of them,
then this constituent is probably the clausal pivot.
It turns out that what is defined as the subject according to its behaviors in the argument structure
indeed is also the pivot in coordination
(\prettyref{sec:grammatical.clause.coordination},
\ref{ex:grammatical.clause.coordination.subject-vp.1}).
Note that two coordinated clauses can also share a topic,
but there are signs which tell us that what is shared is the topic and not the subject
(e.g. \prettyref{sec:grammatical.clause.coordination}, \ref{ex:grammatical.clause.coordination.topic.1}, where the topic is the object of the first clause and the subject of the second clause).

\begin{todobox}{Definition of VP}{nominative} 
    Can a verbal and a nominal predicate be coordinated? 
\end{todobox}

\begin{todobox}{Subject and \ac{tam}}{subject-tam}
    Subject and \ac{tam} in English are closely related. (e.g. control construction)
    What about Classical Chinese?
\end{todobox}

The observation that the argument structure subject in verbal clause
turns out to be the clausal pivot 
and that the TODO: nominal clause 
justify the usage of the term \term{subject} outlined in the beginning of this section.

\subsubsection{Comparison with topic}\label{sec:grammatical.verbal.subject.topic-comparison}

We have already argued that in every Classical Chinese clause,
there is a (possibly empty) subject position,
which is largely \emph{independent} to information structure factors
and therefore is not a topic.
On the other hand, authentic, information structure-related topics
are marked by devices not always available for subjects
(like the particle 者 or a pause; \prettyref{sec:grammatical.clause.topic}).
So indeed subject and topic are two distinct concepts in Classical Chinese.

This does not mean that there are no blurry cases. 





\subsection{Information packaging}\label{sec:grammatical.clause.information}

\subsubsection{Topicalization}\label{sec:grammatical.clause.topic}

Topicalization in Classical Chinese is usually marked by adding the particle 者 after the topic.
In the reading tradition, a pause is often inserted after 者,
which cross-linguistically suggests topicalization
(\ref{ex:grammatical.clause.topic.1}, \ref{ex:grammatical.clause.topic.2}).
In these examples, the subject is topicalized.
We note that the structure of these two examples is comparable to that of the ``judgemental clause''
(\prettyref{sec:grammatical.clause.nominal}),
which obliges us to analyze the judgemental clause as topicalization of the nominal predicate construction.

\begin{exe}
    \ex\label{ex:grammatical.clause.topic.1} 此二人者,实弑寡君
    \ex\label{ex:grammatical.clause.topic.2} 单父人吕公……吕公者,好相人,……
\end{exe}

What is topicalized is of course not restricted to the subject.
This fact is a piece of evidence supporting the distinction between subject and topic in Classical Chinese.
In (\ref{ex:grammatical.clause.topic.predicate.1}),
the comment clearly has a nominal predicate.
What is promoted to the topic position however is not the subject of the nominal predicate construction,
but the predicate. The subject is likely the \emph{focus} and not the topic
\citep[\citepage{138}]{meiguang2018}.
Topicalization of other clausal constituents is also possible
(e.g. \prettyref{sec:grammatical.clause.coordination}, \ref{ex:grammatical.clause.coordination.topic.1}).

\begin{exe}
    \ex\label{ex:grammatical.clause.topic.predicate.1} 
    [仁之实]_{\text{topic: NP_i}},[[事亲]_{\text{subject: ?}} [是]_{\text{predicate: pronoun_i}}]_{\text{comment}} 也 
\end{exe}

We also note that the 者…也… framework is not limited to topicalization of nominal predicate clauses
(i.e. ``judgemental clauses'').
For instance, we have (\ref{ex:grammatical.clause.topic.existential.1}),
in which the sole argument in a existential clause is topicalized,
and the comment receives 也 as its sentence final particle.

\begin{exe}
    \ex\label{ex:grammatical.clause.topic.existential.1} 然而不王者,未之有也
\end{exe}

We also note that topicalization can happen multiple times
(\ref{ex:grammatical.clause.topic.chain.1}).

\begin{exe}
    \ex\label{ex:grammatical.clause.topic.chain.1} 万乘之国,弑其君者,必千乘之家
\end{exe}

\begin{todobox}{Dangling topic}{dangling-topic}
    Are there dangling topics in Classical Chinese?
    If not, it's a another piece of evidence supporting the distinction between subject and topic.
\end{todobox}

\begin{todobox}{A之于B也}{a-zhiyu-b}
    \begin{exe}
        \ex 寡人之于国也,尽心焉耳矣
    \end{exe}

    \begin{exe}
        \ex 其为人也,发愤忘食,乐以忘忧,不知老之将至云尔
    \end{exe}

    A之于B也 (or A之为B也),predicate, or A为B也, predicate.
    The structure seems to be parallel to the English
    \form{I, as a concerned citizen, want to emphasize that \dots},
    where \form{as a concerned citizen} obligatorily modifies the subject \form{I}.
    其为人也 here seems to be a \emph{frame}, somehow comparable to the ``global'' temporal or locational phrase.
    Another issue is that the sentence seems to be unable to represent a specific event:
    *昨日,孔子为人也,发愤忘食, while we have \form{yesterday, as a concerned citizen, I \dots}
    
    \begin{exe}
        \ex 水之积也不厚,则其负大舟也无力。
    \end{exe}

\end{todobox}

\subsubsection{Focalization}

Topicalization is marked by fronting and a pause,
but what is fronted and before a pause is not necessarily a topic.
In (\ref{ex:grammatical.clause.focus.vp-fronted.1}),
for instance, the verbal predicate is fronted,
which likely is not topical.
Note that the sentence final particle is fronted as well.

\begin{todobox}{Fronted SFP}{sfp-fronting}
    The phenomenon can be analyzed in multiple ways.
    We may assume that (\ref{ex:grammatical.clause.focus.vp-fronted.1}) is essentially some sort of cleft construction,
    in which the subject is first separated from the rest of the clause
    and then the rest of the clause is focalized.
    Or we can analyze 矣 as a \ac{tam} marker, and not a marker from the CP layer.
    Or maybe we can argue that markers from CP layers are morphologically verbal
    and have to be attached to either the main verb or the verb phrase.
    Which analysis works best depends on whether they are consistent with other phenomena.
    \begin{itemize}
        \item Can we prove that the SFPs are very ``high-level'' and are above the topic layer? For example, can two clauses with different subjects share a SFP?
        \item Semantically do 矣 carry \ac{tam} meanings?
    \end{itemize}
\end{todobox}

\begin{exe}
    \ex\label{ex:grammatical.clause.focus.vp-fronted.1} 
    \gll [[甚]_{\text{VP_i}} 矣]_{\text{focus}}, [汝 之 不 惠]_{\text{subject: gerundive (\prettyref{sec:grammatical.clause.verbal.gerundive})}} ---_{\text{predicate_i}} \\
    extreme \category{sfp} 2 \category{gen} \category{neg} smart \\
    \glt\translate{You are so stupid! (lit. So extreme is your being unintelligent!)}
\end{exe}

\begin{todobox}{A complete overview of the left periphery}{left-periphery}
    See https://referenceworks.brill.com/display/entries/ECLO/COM-000248.xml
\end{todobox}

\subsection{Speech acts}\label{sec:grammatical.clause.force}

\subsubsection{Sentential aspect}

At the first glance, the sentence final particle 矣 marks the perfect aspect
(\ref{ex:grammatical.clause.force.sentential-aspect.yi.1}),
while 也 is for clauses describing something happening regularly, not a single concrete event (\ref{ex:grammatical.clause.force.sentential-aspect.ye.1})
\citep[\citepages{443-445}]{meiguang2018}.

\begin{exe}
    \ex\label{ex:grammatical.clause.force.sentential-aspect.yi.1} 余助苗长矣
    \ex\label{ex:grammatical.clause.force.sentential-aspect.ye.1} 将发命也
\end{exe}

矣 and 也, however, are different from prototypical aspect markers in several aspects.
First, 矣 appears predominantly in direct quotations in Classical texts,
which suggests that it has conversational functions.
也 frequently appears in narratives as a part of the judgemental construction
(\prettyref{sec:grammatical.clause.nominal.real.judgement}),
which lacks \ac{tam} marking, and 也 cannot be a prototypical aspect marker in that context.
Second, it seems that 矣 and 也 can be shared by two coordinated conjuncts with different subject (\prettyref{sec:grammatical.clause.sfp}),
which is rather unusual for an aspect marker.

Therefore, the grammatical category corresponding to 矣 and 也, whatever it is,
is not a typical \ac{tam} category,
and hence we disagree with \citet{meiguang2018}'s analysis.
Its scope is wider than \ac{tam} categories:
a \ac{tam} category is in relation with a nucleus clause,
while 矣 and 也 are in relation with a sentence,
i.e. an arbitrarily complex clause that is one utterance in a conversational context.
This is consistent with the usual analysis of sentence final particles 
in modern Standard Mandarin
\citep{paul2014particles,pan2021sentence}.

\begin{exe}
    \ex 亦各言其志也已矣
\end{exe}

\begin{infobox}{Alternative analysis}{sfp-alternative-analysis}
    \citet[\citepage{233}]{zhudexigrammar} acknowledges the wide spread of the analysis that sentence final particles are in relation with the whole sentence, not the nucleus clause,
    but insists that certain sentence final particles are a part of the predicate.
    Yet no convincing argument is provided.
    Among the three distributional classes he recognizes in Mandarin,
    two (marking the interrogative/imperative force, and attitude of the speaker)
    are uncontroversially attached to \emph{sentences} and not nucleus clauses.
    The remaining class, which is called the ``tense'' class by \citet{zhudexigrammar}
    and is structurally the innermost,
    resembles the class of 矣 and 也 discussed here,
    seems to be forbidden in most embedded clauses \citep{deng2010},
    just like the other two class do.
    Therefore all the three classes of Mandarin sentence final particles described in \citet{zhudexigrammar}
    are indeed in relation with the sentence and not the nucleus clause,
    which is consistent with the structural status of sentence final particles in Classical Chinese.
    
    We also note that it is not completely impossible for two independent nucleus clauses to share one \ac{tam} marker.
    In Japhug, for example, a series of nucleus clauses with different subjects can be coordinated with the \ac{tam} categories being marked at the end of the compound clause \citep[\citepages{1090-1091}]{jacques2021grammar}.
    However, Japhug lacks clear clause-level subject
    (\citealt[\citesec{2.5.3}]{jacques2021grammar}; although subjecthood is well-defined at the level of argument structure (\prettyref{box:multiple-external-arguments})),
    and therefore coordinated nucleus clauses with a shared \ac{tam} marker but different ``subjects'' is less strange in Japhug than it is in Classical Chinese:
    in the latter, we have a well-defined clausal pivot grammatical relation (\prettyref{sec:grammatical.verbal.subject.clause-pivot}) whose scope is over all \ac{tam} categories,
    making clauses sharing the \ac{tam} marker but not subjects rather unusual,
    but in the former this is probably not the case.

    \citet[\citepages{443-445}]{meiguang2018} relies solely on semantic criteria.
    Although we do not believe his analysis of 也 and 矣 as tense marker is completely correct,
    the two clearly have non-trivial interaction with \ac{tam} categories,
    a phenomenon also observed in Mandarin,
    where the lowest sentence final particle has access to \ac{tam} categories of the nucleus clause \citep[\citepage{258}]{paul2014new}.
\end{infobox}

\subsubsection{Interrogative, exclamative, and imperative}

The interrogative speech act, for example, is marked by 乎 and other particles
(\ref{ex:grammatical.clause.force.interrogative.1}).
The exclamative speech act is similarly marked by sentence final particles
(\ref{ex:grammatical.clause.force.exclamative.1}).
We note that in (\ref{ex:grammatical.clause.force.exclamative.1}),
the sentential aspect marker 矣 appears before the exclamative 夫,
which means that the two systems of particles can coexist.
The reverse order *夫矣 is not possible.

\begin{exe}
    \ex\label{ex:grammatical.clause.force.interrogative.1} 其能久乎?
    \ex\label{ex:grammatical.clause.force.exclamative.1} 吾死矣夫!
\end{exe}

\subsection{Subordination}\label{sec:grammatical.clause.linking}

The term \term{subordination} sometimes means all kinds of clause embedding.
In this section we primarily focus on bipartite clauses
with the structure and meaning of \translate{if \dots then \dots} or \translate{when ...},
and leave relative clauses and complement clauses to TODO: ref

An overview of subordination constructions in Classical Chinese can be found in \citet[\citechap{3}]{meiguang2018}.
In all Classical Chinese conditional constructions,
the condition usually appears before the consequence
(\prettyref{ex:grammatical.clause.linking.conditional.1},
\prettyref{ex:grammatical.clause.linking.conditional.2},
\prettyref{ex:grammatical.clause.linking.conditional.3}).
The consequence can be marked by 则
(\prettyref{ex:grammatical.clause.linking.conditional.1}).
Sometimes the marker 则 is dropped
(\prettyref{ex:grammatical.clause.linking.conditional.2})
but putting it back should never render a sentence ungrammatical
\citep[\citepage{86}]{meiguang2018}.
The marker 乃 is also available as a marker of the consequence clause
\citep[\citepage{87}]{meiguang2018}.

The condition clause can also be marked.
Classical Chinese distinguishes between realis and irrealis conditional constructions:
the former are marked by e.g. 既 (\prettyref{ex:grammatical.clause.linking.conditional.1}),
while the latter are marked by e.g. 若 (\prettyref{ex:grammatical.clause.linking.conditional.3}).
This distinction is relevant to the licensing of \ac{tam} markers
\citep[\citepage{81}]{meiguang2018}.
Other markers for the condition clause are also available \citep[\citechap{3}]{meiguang2018}.
We note that the marker 若 is able to appear \emph{after} the subject of the condition clause
\citep[\citepage{94}]{meiguang2018}.

\begin{exe}
    \ex\label{ex:grammatical.clause.linking.conditional.1} [既 来之]_{\text{condition}},[则安之]_{\text{consequence}}
    \ex\label{ex:grammatical.clause.linking.conditional.2} [杀女]_{\text{condition}},[我伐之]_{\text{consequence}}
    \ex\label{ex:grammatical.clause.linking.conditional.3} [若已食] 则退
\end{exe}

\begin{todobox}{Position of condition marker}{condition-marker-position}
    When the subjects of the two clauses are shared,
    it seems 若 obligatorily appears after the subject of the first clause.
    A possible analysis is to assume that the subordination construction is working at the level of VPs.
\end{todobox}

An interesting phenomenon is that the condition (\ref{ex:grammatical.clause.linking.gerundive.condition.1})
or temporal clause (\ref{ex:grammatical.clause.linking.gerundive.condition.1})
can be a gerundive one 
(\prettyref{sec:grammatical.clause.verbal.gerundive}).
This is not surprising cross-linguistically,
as the condition clause or the temporal clause is usually the ``subordinate'' clause,
while the consequence clause is the ``main'' clause,
and it is not uncommon for the subordinate clause in a clause subordination construction
to have a non-finite structure.
This is observed in for example Japanese and Turkish.
Note that the marker 若 can be attached to the gerundive condition clause as well
\citep[\citepage{98}]{meiguang2018}.
In some condition clauses,
the marker 而, instead of the otherwise genitive marker 之, appears between the subject and the predicate,
forming a clause type that is not gerundive and only appears as an irrealis condition clause
\citep[\citepages{100-102}]{meiguang2018}.

\begin{exe}
    \ex\label{ex:grammatical.clause.linking.gerundive.condition.1} 我之不德,民将弃我
    \ex\label{ex:grammatical.clause.linking.gerundive.temporal.1} 臣之壮也,犹不如人
\end{exe}

\subsection{Coordination}\label{sec:grammatical.clause.coordination}

Explicit marking of coordination is primarily done by the marker 而.
When used as a conjunction marker, 而 can be used to link two clauses 
or two verb phrases with a shared subject (\ref{ex:grammatical.clause.coordination.subject-vp.1}),
but not two nominal constituents.
Note that the functionalities of 而 is not restricted to conjunction \citep[\citepage{183}]{meiguang2018}.

\begin{exe}
    \ex\label{ex:grammatical.clause.coordination.subject-vp.1} 
    声伯四日不食以待之,食使者,而后食
\end{exe}

\subsubsection{Coordination of \acp{vp}}

Recall that a \ac{vp} contains an argument structure (\prettyref{sec:grammatical.clause.verbal.argument}) and a set of \ac{tam} markers (\prettyref{sec:grammatical.clause.verbal.tam}).
Therefore, coordination of two \acp{vp} actually has two structural possibilities:
coordination of two argument structures, resulting in a \emph{single} situation
(and the clause is \emph{not} a prototypical compound clause),
or coordination of two full \acp{vp}
\citep[\citepages{192-201}]{meiguang2018}.
In languages with \ac{tam} inflections, in the first scenario,
it is likely that the two verbs have one \ac{tam} marker in total,
or obligatorily have two identical \ac{tam} markers.
The distinction may also influence relativization \citep[\citepage{207}]{meiguang2018}.

Classical Chinese does not have \ac{tam}-based verbal inflection,
but the distinction between the two can still be told 

\subsubsection{Topic chains as syntactic coordination}\label{sec:grammatical.clause.coordination.topic-chain}

An interesting question is the interaction between topicalization and coordination.
\citet[\citepage{217}]{meiguang2018} contends that ``topic chains'',
i.e. several clauses with a shared topic \citep[\citechap{4} \citesec{3.3}]{meiguang2018},
are discourse structures and not syntactic structures.
Therefore topicalization happens first, and coordination happens then:
after that no further topicalization is possible.
He further argues that clauses in a topic chain cannot be linked together by 而.
(\ref{ex:grammatical.clause.coordination.topic.1}) however seems to be a counterexample.
This example clearly contains two coordinated clauses.
In the first clause 取之于蓝, 之, appearing after the verb, 
can only be a pronoun, and the only sensible reading of the clause
is that 之 (the object) is coreferential with 青 at the initial of the sentence,
and 取之于蓝 then means \translate{(people) extract it (i.e. indigo dye) from \species{Indigofera}.}
Therefore, 青 at the initial of the sentence is the object of the first clause
and the subject of the second clause,
meaning it cannot be the shared subject.
This, together with the traditional pause after the first 青,
means the first 青 likely is a topic,
which means here topicalization happens \emph{after} coordination.

\begin{exe}
    \ex\label{ex:grammatical.clause.coordination.topic.1} 
    \gll [青]_{\text{topic: NP_i}},---_i 取 [之]_{\text{object: Pronoun_i}} 于 蓝 而 ---_i 青 于 蓝 \\
    indigo.dye pick it from \species{Indigofera} \category{conj} {} blue than \species{Indigofera} \\
    \glt\translate{Indigo dye, people extract it from \species{Indigofera}, but it's bluer than \species{Indigofera}.}
\end{exe}

\section{The noun phrase}

The Classical Chinese \ac{np} can be roughly divided into 
the determiner region and the ``core'' region,
the latter known in \citet{cgel} as the \term{nominal}.%
\footnote{
    In this note, when the term \term{nominal} is used as a noun,
    it refers to the determined region in \ac{np}s,
    while when it is used as an adjective,
    it refers to the status of being the head of a \ac{np}. 
}
The latter is just the head noun plus possible complements and modifications,
and the first can be left empty or be a demonstrative, or a ``possessor'',
the role of the latter being not confined to a semantic possessor
\citep[\citepage{61}]{pulleyblank1995outline}.
When the ``possessor'' is present, the particle 之 appears between the possessor and the nominal region
(\ref{ex:grammatical.np.template.gen.1}, \ref{ex:grammatical.np.template.gen.2}).
When only the demonstrative is present, no marking is present
(\ref{ex:grammatical.np.template.dem.1}).

\begin{todobox}{Determiner region}{determiner}
    Give a comprehensive list of determiners.
\end{todobox}

\begin{exe}
    \ex\label{ex:grammatical.np.template.gen.1} 王之诸臣
    \ex\label{ex:grammatical.np.template.gen.2} 马之死者
    \ex\label{ex:grammatical.np.template.dem.1} [此心] 之所以合于王者
\end{exe}

\subsection{The nominal region} 

\paragraph*{Pre-head attributives} 

\begin{todobox}{Pre-head attributive}{pre-head-attributive}
Is the following paragraph right?

An interesting feature of Classical Chinese is 
that adjectives before the head noun seem strongly discouraged. 
The meaning of, say, \translate{an ugly big old bear},
is canonically expressed by several strategies.
One is the 者 construction introduced below, 
which can be described as a relative clause construction (but with caveats)
and seems to have no complexity constraints
(\ref{ex:grammatical.np.nominal.relative.long-1}).
Semantically non-restrictive attributives can always replaced by clausal coordination.

Multiple adjectives are indeed possible.


\end{todobox}

\paragraph*{The marker 者 and the relative clause construction} 
\label{sec:grammatical.noun-phrase.determinative-relative}
The marker 者 looks like a relativizer.
It is different from relativizers in many other languages in that
further structural add-ons can be applied to the fused relative clause formed by it,
while the fused relative clause constructions in many other languages 
are unable to undergo further modification.
This seems to be the only productive way to form complex nominals 
(\ref{ex:grammatical.np.nominal.relative.long-1}).

\begin{exe}
    \ex 马之千里者
    \ex\label{ex:grammatical.np.nominal.relative.long-1} 若[至力农畜,工虞商贾,为权利以成富,大者倾郡,中者倾县,下者倾乡里者],不可胜数 
\end{exe}

\begin{todobox}{Relative clause complexity}{relative-clause-complexity}
    Can a relative clause contain a NP that in turn contains a relative clause?
\end{todobox}

\begin{todobox}{\form{zhi}-\form{zhe} construction}{zhi-zhe}
    The structure of the 之-者 construction may cause some debates.
    It can be analyzed as a possessive construction on top of a fused relative clause construction
    and translated word-to-word into English as 
    \translate{[those who go one thousand miles] of horses}.
    An interesting question then is whether we have any other appearances of the N 之 V 者 construction
    where the relation between N and [V 者] is prototypically possessive.
    It seems this is indeed possible: 城北徐公,齐国之美丽者也.
    
    Under this analysis, 楚人有吹箫于市者 is composed by applying the external possessive construction
    to 楚人之吹箫于市者

    One fact (or is it really a fact?) supporting the determinative analysis of 之-者
    is the construction seems to be unable to receive a further determiner:
    *此马之千里者.
    The sequence 此马之千里者 does appear but it is almost always a nominal predication construction.
\end{todobox}

\begin{todobox}{What can be relativized, and possible external possession}{external-possession-or-relative-clause}
    若至[力农畜,工虞商贾,为权利以成富,大者倾郡,中者倾县,下者倾乡里者],不可胜数
    
    It seems what is relativized here is the subject of the bracketed clause.
    But then what's the role of 大者倾郡?
    If we consider it to be a coordinated clause,
    then it seems an argument is moved from only one branch of a coordination construction:
    a clear violation of the coordinate structure constraint of extraction!
    
    If we consider it to be a coordinated VP,
    then Classical Chinese should have a external possession construction:
    [商人]_{\text{subject}} [大者 倾郡]_{\text{predicate}},
    in which 大者 is a part of 商人.
    
    Or maybe this is a clausal pseudo-coordination:
    \form{what did Alex go to the store and buy}.
\end{todobox}

\subsection{Prepositions}

In Old Chinese, there are only two prepositions: 于 and 於.
The exact usages of the two prepositions are not clear.
In \work{Zuo Zhuan}, 于 is reserved for prepositional complements (\prettyref{sec:grammatical.verbal.argument.prepositional}),
while 於 is for inter-predicate focalization (TODO).
Other Old Chinese works have different conventions. 

It is possible to omit the object of a preposition.

\begin{exe}
    \ex 孔子 [因 ---]_{\text{reason}} 叹
\end{exe}

\chapter{Parts of speech}

The part of speech distinctions in Classical Chinese 
has been discussed in \prettyref{sec:grammatical.pos},
and in this chapter we discuss their behaviors in detail.
In principle, function words can be introduced together with their grammatical functions,
but since the correct analyses of some constructions are still controversial
and it may well be possible that the controversies reflect
real historical linguistic divergence among speakers,
function words are also included in this section for easier reference.

\begin{todobox}{Parts of speech, a chapter}{chap-pos}
    This chapter depends on a list of POS (\prettyref{box:pos-list}).
    The content:
    \begin{itemize}
        \item Noun; 
        \item Verbs; the details about noun-used-as-verb can be placed here, as a source of verbs.
        \item Look-up tables for particles
    \end{itemize}
\end{todobox}

\section{Nouns}


The verbs 出 (\translate{go out}), 入 (\translate{enter}), 亡 (\translate{die, decay})
are regularly derived to 出 (\translate{what goes out}), 入 (\translate{what comes in})
and 亡 (\translate{what dies}).
This derivation pattern however is not 

\begin{todobox}{Deverbalization derivation}{deverbal}
    Summarize deverbal derivations.  
\end{todobox}

\section{Verbs}

\subsection{``Nouns used as verbs''}\label{sec:pos.verb.noun-to-verb}

The conventional term in Mandarin Chinese 名词作动词 \translate{nouns used as verbs} covers two phenomena,
corresponding to multiple functions and zero derivation \citep[\citesec{11.3}]{dixon2010basic2},
and also the rare case of ad hoc re-categorization of a root.

\paragraph*{Multiple functions}
Some roots have both nominal and verbal uses,
and there is usually some semantic connection between the interpretations of the two uses,
but this is not regularly inferrable. 
Here we consider some examples in \citet{yang1991dict}:
\begin{itemize}
    \item 楚 may mean \translate{the Chu state} or \translate{do what Chu people do}.
    \item 床 may mean \translate{bed} or \translate{settle down your bed or sleep on a bed}.
    \item 城 may mean \translate{city, castle} or \translate{build a city}.
\end{itemize}
The interpretation of the verbal usage is usually \emph{not} decided
from the meaning of the root and that the root is used in a verbal environment;
rather, it is instructed by the lexicon.
Therefore, the verbal usage of 城市 only means \translate{build a city}
although the \translate{do city-related things} reading in principle could make sense. 

Therefore, roots like 城, 楚 and 床 have double functions: nominal and verbal,
but the two functions are likely not related to each other by regular grammatical rules.
This corresponds to the ``multiple function'' case in \citet[\citesec{11.3}]{dixon2010basic2}.
Moreover, what is stored in the lexicon is not the bare, non-categorized root 城,
but one noun lexeme 城 \translate{city} that specifies its nominal usage 
and one verb lexeme \translate{build a city} that specifies its verbal usage,
and other seemingly possible ways to categorize the root, although attested elsewhere,
are ruled out by their absence in the lexicon.

The boundary between roots with double functions and roots undergoing zero derivation (see below) 
is somehow blurry,
as the nominal and verbal uses of 城 and 床 still seem to show a common pattern
and may be understood as a rare derivation.
This blurriness leads many grammatical works on Classical Chinese 
to simply refer to the two phenomena uniformly as ``nouns used as verbs''.

\paragraph*{Zero derivation}
In other cases the meaning of the verbal use of a root usually appearing in a nominal context
is regularly derived from the nominal meaning.
This is because although tropative or causative derivations in Classical Chinese
are mainly verb-to-verb,
they can also be applied to nouns.
In this way from 臣 \translate{servant, official, minister}
we have the causative verbal usage \translate{make sb. dependent to},
and from 客 \translate{guest} we have the tropative usage \translate{consider sb. as a guest}.
These verbal usages are nothing different from noun-to-verb derivation observed in other languages,
so we regard the relevant phenomena as zero derivation as in \citet[\citesec{11.3}]{dixon2010basic2}.

In zero derivation, the meaning of the nominal usage has to be recorded in the lexicon,
the meaning of the verbal usage can be automatically decided from the derivation rule.
These derivations are however not completely regular and not for every word:
the lexicon also controls whether a derivational rule applies.

\paragraph*{Ad hoc re-categorization}
There are sporadic verbal usages of nouns that are almost never attested elsewhere,
like 军 in 沛公军霸上.
This means that ad hoc re-categorization of roots is possible in Classical Chinese,
and the meaning is to be decided from the context.
This is also possible in English
(as in \form{I might [guinea pig] it for you.}) 
but usually not accepted in formal texts.
Alleged ad hoc categorized Classical Chinese roots are indeed a possibility, after all,
although their frequency is quite low and cannot be exaggerated to be the norm rather than the exception.

\section{Pronouns}

\begin{todobox}{Third person pronouns}{third-person-pronoun}
    之 seems to be the accusative pronoun in Old Chinese.
    其 seems to be the genitive pronoun,
    and may be a phonological fusion of 之 and a possessive marker.
    
    See Mei, Guang.
\end{todobox}

\section{Particles}

Grammatical particles are not content words
and in principle can be introduced together with the grammatical categories and relations they express.
The long and complicated history evolution of Classical Chinese
however means a particle may have multiple quite different uses
possibly due to grammaticalization,
so a surface form-to-function discussion on particles is of great descriptive value.

\begin{todobox}{Classification of particles}{particle-classification}
    Do I need to classify particles?
\end{todobox}

\paragraph*{者} The particle 者 most frequently appears as a relativizer, a complementizer,
or in the \form{zhe}-\form{ye} construction.
The three functions can be uniformly analyzed as the function of a low-level determiner \citep{aldridge2009old}. 

\paragraph*{之} This 

\chapter{Verb valency}

\begin{todobox}{More topics on argument structure}{argument-structure-topics}
    \begin{itemize}
        \item Morphology?
    \end{itemize}
\end{todobox}

\section{Simple argument structures}
\label{sec:grammatical.clause.verbal.argument.simple}

\subsection{Prototypical \category{do}}
\label{sec:valency.simple.do}

The subject of a \category{do} clause usually has to be animate,
because it voluntarily initiates the event described by the clause
(\ref{ex:valency.simple.do.1}).
The subject is an \term{agent}, as opposed to a \term{causer} (\prettyref{sec:grammatical.clause.verbal.argument-structure.causative})
or a \term{theme} (\prettyref{sec:valency.simple.state-and-change}).

\begin{exe}
    \ex\label{ex:valency.simple.do.1} 桓公杀公子纠
\end{exe}

\subsubsection{Unique properties of \category{do} verbs}\label{sec:valency.simple.do.properties}

\citet[\citepage{272}]{meiguang2018} lists some criteria
to distinguish a transitive \category{do} verb 
from a transitive \category{cause} verb
(\prettyref{sec:grammatical.clause.verbal.argument-structure.causative.synthetic}).

We note that certain \category{cause} verbs may gradually develop a lexicalized meaning
and eventually get reanalyzed as a \category{do} verb
\citep[\citepages{269-271}]{meiguang2018}.

\subsection{Prototypical \category{become} and \category{be} verbs}
\label{sec:valency.simple.state-and-change}

\subsubsection{The intransitive usage}

A \category{be} verb describes a state;
a \category{become} verb describes the change of a state.
In both types of argument structures,
the sole argument is a \term{theme}:
the situation happening to it just happens,
and usually it does not have much control over it nor any volition to trigger it
(\citealt[\citepage{345}]{li2004grammar}; \citealt[\citepage{275}]{meiguang2018}).

In Classical Chinese, just like in other languages,
\category{become}/\category{be} verbs often have established causative usages,
forming \category{cause}-\category{become}/\category{be} argument structures
with the \term{causer} argument being the subject and the \term{theme} argument being internal
(\prettyref{sec:grammatical.clause.verbal.argument-structure.causative.synthetic}).
When a causer is absent, the structure of the clause 
is comparable to what sometimes is known as the middle voice in English
(e.g. \form{the door opened}; c.f. the transitive \category{cause}-\category{become} \form{I opened the door}).
(\ref{ex:grammatical.clause.verbal.stative.1}) is an instance:
in its \category{cause}-\category{be} usage (\ref{ex:grammatical.clause.verbal.argument.causative.synthetic.2}),
the argument that is described as weak is an internal argument appears after the verb,
but in (\ref{ex:grammatical.clause.verbal.stative.1}),
the argument that is described as weak is the \emph{subject}:
the internal theme argument gets promoted to the subject position.

\begin{exe}
    \ex\label{ex:grammatical.clause.verbal.stative.1} 秦强而赵弱
\end{exe}

The ``middle voice'' construction exemplified in (\ref{ex:grammatical.clause.verbal.stative.1})
(known as 内动 in \citet{meiguang2018}) has a subject,
which corresponds to the argument that is the object in the \category{cause}-\category{become}/\category{be} construction
(i.e. the internal argument).
It is however possible (although rare) for the subject position to be unfilled,
and the internal argument remains in-situ.
For instance, the verb 鸣 \translate{chirp} appears in ``middle voice'' clauses (\ref{ex:grammatical.clause.verbal.stative.2}),
but its sole argument can also stay \emph{after} the verb (\ref{ex:grammatical.clause.verbal.stative.3}).
The structure of (\ref{ex:grammatical.clause.verbal.stative.3}) can only be reasonably conceived
if we assume that 鸣 is a \category{be} verb,
denoting a state where bugs continue to make noise,
and that the sole argument 蜩 remains in-situ and is not promoted to the subject position.
No other analysis is available: for instance a \category{do} verb can never have such a behavior
\citep[\citepage{351}]{meiguang2018}.

\begin{exe}
    \ex\label{ex:grammatical.clause.verbal.stative.2}
    \gll 蝼蝈 鸣 \\
    \species{?} chirp \\
    \glt\translate{??? chirp.} (礼记·月令)

    \ex\label{ex:grammatical.clause.verbal.stative.3}
    \gll [五 月]_{\text{temporal}} [鸣]_{\text{predicate}} [蜩]_{\text{internal argument}} \\
    five month chirp cicada \\
    \glt\translate{In the fifth (lunar) month, cicadas chirp.}
\end{exe}

\subsubsection{The alternation between \category{be} and \category{become}}

Alternation between \category{be} and \category{become} verb frames is natural.
Some \category{become} verbs however do not have \category{be} counterparts.

\subsection{Non-conventional \category{be}/\category{become} clauses}
\label{sec:grammatical.clause.verbal.argument.simple.non-conventional-state}

Some verbs license subjects that look like arguments of prototypical \category{become} or \category{be} verbs:
the subject may be animate but it does not volitionally trigger the event.
The situation ``just happens to be the case'', and the subject can be described as a \term{theme} and not an \term{agent}.
What sets them apart from prototypical \category{become} or \category{be} verbs 
in \prettyref{sec:valency.simple.state-and-change}
is the fact that the subject seems quite unlike an internal argument.
In (\ref{ex:grammatical.clause.verbal.argument.non-conventional-theme.1}),
the subject 火 \translate{fire} is definitely a theme and not an agent:
the fire does not get to \emph{decide} if it burns the flag
\citep[\citepage{276}]{meiguang2018}.
Still the clause is not a prototypical \category{become} one
as there is an internal argument 其旗 in it,
and the theme 火 is an \emph{external} theme
\citep[\citepage{353}]{meiguang2018}.
These verbs therefore have difficulties participating in synthetic causativization
(\prettyref{sec:grammatical.clause.verbal.argument-structure.causative.synthetic}).

\begin{exe}
    \ex\label{ex:grammatical.clause.verbal.argument.non-conventional-theme.1} 火焚其旗
\end{exe}

\subsection{Experience verbs}
\label{sec:grammatical.clause.verbal.argument.simple.experience}
Some experience verbs, mostly verbs about emotions, behave like \category{become} verbs
\citep[\citepage{273}]{meiguang2018}:
when used as transitive verbs,
the subject do not look quite agentative and the clause is likely causative
(\ref{ex:grammatical.clause.verbal.argument.simple.experience.2}),
and when used as intransitive verbs,
there is a clear internal change-of-state meaning
(\ref{ex:grammatical.clause.verbal.argument.simple.experience.1}).

\begin{exe}
    \ex\label{ex:grammatical.clause.verbal.argument.simple.experience.1} 孔子成春秋,而乱臣贼子惧 (孟子·滕文公章句下)
    \ex\label{ex:grammatical.clause.verbal.argument.simple.experience.2} 惧之以怒 (左传·昭公十三年)
\end{exe}

On the other hand, perception verbs (e.g. 见 \translate{look}) and cognition verbs (e.g. 知 \translate{know})
are often transitive,
and therefore are not compatible with the synthetic causative construction 
(\prettyref{sec:grammatical.clause.verbal.argument-structure.causative.synthetic};
\citealt[\citepage{274}]{meiguang2018}).
Intuitively, these verbs are \category{do}-like according to the criteria listed in \prettyref{sec:valency.simple.do}.
For instance, they can appear in 所 construction
(\ref{ex:grammatical.clause.verbal.argument.simple.experience.do.1}).

\begin{exe}
    \ex\label{ex:grammatical.clause.verbal.argument.simple.experience.do.1} 異乎吾所聞
\end{exe}

Certain perception verbs however have developed a figurative, fossilized meaning,
and when intransitivized, can participate in synthetic causativization
(\prettyref{sec:grammatical.clause.verbal.argument-structure.causative.synthetic},
\ref{ex:grammatical.clause.verbal.argument.causative.synthetic.3}).
This possibility indicates that these fossilized usages are \category{become}- or \category{be}-like:
见 \translate{meet formally} therefore means \translate{in the state of regularly meeting an important figure}.

\section{Various causative constructions}\label{sec:grammatical.clause.verbal.argument-structure.causative}

A \term{causer} makes a situation to be the case,
but does not always do so intentionally.
It can therefore be inanimate,
as opposed to how an \term{agent} behaves
(\prettyref{sec:valency.simple.do}).

We can divide causative constructions in Classical Chinese into
synthetic and analytic ones.
In the synthetic causative construction,
there is only one verb in the surface form:
the causative valency alternation is supposedly marked by a prefix \form{*s-},
which is invisible in the written texts but is reflected by tonal changes of the verb.
If a root develops a lexicalized usage in the synthetic causative construction,
then \category{cause} verb is formed.

\subsection{Synthetic causative}\label{sec:grammatical.clause.verbal.argument-structure.causative.synthetic}

The synthetic causative construction applies to existing argument structures, or sometimes bare roots.
The synthetic causative construction cannot be applied to a \category{do} construction:
the reason is probably because a \category{do} construction is too ``big'',
already having a full-fledged wannabe subject \citep[\citepage{363-364}]{meiguang2018}.
On the other hand, the syntactic causative construction can be applied to 
``passive'' (\ref{ex:grammatical.clause.verbal.argument.causative.synthetic.1})
and \category{become} or \category{be} (\ref{ex:grammatical.clause.verbal.argument.causative.synthetic.2}) argument structures.
Certain intransitivized experience verbs,
possibly having an argument structure comparable to a \category{become}/\category{do} verb (\prettyref{sec:grammatical.clause.verbal.argument.simple.experience}) with a wannabe subject also have causative usages
(\ref{ex:grammatical.clause.verbal.argument.causative.synthetic.3}),
but their transitive counterparts are never compatible with the synthetic causative construction
\citep[\citepage{274}]{meiguang2018}.
Finally, the synthetic causative construction can be directly applied to a root (\ref{ex:grammatical.clause.verbal.argument.causative.synthetic.4}):
the word 妻 \translate{wife} is sometimes used as a verb,
meaning \translate{to marry daughter to \dots},
inconsistent with the meaning of (\ref{ex:grammatical.clause.verbal.argument.causative.synthetic.4}).
Therefore, in (\ref{ex:grammatical.clause.verbal.argument.causative.synthetic.4}),
妻 is ad hoc categorized into a \category{cause} verb,
its usual verbal usage being irrelevant here.

\begin{exe}
    \ex\label{ex:grammatical.clause.verbal.argument.causative.synthetic.1}
    \gll 是 夭 子蛮, 杀 御叔…… \\
    this die.young \category{name} kill \category{name} \\
    \glt\translate{This woman made Ziman die at a young age, and got Yushu killed\dots}
    
    \ex\label{ex:grammatical.clause.verbal.argument.causative.synthetic.2}
    \gll 以 弱 天下 之 民 \\
    \category{purpose} weak world \category{gen} people \\
    \glt\translate{\dots to weaken the people.}

    \ex\label{ex:grammatical.clause.verbal.argument.causative.synthetic.3}
    \gll 子尾 见 疆 \\
    \category{name} formally.visit \category{name} \\
    \glt\translate{Ziwei let Jiang formally visit (with Xuanzi).} (左传·昭公二年)

    \ex\label{ex:grammatical.clause.verbal.argument.causative.synthetic.4} 妻帝之二女
\end{exe}

The labile S/O alternation between the \category{be}/\category{become} usage
and the \category{cause}-\category{be}/\category{become} usage
is quite regular in Classical Chinese;
verbs allowing this alternation are sometimes known as \term{ergative verbs}
\citep[\citepage{378}]{meiguang2018},
although the phenomenon is about the core argument structure and has nothing to do with ergativity in alignment.
It should be noted that not all \category{become}/\category{be} verbs are compatible
with the synthetic causative construction.
For instance, 鸣 \translate{chirp} in (\ref{ex:grammatical.clause.verbal.stative.3})
does not have a transitive \category{cause}-\category{be} usage.
More examples are given in \citet[\citepage{276}]{meiguang2018}.
On the other hand, some clauses that look like \category{cause}-\category{be}/\category{become} clauses
actually do not have \category{be} or \category{become} counterparts 
(\prettyref{sec:grammatical.clause.verbal.argument-structure.causative.fossilization}).

\subsection{Fossilization of synthetic causative construction}
\label{sec:grammatical.clause.verbal.argument-structure.causative.fossilization}

Some \category{cause} verbs are fossilized,
and do not have clear intransitive counterparts.
For instance, 伤 \translate{hurt} typically is a state transition verb meaning body, etc. being hurt,
and it also has a causative (i.e. \category{cause}-\category{become}) meaning
(\translate{make \dots hurt}).
The \category{cause}-\category{become} verb frame of 伤 however has gained a separate lexicalized specific that can't be transparently inferred from the meaning of the \category{become} usage:
it can mean \translate{let \dots be demaged},
in which the object is not necessarily body or a person.
This usage of 伤 has no \category{become} or other intransitive counterpart.
The absence of a \category{become} counterpart can be proven by 
the ability for this fossilized figurative usage of 伤 to undergo ``passivization'' 
(\ref{ex:grammatical.clause.argument.causative.fossilize.1}),
which is otherwise not possible (\prettyref{sec:grammatical.clause.verbal.argument-structure.pseudo-passive}).

\begin{exe}
    \ex\label{ex:grammatical.clause.argument.causative.fossilize.1} 女红伤则寒之原也
\end{exe}

Verbs like 伤 in like (\ref{ex:grammatical.clause.argument.causative.fossilize.1}) can easily be reanalyzed as \category{do} verbs.
This is likely a diachronic path of the creation of \category{do} verbs.
The verb 败 for example seems to be originally a \category{become} verb (\translate{to get corrupted})
and have later gained a specific meaning of \translate{to defeat} in its \category{cause}-\category{become} usage,
which had eventually evolved into a \category{do} usage
\citep[\citepage{285}]{meiguang2018}.

\section{The ``passive''}\label{sec:grammatical.clause.verbal.argument-structure.passive}

What is often known as the passive in Classical Chinese is not really a passive construction
comparable to the English or Latin passive.
The main problem is the lack of a grammaticalized way to say the agent:
in a ``true'' passive construction, the original subject is somehow demoted
(represented by the appearance of \form{by} in English or the ablative case in Latin)
and sometimes omitted, and an internal argument is promoted to the subject position,
while the so-called ``passive'' constructions in Classical Chinese
are \emph{obligatorily removed} \citep[\citepage{287-289}]{meiguang2018}.

\subsection{The agent-less ``passive''}\label{sec:grammatical.clause.verbal.argument-structure.pseudo-passive}

It is rare to apply the ``passive'' construction to a causative construction
\citep[\citepage{283}]{meiguang2018}.
Suppose we have a bivalence causative construction,
and we want to suppress the external argument and let the internal argument to be the subject.
But such a bivalence causative construction usually has a \category{cause}-\category{become} structure,
and removing the causer leaves us a clause that looks just like a \category{become} clause.
So there are two competing analyses,
and since usually if a verb root is lexically licensed to head a \category{cause}-\category{become} clause,
then its usage in a \category{become} clause is also in the lexicon,
the simpler \category{become} analysis is preferred.
However, where this preference is eliminated, ``passivization'' of a causative construction is possible 
(\citealt[\citepages{284,370-372}]{meiguang2018}; \prettyref{sec:grammatical.clause.verbal.argument-structure.causative.fossilization}).

\subsubsection{The case with multiple internal arguments}\label{sec:grammatical.clause.verbal.argument-structure.pseudo-passive.multiple-argument}

When the pseudo-passive construction is applied to double object clauses,
it is the \emph{recipient} that is promoted to the subject position 
(\ref{ex:valency.decrease.pseudo-passive.multi-valent.1}) \citep[\citepage{421}]{meiguang2018}.
On the other hand, when the pseudo-passive construction is applied to the corresponding prepositional argument construction,
it is the \emph{theme} that is promoted to the subject position
(\ref{ex:valency.decrease.pseudo-passive.multi-valent.2}).
These phenomena establish a hierarchy of \term{externality} of arguments (\prettyref{sec:grammatical.verbal.subject.argument-structure.alternation}).

\begin{exe}
    \ex\label{ex:valency.decrease.pseudo-passive.multi-valent.1} 诸侯,赐弓矢然后征
    \ex\label{ex:valency.decrease.pseudo-passive.multi-valent.2} 药言先献于贵,然后闻于卑
\end{exe}

\section{Experiential valency increasing}

\subsection{The affective constructions}

Classical Chinese has two affective constructions,
in which the subject is an experiencer suffering something bad from the situation described by the latter
\citep[\citepages{354-358}]{meiguang2018}.

The first affective construction simply attaches an experiencer to an argument structure
(\ref{ex:grammatical.clause.verbal.argument.affective.1}).
In (\ref{ex:grammatical.clause.verbal.argument.affective.1.become}),
亡 appears as a \category{become} verb (\prettyref{sec:valency.simple.state-and-change}):
it is intransitive and its subject, the \work{Odes}, did not have control over its being ignored.
In (\ref{ex:grammatical.clause.verbal.argument.affective.1.affective-become}),
a new argument -- the experiencer subject -- is introduced to the argument structure of 亡:
the meaning of the sentence is \translate{the shepherds suffered from the sheeps getting lost.}

\begin{exe}
    \ex\label{ex:grammatical.clause.verbal.argument.affective.1} 
    \begin{xlist}
        \ex\label{ex:grammatical.clause.verbal.argument.affective.1.become} 亡 as a \category{become} verb
        \gll [\focus{诗}]_{\text{subject,theme: NP}} [\focus{亡}]_{\text{\category{become}}} 然后 春秋 作 \\
        \focus{poem} \focus{get.lost} then Spring-Autumn compose \\
        
        \glt\translate{The \work{Odes} got lost, and then the \work{Spring and Autumn Annals} was composed.}
        \ex\label{ex:grammatical.clause.verbal.argument.affective.1.affective-become}
        亡 as a \category{affective}-\category{become} verb
        \gll [\focus{二} \focus{人}]_{\text{subject: NP}} [相 与 牧 羊, 而 俱 [\focus{亡}]_{\text{\category{affective}-\category{become}}} \focus{其} \focus{羊}]_{\text{coordinated VP}} \\
        \focus{two} \focus{person} mutually go.together herd sheep \category{conj} all \focus{get.lost} \focus{\category{poss}} \focus{sheep} \\
        \glt\translate{Two people herded their sheep together, and they both lost their sheep (lit. suffer from their sheep's missing).}
    \end{xlist}
\end{exe}

The second affective construction \emph{obligatorily} has an object,
which is in possession of the subject.
The meaning of (\ref{ex:valency.affective.type-2.1}), for example,
is that Confucius was frustrated by the fact that his tree was cut in Song.

The fact that the object (树 and 胁 here) should not contain any possessive markers
and has to be interpreted as something being possessed by the subject
suggests that the second affective construction is an external possession construction.

\begin{exe}
    \ex\label{ex:valency.affective.type-2.1} \focus{吾} 再 逐 於鲁, 伐 树 於 宋
    \ex\label{ex:valency.affective.type-2.2} 范睢折胁於魏
\end{exe}

\begin{todobox}{External possession}{external-possession}
    External possession as subject
\end{todobox}

\subsection{Tropative}

Tropative is a construction which attaches an experiencer to a \category{be} argument structure,
with the meaning being \translate{$A$ consider $B$ to be \dots}
\citep[\citepages{413-414}]{meiguang2018}
The Classical Chinese tropative is actually not limited to stative verbs:
it also applies to nouns.
It is however not likely that this construction comes from transformation of the nominal predicate construction
(\prettyref{sec:grammatical.clause.nominal}).
The main difference is that in the nominal predicate construction,
the predicate is a noun \emph{phrase},
but the tropative construction never takes a nominal predicate as input.

\section{Applicative constructions}

\begin{todobox}{Applicative constructions}{applicative}
    benefactive; the claim that double object constructions are similar to benefactive constructions;
    what object gets passivized.
    See \citet[\citepage{421}]{meiguang2018}.
\end{todobox}

\chapter{Negation}

\chapter{Discussions on quirky examples}

\begin{exe}
    \ex {} [良人者]_{\text{subject}} [所仰望而终身]_{\text{predicate}} 也
\end{exe}

It seems subjects of ordinary verbal clauses cannot be topicalized
(\ref{ex:grammatical.clause.subject.no-topic-1}),
but if the \ac{vp} is emphasized, topicalization is possible.

\begin{exe}
    \ex\label{ex:grammatical.clause.subject.no-topic-1} \begin{xlist}    
        \ex 三王既以定法度
        \ex *三王,既以定法度
    \end{xlist}
\end{exe}

\begin{exe}
    \ex 秦,虎狼之国,不可信,不如毋行
\end{exe}




\printbibliography[title=References]

\end{document}

\part{单粒子量子力学}

\chapter{量子化的光场}

\section{线性介质中的光场量子化}

电磁场足够强以至于难以看到单光子效应,而又足够弱以至于能量不至于强到需要考虑量子电动力学的圈图修正,这样就可以使用经典电动力学描述整个系统。
为了讨论电磁波的量子涨落(在分析诸如腔内辐射场,或是非线性光学中的DFG过程时非常重要,在做高精度测量时有时也要考虑),即使没有圈图效应,我们也要做光场的量子化。
这个做法的必要性将在后续的章节中多次体现出来,我们这里只是讨论量子化技术本身,暂时不考虑光的量子性在哪些情境下最为明显。

\subsection{真空}\label{sec:quantization-in-vacuum}

我们首先考虑真空中的光场的量子化,此时我们无非是在重复QED中的运算,实际上是在重复无质量矢量场的量子化(见\qftdoc中的\ref{qft-sec:massless-vector-quantize}节)。
QED中矢量场展开为
\begin{equation}
    A_\mu(\vb*{x}, t) = (\frac{\varphi}{c}, - \vb*{A}) = \int \frac{\dd[3]{\vb*{k}}}{(2\pi)^3} \sqrt{\frac{\hbar}{2\omega_{\vb*{k}} \epsilon_0}} \sum_{\sigma=1}^2 \left( a_{\vb*{k} \sigma} \epsilon_\mu^\sigma(\vb*{k}) \ee^{\ii \vb*{k} \cdot \vb*{x} - \ii \omega_{\vb*{k}} t} + a_{\vb*{k} \sigma}^\dagger \epsilon_\mu^\sigma(\vb*{k})^* \ee^{- \ii \vb*{k} \cdot \vb*{x} + \ii \omega_{\vb*{k}} t} \right),
    \label{eq:vector-field-components}
\end{equation}
其中电磁场模式为平面波。
取费曼规范,做一些分部积分并去掉表面项,得到
\begin{equation}
    \mathcal{L} = - \frac{1}{2 \mu_0} \partial_\mu A_\nu \partial^\mu A^\nu,
\end{equation}
从而正则动量为
\begin{equation}
    \pi^\mu = \pdv{\mathcal{L}}{\partial_0 A_\mu} = - \partial^0 A^\mu,
\end{equation}
可以据此写出正则量子化条件,即时间相同时,$A^\mu$同$A^\nu$对易,而
\begin{equation}
    [A^\mu(\vb*{x}, t), \pi^\mu(\vb*{y}, t)] = \ii \eta^{\mu \nu} \delta^{(3)}(\vb*{x} - \vb*{y}).
\end{equation}
哈密顿量为
\[
    \begin{aligned}
        H &= \int \dd[3]{\vb*{r}} (\pi^\mu \partial_0 A_\mu - \mathcal{L}) \\
        &= \int \dd[3]{\vb*{r}} \left( - \frac{1}{c^2} (\partial_t A^\mu)^2 + \frac{1}{2} \partial_\mu A_\nu \partial^\mu A^\nu \right),
    \end{aligned}
\]
这里要注意$x^0 = c t$。代入$A_\mu$的展开式计算得到
\begin{equation}
    H = \sum_{\sigma=1}^2 \int \frac{\dd[2]{\vb*{k}}}{(2\pi)^3} \hbar \omega_{\vb*{k}} \left( a^\dagger_{\vb*{k} \sigma} a_{\vb*{k} \sigma} + \frac{1}{2} \right), \quad \omega_{\vb*{k}} = c \abs*{\vb*{k}}.
\end{equation}
这就得到了量子化的能量。
在量子化过程中我们已经通过限制$\sigma$而施加了规范,不过这个规范并不是辐射规范,而是洛伦兹规范。横波条件通过$\epsilon$矢量和波矢垂直而保证。
不过,既然我们只关心电偶极辐射而有关的相互作用哈密顿量可以完全写成$\vb*{E}$,这也不重要。

从四维矢量计算电场,得到
\begin{equation}
    \vb*{E}(\vb*{r}, t) = \int \frac{\dd[3]{\vb*{k}}}{(2\pi)^3} \sqrt{\frac{\hbar}{2\omega_{\vb*{k}} \epsilon_0}} \sum_{\sigma=1}^2 \left( (- \ii \vb*{k} \epsilon_0^\sigma(\vb*{k}) + \ii \omega_{\vb*{k}} \vb*{\epsilon}^\sigma(\vb*{k})) a_{\vb*{k} \sigma} \ee^{\ii \vb*{k} \cdot \vb*{r} - \ii \omega_{\vb*{k}} t} + \text{h.c.} \right),
\end{equation}
以及
\begin{equation}
    \vb*{B}(\vb*{r}, t) = \int \frac{\dd[3]{\vb*{k}}}{(2\pi)^3} \sqrt{\frac{\hbar}{2\omega_{\vb*{k}} \epsilon_0}} \sum_{\sigma=1}^2 \left( \ii \vb*{k} \times \vb*{\epsilon}_\sigma a_{\vb*{k} \sigma} \ee^{\ii \vb*{k} \cdot \vb*{r} - \ii \omega_{\vb*{k}} t } + \text{h.c.} \right).
\end{equation}
其中
\begin{equation}
    \epsilon^\mu_\sigma = (\epsilon_\sigma^0, \vb*{\epsilon}_\sigma), \quad \frac{\omega_{\vb*{k}}}{c} \epsilon_0^\sigma - \vb*{k} \cdot \vb*{\epsilon}_\sigma = 0, \quad \abs*{\epsilon_\sigma^0}^2 - \abs*{\vb*{\epsilon}_\sigma}^2 = 1.
\end{equation}
通过以上公式,能够验证以下哈密顿量形式:
\begin{equation}
    H = \int \dd[3]{\vb*{r}} \left( \frac{\epsilon_0}{2} \vb*{E}^2 + \frac{1}{2\mu_0} \vb*{B}^2 \right).
    \label{eq:e-and-b-hamiltonian}
\end{equation}
这正是电动力学中常见的形式。因此实际上我们也可以直接用\eqref{eq:vector-field-components}写出$\vb*{E}$和$\vb*{B}$并代入\eqref{eq:e-and-b-hamiltonian}。

在本文讨论的光学问题中,我们可以使用一种对具体计算更加友好的形式,即采用\concept{辐射规范}。
在辐射规范之下,我们有
\begin{equation}
    \begin{aligned}
        \mathcal{L} &= - \frac{1}{4 \mu_0} (\partial_\mu A_\nu - \partial_\nu A_\mu) (\partial^\mu A^\nu - \partial^\nu A^\mu) \\
        &= \frac{1}{2 \mu_0} \frac{1}{c^2} (\partial_t \vb*{A})^2 - \frac{1}{4\mu_0} (\partial_i A_j - \partial_j A_i) (\partial^i A^j - \partial^j A^i) \\
        &= \frac{\epsilon_0}{2} ((\dot{\vb*{A}})^2 - c^2 (\curl{\vb*{A}})^2).
    \end{aligned}
\end{equation}
以这个拉氏量为出发点做正则量子化。做展开
\begin{equation}
    \vb*{A}(\vb*{r}, t) = \int \frac{\dd[3]{\vb*{k}}}{(2\pi)^3} \sqrt{\frac{\hbar}{2 \omega_{\vb*{k}} \epsilon_0}} \sum_{\sigma=1}^2 (a_{\vb*{k} \sigma} \vu*{e}^\sigma \ee^{\ii \vb*{k} \cdot \vb*{r} - \ii \omega_{\vb*{k}} t} + a^\dagger_{\vb*{k} \sigma} (\vu*{e}^\sigma)^* \ee^{- \ii \vb*{k} \cdot \vb*{r} + \ii \omega_{\vb*{k}} t}),
\end{equation}
从而电场和磁场分别为
\begin{equation}
    \vb*{E}(\vb*{r}, t) = \ii \int \frac{\dd[3]{\vb*{k}}}{(2\pi)^3} \sqrt{\frac{\hbar \omega_{\vb*{k}}}{2 \epsilon_0}} \sum_{\sigma=1}^2 (a_{\vb*{k} \sigma} \vu*{e}^\sigma \ee^{\ii \vb*{k} \cdot \vb*{r} - \ii \omega_{\vb*{k}} t} - a^\dagger_{\vb*{k} \sigma} (\vu*{e}^\sigma)^* \ee^{- \ii \vb*{k} \cdot \vb*{r} + \ii \omega_{\vb*{k}} t})
    \label{eq:vacuum-e-field}
\end{equation}
和
\begin{equation}
    \vb*{B}(\vb*{r}, t) = \ii \int \frac{\dd[3]{\vb*{k}}}{(2\pi)^3} \sqrt{\frac{\hbar}{2 \omega_{\vb*{k}} \epsilon_0}} \sum_{\sigma=1}^2 (a_{\vb*{k} \sigma} \vb*{k} \times \vu*{e}_\sigma \ee^{\ii \vb*{k} \cdot \vb*{r} - \ii \omega_{\vb*{k}} t} - a^\dagger_{\vb*{k} \sigma} \vb*{k} \times \vu*{e}_\sigma^* \ee^{- \ii \vb*{k} \cdot \vb*{r} + \ii \omega_{\vb*{k}} t}).
    \label{eq:vacuum-b-field}
\end{equation}
正则动量为
\begin{equation}
    \vb*{\pi} = \epsilon_0 \dot{\vb*{A}},
\end{equation}
施加正则对易关系,会得到正确的
\begin{equation}
    \comm*{a_{\vb*{k} \sigma}}{a_{\vb*{k}' \sigma'}} = (2\pi)^3 \delta(\vb*{k} - \vb*{k}') \delta_{\sigma \sigma'},
\end{equation}
而哈密顿量为
\begin{equation}
    \begin{aligned}
        H &= \int \dd[3]{\vb*{r}} \left(\vb*{\pi} \cdot \pdv{\vb*{A}}{t} - \mathcal{L} \right) = \int \dd[3]{\vb*{r}} \left( \frac{\epsilon_0}{2} \left(\pdv{\vb*{A}}{t}\right)^2 + \frac{\epsilon_0}{2} c^2 (\curl{\vb*{A}})^2 \right) \\
        &= \int \dd[3]{\vb*{r}} \left( \frac{\epsilon_0}{2} \vb*{E}^2 + \frac{1}{2\mu_0} \vb*{B}^2 \right) \\
        &= \sum_{\sigma=1}^2 \int \frac{\dd[3]{\vb*{k}}}{(2\pi)^3} \hbar \omega_{\vb*{k}} \left(a^\dagger_{\vb*{k} \sigma} a_{\vb*{k} \sigma} + \frac{1}{2} \right).
    \end{aligned}
\end{equation}
因此,辐射规范给出的结果和完整的QED计算是完全一致的。
在辐射规范中我们还可以证明一个在横场条件成立时也成立,并且在一般的QED中很难计算的公式:
\begin{equation}
    \comm*{E^i(\vb*{r}, t)}{B^j(\vb*{r}', t)} = - \frac{\ii \hbar}{\epsilon_0} \pdv{x^k} \delta(\vb*{r} - \vb*{r}')
\end{equation}
其中$i, j, k$是$x, y, z$的轮换排列;其它情况下对易子为零。
还能够发现电场和自己的对易子始终为零,磁场亦然。因此电场的三个分量可以同时确定地被测量,磁场亦然。
但是不能同时准确测出$\vb*{E}$和$\vb*{B}$。
由于$(\vb*{E}, \vb*{B})$,$\vb*{A}$和$a_{\vb*{k} \sigma}$之间的关系是线性的,$a$的产生湮灭算符对易关系、$\vb*{A}$和$\vb*{\pi}$的正则对易关系以及$\vb*{E}$和$\vb*{B}$的对易关系是彼此等价的。

\begin{back}{粒子数表象和升降算符}{ladder-operator-particle-number}
    如果
    \begin{equation}
        n = a^\dagger a,
    \end{equation}
    且
    \begin{equation}
        \comm*{a}{a^\dagger} = 1,
    \end{equation}
    那么
    \begin{equation}
        a = \sum_{n} \sqrt{n} \ket*{n-1} \bra{n}, \quad a^\dagger = \sum_{n} \sqrt{n} \ket*{n} \bra*{n-1},
    \end{equation}
    或者说
\end{back}

\subsection{长波光子和介质中的麦克斯韦方程}\label{sec:long-wavelength-photon-maxwell-general}

下面我们讨论和\eqref{eq:material-hamiltonian}匹配的对易关系,以及它对角化之后将给出什么样的能谱。
应当指出,此时真空中的那些对易关系——$\vb*{A}$和$\epsilon_0 \dot{\vb*{A}}$之间的对易关系,$\vb*{E}$和$\vb*{B}$之间的对易关系——可能不能够直接适用。
这是因为正则量子化中,积掉自由度会导致哈密顿量的本征态的意义发生变化,从而算符的意义会发生变化(例如重整化后产生一个介质中的光子实际上意味着产生了如下的态:单真空光子态,叠加上单真空光子加电子空穴对的态,叠加上别的一些东西)。在高能物理中这导致场强重整化,在本文讨论的量子光学中这意味着$\vb*{E}$的各个分量之间的正则对易关系会发生变化(当然,在适当的定义下,单光子产生湮灭算符之间的对易关系仍然不变)。
同样这也会让横波条件的形式发生变化——介质中横波条件是$\div{\vb*{\epsilon} \cdot \vb*{E}} = 0$。

我们需要直接从\eqref{eq:material-hamiltonian}计算正则动量。同样取辐射规范,以$\vb*{A}$为基本自由度,则\eqref{eq:material-hamiltonian}就是
\begin{equation}
    H = \int \dd[3]{\vb*{r}} \left( \frac{1}{2} \dot{\vb*{A}} \cdot \vb*{\epsilon} \cdot \dot{\vb*{A}} + \frac{1}{2} (\curl{\vb*{A}}) \cdot (\vb*{\mu}^{-1}) \cdot (\curl{\vb*{A}}) \right),
\end{equation}
于是正则动量为
\begin{equation}
    \vb*{\pi} = \vb*{\epsilon} \cdot \dot{\vb*{\vb*{A}}}.
\end{equation}
注意到此时的正则动量的形式和真空中是不同的。
我们现在需要展开$\vb*{A}$。此时空间平移不变性不能保持,我们不能使用动量来标记电场的振动模式,
我们将\eqref{eq:photon-in-material}右边的$\vb*{j}$取为零——我们此处在对线性介质做正则量子化,暂时不考虑电流——那就得到了一个广义本征值问题。
这就意味着,我们可以求解出一整套本征函数,它们由下式
\begin{equation}
    \curl{(\mu^{-1} \cdot \curl{\vb*{u}_n})} - \omega_n^2 \epsilon \cdot \vb*{u}_n = 0
\end{equation}
确定,其中$\omega_n$对应着能够在系统中稳定传播的电磁波模式的频率,且有正交归一关系
\begin{equation}
    \int_V \dd[3]{\vb*{r}} \vb*{u}^*_m \cdot \vb*{\epsilon} \cdot \vb*{u}_n = \delta_{mn}, 
\end{equation}
请注意由于$\vb*{A}$的厄米性,$\vb*{u}_m^*$一般是另一个$\vb*{u}_n$。
正交归一关系又意味着
\begin{equation}
    \omega_n^2 \delta_{mn} = \int \dd[3]{\vb*{r}} (\curl{\vb*{u}_m^*}) \cdot (\vb*{\mu}^{-1}) \cdot (\curl{\vb*{u}_n}) + \int \dd{\vb*{S}} \cdot (\vb*{u}_m^* \times ((\vb*{\mu}^{-1}) \cdot (\curl{\vb*{u}_n}))),
\end{equation}
在自由空间中等式右边第二项可以略去,在一个反射性能尚可的反射腔体(如果我们只讨论有限空间中的问题,那么基本上这个问题需要放在一个腔体中,否则无法忽视外界影响)中可以把第二项当成微扰。
本节仅仅给出最为简单的理论,暂时不考虑第二项。
用这组基底$\{\vb*{u}_n\}$做展开
\begin{equation}
    \vb*{A}(\vb*{r}, t) = \sum_n \ii \sqrt{\frac{\hbar}{2\omega_{n}}} \vb*{u}_n(\vb*{r}) a_n \ee^{- \ii \omega_n t} + \text{h.c.},
\end{equation}
得到
\begin{equation}
    - \vb*{E} = \dot{\vb*{A}} = \sum_n \sqrt{\frac{\hbar \omega_n}{2}} \vb*{u}_n(\vb*{r}) a_n \ee^{-\ii \omega_n t} + \text{h.c.},
\end{equation}
以及
\begin{equation}
    \vb*{B} = \curl{\vb*{A}} = \sum_n \ii \sqrt{\frac{\hbar}{2\omega_n}} \curl{\vb*{u}_n(\vb*{r})} a_n \ee^{-\ii \omega_n t} + \text{h.c.}.
\end{equation}
施加正则对易关系
\begin{equation}
    \comm*{A^i(\vb*{r}, t)}{\pi^j(\vb*{r}', t)} = \ii \hbar \delta(\vb*{r} - \vb*{r}') \delta^{ij},
\end{equation}
我们发现我们能够得到我们想要的产生湮灭算符对易关系
\begin{equation}
    \comm*{a_n}{a_m^\dagger} = \delta_{mn}.
\end{equation}
然后,可以计算出哈密顿量为
\begin{equation}
    H = \sum_n \hbar \omega_n \left( a^\dagger_n a_n + \frac{1}{2} \right).
\end{equation}
这个哈密顿量的形式和真空中完全一样,不同的地方在于$\omega_{\vb*{k}}$被$\omega_n$取代,色散关系可能变得非常不一样。

既然$\epsilon$和$\mu$的概念对长波光子在量子情况下仍然适用,反射、折射等概念对长波光子仍然有意义,且和经典情况非常类似。
特别的,光场可能被约束在一个四面都是反射镜的腔体中,此时的光场被所谓的\concept{cavity QED}或者简写为\concept{cQED}描述。

一个介质系统中的量子化光场的自由哈密顿量就是普通的谐振子哈密顿量。
使用本质上是经典的方程\eqref{eq:photon-in-material},得到一系列振动模式,其频率即为这个介质系统中的量子化光场中的模式的频率,振动模式的场强分布就是\eqref{eq:photon-in-material}给出的本征模式。

\subsection{归一化和单光子电场}

现在设想我们在一个有限大小的空间中做光场量子化,不过该空间中还是能够良定义波矢。
这样我们只需要在\autoref{sec:quantization-in-vacuum}中做代换
\[
    \int \frac{\dd[3]{\vb*{k}}}{(2\pi)^3} \longrightarrow \frac{1}{V} \sum_{\vb*{k}},
\]
于是电场和磁场分别为
\begin{equation}
    \vb*{E}(\vb*{r}, t) = \frac{\ii}{V} \sum_{\vb*{k}} \sqrt{\frac{\hbar \omega_{\vb*{k}}}{2 \epsilon_0}} \sum_{\sigma=1}^2 (a_{\vb*{k} \sigma} \vu*{e}^\sigma \ee^{\ii \vb*{k} \cdot \vb*{r} - \ii \omega_{\vb*{k}} t} - a^\dagger_{\vb*{k} \sigma} (\vu*{e}^\sigma)^* \ee^{- \ii \vb*{k} \cdot \vb*{r} + \ii \omega_{\vb*{k}} t})
    \label{eq:vacuum-e-field-cavity-origin}
\end{equation}
和
\begin{equation}
    \vb*{B}(\vb*{r}, t) = \frac{\ii}{V} \sum_{\vb*{k}} \sqrt{\frac{\hbar}{2 \omega_{\vb*{k}} \epsilon_0}} \sum_{\sigma=1}^2 (a_{\vb*{k} \sigma} \vb*{k} \times \vu*{e}_\sigma \ee^{\ii \vb*{k} \cdot \vb*{r} - \ii \omega_{\vb*{k}} t} - a^\dagger_{\vb*{k} \sigma} \vb*{k} \times \vu*{e}_\sigma^* \ee^{- \ii \vb*{k} \cdot \vb*{r} + \ii \omega_{\vb*{k}} t}).
    \label{eq:vacuum-b-field-cavity-origin}
\end{equation}
不过我们注意到,此时哈密顿量为
\[
    H = \frac{1}{V} \sum_{\vb*{k}, \sigma} \hbar \omega_{\vb*{k}} \left( a^\dagger_{\vb*{k} \sigma} a_{\vb*{k} \sigma} + \frac{1}{2} \right),
\]
多出来一个不太美观的因子$V$;此外对易关系也是
\[
    \comm*{a_{\vb*{k} \sigma}}{a_{\vb*{k}' \sigma'}^\dagger} = (2\pi)^3 \delta(\vb*{k} - \vb*{k}') \delta_{\sigma \sigma'} \longrightarrow \frac{1}{V} \delta_{\vb*{k} \vb*{k}'} \delta_{\sigma \sigma'}.
\]
为此我们如下重新定义产生湮灭算符:
\begin{equation}
    a_{\vb*{k} \sigma} \longrightarrow \sqrt{V} a_{\vb*{k} \sigma},
\end{equation}
于是电场和磁场分别为
\begin{equation}
    \vb*{E}(\vb*{r}, t) = \ii \sum_{\vb*{k}} \sqrt{\frac{\hbar \omega_{\vb*{k}}}{2 \epsilon_0 V}} \sum_{\sigma=1}^2 (a_{\vb*{k} \sigma} \vu*{e}^\sigma \ee^{\ii \vb*{k} \cdot \vb*{r} - \ii \omega_{\vb*{k}} t} - a^\dagger_{\vb*{k} \sigma} (\vu*{e}^\sigma)^* \ee^{- \ii \vb*{k} \cdot \vb*{r} + \ii \omega_{\vb*{k}} t})
    \label{eq:vacuum-e-field-cavity-1}
\end{equation}
和
\begin{equation}
    \vb*{B}(\vb*{r}, t) = \ii \sum_{\vb*{k}} \sqrt{\frac{\hbar}{2 \omega_{\vb*{k}} \epsilon_0 V}} \sum_{\sigma=1}^2 (a_{\vb*{k} \sigma} \vb*{k} \times \vu*{e}_\sigma \ee^{\ii \vb*{k} \cdot \vb*{r} - \ii \omega_{\vb*{k}} t} - a^\dagger_{\vb*{k} \sigma} \vb*{k} \times \vu*{e}_\sigma^* \ee^{- \ii \vb*{k} \cdot \vb*{r} + \ii \omega_{\vb*{k}} t}),
    \label{eq:vacuum-b-field-cavity-1}
\end{equation}
哈密顿量为
\begin{equation}
    H = \sum_{\vb*{k}, \sigma} \hbar \omega_{\vb*{k}} \left( a^\dagger_{\vb*{k} \sigma} a_{\vb*{k} \sigma} + \frac{1}{2} \right),
\end{equation}
对易关系为
\begin{equation}
    \comm*{a_{\vb*{k} \sigma}}{a_{\vb*{k}' \sigma'}^\dagger} = \delta_{\vb*{k} \vb*{k}'} \delta_{\sigma \sigma'}.
\end{equation}
我们定义
\begin{equation}
    \mathcal{E}_{\vb*{k} \sigma} = \sqrt{\frac{\hbar \omega_{\vb*{k}}}{2 \epsilon_0 V}},
\end{equation}
则
\begin{equation}
    \vb*{E} = \sum_{\vb*{k}, \sigma} \mathcal{E}_{\vb*{k} \sigma} \vb*{f}_{\vb*{k} \sigma} a_{\vb*{k} \sigma} \ee^{- \ii \omega_{\vb*{k}} t} + \text{h.c.},
\end{equation}
其中
\begin{equation}
    \vb*{f}_{\vb*{k} \sigma} = \ii \ee^{\ii \vb*{k} \cdot \vb*{r}} \vu*{e}^\sigma,
\end{equation}
满足
\begin{equation}
    \frac{1}{V} \int \dd[3]{\vb*{r}} \vb*{f}^*_{\vb*{k} \sigma} \cdot \vb*{f}_{\vb*{k}' \sigma'} = \delta_{\vb*{k} \vb*{k}'} \delta_{\sigma \sigma'}.
\end{equation}
我们称$\mathcal{E}_{\vb*{k} \sigma}$为\concept{单光子电场},因为它给出了增多一个光子,电场大体上增多的幅度,而$\vb*{f}$则称为\concept{模式函数},它给出了光场的稳定振动方式,且不显含体积(体积出现在了积分前的归一化常数中)。

对一个一般的体系,我们会定义
\begin{equation}
    \mathcal{E}_{n} = \sqrt{\frac{\hbar \omega_n}{2 \epsilon_0 V}},
\end{equation}
从而电场为
\begin{equation}
    \vb*{E} = \sum_n \mathcal{E}_n \vb*{f}_{n} a_n \ee^{- \ii \omega_n t} + \text{h.c.},
    \label{eq:general-optical-field}
\end{equation}
其中
\begin{equation}
    \vb*{f}_n = \sqrt{\epsilon_0 V} \vb*{u}_n,
\end{equation}
且
\begin{equation}
    \frac{1}{V} \int \dd[3]{\vb*{r}} \vb*{f}_m^* \cdot \frac{\vb*{\epsilon}}{\epsilon_0} \cdot \vb*{f}_n = \delta_{mn}.
\end{equation}

实际上从这里我们可以看出,经典的麦克斯韦方程本身已经是一个具有一定量子特性的理论了——“单光子波函数”(虽然没有良定义的单光子量子力学,但是我们不妨这么指代$\mel*{0}{A^i(\vb*{r})}{\psi}$)服从的方程就是麦克斯韦方程。
也可以从另一个角度看这件事:在麦克斯韦方程两边乘上$\hbar$,由于$E \sim \hbar \omega$,得到的理论看上去就是一个量子理论。
将光场量子化引入的新物理只有两件事:光束由分立的光子构成,以及存在光子数的量子涨落,但是,在没有非线性光学效应的情况下,光子数目守恒,第一件事完全可以通过手动引入“光子”的概念并指派其波函数为(经过适当归一化的)经典电磁场来做到。
光的量子性只有在下面的地方才会变得重要:
\begin{itemize}
    \item 光子生灭明显,一些光子模式上原本没有电子而一段时间后有光子产生时,即处理非线性光学时,因为此时会有一些原本完全没有光子分布的模式上出现了光子。经典处理只能在有种子光的时候处理光子的产生——并不奇怪,因为光子从零到一产生的过程涉及一个极为弱的场强,弱到经典场论不再使用。
    \item 光子的“颗粒感”较为明显时。如即使多光子Fock态$\ket*{n}$中的$n$很大,由于其光子数目完全确定,我们有$\expval*{\vb*{E}}{n} = 0$而$\expval*{\vb*{E}^2}{n} \neq 0$,因为经过$\vb*{E}$中的产生湮灭算符作用后,光子数目发生变动,从而和原来的态完全正交。
    \item 纠缠重要时。经典电动力学面对“光子增多了”的描述方法是更大的场强,而没有直积的希尔伯特空间这样的概念,从而无法捕捉到纠缠。
\end{itemize}
可以看到这些光的量子性变得明显的情况都涉及多光子Fock态。
我们其实可以在这里看到一个相当有趣的情况:当所研究的问题中涉及非常弱的光场(如特定频率的光子一开始没有,但是一段时间后被产生)时,经典电动力学就失效了,然而经典电动力学的形式却又很像是在处理“单光子波函数”。
当然,这两者并没有矛盾,本质上是因为经典电动力学无法正确处理“多光子形成的多体波函数”:“单光子波函数”不涉及多体波函数,它给出的所有物理就是一个麦克斯韦方程,正好和经典电动力学一致;
相干态下电场的标准差相比于电场期望值本身比较小,从而电场能够被看成经典的量,同样可以被经典电动力学处理。
经典电动力学无法正确处理“多光子形成的多体波函数”,因为没有Fock空间;但是这并不是说经典电动力学就缺乏(相比于经典质点动力学的)一切量子性,如坐标和动量的不确定性等。
实际上,对缺乏纠缠、缺乏粒子生灭和碰撞、粒子数大的有质量粒子系统,单粒子波函数乘以适当的因子也可以诠释为“粒子数的平方根”。在这个意义上它和电磁场的地位是类似的。
当然,有质量粒子系统中有大量的碰撞,其宏观理论通常是动理学方程,且“经典费米场”不是一个物理意义特别明确的东西,因此我们很少看到“电子的宏观场”。
然而在电动力学以外的系统中,由于系统非常接近相干态,量子场能够被看成经典场的例子也是有的,如超流和超导中的序参量。

从本节的计算中也可以看到为什么很多时候经典电动力学已经够用了,因为一般的偶极辐射产生的就是相干光。要产生明显偏离相干光的辐射场实际上是很不容易的。

在本文中“电磁场”可能代表量子化的场算符,也可能代表经典电磁场,也可能代表“单光子波函数”。在不考虑非线性效应时这三者的时间演化是相同的。
在$\vb*{E}$被认为是经典场时,$\vb*{E}^2$——从而$I$——在形式上对应于“光子出现的概率”。
光场中被传输的不是$I$,电磁场的相位信息是很重要的,正如量子力学中叠加的不是概率而是概率振幅一样。
对非相干光(后文将讨论),可以直接将$I$相加,正如高度混合态的系统可以直接使用经典概率论处理一样。

\section{相干态和Wigner波函数}

如果我们直接计算电场在光子的Fock态下的期望值,无疑会得到零,因为$\vb*{E}$算符正比于单个产生算符和湮灭算符之和,从而,设$\ket*{\psi}$是一个光子Fock态,$\mel{\psi}{E_n}{\psi}$中,$E_n$作用到右边后模式$n$上的光子数目发生变化,因此电场期望值为零。

这看起来真是匪夷所思,因为这似乎说明Fock态中没有光子,没有能量,而这当然不正确。
例如,计算电场平方的期望值却又会得到非零结果。电场平方正比于该点的“电场能”,因此Fock态是有能量的。

问题的核心在于Fock态\emph{不是}电场的本征态。我们现在要找到一个和经典的电场接近的量子态,并且考虑如何用一种经典意义明显的方式表示一个任意的量子光学中的光场波函数。

\subsection{相干态}

\begin{back}{相干态和相干态路径积分}{coherent-state}
    设动力学变量$x$和它的正则动量满足正则对易关系
    \begin{equation}
        \comm{x}{p} = \ii,
    \end{equation}
    引入产生湮灭算符
    \begin{equation}
        a = \frac{1}{\sqrt{2}} (x + \ii p), \quad a^\dagger = \frac{1}{\sqrt{2}} (x - \ii p),
    \end{equation}
    则有正确的对易关系
    \begin{equation}
        \comm{a}{a^\dagger} = 1.
    \end{equation}
    我们现在考虑怎么将一个密度矩阵写成某种“准概率分布”的形式,即能够将它和一个函数$W(x, p)$或者$W(a, a^*)$建立线性关系。

    考虑用复参数定义的相干态
    \begin{equation}
        \ket*{\alpha} = \ee^{- \abs*{\alpha}^2 / 2} \sum_{n = 0}^\infty \frac{\alpha^n}{\sqrt{n!}} \ket*{0},
    \end{equation}
    其中$\ket*{n}$是谐振子哈密顿量
    \begin{equation}
        H = \sum_{n \geq 0} \hbar \omega \left( a^\dagger a + \frac{1}{2} \right) 
    \end{equation}
    的第$n$激发态。
    能够证明完备条件
    \begin{equation}
        \frac{1}{\pi} \int \dd[2]{\alpha} \dyad{\alpha} = 1
    \end{equation}
    成立,这里积分测度为
    \begin{equation}
        \dd[2]{\alpha} = \frac{\dd{\alpha^*} \wedge \dd{\alpha}}{2 \ii} = R \dd{R} \wedge \dd{\theta},
    \end{equation}
    其中$R$是$\alpha$的长度而$\theta$是相角,$\dd[2]{\alpha}$是复平面上的积分测度。
    由于不同$\alpha$的相干态彼此不正交——实际上,我们有
    \begin{equation}
        \braket*{\alpha}{\beta} = \ee^{- (\abs{\alpha}^2 + \abs{\beta}^2) / 2 + \alpha^* \beta},
    \end{equation}
    全体相干态实际上是超完备的。

\end{back}

我们在构造路径积分时遇到过形式最为一般的相干态。在量子光学中相干态还有特殊的意义。

设想空间中有一个振动频率给定的电偶极子
\begin{equation}
    \vb*{d}(t) = \vb*{d} \ee^{- \ii \omega t } + \text{h.c.},
\end{equation}
我们考虑空间中的光场的演化情况。设光场波函数为$\ket*{\psi}$,取偶极辐射近似。
在相互作用绘景中有
\begin{equation}
    \ii \pdv{t} \ket*{\psi} = - \vb*{d}(t) \cdot \vb*{E} \ket*{\psi}.
\end{equation}
假定$t=0$时没有任何光子,因此,等价地我们可以认为从$t=0$开始电偶极子才开始振动,在此之前系统中一直没有任何光子。
我们有形式解
\[
    \begin{aligned}
        \ket*{\psi(t)} &= \exp(\frac{\ii}{\hbar} \int_0^t \dd{t'} \sum_n (\vb*{d} \ee^{- \ii \omega t } + \text{h.c.}) \cdot (\mathcal{E}_n \vb*{f}_n a_n \ee^{- \ii \omega_n t'} + \text{h.c.}) ) \ket*{0} \\
        &= \exp(\sum_n (\alpha_n a_n^\dagger - \alpha_n^* a_n) ) \ket*{0},
    \end{aligned}
\]
这里我们定义
\begin{equation}
    \alpha_n \coloneqq \frac{\ii}{\hbar} \int_0^t \dd{t'} \mathcal{E}_n \vb*{f}_n^* \ee^{\ii \omega_n t'} \cdot (\vb*{d} \ee^{- \ii \omega t} + \text{h.c.}).
\end{equation}
由于
\[
    \comm*{a_n}{\comm*{a_n}{a^\dagger_n}} = \comm*{a^\dagger_n}{\comm*{a_n}{a^\dagger_n}} = 0,
\]
我们有
\[
    \begin{aligned}
        \exp(\alpha_n a_n^\dagger - \alpha_n^* a_n) &= \ee^{\alpha_n a_n^\dagger} \ee^{- \alpha_n^* a_n} \ee^{\frac{1}{2} \comm*{\alpha_n a_n^\dagger}{\alpha_n^* a_n}} \\
        &= \ee^{\alpha_n a_n^\dagger} \ee^{- \alpha_n^* a_n} \ee^{\frac{1}{2} \abs{\alpha_n}},
    \end{aligned}
\]
定义其为\concept{位移算符}
\begin{equation}
    D(\{\alpha_n\}) \coloneqq \prod_n \ee^{\alpha_n a_n^\dagger} \ee^{- \alpha_n^* a_n} \ee^{\frac{1}{2} \abs{\alpha_n}},
    \label{eq:displacement-operator}
\end{equation}
它作用在真空态上给出
\begin{equation}
    \ket*{\{\alpha_n\}} \coloneqq D(\{\alpha_n\}) \ket*{0} = \exp(\sum_n (\alpha_n a_n^\dagger - \alpha_n^* a_n) ) \ket*{0} = \prod_n \ee^{- \frac{\abs{\alpha_n}^2}{2}} \ee^{\alpha_n a^\dagger_n} \ket*{0_n}.
\end{equation}
将上式中的$\ee^{\alpha_n a^\dagger_n}$展开,我们发现
\begin{equation}
    \ket*{\{\alpha_n\}} = \prod_n \ee^{- \frac{\abs{\alpha_n}^2}{2}} \sum_{i_n=0}^\infty \frac{\alpha_n^{i_n}}{\sqrt{i_n!}} \ket*{0_n},
\end{equation}
因此它的确就是路径积分量子化中定义的相干态。
位移算符这个说法是比较直观的:它给出了$\alpha$的位移;此外注意到电偶极相互作用哈密顿量$- \vb*{d} \cdot \vb*{E}$就是“一个外场乘在动力学变量上”,在谐振子的物理图像中,相当于一个谐振子从某时刻起被放置在一个匀强外场中,形象地说,就是电梯中放置了一个谐振子,然后电梯突然启动了。

现在我们回来看看$\{\alpha_n\}$到底是什么,或者说相干态到底是什么。
实际上在路径积分量子化中我们已经知道系统处在相干态上意味着系统大体上服从经典场论了,不过还是做一些具体计算来更清楚地展示这一点。

为了更加简便我们考虑一个\concept{单模光场},即只考虑\eqref{eq:general-optical-field}中的一个模式。
这是合理的,因为一个线性体系中的所有模式之间不存在任何关系。
对某个特定的模式对应的产生湮灭算符$(a, a^\dagger)$定义
\begin{equation}
    X_1 = \frac{1}{2} (a + a^\dagger), \quad X_2 = \frac{1}{2\ii} (a - a^\dagger),
\end{equation}
则
\begin{equation}
    \vb*{E} = 2 \mathcal{E} (\vb*{f} X_1 \cos(\omega t) + \vb*{f}^* X_2 \sin(\omega t)).
\end{equation}
经典电动力学中系统总是在相干态附近,将$a$替换成$\alpha$就能够从量子光学过渡到经典光学。
在做了这个过渡之后,$X_1$大体上是$\Re{\alpha}$而$X_2$大体上就是$\Im{\alpha}$。
于是我们就看到了相干态$\ket*{\alpha}$的经典意义:一个相干态$\ket*{\alpha}$对应一个正弦振动的经典电场,其幅度为
\begin{equation}
    A = 2 \mathcal{E} \abs{\vb*{f}} \sqrt{X_1^2 + X_2^2} = 2 \mathcal{E} \abs{\vb*{f}} \abs{\alpha},
\end{equation}
其相位则是
\begin{equation}
    \varphi = \arg \alpha = \arctan \frac{X_2}{X_1}.
\end{equation}

$X_1$和$X_2$当然是不对易的,因此即使是相干态下(当然也包括其它相干态下),实际上电场
\begin{equation}
    E = A \ee^{\ii \varphi} \ee^{- \omega t} + \text{h.c.}
\end{equation}
是不能够完全确定的。或者也可以说,$A$(实际上就是粒子数开平方)和$\varphi$是不能同时完全确定的。

相干态下的光子数服从泊松分布。
因此,相干态的平均值和经典电场类似,但是存在涨落,其涨落行为和真空类似。

\subsection{准概率分布函数}

\begin{back}{量子力学中的相空间和准概率分布函数}{phase-wigner}
    虽然在量子力学中坐标和动量不能同时确定,从而看起来似乎不能够有良定义的相空间,不过注意到,算符$O$的矩阵元一般来说形如$\mel{x}{O}{x'}$,即需要两个标签——$x$和$x'$——标记一个矩阵元,那么我们将$x$和$x'$线性组合一下,将其中一个坐标做傅里叶变换切换到动量空间,似乎还是能够写出$O(x, p)$这样的式子,从而有一个等效的相空间。
    当然,这样得到的“相空间中的分布函数”,即满足下述条件的函数$f(x, p)$:
    \[
        \expval*{O} = \int \dd{x} \dd{p} O(x, p) f(x, p)
    \]
    未必能够赋予概率的意义,因为它可以取负值甚至虚数值。

    我们现在考虑几种$f(x, p)$的定义。最为知名的可能是Wigner函数,它基本上是经典力学中的粒子分布函数的量子推广。
    设$\rho$是一个密度矩阵,\concept{Wigner函数}定义为
    \begin{equation}
        W(x, p) = \frac{1}{2\pi \hbar} \int \dd{y} \mel{x - \frac{y}{2}}{\rho}{x + \frac{y}{2}} \ee^{\ii p y / \hbar}.
    \end{equation}
    当然也可以用$\alpha$和$\alpha^*$——或者说,用二维相平面上的复数$\alpha$——做Wigner函数的宗量。
    将算符$O$写成关于$a$和$a^\dagger$的\emph{对称排序}$O_\text{S}$之后,我们有
    \begin{equation}
        \expval*{O} = \int \dd[2]{\alpha} W(\alpha, \alpha^*) O_\text{S}(\alpha, \alpha^*).
    \end{equation}
    例如,
    \begin{equation}
        \frac{1}{2} \expval*{a a^\dagger + a^\dagger a} = \int \dd[2]{\alpha} W(\alpha, \alpha^*) \alpha \alpha^*.
    \end{equation}
    
    Wigner分布函数可以验证是实数,不过有正有负。能够证明Wigner函数取负值的区域不会太大——在几个$\hbar$以内——这直观地展示了从量子过渡到经典的过程。
    与路径积分类似,Wigner函数也并非只能在坐标-动量构成的相空间中定义——任何有广义坐标和广义动量的能够用哈密顿动力学描述的系统中的密度矩阵都能够用Wigner函数等价地给出。
    因此,Wigner函数实际上可以构成量子力学的另一种形式理论:所谓量子力学,就是将普通的概率分布拓展为Wigner函数的物理理论。

    我们当然也可以寻找一种函数$P(\alpha, \alpha^*)$使得
    \begin{equation}
        \expval*{O} = \int \dd[2]{\alpha} P(\alpha, \alpha^*) O_\text{N}(\alpha, \alpha^*),
    \end{equation}
    这里$O_\text{N}$表示算符$O$的\emph{正则排序},即将湮灭算符排在右边,产生算符排在左边。
    我们有
    \[
        \begin{aligned}
            \expval*{O} &= \sum_{m, n} c_{mn} \trace((a^\dagger)^m a^n \rho) \\
            &= \sum_{m, n} c_{mn} \int \dd[2]{\alpha} \trace((\alpha^*)^m \alpha^n \rho \delta(\alpha^* - a^\dagger) \delta(\alpha - a)) \\
            &= \int \dd[2]{\alpha} \sum_{m,n} c_{mn} (\alpha^*)^m \alpha^n \trace(\rho \delta(\alpha^* - a^\dagger) \delta(\alpha - a)),
        \end{aligned}
    \]
    上式最后一行的形式正是$O_\text{N}$乘以某个分布函数,于是我们得到
    \begin{equation}
        P(\alpha, \alpha^*) = \trace(\rho \delta(\alpha^* - a^\dagger) \delta(\alpha - a)).
    \end{equation}
    简单地验证会发现
    \begin{equation}
        \rho=\int \dd^{2} \alpha P\left(\alpha, \alpha^{*}\right)|\alpha\rangle\langle\alpha| .
    \end{equation}

    我们当然还可以让寻找一种函数$Q(\alpha, \alpha^*)$使得
    \begin{equation}
        \expval*{O} = \int \dd[2]{\alpha} Q(\alpha, \alpha^*) O_\text{A}(\alpha, \alpha^*),
    \end{equation}
    这里$O_\text{A}$表示算符$O$的\emph{反正则排序},即湮灭算符排在左边而产生算符排在右边。和$P$函数一样如法炮制,能够得到
    \begin{equation}
        Q\left(\alpha, \alpha^{*}\right)=\operatorname{tr}\left[\rho \delta(\alpha-a) \delta\left(\alpha^{*}-a^{\dagger}\right)\right].
    \end{equation}
    由于
    \[
        \begin{aligned}
            Q\left(\alpha, \alpha^{*}\right) &=\frac{1}{\pi} \operatorname{tr} \int d^{2} \alpha^{\prime}\left[\rho \delta(\alpha-a)\left|\alpha^{\prime}\right\rangle\left\langle\alpha^{\prime}\right| \delta\left(\alpha^{*}-a^{\dagger}\right)\right] \\
            &=\frac{1}{\pi} \operatorname{tr} \int d^{2} \alpha^{\prime}\left\{\rho \delta\left(\alpha-\alpha^{\prime}\right)\left|\alpha^{\prime}\right\rangle\left\langle\alpha^{\prime}\right| \delta\left[\alpha^{*}-\left(\alpha^{\prime}\right)^{*}\right]\right\} \\
            &=\frac{1}{\pi} \operatorname{tr}(\rho|\alpha\rangle\langle\alpha|) \\
            &=\frac{1}{\pi}\langle\alpha|\rho| \alpha\rangle,
            \end{aligned}
    \]
    我们有
    \begin{equation}
        Q(\alpha, \alpha^*) = \frac{1}{\pi} \mel{\alpha}{\rho}{\alpha},
    \end{equation}

    $P$函数,$W$函数,$Q$函数都能够写成某些算符的期望值的某种积分变换。
    我们有
    \begin{equation}
        \delta\left(\alpha^{*}-a^{\dagger}\right) \delta(\alpha-a) =\frac{1}{\pi^{2}} \int \dd^{2} \beta \exp \left[-\beta\left(\alpha^{*}-a^{\dagger}\right)\right] \exp \left[\beta^{*}(\alpha-a)\right],
    \end{equation}
    于是
    \begin{equation}
        P\left(\alpha, \alpha^{*}\right)=\frac{1}{\pi^{2}} \int \dd^{2} \beta \ee^{- \ii \beta \alpha^{*}- \ii \beta^{*} \alpha} C_\text{N}\left(\beta, \beta^{*}\right),
    \end{equation}
    其中
    \begin{equation}
        C_\text{N}\left(\beta, \beta^{*}\right)=\operatorname{tr}\left(\ee^{\ii \beta a^{\dagger}} \ee^{\ii \beta^{*} a} \rho\right).
    \end{equation}
    类似的
    \begin{equation}
        Q\left(\alpha, \alpha^{*}\right)=\frac{1}{\pi^{2}} \int \dd^{2} \beta \ee^{- \ii \beta \alpha^{*}- \ii \beta^{\cdot} \alpha} C_\text{A}\left(\beta, \beta^{*}\right),
    \end{equation}
    其中
    \begin{equation}
        C_\text{A}\left(\beta, \beta^{*}\right)=\operatorname{tr}\left(\ee^{\ii \beta^{*} a} \ee^{\ii \beta a^{\dagger}} \rho\right).
    \end{equation}
    对Wigner函数也有类似的公式:
    \begin{equation}
        W\left(\alpha, \alpha^{*}\right)=\frac{1}{\pi^{2}} \int \dd^{2} \beta \ee^{-\ii \beta \alpha^{*}- \ii \beta^{*} \alpha} C_\text{S}\left(\beta, \beta^{*}\right),
    \end{equation}
    其中
    \begin{equation}
        C_\text{S}\left(\beta, \beta^{*}\right)=\operatorname{tr}\left(\ee^{\ii \beta a^{\dagger}+ \ii \beta^{*} a} \rho\right).
    \end{equation}
\end{back}

\section{光场的多光子概率分布和相干性}

\subsection{关联函数}\label{sec:correlation-function-photon}

实用的光探测器基本上都是根据偶极相互作用哈密顿量$- \vb*{d} \cdot \vb*{E}$工作的,即一个凝聚态系统——这里是光传感器——吸收了一个光子,将电子激发出来,然后探测由此产生的光电流。
由于光传感器中电子的能量通常很低,不会激发出多少光子,实验中置于光场中的探测器探测到光信号的概率几乎完全由“光场中一个光子湮灭,光场掉落到一个能量更低的状态”这样的过程贡献。
因此,一个可观察量的期望值基本上就是
\[
    \expval*{O} = \sum_\text{final state $f$} \abs{\mel{f}{\text{photon annihilation}}{\psi}}^2 \times \text{a function of $f$},
\]
容易看出,只要将关于末态$\ket*{f}$的函数写成$a$和$a^\dagger$的正规序(注意我们将电子的波函数看成背景的,$\ket*{f}$只是光场的波函数),这样的一个期望值可以展开成一系列形如
\[
    \mel{\psi}{a^\dagger_{\vb*{q}_1} a^\dagger_{\vb*{q}_2} \cdots a_{\vb*{p}_2} a_{\vb*{p}_1}}{\psi}
\]
的正规序关联函数之和;考虑到“关于末态$\ket*{f}$的函数”很多时候就是光子数的函数,实际上大部分可观察量的期望值\emph{就是}一个正规序关联函数。
考虑到光场本身可能热化,更加一般的形式是
\[
    \trace(\rho a^\dagger_{\vb*{q}_1} a^\dagger_{\vb*{q}_2} \cdots a_{\vb*{p}_2 a_{\vb*{p}_2}}) \eqqcolon \expval*{a^\dagger_{\vb*{q}_1} a^\dagger_{\vb*{q}_2} \cdots a_{\vb*{p}_2 a_{\vb*{p}_2}}},
\]
其中$\rho$是光场的密度矩阵。同样,关于电场的有明确物理意义的关联函数均形如
\begin{equation}
    \expval*{\vb*{E}^-_{1'} \vb*{E}^-_{2'} \cdots \vb*{E}^+_2 \vb*{E}^+_1},
    \label{eq:correlation-function-optical-field}
\end{equation}
其中
\begin{equation}
    \vb*{E}^{+} (\vb*{r}, t) = \sum_n \mathcal{E}_n \vb*{f}_n(\vb*{r}) a_n \ee^{-\ii \omega_n t} , \quad \vb*{E}^{-} (\vb*{r}, t) = \sum_n \mathcal{E}_n \vb*{f}^*_n(\vb*{r}) a^\dagger_n \ee^{\ii \omega_n t}, \quad \vb*{E} = \vb*{E}^+ + \vb*{E}^-,
\end{equation}
符号$\vb*{E}^{\pm}_n$表示时空坐标$\vb*{r}_n, t_n$处的$\vb*{E}^\pm$场;为了保证正规序——实际上是为了保证因果性——我们要求对$\vb*{E}^+$算符的乘积序列有$t_1 < t_2 < \cdots$。
如果我们用光子激发出的电偶极矩作为观测量,那么\eqref{eq:correlation-function-optical-field}有直接的物理意义。

我们考虑在空间位置$\vb*{r}$探测到光子的概率,或者等价地说,在$\vb*{r}$处的基于电偶极相互作用的一个探测器附近湮灭了一个光子而有电子首受到激发。
根据费米黄金法则,大体上有
\[
    P_f \propto \mel{f}{- \vb*{d} \cdot \vb*{E}^+(\vb*{r})}{\psi},
\]
如果我们不关心激发出了一个怎样的电子(也即,不关心被湮灭的的是一个怎样的光子,相当于是除了光子粒子数变化了以外什么也不想知道),那么可以忽略末态$f$的细节而将所有$f$的概率直接非相干求和,得到
\[
    P \propto \sum_f \mel{f}{- \vb*{d} \cdot \vb*{E}^+(\vb*{r})}{\psi}.
\]
由于末态的完备性,我们就得到
\begin{equation}
    P(\vb*{r}) = \eta \expval{\vb*{E}^-(\vb*{r}) \cdot \vb*{E}^+(\vb*{r})},
\end{equation}
其中$\eta$是一个(一般来说难以确定的)关于探测器细节,可能还含有诸如$2\pi / \hbar$之类的费米黄金法则引入的因子的常数。
点积$\cdot$出现,是因为$\vb*{d}$可以指向各个方向,于是求和后它正比于单位张量,从而最后$(\vb*{E}^- \cdot \vb*{d}^\dagger) (\vb*{d} \cdot \vb*{E}^+)$就变成$\vb*{E}^- \cdot \vb*{E}^+$。
% TODO:完整的费米黄金法则计算

类似的,在$\vb*{r}_1, \vb*{r}_2, \ldots, \vb*{r}_N$同时探测到光子的概率为
\[
    P \propto \sum_{f} \prod_i (- \vb*{d} \cdot \vb*{E}^-(\vb*{r}_i)),
\]
经过完全一样的化简过程,我们发现最一般的$N$光子联合概率分布函数为
\begin{equation}
    P^{(N)}(\vb*{r}_1, \ldots, \vb*{r}_N) = \eta^N \expval*{\vb*{E}^-(\vb*{r}_1) \cdots \vb*{E}^-(\vb*{r}_N) \cdot \vb*{E}^+(\vb*{r}_N) \cdots \vb*{E}^+(\vb*{r}_1)},
    \label{eq:general-n-photon-correlation}
\end{equation}
其中$\eta$定义如前,点乘$\cdot$还是表示将$\vb*{E}^-(\vb*{r}_N)$和$\vb*{E}^+(\vb*{r}_N)$缩并,将$\vb*{E}^-(\vb*{r}_{N-1})$和$\vb*{E}^+(\vb*{r}_{N-1})$缩并等等。

以上推导都是针对纯态的,然而由于对末态$f$的求和是非相干求和,\eqref{eq:general-n-photon-correlation}对热态也适用。

简单的计算告诉我们,$N$粒子Fock态的

我们还可以定义\concept{相干度},即将$N$点关联函数归一化后得到的结果。
我们有
\begin{equation}
    g^{(N)}(\vb*{r}_1, \vb*{r}_2) = \frac{\expval*{\vb*{E}^-(\vb*{r}_1) \cdots \vb*{E}^-(\vb*{r}_N) \cdot \vb*{E}^+(\vb*{r}_N) \cdots \vb*{E}^+(\vb*{r}_1)}}{\expval*{\vb*{E}^-(\vb*{r}_1) \cdot \vb*{E}^+(\vb*{r}_1)} \cdots \expval*{\vb*{E}^-(\vb*{r}_N) \cdot \vb*{E}^+(\vb*{r}_N)}},
\end{equation}
它给出一个无量纲的,衡量光场中多点相干性的量。
我们对两点相干度或者说一阶相干度比较熟悉。用狄拉克的话说,这是“光子自己和自己干涉”:这里最重要的是计算$\expval*{\vb*{E}^+(\vb*{r}_1) \vb*{E}^-(\vb*{r}_2)}$,它体现的是单光子波函数的空间分布。

能够证明相干态的任意阶相干度都是1。

对Fock态的相干度,我们有
\begin{equation}
    g^{(2)}(\vb*{r}) = 1 - \frac{1}{n},
\end{equation}
其中$n$为光子数。这是非常合理的:例如,$n=1$时$g^{(2)} = 0$,因为系统中确定只有一个光子,这个光子在一个地方被湮灭了以后无论如何不可能出现在另一个地方。

对单模光场的热态
\begin{equation}
    \rho = \sum_n \frac{\bar{n}^n}{(1 + \bar{n})^{n+1}} \dyad{n},
\end{equation}
我们有
\[
    \begin{aligned}
        &\quad \expval{\vb*{E}^-(\vb*{r}_1) \cdots \vb*{E}^-(\vb*{r}_N) \cdot \vb*{E}^+(\vb*{r}_N) \cdots \vb*{E}^+(\vb*{r}_1)} \\
        &= \sum_n 
    \end{aligned}
\]
例如,
\begin{equation}
    g^{(2)} = 2.
\end{equation}
相当奇怪的,在热态中,观察到一个光子会让观察到下一个光子的概率\emph{增加}。

\subsection{Hanbury Brown和Twiss效应}

\subsubsection{测量天狼星的张角}

\begin{figure}
    \centering
    

\tikzset{every picture/.style={line width=0.75pt}} %set default line width to 0.75pt        

\begin{tikzpicture}[x=0.75pt,y=0.75pt,yscale=-1,xscale=1]
%uncomment if require: \path (0,300); %set diagram left start at 0, and has height of 300

%Shape: Star [id:dp7993730556190444] 
\draw  [fill={rgb, 255:red, 144; green, 19; blue, 254 }  ,fill opacity=0.07 ] (115,22.33) -- (117.35,27.3) -- (122.61,28.09) -- (118.8,31.95) -- (119.7,37.41) -- (115,34.83) -- (110.3,37.41) -- (111.2,31.95) -- (107.39,28.09) -- (112.65,27.3) -- cycle ;
%Straight Lines [id:da6010958544087177] 
\draw    (107.39,28.09) -- (119,208.33) ;
\draw [shift={(113.2,118.21)}, rotate = 266.31] [fill={rgb, 255:red, 0; green, 0; blue, 0 }  ][line width=0.08]  [draw opacity=0] (12,-3) -- (0,0) -- (12,3) -- cycle    ;
%Straight Lines [id:da5792212007108453] 
\draw    (122.61,28.09) -- (119,208.33) ;
\draw [shift={(120.8,118.21)}, rotate = 271.15] [fill={rgb, 255:red, 0; green, 0; blue, 0 }  ][line width=0.08]  [draw opacity=0] (12,-3) -- (0,0) -- (12,3) -- cycle    ;
%Shape: Arc [id:dp02266441256288565] 
\draw  [draw opacity=0] (88.36,213.69) .. controls (97.39,210.04) and (108.63,208.04) .. (120.75,208.37) .. controls (130.27,208.63) and (139.16,210.3) .. (146.81,213) -- (119.93,238.36) -- cycle ; \draw   (88.36,213.69) .. controls (97.39,210.04) and (108.63,208.04) .. (120.75,208.37) .. controls (130.27,208.63) and (139.16,210.3) .. (146.81,213) ;
%Curve Lines [id:da2800220853905422] 
\draw    (110,76.33) .. controls (118,77.33) and (118,75.33) .. (122,77) ;
%Straight Lines [id:da7563818948167416] 
\draw    (107.39,28.09) -- (96,209.33) ;
\draw [shift={(101.7,118.71)}, rotate = 273.6] [fill={rgb, 255:red, 0; green, 0; blue, 0 }  ][line width=0.08]  [draw opacity=0] (12,-3) -- (0,0) -- (12,3) -- cycle    ;
%Straight Lines [id:da6535572732710992] 
\draw    (122.61,29.09) -- (96,210.33) ;
\draw [shift={(109.3,119.71)}, rotate = 278.35] [fill={rgb, 255:red, 0; green, 0; blue, 0 }  ][line width=0.08]  [draw opacity=0] (12,-3) -- (0,0) -- (12,3) -- cycle    ;

% Text Node
\draw (126,65.4) node [anchor=north west][inner sep=0.75pt]    {$\theta $};
% Text Node
\draw (82,214.4) node [anchor=north west][inner sep=0.75pt]    {$I_{1}$};
% Text Node
\draw (116,209.4) node [anchor=north west][inner sep=0.75pt]    {$I_{2}$};


\end{tikzpicture}

    \caption{HBT实验测量天狼星的张角}
    \label{fig:hbt-initial-experiment}
\end{figure}

1954年两位电气工程师Robert Hanbury Brown%
\footnote{
    Hanbury Brown是一个复姓。
}%
和Richard Q. Twiss试图想出一种方法测量天狼星的张角。如\autoref{fig:hbt-initial-experiment}所示,我们在地球上安装两个相隔一定距离的探测器,用它们接受天狼星两端的光。
与通常的干涉实验不同,Hanbury Brown和Twiss测量\emph{二阶相干度}

这个说法——遥远星体上相隔很远的两点产生的光不知道怎么回事具有很好的相干性——不出意外地引起了轩然大波。

\subsubsection{经典电动力学解释}

\subsubsection{两个激光器产生的光束的干涉}

两个激光器产生的光似乎是相干的?

实际上激光器产生的是相干态光而不是光子数确定的多光子玻色波函数。

\subsubsection{玻色-爱因斯坦凝聚态中的干涉}

\chapter{线性量子光学过程}

\section{线性过程}

我们这一节要做的事情和\autoref{chap:linear-matter-no-scattering}中没有什么本质上的区别:我们讨论线性过程——或者说没有光子生灭的过程。
对这样的一个过程,我们选择海森堡绘景,设$\{a_k\}$是某组模式,$\{b_l \}$是它经过一个线性过程演化之后的模式,即\begin{equation}
    b_l^\dagger = S_{lk} a^\dagger_k,
\end{equation}
且根据幺正性要求我们有
\begin{equation}
    a_k^\dagger = S_{kl}^* b^\dagger_l,
\end{equation}
则使用$a$算符表述的系统状态是
\begin{equation}
    \ket*{\psi}_\text{in} = f(\{ a_k^\dagger \}) \ket*{0},
\end{equation}
而使用$b$算符表述的系统状态是
\begin{equation}
    \ket*{\psi}_\text{out} = f(\{ S_{kl}^* b_l^\dagger \}) \ket*{0}.
\end{equation}
例如,对相干态我们有
\begin{equation}
    \ket*{\alpha} = 
\end{equation}
因此我们有
\begin{equation}
    \beta_l = S_{lk}^* \alpha_k ,
\end{equation}
即相干态的标签$\alpha$的时间演化方式和$a$算符完全一样,正好是预期的结果。

应当注意这里的$S$矩阵\emph{不是}薛定谔绘景中经过元件前后的光场的量子态的演化算符,而是它的共轭转置。
这个微妙之处在量子场论中就出现过:在一个自由理论中,薛定谔绘景下单粒子波函数的运动方程和海森堡绘景中\emph{湮灭}算符的时间演化方程保持一致,而不是和产生算符。
这里的关键点在于在海森堡绘景中态矢量根本不应该发生任何变化,海森堡绘景中如果某个矢量发生了时间演化,那是因为用于定义这个矢量的算符发生了时间演化,我们自然没有理由要求这种所谓的“矢量的时间演化”遵循薛定谔绘景中的演化方程。

我们还应当注意$S$矩阵实际上有两重地位:它一方面是一个时间演化算符,另一方面其实也是一个类似于边界条件的东西:线性光学元件入射和反射端口处的模式\emph{必定}满足关系式$b^\dagger = S a^\dagger$。
这两重身份是兼容的:前者用高能物理中的话说是在约束$t = -\infty$的入射态和$t = \infty$的出射态之间的关系,后者是在描写散射定态,这两者当然是有关的。
类似的,我们在讨论非线性晶体时是在求解非线性晶体内部的\emph{稳态}电场分布,计算得到的是各种模式上的能量分布和透射深度的关系,但是由于散射定态决定了$t = -\infty$的入射态和$t = \infty$的出射态之间的关系,实际上稳态电场分布给出的电场随着透射深度的变化就反映了在非线性晶体的一侧输入一个脉冲后输出端的脉冲的成分。

\subsection{分束器}

\begin{figure}
    \centering
    \subfigure[]{
        

\tikzset{every picture/.style={line width=0.75pt}} %set default line width to 0.75pt        

\begin{tikzpicture}[x=0.75pt,y=0.75pt,yscale=-1,xscale=1]
%uncomment if require: \path (0,300); %set diagram left start at 0, and has height of 300

%Shape: Square [id:dp7292498374829528] 
\draw   (235,140) -- (261,140) -- (261,166) -- (235,166) -- cycle ;
%Straight Lines [id:da010139268745176677] 
\draw    (235,166) -- (261,140) ;

%Straight Lines [id:da5573883310885233] 
\draw    (168,153) -- (248,153) ;
\draw [shift={(208,153)}, rotate = 180] [fill={rgb, 255:red, 0; green, 0; blue, 0 }  ][line width=0.08]  [draw opacity=0] (12,-3) -- (0,0) -- (12,3) -- cycle    ;
%Straight Lines [id:da26712641506196255] 
\draw    (248,153) -- (353.71,153) ;
\draw [shift={(300.85,153)}, rotate = 180] [fill={rgb, 255:red, 0; green, 0; blue, 0 }  ][line width=0.08]  [draw opacity=0] (12,-3) -- (0,0) -- (12,3) -- cycle    ;
%Straight Lines [id:da6641395542938842] 
\draw    (248,153) -- (248,221.67) ;
\draw [shift={(248,187.33)}, rotate = 270] [fill={rgb, 255:red, 0; green, 0; blue, 0 }  ][line width=0.08]  [draw opacity=0] (12,-3) -- (0,0) -- (12,3) -- cycle    ;
%Straight Lines [id:da6178054140461928] 
\draw    (248,85.33) -- (248,153) ;
\draw [shift={(248,119.17)}, rotate = 270] [fill={rgb, 255:red, 0; green, 0; blue, 0 }  ][line width=0.08]  [draw opacity=0] (12,-3) -- (0,0) -- (12,3) -- cycle    ;
%Straight Lines [id:da6569709591508739] 
\draw    (248,221.67) -- (353.71,221.67) ;
%Shape: Chord [id:dp4222004696727377] 
\draw   (353.82,137.75) .. controls (363.01,137.77) and (370.52,144.37) .. (370.69,152.67) .. controls (370.85,161.09) and (363.38,168.06) .. (354,168.25) -- cycle ;
%Shape: Chord [id:dp5038597778456795] 
\draw   (353.82,206.42) .. controls (363.01,206.43) and (370.52,213.04) .. (370.69,221.33) .. controls (370.85,229.76) and (363.38,236.73) .. (354,236.91) -- cycle ;

% Text Node
\draw (375.71,153) node [anchor=west] [inner sep=0.75pt]    {$n_{1}$};
% Text Node
\draw (373.71,221) node [anchor=west] [inner sep=0.75pt]    {$n_{2}$};
% Text Node
\draw (253,74.4) node [anchor=north west][inner sep=0.75pt]    {$\ket{0}$};


\end{tikzpicture}

        \label{fig:beam-splitter-1}
    }
    \subfigure[]{
        

\tikzset{every picture/.style={line width=0.75pt}} %set default line width to 0.75pt        

\begin{tikzpicture}[x=0.75pt,y=0.75pt,yscale=-1,xscale=1]
%uncomment if require: \path (0,300); %set diagram left start at 0, and has height of 300

%Shape: Square [id:dp6835515765563589] 
\draw   (434,169) -- (460,169) -- (460,195) -- (434,195) -- cycle ;
%Straight Lines [id:da46226803039130826] 
\draw    (434,195) -- (460,169) ;

%Straight Lines [id:da1670024331738862] 
\draw    (367,182) -- (447,182) ;
\draw [shift={(407,182)}, rotate = 180] [fill={rgb, 255:red, 0; green, 0; blue, 0 }  ][line width=0.08]  [draw opacity=0] (12,-3) -- (0,0) -- (12,3) -- cycle    ;
%Straight Lines [id:da9323888654594039] 
\draw    (447,182) -- (521,182) ;
\draw [shift={(484,182)}, rotate = 180] [fill={rgb, 255:red, 0; green, 0; blue, 0 }  ][line width=0.08]  [draw opacity=0] (12,-3) -- (0,0) -- (12,3) -- cycle    ;
%Straight Lines [id:da9013055724976109] 
\draw    (447,182) -- (447,250.67) ;
\draw [shift={(447,216.33)}, rotate = 90] [fill={rgb, 255:red, 0; green, 0; blue, 0 }  ][line width=0.08]  [draw opacity=0] (12,-3) -- (0,0) -- (12,3) -- cycle    ;
%Straight Lines [id:da04689085671987714] 
\draw    (447,114.33) -- (447,182) ;
\draw [shift={(447,148.17)}, rotate = 90] [fill={rgb, 255:red, 0; green, 0; blue, 0 }  ][line width=0.08]  [draw opacity=0] (12,-3) -- (0,0) -- (12,3) -- cycle    ;




\end{tikzpicture}

        \label{fig:beam-splitter-2}
    }
    \caption{分束器}
    \label{fig:beam-splitter}
\end{figure}

一个分束器是一个形如\autoref{fig:beam-splitter}的装置,最简单的分束器是一个有特殊镀膜的玻璃片(\autoref{fig:beam-splitter-1}),也可以是如今实验上更加常见的两个等腰直角三角形夹着特殊镀膜的结构(\autoref{fig:beam-splitter-2})。
它将两束光变换为另外两束光。分束器的变换矩阵为
\begin{equation}
    \pmqty{\tilde{\mathcal{E}}_1 \\ \tilde{\mathcal{E}}_2} = \pmqty{t & r \\ - r^* & t} \pmqty{\mathcal{E}_1 \\ \mathcal{E}_2}.
\end{equation}
这里,$-r^*$的负号来自幺正性的要求;实际上这就是经典电动力学中的半波损失。
例如,我们有\SI{50}{\percent}分束器,其变换矩阵为
\begin{equation}
    S = \frac{1}{\sqrt{2}} \pmqty{1 & 1 \\ -1 & 1}.
\end{equation}

\subsection{反射镜}

\section{线性光路}

\subsection{Aspect实验}

要证明单光子具有量子性,最好的办法是使用一些这样的实验:它的一种版本能够证明光子的粒子性,它的另一个仅仅做了少许修正的版本(比如说探测器被移动到别的位置)能够证明光子的波动性。
两个版本区别很小这件事能够排除实验装置和光的复杂相互作用显著地改变了光的行为这样的说法,而粒子性和波动性同时出现则强烈暗示需要量子理论描述光。
1986年的Aspect实验是这种实验的一个典范。

\begin{equation}
    \begin{aligned}
        S &= \frac{1}{\sqrt{2}} \pmqty{1 & 1 \\ -1 & 1} \pmqty{ \dmat{\ee^{\ii \varphi / 2}, \ee^{- \ii \varphi / 2}} } \frac{1}{\sqrt{2}} \pmqty{1 & 1 \\ -1 & 1}  \\
        &= \pmqty{\cos\frac{\varphi}{2} & \sin\frac{\varphi}{2} \\ - \sin\frac{\varphi}{2} & \cos\frac{\varphi}{2}}.
    \end{aligned}
\end{equation}

\subsection{Zeilinger实验}

A.Zeilinger
一种更加简明的实验是这样的:同样使用分束器和反射镜,构造这样的光路:

\subsection{Hong-Ou-Mondel效应}

\section{线性单光子量子计算}

\subsection{单光子态产生和条件量子态}

\begin{figure}
    \centering
    

\tikzset{every picture/.style={line width=0.75pt}} %set default line width to 0.75pt        

\begin{tikzpicture}[x=0.75pt,y=0.75pt,yscale=-1,xscale=1]
%uncomment if require: \path (0,300); %set diagram left start at 0, and has height of 300

%Shape: Rectangle [id:dp9242602312444221] 
\draw   (193,119) -- (263,119) -- (263,159) -- (193,159) -- cycle ;
%Straight Lines [id:da7075004076272056] 
\draw    (127,140) -- (193,140) ;
\draw [shift={(160,140)}, rotate = 180] [fill={rgb, 255:red, 0; green, 0; blue, 0 }  ][line width=0.08]  [draw opacity=0] (12,-3) -- (0,0) -- (12,3) -- cycle    ;
%Straight Lines [id:da42996678105947117] 
\draw    (263,140) -- (348,79) ;
\draw [shift={(305.5,109.5)}, rotate = 504.33] [fill={rgb, 255:red, 0; green, 0; blue, 0 }  ][line width=0.08]  [draw opacity=0] (12,-3) -- (0,0) -- (12,3) -- cycle    ;
%Straight Lines [id:da7812145894988733] 
\draw    (263,140) -- (348,201) ;
\draw [shift={(305.5,170.5)}, rotate = 215.67000000000002] [fill={rgb, 255:red, 0; green, 0; blue, 0 }  ][line width=0.08]  [draw opacity=0] (12,-3) -- (0,0) -- (12,3) -- cycle    ;
%Straight Lines [id:da29099064424770593] 
\draw    (348,201) -- (426,201) ;
\draw [shift={(387,201)}, rotate = 180] [fill={rgb, 255:red, 0; green, 0; blue, 0 }  ][line width=0.08]  [draw opacity=0] (12,-3) -- (0,0) -- (12,3) -- cycle    ;
%Shape: Chord [id:dp5153738151961664] 
\draw   (425.67,179) .. controls (438.92,179.02) and (449.76,188.55) .. (450,200.52) .. controls (450.23,212.67) and (439.46,222.73) .. (425.93,223) -- cycle ;
%Straight Lines [id:da7719196407492437] 
\draw  [dash pattern={on 4.5pt off 4.5pt}]  (450,201) -- (466,201) ;
%Straight Lines [id:da27041501657550526] 
\draw    (348,79) -- (460,79) ;
\draw [shift={(404,79)}, rotate = 180] [fill={rgb, 255:red, 0; green, 0; blue, 0 }  ][line width=0.08]  [draw opacity=0] (12,-3) -- (0,0) -- (12,3) -- cycle    ;
%Straight Lines [id:da20605124142789344] 
\draw  [dash pattern={on 4.5pt off 4.5pt}]  (466,105.33) -- (466,201) ;
%Shape: Rectangle [id:dp3038430486229844] 
\draw   (471,63) -- (459.92,63) -- (459.92,105.33) -- (471,105.33) -- cycle ;
%Straight Lines [id:da18864355508447694] 
\draw    (471,79) -- (526,79) ;
\draw [shift={(498.5,79)}, rotate = 180] [fill={rgb, 255:red, 0; green, 0; blue, 0 }  ][line width=0.08]  [draw opacity=0] (12,-3) -- (0,0) -- (12,3) -- cycle    ;

% Text Node
\draw (228,139) node   [align=left] {SHG};
% Text Node
\draw (118,115.4) node [anchor=north west][inner sep=0.75pt]    {$E_{\text{p}}$};
% Text Node
\draw (350,54.4) node [anchor=north west][inner sep=0.75pt]    {$E_{1}$};
% Text Node
\draw (346,177.4) node [anchor=north west][inner sep=0.75pt]    {$E_{2}$};


\end{tikzpicture}

    \caption{产生条件量子态的装置}
    \label{fig:condition-photon}
\end{figure}

考虑\autoref{sec:chi-2-wave}的量子版本,它会引入相互作用哈密顿量
\begin{equation}
    V = \int \dd[3]{\vb*{r}} \chi^{(2)} : \vb*{E}_\text{p}^\dagger \vb*{E}_1 \vb*{E}_2 + \text{h.c.},
\end{equation}
其中$\vb*{E}_\text{p}$为泵浦光;通常泵浦光是强激光,从而可以认为处在相干态上。
出射态可以相当好地被下式近似:% TODO
\begin{equation}
    \ket{\psi_\text{out}} = \exp(\beta a_1^\dagger a_2^\dagger - \text{h.c.}) \ket{0}.
\end{equation}

现在我们制造一个非幺正的装置,展示如\autoref{fig:condition-photon}:它首先测量$E_2$模式上的光子数,如果发现光子数不是$1$,那么启动一个挡板,将$E_1$模式上的光子全部吸收掉;反之,如果发现光子数真的是$1$,那么让$E_1$模式上的光子通过。
最终我们会得到一个混合态:它有一定的概率是$E_1$上的单光子态,而剩下的可能性则是什么也没有。
并且,我们能够通过$E_2$处的探测器的探测结果,明确地知道我们是否成功制备了一个单光子态。
我们得到的$E_1$态是\concept{条件量子态}的一个例子。

\subsection{量子门}

条件量子态的产生后我们需要做量子控制,即用一个光子控制另一个光子;这可以称为\concept{单光子非线性性},因为对应的哈密顿量不再能够写成电磁场的二次型,而是必须含有非线性光学过程。
因此,使用前面列出的纯粹线性的操作,是肯定不足以实现通用量子计算的。

最为直接的想法是,我们能够找到一个足够强的二阶非线性晶体,那么就能够实现:这意味着我们可以。
例如,就SHG过程而言,如果它足够强,使得两个不同模式的光子共同存在时就能够聚合为一个另一个模式的光子,那么这就是一个与门。
可惜的是目前在普通的非线性晶体中无法实现这种过程:这些非线性光学过程都实在是太弱了。冷原子体系能够产生强烈的光子-光子耦合,这是一个可能的方向。
我们这里则致力于通过一些巧妙的含有测量的构造来实现\emph{等效}的量子门。

\begin{figure}
    \centering
    

\tikzset{every picture/.style={line width=0.75pt}} %set default line width to 0.75pt        

\begin{tikzpicture}[x=0.75pt,y=0.75pt,yscale=-1,xscale=1]
%uncomment if require: \path (0,300); %set diagram left start at 0, and has height of 300

%Shape: Square [id:dp5812640844481787] 
\draw   (206,108) -- (232,108) -- (232,134) -- (206,134) -- cycle ;
%Straight Lines [id:da23406010312221892] 
\draw    (206,134) -- (232,108) ;

%Straight Lines [id:da12470247976338489] 
\draw    (105,121) -- (219,121) ;
\draw [shift={(162,121)}, rotate = 180] [fill={rgb, 255:red, 0; green, 0; blue, 0 }  ][line width=0.08]  [draw opacity=0] (12,-3) -- (0,0) -- (12,3) -- cycle    ;
%Straight Lines [id:da8231064877854108] 
\draw    (219,121) -- (333,121) ;
%Straight Lines [id:da03344738041998285] 
\draw    (219,121) -- (219,211.33) ;
\draw [shift={(219,166.17)}, rotate = 90] [fill={rgb, 255:red, 0; green, 0; blue, 0 }  ][line width=0.08]  [draw opacity=0] (12,-3) -- (0,0) -- (12,3) -- cycle    ;
%Straight Lines [id:da3041250062025471] 
\draw    (219,53.33) -- (219,121) ;
\draw [shift={(219,87.17)}, rotate = 90] [fill={rgb, 255:red, 0; green, 0; blue, 0 }  ][line width=0.08]  [draw opacity=0] (12,-3) -- (0,0) -- (12,3) -- cycle    ;
%Shape: Square [id:dp9371797224374869] 
\draw   (320,108) -- (346,108) -- (346,134) -- (320,134) -- cycle ;
%Straight Lines [id:da05948010816759908] 
\draw    (320,134) -- (346,108) ;

%Straight Lines [id:da4093214516440322] 
\draw    (333,121) -- (447,121) ;
\draw [shift={(390,121)}, rotate = 180] [fill={rgb, 255:red, 0; green, 0; blue, 0 }  ][line width=0.08]  [draw opacity=0] (12,-3) -- (0,0) -- (12,3) -- cycle    ;
%Straight Lines [id:da3614482552806686] 
\draw    (333,121) -- (333,211.33) ;
\draw [shift={(333,166.17)}, rotate = 90] [fill={rgb, 255:red, 0; green, 0; blue, 0 }  ][line width=0.08]  [draw opacity=0] (12,-3) -- (0,0) -- (12,3) -- cycle    ;
%Straight Lines [id:da7606039605325201] 
\draw    (333,53.33) -- (333,121) ;
\draw [shift={(333,87.17)}, rotate = 90] [fill={rgb, 255:red, 0; green, 0; blue, 0 }  ][line width=0.08]  [draw opacity=0] (12,-3) -- (0,0) -- (12,3) -- cycle    ;
%Straight Lines [id:da4133297014948891] 
\draw    (219,53.33) -- (274,53.33) ;
%Straight Lines [id:da6242316516385615] 
\draw    (333,53.33) -- (388,53.33) ;
%Shape: Chord [id:dp8695097730726187] 
\draw   (274.11,38.08) .. controls (283.31,38.1) and (290.82,44.7) .. (290.98,53) .. controls (291.14,61.42) and (283.68,68.4) .. (274.3,68.58) -- cycle ;
%Shape: Chord [id:dp8381933152038832] 
\draw   (388.11,38.08) .. controls (397.31,38.1) and (404.82,44.7) .. (404.98,53) .. controls (405.14,61.42) and (397.68,68.4) .. (388.3,68.58) -- cycle ;

% Text Node
\draw (14,98.4) node [anchor=north west][inner sep=0.75pt]    {$\alpha \ket{0} +\beta \ket{1} +\gamma \ket{2}$};
% Text Node
\draw (204,214.4) node [anchor=north west][inner sep=0.75pt]    {$\ket{1}$};
% Text Node
\draw (322,212.4) node [anchor=north west][inner sep=0.75pt]    {$\ket{0}$};


\end{tikzpicture}

    \caption{通过观察实现受控$\pi$相位门(或者说受控符号门)}
    \label{fig:controled-phase-gate-by-obs}
\end{figure}

受控相位门可以通过观察以及条件量子态实现。光路见\autoref{fig:controled-phase-gate-by-obs},计算发现末态为
\begin{equation}
    \ket*{\psi_\text{c}'} = \alpha \ket*{0} + \beta \ket*{1} - \gamma \ket*{2}
\end{equation}

为了更加清楚地展示光路中的各种操作的时间顺序,我们通常会将\autoref{fig:controled-phase-gate-by-obs}这样的光路图进一步抽象,将它画成

\chapter{量子光学测量}

本章我们讨论怎样通过测量一个量子光学态来重建关于它的性质。大部分情况下我们只能够测量光子数期望,然而量子光学中的信息远不止光子数期望——例如,形如
\[
    \ket{n_1} + \ee^{\ii \theta} \ket{n_2}
\]
的态就含有有物理意义的相位信息,而通过测量光子数是测不出来这个相位的。

\section{相位的零差探测}

Homodyne

\begin{figure}
    \centering
    

\tikzset{every picture/.style={line width=0.75pt}} %set default line width to 0.75pt        

\begin{tikzpicture}[x=0.75pt,y=0.75pt,yscale=-1,xscale=1]
%uncomment if require: \path (0,300); %set diagram left start at 0, and has height of 300

%Straight Lines [id:da828543766930488] 
\draw    (228,144) -- (322,144) ;
\draw [shift={(275,144)}, rotate = 180] [fill={rgb, 255:red, 0; green, 0; blue, 0 }  ][line width=0.08]  [draw opacity=0] (12,-3) -- (0,0) -- (12,3) -- cycle    ;
%Shape: Square [id:dp6034028374707234] 
\draw   (335,131) -- (309,131) -- (309,157) -- (335,157) -- cycle ;
%Straight Lines [id:da08340572984807304] 
\draw    (335,157) -- (309,131) ;

%Straight Lines [id:da08559705327208378] 
\draw    (322,58) -- (322,144) ;
\draw [shift={(322,101)}, rotate = 270] [fill={rgb, 255:red, 0; green, 0; blue, 0 }  ][line width=0.08]  [draw opacity=0] (12,-3) -- (0,0) -- (12,3) -- cycle    ;
%Straight Lines [id:da571324279939923] 
\draw    (322,144) -- (427,144) ;
\draw [shift={(374.5,144)}, rotate = 180] [fill={rgb, 255:red, 0; green, 0; blue, 0 }  ][line width=0.08]  [draw opacity=0] (12,-3) -- (0,0) -- (12,3) -- cycle    ;
%Shape: Chord [id:dp5477809121009893] 
\draw   (427.11,128.08) .. controls (436.31,128.1) and (443.82,134.7) .. (443.98,143) .. controls (444.14,151.42) and (436.68,158.4) .. (427.3,158.58) -- cycle ;
%Shape: Rectangle [id:dp5820030129707479] 
\draw   (129,123) -- (228,123) -- (228,163) -- (129,163) -- cycle ;
%Straight Lines [id:da3527058340560272] 
\draw    (322,144) -- (322,230) ;
\draw [shift={(322,187)}, rotate = 270] [fill={rgb, 255:red, 0; green, 0; blue, 0 }  ][line width=0.08]  [draw opacity=0] (12,-3) -- (0,0) -- (12,3) -- cycle    ;

% Text Node
\draw (138,123) node [anchor=north west][inner sep=0.75pt]   [align=left] {Source to be \\evaluated};
% Text Node
\draw (322,54.6) node [anchor=south] [inner sep=0.75pt]    {$\ket{\beta _{0}}$};
% Text Node
\draw (259,120.4) node [anchor=north west][inner sep=0.75pt]    {$b_{1}$};
% Text Node
\draw (324,80.4) node [anchor=north west][inner sep=0.75pt]    {$b_{2}$};
% Text Node
\draw (323,206.4) node [anchor=north west][inner sep=0.75pt]    {$a_{2}$};
% Text Node
\draw (369,120.4) node [anchor=north west][inner sep=0.75pt]    {$a_{1}$};
% Text Node
\draw (448,140.7) node [anchor=west] [inner sep=0.75pt]    {$\langle a^{\dagger } a\rangle $};


\end{tikzpicture}

    \caption{相位测量装置1}
    \label{fig:phase-detection-1}
\end{figure}

\autoref{fig:phase-detection-1}是一种可能的探测方式。我们将待测量的光源的态记作
\begin{equation}
    \ket{\text{source}} = f(b_1^\dagger) \ket{0},
\end{equation}
我们将它和一个相干态$\ket{\beta_0}$通过一个分束器,该分束器的变换矩阵为
\begin{equation}
    S = \pmqty{ t & r \\ -r^* & t }.
\end{equation}
使用海森堡绘景,我们将分束器的入射态
\[
    \ket{\Psi} = \ee^{\beta_0 b_2^\dagger - \beta_0^* b_2} f(b_1^\dagger) \ket{0}
\] 
用时间演化之后的$a$算符重写,得到
\begin{equation}
    \begin{aligned}
        \ket{\Psi} &= \\
        &\eqqcolon \ee^{\beta_0 t a_2^\dagger - \beta_0^* a_2} \ee^{\beta a_1^\dagger - \beta^* a_1} f(b_1^\dagger) \ket{0},
    \end{aligned}
\end{equation}
这里我们定义
\begin{equation}
    \beta = - \beta_0 r.
\end{equation}
如果分束器和$\beta_0$被适当选择,使得
\begin{equation}
    \abs*{\beta} \gg 1, \quad t \ll 1,
\end{equation}
就有
\begin{equation}
    \ket{\Psi} \approx \ee^{\beta a_1^\dagger - \beta^* a_1} f( a_1^\dagger) \ket{0} = D(\beta) f()
\end{equation}

\begin{figure}
    

\tikzset{every picture/.style={line width=0.75pt}} %set default line width to 0.75pt        

\begin{tikzpicture}[x=0.75pt,y=0.75pt,yscale=-1,xscale=1]
%uncomment if require: \path (0,300); %set diagram left start at 0, and has height of 300

%Shape: Square [id:dp48465100337411604] 
\draw   (286,119) -- (260,119) -- (260,145) -- (286,145) -- cycle ;
%Straight Lines [id:da009730264904362462] 
\draw    (286,145) -- (260,119) ;

%Straight Lines [id:da841426330546053] 
\draw    (179,132) -- (273,132) ;
\draw [shift={(226,132)}, rotate = 180] [fill={rgb, 255:red, 0; green, 0; blue, 0 }  ][line width=0.08]  [draw opacity=0] (12,-3) -- (0,0) -- (12,3) -- cycle    ;
%Straight Lines [id:da3980567150516523] 
\draw    (273,46) -- (273,132) ;
\draw [shift={(273,89)}, rotate = 270] [fill={rgb, 255:red, 0; green, 0; blue, 0 }  ][line width=0.08]  [draw opacity=0] (12,-3) -- (0,0) -- (12,3) -- cycle    ;
%Straight Lines [id:da7111057972380808] 
\draw    (273,132) -- (273,218) ;
\draw [shift={(273,175)}, rotate = 270] [fill={rgb, 255:red, 0; green, 0; blue, 0 }  ][line width=0.08]  [draw opacity=0] (12,-3) -- (0,0) -- (12,3) -- cycle    ;
%Straight Lines [id:da12521956138792478] 
\draw    (273,132) -- (392.71,132.95) ;
\draw [shift={(332.85,132.48)}, rotate = 180.46] [fill={rgb, 255:red, 0; green, 0; blue, 0 }  ][line width=0.08]  [draw opacity=0] (12,-3) -- (0,0) -- (12,3) -- cycle    ;
%Shape: Chord [id:dp14603723174788152] 
\draw   (393.11,117.08) .. controls (402.31,117.1) and (409.82,123.7) .. (409.98,132) .. controls (410.14,140.42) and (402.68,147.4) .. (393.3,147.58) -- cycle ;
%Straight Lines [id:da09107409379680376] 
\draw    (273,218) -- (392.71,218.95) ;
\draw [shift={(332.85,218.48)}, rotate = 180.46] [fill={rgb, 255:red, 0; green, 0; blue, 0 }  ][line width=0.08]  [draw opacity=0] (12,-3) -- (0,0) -- (12,3) -- cycle    ;
%Shape: Chord [id:dp158377588748835] 
\draw   (393.11,203.08) .. controls (402.31,203.1) and (409.82,209.7) .. (409.98,218) .. controls (410.14,226.42) and (402.68,233.4) .. (393.3,233.58) -- cycle ;
%Straight Lines [id:da39545292367915774] 
\draw    (409.71,132.95) -- (449.71,132.95) ;
%Straight Lines [id:da29050927155087436] 
\draw    (409.71,218.95) -- (449.71,218.95) ;
%Straight Lines [id:da4259702929343543] 
\draw    (449.71,132.95) -- (449.68,162.95) ;
%Straight Lines [id:da376644991684453] 
\draw    (449.68,191.95) -- (449.71,218.95) ;
%Shape: Circle [id:dp47795417904392656] 
\draw   (435.18,177.45) .. controls (435.18,169.45) and (441.67,162.95) .. (449.68,162.95) .. controls (457.69,162.95) and (464.18,169.45) .. (464.18,177.45) .. controls (464.18,185.46) and (457.69,191.95) .. (449.68,191.95) .. controls (441.67,191.95) and (435.18,185.46) .. (435.18,177.45) -- cycle ;
%Straight Lines [id:da9614966812921062] 
\draw    (464.18,177.45) -- (494.71,177.45) ;
%Shape: Rectangle [id:dp9974767092600316] 
\draw   (80,113) -- (179,113) -- (179,153) -- (80,153) -- cycle ;

% Text Node
\draw (279,32.4) node [anchor=north west][inner sep=0.75pt]    {$\mathrm{e}^{\alpha_{2} a^\dagger_2 - \alpha^*_2 a_2}$};
% Text Node
\draw (324.5,103.4) node [anchor=north west][inner sep=0.75pt]    {$b_{1}^{\dagger }$};
% Text Node
\draw (324.5,188.4) node [anchor=north west][inner sep=0.75pt]    {$b_{2}^{\dagger }$};
% Text Node
\draw (449.68,177.45) node    {$-$};
% Text Node
\draw (388,92.4) node [anchor=north west][inner sep=0.75pt]    {$\expval{b_1^\dagger b_1}$};
% Text Node
\draw (389,235.4) node [anchor=north west][inner sep=0.75pt]    {$\expval{b_2^\dagger b_2}$};
% Text Node
\draw (89,113) node [anchor=north west][inner sep=0.75pt]   [align=left] {Source to be \\evaluated};
% Text Node
\draw (219,102.4) node [anchor=north west][inner sep=0.75pt]    {$a_{1}^{\dagger }$};


\end{tikzpicture}

    \caption{相位测量装置2}
\end{figure}

例如,如果入射态是一个相干态$\ket{\alpha}$,计算得到
\begin{equation}
    \bar{n}_{12} = \abs*{\alpha \alpha_0} \cos \Delta \varphi.
\end{equation}

我们现在分析误差、噪声等引入的不确定度。

\begin{equation}
    \var{\Delta\varphi} = \frac{\var{n_{12}}}{\abs{\pdv{n_{12}}{\Delta\varphi}}} = \frac{\abs*{\alpha}}{\abs*{\alpha} \abs*{\alpha_0} \sin \Delta \varphi} = \frac{1}{\abs*{\alpha_0} \sin \Delta \Delta \varphi},
\end{equation}
因此我们有
\begin{equation}
    \var{\Delta \varphi} \geq \frac{1}{\abs*{\alpha_0}} = \frac{1}{\sqrt{\bar{n}}}.
\end{equation}
这个结论称为\concept{标准量子极限}。

Heterodyne

\section{量子层析}

从不同角度测量一个Wigner函数,然后用这些信息“粘”出完整的Winger函数。
因此,只要我们能够确定某个光源总是产生一样的量子光学态,那么我们是可以通过多次测定来(差一个相位地)确定它产生地量子光学态。
这个过程和CT很类似,从而称为\concept{量子层析}。

\part{原子物理}

\chapter{类氢原子}

\section{库伦势场}

本节讨论库伦势场中的电子运动情况。首先我们表明,原子的的确确是某个带正电荷而非常重的核约束了一些电子而得到的体系。
通过阴极射线实验可以证实原子中确确实实有电子,但是正电荷的分布是不清楚的。
采用半经典模型,设一个$\alpha$粒子与原子发生散射,则势能为
\begin{equation}
    V(r) = \frac{1}{4\pi \epsilon_0} \frac{Z_1 Z_2 e^2}{r}, 
\end{equation}
散射角为
\[
    \theta = \pi - 2 \int_{r_\text{min}}^\infty \frac{b \dd{r}}{r^2 \sqrt{1 - V(r)/E - b^2/r^2}},  
\]
而散射截面为
\[
    \dd{\sigma} = \frac{b(\theta)}{\sin \theta} \abs{\dv{b}{\theta}} \dd{\Omega},
\]
最终可以计算出
\begin{equation}
    \dd{\sigma} = \left(\frac{a}{4}\right)^2 \sin^{-4}\frac{\theta}{2} \dd{\Omega}, \quad a = \frac{Z_1 Z_2 e^2}{4\pi \epsilon_0 E}.    
\end{equation}
如果我们使用非常薄的金属箔作为靶标,并假定不同原子核分布非常稀疏(由于原子核非常小,这是正确的),从而不同层的原子核都是错开的,没有互相遮挡,设金属箔数密度为$n$,厚度为$t$,则被散射到立体角$\dd{\Omega}$中的$\alpha$粒子数满足
\[
    \frac{\dd{N}}{N} = \frac{n A t \dd{\sigma}}{A} = nt \dd{\sigma},
\]
于是
\begin{equation}
    \frac{\dd{N}}{N \dd{\Omega}} = n t \left(\frac{a}{4}\right)^2 \sin^{-4}\frac{\theta}{2}.
\end{equation}
此即\concept{卢瑟福散射公式}。卢瑟福散射公式成立的条件包括:
\begin{enumerate}
    \item 非相对论近似,因为使用了牛顿力学的动能公式;
    \item 大角度散射,也就是说瞄准距离$b$比较小,因为只有这样入射的$\alpha$粒子才能够充分接近原子核,从而可以像我们做的那样,忽略外层电子的屏蔽效应;
    \item $r_\text{min}$要大于原子核半径,从而不会发生核反应,且可以将原子核看成点电荷;
    \item 在满足第一个和第三个条件的前提下,入射$\alpha$粒子动能尽可能大,从而外层电子的屏蔽作用可以忽略。
\end{enumerate}
这个公式和实验结果一致,说明原子结构中确实有原子核。
再往下,经典理论就会造成著名的疑难,就是既然电子绕着原子核运动,那么必然会发出辐射而损失能量。
因此我们接下来使用量子力学来分析原子内部电子的运动情况。

库伦势场中的单个电子的哈密顿量为
\begin{equation}
    {H} \psi = \frac{{p}^2}{2m} \psi - \frac{Z}{4\pi \epsilon_0} \frac{e^2}{\abs{\vb*{r}}} \psi.
    \label{eq:columb-electron-hamiltonian}
\end{equation}
我们常常将这样的体系称为类氢原子,因为它和氢原子的结构除了$Z$可能不一样以外完全一致。
此时\eqref{eq:r-equation}为
\[
    \left( - \frac{\hbar^2}{2m r^2} \dv{r} r^2 \dv{r} + \frac{\hbar^2}{2m r^2} l(l+1) - \frac{1}{4\pi \epsilon_0} \frac{e^2}{r} \right) R = E R,
\]
显然这是一个束缚在势阱中的电子的方程,它必定有束缚态解,从而可以提供我们需要的主量子数。

为了获取一些灵感,首先考虑$r\to \infty$的极限,得到渐进解
\[
    R = \exp(- \sqrt{- \frac{2 m E}{\hbar^2}} r).
\]
设
\[
    k^2 = - \frac{2 m E}{\hbar^2},
\]
并令
\[
    \rho = 2 k r, \quad \gamma = \frac{m Z e^2}{4\pi \epsilon_0 k \hbar^2}, 
\]
取试探解
\[
    R(\rho) = \ee^{- \rho / 2} F(\rho),
\]
得到
\[
    \dv[2]{F}{\rho} + \left( \frac{2}{\rho} - 1 \right) \dv{F}{\rho} + \left( \frac{\gamma - 1}{\rho} - \frac{l(l+1)}{\rho^2} \right) F = 0.
\]
这仍然是一个本征值问题,本征值由$\gamma$标记。通过广义幂级数展开可以知道$\gamma$应当为整数,这样就得到了本征值
\begin{equation}
    E_n = - \frac{1}{2} m Z^2 \left( \frac{e^2}{4\pi \epsilon_0 \hbar} \right)^2 \frac{1}{n^2} = - \frac{1}{2} m Z^2 (\alpha c)^2 \frac{1}{n^2}.
\end{equation}
通过广义幂级数展开还可以发现广义幂级数形如
\[
    F(\rho) = \rho^l \sum_j a_j \rho^j.
\]
我们会发现除了$\rho^l$以外的$F(\rho)$的因子实际上服从合流拉盖尔方程,于是得到
\[
    R(\rho) = \ee^{-\rho/2} \rho^l \laguerre_{n+l}^{2l+1}(\rho).
\]
定义\concept{第一波尔半径}
\begin{equation}
    a_1 = \frac{4\pi \epsilon_0 \hbar^2}{m e^2},
\end{equation}
我们下面会看到它是经典原子模型(电子绕着正电荷匀速圆周运动)给出的轨道半径,则能量可以写成
\begin{equation}
    E_n = - \frac{Z^2 e^2}{4 \pi \epsilon_0 } \frac{1}{2 a_1} \frac{1}{n^2} = \frac{E_1}{n^2}, \quad E_1 = - \frac{1}{4\pi \epsilon_0} \frac{e^2}{2 a_1}.
    \label{eq:hydrogen-energy}
\end{equation}
将各个常数放回上式并归一化就得到
\begin{equation}
    R_{n, l} = - \sqrt{\left( \frac{2 Z}{n a_1} \right)^3 \frac{(n-(l+1))!}{2n ((n + l)!)^3}} \exp(- \frac{Z r}{n a_1}) \left( \frac{2 Z r}{n a_1} \right)^l \laguerre_{n+l}^{2l+1}\left(\frac{2 Z r}{n a_1}\right),
\end{equation}
其中$\laguerre_{n+l}^{2l+1}$是合流拉盖尔多项式。
$n-l$决定了径向峰值的数目。为了让合流拉盖尔多项式有良定义,我们有
\begin{equation}
    l = 0, 1, 2, \ldots, n-1.
\end{equation}
至此,库伦中心势下的电子运动情况完全确定。

可以依稀从量子力学中的氢原子看出一些经典的图像。在经典的原子模型中,粒子可以做椭圆运动,但是运动的能量仅仅关乎一个参数即椭圆的半长轴$a$,而和半短轴$b$无关;半短轴$b$则决定角动量等。
因此能量和角动量是分开的。角动量可以有不同的指向,因此角动量长度和它在$z$轴上的投影也没有必然的关系(当然,角动量在$z$轴上的投影不可能超过总的角动量长度)
实际上,如果我们假定:
\begin{enumerate}
    \item 核外电子绕着原子核运动遵循牛顿定律;
    \item 在稳定的轨道上运动的电子不会发射或者吸收电磁波;
    \item 角动量量子化,即
    \begin{equation}
        L = m v r = n \hbar,
        \label{eq:quantum-angular-momentum}
    \end{equation}
\end{enumerate}
我们也可以得到类氢原子能级(这称为\concept{波尔模型})。
做受力分析
\[
    m \frac{v^2}{r} = \frac{1}{4\pi \epsilon_0} \frac{Z e^2}{r^2},
\]
并结合\eqref{eq:quantum-angular-momentum},可以得到
\begin{equation}
    r_n = \frac{4\pi \epsilon_0 \hbar^2}{Zme^2} n^2.
\end{equation}

求解出氢原子的薛定谔方程的完整解之后,可以发现主量子数为$n$的能级上有$n$个可能的角量子数,每个角量子数又允许$2l+1$个磁量子数,而最后还有两个自旋量子数,因此主量子数为$n$的能级上有
\[
    \frac{1}{2} (1 + (2(n-1)+1)) n = n^2
\]
个轨道,有$2n^2$个电子。

我们注意到,一般来说,能量和$n, l$都有关系(和$m$确定没有关系,因为自旋旋转不变性),但在库伦势场中能量和$l$实际上并没有关系。
这意味着库伦势场中其实还有一个隐藏的对称性。从半经典模型的考虑,角动量由半短轴决定,但是能量只和半长轴有关,因此这并不出乎意料。
当然,电子的轨道可以不垂直于我们选取的$xOy$平面,从而角动量的$z$轴投影可能变动,但能量和坐标轴选取无关。

\section{跃迁和偶极辐射}\label{sec:electro-dipole}

电子和电磁场耦合,因此可以在不同能级之间跃迁而发射或吸收光子。
跃迁包括受激跃迁(电子首先吸收光子,然后发生跃迁)以及自发跃迁(电子直接发生跃迁)。
对这一过程的完整计算涉及量子电动力学的束缚态,但通常对能标不是非常高的过程,使用量子化的原子和经典电动力学就足够计算一些问题。
受激跃迁只需要量子化原子加上经典电动力学即可完全解释,而自发跃迁不能使用经典电动力学解释而必须将光场量子化,因为“自发”意味着光场的真空涨落,这要使用量子理论处理。

\subsection{光子}

我们首先从经典电动力学中的平面波来探讨问题,这是可以的,因为实际上光子的产生湮灭算符对应着量子化的电磁场的傅里叶分量的振幅,于是单个光子的经典对应就是一个平面波。
考虑以下单色平面波
\begin{equation}
    \vb*{E} = \vb*{E}_0 \ee^{\ii(\vb*{k} \cdot \vb*{r} - \omega t)}, \quad \vb*{E}_0 = \sum_{m_s} E_{m_s} \vb*{e}_{m_s}, \quad m_s = -1, 0, 1,
\end{equation}
这里的$m_s$指的就是自旋。光场是矢量场,因此自旋为1,这样其内禀旋转自由度有三个方向,正好和矢量有$x, y, z$三个方向对应。
定义
\begin{equation}
    \vb*{e}_{\pm 1} = \mp \frac{1}{\sqrt{2}} (\vb*{e}_x \pm \ii \vb*{e}_y), \quad \vb*{e}_0 = \vb*{e}_z,
\end{equation}
容易看出它们正交,且$\vb*{e}_1$对应着电场垂直于$z$轴,且绕$z$轴逆时针旋转,$\vb*{e}_{-1}$对应着电场垂直于$z$轴,且绕$z$轴顺时针旋转,因此$\vb*{e}_{\pm 1}$是垂直于$z$轴的圆偏振基,称它们为$\sigma^\pm$基,而$\vb*{e}_0$是平行于$z$轴的线偏振基,称为$\pi$基。

照惯例,取$\vb*{k}$的方向为$z$轴,电磁波只有横波模式没有纵波模式(这是$U(1)$规范场的性质:总是可以选取一个规范让纵波消失),因此$\pi$基没有复振幅,光从来不在$\pi$基上有偏振,或者等价地说,光子在其传播方向上的自旋只有$\pm 1$,没有$0$。纵光子是观测不到的。

原子发射光子时,由对称性分析,如果能够良定义一个光子位置$\vb*{r}$,且将位矢零点放在原子中心,那么有
\[
    \vb*{r} \times \vb*{k} = 0,
\]
因此将位矢放在原子中心时由原子发射的光子没有轨道角动量。于是光子的角动量仅含有自旋角动量。

\subsection{爱因斯坦的唯象理论}\label{sec:einstein-phonomenon}

考虑温度为$T$的空腔中有大量相同的原子,显然处于定态$i$和$j$的原子需要满足玻尔兹曼分布率
\[
    N_i \propto \ee^{-\frac{E_i}{k_\text{B} T}},
\]
或者写成
\[
    \frac{N_j}{N_i} = \ee^{-\hbar \omega_{ji} / kT}.
\]
能够达到热力学平衡意味着电子需要在不同能级之间跃迁。
电子和电磁场有耦合,因此电子在不同能级上跃迁确实是可以的。跃迁发生的机制可能有这么几种:
\begin{enumerate}
    \item 自发发射,也就是电子放出一个光子,跃迁到较低的能级;
    \item 受激发射,即电子先吸收一个光子再放出一个光子,然后发生跃迁;
    \item 吸收,即电子吸收一个光子然后跃迁到较高的轨道上。
\end{enumerate}

当然,在经典电动力学的框架下,只应该出现后两种情况。但是这样一来系统实际上不能够达到玻尔兹曼分布。

受激发射和吸收可以使用量子化的原子和一个经典电磁场耦合来计算,但是自发发射在这个框架下是很难解释的,因为一个激发态的原子放在完全没有电磁场的空间内照样会有自发发射。
对这一现象的完整解释显然涉及真空涨落,因此需要量子电动力学。

设温度为$T$的光场中频率为$\omega$附近的能量密度为$u(\omega, T)$。设有两个能级$i$和$j$,且$E_j > E_i$。
我们假定(之后会通过量子力学严格证明)自发发射的跃迁率和$u$无关,而受激发射和吸收的跃迁率正比于$u(\omega_{ji}, T)$,其中
\begin{equation}
    \hbar \omega_{ji} = E_j - E_i,
    \label{eq:photon-energy}
\end{equation}
这样在这两个能级之间的自发发射、受激发射、吸收的跃迁率分别是
\[
    A_{ji} N_j, \quad B_{ji} N_j u(\omega_{ji}, T), \quad C_{ij} N_i u(\omega_{ji}, T).
\]
能级$i$向上跃迁到$j$的跃迁率为
\[
    \lambda_{ij} = C_{ij} u(\omega_{ji}, T),
\]
能级$j$向下跃迁到$i$的跃迁率为
\[
    \lambda_{ji} = B_{ji} u(\omega_{ji}, T) + A_{ji}.
\]
平衡时两者相等,即有
\[
    N_i C_{ij} u(\omega_{ji}, T) = N_j (B_{ji} u(\omega_{ji}, T) + A_{ji})
\]
$T \to \infty$,不同能级上原子分布的个数差别变得很小,$u \to \infty$,而上式仍然成立,因此$C_{ij} = B_{ji}$%
\footnote{请注意对温度的依赖被完全归入$u(\omega, T)$中,系数$C$和$B$由电子和光场耦合的方式决定,因此不依赖温度。}%
,这样就有
\[
    u(\omega_{ji}) = \frac{A_{ji} / B_{ji}}{\ee^{\omega_{ji} \hbar / k T} - 1}.
\]
由于原子能级可以随意调整,我们有
\[
    u(\omega) = \frac{A_{ji} / B_{ji}}{\ee^{\omega \hbar / k T} - 1}.
\]
而由于空腔内的辐射能量密度为
\[
    u = \frac{\hbar \omega^3}{\pi^2 c^3} \frac{1}{\ee^{\hbar \omega / kT} - 1}
\]
自发发射跃迁率为
\[
    A_{ji} = \frac{\omega_{ji}^3}{3\pi \epsilon_0 \hbar c^3}
\]

\subsection{跃迁系数的推导}\label{sec:electro-dipole-hopping}

现在尝试从头计算跃迁率。如前所述,热平衡(实际上不仅仅是热平衡)时受激发射和吸收的跃迁率相等,这是由细致平衡条件以及高温下辐射密度趋于无限大这两个事实保证的,没有用到任何关于辐射机制的细节。
这样就只需要计算自发发射和受激发射的跃迁率。
在上一节中我们用到了空腔内的辐射能量密度,通过经典电动力学推导出的辐射能量密度公式是错误的(红外灾难和紫外灾难),这意味着完整地讨论跃迁需要量子电动力学。
不过,受激发射还是可以使用经典电动力学得到。

仅考虑电偶极辐射,则哈密顿量中需要加入这样一项(这件事的严格推导见\opticsdoc中的第\ref{optics-sec:dipole-radiation}节):
\begin{equation}
    {H}_\text{DE} = - {\vb*{d}} \cdot \vb*{E}, \quad {\vb*{d}} = -e {\vb*{r}}.
\end{equation}
由于是偶极辐射,电场近似认为不存在空间变动,即可以展开成以下仅显含时间的傅里叶分量:
\[
    \vb*{E} = \int \dd{\omega} \vb*{E}(\omega) \ee^{- \ii \omega t}.
\]
实际上,由于电子的运动会释放光子,终究不能够将电场$\vb*{E}$看成一个外部给定的场,而必须把它的值看成是系统状态的一部分,电场的状态和电子的状态直积得到系统状态。
我们接下来将只考虑电子的状态,换而言之,将电场的状态迹掉了,因此电子和不同频率的电场的相互作用得到的概率振幅应当被非相干叠加,即使用经典概率的方法叠加。%
\footnote{实际上,如果完整地使用cQED做计算,把光子和原子全部计入态矢量中,那么的确在计算过程中不会出现混合态,但是当我们开始略去光子的状态而只考虑原子的跃迁时,就已经隐含地迹掉了光子,从而原子的状态就成为混合态了。
在推导原子的跃迁时必然要在某个阶段引入混合态,从而概率振幅非相干叠加,因为\autoref{sec:einstein-phonomenon}中的能量密度是光子的热系综的能量密度,因此必须能够保证我们的推导对热系综也适用。}%
在经典极限下平面波的偏振方向可以取任何方向,因为光子一方面单个能量很弱,另一方面总数又很大,的确可以让电场振动方向指向任意方向。
总之,相互作用哈密顿量为
\[
    {H}_\text{DE} = \int \dd{\omega} e {\vb*{r}} \cdot \vb*{E}(\omega),
\]
且每个$\vb*{E}(\omega)$又允许任意的偏振方向取向。换而言之电子能够和$\omega$、偏振方向随意取的平面波电场模式(实际上是光子模式的经典极限)发生相互作用。

对每个电场模式,使用一阶微扰计算跃迁振幅。一阶含时微扰论的概率振幅为
\[
    \braket{n}{\psi(t)} = - \frac{\ii}{\hbar} \ee^{- \ii E_n t / \hbar} \sum_m \int_0^t \dd{t'} \mel{n}{{H}_\text{DE}}{m} \ee^{\ii \omega_{nm} t} \braket{m}{\psi(0)},
\]
其中
\[
    \hbar \omega_{nm} = E_n - E_m,
\]
$m, n$等表示一组正交基——在这里就是使用$(n, l, m_l, m_s)$表示出的原子态。
这一组原子态是一组偏好基,它们之间的跃迁概率就是以上概率振幅的模长平方。
考虑这样一个过程:一开始原子处于态$\ket{i}$上,然后和频率为$\omega$的电场发生相互作用,跃迁到态$\ket{j}$上。
这个过程的振幅为
\[
    \begin{aligned}
        \braket{j}{\psi(t)} &= - \frac{\ii}{\hbar} \ee^{- \ii E_n t / \hbar} \int_0^t \dd{t'} \mel{j}{e {\vb*{r}} \cdot \vb*{E}(\omega) \ee^{- \ii \omega t'}}{i} \ee^{\ii \omega_{ji} t'} \\
        &= - \frac{\ii}{\hbar} \ee^{- \ii E_n t / \hbar} e \vb*{E}(\omega) \cdot \mel{j}{{\vb*{r}}}{i} \frac{\ee^{\ii (\omega_{ji} - \omega) t} - 1}{\ii (\omega_{ji} - \omega)} \\
        &= - \frac{\ii}{\hbar} \ee^{- \ii E_n t / \hbar} e E(\omega) \mel{j}{{\vb*{r}}}{i} \cos \theta \frac{\ee^{\ii (\omega_{ji} - \omega) t} - 1}{\ii (\omega_{ji} - \omega)},
    \end{aligned}
\]
其中$\theta$是电场偏振方向和$\mel{j}{{\vb*{r}}}{i}$的夹角。

电子从$\ket{i}$到$\ket{j}$的过程可以通过和不同波长、不同偏振方向的电磁波相互作用而发生。由对称性分析,不同偏振方向的电磁波出现的可能性是一样的,也就是说偏振方向在立体角$\dd{\Omega}$中,频率出现在$\omega$到$\omega+\dd{\omega}$的电磁波出现的概率为
\[
    P \dd{\Omega} = p(\omega) \frac{\dd{\Omega}}{4\pi} \dd{\omega},
\]
于是
\[
    \begin{aligned}
        P_{i \to j} &= \int p(\omega) \dd{\omega} \int \frac{\dd{\Omega}}{4\pi} \abs*{\braket{j}{\psi(t)}}^2 \\
        &= \frac{e^2}{\hbar^2} \int \frac{\dd{\Omega}}{4\pi} \cos^2 \theta \int \dd{\omega} p(\omega) E(\omega)^2 \abs*{\mel{j}{{\vb*{r}}}{i}}^2 \frac{\sin^2((\omega_{ji} - \omega) t / 2)}{(\omega_{ji}-\omega)^2} \\
        &= \frac{e^2}{3 \hbar^2} \abs*{\mel{j}{{\vb*{r}}}{i}}^2 \int \dd{\omega} p(\omega) E(\omega)^2 \frac{\sin^2((\omega_{ji} - \omega) t / 2)}{(\omega_{ji}-\omega)^2}.
    \end{aligned}
\]
我们关心的时间尺度通常比较大,而随着$t$增大,$\frac{\sin^2((\omega_{ji} - \omega) t / 2)}{(\omega_{ji}-\omega)^2}$会变得越来越尖锐,只在$\omega = \omega_{ji}$附近有比较明显的非零值,于是近似有(这个近似表明跃迁几乎总是发出频率就是$\omega_{ji}$的电磁波,这当然是正确的)
\[
    \begin{aligned}
        P_{i \to j} &= \frac{e^2}{3 \hbar^2} \abs*{\mel{j}{{\vb*{r}}}{i}}^2 p(\omega_{ji}) E(\omega_{ji})^2 \int_{-\infty}^\infty \dd{\omega} \frac{\sin^2((\omega_{ji} - \omega) t / 2)}{(\omega_{ji}-\omega)^2} \\
        &= \frac{e^2}{3 \hbar^2} \abs*{\mel{j}{{\vb*{r}}}{i}}^2 p(\omega_{ji}) E(\omega_{ji})^2 \pi t.
    \end{aligned}
\]
可见跃迁几率随着时间增大而线性增大。单位时间的跃迁几率为
\[
    \Gamma_{i \to j} = \dv{P_{i \to j}}{t} = \frac{\pi e^2}{3 \hbar^2} \abs*{\mel{j}{{\vb*{r}}}{i}}^2 p(\omega_{ji}) E(\omega_{ji})^2.
\]
考虑到频段$\omega$到$\omega+\dd{\omega}$上的电磁场能量(它是电场能量的两倍)为
\begin{equation}
    u(\omega) = \epsilon_0 p(\omega) E(\omega)^2,
\end{equation}
最后得到
\begin{equation}
    \Gamma_{i \to j} = \frac{\pi e^2 \expval*{\vb*{r}_{ji}}^2}{3 \epsilon_0 \hbar^2} u(\omega_{ji}), \quad B_{ji} = \frac{\pi e^2}{3 \epsilon_0 \hbar^2},
\end{equation}
其中
\begin{equation}
    e \expval*{\vb*{r}_{ji}} = e \int \dd[3]{\vb*{r}} \psi_i^*(\vb*{r}) \vb*{r} \psi_j(\vb*{r})
    \label{eq:electro-dipole}
\end{equation}
为电偶极矩的期望值。

以上是国际单位制的推导,如果改用高斯单位制,由于电磁能量的形式会发生变化,将得到
\begin{equation}
    B_{ji} = \frac{4\pi^2 e^2}{3 \hbar^2} \abs{\expval*{\vb*{r}_{ji}}}^2.
\end{equation}

\subsubsection{选择定则}

\eqref{eq:electro-dipole}中的波函数对$(r, \theta, \varphi)$是分离变量的,而
\[
    \begin{aligned}
        r_x &= r \sin \theta \cos \varphi, \\
        r_y &= r \sin \theta \sin \varphi, \\
        r_z &= r \cos \theta 
    \end{aligned}
\]
也是分离变量的。
记$\psi_1$和$\psi_2$的量子数分别是$n_1, l_1, m_1$和$n_2, l_2, m_2$。
\eqref{eq:electro-dipole}给出非零结果的必要条件是其角部分均不为零。
在$\varphi$方向上,积分是
\[
    \int_0^{2\pi} \dd{\varphi} \ee^{ - \ii m_1 \varphi} \cos \varphi \ee^{\ii m_2 \varphi} \vb*{e}_x + \int_0^{2\pi} \dd{\varphi} \ee^{ - \ii m_1 \varphi} \sin \varphi \ee^{\ii m_2 \varphi} \vb*{e}_y + \int_0^{2\pi} \dd{\varphi} \ee^{ - \ii m_1 \varphi} \ee^{\ii m_2 \varphi} \vb*{e}_z,
\]
让三个分量不全为零的可能取值是:
\[
    m_2 - m_1 = \pm 1, 0.
\]
在$\theta$方向上,积分是
\[
    \int_0^\pi \dd{\theta} \sin \theta \legpoly_{l_1}^{m_1} (\cos \theta) \legpoly_{l_2}^{m_2} (\cos \theta) (\vb*{e}_x + \vb*{e}_y) + \int_0^\pi \dd{\theta} \cos \theta \legpoly_{l_1}^{m_1} (\cos \theta) \legpoly_{l_2}^{m_2} (\cos \theta) \vb*{e}_z,
\]
使用勒让德多项式的性质,可以证明让三个分量不全为零的可能取值为
\[
    l_2 - l_1 = \pm 1.
\]
总之,要让受激发射系数不为零,需要
\begin{equation}
    \Delta m = 0, \pm 1, \quad \Delta l = \pm 1.
\end{equation}
这就是\concept{单电子原子跃迁的选择定则}。
实际上,也可以通过守恒量分析得到这个结论。
由于光子为自旋1的粒子,%
\footnote{这里可能会遇到一个疑难:光子的自旋角动量只在其前进方向上有投影,且只有$\pm 1$两种取值,那么似乎$z$方向角动量守恒意味着只能有$\Delta m = \pm 1$。
然而,光子的前进方向和我们选取的电子$z$方向未必相同,因此光子的自旋角动量投影在$z$方向上还是会有$0, \pm 1$三种取值。
}%

不满足选择定则的跃迁称为\concept{禁戒跃迁}。通过磁偶极跃迁、电四极子跃迁、双光子跃迁甚至原子和原子之间的碰撞等方法,禁戒跃迁也是可以发生的,但是相对来说发生概率不大,从而对应的能级为亚稳态——在电偶极跃迁比较频繁的时间尺度上它不会发生,但是在更长的时间尺度上它的确会发生。

我们通常会讨论的和原子相互作用的电磁场都是比较弱的,因此原子对光子的吸收和发射和其它过程——如原子在外加电场、磁场下的变化——都没有耦合。
这意味着原子光谱提供了一种非常好的、不受其它实验手段影响的检查原子内部能级发生了什么变动的方式。
将原子置于外场中而产生的各种效应基本上都可以通过光谱体现出来。

\section{磁矩和磁场作用}

到目前为止的讨论,磁量子数$m_l$都是能量简并的,而如果加入一个磁场,那么就会有一个特定的空间方向,从而破缺$m_l$简并,导致能级进一步分裂。

\subsection{磁矩}

一些系统在外加静磁场时能量会增加一项
\[
    E_\text{M} = - \vb*{\mu} \cdot \vb*{B},
\]
其中的矢量$\vb*{\mu}$就称为磁矩。
磁矩和电荷的周期性运动具有非常密切的关系。一个没有内部结构的电荷做周期性运动相当于产生了一个环状电流,因此会产生一个磁矩,称为\concept{轨道磁矩}。
首先采用经典理论分析轨道磁矩。电子轨道角动量的公式为
\[
    \vb*{L} = m_\text{e} \vb*{r} \times \dv{\vb*{r}}{t} = 2 m_\text{e} \dv{\vb*{S}}{t},
\]
电子被束缚在原子核周围时做平面周期性运动,这样它就产生了一个大小为
\[
    I = - \frac{e}{\tau}
\]
的电流,其中$\tau$是运动周期。
在电磁学中,一个电流为$I$,围绕的面积为$\vb*{S}$的平面线圈的磁矩为
\[
    \vb*{\mu} = I \vb*{S} = - \frac{e \vb*{S}}{\tau},
\]
而由于电子做周期性运动,由角动量守恒我们有
\[
    \dv{\vb*{S}}{t} = \frac{\vb*{S}}{\tau},
\]
这样轨道磁矩就是
\[
    \vb*{\mu} = - \frac{e}{2m_\text{e}} \vb*{L}.
\]
负号的出现是因为电子携带负电荷,后面推导核子的磁矩时就没有这个负号。
为了与原子物理的背景保持一致,引入下标$l$表示这是来自轨道角动量的磁矩,并且设
\begin{equation}
    \mu_\text{B} = \frac{e\hbar}{2m_\text{e}}
\end{equation}
称为\concept{玻尔磁子},于是
\begin{equation}
    \vb*{\mu}_l = - \frac{\mu_\text{B}}{\hbar} \vb*{L}.
    \label{eq:orbit-magnetic-moment}
\end{equation}
虽然\eqref{eq:orbit-magnetic-moment}是在经典力学中导出的,但它也适用于量子理论。

量子理论中还有自旋角动量,这是不是会引入自旋磁矩?确实会,不过自旋磁矩的值和把电子当成带电小球计算出来的值并不相同。实际上,自旋磁矩是
\begin{equation}
    \vb*{\mu}_s = - \frac{e}{m_\text{e}} \vb*{S}.
\end{equation}
当然这也不奇怪,因为自旋在粒子图像中并没有经典对应。实际上自旋磁矩的严格计算直接来自QED。

总之,原子的总磁矩为
\begin{equation}
    {\vb*{\mu}} = - \frac{\mu_\text{B}}{\hbar} (\vb*{L} + 2\vb*{S}),
\end{equation}
而对应的哈密顿量为
\begin{equation}
    {H}_\text{mag} = - {\vb*{\mu}} \cdot \vb*{B}.
    \label{eq:magnetic-hamiltonian}
\end{equation}

\subsection{半经典图像}

轨道角动量有明确的经典意义,可以使用半经典理论描述它。例如如果角动量的长度和$z$轴分量保持不变,那么就会发生\concept{拉莫尔进动},即角动量矢量在一个对称轴就是$z$轴的锥面上运动,长度保持不变。

磁矩对电子运动的影响无非是让电子受力(即破缺平移不变性)或是受力矩(即破缺旋转不变性)。
如果磁场是均匀的,那么电子肯定不会受力,但会受到一个力矩,因为磁矩的方向的变动会让$\vb*{\mu} \cdot \vb*{B}$发生变化,即
\[
    \pdv{(\vb*{\mu} \cdot \vb*{B})}{\vb*{\varphi}} \neq 0,
\]
但是电子位置的变动当然不会让$\vb*{\mu}$发生任何变化。
如果磁场是不均匀的,那么磁矩不仅受到力矩还受力,因为空间平移不变性被破缺了,或者说
\[
    \pdv{(\vb*{\mu} \cdot \vb*{B})}{\vb*{r}} \neq 0.
\]

\section{相对论修正}

本节讨论相对论修正导致的能级的小幅变化。这会导致环绕原子核的库伦势场中的电子的能级发生小的分裂。

\subsection{精细结构}

本节讨论\concept{精细结构},即相对论修正中最大的两个项。它们分别来自相对论动能和自旋-轨道耦合。
分裂出的两个能级很难发生彼此之间的跃迁,因为两者的角量子数和磁量子数都一样,因此两者之间的跃迁违背选择定则。

首先,我们知道,相对论情况下动能为
\[
    E_\text{k} = \frac{m c^2}{\sqrt{1 - v^2 / c^2}},
\]
用动量表示出来,展开到第二阶,得到
\begin{equation}
    E_\text{k} = \frac{p^2}{2m} - \frac{p^4}{8 m^3 c^2} + \cdots.
\end{equation}
第一项当然就是经典动能,第二项给出了一个微扰。注意到这个微扰仍然具有全部的旋转不变性,因此角动量平方和$z$方向角动量在它之下仍然守恒。
这样虽然类氢原子的波函数有简并,在$\psi_{nlm}$下仍然可以把它们当成非简并的。
一阶微扰为
\begin{equation}
    E^{\text{kin},(1)}_{n, l} = - \frac{(E^{(0)}_n)^2}{2 m c^2} \left( \frac{4 n}{l + 1/2} - 3 \right).
    \label{eq:kinetic-energy-relativity-correction}
\end{equation}
$l$的简并解除了——本该如此,实际上$l$会有简并单纯是库伦场的额外对称性的结果。这大约是$E_n^{(0)}$的$10^{-5}$量级。
$m$仍然有简并,从而对不同的$m$对应的波函数做线性组合,得到的仍然是能量本征态。这件事很重要,因为我们马上要引入一个不能保持$m$守恒的修正。

即使没有外加磁场,电子的自旋和轨道角动量仍然会出现小的耦合。这是相对论效应的结果。
直观地看,这是因为自旋自由度和磁场有耦合,而电子的轨道自由度的移动产生磁场(可以看成电子静止时的库伦电场做洛伦兹变换的结果)。
本节将给出对这一现象的一个半经典讨论。
为方便起见,以下称相对于原子实静止的参考系为$O$系,相对于某一时刻的电子静止的参考系(这仍然是一个惯性系,因为它并不是每一时刻都和电子保持静止)为$e$系。
设电子在$O$系中运行速度为$\vb*{v}$,$O$系中原子实施加给电子一个静电场$\vb*{E}_0$,则$e$系中$\vb*{E}_0$将变换成如下磁场:
\[
    \vb*{B}' = \frac{\vb*{E}_0 \times \vb*{v}}{\sqrt{c^2 - v^2}},
\]
由于$\vb*{v}$相对于光速很小,有
\begin{equation}
    \vb*{B}' = \frac{1}{c^2} \vb*{E}_0 \times \vb*{v}.
\end{equation}
这个磁场会被电子的自旋角动量感受到。
$e$系中任何一个角动量的进动都是% TODO:经典力学
\[
    \vb*{\omega}' = \frac{e}{m_\text{e}} \vb*{B}',
\]
而在$O$系中$e$系的坐标轴以
\[
    \vb*{\omega}_T = - \frac{e}{2 m_\text{e}} \vb*{B}'
\]
的角速度进动,因此最后$O$系中任何一个角动量都在以
\begin{equation}
    \vb*{\omega} = \frac{e}{2 m_\text{e}} \vb*{B}'
\end{equation}
的角速度进动。这等价于$O$系中多出来了一个磁场,% 这也太扯了。。我觉得比较好的推导是,证明$e$系中哈密顿量会多出来一项,然后这一项和$O$系是相同的,当然哈密顿量未必是洛伦兹标量,所以也很麻烦。。。
因此需要在哈密顿量中引入一项
\begin{equation}
    {H}_{LS} = - \vb*{\mu} \cdot \vb*{B}_\text{eff} = \frac{1}{2c^2} \frac{e}{m} \vb*{S} \cdot (\vb*{E}_0 \times \vb*{v}).
\end{equation}
如果是类氢原子,有
\begin{equation}
    {H}_{LS} = \frac{Ze^2}{8 \pi \epsilon_0 c^2 m^2} \frac{\vb*{S} \cdot \vb*{L}}{r^3}.
    \label{eq:spin-ortibal-coupling}
\end{equation}
因此轨道角动量和自旋角动量是有耦合的,即所谓\concept{自旋-轨道耦合},两者同向时能量较高,两者反向时能量较低。
这同样会带来一个能级分裂。由于
\[
    \vb*{S} \cdot \vb*{L} = \frac{1}{2} (J^2 - L^2 - S^2),
\]
应该使用做了L-S耦合(详情见多电子系统的一般讨论\autoref{sec:ls-coupling},这里就是将轨道角动量和自旋角动量做了一个复合)的波函数$\psi_{njl m_j}$,请注意这族波函数也是相对论性动能修正下的能量本征态。
计算得到
\begin{equation}
    E^{\text{LS}, (1)}_{njl} = \frac{(E^{(0)}_n)^2}{mc^2} \frac{2n (j(j+1) - l(l+1) - 3/4)}{l(l+1)(2l+1)}.
\end{equation}
于是最后能级修正为
\begin{equation}
    E^{(1)}_{nj} = \frac{(E^{(0)}_n)^2}{mc^2} \left( \frac{3}{2} - \frac{4 n}{2 j + 1} \right).
\end{equation}
引入无量纲的\concept{精细结构常数}
\begin{equation}
    \alpha = \frac{e^2}{4\pi \epsilon_0 \hbar c} \approx \frac{1}{137},
\end{equation}
做了一阶修正之后的能级为
\begin{equation}
    E_{n j} = \frac{E_1^{(0)}}{n^2} \left( 1 + \left( \frac{Z \alpha}{n} \right)^2 \left(  \frac{2 n}{2 j + 1} - \frac{3}{4} \right) \right).
\end{equation}

\subsection{更精细的物理}

% TODO

实际上,以上讨论还是不能够覆盖所有的物理现象。核磁矩的存在、电四极辐射以及其它一些机制会导致能级进一步分裂,产生\concept{超精细结构}。
这些效应都远远小于精细结构,如核磁矩相比电子磁矩是很小的,

使用狄拉克方程,因为兰姆位移。

\chapter{多电子原子}

多电子处于同一系统时会产生更多有趣的结果。由于电子是费米子,体系的波函数(同时包括轨道部分和自旋部分)一定是交换反对称的。
如果不考虑轨道-自旋耦合,为了保证反对称性,轨道部分对称则自旋部分反对称;轨道部分反对称则自旋部分对称。

\section{双电子原子}

\subsection{反对称化}

首先考虑一个双电子原子。如果两个电子的轨道运动相同,那么轨道部分的波函数一定是对称的(如果是反对称的就变成零了),那么自旋部分的波函数一定是反对称的,并且两个电子的自旋一定不相同(否则所有状态都相同,违反泡利不相容原理)。
这样,自旋部分的波函数就是
\begin{equation}
    \chi = \frac{1}{\sqrt{2}} (\chi_{\uparrow 1} \chi_{\downarrow 2} - \chi_{\uparrow 2} \chi_{\downarrow 1}).
    \label{eq:asym-spin}
\end{equation}
不需要其它任何条件,自旋部分的波函数就完全确定了(可以差一个因子但这无关紧要)。
因此轨道部分相同的两个电子的自旋角动量代数是单态的,即$s=0, m_s=0$。

如果两个电子的轨道运动不同,那么轨道部分的波函数可以是对称的也可以是反对称的。
假定它是对称的,那么自旋部分的波函数一定反对称,因此两个电子的自旋不可能相等。这就意味着自旋部分的波函数还是\eqref{eq:asym-spin}。
而如果轨道部分的波函数是反对称的,那么自旋部分的波函数是对称的。下面我们考虑自旋本征态。
两个电子的自旋如果相等,那么自旋波函数就是以下二者之一:
\[
    \chi_{\uparrow 1} \chi_{\uparrow 2}, \quad \chi_{\downarrow 1} \chi_{\downarrow 2}.
\]
自旋波函数当然也可以是两个不同的自旋的线性组合,即它是$\chi_{\uparrow 1} \chi_{\downarrow 2}$及其交换的线性组合,并且满足对称条件,从而为
\[
    \chi = \frac{1}{\sqrt{2}} (\chi_{\uparrow 1} \chi_{\downarrow 2} + \chi_{\uparrow 2} \chi_{\downarrow 1}).
\]
因此自旋本征态为
\begin{equation}
    \chi = \chi_{\uparrow 1} \chi_{\uparrow 2}, \quad \chi_{\downarrow 1} \chi_{\downarrow 2}, \quad \frac{1}{\sqrt{2}} (\chi_{\uparrow 1} \chi_{\downarrow 2} + \chi_{\uparrow 2} \chi_{\downarrow 1}).
    \label{eq:sym-spin}
\end{equation}
\eqref{eq:sym-spin}是一个三重态,$s=1, m_s=0, \pm 1$。

从角动量代数的角度,两个电子放在一起,它们的角动量代数的复合是两个$s=1/2$的角动量代数的复合,所得结果的$s$取值范围为$0, 1$,和刚才推导得到的一致。
具体$s$取多少由轨道波函数的情况决定。轨道部分如果对称,那么自旋部分必须反对称%
\footnote{注意这是交换对称不是空间对称,空间对称由宇称描述。}%
,这对应$s=0$;反之,轨道部分反对称,则自旋部分必须对称,对应$s=1$。

\subsubsection{交换能}

双电子原子的波函数必须满足交换对称或者反对称条件还意味着,电子之间的库伦能也会发生改变。
将两个电子之间的库伦相互作用看成微扰,计算该微扰造成的能量本征值变化的一阶修正,就是
\begin{equation}
    E = \int \dd[3]{\vb*{r}_1} \dd[3]{\vb*{r}_2} \psi^*(\vb*{r}_1, \vb*{r}_2) \frac{1}{4\pi \epsilon_0} \frac{e^2}{\abs{\vb*{r}_1 - \vb*{r}_2}} \psi(\vb*{r}_1, \vb*{r}_2).
\end{equation}
对对称态或者反对称态
\[
    \psi(\vb*{r}_1, \vb*{r}_2) = \frac{1}{\sqrt{2}} (\psi_1(\vb*{r}_1) \psi_2(\vb*{r}_2) \pm \psi_2(\vb*{r}_1) \psi_1(\vb*{r}_2)),
\]
我们有
\begin{equation}
    \begin{aligned}
        E &= \int \dd[3]{\vb*{r}_1} \dd[3]{\vb*{r}_2} \psi_1^*(\vb*{r}_1) \psi_1(\vb*{r}_1) \frac{1}{4\pi \epsilon_0} \frac{e^2}{\abs{\vb*{r}_1 - \vb*{r}_2}} \psi_2^*(\vb*{r}_2) \psi_2(\vb*{r}_2) \\
        &\pm \int \dd[3]{\vb*{r}_1} \dd[3]{\vb*{r}_2} \psi_2^*(\vb*{r}_1) \psi_1(\vb*{r}_1) \frac{1}{4\pi \epsilon_0} \frac{e^2}{\abs{\vb*{r}_1 - \vb*{r}_2}} \psi_1^*(\vb*{r}_2) \psi_2(\vb*{r}_2).
    \end{aligned}
\end{equation}
等式右边第一项就是将电子云密度看成电荷密度计算出来的库伦能,第二项则是一个没有经典对应的项,称为\concept{交换能}。
可以看到在两个电子的波函数没有很大重叠时交换能可以略去,这也是合理的。

交换能意味着两个电子的自旋角动量发生了耦合。要看出这是为什么,定义
\begin{equation}
    J = \int \dd[3]{\vb*{r}_1} \dd[3]{\vb*{r}_2} \psi_2^*(\vb*{r}_1) \psi_1(\vb*{r}_1) \frac{1}{4\pi \epsilon_0} \frac{e^2}{\abs{\vb*{r}_1 - \vb*{r}_2}} \psi_1^*(\vb*{r}_2) \psi_2(\vb*{r}_2),
\end{equation}
若$l=1$则轨道部分反对称,交换能为$-J$,若$l=0$则轨道部分对称,交换能为$J$。这样,交换能在哈密顿量中就引入这样一项:
\[
    \hat{V}_\text{ex} = -J ( \dyad{\uparrow \uparrow} + \dyad{\downarrow \downarrow} + \frac{1}{2} (\ket{\uparrow \downarrow} + \ket{\downarrow \uparrow}) (\bra{\uparrow \downarrow} + \bra{\downarrow \uparrow}) ) + J \frac{1}{2} (\ket{\uparrow \downarrow} - \ket{\downarrow \uparrow}) (\bra{\uparrow \downarrow} - \bra{\downarrow \uparrow}),
\]
可以验证,这实际上就是
\begin{equation}
    \hat{V}_\text{ex} = - \frac{1}{2} J (1 + 4 \hat{\vb*{s}}_1 \cdot \hat{\vb*{s}}_2),
\end{equation}
现在我们看出,交换能实际上会让电子的自旋倾向于趋于一致。

实际上,以上推导没有用到任何关于库伦相互作用的信息,因此可以看出,只要两个电子之间有相互作用,就肯定会有交换相互作用。

\subsection{中心场近似下的多电子原子}

本节中单电子物理量用小写,整个原子的物理量(通常就是对应的单电子物理量之和)用大写。

\subsubsection{中心场近似}\label{sec:centric-field}

将一个电子受到其它电子的作用看成一个平均场,即认为原子核受到的屏蔽作用是固定不变的。对称性分析表明这个平均场一定是一个有心力场$S(r)$,因此称之为\concept{中心场}。
除了中心场以外的相互作用称为\concept{剩余相互作用}。

在哈密顿量的势能项当中加入中心场之后,会发现能量和角量子数有关,这是因为$n$相同$l$不同的原子径向分布不同,因此受到的屏蔽也不同。角动量大的电子近核概率小,屏蔽效应强,能量高。
$n, l$完全决定了波函数的径向部分。(见\autoref{sec:quantum-number})

原子中所有电子在单电子能级上的分布情况称为\concept{电子组态},它给出了全部电子的能级的组合,也即,给出了$n$和$l$的组合。$n$和$l$相同的电子称为\concept{同科电子}。
不需要知道完整的电子状态就可以得到电子组态。
电子组态可以使用标准的spdf记号给出。
我们称不同的主量子数对应的全部电子组成一个\concept{壳层}。$n=1, 2, 3, \ldots$对应着K,L,M,N,O,P等壳层,但是现在很少用这些字母符号了。
在每一个壳层内部,不同的角量子数$l$给出不同的\concept{支壳层}。
在中心场近似下,每个支壳层内部的电子能量都是一样的。
支壳层内部的$m_l$可以发生变化,每个$m_l$给出一个\concept{轨道},每个轨道容纳自旋不同的两个电子。%
\footnote{当然,这是以$n,l,m_l,m_s$为好量子数之后的半经典叙述。实际上电子可以处于这组表象下的叠加态。}%
$l$支壳层有$2(2l+1)$个电子,即$2l+1$个轨道,$2$个自旋,因此$n$壳层有
\[
    \sum_{l=0}^{n-1} 2(2l+1) = 2n^2
\]
个电子。
外层电子的能量主要由$l$决定;这就导致了所谓的能级交错现象,即主量子数小的支壳层如果角量子数适当能量反而比较大。\concept{洪特规则}给出了不同支壳层能量的大小顺序。

以上图景可以解释一些实验中观察到的现象。
首先是\concept{原子幻数},即
\[
    Z=2, 10, 18, 36, 54, 86, \ldots
\]
时第一电离能位于峰值,然后一下子到达谷值。原子序数为原子幻数的元素即为稀有气体。
稀有气体非常稳定是因为它有电子的最高的能级都是$n$p支壳层,且全满,而$n$p与$(n+1)$s有较大能隙,p支壳层全满的原子不容易激发;此外内满壳层电子云的电荷分布球对称,对价电子吸引强。
为什么碱金属容易电离是因为原子实中电荷均匀球对称分布,几乎就是一个单独的正电荷,因此价电子受到的束缚非常弱。
同理为什么卤素容易接受电子是因为容易失去一个空穴。

\[
    \sum_{m=-l}^l \abs{Y_{lm}(\theta, \varphi)}^2 = \frac{2l+1}{4\pi},
\]
从而
\[
    \rho(\vb*{r}) = -2(2l+1) \frac{e}{4\pi} \chi_{nl}(\vb*{r})
\]

\subsubsection{化学反应}

吸能:失去电子

放能:得到电子、正负电荷中心接近从而降低库伦能

\subsubsection{原子态和光谱}

由于能量和$m_l$、$m_s$无关,如果一个支壳层非空而非全满,那么就有能量简并。$C_{2(2l+1)}^N$

\subsubsection{角动量的合成}

一个很自然的问题是,如何从不同电子的角动量代数推导出整个原子的角动量代数。
这样做的目的至少有两个,首先,在不涉及原子失去或者得到电子的情况下,只需要将原子作为整体讨论其角动量就可以;其次,在中心场近似的基础上将电子之间的相互作用作为微扰引入时,需要使用角动量的合成来获得一组基(见\autoref{sec:ls-coupling}和\autoref{sec:jj-coupling}),这组基中每一个的标签(如总角动量等)在加入微扰前后都是守恒量,这样可以省去简并微扰论需要的繁琐的行列式计算。

角动量代数的合成实际上就是将$SO(3)$的表示的直积(显然也是$SO(3)$的某个表示)分解成一系列不可约表示的直和的过程。
我们已经在\autoref{sec:algebra-of-angular}中讨论了其不可约表示的结构,现在需要分析怎么做分解。

首先讨论两个角动量代数的合成。设我们有算符$\hat{\vb*{J}}_1$和$\hat{\vb*{J}}_2$,它们各自的角量子数和磁量子数为$j_1, m_1$和$j_2, m_2$(略去了$m_j$的下标$j$)。
简单地将两个空间直积起来,可以得到一组正交归一化基
\begin{equation}
    \ket{j_1 j_2 m_1 m_2} = \ket{j_1 m_1} \otimes \ket{j_2 m_2}.
\end{equation}
下面假定$j_1$和$j_2$已经给定,也就是说被合成的是两个不可约有限维表示。这个假设是不失一般性的,因为如果需要合成两个可约表示,总是可以把它们包含的不可约表示分别合成。
这样我们有
\[
    \sum_{m_1,m_2} \dyad{j_1 j_2 m_1 m_2} = 1.
\]
设合成之后的角量子数为$j$(取值不再仅限于一个),磁量子数为$m$,则我们要做的就是计算出$\braket{jm}{j_1 j_2 m_1 m_2}$,它们称为\concept{CG系数}。

首先注意到由于
\[
    \hat{J}_z = \hat{J}_{z1} + \hat{J}_{z2},
\]
可以得到
\[
    m \hbar \ket{jm} = \sum_{m_1, m_2} (m_1 + m_2) \hbar \ket{j_1 j_2 m_1 m_2} \braket{j_1 j_2 m_1 m_2}{j m}, 
\]
从而
\[
    (m - m_1 - m_2) \braket{j_1 j_2 m_1 m_2}{j m} = 0, 
\]
或者也可以写成
\begin{equation}
    \braket{j_1 j_2 m_1 m_2}{j m} = \braket{j_1 j_2 m_1 (m - m_1)}{j m} \delta_{m, m_1 + m_2}.
    \label{eq:m-is-the-sum-of-m1-and-m2}
\end{equation}
换而言之,合成之后一定有
\begin{equation}
    m = m_1 + m_2.
\end{equation}
这当然是角动量叠加的定义导致的结果。这又意味着,让$m_1$和$m_2$扫过它们可以取的值,会得到以下结果:
\begin{itemize}
    \item $m$扫过$-(j_1 + j_2)$到$j_1 + j_2$;
    \item $m$扫过$-(j_1 + j_2 + 1)$到$j_1 + j_2 + 1$;
    \item ……
    \item $m$扫过$-\abs*{j_1 - j_2}$到$\abs*{j_1 - j_2}$。
\end{itemize}
这意味着合成之后得到的角动量代数实际上是
\begin{equation}
    j = \abs*{j_1 - j_2}, \abs*{j_1 - j_2} + 1, \ldots, j_1 + j_2 - 1, j_1 + j_2
\end{equation}
的不可约表示的直和。

\eqref{eq:m-is-the-sum-of-m1-and-m2}意味着
\[
    \ket{jm} = \sum_{\max(-j_1, m-j_2) \leq m_1 \leq \min(j_1, m+j_2)} \ket{j_1 j_2 m_1 (m - m_1)} \braket{j_1 j_2 m_1 (m - m_1)}{jm}.
\]
不断将升降算符作用在上式左右,就能够计算出CG系数。

如果被合成的若干个角动量代数的$j$都是相同的,

如果需要将两个可约表示做合成,由于不同的$j_1$和$j_2$对应的不可约表示可以合成得到具有相同的$j$的表示,需要使用$j_1$和$j_2$区分这些表示。
这样,合成之后的好量子数为$j_1, j_2, j, m$。
可以重复以上步骤将$n$个可约表示做合成,得到的好量子数为$\{j_i\} j m$。

需要格外注意的是,无论怎么将不同的电子的角动量代数做合成,最后得到的本征态必须整体是反对称的,而以上给出的合成方法不能够保证这一点。
如果参与合成的电子全部不同科,那么不会有任何问题,因为能够保证所有电子的单电子态矢量都是不一样的,因此可以随意做对称化和反对称化;
如果参与合成的电子有同科电子,那么由泡利不相容原理,同科电子磁量子数相同必定意味着轨道部分是对称的,从而自旋部分必须是反对称的。

% 偶数定则:交换两个电子会产生一个$(-1)^{l_1 + l_2 - l}$的因子

\subsection{相互作用带来的能量修正}

中心场近似中各个电子之间没有相互作用。实际上,各个电子之间有两种主要的相互作用。
其一是\concept{剩余相互作用}$\hat{H}_1$,即中心场以外的电子间剩余库伦作用(交换能、关联能等全部被收入这一项),其二是自旋轨道耦合$\hat{H}_2$。
这两者分别可以写成
\begin{equation}
    \hat{H}_1 = \frac{1}{2} \sum_{\vb*{r}_1, \vb*{r}_2} \int \dd[3]{\vb*{r}_1} \dd[3]{\vb*{r}_2} \frac{1}{4 \pi \epsilon_0} \frac{e^2}{\abs{\vb*{r}_1 - \vb*{r}_2}} - S(r),
\end{equation}
以及
\begin{equation}
    \hat{H}_2 = \sum_{i=1}^N \xi(r_i) \vb*{l}_i \cdot \vb*{s}_i.
    \label{eq:spin-orbital-many}
\end{equation}
上式中$\xi$的宗量为$\vb*{r}$的模长,这是空间各向同性的推论。

可以估计出
\[
    H_1 \sim Z, \quad H_2 \sim Z^4,
\]
且电子相距越近,剩余相互作用越明显。下面我们会发现,如果剩余相互作用远大于自旋轨道相互作用,则会导致L-S耦合,反之会导致j-j耦合,因此大部分元素的基态、轻元素的低激发态适用L-S耦合,重元素的激发态适用j-j耦合。
在$\hat{H}_1$和$\hat{H}_2$同阶时,会出现中间耦合,这适用于轻元素的高激发态和中等元素的激发态。

\subsubsection{L-S耦合}\label{sec:ls-coupling}

% TODO:同科电子的合成。在将轨道角动量和自旋角动量合成成总角动量时,如果不涉及同科电子,那么只需要放心大胆地根据角动量代数的合成法则做就可以,因为总角动量算符的每一个基矢量都可以做反对称化而不会变成零。但有同科电子时,必须把违反全同性的那些态排除掉。。TODO

如果剩余相互作用的修正远大于自旋-轨道相互作用,即
\begin{equation}
    H_2 \ll H_1 \ll H_0,
\end{equation}
就首先考虑$\hat{H}_1$的作用。此时两个电子的自旋角动量之间存在耦合,从而不同电子的轨道角动量之间也存在耦合,但是自旋角动量和轨道角动量之间尚无耦合,这称为\concept{L-S耦合}。
这样,在加入了$\hat{H}_1$之后的好量子数由$\hat{H}_0, \hat{L}^2, \hat{L}_z, \hat{S}^2, \hat{S}_z$给出,它们具有共同本征函数系。
于是把$\{n_i l_i m_{li} m_{si}\}$表象线性变换成$\{n_i l_i\} L M_L S M_S$表象,$\hat{H}_1$带来的修正在这一组表象下应当是对角的。%
\footnote{由于多个电子角动量合成之后的角动量代数是可约的,即使每个电子的$l_i$都知道了,还是必须明确给出$L$和$S$,从经典图像上说,这是因为我们不知道各个电子的角动量的指向如何。}%
这样,把$\hat{H}_1$当成微扰项,那么能量的一阶修正为
\begin{equation}
    E^{\text{rem}, (1)}_{\{n_i l_i\} LS} = \sum_{s_{z1}, s_{z2}, \ldots, s_{zN}} \int \prod \dd[3]{\vb*{r}_i} \Psi^{(0) *}_{\{n_i l_i\} L M_L S M_S}(\{\vb*{r}_i\}) \hat{H}_1 \Psi^{(0)}_{\{n_i l_i\} L M_L S M_S}(\{\vb*{r}_i\}).
    \label{eq:ls-1}
\end{equation}
中心场近似的能量$E^{(0)}_{\{n_i l_i\}}$仅仅和原子组态有关;实际上,$E^{(1)}_{LS}$只和原子组态以及$L,S$有关。
这是系统的对称性决定的,加入$\hat{H}_1$之后系统同时具有自旋旋转不变性(剩余相互作用中的交换能部分)和轨道旋转不变性(剩余相互作用中的库伦能部分),因此$M_L$和$M_S$对能量没有影响。
这样在只考虑剩余相互作用时,能级简并度为
\begin{equation}
    g_{\{n_i l_i\}LS} = (2L+1)(2S+1).
\end{equation}
从经典图景的角度说,这是因为剩余库伦能和剩余交换能分别由电子的相对角分布和自旋角动量的夹角决定,至于它们绝对地指向什么方向,并不重要。
在经典图景下各个$\{\vb*{l}_i\}$绕着$\vb*{L}$进动,为了把所有量确定下来要知道$\{\vb*{l}_i\}$的大小、$\vb*{L}$的大小和方向;自旋角动量同理。

现在在L-S耦合的系统中再引入$\hat{H}_2$,此时$L_z$和$S_z$也不再是守恒的了,但由于$\hat{H}_2$只是让角动量在自旋和轨道两种形式之间转换,总角动量在$z$轴的分量$J_z$还是守恒的,且$J^2$也是守恒的。同样$L^2$和$S^2$也是守恒的,这样好量子数是$\hat{H}_0, \hat{L}^2, \hat{S}^2, \hat{J}^2, \hat{J}_z$。
这样需要把$\{n_i l_i\} L M_L S M_S$表象再次变换成$\{n_i l_i\} L S J M_J$表象(这组表象由于$L,S$确定,\eqref{eq:ls-1}还是适用的),$\hat{H}_2$引入的修正在这一组表象下是对角的,为
\begin{equation}
    \begin{aligned}
        E^{\text{soi}, (1)}_{\{n_i l_i\} LSJ} &= \sum_{s_{z1}, s_{z2}, \ldots, s_{zN}} \int \prod \dd[3]{\vb*{r}_i} \Psi^{(0) *}_{\{n_i l_i\} L S J M_J}(\{\vb*{r}_i\}) \hat{H}_2 \Psi^{(0)}_{\{n_i l_i\} L S J M_J}(\{\vb*{r}_i\}) \\
        &= \frac{\zeta_{\{n_i l_i\}LS}}{2} \hbar^2 (J(J+1) - L(L+1) - S(S+1)).
        \label{eq:ls-2}
    \end{aligned}
\end{equation}
第二个等号看起来很奇怪不过下面马上会解释它。
由于空间旋转对称性,上式和$M_J$无关。这样,完整地考虑了两种相互作用带来的修正之后,能量修正为
\begin{equation}
    E^{\text{LS}, (1)}_{\{n_i l_i\} LSJ} = E^{\text{rem}, (1)}_{\{n_i l_i\} LS} + E^{\text{soi}, (1)}_{\{n_i l_i\} LSJ},
\end{equation}
它只和$L,S,J$有关,和$M_J$无关,于是能级简并度为
\begin{equation}
    g_{\{n_i l_i\}LSJ} = 2J + 1.
\end{equation}
从经典图景看,各个电子的$\vb*{l}_i$绕着$\vb*{L}$进动,$\vb*{s}_i$绕着$\vb*{S}$进动,而$\vb*{L}$和$\vb*{S}$又绕着$\vb*{J}$进动,由于$\hat{H}_1$远大于$\hat{H}_2$,前者远远快于后者。
这个经典图景有助于理解\eqref{eq:ls-2}的形式:由于$\vb*{l}_i$和$\vb*{s}_i$的进动非常快,可以做近似
\[
    H_2 \approx \bar{H}_2,
\]
从而
\[
    H_2 = \expval{\sum_{i=1}^N \left(\vb*{l}_i \cdot \frac{\vb*{L}}{L} \right) \frac{\vb*{L}}{L} \cdot \left(\vb*{s}_i \cdot \frac{\vb*{S}}{S} \right) \frac{\vb*{S}}{S} } = \zeta_{\{n_i l_i\}LS} \vb*{L} \cdot \vb*{S} = \frac{\zeta_{\{n_i l_i\}LS}}{2} \hbar^2 (J^2 - L^2 - S^2),
\]
这就导致了\eqref{eq:ls-2}的形式。$\zeta_{\{n_i l_i\}LS}$是$\vb*{l}_i$和$\vb*{s}_i$导致的比例系数,它在尚未半满的情况下大于零,否则小于零。

总之,L-S耦合中,能级分裂情况如下面的列表所示:
\begin{enumerate}
    \item 忽略电子间库仑相互作用,得到类氢原子近似,能量完全由主量子数$n$确定;
    \item 引入中心场近似,能量由$n, l$决定,出现能级交错,即具有较低的主量子数的能级可以高于具有较高的主量子数的能级,可以解释稀有气体、碱金属和卤素的性质;
    \item 考虑剩余相互作用,相同组态的能量发生分裂,不同的$L$和$S$之间能量不同,例如各个电子自旋同向的状态能量会低一些,于是出现前两个洪特规则:$S$越大越稳定,即电子倾向于尽可能地取朝向一致的自旋(或者说$S$较大的状态较稳定),这又意味着半满或全满的轨道比较稳定;在$S$确定的情况下,$L$越大越稳定,因为这意味着不同电子的电子云重叠得更加少;
    \item 考虑轨道自旋耦合,不同的$J$会带来不同能量,这就是类氢原子的精细结构在多电子原子中的对应,也是最后一个洪特规则,具体来说,在$S$和$L$也已经确定的情况下,对尚未半满的情况,$J$越小越稳定;否则$J$越大越稳定。
\end{enumerate}
这样,对一个组态已知的原子系统,$L, S, J$完全决定其能量,记这样的能级(或者说\concept{谱项})为$^{2S+1} L_J$,$2S+1$称为\concept{自旋多重度},因为它给出了自旋量子数$M_S$的取值个数。
将所有的$\{l_i\}$合成可以求出$L$的取值范围,将所有的$\{s_i\}$合成可以求出$S$的取值范围,将$L$和$S$合成又可以得到$J$的取值范围。
需要注意的是不是所有的$\{s_i\}$都是可能的,因为必须保持多电子波函数反对称。
精细结构的能级差为
\begin{equation}
    E_{\{n_i l_i\} LSJ} - E_{\{n_i l_i\} LS(J-1)} = J \zeta_{\{n_i l_i\}} \hbar^2.
    \label{eq:lande-gapping}
\end{equation}
这就是\concept{朗德间隔定则}:如果原子遵循L-S耦合则它成立,如果原子不遵循L-S耦合,它通常会被违反。

使用所有$\{l_i\}$合成出$L$是非常繁琐的。
在合成时,我们总是可以首先把一个满支壳层中的所有电子的角动量代数合成,而单个电子无论是磁量子数还是自旋量子数都可以跑遍$\pm l$或$\pm \frac{1}{2}$中的全部整数或半整数值,它们求和会得到零,因此一个满支壳层中所有电子的角动量合成之后,$M_L$和$M_S$一定是零,从而一个满支壳层的$L$和$S$只能是零。因此实际上满支壳层对整个原子的$L$和$S$是没有贡献的。
这个结论还意味着,当要耦合的电子数较大时,可以把它们看成“从满的支壳层拿掉少数几个电子”,由于空穴的角动量代数和电子完全一样,这样得到的角动量代数也是正确的。
换而言之,电子组态$nl^x$和$nl^{2(2l+1)-x}$的谱项是完全一样的;但需要注意这两种电子组态中,固定$L,S$并变动$J$时各个谱项的能级顺序是相反的,因为$\zeta_{\{n_i l_i\}LS}$会变号。

\subsubsection{j-j耦合}\label{sec:jj-coupling}

如果自旋-轨道相互作用的修正远大于剩余相互作用,即
\begin{equation}
    H_1 \ll H_2 \ll H_0,
\end{equation}
就需要首先考虑$\hat{H}_2$,此时每个电子的自旋角动量和轨道角动量有很强的耦合,但是不同电子的总角动量之间并没有很强的耦合,这称为\concept{j-j耦合}。
在加入$\hat{H}_2$之后,电子之间还是没有相互作用,因此我们可以只讨论单个电子的运动情况。
单个电子的哈密顿量包括两项,其一是中心场近似下的
\[
    \hat{h}_0 = \frac{\hat{p}^2}{2m_\text{e}} - \frac{Z e^2}{4\pi \epsilon_0 r} + S(\vb*{r}),
\]
其二是\eqref{eq:spin-orbital-many}导致的
\[
    \hat{h}_2 = \xi(\vb*{r}) \hat{\vb*{l}} \cdot \hat{\vb*{s}}.
\]
和上一节中讨论$\hat{H}_2$的作用时类似,一组好量子数由$\hat{h}_0, \hat{l}^2, \hat{s}^2, \hat{j}^2, \hat{j}_z$给出,于是从$nl m_l m_s$表象切换到$nlj m_j$表象,且记径向部分为$R_{nl}(\vb*{r})$(在中心场近似中,给定组态,各个电子的径向分布就给定了),则$\hat{h}_2$引入的能量修正为
\begin{equation}
    \begin{aligned}
        \epsilon_{nlj}^{\text{soi}, (1)} &= \sum_{s_z} \int \dd[3]{\vb*{r}} \psi^{(0)*}_{nlj m_j} \hat{h}_2 \psi^{(0)}_{nlj m_j} \\
        &= \frac{\xi_{nl}}{2} \hbar^2 (j(j+1) - l(l+1) - s(s+1)),
    \end{aligned}
\end{equation}
其中
\begin{equation}
    \xi_{nl} = \int_0^\infty \dd{r} (r R_{nl}(r))^2 \xi(r).
\end{equation}
由于电子间没有相互作用,这就意味着原子总能量为
\[
    E_{\{n_i l_i j_i\}} = \sum_{n, l, j} N_{nlj} \epsilon_{nlj},
\]
其中$N_{nlj}$为处于状态$(n, l, j)$的电子数目,由电子组态$\{n_i l_i\}$给定。这样$\hat{H}_2$对原子总能量的一阶修正就是
\begin{equation}
    E^{\text{soi}, (1)}_{\{n_i l_i j_i\}} = \sum_{n, l, j} N_{nlj} \epsilon_{nlj}^{\text{soi}, (1)}.
\end{equation}
能量和$m_j$没有任何关系,因此有能级简并。能级简并数为
\begin{equation}
    g_{\{n_i l_i j_i\}} = \prod_{n, l, j} C_{2j+1}^{N_{nlj}}.
\end{equation}
从经典图景看,每个电子的$\vb*{l}$和$\vb*{s}$绕着$\vb*{j}$进动。

现在再讨论剩余相互作用带来的修正。加入剩余相互作用之后,诸$\{m_{ji}\}$不再是好量子数,因为不同电子的角动量会有耦合,不过$\vb*{J}$还是守恒的,且$j^2$也还是守恒的,于是好量子数由$\hat{H}_0, \hat{J}^2, \hat{J}_z$给出。
从$\{n_i l_i j_i m_{ji}\}$表象(每个$i$对应一个单电子的$nl j m_j$表象)切换到$\{n_i l_i j_i\} J M_J$表象,$\hat{H}_1$带来的一阶修正为
\begin{equation}
    E^{\text{rem}, (1)}_{\{n_i l_i j_i\} J} = \sum_{s_{z1}, s_{z2}, \ldots, s_{zN}} \int \prod \dd[3]{\vb*{r}_i} \Psi^{(0) *}_{\{n_i l_i j_i\} J M_J}(\{\vb*{r}_i\}) \hat{H}_1 \Psi^{(0)}_{\{n_i l_i j_i\} J M_J}(\{\vb*{r}_i\}).
\end{equation}
同样空间旋转对称性意味着$M_J$对能量修正没有影响。于是同时考虑了两种相互作用带来的修正,我们有
\begin{equation}
    E^{\text{jj}, (1)}_{\{n_i l_i j_i\} J} = E^{\text{soi}, (1)}_{\{n_i l_i j_i\}} + E^{\text{rem}, (1)}_{\{n_i l_i j_i\} J},
\end{equation}
能量简并度为
\begin{equation}
    g_{\{n_i l_i j_i\}J} = 2J+1.
\end{equation}
从经典图景看,此时每个电子的$\vb*{l}$和$\vb*{s}$绕着$\vb*{j}$进动,各个$\vb*{j}_i$绕着$\vb*{J}$进动。前者明显快于后者。

j-j耦合中,能级分裂情况由以下列表所示:
\begin{enumerate}
    \item 忽略电子间库仑相互作用,得到类氢原子近似,能量完全由主量子数$n$确定;
    \item 引入中心场近似,能量由$n, l$决定,出现能级交错,可以解释稀有气体、碱金属和卤素的性质;
    \item 引入轨道自旋耦合,能量由$\{n_i l_i j_i\}$确定,这是类氢原子的剩余相互作用在多电子原子中的对应,这一步造成的能级分裂的能量大小顺序由$\{j_i\}$决定,$j_i$越大,说明方向相同的自旋角动量和轨道角动量越多,因此能量越高;
    \item 引入剩余相互作用,能量由$\{n_i l_i j_i\} J$确定,这一步中的能级分裂次序没有特别的规律。
\end{enumerate}

j-j耦合中原子组态已知时谱项表示为$(j_1, j_2, \ldots, j_N)_J$。
获得谱线的方式是先将$l$和$s$合成出$j$,然后再把各个$\{j_i\}$合成成$J$。
第一步是非常显然的,第二步则比较繁琐。实际上,此时满支壳层对$J$同样没有贡献。与L-S耦合类似,我们将一个满支壳层中的电子首先来做合成,则
\[
    M = \sum_{j=\abs{l-1/2}}^{l+1/2} \sum_{m_j=-j}^j m_j = 0,
\]
因此满支壳层的角动量代数的磁量子数唯一的取值是零,因此$J=0$,即这个角动量代数对整个原子的总角动量代数没有贡献。
实际上,使用类似的方法,可以证明满支壳层带来的自旋-轨道修正也是零,因为能量修正有的为正有的为负,加起来等于没修正。

\subsection{多电子原子的偶极辐射}

\subsubsection{选择定则}

设电偶极跃迁光子的总角动量为$\vb*{j}_\gamma$,显然有
\[
    \vb*{J} + \vb*{j}_\gamma = \vb*{J}'.
\]
对光子$j_\gamma=1$,则
\[
    \Delta J = \pm 1, 0, 
\]
而且$J$和$J'$不能同时为零,否则角动量不可能守恒。
同样,磁量子数的变化为
\[
    \Delta M_J = \pm 1, 0.
\]
同时我们还有宇称守恒,而光子具有奇宇称,而原子的宇称为
\[
    \Pi = (-1)^{\sum_{j} l_j},
\]
因此
\[
    \Delta L = \pm 1, \pm 3, \ldots,
\]
对低激发态,这意味着%
\footnote{通常只考虑低激发态的原因是,如果一份能量足够让原子的多个电子被激发,那也足够让单个电子被电离。
后者是更为常见的现象。}%
\[
    \Delta l_\text{trans} = \pm 1, \quad \Delta l_\text{other} = 0.
\]
无论如何,总角量子数发生变化意味着电偶极跃迁发生在不同组态的谱项之间。
同一组态中的谱项之间不能发生电偶极跃迁,因为轨道角动量没有发生变化,从而不满足宇称守恒。
这样就得到了任何一个多电子原子的偶极跃迁应遵循的选择定则:
\begin{equation}
    \Delta J = 0, \pm 1, \quad \Delta L = \pm 1, \pm 3, \ldots, \quad \Delta M_J = \pm 1, 0,
    \label{eq:many-electron-selective}
\end{equation}
且$J$不能从0变化到0。

除了以上规则,通过计算跃迁矩阵元,还可以发现一些选择定则(称为附加定则)。对L-S耦合,有
\begin{equation}
    \Delta S = 0, \quad \Delta L = \pm 1.
    \label{eq:l-s-selective}
\end{equation}
对j-j耦合,有
\begin{equation}
    \Delta j_\text{trans} = \pm 1, 0, \quad \Delta j_\text{other} = 0.
    \label{eq:j-j-selective}
\end{equation}

\subsubsection{类氢光谱}

考虑一个类氢原子(即氢原子或者碱金属原子),其基态价壳组态为$n$s。
考虑低激发态,即只有价电子跃迁。这样,角动量——无论是自旋还是轨道——完全来自价电子。

首先采用L-S耦合,则
\[
    L = l, \quad S = s = \frac{1}{2}, 
\]
而
\[
    J = j = \begin{cases}
        l \pm 1/2, &\quad l \neq 0, \\
        1/2, &\quad l = 0.
    \end{cases}
\]
应用L-S耦合的选择定则\eqref{eq:l-s-selective}和\eqref{eq:many-electron-selective},我们有
\[
    \Delta j = 0, \pm 1, \quad \Delta m_j = 0, \pm 1, \quad \Delta s = 0, 
\]
容易看出这正是单电子的选择定则。使用j-j耦合也可以得到同样的结果。

如果是L-S耦合,我们可以根据$nl$分析会有哪些谱项。
\begin{enumerate}
    \item 如果$l=0$,即价电子占据支壳层$n$s,那么$J=1/2$,于是谱项为\lsterm{2}{S}{1/2};
    \item 如果$l=1$,即价电子占据支壳层$n$l,那么$J=1/2, 3/2$,谱项为\lsterm{2}{P}{1/2}和\lsterm{2}{P}{3/2},后者自旋和轨道角动量平行,由\eqref{eq:spin-ortibal-coupling}可以看出后者能量高于前者,这和单电子的精细结构来自同样的物理机制;
    \item $l=3$,按照以上步骤可以得到两个谱项\lsterm{2}{D}{3/2}和\lsterm{2}{D}{5/2},后者能量高于前者;
    \item $l=4$,得到\lsterm{2}{F}{5/2}和\lsterm{2}{F}{7/2}。更高能量的谱项暂不考虑。
\end{enumerate}

现在以钠为例分析可以有哪些跃迁。首先同一$l$的谱项肯定不能相互跃迁。
下面列举了一些常见的跃迁,同一类型的跃迁导致的谱线称为一个\concept{线系}。

\begin{itemize}
    \item 会跃迁到基态3s上的只有$l=1$的谱项,即从3p,4p,5p等跃迁到3s上,这一组谱线都是双线(因为$l=1$有精细结构谱项分裂),其中3p到3s就是钠双黄线(有时也称为\concept{D双线},虽然它并不是漫线系的)。
    这一线系称为\concept{主线系(principal series)}。随着波长变短,精细结构导致的波长分裂也会变小。
    这就是将主线系的出发态$l=1$命名为p态的原因。
    \item 4s,5s,6s等谱线可以跃迁到3p,从一个单能级跳到双能级,从而导致一组双线,这组双线不同波长的两条谱线距离相等(就是\lsterm{2}{P}{1/2}和\lsterm{2}{P}{3/2}的差距),非常清晰,称为\concept{锐线系(sharp series)}。这就是锐线系的出发态$l=0$名为s态的原因。
    \item d态也可以跃迁到p态。d和p都有精细结构分裂,但由于$\Delta j$最大取到1,这实际上是一个三线系,称为\concept{漫线系(diffuse series)}。这就是漫线系的出发态$l=2$称为d态的原因。
    \item 与漫线系类似,f态可以跃迁到d态,产生一个三线系,称为\concept{基线系(fundamental series)}。这就是基线系的出发态$l=3$称为f态的原因。
\end{itemize}

其余的谱线都发生在比较高的能级之间,不容易观察到。

以上提到的都是发射光谱,但显然也可以把产生它们的过程倒转过来而得到吸收光谱。
由于基态为3s,吸收光谱中通常只能看到主线系。

对每个确定的$L$,都最多只有两个$J$值,因为$S$固定为$1/2$。可以计算出这两个$J$值之间的能级差为
\begin{equation}
    \Delta U = \frac{(\alpha Z)^4}{2 n^3 l(l+1)} E_0.
\end{equation}

\subsubsection{类氦光谱}

下面讨论类氦原子,即基态价壳组态为$n$s$^2$的原子,包括氦原子和碱土金属原子。
低激发态只有一个价电子发生跃迁。

使用L-S耦合。设跃迁价电子的角量子数为$l$,则$L=l$。两个电子给出$S=0, 1$,即有自旋单态和三重态。
附加选择定则\eqref{eq:l-s-selective}要求$\Delta S=0$,因此三重态和单态之间的跃迁是禁戒的——这样一来,如果只考虑偶极辐射,三重态的原子永远处于三重态,单态的原子永远处于单态。
我们将三重态的原子称为\concept{正氦},将单态的原子称为\concept{仲氦}。
三重态原子的自旋波函数是对称的,因此轨道波函数必须反对称,因此两个价电子不能出现在同一个轨道上。
由于仅考虑低激发态即只有一个价电子,两个价电子出现在同一轨道上只可能意味着电子组态为基态组态,即$n$s$^2$。
由于自旋平行会让能量降低,同一电子组态的正氦能量低于仲氦。
总之,正氦没有$n$s$^2$价电子组态,且同一电子组态的能量低于仲氦。

基于L-S耦合,仲氦的谱项列举如下:(以下$n=1$指的是价壳层)
\begin{enumerate}
    \item \lsterm{1}{S}{0},$n=1, 2, ,\ldots$;
    \item \lsterm{1}{P}{1},$n=2, 3, \ldots$;
    \item \lsterm{1}{D}{2},$n=3, 4, \ldots$;
    \item \lsterm{1}{F}{3},$n=4, 5, ,\ldots$。
\end{enumerate}
而正氦的谱项列举如下:
\begin{enumerate}
    \item \lsterm{3}{S}{1},$n=2, 3,\ldots$;
    \item \lsterm{3}{P}{0,1,2},$n=2, 3, \ldots$;
    \item \lsterm{3}{D}{1,2,3},$n=3, 4, \ldots$;
    \item \lsterm{3}{F}{2,3,4},$n=4, 5, ,\ldots$。
\end{enumerate}
请注意正氦没有$n=1$的态;此外除了S态以外,正氦的每个电子组态均存在关于$J$的能级三重分裂,$J$越大能量越高。

下面分别分析正氦和仲氦的光谱,由选择定则还是会有主线系、漫线系、锐线系、基线系。
对仲氦,所有这些线系都是单线系。
对正氦,主线系、锐线系是三线系,基线系、漫线系是三线系。

正氦和仲氦还有亚稳态。对仲氦,\lsterm{1}{S}{0}是亚稳态,因为低于它的能级只有基态,但从它到基态的过程$\Delta L = 0$;同样对正氦,\lsterm{3}{S}{1}是亚稳态,它要跃迁到基态必须发生正氦到仲氦的转变。
换而言之,1s2s电子组态是亚稳态,无论是仲氦还是正氦。它要跃迁到基态只能通过原子碰撞、双光子过程、电四极子跃迁、磁偶极子跃迁等微弱得多的过程。

\subsection{磁场中的多电子原子}

现在将\eqref{eq:magnetic-hamiltonian}引入。外加磁场破坏了空间各向同性,从而让磁量子数不再造成简并。
记由此产生的哈密顿量为
\begin{equation}
    \hat{H}_3 = - \hat{\vb*{\mu}} \cdot \vb*{B}.
\end{equation}

\subsubsection{弱磁场近似}

首先假定磁场很弱,比剩余相互作用和自旋-轨道耦合都弱,即
\begin{equation}
    \hat{H}_3 \ll \hat{H}_1, \hat{H}_2 \ll \hat{H}_0,
\end{equation}
从而可以在L-S耦合或者j-j耦合的基础上,计算$\hat{H}_3$引入的一阶微扰。计算结果是,
\begin{equation}
    E^{\text{mag}, (1)}_{JM_J} = \frac{J_z}{\hbar} g_J \mu_\text{B} B = m_J g_J \mu_\text{B} B = - \vb*{\mu}_J \cdot \vb*{B},
\end{equation}
其中
\begin{equation}
    \vb*{\mu}_J = - g_J \frac{\mu_\text{B}}{\hbar} \vb*{J}
    \label{eq:atom-mean-magnetic}
\end{equation}
称为\concept{原子平均磁矩}。这个形式和单电子自旋角动量或轨道角动量对磁矩的贡献非常相似:要乘以一个无量纲修正因子$g_J$(称为\concept{朗德g因子})。
无论如何,能量关于$M_J$的简并就解出了。

可以计算出对L-S耦合在$J$不为零时有
\begin{equation}
    g_J = \frac{3}{2} + \frac{S(S+1) - L(L+1)}{2J(J+1)},
    \label{eq:g-factor-ls}
\end{equation}
$J$为零时没有磁矩。
对j-j耦合$g_J$还和$\{j_i\}$有关。特别的,对L-S耦合下的双电子组态,我们有
\begin{equation}
    g_{j} = \frac{3}{2} + \frac{s(s+1) - l(l+1)}{2j(j+1)},
\end{equation}
这就是所谓的\concept{单电子朗德g因子};而对j-j耦合下的双电子组态,有
\begin{equation}
    g_{J j_1 j_2} = \frac{g_{j_1} + g_{j_2}}{2} + \frac{(g_{j_2} - g_{j_1})(j_2(j_2+1) - j_1(j_1+1))}{2J(J+1)},
\end{equation}
其中$g_{j_1}$和$g_{j_2}$指的是单电子的磁矩和总角动量相差的因子。

\subsubsection{半经典图像}

上面的表达式的导出需要使用严格的量子力学计算,不过实际上它们还是可以使用角动量矢量的经典图像推导并解释。

对L-S耦合,由强到弱的三种微扰分别造成以下影响:

\begin{enumerate}
    \item 剩余相互作用$\hat{H}_1$使得不同电子的轨道磁矩绕总轨道角动量$\vb*{L}$快速进动,自旋磁矩绕总自旋角动量$\vb*{S}$快速进动;
    \item 自旋-轨道耦合$\hat{H}_2$让磁矩$\vb*{\mu}_L$,$\vb*{\mu}_S$和总磁矩$\vb*{\mu}$绕着$\vb*{J}$缓慢进动(由于自旋和轨道角动量和对应磁矩之间的比例关系差一个因子2,$\vb*{\mu}$和$\vb*{J}$并不平行);
    \item 磁场作用$\hat{H}_3$让$\vb*{J}$围绕$\vb*{B}$做拉莫尔进动,这个进动的幅度又远小于上述两种进动。
\end{enumerate}

这就意味着,在拉莫尔进动的时间尺度上,总磁矩平均而言只在$\vb*{J}$的方向上有分量,它垂直于$\vb*{J}$的分量一直在不停地变化,无法产生明显效应。%
\footnote{系统不能对过快的外界扰动产生反应,正如简谐振子展示的那样。}%
因此有效的$\vb*{\mu}$为
\[
    \bar{\vb*{\mu}} = \vb*{\mu}_J = \left(\vb*{\mu} \cdot \frac{\vb*{J}}{\abs{\vb*{J}}}\right) \frac{\vb*{J}}{\abs{\vb*{J}}} = - \underbrace{\frac{(\vb*{L} + 2\vb*{S}) \cdot (\vb*{L} + \vb*{S})}{J^2}}_{g_J} \frac{\mu_\text{B}}{\hbar} \vb*{J}.
\]
这就找到了$J$不为零时的$g_J$的表达式。通过
\[
    \vb*{J} = \vb*{L} + \vb*{S},
\]
就得到了\eqref{eq:g-factor-ls}。
在$J$为零时,虽然一时看不出$g_J$应该取什么值,但由于$M_J$只能够取$0$,总磁矩也应该是零。

j-j耦合由于有大量的$j$,处理起来稍微复杂一些。三种由强到弱的微扰分别造成以下影响:
\begin{enumerate}
    \item 自旋-轨道耦合$\hat{H}_2$让每个电子的角动量合成成$\vb*{j}$,单电子磁矩$\vb*{\mu}$绕着$\vb*{j}$快速进动;
    \item 剩余相互作用$\hat{H}_1$让每个电子的单电子磁矩绕着$\vb*{J}$缓慢进动;
    \item 磁场作用$\hat{H}_3$让$\vb*{J}$绕$\vb*{B}$做最慢的拉莫尔进动。
\end{enumerate}
$\vb*{\mu}_i$绕$\vb*{j}_i$的进动在拉莫尔进动的时间尺度下非常快,因此真正有效的只有$\vb*{\mu}_i$在$\vb*{j}_i$方向上的投影$\vb*{\mu}_{j i}$。
相应的,原子总磁矩$\vb*{\mu}$绕$\vb*{J}$进动也很快,有效的只有$\vb*{\mu}$在$\vb*{J}$方向上的投影,从而
\[
    \bar{\vb*{\mu}} = \vb*{\mu}_J = \sum_i \left( \vb*{\mu}_{j i} \cdot \frac{\vb*{J}}{\abs{\vb*{J}}} \right) \frac{\vb*{J}}{\abs{\vb*{J}}}.
\]
使用和L-S耦合非常相似的方法,我们有
\begin{equation}
    \vb*{\mu}_{j i} = - g_{j i} \frac{\mu_\text{B}}{\hbar} \vb*{j}_i, \quad g_{j i} = \frac{3}{2} + \frac{s_i(s_i + 1) - l_i (l_i + 1)}{2 j_i (j_1 + 1)},
\end{equation}
代入上式即可。

以上两个推导都建立在拉莫尔进动非常慢这一假设上。
可以计算出,与$\vb*{J}$的变化对应的力矩为$\vb*{\mu}_J \times \vb*{B}$,则$\vb*{J}$的运动方程为
\[
    \dv{\vb*{J}}{t} = \vb*{\mu}_J \times \vb*{B},
\]
当然也可以直接从哈密顿量得到这个公式。将\eqref{eq:atom-mean-magnetic}代入上式,就得到
\[
    \dv{\vb*{J}}{t} = \frac{g_J \mu_\text{B} \vb*{B}}{\hbar} \times \vb*{J},
\]
从而推导出\concept{拉莫尔进动角速度}
\begin{equation}
    \vb*{\omega}_\text{L} = \frac{g_J \mu_\text{B} \vb*{B}}{\hbar}.
\end{equation}
这就是总角动量绕着磁场进动的角速度。相应的,可以计算出
\begin{equation}
    E^{\text{mag}, (1)}_{JM_J} = M_J g_J \mu_\text{B} B = M_J \hbar \omega_\text{L},
    \label{eq:small-magnetic-gapping}
\end{equation}
和量子力学计算出的结果一致。在拉莫尔进动中总角动量不守恒,但是磁量子数和角量子数——总角动量的进动锥体的母线和高度——都还是好量子数。
可以看到,磁矩和磁场的夹角越大,能量越高。

\subsubsection{塞曼效应}

既然磁场在能量中引入了$M_J$的依赖,原本L-S耦合和j-j耦合的每一个谱项$2J+1$重简并会发生等间距的分裂。
$M_J$越大意味着角动量和磁场的夹角越小(磁场在$z$方向上)。
分裂产生的$2J+1$个子能级称为\concept{塞曼能级},它会导致弱磁场中的原子光谱出现分裂,即\concept{塞曼效应}。
磁场的加入彻底地消除了所有简并。

塞曼效应可以分成两种,一种是\concept{正常塞曼效应},谱线等间距分裂成三根,另一种是\concept{反常塞曼效应},即不满足以上条件的光谱分裂。
塞曼效应发现时量子力学尚未建立,但是塞曼的老师——洛伦兹——通过经典电偶极振子模型计算出来正常塞曼效应。这也就是“正常”和“反常”这两个概念的来源。

正常塞曼效应的模型大体上是这样的:原子价电子受到正离子线性回复力,做简谐运动,与此同时受到一个磁场力,从而运动方程可以写成
\begin{equation}
    m_\text{e} \dv[2]{\vb*{r}}{t} = - m_\text{e} \omega_0^2 \vb*{r} + \left( - e \dv{\vb*{r}}{t} \times \vb*{B} \right),
\end{equation}
照惯例取$\vb*{B}$的方向指向$z$轴,则得到
\[
    \begin{aligned}
        m_\text{e} \dv[2]{x}{t} &= - m_\text{e} \omega_0^2 x - e \dot{y} B, \\
        m_\text{e} \dv[2]{y}{t} &= - m_\text{e} \omega_0^2 y + e \dot{x} B, \\
        m_\text{e} \dv[2]{z}{t} &= - m_\text{e} \omega_0^2 z.
    \end{aligned}
\]
考虑时谐解,并假定磁场非常小,则可以得到三个振动模式,圆频率分别是
\begin{equation}
    \omega = \omega_0, \quad \omega_0 + \frac{eB}{2m}, \quad \omega_0 - \frac{eB}{2m}.
\end{equation}
因此,原本的一个能级一分为三,三个能级间隔相等,相邻能级之间相差
\begin{equation}
    \Delta E = \frac{\hbar eB}{2m} = \mu_\text{B} B.
\end{equation}

但实际上,实验表明虽然的确会有能级分裂,且能级分裂的数量级的的确确也在$\mu_\text{B} B$的量级上,但前面的系数却并不总是$1$。
使用量子力学中的弱磁场近似\eqref{eq:small-magnetic-gapping}%
\footnote{
    需要注意的是自旋-轨道耦合本身是涉及一个磁场的,这个磁场大小大约是\SI{12}{T},已经是非常强的磁场了。外加磁场是弱还是强需要和这个做比较。
}%
可以发现我们有
\begin{equation}
    \Delta E = g_J \mu_\text{B} B,
\end{equation}
可见实际的能级分裂虽然也是均匀的分裂,但$\mu_\text{B} B$前面的系数的确未必是1,并且能级也未必分裂成三条。
在总自旋为零时,
\[
    g_J = \frac{3}{2} + \frac{-L(L+1)}{2J(J+1)} = 1,
\]
因此能级分裂和通过经典谐振子模型计算出来的相同。
总之,总自旋不为零的能级必然发生反常塞曼效应,总自旋为零且$J=1$的能级发生正常塞曼效应。

\subsubsection{强磁场和帕邢-巴克效应}

当磁场比较强时,若原子在没有磁场时为L-S耦合,则
\begin{equation}
    {H}_2 \ll {H}_3 \ll {H}_1 \ll {H}_0,
\end{equation}
此时做完剩余相互作用的微扰之后就应该考虑$\hat{H}_3$,相比之下自旋-轨道耦合则是非常小的。我们知道$\hat{H}_2$引入的能级分裂满足朗德间隔定则\eqref{eq:lande-gapping},即%
\footnote{这里我们使用某部分哈密顿量引入的能级差来衡量其影响。哈密顿量本身的绝对大小并无意义,因为总是可以加上或者减去任意大小的常数。系统中的量子数发生变化,对应的能级会发生什么变化才真正体现这一部分相互作用是否重要。}%
\[
    H_2 \sim J \zeta_{LS} \hbar^2.
\]
$\hat{H}_3$引入的能级分裂满足\eqref{eq:small-magnetic-gapping},即能级间隙为$\hbar \omega_\text{L}$,于是
\[
    H_3 \sim \hbar \omega_\text{L}.
\]
当磁场强到
\begin{equation}
    \omega_\text{L} \gg \zeta_{LS} \hbar,
\end{equation}
即拉莫尔进动的速度远快于自旋-轨道耦合时,先前的弱磁场近似就不再适用了。
我们在做完$\hat{H}_1$的微扰之后需要直接做$\hat{H}_3$的微扰。
在$\hat{H}_3$的扰动之后,$L, M_L, S, M_S$仍然是好量子数。注意到$\hat{H}_3$形如
\[
    \hat{H}_3 = (\hat{L}_z + 2 \hat{S}_z) \frac{\mu_\text{B} B}{\hbar}, 
\]
它在$L M_L S M_S$的表象下的本征值为
\begin{equation}
    E^{\text{mag}, (1)}_{\{n_i l_i\} L M_L S M_S} = (M_L + 2 M_S) \mu_\text{B} B,
\end{equation}
则先考虑$\hat{H}_1$的微扰,再考虑$\hat{H}_3$的微扰之后,有
\[
    E^{\text{rem,mag}, (0)}_{\{n_i l_i\} L M_L S M_S} = E^{\text{rem}, (1)}_{\{n_i l_i\} LS} + E^{\text{mag}, (1)}_{\{n_i l_i\} L M_L S M_S}.
\]
然后再考虑$\hat{H}_2$的微扰,此时$\vb*{L}$和$\vb*{S}$都在绕着$\vb*{B}$进动,$\vb*{J}$实际上并不守恒,已经失去意义了。
选择定则为
\begin{equation}
    \Delta L = 0, \pm 1, \quad \Delta S = 0, \quad \Delta M_S = 0, \quad \Delta M_L = 0, \pm 1.
\end{equation}
可以看到所有关于$J$的定则都被移除了。若原子放出光子,则
\begin{equation}
    \omega = \omega_0 + \frac{\mu_\text{B} B}{\hbar} \Delta M_L.
\end{equation}
$\Delta M_L$取$1$和$-1$代表光子是$\sigma$光,取$0$则代表光子是$\pi$光。

\subsubsection{磁共振}

外加磁场导致能级分裂,由于选择定则,分裂的能级之间不存在电偶极跃迁。但实际上,这些能级之间可以有磁偶极跃迁。
实际上,这就是\concept{磁共振}的原理。本节首先从一个半经典图像来分析磁共振,然后再考虑严格的量子理论。

由于$\vb*{J}$和$\vb*{\mu}$平行,我们有
\[
    \dv{\vb*{\mu}_J}{t} = \vb*{\omega}_J \times \vb*{\mu}_J,
\]
其中
\[
    \vb*{\omega}_J = \frac{g_J \mu_\text{B} \vb*{B}}{\hbar}.
\]
没有使用下标L是因为我们将要考虑方向随着时间变动的磁场。
现在如果施加两个磁场,一个磁场是恒定的$\vb*{B}_0$,还有一个是以平行于$z$轴的角速度$\vb*{\omega}$左旋的弱磁场
\begin{equation}
    \vb*{B}_1 = B_1 (\cos(\omega t) \vb*{e}_x + \sin(\omega t) \vb*{e}_y),
\end{equation}
则
\begin{equation}
    \vb*{\omega}_J = \vb*{\omega}_\text{L} + \vb*{\omega}_{1}, \quad \vb*{\omega}_\text{L} = \frac{g_J \mu_\text{B} B_0}{\hbar} \vb*{e}_z, \quad \vb*{\omega}_1 = \frac{g_J \mu_\text{B} B_1}{\hbar} (\cos(\omega t) \vb*{e}_x + \sin (\omega t) \vb*{e}_y).
\end{equation}
将实验室坐标系记为$S$,将相对实验室坐标系以角速度$\vb*{\omega}$旋转的参考系记为$S'$,则
\[
    \dv{t} = \vb*{\omega} \times + \dv'{t},
\]
于是
\[
    \dv'{t} \vb*{\mu}_J = (\vb*{\omega}_\text{L} - \vb*{\omega} + \vb*{\omega}_1) \times \vb*{\mu}_J.
\]
由于在$S'$系中$\vb*{B}_1$是恒定的,上式意味着在$S'$系中$\vb*{\mu}_J$以\concept{Rabi角速度}
\begin{equation}
    \vb*{\omega}_\text{R} = \vb*{\omega}_\text{L} - \vb*{\omega} + \vb*{\omega}_1, \quad \omega_\text{R} = \sqrt{(\omega_\text{L} - \omega)^2 + \omega_1^2}
\end{equation}
进动。我们首先考虑一个简化的情况:初始时,$\vb*{J}$和$\vb*{B}_0$平行。这样,以$\vb*{\omega}_\text{R}$在$S'$系中的指向为$z'$轴,并要求$t=0$时$S'$系的$x'$轴和$S$系的$x$轴重合,则在$S'$系中有
\begin{equation}
    \mu_{Jx}' = \mu_J \frac{\omega_1}{\omega_\text{R}} \sin (\omega_\text{R} t), \quad \mu_{Jy}' = - \mu_J \frac{\omega_1}{\omega_\text{R}} \cos (\omega_\text{R} t), \quad \mu_{Jz}' = \mu_J \frac{\omega_\text{L} - \omega}{\omega_\text{R}}.
\end{equation}
切换回$S$系,就有
\begin{equation}
    \mu_{Jz} = \mu_J \left( \frac{(\omega_\text{L} - \omega)^2}{\omega_\text{R}^2} + \frac{\omega_1^2}{\omega_\text{R}^2} \cos (\omega_\text{R} t) \right).
\end{equation}
在没有时变磁场时$\mu_{Jz}$不会有时间变化,因为显然此时$\omega_1=0$。因此,$\mu_{Jz}$的波动范围越大,时变磁场带来的影响越明显。
当$\omega_\text{L}=\omega$时,$\mu_{Jz}$的波动最为明显,即磁矩与外场有效地交换能量,达到共振状态。此时$\vb*{\mu}_J$在$S'$系中完全位于$x'$-$y'$平面内。
此时
\begin{equation}
    \begin{aligned}
        \mu_{Jx} &= \mu_J \sin(\omega_1 t) \cos(\omega_\text{L} t - \pi / 2), \\
        \mu_{Jy} &= \mu_J \sin(\omega_1 t) \sin(\omega_\text{L} t - \pi / 2), \\
        \mu_{Jz} &= \mu_J \cos(\omega_t t).
    \end{aligned}
    \label{eq:rotating-spin}
\end{equation}
$\mu_{Jx}$和$\mu_{Jy}$是水平旋转磁场$\vb*{B}_1$带来的感生磁矩,它有一个相位落后$\pi / 2$。
由于$B_1$比$B_0$小得多,有$\omega_1 \ll \omega_\text{L}$,因此感生磁矩不停地绕着$z$轴以角速度$\omega_\text{L}$转动,而转动平面的经度以$\omega_1$升高。

使用类似的方法也可以得到系统对右旋磁场或者沿着$z$方向线偏振磁场的响应,不过这两种磁场实际上并不能够导致磁共振。
总之,对系统施加一个周期性磁场,只有其左旋成分会导致磁共振。

从更加量子的角度看,磁共振实际上是电子磁矩导致的电子和外部磁场的耦合。
哈密顿量$\hat{H}_3$和$-\vb*{d} \cdot \vb*{E}$形式一致,它会导致\concept{磁偶极辐射}。
在这里由于外界磁场是完全给定的,不需要像考虑电偶极辐射那样将所有可能的电场模式造成的跃迁做非相干叠加。
仿照电偶极辐射的推导(见\autoref{sec:electro-dipole-hopping}),我们有% 下面的说法是错的
\[
    \braket{j}{\psi(t)} = \frac{\ii}{\hbar} \ee^{-\ii E_n t / \hbar} \mel{j}{\hat{\mu}_z}{i} B_1 \frac{\ee^{\ii(\omega_{ji}-\omega) t} - 1}{\ii (\omega_{ji} - \omega)},
\]
因此是否能够发生跃迁取决于两点:首先,外加磁场的频率必须要和$\omega_{ji}$匹配;其次,磁矩的矩阵元不应该是零。
后一个要求就是\concept{磁偶极跃迁的选择定则}。通过计算矩阵元可以发现,有
\begin{equation}
    \Delta J = 0, \pm 1, \quad \Delta M_J = 0, \pm 1, \quad \Delta l = 0, \Delta n = 0,
\end{equation}
且$J$不能从$0$变化到$0$。这说明原子组态不能够发生任何变化。对L-S耦合,还有以下附加定则:
\begin{equation}
    \Delta L = \Delta S = 0.
\end{equation}

例如,因为外加磁场而分裂出来的塞曼能级之间肯定能够发生跃迁。
由于$\Delta M_J=\pm 1$(不能够取$0$否则就没有发生任何跃迁),有
\[
    \hbar \omega_{ij} = g_J \mu_\text{B} B M_J,
\]
于是能够发生的跃迁对应的圆频率就是$\omega_\text{L}$。
这对应微波频段。
特别的,如果只有一个价电子,且没有轨道角动量,那么
\[
    L = 0, \quad S = \frac{1}{2}, \quad J = \frac{1}{2},
\]
此时
\begin{equation}
    \omega = \omega_\text{L} = \frac{2 \mu_\text{B} B}{\hbar},
\end{equation}
即\concept{电子自旋共振}。

计算磁矩的期望值,会发现只要把\eqref{eq:rotating-spin}中的磁矩替换为期望值,结果就是正确的。

实际上,只要有磁场和带磁矩的粒子,就能够发生磁共振,因此完全可以做\concept{核磁共振},通过共振情况推断不同成分的原子的含量。

\chapter{原子核}

以上讨论的都是原子核以外的物理,而下面将要讨论原子核内部的物理。原子核内部的物理涉及强力和弱力,完整地处理这些相互作用需要QCD,本文将不详细讨论这个理论。

\section{原子核的成分}

\subsection{质子和中子的发现}

质子的发现来自1919年卢瑟福使用$\alpha$粒子轰击轻元素而得到的一种“共有成分”。现在我们知道,他大致是发现了这么一个过程:
\[
    \nuclear{He}{4}{2} + \nuclear{X}{A}{Z} \longrightarrow \nuclear{Y}{A+3}{Z+1} + \nuclear{H}{1}{1}.
\]
在此过程中质量数(左上标)和电荷数(左下标)守恒。
那么,质子——事后发现这就是氢原子核——应当是原子核的一个组成部分。

当时还发现了另一个过程:
\[
    \nuclear{X}{A}{Z} \longrightarrow \nuclear{Y}{A}{Z+1} + \text{e},
\]
即\concept{$\beta$衰变}。乍一看似乎非常自然地可以认为原子核同时由质子和电子组成。但实际上这个假设给出了错误的核半径、核自旋、磁矩。
就核半径而言,电子和质子组成的体系可以看成电子嵌入在均匀的正电荷“凝胶”中,则电子实际上是三维简谐振子。计算发现要能够稳定地将电子约束在原子核内,原子核的尺度要达到原子尺度——显然这是错误的。
就核自旋而言,质子和电子都是自旋$1/2$,那么,质量数为偶数,电荷数为奇数的原子核的自旋应该是半整数,而实验结果为整数。
从磁矩的角度,质子磁矩远小于电子,则核磁矩应该是玻尔磁子的数量级,且质子的磁矩可以略去,但实际测量得到的磁矩全部都是质子的量级。
因此,这个核模型是错误的。

1920年,卢瑟福猜测,原子核内存在质量和质子接近的一种粒子。1930年,用$\alpha$轰击Be,得到一种穿透力很强的中性射线,当时假定为$\gamma$射线;1931年,用这种射线轰击石蜡,得到\SI{4.8}{MeV}的质子流,当时假定为$\gamma$射线和质子发生康普顿散射,按此假设,$\gamma$射线的能量要达到\SI{50}{MeV}。
1932年,用氮原子核代替石蜡,得到\SI{1.2}{MeV}的质子流,则对应的$\gamma$射线能量要达到\SI{89}{MeV}。
这么高的光子质量实在是离谱,且使用同一放射源产生这么大能量范围的射线完全解释不通。
查德威克于是断言,此射线不可能是$\gamma$射线,而应该是一种电中性粒子组成的,$\alpha$射线轰击原子核会发生以下过程:
\[
    \nuclear{He}{4}{2} + \nuclear{X}{A}{Z} \longrightarrow \nuclear{Y}{A+3}{Z+2} + \nuclear{n}{1}{0}.
\]
这种质量数为1(和质子相同),不带电的粒子就是中子。据使用此中性射线轰击不同原子量的物质得到的质子流的能量,在非相对论近似下反推可以发现中子质量和质子质量基本一致。

以质子和中子为基本组成的原子核模型与核物理中的实验结果一致。因此,这两种粒子是核物理层次上的基本粒子,即\concept{核子}。
核子实际上仍然可分,可以分解成夸克。

\subsubsection{自旋和磁矩}

质子和中子也有自旋,也有磁矩。设
\begin{equation}
    \mu_\text{N} = \frac{e \hbar}{2 m_\text{p}}
\end{equation}
为\concept{核磁子},它是把波尔磁子中的电子质量替换成质子质量之后得到的结果。质子和中子的轨道角动量和自旋角动量决定了其磁矩,即
\begin{equation}
    \vb*{\mu}_\text{p} = \frac{\mu_\text{N}}{\hbar} (g_{\text{p} l} \vb*{l} + g_{\text{p} s} \vb*{s}), \quad 
    \vb*{\mu}_\text{n} = \frac{\mu_\text{N}}{\hbar} (g_{\text{n} l} \vb*{l} + g_{\text{n} s} \vb*{s}),
\end{equation}
质子携带一个电荷,中子不携带电荷,则
\begin{equation}
    g_{\text{p}l} = 1, \quad g_{\text{n} l} = 0.
\end{equation}
质子和中子的自旋角动量对磁矩的贡献和电子非常不同,实际上
\begin{equation}
    g_{\text{p} s} \approx 5.6, \quad g_{\text{n} s} \approx -3.8.
\end{equation}
会有这种奇怪的行为,是因为核子的“自旋”(即所谓\concept{核自旋})是核子的总角动量扣除核子整体的轨道角动量之后剩下的部分,实际上是在核子静止的情况下,组成核子的各个夸克的自旋加上它们之间的轨道角动量得到的结果。
通常用$\hat{\vb*{I}}$表示核自旋。与之前将一个原子内部的所有电子磁矩合成成一个等效磁矩相似,我们有
\begin{equation}
    \vb*{\mu}_I = \underbrace{g_I \frac{\mu_\text{N}}{\hbar}}_\gamma \vb*{I}, \quad \mu_{Iz} = g_I \mu_\text{N} m_I.
\end{equation}

\subsection{核子的结合}

\subsubsection{结合能}

可以通过电子散射来测量原子核的半径。实验数据表明,设$A$为质量数,我们有这样的公式:
\begin{equation}
    R = r_0 A^{1/3},
\end{equation}
从而
\[
    V = \frac{4}{3} \pi r_0^3 A \propto A.
\]
这表明核几乎不可压缩,核子是非常紧密地挨在一起的。

当核子聚集,受核力力约束而成为原子核时,会放出能量,从而发生\concept{质量亏损}
\begin{equation}
    \Delta m = Z m_\text{p} + (A - Z) m_\text{n} - m_\text{N}.
\end{equation}
\concept{结合能}为
\begin{equation}
    E_\text{B} = \Delta m c^2 = - \left( \sum_i T_i + \sum_{i < j} V_{ij} \right), 
\end{equation}
显然动能为正,上式意味着为了形成稳定的原子核,势能必须为负并且绝对值很大。
结合能当然在任何过程中都存在,但是在化学反应中能量变化在eV级别,而在核反应中能量变化在MeV级别,后者要明显得多。

定义
\begin{equation}
    \epsilon = \frac{E_\text{B}}{A} 
\end{equation}
为\concept{比结合能},比结合能越大说明系统越稳定。随着原子序数增大,比结合能先上升后下降,因此轻核聚变,重核裂变。

\subsubsection{核力}

让带有同种电荷的核子能够聚集在一起的是一种非常强的相互作用:\concept{核力}。其强度比电磁相互作用大两个数量级(预料之中,否则无法把核子聚集起来),且力程非常短,因此几乎只有相邻的核子之间才能够有核力,或者说具有\concept{短程饱和性}。
确定这一点的方式是,如果只有相邻的核子之间才有核力,则应有
\[
    E_\text{B} \sim A,
\]
而如果任意两个核子之间都可以有核力,则应该有
\[
    E_\text{B} \sim A(A-1).
\]
实验支持前者而不是后者。
核力和电荷无关,因此核子数量相同的原子核的结合能的变化仅仅和电荷之间的库伦排斥有关,如$\nuclear{He}{3}{2}$的结合能小于$\nuclear{H}{3}{1}$,因为前者有两个正电荷,库伦排斥更强。

这样,质子和中子在核力上具有完全一样的性质。它们仅有的不同在于电荷这个标签。实际上,这可以对应到一个称为\concept{同位旋}的标签上。
对核子,同位旋为$I = 1/ 2$,
\begin{equation}
    I_z = \begin{cases}
        \frac{1}{2}, \quad &\text{p}, \\
        -\frac{1}{2}, \quad &\text{n},
    \end{cases}
\end{equation}
电荷为
\begin{equation}
    q = \left( I_z + \frac{1}{2} \right) e.
\end{equation}
同位旋的合成和角动量的合成类似。

核力还和自旋的相对取向有关。此外,在\SI{0.8}{fm}内,核力实际上是排斥力,而在\SI{0.8}{fm}到\SI{2}{fm}内,核力才是吸引力。距离超过\SI{10}{fm}时核力基本上观察不到了。

电磁相互作用对应的玻色子是光子,核力也应该有对应的玻色子和玻色场。
在相对论性量子场论的框架下,最容易想到的方案是,取一个带有质量的克莱因-高登场,其稳态解形如
\[
    \phi \propto \frac{\ee^{- r / a}}{r},
\]
从而这是一个短程相互作用,特征尺度为$a$,且可以根据强力的力程\SI{2}{fm}估算出其质量在电子和质子之间。
实验中果然探测到了这样的粒子,即$\pi^{0, \pm 1}$介子,它们寿命有限,有自旋,有电荷。
$\pi$介子可以看成取不同同位旋($I=1,I_z=0, \pm 1$)的同一种粒子。
对介子,有
\begin{equation}
    q = I_z e.
\end{equation}
所有反应都需要保持各个守恒量相同,这样就有
\[
    \text{p} \longleftrightarrow \text{p} + \pi^0, \quad \text{n} \longleftrightarrow \text{n} + \pi^0, \quad \text{p} \longleftrightarrow \text{n} + \pi^+, \quad \text{n} \longleftrightarrow \text{p} + \pi^-. 
\]
以上四个过程各自对应着费曼图的一种“核子-核子-介子”角点。

实际上,以上过程并不是最基本的。核子和介子由夸克组成,将夸克粘在一起的是胶子,上面所说的核子-介子相互作用——也即核力——实际上是类似于范德华力这样的剩余作用力。

\subsection{衰变}

\subsubsection{半衰期和活度}

一个原子核的中子数为
\begin{equation}
    N = Z - A.
\end{equation}
一个原子核中的中子数和质子数显然不能够相差太多,否则或者质子过多而自发溢出,或者中子过多而自发溢出,因为原子核也遵循量子力学,满足泡利不相容原理,有核能级,所以质子和中子都是从低到高填充这些能级,而如果中子数过多,那么中子能级上被占据了太多的状态,和质子能级产生很大的能量差,因此不稳定;质子数过多同理。
对比较重的核,中子通常比质子多一些,因为这样可以稀释质子之间的库伦排斥。

这样,在$N$-$Z$图上就有两条线,一条是质子溢出线,一条是中子溢出线,这两条线夹成的区域以外不能形成束缚态的核;在这两条线中间的区域分成“核心”和“周边”,前者是稳定核素,后者是不稳定核素。

通过费米黄金法则可以计算衰变的跃迁率,它和时间基本无关,因此对一个原子核,设$p(t)$是$t$时刻还不发生衰变的概率,则有以下状态转移方程:
\[
    p(t+\dd{t}) = p(t) (1 - \lambda \dd{t}),
\]
于是
\[
    \dv{p}{t} = - \lambda p.
\]
如果同时考虑大量核,设$N(t)$为$t$时刻还未衰变的核的数目,则
\[
    \dv{N}{t} = - \lambda N,
\]
从而
\begin{equation}
    N(t) = N(0) \ee^{- \lambda t}.
    \label{eq:nuclear-decay}
\end{equation}
这样,虽然对一个单独的核,我们不能预测其什么时候衰变,但大量核服从明确的统计规律。
\eqref{eq:nuclear-decay}意味着核的衰变有固定的半衰期
\begin{equation}
    T_{1/2} = \frac{\ln 2}{\lambda}.    
\end{equation}
平均寿命为
\begin{equation}
    \tau = \int t \cdot P(\text{decay between $t$ and $t+\dd{t}$}) = \int t \cdot p(t) \cdot \lambda \dd{t} = \frac{1}{\lambda} = \frac{T_{1/2}}{\ln 2}.
\end{equation}
单位时间内发生衰变的核的数目给出了一块放射性材料的放射性活性,称为\concept{活度},它是
\begin{equation}
    R = \lambda N = R(0) \ee^{-\lambda t}.
\end{equation}
活度的单位是\si{Bq}或者\si{Ci}(贝克勒尔或者居里),定义为
\[
    \SI{1}{Bq} = \SI{1}{s^{-1}}, \quad \SI{1}{Ci} = \SI{3.7e10}{Bq}.
\]

\subsubsection{常见衰变类型}

\concept{$\alpha$衰变}是指原子核放出一个$\alpha$粒子的衰变,即
\[
    \nuclear{X}{A}{Z} \longrightarrow \nuclear{Y}{A-4}{Z-2} + \nuclear{He}{4}{2}.
\]
可以通过测量衰变之后释放的$\alpha$粒子的动能来测定衰变能,两者的关系为
\begin{equation}
    Q = \frac{A}{A-4} T_\alpha.
\end{equation}
实际上,可以测定到几种分立的$T_\alpha$,这表明原子核的确具有离散的能级。

另一种衰变是$\beta$衰变,即释放电子的衰变。可以尝试将其通式写成
\[
    \nuclear{X}{A}{Z} \longrightarrow \nuclear{Y}{A}{Z+1} + \text{e}^-.
\]
使用和之前类似的方法,可以发现
\begin{equation}
    Q \approx T_\text{e}.
\end{equation}
不过,实验观察到释放出的电子的能量连续分布,这和原子核能级离散矛盾;由于核子数在反应前后不变,$X$和$Y$要么同时是玻色子要么同时是费米子,而电子是费米子,因此反应前后的角动量必定一个是整数一个是半整数,不可能相等。
唯一的可能是,在$\beta$衰变中产生了一种不参与强相互作用、不参与电磁相互作用、质量非常小的粒子。这就是\concept{中微子}。
整个过程如下:
\[
    \text{n} \longrightarrow \text{p} + \text{e}^- + \bar{\nu}_\text{e}.
\]
中微子实际上是一种自旋$1/2$的费米子,考虑它之后$\beta$衰变就是
\[
    \nuclear{X}{A}{Z} \longrightarrow \nuclear{Y}{A}{Z+1} + \text{e}^- + \bar{\nu}_\text{e}.
\]
会产生中微子是因为还有一种相互作用:\concept{弱相互作用},它会让中子衰变成质子和电子型反中微子。
另一种弱相互作用导致的衰变是质子衰变成中子、正电子和电子型中微子,如下:
\[
    \text{p} \longrightarrow \text{n} + \text{e}^+ + \nu_\text{e}.
\]
此外也可以发生\concept{核外电子俘获},即质子俘获一个电子,而形成中子和一个电子型中微子,即
\[
    \text{p} + \text{e}^- \longrightarrow \text{n} + \nu_\text{e}.
\]
以上三种过程发生的几率都在同一数量级上。

弱相互作用也是短程力,它比强相互作用的强度小九个数量级,其力程比强相互作用小三个数量级。
与弱相互作用有关的媒介玻色子为$\text{W}^\pm$和$Z^0$,称为\concept{中间玻色子}。
前两种互为反粒子,带有电荷$\pm 1$,第三种则不带电。

还有一种所谓的$\gamma$衰变,它指的实际上是高核能级的原子核释放$\gamma$光子的过程。


\end{document}