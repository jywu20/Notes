\documentclass[hyperref, a4paper]{article}

\usepackage{geometry}
\usepackage{float}
\usepackage{titling}
\usepackage{titlesec}
% No longer needed, since we will use enumitem package
% \usepackage{paralist}
\usepackage{enumitem}
\usepackage{footnote}
\usepackage{soulutf8}
\usepackage{enumerate}
\usepackage{amsmath, amssymb, amsthm}
\usepackage{mathtools}
\usepackage{bbm}
\usepackage{graphicx}
\usepackage{subfigure}
\usepackage{physics}
\usepackage{tensor}
\usepackage{siunitx}
\usepackage{booktabs}
\usepackage[version=4]{mhchem}
\usepackage{tikz}
\usepackage{xcolor}
\usepackage{listings}
\usepackage{autobreak}
\usepackage[ruled, vlined, linesnumbered]{algorithm2e}
\usepackage[backend=bibtex,doi=false,isbn=false,url=false]{biblatex}
\addbibresource{gw.bib}
\usepackage[colorlinks,unicode]{hyperref} % , linkcolor=black, anchorcolor=black, citecolor=black, urlcolor=black, filecolor=black
\usepackage[most]{tcolorbox}
\usepackage{prettyref}

% Page style
\geometry{left=3.18cm,right=3.18cm,top=2.54cm,bottom=2.54cm}
\titlespacing{\paragraph}{0pt}{1pt}{10pt}[20pt]
\setlength{\droptitle}{-5em}

% More compact lists 
% \setlist[itemize]{itemindent=17pt, leftmargin=1pt}

% Math operators
\DeclareMathOperator{\timeorder}{T}
\DeclareMathOperator{\diag}{diag}
\DeclareMathOperator{\legpoly}{P}
\DeclareMathOperator{\primevalue}{P}
\DeclareMathOperator{\sgn}{sgn}
\newcommand*{\ii}{\mathrm{i}}
\newcommand*{\ee}{\mathrm{e}}
\newcommand*{\const}{\mathrm{const}}
\newcommand*{\suchthat}{\quad \text{s.t.} \quad}
\newcommand*{\argmin}{\arg\min}
\newcommand*{\argmax}{\arg\max}
\newcommand*{\normalorder}[1]{: #1 :}
\newcommand*{\pair}[1]{\langle #1 \rangle}
\newcommand*{\fd}[1]{\mathcal{D} #1}
\DeclareMathOperator{\bigO}{\mathcal{O}}
\DeclareMathOperator{\id}{id}

% TikZ setting
\usetikzlibrary{decorations.text}
\usetikzlibrary{arrows,shapes,positioning}
\usetikzlibrary{arrows.meta}
\usetikzlibrary{decorations.markings}
\tikzstyle arrowstyle=[scale=1]
\tikzstyle directed=[postaction={decorate,decoration={markings,
    mark=at position .5 with {\arrow[arrowstyle]{stealth}}}}]
\tikzstyle ray=[directed, thick]
\tikzstyle dot=[anchor=base,fill,circle,inner sep=1pt]

% Algorithm setting
% Julia-style code
\SetKwIF{If}{ElseIf}{Else}{if}{}{elseif}{else}{end}
\SetKwFor{For}{for}{}{end}
\SetKwFor{While}{while}{}{end}
\SetKwProg{Function}{function}{}{end}
\SetArgSty{textnormal}

\newcommand*{\concept}[1]{{\textbf{#1}}}

\DeclareMathOperator{\im}{im}

% Embedded codes
\lstset{basicstyle=\ttfamily,
  showstringspaces=false,
  commentstyle=\color{gray},
  keywordstyle=\color{blue}
}

\newcommand{\soliddoc}{\href{../solid/solid.pdf}{this note}}
\newcommand{\lastlec}{\href{./2022-3-15.pdf}{the last lecture}}

% Reference formatter
\newrefformat{fig}{Fig.~\ref{#1}}

% Color boxes
\tcbuselibrary{skins, breakable, theorems}
\newtcbtheorem[number within=section]{warning}{Warning}%
  {colback=orange!5,colframe=orange!65,fonttitle=\bfseries, breakable}{warn}
\newtcbtheorem[number within=section]{note}{Note}%
  {colback=green!5,colframe=green!65,fonttitle=\bfseries, breakable}{note}
\newtcbtheorem[number within=section]{info}{Info}%
  {colback=blue!5,colframe=blue!65,fonttitle=\bfseries, breakable}{info}

% Disable unsupported commands in bookmark titles 
\pdfstringdefDisableCommands{%
  \def\\{}%
  \def\texttt#1{<#1>}%
  \def\mathbb#1{#1}%
}
\pdfstringdefDisableCommands{\def\eqref#1{(\ref{#1})}}

\makeatletter
\pdfstringdefDisableCommands{\let\HyPsd@CatcodeWarning\@gobble}
\makeatother

\title{Photon transferring}
\author{Jinyuan Wu}

\begin{document}

\maketitle

\section{Heuristic derivation of the light transportation equation}

In this section we derive the light transportation equation 
in a way inspired by the quantum Boltzmann equation (QBE).
We know in an isotropic medium, the dispersion relation of light is always 
in the form of 
\begin{equation}
    \omega = c k,  
\end{equation}
where $c$ is the effective speed of light in the medium, 
and therefore 
\begin{equation}
    \grad_{\vb*{k}} = c \vu*{k} \eqqcolon c \vu*{s},
\end{equation}
where $\vu*{s}$ is the direction of the photon momentum, 
i.e. the direction of light propagation.
Thus, for photons in an \ul{isotropic and uniform medium}, 
In the LHS of QBE, we have 
\begin{equation}
    \begin{aligned}
        \text{LHS} &= \pdv{f}{t} + \pdv{\omega}{\vb*{k}} \cdot \pdv{f}{\vb*{r}}
        - \pdv{\omega}{\vb*{r}} \cdot \pdv{f}{\vb*{k}} \\
        &= \pdv{f}{t} + c \vu*{s} \cdot \pdv{f}{\vb*{r}}.
    \end{aligned}
\end{equation}
\ul{Ignoring non-linear processes}, due to conservation of energy, 
the magnitude of the $\vb*{k}$ vector is conserved, 
and therefore we may confine ourselves to 
a small segment of $f(\vb*{r}, \vb*{k}, t)$
where $\abs*{\vb*{k}}$ is a given constant.
We name the resulting distribution function $L(\vb*{r}, \vu*{s}, t)$
\concept{radiance}, 
where the degree of freedom $\vb*{k}$ 
is reduced to its direction.
We choose the following normalization scheme: 
\begin{equation}
    I(\vb*{r}, t) = \int \dd{\Omega'} L(\vb*{r}, \vu*{s}, t),
\end{equation}
where $I$ is the radiation intensity at $\vb*{r}, t$. 
There of course has to be some kind of normalization constant 
between $f$ and $L$; 
currently, however, since we are only working with single-photon processes, 
and all scattering events can be thought of as 
scattering with some kind of ``disorders'' in the medium, 
the collision integral in QBE, obtained from Fermi golden rule, 
is linear with respect to $f$: 
it always takes the form of 
\[
    - 2\pi \sum_{\vb*{k'}} \abs*{M(\vb*{k} \to \vb*{k}')}^2 (f(\vb*{r}, \vb*{k} , t) - f(\vb*{r}, \vb*{k}', t) )
    \delta(\omega_{\vb*{k}} - \omega_{\vb*{k}'}),
\] 
where the quantum corrections to the incoming and the outgoing terms 
cancel each other. 
Applying this fact to the isotropic, uniform and linear case 
mentioned above, we find the collision integral in the 
Boltzmann of photon is 
\begin{equation}
    \begin{aligned}
        \text{RHS} &= c \mu_{\text{s}} \int \dd{\Omega'} (L(\vb*{r}, \vu*{s}', t) - L(\vb*{r}, \vu*{s}, t)) P(\vu*{s}, \vu*{s}') \\
        &= c \mu_{\text{s}} \int \dd{\Omega'} L(\vb*{r}, \vu*{s}', t)  P(\vu*{s}, \vu*{s}')
        - c \mu_{\text{s}} L(\vb*{r}, \vu*{s}, t) ,
    \end{aligned}
\end{equation}
where we have decomposed the scattering matrix 
into an overall strength $c \mu_{\text{s}}$ 
(the $c$ factor will cancel with the $c$ factor in the second term of LHS), 
and a function $P(\vu*{s}, \vu*{s}')$ -- for some reason called the \concept{phase function} -- 
measuring how anisotropic the scattering process is, 
and the normalization condition is 
\begin{equation}
    \int \dd{\Omega'} P(\vu*{s}, \vu*{s}') = 1,
\end{equation}
and when the scattering process is also completely isotropic, we have 
\begin{equation}
    P(\vu*{s}, \vu*{s}') = \frac{1}{4 \pi}.
\end{equation}
The QBE of photons therefore becomes the following 
\concept{radiation transfer equation (RTE)}: 
\begin{equation}
    \frac{1}{c} \pdv{L}{t} + \vu*{s} \cdot \grad_{\vb*{r}} L = 
    - \mu_{\text{s}} L + \mu_{\text{s}} \int \dd{\Omega'} 
    L(\vb*{r}, \vu*{s}', t) P(\vu*{s}, \vu*{s}').
    \label{eq:rad-transfer-1}
\end{equation}
The conditions of the validity of the above equation 
includes the aforementioned assumptions that 
\ul{the medium is uniform and isotropic, 
and there is no nonlinear process in the system,} 
and also the fundamental validity condition of QBE: 
\ul{the validity of gradient expansion of the effective density matrix 
in the Wigner representation}.
Physically speaking, this means the system should allow 
the formation of defined wave pockets, 
which has well-defined positions 
but also looks like a plane wave when we zoom in on it. 

The next step is to find the true meaning of $\mu_{\text{s}}$. 
Assuming a clean background and a stationary configuration, 
the second term in the RHS of \eqref{eq:rad-transfer-1} vanishes, 
and we know $I \propto L$, 
and we find 
\[
    \dv{I}{z} = - \mu_{\text{s}} I, 
\]
where $z$ is the propagation length. 
So we find $1 / \mu_{\text{s}}$ is roughly 
the mean path $l_{\text{s}}$ between two scattering incidents, 
and \eqref{eq:rad-transfer-1} can also be alternatively written as 
\begin{equation}
    \frac{1}{c} \pdv{L}{t} + \vu*{s} \cdot \grad_{\vb*{r}} L = 
    \frac{1}{l_{\text{s}}} \int \dd{\Omega'} 
    (L(\vb*{r}, \vu*{s}', t) - L(\vb*{r}, \vu*{s}, t)) P(\vu*{s}, \vu*{s}'),
    \label{eq:rad-transfer-2}
\end{equation}
from which we have 
\begin{equation}
    \dv{I}{z} + \frac{1}{l_{\text{s}}} I = 0 
\end{equation}
when the system reaches the stationary state. 

Some scattering processes turn the photons into other degrees of freedom, 
and in the foreseeable future they will not come back; 
these processes are just absorption processes, 
and we can just add them to \eqref{eq:rad-transfer-1} 
using the relaxation time approximation, and get 
\begin{equation}
    \begin{aligned}
        \frac{1}{c} \pdv{L}{t} + \vu*{s} \cdot \grad_{\vb*{r}} L &= 
        - \mu_{\text{a}} L +
        \underbrace{\frac{1}{l_{\text{s}}}}_{\mu_{\text{s}}} \int \dd{\Omega'} 
        (L(\vb*{r}, \vu*{s}', t) - L(\vb*{r}, \vu*{s}, t)) P(\vu*{s}, \vu*{s}') \\
        &= - \underbrace{(\mu_{\text{a}} + \mu_{\text{s}})}_{\mu_{\text{t}}} L 
        + \mu_{\text{s}} \int \dd{\Omega'} 
        L(\vb*{r}, \vu*{s}', t) P(\vu*{s}, \vu*{s}') ,
    \end{aligned}
    \label{eq:rte-final}
\end{equation}
where $\mu_{\text{t}}$ is known as the extinction coefficient. 

A final correction to RTE is the spatial non-uniformity of optical properties, 
including scattering and change of light speed. 
This means both $\mu_{\text{a/s}}$ and $P(\vu*{s}, \vu*{s}')$ 
may have spatial variance, 
and so does $c$; 
of course, this variance shouldn't break the validity of QBE. 
In this case even though the correct generalization of RTE contains, say, 
$\grad_{\vb*{r}} (c L)$, 
due to the slow spatial variance of $c$, 
we can still rewritten it as $c \grad_{\vb*{r}} L$, 
and get back to \eqref{eq:rte-final}.

\section{From RTE to the diffusion equation}

Under the following assumptions, RTE is reduced to the diffusion equation: 
\begin{itemize}
    \item \ul{The radiance is nearly isotropic 
    -- and therefore if we do spherical harmonic expansion to it, 
    only the $l = 0, 1$ components need to be kept.} 
    This happens when scattering is strong enough 
    so that the direction of light propagation 
    is randomized; 
    to be exact, this means \ul{scattering is much stronger than absorption}:
    $1 / c \mu_{\text{a}}$ is the time scale of 
    how long radiation lasts in the system before being dampened by absorption, 
    and scattering should randomized the directions of light propagation 
    within this period of time. 
    Note that this also means that if the radiance starts with a highly 
    anisotropic configuration, 
    then there is already a (although short) period of time in the beginning 
    during which the diffusion equation is not a good description 
    of the dynamics of the system. 
    \item \ul{The change of current density is much slower 
    than the speed photons pass the mean free path.}
    \item \ul{The scattering property has rotational symmetry (although not completely isotropic)}:
        thus $P(\vu*{s}, \vu*{s}') = P(\vu*{s} \cdot \vu*{s}')$.
        This can be seen as included in the second approximation, 
        since otherwise it's not likely that the radiance finally converges to 
        an isotropic solution.
\end{itemize}

If we only keep the first two spherical harmonic components of $L$
(here the first approximation is used), 
i.e. the $\vu*{s}$ dependence of $L$ 
is either trivial 
or is proportional to something dot $\vu*{s}$, 
then we have
\begin{equation}
    L(\vb*{r}, \vu*{s}, t) = \frac{1}{4 \pi} \Phi(\vb*{r}, t) 
    + \frac{3}{4 \pi} \vb*{J} \cdot \vu*{s},
\end{equation}
where 
\begin{equation}
    \vb*{J}(\vb*{r}, t) = \int \dd{\Omega} \vu*{s} L(\vb*{r}, \vu*{s}, t)
\end{equation}
is recognized as the current density; 
this seems intuitive from the microscopic meaning of $L$,
and we are also going to explicitly show that $\vb*{J}$
appears in its expected position in the diffusion equation.
We can explicitly verify that the normalization conditions are correct: 
for example, we have 
\begin{equation}
    \int \dd{\Omega} \frac{3}{4 \pi} \vb*{J} \cdot \vu*{s} \cdot \underbrace{\cos \theta}_{\hat{s}_z} = 
    J_z, 
\end{equation} 
where the $J_{x, y}$ terms, since they contain a $\sin \varphi$ or $\cos \varphi$ factor, 
vanish under the integral over $\dd \Omega$. 

Now, by applying $\int \dd{\Omega}$ to \eqref{eq:rte-final}, the LHS becomes 
\begin{equation}
    \text{LHS} = \frac{1}{c} \pdv{\Phi}{t} 
    + \int \dd{\Omega} \vu*{s} \cdot \grad_{\vb*{r}} L(\vb*{r}, \vu*{s}, t)
    = \frac{1}{c} \pdv{\Phi}{t} 
    + \div \underbrace{
        \int \dd{\Omega} \vu*{s}  L(\vb*{r}, \vu*{s}, t) 
    }_{\vb*{J}}, 
\end{equation}
where we have used the condition 
\begin{equation}
    \div(\vu*{s} L) = \vu*{s} \cdot \grad L
    \label{eq:move-grad-1}
\end{equation} 
since $\vu*{s}$ has no spatial dependence, 
while the RHS becomes 
\begin{equation}
    \begin{aligned}
        \text{RHS} &= - \mu_{t} \Phi 
        + \mu_{\text{s}} \int \dd{\Omega'} L(\vb*{r}, \vu*{s}', t) \int \dd{\Omega} P(\vu*{s}, \vu*{s}') \\
        &= - \mu_{t} \Phi 
        + \mu_{\text{s}} \int \dd{\Omega'} L(\vb*{r}, \vu*{s}', t) \\
        &= - \mu_{t} \Phi 
        + \mu_{\text{s}} \Phi = - \mu_{\text{a}} \Phi ,
    \end{aligned}
\end{equation}
and thus we have 
\begin{equation}
    \frac{1}{c} \pdv{\Phi}{t} + \div{\vb*{J}} + \mu_{\text{a}} \Phi = 0.
\end{equation}
Now we see $\vb*{J}$ indeed is the current density.

Then, we apply $\int \dd{\Omega} \vu*{s}$ to \eqref{eq:rte-final}.
This time the calculation will be slightly more non-trivial.
The LHS now is 
\begin{equation}
    \text{LHS} = \frac{1}{c} \pdv{\vb*{J}}{t} + 
    \div \int \dd{\Omega} \vu*{s} \vu*{s} L, 
\end{equation}
where we again use the condition \eqref{eq:move-grad-1}. 
A direct evaluation tells us 
\begin{equation}
    \frac{1}{4\pi} \int \dd{\Omega} \vu*{s} \vu*{s} \Phi = \frac{1}{3} \Phi,
\end{equation}
and the contribution from the $\vb*{J}$ part is zero. 
The $(z, z)$ component of $\int \dd{\Omega} \vu*{s} \vu*{s}$, for example, is 
\[
    \int \dd{\Omega} \cos^2 \theta = 
    \int_{0}^{\pi} \sin \theta \dd{\theta} \cos^2 \theta \cdot 2 \pi 
    = \frac{2}{3} \cdot 2 \pi,
\]
and hence the $1/3$ factor. 
On the other hand, the $(x, z)$ or $(y, z)$ components suffer from 
the existence of a vanishing $\int \dd{\varphi} \sin\varphi$ or $\cos \varphi$ factor, 
and for the $(x, y)$ component, $\int \dd{\varphi} \sin \varphi \cos \varphi$ is still zero. 
So now we find the full expression of the LHS: 
\begin{equation}
    \text{LHS} = \frac{1}{c} \pdv{\vb*{J}}{t} + \frac{1}{3} \grad{\Phi}.
\end{equation}
As for the RHS, we have 
\begin{equation}
    \text{RHS} = - \mu_{\text{t}} \vb*{J} + \mu_{\text{s}} \int \dd{\Omega'} 
    L(\vb*{r}, \vu*{s}', t) \int \dd{\Omega} \vu*{s} P(\vu*{s} \cdot \vu*{s}').
\end{equation}
We use $\vu*{s}'$ as the $z$ axis, and then $P(\vu*{s} \cdot \vu*{s}') = P(\cos \theta)$, 
and therefore $\int \dd{\Omega} \vu*{s} P(\vu*{s} \cdot \vu*{s}')$ 
is always parallel to $\vu*{s}'$, 
since the $x$ and $y$ components contain 
$\int \dd{\varphi} \sin \varphi$ and $\int \dd{\varphi} \cos \varphi$
and therefore vanish.
We therefore write 
\begin{equation}
    \int \dd{\Omega} \vu*{s} P(\vu*{s} \cdot \vu*{s}') = g \vu*{s}',
\end{equation}
where $g$ is an unknown factor.
The RHS therefore becomes 
\begin{equation}
    \text{RHS} = - \mu_{\text{t}} \vb*{J} 
    + \mu_{\text{s}} g \int \dd{\Omega'} \vu*{s}' L(\vb*{r}, \vu*{s}', t) 
    = - \mu_{\text{t}} \vb*{J} 
    + \mu_{\text{s}} g \vb*{J}.
\end{equation}
So we have 
\begin{equation}
    \frac{1}{c} \pdv{\vb*{J}}{t} + \frac{1}{3} \grad{\Phi}
    = - \mu_{\text{t}} \vb*{J} 
    + \mu_{\text{s}} g \vb*{J}
    = - (\mu_{\text{a}} + \mu_{\text{s}}') \vb*{J}, 
    \quad \mu_{\text{s}}' = (1 - g) \mu_{\text{s}}.
\end{equation}
Now we invoke the condition that the change of the current is slow 
compared with $c \mu_{\text{t}}$, 
and we approximately, we have 
\begin{equation}
    \frac{1}{3} \grad{\Phi}
    = - \mu_{\text{t}} \vb*{J} 
    + \mu_{\text{s}} g \vb*{J}
    = - (\mu_{\text{a}} + \mu_{\text{s}}') \vb*{J}, 
    \quad \mu_{\text{s}}' = (1 - g) \mu_{\text{s}}.
\end{equation}
This gives the constitutive relation between the current density 
and the gradient of radiance. 

Thus, we find with the three -- or two -- assumptions listed at the start of this section, 
we have 
\begin{equation}
    \frac{1}{c} \pdv{\Phi}{t} + \div{\vb*{J}} + \mu_{\text{a}} \Phi = 0, \quad 
    \vb*{J} = - D \grad{\Phi},
\end{equation}
where 
\begin{equation}
    D = \frac{1}{3 (\mu_{\text{a}} + \mu_{\text{s}}')}, \quad 
    \mu_{\text{s}}' = (1 - g) \mu_{\text{s}}, 
\end{equation}
and 
\begin{equation}
    \int \dd{\Omega} \vu*{s} P(\vu*{s} \cdot \vu*{s}') = g \vu*{s}'
    \Rightarrow 2 \pi \int_{0}^{\pi} \sin \theta \dd{\theta} \cdot \cos \theta P(\cos \theta)
    = g .
\end{equation}
Now we find RTE has been completely reduced to a diffusion equation.

\section{Volume rendering}

In most everyday cases, the time evolution of $L$ can be ignored: 
the system reaches equilibrium as soon as ``the light is turned on'', 
since the speed of light is very fast. 
It's also frequently assumed that $c$ is 
\ul{one hundred percent uniform in space
and not just very slow in its spatial change}. 
In these cases, we get 
\begin{equation}
    \begin{aligned}
        \vu*{s} \cdot \grad_{\vb*{r}} L &= 
        - \mu_{\text{a}} L +
        \underbrace{\frac{1}{l_{\text{s}}}}_{\mu_{\text{s}}} \int \dd{\Omega'} 
        (L(\vb*{r}, \vu*{s}', t) - L(\vb*{r}, \vu*{s}, t)) P(\vu*{s}, \vu*{s}') \\
        &= - \underbrace{(\mu_{\text{a}} + \mu_{\text{s}})}_{\mu_{\text{t}}} L 
        + \mu_{\text{s}} \int \dd{\Omega'} 
        L(\vb*{r}, \vu*{s}', t) P(\vu*{s}, \vu*{s}') .
    \end{aligned}
    \label{eq:se-final}
\end{equation}
The time evolution term is thrown away. 
This equation is sometimes known as the \concept{scattering equation}.


\begin{figure}
    \centering
    

\tikzset{every picture/.style={line width=0.75pt}} %set default line width to 0.75pt        

\begin{tikzpicture}[x=0.75pt,y=0.75pt,yscale=-1,xscale=1]
%uncomment if require: \path (0,300); %set diagram left start at 0, and has height of 300

%Straight Lines [id:da33873959681829824] 
\draw  [dash pattern={on 4.5pt off 4.5pt}]  (108,138) -- (300.17,138) ;
%Straight Lines [id:da9337172607454087] 
\draw  [dash pattern={on 4.5pt off 4.5pt}]  (108,176) -- (300.17,176) ;
%Shape: Rectangle [id:dp2973768399788532] 
\draw  [draw opacity=0][fill={rgb, 255:red, 80; green, 227; blue, 194 }  ,fill opacity=0.1 ] (108,138) -- (300.17,138) -- (300.17,176) -- (108,176) -- cycle ;
%Straight Lines [id:da7495468823358691] 
\draw [color={rgb, 255:red, 245; green, 166; blue, 35 }  ,draw opacity=1 ]   (298.17,63.27) -- (204.08,138) ;
\draw [shift={(249.09,102.25)}, rotate = 321.54] [fill={rgb, 255:red, 245; green, 166; blue, 35 }  ,fill opacity=1 ][line width=0.08]  [draw opacity=0] (7.14,-3.43) -- (0,0) -- (7.14,3.43) -- (4.74,0) -- cycle    ;
%Straight Lines [id:da8449740519670257] 
\draw    (204.08,75.27) -- (204.08,138) ;
\draw [shift={(204.08,72.27)}, rotate = 90] [fill={rgb, 255:red, 0; green, 0; blue, 0 }  ][line width=0.08]  [draw opacity=0] (7.14,-3.43) -- (0,0) -- (7.14,3.43) -- (4.74,0) -- cycle    ;
%Straight Lines [id:da6387995376325981] 
\draw [color={rgb, 255:red, 245; green, 166; blue, 35 }  ,draw opacity=1 ]   (115.17,45.27) -- (204.08,138) ;
\draw [shift={(156.79,88.68)}, rotate = 46.2] [fill={rgb, 255:red, 245; green, 166; blue, 35 }  ,fill opacity=1 ][line width=0.08]  [draw opacity=0] (7.14,-3.43) -- (0,0) -- (7.14,3.43) -- (4.74,0) -- cycle    ;
%Shape: Arc [id:dp10648236478534168] 
\draw  [draw opacity=0][dash pattern={on 4.5pt off 4.5pt}] (204.52,119.25) .. controls (207.51,118.92) and (210.58,119.18) .. (213.58,120.1) .. controls (216.54,121.01) and (219.19,122.48) .. (221.47,124.39) -- (205.28,147.17) -- cycle ; \draw  [dash pattern={on 4.5pt off 4.5pt}] (204.52,119.25) .. controls (207.51,118.92) and (210.58,119.18) .. (213.58,120.1) .. controls (216.54,121.01) and (219.19,122.48) .. (221.47,124.39) ;  
%Shape: Arc [id:dp6672997664685261] 
\draw  [draw opacity=0][dash pattern={on 4.5pt off 4.5pt}] (187.8,121.42) .. controls (189.87,119.22) and (192.37,117.42) .. (195.25,116.17) .. controls (199.78,114.2) and (204.66,113.85) .. (209.34,114.86) -- (206.53,142.14) -- cycle ; \draw  [dash pattern={on 4.5pt off 4.5pt}] (187.8,121.42) .. controls (189.87,119.22) and (192.37,117.42) .. (195.25,116.17) .. controls (199.78,114.2) and (204.66,113.85) .. (209.34,114.86) ;  
%Straight Lines [id:da8579880812195555] 
\draw [color={rgb, 255:red, 0; green, 0; blue, 0 }  ,draw opacity=1 ] [dash pattern={on 0.84pt off 2.51pt}]  (204.08,138) -- (239.28,174.6) ;
%Straight Lines [id:da6808341183836859] 
\draw    (111,141) -- (111,173) ;
\draw [shift={(111,176)}, rotate = 270] [fill={rgb, 255:red, 0; green, 0; blue, 0 }  ][line width=0.08]  [draw opacity=0] (5.36,-2.57) -- (0,0) -- (5.36,2.57) -- (3.56,0) -- cycle    ;
\draw [shift={(111,138)}, rotate = 90] [fill={rgb, 255:red, 0; green, 0; blue, 0 }  ][line width=0.08]  [draw opacity=0] (5.36,-2.57) -- (0,0) -- (5.36,2.57) -- (3.56,0) -- cycle    ;

% Text Node
\draw (204.08,69.27) node [anchor=south] [inner sep=0.75pt]    {$n$};
% Text Node
\draw (300.17,63.27) node [anchor=west] [inner sep=0.75pt]    {$s'$};
% Text Node
\draw (113.17,45.27) node [anchor=east] [inner sep=0.75pt]    {$s$};
% Text Node
\draw (213.63,108.27) node [anchor=west] [inner sep=0.75pt]    {$\theta '$};
% Text Node
\draw (186.63,101.6) node [anchor=west] [inner sep=0.75pt]    {$\theta $};
% Text Node
\draw (131.28,114.38) node [anchor=north west][inner sep=0.75pt]    {$L\neq 0$};
% Text Node
\draw (239.28,177.6) node [anchor=north] [inner sep=0.75pt]    {$L=0$};
% Text Node
\draw (109,157) node [anchor=east] [inner sep=0.75pt]    {$l$};


\end{tikzpicture}

    \caption{The boundary modeled as a region (the green rectangle) where scattering events are frequent}
    \label{fig:boundary-scattering}
\end{figure}

\section{Surface rendering}

If we are not modeling things like clouds, 
and are interested in, say, the image of a cup or an apple, 
the bulk part of the scattering equation is easy, 
since \ul{everything is assumed to be isotropic and uniform in the bulk};
the main problem then becomes how to treat the boundaries appropriately.

We can think the boundary of an object as a region where 
scattering is particularly strong, 
as in \prettyref{fig:boundary-scattering}.
On one side of the region, 
the radiance takes its value at the edge of the bulk, 
and on the other side of the region, 
the radiance decreases to zero. 
Suppose the thickness of the region is $l$.

Now we extend the $\vu*{s}$ line shown in \prettyref{fig:boundary-scattering}
opposite its direction, until we reach the $L=0$ side of the scattering region.
We call this path $C$,
and the length of this path is $l / \cos \theta$.
We integrate the scattering equation \eqref{eq:se-final} along this path, 
and get 
\begin{equation}
    L(\vb*{r}, \vu*{s}) = - \mu_{\text{t}} \int_C \dd{l} L(\vb*{r}', \vu*{s})
    + \mu_{\text{s}} \int \dd{\Omega'} P(\vu*{s}, \vu*{s}') \int_C \dd{l} L(\vb*{r}', \vu*{s}') .
    \label{eq:re-1}
\end{equation}
$L(\vb*{r}', \vu*{s})$ is a linear functional of 
the radiance distribution around the surface. 
(It has to be linear, since the scattering equation is a linear equation.)
In the simplest case, the reflected light on one point ($L(\vb*{r}, \vu*{s})$)
mainly comes from incoming light at the same point; 
thus $L(\vb*{r}', \vu*{s})$ and $L(\vb*{r}', \vu*{s}')$ on path $C$ 
can be rewritten as some kind of damping function of $\abs*{\vb*{r} - \vb*{r}'}$ 
times $L(\vb*{r}, \vu*{s})$ and $L(\vb*{r}, \vu*{s}')$, 
and we have 
\[
    \mu_{\text{s}} \int_C \dd{l} L(\vb*{r}', \vu*{s}') 
    = g(\vu*{s}, \vu*{s}') \frac{\mu_{\text{s}} l}{\cos \theta} L(\vb*{r}, \vu*{s}),
\]
where $g(\vu*{s}, \vu*{s}')$ comes from 
the properties of the scattering layer, 
and from time reversal symmetry we can assume that it's symmetric, 
since the phase function is symmetric. 
This means 
\begin{equation}
    \begin{aligned}
        \mu_{\text{s}} \int \dd{\Omega'} P(\vu*{s}, \vu*{s}') \int_C \dd{l} L(\vb*{r}', \vu*{s}')
        &= \mu_{\text{s}} l \int \dd{\Omega'} P(\vu*{s}, \vu*{s}') 
        g(\vu*{s}, \vu*{s}') \frac{1}{\cos \theta} L(\vb*{r}, \vu*{s}) \\
        &= \mu_{\text{s}} l \int \cos \theta' \dd{\Omega'} 
        \underbrace{
            P(\vu*{s}, \vu*{s}') 
            g(\vu*{s}, \vu*{s}') \frac{1}{\cos \theta \cos \theta'}
        }_{\text{symmetric}} L(\vb*{r}, \vu*{s})  .
    \end{aligned}
\end{equation}
Similarly, we have 
\begin{equation}
    \begin{aligned}
        \mu_{\text{t}} \int_C \dd{l} L(\vb*{r}', \vu*{s})
        &= \mu_{\text{t}} g(\vu*{s}, \vu*{s}) \frac{l}{\cos \theta} L(\vb*{r}, \vu*{s}) \\
        &= \mu_{\text{t}} l \int \cos \theta' \dd{\Omega'} 
        \underbrace{
            \delta(\vu*{s}, \vu*{s}') \frac{1}{\cos \theta \cos \theta'}
        }_{\text{symmetric}} L(\vb*{r}, \vu*{s}'),
    \end{aligned}
\end{equation}
where 
\begin{equation}
    \delta(\vu*{s}, \vu*{s}') = \frac{1}{\sin \theta} \delta(\theta - \theta') \delta(\varphi - \varphi'). 
\end{equation}
So finally, from \eqref{eq:re-1} we get 
\begin{equation}
    L(\vb*{r}, \vu*{s}) = \int f(\vu*{s}, \vu*{s}') L(\vb*{r}, \vu*{s}') \cos \theta' \dd{\Omega'}, 
    \quad f(\vu*{s}, \vu*{s}') = f(\vu*{s}', \vu*{s}).
    \label{eq:re-2}
\end{equation}
We find the exact theory (under the aforementioned premises) of light propagation 
in this case turns out to be \eqref{eq:re-2},
which is known as the \concept{rendering equation}. 
In the rendering equation we have the symmetric function $f(\vu*{s}, \vu*{s}')$,
called the \concept{bidirectional reflectance distribution function (BRDF)}. 
We can always readjust the BRDF so that it is only non-zero 
when $\vu*{s} \cdot \vu*{n}$ and $\vu*{s}' \cdot \vu*{n}$
have different signs. 
This can be seen by noticing that an outgoing (i.e. $\vu*{s} \cdot \vu*{n} > 0$) radiance 
always comes from an ingoing radiance, 
and therefore its effect can be alternatively attributed to the latter. 
This means we can rewrite \eqref{eq:re-2} to
\begin{equation}
    L(\vb*{r}, \vu*{s}) = \int_{\vu*{s}' \cdot \vu*{n} < 0} 
    f(\vu*{s}, \vu*{s}') L(\vb*{r}, \vu*{s}') \cos \theta' \dd{\Omega'}, \quad 
    \vu*{s} \cdot \vu*{n} > 0,
    \label{eq:re-final}
\end{equation}
with the BRDF defined appropriately.
The photon number conservation condition (essentially the energy conservation condition, 
since we don't have any frequency changing mechanism) is 
\begin{equation}
    \int f(\vu*{s}, \vu*{s}') \cos \theta' \dd{\Omega'}  \leq 1.
\end{equation}
This is an inequality, because absorption is also included in the BRDF.

\eqref{eq:re-final} can be derived in a more straightforward way 
from energy conservation, 
where we define 
\begin{equation}
    \dd{L(\vb*{r}, \vu*{s})} = L(\vb*{r}, \vu*{s}') \cos \theta' f(\vu*{s}, \vu*{s}') \dd{\Omega'},
\end{equation}
where the $\cos \theta'$ factor can be obtained intuitively 
by considering the fact that if a beam of light 
is incident on the material with an angle $\theta'$, 
the effective cross section of the beam is reduced by a factor of $\cos \theta'$.

\end{document}