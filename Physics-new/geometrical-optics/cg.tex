\documentclass[hyperref, a4paper]{article}

\usepackage{geometry}
\usepackage{float}
\usepackage{titling}
\usepackage{titlesec}
% No longer needed, since we will use enumitem package
% \usepackage{paralist}
\usepackage{enumitem}
\usepackage{footnote}
\usepackage{soulutf8}
\usepackage{enumerate}
\usepackage{amsmath, amssymb, amsthm}
\usepackage{mathtools}
\usepackage{bbm}
\usepackage{graphicx}
\usepackage{subfigure}
\usepackage{physics}
\usepackage{tensor}
\usepackage{siunitx}
\usepackage{booktabs}
\usepackage[version=4]{mhchem}
\usepackage{tikz}
\usepackage{xcolor}
\usepackage{listings}
\usepackage{autobreak}
\usepackage[ruled, vlined, linesnumbered]{algorithm2e}
\usepackage[backend=bibtex,doi=false,isbn=false,url=false]{biblatex}
\addbibresource{gw.bib}
\usepackage[colorlinks,unicode]{hyperref} % , linkcolor=black, anchorcolor=black, citecolor=black, urlcolor=black, filecolor=black
\usepackage[most]{tcolorbox}
\usepackage{prettyref}

% Page style
\geometry{left=3.18cm,right=3.18cm,top=2.54cm,bottom=2.54cm}
\titlespacing{\paragraph}{0pt}{1pt}{10pt}[20pt]
\setlength{\droptitle}{-5em}

% More compact lists 
% \setlist[itemize]{itemindent=17pt, leftmargin=1pt}

% Math operators
\DeclareMathOperator{\timeorder}{T}
\DeclareMathOperator{\diag}{diag}
\DeclareMathOperator{\legpoly}{P}
\DeclareMathOperator{\primevalue}{P}
\DeclareMathOperator{\sgn}{sgn}
\newcommand*{\ii}{\mathrm{i}}
\newcommand*{\ee}{\mathrm{e}}
\newcommand*{\const}{\mathrm{const}}
\newcommand*{\suchthat}{\quad \text{s.t.} \quad}
\newcommand*{\argmin}{\arg\min}
\newcommand*{\argmax}{\arg\max}
\newcommand*{\normalorder}[1]{: #1 :}
\newcommand*{\pair}[1]{\langle #1 \rangle}
\newcommand*{\fd}[1]{\mathcal{D} #1}
\DeclareMathOperator{\bigO}{\mathcal{O}}
\DeclareMathOperator{\id}{id}

% TikZ setting
\usetikzlibrary{decorations.text}
\usetikzlibrary{arrows,shapes,positioning}
\usetikzlibrary{arrows.meta}
\usetikzlibrary{decorations.markings}
\tikzstyle arrowstyle=[scale=1]
\tikzstyle directed=[postaction={decorate,decoration={markings,
    mark=at position .5 with {\arrow[arrowstyle]{stealth}}}}]
\tikzstyle ray=[directed, thick]
\tikzstyle dot=[anchor=base,fill,circle,inner sep=1pt]

% Algorithm setting
% Julia-style code
\SetKwIF{If}{ElseIf}{Else}{if}{}{elseif}{else}{end}
\SetKwFor{For}{for}{}{end}
\SetKwFor{While}{while}{}{end}
\SetKwProg{Function}{function}{}{end}
\SetArgSty{textnormal}

\newcommand*{\concept}[1]{{\textbf{#1}}}

\DeclareMathOperator{\im}{im}

% Embedded codes
\lstset{basicstyle=\ttfamily,
  showstringspaces=false,
  commentstyle=\color{gray},
  keywordstyle=\color{blue}
}

\newcommand{\soliddoc}{\href{../solid/solid.pdf}{this note}}
\newcommand{\lastlec}{\href{./2022-3-15.pdf}{the last lecture}}

% Reference formatter
\newrefformat{fig}{Fig.~\ref{#1}}

% Color boxes
\tcbuselibrary{skins, breakable, theorems}

\newtcbtheorem[number within=chapter]{infobox}{Box}{
    enhanced,
    boxrule=0pt,
    colback=blue!5,
    colframe=blue!5,
    coltitle=blue!50,
    borderline west={4pt}{0pt}{blue!65},
    sharp corners,
    fonttitle=\bfseries, 
    breakable,
    before upper={\parindent15pt\noindent}}{box}

% Disable unsupported commands in bookmark titles 
\pdfstringdefDisableCommands{%
  \def\\{}%
  \def\texttt#1{<#1>}%
  \def\mathbb#1{#1}%
}
\pdfstringdefDisableCommands{\def\eqref#1{(\ref{#1})}}

\makeatletter
\pdfstringdefDisableCommands{\let\HyPsd@CatcodeWarning\@gobble}
\makeatother

\title{Rendering}
\author{Jinyuan Wu}

\begin{document}

\maketitle

\section{What a renderer does}

A completely accurate surface renderer does the following:
\begin{itemize}
    \item It takes the following input: 
        \begin{itemize}
            \item The positions and shapes of boundaries in the system;
            \item BSDF or BSSDF of each boundaries;
            \item Positions and directional distributions of light sources.
        \end{itemize}
    \item It aims to calculate the radiance $L(\vb*{r}, \omega)$
        for given $\vb*{r}$ and $\omega$;%
        \footnote{
            Here I use the notation in computer graphics: 
            $\omega$ is the direction of the wave vector, 
            and $\dd{\omega}$ is the solid angle element 
            surrounding the direction $\omega$;
            we define 
            \begin{equation}
                \delta(\omega - \omega') = \frac{1}{\sin \theta} \delta(\theta - \theta') \delta(\varphi - \varphi').
            \end{equation}
        }
        at boundaries that allow transmission 
        we need to distinguish between $L(\vb*{r}, \omega)$
        on the two sides of the boundary;
        we can use $L_{\text{i}}$ and $L_{\text{o}}$ 
        to refer to them.

        Since the radiance is conserved along a ray (Pharr et al. chap. 14),
        we can also use $L(\vb*{r} \to \vb*{r}')$ to refer to the radiance, 
        if we are able to find some $\vb*{r}$ and $\vb*{r}'$ on boundaries,%
        \footnote{
            Often we need to deal with lights from infinity,
            but at least in theory we can model ``infinity''
            by a large light source at a distance.
        }
        and $\widehat{\vb*{r}' - \vb*{r}} = \omega$.
        With this notation, the i and o subscripts are no longer necessary.
    \item It enumerates all possible light paths that ends with a specific point $\vb*{r}$.
        For each path, 
        \begin{itemize}
            \item If it comes directly from a light source $\vb*{r}'$, 
                then it contributes $L_{\text{e}}(\vb*{r}' \to \vb*{r})$ to $L(\vb*{r}, \omega)$,
                where $L_{\text{e}}$ is the emission radiance of the light source.
            \item If reflection or transmission happens at $\vb*{r}'$, 
                the contribution of $\vb*{r}'' \to \vb*{r}' \to \vb*{r}$ is 
                \begin{equation}
                    \begin{aligned}
                        \dd{L}(\vb*{r}' \to \vb*{r})
                        &= f(\vb*{r}', \widehat{\vb*{r}' - \vb*{r}''}, \widehat{\vb*{r} - \vb*{r}'})
                        \abs*{\cos \theta } L(\vb*{r}'' \to \vb*{r}') \dd{\omega''} \\
                        &= f(\vb*{r}', \widehat{\vb*{r}' - \vb*{r}''}, \widehat{\vb*{r} - \vb*{r}'})
                        \abs*{\cos \theta } L(\vb*{r}'' \to \vb*{r}')
                        \frac{\dd{A}''}{\abs*{\vb*{r}'' - \vb*{r}'}^2},
                    \end{aligned}
                    \label{eq:three-point-re}
                \end{equation} 
                where 
                \begin{equation}
                    \cos \theta = \vu*{n}' \cdot \widehat{\vb*{r}' - \vb*{r}''},
                \end{equation}
                and $\vu*{n}'$ is the normal vector at $\vb*{r}'$.
                
                Of course, now the problem is how to find $L(\vb*{r}'' \to \vb*{r}')$;
                if $\vb*{r}''$ is on a light source, 
                \eqref{eq:three-point-re} is already evaluated, 
                and if not, 
                we need to evaluate $L(\vb*{r}'' \to \vb*{r}')$
                on the LHS of \eqref{eq:three-point-re}.
        \end{itemize}
    \item $L(\vb*{r}, \omega)$ is then used together with a camera model 
        to create a ``photon'' of the system.
\end{itemize}
Since we need to sum over all rays determined by geometrical optics, 
this is known as ray tracing.

\section{Camera model}

\end{document}