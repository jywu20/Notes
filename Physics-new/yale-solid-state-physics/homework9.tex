\documentclass[hyperref, a4paper]{article}

\usepackage{geometry}
\usepackage{titling}
\usepackage{titlesec}
% No longer needed, since we will use enumitem package
% \usepackage{paralist}
\usepackage{enumitem}
\usepackage{footnote}
\usepackage{enumerate}
\usepackage{amsmath, amssymb, amsthm}
\usepackage{mathtools}
\usepackage{bbm}
\usepackage{cite}
\usepackage{graphicx}
\usepackage{subfigure}
\usepackage{physics}
\usepackage{tensor}
\usepackage{siunitx}
\usepackage[version=4]{mhchem}
\usepackage{tikz}
\usepackage{xcolor}
\usepackage{listings}
\usepackage{autobreak}
\usepackage[ruled, vlined, linesnumbered]{algorithm2e}
\usepackage{nameref,zref-xr}
\zxrsetup{toltxlabel}
\usepackage[colorlinks,unicode]{hyperref} % , linkcolor=black, anchorcolor=black, citecolor=black, urlcolor=black, filecolor=black
\usepackage[most]{tcolorbox}
\usepackage{prettyref}

% Page style
\geometry{left=3.18cm,right=3.18cm,top=2.54cm,bottom=2.54cm}
\titlespacing{\paragraph}{0pt}{1pt}{10pt}[20pt]
\setlength{\droptitle}{-5em}
%\preauthor{\vspace{-10pt}\begin{center}}
%\postauthor{\par\end{center}}

% More compact lists 
\setlist[itemize]{
    itemindent=17pt, 
    leftmargin=1pt,
    listparindent=\parindent,
    parsep=0pt,
}

% Math operators
\DeclareMathOperator{\timeorder}{\mathcal{T}}
\DeclareMathOperator{\diag}{diag}
\DeclareMathOperator{\legpoly}{P}
\DeclareMathOperator{\primevalue}{P}
\DeclareMathOperator{\sgn}{sgn}
\newcommand*{\ii}{\mathrm{i}}
\newcommand*{\ee}{\mathrm{e}}
\newcommand*{\const}{\mathrm{const}}
\newcommand*{\suchthat}{\quad \text{s.t.} \quad}
\newcommand*{\argmin}{\arg\min}
\newcommand*{\argmax}{\arg\max}
\newcommand*{\normalorder}[1]{: #1 :}
\newcommand*{\pair}[1]{\langle #1 \rangle}
\newcommand*{\fd}[1]{\mathcal{D} #1}
\DeclareMathOperator{\bigO}{\mathcal{O}}

% TikZ setting
\usetikzlibrary{arrows,shapes,positioning}
\usetikzlibrary{arrows.meta}
\usetikzlibrary{decorations.markings}
\tikzstyle arrowstyle=[scale=1]
\tikzstyle directed=[postaction={decorate,decoration={markings,
    mark=at position .5 with {\arrow[arrowstyle]{stealth}}}}]
\tikzstyle ray=[directed, thick]
\tikzstyle dot=[anchor=base,fill,circle,inner sep=1pt]

% Algorithm setting
% Julia-style code
\SetKwIF{If}{ElseIf}{Else}{if}{}{elseif}{else}{end}
\SetKwFor{For}{for}{}{end}
\SetKwFor{While}{while}{}{end}
\SetKwProg{Function}{function}{}{end}
\SetArgSty{textnormal}

\newcommand*{\concept}[1]{{\textbf{#1}}}

% Embedded codes
\lstset{basicstyle=\ttfamily,
  showstringspaces=false,
  commentstyle=\color{gray},
  keywordstyle=\color{blue}
}

% Reference formatting
\newrefformat{fig}{Figure~\ref{#1}}

% Color boxes
\tcbuselibrary{skins, breakable, theorems}
\newtcbtheorem[number within=section]{warning}{Warning}%
  {colback=orange!5,colframe=orange!65,fonttitle=\bfseries, breakable}{warn}
\newtcbtheorem[number within=section]{note}{Note}%
  {colback=green!5,colframe=green!65,fonttitle=\bfseries, breakable}{note}
\newtcbtheorem[number within=section]{info}{Info}%
  {colback=blue!5,colframe=blue!65,fonttitle=\bfseries, breakable}{info}

\newenvironment{shelldisplay}{\begin{lstlisting}}{\end{lstlisting}}

\newcommand{\address}[1]{\href{#1}{\url{#1}}}

\title{Homework 9}
\author{Jinyuan Wu}

\begin{document}

\maketitle

\paragraph{Problem 1} Assume we have a material for which the conduction band energy dispersion in the $k_x$ direction has the following form:
$$
E=\frac{\hbar^2 k_x^2}{2 m_0^*}-C \hbar^4 k_x^4
$$
(a) If this relationship is correct for $k_x$ all the way up to the edge of the Brillouin zone at $k_x=\pi / a$, what is the value of the constant $C$ ?
(b) At what value of $k_x$ is the electron velocity have maximum magnitude?

\paragraph{Solution} \begin{itemize}
\item[(a)] At the boundary of the first Brillouin zone we have 
\begin{equation}
0 = \pdv{E}{k_x} = \frac{\hbar^2 k_x}{m_0^*} - 4 C \hbar^4 k_x^3,
\end{equation}
\begin{equation}
    C = \frac{1}{4 m_0^* \hbar^2 k_x^2} = \frac{a^2}{4 m_0^* \hbar^2 \pi^2}.
\end{equation}
\item[(b)] The velocity is given by 
\begin{equation}
    v_x = \pdv{E}{k_x} = \frac{\hbar k_x}{m_0^*} - \frac{\hbar^2 a^2}{\pi^2 m_0^*} k_x^3,
\end{equation}
and its maximum is reached when 
\begin{equation}
    0 = \pdv{v_x}{k_x} = \frac{\hbar^2}{m_0^*} - \frac{\hbar^2 a^2}{\pi^2 m_0^*} 3 k_x^2 \Rightarrow
    k_x = \pm \frac{1}{\sqrt{3}} \frac{\pi}{a}.
\end{equation}

\end{itemize}

\paragraph{Problem 2} A tetragonal metal has a first Brillouin zone (BZ) size of $2 \times 10^{10} \mathrm{~m}^{-1}$ along the $k_x$ direction (this is the size from one side of the $\mathrm{BZ}$ to the other side along $\mathrm{x}$ ). An open electron orbit connects the opposite faces of the BZ along the $k_x$ direction.
a) If a magnetic field of $0.1$ Tesla is applied to the metal normal to the plane of the open orbit, what is the (approximate) period of the electron motion in $\mathrm{k}$ space? Assume the electron velocity is the Fermi velocity $v_F=10^6 \mathrm{~m} / \mathrm{s}$.
b) Describe the motion of an electron moving along this open orbit in real space.

\paragraph{Solution} \begin{itemize}
\item[(a)] Since the orbital is open,
the period is just the time for the electron to move from one side to the other 
in the first Brillouin zone.
From 
\begin{equation}
    \hbar \dot{\vb*{k}} = - e \vb*{v} \times \vb*{B},
    \label{eq:k-v-eq}
\end{equation}
we get 
\begin{equation}
    t = \frac{\hbar}{e v B} \Delta k = \SI{1.3e-10}{s}.
\end{equation}
\item[(b)] Since $\vb*{k}$ is restricted in the $x$ direction,
$\vb*{v}$ is restricted in the $y$ direction because of \eqref{eq:k-v-eq}.
So in the real space, the particle is in uniform motion in the $y$ direction.
\end{itemize}

\paragraph{Problem 3} Consider a 2-dimensional metal in the $x-y$ plane with a rectangular primitive cell with sizes $a=\SI{2}{\angstrom}$ and $b=\SI{4}{\angstrom}$. The metal has one conduction electron per primitive unit cell.
a) Draw the first Brillouin zone of this metal and give its numerical dimensions.
b) What is the radius of the free electron circle (with units)? Draw this circle to scale on your drawing of the BZ from (a).
c) Assuming the nearly free electron approximation is valid for this metal, make a separate drawing of the BZ and the resulting Fermi surface(s) labeling each Fermi surface by the Brillouin zone it comes from (first, second, etc.)
d) Describe the orbits resulting from part (c) when a magnetic field is applied perpendicular to the plane of the metal and the Fermi surface along the $+z$ direction. In addition to describing the character of each orbit, show with arrows the direction of electron motion in $\vb*{k}$-space along each Fermi surface.

\paragraph{Solution} \begin{itemize}
\item[(a)] See \prettyref{fig:problem-3}(a).The dimensions are $\pi / a = \SI{1.57}{\angstrom^{-1}}$, 
$\pi / b = \SI{0.79}{\angstrom^{-1}}$.
\item[(b)] We have 
\[
    \pi k_{\text{F}}^2 \cdot \frac{S}{(2\pi)^2} \cdot 2 = N,
\]
\begin{equation}
    k_{\text{F}} = \sqrt{\frac{2\pi}{ab}} = \SI{0.89}{\angstrom^{-1}}.
\end{equation}
This is plotted in \prettyref{fig:problem-3}(b).
\item[(c)] See \prettyref{fig:problem-3}(c). 
The requirements that the boundary of the first Brillouin zone is perpendicular to Fermi surfaces
and the fact that after band splitting,
the low band goes down and the high band goes up 
means we have two Fermi surfaces that are connected to the boundary of the first Brillouin zone,
and a Fermi pocket (or two pockets, depending on how you count it) 
between the aforementioned Fermi surfaces. 
The two Fermi surfaces connected to the boundary are from 1BZ;
the Fermi pockets, on the other hand, are from 2BZ, 
because they are formed by moving the original free-electron Fermi surface outside 1BZ 
back into 1BZ.
\item[(d)] After a magnetic field is applied,
electrons on the Fermi surfaces connected to the boundary 
are in an open orbit,
endlessly moving in the $+y$ direction or the $-y$ direction.
Here on the right Fermi surface, electrons are moving in the $+y$ direction,
because the effective mass is positive here and we have 
$\dot{\vb*{k}} \propto - \vb*{v} \times \vb*{B} \propto - \vb*{k} \times \vb*{B}$,
and since $\vb*{k}$ is toward the right side, 
$\vb*{k} \times \vb*{B}$ is in the $- y$ direction, 
and so $\dot{\vb*{k}}$ is in the $y$ direction.

For Fermi pockets, the electrons are on a closed orbit.
since they are on the high band after band splitting,
the effective mass is negative,
and we have $\dot{\vb*{k}} \propto \vb*{k} \times \vb*{B}$,
so we get the directions plotted in \prettyref{fig:problem-3}(d).
Note that we should only use $\dot{\vb*{k}} \propto \vb*{k} \times \vb*{B}$
in the first Brillouin zone,
or otherwise we get conflicting results:
On the upper part of the upper Fermi pocket in \prettyref{fig:problem-3}(d),
the electrons should move leftwards,
because on the upper part of the lower Fermi pocket the electrons move leftwards,
but since the upper part of the upper Fermi pocket 
is close to the lower part of the upper Fermi pocket,
on which electrons move rightwards,
it seems on the upper part of the upper Fermi pocket, 
the electrons should move rightwards,
and we get a contradiction.
The second way to label the movement direction is not correct 
because it's outside the first Brillouin zone.

\end{itemize}

\begin{figure}
    \centering
    \begin{tikzpicture}[x=0.75pt,y=0.75pt,yscale=-1,xscale=1]
    %uncomment if require: \path (0,362); %set diagram left start at 0, and has height of 362
    
    %Shape: Rectangle [id:dp969921572562275] 
    \draw   (201,74) -- (201,124) -- (101,124) -- (101,74) -- cycle ;
    %Straight Lines [id:da7281093812667929] 
    \draw    (101,74) ;
    \draw [shift={(101,74)}, rotate = 0] [color={rgb, 255:red, 0; green, 0; blue, 0 }  ][fill={rgb, 255:red, 0; green, 0; blue, 0 }  ][line width=0.75]      (0, 0) circle [x radius= 3.35, y radius= 3.35]   ;
    %Straight Lines [id:da2750232292895938] 
    \draw    (101,124) ;
    \draw [shift={(101,124)}, rotate = 0] [color={rgb, 255:red, 0; green, 0; blue, 0 }  ][fill={rgb, 255:red, 0; green, 0; blue, 0 }  ][line width=0.75]      (0, 0) circle [x radius= 3.35, y radius= 3.35]   ;
    %Straight Lines [id:da2814393689984458] 
    \draw    (201,74) ;
    \draw [shift={(201,74)}, rotate = 0] [color={rgb, 255:red, 0; green, 0; blue, 0 }  ][fill={rgb, 255:red, 0; green, 0; blue, 0 }  ][line width=0.75]      (0, 0) circle [x radius= 3.35, y radius= 3.35]   ;
    %Straight Lines [id:da48518604894585726] 
    \draw    (201,124) ;
    \draw [shift={(201,124)}, rotate = 0] [color={rgb, 255:red, 0; green, 0; blue, 0 }  ][fill={rgb, 255:red, 0; green, 0; blue, 0 }  ][line width=0.75]      (0, 0) circle [x radius= 3.35, y radius= 3.35]   ;
    %Straight Lines [id:da19234315310795913] 
    \draw  [dash pattern={on 0.84pt off 2.51pt}]  (50,99) -- (151,99) ;
    %Straight Lines [id:da5344566028165196] 
    \draw  [dash pattern={on 0.84pt off 2.51pt}]  (50,149) -- (151,149) ;
    %Straight Lines [id:da9360343333608041] 
    \draw  [dash pattern={on 0.84pt off 2.51pt}]  (151,99) -- (151,149) ;
    %Straight Lines [id:da9992594836410511] 
    \draw  [dash pattern={on 0.84pt off 2.51pt}]  (50,99) -- (50,149) ;
    %Shape: Rectangle [id:dp6834723189004037] 
    \draw  [draw opacity=0][fill={rgb, 255:red, 80; green, 227; blue, 194 }  ,fill opacity=0.2 ] (151,99) -- (151,149) -- (51,149) -- (51,99) -- cycle ;
    %Shape: Circle [id:dp0868642087544036] 
    \draw  [color={rgb, 255:red, 74; green, 144; blue, 226 }  ,draw opacity=1 ][fill={rgb, 255:red, 74; green, 144; blue, 226 }  ,fill opacity=0.1 ] (72.63,227) .. controls (72.63,211.33) and (85.33,198.63) .. (101,198.63) .. controls (116.67,198.63) and (129.38,211.33) .. (129.38,227) .. controls (129.38,242.67) and (116.67,255.38) .. (101,255.38) .. controls (85.33,255.38) and (72.63,242.67) .. (72.63,227) -- cycle ;
    %Straight Lines [id:da6839425426180443] 
    \draw  [dash pattern={on 0.84pt off 2.51pt}]  (50,202) -- (151,202) ;
    %Straight Lines [id:da8601879442340727] 
    \draw  [dash pattern={on 0.84pt off 2.51pt}]  (50,252) -- (151,252) ;
    %Straight Lines [id:da09814028918501272] 
    \draw  [dash pattern={on 0.84pt off 2.51pt}]  (151,202) -- (151,252) ;
    %Straight Lines [id:da9219793273010373] 
    \draw  [dash pattern={on 0.84pt off 2.51pt}]  (50,202) -- (50,252) ;
    %Shape: Rectangle [id:dp9228451369099275] 
    \draw  [draw opacity=0][fill={rgb, 255:red, 80; green, 227; blue, 194 }  ,fill opacity=0.2 ] (151,202) -- (151,252) -- (51,252) -- (51,202) -- cycle ;
    %Shape: Ellipse [id:dp29214516692296866] 
    \draw  [color={rgb, 255:red, 74; green, 144; blue, 226 }  ,draw opacity=1 ][dash pattern={on 0.84pt off 2.51pt}] (331.7,98.63) .. controls (331.7,67.9) and (356.6,43) .. (387.33,43) .. controls (418.05,43) and (442.95,67.9) .. (442.95,98.63) .. controls (442.95,129.35) and (418.05,154.25) .. (387.33,154.25) .. controls (356.6,154.25) and (331.7,129.35) .. (331.7,98.63) -- cycle ;
    %Straight Lines [id:da18346972138327478] 
    \draw  [dash pattern={on 0.84pt off 2.51pt}]  (287,49.26) -- (487.33,49.26) ;
    %Straight Lines [id:da5797454286106265] 
    \draw  [dash pattern={on 0.84pt off 2.51pt}]  (287,148.43) -- (487.33,148.43) ;
    %Straight Lines [id:da09162183753727016] 
    \draw  [dash pattern={on 0.84pt off 2.51pt}]  (485.35,49.26) -- (485.35,148.43) ;
    %Straight Lines [id:da6198897499997191] 
    \draw  [dash pattern={on 0.84pt off 2.51pt}]  (287,49.26) -- (287,148.43) ;
    %Shape: Ellipse [id:dp225162492918219] 
    \draw  [color={rgb, 255:red, 144; green, 19; blue, 254 }  ,draw opacity=1 ][fill={rgb, 255:red, 144; green, 19; blue, 254 }  ,fill opacity=0.1 ] (362.92,49.57) .. controls (362.92,45.84) and (373.5,42.81) .. (386.55,42.81) .. controls (399.6,42.81) and (410.17,45.84) .. (410.17,49.57) .. controls (410.17,53.31) and (399.6,56.33) .. (386.55,56.33) .. controls (373.5,56.33) and (362.92,53.31) .. (362.92,49.57) -- cycle ;
    %Shape: Polygon Curved [id:ds7065769403732807] 
    \draw  [draw opacity=0][fill={rgb, 255:red, 74; green, 144; blue, 226 }  ,fill opacity=0.1 ] (355.33,48.92) .. controls (378.83,48.75) and (392.27,48.83) .. (401.05,48.91) .. controls (409.83,48.98) and (413.96,49.04) .. (418.83,48.83) .. controls (418.83,57.67) and (444.39,70.99) .. (444.18,99.5) .. controls (444.64,128.99) and (418.83,139.67) .. (418.83,147.83) .. controls (365.58,148.5) and (418.08,148) .. (355.83,148.25) .. controls (355.83,139.17) and (330.08,126.67) .. (329.99,99.58) .. controls (329.83,68.92) and (355.33,57.67) .. (355.33,48.92) -- cycle ;
    %Curve Lines [id:da6480281352502946] 
    \draw [color={rgb, 255:red, 74; green, 144; blue, 226 }  ,draw opacity=1 ]   (355.83,148.25) .. controls (356,138.67) and (330.98,127) .. (329.99,99.58) .. controls (329,72.17) and (352.5,58.42) .. (355.33,48.92) ;
    %Curve Lines [id:da4080682366333286] 
    \draw [color={rgb, 255:red, 74; green, 144; blue, 226 }  ,draw opacity=1 ]   (418.33,148.17) .. controls (418.17,138.58) and (443.19,126.92) .. (444.18,99.5) .. controls (445.17,72.08) and (421.67,58.33) .. (418.83,48.83) ;
    
    %Shape: Ellipse [id:dp930430868897534] 
    \draw  [color={rgb, 255:red, 144; green, 19; blue, 254 }  ,draw opacity=1 ][fill={rgb, 255:red, 144; green, 19; blue, 254 }  ,fill opacity=0.1 ] (363.54,147.26) .. controls (363.54,143.52) and (374.12,140.49) .. (387.17,140.49) .. controls (400.22,140.49) and (410.8,143.52) .. (410.8,147.26) .. controls (410.8,150.99) and (400.22,154.02) .. (387.17,154.02) .. controls (374.12,154.02) and (363.54,150.99) .. (363.54,147.26) -- cycle ;
    %Straight Lines [id:da20127033717253107] 
    \draw  [dash pattern={on 0.84pt off 2.51pt}]  (287,219.59) -- (487.33,219.59) ;
    %Straight Lines [id:da7733642137813934] 
    \draw  [dash pattern={on 0.84pt off 2.51pt}]  (287,318.76) -- (487.33,318.76) ;
    %Straight Lines [id:da9795385658475235] 
    \draw  [dash pattern={on 0.84pt off 2.51pt}]  (485.35,219.59) -- (485.35,318.76) ;
    %Straight Lines [id:da6434323319988164] 
    \draw  [dash pattern={on 0.84pt off 2.51pt}]  (287,219.59) -- (287,318.76) ;
    %Shape: Ellipse [id:dp43546690840978175] 
    \draw  [color={rgb, 255:red, 144; green, 19; blue, 254 }  ,draw opacity=1 ][fill={rgb, 255:red, 144; green, 19; blue, 254 }  ,fill opacity=0.1 ] (362.92,219.91) .. controls (362.92,216.17) and (373.5,213.14) .. (386.55,213.14) .. controls (399.6,213.14) and (410.17,216.17) .. (410.17,219.91) .. controls (410.17,223.64) and (399.6,226.67) .. (386.55,226.67) .. controls (373.5,226.67) and (362.92,223.64) .. (362.92,219.91) -- cycle ;
    %Shape: Ellipse [id:dp07576067434982403] 
    \draw  [color={rgb, 255:red, 144; green, 19; blue, 254 }  ,draw opacity=1 ][fill={rgb, 255:red, 144; green, 19; blue, 254 }  ,fill opacity=0.1 ] (362.54,319.59) .. controls (362.54,315.86) and (373.12,312.83) .. (386.17,312.83) .. controls (399.22,312.83) and (409.8,315.86) .. (409.8,319.59) .. controls (409.8,323.32) and (399.22,326.35) .. (386.17,326.35) .. controls (373.12,326.35) and (362.54,323.32) .. (362.54,319.59) -- cycle ;
    %Straight Lines [id:da9550791215283965] 
    \draw    (451.33,285.33) -- (451.33,251.89) ;
    \draw [shift={(451.33,249.89)}, rotate = 90] [fill={rgb, 255:red, 0; green, 0; blue, 0 }  ][line width=0.08]  [draw opacity=0] (12,-3) -- (0,0) -- (12,3) -- cycle    ;
    %Straight Lines [id:da13363091354693712] 
    \draw    (321.33,283.33) -- (321.33,249.89) ;
    \draw [shift={(321.33,285.33)}, rotate = 270] [fill={rgb, 255:red, 0; green, 0; blue, 0 }  ][line width=0.08]  [draw opacity=0] (12,-3) -- (0,0) -- (12,3) -- cycle    ;
    %Straight Lines [id:da0021767207481300233] 
    \draw    (368,231.33) -- (405.67,231.33) ;
    \draw [shift={(407.67,231.33)}, rotate = 180] [fill={rgb, 255:red, 0; green, 0; blue, 0 }  ][line width=0.08]  [draw opacity=0] (12,-3) -- (0,0) -- (12,3) -- cycle    ;
    %Straight Lines [id:da6097581516020871] 
    \draw    (403.67,307.33) -- (369,307.33) ;
    \draw [shift={(367,307.33)}, rotate = 360] [fill={rgb, 255:red, 0; green, 0; blue, 0 }  ][line width=0.08]  [draw opacity=0] (12,-3) -- (0,0) -- (12,3) -- cycle    ;
    %Shape: Polygon Curved [id:ds740863845727582] 
    \draw  [draw opacity=0][fill={rgb, 255:red, 74; green, 144; blue, 226 }  ,fill opacity=0.1 ] (356.33,219.92) .. controls (379.83,219.75) and (393.27,219.83) .. (402.05,219.91) .. controls (410.83,219.98) and (414.96,220.04) .. (419.83,219.83) .. controls (419.83,228.67) and (445.39,241.99) .. (445.18,270.5) .. controls (445.64,299.99) and (419.83,310.67) .. (419.83,318.83) .. controls (366.58,319.5) and (419.08,319) .. (356.83,319.25) .. controls (356.83,310.17) and (331.08,297.67) .. (330.99,270.58) .. controls (330.83,239.92) and (356.33,228.67) .. (356.33,219.92) -- cycle ;
    %Curve Lines [id:da8581780493753055] 
    \draw [color={rgb, 255:red, 74; green, 144; blue, 226 }  ,draw opacity=1 ]   (356.83,319.25) .. controls (357,309.67) and (331.98,298) .. (330.99,270.58) .. controls (330,243.17) and (353.5,229.42) .. (356.33,219.92) ;
    %Curve Lines [id:da979676796510633] 
    \draw [color={rgb, 255:red, 74; green, 144; blue, 226 }  ,draw opacity=1 ]   (419.33,319.17) .. controls (419.17,309.58) and (444.19,297.92) .. (445.18,270.5) .. controls (446.17,243.08) and (422.67,229.33) .. (419.83,219.83) ;
    
    
    % Text Node
    \draw (138,28) node [anchor=north west][inner sep=0.75pt]   [align=left] {$\displaystyle \frac{2\pi }{a}$};
    % Text Node
    \draw (215,101) node [anchor=west] [inner sep=0.75pt]   [align=left] {$\displaystyle \frac{2\pi }{b}$};
    % Text Node
    \draw (52,102) node [anchor=north west][inner sep=0.75pt]   [align=left] {1BZ};
    % Text Node
    \draw (101.5,162) node [anchor=north west][inner sep=0.75pt]   [align=left] {(a)};
    % Text Node
    \draw (93,264.38) node [anchor=north west][inner sep=0.75pt]   [align=left] {(b)};
    % Text Node
    \draw (378.67,166.44) node [anchor=north west][inner sep=0.75pt]   [align=left] {(c)};
    % Text Node
    \draw (378.67,340.78) node [anchor=north west][inner sep=0.75pt]   [align=left] {(d)};
    % Text Node
    \draw (299.67,88.78) node [anchor=north west][inner sep=0.75pt]  [color={rgb, 255:red, 74; green, 144; blue, 226 }  ,opacity=1 ] [align=left] {1BZ};
    % Text Node
    \draw (371,23.44) node [anchor=north west][inner sep=0.75pt]  [color={rgb, 255:red, 144; green, 19; blue, 254 }  ,opacity=1 ] [align=left] {2BZ};
    
    
    \end{tikzpicture}
    
    \caption{The first Brillouin zone and Fermi surfaces in Problem 3}
    \label{fig:problem-3}
\end{figure}

\end{document}