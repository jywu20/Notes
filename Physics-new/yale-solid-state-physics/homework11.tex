\documentclass[hyperref, a4paper]{article}

\usepackage{geometry}
\usepackage{titling}
\usepackage{titlesec}
% No longer needed, since we will use enumitem package
% \usepackage{paralist}
\usepackage{enumitem}
\usepackage{footnote}
\usepackage{enumerate}
\usepackage{amsmath, amssymb, amsthm}
\usepackage{mathtools}
\usepackage{bbm}
\usepackage{cite}
\usepackage{graphicx}
\usepackage{subfigure}
\usepackage{physics}
\usepackage{tensor}
\usepackage{siunitx}
\usepackage[version=4]{mhchem}
\usepackage{tikz}
\usepackage{xcolor}
\usepackage{listings}
\usepackage{autobreak}
\usepackage[ruled, vlined, linesnumbered]{algorithm2e}
\usepackage{nameref,zref-xr}
\zxrsetup{toltxlabel}
\usepackage[colorlinks,unicode]{hyperref} % , linkcolor=black, anchorcolor=black, citecolor=black, urlcolor=black, filecolor=black
\usepackage[most]{tcolorbox}
\usepackage{prettyref}

% Page style
\geometry{left=3.18cm,right=3.18cm,top=2.54cm,bottom=2.54cm}
\titlespacing{\paragraph}{0pt}{1pt}{10pt}[20pt]
\setlength{\droptitle}{-5em}
%\preauthor{\vspace{-10pt}\begin{center}}
%\postauthor{\par\end{center}}

% More compact lists 
\setlist[itemize]{
    itemindent=17pt, 
    leftmargin=1pt,
    listparindent=\parindent,
    parsep=0pt,
}

% Math operators
\DeclareMathOperator{\timeorder}{\mathcal{T}}
\DeclareMathOperator{\diag}{diag}
\DeclareMathOperator{\legpoly}{P}
\DeclareMathOperator{\primevalue}{P}
\DeclareMathOperator{\sgn}{sgn}
\newcommand*{\ii}{\mathrm{i}}
\newcommand*{\ee}{\mathrm{e}}
\newcommand*{\const}{\mathrm{const}}
\newcommand*{\suchthat}{\quad \text{s.t.} \quad}
\newcommand*{\argmin}{\arg\min}
\newcommand*{\argmax}{\arg\max}
\newcommand*{\normalorder}[1]{: #1 :}
\newcommand*{\pair}[1]{\langle #1 \rangle}
\newcommand*{\fd}[1]{\mathcal{D} #1}
\DeclareMathOperator{\bigO}{\mathcal{O}}

% TikZ setting
\usetikzlibrary{arrows,shapes,positioning}
\usetikzlibrary{arrows.meta}
\usetikzlibrary{decorations.markings}
\tikzstyle arrowstyle=[scale=1]
\tikzstyle directed=[postaction={decorate,decoration={markings,
    mark=at position .5 with {\arrow[arrowstyle]{stealth}}}}]
\tikzstyle ray=[directed, thick]
\tikzstyle dot=[anchor=base,fill,circle,inner sep=1pt]

% Algorithm setting
% Julia-style code
\SetKwIF{If}{ElseIf}{Else}{if}{}{elseif}{else}{end}
\SetKwFor{For}{for}{}{end}
\SetKwFor{While}{while}{}{end}
\SetKwProg{Function}{function}{}{end}
\SetArgSty{textnormal}

\newcommand*{\concept}[1]{{\textbf{#1}}}

% Embedded codes
\lstset{basicstyle=\ttfamily,
  showstringspaces=false,
  commentstyle=\color{gray},
  keywordstyle=\color{blue}
}

% Reference formatting
\newrefformat{fig}{Figure~\ref{#1}}

% Color boxes
\tcbuselibrary{skins, breakable, theorems}
\newtcbtheorem[number within=section]{warning}{Warning}%
  {colback=orange!5,colframe=orange!65,fonttitle=\bfseries, breakable}{warn}
\newtcbtheorem[number within=section]{note}{Note}%
  {colback=green!5,colframe=green!65,fonttitle=\bfseries, breakable}{note}
\newtcbtheorem[number within=section]{info}{Info}%
  {colback=blue!5,colframe=blue!65,fonttitle=\bfseries, breakable}{info}

\newenvironment{shelldisplay}{\begin{lstlisting}}{\end{lstlisting}}

\newcommand{\address}[1]{\href{#1}{\url{#1}}}

\title{Homework 11}
\author{Jiinyuan Wu}

\begin{document}
    
\maketitle

\paragraph{Problem 1} Consider the band structure of $\operatorname{InP}$ as having only the electron band and the heavy hole band. Use numerical values from Kittel for the material.
a) Where in the band gap is the chemical potential located at $300 \mathrm{~K}$ for an ultrapure sample of InP?
b) Explain briefly the physical reason why the chemical potential has the value you found in part (a) (i.e., why it changes in the direction it does with $\mathrm{T}>0$ compared to $\mathrm{T}=0$ ).

\paragraph{Solution} \begin{itemize}
\item[(a)] In Kittel Section 8.1, we find $E_{\text{g}} = \SI{1.27}{eV}$, 
and in Section 8.2.5, we find $m^*_{\text{e}} = 0.073m$, 
$m_{\text{h}} = 0.4m$, 
so according to 
\begin{equation}
    \mu = E_{\text{v}} + \frac{1}{2} E_{\text{g}} + \frac{3}{4} k_{\text{B}} T \ln(\frac{m_\text{h}^*}{m_{\text{e}}^*}),
\end{equation}
we find $\mu - E_{\text{v}} = \SI{0.68}{eV}$, so it's slightly 
above the middle point between $E_{\text{v}}$ and $E_{\text{c}}$.

\item[(b)] We know the Fermi-Dirac distribution function for electrons above $\mu$
and the Fermi-Dirac distribution function for holes below $\mu$
are symmetric to each other.
Since $m^*_{\text{h}}$ is larger, 
the valence band is flatter,
and near the top of the valence band,
the number of states is larger than that near the bottom of the conduction band.
So if $\mu - E_{\text{v}} = E_{\text{g}} / 2$, 
the Fermi-Dirac factor in $n$ and $p$ is the same,
while the density of state factors are different, 
and we find $p > n$, which is wrong.
So we have to push $\mu$ higher to reduce the $f(E)$ factor in $p$
to ensure $n = p$.

\end{itemize}

\paragraph{Problem 2} At very low temperatures, donor electrons in n-type semiconductors are bound to the impurity (donor) ions (only at higher temperatures will they be ionized and be mobile). Staying at very low temperatures, consider increasing the doping concentration to the point when the bound electron orbits on adjacent impurity atoms being to "touch" (overlap). When this occurs, electrons can become mobile by tunneling from one impurity orbit to the next and an "impurity band" forms (very close in philosophy to the tight-binding picture). Consider n-type InAs and use any needed values from Kittel.
a) What is the donor ionization energy in InAs? Assume the hydrogenic model holds so you can simply scale from the Hydrogen atom.
b) What is the radius of the electron orbit in the bound state on the impurity? Use the same assumptions as (a) and then check your assumptions are selfconsistently correct.
c) Estimate the donor concentration above which the adjacent electron orbits overlap significantly and we start to get an impurity band. (You can define a term like "significantly" in your own sensible way but make sure you explain your definition.)

\paragraph{Solution} \begin{itemize}
\item[(a)] When the hydrogenic model works,
the inner details of the donor atom don't matter: 
the doped material can be seen as a positive charge 
placed in an undoped material,
and the Coulomb field introduced by the former is then screened by the latter.
So the spectrum of an electron near the donor atom can be obtained 
by replacing $m$ by $m^*_{\text{e}}$ and $\epsilon_0$ by $\epsilon \epsilon_0$
in the Bohr theory of hydrogen, 
and we get 
\begin{equation}
    E_n = \frac{1}{n^2} \frac{\mathrm{e}^4 m^*_{\mathrm{e}}}{2\left(4 \pi \epsilon \epsilon_0 \hbar\right)^2}
    = \frac{1}{\epsilon^2} \frac{m^*_{\text{e}}}{m} \frac{1}{n^2} \underbrace{E_0}_{\SI{13.6}{eV}}.
\end{equation}
In Section 8.4.1 we find $\epsilon = 14.55$, and in Section 8.2.5 we find $m_{\text{e}}^* = 0.026 m$,
so finally we find $E_1 = - \SI{1.67e-3}{eV}$,
and therefore the ionization energy is $\SI{1.67e-3}{eV}$.

\item[(b)] Similarly we have 
\begin{equation}
    a = \frac{4 \pi \epsilon \epsilon_0 \hbar^2}{m_{\text{e}}^* e^2} = \frac{\epsilon m}{m_{\text{e}}^*} \cdot \SI{0.53}{\angstrom},
\end{equation}
and for \ce{InAs}, $a = \SI{29.6}{nm}$.
Wikipedia says the lattice constant of \ce{InAs} is $\sim \SI{6.05}{\angstrom}$,
so the radius of the electron donated is much larger than the characteristic length of the crystal structure,
which means our approximation is reasonable,
or otherwise at some momentum values involved,
the hyperbolic approximation of the band is no longer correct,
and the details of the crystal structure is visible 
to the motion of the impurity-bound electron,
and the approximation breaks down.

\item[(c)] When the distance between impurities is comparable to $a$,
hopping between nearest impurities becomes frequent.
So when 
\[
    n \gtrsim \frac{1}{a^3} = \SI{3.86e22}{m^{-3}},
\]
the impurity band becomes visible.

\end{itemize}

\paragraph{Problem 3} Kittel problem 2 of chapter 8 . The temperature here is so low, that the formulae we derived in lecture are not applicable (the chemical potential is now quite close to the donor bound state energy and not all the donors are ionized). You will have to read on the right relations to use for such low temperatures (A\&M
Chapter 28 problem 6 is a good place to look; Kittel also has a brief discussion, but you will have to see if the two of them agree...). 

In a particular semiconductor there are \SI{1e13}{donors/cm^{-3}}
with an ionization energy $E_d$ of $1 \mathrm{meV}$ and an effective mass $0.01 \mathrm{~m}$. (a) Estimate the concentration of conduction electrons at $4 \mathrm{~K}$. (b) What is the value of the Hall coefficent? Assume no acceptor atoms are present and that $E_g \gg k_B T$.

\paragraph{Solution} \begin{itemize}
\item[(a)] Thermal hopping from the valence band to the conduction band can be ignored and 
approximately 
\begin{equation}
    2 \left( \frac{m_\text{e}^* k_{\text{B}} T}{2 \pi \hbar^2} \right)^{3/2} 
    \ee^{- (E_{\text{c}} - \mu) / k_{\text{B}} T} = n = N_{\text{d}} - n_{\text{d}} = N_{\text{d}} \left( 1 - 
    \frac{1}{ 1 + \frac{1}{2} \ee^{ (E_{\text{c}} - E_{\text{d}} - \mu) / k_{\text{B}} T } } \right) ,
\end{equation}
and we can solve the equation and find $\ee^{-(E_{\text{c}} - \mu) / k_{\text{B}} T} = 0.07$, 
and therefore $n = \SI{2.7e18}{m^{-3}}$.

\item[(b)] The Hall coefficient is 
\begin{equation}
    R_{\text{H}} = \frac{1}{en} = 2.31.
\end{equation}

\end{itemize}

\paragraph{Problem 4} A particular donor atom in n-type GaAs has one electron to donate and has a donor binding energy of $E_d=35 \mathrm{meV}$. Consider only one conduction band and one valence band for GaAs with effective masses $m_e^*=0.066 \mathrm{~m}$ and $m_h^*=0.5 \mathrm{~m}$ where $m$ is the bare electron mass. Take the band gap of GaAs to be $E_g=1.4 \mathrm{eV}$. Let the donor density be $N_d=5 \times 10^{15} \mathrm{~cm}^{-3}$. There are no acceptors present. The system as at room temperature which you can take to be $T=300 \mathrm{~K}$.
a) Assuming non-degenerate conditions hold, find the intrinsic chemical potential $\mu_i(T)$ and then the actual chemical potential $\mu(T)$ in eV referenced to the valence band edge energy.
b) What is density of non-ionized donors $n_d(T)$ ? Is the system in the extrinsic or intrinsic limit?
c) Does the non-degenerate assumption hold?
d) Find the density $n(T)$ of electrons in the conduction band.
e) Compare the result of (c) to the intrinsic electron density $n_i(T)$.

\paragraph{Solution} \begin{itemize}
\item[(a)] The intrinsic chemical potential is 
\begin{equation}
    \mu_{\text{i}} = \frac{1}{2} E_{\text{g}} + \frac{3}{4} k_{\text{B}} T \ln(\frac{m^*_{\text{h}} }{m^*_{\text{e}}}) = \SI{0.74}{eV}.
\end{equation}
So 
\begin{equation}
    n_{\text{i}} = 2 \left( \frac{m_\text{e}^* k_{\text{B}} T}{2 \pi \hbar^2} \right)^{3/2} 
    \ee^{- (E_{\text{c}} - \mu_{\text{i}}) / k_{\text{B}} T} 
    = \SI{3.48e12}{m^{-3}}.
\end{equation}
Using Eq. (28.39) in A\&M (further assuming $E_{\text{d}} - \mu \gg k_{\text{B}} T$), which is 
\begin{equation}
    \frac{N_\text{d} - N_\text{a}}{n_\text{i}} = 2 \sinh \frac{\mu - \mu_{\text{i}}}{k_{\text{B}} T},
    \label{eq:quick-mu}
\end{equation}
we get $\mu = \SI{0.74}{eV} + \SI{0.54}{eV} = \SI{1.28}{eV}$.

\item[(b)] We have (here $E_{\text{d}}$ is the donor binding energy)
\begin{equation}
    n_{\text{d}} = \frac{N_{\text{d}}}{1 + \frac{1}{2} \ee^{(E_{\text{c}} - E_{\text{d}} - \mu) / k_{\text{B} T}}}
    = \SI{3.5e14}{cm^{-3}}.
\end{equation}
It can be found that most of the donor electrons indeed enter the conduction band,
and $N_{\text{d}} - n_{\text{d}} \gg n_{\text{i}}$.
The system is therefore in the extrinsic limit.

\item[(c)] We have 
\[
    \frac{E_{\text{c}} - \mu}{k_{\text{B}} T} = 4.6, \quad 
    \frac{E_{\text{c}} - E_{\text{d}} - \mu}{k_{\text{B}} T} = 3.3,
\]
so the non-degenerate assumption that the above ratios should be very large 
doesn't hold -- but since $\Delta E / k_{\text{B}} T$ appears in the exponential function,
in practice it's still good enough.

\item[(d)] We have 
\begin{equation}
    n = \frac{\Delta n + \sqrt{ 4 n_{\text{i}}^2 + \Delta n^2 }}{2}, 
    \quad \Delta n = N_{\text{d}} - n_{\text{d}},
\end{equation}
and the result is $n = \SI{4.65e15}{cm^{-3}}$.

\item[(e)] It can be found that the system is indeed in the extrinsic limit:
we have $n = \Delta n$, 
and the contribution of $n_{\text{i}}$ can be ignored.

\end{itemize}

\paragraph{Problem 5} An acceptor atom in p-type GaAs can donate one hole per atom. It has an acceptor binding energy of $E_a=7 \mathrm{meV}$ and a density of $N_a=1 \times 10^{12} \mathrm{~cm}^{-3}$. There are no donors present. Other materials parameters are identical to the previous problem (problem 4 above).
a) What is $\mu(T)$ at $T=700 \mathrm{~K}$ referenced with respect to the middle of the gap? If you did this the harder way using the actual formula for the doped case, why can you also get a pretty accurate answer in an easier way?
b) Sketch $\mu(T)$ versus $T$ for this system from $0 \mathrm{~K}$ to $1000 \mathrm{~K}$. Explain the shape of the curve and different limit behaviors it may have.

\paragraph{Solution} \begin{itemize}
\item[(a)] When $T = \SI{700}{K}$, we have 
\begin{equation}
    n_{\text{i}} = 2\left(\frac{k_{\mathrm{B}} T}{2 \pi \hbar^2}\right)^{3 / 2}\left(m_{\mathrm{e}} m_{\mathrm{h}}\right)^{3 / 4} \exp \left(-E_{\mathrm{g}} / 2 k_{\mathrm{B}} T\right)
    = \SI{1.1e22}{m^{-3}},
\end{equation}
and (here we assume the middle of the gap has zero energy)
\begin{eqnarray}
    \mu_{\text{i}} = \frac{3}{4} k_{\text{B}} T \ln(\frac{m^*_{\text{h}} }{m^*_{\text{e}}}) = \SI{0.92}{eV}.
\end{eqnarray}
Again using \eqref{eq:quick-mu}, we have $\mu = \SI{0.92}{eV} - \SI{0.07}{eV} = \SI{0.85}{eV}$.
This is close to $\mu_{\text{i}}$: the relative error is 0.08.
The reason is now the system is in the intrinsic limit, 
because doping concentration is much smaller than $n_{\text{i}}$.

\item[(b)] See \prettyref{fig:semiconductor-limits}.

\begin{figure}
    \centering
    \begin{tikzpicture}[x=0.75pt,y=0.75pt,yscale=-1,xscale=1]
    %uncomment if require: \path (0,355); %set diagram left start at 0, and has height of 355
    
    %Straight Lines [id:da12533057960173433] 
    \draw    (150,224) -- (379.5,224) ;
    \draw [shift={(381.5,224)}, rotate = 180] [fill={rgb, 255:red, 0; green, 0; blue, 0 }  ][line width=0.08]  [draw opacity=0] (12,-3) -- (0,0) -- (12,3) -- cycle    ;
    %Straight Lines [id:da46025762315895236] 
    \draw    (150,224) -- (150,54.85) ;
    \draw [shift={(150,52.85)}, rotate = 90] [fill={rgb, 255:red, 0; green, 0; blue, 0 }  ][line width=0.08]  [draw opacity=0] (12,-3) -- (0,0) -- (12,3) -- cycle    ;
    %Straight Lines [id:da9241041282655598] 
    \draw  [dash pattern={on 0.84pt off 2.51pt}]  (150,224) -- (365.5,113.85) ;
    %Curve Lines [id:da902430446942259] 
    \draw    (150,199) .. controls (159.5,104.85) and (191.5,165.85) .. (213.5,170.85) .. controls (235.5,175.85) and (329.5,128.85) .. (365.5,111.85) ;
    %Shape: Circle [id:dp16809872568107043] 
    \draw  [color={rgb, 255:red, 80; green, 227; blue, 194 }  ,draw opacity=1 ][fill={rgb, 255:red, 80; green, 227; blue, 194 }  ,fill opacity=0.1 ] (136.3,199) .. controls (136.3,191.43) and (142.43,185.3) .. (150,185.3) .. controls (157.57,185.3) and (163.7,191.43) .. (163.7,199) .. controls (163.7,206.57) and (157.57,212.7) .. (150,212.7) .. controls (142.43,212.7) and (136.3,206.57) .. (136.3,199) -- cycle ;
    %Shape: Circle [id:dp2672422180325582] 
    \draw  [color={rgb, 255:red, 74; green, 144; blue, 226 }  ,draw opacity=1 ][fill={rgb, 255:red, 74; green, 144; blue, 226 }  ,fill opacity=0.1 ] (184.75,163.55) .. controls (184.75,155.99) and (190.88,149.85) .. (198.45,149.85) .. controls (206.01,149.85) and (212.15,155.99) .. (212.15,163.55) .. controls (212.15,171.12) and (206.01,177.25) .. (198.45,177.25) .. controls (190.88,177.25) and (184.75,171.12) .. (184.75,163.55) -- cycle ;
    %Shape: Circle [id:dp6203728259149861] 
    \draw  [color={rgb, 255:red, 144; green, 19; blue, 254 }  ,draw opacity=1 ][fill={rgb, 255:red, 144; green, 19; blue, 254 }  ,fill opacity=0.1 ] (351.8,111.85) .. controls (351.8,104.29) and (357.93,98.16) .. (365.5,98.16) .. controls (373.07,98.16) and (379.2,104.29) .. (379.2,111.85) .. controls (379.2,119.42) and (373.07,125.55) .. (365.5,125.55) .. controls (357.93,125.55) and (351.8,119.42) .. (351.8,111.85) -- cycle ;
    
    % Text Node
    \draw (270,163.4) node [anchor=north west][inner sep=0.75pt]    {$\mu _{\text{i}}$};
    % Text Node
    \draw (383.5,224) node [anchor=west] [inner sep=0.75pt]    {$T$};
    % Text Node
    \draw (150,49.45) node [anchor=south] [inner sep=0.75pt]    {$\mu $};
    % Text Node
    \draw (20,126) node [anchor=north west][inner sep=0.75pt]   [align=left] {very low $\displaystyle T$,\\donated electrons\\start to be \\excited};
    % Text Node
    \draw (179,62) node [anchor=north west][inner sep=0.75pt]   [align=left] {low $\displaystyle T$,\\donated electrons\\almost fully excited;\\extrinsic limit};
    % Text Node
    \draw (393,69) node [anchor=north west][inner sep=0.75pt]   [align=left] {high T,\\thermal inter-band\\hopping dominates;\\intrinsic limit};
    
    
    \end{tikzpicture}
    
    \caption{The temperature dependence of $\mu$. 
    The energy zero point is $E_{\text{v}} + E_{\text{g}} / 2$.
    When $T$ is very, very close to 0, doped electron }
    \label{fig:semiconductor-limits}
\end{figure}

\end{itemize}

\end{document}