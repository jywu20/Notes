\documentclass[hyperref, a4paper]{article}

\usepackage{geometry}
\usepackage{titling}
\usepackage{titlesec}
% No longer needed, since we will use enumitem package
% \usepackage{paralist}
\usepackage{enumitem}
\usepackage{footnote}
\usepackage{enumerate}
\usepackage{amsmath, amssymb, amsthm}
\usepackage{mathtools}
\usepackage{bbm}
\usepackage{cite}
\usepackage{graphicx}
\usepackage{subfigure}
\usepackage{physics}
\usepackage{tensor}
\usepackage{siunitx}
\usepackage[version=4]{mhchem}
\usepackage{tikz}
\usepackage{xcolor}
\usepackage{listings}
\usepackage{autobreak}
\usepackage[ruled, vlined, linesnumbered]{algorithm2e}
\usepackage{nameref,zref-xr}
\zxrsetup{toltxlabel}
\usepackage[colorlinks,unicode]{hyperref} % , linkcolor=black, anchorcolor=black, citecolor=black, urlcolor=black, filecolor=black
\usepackage[most]{tcolorbox}
\usepackage{prettyref}

% Page style
\geometry{left=3.18cm,right=3.18cm,top=2.54cm,bottom=2.54cm}
\titlespacing{\paragraph}{0pt}{1pt}{10pt}[20pt]
\setlength{\droptitle}{-5em}
%\preauthor{\vspace{-10pt}\begin{center}}
%\postauthor{\par\end{center}}

% More compact lists 
\setlist[itemize]{
    itemindent=17pt, 
    leftmargin=1pt,
    listparindent=\parindent,
    parsep=0pt,
}

% Math operators
\DeclareMathOperator{\timeorder}{\mathcal{T}}
\DeclareMathOperator{\diag}{diag}
\DeclareMathOperator{\legpoly}{P}
\DeclareMathOperator{\primevalue}{P}
\DeclareMathOperator{\sgn}{sgn}
\newcommand*{\ii}{\mathrm{i}}
\newcommand*{\ee}{\mathrm{e}}
\newcommand*{\const}{\mathrm{const}}
\newcommand*{\suchthat}{\quad \text{s.t.} \quad}
\newcommand*{\argmin}{\arg\min}
\newcommand*{\argmax}{\arg\max}
\newcommand*{\normalorder}[1]{: #1 :}
\newcommand*{\pair}[1]{\langle #1 \rangle}
\newcommand*{\fd}[1]{\mathcal{D} #1}
\DeclareMathOperator{\bigO}{\mathcal{O}}

% TikZ setting
\usetikzlibrary{arrows,shapes,positioning}
\usetikzlibrary{arrows.meta}
\usetikzlibrary{decorations.markings}
\tikzstyle arrowstyle=[scale=1]
\tikzstyle directed=[postaction={decorate,decoration={markings,
    mark=at position .5 with {\arrow[arrowstyle]{stealth}}}}]
\tikzstyle ray=[directed, thick]
\tikzstyle dot=[anchor=base,fill,circle,inner sep=1pt]

% Algorithm setting
% Julia-style code
\SetKwIF{If}{ElseIf}{Else}{if}{}{elseif}{else}{end}
\SetKwFor{For}{for}{}{end}
\SetKwFor{While}{while}{}{end}
\SetKwProg{Function}{function}{}{end}
\SetArgSty{textnormal}

\newcommand*{\concept}[1]{{\textbf{#1}}}

% Embedded codes
\lstset{basicstyle=\ttfamily,
  showstringspaces=false,
  commentstyle=\color{gray},
  keywordstyle=\color{blue}
}

% Reference formatting
\newrefformat{fig}{Figure~\ref{#1}}

% Color boxes
\tcbuselibrary{skins, breakable, theorems}
\newtcbtheorem[number within=section]{warning}{Warning}%
  {colback=orange!5,colframe=orange!65,fonttitle=\bfseries, breakable}{warn}
\newtcbtheorem[number within=section]{note}{Note}%
  {colback=green!5,colframe=green!65,fonttitle=\bfseries, breakable}{note}
\newtcbtheorem[number within=section]{info}{Info}%
  {colback=blue!5,colframe=blue!65,fonttitle=\bfseries, breakable}{info}

\newenvironment{shelldisplay}{\begin{lstlisting}}{\end{lstlisting}}

\newcommand{\address}[1]{\href{#1}{\url{#1}}}

\title{Homework 11}
\author{Jiinyuan Wu}

\begin{document}
    
\maketitle

\paragraph{Problem 1} Consider the band structure of $\operatorname{InP}$ as having only the electron band and the heavy hole band. Use numerical values from Kittel for the material.
a) Where in the band gap is the chemical potential located at $300 \mathrm{~K}$ for an ultrapure sample of InP?
b) Explain briefly the physical reason why the chemical potential has the value you found in part (a) (i.e., why it changes in the direction it does with $\mathrm{T}>0$ compared to $\mathrm{T}=0$ ).

\paragraph{Solution} \begin{itemize}
\item[(a)] In Kittel Section 8.1, we find $E_{\text{g}} = \SI{1.27}{eV}$, 
and in Section 8.2.5, we find $m^*_{\text{e}} = 0.073m$, 
$m_{\text{h}} = 0.4m$, 
so according to 
\begin{equation}
    \mu = E_{\text{v}} + \frac{1}{2} E_{\text{g}} + \frac{3}{4} k_{\text{B}} T \ln(\frac{m_\text{h}^*}{m_{\text{e}}^*}),
\end{equation}
we find $\mu - E_{\text{v}} = \SI{0.68}{eV}$, so it's slightly 
above the middle point between $E_{\text{v}}$ and $E_{\text{c}}$.

\item[(b)] We know the Fermi-Dirac distribution function for electrons above $\mu$
and the Fermi-Dirac distribution function for holes below $\mu$
are symmetric to each other.
Since $m^*_{\text{h}}$ is larger, 
the valence band is flatter,
and near the top of the valence band,
the number of states is larger than that near the bottom of the conduction band.
So if $\mu - E_{\text{v}} = E_{\text{g}} / 2$, 
the Fermi-Dirac factor in $n$ and $p$ is the same,
while the density of state factors are different, 
and we find $p > n$, which is wrong.
So we have to push $\mu$ higher to reduce the $f(E)$ factor in $p$
to ensure $n = p$.

\end{itemize}

\paragraph{Problem 2}

\paragraph{Solution} \begin{itemize}
\item[(a)] When the hydrogenic model works,
the inner details of the donor atom don't matter: 
the doped material can be seen as a positive charge 
placed in an undoped material,
and the Coulomb field introduced by the former is then screened by the latter.
So the spectrum of an electron near the donor atom can be obtained 
by replacing $m$ by $m^*_{\text{e}}$ and $\epsilon_0$ by $\epsilon \epsilon_0$
in the Bohr theory of hydrogen, 
and we get 
\begin{equation}
    E_n = \frac{1}{n^2} \frac{\mathrm{e}^4 m^*_{\mathrm{e}}}{2\left(4 \pi \epsilon \epsilon_0 \hbar\right)^2}
    = \frac{1}{\epsilon^2} \frac{m^*_{\text{e}}}{m} \frac{1}{n^2} \underbrace{E_0}_{\SI{13.6}{eV}}.
\end{equation}
In Section 8.4.1 we find $\epsilon = 14.55$, and in Section 8.2.5 we find $m_{\text{e}}^* = 0.026 m$,
so finally we find $E_1 = - \SI{1.67e-3}{eV}$,
and therefore the ionization energy is $\SI{1.67e-3}{eV}$.

\item[(b)] Similarly we have 
\begin{equation}
    a = \frac{4 \pi \epsilon \epsilon_0 \hbar^2}{m_{\text{e}}^* e^2} = \frac{\epsilon m}{m_{\text{e}}^*} \cdot \SI{0.53}{\angstrom},
\end{equation}
and for \ce{InAs}, $a = \SI{29.6}{nm}$.
Wikipedia says the lattice constant of \ce{InAs} is $\sim \SI{6.05}{\angstrom}$,
so the radius of the electron donated is much larger than the characteristic length of the crystal structure,
which means our approximation is reasonable,
or otherwise at some momentum values involved,
the hyperbolic approximation of the band is no longer correct,
and the details of the crystal structure is visible 
to the motion of the impurity-bound electron,
and the approximation breaks down.

\item[(c)] When the distance between impurities is comparable to $a$,
hopping between nearest impurities becomes frequent.
So when 
\[
    n \gtrsim \frac{1}{a^3} = \SI{3.86e22}{m^{-3}},
\]
the impurity band becomes visible.

\end{itemize}

\paragraph{Problem 4} 

\paragraph{Solution} \begin{itemize}
\item[(a)] The intrinsic chemical potential is 
\begin{equation}
    \mu_{\text{i}} = \frac{1}{2} E_{\text{g}} + \frac{3}{4} k_{\text{B}} T \ln(\frac{m^*_{\text{h}} }{e^*_{\text{e}}}) = \SI{0.74}{eV}.
\end{equation}
So 
\begin{equation}
    n_{\text{i}} = 2 \left( \frac{m_\text{e}^* k_{\text{B}} T}{2 \pi \hbar^2} \right)^{3/2} 
    \ee^{- (E_{\text{c}} - \mu_{\text{i}}) / k_{\text{B}} T} 
    = \SI{3.48e12}{m^{-3}}.
\end{equation}
Using Eq. (28.39) in A\&M (further assuming $E_{\text{d}} - \mu \gg k_{\text{B}} T$), which is 
\begin{equation}
    \frac{N_\text{d} - N_\text{a}}{n_\text{i}} = 2 \sinh \frac{\mu - \mu_{\text{i}}}{k_{\text{B}} T},
\end{equation}
we get $\mu = \SI{0.74}{eV} + \SI{0.54}{eV} = \SI{1.28}{eV}$.

\item[(b)] We have (here $E_{\text{d}}$ is the donor binding energy)
\begin{equation}
    n_{\text{d}} = \frac{N_{\text{d}}}{1/2 + \ee^{(E_{\text{c}} - E_{\text{d}} - \mu) / k_{\text{B} T}}}
    = \SI{1.8e14}{cm^{-3}}.
\end{equation}
It can be found that most of the donor electrons indeed enter the conduction band,
and $N_{\text{d}} - n_{\text{d}} \gg n_{\text{i}}$.
The system is therefore in the extrinsic limit.

\item[(c)] We have 
\[
    \frac{E_{\text{c}} - \mu}{k_{\text{B}} T} = 4.6, \quad 
    \frac{E_{\text{c}} - E_{\text{d}} - \mu}{k_{\text{B}} T} = 3.3,
\]
so the non-degenerate assumption that the above ratios should be very large 
doesn't hold -- but since $\Delta E / k_{\text{B}} T$ appears in the exponential function,
in practice it's still good enough.

\item[(d)] We have 
\begin{equation}
    n = \frac{\Delta n + \sqrt{ 4 n_{\text{i}}^2 + \Delta n^2 }}{2}, 
    \quad \Delta n = N_{\text{d}} - n_{\text{d}},
\end{equation}
and the result is $n = \SI{4.82e15}{cm^{-3}}$.

\item[(e)] It can be found that the system is indeed in the extrinsic limit:
we have $n = \Delta n$, 
and the contribution of $n_{\text{i}}$ can be ignored.

\end{itemize}

\end{document}