\documentclass[hyperref, a4paper]{article}

\usepackage{geometry}
\usepackage{titling}
\usepackage{titlesec}
% No longer needed, since we will use enumitem package
% \usepackage{paralist}
\usepackage{enumitem}
\usepackage{footnote}
\usepackage{enumerate}
\usepackage{amsmath, amssymb, amsthm}
\usepackage{mathtools}
\usepackage{bbm}
\usepackage{cite}
\usepackage{graphicx}
\usepackage{subfigure}
\usepackage{physics}
\usepackage{tensor}
\usepackage{siunitx}
\usepackage[version=4]{mhchem}
\usepackage{tikz}
\usepackage{xcolor}
\usepackage{listings}
\usepackage{autobreak}
\usepackage[ruled, vlined, linesnumbered]{algorithm2e}
\usepackage{nameref,zref-xr}
\zxrsetup{toltxlabel}
\usepackage[colorlinks,unicode]{hyperref} % , linkcolor=black, anchorcolor=black, citecolor=black, urlcolor=black, filecolor=black
\usepackage[most]{tcolorbox}
\usepackage{prettyref}

% Page style
\geometry{left=3.18cm,right=3.18cm,top=2.54cm,bottom=2.54cm}
\titlespacing{\paragraph}{0pt}{1pt}{10pt}[20pt]
\setlength{\droptitle}{-5em}
%\preauthor{\vspace{-10pt}\begin{center}}
%\postauthor{\par\end{center}}

% More compact lists 
\setlist[itemize]{
    itemindent=17pt, 
    leftmargin=1pt,
    listparindent=\parindent,
    parsep=0pt,
}

% Math operators
\DeclareMathOperator{\timeorder}{\mathcal{T}}
\DeclareMathOperator{\diag}{diag}
\DeclareMathOperator{\legpoly}{P}
\DeclareMathOperator{\primevalue}{P}
\DeclareMathOperator{\sgn}{sgn}
\newcommand*{\ii}{\mathrm{i}}
\newcommand*{\ee}{\mathrm{e}}
\newcommand*{\const}{\mathrm{const}}
\newcommand*{\suchthat}{\quad \text{s.t.} \quad}
\newcommand*{\argmin}{\arg\min}
\newcommand*{\argmax}{\arg\max}
\newcommand*{\normalorder}[1]{: #1 :}
\newcommand*{\pair}[1]{\langle #1 \rangle}
\newcommand*{\fd}[1]{\mathcal{D} #1}
\DeclareMathOperator{\bigO}{\mathcal{O}}

% TikZ setting
\usetikzlibrary{arrows,shapes,positioning}
\usetikzlibrary{arrows.meta}
\usetikzlibrary{decorations.markings}
\tikzstyle arrowstyle=[scale=1]
\tikzstyle directed=[postaction={decorate,decoration={markings,
    mark=at position .5 with {\arrow[arrowstyle]{stealth}}}}]
\tikzstyle ray=[directed, thick]
\tikzstyle dot=[anchor=base,fill,circle,inner sep=1pt]

% Algorithm setting
% Julia-style code
\SetKwIF{If}{ElseIf}{Else}{if}{}{elseif}{else}{end}
\SetKwFor{For}{for}{}{end}
\SetKwFor{While}{while}{}{end}
\SetKwProg{Function}{function}{}{end}
\SetArgSty{textnormal}

\newcommand*{\concept}[1]{{\textbf{#1}}}

% Embedded codes
\lstset{basicstyle=\ttfamily,
  showstringspaces=false,
  commentstyle=\color{gray},
  keywordstyle=\color{blue}
}

% Reference formatting
\newrefformat{fig}{Figure~\ref{#1}}

% Color boxes
\tcbuselibrary{skins, breakable, theorems}
\newtcbtheorem[number within=section]{warning}{Warning}%
  {colback=orange!5,colframe=orange!65,fonttitle=\bfseries, breakable}{warn}
\newtcbtheorem[number within=section]{note}{Note}%
  {colback=green!5,colframe=green!65,fonttitle=\bfseries, breakable}{note}
\newtcbtheorem[number within=section]{info}{Info}%
  {colback=blue!5,colframe=blue!65,fonttitle=\bfseries, breakable}{info}

\newenvironment{shelldisplay}{\begin{lstlisting}}{\end{lstlisting}}

\newcommand{\address}[1]{\href{#1}{\url{#1}}}

\title{Homework 7}
\author{Jinyuan Wu}

\begin{document}

\maketitle

\paragraph{Problem 2} 

\paragraph{Solution} \begin{itemize}
\item[(a)] From $j = n e v$ we get 
\[
    \abs{\expval{\vb*{v}}} = \frac{j}{n e} = \frac{j a^3}{e} = \SI{4.1e-5}{m/s}.
\]
\item[(b)] The magnitude of Fermi velocity is $\sim \SI{10e8}{m/s}$, 
so the fraction is $\sim 10^{-13}$.
\item[(c)] Since 
\begin{equation}
    \sigma = \frac{n \tau e^2}{m},
\end{equation}
we have 
\[
    \tau = \frac{m}{\rho n e^2} = \frac{m a^3}{\rho e^2} = \SI{8.6e-14}{s^{-1}}.
\]
\end{itemize}

\paragraph{Problem 3}

\paragraph{Solution} \begin{itemize}
\item[(a)] The number density, according to (2.76) in A\&M, is 
\begin{equation}
    n=\int_0^{\epsilon_\text{F}} g(\epsilon) \dd \epsilon +\left\{\left(\mu-\epsilon_\text{F}\right) g\left(\epsilon_\text{F}\right)+\frac{\pi^2}{6}\left(k_B T\right)^2 g^{\prime}\left(\epsilon_\text{F}\right)\right\} ,
\end{equation}
so after adding one electron, we have 
\[
    \frac{1}{V} = \Delta n = \Delta \mu \cdot g(\epsilon_{\text{F}}),
\]
because on the RHS there is only one term that contains $\mu$,
and therefore 
\begin{equation}
    \Delta \mu = \frac{1}{V g(\epsilon_{\text{F}})}.
\end{equation}
\item[(b)] We have
\[
    \begin{aligned}
        \Delta f &= \Delta \mu \cdot \eval{\dv{\mu} 
        \frac{1}{\ee^{\beta (\epsilon - \mu)} + 1}}_{\epsilon = \epsilon_{\text{F}}}  \\
        &= \Delta \mu \cdot \frac{\beta}{4} \\
        &= \frac{\beta}{4 V g(\epsilon_{\text{F}})}. 
    \end{aligned}
\]
Since we have (2.65), which is 
\[
    g(\epsilon_{\text{F}}) = \frac{3 n }{2 \epsilon_{\text{F}}},
\]
we have 
\begin{equation}
    \Delta f =  \frac{\beta}{4 V } \frac{2 \epsilon_{\text{F}}}{3 n} 
    = \frac{\epsilon_{\text{F}}}{6 N k_{\text{B}} T}.
\end{equation}
\end{itemize}

\paragraph{Problem 4}

\paragraph{Solution} \begin{itemize}
\item[(a)] We choose $T = \SI{298}{K}$.
Putting the mass into the formula
\begin{equation}
    \lambda = \frac{h}{\sqrt{3 m k_{\text{B}} T}},
\end{equation}
for an electron, $\lambda = \SI{6.2e-9}{m}$;
for a Na atom, $\lambda = \SI{3.1e-11}{m}$;
for a $^4$He, $\lambda = \SI{7.3e-11}{m}$.
\item[(b)] When 
\begin{equation}
    a \sim \frac{h}{\sqrt{3 m k_{\text{B}} T} },
\end{equation}
or in other words 
\begin{equation}
    T \sim \frac{h^2}{3 m k_{\text{B}} a^2},
\end{equation}
the particle is wave-like enough.
For an electron it's \SI{1.3e5}{K}.
For a Na atom it's \SI{3}{K}.
For $^4$He it's \SI{18}{K}.
So in ordinary conditions, 
electrons should always be treated quantum mechanically.

\item[(c)] Now the criterion is 
\begin{equation}
    \left(\frac{k_{\text{B}} T}{P}\right)^{1/3} = \left(\frac{V}{N}\right)^{1/3} = a \sim \frac{h}{\sqrt{3 m k_{\text{B}} T} },
\end{equation}
and we have 
\begin{equation}
    T \sim \left( \frac{h^6}{(3 m k_{\text{B}})^3} \left( \frac{P}{k_{\text{B}}} \right)^2 \right)^{1/5} 
    = \SI{2.3}{K}.
\end{equation}
This means in ordinary conditions, 
$^4$He is still largely classical.

When $T = \SI{3}{K}$, we have $\lambda = \SI{7.3e-10}{m}$,
which is much smaller than \SI{1}{cm},
so $^4$He atoms in outer space can be treated classically.

\item[(d)] 

\end{itemize}

\end{document}