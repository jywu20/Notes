\documentclass[hyperref, a4paper]{article}

\usepackage{geometry}
\usepackage{titling}
\usepackage{titlesec}
% No longer needed, since we will use enumitem package
% \usepackage{paralist}
\usepackage{enumitem}
\usepackage{footnote}
\usepackage{enumerate}
\usepackage{amsmath, amssymb, amsthm}
\usepackage{mathtools}
\usepackage{bbm}
\usepackage{cite}
\usepackage{graphicx}
\usepackage{subfigure}
\usepackage{physics}
\usepackage{tensor}
\usepackage{siunitx}
\usepackage[version=4]{mhchem}
\usepackage{tikz}
\usepackage{xcolor}
\usepackage{listings}
\usepackage{autobreak}
\usepackage[ruled, vlined, linesnumbered]{algorithm2e}
\usepackage{nameref,zref-xr}
\zxrsetup{toltxlabel}
\usepackage[colorlinks,unicode]{hyperref} % , linkcolor=black, anchorcolor=black, citecolor=black, urlcolor=black, filecolor=black
\usepackage[most]{tcolorbox}
\usepackage{prettyref}

% Page style
\geometry{left=3.18cm,right=3.18cm,top=2.54cm,bottom=2.54cm}
\titlespacing{\paragraph}{0pt}{1pt}{10pt}[20pt]
\setlength{\droptitle}{-5em}
%\preauthor{\vspace{-10pt}\begin{center}}
%\postauthor{\par\end{center}}

% More compact lists 
\setlist[itemize]{
    itemindent=17pt, 
    leftmargin=1pt,
    listparindent=\parindent,
    parsep=0pt,
}

% Math operators
\DeclareMathOperator{\timeorder}{\mathcal{T}}
\DeclareMathOperator{\diag}{diag}
\DeclareMathOperator{\legpoly}{P}
\DeclareMathOperator{\primevalue}{P}
\DeclareMathOperator{\sgn}{sgn}
\newcommand*{\ii}{\mathrm{i}}
\newcommand*{\ee}{\mathrm{e}}
\newcommand*{\const}{\mathrm{const}}
\newcommand*{\suchthat}{\quad \text{s.t.} \quad}
\newcommand*{\argmin}{\arg\min}
\newcommand*{\argmax}{\arg\max}
\newcommand*{\normalorder}[1]{: #1 :}
\newcommand*{\pair}[1]{\langle #1 \rangle}
\newcommand*{\fd}[1]{\mathcal{D} #1}
\DeclareMathOperator{\bigO}{\mathcal{O}}

% TikZ setting
\usetikzlibrary{arrows,shapes,positioning}
\usetikzlibrary{arrows.meta}
\usetikzlibrary{decorations.markings}
\tikzstyle arrowstyle=[scale=1]
\tikzstyle directed=[postaction={decorate,decoration={markings,
    mark=at position .5 with {\arrow[arrowstyle]{stealth}}}}]
\tikzstyle ray=[directed, thick]
\tikzstyle dot=[anchor=base,fill,circle,inner sep=1pt]

% Algorithm setting
% Julia-style code
\SetKwIF{If}{ElseIf}{Else}{if}{}{elseif}{else}{end}
\SetKwFor{For}{for}{}{end}
\SetKwFor{While}{while}{}{end}
\SetKwProg{Function}{function}{}{end}
\SetArgSty{textnormal}

\newcommand*{\concept}[1]{{\textbf{#1}}}

% Embedded codes
\lstset{basicstyle=\ttfamily,
  showstringspaces=false,
  commentstyle=\color{gray},
  keywordstyle=\color{blue}
}

% Reference formatting
\newrefformat{fig}{Figure~\ref{#1}}

% Color boxes
\tcbuselibrary{skins, breakable, theorems}
\newtcbtheorem[number within=section]{warning}{Warning}%
  {colback=orange!5,colframe=orange!65,fonttitle=\bfseries, breakable}{warn}
\newtcbtheorem[number within=section]{note}{Note}%
  {colback=green!5,colframe=green!65,fonttitle=\bfseries, breakable}{note}
\newtcbtheorem[number within=section]{info}{Info}%
  {colback=blue!5,colframe=blue!65,fonttitle=\bfseries, breakable}{info}

\newenvironment{shelldisplay}{\begin{lstlisting}}{\end{lstlisting}}

\newcommand{\address}[1]{\href{#1}{\url{#1}}}

\title{Homework 10}
\author{Jiinyuan Wu}

\begin{document}

\maketitle

\paragraph{Problem 1}

\paragraph{Solution}  \begin{itemize}
\item[(a)] From 
\begin{equation}
    w\left(\vb*{r}-\vb*{r}_n\right)=N^{-1 / 2} \sum_{\vb*{k}} \exp \left(- \ii \vb*{k} \cdot \vb*{r}_n\right) \psi_{\vb*{k}}(\vb*{r})
\end{equation}
we have 
\[
    \begin{aligned}
        \int \dd[3]{\vb*{r}} w(\vb*{r} - \vb*{r}_n) w^*(\vb*{r} - \vb*{r}_m)
        &= \frac{1}{N} \sum_{\vb*{k}, \vb*{k}'} 
        \ee^{- \ii \vb*{k} \cdot \vb*{r}_n} \ee^{\ii \vb*{k} \cdot \vb*{r}_m}
        \int \dd[3]{\vb*{r}} \psi_{\vb*{k}}(\vb*{r}) \psi_{\vb*{k}'}^*(\vb*{r}) \\
        &= \frac{1}{N} \sum_{\vb*{k}, \vb*{k}'} \delta_{\vb*{k} \vb*{k}'} 
        \ee^{- \ii \vb*{k} \cdot \vb*{r}_n} \ee^{\ii \vb*{k} \cdot \vb*{r}_m} \\
        &= \frac{1}{N} \sum_{\vb*{k}} \ee^{\ii \vb*{k} \cdot (\vb*{r}_m - \vb*{r}_n)} \\
        &= \delta_{mn}.
    \end{aligned}
\]
So when $m \neq n$, we have 
\begin{equation}
    \int \dd[3]{\vb*{r}} w(\vb*{r} - \vb*{r}_n) w^*(\vb*{r} - \vb*{r}_m) = 0.
\end{equation}

\item[(b)] We have 
\[
    \begin{aligned}
        w(x - x_n) &= \frac{1}{\sqrt{N}} \sum_{k} 
        \ee^{- \ii k x_n} \underbrace{\frac{1}{\sqrt{N}} \ee^{\ii k x} u_0(x)}_{\psi_k(x)} \\
        &= \frac{1}{N} u_0(x) \sum_k \ee^{\ii k (x - x_n)} \\
        &= \frac{1}{N} u_0(x) \sum_{i=1}^N \ee^{\ii (x - x_n) \cdot 2\pi i / L} \\
        &= \frac{1}{N} u_0(x) \frac{
            \ee^{\ii (x - x_n) \cdot 2\pi / L} (1 - \ee^{\ii (x - x_n) \cdot 2\pi N / L})
        }{
            1 - \ee^{\ii (x - x_n) \cdot 2\pi / L}
        } \\
        &= \frac{1}{N} u_0(x) \ee^{\ii (x - x_n) \cdot 2\pi / L} 
        \frac{\ee^{\ii (x - x_n) \cdot \pi N / L}}{\ee^{\ii (x - x_n) \cdot \pi / L}}
        \frac{
            \ee^{- \ii (x - x_n) \cdot \pi N / L} - \ee^{\ii (x - x_n) \cdot \pi N / L}
        }{
           \ee^{- \ii (x - x_n) \cdot \pi / L} - \ee^{\ii (x - x_n) \cdot \pi / L}
        } \\
        &= \frac{1}{N} u_0(x) \ee^{\ii (x - x_n) \cdot 2\pi / L} 
        \frac{\ee^{\ii (x - x_n) \cdot \pi N / L}}{\ee^{\ii (x - x_n) \cdot \pi / L}}
        \frac{\sin \pi (x - x_n) N / L}{\sin \pi (x - x_n) / L}.
    \end{aligned}
\]
Note that $L = Na$, and therefore 
\[
    \frac{\sin \pi (x - x_n) N / L}{\sin \pi (x - x_n) / L}
    = \frac{\sin \pi (x - x_n) / a}{\sin \pi (x - x_n) / Na}
    \approx \frac{\sin \pi (x - x_n) / a}{\pi (x - x_n) / Na},
\]
and therefore 
\begin{equation}
    w(x - x_n) = \ee^{\ii \cdot \text{some number}} 
    \cdot u_0(x) \frac{\sin \pi (x - x_n) / a}{\pi (x - x_n) / a},
\end{equation}
and we can just throw away the unitary prefactor and therefore 
\begin{equation}
    w(x - x_n) = u_0(x) \frac{\sin \pi (x - x_n) / a}{\pi (x - x_n) / a}.
\end{equation}

\end{itemize}


\paragraph{Problem 2}

\paragraph{Solution} \begin{itemize}
\item[(a)] Following the standard procedure to find the band structure, we have 
\begin{equation}
    \begin{aligned}
        E_{\vb*{k}} &= A + \sum_{\vb*{R}} \ee^{\ii \vb*{k} \cdot \vb*{R}} h(\vb*{R}) \\
        &= A - t \ee^{\ii k a} - t \ee^{- \ii k a} = A - 2 t \cos(ka).
    \end{aligned}
\end{equation}
\item[(b)] The smallest energy is taken when $k = 0$,
and here we have 
\begin{equation}
    m^* = \frac{1}{\hbar^2} \pdv[2]{E_k}{k} = 2 t \frac{a^2}{\hbar^2} \cos ka |_{k = 0} = 2 t \frac{a^2}{\hbar^2}.
\end{equation}
The largest energy is taken when $k = \pm \pi / a$,
and 
\begin{equation}
    m^* = \frac{1}{\hbar^2} \pdv[2]{E_k}{k} = 2 t \frac{a^2}{\hbar^2} \cos ka |_{k = \pi / a} = - 2 t \frac{a^2}{\hbar^2}.
\end{equation}

\item[(c)] The band can contain $2N$ electrons,
because each $\vb*{k}$ position can host a spin-up electron and a spin-down electron.
So if each atom donates one valence electron,
then the band is half-filled,
so the system is a metal, and
\begin{equation}
    E_\text{F} = A,
\end{equation}
and 
\begin{equation}
    k_{\text{F}} a = \pm \frac{\pi}{2}, \quad k_{\text{F}} = \pm \frac{\pi}{2a}.
\end{equation}
If one atom donates two electrons,
then the band is completely filled and the system is an insulator.

\end{itemize}

\paragraph{Problem 3}

\paragraph{Solution} 

\end{document}