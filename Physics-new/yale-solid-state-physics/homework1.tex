\documentclass[hyperref, a4paper]{article}

\usepackage{geometry}
\usepackage{titling}
\usepackage{titlesec}
% No longer needed, since we will use enumitem package
% \usepackage{paralist}
\usepackage{enumitem}
\usepackage{footnote}
\usepackage{enumerate}
\usepackage{amsmath, amssymb, amsthm}
\usepackage{mathtools}
\usepackage{bbm}
\usepackage{cite}
\usepackage{graphicx}
\usepackage{subfigure}
\usepackage{physics}
\usepackage{tensor}
\usepackage{siunitx}
\usepackage[version=4]{mhchem}
\usepackage{tikz}
\usepackage{xcolor}
\usepackage{listings}
\usepackage{autobreak}
\usepackage[ruled, vlined, linesnumbered]{algorithm2e}
\usepackage{nameref,zref-xr}
\zxrsetup{toltxlabel}
\zexternaldocument*[hw2-]{../2/2}[2.pdf]
\usepackage[colorlinks,unicode]{hyperref} % , linkcolor=black, anchorcolor=black, citecolor=black, urlcolor=black, filecolor=black
\usepackage[most]{tcolorbox}
\usepackage{prettyref}

% Page style
\geometry{left=3.18cm,right=3.18cm,top=2.54cm,bottom=2.54cm}
\titlespacing{\paragraph}{0pt}{1pt}{10pt}[20pt]
\setlength{\droptitle}{-5em}
\preauthor{\vspace{-10pt}\begin{center}}
\postauthor{\par\end{center}}

% More compact lists 
\setlist[itemize]{
    itemindent=17pt, 
    leftmargin=1pt,
    listparindent=\parindent,
    parsep=0pt,
}

% Math operators
\DeclareMathOperator{\timeorder}{\mathcal{T}}
\DeclareMathOperator{\diag}{diag}
\DeclareMathOperator{\legpoly}{P}
\DeclareMathOperator{\primevalue}{P}
\DeclareMathOperator{\sgn}{sgn}
\newcommand*{\ii}{\mathrm{i}}
\newcommand*{\ee}{\mathrm{e}}
\newcommand*{\const}{\mathrm{const}}
\newcommand*{\suchthat}{\quad \text{s.t.} \quad}
\newcommand*{\argmin}{\arg\min}
\newcommand*{\argmax}{\arg\max}
\newcommand*{\normalorder}[1]{: #1 :}
\newcommand*{\pair}[1]{\langle #1 \rangle}
\newcommand*{\fd}[1]{\mathcal{D} #1}
\DeclareMathOperator{\bigO}{\mathcal{O}}

% TikZ setting
\usetikzlibrary{arrows,shapes,positioning}
\usetikzlibrary{arrows.meta}
\usetikzlibrary{decorations.markings}
\tikzstyle arrowstyle=[scale=1]
\tikzstyle directed=[postaction={decorate,decoration={markings,
    mark=at position .5 with {\arrow[arrowstyle]{stealth}}}}]
\tikzstyle ray=[directed, thick]
\tikzstyle dot=[anchor=base,fill,circle,inner sep=1pt]

% Algorithm setting
% Julia-style code
\SetKwIF{If}{ElseIf}{Else}{if}{}{elseif}{else}{end}
\SetKwFor{For}{for}{}{end}
\SetKwFor{While}{while}{}{end}
\SetKwProg{Function}{function}{}{end}
\SetArgSty{textnormal}

\newcommand*{\concept}[1]{{\textbf{#1}}}

% Embedded codes
\lstset{basicstyle=\ttfamily,
  showstringspaces=false,
  commentstyle=\color{gray},
  keywordstyle=\color{blue}
}

% Reference formatting
\newrefformat{fig}{Figure~\ref{#1}}

% Color boxes
\tcbuselibrary{skins, breakable, theorems}
\newtcbtheorem[number within=section]{warning}{Warning}%
  {colback=orange!5,colframe=orange!65,fonttitle=\bfseries, breakable}{warn}
\newtcbtheorem[number within=section]{note}{Note}%
  {colback=green!5,colframe=green!65,fonttitle=\bfseries, breakable}{note}
\newtcbtheorem[number within=section]{info}{Info}%
  {colback=blue!5,colframe=blue!65,fonttitle=\bfseries, breakable}{info}

\newenvironment{shelldisplay}{\begin{lstlisting}}{\end{lstlisting}}

\title{Solid State Physics Homework 1}
\author{Jinyuan Wu}

\begin{document}

\maketitle


\paragraph{Solution} 
\begin{itemize}
\item[(a)] Consider \prettyref{fig:volume-between}. Suppose there are $N$ points in an area $A$
on a certain lattice plane.
The volume between $A$ and its counterpart on an adjacent lattice plane is 
\[
    V_{\text{total}} = A d
\]
Now since there are $N$ points in $A$,
there are $N$ primitive unit cells between $A$ and its counterpart, so 
\[
    V_{\text{total}} = N V.
\]
Thus 
\begin{equation}
    NV = Ad, \quad \text{areal density} \coloneqq \frac{N}{A} = \frac{d}{V}.
\end{equation}
\item[(b)] To maximize the areal density of lattice points is to maximize $d$.

\end{itemize}

\begin{figure}
    \centering
    

\begin{tikzpicture}[x=0.75pt,y=0.75pt,yscale=-1,xscale=1]
%uncomment if require: \path (0,300); %set diagram left start at 0, and has height of 300

%Straight Lines [id:da40509447135468535] 
\draw [color={rgb, 255:red, 155; green, 155; blue, 155 }  ,draw opacity=1 ] [dash pattern={on 4.5pt off 4.5pt}]  (86,144.04) -- (102,92.04) ;
%Straight Lines [id:da5046748334738116] 
\draw [color={rgb, 255:red, 155; green, 155; blue, 155 }  ,draw opacity=1 ] [dash pattern={on 4.5pt off 4.5pt}]  (73,160) -- (89,108) ;
%Straight Lines [id:da9240421413092696] 
\draw [color={rgb, 255:red, 155; green, 155; blue, 155 }  ,draw opacity=1 ] [dash pattern={on 4.5pt off 4.5pt}]  (63,144.04) -- (79,92.04) ;
%Straight Lines [id:da4809710792018871] 
\draw [color={rgb, 255:red, 155; green, 155; blue, 155 }  ,draw opacity=1 ] [dash pattern={on 4.5pt off 4.5pt}]  (50,160) -- (66,108) ;
%Shape: Parallelogram [id:dp06799371355315431] 
\draw   (98.7,66.04) -- (175,66.04) -- (142.3,108) -- (66,108) -- cycle ;
%Shape: Parallelogram [id:dp7388899093821453] 
\draw   (82.7,118.04) -- (159,118.04) -- (126.3,160) -- (50,160) -- cycle ;
%Straight Lines [id:da8660880194969045] 
\draw    (50,160) -- (73,160) ;
\draw [shift={(73,160)}, rotate = 0] [color={rgb, 255:red, 0; green, 0; blue, 0 }  ][fill={rgb, 255:red, 0; green, 0; blue, 0 }  ][line width=0.75]      (0, 0) circle [x radius= 3.35, y radius= 3.35]   ;
\draw [shift={(50,160)}, rotate = 0] [color={rgb, 255:red, 0; green, 0; blue, 0 }  ][fill={rgb, 255:red, 0; green, 0; blue, 0 }  ][line width=0.75]      (0, 0) circle [x radius= 3.35, y radius= 3.35]   ;
%Straight Lines [id:da5161526563659284] 
\draw    (73,160) -- (96,160) ;
\draw [shift={(96,160)}, rotate = 0] [color={rgb, 255:red, 0; green, 0; blue, 0 }  ][fill={rgb, 255:red, 0; green, 0; blue, 0 }  ][line width=0.75]      (0, 0) circle [x radius= 3.35, y radius= 3.35]   ;
\draw [shift={(73,160)}, rotate = 0] [color={rgb, 255:red, 0; green, 0; blue, 0 }  ][fill={rgb, 255:red, 0; green, 0; blue, 0 }  ][line width=0.75]      (0, 0) circle [x radius= 3.35, y radius= 3.35]   ;
%Straight Lines [id:da39386897512422125] 
\draw    (50,160) -- (63,144.04) ;
\draw [shift={(63,144.04)}, rotate = 309.17] [color={rgb, 255:red, 0; green, 0; blue, 0 }  ][fill={rgb, 255:red, 0; green, 0; blue, 0 }  ][line width=0.75]      (0, 0) circle [x radius= 3.35, y radius= 3.35]   ;
\draw [shift={(50,160)}, rotate = 309.17] [color={rgb, 255:red, 0; green, 0; blue, 0 }  ][fill={rgb, 255:red, 0; green, 0; blue, 0 }  ][line width=0.75]      (0, 0) circle [x radius= 3.35, y radius= 3.35]   ;
%Straight Lines [id:da7187681043638781] 
\draw    (73,160) -- (86,144.04) ;
\draw [shift={(86,144.04)}, rotate = 309.17] [color={rgb, 255:red, 0; green, 0; blue, 0 }  ][fill={rgb, 255:red, 0; green, 0; blue, 0 }  ][line width=0.75]      (0, 0) circle [x radius= 3.35, y radius= 3.35]   ;
\draw [shift={(73,160)}, rotate = 309.17] [color={rgb, 255:red, 0; green, 0; blue, 0 }  ][fill={rgb, 255:red, 0; green, 0; blue, 0 }  ][line width=0.75]      (0, 0) circle [x radius= 3.35, y radius= 3.35]   ;
%Straight Lines [id:da0904767212465516] 
\draw [color={rgb, 255:red, 155; green, 155; blue, 155 }  ,draw opacity=1 ]   (152,118.04) -- (208,118.04) ;
%Straight Lines [id:da8112713117460995] 
\draw [color={rgb, 255:red, 155; green, 155; blue, 155 }  ,draw opacity=1 ]   (175,66.04) -- (208,66.04) ;
%Straight Lines [id:da8711718170026372] 
\draw [color={rgb, 255:red, 155; green, 155; blue, 155 }  ,draw opacity=1 ] [dash pattern={on 4.5pt off 4.5pt}]  (190,68.04) -- (190,116.04) ;
\draw [shift={(190,118.04)}, rotate = 270] [fill={rgb, 255:red, 155; green, 155; blue, 155 }  ,fill opacity=1 ][line width=0.08]  [draw opacity=0] (12,-3) -- (0,0) -- (12,3) -- cycle    ;
\draw [shift={(190,66.04)}, rotate = 90] [fill={rgb, 255:red, 155; green, 155; blue, 155 }  ,fill opacity=1 ][line width=0.08]  [draw opacity=0] (12,-3) -- (0,0) -- (12,3) -- cycle    ;
%Straight Lines [id:da737172554421827] 
\draw    (63,144.04) -- (86,144.04) ;
\draw [shift={(86,144.04)}, rotate = 0] [color={rgb, 255:red, 0; green, 0; blue, 0 }  ][fill={rgb, 255:red, 0; green, 0; blue, 0 }  ][line width=0.75]      (0, 0) circle [x radius= 3.35, y radius= 3.35]   ;
\draw [shift={(63,144.04)}, rotate = 0] [color={rgb, 255:red, 0; green, 0; blue, 0 }  ][fill={rgb, 255:red, 0; green, 0; blue, 0 }  ][line width=0.75]      (0, 0) circle [x radius= 3.35, y radius= 3.35]   ;
%Straight Lines [id:da9264832738509183] 
\draw    (66,108) -- (89,108) ;
\draw [shift={(89,108)}, rotate = 0] [color={rgb, 255:red, 0; green, 0; blue, 0 }  ][fill={rgb, 255:red, 0; green, 0; blue, 0 }  ][line width=0.75]      (0, 0) circle [x radius= 3.35, y radius= 3.35]   ;
\draw [shift={(66,108)}, rotate = 0] [color={rgb, 255:red, 0; green, 0; blue, 0 }  ][fill={rgb, 255:red, 0; green, 0; blue, 0 }  ][line width=0.75]      (0, 0) circle [x radius= 3.35, y radius= 3.35]   ;
%Straight Lines [id:da5329200824589426] 
\draw    (89,108) -- (112,108) ;
\draw [shift={(112,108)}, rotate = 0] [color={rgb, 255:red, 0; green, 0; blue, 0 }  ][fill={rgb, 255:red, 0; green, 0; blue, 0 }  ][line width=0.75]      (0, 0) circle [x radius= 3.35, y radius= 3.35]   ;
\draw [shift={(89,108)}, rotate = 0] [color={rgb, 255:red, 0; green, 0; blue, 0 }  ][fill={rgb, 255:red, 0; green, 0; blue, 0 }  ][line width=0.75]      (0, 0) circle [x radius= 3.35, y radius= 3.35]   ;
%Straight Lines [id:da2773689191142634] 
\draw    (66,108) -- (79,92.04) ;
\draw [shift={(79,92.04)}, rotate = 309.17] [color={rgb, 255:red, 0; green, 0; blue, 0 }  ][fill={rgb, 255:red, 0; green, 0; blue, 0 }  ][line width=0.75]      (0, 0) circle [x radius= 3.35, y radius= 3.35]   ;
\draw [shift={(66,108)}, rotate = 309.17] [color={rgb, 255:red, 0; green, 0; blue, 0 }  ][fill={rgb, 255:red, 0; green, 0; blue, 0 }  ][line width=0.75]      (0, 0) circle [x radius= 3.35, y radius= 3.35]   ;
%Straight Lines [id:da10009334559985272] 
\draw    (89,108) -- (102,92.04) ;
\draw [shift={(102,92.04)}, rotate = 309.17] [color={rgb, 255:red, 0; green, 0; blue, 0 }  ][fill={rgb, 255:red, 0; green, 0; blue, 0 }  ][line width=0.75]      (0, 0) circle [x radius= 3.35, y radius= 3.35]   ;
\draw [shift={(89,108)}, rotate = 309.17] [color={rgb, 255:red, 0; green, 0; blue, 0 }  ][fill={rgb, 255:red, 0; green, 0; blue, 0 }  ][line width=0.75]      (0, 0) circle [x radius= 3.35, y radius= 3.35]   ;
%Straight Lines [id:da20112614980324728] 
\draw    (79,92.04) -- (102,92.04) ;
\draw [shift={(102,92.04)}, rotate = 0] [color={rgb, 255:red, 0; green, 0; blue, 0 }  ][fill={rgb, 255:red, 0; green, 0; blue, 0 }  ][line width=0.75]      (0, 0) circle [x radius= 3.35, y radius= 3.35]   ;
\draw [shift={(79,92.04)}, rotate = 0] [color={rgb, 255:red, 0; green, 0; blue, 0 }  ][fill={rgb, 255:red, 0; green, 0; blue, 0 }  ][line width=0.75]      (0, 0) circle [x radius= 3.35, y radius= 3.35]   ;

% Text Node
\draw (192,92.04) node [anchor=west] [inner sep=0.75pt]    {$d$};
% Text Node
\draw (114,75.4) node [anchor=north west][inner sep=0.75pt]    {$A$};
% Text Node
\draw (43,115.4) node [anchor=north west][inner sep=0.75pt]    {$V$};
% Text Node
\draw (51,172) node [anchor=north west][inner sep=0.75pt]   [align=left] {$\displaystyle N$ points};


\end{tikzpicture}

    \caption{The volume between two areas on adjacent lattice planes}
    \label{fig:volume-between}
\end{figure}

\paragraph{Solution} 
Below I use the side length of the so-called cubes as the length unit.
\begin{itemize}
\item[(a)] Based-centered cubic is a Bravais lattice but the name is not on the list of 14. 
It is not really cubic, 
because it doesn't have the rotational symmetry around the $x$ and $y$ axes.
It's actually a simple tetragonal lattice with $a= 1 / \sqrt{2}, c = 1$, and
the primitive vectors are 
\begin{equation}
    \vb*{a}_1 = (1/2, -1/2, 0), \quad \vb*{a}_2 = (1/2, 1/2, 0), \quad \vb*{a}_3 = (0, 0, 1).
    \label{eq:based-centered-cubic-vector}
\end{equation}
\item[(b)] Similar to the first case, 
side-centered cubic is also a Bravais lattice but not cubic,
because it also doesn't have the rotational symmetry around the $x$ and $y$ axes
and therefore is not really ``cubic''.
It's actually a body-centered tetragonal lattice with $a= 1 / \sqrt{2}, c = 1$.
The primitive vectors are the same with \eqref{eq:based-centered-cubic-vector}.
\item[(c)] The edge-centered cubic lattice has the symmetry of a cube and therefore has to be a cubic lattice.
By counting the number of lattice points, 
we find \prettyref{fig:unit-cell-edge-centered} shows one unit cell.
No translation symmetry is able to turn one of the points into another,
so the unit cell is actually a primitive one.
Thus the edge-centered cubic lattice is actually a simple cubic lattice: 
in each primitive unit cell,
there are four points.
So the edge-centered cubic lattice is not a Bravais lattice.
It's the simple cubic lattice plus the basis shown as black (or red, or blue) in 
\prettyref{fig:unit-cell-edge-centered}.
\end{itemize}

\begin{figure}
    \centering
    


\begin{tikzpicture}[x=0.75pt,y=0.75pt,yscale=-1,xscale=1]
%uncomment if require: \path (0,300); %set diagram left start at 0, and has height of 300

%Shape: Cube [id:dp27906377889622136] 
\draw   (100,124.66) -- (129.29,95.38) -- (198,95.38) -- (198,163.71) -- (168.71,193) -- (100,193) -- cycle ; \draw   (198,95.38) -- (168.71,124.66) -- (100,124.66) ; \draw   (168.71,124.66) -- (168.71,193) ;
%Straight Lines [id:da8949678203128855] 
\draw    (198,95.38) -- (198,163.71) ;
%Straight Lines [id:da49535075513935123] 
\draw  [dash pattern={on 4.5pt off 4.5pt}]  (129.29,95.38) -- (129.29,163.71) ;
%Straight Lines [id:da9925075456381425] 
\draw  [dash pattern={on 4.5pt off 4.5pt}]  (198,163.71) -- (129.29,163.71) ;
%Straight Lines [id:da4763063388834874] 
\draw  [dash pattern={on 4.5pt off 4.5pt}]  (100,193) -- (129.29,163.71) ;
%Straight Lines [id:da6235209425906698] 
\draw    (168.71,124.66) ;
\draw [shift={(168.71,124.66)}, rotate = 0] [color={rgb, 255:red, 0; green, 0; blue, 0 }  ][fill={rgb, 255:red, 0; green, 0; blue, 0 }  ][line width=0.75]      (0, 0) circle [x radius= 3.35, y radius= 3.35]   ;
%Straight Lines [id:da6830253639832535] 
\draw    (136,124.66) ;
\draw [shift={(136,124.66)}, rotate = 0] [color={rgb, 255:red, 0; green, 0; blue, 0 }  ][fill={rgb, 255:red, 0; green, 0; blue, 0 }  ][line width=0.75]      (0, 0) circle [x radius= 3.35, y radius= 3.35]   ;
%Straight Lines [id:da3555449792412686] 
\draw    (182.71,110.66) ;
\draw [shift={(182.71,110.66)}, rotate = 0] [color={rgb, 255:red, 0; green, 0; blue, 0 }  ][fill={rgb, 255:red, 0; green, 0; blue, 0 }  ][line width=0.75]      (0, 0) circle [x radius= 3.35, y radius= 3.35]   ;
%Straight Lines [id:da30048729914335426] 
\draw    (168.71,157.66) ;
\draw [shift={(168.71,157.66)}, rotate = 0] [color={rgb, 255:red, 0; green, 0; blue, 0 }  ][fill={rgb, 255:red, 0; green, 0; blue, 0 }  ][line width=0.75]      (0, 0) circle [x radius= 3.35, y radius= 3.35]   ;
%Straight Lines [id:da39538770796014844] 
\draw [color={rgb, 255:red, 208; green, 2; blue, 27 }  ,draw opacity=1 ]   (99.71,124.66) ;
\draw [shift={(99.71,124.66)}, rotate = 0] [color={rgb, 255:red, 208; green, 2; blue, 27 }  ,draw opacity=1 ][fill={rgb, 255:red, 208; green, 2; blue, 27 }  ,fill opacity=1 ][line width=0.75]      (0, 0) circle [x radius= 3.35, y radius= 3.35]   ;
%Straight Lines [id:da5417236582392115] 
\draw [color={rgb, 255:red, 208; green, 2; blue, 27 }  ,draw opacity=1 ]   (67,124.66) ;
\draw [shift={(67,124.66)}, rotate = 0] [color={rgb, 255:red, 208; green, 2; blue, 27 }  ,draw opacity=1 ][fill={rgb, 255:red, 208; green, 2; blue, 27 }  ,fill opacity=1 ][line width=0.75]      (0, 0) circle [x radius= 3.35, y radius= 3.35]   ;
%Straight Lines [id:da7317973769989081] 
\draw [color={rgb, 255:red, 208; green, 2; blue, 27 }  ,draw opacity=1 ]   (113.71,110.66) ;
\draw [shift={(113.71,110.66)}, rotate = 0] [color={rgb, 255:red, 208; green, 2; blue, 27 }  ,draw opacity=1 ][fill={rgb, 255:red, 208; green, 2; blue, 27 }  ,fill opacity=1 ][line width=0.75]      (0, 0) circle [x radius= 3.35, y radius= 3.35]   ;
%Straight Lines [id:da6579080343176826] 
\draw [color={rgb, 255:red, 208; green, 2; blue, 27 }  ,draw opacity=1 ]   (99.71,157.66) ;
\draw [shift={(99.71,157.66)}, rotate = 0] [color={rgb, 255:red, 208; green, 2; blue, 27 }  ,draw opacity=1 ][fill={rgb, 255:red, 208; green, 2; blue, 27 }  ,fill opacity=1 ][line width=0.75]      (0, 0) circle [x radius= 3.35, y radius= 3.35]   ;

%Straight Lines [id:da3609355877334228] 
\draw [color={rgb, 255:red, 74; green, 144; blue, 226 }  ,draw opacity=1 ]   (168.71,192.66) ;
\draw [shift={(168.71,192.66)}, rotate = 0] [color={rgb, 255:red, 74; green, 144; blue, 226 }  ,draw opacity=1 ][fill={rgb, 255:red, 74; green, 144; blue, 226 }  ,fill opacity=1 ][line width=0.75]      (0, 0) circle [x radius= 3.35, y radius= 3.35]   ;
%Straight Lines [id:da7479710636712822] 
\draw [color={rgb, 255:red, 74; green, 144; blue, 226 }  ,draw opacity=1 ]   (136,192.66) ;
\draw [shift={(136,192.66)}, rotate = 0] [color={rgb, 255:red, 74; green, 144; blue, 226 }  ,draw opacity=1 ][fill={rgb, 255:red, 74; green, 144; blue, 226 }  ,fill opacity=1 ][line width=0.75]      (0, 0) circle [x radius= 3.35, y radius= 3.35]   ;
%Straight Lines [id:da2086207673220617] 
\draw [color={rgb, 255:red, 74; green, 144; blue, 226 }  ,draw opacity=1 ]   (182.71,178.66) ;
\draw [shift={(182.71,178.66)}, rotate = 0] [color={rgb, 255:red, 74; green, 144; blue, 226 }  ,draw opacity=1 ][fill={rgb, 255:red, 74; green, 144; blue, 226 }  ,fill opacity=1 ][line width=0.75]      (0, 0) circle [x radius= 3.35, y radius= 3.35]   ;
%Straight Lines [id:da8842065819921627] 
\draw [color={rgb, 255:red, 74; green, 144; blue, 226 }  ,draw opacity=1 ]   (168.71,225.66) ;
\draw [shift={(168.71,225.66)}, rotate = 0] [color={rgb, 255:red, 74; green, 144; blue, 226 }  ,draw opacity=1 ][fill={rgb, 255:red, 74; green, 144; blue, 226 }  ,fill opacity=1 ][line width=0.75]      (0, 0) circle [x radius= 3.35, y radius= 3.35]   ;

%Straight Lines [id:da3182715597136758] 
\draw [color={rgb, 255:red, 80; green, 227; blue, 194 }  ,draw opacity=1 ]   (100.71,192.66) ;
\draw [shift={(100.71,192.66)}, rotate = 0] [color={rgb, 255:red, 80; green, 227; blue, 194 }  ,draw opacity=1 ][fill={rgb, 255:red, 80; green, 227; blue, 194 }  ,fill opacity=1 ][line width=0.75]      (0, 0) circle [x radius= 3.35, y radius= 3.35]   ;
%Straight Lines [id:da09238987837864565] 
\draw [color={rgb, 255:red, 80; green, 227; blue, 194 }  ,draw opacity=1 ]   (68,192.66) ;
\draw [shift={(68,192.66)}, rotate = 0] [color={rgb, 255:red, 80; green, 227; blue, 194 }  ,draw opacity=1 ][fill={rgb, 255:red, 80; green, 227; blue, 194 }  ,fill opacity=1 ][line width=0.75]      (0, 0) circle [x radius= 3.35, y radius= 3.35]   ;
%Straight Lines [id:da024868637558111972] 
\draw [color={rgb, 255:red, 80; green, 227; blue, 194 }  ,draw opacity=1 ]   (114.71,178.66) ;
\draw [shift={(114.71,178.66)}, rotate = 0] [color={rgb, 255:red, 80; green, 227; blue, 194 }  ,draw opacity=1 ][fill={rgb, 255:red, 80; green, 227; blue, 194 }  ,fill opacity=1 ][line width=0.75]      (0, 0) circle [x radius= 3.35, y radius= 3.35]   ;
%Straight Lines [id:da6333331707155729] 
\draw [color={rgb, 255:red, 80; green, 227; blue, 194 }  ,draw opacity=1 ]   (100.71,225.66) ;
\draw [shift={(100.71,225.66)}, rotate = 0] [color={rgb, 255:red, 80; green, 227; blue, 194 }  ,draw opacity=1 ][fill={rgb, 255:red, 80; green, 227; blue, 194 }  ,fill opacity=1 ][line width=0.75]      (0, 0) circle [x radius= 3.35, y radius= 3.35]   ;

%Straight Lines [id:da5979371349789433] 
\draw [color={rgb, 255:red, 208; green, 2; blue, 27 }  ,draw opacity=1 ]   (165.71,102.66) -- (99,102.66) ;
\draw [shift={(97,102.66)}, rotate = 360] [fill={rgb, 255:red, 208; green, 2; blue, 27 }  ,fill opacity=1 ][line width=0.08]  [draw opacity=0] (12,-3) -- (0,0) -- (12,3) -- cycle    ;
%Straight Lines [id:da9138705372611224] 
\draw [color={rgb, 255:red, 80; green, 227; blue, 194 }  ,draw opacity=1 ]   (161.71,119.66) -- (94.42,186.59) ;
\draw [shift={(93,188)}, rotate = 315.16] [fill={rgb, 255:red, 80; green, 227; blue, 194 }  ,fill opacity=1 ][line width=0.08]  [draw opacity=0] (12,-3) -- (0,0) -- (12,3) -- cycle    ;
%Straight Lines [id:da42187668274599144] 
\draw [color={rgb, 255:red, 74; green, 144; blue, 226 }  ,draw opacity=1 ]   (192.71,124.66) -- (192.71,191) ;
\draw [shift={(192.71,193)}, rotate = 270] [fill={rgb, 255:red, 74; green, 144; blue, 226 }  ,fill opacity=1 ][line width=0.08]  [draw opacity=0] (12,-3) -- (0,0) -- (12,3) -- cycle    ;




\end{tikzpicture}

    \caption{One unit cell of a edge centered cubic lattice}
    \label{fig:unit-cell-edge-centered}
\end{figure}

\paragraph{Solution} In each conventional cell of the bcc lattice, 
there are 2 lattice points, and so is the case for hcp.
The volume of a conventional cell of the bcc lattice is $a^3$.
In the hcp lattice, $c = 2 \sqrt{6} a / 3$,
so the volume is 
\[
    \frac{\sqrt{3}}{2} a^2 \times c = \sqrt{2} a^3.
\]
So to keep the density invariant,
\[
    \sqrt{2} a_\text{hcp}^3 = a_{\text{bcc}}^3,
\]
and 
\begin{equation}
    a_{\text{hcp}} = 2^{-1/6} a_{\text{bcc}} = \SI{3.77}{\angstrom}.
\end{equation}

\paragraph{Solution} \begin{itemize}
\item[(a)] There is only one Ti atom in the cube given, 
and obviously it's impossible to use a translation symmetry operation 
to turn one atom into another in the cube.
The lattice is a simple cubic lattice,
and the primitive lattice vectors are 
\begin{equation}
    \vb*{a}_1 = (1, 0, 0), \quad \vb*{a}_2 = (0, 1, 0), \quad \vb*{a}_3 = (0, 0, 1).
\end{equation}
\item[(b)] There are five: $1/4 \times 4 = 1$ Sr atom, $1$ Ti atom, and 
$1/2 \times 6 = 3$ O atoms.
The coordinates of the Ti atom are $(1/2, 1/2, 1/2)$.
The coordinates of the Sr atom are $(0, 0, 0)$.
The coordinates of the O atoms are $(1/2, 1/2, 0), (0, 1/2, 1/2), (1/2, 0, 1/2)$.
\item[(c)] The nearest neighbors of a Sr atom are 12 Ti atoms.
The nearest neighbors of a Ti atom are 6 O atoms.
The nearest neighbors of an O atom are 2 Ti atoms.
\end{itemize}

\paragraph{Solution} 

\end{document}