\documentclass[hyperref, a4paper]{article}

\usepackage{geometry}
\usepackage{titling}
\usepackage{titlesec}
% No longer needed, since we will use enumitem package
% \usepackage{paralist}
\usepackage{enumitem}
\usepackage{footnote}
\usepackage{enumerate}
\usepackage{amsmath, amssymb, amsthm}
\usepackage{mathtools}
\usepackage{bbm}
\usepackage{cite}
\usepackage{graphicx}
\usepackage{subfigure}
\usepackage{physics}
\usepackage{tensor}
\usepackage{siunitx}
\usepackage[version=4]{mhchem}
\usepackage{tikz}
\usepackage{xcolor}
\usepackage{listings}
\usepackage{autobreak}
\usepackage[ruled, vlined, linesnumbered]{algorithm2e}
\usepackage{nameref,zref-xr}
\zxrsetup{toltxlabel}
\usepackage[colorlinks,unicode]{hyperref} % , linkcolor=black, anchorcolor=black, citecolor=black, urlcolor=black, filecolor=black
\usepackage[most]{tcolorbox}
\usepackage{prettyref}

% Page style
\geometry{left=3.18cm,right=3.18cm,top=2.54cm,bottom=2.54cm}
\titlespacing{\paragraph}{0pt}{1pt}{10pt}[20pt]
\setlength{\droptitle}{-5em}
%\preauthor{\vspace{-10pt}\begin{center}}
%\postauthor{\par\end{center}}

% More compact lists 
\setlist[itemize]{
    itemindent=17pt, 
    leftmargin=1pt,
    listparindent=\parindent,
    parsep=0pt,
}

% Math operators
\DeclareMathOperator{\timeorder}{\mathcal{T}}
\DeclareMathOperator{\diag}{diag}
\DeclareMathOperator{\legpoly}{P}
\DeclareMathOperator{\primevalue}{P}
\DeclareMathOperator{\sgn}{sgn}
\newcommand*{\ii}{\mathrm{i}}
\newcommand*{\ee}{\mathrm{e}}
\newcommand*{\const}{\mathrm{const}}
\newcommand*{\suchthat}{\quad \text{s.t.} \quad}
\newcommand*{\argmin}{\arg\min}
\newcommand*{\argmax}{\arg\max}
\newcommand*{\normalorder}[1]{: #1 :}
\newcommand*{\pair}[1]{\langle #1 \rangle}
\newcommand*{\fd}[1]{\mathcal{D} #1}
\DeclareMathOperator{\bigO}{\mathcal{O}}

% TikZ setting
\usetikzlibrary{arrows,shapes,positioning}
\usetikzlibrary{arrows.meta}
\usetikzlibrary{decorations.markings}
\tikzstyle arrowstyle=[scale=1]
\tikzstyle directed=[postaction={decorate,decoration={markings,
    mark=at position .5 with {\arrow[arrowstyle]{stealth}}}}]
\tikzstyle ray=[directed, thick]
\tikzstyle dot=[anchor=base,fill,circle,inner sep=1pt]

% Algorithm setting
% Julia-style code
\SetKwIF{If}{ElseIf}{Else}{if}{}{elseif}{else}{end}
\SetKwFor{For}{for}{}{end}
\SetKwFor{While}{while}{}{end}
\SetKwProg{Function}{function}{}{end}
\SetArgSty{textnormal}

\newcommand*{\concept}[1]{{\textbf{#1}}}

% Embedded codes
\lstset{basicstyle=\ttfamily,
  showstringspaces=false,
  commentstyle=\color{gray},
  keywordstyle=\color{blue}
}

% Reference formatting
\newrefformat{fig}{Figure~\ref{#1}}

% Color boxes
\tcbuselibrary{skins, breakable, theorems}
\newtcbtheorem[number within=section]{warning}{Warning}%
  {colback=orange!5,colframe=orange!65,fonttitle=\bfseries, breakable}{warn}
\newtcbtheorem[number within=section]{note}{Note}%
  {colback=green!5,colframe=green!65,fonttitle=\bfseries, breakable}{note}
\newtcbtheorem[number within=section]{info}{Info}%
  {colback=blue!5,colframe=blue!65,fonttitle=\bfseries, breakable}{info}

\newenvironment{shelldisplay}{\begin{lstlisting}}{\end{lstlisting}}

\newcommand{\address}[1]{\href{#1}{\url{#1}}}

\title{Homework 6}
\author{Jinyuan Wu}

\begin{document}

\maketitle

\paragraph{Problem 1} 

\paragraph{Solution} The mistakes are
\begin{itemize}
    \item Electrons change their directions when they are far from nuclei,
    and this may symbolize the Coulomb interaction between electrons,
    which is ignored in Drude model.
    \item I'm not sure whether I understand the video correctly,
    but I notice in the video the collisions are illustrated in a quite ``realistic'' way:
    $\theta_{\text{in}} = \theta_{\text{out}}$,
    if the distance between an electron and an ion is large, then $\Delta \vb*{v}$ isn't large, etc.
    That's not how Drude model sees collisions:
    In Drude model there is not correlation between the velocities before and after a collision.
\end{itemize}

\paragraph{Problem 2} Problem 2 of chapter 1 of A\&M.

\paragraph{Solution} \begin{itemize}
\item[(a)] Suppose $\vb*{p}_{\text{i}}$ is the momentum after the first collision,
and $\vb*{p}_{\text{f}}$ is the momentum after the second collision.
The energy lost in the second collision, on average, is 
\[
    \begin{aligned}
        \Delta E &= \expval{\frac{(\vb*{p}_{\text{i}} + e \vb*{E} t)^2}{2m}} 
        - \expval{\frac{\vb*{p}_{\text{f}}^2}{2m}} \\
        &= \expval{\frac{\vb*{p}_{\text{i}}^2}{2m}} - \expval{\frac{\vb*{p}_{\text{f}}^2}{2m}}
        + \frac{e \vb*{E} t}{m} \cdot \expval{\vb*{p}_{\text{i}}} 
        + \frac{(e \vb*{E} t)^2}{2m}.
    \end{aligned}
\]
Because we assume after the collision the particle is in thermal equilibrium,
the first two terms cancel each other,
and the third term is zero because of spatial mirror symmetry.
So the only remaining term is the last term, and we have 
\begin{equation}
    \Delta E = \frac{(e \vb*{E} t)^2}{2m}.
\end{equation}

\item[(b)] Now $t$ is a random variable and the expected energy loss per electron per collision is 
\[
    \begin{aligned}
        &\quad \int_{0}^\infty \frac{\dd{t}}{\tau} \ee^{- t / \tau} 
        \frac{(e \vb*{E} t)^2}{2m} \\
        &= \frac{e^2 \vb*{E}^2}{2m \tau} \cdot \tau^3 \int_{0}^\infty \frac{\dd{t} t^2}{\tau^3} \ee^{- t / \tau} \\
        &= \frac{e^2 \vb*{E}^2}{2m \tau} \cdot \tau^3 \cdot \int_{0}^\infty \dd{x} x^2 \ee^{-x} \\
        &= \frac{e^2 \vb*{E}^2}{2m \tau} \cdot \tau^3 \cdot 2 = \frac{(e \vb*{E} \tau)^2}{m}.
    \end{aligned} 
\]
So the average energy loss per electron per collision is $(e \vb*{E} \tau)^2 / m$.
\begin{equation}
    \dv{t} \dv{E_{\text{loss}}}{V} = \frac{(e \vb*{E} \tau)^2}{m} \cdot \frac{n}{\tau} 
    = \left( \frac{n e^2 \tau}{m} \right) \vb*{E} = \sigma \vb*{E}^2.
\end{equation}

For a wire of length $L$ and cross section $A$, the current density is 
\[
    j = \frac{I}{A},
\]
so we have 
\[
    E = \frac{j}{\sigma} = \frac{I}{A \sigma}.
\]
Then 
\begin{equation}
    P_{\text{loss}} = \int \dd[3]{\vb*{r}} \sigma \vb*{E}^2 
    = AL \sigma \frac{I^2}{A^2 \sigma^2} = \frac{L}{A} \frac{1}{\sigma} I^2 = I^2 R,
\end{equation}
where 
\[
    R = \frac{L}{A} \rho = \frac{L}{A} \frac{1}{\sigma}.
\]

\end{itemize}

\paragraph{Problem 3} The ionosphere consists of a plasma of electrons and positive ions. It acts like a Drude metal when interacting with electromagnetic waves. AM radio frequencies (up to $1600 \mathrm{kHz}$ ) are reflected by the ionosphere (so you can bounce them over the horizon for long range radio transmissions) while FM frequencies (starting at $88 \mathrm{MHz}$ ) pass through the ionosphere.
a) In what range is the plasma energy (in eV) of the ionosphere?
b) For the larger of those values, what would be the electron density in the ionosphere? To get a sense of scale, compare the mean distance between electrons in this plasma to that of the conduction electrons in solid $\mathrm{Na}$ at STP as well as to the mean distance between molecules in air at STP (STP means $1 \mathrm{~atm}$ and $298 \mathrm{~K}$ ). Define the mean distance between particles as the length of the side of a cube whose volume contains one particle on average.

\paragraph{Solution} \begin{itemize}
\item[(a)] It must be between \SI{1600}{kHz} and \SI{88}{MHz}:
when $\omega < \omega_{\text{p}}$,
the system looks like a mirror to an external beam of light,
and when $\omega > \omega_{\text{p}}$,
the system is transparent.
So in \si{eV}, the range is 
(here the numbers \SI{1600}{kHz} and \SI{88}{MHz} are all ordinary frequencies, not angular frequencies)
between \SI{4.2e-8}{eV} and \SI{2.3e-6}{eV}.

\item[(b)] We have 
\begin{equation}
    \omega_{\text{p}}^2 = \frac{e^2 n}{m \epsilon_0}.
\end{equation}
So when we estimate $\omega_{\text{p}}$ to be $2\pi \cdot \SI{88}{MHz}$, we have 
\begin{equation}
    n = \frac{m \epsilon_0}{e^2} \omega_{\text{p}}^2 = \SI{9.6e13}{m^{-3}}.
\end{equation}
So the mean distance can be estimated as 
\begin{equation}
    l = \sqrt[3]{\frac{1}{n}} = \SI{2.2e-5}{m}.
\end{equation} 

The mass density of solid Na is \SI{0.968}{g/cm^3}, 
and the mass of one Na atom is \SI{22.989769}{u}.
So the number density of Na atoms -- and also the electrons -- is 
$\SI{0.968}{g/cm^3} / \SI{22.989769}{u} = \SI{2.5e28}{m^{-3}}$.
So the mean distance is $(1 / n)^{1/3} = \SI{3.4e-10}{m}$.

The number density of air is 
\[
    \frac{N}{V} = \frac{p}{k_{\text{B}} T} = \SI{2.46e25}{m^{-3}},
\]
and the mean distance is $(1/n)^{1/3} = \SI{3.4e-9}{m}$.

So air is much dilute than solid Na, but ionosphere is much, much more dilute than the first two.

\end{itemize}

\paragraph{Problem 4} Problem 1 of chapter 2 of A\&M.

\paragraph{Solution} \begin{itemize}
\item[(a)] By definition, 
\begin{equation}
    n = \frac{N}{L^2} = \frac{1}{L^2} \cdot 2 \frac{L^2}{(2\pi)^2} \pi k_{\text{F}}^2
    = \frac{k_{\text{F}}^2}{2\pi}.
\end{equation}
\item[(b)] By counting electrons we have 
\[
    N = 2 \frac{L^2}{(2\pi)^2} \pi k_{\text{F}}^2 = \frac{L^2}{\pi r_{\text{s}}^2},
\]
so 
\begin{equation}
    k_{\text{F}} = \frac{\sqrt{2}}{r_{\text{s}}}.
\end{equation}

\item[(c)] By definition
\[
    \begin{aligned}
        g(E) &= 2 \cdot \frac{1}{(2\pi)^2} \int_{\epsilon_{\vb*{k}} = E} 
        \frac{\dd{l_{\vb*{k}}}}{\abs{\grad_{\vb*{k}} \epsilon_{\vb*{k}}}} \\
        &= 2 \cdot \frac{1}{(2\pi)^2} \eval{\frac{2\pi k}{\frac{\hbar^2 k}{m}}}_{ E = \frac{\hbar^2 k^2}{2m} } \\
        &= \begin{cases}
            \frac{m}{\pi \hbar^2}, & \quad E \geq 0, \\
            0                    , & \quad E < 0.
        \end{cases}
    \end{aligned}
\]
When $E < 0$, it's impossible to find a $k$ such that $E = \hbar^2 k^2 / 2m$,
so there is a cutoff.
So when $E > 0$, $g(E)$ is a constant $m / \pi \hbar^2$.

\item[(d)] Since $g(E)$ is a constant when $E > 0$,
from (C.7) in A\&M we have (here the function $H(E)$ is just $g(E)$)
\begin{equation}
    n = \int_{-\infty}^\infty g(E) f(E) \dd{E}  = \int_{-\infty}^\mu g(E) \dd{E} = 
    \frac{m}{\pi \hbar^2} \cdot \mu.
\end{equation}
All the following terms containing derivatives of $g(E)$ vanish, 
so even at a finite temperature,
$n$ doesn't have temperature dependence.

On the other hand, when $T = 0$, we have 
\begin{equation}
    n = \int_{-\infty}^{\epsilon_{\text{F}}} g(E) \dd{E} = \frac{m}{\pi \hbar^2} \cdot \epsilon_{\text{F}}.
\end{equation}
So we find 
\begin{equation}
    \mu = \epsilon_{\text{F}}.
    \label{eq:wrong-mu}
\end{equation}

\item[(e)] The spatial density of electrons is  
\[
    n = \int_{-\infty}^\infty g(E) f(E) \dd{E} 
    = \frac{m}{\pi \hbar^2} \int_{0}^\infty \frac{\dd{E}}{1 + \ee^{\beta (E - \mu)}}.
\]
The integral can be evaluated as 
\[
    \int_{- \mu}^\infty \dd{\xi} \frac{1}{1 + \ee^{\beta \xi}} = 
    \int_{- \mu}^\infty \dd{\xi} \frac{\ee^{- \beta \xi}}{1 + \ee^{- \beta \xi}} =
    - \frac{1}{\beta} \int_{\ee^{\beta \mu}}^0 \frac{\dd{x}}{1 + x} 
    = \frac{1}{\beta} \ln (1 + \ee^{\beta \mu}),
\]
so we get 
\[
    n = \frac{m}{\pi \hbar} \frac{1}{\beta} \ln (1 + \ee^{\beta \mu}).
\]
On the other hand, at the zero temperature, we have 
\[
    n = \frac{m}{\pi \hbar^2} \cdot \epsilon_{\text{F}}.
\]
Changing the temperature doesn't change the number of electrons, so we get 
\begin{equation}
    \begin{aligned}
        \epsilon_{\text{F}} &= \frac{1}{\beta} \ln (1 + \ee^{\beta \mu}) \\
        &= k_{\text{B}} T \left(
            \ln (\ee^{- \mu / k_{\text{B}} T} + 1) + \frac{\mu}{k_{\text{B}} T}
        \right) \\
        &= \mu + k_{\text{B}} T \ln (\ee^{- \mu / k_{\text{B}} T} + 1).
    \end{aligned}
    \label{eq:correct-mu}
\end{equation}

\item[(f)] The relative error of the value of $\mu$ in \eqref{eq:wrong-mu},
according to \eqref{eq:correct-mu} is 
\begin{equation}
    \frac{1}{\mu} k_{\text{B}} T \ln (\ee^{- \mu / k_{\text{B}} T} + 1) \eqqcolon
    f(k_{\text{B}} T / \mu), \quad f(x) = x \ln (\ee^{- 1/x} + 1).
\end{equation}
When $x \to \infty$, the behavior of $f(x)$ is quite normal: $\ee^{-1/x} \to 1$ and $f(x) \approx x \ln 2$.
Around $x = 0$, however, we don't have a Taylor expansion of the function.
This can be seen by plotting the function,
which is almost complete flat in the $0 < x < 0.5$ region,
and this can also be seen by doing Taylor expansion with respect to $\ee^{- 1 / x}$, 
which is small when $x \to 0$: we have 
\[
    f(x) \approx x \ee^{- 1 / x},
\]
which is also almost completely zero around $x = 0$ and also doesn't have a Taylor expansion. 
So the conclusions are:
\begin{itemize}
    \item It's safe to use $\mu = \epsilon_{\text{F}}$ as an approximation,
    because even when $k_{\text{B}} T \sim \mu / 2$, 
    the relative error is still very small.
    \item The reason $\mu = \epsilon_{\text{F}}$ is not \SI{100}{\percent} correct 
    is because $n(T)$ -- and hence $\mu(T)$ -- doesn't have a Taylor expansion near $T = 0$,
    so we don't have the Sommerfield expansion in the first place.
\end{itemize} 

\end{itemize}

\paragraph{Problem 5} The gold atom has the ground state electron configuration $[X e] 4 f^{14} 5 d^{10} 6 s^1$.
(a) In solid gold, how many electrons do you think each Au atom would contribute to the "electron sea" that makes for the metallic behavior and holds the crystal together? Why?
(b) Using that number, determine the electron relaxation time (within the Drude model) of $\mathrm{Au}$ at $273 \mathrm{~K}$ using the resistivity in Table $1.2$ and lattice constant given in Table $4.1$ of A\&M.

\paragraph{Solution} \begin{itemize}
\item[(a)] One gold atom only contributes one electron, the 6s$^1$ one,
because the 4f and 5d shells are closed and electrons in them have lower energies 
when staying nears the nuclei instead of moving around.
\item[(b)] When $T = \SI{273}{K}$, $\rho = \SI{2.04e-8}{\ohm \cdot m}$.
The density of electrons is just the density of Au atoms, which is $1 / a^3$,
$a = \SI{4.08e-10}{m}$,
and therefore 
\begin{equation}
    \sigma = \frac{n e^2 \tau}{m}, \quad 
    \tau = \frac{\sigma m}{n e^2} = \frac{m a^3}{e^2 \rho} = \SI{1.2e-13}{s}.
\end{equation}
\end{itemize}

\paragraph{Problem 6} Show that for a 3D free electron gas within the Sommerfeld model at zero temperature, the kinetic energy per electron is $3 / 5$ the Fermi energy.

\paragraph{Solution} The total kinetic energy is 
\begin{equation}
    E = 2 \cdot \int_{0}^{k_{\text{F}}} \frac{V}{(2\pi)^3} 
    \cdot 4\pi k^2 \dd{k} \cdot \frac{\hbar^2 k^2}{2m} 
    = \frac{V}{(2\pi)^3} \frac{8\pi}{5} \frac{\hbar^2}{2m} k_{\text{F}}^5,
\end{equation}
and the total number of electrons is 
\begin{equation}
    N = 2 \cdot \int_{0}^{k_{\text{F}}} \frac{V}{(2\pi)^3} \cdot 4\pi k^2 \dd{k}
    = \frac{V}{(2\pi)^3} \frac{8\pi}{3} k_{\text{F}}^3.
\end{equation}
So the kinetic energy per electron is 
\begin{equation}
    \frac{E}{V} = \frac{3}{5} \frac{\hbar^2 k_{\text{F}}^2}{2m} = \frac{3}{5} \epsilon_{\text{F}}.
\end{equation}

\end{document}