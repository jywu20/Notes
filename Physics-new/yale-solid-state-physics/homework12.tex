\documentclass[hyperref, a4paper]{article}

\usepackage{geometry}
\usepackage{titling}
\usepackage{titlesec}
% No longer needed, since we will use enumitem package
% \usepackage{paralist}
\usepackage{enumitem}
\usepackage{footnote}
\usepackage{enumerate}
\usepackage{amsmath, amssymb, amsthm}
\usepackage{mathtools}
\usepackage{bbm}
\usepackage{cite}
\usepackage{graphicx}
\usepackage{subfigure}
\usepackage{physics}
\usepackage{tensor}
\usepackage{siunitx}
\usepackage[version=4]{mhchem}
\usepackage{tikz}
\usepackage{xcolor}
\usepackage{listings}
\usepackage{autobreak}
\usepackage[ruled, vlined, linesnumbered]{algorithm2e}
\usepackage{nameref,zref-xr}
\zxrsetup{toltxlabel}
\usepackage[colorlinks,unicode]{hyperref} % , linkcolor=black, anchorcolor=black, citecolor=black, urlcolor=black, filecolor=black
\usepackage[most]{tcolorbox}
\usepackage{prettyref}

% Page style
\geometry{left=3.18cm,right=3.18cm,top=2.54cm,bottom=2.54cm}
\titlespacing{\paragraph}{0pt}{1pt}{10pt}[20pt]
\setlength{\droptitle}{-5em}
%\preauthor{\vspace{-10pt}\begin{center}}
%\postauthor{\par\end{center}}

% More compact lists 
\setlist[itemize]{
    itemindent=17pt, 
    leftmargin=1pt,
    listparindent=\parindent,
    parsep=0pt,
}

% Math operators
\DeclareMathOperator{\timeorder}{\mathcal{T}}
\DeclareMathOperator{\diag}{diag}
\DeclareMathOperator{\legpoly}{P}
\DeclareMathOperator{\primevalue}{P}
\DeclareMathOperator{\sgn}{sgn}
\newcommand*{\ii}{\mathrm{i}}
\newcommand*{\ee}{\mathrm{e}}
\newcommand*{\const}{\mathrm{const}}
\newcommand*{\suchthat}{\quad \text{s.t.} \quad}
\newcommand*{\argmin}{\arg\min}
\newcommand*{\argmax}{\arg\max}
\newcommand*{\normalorder}[1]{: #1 :}
\newcommand*{\pair}[1]{\langle #1 \rangle}
\newcommand*{\fd}[1]{\mathcal{D} #1}
\DeclareMathOperator{\bigO}{\mathcal{O}}

% TikZ setting
\usetikzlibrary{arrows,shapes,positioning}
\usetikzlibrary{arrows.meta}
\usetikzlibrary{decorations.markings}
\tikzstyle arrowstyle=[scale=1]
\tikzstyle directed=[postaction={decorate,decoration={markings,
    mark=at position .5 with {\arrow[arrowstyle]{stealth}}}}]
\tikzstyle ray=[directed, thick]
\tikzstyle dot=[anchor=base,fill,circle,inner sep=1pt]

% Algorithm setting
% Julia-style code
\SetKwIF{If}{ElseIf}{Else}{if}{}{elseif}{else}{end}
\SetKwFor{For}{for}{}{end}
\SetKwFor{While}{while}{}{end}
\SetKwProg{Function}{function}{}{end}
\SetArgSty{textnormal}

\newcommand*{\concept}[1]{{\textbf{#1}}}

% Embedded codes
\lstset{basicstyle=\ttfamily,
  showstringspaces=false,
  commentstyle=\color{gray},
  keywordstyle=\color{blue}
}

% Reference formatting
\newrefformat{fig}{Figure~\ref{#1}}

% Color boxes
\tcbuselibrary{skins, breakable, theorems}
\newtcbtheorem[number within=section]{warning}{Warning}%
  {colback=orange!5,colframe=orange!65,fonttitle=\bfseries, breakable}{warn}
\newtcbtheorem[number within=section]{note}{Note}%
  {colback=green!5,colframe=green!65,fonttitle=\bfseries, breakable}{note}
\newtcbtheorem[number within=section]{info}{Info}%
  {colback=blue!5,colframe=blue!65,fonttitle=\bfseries, breakable}{info}

\newenvironment{shelldisplay}{\begin{lstlisting}}{\end{lstlisting}}

\newcommand{\address}[1]{\href{#1}{\url{#1}}}

\title{Homework 12}
\author{Jiinyuan Wu}

\begin{document}
    
\maketitle

\paragraph{Problem 1} A photon of energy $1.6 \mathrm{eV}$ is absorbed by GaAs at room temperature: in the process, an electron is excited from the heavy hole band to the conduction band. No phonons are involved. Calculate the energies and length of the wave vectors for the electron and hole involved. Use values from Kittel and the approximate dispersion relations on page 201 of Kittel for the heavy hole.

\paragraph{Solution} The data used are $m_{\text{e}} = 0.066m$, 
$m_{\text{h}} = 0.5 m$,
and the conservation laws are 
(note that in conservation laws, 
the energy and momentum of the hole 
have negative signs)
\begin{equation}
    E_{\text{c}} + \frac{p_{\text{e}}^2}{2m_{\text{e}}} - \mu + \mu - \left(E_{\text{v}} - \frac{p_{\text{h}}^2}{2m_{\text{h}}}\right)  = E_{\text{photon}}, \quad 
    p_{\text{e}} - p_{\text{h}} = \frac{E_{\text{photon}}}{c} ,
\end{equation}
and we get 


\paragraph{Problem 2} Why does a small concentration of donors or acceptors can have a big impact on the concentration of free carriers in Ge at room temperature $(300 \mathrm{~K})$ ?

\paragraph{Solution} Because the intrinsic carrier concentration is too small.
For \ce{Ge}, I consult \href{https://www.iue.tuwien.ac.at/phd/palankovski/node40.html}{this webpage},
and find the heavy hole mass is $m_{\text{ph}} = 0.29m$,
and the geometric average of the electron mass is 
$m_{\text{L}} = 0.222m$, 
and the band gap can be found from \href{https://en.wikipedia.org/wiki/Germanium}{Wikipedia},
which is \SI{0.67}{eV} when $T = \SI{300}{K}$,
so we have 
\begin{equation}
    n_{\mathrm{i}}=2\left(\frac{k_{\mathrm{B}} T}{2 \pi \hbar^2}\right)^{3 / 2}\left(m_{\mathrm{e}} m_{\mathrm{h}}\right)^{3 / 4} \exp \left(-E_{\mathrm{g}} / 2 k_{\mathrm{B}} T\right) = \SI{7.56e12}{cm^{-3}},
\end{equation}
which is small compared with usual acceptor or donor concentrations.

\paragraph{Problem 3} 3. [20 points] A silicon p-n junction tunnel diode is doped with equal concentrations of donors (n-side) and acceptors (p-side): $N_d=N_a$. What values of $N_a$ and $N_d$ are needed for the depletion region width $\left(d_n+d_p\right)$ to equal \SI{50}{\angstrom}? Assume that you have $\epsilon e \Delta \phi_0=0.5 \mathrm{eV}$.

\paragraph{Solution} Here we use 
\begin{equation}
    d_{n, p}=105\left\{\frac{\left(N_a / N_d\right)^{\pm 1}}{10^{-18}\left(N_d+N_a\right)}[\epsilon e \Delta \phi]_\text{ev}\right\}^{1 / 2} \unit{\angstrom}
\end{equation}
from A\&M (29.18).
Since $N_a = N_d$, we have $d_n = d_p = \SI{25}{\angstrom}$,
and therefore $N_a = N_d = \SI{4.4e18}{cm^{-3}}$.
(Note that in (29.18), the unit of $N_{a/d}$ is \unit{cm^{-1}}, not \unit{m^{-1}};
and (29.17) is in Gaussian units.)

\paragraph{Problem 4} A Si p-n junction diode at room temperature has its p-side doped with $N_a=4 \times 10^{16} \mathrm{~cm}^{-3}$ and its $\mathrm{n}$-side with $N_d=1 \times 10^{17} \mathrm{~cm}^{-3}$. The donor and acceptor binding energies are $0.025 \mathrm{eV}$ and $0.06 \mathrm{eV}$, respectively. Take $\epsilon e \Delta \phi_0=0.5 \mathrm{eV}$ and $\Delta \phi_0=1 \vee$ for Si. (I realize these values seem strange in that $\epsilon<1$ is implied;
but please take these as effective values that are correct for a junction where our simple theory is not quite right: namely, use $\Delta \phi_0$ as the potential change across the junction at equilibrium that then determines the voltage dependent junction wide and take $\epsilon e \Delta \phi_0$ as the key parameter determining the junction width at equilibrium).
a) What are the widths of the $\mathrm{n}$ - and $\mathrm{p}$-type depletion regions in equilibrium?
b) When a forward bias of $0.8 \mathrm{~V}$ is applied to this diode, what are the values of the $\mathrm{n}$ - and $\mathrm{p}$-type depletion region widths?

\paragraph{Solution} \begin{itemize}
\item[(a)] We are under room temperature so it's safe to assume 
that the doped atoms are fully ionized.
Using A\&M (29.18) again, we have $d_n = \SI{125.5}{\angstrom}$,
and $d_p = \SI{313.7}{\angstrom}$.
\item[(b)] From 
\begin{equation}
    d_{n, p}(V) = \sqrt{1 - \frac{V}{\Delta \phi}} d_{n, p}(0)
\end{equation}
we have 
$d_{n} = \SI{56.1}{\angstrom}$, and 
$d_{p} = \SI{140.3}{\angstrom}$.
\end{itemize}

\paragraph{Problem 5}  The color of an LED is controlled by the direct band gap of the semiconductor used to construct it. In practice, this is done by alloying different semiconductors to change the band gap. Vegard's law applies reasonably well: the size of the band gap is linearly related to the composition of the material; for example, if you make a 50/50 alloy of AIN and $\mathrm{GaN}$ with composition $\mathrm{Al}_{0.5} \mathrm{Ga}_{0.5} \mathrm{~N}$ then its band gap is the average of the band gaps of AIN and GaN. Assume Vegard's law holds. Also, consult the handout on band gaps on the next page. For questions of practicality, know that trying to make a high-quality junction of two materials whose lattice parameters differ significantly is very hard: when the strain (lattice) mismatch is too high, the interface region between the two materials becomes disorderly and defective to relax the strain, and this degrades performance. The more the strain mismatch, the worse the interface.
a) You want to make an LED emit at $620 \mathrm{~nm}$ using an alloy of CdS and CdSe. What composition should you use?
b) Your supply of CdS is depleted due to supply chain issues. However, you do have a supply of $\mathrm{CdSe}, \mathrm{ZnS}$, and $\mathrm{ZnSe}$. Which combination is the best choice for making a high-quality LED emitting at $620 \mathrm{~nm}$ ? Why?
c) Your friend has an alternative approach: they have a supply of $\mathrm{GaP}$ and GaAs and plan to make an alloy to create the $620 \mathrm{~nm}$ LED. What is a problem with their plan?

\paragraph{Solution} \begin{itemize}
\item[(a)] The energy of a \SI{620}{nm} photon is \SI{2.0}{eV}.
The energy gap of \ce{CdSe} is \SI{1.7}{eV},
and the energy gap of \ce{CdS} is \SI{2.5}{eV},
so the percentage of \ce{CdSe} should be \SI{62.5}{\percent},
and the percentage of \ce{CdS} should be \SI{37.5}{\percent}.

\item[(b)] \ce{ZnSe} and \ce{CdSe} should be used.
The percentage of \ce{CdSe} is \SI{72.7}{\percent}, 
and the percentage of \ce{ZnSe} is \SI{27.3}{\percent}.
\ce{ZnS} is not used because the difference between the lattice constants of it and \ce{CdSe} is too large.

\item[(c)] The problem is \ce{GaP} has an indirect band gap, 
and therefore isn't very efficient in light emission.
\end{itemize}

\end{document}