\documentclass[hyperref, a4paper]{article}

\usepackage{geometry}
\usepackage{titling}
\usepackage{titlesec}
% No longer needed, since we will use enumitem package
% \usepackage{paralist}
\usepackage{enumitem}
\usepackage{footnote}
\usepackage{enumerate}
\usepackage{amsmath, amssymb, amsthm}
\usepackage{mathtools}
\usepackage{bbm}
\usepackage{cite}
\usepackage{graphicx}
\usepackage{subfigure}
\usepackage{physics}
\usepackage{tensor}
\usepackage{siunitx}
\usepackage[version=4]{mhchem}
\usepackage{tikz}
\usepackage{xcolor}
\usepackage{listings}
\usepackage{autobreak}
\usepackage[ruled, vlined, linesnumbered]{algorithm2e}
\usepackage{nameref,zref-xr}
\zxrsetup{toltxlabel}
\usepackage[colorlinks,unicode]{hyperref} % , linkcolor=black, anchorcolor=black, citecolor=black, urlcolor=black, filecolor=black
\usepackage[most]{tcolorbox}
\usepackage{prettyref}

% Page style
\geometry{left=3.18cm,right=3.18cm,top=2.54cm,bottom=2.54cm}
\titlespacing{\paragraph}{0pt}{1pt}{10pt}[20pt]
\setlength{\droptitle}{-5em}
%\preauthor{\vspace{-10pt}\begin{center}}
%\postauthor{\par\end{center}}

% More compact lists 
\setlist[itemize]{
    itemindent=17pt, 
    leftmargin=1pt,
    listparindent=\parindent,
    parsep=0pt,
}

% Math operators
\DeclareMathOperator{\timeorder}{\mathcal{T}}
\DeclareMathOperator{\diag}{diag}
\DeclareMathOperator{\legpoly}{P}
\DeclareMathOperator{\primevalue}{P}
\DeclareMathOperator{\sgn}{sgn}
\newcommand*{\ii}{\mathrm{i}}
\newcommand*{\ee}{\mathrm{e}}
\newcommand*{\const}{\mathrm{const}}
\newcommand*{\suchthat}{\quad \text{s.t.} \quad}
\newcommand*{\argmin}{\arg\min}
\newcommand*{\argmax}{\arg\max}
\newcommand*{\normalorder}[1]{: #1 :}
\newcommand*{\pair}[1]{\langle #1 \rangle}
\newcommand*{\fd}[1]{\mathcal{D} #1}
\DeclareMathOperator{\bigO}{\mathcal{O}}

% TikZ setting
\usetikzlibrary{arrows,shapes,positioning}
\usetikzlibrary{arrows.meta}
\usetikzlibrary{decorations.markings}
\tikzstyle arrowstyle=[scale=1]
\tikzstyle directed=[postaction={decorate,decoration={markings,
    mark=at position .5 with {\arrow[arrowstyle]{stealth}}}}]
\tikzstyle ray=[directed, thick]
\tikzstyle dot=[anchor=base,fill,circle,inner sep=1pt]

% Algorithm setting
% Julia-style code
\SetKwIF{If}{ElseIf}{Else}{if}{}{elseif}{else}{end}
\SetKwFor{For}{for}{}{end}
\SetKwFor{While}{while}{}{end}
\SetKwProg{Function}{function}{}{end}
\SetArgSty{textnormal}

\newcommand*{\concept}[1]{{\textbf{#1}}}

% Embedded codes
\lstset{basicstyle=\ttfamily,
  showstringspaces=false,
  commentstyle=\color{gray},
  keywordstyle=\color{blue}
}

% Reference formatting
\newrefformat{fig}{Figure~\ref{#1}}

% Color boxes
\tcbuselibrary{skins, breakable, theorems}
\newtcbtheorem[number within=section]{warning}{Warning}%
  {colback=orange!5,colframe=orange!65,fonttitle=\bfseries, breakable}{warn}
\newtcbtheorem[number within=section]{note}{Note}%
  {colback=green!5,colframe=green!65,fonttitle=\bfseries, breakable}{note}
\newtcbtheorem[number within=section]{info}{Info}%
  {colback=blue!5,colframe=blue!65,fonttitle=\bfseries, breakable}{info}

\newenvironment{shelldisplay}{\begin{lstlisting}}{\end{lstlisting}}

\newcommand{\address}[1]{\href{#1}{\url{#1}}}

\title{Homework 12}
\author{Jiinyuan Wu}

\begin{document}
    
\maketitle

\paragraph{Problem 1} 

\paragraph{Solution} The data used are $m_{\text{e}} = 0.066m$, 
$m_{\text{h}} = 0.5 m$,
and the conservation laws are 
\begin{equation}
    \frac{p_{\text{e}}}{2m_{\text{e}}} + \frac{p_{\text{h}}}{2m_{\text{h}}} = E_{\text{photon}}, \quad 
    p_{\text{e}} + p_{\text{h}} = \frac{E_{\text{photon}}}{c} ,
\end{equation}
and we get 
\begin{equation}
    p_{\text{e}} = \SI{1.65e-25}{kg \cdot m/s}, \quad 
    p_{\text{h}} = \SI{-1.64e-25}{kg \cdot m/s}, \quad 
    \frac{p_{\text{e}}^2}{2m_{\text{e}}} = \SI{1.41}{eV}, \quad 
    \frac{p_{\text{h}}^2}{2m_{\text{h}}} = \SI{0.18}{eV}.
\end{equation}

\paragraph{Problem 2}

\paragraph{Solution} 

\paragraph{Problem 3}

\paragraph{Solution} Here we use 
\begin{equation}
    d_{n, p}=105\left\{\frac{\left(N_a / N_d\right)^{\pm 1}}{10^{-18}\left(N_d+N_a\right)}[\epsilon e \Delta \phi]_\text{ev}\right\}^{1 / 2} \unit{\angstrom}
\end{equation}
from A\&M (29.18).
Since $N_a = N_d$, we have $d_n = d_p = \SI{25}{\angstrom}$,
and therefore $N_a = N_d = \SI{4.4e18}{cm^{-3}}$.

\paragraph{Problem 4}

\paragraph{Solution} 

\paragraph{Problem 5}

\paragraph{Solution} \begin{itemize}
\item[(a)] The energy of a \SI{620}{nm} photon is \SI{2.0}{eV}.
The energy gap of \ce{CdSe} is \SI{1.7}{eV},
and the energy gap of \ce{CdS} is \SI{2.5}{eV},
so the percentage of \ce{CdSe} should be \SI{62.5}{\percent},
and the percentage of \ce{CdS} should be \SI{37.5}{\percent}.

\item[(b)] \ce{ZnSe} and \ce{CdSe} should be used.
The percentage of \ce{CdSe} is \SI{72.7}{\percent}, 
and the percentage of \ce{ZnSe} is \SI{27.3}{\percent}.
\ce{ZnS} is not used because the difference between the lattice constants of it and \ce{CdSe} is too large.

\item[(c)] 
\end{itemize}

\end{document}