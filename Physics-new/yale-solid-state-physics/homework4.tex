\documentclass[hyperref, a4paper]{article}

\usepackage{geometry}
\usepackage{titling}
\usepackage{titlesec}
% No longer needed, since we will use enumitem package
% \usepackage{paralist}
\usepackage{enumitem}
\usepackage{footnote}
\usepackage{enumerate}
\usepackage{amsmath, amssymb, amsthm}
\usepackage{mathtools}
\usepackage{bbm}
\usepackage{cite}
\usepackage{graphicx}
\usepackage{subfigure}
\usepackage{physics}
\usepackage{tensor}
\usepackage{siunitx}
\usepackage[version=4]{mhchem}
\usepackage{tikz}
\usepackage{xcolor}
\usepackage{listings}
\usepackage{autobreak}
\usepackage[ruled, vlined, linesnumbered]{algorithm2e}
\usepackage{nameref,zref-xr}
\zxrsetup{toltxlabel}
\usepackage[colorlinks,unicode]{hyperref} % , linkcolor=black, anchorcolor=black, citecolor=black, urlcolor=black, filecolor=black
\usepackage[most]{tcolorbox}
\usepackage{prettyref}

% Page style
\geometry{left=3.18cm,right=3.18cm,top=2.54cm,bottom=2.54cm}
\titlespacing{\paragraph}{0pt}{1pt}{10pt}[20pt]
\setlength{\droptitle}{-5em}
%\preauthor{\vspace{-10pt}\begin{center}}
%\postauthor{\par\end{center}}

% More compact lists 
\setlist[itemize]{
    itemindent=17pt, 
    leftmargin=1pt,
    listparindent=\parindent,
    parsep=0pt,
}

% Math operators
\DeclareMathOperator{\timeorder}{\mathcal{T}}
\DeclareMathOperator{\diag}{diag}
\DeclareMathOperator{\legpoly}{P}
\DeclareMathOperator{\primevalue}{P}
\DeclareMathOperator{\sgn}{sgn}
\newcommand*{\ii}{\mathrm{i}}
\newcommand*{\ee}{\mathrm{e}}
\newcommand*{\const}{\mathrm{const}}
\newcommand*{\suchthat}{\quad \text{s.t.} \quad}
\newcommand*{\argmin}{\arg\min}
\newcommand*{\argmax}{\arg\max}
\newcommand*{\normalorder}[1]{: #1 :}
\newcommand*{\pair}[1]{\langle #1 \rangle}
\newcommand*{\fd}[1]{\mathcal{D} #1}
\DeclareMathOperator{\bigO}{\mathcal{O}}

% TikZ setting
\usetikzlibrary{arrows,shapes,positioning}
\usetikzlibrary{arrows.meta}
\usetikzlibrary{decorations.markings}
\tikzstyle arrowstyle=[scale=1]
\tikzstyle directed=[postaction={decorate,decoration={markings,
    mark=at position .5 with {\arrow[arrowstyle]{stealth}}}}]
\tikzstyle ray=[directed, thick]
\tikzstyle dot=[anchor=base,fill,circle,inner sep=1pt]

% Algorithm setting
% Julia-style code
\SetKwIF{If}{ElseIf}{Else}{if}{}{elseif}{else}{end}
\SetKwFor{For}{for}{}{end}
\SetKwFor{While}{while}{}{end}
\SetKwProg{Function}{function}{}{end}
\SetArgSty{textnormal}

\newcommand*{\concept}[1]{{\textbf{#1}}}

% Embedded codes
\lstset{basicstyle=\ttfamily,
  showstringspaces=false,
  commentstyle=\color{gray},
  keywordstyle=\color{blue}
}

% Reference formatting
\newrefformat{fig}{Figure~\ref{#1}}

% Color boxes
\tcbuselibrary{skins, breakable, theorems}
\newtcbtheorem[number within=section]{warning}{Warning}%
  {colback=orange!5,colframe=orange!65,fonttitle=\bfseries, breakable}{warn}
\newtcbtheorem[number within=section]{note}{Note}%
  {colback=green!5,colframe=green!65,fonttitle=\bfseries, breakable}{note}
\newtcbtheorem[number within=section]{info}{Info}%
  {colback=blue!5,colframe=blue!65,fonttitle=\bfseries, breakable}{info}

\newenvironment{shelldisplay}{\begin{lstlisting}}{\end{lstlisting}}

\newcommand{\address}[1]{\href{#1}{\url{#1}}}

\title{Solid State Physics Homework 4}
\author{Jinyuan Wu}

\begin{document}

\maketitle

\paragraph{Problem 1} Optical phonons have that name because they can couple to electromagnetic radiation: when the atoms in the unit cell have different charge states (e.g., oppositely charged ions), then if they move "against" each other we have oscillating dipoles which can couple to electromagnetic waves. Explain, using energy and momentum conservation, why you can only excite an optical mode and not an acoustic mode using one photon of light. (To do this, look up a few dispersion curves for phonons in real materials to see the physical ranges of wave vector and frequency that are typical, and then compare to electromagnetic waves.)

\paragraph{Solution} From \cite{messaoudi2015band},
it can be seen the highest frequencies of acoustic phonons in \ce{NaCl} are around $\sim \SI{6}{THz}$.
The lattice constant is \SI{0.563}{nm},
which is also the magnitude of the ``wave length'' of phonons.
So the magnitude of the sound speed is 
\[
    \SI{6}{THz} \times \SI{0.563}{nm} = \SI{3e3}{m/s}.
\]
This is much lower than the speed of light.
This is shown on \prettyref{fig:light-and-optical-phonon}.

Exciting a phonon means converting a photon to a phonon,
and by the conservation of energy and momentum,
this only happens at the intersection of the photon dispersion curve 
and the phonon dispersion curves.
So from the diagram, we see only the optical branch has intersection with the light dispersion curve,
so only optical phonons can be excited.

\begin{figure}
    \centering
    

\tikzset{every picture/.style={line width=0.75pt}} %set default line width to 0.75pt        

\begin{tikzpicture}[x=0.75pt,y=0.75pt,yscale=-1,xscale=1]
%uncomment if require: \path (0,300); %set diagram left start at 0, and has height of 300

%Straight Lines [id:da8088878009479294] 
\draw    (197,234) -- (352.83,234) ;
%Straight Lines [id:da11765036321565625] 
\draw    (272.92,234) -- (272.92,45.52) ;
%Straight Lines [id:da683479585384096] 
\draw  [dash pattern={on 0.84pt off 2.51pt}]  (329.83,234) -- (329.83,45.52) ;
%Straight Lines [id:da6868698450599917] 
\draw  [dash pattern={on 0.84pt off 2.51pt}]  (215.92,234) -- (215.92,45.52) ;
%Curve Lines [id:da0675227528677198] 
\draw [color={rgb, 255:red, 74; green, 144; blue, 226 }  ,draw opacity=1 ]   (216.29,114.44) .. controls (229.79,114.44) and (242.29,79.44) .. (272.92,78.52) ;
%Curve Lines [id:da7768755956644466] 
\draw [color={rgb, 255:red, 74; green, 144; blue, 226 }  ,draw opacity=1 ]   (329.54,114.44) .. controls (316.04,114.44) and (303.54,79.44) .. (272.92,78.52) ;
%Curve Lines [id:da4463535198979316] 
\draw [color={rgb, 255:red, 74; green, 144; blue, 226 }  ,draw opacity=1 ]   (272.92,234) .. controls (260.53,206.05) and (227.73,128.05) .. (216.29,128.08) ;
%Curve Lines [id:da7661771106088253] 
\draw [color={rgb, 255:red, 74; green, 144; blue, 226 }  ,draw opacity=1 ]   (272.92,234) .. controls (284.53,204.85) and (315.7,127.79) .. (329.54,128.08) ;
%Straight Lines [id:da11911287691360406] 
\draw [color={rgb, 255:red, 245; green, 166; blue, 35 }  ,draw opacity=1 ]   (277.83,47.85) -- (272.92,234) ;
%Straight Lines [id:da46839192970078103] 
\draw [color={rgb, 255:red, 245; green, 166; blue, 35 }  ,draw opacity=1 ]   (268,47.85) -- (272.92,234) ;

% Text Node
\draw (272.92,42.12) node [anchor=south] [inner sep=0.75pt]    {$\omega $};
% Text Node
\draw (354.83,234) node [anchor=west] [inner sep=0.75pt]    {$k$};


\end{tikzpicture}

    \caption{The dispersion curves of light and phonons; the orange lines are for the photon,
    and the blue lines are for phonons}
    \label{fig:light-and-optical-phonon}
\end{figure}

\paragraph{Problem 2} Consider two related one-dimensional crystals:
\begin{itemize}
\item Case $\mathrm{A}$ is a monoatomic linear chain with atoms of mass $M$, spring constant $\mathrm{C}$, and atomic spacing $a$. Label the atoms by the integers.
\item Case B is obtained from case A by letting the masses alternate: the even numbered atomic sites are have mass $M_1=M+\Delta$ and the odd numbered ones have mass $M_2=M-\Delta$ where $0<\Delta \ll M$. The spring constants and interatomic spacings are unchanged.
\end{itemize}
\begin{itemize}
\item[(a)] For case A, what is $\omega$ at the Brillouin zone boundary?
\item[(b)] What is the speed of sound for case $A$ ?
\item[(c)] What is the speed of sound for case B? Compare to question (b) and explain what is going on.
\item[(d)] For case $B$, what is the optical phonon $\omega$ at $k=0$ ?
\item[(e)] Sketch the phonon dispersion relation for both crystals on the same plot in the Brillouin zone of case $A$ and indicate the $k$ value for the first Brillouin zone edge for both cases.
\end{itemize}

\paragraph{Solution} \begin{itemize}
\item[(a)] The EOM is 
\begin{equation}
    M \ddot{x}_n = C (x_{n-1} + x_{n+1} - 2 x_n).
\end{equation}
At the edge of the Brillouin zone, 
we have 
\begin{equation}
    x_n = A \ee^{\ii k a n - \ii \omega t}, \quad k = \frac{\pi}{a}.
    \label{eq:x-n-component-1}
\end{equation}
The wave vector $k = - \pi / a$ is also possible, 
but the difference between it and $\pi / a$ is a reciprocal lattice vector,
so we can just work with $k = \pi / a$.
Thus at the edge of the Brillouin zone, we have
\[
    - M \omega^2 A = C (\ee^{- \ii k a} + \ee^{\ii k a} - 2) A = - 4 C A ,
\] 
\begin{equation}
    \omega = \sqrt{\frac{4C}{M}}.
    \label{eq:single-atom-edge}
\end{equation}

\item[(b)] Again we take the ansatz \eqref{eq:x-n-component-1},
and we have 
\[
    - M \omega^2 A = C (\ee^{- \ii k a} + \ee^{\ii k a} - 2) A 
    = CA (2 \cos ka - 2) \approx - CA (ka)^2,
\]
so 
\begin{equation}
    \omega = \sqrt{\frac{C}{M}} k a,
    \label{eq:single-atom-k-0}
\end{equation}
and 
\begin{equation}
    v_{\text{sound}} = \frac{\omega}{k} = \sqrt{\frac{C}{M}} a.
    \label{eq:sound-speed}
\end{equation}

\item[(c)] In Case B, the EOMs are 
\begin{equation}
    \begin{aligned}
        M_1 \ddot{x}_{2n}   &= C (x_{2n-1} + x_{2n+1} - 2x_{2n}), \\
        M_2 \ddot{x}_{2n+1} &= C (x_{2n+2} + x_{2n}   - 2x_{2n+1}).
    \end{aligned}
    \label{eq:eom-2}
\end{equation}
The ansatz is 
\begin{equation}
    \begin{aligned}
        x_{2n}   &= A_1 \ee^{2 \ii k a n      - \ii \omega t}, \\
        x_{2n+1} &= A_2 \ee^{\ii k a (2n + 1) - \ii \omega t},
    \end{aligned}
\end{equation}
and \eqref{eq:eom-2} becomes 
\[
    \begin{aligned}
        - M_1 \omega^2 A_1 \ee^{2 \ii k a n} 
        &= C (A_2 \ee^{\ii k a (2n - 1)} + A_2 \ee^{\ii k a (2n + 1)} - 2 A_1 \ee^{2 \ii k a n}), \\
        - M_2 \omega^2 A_2 \ee^{\ii k a (2n + 1)} 
        &= C (A_1 \ee^{\ii k a (2n + 2)} + A_1 \ee^{2 \ii k a n} - 2 A_2 \ee^{\ii k a (2n + 1)}),
    \end{aligned}
\] 
\begin{equation}
    \pmqty{
        2C - M_1 \omega^2                    &  C (\ee^{- \ii k a} + \ee^{\ii k a}) \\
        C (\ee^{\ii k a} + \ee^{- \ii k a})  &  2C - M_2 \omega^2 
    } \pmqty{A_1 \\ A_2} = 0.
\end{equation}
So we have 
\begin{equation}
    (2C - M_1 \omega^2) (2C - M_2 \omega^2)
    - 4 C^2 \cos^2 ka = 0,
\end{equation}
and therefore 
\begin{equation}
    \omega^2 = \frac{M_1 + M_2 \pm \sqrt{(M_1 + M_2)^2 - 4 M_1 M_2 (1 - \cos^2 ka)}}{M_1 M_2} C.
\end{equation}
The positive and negative branches correspond to the acoustic and optical phonons, respectively.

When $M_1 = M_2$, we get 
\begin{equation}
    \omega^2 = \frac{2M \pm 2M \cos ka}{M^2} C = \frac{2C}{M} (1 \pm \cos ka).
\end{equation}
In the $k \to 0$ limit, we have 
\[
    \frac{2C}{M} (1 + \cos ka) \approx \frac{4C}{M}, \quad 
    \frac{2C}{M} (1 - \cos ka) \approx \frac{C}{M} (ka)^2, 
\]
so 
\begin{equation}
    \omega_{\text{acoustic}} \approx \sqrt{\frac{C}{M}} k a, \quad 
    \omega_{\text{optical}}  \approx \sqrt{\frac{4 C}{M}}.
\end{equation}
So the sound speed is also \eqref{eq:sound-speed}.

\item[(d)] The frequency of the optical phonon near $k=0$ is the same as \eqref{eq:single-atom-edge}.
We notice that the first equation is the same as \eqref{eq:single-atom-k-0},
and the second equation is the same as \eqref{eq:single-atom-edge}.
That's to say, when $\Delta$ is small enough,
the acoustic phonon spectrum of case B near the $\Gamma$ point 
is the same as the phonon spectrum of case A near the $\Gamma$ point,
while the optical phonon spectrum of case B near the $\Gamma$ point 
is the same as the phonon spectrum of case A near the boundary of the Brillouin zone.

What's going on here is in the $\Delta = 0$ case, 
when we view $2a$ -- instead of $a$ -- 
as the lattice parameter,
the size of the Brillouin zone shrinks,
and the phonon spectrum in $\pi / 2a \leq \abs{k} \leq \pi / a$
now has to be fold into the shrunk Brillouin zone.
Thus, the so-called frequency of the optical phonon at $k=0$ is actually 
the frequency of the case A phonon at $k = \pi / 2$,
which is folded back to $k=0$ in case B;
the so-called acoustic phonon branch in case B 
is just the $0 \leq \abs{k} \leq \pi / 2a$ part of the phonon spectrum in case A.

When $\Delta \neq 0$,
by perturbation theory we know at the boundary of the shrunk Brillouin zone,
an energy gap will be opened.
This is what's going on in (e).

\item[(e)] When $k = \pi / (2a)$, i.e. when we are at the boundary of the Brillouin zone of case B,
we have $1 - \cos^2 ka = 1$, and 
\[
    \omega^2 = \frac{M_1 + M_2 \pm (M_1 - M_2)}{M_1 M_2} C,
\]
\begin{equation}
    \omega^2 = \sqrt{\frac{2 C}{M_1}} \approx \sqrt{\frac{2C}{M}} 
    \left(1 - \frac{\Delta}{2M} \right), \quad 
    \text{or } \omega = \sqrt{\frac{2C}{M_2}} \approx \sqrt{\frac{2C}{M}} \left( 1 + \frac{\Delta}{2M} \right).
\end{equation}
In case B, when $\Delta \neq 0$, the $k=0$ value of the optical phonon band is 
\[
    \omega^2 = \frac{2 (M_1 + M_2)}{M_1 M_2} C = \frac{4 M}{M^2 - \Delta^2} C ,
\]
\begin{equation}
    \omega = \sqrt{\frac{4 M}{M^2 - \Delta^2} C} \approx \sqrt{\frac{4 C}{M}}.
\end{equation}
So we sketch the phonon bands in \prettyref{fig:phonon-spectrum}.
Here the blue lines are case B phonons,
the upper band being the optical branch,
the lower band being the acoustic branch.
The boundaries of the Brillouin zone of case A are $\pm \pi / a$,
and the boundaries of the Brillouin zone of case B are $\pm \pi / 2a$.

\begin{figure}
    \centering
    \tikzset{every picture/.style={line width=0.75pt}} %set default line width to 0.75pt        

\begin{tikzpicture}[x=0.75pt,y=0.75pt,yscale=-1,xscale=1]
    %uncomment if require: \path (0,331); %set diagram left start at 0, and has height of 331
    
    %Straight Lines [id:da8356152726041366] 
    \draw    (124,214) -- (381.83,214) ;
    %Straight Lines [id:da021999188245394707] 
    \draw    (252.92,214) -- (252.92,25.52) ;
    %Straight Lines [id:da31671127617153116] 
    \draw  [dash pattern={on 0.84pt off 2.51pt}]  (364.83,214) -- (364.83,25.52) ;
    %Straight Lines [id:da31489859157964584] 
    \draw  [dash pattern={on 0.84pt off 2.51pt}]  (141,214) -- (141,25.52) ;
    %Straight Lines [id:da5781990351052568] 
    \draw  [dash pattern={on 0.84pt off 2.51pt}]  (309.83,214) -- (309.83,25.52) ;
    %Straight Lines [id:da3668794574063743] 
    \draw  [dash pattern={on 0.84pt off 2.51pt}]  (195.92,214) -- (195.92,25.52) ;
    %Curve Lines [id:da40293837097913743] 
    \draw [color={rgb, 255:red, 80; green, 227; blue, 194 }  ,draw opacity=1 ]   (252.92,214) .. controls (261.73,190.45) and (310.53,59.25) .. (364.83,58.52) ;
    %Straight Lines [id:da2195792690812819] 
    \draw  [dash pattern={on 0.84pt off 2.51pt}]  (141,58.52) -- (409.33,58.52) ;
    %Curve Lines [id:da7601015436707863] 
    \draw [color={rgb, 255:red, 74; green, 144; blue, 226 }  ,draw opacity=1 ]   (196.29,94.44) .. controls (209.79,94.44) and (222.29,59.44) .. (252.92,58.52) ;
    %Curve Lines [id:da3619583062947258] 
    \draw [color={rgb, 255:red, 80; green, 227; blue, 194 }  ,draw opacity=1 ]   (252.92,214) .. controls (242.53,188.59) and (192.13,58.59) .. (141,58.52) ;
    %Curve Lines [id:da41813310010635707] 
    \draw [color={rgb, 255:red, 74; green, 144; blue, 226 }  ,draw opacity=1 ]   (309.54,94.44) .. controls (296.04,94.44) and (283.54,59.44) .. (252.92,58.52) ;
    %Curve Lines [id:da7531515572904148] 
    \draw [color={rgb, 255:red, 74; green, 144; blue, 226 }  ,draw opacity=1 ]   (252.92,214) .. controls (240.53,186.05) and (207.73,108.05) .. (196.29,108.08) ;
    %Curve Lines [id:da627146833946499] 
    \draw [color={rgb, 255:red, 74; green, 144; blue, 226 }  ,draw opacity=1 ]   (252.92,214) .. controls (264.53,184.85) and (295.7,107.79) .. (309.54,108.08) ;
    %Straight Lines [id:da6762460883625818] 
    \draw  [dash pattern={on 0.84pt off 2.51pt}]  (141,94.52) -- (364.83,94.52) ;
    %Straight Lines [id:da2972745572690798] 
    \draw  [dash pattern={on 0.84pt off 2.51pt}]  (141,107.52) -- (364.83,107.52) ;
    %Straight Lines [id:da6582741599552242] 
    \draw    (91,69.76) -- (139.21,93.63) ;
    \draw [shift={(141,94.52)}, rotate = 206.34] [fill={rgb, 255:red, 0; green, 0; blue, 0 }  ][line width=0.08]  [draw opacity=0] (12,-3) -- (0,0) -- (12,3) -- cycle    ;
    %Straight Lines [id:da18556256043603003] 
    \draw    (92.2,120.56) -- (139.07,108.04) ;
    \draw [shift={(141,107.52)}, rotate = 165.04] [fill={rgb, 255:red, 0; green, 0; blue, 0 }  ][line width=0.08]  [draw opacity=0] (12,-3) -- (0,0) -- (12,3) -- cycle    ;
    
    % Text Node
    \draw (252.92,22.12) node [anchor=south] [inner sep=0.75pt]    {$\omega $};
    % Text Node
    \draw (383.83,214) node [anchor=west] [inner sep=0.75pt]    {$k$};
    % Text Node
    \draw (364.83,217.4) node [anchor=north] [inner sep=0.75pt]    {$\frac{\pi }{a}$};
    % Text Node
    \draw (141,217.4) node [anchor=north] [inner sep=0.75pt]    {$-\frac{\pi }{a}$};
    % Text Node
    \draw (309.83,217.4) node [anchor=north] [inner sep=0.75pt]    {$\frac{\pi }{2a}$};
    % Text Node
    \draw (195.92,217.4) node [anchor=north] [inner sep=0.75pt]    {$-\frac{\pi }{2a}$};
    % Text Node
    \draw (411.33,58.52) node [anchor=west] [inner sep=0.75pt]    {$\sqrt{\frac{4C}{M}}$};
    % Text Node
    \draw (18.13,142.92) node [anchor=west] [inner sep=0.75pt]    {$\sqrt{\frac{2C}{M}}\left( 1-\frac{\Delta }{2M}\right)$};
    % Text Node
    \draw (18,44.72) node [anchor=west] [inner sep=0.75pt]    {$\sqrt{\frac{2C}{M}}\left( 1+\frac{\Delta }{2M}\right)$};
    
    
    \end{tikzpicture}
    
    \caption{The phonon spectrum}
    \label{fig:phonon-spectrum}
\end{figure}

\end{itemize}

\paragraph{Problem 3} Partly in response to some good questions in lecture: go to this web site \address{https://en.wikipedia.org/wiki/File:Phonon\_k\_3k.gif} which shows a closed loop animation of a monoatomic line of atoms spaced by $a$ in 1D and with the harmonic nearest neighbor approximation. There are two different waves being shown (black and red) and the black balls are the actual atoms that move up and down (the wave is propagating along the horizontal axis and the atoms are moving up and down so this is transverse).
(a) Take $k>0$ to mean wave motion to the right. Indicate the $(k, \omega)$ for both waves on the dispersion relation for this system. Explain briefly how you determined them.
(b) At what speed is energy being transmitted along this chain of atoms?

\paragraph{Solution}

\begin{itemize}
\item[(a)] We can wait until one wave crest of the black wave is at one of the atoms,
and then it's clear that half of the wave length of the black wave is $a$,
so the total wavelength is $2a$ and thus $k = 2\pi / (2a) = \pi / a$;
it's positive because the black wave is heading right.
Similarly, for the red wave, 
2.5 times of the total wavelength is $a$, so the wavelength is $a / 2.5$,
and $k = - 2\pi / (a / 2.5) = - 5 \pi / a$.

Since the two waves are describing the same lattice motion,
$\omega$'s are the same.

\item[(b)] In the animation, 
each atom is moving as if it's a harmonic oscillator away from anything else,
so there is no transmission of energy here:
the speed of energy transmission is zero.
\end{itemize}

\paragraph{Problem 4} Here are five chemical formulae for crystals:
(i) $\mathrm{LiNbO}_3$, (ii) $\mathrm{Ni}_3 \mathrm{Ti}$, (iii) $\mathrm{V}_2 \mathrm{O}_5$, (iv) $\mathrm{Fe}_3 \mathrm{O}_4$, , v) $\mathrm{PbO}_2$
And below are two actual dispersion relations for phonons ( $A$ and $B$ ).
For each dispersion curve, list the crystals that might correspond to it (and you must provide at least one answer per curve). Also please make some reasonable assumptions instead of pleading ignorance of degeneracies and thus saying anything goes.

\paragraph{Solution} There are 12 curves in the first figure, 
so there are $12 / 3 = 4$ atoms in each primitive unit cell.
The only crystal that may satisfy this requirement is \ce{Ni3Ti}.

There are 30 curves in the left part of the second figure,
and 20 curves in the right part.
This difference is likely to be a result of degeneracy,
so we pick up $30 / 3 = 10$ as the expected number of atoms per primitive unit cell.
Among the possible crystals,
only \ce{LiNbO3} has a total atom number that divides $10$,
so we can expect the second figure gives the phonon spectrum of \ce{LiNbO3},
and just like the case of graphene,
there are two Li atoms, two Nb atoms, and six O atoms in one primitive unit cell;
the two Li atoms have different surroundings.

\bibliographystyle{plain}
\bibliography{real-world-bands}

\end{document}