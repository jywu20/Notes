\documentclass[hyperref, a4paper]{article}

\usepackage{geometry}
\usepackage{titling}
\usepackage{titlesec}
% No longer needed, since we will use enumitem package
% \usepackage{paralist}
\usepackage{enumitem}
\usepackage{footnote}
\usepackage{enumerate}
\usepackage{amsmath, amssymb, amsthm}
\usepackage{mathtools}
\usepackage{bbm}
\usepackage{cite}
\usepackage{graphicx}
\usepackage{subfigure}
\usepackage{physics}
\usepackage{tensor}
\usepackage{siunitx}
\usepackage[version=4]{mhchem}
\usepackage{tikz}
\usepackage{xcolor}
\usepackage{listings}
\usepackage{autobreak}
\usepackage[ruled, vlined, linesnumbered]{algorithm2e}
\usepackage{nameref,zref-xr}
\zxrsetup{toltxlabel}
\usepackage[colorlinks,unicode]{hyperref} % , linkcolor=black, anchorcolor=black, citecolor=black, urlcolor=black, filecolor=black
\usepackage[most]{tcolorbox}
\usepackage{prettyref}

% Page style
\geometry{left=3.18cm,right=3.18cm,top=2.54cm,bottom=2.54cm}
\titlespacing{\paragraph}{0pt}{1pt}{10pt}[20pt]
\setlength{\droptitle}{-5em}
%\preauthor{\vspace{-10pt}\begin{center}}
%\postauthor{\par\end{center}}

% More compact lists 
\setlist[itemize]{
    itemindent=17pt, 
    leftmargin=1pt,
    listparindent=\parindent,
    parsep=0pt,
}

% Math operators
\DeclareMathOperator{\timeorder}{\mathcal{T}}
\DeclareMathOperator{\diag}{diag}
\DeclareMathOperator{\legpoly}{P}
\DeclareMathOperator{\primevalue}{P}
\DeclareMathOperator{\sgn}{sgn}
\newcommand*{\ii}{\mathrm{i}}
\newcommand*{\ee}{\mathrm{e}}
\newcommand*{\const}{\mathrm{const}}
\newcommand*{\suchthat}{\quad \text{s.t.} \quad}
\newcommand*{\argmin}{\arg\min}
\newcommand*{\argmax}{\arg\max}
\newcommand*{\normalorder}[1]{: #1 :}
\newcommand*{\pair}[1]{\langle #1 \rangle}
\newcommand*{\fd}[1]{\mathcal{D} #1}
\DeclareMathOperator{\bigO}{\mathcal{O}}

% TikZ setting
\usetikzlibrary{arrows,shapes,positioning}
\usetikzlibrary{arrows.meta}
\usetikzlibrary{decorations.markings}
\tikzstyle arrowstyle=[scale=1]
\tikzstyle directed=[postaction={decorate,decoration={markings,
    mark=at position .5 with {\arrow[arrowstyle]{stealth}}}}]
\tikzstyle ray=[directed, thick]
\tikzstyle dot=[anchor=base,fill,circle,inner sep=1pt]

% Algorithm setting
% Julia-style code
\SetKwIF{If}{ElseIf}{Else}{if}{}{elseif}{else}{end}
\SetKwFor{For}{for}{}{end}
\SetKwFor{While}{while}{}{end}
\SetKwProg{Function}{function}{}{end}
\SetArgSty{textnormal}

\newcommand*{\concept}[1]{{\textbf{#1}}}

% Embedded codes
\lstset{basicstyle=\ttfamily,
  showstringspaces=false,
  commentstyle=\color{gray},
  keywordstyle=\color{blue}
}

% Reference formatting
\newrefformat{fig}{Figure~\ref{#1}}

% Color boxes
\tcbuselibrary{skins, breakable, theorems}
\newtcbtheorem[number within=section]{warning}{Warning}%
  {colback=orange!5,colframe=orange!65,fonttitle=\bfseries, breakable}{warn}
\newtcbtheorem[number within=section]{note}{Note}%
  {colback=green!5,colframe=green!65,fonttitle=\bfseries, breakable}{note}
\newtcbtheorem[number within=section]{info}{Info}%
  {colback=blue!5,colframe=blue!65,fonttitle=\bfseries, breakable}{info}

\newenvironment{shelldisplay}{\begin{lstlisting}}{\end{lstlisting}}

\title{Solid State Physics Homework 4}
\author{Jinyuan Wu}

\begin{document}

\maketitle

\paragraph{Problem 1}

\paragraph{Solution} From \cite{messaoudi2015band},
it can be seen the frequencies of optical phonons in \ce{NaCl} are around $\sim \SI{6}{THz}$
$\sim \SI{0.06}{eV}$.
The lattice constant is \SI{0.563}{nm},
which is also the magnitude of the ``wave length'' of phonons.
The magnitude of wave lengths in the visible light spectrum is $\sim \SI{500}{nm}$,
which is much longer than the wave length of phonons, 
so its momentum is much smaller than the momentum of phonons.

TODO

\paragraph{Problem 3} 

\paragraph{Solution}

\begin{itemize}
\item[(a)] We can wait until one wave crest of the black wave is at one of the atoms,
and then it's clear that half of the wave length of the black wave is $a$,
so the total wavelength is $2a$ and thus $k = 2\pi / (2a) = \pi / a$;
it's positive because the black wave is heading right.
Similarly, for the red wave, 
2.5 times of the total wavelength is $a$, so the wavelength is $a / 2.5$,
and $k = - 2\pi / (a / 2.5) = - 5 \pi / a$.

Since the two waves are describing the same lattice motion,
$\omega$'s are the same.

\item[(b)] In the animation, 
each atom is moving as if it's a harmonic oscillator away from anything else,
so there is no transmission of energy here:
the speed of energy transmission is zero.
\end{itemize}

\paragraph{Problem 4}

\paragraph{Solution} There are 12 curves in the first figure, 
so there are $12 / 3 = 4$ atoms in each primitive unit cell.
The only crystal that may satisfy this requirement is \ce{Ni3Ti}.

There are 30 curves in the left part of the second figure,
and 20 curves in the right part.
This difference is likely to be a result of degeneracy,
so we pick up $30 / 3 = 10$ as the expected number of atoms per primitive unit cell.
Among the possible crystals,
only \ce{LiNbO3} has a total atom number that divides $10$,
so we can expect the second figure gives the phonon spectrum of \ce{LiNbO3},
and just like the case of graphene,
there are two Li atoms, two Nb atoms, and six O atoms in one primitive unit cell;
the two Li atoms have different surroundings.

\bibliographystyle{plain}
\bibliography{real-world-bands}

\end{document}