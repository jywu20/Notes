\documentclass[hyperref, a4paper]{article}

\usepackage{geometry}
\usepackage{titling}
\usepackage{titlesec}
% No longer needed, since we will use enumitem package
% \usepackage{paralist}
\usepackage{enumitem}
\usepackage{footnote}
\usepackage{enumerate}
\usepackage{amsmath, amssymb, amsthm}
\usepackage{mathtools}
\usepackage{bbm}
\usepackage{cite}
\usepackage{graphicx}
\usepackage{subfigure}
\usepackage{physics}
\usepackage{tensor}
\usepackage{siunitx}
\usepackage[version=4]{mhchem}
\usepackage{tikz}
\usepackage{xcolor}
\usepackage{listings}
\usepackage{autobreak}
\usepackage[ruled, vlined, linesnumbered]{algorithm2e}
\usepackage{nameref,zref-xr}
\zxrsetup{toltxlabel}
\usepackage[colorlinks,unicode]{hyperref} % , linkcolor=black, anchorcolor=black, citecolor=black, urlcolor=black, filecolor=black
\usepackage[most]{tcolorbox}
\usepackage{prettyref}

% Page style
\geometry{left=3.18cm,right=3.18cm,top=2.54cm,bottom=2.54cm}
\titlespacing{\paragraph}{0pt}{1pt}{10pt}[20pt]
\setlength{\droptitle}{-5em}
%\preauthor{\vspace{-10pt}\begin{center}}
%\postauthor{\par\end{center}}

% More compact lists 
\setlist[itemize]{
    itemindent=17pt, 
    leftmargin=1pt,
    listparindent=\parindent,
    parsep=0pt,
}

% Math operators
\DeclareMathOperator{\timeorder}{\mathcal{T}}
\DeclareMathOperator{\diag}{diag}
\DeclareMathOperator{\legpoly}{P}
\DeclareMathOperator{\primevalue}{P}
\DeclareMathOperator{\sgn}{sgn}
\newcommand*{\ii}{\mathrm{i}}
\newcommand*{\ee}{\mathrm{e}}
\newcommand*{\const}{\mathrm{const}}
\newcommand*{\suchthat}{\quad \text{s.t.} \quad}
\newcommand*{\argmin}{\arg\min}
\newcommand*{\argmax}{\arg\max}
\newcommand*{\normalorder}[1]{: #1 :}
\newcommand*{\pair}[1]{\langle #1 \rangle}
\newcommand*{\fd}[1]{\mathcal{D} #1}
\DeclareMathOperator{\bigO}{\mathcal{O}}

% TikZ setting
\usetikzlibrary{arrows,shapes,positioning}
\usetikzlibrary{arrows.meta}
\usetikzlibrary{decorations.markings}
\tikzstyle arrowstyle=[scale=1]
\tikzstyle directed=[postaction={decorate,decoration={markings,
    mark=at position .5 with {\arrow[arrowstyle]{stealth}}}}]
\tikzstyle ray=[directed, thick]
\tikzstyle dot=[anchor=base,fill,circle,inner sep=1pt]

% Algorithm setting
% Julia-style code
\SetKwIF{If}{ElseIf}{Else}{if}{}{elseif}{else}{end}
\SetKwFor{For}{for}{}{end}
\SetKwFor{While}{while}{}{end}
\SetKwProg{Function}{function}{}{end}
\SetArgSty{textnormal}

\newcommand*{\concept}[1]{{\textbf{#1}}}

% Embedded codes
\lstset{basicstyle=\ttfamily,
  showstringspaces=false,
  commentstyle=\color{gray},
  keywordstyle=\color{blue}
}

% Reference formatting
\newrefformat{fig}{Figure~\ref{#1}}

% Color boxes
\tcbuselibrary{skins, breakable, theorems}
\newtcbtheorem[number within=section]{warning}{Warning}%
  {colback=orange!5,colframe=orange!65,fonttitle=\bfseries, breakable}{warn}
\newtcbtheorem[number within=section]{note}{Note}%
  {colback=green!5,colframe=green!65,fonttitle=\bfseries, breakable}{note}
\newtcbtheorem[number within=section]{info}{Info}%
  {colback=blue!5,colframe=blue!65,fonttitle=\bfseries, breakable}{info}

\newenvironment{shelldisplay}{\begin{lstlisting}}{\end{lstlisting}}

\newcommand{\address}[1]{\href{#1}{\url{#1}}}

\title{Homework 5}
\author{Jinyuan Wu}

\begin{document}

\maketitle

\paragraph{Problem 1} Consider a particular vibrational mode of a crystal with a frequency (not angular $\omega$ but ``regular'' $\nu$) of $10^{12} \mathrm{~Hz}$. At what temperature is the mean energy of this mode equally divided between the zero-point energy contribution and the temperature dependent contribution?

\paragraph{Solution} We have 
\begin{equation}
    \expval{E} = \sum_{\vb*{k}, \sigma} \hbar \omega_{\vb*{k} \sigma}  
    \left( \frac{1}{2} + \frac{1}{\ee^{{\hbar \omega_{\vb*{k} \sigma}} / {k_{\text{B}} T}} - 1} \right).
\end{equation}
So to make the contribution of the zero-point energy and temperature dependent term the same,
we have 
\[
    \frac{1}{2} = \frac{1}{\ee^{{\hbar \omega_{\vb*{k} \sigma}} / {k_{\text{B}} T}} - 1},
\]
\begin{equation}
    T = \frac{\hbar \omega_{\vb*{k} \sigma}}{k_{\text{B}} \ln 3} = \frac{h \nu}{k_{\text{B}} \ln 3} 
    \sim \SI{43}{K}.
\end{equation}

\paragraph{Problem 2} Problem 1(b) only of Kittel Chapter 5: Suppose that an optical phonon branch has the form $\omega(K)=\omega_0-A K^2$, near $K=0$ in three dimensions. Show that $D(\omega)=$ $(L / 2 \pi)^3\left(2 \pi / A^{3 / 2}\right)\left(\omega_0-\omega\right)^{1 / 2}$ for $\omega<\omega_0$ and $D(\omega)=0$ for $\omega>\omega_0$. Here the density of modes is discontinuous.

\paragraph{Solution} The relation between $\omega$ and $K$ is 
\begin{equation}
    K = \sqrt{\frac{\omega_0 - \omega}{A}}.
\end{equation}
Note that for $\omega_0 < \omega$,
there is no $K$: so $D(\omega) = 0$ for $\omega > \omega_0$.
For $\omega < \omega_0$, we have 
\begin{equation}
    \begin{aligned}
        D(\omega) \dd{\omega} &= \int_{\omega = \omega_K} \dd[3]{\vb*{K}} \cdot \left(\frac{L}{2\pi}\right)^3 \\
        &= \int_{\omega = \omega_K} \dd[2]{\vb*{S}_{\vb*{K}}} \frac{\dd{\omega}}{\abs{\grad_{\vb*{K}} \omega}}
        \cdot \left(\frac{L}{2\pi}\right)^3 ,
    \end{aligned}
\end{equation}
\begin{equation}
    D(\omega) = \left(\frac{L}{2\pi}\right)^3 
    \int_{\omega=\omega_{\vb*{K}}} \frac{\dd[2]{\vb*{S}_{\vb*{K}}}}{\abs{\grad_{\vb*{K}} \omega}}
    = \left(\frac{L}{2\pi}\right)^3 \frac{4\pi K^2}{2 A K} 
    = \left(\frac{L}{2\pi}\right)^3 \frac{2\pi}{A^{3/2}} \sqrt{\omega_0 - \omega}.
\end{equation}
So 
\begin{equation}
    D(\omega) = \begin{cases}
        0, & \omega > \omega_0, \\
        \left(\frac{L}{2\pi}\right)^3 \frac{2\pi}{A^{3/2}} \sqrt{\omega_0 - \omega},
        & \omega < \omega_0.
    \end{cases}
\end{equation}

\paragraph{Problem 3} Minor variation of Problem 5 of Chapter 25 of A\&M: anharmonic terms mean that the harmonic phonon modes are no longer eigenstates and have finite lifetimes. This is because a phonon can decay into two other phonons or two phonons combine (disappear) into a new third phonon. Consider a 3D monoatomic crystal and a particular direction in $\mathrm{k}$ space so we can simply discuss $\mathrm{k}$ as a number. The LA mode lie above the TA modes in frequency for all $\mathrm{k}$, and all the modes have negative curvature $\dd^2 \omega / \dd k^2<0$ for all $k \neq 0$. Assume the two TA modes are degenerate (have the same dispersion relation).
(a) Show that both the above two processes are impossible if all three phonons belong to the same phonon branch.
(b) Show that the only possible processes are $\mathrm{TA}+\mathrm{TA} \leftrightarrow \mathrm{LA}$ and $\mathrm{TA} +\mathrm{LA} \leftrightarrow \mathrm{LA}$ where the branch of each phonon is indicated by LA or TA and the double arrow means the process can go left or right (our two processes).

\paragraph{Solution} \begin{itemize}
\item[(a)] What we need to prove is it's impossible to have 
\begin{equation}
    k_1 + k_2 = k_3, \quad 
    \omega_{k_1} + \omega_{k_2} = \omega_{k_3},
\end{equation}
or in other words, it's impossible to have 
\begin{equation}
    \Delta \omega \coloneqq \omega_{k_1 + k_2} - \omega_{k_1} - \omega_{k_2} = 0.
    \label{eq:impossible-2}
\end{equation}
Since $\pdv[2]{\omega}{k} < 0$, we find  
\[
    \pdv{\Delta \omega}{k_2} = \eval{\pdv{\omega}{k}}_{k_1 + k_2} - \eval{\pdv{\omega}{k}}_{k_1}
\]
has the opposite sign to $k_2$. 
So when $k_1$ is fixed, $\Delta \omega$ reaches its maximum when $k_2 = 0$.
But when $k_2 = 0$, we have 
\[
    \Delta \omega = \omega_{k_1} - \omega_{k_1} - \omega_{k_2} < 0,
\]
so $\Delta \omega$ is always smaller than zero, so \eqref{eq:impossible-2} can never be reached,
and hence it's impossible to have a three-photon process on a single band.

\item[(b)] The $\mathrm{TA} + \mathrm{TA} \leftrightarrow \mathrm{TA}$ 
and $\mathrm{LA} + \mathrm{LA} \leftrightarrow \mathrm{LA}$ 
processes are already discarded.
The $\mathrm{TA} + \mathrm{LA} \leftrightarrow \mathrm{TA}$ process 
is impossible simply because of energy conservation:
the energy of the LA phonon on the LHS is already higher than the energy of the TA phonon on the RHS.
The $\mathrm{LA} + \mathrm{LA} \leftrightarrow \mathrm{TA}$ process 
is similarly impossible.
So the only remaining processes are $\mathrm{TA} + \mathrm{TA} \leftrightarrow \mathrm{LA}$
and $\mathrm{TA} + \mathrm{LA} \leftrightarrow \mathrm{LA}$.
It's easy to construct instances of the two processes:
we can just pick two points on the TA spectrum, 
and it's always possible to draw a concave curve that passes the sum of the two,
which is the LA spectrum;
similarly it's always possible to draw a concave curve -- which is the TA spectrum -- that passes the difference 
between two points on the LA spectrum.

\end{itemize}

\end{document}