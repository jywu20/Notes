\documentclass[hyperref, a4paper]{article}

\usepackage{geometry}
\usepackage{titling}
\usepackage{titlesec}
% No longer needed, since we will use enumitem package
% \usepackage{paralist}
\usepackage{enumitem}
\usepackage{footnote}
\usepackage{enumerate}
\usepackage{amsmath, amssymb, amsthm}
\usepackage{mathtools}
\usepackage{bbm}
\usepackage{cite}
\usepackage{graphicx}
\usepackage{subfigure}
\usepackage{physics}
\usepackage{tensor}
\usepackage{siunitx}
\usepackage[version=4]{mhchem}
\usepackage{tikz}
\usepackage{xcolor}
\usepackage{listings}
\usepackage{autobreak}
\usepackage[ruled, vlined, linesnumbered]{algorithm2e}
\usepackage{nameref,zref-xr}
\zxrsetup{toltxlabel}
\usepackage[colorlinks,unicode]{hyperref} % , linkcolor=black, anchorcolor=black, citecolor=black, urlcolor=black, filecolor=black
\usepackage[most]{tcolorbox}
\usepackage{prettyref}

% Page style
\geometry{left=3.18cm,right=3.18cm,top=2.54cm,bottom=2.54cm}
\titlespacing{\paragraph}{0pt}{1pt}{10pt}[20pt]
\setlength{\droptitle}{-5em}
%\preauthor{\vspace{-10pt}\begin{center}}
%\postauthor{\par\end{center}}

% More compact lists 
\setlist[itemize]{
    itemindent=17pt, 
    leftmargin=1pt,
    listparindent=\parindent,
    parsep=0pt,
}

% Math operators
\DeclareMathOperator{\timeorder}{\mathcal{T}}
\DeclareMathOperator{\diag}{diag}
\DeclareMathOperator{\legpoly}{P}
\DeclareMathOperator{\primevalue}{P}
\DeclareMathOperator{\sgn}{sgn}
\newcommand*{\ii}{\mathrm{i}}
\newcommand*{\ee}{\mathrm{e}}
\newcommand*{\const}{\mathrm{const}}
\newcommand*{\suchthat}{\quad \text{s.t.} \quad}
\newcommand*{\argmin}{\arg\min}
\newcommand*{\argmax}{\arg\max}
\newcommand*{\normalorder}[1]{: #1 :}
\newcommand*{\pair}[1]{\langle #1 \rangle}
\newcommand*{\fd}[1]{\mathcal{D} #1}
\DeclareMathOperator{\bigO}{\mathcal{O}}

% TikZ setting
\usetikzlibrary{arrows,shapes,positioning}
\usetikzlibrary{arrows.meta}
\usetikzlibrary{decorations.markings}
\tikzstyle arrowstyle=[scale=1]
\tikzstyle directed=[postaction={decorate,decoration={markings,
    mark=at position .5 with {\arrow[arrowstyle]{stealth}}}}]
\tikzstyle ray=[directed, thick]
\tikzstyle dot=[anchor=base,fill,circle,inner sep=1pt]

% Algorithm setting
% Julia-style code
\SetKwIF{If}{ElseIf}{Else}{if}{}{elseif}{else}{end}
\SetKwFor{For}{for}{}{end}
\SetKwFor{While}{while}{}{end}
\SetKwProg{Function}{function}{}{end}
\SetArgSty{textnormal}

\newcommand*{\concept}[1]{{\textbf{#1}}}

% Embedded codes
\lstset{basicstyle=\ttfamily,
  showstringspaces=false,
  commentstyle=\color{gray},
  keywordstyle=\color{blue}
}

% Reference formatting
\newrefformat{fig}{Figure~\ref{#1}}

% Color boxes
\tcbuselibrary{skins, breakable, theorems}
\newtcbtheorem[number within=section]{warning}{Warning}%
  {colback=orange!5,colframe=orange!65,fonttitle=\bfseries, breakable}{warn}
\newtcbtheorem[number within=section]{note}{Note}%
  {colback=green!5,colframe=green!65,fonttitle=\bfseries, breakable}{note}
\newtcbtheorem[number within=section]{info}{Info}%
  {colback=blue!5,colframe=blue!65,fonttitle=\bfseries, breakable}{info}

\newenvironment{shelldisplay}{\begin{lstlisting}}{\end{lstlisting}}

\title{Solid State Physics Homework 3}
\author{Jinyuan Wu}

\begin{document}

\maketitle

\paragraph{Problem 1} Consider the ionic crystal $\mathrm{NaCl}$. Model the total energy of the system in the standard way as a sum of an attractive electrostatic term (Madelung part) and a repulsive part with the usual $1 / r^{12}$ dependence between atom pairs. Show that at the equilibrium conventional lattice parameter $a$ (or equivalently at the equilibrium nearest neighbor distance $R$ ), the total energy is dominated to high precision by the electrostatic part and that the repulsive contribution is small. [Note: this happens because the electrostatic part is gentle and changing like $1 / r$ while the repulsive one is very steep.]

\paragraph{Solution} The total energy is 
\begin{equation}
    E = \sum_{i \neq j} E_{ij} = \frac{N}{2} \sum_{i \neq 0} E_{0i}.
    \label{eq:total-energy}
\end{equation}
We can replace the integer index $\vb*{i}$ by $\vb*{i}$,
where $ a /2 \cdot (i_x, i_y, i_z)$ are the coordinates of each atom in the crystal.
(That is, $\vb*{i}$ is \emph{not} the index of unit cells.)
Without the loss of generality,
we assume at $\vb*{i}=0$ lies a Na atom.
Then, if $i_x + i_y + i_z$ is an odd number,
then the atom at $\vb*{i}$ is a Cl atom; 
otherwise it's a Na atom.
So we have 
\begin{equation}
    E_{0i} = \frac{e^2}{4\pi \epsilon_0} \frac{(-1)^{i_x + i_y + i_z}}{\frac{a}{2} \sqrt{i_x^2 + i_y^2 + i_z^2}} 
    + \frac{C}{\left(\frac{a}{2}\right)^{12} \left( i_x^2 + i_y^2 + i_z^2 \right)^6}.
    \label{eq:separate-energy}
\end{equation}
The values of
\begin{equation}
    \sum_{\vb*{i} \neq 0} \frac{(-1)^{i_x + i_y + i_z}}{\sqrt{i_x^2 + i_y^2 + i_z^2}} 
    \label{eq:coefficient1}
\end{equation}
and 
\begin{equation}
    \sum_{\vb*{i} \neq 0} \frac{1}{ \left( i_x^2 + i_y^2 + i_z^2 \right)^6}.
    \label{eq:coefficient2}
\end{equation}
have already been calculated in the literature,
but they actually don't matter here:
from \eqref{eq:total-energy} and \eqref{eq:separate-energy} we know 
\begin{equation}
    E = \frac{C_1}{a} - \frac{C_2}{a^{12}}, \quad C_1, C_2 > 0.
    \label{eq:abstract-energy}
\end{equation}
The first term is the electrostatic term,
and the second term is the Lennard-Jones term.
Now by taking the derivative we can find the minimum point:
\[
    \dv{E}{a} = - \frac{C_1}{a^2} + \frac{12 C_2}{a^{13}} = 0 \Rightarrow a^{11} = \frac{12 C_2}{C_1}.
\]
Putting this back to \eqref{eq:abstract-energy}, we get 
\begin{equation}
    E = - \frac{C_1}{a_{\text{min}}} \left( 1 - \frac{C_2}{C_1} \frac{1}{a_{\text{min}}^{11}} \right) 
    = - \frac{11}{12} \frac{C_1}{a} = 0.92 E_{\text{electrostatic}}.
\end{equation}
Note that this result doesn't depend on the exact value of $C_1$ and $C_2$:
as long as the repulsive part is $\sim a^{-12}$,
the total energy is 0.92 times of the electrostatic energy.
When the repulsive potential goes steeper,
the total energy and the electrostatic energy get closer
($12/13$, $13/14$, and so on).

\paragraph{Problem 2} Consider the compound $\mathrm{La}_2 \mathrm{O}_3$ in its lowest energy crystal structure. What do you believe is the dominant type of bonding in this material? Look up the ionization energies of $\mathrm{Y}$ and $\mathrm{La}$ and thereby decide which of $\mathrm{La}_2 \mathrm{O}_3$ and $\mathrm{Y}_2 \mathrm{O}_3$ has more covalent character (assume for this problem that they have same crystal structure).

\paragraph{Solution} 
\href{https://www.angelo.edu/faculty/kboudrea/periodic/trends\_ionization\_energy.htm}{This webpage}
lists the ionization energies together with the periodic table.
The ionization energy of Y is \SI{616}{kJ/mol}.
The ionization energy of La is \SI{538}{kJ/mol}.
The electron affinity of O is \SI{141}{kJ/mol}, 
so O atoms easily get electrons and La atoms easily lose electrons
and thus the dominant type of bonding in \ce{La2O3} is likely to be ionic bond.
The ionization energy of Y is higher,
so Y is harder to ionize,
so \ce{Y2O3} has more covalent character.

Note that it's not sufficient to just look at the ionization of O
and conclude that O atoms easily get electrons,
because it's of course possible for an atom to reject both ionization and electron injection.

\paragraph{Problem 3} Consider a one-dimensional monoatomic chain of atoms as discussed in Kittel or A\&M: e.g., the situation described by equation (2) of Chapter 4 of Kittel: $M \ddot{u}_s=C\left(u_{s+1}+u_{s-1}-2 u_s\right)$. Following Kittel's notations, $C$ is the spring constant between neighbors and $M$ is the mass of each atom. Show that for long wavelength vibrational modes, the displacements $u$ obey the continuum wave equation
$$
\frac{\partial^2 u}{\partial t^2}=v^2 \frac{\partial^2 u}{\partial x^2}
$$
Where $u$ is now a function of continuous $x$ and $v$ is the velocity of sound. Hint: the main issue to convert from discrete positions (integer $s$ in Kittel labelling the atoms that spaced by $a$ ) to continuous ones and figure out how to convert the derivative. The simplest approach is to use $x=a$ and then assume that changing by one lattice unit $\pm a$ is a small displacement so you can Taylor expand in powers of the shift $a$. Long wavelength means $u$ varies gently in space.

\paragraph{Solution} In the long wavelength limit,
the lattice is almost continuous,
and the distance between neighbor atoms is small enough.
Thus 
\begin{equation}
    \pdv{u}{x} \approx \frac{u(x) - u(x-a)}{a} = \frac{u_{i} - u_{i-1}}{a},
\end{equation}
and 
\begin{equation}
    \begin{aligned}
        \pdv[2]{u}{x} &= \frac{1}{a} 
        \left( \eval{\pdv{u}{x}}_{x + a} - \eval{\pdv{u}{x}}_{x} \right) \\
        &= \frac{1}{a} \left( \frac{u_{i+1} - u_i}{a} - \frac{u_i - u_{i-1}}{a} \right) \\
        &= \frac{u_{i+1} - 2u_i + u_{i-1}}{a^2}.
    \end{aligned}
\end{equation}
So the EOM of atoms 
\begin{equation}
    M \ddot{u}_i = C (u_{i+1} + u_{i-1} - 2u_i)
\end{equation}
now reads 
\[
    M \pdv[2]{u}{t} = C a^2 \pdv[2]{u}{x},
\]
and we get 
\begin{equation}
    \pdv[2]{u}{t} = v^2 \pdv[2]{u}{x}, \quad v^2 = \frac{C a^2}{M}.
\end{equation}

\end{document}