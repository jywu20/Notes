\documentclass[hyperref, a4paper]{article}

\usepackage{geometry}
\usepackage{titling}
\usepackage{titlesec}
% No longer needed, since we will use enumitem package
% \usepackage{paralist}
\usepackage{enumitem}
\usepackage{footnote}
\usepackage{enumerate}
\usepackage{amsmath, amssymb, amsthm}
\usepackage{mathtools}
\usepackage{bbm}
\usepackage{cite}
\usepackage{graphicx}
\usepackage{subfigure}
\usepackage{physics}
\usepackage{tensor}
\usepackage{siunitx}
\usepackage[version=4]{mhchem}
\usepackage{tikz}
\usepackage{xcolor}
\usepackage{listings}
\usepackage{autobreak}
\usepackage[ruled, vlined, linesnumbered]{algorithm2e}
\usepackage{nameref,zref-xr}
\zxrsetup{toltxlabel}
\zexternaldocument*[hw2-]{../2/2}[2.pdf]
\usepackage[colorlinks,unicode]{hyperref} % , linkcolor=black, anchorcolor=black, citecolor=black, urlcolor=black, filecolor=black
\usepackage[most]{tcolorbox}
\usepackage{prettyref}

% Page style
\geometry{left=3.18cm,right=3.18cm,top=2.54cm,bottom=2.54cm}
\titlespacing{\paragraph}{0pt}{1pt}{10pt}[20pt]
\setlength{\droptitle}{-5em}
\preauthor{\vspace{-10pt}\begin{center}}
\postauthor{\par\end{center}}

% More compact lists 
\setlist[itemize]{
    itemindent=17pt, 
    leftmargin=1pt,
    listparindent=\parindent,
    parsep=0pt,
}

% Math operators
\DeclareMathOperator{\timeorder}{\mathcal{T}}
\DeclareMathOperator{\diag}{diag}
\DeclareMathOperator{\legpoly}{P}
\DeclareMathOperator{\primevalue}{P}
\DeclareMathOperator{\sgn}{sgn}
\newcommand*{\ii}{\mathrm{i}}
\newcommand*{\ee}{\mathrm{e}}
\newcommand*{\const}{\mathrm{const}}
\newcommand*{\suchthat}{\quad \text{s.t.} \quad}
\newcommand*{\argmin}{\arg\min}
\newcommand*{\argmax}{\arg\max}
\newcommand*{\normalorder}[1]{: #1 :}
\newcommand*{\pair}[1]{\langle #1 \rangle}
\newcommand*{\fd}[1]{\mathcal{D} #1}
\DeclareMathOperator{\bigO}{\mathcal{O}}

% TikZ setting
\usetikzlibrary{arrows,shapes,positioning}
\usetikzlibrary{arrows.meta}
\usetikzlibrary{decorations.markings}
\tikzstyle arrowstyle=[scale=1]
\tikzstyle directed=[postaction={decorate,decoration={markings,
    mark=at position .5 with {\arrow[arrowstyle]{stealth}}}}]
\tikzstyle ray=[directed, thick]
\tikzstyle dot=[anchor=base,fill,circle,inner sep=1pt]

% Algorithm setting
% Julia-style code
\SetKwIF{If}{ElseIf}{Else}{if}{}{elseif}{else}{end}
\SetKwFor{For}{for}{}{end}
\SetKwFor{While}{while}{}{end}
\SetKwProg{Function}{function}{}{end}
\SetArgSty{textnormal}

\newcommand*{\concept}[1]{{\textbf{#1}}}

% Embedded codes
\lstset{basicstyle=\ttfamily,
  showstringspaces=false,
  commentstyle=\color{gray},
  keywordstyle=\color{blue}
}

% Reference formatting
\newrefformat{fig}{Figure~\ref{#1}}

% Color boxes
\tcbuselibrary{skins, breakable, theorems}
\newtcbtheorem[number within=section]{warning}{Warning}%
  {colback=orange!5,colframe=orange!65,fonttitle=\bfseries, breakable}{warn}
\newtcbtheorem[number within=section]{note}{Note}%
  {colback=green!5,colframe=green!65,fonttitle=\bfseries, breakable}{note}
\newtcbtheorem[number within=section]{info}{Info}%
  {colback=blue!5,colframe=blue!65,fonttitle=\bfseries, breakable}{info}

\newenvironment{shelldisplay}{\begin{lstlisting}}{\end{lstlisting}}

\title{Solid State Physics Homework 2}
\author{Jinyuan Wu}

\begin{document}

\maketitle

\paragraph{Problem 1}

\paragraph{Solution} $(100)$ planes are parallel to $\vb*{a}_3 - \vb*{a}_1$ and $\vb*{a}_2$,
so the $\vb*{a}_1$-intercept and the $\vb*{a}_3$-intercept are the same,
while the $\vb*{a}_2$-intercept is $\infty$.
So the new indices are 
\[
    1 : \frac{1}{\infty} : 1 = 1 : 0 : 1,
\]
i.e. $(101)$. Similarly, the new indices for the $(001)$ planes are $(011)$.

\paragraph{Problem 2}

\paragraph{Solution} 
\begin{itemize}
\item[(a)] Since 
\[
    \vb*{a} \cdot \vb*{b}_j = 2 \pi \delta_{ij},
\]
suppose $\{x_i\}$ are the coordinates based on $\{\vb*{a}_i\}$, i.e. 
\begin{equation}
    \vb*{r} = \sum_{i=1}^3 x_i \vb*{a}_i,
\end{equation}
we have 
\begin{equation}
    x_i = \frac{1}{2\pi} \vb*{b}_i \cdot \vb*{r}.
\end{equation}
So 
\begin{equation}
    \grad{x_i} = \frac{1}{2 \pi} \vb*{b}_i.
\end{equation}
The equation of a $(hkl)$ plane is 
\begin{equation}
    h x_1 + k x_2 + l x_3 = \text{const},
\end{equation}
so one of the normal vector of the plane is given by 
\begin{equation}
    \grad{(h x_1 + k x_2 + l x_3)} = \frac{h}{2\pi} \vb*{b}_1 + \frac{k}{2\pi} \vb*{b}_2 + \frac{l}{2\pi} \vb*{b}_3,
\end{equation}
and of course this is parallel to $\vb*{G}$,
so $\vb*{G}$ is perpendicular to the plane.

\item[(b)] There has to be a lattice vector connecting two $(hkl)$ planes.
The distance between the two planes is therefore 
\begin{equation}
    d = \frac{\abs{\vb*{G}' \cdot \vb*{G}}}{\abs{\vb*{G}}} 
    = \frac{2\pi \abs{h x_1' + k x_2' + l x_3'}}{\abs{\vb*{G}}}, \quad 
    \vb*{G}' = \sum_{i=1}^3 x_i' \vb*{b}_i.
\end{equation}
For two adjacent planes, we should take the non-zero minimum value of $d$.
Of course 
\[
    \abs{h x_1' + k x_2' + l x_3'} \in \mathbb{N},
\]
and a elementary number theory theorem tells us 
that 1 is a linear combination of any coprime integers,
so the non-zero minimum of $\abs{h x_1' + k x_2' + l x_3'}$ is 1,
and thus 
\begin{equation}
    d_{\min} = \frac{2\pi}{\abs{\vb*{G}}}.
\end{equation}

\item[(c)] For a simple cubic lattice, 
$\vb*{a}_i$ vectors are orthogonal to each other,
and so are $\vb*{b}_i$ vectors.
The condition $\vb*{a}_i \cdot \vb*{b}_j = 2\pi \delta_{ij}$ means 
the length of all $\vb*{b}_i$ vectors is $2\pi / a$,
so 
\[
    \abs{\vb*{G}} = \sqrt{ \left(\frac{2\pi}{a}\right)^2 (h^2 + k^2 + l^2) },
\]
and therefore 
\begin{equation}
    d^2 = \frac{a^2}{h^2 + k^2 + l^2}.
\end{equation}

\end{itemize}

\paragraph{Problem 3}

\paragraph{Solution} 

\paragraph{Problem 4}

\paragraph{Solution} Since the volume of all primitive cells of the same lattice is the same,
we can just calculate the volume of the parallelepiped spanned by $\{\vb*{b}_i\}$.
The condition $\vb*{a}_i \cdot \vb*{b}_j = 2\pi \delta_{ij}$ can be rewritten as 
\[
    \pmqty{ \vb*{b}_1^\top \\ \vb*{b}_2^\top \\ \vb*{b}_3^\top } \pmqty{ \vb*{a}_1 & \vb*{a}_2 & \vb*{a}_3 } = 2\pi I_{3 \times 3},
\]
and by taking the determinant of the equation we get 
\[
    \det \pmqty{ \vb*{b}_1^\top \\ \vb*{b}_2^\top \\ \vb*{b}_3^\top } \det \pmqty{ \vb*{a}_1 & \vb*{a}_2 & \vb*{a}_3 } = (2\pi)^3 = \det \pmqty{ \vb*{b}_1 & \vb*{b}_2 & \vb*{b}_3 } \det \pmqty{\vb*{a}_1 & \vb*{a}_2 & \vb*{a}_3},
\]
so 
\[
    V_{\text{inverse space unit cell}} V_{\text{unit cell}} = (2\pi)^3,
\]
\begin{equation}
    V_{\text{Brillouin}} = \frac{(2\pi)^3}{V_{\text{unit cell}}}.
\end{equation}

\paragraph{Problem 5} 

\end{document}