\documentclass[hyperref, a4paper]{article}

\usepackage{geometry}
\usepackage{titling}
\usepackage{titlesec}
% No longer needed, since we will use enumitem package
% \usepackage{paralist}
\usepackage{enumitem}
\usepackage{footnote}
\usepackage{enumerate}
\usepackage{amsmath, amssymb, amsthm}
\usepackage{mathtools}
\usepackage{bbm}
\usepackage{cite}
\usepackage{soulutf8}
\usepackage{graphicx}
\usepackage{subfigure}
\usepackage{physics}
\usepackage{tensor}
\usepackage{siunitx}
\usepackage[version=4]{mhchem}
\usepackage{tikz}
\usepackage{xcolor}
\usepackage{listings}
\usepackage{autobreak}
\usepackage[ruled, vlined, linesnumbered]{algorithm2e}
\usepackage{nameref,zref-xr}
\zxrsetup{toltxlabel}
\usepackage[colorlinks,unicode]{hyperref} % , linkcolor=black, anchorcolor=black, citecolor=black, urlcolor=black, filecolor=black
\usepackage[most]{tcolorbox}
\usepackage{prettyref}

% Page style
\geometry{left=3.18cm,right=3.18cm,top=2.54cm,bottom=2.54cm}
\titlespacing{\paragraph}{0pt}{1pt}{10pt}[20pt]
\setlength{\droptitle}{-5em}
%\preauthor{\vspace{-10pt}\begin{center}}
%\postauthor{\par\end{center}}

% More compact lists 
\setlist[itemize]{
    itemindent=17pt, 
    leftmargin=1pt,
    listparindent=\parindent,
    parsep=0pt,
}

% Math operators
\DeclareMathOperator{\timeorder}{\mathcal{T}}
\DeclareMathOperator{\diag}{diag}
\DeclareMathOperator{\legpoly}{P}
\DeclareMathOperator{\primevalue}{P}
\DeclareMathOperator{\sgn}{sgn}
\newcommand*{\ii}{\mathrm{i}}
\newcommand*{\ee}{\mathrm{e}}
\newcommand*{\const}{\mathrm{const}}
\newcommand*{\suchthat}{\quad \text{s.t.} \quad}
\newcommand*{\argmin}{\arg\min}
\newcommand*{\argmax}{\arg\max}
\newcommand*{\normalorder}[1]{: #1 :}
\newcommand*{\pair}[1]{\langle #1 \rangle}
\newcommand*{\fd}[1]{\mathcal{D} #1}
\DeclareMathOperator{\bigO}{\mathcal{O}}

% TikZ setting
\usetikzlibrary{arrows,shapes,positioning}
\usetikzlibrary{arrows.meta}
\usetikzlibrary{decorations.markings}
\tikzstyle arrowstyle=[scale=1]
\tikzstyle directed=[postaction={decorate,decoration={markings,
    mark=at position .5 with {\arrow[arrowstyle]{stealth}}}}]
\tikzstyle ray=[directed, thick]
\tikzstyle dot=[anchor=base,fill,circle,inner sep=1pt]

% Algorithm setting
% Julia-style code
\SetKwIF{If}{ElseIf}{Else}{if}{}{elseif}{else}{end}
\SetKwFor{For}{for}{}{end}
\SetKwFor{While}{while}{}{end}
\SetKwProg{Function}{function}{}{end}
\SetArgSty{textnormal}

\newcommand*{\concept}[1]{{\textbf{#1}}}

% Embedded codes
\lstset{basicstyle=\ttfamily,
  showstringspaces=false,
  commentstyle=\color{gray},
  keywordstyle=\color{blue}
}

% Reference formatting
\newrefformat{fig}{Figure~\ref{#1}}

% Color boxes
\tcbuselibrary{skins, breakable, theorems}
\newtcbtheorem[number within=section]{warning}{Warning}%
  {colback=orange!5,colframe=orange!65,fonttitle=\bfseries, breakable}{warn}
\newtcbtheorem[number within=section]{note}{Note}%
  {colback=green!5,colframe=green!65,fonttitle=\bfseries, breakable}{note}
\newtcbtheorem[number within=section]{info}{Info}%
  {colback=blue!5,colframe=blue!65,fonttitle=\bfseries, breakable}{info}

\newenvironment{shelldisplay}{\begin{lstlisting}}{\end{lstlisting}}

\newcommand{\address}[1]{\href{#1}{\url{#1}}}

%region Constants 

\newcommand{\kB}{k_{\text{B}}}
\newcommand{\Eg}{E_{\text{e}}}
\newcommand{\Ec}{E_{\text{c}}}
\newcommand{\Ev}{E_{\text{v}}}
\newcommand{\me}{m_{\text{e}}}
\newcommand{\mh}{m_{\text{h}}}
\newcommand{\Ed}{E_{\text{d}}}
\newcommand{\Ea}{E_{\text{a}}}
\newcommand{\nd}{n_{\text{d}}}
\newcommand{\Nd}{N_{\text{d}}}
\newcommand{\pa}{p_{\text{a}}}
\newcommand{\Na}{N_{\text{a}}}

%endregion

\title{Semiconductors}
\author{Jiinyuan Wu}

\begin{document}

\maketitle

In the next several sections, we discuss the following topics: 
\begin{itemize}
    \item The behavior of the semiconductor when doping is 
    low/relatively high/so high that an ``impurity band'' forms.
    \item The role of temperature.
    \item Junctions.
\end{itemize}

\section{Two-band model}

At $T = 0$ there is strictly no such thing as a semiconductor:
there are just band metals or band insulators.
When $T > 0$, however, 
the following two mechanisms provide a band insulator 
with non-zero carrier concentration:
\begin{itemize}
    \item The first is that electrons jump from the valence band to the conduction band
    because of thermal fluctuation, or more specifically, 
    because of Fermi-Dirac distribution.
    \item The second is dopoing adds energy levels 
    near the highest point of the valence band ($E_{\text{v}}$)
    or the lowest point of the conduction band ($E_{\text{c}}$),
    and again, thermally excited electrons will jump to them.
    From another point of view we may say doped atoms eat electrons or give electrons,
    so some electrons are missing if we only look at Bloch states in the spectrum.
\end{itemize}
In both mechanisms,
we have electrons in the conduction band and holes in the valence band,
which are \concept{carriers} of electric current.

In the discussion below, we assume that 
\ul{the band gap is \emph{much larger} than $k_{\text{B}} T$},
and this means when $E \geq E_{\text{c}}$,
\begin{equation}
    n_{\text{electron}} = \ee^{- (E - \mu) / k_{\text{B}} T},
\end{equation}
and when $E \leq E_{\text{v}}$
\begin{equation}
    n_{\text{hole}} = \ee^{- (\mu - E) / k_{\text{B}} T}.
\end{equation}
The conditions are usually satisfied 
in most semiconductors; 
in more accurate numerical predictions, of course, 
we don't really need the approximations. 
In semiconductor physics we usually use the following abbreviations:
\begin{itemize}
    \item $p$ (positive), v (valence), h (hole) mean holes in the valence band, and 
    \item $n$ (negative), c (conduction), e (electron) mean electrons in the conduction band.
\end{itemize}
Thus the density of electrons and holes are given by 
\begin{equation}
    n = \int_{E_\text{c}}^\infty D_{\text{c}} (E) \dd{E} \ee^{- (E - \mu) / k_{\text{B}} T},
\end{equation}
and 
\begin{equation}
    p = \int^{E_{\text{v}}}_{-\infty} D_{\text{v}} (E) \dd{E} \ee^{- (\mu - E) / k_{\text{B}} T}.
\end{equation}

Consider the simplest case, where we have a band gap (possibly indirect) 
\begin{equation}
    E_{\text{g}} = E_{\text{c}} - E_{\text{v}}
\end{equation}
between the highest valence band the the lowest conduction band.
The $\Delta E \gg k_{\text{B}} T$ condition 
indicates that \ul{only the top of the highest valence band 
and the bottom of the lowest conduction band matters,}  
and therefore the hyperbolic approximation works, 
and the conduction band is completely characterized by its effective mass 
and the momentum with lowest energy,
and so is the case for the valence band.
We have 
\begin{equation}
    D(E) = \frac{1}{2 \pi^2} \left( \frac{2 m^*}{\hbar^2} \right) \sqrt{E - E_{\text{max/min}}}
\end{equation}
for hyperbolic bands,
and therefore we have (note that in the valence band, the effective mass of the electron is negative,
so the effective mass of the hole is positive) 
\begin{equation}
    n = 2 \left( \frac{m_\text{e}^* k_{\text{B}} T}{2 \pi \hbar^2} \right)^{3/2} 
    \ee^{- (E_{\text{c}} - \mu) / k_{\text{B}} T},
    \label{eq:n-def}
\end{equation}
\begin{equation}
    p = 2 \left( \frac{m_\text{h}^* k_{\text{B}} T}{2 \pi \hbar^2} \right)^{3/2} 
    \ee^{- (\mu - E_{\text{v}}) / k_{\text{B}} T}.
\end{equation}
Note that if there is doping, electrons donated or attracted by the doped atoms 
are not included in the above equations.
Thus, we find 
\begin{equation}
    n \cdot p = 4 \left(
        \frac{\kB T}{2 \pi \hbar^2} 
    \right)^3
    (m_{\text{e}}^* m_{\text{h}}^*)^{3/2} \ee^{- E_{\text{g}} / \kB T}. 
    \label{eq:n-p-product}
\end{equation}
Note that this result doesn't contain $\mu$;
specifically, its derivation involves no information about doping: 
whatever the doping is, 
the product of the number of holes in the valence band 
and the number of electrons in the conduction band 
is always decided by the shape of the two bands,
the band gap, and the temperature.

\section{The intrinsic semiconductor limit}

In the \concept{intrinsic semiconductor limit},
\ul{there is no doping}, and only the first mechanism 
-- thermal excitation in the conduction band and the valence band --
works. 
This limit is useful 
when $T$ is very high so the effect of doping can be ignored.
So in the opposite, 
\emph{when $T$ is low enough, 
even materials that are very clean can't be described well by the intrinsic semiconductor limit},
because in this case, if the material has any semiconductor property,
then it has to come from doping.

\subsection{Isotropic two-band model}

Since \ul{there is no doping, we have $n = p$ because of charge neutrality,
and we have $n = \sqrt{np}$},
and thus from \eqref{eq:n-p-product}, we find 
\begin{equation}
    \mu = E_{\text{v}} + \frac{1}{2} E_{\text{g}} + \frac{3}{4} k_{\text{B}} T \ln(\frac{m_\text{h}^*}{m_{\text{e}}^*}).
    \label{eq:mu-nat}
\end{equation}
Here we can see that when $T = 0$, 
actually $\mu \neq E_{\text{F}} = E_{\text{v}}$,
but this doesn't matter:
when $T$ is zero and we are working with an insulator,
putting $\mu$ anywhere between $E_{\text{c}}$ and $E_{\text{v}}$ is acceptable.

Here Drude model works, because the density of electrons and holes is small 
(TODO: why no strong correlation, like Wigner crystal?),
and therefore the average distance between electrons 
is much larger than the thermal de Broglie wave length,
and the quantum nature of electrons isn't apparent.
(TODO: this is what Sohrab said; but Drude model still works 
for conductivity even in the quantum case so I don't know what he meant here)
So we have 
\begin{equation}
    \sigma_\text{e} = n \mu_{\text{e}}, \quad 
    \sigma_{\text{H}} = p \mu_{\text{H}},
\end{equation}
and the mobilities are 
\begin{equation}
    \mu_{\text{e}} = - \frac{\tau e}{m^*_{\text{e}}}, \quad 
    \mu_{\text{h}} =   \frac{\tau e}{m^*_{\text{h}}}.
    \label{eq:mobilities}
\end{equation}
Usually the mobility of holes isn't as good as electrons because in general 
$m_{\text{h}}^* > m_{\text{e}}^*$,
and for large effective mass we expect low mobility.
This can be found by looking at \eqref{eq:mobilities},
and \eqref{eq:mobilities} comes from the semiclassical EOM of electrons and holes,
so we can also explain the fact from a quantum perspective.
Suppose a hole is stuck in the lattice 
(it may slightly distort the lattice,
so what is stuck is actually a hole and some phonons, 
or a polaron).
It may still tunnel away because $x$ is localized 
and $p$ is uncertain, which may give it a kinetic energy large enough to escape,
but if $m^*_{\text{h}}$ is large enough, 
the kinetic energy fluctuation can be ignored,
so the hole is safely localized.

\subsection{The anisotropic two-band model}

When the anisotropy of the lattice is strong, we have 
\begin{equation}
    n = 2 \left( \frac{k_{\text{B}} T}{2 \pi \hbar^2} \right)^{3/2} (\tilde{m}_{\text{e}}^* \tilde{m}_{\text{h}}^*)^{3/4} \ee^{- E_{\text{g}} / k_{\text{B}} T},
\end{equation}
where $\tilde{m}$ means $(\det \vb*{M})^{1/3} = \sqrt[3]{m_1^* m_2^* m_3^*}$,
with the three $m^*$'s being the eigenvalues of the mass matrix,
which is defined as 
\begin{equation}
    E = E_0 \pm \frac{1}{2} \vb*{k} \cdot \vb*{M}^{-1} \cdot \vb*{k},
\end{equation}
where $E_0$ is the bottom/top of the band.

\subsection{More valleys}

It's possible that we have two valleys in the conduction bands
or in the valence bands; 
the latter happens more frequently, for some reasons.
In this case a analytic solution is hard to get.

\section{Doping}

The overall effect of doping is to introduce a difference 
between the number of electrons/holes in the conduction band/valence band:
\begin{equation}
    \Delta n = n - p \neq 0.
\end{equation}
Suppose $n_{\text{i}}$ and $p_{\text{i}}$ are number of electrons and holes 
in the intrinsic semiconductor, 
given by \eqref{eq:n-def} and \eqref{eq:mu-nat}.
Recall that \eqref{eq:n-p-product} is obtained 
without assuming that the system is intrinsic,
and that $n_{\text{i}}, p_{\text{i}}$ also satisfy the equation,
and we find 
\begin{equation}
    n \cdot p = n_{\text{i}}^2 = p_{\text{i}}^2,
\end{equation}
which in turns reads  
\begin{equation}
    n = \frac{\Delta n + \sqrt{4 n_{\text{i}}^2 + \Delta n^2}}{2}, \quad 
    p = \frac{-\Delta n + \sqrt{4 n_{\text{i}}^2 + \Delta n^2}}{2}.
\end{equation}

The \concept{extrinsic limit} is when $\Delta n \gg n_{\text{i}}$.
Now the problem is how to find $\Delta n$
(and therefore to find how we can achieve the extrinsic limit).
This can be done by calculating the particle number expectation of each impurity,
which, in turn, reduces to the problem 
of calculating the electronic energy levels 
introduced by the impurities.

\begin{figure}
    \centering
    \begin{tikzpicture}[x=0.75pt,y=0.75pt,yscale=-0.9,xscale=0.9]
    %uncomment if require: \path (0,332); %set diagram left start at 0, and has height of 332
    
    %Curve Lines [id:da5188436226725119] 
    \draw [color={rgb, 255:red, 74; green, 144; blue, 226 }  ,draw opacity=1 ][line width=1.5]    (272,52) .. controls (309.17,144.94) and (332.17,146.94) .. (366,52) ;
    %Curve Lines [id:da8348945462819415] 
    \draw [color={rgb, 255:red, 208; green, 2; blue, 27 }  ,draw opacity=1 ][line width=1.5]    (273,201) .. controls (296.17,180.94) and (340.17,177.94) .. (367,200) ;
    %Straight Lines [id:da16499394209866347] 
    \draw  [dash pattern={on 0.84pt off 2.51pt}]  (322,141) -- (385.17,141) ;
    %Straight Lines [id:da49742747833693035] 
    \draw [color={rgb, 255:red, 155; green, 155; blue, 155 }  ,draw opacity=1 ][line width=1.5]    (310.92,141) -- (333.08,141) ;
    %Straight Lines [id:da9573480168402559] 
    \draw  [dash pattern={on 0.84pt off 2.51pt}]  (319,123) -- (386.17,123) ;
    %Straight Lines [id:da2519812942273514] 
    \draw    (386.17,126) -- (386.17,141) ;
    \draw [shift={(386.17,123)}, rotate = 90] [fill={rgb, 255:red, 0; green, 0; blue, 0 }  ][line width=0.08]  [draw opacity=0] (8.04,-3.86) -- (0,0) -- (8.04,3.86) -- (5.34,0) -- cycle    ;
    %Straight Lines [id:da35917903180614763] 
    \draw    (64,80) -- (164,180) ;
    %Straight Lines [id:da5449127693447489] 
    \draw    (164,80) -- (64,180) ;
    %Shape: Circle [id:dp8469242700102309] 
    \draw  [draw opacity=0][fill={rgb, 255:red, 255; green, 255; blue, 255 }  ,fill opacity=1 ] (97.42,130) .. controls (97.42,120.84) and (104.84,113.42) .. (114,113.42) .. controls (123.16,113.42) and (130.58,120.84) .. (130.58,130) .. controls (130.58,139.16) and (123.16,146.58) .. (114,146.58) .. controls (104.84,146.58) and (97.42,139.16) .. (97.42,130) -- cycle ;
    %Shape: Circle [id:dp39716505187272055] 
    \draw  [draw opacity=0][fill={rgb, 255:red, 255; green, 255; blue, 255 }  ,fill opacity=1 ] (47.42,80) .. controls (47.42,70.84) and (54.84,63.42) .. (64,63.42) .. controls (73.16,63.42) and (80.58,70.84) .. (80.58,80) .. controls (80.58,89.16) and (73.16,96.58) .. (64,96.58) .. controls (54.84,96.58) and (47.42,89.16) .. (47.42,80) -- cycle ;
    %Shape: Circle [id:dp4211717469825407] 
    \draw  [draw opacity=0][fill={rgb, 255:red, 255; green, 255; blue, 255 }  ,fill opacity=1 ] (147.42,80) .. controls (147.42,70.84) and (154.84,63.42) .. (164,63.42) .. controls (173.16,63.42) and (180.58,70.84) .. (180.58,80) .. controls (180.58,89.16) and (173.16,96.58) .. (164,96.58) .. controls (154.84,96.58) and (147.42,89.16) .. (147.42,80) -- cycle ;
    %Shape: Circle [id:dp7734322446909323] 
    \draw  [draw opacity=0][fill={rgb, 255:red, 255; green, 255; blue, 255 }  ,fill opacity=1 ] (47.42,180) .. controls (47.42,170.84) and (54.84,163.42) .. (64,163.42) .. controls (73.16,163.42) and (80.58,170.84) .. (80.58,180) .. controls (80.58,189.16) and (73.16,196.58) .. (64,196.58) .. controls (54.84,196.58) and (47.42,189.16) .. (47.42,180) -- cycle ;
    %Shape: Circle [id:dp5140224677059457] 
    \draw  [draw opacity=0][fill={rgb, 255:red, 255; green, 255; blue, 255 }  ,fill opacity=1 ] (147.42,180) .. controls (147.42,170.84) and (154.84,163.42) .. (164,163.42) .. controls (173.16,163.42) and (180.58,170.84) .. (180.58,180) .. controls (180.58,189.16) and (173.16,196.58) .. (164,196.58) .. controls (154.84,196.58) and (147.42,189.16) .. (147.42,180) -- cycle ;
    %Straight Lines [id:da262249620774625] 
    \draw [color={rgb, 255:red, 74; green, 144; blue, 226 }  ,draw opacity=0.5 ]   (120.17,173.94) .. controls (122.46,174.46) and (123.35,175.87) .. (122.83,178.17) .. controls (122.3,180.47) and (123.19,181.88) .. (125.49,182.4) .. controls (127.79,182.92) and (128.68,184.33) .. (128.16,186.63) .. controls (127.63,188.93) and (128.52,190.34) .. (130.82,190.86) -- (131.31,191.63) -- (135.57,198.4) ;
    \draw [shift={(137.17,200.94)}, rotate = 237.8] [fill={rgb, 255:red, 74; green, 144; blue, 226 }  ,fill opacity=0.5 ][line width=0.08]  [draw opacity=0] (8.04,-3.86) -- (0,0) -- (8.04,3.86) -- (5.34,0) -- cycle    ;
    %Straight Lines [id:da7557410318305315] 
    \draw [color={rgb, 255:red, 74; green, 144; blue, 226 }  ,draw opacity=1 ]   (116,164) ;
    \draw [shift={(116,164)}, rotate = 0] [color={rgb, 255:red, 74; green, 144; blue, 226 }  ,draw opacity=1 ][fill={rgb, 255:red, 74; green, 144; blue, 226 }  ,fill opacity=1 ][line width=0.75]      (0, 0) circle [x radius= 3.35, y radius= 3.35]   ;
    %Straight Lines [id:da2678167044323243] 
    \draw [color={rgb, 255:red, 155; green, 155; blue, 155 }  ,draw opacity=1 ] [dash pattern={on 0.84pt off 2.51pt}]  (62.25,130) -- (94.42,130) ;
    \draw [shift={(97.42,130)}, rotate = 180] [fill={rgb, 255:red, 155; green, 155; blue, 155 }  ,fill opacity=1 ][line width=0.08]  [draw opacity=0] (8.04,-3.86) -- (0,0) -- (8.04,3.86) -- (5.34,0) -- cycle    ;
    %Shape: Rectangle [id:dp7740652491535642] 
    \draw  [draw opacity=0][fill={rgb, 255:red, 74; green, 144; blue, 226 }  ,fill opacity=1 ] (447,149) -- (488.17,149) -- (488.17,213.94) -- (447,213.94) -- cycle ;
    %Shape: Rectangle [id:dp0011108970125532913] 
    \draw  [draw opacity=0][fill={rgb, 255:red, 255; green, 255; blue, 255 }  ,fill opacity=1 ] (447,36.94) -- (488.17,36.94) -- (488.17,87) -- (447,87) -- cycle ;
    %Straight Lines [id:da44871219468632595] 
    \draw [color={rgb, 255:red, 155; green, 155; blue, 155 }  ,draw opacity=1 ]   (447,144) -- (488.17,144) ;
    %Straight Lines [id:da9365477102540947] 
    \draw [color={rgb, 255:red, 155; green, 155; blue, 155 }  ,draw opacity=1 ]   (447,137) -- (488.17,137) ;
    %Straight Lines [id:da21868809074821405] 
    \draw [color={rgb, 255:red, 155; green, 155; blue, 155 }  ,draw opacity=1 ]   (447,92) -- (488.17,92) ;
    %Straight Lines [id:da7310429080965457] 
    \draw [color={rgb, 255:red, 155; green, 155; blue, 155 }  ,draw opacity=1 ]   (447,102) -- (488.17,102) ;
    %Straight Lines [id:da667054670910258] 
    \draw [color={rgb, 255:red, 74; green, 144; blue, 226 }  ,draw opacity=1 ]   (467.58,92) ;
    \draw [shift={(467.58,92)}, rotate = 0] [color={rgb, 255:red, 74; green, 144; blue, 226 }  ,draw opacity=1 ][fill={rgb, 255:red, 74; green, 144; blue, 226 }  ,fill opacity=1 ][line width=0.75]      (0, 0) circle [x radius= 3.35, y radius= 3.35]   ;
    %Straight Lines [id:da1349832059057845] 
    \draw [color={rgb, 255:red, 74; green, 144; blue, 226 }  ,draw opacity=1 ]   (467.58,102) ;
    \draw [shift={(467.58,102)}, rotate = 0] [color={rgb, 255:red, 74; green, 144; blue, 226 }  ,draw opacity=1 ][fill={rgb, 255:red, 74; green, 144; blue, 226 }  ,fill opacity=1 ][line width=0.75]      (0, 0) circle [x radius= 3.35, y radius= 3.35]   ;
    %Straight Lines [id:da5672556280582355] 
    \draw [color={rgb, 255:red, 155; green, 155; blue, 155 }  ,draw opacity=1 ]   (447,149) -- (447,213.94) ;
    %Straight Lines [id:da3182397944648898] 
    \draw [color={rgb, 255:red, 155; green, 155; blue, 155 }  ,draw opacity=1 ]   (488.17,149) -- (488.17,213.94) ;
    %Straight Lines [id:da042197917825606] 
    \draw [color={rgb, 255:red, 155; green, 155; blue, 155 }  ,draw opacity=1 ]   (447,149) -- (488.17,149) ;
    %Straight Lines [id:da16997068385226588] 
    \draw [color={rgb, 255:red, 155; green, 155; blue, 155 }  ,draw opacity=1 ]   (488.17,36.94) -- (488.17,87) ;
    %Straight Lines [id:da4187219489813663] 
    \draw [color={rgb, 255:red, 155; green, 155; blue, 155 }  ,draw opacity=1 ]   (447,36.94) -- (447,87) ;
    %Straight Lines [id:da4341449959145962] 
    \draw [color={rgb, 255:red, 155; green, 155; blue, 155 }  ,draw opacity=1 ]   (447,87) -- (488.17,87) ;
    %Shape: Rectangle [id:dp6646256065898335] 
    \draw  [draw opacity=0][fill={rgb, 255:red, 74; green, 144; blue, 226 }  ,fill opacity=1 ] (531,149) -- (572.17,149) -- (572.17,213.94) -- (531,213.94) -- cycle ;
    %Shape: Rectangle [id:dp6319220535315351] 
    \draw  [draw opacity=0][fill={rgb, 255:red, 255; green, 255; blue, 255 }  ,fill opacity=1 ] (531,36.94) -- (572.17,36.94) -- (572.17,87) -- (531,87) -- cycle ;
    %Straight Lines [id:da710563271317818] 
    \draw [color={rgb, 255:red, 155; green, 155; blue, 155 }  ,draw opacity=1 ]   (531,144) -- (572.17,144) ;
    %Straight Lines [id:da36401966636676586] 
    \draw [color={rgb, 255:red, 155; green, 155; blue, 155 }  ,draw opacity=1 ]   (531,137) -- (572.17,137) ;
    %Straight Lines [id:da15588078610133516] 
    \draw [color={rgb, 255:red, 155; green, 155; blue, 155 }  ,draw opacity=1 ]   (531,92) -- (572.17,92) ;
    %Straight Lines [id:da8182354772588185] 
    \draw [color={rgb, 255:red, 155; green, 155; blue, 155 }  ,draw opacity=1 ]   (531,102) -- (572.17,102) ;
    %Straight Lines [id:da037053585016112445] 
    \draw [color={rgb, 255:red, 74; green, 144; blue, 226 }  ,draw opacity=1 ]   (551.58,81) ;
    \draw [shift={(551.58,81)}, rotate = 0] [color={rgb, 255:red, 74; green, 144; blue, 226 }  ,draw opacity=1 ][fill={rgb, 255:red, 74; green, 144; blue, 226 }  ,fill opacity=1 ][line width=0.75]      (0, 0) circle [x radius= 3.35, y radius= 3.35]   ;
    %Straight Lines [id:da6754502602299528] 
    \draw [color={rgb, 255:red, 74; green, 144; blue, 226 }  ,draw opacity=1 ]   (551.58,92) ;
    \draw [shift={(551.58,92)}, rotate = 0] [color={rgb, 255:red, 74; green, 144; blue, 226 }  ,draw opacity=1 ][fill={rgb, 255:red, 74; green, 144; blue, 226 }  ,fill opacity=1 ][line width=0.75]      (0, 0) circle [x radius= 3.35, y radius= 3.35]   ;
    %Straight Lines [id:da4636752396565975] 
    \draw [color={rgb, 255:red, 155; green, 155; blue, 155 }  ,draw opacity=1 ]   (531,149) -- (531,213.94) ;
    %Straight Lines [id:da5161027155227682] 
    \draw [color={rgb, 255:red, 155; green, 155; blue, 155 }  ,draw opacity=1 ]   (572.17,149) -- (572.17,213.94) ;
    %Straight Lines [id:da8565602497762101] 
    \draw [color={rgb, 255:red, 155; green, 155; blue, 155 }  ,draw opacity=1 ]   (531,149) -- (572.17,149) ;
    %Straight Lines [id:da6303648939492528] 
    \draw [color={rgb, 255:red, 155; green, 155; blue, 155 }  ,draw opacity=1 ]   (572.17,36.94) -- (572.17,87) ;
    %Straight Lines [id:da4064574325852466] 
    \draw [color={rgb, 255:red, 155; green, 155; blue, 155 }  ,draw opacity=1 ]   (531,36.94) -- (531,87) ;
    %Straight Lines [id:da561771783283219] 
    \draw [color={rgb, 255:red, 155; green, 155; blue, 155 }  ,draw opacity=1 ]   (531,87) -- (572.17,87) ;
    %Straight Lines [id:da5570179685924626] 
    \draw [color={rgb, 255:red, 74; green, 144; blue, 226 }  ,draw opacity=1 ]   (551.58,144) ;
    \draw [shift={(551.58,144)}, rotate = 0] [color={rgb, 255:red, 74; green, 144; blue, 226 }  ,draw opacity=1 ][fill={rgb, 255:red, 74; green, 144; blue, 226 }  ,fill opacity=1 ][line width=0.75]      (0, 0) circle [x radius= 3.35, y radius= 3.35]   ;
    %Straight Lines [id:da28248586607353166] 
    \draw [color={rgb, 255:red, 255; green, 255; blue, 255 }  ,draw opacity=1 ]   (551.58,155) ;
    \draw [shift={(551.58,155)}, rotate = 0] [color={rgb, 255:red, 255; green, 255; blue, 255 }  ,draw opacity=1 ][fill={rgb, 255:red, 255; green, 255; blue, 255 }  ,fill opacity=1 ][line width=0.75]      (0, 0) circle [x radius= 3.35, y radius= 3.35]   ;
    %Straight Lines [id:da4321991989504319] 
    \draw [color={rgb, 255:red, 155; green, 155; blue, 155 }  ,draw opacity=1 ] [dash pattern={on 0.84pt off 2.51pt}]  (572.17,87) -- (639.17,87) ;
    %Straight Lines [id:da43667319847309294] 
    \draw [color={rgb, 255:red, 155; green, 155; blue, 155 }  ,draw opacity=1 ] [dash pattern={on 0.84pt off 2.51pt}]  (572.17,149) -- (639.17,149) ;
    %Straight Lines [id:da37805847900241685] 
    \draw    (631.17,90) -- (631.17,149) ;
    \draw [shift={(631.17,87)}, rotate = 90] [fill={rgb, 255:red, 0; green, 0; blue, 0 }  ][line width=0.08]  [draw opacity=0] (8.04,-3.86) -- (0,0) -- (8.04,3.86) -- (5.34,0) -- cycle    ;
    
    % Text Node
    \draw (308.92,141) node [anchor=east] [inner sep=0.75pt]  [color={rgb, 255:red, 155; green, 155; blue, 155 }  ,opacity=1 ] [align=left] {donor energy level};
    % Text Node
    \draw (396.17,132) node [anchor=west] [inner sep=0.75pt]    {$\Ed$};
    % Text Node
    \draw (114,130) node   [align=left] {P};
    % Text Node
    \draw (164,80) node   [align=left] {Si};
    % Text Node
    \draw (164,180) node   [align=left] {Si};
    % Text Node
    \draw (64,80) node   [align=left] {Si};
    % Text Node
    \draw (64,180) node   [align=left] {Si};
    % Text Node
    \draw (60.25,130) node [anchor=east] [inner sep=0.75pt]  [color={rgb, 255:red, 155; green, 155; blue, 155 }  ,opacity=1 ] [align=left] {donor};
    % Text Node
    \draw (633.17,118) node [anchor=west] [inner sep=0.75pt]    {$\Eg$};
    % Text Node
    \draw (98,226) node [anchor=north west][inner sep=0.75pt]   [align=left] {(a)};
    % Text Node
    \draw (310,226) node [anchor=north west][inner sep=0.75pt]   [align=left] {(b)};
    % Text Node
    \draw (497,226) node [anchor=north west][inner sep=0.75pt]   [align=left] {(c)};
    % Text Node
    \draw (445,10) node [anchor=north west][inner sep=0.75pt]    {$T=0$};
    % Text Node
    \draw (530,9) node [anchor=north west][inner sep=0.75pt]    {$T >0$};
    
    
    \end{tikzpicture}
    
    \caption{Doping: an additional electron is introduced into the material
    after an electron donor is doped (a), 
    which is bound around the donor at the ground state 
    and therefore a donor energy level is formed (b); 
    the electron may be thermally excited into the conduction band (c).}
    \label{fig:donor}
\end{figure}

\subsection{What does doping do?}

Let's consider the example in \prettyref{fig:donor},
where a P atom is doped into a silicon single crystal.
P has one more electron than Si, 
so if the electron is able to go around in the crystal, 
it should appear in the conduction band. 
But of course it's still possible that 
the electron is bound around the P atom.
The Hamiltonian of an electron in the conduction band is therefore roughly 
\begin{equation}
    H = - \frac{\hbar^2 \grad^2}{2 m^*_{\text{c}}} 
    - \frac{e^2}{\epsilon \abs*{\vb*{r}}},
\end{equation}
where $\epsilon$ comes from the screening effects of other electrons, 
which may be estimated by, say, $GW$ approximation.
The Hamiltonian has a bound state with negative energy 
as the ground state, 
and therefore we see a donor energy level formed 
in the spectrum of the system, illustrated in \prettyref{fig:donor}.
Similarly, when an electron attractor is introduced 
in the system, in the electron representation 
it forms an attractor energy level 
above the valence band
(and therefore in the hole representation, 
an energy level below the hole band 
which is the inverse of the valence band).
We use $\Ed$ and $\Ea$ to refer to the gap 
between the donor energy level and the bottom of the conduction band 
and the attractor energy level and the top of the valence band, respectively.

When $T = 0$, electrons stay at donor energy levels, 
and attractor energy levels are empty, 
or in other words are filled by holes (\prettyref{fig:donor}(c)); 
when the temperature rises, 
electrons go to the conduction band
and the attractor energy levels, 
the latter also known as ``holes go to the valence band''.

\subsection{Doped electron/hole concentration}

Usually, we assume that there is at most one electron at each impurity energy level, 
or otherwise strong Hubbard expulsion occurs.
The Hubbard $U$ as estimated as 
\begin{equation}
    U \sim \frac{e^2}{a_0^* \epsilon} \sim \SI{100}{meV} \gg \kB T,
\end{equation}
where $a_0^*$ is the Bohr radius 
in which $m$ is replaced by $m^*$.
Thus, we find \ul{the double occupation configurations 
are highly unlikely in the temperature range 
that most condensed matter experiments are carried out}.
The partition function of a donor energy level is therefore 
\begin{equation}
    Z_{\text{d}} \approx 1 + 2 \ee^{- (\Ed - \mu) / \kB T},
\end{equation}
and the electron occupation expectation at the donor energy level is 
\begin{equation}
    \expval{n_{\text{d, single}}} = \frac{1}{Z} (0 \cdot 1 + 1 \cdot 2 \ee^{- (\Ed - \mu) / \kB T}).
\end{equation}
If \ul{there is only one kind of donor doped into the material}, 
and \ul{we assume that the concentration of donors $\Nd$
is small enough so that there is no interaction between donors},
then the total number of electrons around donors is 
\begin{equation}
    \nd = \Nd \cdot \expval{n_{\text{d, single}}} 
    = \frac{1}{
        1 + \frac{1}{2} 
        \ee^{(\Ed - \mu) / \kB T}
    }.
\end{equation}

\subsection{Charge conservation}

Since the expression of $\nd$ and $\pa$ are already known, 
we can invoke the charge conservation equation 
\begin{equation}
    n + \nd - p - \pa = \Nd - \Na,
\end{equation}
and 

\section{Spatial inhomogeneity}

It should be noted that in an equilibrium diode, 
assuming a space-dependent chemical potential 
and assuming a space-dependent band structure 
are \emph{equivalent}:
once one of them is chosen,
the other has to be given up.
On the other hand, 
when an external electric field is applied,
we should allow spatial dependence of \emph{both} $\mu$ and $\varphi_{n \vb*{k}}$.
The point here is in when a diode is made,
the charge density distribution deviates from 
the distribution in homogeneous p- or n-type materials,
while when an external electric field is applied,
we \emph{don't} expect any change in charge distribution after a stable current is formed.
We change the band distortion in the depletion area 
to model the decrease/increase of energy barrier,
but then the chemical potential also has to have spatial dependence
or otherwise the fact that 
the electron distributions in the p-part and n-part are the same can't be reflected.

It should be noted that the way of thinking is different in classical circuit analysis
from the way in condensed matter physics:
after the (usually stationary) relation between the quantities of a system is solved,
it is used as a \emph{constraint} 
in circuit equations.
In condensed matter physics, we talk about response functions,
but in electronics the cause-effect relation is not emphasized.
This is comparable to the way of thinking in scattering theory,
where we focus on the scattering stationary states,
and indeed, in quantum optics we also talk about scattering matrices,
which can be derived from scattering stationary states 
and relates $a_{\text{in}}$'s and $a_{\text{out}}$'s.
But the formalism in electronics is more generalized:
non-unitary processes can also be modeled as constraints,
the most famous example being the Ohm's law.

\end{document}