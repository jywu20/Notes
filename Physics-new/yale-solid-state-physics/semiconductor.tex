\documentclass[hyperref, a4paper]{article}

\usepackage{geometry}
\usepackage{titling}
\usepackage{titlesec}
% No longer needed, since we will use enumitem package
% \usepackage{paralist}
\usepackage{enumitem}
\usepackage{footnote}
\usepackage{enumerate}
\usepackage{amsmath, amssymb, amsthm}
\usepackage{mathtools}
\usepackage{bbm}
\usepackage{cite}
\usepackage{graphicx}
\usepackage{subfigure}
\usepackage{physics}
\usepackage{tensor}
\usepackage{siunitx}
\usepackage[version=4]{mhchem}
\usepackage{tikz}
\usepackage{xcolor}
\usepackage{listings}
\usepackage{autobreak}
\usepackage[ruled, vlined, linesnumbered]{algorithm2e}
\usepackage{nameref,zref-xr}
\zxrsetup{toltxlabel}
\usepackage[colorlinks,unicode]{hyperref} % , linkcolor=black, anchorcolor=black, citecolor=black, urlcolor=black, filecolor=black
\usepackage[most]{tcolorbox}
\usepackage{prettyref}

% Page style
\geometry{left=3.18cm,right=3.18cm,top=2.54cm,bottom=2.54cm}
\titlespacing{\paragraph}{0pt}{1pt}{10pt}[20pt]
\setlength{\droptitle}{-5em}
%\preauthor{\vspace{-10pt}\begin{center}}
%\postauthor{\par\end{center}}

% More compact lists 
\setlist[itemize]{
    itemindent=17pt, 
    leftmargin=1pt,
    listparindent=\parindent,
    parsep=0pt,
}

% Math operators
\DeclareMathOperator{\timeorder}{\mathcal{T}}
\DeclareMathOperator{\diag}{diag}
\DeclareMathOperator{\legpoly}{P}
\DeclareMathOperator{\primevalue}{P}
\DeclareMathOperator{\sgn}{sgn}
\newcommand*{\ii}{\mathrm{i}}
\newcommand*{\ee}{\mathrm{e}}
\newcommand*{\const}{\mathrm{const}}
\newcommand*{\suchthat}{\quad \text{s.t.} \quad}
\newcommand*{\argmin}{\arg\min}
\newcommand*{\argmax}{\arg\max}
\newcommand*{\normalorder}[1]{: #1 :}
\newcommand*{\pair}[1]{\langle #1 \rangle}
\newcommand*{\fd}[1]{\mathcal{D} #1}
\DeclareMathOperator{\bigO}{\mathcal{O}}

% TikZ setting
\usetikzlibrary{arrows,shapes,positioning}
\usetikzlibrary{arrows.meta}
\usetikzlibrary{decorations.markings}
\tikzstyle arrowstyle=[scale=1]
\tikzstyle directed=[postaction={decorate,decoration={markings,
    mark=at position .5 with {\arrow[arrowstyle]{stealth}}}}]
\tikzstyle ray=[directed, thick]
\tikzstyle dot=[anchor=base,fill,circle,inner sep=1pt]

% Algorithm setting
% Julia-style code
\SetKwIF{If}{ElseIf}{Else}{if}{}{elseif}{else}{end}
\SetKwFor{For}{for}{}{end}
\SetKwFor{While}{while}{}{end}
\SetKwProg{Function}{function}{}{end}
\SetArgSty{textnormal}

\newcommand*{\concept}[1]{{\textbf{#1}}}

% Embedded codes
\lstset{basicstyle=\ttfamily,
  showstringspaces=false,
  commentstyle=\color{gray},
  keywordstyle=\color{blue}
}

% Reference formatting
\newrefformat{fig}{Figure~\ref{#1}}

% Color boxes
\tcbuselibrary{skins, breakable, theorems}
\newtcbtheorem[number within=section]{warning}{Warning}%
  {colback=orange!5,colframe=orange!65,fonttitle=\bfseries, breakable}{warn}
\newtcbtheorem[number within=section]{note}{Note}%
  {colback=green!5,colframe=green!65,fonttitle=\bfseries, breakable}{note}
\newtcbtheorem[number within=section]{info}{Info}%
  {colback=blue!5,colframe=blue!65,fonttitle=\bfseries, breakable}{info}

\newenvironment{shelldisplay}{\begin{lstlisting}}{\end{lstlisting}}

\newcommand{\address}[1]{\href{#1}{\url{#1}}}

\title{Semiconductors}
\author{Jiinyuan Wu}

\begin{document}

\maketitle

\section{Two-band model}

At $T = 0$ there is strictly no such thing as a semiconductor:
there are just band metals or band insulators.
When $T > 0$, however, 
the following two mechanisms happen.
The first is that electrons jump from the valence band to the conduction band.
The second is if there are energy levels 
near the highest point of the valence band ($E_{\text{v}}$)
or the lowest point of the conduction band ($E_{\text{c}}$),
thermally excited electrons will jump to them.
(From another point of view we may say doped atoms eat electrons or give electrons,
so some electrons are missing if we only look at Bloch states in the spectrum.)
In both mechanisms,
we have electrons in the conduction band and holes in the valence band,
which are \concept{carriers} of electric current.

In the discussion below, we assume that the band gap is \emph{larger} compared with $k_{\text{B}} T$,
and this means when $E \geq E_{\text{c}}$,
\begin{equation}
    n_{\text{electron}} = \ee^{- (E - \mu) / k_{\text{B}} T},
\end{equation}
and when $E \leq E_{\text{v}}$
\begin{equation}
    n_{\text{hole}} = \ee^{- (\mu - E) / k_{\text{B}} T}.
\end{equation}
In semiconductor physics we usually use the following abbreviations:
p (positive), v (valence), h (hole) mean holes in the valence band,
and n (negative), c (conduction), e (electron) mean electrons in the conduction band.
Thus the density of electrons and holes are given by 
\begin{equation}
    n = \int_{E_\text{c}}^\infty D_{\text{c}} (E) \dd{E} \ee^{- (E - \mu) / k_{\text{B}} T},
\end{equation}
and 
\begin{equation}
    p = \int^{E_{\text{v}}}_{-\infty} D_{\text{v}} (E) \dd{E} \ee^{- (\mu - E) / k_{\text{B}} T}.
\end{equation}

Consider the simplest case, where we have a band gap (possibly indirect) 
\begin{equation}
    E_{\text{g}} = E_{\text{c}} - E_{\text{v}}
\end{equation}
between the highest valence band the the lowest conduction band.
The $\Delta E \gg k_{\text{B}} T$ condition also means there should be 
electrons or holes with very high energy,
and therefore the hyperbolic approximation works, 
and the conduction band is completely characterized by its effective mass 
and the momentum with lowest energy,
and so is the case for the valence band.
We have 
\begin{equation}
    D(E) = \frac{1}{2 \pi^2} \left( \frac{2 m^*}{\hbar^2} \right) \sqrt{E - E_{\text{max/min}}}
\end{equation}
for hyperbolic bands,
and therefore we have (note that in the valence band, the effective mass of the electron is negative,
so the effective mass of the hole is positive) 
\begin{equation}
    n = 2 \left( \frac{m_\text{e}^* k_{\text{B}} T}{2 \pi \hbar^2} \right)^{3/2} 
    \ee^{- (E_{\text{c}} - \mu) / k_{\text{B}} T},
    \label{eq:n-def}
\end{equation}
\begin{equation}
    p = 2 \left( \frac{m_\text{h}^* k_{\text{B}} T}{2 \pi \hbar^2} \right)^{3/2} 
    \ee^{- (\mu - E_{\text{v}}) / k_{\text{B}} T}.
\end{equation}
Note that if there is doping, electrons donated or attracted by the doped atoms 
are not included in the above equations.

\section{The intrinsic semiconductor limit}

In the \concept{intrinsic semiconductor limit},
there is no doping, and only the first mechanism 
-- thermal excitation in the conduction band and the valence band --
works. 
This limit is useful 
when $T$ is very high so the effect of doping can be ignored.
(So in the opposite, 
when $T$ is low enough, 
even materials that are very clean can't be described well by the intrinsic semiconductor limit,
because in this case, if the material has any semiconductor property,
then it has to come from doping.)
Since there is no doping, we have $n = p$ because of charge neutrality,
and we have $n = \sqrt{np}$,
and solving this equation we find 
\begin{equation}
    \mu = E_{\text{v}} + \frac{1}{2} E_{\text{g}} + \frac{3}{4} k_{\text{B}} T \ln(\frac{m_\text{h}^*}{m_{\text{e}}^*}).
    \label{eq:mu-nat}
\end{equation}
Here we can see that when $T = 0$, 
actually $\mu \neq E_{\text{F}} = E_{\text{v}}$,
but it doesn't matter:
when $T$ is zero and we are working with an insulator,
putting $\mu$ anywhere between $E_{\text{c}}$ and $E_{\text{v}}$ is acceptable.

Here Drude model works, because the density of electrons and holes is small 
(TODO: why no strong correlation, like Wigner crystal?),
and the quantum nature of electrons isn't apparent.
So we have 
\begin{equation}
    \sigma_\text{e} = n \mu_{\text{e}}, \quad 
    \sigma_{\text{H}} = p \mu_{\text{H}},
\end{equation}
and the mobilities are 
\begin{equation}
    \mu_{\text{e}} = - \frac{\tau e}{m^*_{\text{e}}}, \quad 
    \mu_{\text{h}} =   \frac{\tau e}{m^*_{\text{h}}}.
    \label{eq:mobilities}
\end{equation}
Usually the mobility of holes isn't as good as electrons because in general 
$m_{\text{h}}^* > m_{\text{e}}^*$,
and for large effective mass we expect low mobility.
This can be found by looking at \eqref{eq:mobilities},
and \eqref{eq:mobilities} comes from the semiclassical EOM of electrons and holes,
so we can also explain the fact from a quantum perspective.
Suppose a hole is stuck in the lattice 
(it may slightly distort the lattice,
so what is stuck is actually a hole and some phonons, 
or a polaron).
It may still tunnel away because $x$ is localized 
and $p$ is uncertain, which may give it a kinetic energy large enough to escape,
but if $m^*_{\text{h}}$ is large enough, 
the kinetic energy fluctuation can be ignored,
so the hole is safely localized.

When the anisotropy of the lattice is strong, we have 
\begin{equation}
    n = 2 \left( \frac{k_{\text{B}} T}{2 \pi \hbar^2} \right)^{3/2} (\tilde{m}_{\text{e}}^* \tilde{m}_{\text{h}}^*)^{3/4} \ee^{- E_{\text{g}} / k_{\text{B}} T},
\end{equation}
where $\tilde{m}$ means $(\det \vb*{M})^{1/3} = \sqrt[3]{m_1^* m_2^* m_3^*}$,
with the three $m^*$'s being the eigenvalues of the mass matrix,
which is defined as 
\begin{equation}
    E = E \pm \frac{1}{2} \vb*{k} \cdot \vb*{M}^{-1} \cdot \vb*{k}.
\end{equation}

\section{Doping}

The overall effect of doping is to introduce 
\begin{equation}
    \Delta n = n - p \neq 0.
\end{equation}
Still we have 
\begin{equation}
    n \cdot p = n_{\text{i}}^2 = p_{\text{i}}^2,
\end{equation}
where $n_{\text{i}}$ is the $n$ given by \eqref{eq:n-def} and \eqref{eq:mu-nat}.
This gives 
\begin{equation}
    n = \frac{\Delta n + \sqrt{4 n_{\text{i}}^2 + \Delta n^2}}{2}, \quad 
    p = \frac{-\Delta n + \sqrt{4 n_{\text{i}}^2 + \Delta n^2}}{2}.
\end{equation}
And then the chemical potential can be found by \eqref{eq:n-def}.

Now the problem is how to find $\Delta n$.
This can be done by calculating the particle number expectation of each impurity.

It should be noted that in an equilibrium diode, 
assuming a space-dependent chemical potential 
and assuming a space-dependent band structure 
are \emph{equivalent}:
once one of them is chosen,
the other has to be given up.
On the other hand, 
when an external electric field is applied,
we should allow spatial dependence of \emph{both} $\mu$ and $\varphi_{n \vb*{k}}$.
The point here is in when a diode is made,
the charge density distribution deviates from 
the distribution in homogeneous p- or n-type materials,
while when an external electric field is applied,
we \emph{don't} expect any change in charge distribution after a stable current is formed.
We change the band distortion in the depletion area 
to model the decrease/increase of energy barrier,
but then the chemical potential also has to have spatial dependence
or otherwise the fact that 
the electron distributions in the p-part and n-part are the same can't be reflected.

It should be noted that the way of thinking is different in classical circuit analysis
from the way in condensed matter physics:
after the (usually stationary) relation between the quantities of a system is solved,
it is used as a \emph{constraint} 
in circuit equations.
In condensed matter physics, we talk about response functions,
but in electronics the cause-effect relation is not emphasized.
This is comparable to the way of thinking in scattering theory,
where we focus on the scattering stationary states,
and indeed, in quantum optics we also talk about scattering matrices,
which can be derived from scattering stationary states 
and relates $a_{\text{in}}$'s and $a_{\text{out}}$'s.
But the formalism in electronics is more generalized:
non-unitary processes can also be modeled as constraints,
the most famous example being the Ohm's law.

\end{document}