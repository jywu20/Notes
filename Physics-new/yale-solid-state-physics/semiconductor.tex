\documentclass[hyperref, a4paper]{article}

\usepackage{geometry}
\usepackage{titling}
\usepackage{titlesec}
% No longer needed, since we will use enumitem package
% \usepackage{paralist}
\usepackage{enumitem}
\usepackage{footnote}
\usepackage{enumerate}
\usepackage{amsmath, amssymb, amsthm}
\usepackage{mathtools}
\usepackage{bbm}
\usepackage{cite}
\usepackage{graphicx}
\usepackage{subfigure}
\usepackage{physics}
\usepackage{tensor}
\usepackage{siunitx}
\usepackage[version=4]{mhchem}
\usepackage{tikz}
\usepackage{xcolor}
\usepackage{listings}
\usepackage{autobreak}
\usepackage[ruled, vlined, linesnumbered]{algorithm2e}
\usepackage{nameref,zref-xr}
\zxrsetup{toltxlabel}
\usepackage[colorlinks,unicode]{hyperref} % , linkcolor=black, anchorcolor=black, citecolor=black, urlcolor=black, filecolor=black
\usepackage[most]{tcolorbox}
\usepackage{prettyref}

% Page style
\geometry{left=3.18cm,right=3.18cm,top=2.54cm,bottom=2.54cm}
\titlespacing{\paragraph}{0pt}{1pt}{10pt}[20pt]
\setlength{\droptitle}{-5em}
%\preauthor{\vspace{-10pt}\begin{center}}
%\postauthor{\par\end{center}}

% More compact lists 
\setlist[itemize]{
    itemindent=17pt, 
    leftmargin=1pt,
    listparindent=\parindent,
    parsep=0pt,
}

% Math operators
\DeclareMathOperator{\timeorder}{\mathcal{T}}
\DeclareMathOperator{\diag}{diag}
\DeclareMathOperator{\legpoly}{P}
\DeclareMathOperator{\primevalue}{P}
\DeclareMathOperator{\sgn}{sgn}
\newcommand*{\ii}{\mathrm{i}}
\newcommand*{\ee}{\mathrm{e}}
\newcommand*{\const}{\mathrm{const}}
\newcommand*{\suchthat}{\quad \text{s.t.} \quad}
\newcommand*{\argmin}{\arg\min}
\newcommand*{\argmax}{\arg\max}
\newcommand*{\normalorder}[1]{: #1 :}
\newcommand*{\pair}[1]{\langle #1 \rangle}
\newcommand*{\fd}[1]{\mathcal{D} #1}
\DeclareMathOperator{\bigO}{\mathcal{O}}

% TikZ setting
\usetikzlibrary{arrows,shapes,positioning}
\usetikzlibrary{arrows.meta}
\usetikzlibrary{decorations.markings}
\tikzstyle arrowstyle=[scale=1]
\tikzstyle directed=[postaction={decorate,decoration={markings,
    mark=at position .5 with {\arrow[arrowstyle]{stealth}}}}]
\tikzstyle ray=[directed, thick]
\tikzstyle dot=[anchor=base,fill,circle,inner sep=1pt]

% Algorithm setting
% Julia-style code
\SetKwIF{If}{ElseIf}{Else}{if}{}{elseif}{else}{end}
\SetKwFor{For}{for}{}{end}
\SetKwFor{While}{while}{}{end}
\SetKwProg{Function}{function}{}{end}
\SetArgSty{textnormal}

\newcommand*{\concept}[1]{{\textbf{#1}}}

% Embedded codes
\lstset{basicstyle=\ttfamily,
  showstringspaces=false,
  commentstyle=\color{gray},
  keywordstyle=\color{blue}
}

% Reference formatting
\newrefformat{fig}{Figure~\ref{#1}}

% Color boxes
\tcbuselibrary{skins, breakable, theorems}
\newtcbtheorem[number within=section]{warning}{Warning}%
  {colback=orange!5,colframe=orange!65,fonttitle=\bfseries, breakable}{warn}
\newtcbtheorem[number within=section]{note}{Note}%
  {colback=green!5,colframe=green!65,fonttitle=\bfseries, breakable}{note}
\newtcbtheorem[number within=section]{info}{Info}%
  {colback=blue!5,colframe=blue!65,fonttitle=\bfseries, breakable}{info}

\newenvironment{shelldisplay}{\begin{lstlisting}}{\end{lstlisting}}

\newcommand{\address}[1]{\href{#1}{\url{#1}}}

\title{Semiconductors}
\author{Jiinyuan Wu}

\begin{document}

\maketitle

\section{Two-band model}

At $T = 0$ there is strictly no such thing as a semiconductor:
there are just band metals or band insulators.
When $T > 0$, however, 
the following two mechanisms happen.
The first is that electrons jump from the valence band to the conduction band.
The second is if there are energy levels 
near the highest point of the valence band ($E_{\text{v}}$)
or the lowest point of the conduction band ($E_{\text{c}}$),
thermally excited electrons will jump to them.
(From another point of view we may say doped atoms eat electrons or give electrons,
so some electrons are missing if we only look at Bloch states in the spectrum.)
In both mechanisms,
we have electrons in the conduction band and holes in the valence band,
which are \concept{carriers} of electric current.

In the discussion below, we assume that the band gap is \emph{larger} compared with $k_{\text{B}} T$,
and this means when $E \geq E_{\text{c}}$,
\begin{equation}
    n_{\text{electron}} = \ee^{- (E - \mu) / k_{\text{B}} T},
\end{equation}
and when $E \leq E_{\text{v}}$
\begin{equation}
    n_{\text{hole}} = \ee^{- (\mu - E) / k_{\text{B}} T}.
\end{equation}
In semiconductor physics we usually use the following abbreviations:
p (positive), v (valence), h (hole) mean holes in the valence band,
and n (negative), c (conduction), e (electron) mean electrons in the conduction band.
Thus the density of electrons and holes are given by 
\begin{equation}
    n = \int_{E_\text{c}}^\infty D_{\text{c}} (E) \dd{E} \ee^{- (E - \mu) / k_{\text{B}} T},
\end{equation}
and 
\begin{equation}
    p = \int^{E_{\text{v}}}_{-\infty} D_{\text{v}} (E) \dd{E} \ee^{- (\mu - E) / k_{\text{B}} T}.
\end{equation}

Consider the simplest case, where we have a band gap (possibly indirect) 
\begin{equation}
    E_{\text{g}} = E_{\text{c}} - E_{\text{v}}
\end{equation}
between the highest valence band the the lowest conduction band.
The $\Delta E \gg k_{\text{B}} T$ condition also means there should be 
electrons or holes with very high energy,
and therefore the hyperbolic approximation works, 
and the conduction band is completely characterized by its effective mass 
and the momentum with lowest energy,
and so is the case for the valence band.
We have 
\begin{equation}
    D(E) = \frac{1}{2 \pi^2} \left( \frac{2 m^*}{\hbar^2} \right) \sqrt{E - E_{\text{max/min}}}
\end{equation}
for hyperbolic bands,
and therefore we have 
\begin{equation}
    n = 2 \left( \frac{m_\text{e}^* k_{\text{B}} T}{2 \pi \hbar^2} \right)^{3/2} 
    \ee^{- (E_{\text{c}} - \mu) / k_{\text{B}} T},
\end{equation}
\begin{equation}
    p = 2 \left( \frac{m_\text{h}^* k_{\text{B}} T}{2 \pi \hbar^2} \right)^{3/2} 
    \ee^{- (\mu - E_{\text{v}}) / k_{\text{B}} T}.
\end{equation}
Note that if there is doping, electrons donated or attracted by the doped atoms 
are not included in the above equations.

\section{The intrinsic semiconductor limit}

In the \concept{intrinsic semiconductor limit},
there is no doping, and only the first mechanism 
-- thermal excitation in the conduction band and the valence band --
works. 
This limit is useful 
when $T$ is very high so the effect of doping can be ignored.
(So in the opposite, 
when $T$ is low enough, 
even materials that are very clean can't be described well by the intrinsic semiconductor limit,
because in this case, if the material has any semiconductor property,
then it has to come from doping.)
Since there is no doping, we have $n = p$ because of charge neutrality,
and we have $n = \sqrt{np}$,
and solving this equation we find 
\begin{equation}
    \mu = E_{\text{v}} + \frac{1}{2} E_{\text{g}} + \frac{3}{4} k_{\text{B}} T \ln(\frac{m_\text{h}^*}{m_{\text{e}^*}}).
\end{equation}
Here we can see that when $T = 0$, 
actually $\mu \neq E_{\text{F}} = E_{\text{v}}$,
but it doesn't matter:
when $T$ is zero and we are working with an insulator,
putting $\mu$ anywhere between $E_{\text{c}}$ and $E_{\text{v}}$ is acceptable.


\end{document}