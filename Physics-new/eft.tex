\documentclass{beamer}
\usepackage{physics}
\usepackage{amsmath}
\usepackage{tikz}
\usepackage{mathdots}
\usepackage{yhmath}
\usepackage{cancel}
\usepackage{color}
\usepackage{siunitx}
\usepackage{array}
\usepackage{multirow}
\usepackage{amssymb}
\usepackage{textcomp, gensymb}
\usepackage{tabularx}
\usepackage{extarrows}
\usepackage{booktabs}
\usetikzlibrary{fadings}
\usetikzlibrary{patterns}
\usetikzlibrary{shadows.blur}
\usetikzlibrary{shapes}
\usepackage[style=verbose,backend=bibtex]{biblatex}
\addbibresource{arpes.bib}
\addbibresource{green.bib}
\usepackage{listings}
\usepackage{hyperref}

\newcommand{\pair}[1]{\langle #1 \rangle}
\DeclareMathOperator{\ee}{e}
\DeclareMathOperator{\ii}{i}

\newcommand{\concept}[1]{\textbf{#1}}

%Information to be included in the title page:
\title{Effective theories}
\author{Jinyuan Wu}

\usetheme{Madrid}

\begin{document}

\frame{\titlepage}

\begin{frame}
\frametitle{Effective theories}

\begin{itemize}
    \item Particle physicists and cosmologists come up with weird ideas to explain the world \dots
    \item but material scientists don't care. 
\end{itemize}

\textbf{Why?}

\begin{itemize}
    \item A phenomenon can be well described by a theory that \emph{fits its scale}
    \item \concept{Effective theory}
\end{itemize}

\end{frame}

\end{document}