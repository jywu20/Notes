\documentclass[hyperref, a4paper]{article}

\usepackage{geometry}
\usepackage{titling}
\usepackage{titlesec}
% No longer needed, since we will use enumitem package
% \usepackage{paralist}
\usepackage{enumitem}
\usepackage{footnote}
\usepackage{enumerate}
\usepackage{amsmath, amssymb, amsthm}
\usepackage{mathtools}
\usepackage{bbm}
\usepackage{cite}
\usepackage{graphicx}
\usepackage{subfigure}
\usepackage{physics}
\usepackage{tensor}
\usepackage{siunitx}
\usepackage[version=4]{mhchem}
\usepackage{tikz}
\usepackage{xcolor}
\usepackage{listings}
\usepackage{autobreak}
\usepackage[ruled, vlined, linesnumbered]{algorithm2e}
\usepackage{nameref,zref-xr}
\zxrsetup{toltxlabel}
\zexternaldocument*[optics-]{../optics/optics}[optics.pdf]
\zexternaldocument*[solid-]{../solid/solid}[solid.pdf]
\usepackage[colorlinks,unicode]{hyperref} % , linkcolor=black, anchorcolor=black, citecolor=black, urlcolor=black, filecolor=black
\usepackage{prettyref}

% Page style
\geometry{left=3.18cm,right=3.18cm,top=2.54cm,bottom=2.54cm}
\titlespacing{\paragraph}{0pt}{1pt}{10pt}[20pt]
\setlength{\droptitle}{-5em}
\preauthor{\vspace{-10pt}\begin{center}}
\postauthor{\par\end{center}}

% More compact lists 
\setlist[itemize]{
    itemindent=17pt, 
    leftmargin=1pt,
    listparindent=\parindent,
    parsep=0pt,
}

% Math operators
\DeclareMathOperator{\timeorder}{\mathcal{T}}
\DeclareMathOperator{\diag}{diag}
\DeclareMathOperator{\legpoly}{P}
\DeclareMathOperator{\primevalue}{P}
\DeclareMathOperator{\sgn}{sgn}
\newcommand*{\ii}{\mathrm{i}}
\newcommand*{\ee}{\mathrm{e}}
\newcommand*{\const}{\mathrm{const}}
\newcommand*{\suchthat}{\quad \text{s.t.} \quad}
\newcommand*{\argmin}{\arg\min}
\newcommand*{\argmax}{\arg\max}
\newcommand*{\normalorder}[1]{: #1 :}
\newcommand*{\pair}[1]{\langle #1 \rangle}
\newcommand*{\fd}[1]{\mathcal{D} #1}
\DeclareMathOperator{\bigO}{\mathcal{O}}

% TikZ setting
\usetikzlibrary{arrows,shapes,positioning}
\usetikzlibrary{arrows.meta}
\usetikzlibrary{decorations.markings}
\tikzstyle arrowstyle=[scale=1]
\tikzstyle directed=[postaction={decorate,decoration={markings,
    mark=at position .5 with {\arrow[arrowstyle]{stealth}}}}]
\tikzstyle ray=[directed, thick]
\tikzstyle dot=[anchor=base,fill,circle,inner sep=1pt]

% Algorithm setting
% Julia-style code
\SetKwIF{If}{ElseIf}{Else}{if}{}{elseif}{else}{end}
\SetKwFor{For}{for}{}{end}
\SetKwFor{While}{while}{}{end}
\SetKwProg{Function}{function}{}{end}
\SetArgSty{textnormal}

\newcommand*{\concept}[1]{{\textbf{#1}}}

% Embedded codes
\lstset{basicstyle=\ttfamily,
  showstringspaces=false,
  commentstyle=\color{gray},
  keywordstyle=\color{blue}
}

\newcommand{\opticsdoc}{\href{../optics/optics}{the optics note}}
\newcommand{\soliddoc}{\href{../solid/solid}{the solid state physics note}}

\newrefformat{fig}{Figure~\ref{#1} on page~\pageref{#1}}
\newrefformat{sec}{Section~\ref{#1}}

\title{Soft Condensed Matter by Prof. Peng Tan}
\author{Jinyuan Wu}
\date{October 21, 2021}

\begin{document}

\maketitle

\section{Thermodynamics of phase transition}

Thermodynamics can provide a phenomenology description of phase transition.
Let $\Phi$ be the thermodynamic potential in a given condition.
The equilibrium configuration can be find by extremize it, and its higher order derivatives can be used to determine whether the configuration is stable, meta-stable or unstable.

\begin{equation}
    F = - k_\text{B} T \ln Z = U - TS, 
\end{equation}

The free energy (or other thermodynamic potentials that are used under the circumstances of interest) is always continuous.
A \concept{first order phase transition} is a phase transition in which the first order derivative (for example, the entropy) of the thermodynamic potential is not continuous.
A \concept{second order phase transition} is a phase transition in which the first order derivative of the thermodynamic potential is continuous, but the second order derivative (for example, the special heat) is not.

Hard core gases are often used as a physical picture of phase transitions.
They have gas, liquid and crystal phases.
The fact, actually, is not merely a toy model.
We have what is called \concept{colloidal crystals}, where \emph{colloidal particles} - instead of atoms or molecules - form crystalline structures.
The various interaction between these crystals make their behaviors much more complex.

It should be noted that thermodynamic stability is \emph{not} the sole factor determining what phases we can observe.
We know under the same temperature and pressure, graphite is stabler than diamond, but we can still have stable diamonds.
\emph{Dynamic} factors are important in phase transitions.

\section{Classical nucleation theory}

Now we consider a good example demonstrating the importance of kinetics in first order phase transitions: nucleation, or how liquid drops come out of vapor.
The free energy of a vapor system with a droplet with radius $R$ is 
\begin{equation}
    F(R) = \frac{4}{3} \pi R^3 \times \rho \Delta f + 4 \pi R^2 \times \sigma,
\end{equation}
where 
\begin{equation}
    \Delta f = f_\text{liquid} - f_\text{vapor} < 0.
\end{equation}
Intuitively, the surface tension wants to push the droplet ``back'' into vapor, while the fact that liquid has a smaller free energy mass density then the vapor.
As $R$ goes larger, $F$ goes higher at first, and it reaches itx maximum 
\begin{equation}
    F_\text{c} = \frac{16\pi}{3} \frac{\sigma^3}{(\Delta f)^2}
    \label{eq:max-freeenergy-drop}
\end{equation}
at 
\begin{equation}
    R_\text{c} = - \frac{2\sigma}{\Delta f}.
\end{equation}
When $R$ passes $R_\text{c}$ the free energy goes down.

The problem, now, is whether a drop can really emerge from the vapor.
Due to thermal fluctuation, droplets constantly emerge, grow, shrink and disappear, as if the system is ``trying'' different configurations to find a more stabler position.
The nucleation rate, in quasi-equilibrium systems, is 
\begin{equation}
    P \simeq A \ee^{- \Delta F_\text{c} / k_\text{B} T}.
\end{equation}
If $\sigma$ is just too large, then despite the fact that a state with a large droplet actually has a lower free energy - and therefore, stabler - than a state in which there is only vapor, droplets are unlikely to exist since the energy barrier \eqref{eq:max-freeenergy-drop} is just too high, and the nucleation rate is just too small.
If, on the other hand, $\sigma$ is very small compared to $T$, then the vapor is very unstable: with just a small perturbation it will turn into liquid.
This is called \concept{spinodal decomposition}, where one thermodynamic phase spontaneously (i.e., without nucleation) separates into two phases.

The above mechanism also explains why condensation nuclei can accelerate nucleation: because a liquid droplet attached to a condensation nuclei only has half surface area, and therefore the surface tension term is less important. 

\section{Melting}

Melting is another example of first order phase transition, where we must acknowledge that first order phase transitions are far from being well understood.
We have dimensionality, defects, lattice vibrations and other rich behaviors .

Colloidal particles, again, have \emph{crystallization}, \emph{melting}, \emph{sublimitation} and \emph{solid-solid transitions}, just like ordinary materials do.
They are \emph{tunable particles} or \emph{designer's atoms}, whose size, number and interaction can be tuned manually, and unlike systems made of molecules or atoms, now we can keep an eye on every ``atom'' in the system.

We take hard-core spheres as an example. We usually only change the volume fraction, because heating may make the colloid unstable.
When $\phi < \SI{49.4}{\percent}$, the system freezes into liquid, and when $\phi > \SI{54.5}{\percent}$ the system melts.
Between these two volume fractions we have \emph{supercooling}.
Liquid and solid are both stable phases for hard spheres; they can also form a meta-stable \emph{glass} state.

Some interesting results have already been observed in these systems. We already know that melting starts from the \emph{edge} of a finite size pure (i.e. no defects) crystal, where the liquid-crystal boundary sweeps through the system, and from the inner defects in an almost infinite system.
This demonstrates the general concept of \concept{heterogenous melting and homogenous melting}.
These behaviors are dynamic rather than thermodynamic.

Spinodal decomposition also exists in hard sphere systems. When the system is \concept{superheated} (not by actually heat but by vibration), no nucleation is needed, and solid-liquid transition happens immediately.

The transition kinetics can also be studied in hard sphere systems. There is a big question in crystallization:
how does the system break translational symmetry and build the lattice symmetry?
Actually the process is \emph{staged}. The liquid first transits into \concept{precursors}, which are still in the liquid phase but have already gain some features of solid.
Then, some of the precursors turn into \concept{meta-stable solid}, which later become solid.

\section{Glasses}

A liquid cooling too fast for crystallization falls into a \concept{glass}. 
It is hard to say what really is a glass. 
In everyday language glass is definitely solid; it, though, has some features of liquid.
It is also not correct to classify glass as a kind of liquid, because, for example, ``glasses'' as we know it in \emph{glass windows} are \emph{amorphous} silica, while quartz is \emph{crystalline} silica.
The chemical bonds are exactly the same.

\end{document}