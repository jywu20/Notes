\documentclass[hyperref, a4paper]{article}

\usepackage{geometry}
\usepackage{titling}
\usepackage{titlesec}
% No longer needed, since we will use enumitem package
% \usepackage{paralist}
\usepackage{enumitem}
\usepackage{footnote}
% Conflicts with enumitem
%\usepackage{enumerate}
\usepackage{amsmath, amssymb, amsthm}
\usepackage{mathtools}
\usepackage{bbm}
\usepackage{cite}
\usepackage{graphicx}
\usepackage{subfigure}
\usepackage{physics}
\usepackage{tensor}
\usepackage{siunitx}
\usepackage[version=4]{mhchem}
\usepackage{tikz}
\usepackage{xcolor}
\usepackage{listings}
\usepackage{autobreak}
\usepackage[ruled, vlined, linesnumbered]{algorithm2e}
\usepackage{nameref,zref-xr}
\zxrsetup{toltxlabel}
\zexternaldocument*[optics-]{../optics/optics}[optics.pdf]
\zexternaldocument*[solid-]{../solid/solid}[solid.pdf]
\usepackage[colorlinks,unicode]{hyperref} % , linkcolor=black, anchorcolor=black, citecolor=black, urlcolor=black, filecolor=black
\usepackage[most]{tcolorbox}
\usepackage{prettyref}

% Page style
\geometry{left=3.18cm,right=3.18cm,top=2.54cm,bottom=2.54cm}
\titlespacing{\paragraph}{0pt}{1pt}{10pt}[20pt]
\setlength{\droptitle}{-5em}
\preauthor{\vspace{-10pt}\begin{center}}
\postauthor{\par\end{center}}

% More compact lists 
\setlist[itemize]{
    itemindent=17pt, 
    leftmargin=1pt,
    listparindent=\parindent,
    parsep=0pt,
}

% Math operators
\DeclareMathOperator{\timeorder}{\mathcal{T}}
\DeclareMathOperator{\diag}{diag}
\DeclareMathOperator{\legpoly}{P}
\DeclareMathOperator{\primevalue}{P}
\DeclareMathOperator{\sgn}{sgn}
\newcommand*{\ii}{\mathrm{i}}
\newcommand*{\ee}{\mathrm{e}}
\newcommand*{\const}{\mathrm{const}}
\newcommand*{\suchthat}{\quad \text{s.t.} \quad}
\newcommand*{\argmin}{\arg\min}
\newcommand*{\argmax}{\arg\max}
\newcommand*{\normalorder}[1]{: #1 :}
\newcommand*{\pair}[1]{\langle #1 \rangle}
\newcommand*{\fd}[1]{\mathcal{D} #1}
\DeclareMathOperator{\bigO}{\mathcal{O}}

% TikZ setting
\usetikzlibrary{arrows,shapes,positioning}
\usetikzlibrary{calc}
\usetikzlibrary{arrows.meta}
\usetikzlibrary{decorations.markings}
\tikzstyle arrowstyle=[scale=1]
\tikzstyle directed=[postaction={decorate,decoration={markings,
    mark=at position .5 with {\arrow[arrowstyle]{stealth}}}}]
\tikzstyle ray=[directed, thick]
\tikzstyle dot=[anchor=base,fill,circle,inner sep=1pt]

% Algorithm setting
% Julia-style code
\SetKwIF{If}{ElseIf}{Else}{if}{}{elseif}{else}{end}
\SetKwFor{For}{for}{}{end}
\SetKwFor{While}{while}{}{end}
\SetKwProg{Function}{function}{}{end}
\SetArgSty{textnormal}

\newcommand*{\concept}[1]{{\textbf{#1}}}

% Embedded codes
\lstset{basicstyle=\ttfamily,
  showstringspaces=false,
  commentstyle=\color{gray},
  keywordstyle=\color{blue}
}

% Reference formatting
\newrefformat{fig}{Figure~\ref{#1} on page~\pageref{#1}}

% Color boxes
\tcbuselibrary{skins, breakable, theorems}
\newtcbtheorem[number within=section]{warning}{Warning}%
  {colback=orange!5,colframe=orange!65,fonttitle=\bfseries, breakable}{warn}
\newtcbtheorem[number within=section]{note}{Note}%
  {colback=green!5,colframe=green!65,fonttitle=\bfseries, breakable}{note}

\title{Free Fermion Theories by Tian Yuan}
\author{Jinyuan Wu}
\date{December 4, 2021}    

\begin{document}

\maketitle

This presentation is mostly based on Commun.Math.Phys. 257, 725-771. This article is about what constraints and 
symmetry can be added to the Hilbert space of a free fermion system, and we can get the space of all possible 
Hamiltonians. 

From now on we use $\overline{(\cdot)}$ to denote complex conjugation and use $(\cdot)^*$ to denote adjoint.

We define a \concept{Nambu space} to be a Hilbert space $W = V \oplus V^*$, where $V$ is also a Hilbert space,
and $V^*$ is the dual space of $V$ and an inner product can be induced. An inner product can be defined on $W$
if we assign the inner product of $V^*$ to be  
\begin{equation}
    \expval*{f, f'} = \overline{\expval*{Cf, Cf'}_V}, \quad f, f' \in V^*,
    \label{eq:inner-w}
\end{equation}
where $C: W \to W$ is defined as 
\begin{equation}
    C: V \mapsto \expval*{V, -}.
\end{equation} 
The complex conjugate in \eqref{eq:inner-w} comes from the fact that $C$ is anti-linear and the requirement that 
the first element in an inner product is anti-linear. We also assume there is a canonical bilinear form 
\begin{equation}
    b: W \otimes W \to \mathbb{C}, \quad b(v+f, v'+f') = f(v') + f'(v).
\end{equation}

We show the motivation to define Nambu spaces. The Hilbert space $V$ is actually the linear space spanned by 
$\{c_i^\dagger\}$, and $V^*$ is the linear space spanned by $\{c_i\}$. The many-body Fock space is 
\begin{equation}
    \wedge V = \mathbb{C} \oplus V \oplus \wedge^2 V \oplus \cdots
\end{equation}
and using this construction we can verify that $V^*$ is indeed the dual space of $V$.
$C$ is similar to a particle-hole transformation, though a physical particle-hole transformation may also contain
some unitary transformations. Now we see $W$ is the space of all field operators, and $b$ actually imposed the
anti-commutative relation. Therefore the definition of Nambu spaces summarizes the properties of Fermi operators.

It can be verified that the $b(\cdot, \cdot)$ structure and the $\expval*{\cdot, \cdot}$ structure are not
independent. We make the decomposition 
\begin{equation}
    w = v + f, \quad v \in V, f \in V^*,
\end{equation}
and we have
\begin{equation}
    \expval*{w, w'} = \expval*{v, v'} + \expval*{Cf, Cf'},
\end{equation}

A generalized quadratic Hamiltonian can be written as 
\begin{equation}
    H = \sum_{i,j} A_{ij} c^\dagger_i c_j + 
    \frac{1}{2} \sum_{i,j} (B_{ij} c_i^\dagger c_j^\dagger + C_{ij} c_j c_i).
\end{equation}
We require $H = H^*$, and this in turn requires 
\begin{equation}
    A_{ij} = \overline{A_{ji}}, \quad C_{ij} = \overline{B_{ij}},
\end{equation}
or in other words 
\begin{equation}
    \vb{A} = \vb{A}^*, \quad \vb{B}^\top = - \vb{B}.
\end{equation}
Suppose 
\begin{equation}
    \psi = \pmqty{v_1 \\ \vdots \\ v_n \\ f_1 \\ \vdots \\ f_n},
\end{equation}
the EOM
\[
    \ii \hbar \dot{\psi} = \comm*{\psi}{H}
\]
now becomes 
\begin{equation}
    \dot{\psi} = \frac{\ii}{\hbar} \pmqty{\vb{A} & \vb{B} \\ \vb{B}^* & - \vb{A}^\top} \psi.
\end{equation}
The linear space of all possible Hamiltonians is actually an algebra on $W$, which we denote as $C(W)$.
We have 
\begin{equation}
    C(W) = C^0(W) \oplus C^1(W) \oplus C^2(W) \oplus \cdots,
\end{equation} 
where $C^n(W)$ is all possible Hamiltonians containing only $n$-operator terms. Note that the product 
structure on $C(W)$ is ordinary operator product, but $C^n(W)$ is not closed under ordinary operator product,
so $C^n(W)$ is actually a Lie algebra.

Now we are able to write down a theorem: 
\emph{We have the isomorphism $C^2(W) \simeq \mathfrak{so}(W, b)$}.

Now we can define symmetry on a Nambu space. An arbitrary unitary element $g$ and an anti-unitary element $h$ 
in symmetric group $G$ satisfied the conditions that 
\begin{equation}
    \expval*{\psi, \psi'} = \expval*{g \psi, g' \psi} = \overline{\expval*{h \psi, h' \psi}},
\end{equation}
and 
\begin{equation}
    b(\psi, \psi') = b(g \psi, g' \psi) = \overline{b(h \psi, h' \psi)},
\end{equation}
and we have 
\begin{equation}
    g H g^{-1} = H.
\end{equation}


\end{document}