\documentclass[hyperref, a4paper]{article}

\usepackage{geometry}
\usepackage{titling}
\usepackage{titlesec}
% No longer needed, since we will use enumitem package
% \usepackage{paralist}
\usepackage{enumitem}
\usepackage{footnote}
% Conflicts with enumitem
%\usepackage{enumerate}
\usepackage{amsmath, amssymb, amsthm}
\usepackage{mathtools}
\usepackage{bbm}
\usepackage{cite}
\usepackage{graphicx}
\usepackage{subfigure}
\usepackage{physics}
\usepackage{tensor}
\usepackage{siunitx}
\usepackage[version=4]{mhchem}
\usepackage{tikz}
\usepackage{xcolor}
\usepackage{listings}
\usepackage{autobreak}
\usepackage[ruled, vlined, linesnumbered]{algorithm2e}
\usepackage{nameref,zref-xr}
\zxrsetup{toltxlabel}
\zexternaldocument*[optics-]{../optics/optics}[optics.pdf]
\zexternaldocument*[solid-]{../solid/solid}[solid.pdf]
\usepackage[colorlinks,unicode]{hyperref} % , linkcolor=black, anchorcolor=black, citecolor=black, urlcolor=black, filecolor=black
\usepackage[most]{tcolorbox}
\usepackage{prettyref}

% Page style
\geometry{left=3.18cm,right=3.18cm,top=2.54cm,bottom=2.54cm}
\titlespacing{\paragraph}{0pt}{1pt}{10pt}[20pt]
\setlength{\droptitle}{-5em}
\preauthor{\vspace{-10pt}\begin{center}}
\postauthor{\par\end{center}}

% More compact lists 
\setlist[itemize]{
    itemindent=17pt, 
    leftmargin=1pt,
    listparindent=\parindent,
    parsep=0pt,
}

% Math operators
\DeclareMathOperator{\timeorder}{\mathcal{T}}
\DeclareMathOperator{\diag}{diag}
\DeclareMathOperator{\legpoly}{P}
\DeclareMathOperator{\primevalue}{P}
\DeclareMathOperator{\sgn}{sgn}
\newcommand*{\ii}{\mathrm{i}}
\newcommand*{\ee}{\mathrm{e}}
\newcommand*{\const}{\mathrm{const}}
\newcommand*{\suchthat}{\quad \text{s.t.} \quad}
\newcommand*{\argmin}{\arg\min}
\newcommand*{\argmax}{\arg\max}
\newcommand*{\normalorder}[1]{: #1 :}
\newcommand*{\pair}[1]{\langle #1 \rangle}
\newcommand*{\fd}[1]{\mathcal{D} #1}
\DeclareMathOperator{\bigO}{\mathcal{O}}

% TikZ setting
\usetikzlibrary{arrows,shapes,positioning}
\usetikzlibrary{calc}
\usetikzlibrary{arrows.meta}
\usetikzlibrary{decorations.markings}
\tikzstyle arrowstyle=[scale=1]
\tikzstyle directed=[postaction={decorate,decoration={markings,
    mark=at position .5 with {\arrow[arrowstyle]{stealth}}}}]
\tikzstyle ray=[directed, thick]
\tikzstyle dot=[anchor=base,fill,circle,inner sep=1pt]

% Algorithm setting
% Julia-style code
\SetKwIF{If}{ElseIf}{Else}{if}{}{elseif}{else}{end}
\SetKwFor{For}{for}{}{end}
\SetKwFor{While}{while}{}{end}
\SetKwProg{Function}{function}{}{end}
\SetArgSty{textnormal}

\newcommand*{\concept}[1]{{\textbf{#1}}}

% Embedded codes
\lstset{basicstyle=\ttfamily,
  showstringspaces=false,
  commentstyle=\color{gray},
  keywordstyle=\color{blue}
}

% Reference formatting
\newrefformat{fig}{Figure~\ref{#1} on page~\pageref{#1}}

% Color boxes
\tcbuselibrary{skins, breakable, theorems}
\newtcbtheorem[number within=section]{warning}{Warning}%
  {colback=orange!5,colframe=orange!65,fonttitle=\bfseries, breakable}{warn}
\newtcbtheorem[number within=section]{note}{Note}%
  {colback=green!5,colframe=green!65,fonttitle=\bfseries, breakable}{note}

\title{Stochastic Series Expansion by Yuanda Liao}
\author{Jinyuan Wu}
\date{November 27, 2021}    

\begin{document}

\maketitle

\section{The general formalism}

\concept{Stochastic Series Expansion (SSE)} is a Monte Carlo flavor that does not use discrete path integral
and therefore has no Trotter error.
Consider the partition function 
\[
    Z = \Trace \ee^{- \beta H}.
\]
We use the basis $\{\alpha\}$, and by Taylor series we have 
\[
    \begin{aligned}
        Z &= \sum_{\alpha} \expval{\sum_{n=0}^\infty \frac{1}{n!} (-\beta)^n H^n}{\alpha} \\
        &= \sum_{\{\alpha_i\}} \sum_{n=0}^\infty \frac{\beta^n}{n!} 
        \mel{\alpha_0}{-H}{\alpha_1} \mel{\alpha_1}{-H}{\alpha_2} \cdots \mel{\alpha_{n-1}}{-H}{\alpha_0}.
    \end{aligned}
\]
In the equation above we find for the $n$th term we have $n$ matrix element factors. 
Suppose we stop at the $M$th term, and to make the terms look more symmetric, we rephrase the partition 
function into 
\begin{equation}
    Z = \sum_{\alpha_1, \ldots, \alpha_M} \sum_{n=0}^\infty \frac{\beta^n (M-n)!}{M!} 
    \sum_{\{A_i\}} \prod_{i=1}^M \mel{\alpha_{i-1}}{A_i}{\alpha_i}, 
    \label{eq:partition-1}
\end{equation}
where $\alpha_0 = \alpha_M$, and there are $n$ $(-H)$ operators in the $\{A_i\}$ series, 
the rest being the unit operator. 

Now we consider a piecewise Hamiltonian 
\begin{equation}
    H = - \sum_a \sum_b H_{a, b},
\end{equation}
where the $a$ index refers to the operator type, which may be null operator, diagonal operator and off-diagonal 
operator. The $b$ index is the site index. For example for a 2D $L \times L$ square lattice, $b$ runs over 
$1$ to $L^2$. We denote $(a, b)$ as $S$, and now 
\begin{equation}
    Z = \sum_{S_1, \ldots, S_M} \sum_{\alpha_1, \ldots, \alpha_M} 
    \sum_{n=0}^\infty \frac{\beta^n (M-n)!}{M!} \prod_{i=1}^M \mel{\alpha_{i-1}}{(H_{i})_{a, b}}{\alpha_i}.
    \label{eq:partition-2}
\end{equation}

Now we see the configuration space: each configuration is to put 

\section{The 1D Heisenberg chain}

We consider the example of Heisenberg model. The diagonal operator is 
\begin{equation}
    H_{1, b} = \frac{1}{4} - S^z_{\vb*{i}} S^z_{\vb*{j}},
\end{equation}
and the off-diagonal operator is 
\begin{equation}
    H_{2, b} = \frac{1}{2} (S_{\vb*{i}}^+ S_{\vb*{j}}^- + \text{h.c.}).
\end{equation}
The Hamiltonian is 
\begin{equation}
    H = - \sum_b (H_{1, b} - H_{2, b}) + \frac{N}{4}.
\end{equation}
The basis can be chose as 
\begin{equation}
    \ket*{\alpha} = \ket*{\uparrow \downarrow}, \ket*{\uparrow \uparrow}, \ket*{\downarrow \downarrow}, \ket*{\downarrow \uparrow}.
\end{equation}
We find that 
\[
    \mel{\alpha}{-H_{2, b}}{\alpha} = - \frac{1}{2},
\]
which means \eqref{eq:partition-2} has sign problem. What we really deal with is the model 
\begin{equation}
    H = J \sum_{\pair{\vb*{i}, \vb*{j}}} (S^z_{\vb*{i}} S^z_{\vb*{j}} - S^x_{\vb*{i}} S^x_{\vb*{j}} - S^y_{\vb*{i}} S^y_{\vb*{j}}) = - \sum_b (H_{1, b} + H_{2, b}) + \frac{N}{4}.
\end{equation}

\begin{figure}
    \centering
    

\tikzset{every picture/.style={line width=0.75pt}} %set default line width to 0.75pt        

\begin{tikzpicture}[x=0.75pt,y=0.75pt,yscale=-1,xscale=1]
%uncomment if require: \path (0,432); %set diagram left start at 0, and has height of 432

%Shape: Circle [id:dp31722825590643344] 
\draw   (100,112.51) .. controls (100,108.36) and (103.36,105) .. (107.51,105) .. controls (111.66,105) and (115.02,108.36) .. (115.02,112.51) .. controls (115.02,116.66) and (111.66,120.02) .. (107.51,120.02) .. controls (103.36,120.02) and (100,116.66) .. (100,112.51) -- cycle ;
%Shape: Circle [id:dp44246308269501955] 
\draw   (132,112.51) .. controls (132,108.36) and (135.36,105) .. (139.51,105) .. controls (143.66,105) and (147.02,108.36) .. (147.02,112.51) .. controls (147.02,116.66) and (143.66,120.02) .. (139.51,120.02) .. controls (135.36,120.02) and (132,116.66) .. (132,112.51) -- cycle ;
%Shape: Circle [id:dp2837801121401218] 
\draw  [fill={rgb, 255:red, 0; green, 0; blue, 0 }  ,fill opacity=1 ] (163,112.51) .. controls (163,108.36) and (166.36,105) .. (170.51,105) .. controls (174.66,105) and (178.02,108.36) .. (178.02,112.51) .. controls (178.02,116.66) and (174.66,120.02) .. (170.51,120.02) .. controls (166.36,120.02) and (163,116.66) .. (163,112.51) -- cycle ;
%Shape: Circle [id:dp47001734296182285] 
\draw   (193,112.51) .. controls (193,108.36) and (196.36,105) .. (200.51,105) .. controls (204.66,105) and (208.02,108.36) .. (208.02,112.51) .. controls (208.02,116.66) and (204.66,120.02) .. (200.51,120.02) .. controls (196.36,120.02) and (193,116.66) .. (193,112.51) -- cycle ;
%Shape: Circle [id:dp2697930579085299] 
\draw   (224,112.51) .. controls (224,108.36) and (227.36,105) .. (231.51,105) .. controls (235.66,105) and (239.02,108.36) .. (239.02,112.51) .. controls (239.02,116.66) and (235.66,120.02) .. (231.51,120.02) .. controls (227.36,120.02) and (224,116.66) .. (224,112.51) -- cycle ;
%Shape: Circle [id:dp1867050255182645] 
\draw  [fill={rgb, 255:red, 0; green, 0; blue, 0 }  ,fill opacity=1 ] (256,112.51) .. controls (256,108.36) and (259.36,105) .. (263.51,105) .. controls (267.66,105) and (271.02,108.36) .. (271.02,112.51) .. controls (271.02,116.66) and (267.66,120.02) .. (263.51,120.02) .. controls (259.36,120.02) and (256,116.66) .. (256,112.51) -- cycle ;
%Shape: Circle [id:dp7190507439832894] 
\draw   (287,112.51) .. controls (287,108.36) and (290.36,105) .. (294.51,105) .. controls (298.66,105) and (302.02,108.36) .. (302.02,112.51) .. controls (302.02,116.66) and (298.66,120.02) .. (294.51,120.02) .. controls (290.36,120.02) and (287,116.66) .. (287,112.51) -- cycle ;
%Shape: Circle [id:dp957708285634497] 
\draw   (317,112.51) .. controls (317,108.36) and (320.36,105) .. (324.51,105) .. controls (328.66,105) and (332.02,108.36) .. (332.02,112.51) .. controls (332.02,116.66) and (328.66,120.02) .. (324.51,120.02) .. controls (320.36,120.02) and (317,116.66) .. (317,112.51) -- cycle ;
%Shape: Circle [id:dp10934977824206427] 
\draw   (100,141.51) .. controls (100,137.36) and (103.36,134) .. (107.51,134) .. controls (111.66,134) and (115.02,137.36) .. (115.02,141.51) .. controls (115.02,145.66) and (111.66,149.02) .. (107.51,149.02) .. controls (103.36,149.02) and (100,145.66) .. (100,141.51) -- cycle ;
%Shape: Circle [id:dp2408504361244317] 
\draw   (132,141.51) .. controls (132,137.36) and (135.36,134) .. (139.51,134) .. controls (143.66,134) and (147.02,137.36) .. (147.02,141.51) .. controls (147.02,145.66) and (143.66,149.02) .. (139.51,149.02) .. controls (135.36,149.02) and (132,145.66) .. (132,141.51) -- cycle ;
%Shape: Circle [id:dp7688026290964838] 
\draw  [fill={rgb, 255:red, 0; green, 0; blue, 0 }  ,fill opacity=1 ] (163,141.51) .. controls (163,137.36) and (166.36,134) .. (170.51,134) .. controls (174.66,134) and (178.02,137.36) .. (178.02,141.51) .. controls (178.02,145.66) and (174.66,149.02) .. (170.51,149.02) .. controls (166.36,149.02) and (163,145.66) .. (163,141.51) -- cycle ;
%Shape: Circle [id:dp019219278160686804] 
\draw   (193,141.51) .. controls (193,137.36) and (196.36,134) .. (200.51,134) .. controls (204.66,134) and (208.02,137.36) .. (208.02,141.51) .. controls (208.02,145.66) and (204.66,149.02) .. (200.51,149.02) .. controls (196.36,149.02) and (193,145.66) .. (193,141.51) -- cycle ;
%Shape: Circle [id:dp07989988676765969] 
\draw   (224,141.51) .. controls (224,137.36) and (227.36,134) .. (231.51,134) .. controls (235.66,134) and (239.02,137.36) .. (239.02,141.51) .. controls (239.02,145.66) and (235.66,149.02) .. (231.51,149.02) .. controls (227.36,149.02) and (224,145.66) .. (224,141.51) -- cycle ;
%Shape: Circle [id:dp3393179505408068] 
\draw  [fill={rgb, 255:red, 0; green, 0; blue, 0 }  ,fill opacity=1 ] (256,141.51) .. controls (256,137.36) and (259.36,134) .. (263.51,134) .. controls (267.66,134) and (271.02,137.36) .. (271.02,141.51) .. controls (271.02,145.66) and (267.66,149.02) .. (263.51,149.02) .. controls (259.36,149.02) and (256,145.66) .. (256,141.51) -- cycle ;
%Shape: Circle [id:dp9888009724890388] 
\draw   (287,141.51) .. controls (287,137.36) and (290.36,134) .. (294.51,134) .. controls (298.66,134) and (302.02,137.36) .. (302.02,141.51) .. controls (302.02,145.66) and (298.66,149.02) .. (294.51,149.02) .. controls (290.36,149.02) and (287,145.66) .. (287,141.51) -- cycle ;
%Shape: Circle [id:dp8912386073932812] 
\draw   (317,141.51) .. controls (317,137.36) and (320.36,134) .. (324.51,134) .. controls (328.66,134) and (332.02,137.36) .. (332.02,141.51) .. controls (332.02,145.66) and (328.66,149.02) .. (324.51,149.02) .. controls (320.36,149.02) and (317,145.66) .. (317,141.51) -- cycle ;
%Shape: Circle [id:dp5977394861316492] 
\draw   (100,170.51) .. controls (100,166.36) and (103.36,163) .. (107.51,163) .. controls (111.66,163) and (115.02,166.36) .. (115.02,170.51) .. controls (115.02,174.66) and (111.66,178.02) .. (107.51,178.02) .. controls (103.36,178.02) and (100,174.66) .. (100,170.51) -- cycle ;
%Shape: Circle [id:dp20762989990704117] 
\draw   (132,170.51) .. controls (132,166.36) and (135.36,163) .. (139.51,163) .. controls (143.66,163) and (147.02,166.36) .. (147.02,170.51) .. controls (147.02,174.66) and (143.66,178.02) .. (139.51,178.02) .. controls (135.36,178.02) and (132,174.66) .. (132,170.51) -- cycle ;
%Shape: Circle [id:dp01484879103231096] 
\draw  [fill={rgb, 255:red, 0; green, 0; blue, 0 }  ,fill opacity=1 ] (163,170.51) .. controls (163,166.36) and (166.36,163) .. (170.51,163) .. controls (174.66,163) and (178.02,166.36) .. (178.02,170.51) .. controls (178.02,174.66) and (174.66,178.02) .. (170.51,178.02) .. controls (166.36,178.02) and (163,174.66) .. (163,170.51) -- cycle ;
%Shape: Circle [id:dp7801188162409032] 
\draw   (193,170.51) .. controls (193,166.36) and (196.36,163) .. (200.51,163) .. controls (204.66,163) and (208.02,166.36) .. (208.02,170.51) .. controls (208.02,174.66) and (204.66,178.02) .. (200.51,178.02) .. controls (196.36,178.02) and (193,174.66) .. (193,170.51) -- cycle ;
%Shape: Circle [id:dp1404064779802885] 
\draw   (224,170.51) .. controls (224,166.36) and (227.36,163) .. (231.51,163) .. controls (235.66,163) and (239.02,166.36) .. (239.02,170.51) .. controls (239.02,174.66) and (235.66,178.02) .. (231.51,178.02) .. controls (227.36,178.02) and (224,174.66) .. (224,170.51) -- cycle ;
%Shape: Circle [id:dp36048277842311505] 
\draw  [fill={rgb, 255:red, 0; green, 0; blue, 0 }  ,fill opacity=1 ] (256,170.51) .. controls (256,166.36) and (259.36,163) .. (263.51,163) .. controls (267.66,163) and (271.02,166.36) .. (271.02,170.51) .. controls (271.02,174.66) and (267.66,178.02) .. (263.51,178.02) .. controls (259.36,178.02) and (256,174.66) .. (256,170.51) -- cycle ;
%Shape: Circle [id:dp7700718318126718] 
\draw   (287,170.51) .. controls (287,166.36) and (290.36,163) .. (294.51,163) .. controls (298.66,163) and (302.02,166.36) .. (302.02,170.51) .. controls (302.02,174.66) and (298.66,178.02) .. (294.51,178.02) .. controls (290.36,178.02) and (287,174.66) .. (287,170.51) -- cycle ;
%Shape: Circle [id:dp5106282363761969] 
\draw   (317,170.51) .. controls (317,166.36) and (320.36,163) .. (324.51,163) .. controls (328.66,163) and (332.02,166.36) .. (332.02,170.51) .. controls (332.02,174.66) and (328.66,178.02) .. (324.51,178.02) .. controls (320.36,178.02) and (317,174.66) .. (317,170.51) -- cycle ;
%Shape: Circle [id:dp47933475298653216] 
\draw   (100,198.51) .. controls (100,194.36) and (103.36,191) .. (107.51,191) .. controls (111.66,191) and (115.02,194.36) .. (115.02,198.51) .. controls (115.02,202.66) and (111.66,206.02) .. (107.51,206.02) .. controls (103.36,206.02) and (100,202.66) .. (100,198.51) -- cycle ;
%Shape: Circle [id:dp8564272353881015] 
\draw   (132,198.51) .. controls (132,194.36) and (135.36,191) .. (139.51,191) .. controls (143.66,191) and (147.02,194.36) .. (147.02,198.51) .. controls (147.02,202.66) and (143.66,206.02) .. (139.51,206.02) .. controls (135.36,206.02) and (132,202.66) .. (132,198.51) -- cycle ;
%Shape: Circle [id:dp950149932276249] 
\draw  [fill={rgb, 255:red, 0; green, 0; blue, 0 }  ,fill opacity=1 ] (163,198.51) .. controls (163,194.36) and (166.36,191) .. (170.51,191) .. controls (174.66,191) and (178.02,194.36) .. (178.02,198.51) .. controls (178.02,202.66) and (174.66,206.02) .. (170.51,206.02) .. controls (166.36,206.02) and (163,202.66) .. (163,198.51) -- cycle ;
%Shape: Circle [id:dp8117697126360914] 
\draw   (193,198.51) .. controls (193,194.36) and (196.36,191) .. (200.51,191) .. controls (204.66,191) and (208.02,194.36) .. (208.02,198.51) .. controls (208.02,202.66) and (204.66,206.02) .. (200.51,206.02) .. controls (196.36,206.02) and (193,202.66) .. (193,198.51) -- cycle ;
%Shape: Circle [id:dp7055013815234348] 
\draw  [fill={rgb, 255:red, 0; green, 0; blue, 0 }  ,fill opacity=1 ] (224,198.51) .. controls (224,194.36) and (227.36,191) .. (231.51,191) .. controls (235.66,191) and (239.02,194.36) .. (239.02,198.51) .. controls (239.02,202.66) and (235.66,206.02) .. (231.51,206.02) .. controls (227.36,206.02) and (224,202.66) .. (224,198.51) -- cycle ;
%Shape: Circle [id:dp10078661261381372] 
\draw   (256,198.51) .. controls (256,194.36) and (259.36,191) .. (263.51,191) .. controls (267.66,191) and (271.02,194.36) .. (271.02,198.51) .. controls (271.02,202.66) and (267.66,206.02) .. (263.51,206.02) .. controls (259.36,206.02) and (256,202.66) .. (256,198.51) -- cycle ;
%Shape: Circle [id:dp7079689616434257] 
\draw   (287,198.51) .. controls (287,194.36) and (290.36,191) .. (294.51,191) .. controls (298.66,191) and (302.02,194.36) .. (302.02,198.51) .. controls (302.02,202.66) and (298.66,206.02) .. (294.51,206.02) .. controls (290.36,206.02) and (287,202.66) .. (287,198.51) -- cycle ;
%Shape: Circle [id:dp750266960300866] 
\draw   (317,198.51) .. controls (317,194.36) and (320.36,191) .. (324.51,191) .. controls (328.66,191) and (332.02,194.36) .. (332.02,198.51) .. controls (332.02,202.66) and (328.66,206.02) .. (324.51,206.02) .. controls (320.36,206.02) and (317,202.66) .. (317,198.51) -- cycle ;
%Shape: Circle [id:dp7677972410556284] 
\draw   (100,227.51) .. controls (100,223.36) and (103.36,220) .. (107.51,220) .. controls (111.66,220) and (115.02,223.36) .. (115.02,227.51) .. controls (115.02,231.66) and (111.66,235.02) .. (107.51,235.02) .. controls (103.36,235.02) and (100,231.66) .. (100,227.51) -- cycle ;
%Shape: Circle [id:dp8213574743077503] 
\draw   (132,227.51) .. controls (132,223.36) and (135.36,220) .. (139.51,220) .. controls (143.66,220) and (147.02,223.36) .. (147.02,227.51) .. controls (147.02,231.66) and (143.66,235.02) .. (139.51,235.02) .. controls (135.36,235.02) and (132,231.66) .. (132,227.51) -- cycle ;
%Shape: Circle [id:dp43188470242206933] 
\draw  [fill={rgb, 255:red, 0; green, 0; blue, 0 }  ,fill opacity=1 ] (163,227.51) .. controls (163,223.36) and (166.36,220) .. (170.51,220) .. controls (174.66,220) and (178.02,223.36) .. (178.02,227.51) .. controls (178.02,231.66) and (174.66,235.02) .. (170.51,235.02) .. controls (166.36,235.02) and (163,231.66) .. (163,227.51) -- cycle ;
%Shape: Circle [id:dp08112199979895318] 
\draw   (193,227.51) .. controls (193,223.36) and (196.36,220) .. (200.51,220) .. controls (204.66,220) and (208.02,223.36) .. (208.02,227.51) .. controls (208.02,231.66) and (204.66,235.02) .. (200.51,235.02) .. controls (196.36,235.02) and (193,231.66) .. (193,227.51) -- cycle ;
%Shape: Circle [id:dp541735165236882] 
\draw  [fill={rgb, 255:red, 0; green, 0; blue, 0 }  ,fill opacity=1 ] (224,227.51) .. controls (224,223.36) and (227.36,220) .. (231.51,220) .. controls (235.66,220) and (239.02,223.36) .. (239.02,227.51) .. controls (239.02,231.66) and (235.66,235.02) .. (231.51,235.02) .. controls (227.36,235.02) and (224,231.66) .. (224,227.51) -- cycle ;
%Shape: Circle [id:dp320941185699114] 
\draw   (256,227.51) .. controls (256,223.36) and (259.36,220) .. (263.51,220) .. controls (267.66,220) and (271.02,223.36) .. (271.02,227.51) .. controls (271.02,231.66) and (267.66,235.02) .. (263.51,235.02) .. controls (259.36,235.02) and (256,231.66) .. (256,227.51) -- cycle ;
%Shape: Circle [id:dp5272127829686224] 
\draw   (287,227.51) .. controls (287,223.36) and (290.36,220) .. (294.51,220) .. controls (298.66,220) and (302.02,223.36) .. (302.02,227.51) .. controls (302.02,231.66) and (298.66,235.02) .. (294.51,235.02) .. controls (290.36,235.02) and (287,231.66) .. (287,227.51) -- cycle ;
%Shape: Circle [id:dp7656996328639087] 
\draw   (317,227.51) .. controls (317,223.36) and (320.36,220) .. (324.51,220) .. controls (328.66,220) and (332.02,223.36) .. (332.02,227.51) .. controls (332.02,231.66) and (328.66,235.02) .. (324.51,235.02) .. controls (320.36,235.02) and (317,231.66) .. (317,227.51) -- cycle ;
%Straight Lines [id:da5247408159772211] 
\draw    (87.51,79.51) -- (108.51,79.51) ;
\draw [shift={(108.51,79.51)}, rotate = 360] [color={rgb, 255:red, 0; green, 0; blue, 0 }  ][fill={rgb, 255:red, 0; green, 0; blue, 0 }  ][line width=0.75]      (0, 0) circle [x radius= 3.35, y radius= 3.35]   ;
%Straight Lines [id:da09164979762772574] 
\draw    (108.51,79.51) -- (139.51,79.51) ;
\draw [shift={(139.51,79.51)}, rotate = 0] [color={rgb, 255:red, 0; green, 0; blue, 0 }  ][fill={rgb, 255:red, 0; green, 0; blue, 0 }  ][line width=0.75]      (0, 0) circle [x radius= 3.35, y radius= 3.35]   ;
%Straight Lines [id:da0345079423695851] 
\draw    (139.51,79.51) -- (170.51,79.51) ;
\draw [shift={(170.51,79.51)}, rotate = 0] [color={rgb, 255:red, 0; green, 0; blue, 0 }  ][fill={rgb, 255:red, 0; green, 0; blue, 0 }  ][line width=0.75]      (0, 0) circle [x radius= 3.35, y radius= 3.35]   ;
%Straight Lines [id:da9841114153630979] 
\draw    (169.51,79.51) -- (200.51,79.51) ;
\draw [shift={(200.51,79.51)}, rotate = 0] [color={rgb, 255:red, 0; green, 0; blue, 0 }  ][fill={rgb, 255:red, 0; green, 0; blue, 0 }  ][line width=0.75]      (0, 0) circle [x radius= 3.35, y radius= 3.35]   ;
%Straight Lines [id:da5006765663112398] 
\draw    (200.51,79.51) -- (231.51,79.51) ;
\draw [shift={(231.51,79.51)}, rotate = 0] [color={rgb, 255:red, 0; green, 0; blue, 0 }  ][fill={rgb, 255:red, 0; green, 0; blue, 0 }  ][line width=0.75]      (0, 0) circle [x radius= 3.35, y radius= 3.35]   ;
%Straight Lines [id:da9158191167258407] 
\draw    (231.51,79.51) -- (263.51,79.51) ;
\draw [shift={(263.51,79.51)}, rotate = 0] [color={rgb, 255:red, 0; green, 0; blue, 0 }  ][fill={rgb, 255:red, 0; green, 0; blue, 0 }  ][line width=0.75]      (0, 0) circle [x radius= 3.35, y radius= 3.35]   ;
%Straight Lines [id:da7691375830787559] 
\draw    (263.51,79.51) -- (294.51,79.51) ;
\draw [shift={(294.51,79.51)}, rotate = 0] [color={rgb, 255:red, 0; green, 0; blue, 0 }  ][fill={rgb, 255:red, 0; green, 0; blue, 0 }  ][line width=0.75]      (0, 0) circle [x radius= 3.35, y radius= 3.35]   ;
%Straight Lines [id:da06222142347636783] 
\draw    (294.51,79.51) -- (325.51,79.51) ;
\draw [shift={(325.51,79.51)}, rotate = 0] [color={rgb, 255:red, 0; green, 0; blue, 0 }  ][fill={rgb, 255:red, 0; green, 0; blue, 0 }  ][line width=0.75]      (0, 0) circle [x radius= 3.35, y radius= 3.35]   ;
%Straight Lines [id:da2161972505625398] 
\draw    (325.51,79.51) -- (354.52,79.51) ;
\draw [shift={(356.52,79.51)}, rotate = 180] [fill={rgb, 255:red, 0; green, 0; blue, 0 }  ][line width=0.08]  [draw opacity=0] (12,-3) -- (0,0) -- (12,3) -- cycle    ;
%Straight Lines [id:da7216976495942904] 
\draw    (40,84) -- (40,251.14) ;
\draw [shift={(40,253.14)}, rotate = 270] [fill={rgb, 255:red, 0; green, 0; blue, 0 }  ][line width=0.08]  [draw opacity=0] (12,-3) -- (0,0) -- (12,3) -- cycle    ;
%Shape: Rectangle [id:dp7405651742216428] 
\draw   (136,125) -- (172.02,125) -- (172.02,128.14) -- (136,128.14) -- cycle ;
%Shape: Rectangle [id:dp7279185560469399] 
\draw  [fill={rgb, 255:red, 0; green, 0; blue, 0 }  ,fill opacity=1 ] (231,183) -- (267.02,183) -- (267.02,186.14) -- (231,186.14) -- cycle ;
%Shape: Circle [id:dp1762261082584451] 
\draw   (100,254.51) .. controls (100,250.36) and (103.36,247) .. (107.51,247) .. controls (111.66,247) and (115.02,250.36) .. (115.02,254.51) .. controls (115.02,258.66) and (111.66,262.02) .. (107.51,262.02) .. controls (103.36,262.02) and (100,258.66) .. (100,254.51) -- cycle ;
%Shape: Circle [id:dp8500017347496296] 
\draw   (132,254.51) .. controls (132,250.36) and (135.36,247) .. (139.51,247) .. controls (143.66,247) and (147.02,250.36) .. (147.02,254.51) .. controls (147.02,258.66) and (143.66,262.02) .. (139.51,262.02) .. controls (135.36,262.02) and (132,258.66) .. (132,254.51) -- cycle ;
%Shape: Circle [id:dp6577932075047741] 
\draw  [fill={rgb, 255:red, 0; green, 0; blue, 0 }  ,fill opacity=1 ] (163,254.51) .. controls (163,250.36) and (166.36,247) .. (170.51,247) .. controls (174.66,247) and (178.02,250.36) .. (178.02,254.51) .. controls (178.02,258.66) and (174.66,262.02) .. (170.51,262.02) .. controls (166.36,262.02) and (163,258.66) .. (163,254.51) -- cycle ;
%Shape: Circle [id:dp763687693742481] 
\draw   (193,254.51) .. controls (193,250.36) and (196.36,247) .. (200.51,247) .. controls (204.66,247) and (208.02,250.36) .. (208.02,254.51) .. controls (208.02,258.66) and (204.66,262.02) .. (200.51,262.02) .. controls (196.36,262.02) and (193,258.66) .. (193,254.51) -- cycle ;
%Shape: Circle [id:dp3406498715354074] 
\draw   (224,254.51) .. controls (224,250.36) and (227.36,247) .. (231.51,247) .. controls (235.66,247) and (239.02,250.36) .. (239.02,254.51) .. controls (239.02,258.66) and (235.66,262.02) .. (231.51,262.02) .. controls (227.36,262.02) and (224,258.66) .. (224,254.51) -- cycle ;
%Shape: Circle [id:dp8600534538743556] 
\draw  [color={rgb, 255:red, 0; green, 0; blue, 0 }  ,draw opacity=1 ][fill={rgb, 255:red, 0; green, 0; blue, 0 }  ,fill opacity=1 ] (256,254.51) .. controls (256,250.36) and (259.36,247) .. (263.51,247) .. controls (267.66,247) and (271.02,250.36) .. (271.02,254.51) .. controls (271.02,258.66) and (267.66,262.02) .. (263.51,262.02) .. controls (259.36,262.02) and (256,258.66) .. (256,254.51) -- cycle ;
%Shape: Circle [id:dp8913282673874046] 
\draw   (287,254.51) .. controls (287,250.36) and (290.36,247) .. (294.51,247) .. controls (298.66,247) and (302.02,250.36) .. (302.02,254.51) .. controls (302.02,258.66) and (298.66,262.02) .. (294.51,262.02) .. controls (290.36,262.02) and (287,258.66) .. (287,254.51) -- cycle ;
%Shape: Circle [id:dp6419517529803846] 
\draw   (317,254.51) .. controls (317,250.36) and (320.36,247) .. (324.51,247) .. controls (328.66,247) and (332.02,250.36) .. (332.02,254.51) .. controls (332.02,258.66) and (328.66,262.02) .. (324.51,262.02) .. controls (320.36,262.02) and (317,258.66) .. (317,254.51) -- cycle ;
%Shape: Rectangle [id:dp8950525315085944] 
\draw   (166,211) -- (202.02,211) -- (202.02,214.14) -- (166,214.14) -- cycle ;
%Shape: Rectangle [id:dp4451776349406389] 
\draw  [fill={rgb, 255:red, 0; green, 0; blue, 0 }  ,fill opacity=1 ] (230,240) -- (266.02,240) -- (266.02,243.14) -- (230,243.14) -- cycle ;

% Text Node
\draw (358.52,79.51) node [anchor=west] [inner sep=0.75pt]    {$i$};
% Text Node
\draw (384.52,79.51) node [anchor=west] [inner sep=0.75pt]   [align=left] {site index};
% Text Node
\draw (40,256.54) node [anchor=north] [inner sep=0.75pt]    {$n$};


\end{tikzpicture}

    \caption{An SSE configuration with $M=6$}
    \label{fig:chain-config}
\end{figure}

We consider a 1D Heisenberg chain. \prettyref{fig:chain-config} gives a schematic configuration.

The update scheme is the follows:
\begin{enumerate}
    \item Sampling $\alpha_0$.
    \item Diagonal update. 
    \item Non-diagonal update.
\end{enumerate} 

Diagonal update can be done by Metropolis update. Whenever we meet a 

\end{document}