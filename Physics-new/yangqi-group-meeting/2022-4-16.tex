\documentclass[hyperref, a4paper]{article}

\usepackage{geometry}
\usepackage{titling}
\usepackage{titlesec}
% No longer needed, since we will use enumitem package
% \usepackage{paralist}
% \usepackage{enumitem}
\usepackage{footnote}
\usepackage{marginnote}
\usepackage{enumerate}
\usepackage{amsmath, amssymb, amsthm}
\usepackage{mathtools}
\usepackage{bbm}
\usepackage{cite}
\usepackage{graphicx}
\usepackage{subfigure}
\usepackage{physics}
\usepackage{tensor}
\usepackage{siunitx}
\usepackage[version=4]{mhchem}
\usepackage{tikz}
\usepackage{xcolor}
\usepackage{listings}
\usepackage{autobreak}
\usepackage[ruled, vlined, linesnumbered]{algorithm2e}
\usepackage{nameref,zref-xr}
\zxrsetup{toltxlabel}
\zexternaldocument*[gr-]{../relativity/relativistic}[relativistic.pdf]
\zexternaldocument*[optics-]{../optics/optics}[optics.pdf]
\zexternaldocument*[solid-]{../solid/solid}[solid.pdf]
\zexternaldocument*[kt-]{../topological-phases-reading-notes/kt}[kt.pdf]
\usepackage[colorlinks,unicode]{hyperref} % , linkcolor=black, anchorcolor=black, citecolor=black, urlcolor=black, filecolor=black
\usepackage[most]{tcolorbox}
\usepackage{prettyref}

% Page style
\geometry{left=3.18cm,right=3.18cm,top=2.54cm,bottom=2.54cm}
\titlespacing{\paragraph}{0pt}{1pt}{10pt}[20pt]
\setlength{\droptitle}{-5em}
%\preauthor{\vspace{-10pt}\begin{center}}
%\postauthor{\par\end{center}}

% More compact lists 
%\setlist[itemize]{
%    itemindent=17pt, 
%    leftmargin=1pt,
%    listparindent=\parindent,
%    parsep=0pt,
%}

% Math operators
\DeclareMathOperator{\timeorder}{\mathcal{T}}
\DeclareMathOperator{\diag}{diag}
\DeclareMathOperator{\legpoly}{P}
\DeclareMathOperator{\primevalue}{P}
\DeclareMathOperator{\sgn}{sgn}
\newcommand*{\ii}{\mathrm{i}}
\newcommand*{\ee}{\mathrm{e}}
\newcommand*{\const}{\mathrm{const}}
\newcommand*{\suchthat}{\quad \text{s.t.} \quad}
\newcommand*{\argmin}{\arg\min}
\newcommand*{\argmax}{\arg\max}
\newcommand*{\normalorder}[1]{: #1 :}
\newcommand*{\pair}[1]{\langle #1 \rangle}
\newcommand*{\fd}[1]{\mathcal{D} #1}
\DeclareMathOperator{\bigO}{\mathcal{O}}

% TikZ setting
\usetikzlibrary{arrows,shapes,positioning}
\usetikzlibrary{arrows.meta}
\usetikzlibrary{decorations.markings}
\tikzstyle arrowstyle=[scale=1]
\tikzstyle directed=[postaction={decorate,decoration={markings,
    mark=at position .5 with {\arrow[arrowstyle]{stealth}}}}]
\tikzstyle ray=[directed, thick]
\tikzstyle dot=[anchor=base,fill,circle,inner sep=1pt]
\usetikzlibrary{fadings}
\usetikzlibrary{patterns}
\usetikzlibrary{shadows.blur}
\usetikzlibrary{shapes}

% Algorithm setting
% Julia-style code
\SetKwIF{If}{ElseIf}{Else}{if}{}{elseif}{else}{end}
\SetKwFor{For}{for}{}{end}
\SetKwFor{While}{while}{}{end}
\SetKwProg{Function}{function}{}{end}
\SetArgSty{textnormal}

\newcommand*{\concept}[1]{{\textbf{#1}}}

% Support for tensor double arrows.
\renewcommand{\tensor}[1]{ \stackrel{\leftrightarrow}{\vb*{#1}}}

% Embedded codes
\lstset{basicstyle=\ttfamily,
  showstringspaces=false,
  commentstyle=\color{gray},
  keywordstyle=\color{blue}
}

% Reference formatting
\newrefformat{fig}{Figure~\ref{#1} on page~\pageref{#1}}

% Color boxes
\tcbuselibrary{skins, breakable, theorems}
\newtcbtheorem[number within=section]{warning}{Warning}%
  {colback=orange!5,colframe=orange!65,fonttitle=\bfseries, breakable}{warn}
\newtcbtheorem[number within=section]{note}{Note}%
  {colback=green!5,colframe=green!65,fonttitle=\bfseries, breakable}{note}

% Correctly displaying equation number in section titles
\pdfstringdefDisableCommands{\def\eqref#1{(\ref{#1})}}

\title{Kun Chen on effective field theory of Fermi liquid}
\author{Jinyuan Wu}

\begin{document}

\maketitle

\section{UV-complete effective field theory of Fermi liquid}

Fermi liquid theory, despite being introduced in all decent condensed matter physics textbooks, is kind of 
confusing for every beginner. A root cause is there are actually several independent formulation of 
Fermi liquid theory. The original theory is about the energy functional of eigenstates. 
Polchinski and Shankar defined a non-UV-complete theory with path integral formulation \cite{Polchinski-fermi},
and it completely ignored the details inside the Fermi sea. Under this framework, we have several possible low-energy
interaction channels (or effective vortices)
\[
    k_1 , \  k_2 \stackrel{\Gamma_4}{\longrightarrow} k_1 - q , \  k_2 + q
\]
all of which are near the Fermi surface, including
\begin{itemize}
    \item (non-)Fermi liquid, where the forward scattering channel ($q \to 0$) is the most important;
    \item superconductivity, where the Cooper pair channel ($k_1 + k_2 \to 0$) is the dominant;
    \item CDW or SDW order, where the $q \to q_\text{c}$ channel is the most important. Note that 
    these orders already call for a UV-complete theory, because strictly speaking, only forward
    scattering and 
\end{itemize}
These orders are separate in the phase diagram, but we know they are all emergent from the condensed matter 
field theory, and thus it's tempting to find a \emph{grand unification theory} of Fermi-liquid-like 
field theories, which is still a Fermi-liquid-like EFT of the complete Coulomb interacting electron gas 
but has more to say about the Fermi sea, because the Cooper pair channel is a high-energy channel for a 
Fermi liquid, and vice versa, so a theory taking into account all possible Fermi liquid phases involves 
energy scales higher than the usual energy scale around the Fermi surface and therefore has to account for 
how the \emph{deep} Fermi sea is disrupted. 

Compared to a free electron gas, charged Fermi liquid undergoes significant correction in 
\begin{itemize}
    \item renormalization of the two-point correlation, including $m^*$ and $Z < 1$, and
    \item renormalization of the four-particle vortex and introduction of Landau parameters.
\end{itemize}

TODO: why there appears expressions like $v_q + f_0^+$, which means the Landau parameters are channels independent to Coulomb interaction?
Reference: PRB 20.550 1979, Kukkonen-Overhauser interaction; K. Haule \& K. Chen, Sci Rep 12, 2292, 2002

For how to define Landau parameters, see \cite{PhysRevLett.75.689}

For Coulomb screening as a stable RG fixed-point, see \cite{PhysRevB.104.195142}
The fixed point is a point where the Coulomb interaction is ``perfectly'' screened, while the $r_\text{s} 
> r_\text{s, fixed}$ region is the over-screened region and $r_\text{s} < r_\text{s, fixed}$ is the under-screened region. 
It can be verified that Li, Na, K and Cs are close to the perfectly screened point, which explains why GW or RPA methods fail for them -- these methods are too close to under-screened limit, while in Li, Na, K and Cs the Coulomb interaction is almost completely screened.
No stable metal known exists in the over-screened region, where electrons attract each other. \cite{RevModPhys.53.81} showed that electrons are still stable in this case, but this may not be the case for ions. Further molecular dynamics research is required to check whether this is the case.

Renormalized perturbation theory beyond the tree level, very similar to techniques used when dealing with QED, can be used to find perturbation around the Fermi-liquid state.

\section{Superconductivity without phonons}

There are already several superconductivity mechanisms without phonons. 
In Kohn-Luttinger mechanism \cite{PhysRevLett.15.524}, singularity of polarization at $q = 2 k_\text{F}$ induces an attractive effective interaction, so with sufficiently low temperature, it seems all pure charged Fermi liquid falls into superconductivity state.
In dynamic screening mechanism \cite{PhysRevB.28.5100}, TODO

Anderson pseudopotentials: superconductivity in real materials, which can also be explained by the theory here?

\section{Discrete Lehmann representation}

\concept{Discrete Lehmann representation} is a numerical technique proposed in \cite{PhysRevB.104.195142}.
A generic Lehmann representation is in the form of 
\begin{equation}
    G(\ii \omega_n) = \int \dd{\omega} \frac{1}{\ii \omega_n + \omega} \rho(\omega),
\end{equation}
and if we just take a few peaks into account, we have 
\begin{equation}
    G(\ii \omega_n) = \sum_k \frac{1}{\ii \omega_n + \omega_k} g_k.
\end{equation}
After inverse Fourier transformation, we have 
\begin{equation}
    G(\tau) = \sum_k \underbrace{\frac{\ee^{- \omega_k \tau}}{1 + \ee^{- \beta \omega_k }}}_{\eqqcolon K(\omega_k, \tau)} g_k.
    \label{eq:discrete-lehmann-representation}
\end{equation}
Comparing this with the original 
\begin{equation}
    G(\tau) = \int \dd{\omega} \rho(\omega) K(\omega, \tau),
\end{equation}
we find that in principle, \eqref{eq:discrete-lehmann-representation} as an ansatz is a good approximation of
the Green function, even though it involves much fewer frequencies.

\section{\texttt{NumericalEFT.jl}}



\bibliographystyle{plain}
\bibliography{../solid/fermi-liquid,non-bcs-sc}

\end{document}