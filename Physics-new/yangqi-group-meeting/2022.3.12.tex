\documentclass[hyperref, a4paper]{article}

\usepackage{geometry}
\usepackage{titling}
\usepackage{titlesec}
% No longer needed, since we will use enumitem package
% \usepackage{paralist}
\usepackage{enumitem}
\usepackage{footnote}
% Conflicts with enumitem
%\usepackage{enumerate}
\usepackage{amsmath, amssymb, amsthm}
\usepackage{mathtools}
\usepackage{bbm}
\usepackage{cite}
\usepackage{graphicx}
\usepackage{subfigure}
\usepackage{physics}
\usepackage{tensor}
\usepackage{siunitx}
\usepackage[version=4]{mhchem}
\usepackage{tikz}
\usepackage{xcolor}
\usepackage{listings}
\usepackage{autobreak}
\usepackage[ruled, vlined, linesnumbered]{algorithm2e}
\usepackage{nameref,zref-xr}
\zxrsetup{toltxlabel}
\zexternaldocument*[optics-]{../optics/optics}[optics.pdf]
\zexternaldocument*[solid-]{../solid/solid}[solid.pdf]
\usepackage[colorlinks,unicode]{hyperref} % , linkcolor=black, anchorcolor=black, citecolor=black, urlcolor=black, filecolor=black
\usepackage[most]{tcolorbox}
\usepackage{prettyref}

% Page style
\geometry{left=3.18cm,right=3.18cm,top=2.54cm,bottom=2.54cm}
\titlespacing{\paragraph}{0pt}{1pt}{10pt}[20pt]
\setlength{\droptitle}{-5em}
\preauthor{\vspace{-10pt}\begin{center}}
\postauthor{\par\end{center}}

% More compact lists 
\setlist[itemize]{
    itemindent=17pt, 
    leftmargin=1pt,
    listparindent=\parindent,
    parsep=0pt,
}

% Math operators
\DeclareMathOperator{\timeorder}{\mathcal{T}}
\DeclareMathOperator{\diag}{diag}
\DeclareMathOperator{\legpoly}{P}
\DeclareMathOperator{\primevalue}{P}
\DeclareMathOperator{\sgn}{sgn}
\newcommand*{\ii}{\mathrm{i}}
\newcommand*{\ee}{\mathrm{e}}
\newcommand*{\const}{\mathrm{const}}
\newcommand*{\suchthat}{\quad \text{s.t.} \quad}
\newcommand*{\argmin}{\arg\min}
\newcommand*{\argmax}{\arg\max}
\newcommand*{\normalorder}[1]{: #1 :}
\newcommand*{\pair}[1]{\langle #1 \rangle}
\newcommand*{\fd}[1]{\mathcal{D} #1}
\DeclareMathOperator{\bigO}{\mathcal{O}}

% TikZ setting
\usetikzlibrary{arrows,shapes,positioning}
\usetikzlibrary{calc}
\usetikzlibrary{arrows.meta}
\usetikzlibrary{decorations.markings}
\tikzstyle arrowstyle=[scale=1]
\tikzstyle directed=[postaction={decorate,decoration={markings,
    mark=at position .5 with {\arrow[arrowstyle]{stealth}}}}]
\tikzstyle ray=[directed, thick]
\tikzstyle dot=[anchor=base,fill,circle,inner sep=1pt]

% Algorithm setting
% Julia-style code
\SetKwIF{If}{ElseIf}{Else}{if}{}{elseif}{else}{end}
\SetKwFor{For}{for}{}{end}
\SetKwFor{While}{while}{}{end}
\SetKwProg{Function}{function}{}{end}
\SetArgSty{textnormal}

\newcommand*{\concept}[1]{{\textbf{#1}}}

% Embedded codes
\lstset{basicstyle=\ttfamily,
  showstringspaces=false,
  commentstyle=\color{gray},
  keywordstyle=\color{blue}
}

% Reference formatting
\newrefformat{fig}{Figure~\ref{#1} on page~\pageref{#1}}

% Color boxes
\tcbuselibrary{skins, breakable, theorems}
\newtcbtheorem[number within=section]{warning}{Warning}%
  {colback=orange!5,colframe=orange!65,fonttitle=\bfseries, breakable}{warn}
\newtcbtheorem[number within=section]{note}{Note}%
  {colback=green!5,colframe=green!65,fonttitle=\bfseries, breakable}{note}
\newtcbtheorem[number within=section]{info}{Info}%
  {colback=blue!5,colframe=blue!65,fonttitle=\bfseries, breakable}{info}

\title{Xucheng Wang on Superconducting Phase Fluctuations and the Pseudogap}
\author{Jinyuan Wu}
\date{March 12, 2022}

\begin{document}

\maketitle

High $T_\text{c}$ superconducting has been discovered for several decades, but still with fundamental puzzles like
\begin{itemize}
    \item Symmetry of order parameter: $s$-wave or $p$-wave?
    \item Mechanism of pairing
    \item The role of spin fluctuation.
\end{itemize}
The main topic today is \concept{pseudogap}. When there is no doping, the system is an AF insulator, and with some 
hole doping we have the superconducting phase, and with higher hole concentration, we get a metal state.
When the hole concentration is in the superconducting range but the temperature is risen, we will 
enter the \concept{pseudogap phase}, where there is no superconductivity but we still have remaining
pairing, with a ``pseudo'' superconducting gap. The region may come from AFM or CDW order competing with 
SC, and it is also possible that in this region electrons already form pairs, but the pairs do not \emph{condense}.
It is also theoretically predicted that pseudogap-like features can arise in BCS theory because of pairing of 
vortex of the phase $\theta$ of the order parameter. Recently, a pseudogap state was reported in strongly 
disordered conventional superconductors.

Several models, like the Hubbard model, have been confirmed to possess pseudogap-like behaviors. 
In this presentation we consider a BCS model:
\begin{equation}
    H = H_0 + V, V = \int \dd[2]{\vb*{r}} \Delta(\vb*{r}) \psi^\dagger_{\uparrow} \psi^\dagger_{\downarrow} + \text{h.c.},
\end{equation}
and $H_0$ is the Hamiltonian of free electrons. The mean field order parameter $\Delta$ is disordered, where 
we assume 
\begin{equation}
    \expval*{\Delta(\vb*{r})}_\text{dis} = \expval{\Delta^*(\vb*{r})}_\text{dis} = 0, \quad 
    \expval*{\Delta(\vb*{r}) \Delta^*(\vb*{r}')}_\text{dis} = g(\vb*{r} - \vb*{r}').
    \label{eq:disordered-delta}
\end{equation}

There are two types of disorder averaging. In \concept{quenched disorder}, the time scale of quenched disorder is 
much longer than the rest of the system, i.e. the disorder configuration is frozen when the rest of the 
system evolves, and the free energy is 
\begin{equation}
    F_\text{q} = - \frac{1}{\beta} \expval{\ln Z(\beta)}_\text{dis},
\end{equation}
which means we first calculate physical quantities and \emph{then} calculate the disorder average.
On the other hand, we have \concept{annealed disorder}, where the disorder configuration evolves \emph{together}
with the rest of the system, and we have 
\begin{equation}
    F_\text{a} = - \frac{1}{\beta} \ln \expval{Z(\beta)}_\text{dis}.
\end{equation}
In this setting, we first calculate the partition function after disorder averaging, and then give a 
definite physical quantity prediction.

We can then repeat the procedure as in Anderson localization and represent the disorder as a self-energy 
correction. Then the problem is feasible using existing numerical simulation methods. 
We can also make some approximation when the disorder is not strong. We can evaluate the imaginary part 
of the self energy explicitly and find its asymptotic behavior when the decoherence length $\xi$ 
introduced by disorder is small. After replacing the imaginary part of $\Sigma$ with a function easier 
to deal with, we can do inverse analytical continuation and get an approximate $\Sigma$.

We then will find expected split peak in the spectrum function. We find after introducing $\xi$, the peaks 
are broadened with 
\begin{equation}
    \Delta \omega = \pi \Delta_0 \frac{\xi}{\xi_\text{BCS}},
\end{equation}
where $\xi_{\text{BCS}}$ is the coherence length of the BCS system. 

We can then verify whether the above picture works in a 2D Hubbard model. In this model, at half-filling 
we have SC and CDW together, and the latter suppresses $T_\text{c}$ to zero, and when doped away from 
half-filling, CDW is suppressed and SC comes out. Indeed we find pseudogap in the spectrum function.

Note: here we are working in 2D and usually the phase transition is a KT phase transition, and as the 
temperature goes down, first we get electron pairing and then a KT phase transition, and the region 
between the two is the pseudogap phase. (??? why after KT phase transition we get SC?)

It should also be noted that the broadening comes from both the disordered $\Delta$ and something else,
because experimentally, when the temperature is very low, we still have broadened peaks in the spectrum 
function, which is not the case in \eqref{eq:disordered-delta}.

\end{document}