\documentclass[hyperref, a4paper]{article}

\usepackage{geometry}
\usepackage{float}
\usepackage{titling}
\usepackage{titlesec}
% No longer needed, since we will use enumitem package
% \usepackage{paralist}
\usepackage{enumitem}
\usepackage{footnote}
\usepackage{enumerate}
\usepackage{amsmath, amssymb, amsthm}
\usepackage{mathtools}
\usepackage{bbm}
\usepackage{cite}
\usepackage{graphicx}
\usepackage{subfigure}
\usepackage{physics}
\usepackage{tensor}
\usepackage{siunitx}
\usepackage{booktabs}
\usepackage[version=4]{mhchem}
\usepackage{tikz}
\usepackage{xcolor}
\usepackage{listings}
\usepackage{autobreak}
\usepackage[ruled, vlined, linesnumbered]{algorithm2e}
\usepackage{xr-hyper}
\usepackage[colorlinks,unicode]{hyperref} % , linkcolor=black, anchorcolor=black, citecolor=black, urlcolor=black, filecolor=black
\usepackage{prettyref}

% Page style
\geometry{left=3.18cm,right=3.18cm,top=2.54cm,bottom=2.54cm}
\titlespacing{\paragraph}{0pt}{1pt}{10pt}[20pt]
\setlength{\droptitle}{-5em}
\preauthor{\vspace{-10pt}\begin{center}}
\postauthor{\par\end{center}}

% More compact lists 
\setlist[itemize]{itemindent=17pt, leftmargin=1pt}

% Math operators
\DeclareMathOperator{\timeorder}{T}
\DeclareMathOperator{\diag}{diag}
\DeclareMathOperator{\legpoly}{P}
\DeclareMathOperator{\primevalue}{P}
\DeclareMathOperator{\sgn}{sgn}
\newcommand*{\ii}{\mathrm{i}}
\newcommand*{\ee}{\mathrm{e}}
\newcommand*{\const}{\mathrm{const}}
\newcommand*{\suchthat}{\quad \text{s.t.} \quad}
\newcommand*{\argmin}{\arg\min}
\newcommand*{\argmax}{\arg\max}
\newcommand*{\normalorder}[1]{: #1 :}
\newcommand*{\pair}[1]{\langle #1 \rangle}
\newcommand*{\fd}[1]{\mathcal{D} #1}
\DeclareMathOperator{\bigO}{\mathcal{O}}

% TikZ setting
\usetikzlibrary{arrows,shapes,positioning}
\usetikzlibrary{arrows.meta}
\usetikzlibrary{decorations.markings}
\tikzstyle arrowstyle=[scale=1]
\tikzstyle directed=[postaction={decorate,decoration={markings,
    mark=at position .5 with {\arrow[arrowstyle]{stealth}}}}]
\tikzstyle ray=[directed, thick]
\tikzstyle dot=[anchor=base,fill,circle,inner sep=1pt]

% Algorithm setting
% Julia-style code
\SetKwIF{If}{ElseIf}{Else}{if}{}{elseif}{else}{end}
\SetKwFor{For}{for}{}{end}
\SetKwFor{While}{while}{}{end}
\SetKwProg{Function}{function}{}{end}
\SetArgSty{textnormal}

\newcommand*{\concept}[1]{{\textbf{#1}}}

% Embedded codes
\lstset{basicstyle=\ttfamily,
  showstringspaces=false,
  commentstyle=\color{gray},
  keywordstyle=\color{blue}
}

\title{Monte Carlo by Yuanda Liao}
\author{Jinyuan Wu}

\begin{document}

\maketitle

This article is a note of Yuanda Liao's speech on Prof. Yang Qi's group meeting on October 16. 2021.

\begin{itemize}
    \item Basic principles of Monte Carlo simulations.
    \item Metropolis algorithm.
    \item Wolff algorithm and why it works; some details about how to expand a cluster.
    \item Discrete path integral of quantum Ising models.
\end{itemize}

\section{Cluster Update Algorithms}

Monte Carlo algorithms in physics are usually performed by \emph{importance sampling} and \emph{Markov chain}, i.e. we construct a Markov chain of which the stable probability of a configuration is 
\begin{equation}
    p(C) = \frac{1}{Z} \ee^{- F(C)}.
\end{equation}
The Markov chain should be ergodic, and detailed balance condition
\begin{equation}
    p(C) p(C \to C') = p(C') p(C' \to C).
\end{equation}

\begin{algorithm}

    \DontPrintSemicolon
    \SetAlgoLined
  
    \SetKwFunction{Kssolve}{kssolve}
    \Function(\tcc*[h]{comment of functions}){\Kssolve{i}}{
      \eIf{Condition}{Then}{else}
  
    \For(\tcc*[h]{for condition comment}){for condition}{
      Do sth \tcc*[r]{asdf}
    }
  
    \While{the condition}{
        do something \;
        $i = i + 1$ \;
        
    }
    
    \Return{return value}\;
    \Begin{Mad}
    }
    
    \caption{Wolff update for classical Ising model}
    \label{alg:basic-kohn-sham}
\end{algorithm}

\section{Discrete Path Integral of Quantum Ising Models}

The \concept{transverse field Ising model} is defined as 
\begin{equation}
    H = -J \sum_{\pair{\vb*{i}, \vb*{j}}} \sigma^z_{\vb*{i}} \sigma^z_{\vb*{j}} - h \sum_{\vb*{i}} \sigma^x_{\vb*{i}}.
\end{equation}
The (discrete) path integral in the $\sigma^z$ basis is 
\begin{equation}
    \begin{aligned}
        Z &= \trace \ee^{-\beta H} \\
        &= \sum_\tau \ee^{\Delta\tau J \sum_{\vb*{i}, \vb*{j}} \sigma^z_{\vb*{i}}(\tau) \sigma^z_{\vb*{j}}(\tau)} \mel{\sigma^z(\tau + \Delta\tau)}{\ee^{\Delta\tau h \sum_{\vb*{i}} \sigma^x_{\vb*{i}}}}{\sigma^z(\tau)}
    \end{aligned}
\end{equation}
With an error introduced by Trotter decomposition, we rephrase a problem in terms of operators into a problem in terms of path integral which only involves plus and multiplication of only ordinary numbers.

The temperature is introduced by the finite system effects in the temporal dimension. 
We usually fix $\Delta \tau$ and change the imaginary time steps, for several reasons:
\begin{itemize}
    \item The Trotter error depends on $\Delta \tau$. Changing $\Delta \tau$ during calculation means the Trotter error differs significantly at different choices of $(h, J)$.
    It is therefore very hard to systematically analyze the error.
    \item In finite size scaling, we should scale \emph{both} space and time, so changing the number of imaginary time steps is not correct if we want to do finite size scaling.  
\end{itemize}

\end{document}