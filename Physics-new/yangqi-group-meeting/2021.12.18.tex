\documentclass[hyperref, a4paper]{article}

\usepackage{geometry}
\usepackage{titling}
\usepackage{titlesec}
% No longer needed, since we will use enumitem package
% \usepackage{paralist}
\usepackage{enumitem}
\usepackage{footnote}
% Conflicts with enumitem
%\usepackage{enumerate}
\usepackage{amsmath, amssymb, amsthm}
\usepackage{mathtools}
\usepackage{bbm}
\usepackage{cite}
\usepackage{graphicx}
\usepackage{subfigure}
\usepackage{physics}
\usepackage{tensor}
\usepackage{siunitx}
\usepackage[version=4]{mhchem}
\usepackage{tikz}
\usepackage{xcolor}
\usepackage{listings}
\usepackage{autobreak}
\usepackage[ruled, vlined, linesnumbered]{algorithm2e}
\usepackage{nameref,zref-xr}
\zxrsetup{toltxlabel}
\zexternaldocument*[optics-]{../optics/optics}[optics.pdf]
\zexternaldocument*[solid-]{../solid/solid}[solid.pdf]
\usepackage[colorlinks,unicode]{hyperref} % , linkcolor=black, anchorcolor=black, citecolor=black, urlcolor=black, filecolor=black
\usepackage[most]{tcolorbox}
\usepackage{prettyref}

% Page style
\geometry{left=3.18cm,right=3.18cm,top=2.54cm,bottom=2.54cm}
\titlespacing{\paragraph}{0pt}{1pt}{10pt}[20pt]
\setlength{\droptitle}{-5em}
\preauthor{\vspace{-10pt}\begin{center}}
\postauthor{\par\end{center}}

% More compact lists 
\setlist[itemize]{
    itemindent=17pt, 
    leftmargin=1pt,
    listparindent=\parindent,
    parsep=0pt,
}

% Math operators
\DeclareMathOperator{\timeorder}{\mathcal{T}}
\DeclareMathOperator{\diag}{diag}
\DeclareMathOperator{\legpoly}{P}
\DeclareMathOperator{\primevalue}{P}
\DeclareMathOperator{\sgn}{sgn}
\newcommand*{\ii}{\mathrm{i}}
\newcommand*{\ee}{\mathrm{e}}
\newcommand*{\const}{\mathrm{const}}
\newcommand*{\suchthat}{\quad \text{s.t.} \quad}
\newcommand*{\argmin}{\arg\min}
\newcommand*{\argmax}{\arg\max}
\newcommand*{\normalorder}[1]{: #1 :}
\newcommand*{\pair}[1]{\langle #1 \rangle}
\newcommand*{\fd}[1]{\mathcal{D} #1}
\DeclareMathOperator{\bigO}{\mathcal{O}}

% TikZ setting
\usetikzlibrary{arrows,shapes,positioning}
\usetikzlibrary{calc}
\usetikzlibrary{arrows.meta}
\usetikzlibrary{decorations.markings}
\tikzstyle arrowstyle=[scale=1]
\tikzstyle directed=[postaction={decorate,decoration={markings,
    mark=at position .5 with {\arrow[arrowstyle]{stealth}}}}]
\tikzstyle ray=[directed, thick]
\tikzstyle dot=[anchor=base,fill,circle,inner sep=1pt]

% Algorithm setting
% Julia-style code
\SetKwIF{If}{ElseIf}{Else}{if}{}{elseif}{else}{end}
\SetKwFor{For}{for}{}{end}
\SetKwFor{While}{while}{}{end}
\SetKwProg{Function}{function}{}{end}
\SetArgSty{textnormal}

\newcommand*{\concept}[1]{{\textbf{#1}}}

% Embedded codes
\lstset{basicstyle=\ttfamily,
  showstringspaces=false,
  commentstyle=\color{gray},
  keywordstyle=\color{blue}
}

% Reference formatting
\newrefformat{fig}{Figure~\ref{#1} on page~\pageref{#1}}

% Color boxes
\tcbuselibrary{skins, breakable, theorems}
\newtcbtheorem[number within=section]{warning}{Warning}%
  {colback=orange!5,colframe=orange!65,fonttitle=\bfseries, breakable}{warn}
\newtcbtheorem[number within=section]{note}{Note}%
  {colback=green!5,colframe=green!65,fonttitle=\bfseries, breakable}{note}

\title{Entanglement spectrum calculation using quantum Monte Carlo by Dr. Zheng yan}
\author{Jinyuan Wu}

\begin{document}

\maketitle

The idea first came from PDE, i.e. can one hear the shape of a drum. Am. Math. Mon. 73, 1 (1966) 
It is natural to generalize the idea to the free energy. Nucl. Phys. B 300, 377, 1988
\begin{equation}
    F = f_\text{B} \abs*{A} + f_\text{S} L - \frac{1}{6} c \chi \ln L + \bigO(1).
\end{equation}
where $1/6$ is the conformal anomaly number, and $\chi$ is the Euler characteristic.
Finally we arrive the idea of the relation boundary and entanglement entropy: We have 
\begin{equation}
    S = F_A + F_B - F_{A \cup B},
\end{equation}
and we find 
\begin{equation}
    S = f_\text{S} L_{A \cap B} - \frac{1}{6} c 
\end{equation}
the last term being the topological entanglement entropy.

Now we discuss how to calculate entanglement entropy. The explicit definition of von Nuemann entanglement 
entropy 
\begin{equation}
    S_A = - \trace_A \rho_A \ln \rho_A, \quad \rho_A = \trace_B \rho_B,
\end{equation}
but it is hard to calculate. We usually calculate the generalized 
\begin{equation}
    S_A^{(n)} = \frac{1}{1 - n} \trace_A \rho_A \ln \rho_A^n,
\end{equation}
and the usual entanglement entropy is the $n \to 1$ limit.

\begin{equation}
    S_A^{(n)} = \frac{1}{1-n} \ln\frac{Z_A^{(n)}}{Z_\emptyset^{(n)}},
    \label{eq:s-n-partition}
\end{equation}
where $Z_A^{(n)}$ is the partition function of all possible field configuration obtained by gluing $n$ 
configuration of $A$ together, and $Z_\emptyset^{(n)}$ is simply multiplication of $n$ partition functions 
of $A$. 

Naive Monte Carlo calculation of \eqref{eq:s-n-partition} suffers from small probabilities. A trick to 
work around this is Phys. Rev. Lett. 124, 110602, 2020, Entanglement Entropy from Nonequilibrium work.
We have 
\begin{equation}
    S_A^{(n)} = \frac{1}{1-n} \int_0^1 \dd{\lambda} \pdv{\ln Z_A^{(n)}(\lambda)}{\lambda},
    \label{eq:s-n-integral}
\end{equation}
where 
\begin{equation}
    Z_A^{(n)}(\lambda) = \sum_{B \subseteq A} \lambda^{N_B} (1 - \lambda)^{N_A - N_B} Z_B^{(n)}, \quad Z_A^{(n)}(0) = Z_\emptyset^{(n)}, \quad Z_A^{(n)}(1) = Z_A^{(n)}.
\end{equation}
\begin{equation}
    \pdv{\ln Z_A^{(n)}(\lambda)}{\lambda} = 
\end{equation}
A straight forward approach is to run a Monte Carlo process for each $\lambda$ separately, but note that 
nonequilibrium statistic mechanics tells us 
% TODO
regardless of whether the system has reached equilibrium, we can slowly change $\lambda$ during \emph{one} 
Monte Carlo simulation, and therefore efficiently calculate \eqref{eq:s-n-integral}.

\end{document}
