\chapter{光学谐振腔}

前几章的电磁波传播都是没有任何边界条件约束的,此时如果介质均匀,那么电磁波可以以平面波的形式稳定传播。
本章则讨论束缚在有限区域内的电磁波模式。

原则上当然可以通过求解关于标势和矢势的方程来分析谐振腔的行为,例如解\eqref{eq:wave-eq}即可。
通常谐振腔内部没有源,于是\eqref{eq:wave-eq}中的各个方程都是齐次的,且各个矢势分量之间没有关系,从而似乎求解标量亥姆霍兹方程即可获知谐振腔的行为。
然而,边界条件实际上还是会让各个矢势分量和标势等都产生关系。
因此使用标势和矢势并不能简化问题。于是,本章还是直接求解\eqref{eq:e-in-tensor-material}。
特别的,各向同性体态中有
\begin{equation}
    \laplacian \vb*{E} + k_0^2 \vb*{E} = 0, \quad \div{\vb*{E}} = 0, \quad k_0 = \frac{\omega}{c},
    \label{eq:isotropic-cavity-problem-origin}
\end{equation}
这里$\vb*{E} = \vb*{E}(\vb*{r}, \omega)$。
通常我们会定义
\begin{equation}
    \tilde{\vb*{E}} = \sqrt{\epsilon} \vb*{E}(\vb*{r}, \omega), \quad \tilde{\vb*{B}} = \frac{\ii}{\sqrt{\mu}} \vb*{B}(\vb*{r}, \omega)
\end{equation}
来让$\vb*{E}$和$\vb*{B}$相差不要太大,此时
\begin{equation}
    \tilde{\vb*{B}} = \frac{1}{k_0} \curl{\tilde{\vb*{E}}}.
    \label{eq:tilde-b-tilde-e-cavity}
\end{equation}
将\eqref{eq:isotropic-cavity-problem-origin}用$\tilde{\vb*{E}}$和$\tilde{\vb*{B}}$写出来,就是
\begin{equation}
    \begin{bigcase}
        &\laplacian \tilde{\vb*{E}} + k_0^2 \tilde{\vb*{E}} = 0, \quad \laplacian \tilde{\vb*{B}} + k_0^2 \tilde{\vb*{B}} = 0, \\
        &\tilde{\vb*{B}} = \frac{1}{k_0} \curl{\tilde{\vb*{E}}}, \quad \tilde{\vb*{E}} = \frac{1}{k_0} \curl{\tilde{\vb*{B}}}.
    \end{bigcase}
    \label{eq:isotropic-cavity-problem}
\end{equation}
最后一个方程只需要在倒数第二个方程两边作用散度,通过一些计算即可看出。

\section{方形谐振腔}

\subsection{立方体谐振腔}

\subsection{方形波导}

\concept{波导}是指一个长条状的、电磁波可以在其中传递的装置。我们讨论一个柱状的波导,它是一个形状任意的闭合曲线沿着垂直于截面的方向平移而形成的直导管,其壁为导体,内部填充了某种均匀介质。
基本上,能够称为电磁波的电磁场构型都有很强的趋肤效应,因此接下来如无特殊说明我们认为波导的壁是理想导体,即认为导体内部没有任何场分布,即认为边界条件为
\begin{equation}
    \vb*{n} \cdot \vb*{B} = 0, \quad \vb*{n} \times \vb*{E} = 0.
\end{equation}
表面上,从边界条件$\vb*{n} \cdot (\vb*{D}_1 - \vb*{D}_2) = 0$出发,并利用导体内部没有电场分布这一条件,似乎可以得到$\vb*{n} \cdot \vb*{D}_1=0$,但这是错误的:如果我们要求$\vb*{n} \cdot (\vb*{D}_1 - \vb*{D}_2) = 0$成立,即将导体内部的电流归入$\epsilon$中,即给$\epsilon$一个虚部,那么随着导体电导率的上升,导体内部的$\vb*{E}$的确会下降,但是$\epsilon$会上升,最后在边界上会留下一个不为零的$\vb*{n} \cdot \vb*{D}_2$。
而如果我们不将电流归入$\epsilon$,那么就有$\vb*{n} \cdot (\vb*{D}_2 - \vb*{D}_1) = \sigma$,而$\vb*{D}_2$正常地衰减,于是边界条件就是$\vb*{n} \cdot \vb*{D}_1 = \sigma$。
我们并不知道$\sigma$到底是什么,因此这个边界条件实际上是用来在$\vb*{E}$已知后返回来求解$\sigma$的。
$\vb*{n} \times (\vb*{H}_2 - \vb*{H}_1) = \vb*{j}$同理。

考虑时谐场。由于$z$方向上的平移不变性,我们可以认为
\[
    \vb*{E}, \vb*{B} \propto \ee^{\ii (k_z z - \omega t)},
\]
在导管内部,波动方程为
\begin{equation}
    \left( \pdv[2]{x} + \pdv[2]{y} + k_0^2 - k_z^2 \right) \pmqty{\vb*{E} \\ \vb*{B}} = 0, \quad k_0 = \frac{\omega}{c}.
\end{equation}
以上方程并不能定解,但是实际上通过使用方程$\curl{\vb*{E}}=-\partial_t \vb*{B}$以及$\curl{\vb*{H}} = \partial_t \vb*{E}$,$x, y$方向上的场可以写成$z$方向上的场及其导数的线性函数,因此我们只需要求解
\begin{equation}
    \left( \pdv[2]{x} + \pdv[2]{y} + k_\text{c}^2 \right) \pmqty{E_z \\ B_z} = 0, \quad k_\text{c}^2 = k_0^2 - k_z^2.
\end{equation}
我们不能指望$E_z$和$B_z$都是零,因为此时没有非平庸解,即波导的约束意味着严格的横波是不可能的。
可能的偏振模式可以分成$B_z=0$的\concept{横磁波}(TM)和$E_z=0$的\concept{横电波}(TE)两类。

简单的计算表明对横电波我们有(本段中所有的$\grad$都是二维平面上的,我们暂时忽略电磁场在$z$方向上的周期性波动)
\begin{equation}
    \vb*{B}_\text{t} = \frac{\ii k_z}{k_\text{c}^2} \grad{B_z}, \quad \vb*{E}_\text{t} = - \ii \frac{c k_0}{k_\text{c}^2} \vb*{e}_z \times \grad{B_z},
\end{equation}
于是从$\vb*{n} \cdot \vb*{B} = 0$得到$B_z$满足的边界条件
\begin{equation}
    \pdv{B_z}{n} = 0,
\end{equation}
并且这个条件也能够让$\vb*{n} \times \vb*{E} = 0$成立,于是据此条件求解$B_z$满足的亥姆霍兹方程就确定了一切。对横磁波类似的有
\begin{equation}
    \vb*{E}_\text{t} = \frac{\ii k_z}{k_\text{c}^2} \grad{E_z}, \quad \vb*{B}_\text{t} = \ii \frac{k_0}{ck_\text{c}^2} \vb*{e}_z \times \grad{E_z},
\end{equation}
边界条件为
\begin{equation}
    E_z = 0.
\end{equation}
这个边界条件是$\vb*{n} \times \vb*{E}=0$的直接推论,但是由于它让$\grad{E_z}$在边界上一定沿着$\vb*{n}$,可以验证$\vb*{n} \cdot \vb*{B}=0$也是成立的。

在$xy$平面上求解可能的TE或是TM模式,得到的是离散谱,而电磁场在$z$方向的传播却是散射态,即$\omega$和$k_z$都可以连续取值,于是波导内的模式的能谱形如
\begin{equation}
    \omega = c \sqrt{k_z^2 + k_{\text{c}, mn}^2},
\end{equation}
其中$m, n$为标记$xy$平面上的模式的整数编号。可以看到这个能谱是有能隙的,能量低于
\begin{equation}
    \omega_\text{c} = \min (c k_{\text{c}, mn})
\end{equation}
的电磁波入射波导之后会快速衰减。

\section{导引矢量法}\label{sec:guiding-vector}

注意到问题\eqref{eq:isotropic-cavity-problem}中$\tilde{\vb*{E}}$和$\tilde{\vb*{B}}$的形式高度对称,我们可以尝试通过一个特殊的构造产生它的解。

由于\eqref{eq:isotropic-cavity-problem}中的电场和磁场均满足横波条件,它们总是可以写成某个东西的旋度。
设某个矢量场$\vb*{M}$是某个东西的旋度,即满足
\begin{equation}
    {\vb*{M}} = \curl{(\psi \vb*{c})},
    \label{eq:guiding-vector-construction}
\end{equation}
这样关于$\vb*{M}$的亥姆霍兹方程就变为
\begin{equation}
    0 = \laplacian {\vb*{M}} + k_0^2 {\vb*{M}} = \curl{((\laplacian \psi + k_0^2 \psi) \vb*{c} + \psi \laplacian \vb*{c})}.
    \label{eq:m-helmholtz-original}
\end{equation}
注意此处$\vb*{c}$和$\psi$的定义不唯一。我们总是能够找到(虽然一般都不容易解析地找到)一个$\vb*{c}$满足
\begin{equation}
    \div{\vb*{c}} = \text{const}, \quad \curl{\vb*{c}} = 0.
\end{equation}
这是因为,设
\[
    \vb*{M} = \curl{\vb*{M}'},
\]
我们注意到关于某个标量场$\lambda$的方程
\[
    \div{(\lambda \vb*{M}')} = \text{const}, \quad \curl{(\lambda \vb*{M}')} = 0
\]
一定有解,因为第二个方程的三个分量方程实际上只有两个独立。
因此,我们设
\[
    \vb*{M}' = \psi \vb*{c}, \quad \psi = \frac{1}{\lambda}
\]
即可得到\eqref{eq:guiding-vector-construction},即\eqref{eq:guiding-vector-construction}中的$\vb*{c}$和$\psi$总是可以构造出来的。
我们称$\vb*{c}$为\concept{导引矢量},因为它大体上描绘了$\vb*{M}$的“指向”。

对$\vb*{c}$我们有
\[
    \laplacian \vb*{c} = \grad(\div{\vb*{c}}) - \curl{(\curl{\vb*{c}})} = 0,
\]
于是
\begin{equation}
    \vb*{M} = \grad{\psi} \times \vb*{c}, \quad \vb*{M} \bot \vb*{c},
\end{equation}
关于$\vb*{M}$的亥姆霍兹方程\eqref{eq:m-helmholtz-original}等价于
\begin{equation}
    \laplacian \psi + k_0^2 \psi = 0.
    \label{eq:scalar-cavity-eq}
\end{equation}
在解出$\vb*{M}$之后我们会注意到矢量场
\begin{equation}
    \vb*{N} = \frac{1}{k_0} \curl{\vb*{M}}
    \label{eq:cavity-n-def}
\end{equation}
满足
\begin{equation}
    \vb*{M} = \frac{1}{k_0} \curl{\vb*{N}}, 
\end{equation}
并且它也满足和$\vb*{M}$满足的亥姆霍兹方程完全一样的方程
\begin{equation}
    \laplacian \vb*{N} + k_0^2 \vb*{N} = 0.
\end{equation}

对比$\vb*{M}$和$\vb*{N}$矢量场满足的各个方程和\eqref{eq:isotropic-cavity-problem},我们发现可以将$\vb*{M}$看成$\tilde{\vb*{E}}$,将$\vb*{N}$看成$\tilde{\vb*{B}}$,也可以反过来将$\vb*{M}$看成$\tilde{\vb*{B}}$,将$\vb*{N}$看成$\tilde{\vb*{E}}$。
因此我们得到了一种原则上一般的求解光学谐振腔中的模式的方法:求解标量方程\eqref{eq:scalar-cavity-eq},然后将各个模式代入\eqref{eq:guiding-vector-construction}和\eqref{eq:cavity-n-def},这样就得到了全部的电磁波模式。

\subsection{平凡的例子:平面波}

平面波的经验给出了一种挑选$\vb*{c}$的方法:尽可能让$\vb*{c}$垂直于主要的界面方向。

\subsection{圆柱波导中的柱面波}

在圆柱波导中可以验证将$\vb*{c}$选择为
\begin{equation}
    \vb*{c} = \vb*{e}_\rho
\end{equation}
是可行的,此时需要求解
\begin{equation}
    \frac{1}{\rho} \pdv{\rho} \left( \rho \pdv{\psi}{\rho} \right) + \frac{1}{\rho^2} \pdv[2]{\psi}{\phi} + \pdv[2]{\psi}{z} + k_0^2 \psi = 0.
\end{equation}
这个方程的求解是已知的:它最终转化为柱贝塞尔方程的求解。


我们分析几种极限情况。$k \rho \to 0$的情况对应于在我们关心的距离尺度内电磁波的传播速度可以忽略的情况(或者$\rho$很小,或者$c$很大以至于$k$很小),而$x \to 0$时 % TODO:贝塞尔函数的渐近行为
此时的解就是静电势的通解。

圆柱波导中的电磁波模式本质上还是标量的。

\subsection{球面波}

选取
\begin{equation}
    \vb*{c} = r \vb*{e}_r.
\end{equation}
关于$\psi$的亥姆霍兹方程为
\begin{equation}
    \frac{1}{r^2} \pdv{r}\left( r^2 \pdv{\psi}{r} \right) + \frac{1}{r^2 \sin \theta} \pdv{\theta} \left( \sin \theta \pdv{\psi}{\theta} \right) + \frac{1}{r^2 \sin \theta} \pdv[2]{\psi}{\phi} + k_0^2 \psi = 0.
\end{equation}

以上求解过程说明三维球腔中不存在s波。
从数学上看这来自所谓\emph{毛球定理}:$S^1$上可以有一个连续而处处不为零的切向量场,但是$S^2$上不可能有这样的切向量场。

我们将球面波推导出的$\vb*{N}$和$\vb*{M}$称为\concept{球波函数}。
球波函数显然可以用于做多极矩展开。实际上,它比我们前面通过泰勒级数得到的多极矩展开更加优越,因为后者只在远场情况下能够毫无疑难地定义,在近场时会有一定的模糊性。
一个重要的例子就是\concept{环形磁偶极矩}。

\chapter{非均匀折射率导致的散射}