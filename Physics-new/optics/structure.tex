\chapter{光学谐振腔}

前几章的电磁波传播都是没有任何边界条件约束的,此时如果介质均匀,那么电磁波可以以平面波的形式稳定传播。
本章则讨论束缚在有限区域内的电磁波模式。

\section{波导}

\concept{波导}是指一个长条状的、电磁波可以在其中传递的装置。我们讨论一个柱状的波导,它是一个形状任意的闭合曲线沿着垂直于截面的方向平移而形成的直导管,其壁为导体,内部填充了某种均匀介质。
基本上,能够称为电磁波的电磁场构型都有很强的趋肤效应,因此接下来如无特殊说明我们认为波导的壁是理想导体,即认为导体内部没有任何场分布,即认为边界条件为
\begin{equation}
    \vb*{n} \cdot \vb*{B} = 0, \quad \vb*{n} \times \vb*{E} = 0.
\end{equation}
表面上,从边界条件$\vb*{n} \cdot (\vb*{D}_1 - \vb*{D}_2) = 0$出发,并利用导体内部没有电场分布这一条件,似乎可以得到$\vb*{n} \cdot \vb*{D}_1=0$,但这是错误的:如果我们要求$\vb*{n} \cdot (\vb*{D}_1 - \vb*{D}_2) = 0$成立,即将导体内部的电流归入$\epsilon$中,即给$\epsilon$一个虚部,那么随着导体电导率的上升,导体内部的$\vb*{E}$的确会下降,但是$\epsilon$会上升,最后在边界上会留下一个不为零的$\vb*{n} \cdot \vb*{D}_2$。
而如果我们不将电流归入$\epsilon$,那么就有$\vb*{n} \cdot (\vb*{D}_2 - \vb*{D}_1) = \sigma$,而$\vb*{D}_2$正常地衰减,于是边界条件就是$\vb*{n} \cdot \vb*{D}_1 = \sigma$。
我们并不知道$\sigma$到底是什么,因此这个边界条件实际上是用来在$\vb*{E}$已知后返回来求解$\sigma$的。
$\vb*{n} \times (\vb*{H}_2 - \vb*{H}_1) = \vb*{j}$同理。

考虑时谐场。由于$z$方向上的平移不变性,我们可以认为
\[
    \vb*{E}, \vb*{B} \propto \ee^{\ii (k_z z - \omega t)},
\]
在导管内部,波动方程为
\begin{equation}
    \left( \pdv[2]{x} + \pdv[2]{y} + k_0^2 - k_z^2 \right) \pmqty{\vb*{E} \\ \vb*{B}} = 0, \quad k_0 = \frac{\omega}{c}.
\end{equation}
以上方程并不能定解,但是实际上通过使用方程$\curl{\vb*{E}}=-\partial_t \vb*{B}$以及$\curl{\vb*{H}} = \partial_t \vb*{E}$,$x, y$方向上的场可以写成$z$方向上的场及其导数的线性函数,因此我们只需要求解
\begin{equation}
    \left( \pdv[2]{x} + \pdv[2]{y} + k_\text{c}^2 \right) \pmqty{E_z \\ B_z} = 0, \quad k_\text{c}^2 = k_0^2 - k_z^2.
\end{equation}
我们不能指望$E_z$和$B_z$都是零,因为此时没有非平庸解,即波导的约束意味着严格的横波是不可能的。
可能的偏振模式可以分成$B_z=0$的\concept{横磁波}(TM)和$E_z=0$的\concept{横电波}(TE)两类。

简单的计算表明对横电波我们有(本段中所有的$\grad$都是二维平面上的,我们暂时忽略电磁场在$z$方向上的周期性波动)
\begin{equation}
    \vb*{B}_\text{t} = \frac{\ii k_z}{k_\text{c}^2} \grad{B_z}, \quad \vb*{E}_\text{t} = - \ii \frac{c k_0}{k_\text{c}^2} \vb*{e}_z \times \grad{B_z},
\end{equation}
于是从$\vb*{n} \cdot \vb*{B} = 0$得到$B_z$满足的边界条件
\begin{equation}
    \pdv{B_z}{n} = 0,
\end{equation}
并且这个条件也能够让$\vb*{n} \times \vb*{E} = 0$成立,于是据此条件求解$B_z$满足的亥姆霍兹方程就确定了一切。对横磁波类似的有
\begin{equation}
    \vb*{E}_\text{t} = \frac{\ii k_z}{k_\text{c}^2} \grad{E_z}, \quad \vb*{B}_\text{t} = \ii \frac{k_0}{ck_\text{c}^2} \vb*{e}_z \times \grad{E_z},
\end{equation}
边界条件为
\begin{equation}
    E_z = 0.
\end{equation}
这个边界条件是$\vb*{n} \times \vb*{E}=0$的直接推论,但是由于它让$\grad{E_z}$在边界上一定沿着$\vb*{n}$,可以验证$\vb*{n} \cdot \vb*{B}=0$也是成立的。

在$xy$平面上求解可能的TE或是TM模式,得到的是离散谱,而电磁场在$z$方向的传播却是散射态,即$\omega$和$k_z$都可以连续取值,于是波导内的模式的能谱形如
\begin{equation}
    \omega = c \sqrt{k_z^2 + k_{\text{c}, mn}^2},
\end{equation}
其中$m, n$为标记$xy$平面上的模式的整数编号。可以看到这个能谱是有能隙的,能量低于
\begin{equation}
    \omega_\text{c} = \min (c k_{\text{c}, mn})
\end{equation}
的电磁波入射波导之后会快速衰减。

\section{立方体谐振腔}

\section{球形谐振腔}

\begin{equation}
    \tilde{\vb*{E}} = \sqrt{\epsilon} \vb*{E}(\vb*{r}, \omega), \quad \tilde{\vb*{B}} = \frac{\ii}{\sqrt{\mu}} \vb*{B}(\vb*{r}, \omega).
\end{equation}

\chapter{非均匀折射率导致的散射}