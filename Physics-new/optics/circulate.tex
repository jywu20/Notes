\chapter{似稳场和电路}

\section{基本方程}

\subsection{似稳条件}

本节开始我们讨论随时间发生变化的系统。
介质中的麦克斯韦方程看起来是时间平移不变的,当然,本构关系可以显含时间,因此它也许并不真的是时间平移不变的,但这种情况非常少见。
为方便起见我们经常求解\concept{时谐场},即假定$\vb*{E} \propto \ee^{- \ii \omega t}$,这相当于将场的时间部分切换到频域。
傅里叶变换意味着这当然不会丢失任何一般性。

当然,完整地解麦克斯韦方程是最精确的,但是很多情况下我们发现这类系统并没有特别明显的电磁辐射。
只要电场生磁场、磁场生电场,就可以有电磁辐射,因此电磁辐射不明显的系统中要么基本上没有电场产生的磁场,要么没有磁场产生的电场。 % TODO:,要么两者都有但是可以在感生电场和感生磁场之间建立直接的关系从而简化
这就是\concept{似稳场}或者\concept{准静态场}。体系中的电流被束缚在一些体积相对于电磁波波长不大的导体中的情况,也\concept{电路},经常可以用似稳场处理。
本节将主要讨论没有电场产生的磁场的情况,即忽略了位移电流的情况。
如果特殊需求,假定系统中的各个本构关系都是线性的。
此时麦克斯韦方程为
\begin{equation}
    \begin{bigcase}
        \div{\vb*{D}} &= \rho, \\
        \curl{\vb*{E}} &= - \pdv{\vb*{B}}{t}, \\
        \div{\vb*{B}} &= 0, \\
        \curl{\vb*{H}} &= \vb*{j}.
    \end{bigcase}
    \label{eq:quasi-stable-field}
\end{equation}

何时能够使用似稳场近似?对良导体,位移电流肯定要充分小,即
\[
    \pdv*{\vb*{D}}{t} \ll \sigma \vb*{E},
\]
即
\begin{equation}
    \omega \ll \omega_\sigma = \frac{\sigma}{\epsilon}.
    \label{eq:quasi-stable-field-cond-1}
\end{equation}
表面上看有这个条件就够了,但实际上这里有一个微妙的地方。在电场频率很大时,很多材料中电子在外场作用下不断“折返跑”,不会有宏观上的定向移动。此时的电流更像是束缚电流而不是传导电流,有
\[
    \curl{\vb*{H}} = \vb*{j} \sim \pdv{\vb*{E}}{t},
\]
从而又有了位移电流,因此电场实际的行为更像电磁波而不是似稳场。在频域上看,在$\omega$增大时,$\sigma$会有较大的、随着$\omega$变化的虚部,从而\eqref{eq:quasi-stable-field}的解和我们马上要看到的场的扩散方程(在其中$\sigma$就是一个常数)非常不同。
因此如果我们将直流电阻代入\eqref{eq:quasi-stable-field-cond-1}那这个判据太弱了。

对绝缘体,即在导电区域以外,显然只有$\omega=0$即完全静态的情况下才有\eqref{eq:quasi-stable-field-cond-1}成立。
然而,这仅仅意味着我们没有全空间的似稳场近似,并不意味着在系统的空间尺度较小时没有似稳场近似。
在绝缘体条件下,以$\epsilon$和$\mu$代替真空中的李纳-维谢尔势中的$\epsilon_0$和$\mu_0$,于是
\[
    \begin{aligned}
        \vb*{B}(\vb*{r}, t) &\approx \frac{\mu}{4\pi} \int \dd[3]{\vb*{r}'} \frac{\vb*{j}(\vb*{r}', t - R / c) \times \vb*{R}}{R^3} \\
        &= \frac{\mu}{4\pi} \int \dd[3]{\vb*{r}'} \frac{\vb*{j}(\vb*{r}', t) \ee^{- \ii \omega (t - R / c)} \times \vb*{R}}{R^3},
    \end{aligned}
\]
如果某一点的电流变化要瞬间传递到系统的各处,应有
\begin{equation}
    R \ll \frac{c}{\omega}.
\end{equation}
这当然是非常合理的:扰动基本上以光速传递,因此如果系统足够小,那么系统内一点的扰动总是可以快速传遍整个系统。

如果一个系统能够用似稳场分析,这通常意味着我们可以把系统看成某种电路:磁场可以写成电流的函数,电场可以分成两部分,一部分是磁场的变化率的函数,一部分是电荷的函数,而电流又正比于总电场,因此可以写出一个类似于“电流乘以电导率=由磁场变化导致的电场+电荷导致的电场+外加电场”这样的方程,这正好是电路的方程,其中考虑了四种效应:电阻、电感、电容、电源。
反之,则需要将系统看成某种传导电磁波的介质。

\subsection{场的扩散方程}

% TODO:似乎涉及电荷的重新分布等问题的情况不能用似稳场近似,因为在似稳场情况下电流散度为零

在系统中各处的本构关系都是空间均匀的情况下,经过大约为$\epsilon/\sigma$量级的时间,电荷密度为零(请注意这个结论和是否有外加场、外加场是否变化无关),因此大部分时候我们只需要求解
\[
    \begin{bigcase}
        \div{\vb*{D}} &= 0, \\
        \curl{\vb*{E}} &= - \mu \pdv{\vb*{H}}{t}, \\
        \div{\vb*{H}} &= 0, \\
        \curl{\vb*{H}} &= \sigma \vb*{E},
    \end{bigcase}
\]
从而
\begin{equation}
    \laplacian{\vb*{E}} = \mu \sigma \pdv{\vb*{E}}{t}, \quad \laplacian{\vb*{H}} = \mu \sigma \pdv{\vb*{H}}{t}.
\end{equation}
这就是\concept{场的扩散方程}。可以发现,对良好的导体,场的扩散反而是非常慢的,这是正确的,因为静电场中导体内部不应该有电场,因此在似稳场下导体内部的电场应该很弱,正好说明场的扩散很差。

在频域下,我们有
\begin{equation}
    \laplacian{\vb*{E}} = - \ii \omega \mu \sigma \vb*{E}, \quad \laplacian{\vb*{H}} = - \ii \omega \mu \sigma \vb*{H}.
    \label{eq:semi-stable-omega}
\end{equation}
如果$\sigma$有很大的虚部,以上方程的行为看起来就更像亥姆霍兹方程,从而在时域给出传递的波动。
在$\omega$很大时$\sigma$通常会有很大的虚部,因此此时似稳场不适用。

在似稳场确实适用的情况下,导体内部基本上没有场强分布,即出现\concept{趋肤效应}。
这可以通过在导体表面求解\eqref{eq:semi-stable-omega}看出。在一个无穷大平面边界上,设
\begin{equation}
    \vb*{E} = \vb*{E}_0 \ee^{-\alpha z},
    \label{eq:damping-surface-field}
\end{equation}
% TODO
趋肤深度为
\begin{equation}
    \delta = \sqrt{\frac{2}{\mu \omega \sigma}}.
\end{equation}
对理想导体,$\sigma \to \infty$,因此任何频率下电场都不会进入导体内部。
对实际导体,频率越高,趋肤效应越明显,但是当$\omega$继续增大以至于似稳场不再适用时,趋肤效应就消失了,此时的导体是透明的。

这里有一个看起来的佯谬:对有限大小的电导率,$\omega$很小时似乎有$\delta \to \infty$,也即,静电场可以直接穿透导体!
但是应当注意到一点:似稳场近似不仅包括了静电场的那些模式,也包括了\concept{恒定电场}即导体有稳定的、不随时间变化的电流输入的那些模式。
后者的确不存在任何趋肤效应:很容易验证,$\vb*{E}$在导体内处处均匀分布,指向同一个方向的模式是存在的,这里没有任何趋肤效应。
实际上从麦克斯韦方程可以看出,要产生场的扩散方程,感生电场是必须的(注意$\mu$出现在了场的扩散方程中),因此趋肤效应实际上是因为导体内部感应出的涡旋电场抵消了外加电场。这个机制和静电屏蔽是不一样的。

静电屏蔽实际上来自边界条件。设电场从绝缘体被打到导体上,用1标记绝缘体,2标记导体,有
\[
    \vb*{n} \cdot (\vb*{D}_2 - \vb*{D}_1) = \sigma, \quad \pdv{\rho}{t} = \vb*{n} \cdot \vb*{j},
\]
在频域下就有
\[
    \vb*{n} \cdot (\vb*{D}_2 - \vb*{D}_1) = \sigma, \quad - \ii \omega \sigma = \vb*{n} \cdot \vb*{j},
\]
于是
\[
    \vb*{n} \cdot \vb*{E}_2 = \frac{\vb*{n} \cdot \vb*{D}_1}{\epsilon-0 + \frac{\sigma}{\ii \omega}},
\]
在$\omega \to 0$时$\vb*{n} \cdot \vb*{E}_2$趋于零,即使趋肤深度很大,电场也不能进入导体。因此与静电屏蔽相关的电场衰减的尺度是微观尺度上的:在导体和绝缘体的交接层上电场已经衰减了;另一方面,趋肤效应的尺度虽然很小,但仍然是宏观的,它发生在导体体块内部,而不是导体和绝缘体的交接层上。
我们在这里只讨论了垂直于表面的电场,因为平行于表面的电场不会出现在静电场中,但是可以出现在恒定电场中;如果导体表面附近有平行于表面的电场,那么由$\vb*{n} \times (\vb*{E}_2 - \vb*{E}_1)=0$导体内部肯定也有平行于表面的电场,因此会有电流,就和“静电场”的条件冲突了。
总之,无论外界的静电场如何分布,导体内部根本不会有静电场,因此导体内部$\omega=0$的电场全部都是恒定电场。

还有另一个可能造成疑难的地方:我们知道导体可以远距离传输电流,从而必然可以远距离传输电场,但是上面的论证似乎是说,导体中的电场一定会快速衰减!
我们来分析一下远距离传输的电场可能出现在哪里,也即,持续的、远距离的稳定电流可能出现在哪里。
趋肤效应的推导是非常一般的,因此这样的电流只能出现在导体边界附近,并且在没有外加电流流入的地方一定平行于导体边界。
这些电流一定最后会撞上另一个导体边界,因为电流不可能在缺乏非静电力的导体内部环流。
因此后一个导体边界的边界条件中,电流主要分布在这个边界的边缘上,如对柱状导体,主要分布在柱子的上下底面的棱上。
这样的场构型让\eqref{eq:damping-surface-field}在这些表面上失效,因为此时电磁场在$z$方向和$x, y$方向上都有很大变化。
在静电学中只有静电力,我们不会产生源源不断的电流,也不会将这样的电流从某处导入导体中,因此从来不会激发电流——从而电场——远距离传输的模式。
在导体和绝缘体交界的界面上根本无所谓电流输入,同样不会激发电流和电场远距离传播的模式;在导体和绝缘体交界的界面上,边界条件\eqref{eq:damping-surface-field}是很好的近似,从而在这些界面附近总是能够看到趋肤效应。