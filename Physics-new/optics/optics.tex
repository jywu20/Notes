\documentclass[UTF8, a4paper]{ctexbook}

\usepackage{geometry}
\usepackage{titling}
\usepackage{titlesec}
\usepackage{paralist}
\usepackage{footnote}
\usepackage{enumerate}
\usepackage{amsmath, amssymb, amsthm}
\usepackage{mathtools}
\usepackage{cite}
\usepackage{graphicx}
\usepackage{subfigure}
\usepackage{physics}
\usepackage{siunitx}
\usepackage{tikz-feynhand}
\usepackage{xr-hyper}
\usepackage[colorlinks, linkcolor=black, anchorcolor=black, citecolor=black, filecolor=black]{hyperref}
\usepackage{prettyref}

\externaldocument[qft-]{../relativistic-qft/relativistic-qft}[relativistic-qft.pdf]
\externaldocument[solid-]{../../solid/solid}[solid.pdf]

\geometry{left=3.28cm,right=3.28cm,top=2.54cm,bottom=2.54cm}
\titlespacing{\paragraph}{0pt}{1pt}{10pt}[20pt]
\setlength{\droptitle}{-5em}
\preauthor{\vspace{-10pt}\begin{center}}
\postauthor{\par\end{center}}

\newcommand*{\ee}{\mathrm{e}}
\newcommand*{\ii}{\mathrm{i}}
\newcommand*{\const}{\mathrm{const}}
\newcommand*{\natnums}{\mathbb{N}}
\newcommand*{\reals}{\mathbb{R}}
\newcommand*{\complexes}{\mathbb{C}}
\DeclareMathOperator{\timeorder}{T}
\newcommand*{\ogroup}[1]{\mathrm{O}(#1)}
\newcommand*{\sogroup}[1]{\mathrm{SO}(#1)}
\DeclareMathOperator{\legpoly}{P}

\newrefformat{sec}{第\ref{#1}节}
\newrefformat{note}{注\ref{#1}}
\newrefformat{fig}{图\ref{#1}}
\newrefformat{part}{第\ref{#1}部分}
\renewcommand{\autoref}{\prettyref}

\newcommand{\concept}[1]{\underline{#1}}
\renewcommand{\emph}{\textbf}

\sisetup{parse-numbers = false}

\usetikzlibrary{arrows,shapes,positioning}
\usetikzlibrary{arrows.meta}
\usetikzlibrary{decorations.markings}
\tikzstyle arrowstyle=[scale=1]
\tikzstyle directed=[postaction={decorate,decoration={markings,
    mark=at position .5 with {\arrow[arrowstyle]{stealth}}}}]
\tikzstyle ray=[directed, thick]

\tikzfeynhandset{
    every boldphoton@@/.style={
    /tikz/draw=none,
    /tikz/postaction={
            /tikz/draw,
            /tikz/double,
            /tikz/line width = \feynhandlinesize,
            /tikz/decoration={
                complete sines,
                amplitude=3\feynhandlinesize,
                segment length=7.5\feynhandlinesize,
            },
            /tikz/decorate=true,
        },
    },
    every boldphoton/.style={/tikzfeynhand/every boldphoton@@/.append style={#1}},
    boldphoton/.style={
    /tikzfeynhand/every boldphoton@@,
    },
}

\tikzfeynhandset{
    every extphoton@@/.style={
        /tikz/draw=none,
        /tikz/decoration={name=none},
        /tikz/postaction={
          /tikz/draw,
          /tikz/line width = \feynhandlinesize,
          /tikzfeynhand/with arrow=0.9,
        }
    },
    every extphoton/.style={/tikzfeynhand/every extphoton@@/.append style={#1}},
    extphoton/.style={
    /tikzfeynhand/every extphoton@@,
    }
}

\tikzfeynhandset{
    every outphoton@@/.style={
    /tikz/draw=none,
    /tikzfeynhand/with arrow=0.9,
    /tikz/postaction={
            /tikz/draw,
            /tikz/line width = \feynhandlinesize,
            /tikz/decoration={
                complete sines,
                amplitude=3\feynhandlinesize,
                segment length=7.5\feynhandlinesize,
            },
            /tikz/decorate=true,
        },
    },
    every outphoton/.style={/tikzfeynhand/every outphoton@@/.append style={#1}},
    outphoton/.style={
    /tikzfeynhand/every outphoton@@,
    },
}

\newenvironment{bigcase}{\left\{\quad\begin{aligned}}{\end{aligned}\right.}

\numberwithin{equation}{chapter}

\newcommand{\qftdoc}{\href{../relativistic-qft/relativistic-qft.pdf}{相对论性量子场论笔记}}
\newcommand{\soliddoc}{\href{../solid/solid.pdf}{固体物理笔记}}

\title{光学}
\author{吴晋渊}

\begin{document}

\maketitle

\part{电动力学的基本原理}

\chapter{QED的低能极限}



\section{非相对论性粒子和光场的耦合}

\subsection{非相对论性粒子的哈密顿量}

考虑与电磁场发生相互作用的粒子,我们通常将这些粒子称为物质而将电磁场称为光场或是辐射,虽然严格说起来辐射也算是一种物质。
我们假定粒子做低速运动,从而不需要使用相对论性的理论描述粒子。
粒子轨道部分的哈密顿量是以下保证局部$U(1)$规范对称性的极小耦合:
\begin{equation}
    {H}_\text{orbit} = \frac{1}{2m} ({\vb*{p}} - q \vb*{A})^2 + q \phi,
    \label{eq:minimal-coupling}
\end{equation}
自旋-磁场相互作用还会引入以下哈密顿量:
\begin{equation}
    {H}_\text{spin} = - \frac{q}{m} {\vb*{S}} \cdot \vb*{B} = - \vb*{\mu} \cdot \vb*{B},
\end{equation}
而场的哈密顿量是
\begin{equation}
    {H}_\text{field} = \frac{\epsilon_0}{2} \int \dd[3]{\vb*{r}} (\vb*{E}^2 + c^2 \vb*{B}^2),
\end{equation}
则体系的总哈密顿量
\begin{equation}
    {H} = \sum_i \left( \frac{1}{2m_i} ({\vb*{p}_i} - q_i \vb*{A})^2 + q_i \varphi - \frac{q_i}{m_i} \vb*{S}_i \cdot \vb*{B} \right) + {H}_\text{field} + {H}_\text{int} + {H}_\text{ext},
\end{equation}
其中${H}_\text{int}$和${H}_\text{ext}$分别表示粒子间相互作用和外加势场。
粒子部分——包括轨道和自旋——的拉氏量也可以写成
\begin{equation}
    L = \sum_i \left( \frac{1}{2} m_i \vb*{v}_i^2 - q_i \varphi + q_i \vb*{v}_i \cdot \vb*{A} + \vb*{\mu}_i \cdot \vb*{B} \right).
\end{equation}
具体什么是粒子-粒子相互作用其实有一定人为因素,比如说凝聚态场论中默认电子之间的相互作用是库伦相互作用,但是库伦相互作用其实也是交换光子导致的,实际上是近场辐射的一个无时间延迟近似。
同样,“外加势场”也有人为因素。
不过,由于本文将要讨论光学,实际上可以以一种比较前后一致的方式确定哪些电磁场模式被粒子-粒子等效相互作用替代,哪些被纳入考虑。
我们总是可以将电磁波模式分解成无源有旋的和无旋有源的。通过简单的QED计算可以发现,全体电磁波模式造成的粒子间散射几乎压倒性地来自一个电子发射、一个电子接受的$\varphi$模式,切换到电场中基本上就是库伦场,这是有源无旋的;另一方面,介质中的电磁波宏观地看都满足横波条件$\div{\vb*{E}}=0$。%
\footnote{
    我们称它为横波条件是因为在无穷大空间中这等价于$\vb*{k} \cdot \vb*{E} = 0$,但是这\emph{并不}意味着任何能够称为波矢的$\vb*{k}$都满足$\vb*{k} \cdot \vb*{E} = 0$。
    波导就是一个典型的例子。
}%
因此我们可以只在$\varphi$和$\vb*{A}$中保留满足横波条件、看起来就像真空中电磁波的电磁波模式,这些模式本身就不易被积掉;剩下的不满足横波条件,同时的确很容易积掉的模式——如库伦场——就归入粒子-粒子等效相互作用。
至于外加势场,它或者就是库伦场,或者是外加磁场,后者同样是一个容易被积掉,并且和电磁波非常不相似的模式。

总之,在量子光学问题中,\eqref{eq:couple-ham}原则上给出了所有值得关注的信息,并且我们可以取横波条件$\div{\vb*{E}}=0$。
这可以让我们施加一个比一般的情况更加严格的规范。根据$\div{\vb*{E}}=0$我们有
\[
    \laplacian{\varphi} + \pdv{t} \div{\vb*{A}} = 0,
\]
此时我们没有加入任何限制。我们总是可以取$\varphi=0$,此时
\[
    \pdv{t} \div{\vb*{A}} = 0,
\]
即$\div{\vb*{A}}$是不随时间变化的。那么,总是可以找到一个不随着时间变化的标量场$\chi$,使得
\[
    \div{(\vb*{A} + \grad{\chi})} = 0,
\]
因为这个条件等价于调和方程
\[
    \laplacian{\chi} = - \div{\vb*{A}}.
\]
因此,我们可以做规范变换
\[
    \varphi' = \varphi - \pdv{\chi}{t} = \varphi, \quad \vb*{A}' = \vb*{A} + \grad{\chi},
\]
变换后就有$\varphi=0$和$\div{\vb*{A}}=0$同时成立。
因此,在量子光学中,我们可以同时施加以下两个规范:
\begin{equation}
    \varphi = 0, \quad \div{\vb*{A}} = 0,
\end{equation}
而不用担心产生冲突。这就是\concept{辐射规范}。辐射规范下$\div{\vb*{A}}=0$这一条件保证了$\vb*{p}$和$\vb*{A}$是可交换的。

\eqref{eq:minimal-coupling}中的$\vb*{p}$是正则动量,而不是机械动量。
然而,这反倒有好处:我们要讨论的是“向一个物理系统入射光会得到怎样的出射光”,根本不需要去测量机械动量。
这种情况下我们完全没有必要关注$\vb*{p}$是正则动量这回事:完全可以打开括号$(\vb*{p} - q \vb*{A})^2$,然后求解束缚态问题
\begin{equation}
    H = \sum_i \frac{\vb*{p}_i^2}{2m_i} + H_\text{ext} + H_\text{int},
    \label{eq:levels-ham}
\end{equation}
具体求解时可以直接援引将$\vb*{p}$当成机械动量而得到的现成的解,得到能谱之后引入电子-电磁波耦合项
\begin{equation}
    H_\text{couple} = q \varphi - \frac{q}{2m} (\vb*{p} \cdot \vb*{A} + \vb*{A} \cdot \vb*{p}) + \frac{q^2}{2m} \vb*{A}^2,
    \label{eq:couple-ham}
\end{equation}
计算物质和光场的耦合。($\vb*{A}^2$项中含有粒子的位置,因此也是耦合项)

\subsection{束缚态系统,微扰论和多极矩展开}

将\eqref{eq:couple-ham}当成微扰做微扰论的适用条件是$H_\text{couple}$相对于\eqref{eq:levels-ham}来说很小。
如果微扰论适用,那么显然$q \vb*{A} \ll \vb*{p}$,从而$\vb*{A}^2$项相较于$\vb*{p} \cdot \vb*{A}$项总是非常小的。%
\footnote{
    一个可以抬杠的地方是$\vb*{p}$很小时,似乎$\vb*{p} \cdot \vb*{A}$项远小于$\vb*{A}^2$项。
    然而,由能量守恒,$\vb*{A}^2$项相比于动能加上势能的\eqref{eq:levels-ham}总是很小的。
    如果我们只要求$\vb*{A}^2$级别的精度,那么在$\vb*{p}$大时显然$\vb*{p} \cdot \vb*{A}$项比$\vb*{A}^2$项重要,而$\vb*{p}$小时$\vb*{A}^2$项小于我们的精度要求。
    无论如何,$\vb*{A}^2$项都不如$\vb*{p} \cdot \vb*{A}$重要——后者重要时前者不重要,后者不重要时前者也没有重要到哪儿去。
}%
$q \vb*{A} \ll \vb*{p}$的条件实际上是不那么平凡的。
对散射态系统,机械动量估计为
\[
    m v \sim m \omega x,
\]
而
\[
    q E = m \ddot{x} \sim m \omega^2 x,
\]
最后有
\[
    E \sim - \pdv{A}{t} \sim \omega A,
\]
于是我们会发现$mv$和$eA$实际上是同个量级的。反之,对束缚态系统,$\vb*{p}$的最大值或者说振幅可以估计为
\[
    m \omega^2 x \sim q \grad{V_\text{ext}},
\]
而
\[
    mv \sim m \omega x,
\]
于是$p \gg eA$,等价于$mv \ll eA$,就等价于
\[
    mv \sim \frac{q}{\omega} \grad{V_\text{ext}} \gg q A,
\]
即等价于
\begin{equation}
    \grad{V_\text{ext}} \gg \omega A \sim E_\text{light},
\end{equation}
即束缚电场远大于光场。这应该是能够保证的,否则就不是束缚态了,此时介质就被打穿为等离子体了,并且,这种情况下,将光场撤去,介质也未必会恢复为原状,即出现了光学损伤。

在知道了能将\eqref{eq:couple-ham}当成微扰的系统中的带电粒子高度定域之后,我们立刻想到,由于这些带电粒子的位置高度有界,可以做多极矩展开。
实际上我们看到,多极矩展开合法、带电粒子位置高度定域(这意味着带电粒子)、$e \vb*{A} \ll \vb*{p}$这几个条件是等价的。
应该说\eqref{eq:couple-ham}是很不直观的,因为它是关于$\vb*{A}$的而不是$\vb*{E}$和$\vb*{B}$的,做完多极矩展开之后我们就可以讨论“某个过程在电偶极矩跃迁下可以发生,另一个过程需要电四极矩跃迁,从而很弱”,等等。
以下我们用$0$作为带点粒子位置的“原点”,$\vb*{r}$不会偏离$0$太远。

下面我们尝试使用多种近似手段。我们很快会发现,这些方法都指向同一个事实:对介质中的量子光学,基本上只有电偶极子相互作用是重要的。

直接丢弃$\vb*{A}^2$项。此时如果采取辐射规范,根据$\vb*{p}$和$\vb*{A}$的可交换性,我们就有
\begin{equation}
    H_\text{couple} = - \frac{q}{m} \vb*{A} \cdot \vb*{p}.
    \label{eq:velocity-gauge}
\end{equation}
这称为\concept{速度规范}下的哈密顿量。如果我们进一步,假定$\vb*{A}$在电子运动的区域内没有明显的空间变化,则在一个规范变换之下我们可以得到
\begin{equation}
    H_\text{couple} = - q \vb*{r} \cdot \vb*{E} = - \vb*{d} \cdot \vb*{E}.
    \label{eq:electric-dipole}
\end{equation}
或者,由于$\vb*{A}$在电子运动的区域内没有明显的空间变化,我们根据\eqref{eq:velocity-gauge}可以写出(这里我们假装$\vb*{p}$就是机械动量,但是因为$\vb*{p}$的实际物理意义在做了近似\eqref{eq:velocity-gauge}不再影响哈密顿量的形式,这是可以的)
\[
    \begin{aligned}
        S &= \int \dd{t} \left( \frac{1}{2} m \vb*{v}^2 + \frac{q}{m} \vb*{A} \cdot (m \vb*{v}) \right) \\
        &= \int \dd{t} \left( \frac{1}{2} m \vb*{v}^2 - q \dv{\vb*{A}}{t} \cdot \vb*{r} \right) \\
        &= \int \dd{t} \left( \frac{1}{2} m \vb*{v}^2 + q \vb*{r} \cdot \vb*{E} \right),
    \end{aligned} 
\]
第二个等号用到了分部积分法。再做勒让德变换,就得到\eqref{eq:electric-dipole}。
实际上,我们会注意到以上构造拉氏量以后用分布积分法的方法只用到了一个条件,就是$\vb*{A}$的空间变化不大(从而它对时间的全导数就是它对时间的偏导数,就是电场的相反数),因此只需要“$\vb*{A}$的空间变化不大”就足够推导出\eqref{eq:electric-dipole}。
我们称\eqref{eq:electric-dipole}为\concept{长度规范}下的哈密顿量。

我们现在考虑$\vb*{A}^2$项能够丢弃,但是$\vb*{A}$尚有比较大的空间变化的情况;当然,这是为了将磁场和轨道自由度做耦合。
乍一看,我们可以使用磁标势方法来得到磁场,但是这是行不通的:我们在处理的并非静磁学问题,位移电流项是到处都在的,从而如果要用磁标势方法,磁壳必须取在我们讨论的电子的周围,从而让磁标势毫无用处。
我们会发现取
\begin{equation}
    \vb*{A} = \frac{1}{2} \vb*{B} \times \vb*{r} - \int_0^t \dd{t'} \grad(\vb*{r} \cdot \vb*{E}(\vb*{r}, t'))
    \label{eq:a-containing-e-and-b}
\end{equation}
能够提供足够好的近似。直接计算就会发现
\[
    \curl{\vb*{A}} = \vb*{B},
\]
而
\[
    \begin{aligned}
        \pdv{\vb*{A}}{t} &= \frac{1}{2} \pdv{\vb*{B}}{t} \times \vb*{r} - \grad{(\vb*{r} \cdot \vb*{E})} = \frac{1}{2} \vb*{r} \times (\curl{\vb*{E}}) - \grad{(\vb*{r} \cdot \vb*{E})} \\
        &= \frac{1}{2} \left( \grad{(\vb*{r} \cdot \vb*{E})} - (\vb*{r} \cdot \grad) \vb*{E} - (\vb*{E} \cdot \grad) \vb*{r} - \vb*{E} \times (\curl{\vb*{r}}) \right) - \grad{(\vb*{r} \cdot \vb*{E})} \\
        &= \frac{1}{2} \left( \grad{(\vb*{r} \cdot \vb*{E})} - (\vb*{r} \cdot \grad) \vb*{E} - \vb*{E} \right)- \grad{(\vb*{r} \cdot \vb*{E})} .
    \end{aligned}
\]
如果假定$\vb*{E}$和$\vb*{B}$在空间上没有什么变化,那么就有
\[
    \pdv{\vb*{A}}{t} = \frac{1}{2} (\vb*{E} - \vb*{E}) - \vb*{E} = - \vb*{E}.
\]
因此,在电场和磁场在我们关心的区域基本均匀的情况下,\eqref{eq:a-containing-e-and-b}近似是辐射规范下的矢势。
现在我们再做一个规范变换:
\[
    \vb*{A} \longrightarrow \vb*{A} + \grad{\chi}, \quad \varphi \longrightarrow \varphi - \pdv{\chi}{t}, \quad \chi = \int_0^t \dd{t'} \vb*{r} \cdot \vb*{E}(\vb*{r}, t'),
\]
就有
\[
    \begin{aligned}
        H_\text{couple} &= q \varphi - \frac{q}{m} \vb*{A} \cdot \vb*{p} \\
        &= - q \vb*{r} \cdot \vb*{E} - \frac{q}{m} \frac{1}{2} (\vb*{B} \times \vb*{r}) \cdot \vb*{p} \\
        &= - \vb*{d} \cdot \vb*{E} - \frac{q}{2m} \vb*{B} \cdot (\vb*{r} \times \vb*{p}),
    \end{aligned}
\]
从而
\begin{equation}
    H_\text{couple} = - \vb*{d} \cdot \vb*{E} - \frac{q}{2m} \vb*{L} \cdot \vb*{B}.
\end{equation}
这里多出来了一项,即磁场和轨道角动量的耦合。
% TODO:但是此时电四极矩也开始变得重要了

现在我们有了三种相互作用通道,有电偶极跃迁
\begin{equation}
    H_1 = - \vb*{d} \cdot \vb*{E},
\end{equation}
有自旋取向作用
\begin{equation}
    H_2 = - \frac{q}{m} \vb*{S} \cdot \vb*{B},
\end{equation}
还有轨道角动量取向作用
\begin{equation}
    H_3 = - \frac{q}{2m} \vb*{L} \cdot \vb*{B}.
\end{equation}
实际上,磁场对自旋的取向作用${H}_2$是很弱的。设电磁波波长的尺度为$\lambda$,则
\[
    \vb*{B} = \curl{\vb*{A}} \sim \frac{A}{\lambda},
\]
电子的活动范围的尺度和原子半径$a_0$同阶,由不确定性关系,
\[
    p a_0 \sim \hbar.
\]
于是
\[
    \frac{H_2}{H_1} \sim \frac{\hbar \frac{A}{\lambda}}{\frac{\hbar}{a_0} A} = \frac{a_0}{\lambda}.
\]
波长通常在几百纳米级别,而原子半径在纳米级别以下,从而${H}_1$远大于${H}_2$。

\subsection{散射态系统和等离子体}

\chapter{线性介质概述}

我们总是先讨论自由理论再加入相互作用,因此本文首先考虑线性介质中的电磁波,然后再考虑非线性效应。

\section{有介质情况下的麦克斯韦方程——一个经典的推导}

在\autoref{sec:long-wavelength-photon-maxwell-general}中我们将要从对称性的角度说明,线性介质对麦克斯韦方程的修正的形式是非常有限的,并且这种修正在量子光学中同样适用。
本节则将介质暂且看成完全经典的东西,即将物质场抽象为电荷和电流,试图为线性介质中的麦克斯韦方程中的各项提供直观的、经典的意义。

真空中的麦克斯韦方程组为我们熟知的形式:
\begin{equation}
    \begin{bigcase}
        \div{\vb*{E}} &= \frac{\rho}{\epsilon_0} \\
        \curl{\vb*{E}} &= - \pdv{\vb*{B}}{t} \\
        \div{\vb*{B}} &= 0 \\
        \curl{\vb*{B}} &= \mu_0 \vb*{j} + \mu_0 \epsilon_0 \pdv{\vb*{E}}{t}
    \end{bigcase}
    \label{eq:original-maxwell}
\end{equation}
介质的存在事实上在微观层面不会改变\eqref{eq:original-maxwell}的形式。
介质起作用的方式是,其内部已经有一个电荷分布,当外加电场的时候电荷重新排列、发生运动,在此过程中产生额外的电流、电场、磁场。
于是假定电荷和电流可以做以下分解:
\[
    \begin{bigcase}
        &\vb*{j} = \vb*{j}_\text{f} + \vb*{j}_\text{r}, \quad \rho = \rho_\text{f} + \rho_\text{r}, \\
        &\pdv{\rho_\text{f}}{t} + \div{\vb*{j}_\text{f}} = 0, \\
        &\pdv{\rho_\text{r}}{t} + \div{\vb*{j}_\text{r}} = 0
    \end{bigcase}
\]
其中$\vb*{j}_\text{f}$是所谓的自由电流,而$\vb*{j}_\text{r}$是介质的响应。但是这种二分法实际上很大程度上是任意的。
例如,金属能导电,因为其内部含有大量几乎是自由的电子——那么,外加电场产生的金属中的电流就应该是自由电流了;
但是分析金属的光学属性的时候,这些由于外加电场产生的电流又无疑是介质的响应。
因此$\vb*{j}_\text{f}$和$\vb*{j}_\text{r}$只是辅助量,没有特殊的物理含义。

为了能够将$\vb*{j}_\text{f}$和$\vb*{j}_\text{r}$整合进两个形式上和电场和磁感应强度很像的辅助量,
从而在形式上让\eqref{eq:original-maxwell}变成一个只和自由电荷和自由电流有关的方程组,我们进一步做下面的分解:
\[
    \vb*{j}_\text{r} = \vb*{j}_\text{s} + \vb*{j}_\text{c}
\]
且$\vb*{j}_\text{c}$是一个有旋无源场。光有这个条件不足以在给定$\vb*{j}_\text{r}$时唯一地确定下$\vb*{j}_\text{s}$和$\vb*{j}_\text{c}$,
因此还可以引入一个假设而不至于让$\vb*{j}_\text{s}$和$\vb*{j}_\text{c}$无解。
为了让\eqref{eq:original-maxwell}中第一式的右边只剩下自由电荷,假定
\[
    \rho_\text{r} = - \div{\vb*{P}}
\]
这个假设\concept{没有}缩小$\vb*{j}_\text{s}$和$\vb*{j}_\text{c}$的选择范围,因为任意给定性质足够良好的$\rho_\text{r}$,相对应的$\vb*{P}$总是存在的(而且显然不唯一)。
同时由于$\vb*{j}_\text{c}$是一个有旋无源场,可以再引进一个辅助量$\vb*{M}$使
\[
    \vb*{j}_\text{c} = \curl{\vb*{M}}
\]
此时$\rho_\text{r}$的输运方程成为
\[
    \pdv{\rho_\text{r}}{t} + \div{\vb*{j}_\text{s}} = 0
\]
因为$\curl{\vb*{j}_\text{c}}$的散度为零。这个式子又可以写成
\[
    \div{\left(\vb*{j}_\text{s}-\pdv{\vb*{P}}{t}\right)} = 0
\]
受到这个式子的启发,我们\concept{假设}(不是推出,因为光有上式不能定解,而先前我们只对$\vb*{j}_\text{c}$做过假设而没有对$\vb*{j}_\text{s}$做过假设,因此后者的取值仍然是任意的)有
\[
    \vb*{j}_\text{s} = \pdv{\vb*{P}}{t}
\]
这个假设不会让$\vb*{j}_\text{s}$和$\vb*{j}_\text{c}$无解。

将以上引入的所有物理量代入\eqref{eq:original-maxwell},得到
\[
    \begin{bigcase}
        \epsilon_0 \div{\vb*{E}} &= \rho_\text{f} - \div{\vb*{P}}, \\
        \curl{\vb*{E}} &= - \pdv{\vb*{B}}{t}, \\
        \div{\vb*{B}} &= 0, \\
        \curl{\frac{\vb*{B}}{\mu_0}} &= \vb*{j}_\text{f} + \curl{\vb*{M}} + \pdv{\vb*{P}}{t} + \epsilon_0 \pdv{\vb*{E}}{t}
    \end{bigcase}
\]
引入辅助量
\[
    \vb*{D} = \epsilon_0 \vb*{E} + \vb*{P}, \quad \vb*{H} = \frac{\vb*{B}}{\mu_0} - \vb*{M}
\]
就得到了
\begin{equation}
    \begin{bigcase}
        \div{\vb*{D}} &= \rho_\text{f}, \\
        \curl{\vb*{E}} &= - \pdv{\vb*{B}}{t}, \\
        \div{\vb*{B}} &= 0, \\
        \curl{\vb*{H}} &= \vb*{j}_\text{f} + \pdv{\vb*{D}}{t}
    \end{bigcase}
    \label{eq:maxwell-material}
\end{equation}

方程组\eqref{eq:maxwell-material}除去了\eqref{eq:original-maxwell}中由于介质产生的电荷密度和电流密度,形式上更加简洁,
但是即使在自由电荷密度和电流密度已经给定的情况下,只靠\eqref{eq:maxwell-material}本身也没有办法定解,因为未知数太多了。
考虑到从$\vb*{E}, \vb*{B}$到$\vb*{D}, \vb*{H}$的变换是线性的,
这就意味着\eqref{eq:original-maxwell}在自由电荷密度和电流密度已经给定的情况下其实也不能定解。
这是理所当然的。

下面的问题是,在自由电荷密度和电流密度已经给定的情况下,增加什么方程能够让\eqref{eq:maxwell-material}定解?
当然,只要知道了从$\vb*{E}, \vb*{B}$到$\vb*{D}, \vb*{H}$的变换的具体计算式(而不是显含$\vb*{j}_\text{r}$的定义式)
就能够定解。
更进一步,在什么都不知道,只有初始条件和边界条件的情况下,怎样能够让\eqref{eq:maxwell-material}定解?
只需要增补$\vb*{j}_\text{f}$和$\vb*{E}$的显式关系,以及输运方程
\begin{equation}
    \pdv{\rho_\text{f}}{t} + \div{\vb*{j}_\text{f}} = 0
    \label{eq:transportation}
\end{equation}
就能够定解。

因此要求解出介质中的电磁场变化情况,首先需要\concept{物理方程}\eqref{eq:maxwell-material},
然后是\concept{本构关系}也就是$\vb*{D}$,$\vb*{H}$,$\vb*{j}_\text{f}$关于其他量的表达式,最后是\concept{几何关系}\eqref{eq:transportation},
再加上适当的\concept{边界条件}和\concept{初始条件},就能够定解。

关于本构关系实际上有一个问题,就是从$\vb*{E}$,$\vb*{B}$,$\vb*{j}_\text{f}$到$\vb*{D}$和$\vb*{H}$是不是真的有一个函数关系。
如果相同的$\vb*{E}$,$\vb*{B}$,$\vb*{j}_\text{f}$实际上对应着不同的系统状态,那就糟糕了。
但是在经典电动力学中确实只需要$\vb*{E}$,$\vb*{B}$是仅有的场,
而如果对$\vb*{j}_\text{s}$和$\vb*{j}_\text{c}$加上足够的限制,总是可以使用$\vb*{j}_\text{f}$确定下整个$\vb*{j}$的分布,从而$\rho$的分布,
因此$\vb*{E}$,$\vb*{B}$,$\vb*{j}_\text{f}$能够完全确定系统状态,从而本构关系总是可以写出来的。

\section{亥姆霍兹分解与线性介质中常见的电磁场模式}

在电动力学中我们基本上只需要使用散度和旋度。关于这件事有著名的\concept{亥姆霍兹分解}:任意一个矢量场$\vb*{X}$只要在无穷远处衰减得足够快(至少比$1 / r$快),则可以做如下分解:
\begin{equation}
    \vb*{X} = - \grad{U} + \curl{\vb*{W}},
    \label{eq:ht-decomp}
\end{equation}
其中$U$和$\vb*{X}$可以分别表示为
\begin{equation}
    U = \frac{1}{4\pi} \int \dd[3]{\vb*{r}'} \frac{\grad' \cdot \vb*{X}(\vb*{r}')}{\abs{\vb*{r} - \vb*{r}'}} - \frac{1}{4\pi} \oint_S \dd{S} \vu*{n} \cdot \frac{\vb*{X}(\vb*{r}')}{\abs{\vb*{r} - \vb*{r}'}},
    \label{eq:ht-decomp-u-def}
\end{equation}
以及
\begin{equation}
    \vb*{W} = \frac{1}{4\pi} \int \dd[3]{\vb*{r}'} \frac{\grad' \times \vb*{X}(\vb*{r}')}{\abs{\vb*{r} - \vb*{r}'}} - \frac{1}{4\pi} \oint_S \dd{S} \vu*{n} \times \frac{\vb*{X}(\vb*{r}')}{\abs{\vb*{r} - \vb*{r}'}}.
    \label{eq:ht-decomp-w-def}
\end{equation}
证明是相对简单的,因为\eqref{eq:ht-decomp-u-def}和\eqref{eq:ht-decomp-w-def}是完全构造性的,我们只需要验证它们的确满足\eqref{eq:ht-decomp}即可,这就证明了分解\eqref{eq:ht-decomp}总是可行的。
应注意场的衰减条件还是重要的,因为在一些情况中(如静电学问题中)我们的确会在无穷远处放置一些源(比如说一块其上有感应电荷的金属板),那么场可能衰减得没有那么快。

对电场和磁场作用亥姆霍兹分解,能够得到
\begin{equation}
    \begin{aligned}
        \vb*{E} &= - 
    \end{aligned}
\end{equation}
从这个分解中能够看到线性介质中常见的几种电磁波模式。如果磁场不重要,那么我们就得到静电学,其中
静磁学
将$c \to \infty$,得到准静态近似
最后是电磁波

直观地看,准静态近似实际上是在描写一个天线:

亥姆霍兹分解本身无助于求解麦克斯韦方程,它更多用于在已知结果后诠释它。

\section{线性介质中的光场量子化}

电磁场足够强以至于难以看到单光子效应,而又足够弱以至于能量不至于强到需要考虑量子电动力学的圈图修正,这样就可以使用经典电动力学描述整个系统。
为了讨论电磁波的量子涨落(在分析诸如腔内辐射场,或是非线性光学中的DFG过程时非常重要),即使没有圈图效应,我们也要做光场的量子化。

\subsection{真空}

我们首先考虑真空中的光场的量子化,此时我们无非是在重复QED中的运算。
QED中矢量场展开为
\begin{equation}
    A_\mu(\vb*{x}, t) = (\frac{\varphi}{c}, - \vb*{A}) = \int \frac{\dd[3]{\vb*{k}}}{(2\pi)^3} \sqrt{\frac{\hbar}{2\omega_{\vb*{k}} \epsilon_0}} \sum_{\sigma=1}^2 \left( a_{\vb*{k} \sigma} \epsilon_\mu^\sigma(\vb*{k}) \ee^{\ii \vb*{k} \cdot \vb*{x} - \ii \omega_{\vb*{k}} t} + a_{\vb*{k} \sigma}^\dagger \epsilon_\mu^\sigma(\vb*{k})^* \ee^{- \ii \vb*{k} \cdot \vb*{x} + \ii \omega_{\vb*{k}} t} \right),
\end{equation}
其中电磁场模式为平面波。
取费曼规范,做一些分部积分并去掉表面项,得到
\begin{equation}
    \mathcal{L} = - \frac{1}{2 \mu_0} \partial_\mu A_\nu \partial^\mu A^\nu,
\end{equation}
从而正则动量为
\begin{equation}
    \pi^\mu = \pdv{\mathcal{L}}{\partial_0 A_\mu} = - \partial^0 A^\mu,
\end{equation}
可以据此写出正则量子化条件,即时间相同时,$A^\mu$同$A^\nu$对易,而
\begin{equation}
    [A^\mu(\vb*{x}, t), \pi^\mu(\vb*{y}, t)] = \ii \eta^{\mu \nu} \delta^{(3)}(\vb*{x} - \vb*{y}).
\end{equation}
哈密顿量为
\[
    \begin{aligned}
        H &= \int \dd[3]{\vb*{r}} (\pi^\mu \partial_0 A_\mu - \mathcal{L}) \\
        &= \int \dd[3]{\vb*{r}} \left( - \frac{1}{c^2} (\partial_t A^\mu)^2 + \frac{1}{2} \partial_\mu A_\nu \partial^\mu A^\nu \right),
    \end{aligned}
\]
这里要注意$x^0 = c t$。代入$A_\mu$的展开式计算得到
\begin{equation}
    H = \sum_{\sigma=1}^2 \int \frac{\dd[2]{\vb*{k}}}{(2\pi)^3} \hbar \omega_{\vb*{k}} \left( a^\dagger_{\vb*{k} \sigma} a_{\vb*{k} \sigma} + \frac{1}{2} \right), \quad \omega_{\vb*{k}} = c \abs*{\vb*{k}}.
\end{equation}
这就得到了量子化的能量。
在量子化过程中我们已经通过限制$\sigma$而施加了规范,不过这个规范并不是辐射规范,而是洛伦兹规范。横波条件通过$\epsilon$矢量和波矢垂直而保证。
不过,既然我们只关心电偶极辐射而有关的相互作用哈密顿量可以完全写成$\vb*{E}$,这也不重要。

从四维矢量计算电场,得到
\begin{equation}
    \vb*{E}(\vb*{r}, t) = \int \frac{\dd[3]{\vb*{k}}}{(2\pi)^3} \sqrt{\frac{\hbar}{2\omega_{\vb*{k}} \epsilon_0}} \sum_{\sigma=1}^2 \left( (- \ii \vb*{k} \epsilon_0^\sigma(\vb*{k}) + \ii \omega_{\vb*{k}} \vb*{\epsilon}^\sigma(\vb*{k})) a_{\vb*{k} \sigma} \ee^{\ii \vb*{k} \cdot \vb*{r} - \ii \omega_{\vb*{k}} t} + \text{h.c.} \right),
\end{equation}
以及
\begin{equation}
    \vb*{B}(\vb*{r}, t) = \int \frac{\dd[3]{\vb*{k}}}{(2\pi)^3} \sqrt{\frac{\hbar}{2\omega_{\vb*{k}} \epsilon_0}} \sum_{\sigma=1}^2 \left( \ii \vb*{k} \times \vb*{\epsilon}_\sigma a_{\vb*{k} \sigma} \ee^{\ii \vb*{k} \cdot \vb*{r} - \ii \omega_{\vb*{k}} t } + \text{h.c.} \right).
\end{equation}
其中
\begin{equation}
    \epsilon^\mu_\sigma = (\epsilon_\sigma^0, \vb*{\epsilon}_\sigma), \quad \frac{\omega_{\vb*{k}}}{c} \epsilon_0^\sigma - \vb*{k} \cdot \vb*{\epsilon}_\sigma = 0, \quad \abs*{\epsilon_\sigma^0}^2 - \abs*{\vb*{\epsilon}_\sigma}^2 = 1.
\end{equation}
通过以上公式,能够验证以下哈密顿量形式:
\begin{equation}
    H = \int \dd[3]{\vb*{r}} \left( \frac{\epsilon_0}{2} \vb*{E}^2 + \frac{1}{2\mu_0} \vb*{B}^2 \right).
    \label{eq:e-and-b-hamiltonian}
\end{equation}

在本文讨论的光学问题中,我们可以使用一种对具体计算更加友好的形式,即采用\concept{辐射规范}。
在辐射规范之下,我们有
\begin{equation}
    \begin{aligned}
        \mathcal{L} &= - \frac{1}{4 \mu_0} (\partial_\mu A_\nu - \partial_\nu A_\mu) (\partial^\mu A^\nu - \partial^\nu A^\mu) \\
        &= \frac{1}{2 \mu_0} \frac{1}{c^2} (\partial_t \vb*{A})^2 - \frac{1}{4\mu_0} (\partial_i A_j - \partial_j A_i) (\partial^i A^j - \partial^j A^i) \\
        &= \frac{\epsilon_0}{2} ((\dot{\vb*{A}})^2 - c^2 (\curl{\vb*{A}})^2).
    \end{aligned}
\end{equation}
以这个拉氏量为出发点做正则量子化。做展开
\begin{equation}
    \vb*{A}(\vb*{r}, t) = \int \frac{\dd[3]{\vb*{k}}}{(2\pi)^3} \sqrt{\frac{\hbar}{2 \omega_{\vb*{k}} \epsilon_0}} \sum_{\sigma=1}^2 (a_{\vb*{k} \sigma} \vu*{e}^\sigma \ee^{\ii \vb*{k} \cdot \vb*{r} - \ii \omega_{\vb*{k}} t} + a^\dagger_{\vb*{k} \sigma} (\vu*{e}^\sigma)^* \ee^{- \ii \vb*{k} \cdot \vb*{r} + \ii \omega_{\vb*{k}} t}),
\end{equation}
从而电场和磁场分别为
\begin{equation}
    \vb*{E}(\vb*{r}, t) = \ii \int \frac{\dd[3]{\vb*{k}}}{(2\pi)^3} \sqrt{\frac{\hbar \omega_{\vb*{k}}}{2 \epsilon_0}} \sum_{\sigma=1}^2 (a_{\vb*{k} \sigma} \vu*{e}^\sigma \ee^{\ii \vb*{k} \cdot \vb*{r} - \ii \omega_{\vb*{k}} t} - a^\dagger_{\vb*{k} \sigma} (\vu*{e}^\sigma)^* \ee^{- \ii \vb*{k} \cdot \vb*{r} + \ii \omega_{\vb*{k}} t})
\end{equation}
和
\begin{equation}
    \vb*{B}(\vb*{r}, t) = \ii \int \frac{\dd[3]{\vb*{k}}}{(2\pi)^3} \sqrt{\frac{\hbar}{2 \omega_{\vb*{k}} \epsilon_0}} \sum_{\sigma=1}^2 (a_{\vb*{k} \sigma} \vb*{k} \times \vu*{e}_\sigma \ee^{\ii \vb*{k} \cdot \vb*{r} - \ii \omega_{\vb*{k}} t} - a^\dagger_{\vb*{k} \sigma} \vb*{k} \times \vu*{e}_\sigma^* \ee^{- \ii \vb*{k} \cdot \vb*{r} + \ii \omega_{\vb*{k}} t}).
\end{equation}
正则动量为
\begin{equation}
    \vb*{\pi} = \epsilon_0 \dot{\vb*{A}},
\end{equation}
施加正则对易关系,会得到正确的
\begin{equation}
    \comm*{a_{\vb*{k} \sigma}}{a_{\vb*{k}' \sigma'}} = (2\pi)^3 \delta(\vb*{k} - \vb*{k}') \delta_{\sigma \sigma'},
\end{equation}
而哈密顿量为
\begin{equation}
    \begin{aligned}
        H &= \int \dd[3]{\vb*{r}} \left(\vb*{\pi} \cdot \pdv{\vb*{A}}{t} - \mathcal{L} \right) = \int \dd[3]{\vb*{r}} \left( \frac{\epsilon_0}{2} \left(\pdv{\vb*{A}}{t}\right)^2 + \frac{\epsilon_0}{2} c^2 (\curl{\vb*{A}})^2 \right) \\
        &= \int \dd[3]{\vb*{r}} \left( \frac{\epsilon_0}{2} \vb*{E}^2 + \frac{1}{2\mu_0} \vb*{B}^2 \right) \\
        &= \sum_{\sigma=1}^2 \int \frac{\dd[3]{\vb*{k}}}{(2\pi)^3} \hbar \omega_{\vb*{k}} \left(a^\dagger_{\vb*{k} \sigma} a_{\vb*{k} \sigma} + \frac{1}{2} \right).
    \end{aligned}
\end{equation}
因此,辐射规范给出的结果和完整的QED计算是完全一致的。
在辐射规范中我们还可以证明一个在横场条件成立时也成立,并且在一般的QED中很难计算的公式:
\begin{equation}
    \comm*{E^i(\vb*{r}, t)}{B^j(\vb*{r}', t)} = - \frac{\ii \hbar}{\epsilon_0} \pdv{x^k} \delta(\vb*{r} - \vb*{r}')
\end{equation}
其中$i, j, k$是$x, y, z$的轮换排列;其它情况下对易子为零。
还能够发现电场和自己的对易子始终为零,磁场亦然。因此电场的三个分量可以同时确定地被测量,磁场亦然。
但是不能同时准确测出$\vb*{E}$和$\vb*{B}$。
由于$(\vb*{E}, \vb*{B})$,$\vb*{A}$和$a_{\vb*{k} \sigma}$直接的关系是线性的,$a$的产生湮灭算符对易关系、$\vb*{A}$和$\vb*{\pi}$的正则对易关系以及$\vb*{E}$和$\vb*{B}$的对易关系是彼此等价的。

\subsection{长波光子和介质中的麦克斯韦方程}\label{sec:long-wavelength-photon-maxwell-general}

现在我们考虑介质作用。直接将QED和介质耦合起来,虽然的确是正确的,在实际计算时却会产生一些理论上的问题。
例如,我们知道,介质通常出于热态,因此,一个光子和介质发生相互作用之后就处于混合态了,似乎不能写出一个场论来描述介质中光子;从介质吸收光子到发射光子会有时间延迟;介质微观上是非常不均匀的,从而平面波进入介质后波阵面将面目全非。
第一个问题的严格处理显然需要用到非平衡态场论,但我们可以认为我们将介质积掉了,从而用介质中的电磁场-电磁场关联函数代替了介质作用。
这样,介质的线性效应体现为电磁场的作用量的二次型部分出现一个修正,非线性效应体现为电磁场的自相互作用,非幺正的部分体现为以上修正中的虚部。
后两个问题可以采用和经典电动力学类似的方法解决,即我们只处理“经过空间平均”的电磁场,这相当于做了一个动量截断,只讨论波长足够长的那部分电磁波模式,则介质中发生的过程相比于我们讨论的过程来说是非常快、且空间细节不甚清楚的,从而,介质导致的电磁场关联函数的修正可以认为没有时间上的延迟效应或是空间上的非局域效应。
在电磁场的波长和晶格常数接近时,这么做就失效了,此时必须使用完整的第一性原理做计算。

在以上条件——只需要考虑波长远大于介质的微观不均匀性的空间尺度的光子——成立时,形式上,我们可以直接将介质中的麦克斯韦方程做正则量子化。要看出这是为什么,首先考虑线性部分,描述光场的宏观的线性介质中的麦克斯韦方程是
\[
    \begin{aligned}
        &\div{\vb*{D}} = 0, \quad \curl{\vb*{E}} = - \pdv{\vb*{B}}{t}, \\
        &\div{\vb*{B}} = 0, \quad \curl{\vb*{H}} = \pdv{\vb*{D}}{t} + \vb*{j},
    \end{aligned}
\]
这里我们保留了传导电流,这是为了提示系统哈密顿量中外加激励项$\vb*{j} \cdot \vb*{A}$的存在。取规范$\varphi=0$,并切换到频域,我们会发现以上方程等价于辐射规范加上
\begin{equation}
    \curl{(\mu^{-1} \cdot \curl{\vb*{A}})} - \omega^2 \epsilon \cdot \vb*{A} = \vb*{j}.
    \label{eq:photon-in-material}
\end{equation}
如果介质修正后的电磁场关联函数实际上就是上式的格林函数,我们就可以直接将线性介质中的麦克斯韦方程中的电场和磁场提升为算符,完成正则量子化。

对称性告诉我们,在长波光子条件成立时,破缺空间平移对称性和空间各向同性,但保留局域性,则\eqref{eq:photon-in-material}是最一般的方程。
可以在整个方程左边再乘上一个张量,但是我们随即可以将这个张量吸收到$\vb*{j}$的定义中;$\curl{\vb*{A}}$的形式不能改变,因为无论如何,从$\vb*{A}$出发能够得到的局域的规范不变矢量除了$\pdv*{\vb*{A}}{t}$——在频域下就正比于$\vb*{A}$——以外就只有它了。
因此,的确,对波长远大于介质微观不均匀性(晶格常数等)的光子(大部分能够称为“光学”的问题都是这样的,因为晶格常数差不多几百皮米,已经对应X射线的波长了),至少线性介质中的麦克斯韦方程可以被理解为海森堡绘景下的方程。
非线性项可以如法炮制。因此,在这里,我们的思路和高能物理类似,先获得一个自由理论,再加入相互作用;另一种思路是使用量子版本的极化矢量。

从哈密顿量的角度出发可以更加容易地看出为什么线性麦克斯韦方程\eqref{eq:photon-in-material}可以直接量子化。
破缺空间平移对称性和空间各向同性之后,\eqref{eq:e-and-b-hamiltonian}能够有的修正方式是非常有限的:如果保持哈密顿量为二次型,我们只能够让$\vb*{E}^2$项和$\vb*{B}^2$项变得各向异性,即让它变成
\begin{equation}
    H = \int \dd[3]{\vb*{r}} \left( \frac{1}{2} \vb*{E} \cdot \vb*{\epsilon} \cdot \vb*{E} + \frac{1}{2} \vb*{B} \cdot (\vb*{\mu}^{-1}) \cdot \vb*{B} \right) = \frac{1}{2} \int \dd[3]{\vb*{r}} (\vb*{D} \cdot \vb*{E} + \vb*{B} \cdot \vb*{H}).
    \label{eq:material-hamiltonian}
\end{equation}
它和\eqref{eq:photon-in-material}是等价的。哈密顿量被修正在物理上对应着积掉介质,如果只考虑长波光子,那么这个过程应该给出在时间上和空间上都是局域的等效光子相互作用。
原则上可以产生$\vb*{E}$和$\vb*{B}$的任意次方项,只保留两项就得到\eqref{eq:material-hamiltonian},保留更多项就得到非线性光学效应。

下面我们讨论和\eqref{eq:material-hamiltonian}匹配的对易关系,以及它对角化之后将给出什么样的能谱。
应当指出,此时真空中的那些对易关系——$\vb*{A}$和$\epsilon_0 \dot{\vb*{A}}$之间的对易关系,$\vb*{E}$和$\vb*{B}$之间的对易关系——可能不能够直接适用。
这是因为正则量子化中,积掉自由度会导致哈密顿量的本征态的意义发生变化,从而算符的意义会发生变化。在高能物理中这导致场强重整化,在本文讨论的量子光学中则还会让对易关系发生变化。
同样这也会让横波条件的形式发生变化——介质中横波条件是$\div{\vb*{\epsilon} \cdot \vb*{E}} = 0$。

我们需要直接从\eqref{eq:material-hamiltonian}计算正则动量。同样取辐射规范,以$\vb*{A}$为基本自由度,则\eqref{eq:material-hamiltonian}就是
\begin{equation}
    H = \int \dd[3]{\vb*{r}} \left( \frac{1}{2} \dot{\vb*{A}} \cdot \vb*{\epsilon} \cdot \dot{\vb*{A}} + \frac{1}{2} (\curl{\vb*{A}}) \cdot (\vb*{\mu}^{-1}) \cdot (\curl{\vb*{A}}) \right),
\end{equation}
于是正则动量为
\begin{equation}
    \vb*{\pi} = \vb*{\epsilon} \cdot \dot{\vb*{\vb*{A}}}.
\end{equation}
我们现在需要展开$\vb*{A}$。此时空间平移不变性不能保持,我们不能使用动量来标记电场的振动模式,
我们将\eqref{eq:photon-in-material}右边的$\vb*{j}$取为零——我们此处在对线性介质做正则量子化,暂时不考虑电流——那就得到了一个广义本征值问题。
这就意味着,我们可以求解出一整套本征函数,它们由下式
\begin{equation}
    \curl{(\mu^{-1} \cdot \curl{\vb*{u}_n})} - \omega_n^2 \epsilon \cdot \vb*{u}_n = 0
\end{equation}
确定,其中$\omega_n$对应着能够在系统中稳定传播的电磁波模式的频率,且有正交归一关系
\begin{equation}
    \int_V \dd[3]{\vb*{r}} \vb*{u}^*_m \cdot \vb*{\epsilon} \cdot \vb*{u}_n = \delta_{mn}, 
\end{equation}
请注意由于$\vb*{A}$的厄米性,$\vb*{u}_m^*$一般是另一个$\vb*{u}_n$。
正交归一关系又意味着
\begin{equation}
    \omega_n^2 \delta_{mn} = \int \dd[3]{\vb*{r}} (\curl{\vb*{u}_m^*}) \cdot (\vb*{\mu}^{-1}) \cdot (\curl{\vb*{u}_n}) + \int \dd{\vb*{S}} \cdot (\vb*{u}_m^* \times ((\vb*{\mu}^{-1}) \cdot (\curl{\vb*{u}_n}))),
\end{equation}
在自由空间中等式右边第二项可以略去,在一个反射性能尚可的反射腔体(如果我们只讨论有限空间中的问题,那么基本上这个问题需要放在一个腔体中,否则无法忽视外界影响)中可以把第二项当成微扰。
本节仅仅给出最为简单的理论,暂时不考虑第二项。
用这组基底$\{\vb*{u}_n\}$做展开
\begin{equation}
    \vb*{A}(\vb*{r}, t) = \sum_n \ii \sqrt{\frac{\hbar}{2\omega_{\vb*{k}}}} \vb*{u}_n(\vb*{r}) a_n \ee^{- \ii \omega_n t} + \text{h.c.},
\end{equation}
得到
\begin{equation}
    - \vb*{E} = \dot{\vb*{A}} = \sum_n \sqrt{\frac{\hbar \omega_n}{2}} \vb*{u}_n(\vb*{r}) a_n \ee^{-\ii \omega_n t} + \text{h.c.},
\end{equation}
以及
\begin{equation}
    \vb*{B} = \curl{\vb*{A}} = \sum_n \ii \sqrt{\frac{\hbar}{2\omega_n}} \curl{\vb*{u}_n(\vb*{r})} a_n \ee^{-\ii \omega_n t} + \text{h.c.}.
\end{equation}
施加正则对易关系
\begin{equation}
    \comm*{A^i(\vb*{r}, t)}{\pi^j(\vb*{r}', t)} = \ii \hbar \delta(\vb*{r} - \vb*{r}') \delta^{ij},
\end{equation}
我们发现我们能够得到我们想要的产生湮灭算符对易关系
\begin{equation}
    \comm*{a_n}{a_m^\dagger} = \delta_{mn}.
\end{equation}
然后,可以计算出哈密顿量为
\begin{equation}
    H = \sum_n \hbar \omega_n \left( a^\dagger_n a_n + \frac{1}{2} \right).
\end{equation}
这个哈密顿量的形式和真空中完全一样,不同的地方在于$\omega_{\vb*{k}}$被$\omega_n$取代,色散关系可能变得非常不一样。

既然$\epsilon$和$\mu$的概念对长波光子在量子情况下仍然适用,反射、折射等概念对长波光子仍然有意义,且和经典情况非常类似。
特别的,光场可能被约束在一个四面都是反射镜的腔体中,此时的光场被所谓的\concept{cavity QED}或者简写为\concept{cQED}描述。

总之,有两种方法处理介质:一种是显式地根据光场和介质的电偶极子耦合做微扰计算;但是,也可以首先计算介质中的电磁场关联函数,然后根据\eqref{eq:photon-in-material}得到$\epsilon$和$\mu$,代入算符版本的介质中麦克斯韦方程。
算符版本的介质中麦克斯韦方程适用于波长长于介质微观结构尺度(如晶格常数)的光子,因此适用范围是很大的。
一个介质系统中的量子化光场的自由哈密顿量就是普通的谐振子哈密顿量。
使用本质上是经典的方程\eqref{eq:photon-in-material},得到一系列振动模式,其频率即为这个介质系统中的量子化光场中的模式的频率,振动模式的场强分布就是\eqref{eq:photon-in-material}给出的本征模式。

实际上从这里我们可以看出,经典的麦克斯韦方程本身已经是一个具有一定量子特性的理论了——“单光子波函数”(虽然没有良定义的单光子量子力学,但是我们不妨这么指代$\mel*{0}{A^i(\vb*{r})}{\psi}$)服从的方程就是麦克斯韦方程。
也可以从另一个角度看这件事:在麦克斯韦方程两边乘上$\hbar$,由于$E \sim \hbar \omega$,得到的理论看上去就是一个量子理论。
将光场量子化引入的新物理只有两件事:光束由分立的光子构成,以及存在光子数的量子涨落,但是,在没有非线性光学效应的情况下,光子数目守恒,第一件事完全可以通过手动引入“光子”的概念并指派其波函数为(经过适当归一化的)经典电磁场来做到。
光的量子性只有在下面的地方才会变得重要:
\begin{itemize}
    \item 光子生灭明显,一些光子模式上原本没有电子而一段时间后有光子产生时,即处理非线性光学时,因为此时会有一些原本完全没有光子分布的模式上出现了光子。经典处理只能在有种子光的时候处理光子的产生——并不奇怪,因为光子从零到一产生的过程涉及一个极为弱的场强,弱到经典场论不再使用。
    \item 纠缠重要时。经典电动力学面对“光子增多了”的描述方法是更大的场强,而没有直积的希尔伯特空间这样的概念,从而无法捕捉到纠缠。
\end{itemize}
可以看到这些光的量子性变得明显的情况都涉及多光子态。我们其实可以在这里看到一个相当有趣的情况:当所研究的问题中涉及非常弱的光场(如特定频率的光子一开始没有,但是一段时间后被产生)时,经典电动力学就失效了,然而经典电动力学的形式却又很像是在处理“单光子波函数”。
当然,这两者并没有矛盾,本质上是因为经典电动力学无法正确处理“多光子形成的多体波函数”:“单光子波函数”不涉及多体波函数,它给出的所有物理就是一个麦克斯韦方程,正好和经典电动力学一致;光子数足够多时依照大量子数极限,多体波函数的形式变得不重要,大部分时候可以被看成一个没有纠缠的非常平凡的近独立玻色气体波函数,同样可以被经典电动力学处理。
经典电动力学无法正确处理“多光子形成的多体波函数”,因为没有Fock空间;但是这并不是说经典电动力学就缺乏(相比于经典质点动力学的)一切量子性,如坐标和动量的不确定性等。
实际上,对缺乏纠缠、缺乏粒子生灭和碰撞、粒子数大的有质量粒子系统,单粒子波函数乘以适当的因子也可以诠释为“粒子数的平方根”。在这个意义上它和电磁场的地位是类似的。
当然,实际研究中,有质量粒子系统中有大量的碰撞,其宏观理论通常是动理学方程,且“经典费米场”不是一个物理意义特别明确的东西。

在本文中“电磁场”可能代表量子化的场算符,也可能代表经典电磁场,也可能代表“单光子波函数”。在不考虑非线性效应时这三者的时间演化是相同的。
在$\vb*{E}$被认为是经典场时,$\vb*{E}^2$——从而$I$——在形式上对应于“光子出现的概率”。
光场中被传输的不是$I$,电磁场的相位信息是很重要的,正如量子力学中叠加的不是概率而是概率振幅一样。
对非相干光(后文将讨论),可以直接将$I$相加,正如高度混合态的系统可以直接使用经典概率论处理一样。



\part{线性介质中光的传播}

\part{线性介质}

\chapter{线性介质与经典麦克斯韦方程}

\section{有介质情况下的麦克斯韦方程}

真空中的麦克斯韦方程组为我们熟知的形式:
\begin{equation}
    \begin{bigcase}
        \div{\vb*{E}} &= \frac{\rho}{\epsilon_0} \\
        \curl{\vb*{E}} &= - \pdv{\vb*{B}}{t} \\
        \div{\vb*{B}} &= 0 \\
        \curl{\vb*{B}} &= \mu_0 \vb*{j} + \mu_0 \epsilon_0 \pdv{\vb*{E}}{t}
    \end{bigcase}
    \label{eq:original-maxwell}
\end{equation}
介质的存在事实上在微观层面不会改变\eqref{eq:original-maxwell}的形式。
介质起作用的方式是,其内部已经有一个电荷分布,当外加电场的时候电荷重新排列、发生运动,在此过程中产生额外的电流、电场、磁场。
于是假定电荷和电流可以做以下分解:
\[
    \begin{bigcase}
        &\vb*{j} = \vb*{j}_\text{f} + \vb*{j}_\text{r}, \quad \rho = \rho_\text{f} + \rho_\text{r}, \\
        &\pdv{\rho_\text{f}}{t} + \div{\vb*{j}_\text{f}} = 0, \\
        &\pdv{\rho_\text{r}}{t} + \div{\vb*{j}_\text{r}} = 0
    \end{bigcase}
\]
其中$\vb*{j}_\text{f}$是所谓的自由电流,而$\vb*{j}_\text{r}$是介质的响应。但是这种二分法实际上很大程度上是任意的。
例如,金属能导电,因为其内部含有大量几乎是自由的电子——那么,外加电场产生的金属中的电流就应该是自由电流了;
但是分析金属的光学属性的时候,这些由于外加电场产生的电流又无疑是介质的响应。
因此$\vb*{j}_\text{f}$和$\vb*{j}_\text{r}$只是辅助量,没有特殊的物理含义。

为了能够将$\vb*{j}_\text{f}$和$\vb*{j}_\text{r}$整合进两个形式上和电场和磁感应强度很像的辅助量,
从而在形式上让\eqref{eq:original-maxwell}变成一个只和自由电荷和自由电流有关的方程组,我们进一步做下面的分解:
\[
    \vb*{j}_\text{r} = \vb*{j}_\text{s} + \vb*{j}_\text{c}
\]
且$\vb*{j}_\text{c}$是一个有旋无源场。光有这个条件不足以在给定$\vb*{j}_\text{r}$时唯一地确定下$\vb*{j}_\text{s}$和$\vb*{j}_\text{c}$,
因此还可以引入一个假设而不至于让$\vb*{j}_\text{s}$和$\vb*{j}_\text{c}$无解。
为了让\eqref{eq:original-maxwell}中第一式的右边只剩下自由电荷,假定
\[
    \rho_\text{r} = - \div{\vb*{P}}
\]
这个假设\concept{没有}缩小$\vb*{j}_\text{s}$和$\vb*{j}_\text{c}$的选择范围,因为任意给定性质足够良好的$\rho_\text{r}$,相对应的$\vb*{P}$总是存在的(而且显然不唯一)。
同时由于$\vb*{j}_\text{c}$是一个有旋无源场,可以再引进一个辅助量$\vb*{M}$使
\[
    \vb*{j}_\text{c} = \curl{\vb*{M}}
\]
此时$\rho_\text{r}$的输运方程成为
\[
    \pdv{\rho_\text{r}}{t} + \div{\vb*{j}_\text{s}} = 0
\]
因为$\curl{\vb*{j}_\text{c}}$的散度为零。这个式子又可以写成
\[
    \div{\left(\vb*{j}_\text{s}-\pdv{\vb*{P}}{t}\right)} = 0
\]
受到这个式子的启发,我们\concept{假设}(不是推出,因为光有上式不能定解,而先前我们只对$\vb*{j}_\text{c}$做过假设而没有对$\vb*{j}_\text{s}$做过假设,因此后者的取值仍然是任意的)有
\[
    \vb*{j}_\text{s} = \pdv{\vb*{P}}{t}
\]
这个假设不会让$\vb*{j}_\text{s}$和$\vb*{j}_\text{c}$无解。

将以上引入的所有物理量代入\eqref{eq:original-maxwell},得到
\[
    \begin{bigcase}
        \epsilon_0 \div{\vb*{E}} &= \rho_\text{f} - \div{\vb*{P}}, \\
        \curl{\vb*{E}} &= - \pdv{\vb*{B}}{t}, \\
        \div{\vb*{B}} &= 0, \\
        \curl{\frac{\vb*{B}}{\mu_0}} &= \vb*{j}_\text{f} + \curl{\vb*{M}} + \pdv{\vb*{P}}{t} + \epsilon_0 \pdv{\vb*{E}}{t}
    \end{bigcase}
\]
引入辅助量
\[
    \vb*{D} = \epsilon_0 \vb*{E} + \vb*{P}, \quad \vb*{H} = \frac{\vb*{B}}{\mu_0} - \vb*{M}
\]
就得到了
\begin{equation}
    \begin{bigcase}
        \div{\vb*{D}} &= \rho_\text{f}, \\
        \curl{\vb*{E}} &= - \pdv{\vb*{B}}{t}, \\
        \div{\vb*{B}} &= 0, \\
        \curl{\vb*{H}} &= \vb*{j}_\text{f} + \pdv{\vb*{D}}{t}
    \end{bigcase}
    \label{eq:maxwell-material}
\end{equation}

方程组\eqref{eq:maxwell-material}除去了\eqref{eq:original-maxwell}中由于介质产生的电荷密度和电流密度,形式上更加简洁,
但是即使在自由电荷密度和电流密度已经给定的情况下,只靠\eqref{eq:maxwell-material}本身也没有办法定解,因为未知数太多了。
考虑到从$\vb*{E}, \vb*{B}$到$\vb*{D}, \vb*{H}$的变换是线性的,
这就意味着\eqref{eq:original-maxwell}在自由电荷密度和电流密度已经给定的情况下其实也不能定解。
这是理所当然的。

下面的问题是,在自由电荷密度和电流密度已经给定的情况下,增加什么方程能够让\eqref{eq:maxwell-material}定解?
当然,只要知道了从$\vb*{E}, \vb*{B}$到$\vb*{D}, \vb*{H}$的变换的具体计算式(而不是显含$\vb*{j}_\text{r}$的定义式)
就能够定解。
更进一步,在什么都不知道,只有初始条件和边界条件的情况下,怎样能够让\eqref{eq:maxwell-material}定解?
只需要增补$\vb*{j}_\text{f}$和$\vb*{E}$的显式关系,以及输运方程
\begin{equation}
    \pdv{\rho_\text{f}}{t} + \div{\vb*{j}_\text{f}} = 0
    \label{eq:transportation}
\end{equation}
就能够定解。

因此要求解出介质中的电磁场变化情况,首先需要\concept{物理方程}\eqref{eq:maxwell-material},
然后是\concept{本构关系}也就是$\vb*{D}$,$\vb*{H}$,$\vb*{j}_\text{f}$关于其他量的表达式,最后是\concept{几何关系}\eqref{eq:transportation},
再加上适当的\concept{边界条件}和\concept{初始条件},就能够定解。

关于本构关系实际上有一个问题,就是从$\vb*{E}$,$\vb*{B}$,$\vb*{j}_\text{f}$到$\vb*{D}$和$\vb*{H}$是不是真的有一个函数关系。
如果相同的$\vb*{E}$,$\vb*{B}$,$\vb*{j}_\text{f}$实际上对应着不同的系统状态,那就糟糕了。
但是在经典电动力学中确实只需要$\vb*{E}$,$\vb*{B}$是仅有的场,
而如果对$\vb*{j}_\text{s}$和$\vb*{j}_\text{c}$加上足够的限制,总是可以使用$\vb*{j}_\text{f}$确定下整个$\vb*{j}$的分布,从而$\rho$的分布,
因此$\vb*{E}$,$\vb*{B}$,$\vb*{j}_\text{f}$能够完全确定系统状态,从而本构关系总是可以写出来的。

\section{线性介质假设到光学方程}

\subsection{单色波解}

现在我们认为不存在自由电流和自由电荷,从而\eqref{eq:maxwell-material}成为
\begin{equation}
    \begin{bigcase}
        \div{\vb*{D}} &= 0, \\
        \curl{\vb*{E}} &= - \pdv{\vb*{B}}{t}, \\
        \div{\vb*{B}} &= 0, \\
        \curl{\vb*{H}} &= \pdv{\vb*{D}}{t}
    \end{bigcase}
    \label{eq:no-free-charge}
\end{equation}
并且假定所有本构关系都是线性的(这里由于自由电荷被认为是零,不再需要考虑关于$\vb*{j}_\text{f}$的本构关系),
且$\vb*{D}$只和$\vb*{E}$有关,$\vb*{H}$只和$\vb*{B}$有关。
这样就可以写出一般的本构关系
\[
    \begin{aligned}
        \vb*{D}(\vb*{r}, t) &= \int K_E(\vb*{r} - \vb*{r}', t - t') \vb*{E}(\vb*{r}', t') \dd^3 \vb*{r}' \dd t', \\
        \vb*{H}(\vb*{r}, t) &= \int K_B(\vb*{r} - \vb*{r}', t - t') \vb*{B}(\vb*{r}', t') \dd^3 \vb*{r}' \dd t'
    \end{aligned}
\]
其中的$K_E$和$K_B$都是张量。我们再要求本构关系是局部的,则应有%
\footnote{在本节中$K_E$和$K_B$代表不同的常(张)量,其值在不同地方可能不同}
\[
    \begin{aligned}
        \vb*{D}(\vb*{r}, t) &= \int K_E(\vb*{r}, t-t') \vb*{E}(\vb*{r}, t') \dd t', \\
        \vb*{H}(\vb*{r}, t) &= \int K_B(\vb*{r}, t-t') \vb*{B}(\vb*{r}, t') \dd t'
    \end{aligned}
\]
这是一个时间上的卷积运算。由傅里叶变换,我们只需要讨论平面波的本构关系就可以了,因为其它所有形式的场都能够写成平面波的叠加,既然\eqref{eq:no-free-charge}是线性齐次的(这就是假定没有自由电荷的好处!)。于是,不失一般性的,假设所有的场都取
\[
    \vb*{A}(\vb*{r}) \ee^{-\ii \omega t}
\]
的形式,其中$\vb*{A}(\vb*{r})$可以有虚部,但是一定能够写成某个实矢量乘以$\ee^{\ii \phi}$的形式%
\footnote{在这个条件下,可能的$\vb*{A}$并不能覆盖整个$\complexes^3$,但附加这个条件并不会失去一般性,因为允许$\vb*{A}$有虚部是为了表示$\vb*{A}_0 \cos (\omega t + \phi)$这样的函数,而这一点在限制$\vb*{A}(\vb*{r})$取某个实矢量乘以$\ee^{\ii \phi}$的形式时足以满足。这是三维下的实函数的傅里叶变换的一个特点。}
,则只需要形如
\begin{equation}
    \vb*{D}(\vb*{r}) = K_E(\vb*{r}, \omega) \vb*{E}(\vb*{r}), \quad \vb*{H}(\vb*{r}) = K_B(\vb*{r}, \omega) \vb*{B}(\vb*{r})
    \label{eq:linear-constitutive}
\end{equation}
的本构关系就可以了。\eqref{eq:linear-constitutive}中的所有变量可以取它们完整的表达式,也可以取它们的振幅。注意这些量可以是复数(用以表示相位)。
此时的麦克斯韦方程组成为
\begin{equation}
    \begin{bigcase}
        \div{\vb*{D}} &= 0, \\
        \curl{\vb*{E}} &= \ii \omega \vb*{B}, \\
        \div{\vb*{B}} &= 0, \\
        \curl{\vb*{H}} &= - \ii \omega \vb*{D}
    \end{bigcase}
    \label{eq:sin-wave-eqs}
\end{equation}

现在转而讨论有自由电流和自由电荷的情况。我们认为自由电流正比于电场,即服从\concept{欧姆定律}。
同样不失一般性地设所有的场都可以写成一个复振幅乘以$\ee^{- \ii \omega t}$的形式。
使用导出$\vb*{D}$和$\vb*{E}$的关系、$\vb*{H}$和$\vb*{B}$的关系同样的方法,有
\begin{equation}
    \vb*{j}_\text{f}(\vb*{r}) = \sigma(\vb*{r}, \omega) \vb*{E}(\vb*{r})
    \label{eq:ohm-law-sin-wave}
\end{equation}
那么从\eqref{eq:maxwell-material}和\eqref{eq:transportation}可以得到
\[
    \begin{bigcase}
        \div{\vb*{D}} &= \rho_\text{f}, \\
        \curl{\vb*{E}} &= \ii \omega \vb*{B}, \\
        \div{\vb*{B}} &= 0, \\
        \curl{\vb*{H}} &= \vb*{j}_\text{f} - \ii \omega \vb*{D}, \\
        - \ii \omega \rho_\text{f} + \div{\vb*{j}_\text{f}} &= 0,
    \end{bigcase}
\]
我们希望能够消去$\rho_\text{f}$和$\vb*{j}_\text{f}$,为此反复使用输运方程得到
\[
    \begin{bigcase}
        \div{\left( \vb*{D} - \frac{\vb*{j}_\text{f}}{\ii \omega} \right)   } &= 0, \\
        \curl{\vb*{E}} &= \ii \omega \vb*{B}, \\
        \curl{\vb*{H}} &=  - \ii \omega \left( \vb*{D} - \frac{\vb*{j}_\text{f}}{\ii \omega} \right)
    \end{bigcase}
\]
代入本构关系\eqref{eq:ohm-law-sin-wave}得到
\[
    \begin{bigcase}
        \div{\left( \vb*{D} - \frac{\sigma \vb*{E}}{\ii \omega} \right)   } &= 0, \\
        \curl{\vb*{E}} &= \ii \omega \vb*{B}, \\
        \curl{\vb*{H}} &=  - \ii \omega \left( \vb*{D} - \frac{\sigma \vb*{E}}{\ii \omega} \right)
    \end{bigcase}
\]
注意到$\vb*{D}$和$\vb*{E}$、$\vb*{H}$和$\vb*{B}$之间的关系(见\eqref{eq:linear-constitutive}),做下面的替换
\[
    \vb*{D} - \frac{\sigma \vb*{E}}{\ii \omega} \longrightarrow \vb*{D}, \quad K_E - \frac{\sigma}{\ii \omega} \longrightarrow K_E
\]
得到的方程和\eqref{eq:sin-wave-eqs}形式完全一致,本构关系的形式也还是\eqref{eq:linear-constitutive}。
唯一的不同是,$K_E$和$K_B$都是复数,并且
\begin{equation}
    \Im K_E = \frac{\sigma}{\omega}
\end{equation}

\subsection{各向同性介质中的亥姆霍兹方程}

我们特别感兴趣的是\eqref{eq:linear-constitutive}中$K_E$和$K_B$都是标量(不一定是实数)的情况,
也就是说,\eqref{eq:linear-constitutive}中给出的响应是\concept{各向同性}的。
此时有
\begin{equation}
    \vb*{D}(\vb*{r}) = \epsilon(\vb*{r}, \omega) \vb*{E}(\vb*{r}), \quad \vb*{H}(\vb*{r}) = \frac{1}{\mu(\vb*{r}, \omega)} \vb*{B}(\vb*{r})
    \label{eq:scalar-constitutive}
\end{equation}
第二个公式的比例系数特意被放到了分母中,以达到和\eqref{eq:original-maxwell}相同的形式。
将\eqref{eq:scalar-constitutive}代入\eqref{eq:sin-wave-eqs}中,得到
\begin{equation}
    \begin{bigcase}
        \div{(\epsilon \vb*{E})} &= 0, \\
        \curl{\vb*{E}} &= \ii \omega \vb*{B}, \\
        \div{\vb*{B}} &= 0, \\
        \curl{\left(\frac{\vb*{B}}{\mu}\right)} &= - \ii \omega \epsilon \vb*{E}
    \end{bigcase}
    \label{eq:scalar-cons-maxwell-e-and-b}
\end{equation}
上式中第三式可以通过第二式推导而来;将第二式表示出的$\vb*{B}$代入第四式可以消去$\vb*{B}$。于是将\eqref{eq:scalar-cons-maxwell-e-and-b}简化为
\begin{equation}
    \begin{bigcase}
        \div{(\epsilon \vb*{E})} &= 0, \\
        \curl{ \left(\frac{1}{\mu} \curl{\vb*{E}} \right) } &= \omega^2 \epsilon \vb*{E}, \\
        \curl{\vb*{E}} &= \ii \omega \vb*{B}
    \end{bigcase}
    \label{eq:e-in-material}
\end{equation}
这个方程关于电场的部分等价于
\[
    \begin{bigcase}
        \div{(\epsilon \vb*{E})} &= 0, \\
        \laplacian \vb*{E} + \mu \epsilon \omega^2 \vb*{E} &= - \frac{1}{\epsilon} \grad{(\vb*{E} \cdot \grad{\epsilon})} - \frac{1}{\mu} \grad{\mu} \cross (\curl{\vb*{E}})
    \end{bigcase}
\]
以上方程是基于$\vb*{E}$的,也可以写出基于$\vb*{H}$的类似的方程,遵从同样的步骤,可以得到
\begin{equation}
    \begin{bigcase}
        \div{\mu \vb*{H}} &= 0, \\
        \curl{\left( \frac{1}{\epsilon} \curl{\vb*{H}} \right)} &=  \mu \omega^2 \vb*{H}, \\
        \curl{\vb*{E}} &= \ii \omega \mu \vb*{H}.
    \end{bigcase}
    \label{eq:h-in-material}
\end{equation}
\eqref{eq:e-in-material}或\eqref{eq:h-in-material}称为\concept{主方程}。

特别的,在$\epsilon, \mu$在空间中处处相等或者变化得比较缓慢时,有
\begin{equation}
    \laplacian \vb*{E} + \frac{\omega^2}{c^2} \vb*{E} = 0, \quad c^2 = \frac{1}{\epsilon \mu}
    \label{eq:halmholtz-eq}
\end{equation}
\eqref{eq:halmholtz-eq}就是所谓的\concept{亥姆霍兹方程},求解出它就求解出了整个\eqref{eq:e-in-material}。
需要注意的是并不是所有\eqref{eq:halmholtz-eq}的解都是\eqref{eq:e-in-material}的解,因为\eqref{eq:halmholtz-eq}没有包含\eqref{eq:e-in-material}的第一式。当然这个信息可以在求解\eqref{eq:halmholtz-eq}的时候使用一些边界条件加上去。
这里占用了$c$表示介质中光速,相对应地,设真空中光速为$c_0$,
其值为$1/\sqrt{\epsilon_0 \mu_0}$,因为简单地令$\epsilon=\epsilon_0$,$\mu = \mu_0$,
\eqref{eq:scalar-cons-maxwell-e-and-b}就退化到了真空的情况。

需要注意的是由于$\mu$和$\epsilon$可能有虚部(对应着吸收等情况),$c^2$不一定是实数。这就产生了一个问题:开根号在复平面上是多值的,那么$c$应该取哪一个值呢?

当$c$在某区域中基本上可以看成常的实数时,它就对应着平面波传播的速度。\eqref{eq:halmholtz-eq}有下面的平面波解
\begin{equation}
    \vb*{E} = \vb*{E}_0 \ee^{\ii(\vb*{k} \cdot \vb*{r} - \omega t)}, \quad k = \frac{\omega}{c}
    \label{eq:plane-wave}
\end{equation}
其中$\vb*{k}$是实矢量。
由空间中的傅里叶变换,\eqref{eq:halmholtz-eq}的所有解都可以写成不同$\vb*{k}$的平面波的叠加。

以上完成了关于介质中的控制方程的探讨。下面考虑界面上的衔接条件。最自然的想法,由\eqref{eq:scalar-cons-maxwell-e-and-b},可以写出自然边界条件
\[
    \begin{bigcase}
        \vb*{n} \cdot (\epsilon_i \vb*{E}_1 - \epsilon_t \vb*{E}_2) &= 0, \\
        \vb*{n} \times (\vb*{E}_1 - \vb*{E}_2) &= 0, \\
        \vb*{n} \cdot (\vb*{B}_1 - \vb*{B}_2) &= 0, \\
        \vb*{n} \times \left( \frac{\vb*{B}_1}{\mu_1} - \frac{\vb*{B}_2}{\mu_2} \right) &= 0
    \end{bigcase}
\]
但是正如\eqref{eq:scalar-cons-maxwell-e-and-b}实际上有冗余一样,上式也有方程是多余的。实际上,由\eqref{eq:e-in-material}可以得出的关于电场的相互独立的边界条件为
\begin{equation}
    \left\{\quad
        \begin{aligned}
            \epsilon_1 \vb*{n} \cdot \vb*{E}_1 = \epsilon_2 \vb*{n} \cdot \vb*{E}_2, \\
            \vb*{n} \times \vb*{E}_1 = \vb*{n} \times \vb*{E}_2, \\
            \vb*{n} \times \left( \frac{1}{\mu_1} \curl{\vb*{E}_1} \right) = \vb*{n} \times \left(\frac{1}{\mu_2} \curl{\vb*{E}_2} \right)
        \end{aligned}
    \right.
    \label{eq:e-bound-condition}
\end{equation}
只需要这些边界条件结合\eqref{eq:e-in-material}就能够定解。

\subsection{电各向异性光学介质}

现在考虑一种稍为推广的情况。很多介质——比如晶体——都具有空间上的各向异性,这是因为从不同的方向施加电场可以导致不同强度的极化。
在几乎所有常见的情况中,各向异性仅限于$\vb*{D}$和$\vb*{E}$的关系中,于是\eqref{eq:scalar-constitutive}修正为
\begin{equation}
    \vb*{D}(\vb*{r}) = \vb*{\epsilon}(\vb*{r}, \omega) \vb*{E}(\vb*{r}), \quad \vb*{H}(\vb*{r}) = \frac{1}{\mu(\vb*{r}, \omega)} \vb*{B}(\vb*{r})
    \label{eq:e-tensor-constitutive}
\end{equation}
相应的,\eqref{eq:e-in-material}修改为
\begin{equation}
    \begin{bigcase}
        \div{(\vb*{\epsilon} \cdot \vb*{E})} &= 0, \\
        \curl{ \left(\frac{1}{\mu} \curl{\vb*{E}} \right) } &= \omega^2 \vb*{\epsilon} \cdot \vb*{E}, \\
        \curl{\vb*{E}} &= \ii \omega \vb*{B}
    \end{bigcase}
    \label{eq:e-in-tensor-material}
\end{equation}

我们要研究$\mu$和$\vb*{\epsilon}$变化不大时\eqref{eq:e-in-tensor-material}的平面波解。
此时关于电场的全部方程为
\[
    \begin{aligned}
        \div{\vb*{\epsilon} \cdot \vb*{E}} = 0, \\
        \grad{(\div{\vb*{E}})} - \laplacian{\vb*{E}} = \omega^2 \mu \vb*{\epsilon} \cdot \vb*{E}
    \end{aligned}
\]
不过不难注意到第一个方程可以从第二个方程推导出来,因此方程
\begin{equation}
    \grad{(\div{\vb*{E}})} - \laplacian{\vb*{E}} = \omega^2 \mu \vb*{\epsilon} \cdot \vb*{E}
    \label{eq:anisotropy}
\end{equation}
如果$\vb*{E}$是一个波矢为$\vb*{k}$,频率为$\omega$的单色波,那么这两个方程就推出
\[
    \begin{aligned}
        - (\vb*{k} \cdot \vb*{E}) \vb*{k} + k^2 \vb*{E} = \omega^2 \mu \vb*{\epsilon} \cdot \vb*{E}, \\
    (\vb*{k} \cdot \vb*{\epsilon}) \cdot \vb*{E} = 0
    \end{aligned}
\]
但是这两个方程不是相互独立的——在第一个方程两边点乘$\vb*{k}$就可以推出第二个。
顺带提一句:第二个方程表明,在介质是电各向异性的情况下,波矢$\vb*{k}$并不垂直于电场,而是垂直于$\vb*{D}$。

那么电各向异性光学介质中的平面波就完全地被下式描写:
\[
    k^2 \vb*{E} - (\vb*{k} \cdot \vb*{E}) \vb*{k} - \omega^2 \mu \vb*{\epsilon} \cdot \vb*{E} = 0
\]
也就是说
\begin{equation}
    \left( k^2 \vb*{\delta} - \vb*{k} \vb*{k} - \omega^2 \mu \vb*{\epsilon} \right) \cdot \vb*{E} = 0
    \label{eq:plain-wave-in-anistrophy}
\end{equation}
其中$\vb*{\delta}$为单位张量。

\eqref{eq:plain-wave-in-anistrophy}将电场的方向和波矢的方向联系了起来。要观察波矢自己要满足什么条件,只需要求解
\begin{equation}
    \det \left( k^2 \vb*{\delta} - \vb*{k} \vb*{k} - \omega^2 \mu \vb*{\epsilon} \right) = 0
    \label{eq:k-det}
\end{equation}
即可。

直接处理矢量形式的方程会比较棘手。但是注意到由于能量守恒的原因,$\vb*{D}$不可能和$\vb*{E}$反向,也就是
\[
    \vb*{D} \cdot \vb*{E} = \vb*{E} \cdot \vb*{\epsilon} \cdot \vb*{E} > 0
\]
那么我们可以确定$\vb*{\epsilon}$是正定的(TODO:有虚部时怎么办?)另一方面,晶体的空间对称性意味着$\vb*{\epsilon}$一定可以对角化,因此我们总是可以找到一个坐标系(未必是正交坐标系),在其中$\vb*{\epsilon}$被对角化,且对角元均为正数。
这个坐标系就是所谓的\concept{主轴系}。

\section{介质中的能流}

为了讨论能量,这一节我们再次引入自由电荷的概念,用于讨论场中有实物粒子的情况。(如果实物粒子不带电荷,那么它就不参与电磁相互作用,此时不方便从牛顿力学出发讨论能量)
我们只追踪自由电荷的能量,而不追踪所有电荷的能量。
于是,使用\eqref{eq:maxwell-material}以及洛伦兹力公式
\[
    \vb*{f} = q \vb*{E} + q \vb*{v} \times \vb*{B}
\]
可以得到
\[
    \vb*{f} \cdot \vb*{v} = - \div{(\vb*{E} \times \vb*{H})} - \vb*{H} \cdot \pdv{\vb*{B}}{t} - \vb*{E} \cdot \pdv{\vb*{D}}{t}
\]
则可以选取最简单的电磁能量密度和电磁能流为
\begin{equation}
    \begin{bigcase}
        \vb*{S} &= \vb*{E} \times \vb*{H}, \\
        \pdv{w}{t} &= \vb*{E} \cdot \pdv{\vb*{D}}{t} + \vb*{H} \cdot \pdv{\vb*{B}}{t}
    \end{bigcase}
    \label{eq:energy-in-material}
\end{equation}
其中的$\vb*{S}$就是\concept{坡印廷矢量}。公式\eqref{eq:energy-in-material}的导出没有使用线性介质假设。

在已知$\vb*{D}$和$\vb*{E}$的关系线性、$\vb*{H}$和$\vb*{B}$的关系也是线性的情况下,有
\begin{equation}
    w = \frac{1}{2} (\vb*{E} \cdot \vb*{D} + \vb*{H} \cdot \vb*{B})
    \label{eq:energy-density-linear-material}
\end{equation}

需要注意的是\eqref{eq:energy-in-material}和\eqref{eq:energy-density-linear-material}都不是线性的,
因此在这些公式中不能让各个场有虚部。

在稳态时各个场做正弦变化,这也是本文关注的主要场景。首先讨论介质中只有一个单色波的情况。不失一般性地认为$\vb*{E}$的相位零点在$t=0$处,则有
\[
    \vb*{S} = \vb*{E}_0 \times \vb*{H}_0 \cos(\omega t) \cos(\omega t + \phi)
\]
其中$\phi$是两者的相位差,$\vb*{E}_0, \vb*{H}_0$分别是振幅,这三者均只有实部。我们更加关心一段时间内的平均能流,则有
\[
    \expval*{\vb*{S}} = \frac{1}{2} \vb*{E}_0 \times \vb*{H}_0 \cos \phi
\]
现在考虑含有虚部的$\vb*{E}, \vb*{H}$表达式
\[
    \vb*{E} = \vb*{E}_0 \ee^{\ii \omega t}, \vb*{H} = \vb*{H}_0 \ee^{\ii \phi} \ee^{\ii \omega t}
\]
我们会发现一个有趣的结果:
\[
    \Re \vb*{E}^* \times \vb*{H} = \expval*{\vb*{S}}
\]
直观地看,可以将$\vb*{E}^* \times \vb*{H}$的实部看成“有功分量”,虚部看作“无功分量”。
这样可以定义\concept{实数版本的辐照度}
\begin{equation}
    \vb*{I} = \expval*{\vb*{S}} = \frac{1}{2} \vb*{E}_0 \times \vb*{H}_0 \cos \phi
    \label{eq:radiation-real}
\end{equation}
也可以定义\concept{复数版本的辐照度}
\begin{equation}
    \vb*{I} = \frac{1}{2} \Re \vb*{E}^* \times \vb*{H}
    \label{eq:radiation-complex}
\end{equation}
\eqref{eq:radiation-complex}的实部就是\eqref{eq:radiation-real}。

接着再讨论平面波的能量密度。由于
% \[
%     \vb*{k} \cdot \vb*{B} = 0, \quad \vb*{D} = - \frac{\vb*{k}}{\omega} \cross \vb*{H},
% \]
% 我们有
% \[
%     D^2 = \frac{k^2}{\mu^2 \omega^2} B^2
% \]
% 则
% \[
%     \frac{1}{2} \vb*{H} \cdot \vb*{B} = \frac{\mu \omega^2}{2 k^2} D^2, 
% \]
% \[
%     \frac{1}{2} \vb*{E} \cdot \vb*{D} = \frac{1}{2} \vb*{E} \cdot \vb*{\epsilon} \cdot \vb*{E}
% \]
% 于是
% \begin{equation}
%     w = w_E + w_B = \frac{1}{2} \vb*{E} \cdot \vb*{\epsilon} \cdot \vb*{E} + \frac{\mu \omega^2}{2 k^2} D^2
%     \label{eq:energy-density}
% \end{equation}
\[
    w = w_E + w_B,
\]
分别化简两项,有
\[
    w_E = \frac{1}{2} \vb*{E} \cdot \vb*{D} 
        = \frac{1}{2} \vb*{E} \cdot \left( - \frac{\vb*{k}}{\omega} \times \vb*{H} \right) 
        = \frac{1}{2} \frac{\vb*{k}}{\omega} \cdot (\vb*{E} \times \vb*{H}),
\]
且
\[
    w_B = \frac{1}{2} \vb*{H} \cdot \vb*{B} 
        = \frac{1}{2} \vb*{H} \cdot \left( \frac{\vb*{k}}{\omega} \cdot \vb*{E} \right) 
        = \frac{1}{2} \frac{\vb*{k}}{\omega} \cdot (\vb*{E} \times \vb*{H}),
\]
因此电场能和磁场能各占总能量的一半,且
\begin{equation}
    w = \frac{\vb*{k}}{\omega} \cdot (\vb*{E} \times \vb*{H}) = \frac{k S}{\omega} \cos \alpha
    \label{eq:energy-density-and-s}
\end{equation}
其中$\alpha$是$\vb*{k}$和$\vb*{S}$的夹角。
相应的,能量传递的速度为
\begin{equation}
    \vb*{v} = \frac{\vb*{S}}{w} = \frac{\omega}{k}  \cos \alpha \vu*{S}
\end{equation}


\chapter{线性介质的光学性能的微观原理}

\section{线性极化}

\subsection{经典谐振子模型}

\subsubsection{洛伦兹模型}

我们现在转而考虑存在阻尼的谐振子,有
\begin{equation}
    m \ddot{\vb*{r}} = - m \gamma \dot {\vb*{r}} + q (\vb*{E} + \vb*{v} \times \vb*{B}),
\end{equation}
以$\vb*{B}$的指向为$z$轴,则$\vb*{E}$一定在$xy$平面内,则上式成为% TODO
\begin{equation}
    v_x = \frac{q}{m} \tau E_x + \omega_\text{L} \tau v_y,
\end{equation}
其中
\begin{equation}
    \omega_\text{L} = \frac{qB}{m}
\end{equation}
为\concept{拉莫频率},而
\begin{equation}
    \tau = \frac{1}{\gamma}.
\end{equation}

\chapter{均匀线性介质中的电磁波}

\section{各向同性线性介质中的平面波}\label{sec:light-propagate}

光无非是成束的电磁场,因此接下来可以通过求解\eqref{eq:e-in-material}获得光的传播情况。
在光学中常见的介质或者是性能变化比较均匀的,或者是性能变化非常剧烈的(也就是介质界面附近,例如水和空间交界处)。
前者中的

\subsection{均匀介质内部光的传播}\label{sec:in-interior-uniform}

在均匀介质内部光的传播情况由\eqref{eq:halmholtz-eq}控制。
当介质内部的$\epsilon$和$\mu$都是实数时可以直接使用平面波解\eqref{eq:plane-wave}。而当$\epsilon$和$\mu$含有虚部——也就是说,介质含有吸收等性质——那么根据解析延拓的原理,\eqref{eq:plane-wave}的形式应该得到保持,但是$\vb*{k}$需要有虚部。%
\footnote{从积分变换的角度来看,$\epsilon\mu$含有虚部意味着有关的方程难以直接通过傅里叶展开化简,因为傅里叶展开在实函数上比较简单。为此我们将$\vb*{k}$推广到复数的情况,实际上就是从空间傅里叶变换推广到了空间拉普拉斯变换。}
实际上,即使是$\epsilon$和$\mu$都取实数值的时候,\eqref{eq:plane-wave}中的$\vb*{k}$也可以有虚部。当然,此时的电场不再是一个基本解了,因为它可以使用若干个实数$\vb*{k}$的真正的平面波叠加出来。

在最一般的、$\epsilon$和$\mu$是否有虚部不知道的情况下,将\eqref{eq:plane-wave}代入\eqref{eq:halmholtz-eq}得到
\[
    - (\Re \vb*{k} + \Im \vb*{k})^2 + \omega^2 \epsilon\mu = 0
\]
化简,由于交叉项是纯虚数而其他项都是纯实数,可得
\[
    \begin{bigcase}
        (\Re \vb*{k})^2 - (\Im \vb*{k})^2 &= \omega^2\Re \epsilon\mu, \\
        2 (\Re \vb*{k}) \cdot (\Im \vb*{\vb*{k}}) &= \omega^2\Im \epsilon\mu.
    \end{bigcase}
\]
因此,在一个$\epsilon$和$\mu$都没有虚部的介质中,有
\[
    (\Re \vb*{k}) \cdot (\Im \vb*{\vb*{k}}) = 0
\]
还有\eqref{eq:e-in-material}的第一式没有使用。它意味着
\[
    \vb*{k} \cdot \vb*{E} = (\Re \vb*{k} + \ii \Im \vb*{k}) \cdot \vb*{E} = 0
\]

这样一来我们得出结论:在均匀的、可能有吸收等因素,从而使$c^2$有虚部的介质中,有形如下式的解:
\begin{equation}
    \left\{
        \begin{aligned}
            &\vb*{E} = \vb*{E}_0 \ee^{\ii (\vb*{k} \cdot \vb*{r} - \omega t)}, \\
            &(\Re \vb*{k})^2 - (\Im \vb*{k})^2 = \omega^2 \Re \epsilon\mu, \\
            &2 (\Re \vb*{k}) \cdot (\Im \vb*{\vb*{k}}) = \omega^2 \Im \epsilon\mu, \\
            &\vb*{k} \cdot \vb*{E} = 0
        \end{aligned}
    \right.
    \label{eq:uniform-wave}
\end{equation}

在$\epsilon\mu$完全就是实数的时候\eqref{eq:uniform-wave}意味着$\vb*{k}$的实部、虚部相互正交(电场的方向和$\vb*{k}$的实部、虚部没有特别明显的关系)。
这种情况只有可能发生在界面附近,并且要求$\Re \vb*{k}$平行于界面而$\Im \vb*{k}$要垂直于界面且方向从界面指向介质内部,
否则电场会在远处发散到无穷大,不是物理解。%
\footnote{关于物理解有必要说明这一点:实际上,平面波本身也不是真正具有物理意义的波,因为不可能让电场充满整个空间,且同时只具有一个频率。
然而,实际的光的分布满足的特定边界条件意味着我们确实没有必要讨论所有可能的平面波——只需要讨论满足这些边界条件的平面波即可。
例如,光通常是从一个光源打出来的,这就暗示了两个边界条件:
首先,某个平面(也就是光源所在的平面)上的场强是给定的,需要计算的是这个平面某一侧的场强,另一侧的场强没有意义(因为它在光源之后);
其次,光源两侧无穷远处的场强应趋于零。
这就意味着,组合成这个光的平面波在垂直于$\Re \vb*{k}$的方向上不能发散,在与$\Re\vb*{k}$相反的方向上可以发散,在与$\Re\vb*{k}$同向的方向上不能发散。
这就是我们使用的“物理解”的条件。
}
我们将在\ref{sec:total-reflect}节看到这种解的一个例子。
而当$\epsilon\mu$含有虚部时,可以做下面的分解:
\[
    \Im \vb*{k} = \vb*{k}_\parallel + \vb*{k}_\bot
\]
使$\vb*{k}_\parallel$平行于$\Re\vb*{k}$,$\vb*{k}_\bot$垂直于$\Re\vb*{k}$。
这两个分量当然都会让电场在远处发散到无穷大,但是物理解的要求意味着$\vb*{k}_\parallel$与$\Re \vb*{k}$同向,
而如果$\vb*{k}_\bot$非零,那么必定有$\Re \vb*{k}$(从而$\vb*{k}_\parallel$)平行于界面而$\vb*{k}_\bot$要垂直于界面且方向从界面指向介质内部。

既然$\vb*{k}_\parallel$和$\Re \vb*{k}$总是同向,我们有理由认为这两者可以看成同一个对象的实部和虚部。
因此,在\ref{sec:light-propagate}节的剩下部分,我们用$\vb*{k}$代替原本的$\Re \vb*{k} + \ii \vb*{k}_\parallel$,
而使用$\vb*{\beta}$代替原本的$\vb*{k}_\bot$,这样\eqref{eq:uniform-wave}就需要改写为
\begin{equation}
    \left\{\quad
        \begin{aligned}
            \vb*{E} = \vb*{E}_0 \ee^{- \vb*{\beta} \cdot \vb*{r}} \ee^{\ii (\vb*{k} \cdot \vb*{r} - \omega t)}, \\
            \vb*{k}^2 - \vb*{\beta}^2 = \mu\epsilon\omega^2, \quad \vb*{k} \cdot \vb*{\beta} = 0, \\
            (\vb*{k} + \ii \vb*{\beta}) \cdot \vb*{E} = 0
        \end{aligned}
    \right.
    \label{eq:beta-k-uniform-wave}
\end{equation}
其中$\vb*{k}$与波前传播方向同向。$\vb*{k}$,$\vb*{\beta}$,$\vb*{E}$三者相互垂直,$\vb*{k}$可以有虚部而$\vb*{\beta}$没有。
无论$\vb*{k}$有没有虚部,它都可以写成一个复数乘以一个实矢量的形式,因此可以非常良好地定义$\vb*{k}$方向上的单位矢量$\hat{\vb*{k}}$。
反之,在\eqref{eq:uniform-wave}中,由于没有将$\vb*{k}$做适当的分解,可能难以良定义一个单位矢量,因为此时不同基向量上的分量可能具有不成比例的实部和虚部。

同样,为了让解是物理解,应当有$\vb*{\beta}$垂直于某个界面且方向从界面指向介质内部,否则解会发散。

特别的,在$\vb*{\beta}$为零的时候,我们有
\begin{equation}
    \left\{
        \begin{aligned}
            \vb*{E} = \vb*{E}_0 \ee^{\ii(\vb*{k} \cdot \vb*{r} - \omega t)}, \\
            \vb*{k}^2 = \mu\epsilon \omega^2, \\
            \vb*{k} \cdot \vb*{E} = 0.
        \end{aligned}
    \right.
    \label{eq:beta-zero-uniform-wave}
\end{equation}

关于$\vb*{k}$的取值需要额外的注记。我们设$\vb*{k}=k \hat{\vb*{k}}$,其中$\hat{\vb*{k}}$正是\eqref{eq:uniform-wave}中的那个$\Re\vb*{k}$(和\eqref{eq:beta-k-uniform-wave}中的$\vb*{k}$同向但是不相同)的单位向量。
那么从\eqref{eq:beta-k-uniform-wave}就可以得出
\begin{equation}
    k^2 - \beta^2 = \mu \epsilon \omega^2
\end{equation}
其中$\beta$的取值通常和边界条件有关,确定了$\beta$就能够得到$k$。
但是还有一个额外的问题:$\mu \epsilon$含有虚部,因此下面的表达式
\[
    k = \sqrt{\beta^2 + \mu \epsilon \omega^2}
\]
就是多值的。具体取哪一个值需要根据物理条件确定。在能够确定具体取那个值的时候,我们规定\concept{折射率}为
\begin{equation}
    n(\omega) = c_0 \sqrt{\epsilon(\omega) \mu(\omega)}, 
    \label{eq:refractivity}
\end{equation}
这时就有
\begin{equation}
    \quad \frac{\omega}{k} = \frac{c_0}{n} \equiv c
    \label{eq:k-and-omega-and-n}
\end{equation}
在$n$没有虚部时$c$就是介质中\eqref{eq:beta-k-uniform-wave}的波前传播速度。

\subsection{两种透明均匀介质界面上的折射和反射}\label{sec:two-isotrophy-surface}

所谓透明介质指的是折射率完全为实数的介质。在这种介质中,
如图\ref{fig:ray-onto-flat-surface}所示,我们假定有两个几乎无限大的介质以一个完全平坦的界面隔开。
\begin{figure}
    \centering
    \begin{tikzpicture}
        % 两种介质
        \node[above] at (-4, 2) {$n_i$};
        \node[above] at (-4, -3) {$n_t$};
        % 介质界面
        \draw (-4,0) -- (4,0);
        \node[above] at (-3.5,0) {\small 介质界面};
        % 法线
        \draw[dash pattern=on5pt off3pt] (0,4) -- (0,-4);
        % 入射光线
        \draw[ray] (130:5.2) -- (0,0);
        \node[] at (-2,3) {$\vb*{k}_i$};
        \draw (0,1) arc (90:130:1);
        \node[] at (110:1.4) {$\theta_i$};
    \end{tikzpicture}
    \caption{光入射平整表面}
    \label{fig:ray-onto-flat-surface}
\end{figure}
现在让一束光照射到这个界面上。为了方便起见,假定在我们感兴趣的尺度内光强\concept{处处相同},也就是入射光形式为
\[
    \vb*{E}_i = \vb*{E}_{i0} \ee^{\ii(\vb*{k}_i \cdot \vb*{r} - \omega t)}
\]
(图\ref{fig:ray-onto-flat-surface}中的实线只是代表波的传播方向,不代表光束。)

\begin{figure}
    \centering
    \begin{tikzpicture}
        \draw [-{Stealth}] (-1, 1) -- (1, -1);
        \node[above] at (0,-1) {$\vb*{k}$};
        \draw [-{Stealth}, thick] (0.5,0.5) -- (1.5,1.5);
        \node at (0.4,1) {$\vb*{E}_p$};
        \node at (1.5, 0) {$\odot$};
        \node at (2, 0) {$\vb*{E}_s$};
    \end{tikzpicture}
    \caption{电场的分解}
    \label{fig:decomposition-of-e}
\end{figure}

同时为了将矢量方程\eqref{eq:e-bound-condition}标量化,将电场按照图\ref{fig:decomposition-of-e}的方式做分解。
我们没有给出平行于$\vb*{k}$的电场分量,因为按照\eqref{eq:beta-k-uniform-wave},这个电场分量不存在。

\subsubsection{有透射光的情况}

\begin{figure}
    \centering
    \begin{tikzpicture}
        % 两种介质
        \node[above] at (-4, 2) {$n_i$};
        \node[above] at (-4, -3) {$n_t$};
        % 介质界面
        \draw (-4,0) -- (4,0);
        \node[above] at (-3.5,0) {\small 介质界面};
        % 法线
        \draw[dash pattern=on5pt off3pt] (0,4) -- (0,-4);
        \draw [-{Stealth}, thick] (0,0) -- (0,2);
        \node [above] at (0.3, 2.1) {$\vb*{n}$};
        % 入射光线
        \draw[ray] (130:5.2) -- (0,0);
        \node[] at (-2,3) {$\vb*{k}_i$};
        \draw (0,1) arc (90:130:1);
        \node[] at (110:1.4) {$\theta_i$};
        % 反射光线
        \draw[ray] (0,0) -- (50:5.2) ;
        \node[] at (2,3) {$\vb*{k}_r$};
        \draw (0,1.2) arc (90:50:1.2);
        \node[] at (70:1.4) {$\theta_r$};
        % 折射光线
        \draw[ray] (0,0) -- (-70:4.5);
        \node[] at (1.5,-3) {$\vb*{k}_t$};
        \draw (0,-1.1) arc (-90:-70:1.1);
        \node[] at (-80:1.4) {$\theta_t$};
    \end{tikzpicture}
    \caption{光入射平整表面之后发生反射和折射}
    \label{fig:ray-refraction}
\end{figure}

凭借经验,我们会认为入射波$\vb*{E}_i$会导致两个平面波,分别称为\concept{反射波}和\concept{折射波},图示为图\ref{fig:ray-refraction},形式如下:
\[
    \begin{aligned}
        \vb*{E}_r &= \vb*{E}_{r0} \ee^{\ii(\vb*{k}_r \cdot \vb*{r} - \omega t)}, \\
        \vb*{E}_t &= \vb*{E}_{t0} \ee^{\ii(\vb*{k}_t \cdot \vb*{r} - \omega t)}
    \end{aligned}
\]

这样在介质1中和在介质2中分别有
\[
    \begin{aligned}
        \vb*{E}_1 = \vb*{E}_{i0} \ee^{\ii(\vb*{k}_i \cdot \vb*{r} - \omega t)} +  \vb*{E}_{r0} \ee^{\ii(\vb*{k}_r \cdot \vb*{r} - \omega t)}, \\
        \vb*{E}_2 = \vb*{E}_{t0} \ee^{\ii(\vb*{k}_t \cdot \vb*{r} - \omega t)}
    \end{aligned}
\]
使用\eqref{eq:e-bound-condition},可以得到形如下式的结果:
\[
    \text{something } \ee^{\ii(\vb*{k}_i \cdot \vb*{r} - \omega t)} + \text{something } \ee^{\ii(\vb*{k}_r \cdot \vb*{r} - \omega t)} = \text{something } \ee^{\ii(\vb*{k}_t \cdot \vb*{r} - \omega t)}
\]
从而必须有
\[
    (\vb*{k}_i - \vb*{k}_r) \cdot \vb*{r} = \const, \quad (\vb*{k}_i - \vb*{k}_t) \cdot \vb*{r} = \const
\]
考虑到这些方程描写了一个平面,而它们都在界面(一个平面)上恒成立,有%
\footnote{能够导出这个结果是因为$\ee^{\ii \vb*{k} \cdot \vb*{r}}$是函数基。上面的方程在界面上恒成立,因此在界面上$\vb*{k}_i$,$\vb*{k}_r$,$\vb*{k}_t$的投影,也就是$\vb*{n} \times \vb*{k}$都一样。}
\begin{equation}
    \vb*{n} \times (\vb*{k}_i - \vb*{k}_r) = \vb*{n} \times (\vb*{k}_i - \vb*{k}_t) = 0
    \label{eq:k-and-n}
\end{equation}

假设入射波、反射波、折射波\concept{它们自己}都能够成为\eqref{eq:e-in-material}的解。
考虑到介质被认为是均匀的,假设这三个波都能够写成\eqref{eq:uniform-wave}的形式。
进一步,假设这三个波均没有\eqref{eq:beta-k-uniform-wave}中的$\vb*{\beta}$。
这些假设是太多了,我们将发现它们只能够在一部分场景下成立。

在这种情况下,三个波的$\vb*{k}$都可以写成一个复数乘以一个实矢量的形式,因此可以良好地定义“波前前进的方向”。
于是就有
\[
    \vb*{k}_i = k_i \hat{\vb*{k}}_i, \; \vb*{k}_r = k_r \hat{\vb*{k}}_r, \; \vb*{k}_t = k_t \hat{\vb*{k}}_t
\]
由于入射波和反射波都位于介质1中,我们确定$k_i=k_r$。

按照图\ref{fig:ray-refraction}中展示的方式标记角度。
(注意$\vb*{k}_i$和法向量的夹角为$\pi-\theta_i$)
\eqref{eq:k-and-n}可以写作
\[
    k_i \vb*{n} \times \vu*{k}_i = k_r \vb*{n} \times \vu*{k}_r, \quad k_i \vb*{n} \times \vu*{k}_i = k_t \vb*{n} \times \vu*{k}_t
\]

\begin{equation}
    \theta_i = \theta_r, \quad \frac{\sin \theta_i}{\sin \theta_t} = \frac{n_t}{n_i}
    \label{eq:snell}
\end{equation}

最终得到\concept{菲涅尔公式}
\begin{equation}
    \begin{bigcase}
        r_\bot &= \frac{E_{rs}}{E_{is}} = 
        \frac{\frac{n_i}{\mu_i} \cos \theta_i - \frac{n_t}{\mu_t} \cos \theta_t}{\frac{n_i}{\mu_i} \cos \theta_i + \frac{n_t}{\mu_t} \cos \theta_t}, \\
        t_\bot &= \frac{E_{ts}}{E_{is}} = 
        \frac{2 \frac{n_i}{\mu_i} \cos \theta_i}{\frac{n_i}{\mu_i} \cos \theta_i + \frac{n_t}{\mu_t} \cos \theta_t}, \\
        r_\parallel &= \frac{E_{rp}}{E_{ip}} = 
        \frac{\frac{n_t}{\mu_t} \cos \theta_i - \frac{n_i}{\mu_i} \cos \theta_t}{\frac{n_i}{\mu_i} \cos \theta_t + \frac{n_t}{\mu_t} \cos \theta_i}, \\
        t_\parallel &= \frac{E_{tp}}{E_{ip}} =
        \frac{2 \frac{n_i}{\mu_i} \cos \theta_i}{\frac{n_i}{\mu_i} \cos \theta_t + \frac{n_t}{\mu_t} \cos \theta_i}
    \end{bigcase}
    \label{eq:fresnel-formulas}
\end{equation}

当$\theta_i \to 0$,也就是入射光垂直于界面时,我们有
\begin{equation}
    r_\bot = \frac{n_i / \mu_i - n_t / \mu_t}{n_i / \mu_i + n_t / \mu_t}, \quad 
    r_\parallel = \frac{n_t / \mu_t - n_i / \mu_i}{n_i / \mu_i + n_t / \mu_t}
\end{equation}
看起来这样很奇怪,因为此时不能区分p光和s光,而两者的反射系数却差了一个负号。
但实际上这是错觉。回顾图\ref{fig:decomposition-of-e},我们会发现入射和反射的p光的基矢量在$\theta \to 0$时是反向的,而入射和反射s光的基矢量在$\theta \to 0$时同向,因此反射系数就应该差一个负号——最后无论将入射光当成p光还是当成s光,都能够得到同样的反射光矢量,于是可以写出
\begin{equation}
    \vb*{E}_r = \frac{n_i / \mu_i - n_t / \mu_t}{n_i / \mu_i + n_t / \mu_t} \vb*{E}_i
\end{equation}

关于能量有下面的等式:
\begin{equation}
    T = \left( \frac{n_t \cos \theta_t / \mu_t}{n_i \cos \theta_i / \mu_i} \right) t^2, \quad R = r^2
\end{equation}

\subsubsection{全反射}\label{sec:total-reflect}

TODO:隐逝波的方向不完全

当$n_i > n_t$而
\[
    \theta_i > \arcsin \frac{n_t}{n_i}
\]
时,先前做的“入射波产生反射波和折射波,这三个波都是平面波”的假设就失效了,因为这个假设导致方程组\eqref{eq:e-bound-condition}无解。
为了在这种情况下求解\eqref{eq:e-bound-condition},尝试放松一个假设。
实验上,在全反射发生的时候没有观察到折射波,因此假定折射波在边界处快速衰减了。因此此时我们假定反射波、折射波采取下面的形式:
\[
    \begin{aligned}
        \vb*{E}_r &= \vb*{E}_{r0} \ee^{\ii (\vb*{k}_r \cdot \vb*{r} - \omega t)}, \\
        \vb*{E}_t &= \vb*{E}_{t0} \ee^{-\vb*{\beta} \cdot \vb*{r}} \ee^{\ii (\vb*{k}_t \cdot \vb*{r} - \omega t)}
    \end{aligned}
\]
其中$\vb*{\beta}$垂直于界面。所有的参量都是实的。

此时在边界上会有类似于这样的表达式:
\[
    \mathrm{something} \; \ee^{\ii (\vb*{k}_r \cdot \vb*{r} - \omega t)} + \mathrm{something} \; \ee^{\ii (\vb*{k}_r \cdot \vb*{r} - \omega t)} = \mathrm{something} \; \ee^{-\vb*{\beta} \cdot \vb*{r}} \ee^{\ii (\vb*{k}_t \cdot \vb*{r} - \omega t)}
\]

\subsubsection{公式形式的统一处理}

\subsection{各向同性介质中平面波的能量}

由于在各向同性介质中

\section{各向异性线性介质中的平面波}

以上讨论的都是各向同性介质中光的传播,也就是说,$\vb*{D}$和$\vb*{E}$、$\vb*{H}$和$\vb*{B}$之间的联系都是标量倍数。
现在稍微放松这个假设,假定$\vb*{E}$和之间的联系是一个张量,也就是说,
\[
    \vb*{D} = \vb*{\epsilon} \cdot \vb*{E}
\]
其中$\vb*{\epsilon}$是一个张量,或者也可以写成
\[
    D_i = \epsilon_{ij} E_j
\]

\subsection{单光轴透明介质内部的平面波}\label{sec:one-axis-transparent}

首先讨论一种最简单的情况。此时介质有一个对称轴,绕着这个对称轴有旋转不变性。
这就意味着$\vb*{\epsilon}$有两个特征值,其中一个对应着一个唯一的特征向量,另一个对应着两个特征向量,
且后面两个特征向量垂直于前一个特征向量(否则不能保证旋转不变性)。
% TODO:群论
此时可以将$\vb*{\epsilon}$正交对角化。
于是,可以找到一个直角坐标系$x,y,z$,使$z$方向对应着前一个特征向量,$x,y$方向对应着后两个特征向量(需要对它们做正交化)
此时没有必要区分逆变协变,可以直接写出矩阵形式
\begin{equation}
    [\mu \epsilon_{ij}]_{ij} = \frac{1}{c_0^2} \bmqty{\dmat{n_o^2,n_o^2,n_e^2}}
    \label{eq:one-axis-matrix}
\end{equation}
其中$n_e, n_o > 0$。我们下这个断言是因为介质对外加电场产生的响应不可能使总电场和外加电场方向相反,也就是说,如果$\vb*{E}$取某个适当的方向使$\vb*{E}$和$\vb*{D}$之间只差一个标量倍数,那么这个标量倍数一定大于零,因此这个倍数一定有正的平方根,从而,$\vb*{\epsilon}$的特征值一定大于零。

此时\eqref{eq:k-det}等价于
\[
    \mdet{
        k_y^2 + k_z^2 - \frac{\omega^2}{c_0^2} n_o^2 & -k_x k_y & - k_x k_z \\
        - k_x k_y & k_x^2 + k_z^2 - \frac{\omega^2}{c_0^2} n_o^2 & -k_y k_z \\
        -k_x k_z & -k_y k_z & k_x^2 + k_y^2 - \frac{\omega^2}{c_0^2} n_e^2
    } = 0
\]
化简得到
\begin{equation}
    \left( \frac{k_x^2}{n_o^2} + \frac{k_y^2}{n_o^2} + \frac{k_z^2}{n_o^2} - \frac{\omega^2}{c_0^2} \right) \left( \frac{k_x^2}{n_e^2} + \frac{k_y^2}{n_e^2} + \frac{k_z^2}{n_o^2} - \frac{\omega^2}{c_0^2} \right) = 0
    \label{eq:uniaxial-crystal}
\end{equation}
因此,$\vb*{k}$只需要让其中的一个因式为零,就是可能的解。

下面讨论\eqref{eq:uniaxial-crystal}的两个解。第一种情况是
\[
    \frac{k_x^2}{n_o^2} + \frac{k_y^2}{n_o^2} + \frac{k_z^2}{n_o^2} - \frac{\omega^2}{c_0^2} = 0
\]
我们称此时的波为\concept{o光}。o光的波矢需要且只需要满足
\[
    \abs{\vb*{k}_o} = \frac{n_o \omega}{c_0}
\]
o光的电场方向需要满足什么条件?可以将\eqref{eq:plain-wave-in-anistrophy}写成
\[
    \bmqty{
        k_y^2 + k_z^2 - \frac{\omega^2}{c_0^2} n_e^2 & -k_x k_y & - k_x k_z \\
        - k_x k_y & k_x^2 + k_z^2 - \frac{\omega^2}{c_0^2} n_e^2 & -k_y k_z \\
        -k_x k_z & -k_y k_z & k_x^2 + k_y^2 - \frac{\omega^2}{c_0^2} n_o^2
    }
    \bmqty{
        E_x \\ E_y \\ E_z
    } = 0
\]
在球坐标系中写出
\[
    k_x = \frac{n_o \omega}{c_0} \sin \varphi \cos \theta, 
    \quad k_y = \frac{n_o \omega}{c_0} \sin \varphi \sin \theta, \quad k_z = \frac{n_o \omega}{c_0} \cos \varphi
\]
然后代入上面的矩阵表达式,经过一系列初等变换得到
\[
    \bmqty{
        \sin \varphi \cos \theta & \sin \varphi \sin \theta & \cos \varphi \\
        0 & 0 & 1
    }
    \bmqty{E_x \\ E_y \\ E_z} = 0
\]
于是得到了o光需要(且只需要)满足的条件:
\begin{equation}
    \abs{\vb*{k}_o} = \frac{n_o \omega}{c_0}, \quad\vb*{e}_z \cdot \vb*{E}_o = \vb*{k}_o \cdot \vb*{E}_o = 0
    \label{eq:o-light}
\end{equation}
因此,除了o光永远不会有平行于光轴的分量以外,o光在介质中的传播方式和各向同性介质中的光完全一样:$\vb*{k}$和$\vb*{E}, \vb*{D}$均垂直(并且可以验证$\vb*{D}$和$\vb*{E}$平行),且$k$和$\omega$之间的关系就是普通的折射率确定的关系。

第二种情况是
\[
    \frac{k_x^2}{n_e^2} + \frac{k_y^2}{n_e^2} + \frac{k_z^2}{n_o^2} - \frac{\omega^2}{c_0^2} = 0
\]
称此时的波为\concept{e光}。使用和上面相同的方法,写出椭球坐标之下的$\vb*{k}_e$表达式
\[
    k_x = \frac{n_e \omega}{c_0} \sin \theta \cos \varphi, \quad k_y = \frac{n_e \omega}{c_0} \sin \theta \sin \varphi, \quad k_z = \frac{n_o \omega}{c_0} \cos \theta
\]
将\eqref{eq:plain-wave-in-anistrophy}写成矩阵形式之后代入椭球坐标下的$\vb*{k}_e$表达式,然后做初等变换,得到
\[
    \bmqty{
        n_0 \sin \theta \cos \varphi & n_0 \sin \theta \sin \varphi & n_e \cos \theta \\
        -\sin \varphi & \cos \varphi & 0
    }
    \bmqty{E_x \\ E_y \\ E_z} = 0
\]
这是一个有两个方程组成的方程组。
容易验证,在本坐标系中$\bmqty{n_0 \sin \theta \cos \varphi & n_0 \sin \theta \sin \varphi & n_e \cos \theta}$
和$\vb*{k} \cdot \vb*{\epsilon}$共线,$\bmqty{-\sin \varphi & \cos \varphi & 0}$与$\vb*{e}_z \times \vb*{k}$共线。
因此e光需要且只需要满足的方程为
\begin{equation}
    \frac{k_{ex}^2}{n_e^2} + \frac{k_{ey}^2}{n_e^2} + \frac{k_{ez}^2}{n_o^2} =
    \frac{\omega^2}{c_0^2}, \quad \vb*{k}_e \cdot \vb*{\epsilon} \cdot \vb*{E}_e = 0, \quad (\vb*{e}_z \times \vb*{k}_e) \cdot \vb*{E}_e = 0
    \label{eq:e-light}
\end{equation}
因此,e光的振动方向被限制在了光轴和波矢确定的平面上。
此时通过$k$和$\omega$之间的关系仍然可以定义等效的折射率,它是
\begin{equation}
    n = \frac{c_0 \abs*{\vb*{k}_e}}{\omega} , \quad \frac{1}{n^2} = \frac{\sin^2 \theta}{n_e^2} + \frac{\cos^2 \theta}{n_o^2}.
    \label{eq:e-light-effective-index}
\end{equation}

e光被认为是“反常的”,因为它具有许多各向同性介质中的光完全不显示的性质。

注意到\eqref{eq:o-light}和\eqref{eq:e-light}中关于$\vb*{k}$的方程无论$\vb*{k}$的方向是什么样都是有解的,而且有唯一解。
因此,一旦$\vb*{k}$的方向确定了,e光和o光的$\vb*{k}$以及可能的振动方向也就完全确定了。

总之,各向异性线性介质中的光一般来说是满足\eqref{eq:o-light}的o光和满足\eqref{eq:e-light}的e光的叠加。

% TODO:走移角

\subsection{双光轴透明介质内部的平面波}

\subsubsection{菲涅尔法线方程}

当我们试图将\ref{sec:one-axis-transparent}节中的方法原封不动地推广到一般的各向异性介质中时,会遇到一个严重的困难:无法得到像\eqref{eq:uniaxial-crystal}这样的因式分解好了的关于$\vb*{k}$的方程。

设介质中有一列平面波,记\concept{这一列波的折射率}为
\begin{equation}
    n = \frac{k c_0}{\omega} = \frac{k}{\omega} \frac{1}{\sqrt{\mu_0 \epsilon_0}}
\end{equation}
那么就有
\[
    \vb*{k} = n \omega \sqrt{\mu_o \epsilon_0} \vu*{k}
\]
由\eqref{eq:k-det},可以得到
\[
    \det \left( n^2 \omega^2 \epsilon_0 \mu_0 (\vb*{\delta} - \vu*{k} \vu*{k}) - \omega^2 \mu \vb*{\epsilon} \right) = 0
\]
\[
    \det \left( \vb*{\delta} - \vu*{k} \vu*{k} - \frac{\mu_r \vb*{\epsilon}_r}{n^2} \right) = 0
\]
然后求解这个方程。由于此时$\vu*{k}$已经是给定的了,我们将使用$\vu*{k}, \mu_r, \vb*{\epsilon}_r$表示出$n$。
在主轴坐标系中表示$\vb*{k}$和其它矢量,并且设对角化之后%
\footnote{
    所谓对角化,在这里指的是写成
    \[
        \vb*{\epsilon}_r = \epsilon_{r \underline{i}} \vb*{g}_i \vb*{g}^i
    \]
    的形式。
}%
的$\vb*{\epsilon}_r$的三个元素为$\epsilon_{rx}, \epsilon_{ry}, \epsilon_{rz}$
则有
\begin{equation}
    \frac{k_x k^x}{\frac{1}{n^2} - \frac{1}{\mu_r \epsilon_{rx}}} + \frac{k_y k^y}{\frac{1}{n^2} - \frac{1}{\mu_r \epsilon_{ry}}} + \frac{k_z k^z}{\frac{1}{n^2} - \frac{1}{\mu_r \epsilon_{rz}}} = 0
    \label{eq:fresnel-k-n}
\end{equation}
将\eqref{eq:fresnel-k-n}展开为$1/n$的多项式之后会发现这是一个关于$1/n^2$的二次多项式,且在实数域内有解,因此\eqref{eq:fresnel-k-n}有两个正根两个负根。
仅考虑物理解,能够得到两个正根。
这表明了任意的各向异性介质的特点:给定一个$\vb*{k}$,可以有两种相位传播速度不同的波。

若设$\vb*{\epsilon}$在主轴系中被对角化为
\begin{equation}
    c_0^2 [\mu \epsilon_{ij}]_{ij} = [\mu_r \epsilon_{r\;ij}]_{ij} = \bmqty{\dmat{n_x^2, n_y^2, n_z^2}}
    \label{eq:diag-two-axis}
\end{equation}
则有
\begin{equation}
    \frac{k_x k^x}{\frac{1}{n^2} - \frac{1}{n_x^2}} + \frac{k_y k^y}{\frac{1}{n^2} - \frac{1}{n_y^2}} + \frac{k_z k^z}{\frac{1}{n^2} - \frac{1}{n_z^2}} = 0
\end{equation}
这个方程称为\concept{菲涅尔法线方程}。虽然我们使用$\vb*{k}$的各个分量写出了它,由于其齐次性,完全可以将所有$\vb*{k}$的分量替换为$\vu*{k}$的分量。

需要注意的是对应于$n$的两个根的波的振动方向并不是任意的。使用本节的记号,可以将\eqref{eq:plain-wave-in-anistrophy}写成
\begin{equation}
    \left(\vb*{\delta} - \vu*{k}\vu*{k} - \frac{\mu_r \vb*{\epsilon}_r}{n^2} \right) \cdot \vb*{E} = 0
\end{equation}
它意味着:首先,$n$应当被适当地选定,让方程左边的张量的行列式为零,这等价于\eqref{eq:fresnel-k-n};其次,$n$被确定后,$\vb*{E}$可能的方向也被确定了下来。$\vb*{E}$可能的取值就是$\vb*{\delta} - \vu*{k}\vu*{k} - \mu_r \vb*{\epsilon}_r / n^2$的零空间。

一个可能的问题:在已经选定了$\vu*{k}$之后,我们能够得到两个$n$,从而两个$\vb*{E}$振动的方向,那么为什么不是三个方向?
原因在于我们有约束$\vb*{k} \cdot \vb*{D} = \vb*{k} \cdot \vb*{\epsilon} \cdot \vb*{E} = 0$,因此实际能够取的$\vb*{E}$(或者$\vb*{D}$)分布在一个二维的空间中,而不是三维的空间。

此外,注意到相速度$v_\text{p}$就是$c_0 / n$,因此在波矢方向已经给定的情况下从\eqref{eq:fresnel-k-n}可以解出两个相速度。当然,这就是对应于两个$n$的平面波的传播速度。

\subsubsection{折射率椭球}

% TODO:各向异性是不是还是电场能和磁场能各占一半?
现在考虑区域内电场能量密度
\[
    w_E = \frac{1}{2} \vb*{E} \cdot \vb*{D} = \frac{1}{2} (E^x D_x + E^y D_y + E^z D_z),
\]
同样在主轴坐标系当中工作,由于
\[
    D^x = \epsilon_0 \epsilon_{rx} E^x, \quad D^y = \epsilon_0 \epsilon_{ry} E^y, \quad D^z = \epsilon_0 \epsilon_{rz} E^z
\]
我们有
\begin{equation}
    \frac{2 \epsilon_0 w_E}{\mu_r} = \frac{D_x D^x}{n_x^2} + \frac{D_y D^y}{n_y^2} + \frac{D_z D^z}{n_z^2}
\end{equation}
因此如果固定电磁能密度不变,那么这一点的$\vb*{D}$扫过一个椭球面。这个椭球面称为\concept{折射率椭球}。

\subsubsection{各矢量方向的分析}

首先,下面三个方程还是成立的,正如在各向同性介质中一样:
\[
    \vb*{k} \cdot \vb*{D} = 0, \quad \vb*{k} \cdot \vb*{H} = 0, \quad \vb*{k} \times \vb*{H} = - \omega \vb*{D}
\]
这意味着$\vb*{D}, \vb*{H}, \vb*{k}$构成一组右手系。$\vb*{B}$的方向和$\vb*{H}$完全一致,因此无需单独讨论其方向。

反之,由于$\vb*{\epsilon}$的各向异性,$\vb*{E}$的方向需要特别注意。
由于
\[
    \vb*{k} \times \vb*{E} = \omega \vb*{B}, \quad \vb*{k} \times \vb*{H} = - \omega \vb*{D}
\]
可以导出下面的方程
\begin{equation}
    \vb*{D} = \mu_r \epsilon_0 n^2 (\vb*{E} - (\vu*{k} \cdot \vb*{E}) \vu*{k}) = \mu_r \epsilon_0 n^2 \vb*{E}_{\bot}
    \label{eq:first-crystal-eq}
\end{equation}
即所谓\concept{晶体光学第一方程},其中$\bot$表示在垂直于$\vb*{k}$的方向上做投影。实际上,通过将$\vb*{D} = \vb*{\epsilon} \cdot \vb*{E}$代入上式,也能够导出菲涅尔法线方程\eqref{eq:fresnel-k-n}。

% TODO:画图
\eqref{eq:first-crystal-eq}使用$\vb*{E}$表示了$\vb*{D}$;我们也可以反过来尝试使用$\vb*{D}$表示$\vb*{E}$。注意到$\vb*{E}$虽然和$\vb*{D}$未必重合,但是它一定落在垂直于$\vb*{H}$的平面内;而$\vu*{k} \propto \vb*{D} \times \vb*{H}$,$\vu*{S} \propto \vb*{E} \times \vb*{H}$,于是几何观察告诉我们,将$\vu*{D},\vu*{k}$做一个旋转角为$\alpha$(它正是\eqref{eq:energy-density-and-s}中的那个$\alpha$),在垂直于$\vb*{H}$的平面上的旋转就得到了$\vu*{E}, \vu*{S}$。
于是记
\begin{equation}
    n_r = n \cos \alpha,
\end{equation}
就得到了
\begin{equation}
    \vb*{E} = \frac{1}{\mu_r \epsilon_0 n_r^2} (\vb*{D} - (\vu*{S} \cdot \vb*{D}) \vu*{S})
    \label{eq:second-crystal-eq}
\end{equation}
即所谓的\concept{晶体光学第二方程}。

联立这两个方程\eqref{eq:first-crystal-eq}和\eqref{eq:second-crystal-eq}中的其中一个和本构关系$\vb*{D} = \vb*{\epsilon} \vb*{E}$,
可以得到仅仅关于$\vb*{E}$或仅仅关于$\vb*{D}$的方程。
仅仅关于$\vb*{E}$的方程已经被建立了,它就是\eqref{eq:plain-wave-in-anistrophy},它有非零解的条件就是\eqref{eq:fresnel-k-n}。
联立晶体光学第二方程\eqref{eq:second-crystal-eq}和本构关系,尝试得到仅仅关于$\vb*{D}$的方程。
在主轴系下进行计算,此时本构关系为\eqref{eq:diag-two-axis},
就得到
\begin{equation}
    \frac{S_x S^x}{n_x^2 - \mu_r n_r^2} + \frac{S_y S^y}{n_y^2 - \mu_r n_r^2} + \frac{S_z S^z}{n_z^2 - \mu_r n_r^2} = 0
\end{equation}
它是使用$\vb*{S}$和$n_r$表示的\eqref{eq:fresnel-k-n}的对应物。
定义

\subsubsection{能量和能流}



\section{衍射}

TODO: 相干性;实际上一个光源产生的光可以是一个随机变量(在量子光学中就需要密度矩阵)

何时衍射不明显:波长非常短的时候肯定不明显;相干性差的时候也是;这和“量子效应什么时候不明显”是类似的:

\chapter{光学谐振腔}

前几章的电磁波传播都是没有任何边界条件约束的,此时如果介质均匀,那么电磁波可以以平面波的形式稳定传播。
本章则讨论束缚在有限区域内的电磁波模式。

原则上当然可以通过求解关于标势和矢势的方程来分析谐振腔的行为,例如解\eqref{eq:wave-eq}即可。
通常谐振腔内部没有源,于是\eqref{eq:wave-eq}中的各个方程都是齐次的,且各个矢势分量之间没有关系,从而似乎求解标量亥姆霍兹方程即可获知谐振腔的行为。
然而,边界条件实际上还是会让各个矢势分量和标势等都产生关系。
因此使用标势和矢势并不能简化问题。于是,本章还是直接求解\eqref{eq:e-in-tensor-material}。
特别的,各向同性体态中有
\begin{equation}
    \laplacian \vb*{E} + k_0^2 \vb*{E} = 0, \quad \div{\vb*{E}} = 0, \quad k_0 = \frac{\omega}{c},
    \label{eq:isotropic-cavity-problem-origin}
\end{equation}
这里$\vb*{E} = \vb*{E}(\vb*{r}, \omega)$。
通常我们会定义
\begin{equation}
    \tilde{\vb*{E}} = \sqrt{\epsilon} \vb*{E}(\vb*{r}, \omega), \quad \tilde{\vb*{B}} = \frac{\ii}{\sqrt{\mu}} \vb*{B}(\vb*{r}, \omega)
\end{equation}
来让$\vb*{E}$和$\vb*{B}$相差不要太大,此时
\begin{equation}
    \tilde{\vb*{B}} = \frac{1}{k_0} \curl{\tilde{\vb*{E}}}.
    \label{eq:tilde-b-tilde-e-cavity}
\end{equation}
将\eqref{eq:isotropic-cavity-problem-origin}用$\tilde{\vb*{E}}$和$\tilde{\vb*{B}}$写出来,就是
\begin{equation}
    \begin{bigcase}
        &\laplacian \tilde{\vb*{E}} + k_0^2 \tilde{\vb*{E}} = 0, \quad \laplacian \tilde{\vb*{B}} + k_0^2 \tilde{\vb*{B}} = 0, \\
        &\tilde{\vb*{B}} = \frac{1}{k_0} \curl{\tilde{\vb*{E}}}, \quad \tilde{\vb*{E}} = \frac{1}{k_0} \curl{\tilde{\vb*{B}}}.
    \end{bigcase}
    \label{eq:isotropic-cavity-problem}
\end{equation}
最后一个方程只需要在倒数第二个方程两边作用散度,通过一些计算即可看出。

\section{方形谐振腔}

\subsection{立方体谐振腔}

\subsection{方形波导}

\concept{波导}是指一个长条状的、电磁波可以在其中传递的装置。我们讨论一个柱状的波导,它是一个形状任意的闭合曲线沿着垂直于截面的方向平移而形成的直导管,其壁为导体,内部填充了某种均匀介质。
基本上,能够称为电磁波的电磁场构型都有很强的趋肤效应,因此接下来如无特殊说明我们认为波导的壁是理想导体,即认为导体内部没有任何场分布,即认为边界条件为
\begin{equation}
    \vb*{n} \cdot \vb*{B} = 0, \quad \vb*{n} \times \vb*{E} = 0.
\end{equation}
表面上,从边界条件$\vb*{n} \cdot (\vb*{D}_1 - \vb*{D}_2) = 0$出发,并利用导体内部没有电场分布这一条件,似乎可以得到$\vb*{n} \cdot \vb*{D}_1=0$,但这是错误的:如果我们要求$\vb*{n} \cdot (\vb*{D}_1 - \vb*{D}_2) = 0$成立,即将导体内部的电流归入$\epsilon$中,即给$\epsilon$一个虚部,那么随着导体电导率的上升,导体内部的$\vb*{E}$的确会下降,但是$\epsilon$会上升,最后在边界上会留下一个不为零的$\vb*{n} \cdot \vb*{D}_2$。
而如果我们不将电流归入$\epsilon$,那么就有$\vb*{n} \cdot (\vb*{D}_2 - \vb*{D}_1) = \sigma$,而$\vb*{D}_2$正常地衰减,于是边界条件就是$\vb*{n} \cdot \vb*{D}_1 = \sigma$。
我们并不知道$\sigma$到底是什么,因此这个边界条件实际上是用来在$\vb*{E}$已知后返回来求解$\sigma$的。
$\vb*{n} \times (\vb*{H}_2 - \vb*{H}_1) = \vb*{j}$同理。

考虑时谐场。由于$z$方向上的平移不变性,我们可以认为
\[
    \vb*{E}, \vb*{B} \propto \ee^{\ii (k_z z - \omega t)},
\]
在导管内部,波动方程为
\begin{equation}
    \left( \pdv[2]{x} + \pdv[2]{y} + k_0^2 - k_z^2 \right) \pmqty{\vb*{E} \\ \vb*{B}} = 0, \quad k_0 = \frac{\omega}{c}.
\end{equation}
以上方程并不能定解,但是实际上通过使用方程$\curl{\vb*{E}}=-\partial_t \vb*{B}$以及$\curl{\vb*{H}} = \partial_t \vb*{E}$,$x, y$方向上的场可以写成$z$方向上的场及其导数的线性函数,因此我们只需要求解
\begin{equation}
    \left( \pdv[2]{x} + \pdv[2]{y} + k_\text{c}^2 \right) \pmqty{E_z \\ B_z} = 0, \quad k_\text{c}^2 = k_0^2 - k_z^2.
\end{equation}
我们不能指望$E_z$和$B_z$都是零,因为此时没有非平庸解,即波导的约束意味着严格的横波是不可能的。
可能的偏振模式可以分成$B_z=0$的\concept{横磁波}(TM)和$E_z=0$的\concept{横电波}(TE)两类。

简单的计算表明对横电波我们有(本段中所有的$\grad$都是二维平面上的,我们暂时忽略电磁场在$z$方向上的周期性波动)
\begin{equation}
    \vb*{B}_\text{t} = \frac{\ii k_z}{k_\text{c}^2} \grad{B_z}, \quad \vb*{E}_\text{t} = - \ii \frac{c k_0}{k_\text{c}^2} \vb*{e}_z \times \grad{B_z},
\end{equation}
于是从$\vb*{n} \cdot \vb*{B} = 0$得到$B_z$满足的边界条件
\begin{equation}
    \pdv{B_z}{n} = 0,
\end{equation}
并且这个条件也能够让$\vb*{n} \times \vb*{E} = 0$成立,于是据此条件求解$B_z$满足的亥姆霍兹方程就确定了一切。对横磁波类似的有
\begin{equation}
    \vb*{E}_\text{t} = \frac{\ii k_z}{k_\text{c}^2} \grad{E_z}, \quad \vb*{B}_\text{t} = \ii \frac{k_0}{ck_\text{c}^2} \vb*{e}_z \times \grad{E_z},
\end{equation}
边界条件为
\begin{equation}
    E_z = 0.
\end{equation}
这个边界条件是$\vb*{n} \times \vb*{E}=0$的直接推论,但是由于它让$\grad{E_z}$在边界上一定沿着$\vb*{n}$,可以验证$\vb*{n} \cdot \vb*{B}=0$也是成立的。

在$xy$平面上求解可能的TE或是TM模式,得到的是离散谱,而电磁场在$z$方向的传播却是散射态,即$\omega$和$k_z$都可以连续取值,于是波导内的模式的能谱形如
\begin{equation}
    \omega = c \sqrt{k_z^2 + k_{\text{c}, mn}^2},
\end{equation}
其中$m, n$为标记$xy$平面上的模式的整数编号。可以看到这个能谱是有能隙的,能量低于
\begin{equation}
    \omega_\text{c} = \min (c k_{\text{c}, mn})
\end{equation}
的电磁波入射波导之后会快速衰减。

\section{导引矢量法}\label{sec:guiding-vector}

注意到问题\eqref{eq:isotropic-cavity-problem}中$\tilde{\vb*{E}}$和$\tilde{\vb*{B}}$的形式高度对称,我们可以尝试通过一个特殊的构造产生它的解。

由于\eqref{eq:isotropic-cavity-problem}中的电场和磁场均满足横波条件,它们总是可以写成某个东西的旋度。
设某个矢量场$\vb*{M}$是某个东西的旋度,即满足
\begin{equation}
    {\vb*{M}} = \curl{(\psi \vb*{c})},
    \label{eq:guiding-vector-construction}
\end{equation}
这样关于$\vb*{M}$的亥姆霍兹方程就变为
\begin{equation}
    0 = \laplacian {\vb*{M}} + k_0^2 {\vb*{M}} = \curl{((\laplacian \psi + k_0^2 \psi) \vb*{c} + \psi \laplacian \vb*{c})}.
    \label{eq:m-helmholtz-original}
\end{equation}
注意此处$\vb*{c}$和$\psi$的定义不唯一。我们总是能够找到(虽然一般都不容易解析地找到)一个$\vb*{c}$满足
\begin{equation}
    \div{\vb*{c}} = \text{const}, \quad \curl{\vb*{c}} = 0.
\end{equation}
这是因为,设
\[
    \vb*{M} = \curl{\vb*{M}'},
\]
我们注意到关于某个标量场$\lambda$的方程
\[
    \div{(\lambda \vb*{M}')} = \text{const}, \quad \curl{(\lambda \vb*{M}')} = 0
\]
一定有解,因为第二个方程的三个分量方程实际上只有两个独立。
因此,我们设
\[
    \vb*{M}' = \psi \vb*{c}, \quad \psi = \frac{1}{\lambda}
\]
即可得到\eqref{eq:guiding-vector-construction},即\eqref{eq:guiding-vector-construction}中的$\vb*{c}$和$\psi$总是可以构造出来的。
我们称$\vb*{c}$为\concept{导引矢量},因为它大体上描绘了$\vb*{M}$的“指向”。

对$\vb*{c}$我们有
\[
    \laplacian \vb*{c} = \grad(\div{\vb*{c}}) - \curl{(\curl{\vb*{c}})} = 0,
\]
于是
\begin{equation}
    \vb*{M} = \grad{\psi} \times \vb*{c}, \quad \vb*{M} \bot \vb*{c},
\end{equation}
关于$\vb*{M}$的亥姆霍兹方程\eqref{eq:m-helmholtz-original}等价于
\begin{equation}
    \laplacian \psi + k_0^2 \psi = 0.
    \label{eq:scalar-cavity-eq}
\end{equation}
在解出$\vb*{M}$之后我们会注意到矢量场
\begin{equation}
    \vb*{N} = \frac{1}{k_0} \curl{\vb*{M}}
    \label{eq:cavity-n-def}
\end{equation}
满足
\begin{equation}
    \vb*{M} = \frac{1}{k_0} \curl{\vb*{N}}, 
\end{equation}
并且它也满足和$\vb*{M}$满足的亥姆霍兹方程完全一样的方程
\begin{equation}
    \laplacian \vb*{N} + k_0^2 \vb*{N} = 0.
\end{equation}

对比$\vb*{M}$和$\vb*{N}$矢量场满足的各个方程和\eqref{eq:isotropic-cavity-problem},我们发现可以将$\vb*{M}$看成$\tilde{\vb*{E}}$,将$\vb*{N}$看成$\tilde{\vb*{B}}$,也可以反过来将$\vb*{M}$看成$\tilde{\vb*{B}}$,将$\vb*{N}$看成$\tilde{\vb*{E}}$。
因此我们得到了一种原则上一般的求解光学谐振腔中的模式的方法:求解标量方程\eqref{eq:scalar-cavity-eq},然后将各个模式代入\eqref{eq:guiding-vector-construction}和\eqref{eq:cavity-n-def},这样就得到了全部的电磁波模式。

\subsection{平凡的例子:平面波}

平面波的经验给出了一种挑选$\vb*{c}$的方法:尽可能让$\vb*{c}$垂直于主要的界面方向。

\subsection{圆柱波导中的柱面波}

在圆柱波导中可以验证将$\vb*{c}$选择为
\begin{equation}
    \vb*{c} = \vb*{e}_\rho
\end{equation}
是可行的,此时需要求解
\begin{equation}
    \frac{1}{\rho} \pdv{\rho} \left( \rho \pdv{\psi}{\rho} \right) + \frac{1}{\rho^2} \pdv[2]{\psi}{\phi} + \pdv[2]{\psi}{z} + k_0^2 \psi = 0.
\end{equation}
这个方程的求解是已知的:它最终转化为柱贝塞尔方程的求解。


我们分析几种极限情况。$k \rho \to 0$的情况对应于在我们关心的距离尺度内电磁波的传播速度可以忽略的情况(或者$\rho$很小,或者$c$很大以至于$k$很小),而$x \to 0$时 % TODO:贝塞尔函数的渐近行为
此时的解就是静电势的通解。

圆柱波导中的电磁波模式本质上还是标量的。

\subsection{球面波}

选取
\begin{equation}
    \vb*{c} = r \vb*{e}_r.
\end{equation}
关于$\psi$的亥姆霍兹方程为
\begin{equation}
    \frac{1}{r^2} \pdv{r}\left( r^2 \pdv{\psi}{r} \right) + \frac{1}{r^2 \sin \theta} \pdv{\theta} \left( \sin \theta \pdv{\psi}{\theta} \right) + \frac{1}{r^2 \sin \theta} \pdv[2]{\psi}{\phi} + k_0^2 \psi = 0.
\end{equation}

以上求解过程说明三维球腔中不存在s波。
从数学上看这来自所谓\emph{毛球定理}:$S^1$上可以有一个连续而处处不为零的切向量场,但是$S^2$上不可能有这样的切向量场。

我们将球面波推导出的$\vb*{N}$和$\vb*{M}$称为\concept{球波函数}。
球波函数显然可以用于做多极矩展开。实际上,它比我们前面通过泰勒级数得到的多极矩展开更加优越,因为后者只在远场情况下能够毫无疑难地定义,在近场时会有一定的模糊性。
一个重要的例子就是\concept{环形磁偶极矩}。

\chapter{非均匀折射率导致的散射}

\part{光源}\label{part:source}

\input{source.tex}

\part{非线性光学}

\chapter{非线性极化}

极化矢量和电场之间的关系当然不完全是线性的。我们

\section{非线性光学过程的经典模型}\label{sec:classical-models}

\subsection{非线性谐振子模型}\label{sec:classical-oscillator}

本节将材料当成一系列谐振子的组合,并且暂时不考虑谐振子之间的相互作用。
这是相对合理的,因为能够长距离传输光的介质一般不是金属,从而电子是相对定域的。
然而,这并不意味着我们的理论是自由的。
我们知道一个标准的经典谐振子可以用
\begin{equation}
    m \dv[2]{x}{t} + m \gamma \dv{x}{t} + m \omega_0^2 x = q E
\end{equation}
来描述,而如果我们加上诸如$x^3$这样的项,即让谐振子的回复力为非线性的,就可以造成谐振子模式发生自相互作用。
我们讨论的问题的能量都并不高,谐振子运动不会特别快,因此可以认为谐振子只产生电场,并且其方式为“电偶极子产生库伦场”。
用作用量表示,就是% TODO:有误,这里的关键在于qxE可以给出电场对电荷的作用,但是是否能够给出电荷对电场的作用??或者说,电场和电荷的相互作用拉氏量或是哈密顿量要怎么写??
\[
    S = \int \dd{t} \left( \frac{1}{2} m \dot{x}^2 - \frac{1}{2} k x^2 + \text{higher order $x^n$} + qxE \right).
\]
现在积掉谐振子,就能够得到非线性的光子-光子过程,即多个光子和一或是一个光子分裂为多个光子。

\subsubsection{二阶非线性极化}

我们现在在谐振子能量中加入一个三次项,即在运动方程中加入一个二次回复力:
\begin{equation}
    m \dv[2]{x}{t} + m \gamma \dv{x}{t} + m \omega_0^2 x + m a x^2 = q E.
    \label{eq:x3-eq}
\end{equation}
这相当于在能量中加入了一个$\frac{1}{3} m a x^3$项。这个项破坏了系统的中心反演对称性。
我们实际上是要从电场计算$x$(从$x$计算响应电场的公式是显然的,就是大量偶极子加总为$\vb*{P}$然后从$\vb*{P}$出发算电场)。

对这个问题的标准的处理是微扰求解微分方程,但是实际上可以使用费曼图分析这个问题。由于只考虑经典情况,无需计算圈图。
“经典情况”到底指的是什么需要进一步说明:在经典情况下我们没有二次量子化,粒子图景的经典理论和场的图景的经典理论还不一样。
粒子图景下运动方程是关于各个粒子的位置和动量的,入射和出射外线没有任何限制。
场的图景下,我们求解系统的基本自由度(在这里是电场和谐振子坐标)的运动方程,得到场变量随时间的变化情况,即实际上在求解$\expval*{\phi}$,因此只能有一条出射外线,入射外线应当被当成外源。
在经典极限下这两种图景不会造成太大差别:同一张图的外线数目是固定的,在场的图景下,出射外线多了外源就少,由于我们要求场强满足$\phi / \hbar \ll 1$(但与此同时能标又没有高到多顶角图非常重要,从而圈图修正有必要计算),外源较少的过程是非常不重要的。

我们采用后一种图景,因为我们实际上就是在微扰求解\eqref{eq:x3-eq}。
将电磁场和带有$x^3$形势能的非线性振子耦合,则费曼图中应该有\autoref{fig:x3-vertex}和\autoref{fig:light-osci-couple}两种基本元件。
应当注意这里的短直线代表的是$x$的某个频率的分量,如果做量子化,就是一个谐振子模式。
这里的传播子并不代表谐振子的状态本身。这也就是\autoref{fig:light-osci-couple}顶角中只有一条短直线而不是两条的原因:它代表一个入射光子激发出一个谐振子模式,而不是谐振子整体吸收一个光子之后变成另一个状态。
由于电场和谐振子的耦合是完全线性的(并且由于谐振子是一个没有空间分布的点,实际上耦合项就是电偶极子能量),且我们关心的是“非线性介质中有哪些光学过程”,可以将电场暂时当成背景场,于是\autoref{fig:light-osci-couple}应该被\autoref{fig:external-field}取代。
例如,我们只需要分析一阶过程\autoref{fig:first-order-x3-external}就能知道\autoref{fig:first-order-x3-photon}的来源——如果$E$让$x$产生非线性响应,那就有光子分裂和合并的过程。

费曼规则可以很容易地写出:(我们认为频率为$\omega$的成分携带$\ee^{- \ii \omega t}$因子)
\begin{itemize}
    \item 传播子为
    \[
        \begin{tikzpicture}
            \begin{feynhand}
                \vertex (a) at (0, 0);
                \vertex (b) at (1, 0);
                \propag [plain, mom={$\omega$}] (a) to (b); 
            \end{feynhand}
        \end{tikzpicture} = \frac{\ii}{m (\omega^2 + \ii \gamma \omega - \omega_0^2)}.
    \]
    \item 顶角为
    \[
        \begin{tikzpicture}
            \begin{feynhand}
                \vertex (a) at (-1,-1); \vertex (b) at (1,-1); \vertex (c) at (0,1);
                \vertex (o) at (0,0); 
                \propag [plain] (a) to (o);
                \propag [plain] (b) to (o); 
                \propag [plain] (c) to (o);    
            \end{feynhand}
        \end{tikzpicture} = - \ii 2 m a \cdot 2\pi \delta(\sum \omega).
    \]
    注意正常情况下$x^3$相互作用要配一个$1/3!$的因子但是这里只有$1/3$,因此顶角实际上是$2ma$而不是$ma$。
    \item 外源为
    \[
        \begin{tikzpicture}
            \begin{feynhand}
                \vertex [crossdot] (a) at (0, 0){};
                \vertex (b) at (1, 0);
                \propag [plain, mom={$\omega$}] (a) to (b); 
            \end{feynhand}
        \end{tikzpicture} = \ii q E(\omega).
    \]
    请注意这里没有负号,而$x^3$是负号的,这是因为均匀电场会倾向于把谐振子拉向无穷远处而回复力则会将谐振子拉回来。
    本节采取的傅里叶变换约定为
    \[
        E(t) = \int \dd{\omega} E(\omega) \ee^{- \ii \omega t},
    \]
    没有加入$2\pi$是因为很多时候入射光并不是连续谱,而是离散的几个频域分量加起来。
\end{itemize}

\begin{figure}
    \centering
    \subfigure[$x^3$自相互作用顶角]{
        \begin{tikzpicture}
            \begin{feynhand}
                \vertex (a) at (-1,-1); \vertex (b) at (1,-1); \vertex (c) at (0,1);
                \vertex [dot] (o) at (0,0) {}; 
                \propag [plain] (a) to (o);
                \propag [plain] (b) to (o); 
                \propag [plain] (c) to (o);    
            \end{feynhand}
        \end{tikzpicture}
        \label{fig:x3-vertex}
    }
    \subfigure[光子激发出一个谐振子模式]{
        \begin{tikzpicture}
            \begin{feynhand}
                \vertex (a) at (-1, 1.5);
                \vertex (b) at (0, 1.5);
                \vertex (c) at (1, 1.5);
                \propag [photon] (a) to (b);
                \propag [plain] (b) to (c);
            \end{feynhand}
        \end{tikzpicture}
        \label{fig:light-osci-couple}
    }
    \subfigure[外源驱动谐振子,即\autoref{fig:light-osci-couple}中的光子被当成无动力学的外场后得到的图形]{
        \begin{tikzpicture}
            \begin{feynhand}
                \vertex [crossdot] (a) at (0, 0) {};
                \vertex (b) at (1, 0);
                \propag [plain] (a) to (b);
            \end{feynhand}
        \end{tikzpicture}
        \label{fig:external-field}
    }
    \caption{加入$\frac{1}{3} m a x^3$势能之后的费曼图元件}
\end{figure}

\begin{figure}
    \centering
    \subfigure[外场导致的响应的一阶近似]{
        \begin{tikzpicture}
            \begin{feynhand}
                \vertex [crossdot] (a) at (-1,-1) {};
                \vertex [crossdot] (b) at (1,-1) {}; 
                \vertex (c) at (0,1);
                \vertex (o) at (0,0) ; 
                \propag [plain, mom={$\omega_1$}] (a) to (o);
                \propag [plain, mom={$\omega_2$}] (b) to (o); 
                \propag [plain, mom={$\omega_1 + \omega_2$}] (o) to (c);
            \end{feynhand}
        \end{tikzpicture}
        \label{fig:first-order-x3-external}
    }
    \subfigure[\autoref{fig:first-order-x3-external}导致的非线性光学过程]{
        \begin{tikzpicture}
            \begin{feynhand}
                \vertex (a0) at (-1.5, -1.5);
                \vertex (a) at (-0.5,-0.5);
                \vertex (b0) at (1.5, -1.5);
                \vertex (b) at (0.5,-0.5); 
                \vertex (c) at (0,0.5);
                \vertex (c0) at (0, 1.5);
                \vertex (o) at (0,0) ; 
                \propag [photon, mom={$\omega_1$}] (a0) to (a);
                \propag [plain] (a) to (o);
                \propag [photon, mom={$\omega_2$}] (b0) to (b);
                \propag [plain] (b) to (o); 
                \propag [plain] (o) to (c);
                \propag [photon, mom={$\omega_1 + \omega_2$}] (c) to (c0);
            \end{feynhand}
        \end{tikzpicture}
        \label{fig:first-order-x3-photon}
    }
    \caption{一阶过程}
    \label{fig:x3-first-order}
\end{figure}

据此,线性响应(零阶,没有发生任何非线性效应)为
\begin{equation}
    x_1(\omega) = \ii q E(\omega) \frac{\ii}{m (\omega^2 + \ii \gamma \omega - \omega_0^2)} = \frac{q / m}{\omega_0^2 - \omega^2 - \ii \gamma \omega} E(\omega).
\end{equation}
一阶过程(即\autoref{fig:first-order-x3-external})给出如下修正:
\begin{equation}
    \begin{aligned}
        x_2(\omega) &= \frac{1}{2} \int \dd{\omega_1} \int \dd{\omega_2} (\ii q E(\omega_1)) (\ii q E(\omega_2)) \frac{\ii}{m(\omega_1^2 + \ii \gamma \omega_1 - \omega_0^2)} \frac{\ii}{m(\omega_2^2 + \ii \gamma \omega_2 - \omega_0^2)} \\ 
        &\quad \quad \times (-\ii 2 m a) \frac{\ii}{(m(\omega_1 + \omega_2)^2 + \ii \gamma (\omega_1 + \omega_2) - \omega_0^2)} 2\pi \delta(\omega_1 + \omega_2 - \omega) \\
        &= \int \dd{\omega_1} \int \dd{\omega_2} \frac{a (q / m)^2}{(\omega_1^2 + \ii \gamma \omega_1 - \omega_0^2) (\omega_2^2 + \ii \gamma \omega_2 - \omega_0^2) (\omega^2 + \ii \gamma \omega - \omega_0^2)}  \\
        &\quad \quad \times 2\pi \delta(\omega_1 + \omega_2 - \omega) \times E(\omega_1) E(\omega_2).
    \end{aligned} 
    \label{eq:continuous-x3-first-order}
\end{equation}
这里需要注意一点:\autoref{fig:first-order-x3-external}中外场出现了两次,而
\[
    E(t)^2 = \int \dd{\omega_1} \int \dd{\omega_2} E(\omega_1) E(\omega_2) \ee^{-\ii (\omega_1 + \omega_2) t},
\]
如果$\omega_1 \neq \omega_2$那么$E(\omega_1) E(\omega_2)$项实际上会被求和两次;同样,此时\eqref{eq:continuous-x3-first-order}中的$E(\omega_1) E(\omega_2)$项也会被求和两次。
直观地看,外场是给定的而不能随意交换,所以\autoref{fig:first-order-x3-external}中的两个外场从左到右为$\omega_1$和$\omega_2$的图和从左到右为$\omega_2$和$\omega_1$的图虽然给出一样的结果,但是是两张图,不能看成一张图,它们加起来会导致因子$2$出现。
如果我们令$E(t)$实际上只有两个频率分量,这一点会显得尤其明显。
$\omega_1 = \omega_2$的情况中没有因子$2$,我们常将这样的过程称为\emph{简并的}。

现在我们采取更加常规的,微扰求解微分方程的做法。费曼图计算已经告诉我们主要的光学过程来自\autoref{fig:first-order-x3-external}。
因此,我们将输入的$E$设置为
\begin{equation}
    E = E_1 (\ee^{\ii \omega_1 t} + \ee^{-\ii \omega_1 t}) + E_2 (\ee^{\ii \omega_2 t} + \ee^{-\ii \omega_2 t}), 
    \label{eq:input-two-freq-e}
\end{equation}
做展开
\begin{equation}
    x = x_1 + x_2 + \ldots, \quad x_n \sim E^{n },
\end{equation}
并记
\begin{equation}
    x_i = \sum_n x_i(\omega_n) + \text{c.c.}.
\end{equation}
线性项$x_1$由
\[
    m \ddot{x}_1 + m \gamma \dot{x_1} +  m \omega_0^2 x_1 = q E
\]
给出,为
\begin{equation}
    x_1(\omega_1) = \frac{(q/m) E_i}{\omega_0^2 } \ee^{-\ii \omega_n t},
\end{equation}
二阶项由
% TODO:懒得写了
这些项分别称为:
\begin{itemize}
    \item \concept{和频(SFG, sum frequency generation)},
    \item \concept{差频(DFG, difference frequency generation)},
    \item \concept{倍频(SHG, second harmonic generation)},
    \item \concept{光学整流(OR, optic rectification)}(因为输入交流波而得到直流波,那当然是整流了)。
\end{itemize}
倍频是和频的特殊情况,光学整流是差频的特殊情况。当然,直接将\eqref{eq:input-two-freq-e}代入\eqref{eq:continuous-x3-first-order}也能够得到这些过程。

乍一看,费曼图方法不仅能够得到两个光子合并为一个光子的过程,也能够得到一个光子分裂成两个光子的过程(所谓的\concept{SPDC过程(Spontaneous parametric down-conversion)},也称为\concept{OPG过程(Optical parametric generation)}),但是我们后面将看到,微分方程方法似乎只能给出两个光子合并为一个光子的过程——如果我们在$E$中放入只有一个频率$\omega_0$的波,那么非线性效应似乎只会给出$\omega=0$和$\omega=2\omega_0$两种波。
但是其实这里并没有矛盾:SPDC过程需要两条出射外线;如前所述,我们采用场的图景,由于我们采用微分方程的写法,即从$E$求解$x$(其实是$\expval*{x}$),然后用“谐振子位置的偏离导致极化电场产生”计算总电场的变化,对$x$也要采用场的图景,所以的确只应该考虑只有一条外线的费曼图。
这暗示着SPDC过程实际上是非常弱的(本该如此,和频过程的振幅正比于$E^2$而SPDC过程的振幅正比于$E$),因此在经典图景下这个过程根本就不会出现。%
\footnote{
    从这里也可以看出光学中量子理论的重要性,即使我们讨论的能标自始至终都没有高到让只有QED才有的过程(如四光子等效相互作用)出现。
    经典理论对电磁波的描述是非常粗糙的:如果我们要描述一个物理状态中有两种不同频率的光子,应该怎么做?
    在经典理论中只有一种方法:设
    \[
        \vb*{E} = \vb*{E}_1 \ee^{\ii \omega_1 t} + \vb*{E}_2 \ee^{\ii \omega_2 t} + \text{c.c.}.
    \]
    现在如果要将从一个单频波到以上状态的过程画成费曼图,由于只能画一条外线的图,势必只能画出$\omega_0 \to \omega_1$和$\omega_0 \to \omega_2$两个图,然后能量就不守恒了。
}%

这并不是说SPDC过程——或者说OPG过程——在适用经典近似的体系中完全看不到,因为我们可以在OPG过程后面再放一个DFG过程。
DFG过程也可以称为\concept{OPA过程(Optical parametric amplification)},因为它让入射的两束光的一束变弱而另一束变强。
例如,设我们希望将一束频率为$\omega_1$的光分裂成两束光,频率分别是$\omega_2$和$\omega_3$。
我们可以将一个有二阶非线性极化的光学晶体放在一个内壁对频率为$\omega_2$和$\omega_3$的反射率很高的谐振腔中。
按照后面会提到的\eqref{eq:sfg-intensity},如果有相位匹配条件成立,那么$\omega_1 \to \omega_2 + \omega_3$的OPG过程转化效率很高(在那里是SFG过程效率很高,这里就是OPG过程转化效率很高),于是产生足够强的$\omega_2$光束和$\omega_3$光束,这些光束被谐振腔反射回来,回到非线性晶体内部,于是发生很强的OPA过程。
因此,我们仅仅通过一束单频入射光就得到了两束不同频率的出射光。
% TODO:怎么定量算?
在经典理论中OPA过程是允许的,因此时间反演对称性并没有丧失:的的确确可以有光子的分裂。
但是,经典理论中所有电磁波模式上的光子都是足够多的,因此从“完全没有光子”到“有一个光子”的过程在经典理论中无法被描述。这就是OPG过程看不到的原因。
换而言之,经典理论中的光子分裂,即OPA,不仅需要入射的泵浦光,还需要一个(直观上看,引导泵浦光分裂成哪些频率的光的)\concept{种子光}。
一旦种子光入射了,随着光的传播它会增强。

\begin{figure}
    \centering
    \subfigure[外场导致的响应的二阶近似]{
        \begin{tikzpicture}
            \begin{feynhand}
                \vertex [crossdot] (a) at (-1,-1) {};
                \vertex [crossdot] (b) at (1,-1) {}; 
                \vertex (c) at (0,1);
                \vertex (o) at (0,0) ;
                \vertex [crossdot] (e) at (1, 2) {};
                \vertex  (f) at (-1, 2) ; 
                \propag [plain, mom={$\omega_1$}] (a) to (o);
                \propag [plain, mom={$\omega_2$}] (b) to (o); 
                \propag [plain, mom={$\omega_1 + \omega_2$}] (o) to (c);
                \propag [plain, mom={$\omega_3$}] (e) to (c);
                \propag [plain, mom={$\omega_1 + \omega_2 + \omega_3$}] (c) to (f);
            \end{feynhand}
        \end{tikzpicture}
        \label{fig:second-order-x3-external}
    }
    \subfigure[\autoref{fig:second-order-x3-external}导致的非线性光学过程]{
        \begin{tikzpicture}
            \begin{feynhand}
                \vertex (a0) at (-1.5, -1.5);
                \vertex (a) at (-0.5,-0.5);
                \vertex (b0) at (1.5, -1.5);
                \vertex (b) at (0.5,-0.5); 
                \vertex (c) at (0,0.5);
                \vertex (o) at (0,0) ; 
                \vertex (d) at (0.5, 1);
                \vertex (e) at (-0.5, 1);
                \vertex (d0) at (1.5, 2);
                \vertex (e0) at (-1.5, 2);
                \propag [photon, mom={$\omega_1$}] (a0) to (a);
                \propag [plain] (a) to (o);
                \propag [photon, mom={$\omega_2$}] (b0) to (b);
                \propag [plain] (b) to (o); 
                \propag [plain] (o) to (c);
                \propag [plain] (c) to (e);
                \propag [plain] (c) to (d);
                \propag [photon, mom={$\omega_3$}] (d0) to (d);
                \propag [photon, mom={$\omega_1 + \omega_2 + \omega_3$}] (e) to (e0);
            \end{feynhand}
        \end{tikzpicture}
        \label{fig:second-order-x3-photon}
    }
    \caption{二阶过程}
    \label{fig:x3-second-order}
\end{figure}

还可以进一步往上计算微扰。例如,二阶微扰将给出\autoref{fig:x3-second-order}。
这里给出的四光子相互作用和\autoref{fig:first-order-x3-photon}产生的等效四光子相互作用不同,后者需要两个光子先合并,产生的光子传播一会,然后再和另一个光子合并。
\autoref{fig:second-order-x3-photon}给出的四光子相互作用是直截了当的。

我们来对各阶微扰的量级做一个估计。如果$\omega$和$\omega_0$比较接近,那么微扰论根本就不适用:此时共振发生,$x$是非常大的,可能高阶修正比低阶修正还大。
此时需要从头做光和物质耦合的计算而不能使用加入微弱非线性因素的振子模型。
如果$\omega$远大于$\omega_0$,我们将得到等离子体,此时彼此无关的、振幅不大的振子的图像更加失效了,可能晶格都已经被破坏了,电子的运动状况主要受电场控制。
在这两种情况下本节给出的非线性振子模型都不适用。(等离子体情况下有另一个非线性来源,即协变导数的输运项;见后文)
对$\omega \ll \omega_0$的情况,线性响应的振幅的量级为
\[
    x_1 \sim \frac{(q/m) E}{\omega_0^2},
\]
而
\[
    x_2 \sim \frac{a (q/m)^2 E^2}{\omega_0^6},
\]
因此
\begin{equation}
    \frac{x_2}{x_1} \sim \frac{a q E}{m \omega_0^4}.
\end{equation}
设原子对电子的束缚电场的量级为$E_\text{atom}$,则总位移$x$的振幅可以估计为
\[
    q E_\text{atom} \sim m \omega_0^2 x .
\]
$x$的量级具体有多大是不确定的,它包括没有外加电场时由$q E_\text{atom}$做回复力的内禀振荡,线性响应$x_1$和非线性响应$x_2$。
我们不妨采取一个非常极端的假设,认为线性回复力和非线性回复力已经一样大了(如果非线性回复力很小,那么当然只需要计算一阶图),此时
\[
    m \omega_0^2 x \sim m a x^2,
\]
于是
\[
    q E_\text{atom} \sim m \omega_0^2 \frac{\omega_0^2}{a},
\]
最后
\begin{equation}
    \frac{x_2}{x_1} \sim \frac{E}{E_\text{atom}}.
\end{equation}
通常原子内部电场的数量级为\SI{3e8}{V/m},因此即使认为非线性回复力和线性回复力一样大,一般来说$x_2$也远小于$x_1$,即此时非线性极化相对于线性极化来说还是不大的。
介质的光学性能由极化给出,和回复力没有直接关系,因此非线性极化一般来说总是比非线性极化小得多的。
类似地实际上可以证明
\begin{equation}
    \frac{x_{n+1}}{x_n} \sim \frac{E}{E_\text{atom}}.
\end{equation}

\subsubsection{三阶非线性极化}

\subsection{自由电子气的输运项}

本节将自由电子气视为带电荷的连续介质。非线性效应来自$(\vb*{v} \cdot \grad) \vb*{v}$。
我们写下电荷的运动方程:
\begin{equation}
    \pdv{\vb*{v}}{t} + (\vb*{v} \cdot \grad) \vb*{v} = - e (\vb*{E} + \vb*{v} \times \vb*{B}),
\end{equation}
这里会导致非线性效应的包括输运项$(\vb*{v} \cdot \grad) \vb*{v}$和洛伦兹力项。

我们现在加载单频周期性外场。设频率为$\omega$的电场分量为
\begin{equation}
    \vb*{E}(\omega) = A \ee^{- \ii \omega t} + A^* \ee^{\ii \omega t},
\end{equation}
在忽略所有微扰的情况下显然
\begin{equation}
    \vb*{v}_1 = \frac{e \vb*{E}}{\ii m \omega}.
\end{equation}
于是
\[
    \pdv{\vb*{v}_2}{t} + (\vb*{v}_1 \cdot \grad) \vb*{v}_1 = - e \vb*{v}_1 \times \vb*{B},
\]
最终计算得到
\begin{equation}
    \vb*{j}(2\omega) = \rho_1 \vb*{v}_1 + \rho_0 \vb*{v}_2 = \frac{\epsilon_0 e}{\ii m \omega} (\div{\vb*{E}(\omega)}) \vb*{E}(\omega) + \rho_0 \frac{\ii}{4\omega} \div(\vb*{E}(\omega) \cdot \vb*{E}(\omega)).
    \label{eq:double-freq-current}
\end{equation}

\subsection{金属表面效应}

金属表面也能够产生二次谐波。将金属中的电子视为上一节讨论的带负电荷的连续介质。
金属表面发生了一个突变:内侧是带负电荷的连续介质而外侧则什么也没有。
由于金属外侧没有任何电荷分布,以下提到电流密度等时都是在金属内部讨论。

在金属表面附近,平行于表面的电流密度分量只有\eqref{eq:double-freq-current}的第一项,于是
\begin{equation}
    \vb*{j}_\parallel(2\omega) = \frac{\epsilon_0 e}{\ii m \omega} \pdv{E_\bot(\omega)}{z} \vb*{E}_\parallel(\omega) , 
\end{equation}
对它求积分,得到
\[
    \int_{0^-}^{0^+} \dd{z} \vb*{j}_\parallel(2\omega) = \frac{\epsilon_0 e}{\ii m \omega} \vb*{E}_{\parallel} (\omega) E_\bot(\omega) (1 - \epsilon),
\]
而垂直分量则是
\begin{equation}
    \vb*{j}_\bot(2\omega) = \frac{\epsilon_0 e}{\ii m \omega} \pdv{E_z}{z} E_z + \rho \frac{\ii}{4\omega} \left(\frac{e}{m\omega}\right)^2 \pdv{z} (E_z^2),
\end{equation}

\section{非线性极化的量子理论}

\subsection{单电子系统与经典光场耦合}

绝缘体中,电子-电子库伦散射一般来说是不重要的(除了在一些比较特殊的点由于库仑相互作用打开能隙等)。
从而,对透明晶体——一般来说不导电——以及普通的不导电的气体,的光学性能的计算通常可以使用束缚态单电子模型,即电子的本征态由能级编号标记,然后可以分析光如何让电子在能级之间跃迁。
\autoref{sec:classical-oscillator}中我们以光子和经典谐振子的振动模式为基本自由度,通过为谐振子引入一个非二次型的势来得到非线性效应。
这种做法在量子理论中当然也是成立的,并且此时谐振子的振动模式真的就像一个个粒子一样。
然而,应当注意,这种“光子和谐振子振动模式相互作用,谐振子的振动模式通过非简谐的势能相互作用”的理论并不是最方便的,因为无论为振子——在这里实际上就是原子中的电子——引入怎样的非线性势,由于是束缚态,电子自身的能谱都可以被一系列能级完整描述,并且这些能级是比较容易算出来的。
光子与电子碰撞会让电子从一个能级跳到另一个能级,并且能量守恒条件——其中电子的能量由已经经过非二次型势能修正的能级给出——必须成立。
这意味着,首先考虑非简谐的势能的作用,计算出电子能级,然后考虑光子让电子在这些能级之间跃迁,是更加方便的。
这个图景在经典理论中无法使用,因为此时的光子吸收相互作用顶角有一条入射线,两条出射线,从而一个有光子出射的过程一定有多条出射线(至少一条光子线,以及一条雷打不动的电子出射线),从而无法在经典理论中表达。

在基于电子能级的图景中,非线性光学效应来自高阶微扰论,因为相互作用顶角上连接了两条电子线和一条光子线,从而,一张费曼图中可以有数量任意的入射和出射光子线,另一方面,电子线除了和光子相互作用以外,没有别的相互作用。
这和基于非线性振子的图景非常不同,在后者中一个光子只能连接到一条代表振子振动模式(而不是电子本身)的内线上,即光子到振子振动模式的转换始终是线性的,但是振子振动模式之间可以碰撞,从而有非线性过程。
遮去电子线,将电子的非线性响应表示为宏观的“极化”,以上两种图景统一变成了\autoref{sec:non-linear-maxwell}中的图景。
电子与光场通过电偶极跃迁耦合,在耦合哈密顿量中电场是线性的,从而,电子电偶极矩的$n$次方就对应一个$n$光子顶角。

\subsubsection{电子状态的含时微扰论}

本节考虑一个受到经典电磁场扰动的单电子系统。设系统一开始位于某个态$\ket*{g}$上,电磁场扰动会让它在各个瞬时的状态变得不确定起来。
我们用$m, n$等标记电子能级,用$p, q$等标记光子模式,并做傅里叶分解
\begin{equation}
    \vb*{E}(t) = \sum_p \vb*{E}(\omega_p) \ee^{- \ii \omega_p t}.
\end{equation}

我们首先计算电场扰动下的电子波函数$\ket*{\psi}$。单光子吸收过程$\braket*{m}{\psi^{(1)}}$为
\begin{equation}
    \begin{aligned}
        \begin{gathered}
            \begin{tikzpicture}
                \begin{feynhand}
                    \vertex (a) at (-1.3, 0) {$g$};
                    \vertex (o) at (0, 0);
                    \vertex (b) at (1.3, 0) {$m$};
                    \vertex (c) at (-0.5, 0.87) ;
                    
                    \propag[fermion] (a) to (o);
                    \propag[fermion] (o) to (b);
                    \propag[photon, mom={$p$}] (c) to (o);
                \end{feynhand}
            \end{tikzpicture}
        \end{gathered} &= \frac{1}{\hbar} \sum_p \frac{1}{\omega_g + \omega_p - \omega_m} \mel{m}{- \vb*{d} \cdot \vb*{E}(\omega_p)}{g} \ee^{- \ii (\omega_g + \omega_p - \omega_m) t} \\
        &= \frac{1}{\hbar} \sum_p \frac{\vb*{d}_{mg} \cdot \vb*{E}(\omega_p)}{\omega_{mg} - \omega_p} \ee^{\ii (\omega_{mg} - \omega_p) t},
    \end{aligned}
\end{equation}
双光子吸收过程$\braket*{n}{\psi^{(2)}}$为%
\footnote{
    这里使用的实际上是time ordered perturbation theory, 但是它的各阶微扰和covariant perturbation theory基本上是一样的。
    有两种方式看出这一点。我们可以将开头的$\ket{\text{g}}$也看成一个外线,从而第一个传播子中的$\omega_g + \omega_p$可以看成通过能量守恒定律计算出的能量,而$\omega_m$可以看成从能谱中读出的$m$模式的能量,整个传播子和$\omega - \vb*{p}^2 / 2m$是差不多的。
    这样末端的$l$线可以看成“有外场存在时的自发极化”,然后从此出发即可计算出$\chi$。
    我们也可以在covariant perturbation theory中积分掉各种中间变量来得到推导出此处的分母。
}%
\begin{equation}
    \begin{aligned}
        \begin{gathered}
            \begin{tikzpicture}
                \begin{feynhand}
                    \vertex (a) at (-1.3, 0) {$g$};
                    \vertex (o1) at (0, 0);
                    \vertex (o2) at (1.7, 0);
                    \vertex (b) at (3.0, 0) {$n$};
                    \vertex (c) at (-0.5, 0.87) ;
                    \vertex (d) at (1.2, 0.87);
                    
                    \propag[fermion] (a) to (o1);
                    \propag[fermion] (o1) to[edge label={$m$}] (o2);
                    \propag[fermion] (o2) to (b);
                    \propag[photon, mom={$p$}] (c) to (o1);
                    \propag[photon, mom={$q$}] (d) to (o2);
                \end{feynhand}
            \end{tikzpicture}
        \end{gathered} &= \frac{1}{\hbar^2} \sum_{p, q} \sum_m \frac{1}{\omega_g + \omega_p + \omega_q - \omega_n} \mel{n}{- \vb*{d} \cdot \vb*{E}(\omega_p)}{m} \\ 
        &\quad \quad \times \frac{1}{\omega_g + \omega_p - \omega_m} \mel{m}{- \vb*{d} \cdot \vb*{E}(\omega_p)}{g} \ee^{- \ii (\omega_g + \omega_p + \omega_q - \omega_n) t} \\
        &= \frac{1}{\hbar^2} \sum_{p, q} \sum_m \frac{(\vb*{d}_{nm} \cdot \vb*{E}(\omega_q)) (\vb*{d}_{mg} \cdot \vb*{E}(\omega_p))}{(\omega_{ng} - \omega_q - \omega_p) (\omega_{mg} - \omega_p)} \ee^{\ii (\omega_{ng} - \omega_p - \omega_q) t},
    \end{aligned}
\end{equation}
同理还能够得到三光子吸收过程$\braket*{l}{\psi}$为
\begin{equation}
    \begin{aligned}
        &\quad \begin{gathered}
            \begin{tikzpicture}
                \begin{feynhand}
                    \vertex (a) at (-1.3, 0) {$g$};
                    \vertex (o1) at (0, 0);
                    \vertex (o2) at (1.7, 0);
                    \vertex (o3) at (3.4, 0);
                    \vertex (b) at (4.7, 0) {$l$};
                    \vertex (c) at (-0.5, 0.87) ;
                    \vertex (d) at (1.2, 0.87);
                    \vertex (e) at (2.9, 0.87);
                    
                    \propag[fermion] (a) to (o1);
                    \propag[fermion] (o1) to[edge label={$m$}] (o2);
                    \propag[fermion] (o2) to[edge label={$n$}] (o3);
                    \propag[fermion] (o3) to (b);
                    \propag[photon, mom={$p$}] (c) to (o1);
                    \propag[photon, mom={$q$}] (d) to (o2);
                    \propag[photon, mom={$r$}] (e) to (o3);
                \end{feynhand}
            \end{tikzpicture}
        \end{gathered} \\
        &= \frac{1}{\hbar^3} \sum_{p, q, r} \sum_{m, n} \frac{(\vb*{d}_{ln} \cdot \vb*{E}(\omega_r)) (\vb*{d}_{nm} \cdot \vb*{E}(\omega_q)) (\vb*{d}_{mg} \cdot \vb*{E}(\omega_p))}{(\omega_{lg} - \omega_p - \omega_q - \omega_r) (\omega_{ng} - \omega_q - \omega_p) (\omega_{mg} - \omega_p)} \ee^{\ii (\omega_{lg} - \omega_p - \omega_q - \omega_r) t}.
    \end{aligned}
\end{equation}

以上三个过程都是严格按照电偶极辐射哈密顿量计算出来的;实际的系统中除了电偶极辐射以外,还有各种各样的噪声扰动。
我们假定噪声的主要效果是让系统倾向于自动地弛豫到基态,于是唯象地向传播子中加入有限大小的虚部$\ii \gamma_{mn}$以产生某种阻尼,为了简化书写,令
\begin{equation}
    \omega_{mn} = \omega_{m} - \omega_{n} - \ii \gamma_{mn},
\end{equation}
从而以上三个过程的表达式仍然是正确的,但是此时$\omega_{mn}$具有虚部,其复共轭不等于它本身。
这在计算期望值$\mel*{\psi}{\cdot}{\psi}$时非常重要。
我们还假定
\begin{equation}
    \gamma_{mn} = \gamma_{nm},
\end{equation}
这个假设的合理性需要在\autoref{sec:electron-density-matrix}中看到。

我们在这里用电子波函数描述电子,而用经典的电磁波描述光,一方面没有做光场量子化就能够得到非线性响应,一方面确保了非线性响应的量子本质能够被体现出来。
在经典理论中是画不出以上三个过程的,原因是显然的:在经典理论中光和电子的耦合方式就是\autoref{fig:light-osci-couple},只有这个顶角的话,在经典的费曼图(严格区分“先发生”和“后发生”的过程,从而禁止相当一大类圈图)是画不出来电子吸收多个光子的过程的。
我们此处用实线表示的实际上是\emph{电子场}而不是\emph{电子坐标},或者说不是(像\autoref{fig:light-osci-couple}那样的)电子的振动模式。
二次量子化之后电场和电子的耦合实际上形如$A_\mu \bar{\psi} \psi$,因此的确有两条电子线。
虽然电场和电子的耦合中电场是线性的,电子线却有两条,因此仅仅用电子-光子相互作用顶角就能够构造出多光子的图。

\subsubsection{双侧费曼图方法}\label{sec:pure-double-sided-feynman}

耦合项$- \vb*{d} \cdot \vb*{E}$会导致电子受到电场影响,自然也会导致电场被电子激发出来。
本节讨论经典电磁场,从而不能真的用光子入射散射等概念计算等效光子-光子顶角。
经典电磁场中介质极化是新的波源,而极化矢量为
\begin{equation}
    \vb*{P} = N \expval*{\vb*{d}} = N \mel*{\psi}{\vb*{d}}{\psi},
\end{equation}
将$\vb*{P}$代入介质中的麦克斯韦方程,即可得到介质中光的行为。
我们这里直接计算$\vb*{d}$的期望值,以得到极化矢量,这个做法的合理性在于我们本质上还是在积掉电子,即在配分函数中保留电磁场不动,积掉电子场,计算(两边是基态的)关联函数。

这样,极化矢量对电场的一阶响应为
\begin{equation}
   \begin{aligned}
    &\quad \mel*{\psi^{(0)}}{\vb*{d}}{\psi^{(1)}} + \mel*{\psi^{(1)}}{\vb*{d}}{\psi^{(0)}} \\
    &= \sum_m \ee^{- \ii \omega_m t} \vb*{d}_{gm} \ee^{\ii \omega_g t} \frac{1}{\hbar} \sum_p \frac{\vb*{d}_{mg} \cdot \vb*{E}(\omega_p)}{\omega_{mg} - \omega_p} \ee^{\ii (\omega_{mg} - \omega_p) t} + \text{h.c.} \\
    &= \frac{1}{\hbar} \sum_{m, p} \left( \frac{ \vb*{d}_{gm} (\vb*{d}_{mg} \cdot \vb*{E}(\omega_p))}{\omega_{mg} - \omega_p} \ee^{- \ii \omega_p t} + \frac{(\vb*{d}_{gm} \cdot \vb*{E}(\omega_p) )^* \vb*{d}_{mg}}{\omega_{mg}^* - \omega_p} \ee^{\ii \omega_p t} \right) \\
    &= \frac{1}{\hbar} \sum_{m, p} \left( \frac{ \vb*{d}_{gm} (\vb*{d}_{mg} \cdot \vb*{E}(\omega_p))}{\omega_{mg} - \omega_p} \ee^{- \ii \omega_p t} + \frac{(\vb*{d}_{gm} \cdot \vb*{E}(\omega_p) ) \vb*{d}_{mg}}{\omega_{mg}^* + \omega_p} \ee^{- \ii \omega_p t} \right), 
   \end{aligned} 
   \label{eq:dipole-first-perturbation}
\end{equation}
其中第三个等号将第二项中的$\omega_p$换成了$-\omega_p$。更高阶的响应也可以用类似的方式获得。
在计算更高阶的响应时,直接展开计算是非常繁琐的,例如,$\vb*{d}$的期望值中电场的二次项为
\begin{equation}
    \begin{aligned}
        &\quad \mel*{\psi^{(0)}}{\vb*{d}}{\psi^{(2)}} + \mel*{\psi^{(2)}}{\vb*{d}}{\psi^{(0)}} + \mel*{\psi^{(1)}}{\vb*{d}}{\psi^{(1)}} \\
        &= \frac{1}{\hbar^2} \sum_{p, q} \sum_{m, n} \frac{\vb*{d}_{gn} (\vb*{d}_{nm} \cdot \vb*{E}(\omega_q)) (\vb*{d}_{mg} \cdot \vb*{E}(\omega_p))}{(\omega_{ng} - \omega_p - \omega_q) (\omega_{mg} - \omega_p)} \ee^{-\ii (\omega_p + \omega_q) t} \\
        &\quad + \frac{1}{\hbar^2} \sum_{p, q} \sum_{m, n} \frac{(\vb*{d}_{ng} \cdot \vb*{E}(\omega_q))^* \vb*{d}_{nm} (\vb*{d}_{mg} \cdot \vb*{E}(\omega_p))}{(\omega_{ng}^* - \omega_q) (\omega_{mg} - \omega_p)} \ee^{-\ii (\omega_p - \omega_q) t} \\
        &\quad + \frac{1}{\hbar^2} \sum_{p, q} \sum_{m, n} \frac{ (\vb*{d}_{ng} \cdot \vb*{E}(\omega_q))^* (\vb*{d}_{mn} \cdot \vb*{E}(\omega_p))^* \vb*{d}_{mg} }{(\omega_{ng}^* - \omega_q) (\omega_{mg}^* - \omega_p - \omega_q)} \ee^{\ii (\omega_p + \omega_q) t},
    \end{aligned}
\end{equation}
或者,通过调整求和变量让$\ee$指数完全成为$\ee^{- \ii (\omega_p + \omega_q) t}$,上式就是
\begin{equation}
    \begin{aligned}
        &\quad \mel*{\psi^{(0)}}{\vb*{d}}{\psi^{(2)}} + \mel*{\psi^{(2)}}{\vb*{d}}{\psi^{(0)}} + \mel*{\psi^{(1)}}{\vb*{d}}{\psi^{(1)}} \\
        &= \frac{1}{\hbar^2} \sum_{p, q} \sum_{m, n} \frac{\vb*{d}_{gn} (\vb*{d}_{nm} \cdot \vb*{E}(\omega_q)) (\vb*{d}_{mg} \cdot \vb*{E}(\omega_p))}{(\omega_{ng} - \omega_p - \omega_q) (\omega_{mg} - \omega_p)} \ee^{-\ii (\omega_p + \omega_q) t} \\
        &\quad + \frac{1}{\hbar^2} \sum_{p, q} \sum_{m, n} \frac{(\vb*{d}_{gn} \cdot \vb*{E}(\omega_q)) \vb*{d}_{nm} (\vb*{d}_{mg} \cdot \vb*{E}(\omega_p))}{(\omega_{ng}^* + \omega_q) (\omega_{mg} - \omega_p)} \ee^{-\ii (\omega_p + \omega_q) t} \\
        &\quad + \frac{1}{\hbar^2} \sum_{p, q} \sum_{m, n} \frac{ (\vb*{d}_{gn} \cdot \vb*{E}(\omega_q)) (\vb*{d}_{nm} \cdot \vb*{E}(\omega_p)) \vb*{d}_{mg} }{(\omega_{ng}^* + \omega_q) (\omega_{mg}^* + \omega_p + \omega_q)} \ee^{- \ii (\omega_p + \omega_q) t},
    \end{aligned}
\end{equation}
而三阶项就更加繁琐了。实际上,我们可以将这些计算总结为一套费曼图,其规则如下:
\begin{itemize}
    \item 电子线包括从下而上的左侧线和从上而下的右侧线,即所谓\emph{双边费曼图(double sided Feynman diagram)};
    \item 在左侧线最顶端放置$\vb*{d}$算符,使用一根波浪线代表它产生电磁场;
    \item 由于顶角总是有两条电子线,传播子和顶角可以合并成一个组件。在左侧线上,从下到上第$i$个顶角给出
    \begin{equation}
        \begin{gathered}
            \begin{tikzpicture}
                \begin{feynhand}
                    \vertex (g) at (0, 0);
                    \vertex (t) at (0, 2);
                    \vertex (l) at (-1, 0.5) {$\omega_i$};
                    \vertex (v) at (0, 1);
                    
                    \propag[plain] (g) to[edge label={$m$}] (v) ;
                    \propag[plain] (v) to[edge label={$n$}] (t);
                    \propag[extphoton] (l) to (v);
                \end{feynhand}
            \end{tikzpicture}
        \end{gathered} = \frac{\vb*{d}_{nm} \cdot \vb*{E}(\omega_p) }{\omega_{ng} - \sum_{j=1}^i \omega_j},
        \label{eq:feynman-diagram-left}
    \end{equation}
    其中$\omega_k$表示左侧线从下到上第$j$个光子线的频率。
    而在右侧线上,从下到上第$i$个顶角给出
    \begin{equation}
        \begin{gathered}
            \begin{tikzpicture}
                \begin{feynhand}
                    \vertex (g) at (0, 0);
                    \vertex (t) at (0, 2);
                    \vertex (l) at (1, 0.5) {$\omega_i$};
                    \vertex (v) at (0, 1);
                    
                    \propag[plain] (g) to[edge label={$m$}] (v) ;
                    \propag[plain] (v) to[edge label={$n$}] (t);
                    \propag[extphoton] (l) to (v);
                \end{feynhand}
            \end{tikzpicture}
        \end{gathered} = \frac{\vb*{d}_{mn} \cdot \vb*{E}(\omega_p) }{\omega_{ng}^* + \sum_{j=1}^i \omega_j}.
        \label{eq:feynman-diagram-right}
    \end{equation}
    其中$\omega_k$表示右侧线从下到上第$j$个光子线的频率。为了区分外场和$\mel*{\psi}{\vb*{d}}{\psi}$,我们用直线表示前者而用波浪线代表后者,虽然在凝聚态场论中我们通常用带有$\otimes$的波浪线代表前者。
\end{itemize}
例如,\eqref{eq:dipole-first-perturbation}可以用费曼图表示如下:
\begin{equation}
    \begin{gathered}
        \begin{tikzpicture}
            \begin{feynhand}
                \vertex (g1) at (-0.25, 0) {$g$};
                \vertex (g2) at (0.25, 0) {$g$};
                \vertex (t1) at (-0.25, 2.5);
                \vertex (t2) at (0.25, 2.5);
                \propag[plain] (t1) to[out=90, in=90] (t2);

                \vertex (v1) at (-0.25, 1) ;
                \vertex (l1) at (-1.55, 0.5) {$\omega_p$};
                \propag[extphoton] (l1) to (v1);
                \propag[plain] (g1) to (v1) ;

                \vertex (o) at (-0.25, 2);
                \vertex (e) at (-1.55, 2.5) {$\omega_p$};
                \propag[outphoton] (o) to (e);
                \propag[plain] (v1) to[edge label={$m$}] (o);

                \propag[plain] (o) to (t1);

                \propag[plain] (t2) to (g2);
            \end{feynhand}
        \end{tikzpicture}
    \end{gathered} = \frac{1}{\hbar} \sum_{m, p} \frac{ \vb*{d}_{gm} (\vb*{d}_{mg} \cdot \vb*{E}(\omega_p))}{\omega_{mg} - \omega_p} \ee^{- \ii \omega_p t} ,
    \label{eq:left-in-one-order-perturbation}
\end{equation}
以及
\begin{equation}
    \begin{gathered}
        \begin{tikzpicture}
            \begin{feynhand}
                \vertex (g1) at (-0.25, 0) {$g$};
                \vertex (g2) at (0.25, 0) {$g$};
                \vertex (t1) at (-0.25, 2.5);
                \vertex (t2) at (0.25, 2.5);
                \propag[plain] (t1) to[out=90, in=90] (t2);

                \vertex (v1) at (0.25, 1) ;
                \vertex (l1) at (1.55, 0.5) {$\omega_p$};
                \propag[extphoton] (l1) to (v1);
                \propag[plain] (g2) to (v1) ;

                \propag[plain] (v1) to[edge label'={$m$}] (t2);
                \propag[plain] (t1) to (o);

                \vertex (o) at (-0.25, 2);
                \vertex (e) at (-1.55, 2.5) {$\omega_p$};
                \propag[outphoton] (o) to (e);

                \propag[plain] (o) to (g1);
            \end{feynhand}
        \end{tikzpicture}
    \end{gathered} = \frac{1}{\hbar} \sum_{m, p} \frac{ \vb*{d}_{mg} (\vb*{d}_{gm} \cdot \vb*{E}(\omega_p) )}{\omega_{mg}^* + \omega_p} \ee^{- \ii \omega_p t}.
    \label{eq:right-in-one-order-perturbation}
\end{equation}
更高阶的响应也可以用类似的方式用费曼图计算,如二阶响应对应如下三个图:
\begin{equation}
    \begin{gathered}
        \begin{tikzpicture}
            \begin{feynhand}
                \vertex (g1) at (-0.25, 0) {$g$};
                \vertex (g2) at (0.25, 0) {$g$};
                \vertex (t1) at (-0.25, 3.5);
                \vertex (t2) at (0.25, 3.5);
                \propag[plain] (t1) to[out=90, in=90] (t2);

                \vertex (v1) at (-0.25, 1) ;
                \vertex (l1) at (-1.55, 0.5) {$\omega_p$};
                \propag[extphoton] (l1) to (v1);

                \vertex (v2) at (-0.25, 2);
                \vertex (l2) at (-1.55, 1.5) {$\omega_q$};
                \propag[extphoton] (l2) to (v2);

                \propag[plain] (g1) to (v1) ;
                \propag[plain] (v1) to[edge label={$m$}] (v2);

                \vertex (o) at (-0.25, 3);
                \vertex (e) at (-1.55, 3.5) {$\omega_p + \omega_q$};
                \propag[outphoton] (o) to (e);
                \propag[plain] (v2) to[edge label={$n$}] (o);

                \propag[plain] (o) to (t1);

                \propag[plain] (t2) to (g2);
            \end{feynhand}
        \end{tikzpicture}
    \end{gathered} = \frac{1}{\hbar^2} \sum_{p, q} \sum_{m, n} \frac{\vb*{d}_{gn} (\vb*{d}_{nm} \cdot \vb*{E}(\omega_q)) (\vb*{d}_{mg} \cdot \vb*{E}(\omega_p))}{(\omega_{ng} - \omega_p - \omega_q) (\omega_{mg} - \omega_p)} \ee^{-\ii (\omega_p + \omega_q) t},
    \label{eq:second-response-1}
\end{equation}
\begin{equation}
    \begin{gathered}
        \begin{tikzpicture}
            \begin{feynhand}
                \vertex (g1) at (-0.25, 0) {$g$};
                \vertex (g2) at (0.25, 0) {$g$};
                \vertex (t1) at (-0.25, 3.5);
                \vertex (t2) at (0.25, 3.5);
                \propag[plain] (t1) to[out=90, in=90] (t2);

                \vertex (v1) at (0.25, 1) ;
                \vertex (l1) at (1.55, 0.5) {$\omega_q$};
                \propag[extphoton] (l1) to (v1);
                \propag[plain] (g2) to (v1) ;

                \propag[plain] (v1) to[edge label'={$n$}] (t2);
                \propag[plain] (t1) to (o);

                \vertex (o) at (-0.25, 3);
                \vertex (e) at (-1.55, 3.5) {$\omega_p + \omega_q$};
                \propag[outphoton] (o) to (e);

                \propag[plain] (g1) to (v2);
                \vertex (v2) at (-0.25, 2);
                \vertex (l2) at (-1.55, 1.5) {$\omega_p$};
                \propag[extphoton] (l2) to (v2);
                \propag[plain] (v2) to[edge label={$m$}] (o);
            \end{feynhand}
        \end{tikzpicture}
    \end{gathered} = \frac{1}{\hbar^2} \sum_{p, q} \sum_{m, n} \frac{(\vb*{d}_{gn} \cdot \vb*{E}(\omega_q)) \vb*{d}_{nm} (\vb*{d}_{mg} \cdot \vb*{E}(\omega_p))}{(\omega_{ng}^* + \omega_q) (\omega_{mg} - \omega_p)} \ee^{-\ii (\omega_p + \omega_q) t} ,
\end{equation}
以及
\begin{equation}
    \begin{gathered}
        \begin{tikzpicture}
            \begin{feynhand}
                \vertex (g1) at (-0.25, 0) {$g$};
                \vertex (g2) at (0.25, 0) {$g$};
                \vertex (t1) at (-0.25, 3.5);
                \vertex (t2) at (0.25, 3.5);
                \propag[plain] (t1) to[out=90, in=90] (t2);

                \vertex (v1) at (0.25, 1) ;
                \vertex (l1) at (1.55, 0.5) {$\omega_q$};
                \propag[extphoton] (l1) to (v1);
                \propag[plain] (g2) to (v1) ;

                \vertex (v2) at (0.25, 2);
                \vertex (l2) at (1.55, 1.5) {$\omega_p$};
                \propag[extphoton] (l2) to (v2);
                \propag[plain] (v2) to[edge label={$n$}] (v1);

                \propag[plain] (v2) to[edge label'={$m$}] (t2);
                \propag[plain] (t1) to (o);

                \vertex (o) at (-0.25, 3);
                \vertex (e) at (-1.55, 3.5) {$\omega_p + \omega_q$};
                \propag[outphoton] (o) to (e);

                \propag[plain] (o) to (g1);
            \end{feynhand}
        \end{tikzpicture}
    \end{gathered} = \frac{1}{\hbar^2} \sum_{p, q} \sum_{m, n} \frac{ (\vb*{d}_{gn} \cdot \vb*{E}(\omega_q)) (\vb*{d}_{nm} \cdot \vb*{E}(\omega_p)) \vb*{d}_{mg} }{(\omega_{ng}^* + \omega_q) (\omega_{mg}^* + \omega_p + \omega_q)} \ee^{- \ii (\omega_p + \omega_q) t}.
\end{equation}

\subsubsection{极化矢量和极化率}

现在我们获得了$\vb*{d}$,于是可以计算出$\vb*{P}$,于是进一步可以计算出$\chi^{(1)}$, $\chi^{(2)}$等等。
在假定介质稀薄,或者说假定光只会被散射单次,并且介质中所有电子吸收入射光的概率均相同时,我们有
\[
    \vb*{P} = N \expval*{\vb*{d}} = \epsilon_0 \chi^{(1)}_{ij} E^j + \epsilon_0 \chi^{(2)}_{ij} E^i E^j + \cdots,
\]
即可得到
\begin{equation}
    \chi^{(1)}_{ij}(\omega_p) = \frac{N}{\epsilon_0 \hbar} \sum_{m} \left( \frac{ d_{gm}^i {d}_{mg}^j}{\omega_{mg} - \omega_p} + \frac{{d}_{mg}^i d_{gm}^j}{\omega_{mg}^* + \omega_p} \right)  ,
    \label{eq:linear-chi-pure}
\end{equation}
以及
\begin{equation}
    \begin{aligned}
        \chi^{(2)}_{ijk}(\omega_p + \omega_q, \omega_p, \omega_q) &= \frac{N}{\epsilon_0 \hbar^2} \mathcal{P}_\text{I} \sum_{m, n} \Bigg( \frac{d_{gn}^i d_{nm}^j d_{mg}^k }{(\omega_{ng} - \omega_p - \omega_q) (\omega_{mg} - \omega_p)} \\
        &\quad + \frac{d_{gn}^j d_{nm}^i d_{mg}^k }{(\omega_{ng}^* + \omega_q) (\omega_{mg} - \omega_p)} \\
        &\quad + \frac{ d_{gn}^j d_{nm}^k d_{mg}^i }{(\omega_{ng}^* + \omega_q) (\omega_{mg}^* + \omega_p + \omega_q)} \Bigg) .
    \end{aligned}
    \label{eq:second-chi-pure}
\end{equation}
等式坐标括号内的第一个频率是转换产生的光的频率,后面的频率是输入光的频率。
这里元算符$\mathcal{P}_\text{I}$将其后的表达式中的$\omega_p$和$\omega_q$交换,将$j$与$k$交换,并且求和所有可能的交换情况。
它的出现是因为以上所有求和式中$\sum_{p, q}$均未固定各个频率出现的顺序,从而,例如,如果输入光有两个频率分量$\omega_1, \omega_2$,那么$\omega_p = \omega_1, \omega_q = \omega_2$和$\omega_p = \omega_2, \omega_q = \omega_1$这两种情况都必须考虑进去;由于指标$j$和$\vb*{E}(\omega_q)$缩并,指标$k$和$\vb*{E}(\omega_p)$缩并,交换$\omega_p$和$\omega_q$也要求交换$j$和$k$。
因此,\eqref{eq:second-chi-pure}在去掉$\mathcal{P}_\text{I}$之后其实有六项。

从\eqref{eq:linear-chi-pure}和\eqref{eq:second-chi-pure}出发可以证明关于极化率的一些性质。
首先是熟知的二阶非线性极化对应空间反演对称性破缺。
在空间反演下,正比于$\vb*{r}$的电偶极矩$\vb*{d}$变号,此时$\chi^{(1)}$不变而$\chi^{(2)}$变号。
对具有空间反演不变性的系统,空间反演下张量不应该有变动,因此对具有空间反演不变性的系统,二阶非线性极化为零。
类似的其它从晶体的对称性在宏观层面分析得到的非线性极化率的性质也可以通过观察这些对称性对$\vb*{d}$的矩阵元的约束而获得。

在远离共振的频段,阻尼不会造成特别大的影响,从而\eqref{eq:second-chi-pure}中所有的$\omega_{mn}$都是实数。
此时\eqref{eq:second-chi-pure}可以进一步化简。
设
\begin{equation}
    \omega_s = \omega_p + \omega_q,
\end{equation}
我们会发现,如果我们定义元算符$\mathcal{P}_\text{F}$为将其后的表达式中的$(\omega_q, j)$,$(\omega_p, k)$和$(- \omega_s, i)$做所有可能的置换并求和,则可以验证
\begin{equation}
    \chi^{(2)}_{ijk}(\omega_s, \omega_p, \omega_q) = \frac{N}{\epsilon_0 \hbar^2} \mathcal{P}_\text{F} \sum_{m, n} \frac{d_{gn}^i d_{nm}^j d_{mg}^k }{(\omega_{ng} - \omega_s) (\omega_{mg} - \omega_p)}.
\end{equation}
这个表达式说明了另一个性质:在无阻尼的情况下,二阶非线性极化率中的$\omega_p, \omega_q, -\omega_s$是等价的:把它们连通$i, j, k$一起轮换不改变$\chi^{(2)}_{ijk}(\omega_s, \omega_p, \omega_q)$的值。
这是因为在没有阻尼时,二阶非线性极化可以通过在哈密顿量中加入一个等效相互作用
\begin{equation}
    {H}_\text{int} = \int \dd[3]{\vb*{r}} \frac{1}{3} \epsilon_0 \chi^{(2)} : \vb*{E} \vb*{E} \vb*{E}
\end{equation}
得到,从而$\chi^{(2)}$的$i, j, k$指标是等价的,切换到频域下后,三个电场分别用$\omega_p, \omega_q, \omega_s$标记,则$\omega_p, \omega_q, \omega_s$也应该是等价的。
轮换三个电场相当于同时轮换$i, j, k$和$\omega_p, \omega_q, \omega_s$,这对系统的行为不造成任何改变,这就解释了前述的$\chi^{(2)}$的性质。
$\omega_s$在轮换时带有负号可以用下图理解:
\begin{equation}
    \begin{gathered}
        \begin{tikzpicture}
            \begin{feynhand}
                \vertex (o) at (0, 0);
                \vertex (a) at (1.5, 0);
                \vertex (b) at (-0.75, 1.3);
                \vertex (c) at (-0.75, -1.3);
                
                \propag[photon, mom={$\omega_s$}] (o) to (a);
                \propag[photon, mom={$\omega_p$}] (b) to (o);
                \propag[photon, mom={$\omega_q$}] (c) to (o);
            \end{feynhand}
        \end{tikzpicture}
    \end{gathered} = \begin{gathered}
        \begin{tikzpicture}
            \begin{feynhand}
                \vertex (o) at (0, 0);
                \vertex (a) at (1.5, 0);
                \vertex (b) at (-0.75, 1.3);
                \vertex (c) at (-0.75, -1.3);
                
                \propag[photon, mom={$- \omega_s$}] (a) to (o);
                \propag[photon, mom={$\omega_p$}] (b) to (o);
                \propag[photon, mom={$\omega_q$}] (c) to (o);
            \end{feynhand}
        \end{tikzpicture}
    \end{gathered}
\end{equation}
需注意我们这里实际上还是在处理经典电磁场,从而需要严格区分一个顶角上的入射线和出射线,因此出射的$\omega_s$和入射的$\omega_q$和$\omega_p$在推导$\chi^{(2)}$时的地位看起来不同,但是在最终结果中一定有$\omega_p, \omega_q, -\omega_s$以及附带的$i, j, k$的轮换对称性成立。
下面给出这种轮换性的一些显式表达式,这里我们为了强调“转化”,将$\chi^{(2)}(\omega_s, \omega_q, \omega_p)$写成$\chi^{(2)}(\omega_s = \omega_q + \omega_p)$:
\begin{equation}
    \chi^{(2)}_{ijk}(\omega_s = \omega_q + \omega_p) = \chi^{(2)}_{jik}(- \omega_q = - \omega_s + \omega_p) = \chi^{(2)}_{jik}(\omega_q = \omega_s - \omega_p)^*, 
\end{equation}
\begin{equation}
    \chi^{(2)}_{ijk}(\omega_s = \omega_q + \omega_p) = \chi^{(2)}_{ikj}(\omega_s = \omega_p + \omega_q).
\end{equation}
容易看出更一般的轮换关系对任意阶的远离共振(从而阻尼可以忽略)的非线性极化率都成立,且当$\chi^{(n)}(\omega_s = \sum_i \omega_i)$括号内等式两边的频率从一侧移到另一侧时需要加复共轭。
例如对线性极化就有熟知的
\begin{equation}
    \chi^{(1)}_{ij}(\omega) = \chi^{(1)}_{ji}(-\omega) = \chi^{(1)}_{ji}(\omega)^* = \chi^{(1)}_{ij}(-\omega).
\end{equation}

总之,在量子力学的框架下,产生非线性极化并不需要介质内有非线性势或者类似的非线性相互作用。
在量子力学的框架下电子和光子的相互作用方式——一个电子吸收或者放出一个光子——自然地允许出现\eqref{eq:second-response-1}这样的过程;而另一方面,经典理论禁止多条外线的出现,从而也无法在介质内部没有非线性相互作用的情况下得到非线性响应。

我们此处做的计算积掉了电子自由度。电子在电磁波的作用下没有处在某个明确的能级上,虽然费曼图中我们说电子可能处在$m$或$n$号能级,但这都是离壳的。
光学上,我们说此时电子处在“虚能级”上。后文中当光和介质中实实在在的模式(如声子)耦合时,将会涉及“实能级”,因为声子的发射和吸收让系统从一个明确的能级跃迁到另一个明确的能级上。

实际上,虽然本节在一开始似乎假定了系统处在束缚态,但以上推导完全没有用到不同能级之间有有限的能隙这个条件。(这和诸如拓扑序的波函数重整化之类的东西是不同的,在那些话题中,能隙是至关重要的,因为要保证小的激发不会让系统跑到激发态上去的概率变得很大;本节所述的过程都是求和了所有可能的跃迁,所以有没有能隙不重要)
因此,对那些电子可以长距离运动的系统,如果能够适用能带理论,即单电子近似仍然成立,我们以上的推导在考虑了诸如电子的激发在费米面附近等费米统计带来的特殊效应后仍然是适用的。
一旦通过各种方法——比如DFT计算——得到系统能谱,就能够计算出非线性极化率。

最后我们需要指出,我们目前做的所有计算都依赖两个假设:一个是电子之间的库仑相互作用可以忽略,即我们无需真的去分析一个多体系统;另一个假设是介质充分稀薄。
这第二个假设不成立时我们就有局域场修正,即介质外部的电磁场进入到介质分子或原子附近时会因为邻近的原子的响应而受到“修饰”或者说“增强”。

\subsubsection{与凝聚态场论方法的一致性}

双边费曼图的规则看起来非常奇特,但是实际上可以将它理解为凝聚态场论中的图。
它和一般的凝聚态场论的不同之处在于,本节讨论的带有弛豫的模型是非幺正的,时间反演不变性破缺,从而传播子的书写有一些需要注意的地方。
大体上说我们是要计算
\[
    \expval*{{c}^\dagger_m(t) \vb*{d}_{mn} c_n(t) }, \quad \ket*{\Omega} = c^\dagger_g(-\infty) \ket*{0}.
\]
我们有两种做法,其中之一是直接计算$\mel*{0}{c_g(-\infty) {c}^\dagger_m(t) \vb*{d}_{mn} c_n(t) c^\dagger_g(-\infty)}{0}$,对应的费曼图形如
\begin{equation}
    \sum_{m, n} \vb*{d}_{mn} \begin{gathered}
        \begin{tikzpicture}
            \begin{feynhand}
                \vertex [grayblob] (o) at (0, 0) {};
                \vertex (a) at (-1, 1) {$g, -\infty$};
                \vertex (b) at (-1, -1) {$g, -\infty$};
                \vertex (c) at (1, 1) {$m, t$};
                \vertex (d) at (1, -1) {$n, t$};
    
                \propag[fermion] (a) to (o);
                \propag[fermion] (o) to (b);
                \propag[fermion] (o) to (c);
                \propag[fermion] (d) to (o);
            \end{feynhand}
        \end{tikzpicture}
    \end{gathered},
    \label{eq:double-sided-feynman-diagram-in-condensed}
\end{equation}
其中$m, n$两端均可以不在壳,即有对应的传播子。这是可以理解的,因为如果电磁场也是量子化的,上图实际上就是
\[
    \begin{gathered}
        \begin{tikzpicture}
            \begin{feynhand}
                \vertex [grayblob] (o) at (0, 0) {};
                \vertex (a) at (-1, 1) {$g, -\infty$};
                \vertex (b) at (-1, -1) {$g, -\infty$};
                \vertex (e) at (1.75, 0);
                \vertex (f) at (2.75, 0);
    
                \propag[fermion] (a) to (o);
                \propag[fermion] (o) to (b);
                \propag[fermion] (o) to[out=45, in=135] (e);
                \propag[fermion] (e) to[in=315, out=225] (o);
                \propag[photon] (e) to (f);
            \end{feynhand}
        \end{tikzpicture}
    \end{gathered}.
\]
在本节中我们无需计算上图最右边的产生光子的顶角,光子产生是在非线性麦克斯韦方程中通过极化矢量引入的。

\eqref{eq:double-sided-feynman-diagram-in-condensed}中的$g$外线实际上是系统基态的一部分,因此我们不能将两个$g$外线连接起来,否则得到的实际上是真空气泡图。因此,我们只需要考虑下图
\begin{equation}
    \sum_{m, n} \vb*{d}_{mn} \ee^{- \ii (\omega_m - \omega_n) t} \begin{gathered}
        \begin{tikzpicture}
            \begin{feynhand}
                \vertex (a) at (-2, 1) {$g, -\infty$};
                \vertex (b) at (-2, 0) {$g, -\infty$};
                \vertex (c) at (2, 1) {$m, t$};
                \vertex (d) at (2, 0) {$n, t$};
                \vertex[grayblob] (o1) at (0, 1) {};
                \vertex[grayblob] (o2) at (0, 0) {};
                
                \propag[fermion] (a) to (o1);
                \propag[fermion] (o1) to (c);
                \propag[fermion] (o2) to (b);
                \propag[fermion] (d) to (o2);
            \end{feynhand}
        \end{tikzpicture}
    \end{gathered}
    \label{eq:general-form-condensed}
\end{equation}
即可,其中圆圈内是一个或多个光子吸收或发射过程。由于本节的系统时间反演对称性破缺,因为存在弛豫项,图
\[
    \begin{gathered}
        \begin{tikzpicture}
            \begin{feynhand}
                \vertex (a) at (-2, 1) {$g, -\infty$};
                \vertex (c) at (2, 1) {$m, t$};
                \vertex[grayblob] (o1) at (0, 1) {};
                
                \propag[fermion] (a) to (o1);
                \propag[fermion] (o1) to (c);
            \end{feynhand}
        \end{tikzpicture}
    \end{gathered}
\]
中的频域电子传播子是
\[
    \int \ee^{\ii \omega t} \dd{t} T\expval*{\psi_m(t) \bar{\psi}_g} = \frac{\ii}{\omega - \omega_m + \ii \gamma},
\]
而
\[
    \begin{gathered}
        \begin{tikzpicture}
            \begin{feynhand}
                \vertex (a) at (-2, 1) {$g, -\infty$};
                \vertex (c) at (2, 1) {$m, t$};
                \vertex[grayblob] (o1) at (0, 1) {};
                
                \propag[fermion] (o1) to (a);
                \propag[fermion] (c) to (o1);
            \end{feynhand}
        \end{tikzpicture}
    \end{gathered}
\]
中的频域电子传播子是
\[
    \int \ee^{\ii \omega t} \dd{t} T\expval*{{\psi}_g \bar{\psi}_m(t)} = \frac{\ii}{\omega - \omega_m - \ii \gamma}.
\]
$\mel{\psi}{\vb*{d}}{\psi}$中有一系列从$-\infty$演化到我们要计算的时间点的粒子线,也有一系列从我们要计算的时间点演化回$-\infty$的粒子线,它们的电子传播子是不同的:在时间反演变换下$\gamma$需要加上一个负号。
从\eqref{eq:dipole-first-perturbation}出发,我们也可以将$\gamma$的负号理解为是$\bra{\psi}$相比于$\ket*{\psi}$取了复共轭而产生的。
对$\gamma$的负号的两种理解是一致的,因为时间反演和复共轭紧密相关。
\eqref{eq:general-form-condensed}中的$-\infty$演化到我们要计算的时间点的一系列电子线实际上就是双边费曼图的左侧电子线,而\eqref{eq:general-form-condensed}中从我们要计算的时间点演化到$-\infty$的一系列电子线实际上就是双边费曼图中的右侧电子线。
右边的传播子中的$\gamma$要多一个负号。

基于以上考虑,把这些费曼图当成凝聚态场论的图理解,可以写出
\begin{equation}
    \begin{aligned}
        &\quad \sum_{m, n} \vb*{d}_{mn} \ee^{-\ii (\omega_m - \omega_n) t} \times \left( \begin{gathered}
            \begin{tikzpicture}
                \begin{feynhand}
                    \vertex[crossdot] (l) at (0, 2) {};
                    \vertex (a) at (-2, 1) {$g, -\infty$};
                    \vertex (b) at (-2, 0) {$g, -\infty$};
                    \vertex (c) at (2, 1) {$m, t$};
                    \vertex (d) at (2, 0) {$n, t$};
                    \vertex (v) at (0, 1);

                    \propag[photon, mom={$\omega_p$}] (l) to (v);
                    \propag[fermion] (a) to (v);
                    \propag[fermion] (v) to (c);
                    \propag[fermion] (d) to (b);
                \end{feynhand}
            \end{tikzpicture}
        \end{gathered} \right) \\
        &= \sum_p \vb*{d}_{gm} \ee^{-\ii \omega_p t} \sum_{m} \frac{\ii}{\omega_p + \omega_g - \omega_m + \ii \gamma_{mg}} \frac{(\ii \vb*{d}_{mg} \cdot \vb*{E}(\omega_p))}{\hbar},
    \end{aligned}
\end{equation}
计算结果和\eqref{eq:left-in-one-order-perturbation}是一致的。同样我们使用凝聚态场论的理解方式计算入射光子在右侧电子线的图,有
\begin{equation}
    \begin{aligned}
        &\quad \sum_{m, n} \vb*{d}_{mn} \ee^{-\ii (\omega_m - \omega_n) t} \times \left( \begin{gathered}
            \begin{tikzpicture}
                \begin{feynhand}
                    \vertex[crossdot] (l) at (0, -1) {};
                    \vertex (a) at (-2, 1) {$g, -\infty$};
                    \vertex (b) at (-2, 0) {$g, -\infty$};
                    \vertex (c) at (2, 1) {$m, t$};
                    \vertex (d) at (2, 0) {$n, t$};
                    \vertex (v) at (0, 0);

                    \propag[photon, mom={$\omega_p$}] (l) to (v);
                    \propag[fermion] (a) to (c);
                    \propag[fermion] (d) to (v);
                    \propag[fermion] (v) to (b);
                \end{feynhand}
            \end{tikzpicture}
        \end{gathered} \right) \\
        &= \sum_p \vb*{d}_{mg} \ee^{- \ii \omega_p t} \sum_m \frac{\ii \vb*{d}_{gm} \cdot \vb*{E}(\omega_p)}{\hbar} \frac{\ii}{\omega_g - \omega_p - \omega_m + \ii \gamma_{mg}} ,
    \end{aligned}
\end{equation}
和\eqref{eq:right-in-one-order-perturbation}一致。
类似的可以验证,\eqref{eq:feynman-diagram-left}和\eqref{eq:feynman-diagram-right}两个费曼规则也是可以通过凝聚态场论导出的,其中诸如$\sum \omega_i$这样奇怪的表达式实际上是能量守恒条件的结果。

\subsubsection{密度矩阵表述}\label{sec:electron-density-matrix}

以上所有的计算中的非幺正效应都是手动添加在传播子中的,并且我们到最后实际上又忽略了它们。
本节给出对电子弛豫到基态这件事的稍微严格一些的处理。
我们使用密度矩阵描述电子状态。实际系统中总是存在各种各样的过程让处于激发态的电子回到基态,如自发辐射、碰撞等。唯象地引入参数$\gamma_{mn}$,写出密度矩阵的时间演化方程
\begin{equation}
    \dot{\rho}_{mn} = - \frac{\ii}{\hbar} \comm*{H}{\rho}_{mn} - \gamma_{mn} (\rho_{mn} - \rho^\text{eq}_{mn}),
    \label{eq:electron-density-matrix-evolve}
\end{equation}
通常可以假定电子处在热态,而由于$\{\ket*{m}\}$是能量本征态,有
\begin{equation}
    \rho^\text{eq}_{mn} = \rho^\text{eq}_{mm} \delta_{mn},
\end{equation}
其中$\rho^\text{eq}_{mn}$给出热平衡态分布。

方程\eqref{eq:electron-density-matrix-evolve}

实际上,\eqref{eq:electron-density-matrix-evolve}的解也可以画成双边费曼图,在每个时间点,有\emph{两个}状态,分别对应密度矩阵的左矢和右矢,因此电子线包括左右两部分。
但此时的双边费曼图和\autoref{sec:pure-double-sided-feynman}中有所差别,主要体现在传播子的表达式不同,以及会多出来几个图。
例如,密度矩阵的双边费曼图同时包括下面的两张图:
\[
    \begin{gathered}
        \begin{tikzpicture}
            \begin{feynhand}
                \vertex (g1) at (-0.25, 0) {$\ket*{g}$};
                \vertex (g2) at (0.25, 0) {$\bra*{g}$};
                \vertex (t1) at (-0.25, 4) {$\ket*{n}$};
                \vertex (t2) at (0.25, 4) {$\bra*{n}$};

                \vertex (v1) at (0.25, 1) ;
                \vertex (l1) at (1.55, 0.5) {$\omega_q$};
                \propag[extphoton] (l1) to (v1);
                \propag[plain] (g2) to (v1) ;
                
                \propag[plain] (v1) to (t2);

                \vertex (v2) at (-0.25, 2);
                \vertex (l2) at (-1.55, 1.5) {$\omega_p$};
                \propag[extphoton] (l2) to (v2);
                \propag[plain] (g1) to (v2);

                \vertex (o) at (-0.25, 3);
                \vertex (e) at (-1.55, 3.5) {$\omega_p + \omega_q$};
                \propag[outphoton] (o) to (e);
                \propag[plain] (v2) to[edge label={$\ket*{m}$}] (o);
                \propag[plain] (o) to (t1);
            \end{feynhand}
        \end{tikzpicture}
    \end{gathered} , \quad \begin{gathered}
        \begin{tikzpicture}
            \begin{feynhand}
                \vertex (g1) at (-0.25, 0) {$\ket*{g}$};
                \vertex (g2) at (0.25, 0) {$\bra*{g}$};
                \vertex (t1) at (-0.25, 4) {$\ket*{n}$};
                \vertex (t2) at (0.25, 4) {$\bra*{n}$};

                \vertex (v1) at (0.25, 2) ;
                \vertex (l1) at (1.55, 1.5) {$\omega_q$};
                \propag[extphoton] (l1) to (v1);
                \propag[plain] (g2) to (v1) ;
                
                \propag[plain] (v1) to (t2);

                \vertex (v2) at (-0.25, 1);
                \vertex (l2) at (-1.55, 0.5) {$\omega_p$};
                \propag[extphoton] (l2) to (v2);
                \propag[plain] (g1) to (v2);

                \vertex (o) at (-0.25, 3);
                \vertex (e) at (-1.55, 3.5) {$\omega_p + \omega_q$};
                \propag[outphoton] (o) to (e);
                \propag[plain] (v2) to[edge label={$\ket*{m}$}] (o);
                \propag[plain] (o) to (t1);
            \end{feynhand}
        \end{tikzpicture}
    \end{gathered} ,
\]
第一张图中电子状态的演化路径是
\[
    \dyad{g} \longrightarrow \dyad*{g}{n} \longrightarrow \dyad*{m}{n} \longrightarrow \dyad*{n}{n},
\]
第二张图中电子状态的演化路径为
\[
    \dyad*{g} \longrightarrow \dyad*{m}{g} \longrightarrow \dyad*{m}{n} \longrightarrow \dyad*{n}{n},
\]
两者不同。因此的确,左右两边的入射光子线如果上下位置不同,则产生不同的图。
实际上,此处的密度矩阵的双边费曼图实际上是Keldysh场论的一个特例(见\cite{Hansen_2012})。
最终可以证明,在远离共振的情况下两者是一致的。

\subsection{空间相位}

TODO:多个分子之间的间距如果恰到好处地让它们的辐射相位正好差了$\pi$,有可能在远处观察不到辐射



\section{二阶非线性极化的波动光学}

之前已经讨论过一个问题:具有两个频率分量的泵浦光被打入具有二阶非线性极化的材料中,计算和频光的强度。本节用非线性波动方程重新计算它。
二阶非线性极化意味着材料的中心反演对称性破缺。

本节中如无说明,$\omega>0$,能量守恒条件形如$\omega_3 = \omega_1 + \omega_2$,我们总是使用$\omega_3$表示频率最大的模式。

\subsection{SFG过程的一阶微扰}

\subsubsection{输入光为两束泵浦光}

我们考虑一个最为简单的情况:各向同性介质,泵浦光足够强,以至于在介质中泵浦光几乎没有衰减,即做了\concept{泵浦波无衰减近似}。
假定泵浦光中有两个频率分量的光,它们的偏振方向均相同。取其波矢指向为$z$轴。
我们用$E_i$指代第$i$种傅里叶分量,用$A_i$表示其大小,即
\[
    E_i = \ee^{\ii k_i z - \ii \omega_i t} A_i + \ee^{- \ii k_i z + \ii \omega_i t} A_i^*.
\]
在线性光学中只取正频率或是负频率的分量不会造成任何问题,并且通常会让问题更加容易,但是我们现在在讨论非线性过程,因此$E$必须是实际的入射光,显然必须是实数。
这样,非线性极化导致的一阶微扰中%
\footnote{
    “阶”在本文中有两个意思,一个是非线性光学过程的阶数,$n$阶非线性光学过程就是$n$束光变成一束光;还有一个意思是微扰计算的阶数。
    前者描述单个顶角的粒子线数目,后者描述一张图中顶角的总数。
}%
,SFG过程给出修正(我们为何只考虑SFG过程,而且只考虑$\omega_2$和$\omega_3$指定的SFG过程马上就可以看到;相位匹配条件决定了SFG过程和DFG过程等不太可能同时很重要)
\[
    \left( \pdv[2]{z} - \frac{\epsilon_\text{r}}{c^2} \pdv[2]{t} \right) E_3 = \frac{1}{\epsilon_0 c^2} \pdv[2]{t} \left( 2 \epsilon_0  \chi^{(2)} A_1 A_2 \ee^{\ii (k_1 + k_2) z} \ee^{-\ii \omega_3 t} \right) + \text{h.c.},
\]
其中我们设$\omega_3 = \omega_1 + \omega_2$,$\chi^{(2)}$指的是$\chi^{(2)}(\omega_3=\omega_1+\omega_2)$。
上式右边的因子$2$来自$\chi^{(2)} \vb*{E} \vb*{E}$展开之后——实际上,也可以用一种更绕的办法得出这个因子:设拉氏量中产生了$\chi^{(2)} \vb*{E} \vb*{E}$项的项的耦合常数为$\alpha$,则大体上
\[
    \dv{\vb*{E}} (\alpha \vb*{E} \vb*{E} \vb*{E}) = \chi^{(2)} \vb*{E} \vb*{E},
\]
有
\[
    \alpha = \frac{1}{3} \chi^{(2)},
\]
由于$\alpha$对应一个三条粒子线的顶角,它需要乘上对称性因子$3!$,于是最后就有因子$3! / 3 = 2$。
请注意没有什么能够保证线性极化是频率无关的——没有理由认为$\omega=\omega(k)$是线性的。
$E_3$的频率当然还是$\omega_3$。我们假定$E_3$大体上仍然是一个平面波(即做了\concept{平面波近似}),但是随着$z$增大,$A_3$会有变化。这样就有
\[
    \ee^{\ii k_3 z - \ii \omega_3 t} \left( \pdv[2]{z} + 2 \ii k_3 \pdv{z} - k_3^2 + \frac{\epsilon_\text{r}}{c^2} \omega^2 \right) A_3 = - \frac{\omega_3^2}{\epsilon_0 c^2} 2 \epsilon_0  \chi^{(2)} A_1 A_2 \ee^{\ii (k_1 + k_2) z} \ee^{-\ii \omega_3 t} .
\]
由于色散关系可以不是线性的,从而$k_3$和$k_1 + k_2$完全可以不同。
上式左边括号中最后两项抵消了。再做\emph{慢变振幅近似}(几何光学中就做了这一近似,也可以称为旁轴近似),就有
\[
    \pdv{z} A_3 = \frac{\ii \omega_3^2}{k_3 c^2} \chi^{(2)} A_1 A_2 \ee^{\ii (k_1 + k_2 - k_3) z}.
\]
求解这一方程,考虑到泵浦光中没有$E_3$分量,从而$z=0$处$A_3=0$,以及晶体厚度为$L$,就得到
\begin{equation}
    A_3(L) = \frac{\ii \chi^{(2)} \omega_3^2 A_1 A_2}{k_3 c^2} \left( \frac{\ee^{\ii \Delta k L} - 1}{\ii \Delta k} \right),
    \label{eq:two-pump-a3}
\end{equation}
其中
\begin{equation}
    \Delta k = k_1 + k_2 - k_3.
\end{equation}
$E_3$分量的强度是
\[
    I_3 = \expval*{\epsilon_0 n c E_3^2} = 2 \epsilon_0 n c \abs*{A_3}^2,
\]
于是就得到
\begin{equation}
    I_3 = \frac{8 \epsilon_0 (\chi^{(2)})^2 \abs*{A_1}^2 \abs*{A_2}^2 \omega_3^2}{nc} \frac{\sin^2(\Delta k L / 2)}{(\Delta k)^2} = \frac{(\chi^{(2)})^2 \omega_3^2 I_1 I_2}{2 \epsilon_0 c^3 n_1 n_2 n_3} \mathrm{sinc}^2 \left( \frac{\Delta k L}{2} \right) L^2.
    \label{eq:sfg-intensity}
\end{equation}

如果\concept{相位匹配条件}——即$k_3 = k_1 + k_2$——的确成立,那么$I_3$就稳定地随着$L^2$增大而线性增大,此时的SFG转换效率是非常高的。
当然,$L$大到一定程度时,$I_1$和$I_2$就开始随着$L$增大而下降,并且吸收也开始明显了。
大部分情况下相位匹配条件是没法成立的,因为色散非线性。此时随着$L$的增大,能量先是从$E_1$和$E_2$移向$E_3$模式,然后再从$E_3$模式移向$E_1$和$E_2$模式。
大约需要
\begin{equation}
    L_\text{coh} = \frac{1}{\Delta k}
\end{equation}
的距离可以看到明显的能量转移,这个长度称为\concept{相干长度}。

相位匹配条件叫做这个名字并非没有原因。在某个位置$z$处新产生的$E_3$光大体上形如
\[
    E_3 \sim E_1 E_2 \sim A_1 A_2 \ee^{\ii (k_1 + k_2) z - \ii (\omega_1 + \omega_2) t},
\]
如果$\Delta k \neq 0$,那么新产生的$E_3$光和原来已有的$E_3$光就会有一个不断变化的相位差,在一些地方它们会发生相消干涉。
这就是$\Delta k \neq 0$时$E_3$光强度发生振荡的原因。

相位匹配时才有较高效的能量转换这一事实实际上大大简化了我们的分析。
原则上,我们将频率为$\omega_1$和$\omega_2$的光输入一个非线性晶体,所有过程——和频,差频,倍频,光学整流——都会发生,因此我们\emph{不能}只取一个过程并按照它写出振幅分布方程。
例如,种子光可以因为DFG过程而指数上升,也可以因为SFG过程而振荡,这两个效应按理说应该叠加,然后反过来影响SFG和DFG产生的光的强度。
然而,很多时候,入射光给定之后SFG和DFG过程或者其它一些过程不能同时满足相位匹配条件,从而\eqref{eq:sfg-intensity}分母中的$\Delta k$会压低$I_3$能够达到的最大值。
因此,其实可以分开计算每个过程。
在有广谱的输入光——如制作光谱仪时——这个近似可能不再适用,此时需要做完整的三波混频方程求解。

\subsubsection{输入光为泵浦光和种子光}

本节考虑一个有些不同的情况:输入光还是有两个频率分量,但是其中一个频率分量是很弱的,从而SFG过程产生的光的幅度振荡的同时,那个比较弱的光的幅度也会振荡。
用1表示泵浦光,2表示那个频率分量较弱的输入光(即所谓种子光),3表示SFG过程产生的光。
此时的非线性麦克斯韦方程为
\[
    \begin{aligned}
        \left( \pdv[2]{z} - \frac{\epsilon_\text{r}}{c^2} \pdv[2]{t} \right) E_3 &= \frac{1}{\epsilon_0 c^2} \pdv[2]{t} \left( 2 \epsilon_0  \chi^{(2)} A_1 A_2 \ee^{\ii (k_1 + k_2) z} \ee^{-\ii \omega_3 t} \right) + \text{h.c.}, \\
        \left( \pdv[2]{z} - \frac{\epsilon_\text{r}}{c^2} \pdv[2]{t} \right) E_2 &= \frac{1}{\epsilon_0 c^2} \pdv[2]{t} \left( 2 \epsilon_0  \chi^{(2)} A_1^* A_3 \ee^{\ii (- k_1 + k_3) z} \ee^{-\ii \omega_2 t} \right) + \text{h.c.}.
    \end{aligned}
\]
使用和上一节类似的慢变振幅近似,我们有
\begin{equation}
    \begin{aligned}
        \pdv{A_3}{z} &= \frac{\ii \omega_3^2}{k_3 c^2} \chi^{(2)} A_1 A_2 \ee^{\ii \Delta k z}, \\
        \pdv{A_2}{z} &= \frac{\ii \omega_2^2}{k_2 c^2} \chi^{(2)} A_1^* A_3 \ee^{- \ii \Delta k z}.
    \end{aligned}
    \label{eq:pump-and-seed-sfg}
\end{equation}
我们需要尝试将$A_2$和$A_3$的方程解耦。在相位匹配条件成立的情况下,只需要让以上两个方程两边各自对$z$再求一次导数,即可得到分别关于$A_2$和$A_3$的两个二阶微分方程
\[
    \pdv[2]{A_3}{z} = - \frac{\omega_3^2 \omega_2^2}{k_2 k_3 c^4} \abs*{A_1}^2 (\chi^{(2)})^2 A_3, \quad \pdv[2]{A_2}{z} = - \frac{\omega_3^2 \omega_2^2}{k_2 k_3 c^4} \abs*{A_1}^2 (\chi^{(2)})^2 A_2.
\]
在相位匹配条件不成立的时候,对\eqref{eq:pump-and-seed-sfg}中每个方程求导时等式右边会多出来一个$A_2$或$A_3$项,让求解变得困难。
然而,根据之前“相位不匹配会导致新产生的光和已有的光在一些地方发生相消干涉”的物理图像,我们仍然可以确信,相位不匹配会导致能量转化效率下降。

在相位匹配条件成立时,代入边界条件$A_3(0) = 0$,解得
\begin{equation}
    A_2(z) = A_2(0) \cos(k z), \quad A_3(z) = \ii \sqrt{\frac{\omega_3 n_2}{\omega_2 n_3}} \frac{A_1}{\abs*{A_1}} A_2(0) \sin(k z),
\end{equation}
其中
\begin{equation}
    k^2 = \frac{\omega_3^2 \omega_2^2}{k_2 k_3 c^4} \abs*{A_1}^2 (\chi^{(2)})^2. 
\end{equation}
我们发现一开始,能量从$\omega_2$模式转向$\omega_3$模式,但是一段时间之后能量又从$\omega_3$模式转向$\omega_2$模式。
两个耦合的模式如此来回振荡。存在能量从$E_3$模式流向$E_2$模式的过程意味着在完全经典的计算中,SFG过程的逆过程也是可行的,虽然由于没有量子涨落,必须有种子光这个过程才能发生。
在本节计算的例子中,泵浦光$E_1$起到了$\omega_3$转化为$\omega_1$和$\omega_2$的种子光的作用。

\subsection{DFG过程的一阶微扰}

我们现在转而考虑DFG过程。前面提到过,经典理论中DFG过程不能从一束光凭空产生两束光。
于是我们考虑一个这样的过程:$E_3$是基本上不衰减的泵浦光,它将要产生$E_1$和$E_2$两束光;$E_1$光也是输入光,即$E_1$模式上有种子光;$E_2$在$z=0$处为零。
使用慢变振幅近似,有以下方程:
\[
    \begin{aligned}
        \pdv{A_1}{z} &= \frac{\ii \chi^{(2)}(\omega_1 = \omega_3 - \omega_2) \omega_1^2}{k_1 c^2} A_3 A_2^* \ee^{-\ii \Delta k z}, \\
        \pdv{A_2}{z} &= \frac{\ii \chi^{(2)}(\omega_2 = \omega_3 - \omega_1) \omega_2^2}{k_2 c^2} A_3 A_1^* \ee^{-\ii \Delta k z}.
    \end{aligned}
\]
显然我们可以指定
\begin{equation}
    \chi^{(2)}(\omega_1 = \omega_3 - \omega_2) = \chi^{(2)}(\omega_2 = \omega_3 - \omega_1) = \chi^{(2)}.
\end{equation}
求解以上方程,得到
\begin{equation}
    A_1(z) = A_1(0) \cosh(\kappa z), \quad A_2(z) = \ii \sqrt{\frac{n_1 \omega_2}{n_2 \omega_1}} \frac{A_3}{\abs*{A_3}} A_1^*(0) \sinh(\kappa z),
    \label{eq:dfg-amplitude}
\end{equation}
其中
\begin{equation}
    \kappa^2 = \frac{\omega_1^2 \omega_2^2}{k_1 k_2 c^4} \abs*{A_3}^2 (\chi^{(2)})^2 .
\end{equation}
这里,$\omega_1$和$\omega_2$光都指数增长。这里的关键点在于关于$A_1$的方程右边的$A_2$取了复共轭。
物理图像上,单泵浦光的DFG过程中,$\omega_1$光子的出现能够刺激$\omega_2$光子的出现,反之亦然,因此一旦有$\omega_1$光和泵浦光同时出现,$\omega_1$光和$\omega_2$光就会不断扩增,从而指数增长。
反之,在单泵浦光输入的SFG过程中,产生$\omega_3$光子会消耗$\omega_2$光子,但是$\omega_3$光子不能诱发更多$\omega_1$光子转化为$\omega_2$光子,从而$\omega_2$光子和$\omega_3$光子存在竞争关系。

从以上解可以看到一个有趣的现象,就是种子光$\omega_1$的相位其实是自行决定的,而$\omega_2$光的相位同时由泵浦光$\omega_3$和和种子光$\omega_1$的相位决定。
这意味着非线性晶体不仅可以用来从一个频率的光源产生另一个频率的光源,还可以用于产生相位特定的新光源。

\subsection{相位匹配条件的实现}

如前所述,只有相位匹配时SFG过程或是DFG过程才足够明显。本节讨论给定三个任意频率的光,如何让它们能够满足相位匹配条件。
由于介质色散的存在,相位匹配条件不总是能够完成的,因为联立方程
\[
    \omega_3 = \omega_1 + \omega_2, \quad \omega_3 n(\omega_3) = \omega_2 n(\omega_2) + \omega_1 n(\omega_1)
\]
未必有解。实际上,在所谓的正常折射率的情况下——即在$n$随着$\omega$增大而增大的情况下——这个方程就是无解的,因为显然
\[
    \omega_3 > \omega_1, \quad \omega_3 > \omega_2,
\]
从而
\[
    \omega_3 n(\omega_3) > \omega_1 n(\omega_3) + \omega_2 n(\omega_3) > \omega_1 n(\omega_1) + \omega_2 n(\omega_2).
\]
当然,为了实现相位匹配,我们可以去寻找一种特殊的材料,它在$\omega_3$附近有反常折射率,从而让相位匹配条件能够成立。但这样的材料显然不那么好找。

\subsubsection{双折射}

一种获得相位匹配的方式是使用双折射晶体或者说单轴晶体。设我们有一块同时展现非线性光学效应和双折射的晶体——这样的晶体并不难找,因为大部分晶体的折射率都具有各向异性。
大体上,我们有两种方法可以得到较强的二阶非线性过程:一种是所谓的Type I相位匹配,其中$\omega_1$光和$\omega_2$光的偏振方向一致。对负单光轴晶体我们要设法让
\begin{equation}
    n^\text{e}_3 \omega_3 = n^\text{o}_1 \omega_1 + n^\text{o}_2 \omega_2
\end{equation}
成立,而对正光轴晶体我们则设法让
\begin{equation}
    n^\text{o}_3 \omega_3 = n^\text{e}_1 \omega_1 + n^\text{e}_2 \omega_2
\end{equation}
成立。
还有一种是所谓的Type II相位匹配,其中$\omega_1$光和$\omega_2$光的偏振方向不同。对负光轴晶体我们要设法让
\begin{equation}
    n^\text{e}_3 \omega_3 = n^\text{o}_1 \omega_1 + n^\text{e}_2 \omega_2
\end{equation}
成立,对正光轴晶体我们要设法让
\begin{equation}
    n^\text{o}_3 \omega_3 = n^\text{o}_1 \omega_1 + n^\text{e}_2 \omega_2
\end{equation}
成立。
如果二阶非线性过程的产出在o光偏振方向或是e光偏振方向上有足够大的分量(既然晶体已经确定是各向异性的,输入o光大抵是能够产生e光的,反之亦然),我们就可以获得高效率的二阶非线性转化了。

让这些式子之一成立似乎还是很难,但应注意到这里所谓的“$n^\text{e}$”\emph{可以不是}\eqref{eq:one-axis-matrix}中定义的那个——以任何一个方向为波矢方向的模式都可以是o光模式也可以是e光模式,如果它是e光模式,它的等效折射率由\eqref{eq:e-light-effective-index}给出。
因此我们可以通过调整入射光波矢和晶体光轴方向之间的夹角来调整以上相位匹配条件中的$n^\text{e}$。
为了避免讨厌的反射,我们可以让入射光垂直入射,此时所谓“调整入射光波矢和晶体光轴方向之间的夹角”就是调整光轴和非线性晶体表面的夹角。
这就是所谓的\concept{角度调节}。

角度调节的问题是,这样的光路中一般会存在走移角,这是宏观的一个量,从而可能会让我们希望能够重叠的光束无法重叠。
例如,如果使用Type II相位匹配做SFG,那么$\omega_1$光和$\omega_2$光很快会因为走移角而分开,从而无法发生非线性相互作用。
在Type I相位匹配中非线性相互作用可以持续发生,但是最终产生的光束在空间上会被展宽很多。
作为替代,我们可以选择一些光学性能和温度关系很大的非线性晶体,固定$\theta = \SI{90}{\degree}$,此时没有走移角,并使用温度来调节$n^\text{o}$和$n^\text{e}$的相对关系。
这种方案中晶体可以做得比较厚,而不必担心走移角。

使用二阶非线性双折射晶体有一个好处:可以通过偏振滤除输出光中的一些我们不需要的成分。
例如,设我们要用两束泵浦光产生和频光子,并且使用Type I相位匹配,那就可以通过适当放置偏振片,滤掉未发生转化的泵浦光。
无需使用不同频率的滤光片——偏振片就够了。

\subsubsection{准相位匹配}

在$\Delta k$非零但不大——即所谓\concept{准相位匹配}——时,可以通过这样的方法获得高效率的SFG转换:将一系列$\chi^{(2)}$指向周期性倒转的二阶非线性晶体贴在一起,让第一块晶体的厚度是$\pi L_\text{coh} / 2$,后面所有的晶体的厚度都是$\pi L_\text{coh}$。
这样,根据\eqref{eq:sfg-intensity},走过第一块晶体时,$I_3$强度打到最大值,此时能和$E_3$光发生相长干涉的光的相位是$\pi$;随后进入第二块晶体,$\chi^{(2)}$倒转,根据\eqref{eq:two-pump-a3},这意味着新产生的$E_3$光获得一个$\pi$的相位,于是第二块晶体内仍然发生了相长干涉;在第二块晶体和第三块晶体的交界处,能和$E_3$发生相长干涉的光的相位是$0$,$\chi^{(2)}$再次倒转,根据\eqref{eq:two-pump-a3},在第三块晶体中产生的$E_3$光的相位是$0$,于是还是相长干涉……
如此重复即可获得持续增加的$\omega_3$光,虽然其增速不如相位完全匹配时的情况。

准相位匹配已经是成熟的技术。PPLN装置等铁电体阵列是实现准相位匹配的常用装置。

准相位匹配实际上说明了一点,就是晶体的性质有空间起伏时,之前的诸如相位匹配条件的东西都是不能直接适用的。
这就提示我们,晶体内部的元激发——声子或者别的什么——可以借此和光耦合。

\subsection{二阶非线性效应的各种应用}

\subsubsection{波长调节}

设我们手头上只有一个频率的光——比如说\SI{800}{nm}的光——而需要得到一束频率低一些的光——比如说\SI{1500}{nm}的光。没有别的光源可用。
在光纤通信中经常会遇到这样的任务,如需要做这样的转换来避开容易受到干扰或是吸收的频段。
这时候可以这么做:首先将\SI{800}{nm}光尽可能聚焦到一个材料中,让supercontinuum generation发生,从而得到一个非常宽的频谱(并且很弱),然后将纯净的\SI{800}{nm}光和这个宽谱光入射到一个二阶非线性晶体当中,这样这两者就分别起到了泵浦光和种子光的作用。
当然,DFG过程产生的光的频率也是宽谱的,但是请注意只有满足相位匹配条件的过程才是最可能发生的。
因此我们可以通过选用适当的非线性晶体,调整温度、入射角度等,让\SI{800}{nm}光变成\SI{1500}{nm}光的过程正好符合相位匹配条件,于是就得到了相当纯净的\SI{1500}{nm}输出光。

\subsubsection{光学参数共振器(OPO)}

\concept{光学参数共振器(OPO)}是一个内部放置了一块二阶非线性晶体的光学共振腔。
一束强泵浦光$\vb*{E}_3$始终穿过这块非线性晶体,其强度衰减不明显。
如果在一定条件下只有DFG过程$\omega_3 = \omega_1 + \omega_2$发生,由于泵浦光恒定不变,$\omega_1$光和$\omega_2$光之间实际上只有线性耦合。
如果从两个不同的方向向非线性晶体输入光,则方向1的$\omega_2$光不会和方向2的$\omega_2$光耦合,$\omega_3$光同理。
方向1的$\omega_2$光和方向2的$\omega_3$光也不会有相互作用,因为此时相位匹配条件无法满足。
因此,只有方向1上的$\omega_2$光和$\omega_3$光会有耦合,方向2上的$\omega_2$光和$\omega_3$光会有耦合,并且都是线性耦合。
这种线性耦合服从\eqref{eq:dfg-amplitude},即这两束光会指数增长,能够定义一个增益$g$。

现在假定泵浦光已经输入了腔体。腔体本身有确定的、非常密集的离散谱,其量子涨落会让这些模式上的光子随机地产生,这就是DFG过程需要的种子光。
相位匹配条件决定了只有频率在特定的$\omega_1$和$\omega_2$频率附近的光能够放大。
设一个频率大体上是$\omega_1$的光子产生,它来回穿过非线性晶体,从而被一次次指数放大。
这个过程当然不会永远持续下去,稳定时,在腔体内来回走一趟的增益和损耗(包括吸收和溢出腔体)彼此抵消。
能够建立这样一个平衡,那么泵浦光就能够持续、高效率地转化为频率大体上是$\omega_1$和$\omega_2$、是腔体的某个本征频率的光。
否则,这样的光传播几次就衰减到零,不能形成稳定的输出。

\begin{equation}
    \ee^{2 g L} = 1 - R^2,
\end{equation}
由于增益很小而$R$很接近1,我们有
\begin{equation}
    g L = 1 - R.
\end{equation}

\subsubsection{荧光信号的时间分辨}

荧光持续的时间很长,但是很弱,我们可以在需要仔细分析的时间段制备一个脉冲,将荧光和这个脉冲同时输入一个非线性晶体,就可以把我们需要仔细观察的那一段分离出来。

\subsubsection{光谱学}

\section{三阶非线性极化的波动光学}

受激拉曼效应,受激光栅(让光自己产生干涉,然后非线性效应让折射率发生周期性变化,再来一束光,就发生了衍射),CARS

将一个波包压缩成一个阿秒级别的脉冲

三阶非线性极化允许这样的过程发生:$\omega = \omega - \omega + \omega$,即一束光可以同时提供三个光子,产生同一频率的一个光子。
也即,三阶非线性极化允许单频光自相互作用,或者说允许只涉及一个频率的简并混频。在二阶非线性极化中没有这种现象。
仅仅这个过程就足够导致很多新奇的现象出现。

\subsection{Kerr效应和相关的自相互作用}

\subsubsection{自聚焦}

波列很长的波进入一个各向同性的、中心反演对称的非线性晶体,我们会发现它的折射率大体上是
\begin{equation}
    n = n_0 + \bar{n}_2 \expval*{\vb*{E}(t)^2} = n_0 + 2 \bar{n}_2(\omega) \abs*{\vb*{E}(\omega)}^2,
    \label{eq:ref-index-change}
\end{equation}
其中$n_0$是弱光的折射率而$n_2$是一个二阶折射系数。这种折射率因为入射光而发生变化的效应称为\concept{Kerr效应}。

Kerr效应可能是三阶非线性极化的结果。如果我们只考虑三阶非线性极化,就有
\[
    \vb*{P}_\text{NL} = 3 \epsilon_0 \chi^{(3)}(\omega=\omega + \omega - \omega) : \vb*{E}(\omega) \vb*{E}(\omega) \vb*{E}(\omega)^*,
\]
因子$3$可以直接通过展开$\chi^{(3)} \vb*{E} \vb*{E} \vb*{E}$获得,也可以再一次通过费曼图对称性因子获得:设拉氏量中产生了$\chi^{(3)} \vb*{E} \vb*{E} \vb*{E}$的那一项的耦合常数为$\alpha$,则
\[
    \dv{\vb*{E}} (\alpha \vb*{E} \vb*{E} \vb*{E} \vb*{E}) = \chi^{(2)} \vb*{E} \vb*{E} \vb*{E},
\]
于是
\[
    \alpha = \frac{1}{4} \chi^{(2)},
\]
而由于$\alpha \vb*{E} \vb*{E} \vb*{E} \vb*{E}$项有四条外线,含有它的图需要乘以一个对称性因子$4!$。
在本节的自相互作用的情况下,由于有两条输入线的频率相同,需要除以一个因子$2$。(频率为$-\omega$的输入线和频率为$\omega$的输出线不等价,因为前者是外场线而后者不是)
于是最终的因子是$4! / (4 \cdot 2) = 3$。
在介质各向同性的情况下可以将$\chi^{(3)}$用一个标量代替,它后面的三个电场的乘积和单位张量缩并,于是
\[
    \begin{aligned}
        \vb*{P} &= \epsilon_0 \chi^{(1)}(\omega) \vb*{E}(\omega) +  3 \chi^{(3)}(\omega=\omega + \omega - \omega) : \vb*{E}(\omega) \vb*{E}(\omega) \vb*{E}(\omega)^* \\
        &= \epsilon_0 (\chi^{(1)}(\omega) + 3 \chi^{(3)}(\omega=\omega+\omega-\omega) \abs*{\vb*{E}(\omega)}^2) \vb*{E}(\omega).
    \end{aligned}
\]
折射率的定义为
\[
    n^2 = 1 + \chi,
\]
取小量近似,就得到
\begin{equation}
    n_0(\omega)^2 = 1 + \chi^{(1)}(\omega), \quad \bar{n}_2(\omega) = \frac{3}{4n_0} \chi^{(3)}(\omega=\omega-\omega+\omega).
\end{equation}

我们还可以将折射率写成光的强度
\begin{equation}
    I = \frac{\epsilon_0 c n_0}{2} \abs*{\vb*{E}(\omega)}^2
\end{equation}
的函数,即
\begin{equation}
    n = n_0 + n_2 I,
    \label{eq:reflective-index-indensity-dependence}
\end{equation}
其中
\begin{equation}
    n_2 = \frac{4}{\epsilon_0 c n_0} \bar{n}_2.
\end{equation}

折射率会随着入射光强而变化这件事意味着进入三阶非线性晶体的光会\concept{自聚焦}。
如果$\bar{n}_2 > 0$,那么光束中心的光被偏折得更厉害,光束相当于经过了一个凸面镜;反之光束相当于经过了一个凹面镜。
这个等效“透镜”的行为和频率相关,因此自聚焦可以用来设计一个频率筛选装置,即可以用于锁频。
例如,可以在一个谐振腔内部放置两个自聚焦晶体,则只有频率适当的光能够在谐振腔中稳定地来回传播,频率不适当的光经过多次成像,会散得越来越开。

\subsubsection{自相位调节}

光波经过介质之后(相比另一束没有经过介质的光)会有相位变化,而由于$\bar{n}_2$的存在,光波经过一个三阶非线性晶体之后会有一个额外的相位变化
\begin{equation}
    \phi_\text{NL}(t) = \frac{\omega}{c} L n_2 I(t).
    \label{eq:self-phase-adjustement}
\end{equation}
仪器把一束光大体上当成单频光而测它的频率(即所谓\concept{即时频率},对单色光它就是光的频率,对有多频率光它大概是波包的中心频率),就得到
\[
    \omega_\text{temp} = \pdv{\phi}{t},
\]
于是从三阶非线性晶体出来的光的即时频率会因为非线性效应而变化
\begin{equation}
    \Delta \omega_\text{temp} = - \frac{n_2 \omega L}{c} \pdv{I}{t},
    \label{eq:temp-phase-change}
\end{equation}
即一束多频率的光经过时前面的会看起来更红,后面的会看起来更蓝。
把这束光做频谱分析会发现频谱变宽了。频谱变宽了,时域的波包尺度就会变窄——因此可以用三阶非线性晶体做一个超快激光。
例如可以用三阶非线性晶体做一系列薄片,波包每经过一个薄片,在时域的展宽就窄一些。
我们这里不用一个非常厚的晶体,因为随着频谱变宽,波包的强度会下降,从而$\pdv*{I}{t}$也会下降。因此一个特别长的晶体除了让光束散开来以外并没有什么用处。
同样,表面上,虽然弱光的$I$并不大,但我们可以增大$L$去调节它的相位,但其实这是不现实的。

设脉冲在时间上持续了$\tau$,则频谱宽度为
\[
    \Delta \omega \sim \frac{2\pi}{\tau},
\]
而即时频率的变化为$\Delta \phi_\text{NL} / \tau$,因此为了让频率展宽足够明显,应当有
\begin{equation}
    \Delta \phi_\text{NL} \sim 2\pi.
\end{equation}

\subsubsection{光学孤子}

自相位调节其实提醒我们一点:可能可以使用一些特殊的非线性效应反过来补偿色散导致的不同频率的光的相位差,从而让介质中能够产生孤子,这是一个波包,它能够稳定地在介质中传播,而不发生波包展宽。
在光纤中这已经有了应用,一些时候可以制造一个展宽为微米级的孤子。

我们设有一个波包
\begin{equation}
    E(z, t) = A(z, t) \ee^{\ii (k_0 - \omega_0 t)},
    \label{eq:wave-package}
\end{equation}
其中$k_0$和$\omega_0$是中心波矢和频率,$A(z, t)$的时间和频率依赖给出波包的(随时间变化的)包络线。我们通常认为中心波矢和频率之间的关系是遵从线性折射率的,即
\begin{equation}
    k_0 = n_0(\omega) \frac{\omega_0}{c}.
\end{equation}
我们不区分线性和非线性效应,统一地求解非线性波动方程
\[
    \pdv[2]{E}{z} - \frac{1}{\epsilon_0 c^2} \pdv[2]{D}{t} = 0.
\]
做傅里叶变换
\[
    E(z, t) = \int \frac{\dd{\omega}}{2\pi} \ee^{- \ii \omega t} E(z, \omega), \quad D(z, t) = \int \frac{\dd{\omega}}{2\pi} \ee^{- \ii \omega t} D(z, \omega),
\]
并且
\[
    D(z, \omega) = \epsilon(\omega) E(z, \omega),
\]
其中$\epsilon(\omega)$是(带有非线性效应,依赖于$E$的)介电常数。
我们于是得到非线性版本的亥姆霍兹方程
\begin{equation}
    \pdv[2]{E(z, \omega)}{z} + \epsilon(\omega) \frac{\omega^2}{c^2} E(z, \omega) = 0.
    \label{eq:nonlinear-freq-domain-eq}
\end{equation}
我们能有幸得到形式这么好的方程当然归功于材料本身没有时间演化,否则在频域中$D$和$E$之间的关系就不是简单的“乘以一个系数”了。
我们根据\eqref{eq:wave-package}以及$k_0$和$\omega_0$之间的关系是线性色散关系,得到
\begin{equation}
    E(z, \omega) = A(z, \omega - \omega_0) \ee^{\ii k_0 z } + \text{c.c.}.
\end{equation}

我们考虑慢变振幅近似。需要注意的是,此时这个近似是可能会失效的,几百飞秒的波包仍然满足这个近似,再小一些可能就失效了。
我们进一步假定波包的频谱展宽相比于$\omega_0$是很小的。
这样,\eqref{eq:nonlinear-freq-domain-eq}就变成
\begin{equation}
    2 k_0 \pdv{A(z, \omega - \omega_0)}{z} + (k^2 - k_0^2) A(z, \omega - \omega_0) = 0.
    \label{eq:wave-package-amplitude}
\end{equation}
做泰勒展开
\[
    k = k_0 + \Delta k_\text{NL} + k_1 (\omega - \omega_0) + \frac{1}{2} k_2 (\omega - \omega_0)^2 + \cdots,
\]
其中$k_\text{NL}$为非线性效应导致的自相位调节。这样,\eqref{eq:wave-package-amplitude}就成为
\begin{equation}
    \pdv{A}{z} - \ii \Delta k_\text{NL} A - \ii k_1 (\omega - \omega_0) A - \frac{1}{2} \ii k_2 (\omega - \omega_0)^2 A = 0.
\end{equation}
做傅里叶反变换
\[
    A(z, t) = \int \frac{\dd{\omega}}{2\pi} A(z, \omega - \omega_0) \ee^{- \ii (\omega - \omega_0) t},
\]
得到
\begin{equation}
    \pdv{A}{z} + k_1 \pdv{A}{t} + \frac{1}{2} \ii k_2 \pdv[2]{A}{t} = \ii \Delta k_\text{NL} A.
    \label{eq:wave-package-evolve}
\end{equation}
这其中,$k_1$和$k_2$是线性色散的一阶和二阶泰勒展开系数,它们是
\begin{equation}
    k_1 = \left(\pdv{k}{\omega}\right)_{\omega = \omega_0} = \left( \frac{1}{v_\text{g}} \right)_{\omega = \omega_0},
\end{equation}
以及
\begin{equation}
    k_2 = \left( \pdv[2]{k}{\omega} \right)_{\omega = \omega_0} = - \left( \frac{1}{v_\text{g}} \dv{v_\text{g}}{\omega} \right)_{\omega = \omega_0}.
\end{equation}
自相位调节为(把\eqref{eq:self-phase-adjustement}右边除以$L$就得到)
\begin{equation}
    \Delta k_\text{NL} = n_2 I \frac{\omega_0}{c}.
\end{equation}

如果\eqref{eq:wave-package-evolve}中完全没有非线性光学效应,并且$k$和$\omega$之间的关系不是线性的,那么就会出现波包展宽,因为此时$k_2 \neq 0$,做代换
\[
    \tau = t - \frac{z}{v_\text{g}},
\]
得到% TODO:衰减
同理,自相位调节也会导致波包展宽。如果我们要求
\begin{equation}
    \frac{1}{2} k_2 \pdv[2]{A}{t} = \Delta k_\text{NL} A,
\end{equation}
那么就不会有任何波包展宽。一个例子是
\begin{equation}
    A(z, \tau) = A_0 \sech(\tau / \tau_0) \ee^{\ii k z},
\end{equation}
虽然以上求解过程似乎要求波包要具有特定的形状,这样才能形成孤子,但是实际上,一些形状不那么好的波包输入材料之后其实也能形成孤子,因为不符合波包形状要求的那些频率成分由于色散都各自跑远了,只留下一个孤子波包。

\subsubsection{自陷}

想象一束已经被聚焦过了,然后被输入一个三阶非线性晶体。例如,可以将三阶非线性晶体放在一个高斯光的光腰上。
自聚焦现象如期发生,让光变得更强,然后自聚焦进一步增强……如果几何光学总是适用,那么最终光束将终结到一个点上。
当然,在此之前衍射已经变得明显了。
这里发生的事情就好像“外压和量子涨落的竞争”(我们会看到“量子涨落”并不只是比喻),最终形成一个光束尺寸相对稳定的\concept{光丝}。这就是所谓的\concept{自陷}。

光丝并不是一个非常稳定的状态,因为如果介质中有什么东西散射了一下光丝,它的直径就会增大,光强变小,于是自聚焦的逆过程开始发生,最后光束又四散开去。
但如果光丝足够强,它可能已经将介质内部打出一个等离子体通道了,这个时候支配光丝所在区域的光学性质的不是三阶非线性极化,而是等离子体的光学,光丝也就这样一直传播下去了。

我们来估算什么时候自聚焦会发生。假定自聚焦区域内的折射率大体上是均一的(从而自聚焦区域内的$I$是常数,一旦出了自聚焦区域,就快速衰减为零),则自聚焦区域的边界上的临界角为
\[
    \cos \theta_0 = \frac{n_0}{n_0 + \var{n}}, \quad \var{n} = n_2 I,
\]
射向边界而入射角大于这个角的光将被反射回去,即不会溢出光丝。
另一方面,光丝会有衍射,即在偏离入射光的波矢的地方仍然有光传播,本质上这是因为长得像平面波的光束不可能具有有限直径——光丝内部类似于平面波,但是它有有限大小的直径,我们在偏离入射光的波矢的方向上计算总电场,是能够得到非零结果的。
我们也可以说这是位置和动量的不确定性:光丝的位置是比较确定的,从而“传播方向”是不完全确定的。
我们借用孔径衍射的公式(因为这可以算是一个孔径衍射),衍射角为
\[
    \theta_d = \frac{0.61 \lambda_0}{n_0 d}.
\]
如果很多衍射光的衍射角小于全反射临界角,那么衍射会破坏光丝,而反之光丝可以进一步聚焦。
因此,平衡时,$\theta_0 \sim \theta_d$。做近似
\[
    \cos\theta = 1 - \frac{\theta^2}{2},
\]
并且注意到$n_2 I$无论如何相比$n_0$都是非常小的,我们能够得到
\begin{equation}
    d \sim \frac{0.61 \lambda_0}{\sqrt{2 n_0 n_2 I}} .
\end{equation}
这给出了指定波长、线性折射率和自聚焦效应之后,形成稳定光丝的直径。
我们马上可以,这对应一个功率
\begin{equation}
    P_\text{cr} = \frac{\pi d^2}{4} I = \frac{\pi 0.61^2 \lambda_0^2}{8 n_0 n_2}.
\end{equation}
这是一个完全确定的功率值,不多也不少。如果入射光束的总功率大于$P_\text{cr}$,那么光丝实际上只使用了入射光束的一部分,而如果入射光束的总功率小于$P_\text{cr}$则无法形成光丝。

入射光仍然需要走过一段距离才能够形成光丝。

\subsection{频率相同的多束光}

\subsubsection{光学相位共轭}

四波混频还能够导致另一个神奇的现象:\concept{光学相位共轭},即在被一束入射光激励之后能够产生另一束波矢方向和入射光完全相反而频率不变的的出射光。
设想我们向一块三阶非线性晶体输入两束波矢完全相反的泵浦光,用1和2标记它们,然后再输入一束随便什么光$\vb*{E}_3$。这样我们就有
\begin{equation}
    \begin{aligned}
        \left( \laplacian - \frac{\epsilon_\text{r}}{c^2} \pdv[2]{t} \right) E_4 &= \frac{1}{\epsilon_0 c^2} \pdv[2]{t} (6 \epsilon_0 \chi^{(3)} A_1 A_2 A_3^* \ee^{\ii (\vb*{k}_1 + \vb*{k}_2 - \vb*{k}_3)} \cdot \vb*{r}) \ee^{- \ii \omega_4 t} + \text{h.c.} \\
        &= \frac{1}{\epsilon_0 c^2} \pdv[2]{t} (6 \epsilon_0 \chi^{(3)} A_1 A_2 A_3^* \ee^{- \ii \vb*{k}_3} \cdot \vb*{r}) \ee^{- \ii \omega_4 t} + \text{h.c.} ,
    \end{aligned}
    \label{eq:four-wave-omega-4}
\end{equation}
其中
\begin{equation}
    \omega_4 = \omega_1 + \omega_2 - \omega_3.
\end{equation}
因此我们已经得到了一个波矢刚好反过来的$\omega_4$光。为了让$\omega_4$光和$\omega_3$光光学共轭,我们需要让
\[
    \omega_4 = \omega_1 + \omega_2 - \omega_3 = \omega_3,
\]
于是
\begin{equation}
    \omega_1 = \omega_2 = \omega_3 = \omega_4 = \omega.
\end{equation}
此时我们实际上还是在处理单频光;但是这不再是自相互作用了,因为$\vb*{k}_1$和$\vb*{k}_3$可以不平行和不相反:我们只需要
\begin{equation}
    \vb*{k}_1 = - \vb*{k}_2, \quad \vb*{k}_3 = - \vb*{k}_4,
\end{equation}
而$\vb*{k}_1$和$\vb*{k}_3$之间可以没有任何关系。上式满足之后相位匹配条件自然成立,因此转化效率是非常高的。% TODO:三阶晶体的相位匹配
由于是单频光,\eqref{eq:four-wave-omega-4}中的因子$6$有时候也需要调整。

% TODO:解方程

干涉条纹的空间频率是波矢差;干涉条纹因为自相位调节导致折射率光栅

\subsubsection{光学双稳}

设我们在一个法布里-波洛腔中放置了一个三阶非线性晶体。本节考虑一个最简单的情况,即三阶非线性晶体完全充满整个谐振腔。
设外界入射光振幅为$A_1$,晶体内的折射光振幅为$A_2$,从谐振腔的另一侧透出去的光的振幅为$A_3$,从谐振腔的另一侧反射回来的光的振幅为$A_3'$,与入射光方向相反,从入射侧返回的光的振幅为$A_1'$。
暂时忽略介质吸收,则
\begin{equation}
    A_2' = r A_2 \ee^{2 \ii k L}, \quad A_2 = t A_1 + r A_2'
\end{equation}
其中$r$和$t$是谐振腔壁的反射和透射系数。从以上方程我们解出
\begin{equation}
    A_2 = \frac{\tau A_1}{1 - \rho^2 \ee^{2 \ii k L}} = \frac{\tau A_1}{1 - R \ee^{2 \ii \phi + 2 \ii k L}} = \frac{\tau A_1}{1 - R \ee^{\ii \delta}},
    \label{eq:a2-from-a1-cavity}
\end{equation}
其中$\phi$是$r$的辐角(需要考虑这一点,因为腔壁可能有金属镀层)。
需要注意$k$由于三阶非线性极化会导致折射率依赖于$I$。
这里不能直接用\eqref{eq:reflective-index-indensity-dependence},因为\eqref{eq:reflective-index-indensity-dependence}给出的是单束光的自相互作用,而在本节中谐振腔中有方向相反的两束光$A_2$和$A_2'$。
正比于$A_2 A_2^*$的那部分非线性极化和正比于$A_2' (A_2')^*$的那部分非线性极化都会修正折射率,但是正比于$A_2 A_2'$的那部分非线性极化给出的电场的频率是$3 \omega$,而正比于$A_2^* A_2'$的那部分非线性极化不满足相位匹配条件,因此实际上我们有
\begin{equation}
    \delta = 2 \phi + 2 n_0 \frac{\omega}{c} L + 2 n_2 I_2 \frac{\omega}{c} L + 2 n_2 I_{2}' \frac{\omega}{c} L = 2 \phi + 2 \frac{\omega}{c} L (n_0 + \underbrace{(1+R) I_2}_{\approx 2 I_2}),
    \label{eq:phase-cavity}
\end{equation}
而不是简单地将$\abs*{A_2}^2 + \abs*{A_2'}^2$代入\eqref{eq:reflective-index-indensity-dependence}。
将\eqref{eq:a2-from-a1-cavity}取模长,就得到
\begin{equation}
    \begin{aligned}
        I_2 &= \frac{T I_1}{(1 - R \ee^{\ii \delta}) (1 - R \ee^{- \ii \delta})} = \frac{T I_1}{(1 - R)^2 + 4 R \sin^2 \frac{\delta}{2}} \\
        &= \frac{T I_1}{T^2 + 4 R \sin^2 \frac{\delta}{2}} = \frac{I_1 / T}{1 + (4 R / T^2) \sin^2 \frac{\delta}{2}}.
    \end{aligned}
    \label{eq:relation-i1-i2-cavity}
\end{equation}
联立\eqref{eq:phase-cavity}和\eqref{eq:relation-i1-i2-cavity},即求解
\begin{equation}
    I_1 = T I_2 \left( 1 + \frac{4R}{T^2} \sin^2\left( \phi + \frac{\omega}{c} L (n_0 + (1+R) n_2 I_2) \right) \right),
    \label{eq:cavity-bistability}
\end{equation}
即可得到谐振腔中的光强分布情况。

显然,\eqref{eq:cavity-bistability}给出的$I_2$有可能不止一个。这意味着给定$I_1$实际上是\emph{不能}唯一确定$I_2$的:$I_2$和过去的$I_1$是有关系的。
在存在两个$I_2$时我们说腔体存在\concept{光学双稳},存在更多$I_2$时称为光学多稳。

光开关:见Boyd 7.3.3,大体上说就是光强可以调控折射率从而可以调控光的干涉,从而控制不同方向上的光强。

无损光强测量

\section{光和介质中模式的散射}

布里渊散射,拉曼散射;

\subsection{经典电磁波}

Raman散射可以用一些经典图景分析。我们知道介质中的电子在振动,于是将$\alpha$泰勒展开到一阶,得到
\begin{equation}
    \vb*{P} = \alpha \vb*{E}, \quad \alpha(t) = \alpha_0 + \pdv{\alpha}{Q} Q(t),
\end{equation}
设介质中谐振子正在以$\omega_q$振荡,泵浦光频率为$\omega_l$,则
\begin{equation}
    \vb*{P}(t) = \alpha_0 \vb*{E}_0 \cos(\omega_l t) + \frac{1}{2} \pdv{\alpha}{Q} \vb*{E}_0 Q_0 (\cos(\omega_l + \omega_q) t + \cos(\omega_l - \omega_q)t).
\end{equation}
可以看到我们有三个过程:一个是普通的$\omega_l$光的传播,一个是\concept{Stokes过程},即输出光频率为$\omega_l - \omega_q$,还有一个是\concept{反Stokes过程},即输出频率为$\omega_l + \omega_q$。

以上经典图景无法给出一些需要“能级布居数”才能解释的东西。例如,基本上低能级上的电子数目要远大于高能级,因此Stokes过程总是比反Stokes过程容易发生。
实际上这个原理可以用来测定一个已知能谱的系统的温度,因为通过比较Stokes过程和反Stokes过程的发生几率来确定能级布居数,从而推算出温度。
例如,通过声子和光子的耦合,我们可以测量出固体晶格的温度。
分析Raman过程中的$\omega_q$也可以用于确定系统中的各个能级的相对能量差。

在有磁场存在,且两个能级的间距相比磁场并不大的时候,Stokes

现在我们改用量子力学做计算。由于并没有对光场做量子化,只需要做含时微扰即可。
\[
    \dv{W}{\omega} = \frac{2\pi}{\hbar} g(\omega) \abs{\sum_n \left(
        \frac{\mel*{f}{e \vb*{r} \cdot \vb*{E}}{n} \mel*{n}{e \vb*{r} \cdot \vb*{E}}{g}}{\omega_l - }
    \right)}^2
\]

我们在这里遇到了之前遇到过的类似的问题:如果没有频率为$\omega_l + \omega_q$的种子光,似乎拉曼光不能产生。
当然,这是因为我们没有考虑电磁场的量子涨落。我们现在将光场也视为量子化的,即

\begin{equation}
    \dv]{W}{\omega_s} = \frac{8\pi^3 N \omega_l \omega_s}{n_s^2 n_l^2 V} \abs*{\mel{f}{M}{g}}^2 g(\omega) \abs*{\mel*{\alpha_f}{a_s^\dagger a_l}{\alpha_s}}^2
\end{equation}

受激Raman效应中Stokes光似乎可以以$\ee$指数增加。这个机制称为\concept{Stokes激光},是一种产生激光的机制。
Stokes光的相位和
重点:是否能够保证$\omega_s$是相干的;如果一个材料中的元激发(光学声子、自旋波等)、泵浦光和Stokes光能够持续耦合,“滚着往前跑”,那么就会有比较棘手的东西出现。

受激Raman效应可以用于展宽光谱。

在计算完单次光子-材料中模式的Raman散射之后,我们可以把有关计算结果用于确定宏观参数。我们有
\begin{equation}
    \abs{\vb*{E}_s}^2 = \ee^{G_R z - \alpha_s z} \abs{\vb*{E}_s(0)}^2,
\end{equation}
其中
\begin{equation}
    G_R = \gamma m_l \propto \frac{N}{V} \dv{\sigma}{\Omega} \frac{m_l}{\Gamma},
\end{equation}
其中
\[
    fuck%\frac{N}{V} \dv{\sigma}{\Omega} \sim \SI{10^{-8}}{cm^{-1}}, \quad \Gamma \sim \SI{1}{cm^{-1}}, \quad \gamma \sim %\SI{10^{-3}}{\cm/MW}
\]

使用拉曼散射还可以做高精度测量。其测定精度

拉曼散射

\subsection{自发发射}



\subsection{广义的三波混频}

传播相位和天线相位不同导致相位差。

\subsection{基于拉曼散射的激光冷却}

由于斯托克斯过程和反斯托克斯过程完全是等价的,可以设想,既然能够将激光照在介质上向介质提供能量,当然也可以设法将介质的能量转移到激光中。
实际上,我们可以制备一个腔共振线正好落在反斯托克斯过程上的谐振腔,然后将一些分子放在腔中,并入射激光。
其结果是反斯托克斯过程受到激励,分子不仅没有从激光吸收能量,反而被冷却了。

\subsection{应用}

\subsubsection{生物成像}

生物体中不同组分的吸收波长的重合不大。
自发拉曼散射足够用于做静态成像。如果要做动态成像,自发拉曼散射发生的速率就太慢了,必须要通过受激拉曼散射完成测定。

\section{布里渊散射}

设介质中有声波,则介质性能出现空间起伏,有
\begin{equation}
    \chi(\vb*{r}) = \chi_0 \cos(\vb*{G} \cdot (\vb*{r} - \vb*{v}_{\vb*{G}} t)), \quad \omega_\text{S} = \vb*{G} \cdot \vb*{v}_{\vb*{G}}.
\end{equation}
毫无疑问这会导致一定的光学效应:我们想象一系列折射率不同的介质被贴在一起,一束光被打入其中,则每一层界面都会有微弱的反射,在条件适当的情况下这些反射光相长干涉,产生明显的总的反射。%
\footnote{
    实际上,这样可以获得非常好的反射镜,比镀银的好得多。通过这种方法可以获得5到6个9的反射率。
}%
$\chi$出现周期性起伏的介质就是这样一种系统。
另一个值得注意的地方是反射光会有小的频率变化,因为反射面在动,会有多普勒效应。

在有了以上直觉性的考虑之后我们开始解方程。做通常的拟设

受激布里渊散射会限制光线通讯的激光功率。光纤存在热涨落,时不时就会自发出现声子,

TODO:Kerr effect

\section{超快脉冲}

\subsection{离子化}

direct ionization, multiphoton ionization, tunnel ionization

\part{光的量子性质}

\documentclass[UTF8, a4paper]{ctexart}

\usepackage{geometry}
\usepackage{titling}
\usepackage{titlesec}
\usepackage{paralist}
\usepackage{footnote}
\usepackage{enumerate}
\usepackage{amsmath, amssymb, amsthm}
\usepackage{cite}
\usepackage{graphicx}
\usepackage{subfigure}
\usepackage{physics}
\usepackage{tikz}
\usepackage[colorlinks, linkcolor=black, anchorcolor=black, citecolor=black]{hyperref}

\geometry{left=3.18cm,right=3.18cm,top=2.54cm,bottom=2.54cm}
\titlespacing{\paragraph}{0pt}{1pt}{10pt}[20pt]
\setlength{\droptitle}{-5em}
\preauthor{\vspace{-10pt}\begin{center}}
\postauthor{\par\end{center}}

\DeclareMathOperator{\timeorder}{T}
\DeclareMathOperator{\diag}{diag}
\newcommand*{\ii}{\mathrm{i}}
\newcommand*{\ee}{\mathrm{e}}
\newcommand*{\diff}{\mathop{}\!\mathrm{d}}
\newcommand*{\st}{\quad \text{s.t.} \quad}
\newcommand*{\const}{\mathrm{const}}
\newcommand*{\comment}{\paragraph{注记}}
\newcommand*{\scheq}{Schr\"odinger's Equation}
\newcommand*{\reals}{\mathbb{R}}

\newenvironment{bigcase}{\left\{\quad\begin{aligned}}{\end{aligned}\right.}

\title{量子物理基本概念}
\author{wujinq}

\begin{document}

\maketitle

% 总之我在这几篇文章中把“可观察量”一词玩坏了,它可以表示厄米算符也可以表示其期望……
% 类似的还有“多粒子态”……

\section{抽象代数}

\subsection{指标升降}

符号约定:$\vb*{A}^2$代表$\vb*{A}$的模长平方,$A^2$则表示分量。

\subsection{算符和态}

\[
    \comm{\hat{A}}{\hat{B}}^\dagger = \comm{\hat{B}}{\hat{A}},
\]
因此两个算符对易当且仅当它们的共轭转置对易。

% TODO:明确地点出动力学变量这个概念

% TODO:分析三种绘景下的态
% 两个态表示了同样的物理状态,当且仅当,
% $\ket{\psi}$和$\hat{A}$组成的系统和$\hat{Q}\ket{\psi}$和$\hat{Q} \hat{A} \hat{Q}^\dagger$组成的系统等价,其中$\hat{Q}$是一个幺正算符;反之,如果两个长度等价的向量描述等价的系统,
% TODO: 设算符$\hat{A}$是CSCO,且它在幺正变换$\hat{P}$下不变,那么对任何一个本征值$A_i$,有一个单位复数,使得$\hat{P} \ket{A_i} = c \ket{A_i}$.
% TODO:虽然描写一个态空间可以需要不止一个算符(或者说这个空间的CSCO的大小不为1),但往往可以将这些CSCO拼凑成一个:
% \hat{A} \ket{a_1 a_2 \cdots} = \pmqty{a_1 & a_2 & \cdots} \ket{a_1 a_2 \cdots}
% 只要推导中不涉及本征值的乘除,这样做就没有任何问题。
% 因此下文中将常常这么写。
% 谱结构和对易关系之间有什么联系?

\textbf{表象}指的是态空间的一组正交完备基。由于通常这样一组基是某个CSCO$\hat{M}$的本征态,我们通常使用对应的$CSCO$来标记表象。例如,我们有坐标表象、动量表象,等等。
表象变换公式
\begin{equation}
    \braket{A_i}{\psi} = \sum_j \braket{A_i}{B_j} \braket{B_j}{\psi}
\end{equation}
是基的完备性的推论。

变换
\[
    \braket{A_i}{\psi} \longrightarrow \mel{A_i}{\hat{B}}{\psi}
\]
称为算符$\hat{B}$在$A$表象下的表示。显然,$\hat{A}$在$A$表象下的表示就是
\[
    \braket{A_i}{\psi} \longrightarrow A_i \braket{A_i}{\psi}.
\]

在离散谱的情况下,归一化条件相当简单:
\[
    \braket{A_i}{A_j} = \delta_{ij}
\]
在连续谱的情况下,需要使用积分代替求和,使用$\delta$函数代替$\delta$符号。
多分量算符的本征值有可能不是按照$\reals^n$的方式分布的,而是分布在一个弯曲的空间上(例如,分布在一个球面上)。此时通常需要使用类似于
\[
    \int \dd[n]{x} \delta(f(x))
\]
这样的测度,其中$f(x)$为描述弯曲空间的方程。其结果是,即使两个表象中的态矢量能够做到一一对应,由于使用的测度不同,它们仍然可以差一个模长不为1的系数。
换而言之,相同的态在不同的表象中会以不同的内积被归一化。

\subsection{李群和李代数,以及它们的表示}
% TODO:形如$\exp(\phi_1 G_1 + \phi_2 G_2 + \ldots)$的映射是不是一定可以写成$\exp (\phi_1' G_1) \exp (\phi_2' G_2) \ldots$?
在讨论对称性和守恒量的联系的时候

李群$U(t,t_0)$对应的无穷小生成元$H(t)$定义为
\begin{equation}
    H(t) = \lim_{\Delta t \to 0} \frac{U(t+\Delta t, t)}{\Delta t}
\end{equation}
使用编时算符$\timeorder$可以写出形式解
\begin{eqnarray}
    U()
\end{eqnarray}

如果$U(a+b)=U(a)U(b)$则生成元是常量。

有必要分析一下升降算符的东西:作用在一个本征态上得到另一个本征态??
可以看成一种平移:
\[
    U(\epsilon) \ket{x} = \ket{x + \epsilon}
\]

李代数中的$XY$和$YX$都未必是李代数的元素,但是$[X,Y]$一定是。

需要注意的是即使一个群的生成元是
\[
    \det \ee^{A} = \ee^{\trace A}
\]
只要知道了李括号就完全确定了整个李代数的结构。

生成元会变的情况??编时算符。

特别的,如果一个变换不改变哈密顿量,或者说“不改变物理规律”,且这个变换的生成元不显含时间,那么其生成元就是守恒量。(也就是说哈密顿量是这个群的卡西米尔算符??)

不可约表示中的卡西米尔算符的表示一定是恒等矩阵的倍数。例如,$\laplacian$是空间平移群的卡西米尔算符,而空间平移群在形如$A\exp (\ii \vb*{k} \cdot \vb*{r})$这样的平面波组成的线性空间上的表示是不可约表示(它自己就是一个不变空间,没有更小的不变子空间),那么$\laplacian u = - k^2 u$,可见确实是恒等变换的倍数。

幺正的李群按照$\exp(\theta J)$的形式得到的生成元是反厄米的,按照$\exp(\ii \theta J)$的形式得到的生成元是厄米的。

交换左右手坐标系的坐标变换行列式是负的,否则是正的。

\subsection{常用公式罗列}

使用算符代数的时候需要特别小心,因为不对易性很容易让我们习以为常的公式失效。


我们有
\[
    \dv{t} \ee^{t A} = A \ee^{t A} = \ee^{t A} A
\]
然而,
\[
    \dv{t} \ee^{A(t)} = \dv{A}{t} \ee^{A(t)} = \ee^{A(t)} \dv{A}{t}
\]
并不一般成立。

回顾经典物理,我们会发现对任何一种系统我们都尝试使用一系列固定的物理量描述它,例如一个粒子有位置、动量,一个场有各点的场量,等等。
有时也可以使用另一些物理量描述它,例如我们可以在速度和动量之间切换,可以使用不同的坐标系,等等。
因此,虽然实际计算中常常使用由一系列物理量的值组成的列表描述物理系统(举例:“粒子质量多少多少、位于$x$坐标多少多少、$y$坐标多少多少的位置”),
但观念上我们使用了两种对象:其一是系统的状态,它是某种流形上的一个点(例如在经典哈密顿力学中它是辛流形上的点),
其二是物理量,它是从这个流形到实数、复数、矢量、张量等“量”的映射。

在经典体系中“态”能够做的运算无非是从一个态转移到另一个态;态和态之间是完全孤立的。
然而,无论是理论上的推广(如将经典力学看成某个波动方程的程函方程)还是实验上的发现(如双缝干涉实验)都表明,这并不是完美描述自然界的正确方式。
实际上,态是可以像矢量一样叠加的(例如,干涉条纹意味着电子在空间中的分布可以看成某种场,
这种场满足叠加原理,那么电子在空间中的分布就是$\delta$函数为基底张成的向量空间中的元素)。
因此,一个\textbf{量子系统}指的是其状态可以使用某个希尔伯特空间$\mathcal{H}$中的向量$\ket{\psi(t)}$来描述、并且可以做叠加、数乘等运算的系统。

我们还需要一个额外的假设:一个态矢量数乘上一个复数得到的态矢量和原态矢量代表了同一个态。因此,我们将使用归一化的态矢量$\ket{\psi}$,并简称它们为“态”。
长度为零的态矢量不能归一化,我们认为它是非物理的,仅仅用于保证正确的代数结构,而不起多大作用。
同时我们也假定态矢量如果随时间发生演化,那么它一直是归一化的;进一步,任何作用于态之上的可逆变换都应该保持态的归一化。
或者说,任何作用于态上的可逆变换都应该是\textbf{幺正}的。

下一个问题是,我们怎样“诊断”或者说“读取”这系统的状态,也就是说怎样构造量子体系中的物理量。
实验上的观察(如双缝干涉实验中如果输入电子束密度足够低,是能够捕捉到单个电子的,但是其位置不固定)表明,
一个态$\ket{\psi}$并不对应着一个固定的物理量取值(刚才的例子表明一个态通常并不对应一个固定的位置)。
但是注意到一旦物理量的定义给定了(比如,假定我们接下来要测定位置),的确有\textbf{一些}态能够毫无疑义地确定物理量的取值
(例如,$\delta(x-x_0)$当然就对应一个位于$x_0$处的尖峰)。
因此量子物理中的物理量应该是这种“能够确定物理量取值”的态连同对应的物理量打包而成的结构。
至于那些不能明确确定物理量取值的态,可以把它写成能够明确确定物理量取值的态的线性组合来判断它对应哪些可能的物理量取值,这些物理量取值占比多少。
什么样的结构可以用来做这件事?一个自然的想法是\textbf{算符}:设诸$A_i$为可能的物理量$A$的取值,$\ket{A_i}$为这些取值对应的一组非零态,定义
\begin{equation}
    \hat{A} = \sum_i A_i \dyad{A_i}
\end{equation}
为该物理量对应的算符%
\footnote{一个细节:如果同一个$A_i$对应多个可能的$\ket{A_i}$,则容易看出这个$A_i$对应的所有态矢量对应一个向量空间。
此时需要写出这个向量空间的一组基矢量$\ket*{A_i^{(1)}}, \ket*{A_i^{(2)}}, \ldots$,然后用
\[
    A_i \left(\ket{A_i^{(1)}} + \ket{A_i^{(2)}} + \cdots\right)
\]
代替$A_i \ket{A_i}$。}
。这样一来,“物理量的取值能够确定”的态就是算符$\hat{A}$的本征态,于是我们就可以使用线性代数来处理有关的问题。
物理量随着时间演化就意味着我们有
\begin{equation}
    \hat{A}(t) = \sum_i A_i(t) \dyad{A_i(t)},
\end{equation}
也就是说每一个时间点都对应一个算符。
由于量子物理中大部分有意义的物理量都对应算符,接下来我们将经常混用物理量和算符这两个词%
\footnote{当然的确有一些物理量和算符关系不大,比如质量等,但因为它们总是被当成常数处理因此无大碍。}
。

需要注意的是如果一个态和另一个态的内积$\braket{\psi}{\phi}$不为零,那
%TODO:正交性的意义
因此我们之后均认定对应不同$A_i$的$\ket{A_i}$彼此正交,从而可以毫无顾虑地使用bra-ket记号。
另一方面有意义的物理量值都是实数(有时引入复数单纯是为了方便计算,如表示相位,等等),这就意味着物理量对应的算符都是\textbf{厄米算符}。

\subsection{对系统的等价描述}

设两个希尔伯特空间$\mathcal{H}$和$\mathcal{H}'$,它们使用一个可逆算符$\hat{A}$相关联,当然$\hat{A}$是幺正的。也就是说
\[
    \ket{\psi'} = \hat{A} \ket{\psi}, \quad \ket{\psi} \in \mathcal{H},  \ket{\psi}' \in \mathcal{H}'
\]
$\hat{A}$是一个同构。容易看出,若$\hat{O}$是$\mathcal{H}$中的一个算符,
那么
\begin{equation}
    \hat{O}' = \hat{A} \hat{O} \hat{A}^{-1} = \hat{A} \hat{O} \hat{A}^\dagger
\end{equation}
就是对应的$\mathcal{H}'$中保持代数结构不变的算符,这是下面几个式子的结果:
\[
    \begin{split}
        \ket{\psi'} = \hat{A} \ket{\psi}, \quad \ket{\phi'} = \hat{A} \ket{\phi}, \\
        \ket{\phi} = \hat{O} \ket{\psi}, \quad \ket{\phi'} = \hat{O}' \ket{\psi'}
    \end{split}
\]
一种常见的情况是,$\mathcal{H}$与$\mathcal{H}'$实际上是同一个空间,
则$\hat{O}$在变换$\hat{A}$下不变的充要条件是$\hat{O}'=\hat{O}$,也就是说$\hat{O}$与$\hat{A}$对易。
进一步,如果算符$\hat{O}$在一个李群作用下不变,那么它和每个群元都对易,这又等价于它和这个李群的所有生成元都对易。

上面我们讨论了对希尔伯特空间做一个变换会导致其上的算符做对应的变换。
现在我们讨论反过来的问题:如果两个算符的代数结构彼此对应,那么它们作用的希尔伯特空间之间会有什么样的关系。

% TODO:写串词
设有态矢量$\ket{\psi}$、算符$\hat{O}$,以及$\ket{\psi'}$和$\hat{O}'$,
若$\hat{O}$和$\hat{O}'$的谱结构相同(不变子空间同构,对应的本征值相同),且两个态矢量中含有的可观察量的各本征态的占比一致
则认为两系统等价,因为它们的代数结构不可区分。这时可以证明
\begin{equation}
    \mel{\psi}{A}{\psi} = \mel{\psi'}{A'}{\psi'}
\end{equation}
实际上,像这样的等价系统能够且只能够使用下面的方式产生:
\begin{equation}
    A' = U A U^\dagger, \quad \ket{\psi'} = U \ket{\psi}
\end{equation}
其中$U$为酉算符。要求$U$是酉算符是为了确保变换之后的$A'$的本征态的正交性,从而确保它确实是可观察量。
(由此也可以看出,要求使用复希尔伯特空间来描述系统而又一定要求可观察量的取值为实数实际上是很强的条件)

\subsubsection{从李群到李代数}

本文中我们将不对李群的流形结构进行正式的分析,而仅仅满足于使用一定的群参数把一个李群完整地表示出来。
一个李群中的成员可以一般地写成
\begin{equation}
    g = \exp(\ii \theta_i \sigma_i) \equiv \exp (\ii \theta^i \sigma_i) = \exp (\ii \vb*{\theta} \vb*{\sigma}),
    \label{eq:lie-group-element}
\end{equation}
其中$\theta_i$指的是群参数,而$\sigma_i$指的是生成元。
通常要求群参数为实数。
$\ii$是一个无关紧要的系数,加上它和不加上它唯一的区别就是$\sigma$需不需要乘上一个$\ii$。
为了方便,常常将诸$\theta$排成行向量,$\sigma$排成列向量。由于没有度规,无需区分上下指标。
对应的,设$\theta$是一个群参数,对应的生成元为
\begin{equation}
    \sigma = \frac{1}{\ii} \dv{g}{\theta}.
\end{equation}
需注意\eqref{eq:lie-group-element}假定了
\[
    g(\theta_1) g(\theta_2) = g(\theta_1 + \theta_2),
\]
这又等价于,无论$\theta$取什么值,$g$对$\theta$求导都会得到完全相同的结果。
在大多数情况下可以不失一般性地要求这个性质成立,因为群参数到底是什么并不重要
——我们总是可以巧妙地定义$\theta$使得$g$对$\theta$求导的结果与$\theta$无关%
\footnote{这是来自常微分方程的基本结论:设$X$是一个生成元,那么必定可以找到李群的一个单参数子群$c(t)$,使得
\[
    \dv{t} c(t) = c(t) \cdot X,
\]
从而可以定义指数映射。这是解析映射,因此可以使用诸如求导等运算。},
% 但是真的如此吗?时间演化一定构成李群吗?
% 一种可能的质疑是:在球面上随意画一条闭合轨迹,它显然描述了起点位于球心,终点位于球上面的矢量的一个连续变换,
% 然而它却不能使用$\exp (\alpha G)$的形式表示出来。
% 但这个质疑本身不成立,因为通常的李群总是可以作用在线性空间上的,然而上述变换显然没有线性性。
% 感觉还是很奇怪。
但是有一个重要的例外:时间演化。
我们关注的是“正常人眼中的时间”,而不能随意定义时间流逝的速率,
因此并没有什么能够保证不同$t$处时间演化算符对$t$求导的结果都是$t=0$(也就是恒等映射附近)时间演化算符对$t$求导的结果。
记$U(t, t_0)$为从$t_0$演化到$t$的算符,也即
\[
    U(t, t_0) U(t_0) = U(t),
\]
由于$t$不再能够任意选取,我们不能够写出\eqref{eq:lie-group-element}这样的指数映射,但是可以证明,一定存在一个$H(t)$使得
\begin{equation}
    U(t, t_0) = T \exp \left( \int_{t_0}^t \dd{t} H(t) \right).
    \label{eq:time-dependent-lie-group}
\end{equation}
这里我们略去了\autoref{sec:time-evolution}中的公式中的因子$- \ii /\hbar$,不过这无关紧要。$T$为编时算符。
在不同时刻的$H(t)$彼此对易的情况下可以把$T$去掉,因为此时重排各算符顺序不会产生任何影响。

\eqref{eq:lie-group-element}和\eqref{eq:time-dependent-lie-group}的区别体现在很多地方。
\eqref{eq:lie-group-element}意味着
\[
    g^{-1}(\theta) = g(-\theta),
\]
或者说
\[
    \left( \exp(\theta \sigma) \right)^{-1} = \exp(- \theta \sigma),
\]
但是在不同时刻的$H(t)$彼此不对易时,
\[
    T \exp(\int \dd{t} H(t))^{-1} \neq T \exp(- \int \dd{t} H(t)).
\]
相应的,
\[
    \dv{t} \left(T \exp(\int \dd{t} H(t))^{-1}\right) \neq -H.
\]
这就是\autoref{sec:time-evolution}中做绘景变换时不同绘景下的哈密顿算符不相等的根本原因。

李代数是李群在单位元附近的切空间,也就是说,是$g$在$\theta=0$附近沿着任意方向对$\theta$求导之后得到的结果组成的代数。
接下来我们将讨论\eqref{eq:lie-group-element}的李群,因为“不同点处求导结果不同”基本上只会在处理时间演化时用到,
而此时只有一个生成元(就是哈密顿量),没有必要讨论李代数。
由于李代数的封闭性,设$g_1, g_2, \ldots$是一组相互独立的生成元,它们中任意两个的李括号$\comm*{g_1}{g_2}$一定也是一个生成元,
这意味着它可以使用$g_1, g_2, \ldots$线性表示。
从而我们有
\begin{equation}
    \comm*{g_i}{g_j} = f_{ij}^k g_k.
    \label{eq:structure-of-lie-algebra}
\end{equation}
如果我们只讨论抽象的李代数的性质而不考虑它作用在某些对象上产生的结果,那么\eqref{eq:structure-of-lie-algebra}就完全刻画了一个李代数的结构。
因此,称$f_{ij}^k$为\textbf{结构常数}。

\subsubsection{李代数的具体计算}

% TODO:把前面用到这一节的内容的部分写得更加简洁一些
若
\[
    \comm*{\hat{q}}{\hat{p}} = c,
\]
则
\[
    \comm*{\hat{q}}{\hat{p}^n} = n c \hat{p}^{n-1}.
\]

\subsubsection{表示论}\label{sec:rep-th}

接下来需要讨论李群和李代数的表示。
通常考虑两种表示,其一是李群和李代数在向量空间上的作用,
也就是说,我们在李群、李代数和向量空间上的算符组成的群(以算符的复合为乘法)之间建立一个同态,
一旦建立起这个同态,我们实际上就得到了李群或李代数的一个表示。
比较方便的做法是,先讨论李代数在特定向量空间上的表示,然后使用指数映射获得对应的李群的表示。
第二种表示是,李群和李代数在向量空间上的算符构成的向量空间上的作用。
这种表示和第一种表示是紧密相关的。
设李群$G$在向量空间$V$上的表示为$G_V$,则$G_V \subset GL(V)$。这就自然地诱导出了李群在$GL(V)$上的表示。
算符$\hat{B} \in GL(V)$建立起了这样的关系:
\[
    \phi = \hat{B} \psi,
\]
现在我们把$\hat{A} \in G_V$作用在$\phi$和$\psi$上面,就得到
\[
    \phi' = \hat{A} \phi, \quad \psi' = \hat{A} \psi,
\]
如果我们还是希望在$\phi'$和$\psi'$之间建立关系
\[
    \phi' = \hat{B}' \psi',
\]
应该怎么选取$\hat{B}$?
考虑到$\phi$和$\psi$的任意性,容易看出,
\[
    \hat{B}' = \hat{A} \hat{B} \hat{A}^{-1}.
\]
我们没有规定$\hat{B}$是什么——它是完全任意选取的。这样一来,$G_V$中的每一个元素$\hat{A}$都对应到下面的映射:
\begin{equation}
    \hat{B} \longrightarrow \hat{A} \hat{B} \hat{A}^{-1},
    \label{eq:group-action-on-operators}
\end{equation}
\eqref{eq:group-action-on-operators}是一个从$GL(V)$到$GL(V)$的映射,也就是满足封闭性。
请注意该映射是$GL(GL(V))$的成员,而不是$GL(V)$的成员——它作用在$V$上的算符上而不是$V$中的向量上。
因此,我们通常只讨论简单的向量空间上的群表示,因为这些向量空间上的算符组成的向量空间上的群表示可以使用前者按照\eqref{eq:group-action-on-operators}写出。
另外注意,不同的$\hat{A}$可能对应着同一个\eqref{eq:group-action-on-operators}型的从算符到算符的映射。
这一点在处理旋转群时体现得很明显。

李群和李代数通常被作用在几类向量空间上。
首先是有有限个分量的向量空间。李群在其上的作用形如
\[
    v \longrightarrow v', \quad (v')^\mu = R_{\nu}^\mu (\Lambda) v^\nu.
\]
其中$\Lambda$指抽象的李群。
在有限维向量空间$V$上的表示可能有不变子空间,也就是说,存在$V$的一个子空间$V'$,使得李群中的任何一个成员作用在$v \in V'$上之后得到的结果都还是在$V'$中。当然,$V$以及$\{0\}$一定是不变子空间。
如果一个表示有不是这两个空间的不变子空间,那么这就是一个\textbf{可约表示},反之则为\textbf{不可约表示}。
可以证明,任何一个可约表示都可以写成一系列不可约表示的直和。因此对有限维表示而言,只需要讨论不可约表示就可以了,因为可约表示可以使用不可约表示组装出来。
现在讨论不可约有限维表示。
首先可以证明,任何李群的生成元至少有一个(当然也可以有很多个)可以相似变换为对角矩阵。
% TODO:是不是每一个生成元都可以?
这些被对角化的生成元的集合称为Cartan子代数,它是对应的李群的李代数的表示的子代数。
Cartan子代数中的诸算符共享一组可以张成整个$V$的本征矢量,对应的各生成元的本征值——也就是对角矩阵的对角元——可以用来标记这个不可约表示。
要找到一组Cartan子代数并不难:只需要从李群中找到一个交换子代数,然后尝试对角化这个交换子代数中的某一个成员就可以了。
% TODO:李代数在怎样的程度上决定了对应的算符的谱结构?
非奇异矩阵表示一定可以通过相似变换而变成幺正表示(就是所有矩阵都是幺正的表示)。
这也就是我们频繁地讨论幺正表示的原因。但有许多重要的群——例如洛伦兹群——都不是紧致的(或者说群对应的流形无界),因此它们实际上并没有有限维的幺正表示。就洛伦兹群而言,我们将会看到,其推动生成元的有限维表示不是厄米的,因此整个群也没有幺正的有限维表示。

容易验证,设$\hat{X}$是厄米算符,且
\begin{equation}
    \comm*{\hat{a}^\dagger}{\hat{X}} = c \hat{a}^\dagger,
    \label{eq:raising-operator}
\end{equation}
那么
\[
    \hat{a}^\dagger \ket{X} \propto \ket{X+c},
\]
相应的,
\[
    \hat{a} \ket{X} \propto \ket{X-c}.
\]
因此称$\hat{a}^\dagger$为$\hat{X}$的\textbf{升算符},$\hat{a}$为$\hat{X}$的\textbf{降算符}。
数学上可以证明,在李代数的有限维表示上可以定义内积
\begin{equation}
    \langle \hat{A}, \hat{B} \rangle = \trace \hat{A} \hat{B},
\end{equation}
通过合适的线性组合,能够写出一组正交归一化的生成元。
此时非Cartan子代数的生成元中的每一个都是Cartan子代数中的每一个成员的升降算符,
并且任意两个非Cartan子代数的生成元的对易子都可以使用Cartan子代数的成员线性表示。
% TODO:看起来Cartan子代数似乎构成它的不可约表示空间的一个CSCO
% Symmetry and the Standard Model, p108
因此对一个一般的、没有正交归一化的李代数的有限维表示,我们总是可以从李代数的成员构造出一个升算符。设$\hat{X}$为$g_i$,且
\[
    \hat{a}^\dagger = \lambda^j g_j,
\]
则\eqref{eq:raising-operator}等价于
\[
    \comm*{\lambda^j g_j}{g_i} = c \lambda^j g_j,
\]
代入\eqref{eq:structure-of-lie-algebra},上式又等价于
\begin{equation}
    \left( f^k_{ji} - c \delta_j^k \right) \lambda^j = 0,
    \label{eq:determine-ladder-operators}
\end{equation}
于是通过求解
\begin{equation}
    \det \left( f^k_{ji} - c \delta_j^k \right) = 0
    \label{eq:possible-c}
\end{equation}
就可以得到所有可能的$c$,然后将它们代入\eqref{eq:determine-ladder-operators}就能够得到所有能够被非Cartan子代数表示出来的升降算符。
最后,由于是有限维表示,通过以上手法得到的升降算符实际上就是全部可能的升降算符,因此从一个本征态出发,通过它们可以构造出所有的本征态。
有限维表示还意味着,设$\hat{a}^\dagger$是某个升算符,那么对充分大的$N$,$(\hat{a}^\dagger)^N = 0$,$\hat{a}^N=0$,因为本征态的个数有限。
这些条件可用于确定什么样的不可约表示是被允许的。
% TODO:数学证明,不过多半鸽了
这些操作的一个典型的例子见对旋转群的处理。

现在我们分析一种比较特殊的情况。以上我们都是在“李代数可以分解成一个Cartan子代数和非Cartan元素,后者构成前者的升降算符”的框架下分析问题,那么如果李代数中所有元素都对易,那此时它会有怎样的表示?
由于没有非Cartan元素,这样的一个李代数——从而它的李群——不会有有限维的不可约表示。
通常这样的李群对应着某种空间平移操作。

% TODO:连续谱的情况
以上讨论的不可约表示都是有限维的。无限维表示——这里指的是函数空间的表示——则需要一套不同的框架。设$\hat{q}$具有连续谱,且
\begin{equation}
    \comm*{\hat{q}}{\hat{p}} = \ii,
\end{equation}
则
\begin{equation}
    \exp \left( \ii \lambda \hat{p} \right) \ket{q} = \ket{q + \lambda}.
\end{equation}
也就是说$\exp (\ii \lambda \hat{p})$是让$\hat{q}$的本征矢对应的本征值上升$\lambda$的升算符。

由于空间坐标无非是一种向量,李群和李代数也可以被作用在坐标上。
作用在坐标上的有限维表示又诱导出了作用在函数上的无限维表示%
\footnote{在有限维表示中,上下标$\mu$标记向量的诸分量;在函数空间中,坐标$x^\mu$标记“向量”——也就是函数——的诸“分量”——也就是函数在这一点的值。
李群在有限维向量空间上的表示通常是某个矩阵群,它将不同分量混合在一起,即
\[
    \psi^\mu \longrightarrow R^\mu_\nu \psi^\nu.    
\]
李群在无限维向量空间上的表示通常是“改变坐标$x^\mu$”。
}%
。设$f=f(x)$,若李群在坐标上的表示为
\[
    x \longrightarrow x', \quad (x')^\mu = R_\nu^\mu (\Lambda) x^\nu,
\]
则它在关于坐标的函数——也就是“场”——组成的无限维向量空间上的表示就是
\[
    f \longrightarrow f', \quad f(x) = f'(x') = f'(R(\Lambda) x),
\]
或者等价的,
\begin{equation}
    (x \mapsto f(x)) \longrightarrow (x \mapsto f'(x) = f(R(\Lambda)^{-1} x)).
    \label{eq:infinite-dim-rep}
\end{equation}
换而言之,坐标变动“牵引”了从坐标到场值的映射。
考虑到$f$可能是某个多分量对象(比如矢量、矢量的张量积,或者接下来要看到的旋量)的分量,
李群在此多分量场上的作用还包括通常的有限维表示,也就是
\[
    \psi^a \longrightarrow M(\Lambda)^a_b \psi^b.
\]
需注意此处我们使用了另外一个表示$M^a_b$而不是$R^\mu_\nu$,因为不能够保证$\Lambda$在多分量场$\psi$上的作用和它在坐标向量上的作用来自同一个有限维表示。
由于大部分情况下我们都是从一个群在通常意义上的矢量的作用出发讨论其结构的,可以将$R(\Lambda) x$简记为$\Lambda x$,也就是群元$\Lambda$在$x$上的作用。
这样上式就可以简洁地写成
\begin{equation}
    \psi^a(x) \longrightarrow {\psi'}^a (x) = M^a_b (\Lambda) \psi^b (\Lambda^{-1} x).
    \label{eq:wigner-transform}
\end{equation}
这种同时考虑了多分量场在李群作用下各分量重新混合(这是一个有限维表示)和李群作用下坐标拖曳而改变场(这对坐标而言是另一个有限维表示,对场而言是一个无限维表示)的李群的表示就是\textbf{场表示}。
需要注意的是,不同的$\Lambda$作用到坐标上可能会得出同样的结果,而它们对应的$M$作用到场上却有不同的结果,正如$SU(2)$和$SO(3)$的关系告诉我们的那样。

\eqref{eq:wigner-transform}给出的是李群的场表示的一般形式,但此时我们还只有形式上的变换而没有显式的表达式。
我们来分析其李代数。取%
\footnote{虽然可以任意地调整群参数,从而让生成元前面的系数随意变动,但是通常对有限维表示和无限维表示我们总是采用同样的群参数。这就意味着,在有限维表示确定之后不能随意调节无限维表示的生成元前面的系数,不能随意加一个$\ii$或者改变正负号。这也就是我们在场表示中一并处理有限维表示和无限维表示的原因,因为此时两者的群参数自动地就是相同的。

下式中的$g$的定义可以是\[
    g = \frac{1}{\ii} \pdv{G}{g},
\]
但也可以是像我们定义旋转生成元时的那样,取
\[
    g = \ii \pdv{G}{g},
\]
只需要将$\epsilon$取为负值就可以了。无论$g$是怎么定义的,下式都是成立的。}%
\[
    \Lambda = I + \ii \epsilon g,
\]
其中$g$是一个生成元,我们就有
\[
    \begin{aligned}
        \psi^a \longrightarrow {\psi'}^a &= M^a_b (\Lambda) \psi^b (\Lambda^{-1} x) \\
        &= (I + \ii \epsilon M^a_b(g)) \psi^b (x - \ii \epsilon g x) \\
        &= (I + \ii \epsilon M^a_b(g)) (\psi^b - \ii \epsilon g x \cdot \grad{\psi^b}) \\
        &= \psi^b + \ii \epsilon M^a_b(g) \psi^b - \ii \epsilon g x \cdot \grad{\psi^b},
    \end{aligned}
\]
于是
\[
    {\psi'}^a = (I + \ii \epsilon  (M^a_b(g) - g x \cdot \grad)) \psi^a,
\]
于是场表示的生成元可以写成
\begin{equation}
    M_\text{field} = M_\text{fin} + M_\text{inf}, \quad M_\text{fin} = M^a_b(g), \quad M_\text{inf} = - (g x) \cdot \grad.
    \label{eq:fin-and-inf-rep}
\end{equation}
其中$M_\text{fin}$就是我们所熟悉的李群在有限维向量空间上的矩阵表示,而$M_\text{inf}$则是李群作用在坐标上,拖曳坐标而对场产生的影响。
显然,它们和$g$之间能够建立同态关系。$gx$和$\Lambda x$一样,都是“$g$在坐标空间上的有限维矩阵表示作用于$x$”的简写。
与通常物理中的记号不同,此处的梯度算符作用在所有坐标上,不仅仅是空间坐标,还包括时间坐标。

在以上讨论的基础上我们讨论态矢量。我们总是使用李群在希尔伯特空间上的幺正表示,因为需要保证变换前后的态矢量都是物理的,也就是说,都是正交归一化的。
我们刚才讨论了李群的场表示,这个场表示当然可以被作用在算符场上。但是注意到算符场是态空间上的算符,因此按照\eqref{eq:group-action-on-operators},李群的场表示自然地如下导出了李群在希尔伯特空间上的表示:
\begin{equation}
    \hat{U}(\Lambda) \hat{\psi}^b(\vb*{x}) \hat{U}^{-1}(\Lambda) = M^a_b (\Lambda) \psi^b (\Lambda^{-1} x).
    \label{eq:field-rep-and-state-rep-lie-group}
\end{equation}
由于对$\hat{\phi}$的变换等价于对其本征值做变换,这又等价于保持本征值不变而重新安排本征态,按照上式诱导出的在希尔伯特空间上的李群表示$\hat{U}$也是幺正的。

相应的,\eqref{eq:field-rep-and-state-rep-lie-group}也导致了对应的李代数在希尔伯特空间上的表示。对\eqref{eq:field-rep-and-state-rep-lie-group}取微元,得到
\[
    (1 + \ii \epsilon M_\text{state}) \hat{\psi}^b (1 - \ii \epsilon M_\text{state}) = \ii \epsilon M_\text{field} \hat{\psi},
\]
从而
\begin{equation}
    \comm*{M_\text{state}}{\psi} = M_\text{field} \psi.
    \label{eq:field-rep-and-state-rep-gen}
\end{equation}
实际上,时间演化方程\eqref{eq:quantum-evolution}就是一个例子:时间平移群在希尔伯特空间上的表示是哈密顿算符$\hat{H}$,在场——这里是任何一种物理量——上的表示是$\frac{1}{\ii} \dv{t}$,那么
\[
    \comm*{\hat{H}}{\hat{A}} = \frac{1}{\ii} \dv{t},
\]
这就是时间演化方程。

考虑一个简单的单粒子量子力学的例子:$\hat{x} + a$是将大小为$a$的平移作用在$\hat{x}$上的结果,而考虑被$\hat{x}$完全描述的一个希尔伯特空间,在其上有
\[
    \hat{x} + a = \int \dd{x} x \dyad{x} + a \int \dd{x} \dyad{x} 
    = \int \dd{x} (x + a) \dyad{x} = \int \dd{x'} x' \dyad{x'-a},
\]
因此作用在$\hat{x}$上的大小为$a$的平移就等价于作用在态空间基矢量上的大小为$-a$的平移。
更一般的,将某一个李群$Q(a)$作用在某一算符上就相当于将这一李群的群参数倒转过来得到新的李群$Q'$,
也就是定义$Q'(a) = Q(a)^{-1}$(由于是群,$Q'$和$Q$同构),然后将$Q'(a)$作用在态空间的基矢量上。
由于$Q'$和$Q$同构,两者的区别仅仅是重新规定了群参数,因此它们对应着同样的对称性。
% TODO:以上说法的推广
总之,我们既可以直接从某种李群的场表示出发,推导它允许的算符场有哪些,然后使用二次量子化的有关知识导出其对应的单粒子态,%
\footnote{关于何为“粒子”需要说明:一般把能够使用一个不很复杂的CSCO描述的量子系统称为粒子,例如可以使用$\hat{\vb*{x}}$或$\hat{\vb*{p}}$描述一个粒子。但按照这种定义,原子能级也可以算粒子了——实际上这并不是胡思乱想,在处理量子光学等领域的一些问题时确实可以将能级看成一种粒子,定义其产生湮灭算符,得到费米场,等等——因此,何为粒子更多的是一种约定的说法。实际上任何一个哈密顿量都可以对角化,写出能级之后将不同能级看成不同粒子,然后使用二次量子化的语言描述它。}%
也可以从李群在希尔伯特空间上的表示出发,直接得到单粒子态然后构造算符场。
两种方法是完全一致的。舒尔引理告诉我们,卡西米尔算符(和所有生成元都对易)在不可约表示中一定是恒等算符的常数倍。这个常数可以用来标记相应的不可约表示;事实上这一类常数往往会出现在相应的表示描写的场/粒子的运动方程中,因为运动方程中会出现卡西米尔算符的场表示。
相对而言,在推导运动方程的时候,使用场的观点更加方便,因为相对论情况下粒子数通常是不确定的,因此使用单粒子态难以写出哈密顿量。

概括以下我们至今得到的结果:李群和李代数的表示有下面几种,它们彼此之间有非常密切的关系。
首先,李群和李代数在有限维向量空间上的表示是矩阵,它们或者是可约表示,或者不可约,前者可以通过直和运算由后者组装出来。
不可约有限维表示的结构可以通过使用李代数中的非Cartan元素构造Cartan子代数的升降算符来确定。
通过将有限维表示作用在坐标上,我们得到了作用在关于坐标的函数组成的向量空间上的无限维表示。
将作用在多分量对象上的有限维表示和作用在坐标函数上的无限维表示结合起来,就得到了场表示。
李群在向量空间上的表示很自然地就诱导出了李群在作用在向量空间上的算符上的表示。

\section{动力学}

% 似乎拉格朗日动力学中含有虚部的场要看成两个场,而哈密顿动力学中含有虚部的场只需要看成一个场。
在进一步展开下面的叙述之前,我们先回顾现代物理的数学框架。总的来说,有两套可用的框架,
其一是拉格朗日动力学,路径积分方法是它的量子版本;其二是哈密顿动力学,正则量子化是它的量子版本。
尽管这两个框架在数学上是独立的,我们仍然可以找到它们之间非常深厚的联系。

本节首先从经典拉氏量出发,然后得到经典哈密顿量,然后再过渡到量子形式。
常见的物理问题涉及$3+1$维闵可夫斯基时空中或$0+1$维时空,而后者可以看成前者的一个退化情况,
于是我们将局限在$3+1$维闵可夫斯基时空中,
虽然无论是拉格朗日动力学还是哈密顿力学都适用于比这广得多的体系。
所谓闵可夫斯基时空指的是度规可以化为
\begin{equation}
    \eta_{\mu\nu} = \diag (1, -1, -1, -1)
\end{equation}
的四维几何。通常使用$t, x, y, z$或者$x^0, x^1, x^2, x^3$来依次标记这4个坐标。
容易看出$x, y, z$或者说$x^1, x^2, x^3$就构成一个三维欧氏几何,它们是\textbf{空间维}。
$x^0$则是\textbf{时间维}。

我们还将假设,所有场量在无穷远处的值都是零。
我们将要分析的对象是时空中的场,它是从闵可夫斯基时空到某一线性空间的光滑映射。

\subsection{拉格朗日动力学}

所谓\textbf{拉氏量密度}$\mathcal{L}$——在场论中简称为\textbf{拉氏量}——是这样一个量,它是场的局域泛函,
这就是说,它可以写成$\phi, \partial_\mu \phi, \ldots$以及时空坐标的函数。
本文假定所有的拉氏量仅含有一阶导数,这是为了避免含有高阶导数的拉氏量产生“可以无穷下降的能量”等反直觉现象,并且简化计算。
幸运的是,已有的实验数据并不要求我们考虑更高阶的拉氏量。
我们还假定物理规律在时空上是均匀的,因此我们不认为拉氏量中显含时空坐标。%
\footnote{
    需要注意的是在系统中有相互作用且其中一部分的运动状态已知的情况下,另一部分的等效拉氏量中是有可能出现时空坐标的,
    例如粒子在势场中的运动就是一个典型例子,在那里由于产生势场的物理机制远远比粒子本身要强,因此势场可以看成是给定的,
    于是粒子具有的等效拉氏量就显含了空间坐标。}%
从而我们有
\begin{equation}
    \mathcal{L} = \mathcal{L}(\phi, \partial_\mu \phi).
    \label{eq:lagrangian}
\end{equation}
需要注意的是\eqref{eq:lagrangian}中的$\phi$可以代表任何一个“从时空坐标到数量”的映射,
它可能是一个标量场也可能是一个矢量场的分量,或者是别的什么东西。
\textbf{作用量}是拉氏量在整个闵可夫斯基时空上的积分。

现在我们将一个任意的无穷小变换作用在泛函$S$上,观察其无穷小变动。
需要注意的是无穷小变换同时作用在$\phi$的场值和坐标上,从而$\phi$完整的变化%
\footnote{在实际计算时往往更加容易求出$\var{\phi}$,因为一旦把$\phi'(x')$完全写出,只需要计算$\phi'(x')-\phi(x)$ 即可。}%
同时包含两部分:
\begin{equation}
    \var{\phi} = \bar{\var} \phi + \partial_\mu \phi \var{x}^\mu,
    \label{eq:variance-of-phi}
\end{equation}
其中第一项指的是场值本身的变化%
\footnote{这个变化又有可能来自两个方面。
其一是“场的平移”,也就是我们手动把场$\phi$加减特定值;
其二是“场的旋转”,当$\phi$实际上是某个更大的对象(如矢量)的某个分量时,基矢量的旋转会导致不同的分量混在一起。
通常我们使用一样的基矢量来书写场的分量和坐标的分量,因此除了坐标平移外,坐标变换也伴随着非零的$\bar{\var}{\phi}$。}%
,第二项指的是坐标变换的“拖曳”作用。
坐标的变化还会导致导数算符和积分测度发生变化。这两个几何效应的具体表达式为
\begin{equation}
    \begin{bigcase}
        \partial_{\mu'} = \partial_\mu - \partial_\mu \var{x^\nu} \partial_\nu, \\
        \dd[4]{x'} = (1 + \partial_\mu \var{x^\mu}) \dd[4]{x}.
    \end{bigcase}
\end{equation}
由于$\partial_\mu$算符随着坐标变换会发生变化,我们发现$\partial_\mu \phi$的变化量的形式和$\phi$不完全一致:
\begin{equation}
    \var{\partial_\mu \phi} = \partial_\mu \bar{\var}{\phi} + \partial_\mu \partial_\nu \phi \var{x^\nu}.
\end{equation}
这样一来我们可以计算出
\begin{equation}
    \var{S} = \int \dd[4]{x} \left(
        \left( \pdv{\mathcal{L}}{\phi} - \partial_\mu \pdv{\mathcal{L}}{\partial_\mu \phi} \right) \bar{\var}{\phi} + 
        \partial_\mu \left( \mathcal{L} \var{x^\mu} + \pdv{\mathcal{L}}{\partial_\mu \phi} \bar{\var}{\phi} \right)
    \right).
    \label{eq:variance-of-s}
\end{equation}
在推导\eqref{eq:variance-of-s}时我们没有使用任何关于$\var{\phi}$和$\var{x}$的假设,因此它给出的是最一般的$\var{S}$形式。

实际的场的动力学由保持时空坐标$x$不变且$\phi$在无穷远处固定为零(从而无穷远处$\bar{\var}{\phi}$为零)的情况下的泛函极值问题
\begin{equation}
    \var{S} = \var{\int \dd[4]x \mathcal{L}(\phi, \partial_\mu \phi)}
    \label{eq:min-action}
\end{equation}
给出。
显然这个泛函极值问题的解就是
\begin{equation}
    \pdv{\mathcal{L}}{\phi} - \partial_\mu \pdv{\mathcal{L}}{\partial_\mu \phi} = 0.
    \label{eq:el-eq}
\end{equation}
这就是欧拉-拉格朗日方程。
由于推导欧拉-拉格朗日方程时用到了$\var{\phi}$的任意性,这意味着$\phi$被假定是一个实的场。
如果某些场有虚部,那么在使用\eqref{eq:el-eq}以及相关结论的时候需要把它的实部和虚部分开,当成两个场来处理。
并且,容易证明,设复场$\phi$的实部和虚部分别是$\phi_1$和$\phi_2$,且
\[
    \pmqty{\psi_1 \\ \psi_2} = \pmqty{a & b \\ c & d} \pmqty{\phi_1 \\ \phi_2},
\]
其中$a,b,c,d$为复常数,则$\psi_1$和$\psi_2$的运动方程也可以从\eqref{eq:el-eq}得出。
常见的选择包括取
\[
    \psi_1 = \phi, \psi_2 = \phi^\dagger,
\]
或者如果$\phi$是多分量场,设有一系列复矩阵(不必都是复矩阵,有一个是复的就可以)$\gamma^\mu$,取
\[
    \psi_1 = \phi, \psi_2 = \gamma^\mu \phi_\mu.
\]

需要注意如果两个拉氏量的形式不同,这并不意味着它们描述了不同的物理过程。
实际上容易看出,两个拉氏量描述了相同的物理过程,
当且仅当,它们给出的作用量$S$只相差一个相对于$\dd[4]{x}$的零测集上的积分(这样的积分不影响泛函极值问题的求解,因为它“太小”),
这又等价于这两个拉氏量相差一个散度项,即存在一个$\Lambda^\mu$使得
\begin{equation}
\mathcal{L}' = \mathcal{L} + \partial_\mu \Lambda^\mu.
\end{equation}

当场量$\phi$是物理解的时候,将$\phi$代入到$S$中,然后再做一个无穷小变换,此时\eqref{eq:variance-of-s}中的第一项为零,
于是我们有
\[
    \var{S} = \int \dd[4]{x} \partial_\mu \left( \mathcal{L} \var{x}^\mu + \pdv{\mathcal{L}}{\partial_\mu \phi} \bar{\var}\phi \right).
\]
如果这个无穷小变换实际上不改变系统的动力学,也就是说系统在这个无穷小变化下是对称的,
那么$\var{S}$就应该能够写成一个表面积分,于是我们得到
\begin{equation}
    \partial_\mu \left(\pdv{\mathcal{L}}{\partial_\mu \phi} \bar{\var}\phi + \mathcal{L} \var{x^\mu} + \Lambda^\mu\right) = 0.
    \label{eq:noether}
\end{equation}
当然,如果无穷小变换更进一步不改变拉氏量,那么$\Lambda=0$。

如果无穷小变换是一个李群的李代数的表示,那么$\bar{\var}{\phi},\var{x^\mu}, \Lambda^\mu$都是完全确定的。可以使用小量近似将$\bar{\var}{\phi}$写成小量$ \ii \epsilon$乘以李代数的场表示\eqref{eq:fin-and-inf-rep},$\var{x^\mu}$写成小量$\ii \epsilon$乘以李代数的四维矢量表示,
于是我们在\eqref{eq:noether}中除去一个$\epsilon$,就得到了一个守恒流。
于是\eqref{eq:noether}的括号中的内容能够完全写成坐标的函数。
这就是\textbf{诺特定理}:系统的无穷小对称性诱导出一个守恒流。
由于是四维闵可夫斯基时空,四维的一个守恒流
\begin{equation}
    \partial_\mu j^\mu = 0
\end{equation}
就意味着三维的一个输运方程
\begin{equation}
    \partial_t j^0 + \partial_a j^a = 0.
\end{equation}
从而,
\begin{equation}
    Q = \int \dd[3]x j^0
\end{equation}
就是一个\textbf{守恒荷}。如果其积分范围是一个有限的区域,那么它就是一个局域守恒量,也就是
\[
    \dv{t} Q = - \int \dd{\vb*{S}} \cdot \vb*{j},
\]
而如果其积分范围是全空间,那么它就是守恒的。

我们来检查一下常见的对称性导致的守恒量。%
\footnote{表面上看,下面的讨论在体系并不非常对称的情况下并无意义,而不非常对称的体系占了多数。
不对称性带来的后果是,我们不再有完美的守恒流方程,取而代之的是一个有源的输运方程
\[
    \partial_\mu j^\mu = \text{something},
\]
由于对称性分析无助于找到源的具体形式,使用对称性诱导出特定的物理量似乎并没有什么意义。
然而,我们相信,最基本的物理定律总应该是对称的,因此大部分体系的不对称性可以归结为我们人为地将它从环境中隔离出来进行研究,从而导致类似下面的方程:
\[
    \partial_\mu (j^\mu_\text{sys} + j^\mu_\text{env}), \quad \partial j^\mu_\text{sys} = - j^\mu_\text{env}
\]
第二个方程给出了我们想要的含源的输运方程。因此在分析基本的物理框架时我们可以不讨论“不对称”的情况,
而是导出了基本的方程之后再通过“隔离出一部分系统”来引入不对称性。
}%
假定拉氏量在变换下不变。下面处理的问题都只含有一个场,不过由拉氏量的叠加性,在拉氏量含有多个场的时候只需要把各部分加起来即可。
首先是最简单的平移。处理平移时假定场是标量场,这无损一般性,因为平移没有有限维表示,因此不会导致场分量发生混合。
平移变换作用于场上得到的结果是:
\[
    \begin{split}
        x^\mu \longrightarrow x^{\mu'} = x^\mu + a^\mu, \\
        \var{\phi} = \phi'(x') - \phi(x) = 0.
    \end{split}
\]
% TODO:群作用怎么取
按照\eqref{eq:variance-of-phi},可以计算出
\[
    \bar{\var}{\phi} = - \partial_\mu \phi \var{a^\mu},
\]
或者,由于场在坐标拖曳下的变动实际上就是平移变换的无限维表示,可以直接使用平移变换的无限维表示
\[
    P_\mu = - \ii \partial_\mu
\]
得到上式。
于是对应的守恒流为
\[
    0 = \partial_\mu \left( - \pdv{\mathcal{L}}{\partial_\mu \phi} \partial_\nu \phi \var{a^\nu} + \mathcal{L} \var{a^\mu} \right) 
    = \partial_\mu \left( - \pdv{\mathcal{L}}{\partial_\mu \phi} \partial_\nu \phi + \mathcal{L} \delta^\mu_\nu \right) \var{a^\nu},
\]
考虑到$\var{a^\mu}$的任意性,我们有
\begin{equation}
    T_\mu^\nu = \pdv{\mathcal{L}}{\partial_\nu \phi} \partial_\mu \phi - \mathcal{L} \delta^\nu_\mu, \quad \partial_\nu T_\mu^\nu = 0.
\end{equation}
我们称$T^\nu_\mu$为\textbf{能动张量}。它给出了4个守恒荷,其中一个是来自时间平移不变性的\textbf{能量}
\begin{equation}
    E = \int \dd[3]{x} T^0_0 = \int \dd[3]{x} \left( \pdv{\mathcal{L}}{\partial_0 \phi} \partial_0 \phi - \mathcal{L} \right) ,
    \label{eq:field-energy}
\end{equation}
另外三个是来自空间平移不变性的\textbf{动量}
\begin{equation}
    P_i = \int \dd[3]{x} T^0_i = \int \dd[3]{x} \pdv{\mathcal{L}}{\partial_0 \phi} \partial_i \phi .
    \label{eq:field-momentum}
\end{equation}
能动张量的纯空间部分是能量和动量的输运流,因此就是\textbf{应力张量}。%
\footnote{在非相对论连续介质力学中这些结果也是成立的,因为时间和空间平移同时出现在伽利略群和庞加莱群中。}
相应的,
\begin{equation}
    \mathcal{P}_\mu = \pdv{\mathcal{L}}{\partial_0 \phi} \partial_\mu \phi - g_\mu^0 \mathcal{L}
\end{equation}
为四维动量$(E, \vb*{p})$的密度。
在计算场的三维动量时要注意一点:由于闵可夫斯基度规为$(+, -, -, -)$,闵可夫斯基时空中空间部分的基矢量实际上是指向空间坐标减少的方向的。从而,
\[
    \begin{aligned}
        \vb*{P} &= \int \dd[3]{x} \pdv{\mathcal{L}}{\partial_0 \phi} \partial_i \phi \vb*{g}^i \\
        &= - \int \dd[3]{x} \pdv{\mathcal{L}}{\partial_0 \phi} \partial_i \phi \vb*{g}^i_{\text{3dim}},
    \end{aligned}
\]
也即
\begin{equation}
    \vb*{P} = - \int \dd[3]{\vb*{x}} \pi \grad{\phi}.
\end{equation}

接下来是旋转对称性。%
\footnote{同样,这个对称性无论是在相对论性场论还是非相对论性场论中都是成立的。}%
旋转对称性不涉及时间维,于是我们有
\[
    \var{x^i} = \epsilon^i_{\ jk}  x^j \theta^k,
\]
相应的
\[
    \bar{\var}{\phi^a} = \ii (J_i)^a_{\ b} \theta^i \phi^b - \epsilon^i_{\ jk}  x^j \theta^k \partial_i \phi^a.
\]
其中指标$a,b$跑遍$\phi$的所有分量,$J$指的是旋转生成元在$\phi$所属的向量空间上的表示。

然后我们分析场的内禀对称性带来的守恒量。容易看出场的平移,也就是
\[
    \bar{\var}{\phi} = a, \; \var{x} = 0
\]
对应着守恒流
\[
    \partial_\mu \pdv{\mathcal{L}}{\partial_\mu \phi } a = 0,
\]
其守恒荷为
\begin{equation}
    \Pi = \int \dd[3]x \pdv{\mathcal{L}}{\partial_0 \phi}.
\end{equation}
这称为$\phi$的\textbf{共轭动量},相应的其密度
\begin{equation}
    \pi = \pdv{\mathcal{L}}{\partial_0 \phi}
    \label{eq:def-pi}
\end{equation}
就是\textbf{共轭动量密度}。
需注意此“动量”的名称只是类比而得,它未必和$P_i$有特别紧密的联系。

\subsection{哈密顿动力学}

\subsubsection{经典哈密顿动力学}
原本可以直接从拉氏量通过一个勒让德变换得到哈密顿动力学,但当底流形有多个坐标时我们需要选择合适的一个或几个坐标来充当“时间”,也就是哈密顿系统的参数。
共轭动量使我们有了一个很好的选择。本文取$t=x^0$为哈密顿系统的单参数。接下来我们要观察共轭动量的变化情况,从而凑出一个哈密顿系统。

容易看出$\Pi$的运动方程为%
\footnote{本节的结果也不仅仅适用于相对论性场论。任何能够良定义场的平移并且保证场平移不改变拉氏量的拉格朗日动力学场论都可以使用本节的方法构造对应的哈密顿表述,因为本节只用到了场的内禀平移不变性诱导出的结构。}
\[
    \dv{\Pi}{t} = \int \dd[3]x \partial_0 \pdv{\mathcal{L}}{\partial_0 \phi} = \int \dd[3]x \left( \pdv{\mathcal{L}}{\phi} - \partial_i \pdv{\mathcal{L}}{\partial_i \phi} \right).
\]
被积函数是$\int \dd[3]x \mathcal{L}$在将$x^0$当成常数后对$\phi$泛函求导的结果。于是定义%
\footnote{在$\phi$是多分量场的时候,我们把它看成列向量,记号$\partial \mathcal{L} / \partial \phi$定义为一个行向量,从而所有公式形式上仍然成立。例如,
\[
    \pdv{\vb*{a} \cdot \vb*{x}}{\vb*{x}} = \vb*{a}^\top.
\]
% TODO:需要使用度规吗?
}%
\begin{equation}
    H = \int \dd[3]x \mathcal{H} 
    = \int \dd[3]x \eval{\left( \pdv{\mathcal{L}}{\partial_0 \phi} \partial_0 \phi - \mathcal{L} \right)}_{\partial_0 \phi \to \pi} 
    = \int \dd[3]x \eval{\left( \pi \partial_0 \phi - \mathcal{L} \right)}_{\partial_0 \phi \to \pi}.
    \label{eq:lagrangian-to-hamitonian}
\end{equation}
我们通过将$\partial_0 \phi$用$\pi$表示使得任何对$H$的泛函求导都不会将$\partial_0 \phi$当成变量。
看出,$H$对$\phi$泛函求导就是$-\int \dd[3]x \mathcal{L}$对$\phi$泛函求导,于是我们有
\[
    \dv{\Pi}{t} = - \int \dd[3]x \fdv{H}{\phi}.
\]
另一方面由于$H$不显含任何$\pi$的导数,我们有
\[
    \begin{aligned}
        \fdv{H}{\pi} &= \pdv{\pi} \eval{\left( \pi \partial_0 \phi - \mathcal{L} \right)}_{\partial_0 \phi \to \pi} 
        = \partial_0 \phi + \pi \pdv{\partial_0 \phi}{\pi} - \pdv{\pi} \eval{\mathcal{L}}_{\partial_0 \to \pi} \\
        &= \partial_0 \phi + \pi \pdv{\partial_0 \phi}{\pi} - \pdv{\mathcal{L}}{\partial_0} \pdv{\partial_0}{\pi} = \partial_0 \phi.
    \end{aligned}
\]
于是就得到了3+1维场论的哈密顿表述:
\begin{equation}
    \dv{\pi}{t} = - \fdv{H}{\phi}, \quad \dv{\phi}{t} = \fdv{H}{\pi}.
    \label{eq:hamitonian-eq}
\end{equation}
其中$H$仅仅是$\phi, \partial_i \phi$和$\pi$的函数。
方程中的全导数也可以写成偏导数,我们把它写成全导数是因为我们通常只在一个固定的空间点观察场的变化,也就是说在\eqref{eq:hamitonian-eq}中我们只把时间看成变量而将空间坐标看成“标签”(见\autoref{note:spacial-label})。
由于我们讨论的基本上是场论问题,常常使用下面的记号:%
\footnote{在一些上下文中,场的时间全导数常常被定义为某个位置会随时间发生变化的场点处的场的导数,也就是
\[
    \dot{\phi} = \dv{t} \phi(\vb*{x}, t) = \pdv{\phi}{t} + \dot{\vb*{x}} \cdot \grad{\phi} = \partial_0 A = \partial^0 A.
\]
本文不涉及这样的问题,因此不使用这个记号。
}%
\[
    \dot{A} = \dv{A}{t} = \pdv{A(\vb*{x}, t)}{t}.
\]
在同一个场有多个分量的情况下,我们记各场为$\phi^i$,如果还是希望维持形式上的指标升降关系,$\pi$就可以写成$\pi_i$。

总之,使用拉氏量描述的3+1维经典场也能够使用一个哈密顿动力学描述,这个哈密顿动力学的演化参数为$x^0$也就是时间维,而使用空间维作为连续的“正则坐标”的标记。%
\footnote{也就是说,空间坐标$x^1, x^2, x^3$对应离散情况下的场量标签,
如$\phi^1(x, y, z)$指的是以$1, x, y, z$为标签的一个正则坐标,正如离散时的$q^{1}$代表以$1$为标签的一个正则坐标。
注意到这种哈密顿表述并没有以统一的方式对待时间和空间。\label{note:spacial-label}}%
任何物理量都是$\phi$和$\partial_\mu \phi$的函数,因此它们能够写成$\phi, \partial_i \phi$和$\pi$的函数,从而它们的演化都可以使用\eqref{eq:hamitonian-eq}确定,因为
\begin{equation}
    \dv{A}{t} = \pdv{A}{\phi} \dv{\phi}{t} + \pdv{A}{\partial_i \phi} \partial_i \dv{\phi}{t} + \pdv{A}{\pi} \dv{\pi}{t}.
    \label{eq:evolution-of-any-quantity}
\end{equation}

哈密顿动力学(无论是经典哈密顿动力学还是下一节讨论的正则量子化)中如果场是复的,仍然可以使用\eqref{eq:lagrangian-to-hamitonian}从拉氏量得到哈氏量,但此时不能够保证$\phi$、$\phi^\dagger$、$\pi$、$\pi^\dagger$彼此独立。
% TODO

\subsubsection{正则量子化}\label{sec:canonical-quantization}

下面我们转而讨论量子情况下的哈密顿动力学。这种使用哈密顿动力学建立量子理论的方法称为\textbf{正则量子化}。
完整地描述一个量子系统的状态和演化情况需要一个三元组:
首先是一个希尔伯特空间,称为\textbf{态空间},其中的矢量称为\textbf{态矢量},它们表示了系统的状态,
并且我们认为只差了一个倍数的态矢量等价,从而我们可以仅使用单位长度的态矢量描述任何的系统;
其次是一组\textbf{可观察量},它们是希尔伯特空间上的厄米算符,这意味着它们可以被幺正对角化,并且本征值都是实数%
\footnote{后面会提到,如果一个理论在正则量子化时必须选择反对易子的量子化方案,那么实际上它描写的场算符的本征值是格拉斯曼数。
但是如果我们在正则量子化的框架下工作,就从来不关注这种理论对应的场算符的本征值到底是多少,因此没有必要特意讨论它们。
在路径积分量子化的框架中,由于需要讨论费米子的经典场,格拉斯曼数是比较重要的。}%
;
最后是一个\textbf{哈密顿量}或者说\textbf{哈密顿算符},它自身也是一个可观察量(在经典极限下就是经典哈密顿量),且它指示了系统的演化方式。
经典哈密顿理论中同样有对应的三元组,
只不过态空间并不是一个可以做线性叠加的向量空间,从而可观察量也只是从态到实数的映射而不是希尔伯特空间上的算符。
由于所谓的场量$\phi$需要使用算符$\hat{\phi}$代替,因此不再能够良好地定义$\mathcal{H}$对各场量的偏导数,
从而我们也不能良好地定义$\var{H}/\var{\phi}$,等等。
现在动力学方程由
\begin{equation}
    \dv{\hat{A}}{t} = \frac{1}{\ii \hbar} [\hat{A}, \hat{H}]
    \label{eq:quantum-evolution}
\end{equation}
确定。%
\footnote{我们不讨论其定义显含时间的算符,因为它们不会出现在基本的物理规律中。}%
此时有意义的物理量虽然是算符,但在正则量子化之下仍然能够写成场算符$\phi, \partial_i \phi$和$\pi$的函数,
因此一旦$\phi$和$\pi$的演化确定了,\eqref{eq:evolution-of-any-quantity}就以一种和经典情况完全一致的方式确定了所有物理量的演化。
换而言之,\eqref{eq:hamitonian-eq-quantum}完全等价于
\begin{equation}
    \dv{\hat{\phi}}{t} = \frac{1}{\ii \hbar} [\hat{\phi}, \hat{H}], 
    \quad \dv{\hat{\pi}}{t} = \frac{1}{\ii \hbar} [\hat{\pi}, \hat{H}].
    \label{eq:hamitonian-eq-quantum}
\end{equation}

运动方程\eqref{eq:quantum-evolution}意味着,哈密顿量就是与时间平移不变性对应的守恒量。我们称其本征值为\textbf{能量},相应的,其本征态就是\textbf{能量本征态}。

要确定系统的动力学,只需要讨论$[\hat{\phi}, \hat{H}]$和$[\hat{\pi}, \hat{H}]$就可以,
而要讨论这两者又只需要讨论所有有关的场之间的对易关系即可,因为我们总是可以把$H$写成这些场的多项式。
(下文中讨论量子化方案时有对这一点的形象说明)
因此,取对易子为李括号,一个理论中涉及的所有算符就构成了一个李代数,而基本的场之间的对易关系又完全确定了这个李代数的结构。

仅仅有一个抽象的李代数并不能获得完整的理论。
例如,单粒子体系中$\hat{\vb*{x}}$和$\hat{\vb*{p}}$之间的李代数和多粒子体系中每一个粒子的$\hat{\vb*{x}}$和$\hat{\vb*{p}}$之间的李代数具有完全一样的结构,但是显然单粒子体系不是多粒子体系。
例如单粒子体系中$\vb*{x}$的谱没有简并而多粒子体系中$\vb*{x}$的谱有简并。
要获得完整的理论,我们还需要讨论态空间的结构。
我们将不讨论完整的数学,而只是对物理上常用的一些操作做一些说明。

当我们选定一个希尔伯特空间并且将(抽象的李代数中的)算符作用于其上时,实际上是对这个算符做了一个幺正表示。
进一步,当我们说一个系统的希尔伯特空间$H$能够被一组相互对易的算符$S$完全描述时,
我们实际上是说,算符集合$S$在$H$上的(幺正)表示组成了$H$上的一个完备相容算符集合,也即,$S$中各个算符在$H$上的表示共享的本征矢量构成$H$的一组基。
可以证明,如果$S_1,S_2$是$S$的一个划分,且$S_1$完全描述了$H_1$而$S_2$完全描述了$H_2$,那么就有$H$和$H_1 \otimes H_2$同构。
因此我们把$H$完全分解成了若干空间的直积,这些空间中的每一个都由完整描述系统需要的算符中的其中一个完全刻画。

一旦同时知道了各算符的对易关系(从而建立起它们的李代数),以及完整描述系统需要的完备相容算符集合,
我们就可以完整地推导出这个系统每一时刻的状态以及其演化方式了。
实际上,我们真正关注的是完备相容算符集合中各算符的谱结构。

根据上下文,我们可以容易地分辨作为抽象的李代数成员的算符,以及它们在各个希尔伯特空间上的表示,
因此为方便陈述,以下不再对这些略有不同的对象做详细的区分。
对算符而言这样做是合理的,因为从某个表示中得到的代数关系只要不涉及具体的表示的细节,就在抽象的李代数中也成立。
例如如果我们在某个表示中推导出$[\hat{x}, \hat{p}] = \ii \hbar \hat{I}$,那么在抽象的李代数中必定也有这个式子成立,
因为其中只牵扯到算符而没有牵扯到态矢量。
同样,可以比较容易地分辨各个希尔伯特空间中的态矢量,因此在不引起混淆的情况下我们也不刻意区分它们。

现在我们要做的是,分析$\phi$和$\pi$之间要具有什么样的代数关系%
\footnote{实际上只需要分析同一个时间$t$下$\phi(\vb*{x}, t)$和$\pi(\vb*{y}, t)$之间的关系就可以了,
因为\eqref{eq:hamitonian-eq-quantum}中从来不会出现不同时间的量之间的对易子。
这是量子版本的哈密顿动力学不适宜用于分析洛伦兹协变性的一个例子。
另外请注意这套理论并不能原封不动地适用于广义相对论时空,因为那里会需要讨论“不同时间处的量之间的关系”。},
才能够让\eqref{eq:hamitonian-eq-quantum}在$\hbar \to 0$时退化到经典情况\eqref{eq:hamitonian-eq}。
选定这样一个代数关系就称为选取一种\textbf{量子化方案},因为一旦给定了这样的代数关系,我们就把\eqref{eq:hamitonian-eq}推广到了量子理论中。
我们将不试图穷举所有可以的量子化方案,而只是举两个行之有效的例子——也就是说,实验数据要求使用这样的量子化方案。%
\footnote{我们说“穷举所有情况”意味着,面对同一个经典哈密顿量密度$\mathcal{H}$,
有不止一种指定$\hat{\phi}$和$\hat{\pi}$的方式,
使得我们能够得到一个量子动力学\eqref{eq:hamitonian-eq-quantum},
并且在$\hbar \to 0$的极限情况下回退到经典动力学\eqref{eq:hamitonian-eq}。
这是可以预期的,因为“取$\hbar\to 0$的极限”这个操作显然不是一一对应的,或者说量子化方案可以不止一种。
}

第一个方案是指定对易子为
\begin{equation}
    [\hat{\phi}^i(\vb*{x}, t), \hat{\pi}_j(\vb*{y}, t)] = \ii \hbar \delta^i_j \delta^3(\vb*{x} - \vb*{y}), 
    \quad [\hat{\phi}^i(\vb*{x}, t), \hat{\phi}^j(\vb*{y}, t)] = [\hat{\pi}_i(\vb*{x}, t), \hat{\pi}_j(\vb*{y}, t)] = 0.
    \label{eq:symmetry-commutator}
\end{equation}
从中可以容易地看出
\begin{equation}
    [\partial_\mu \hat{\phi}^i(\vb*{x}, t), \hat{\phi}^j(\vb*{y}, t)] = 0, 
    \quad [\grad{\hat{\phi}^i(\vb*{x}, t)}, \hat{\pi}_j(\vb*{y}, t)] = \ii \hbar \delta^i_j \grad{\delta^3(\vb*{x} - \vb*{y})}.
    \label{eq:symmetry-partial-mu-commutator}
\end{equation}
只需要将导数看成非常接近的两个量之差,然后利用对易子的线性性即可导出上式。

现在我们来推导$\hbar\to 0$时\eqref{eq:hamitonian-eq-quantum}的极限。
为方便起见,推导过程中假定$\hat{\phi}$是标量场。这并不会有损一般性,因为推导过程中没有用到任何关于坐标变换导致场变化的知识,从而我们可以把同一个多分量场的不同分量看成不同的标量场。
使用\eqref{eq:symmetry-commutator},已知$\hat{\phi} \hat{\pi}$就能够写出$\hat{\pi} \hat{\phi}$。
将$\hat{\mathcal{H}}$写成关于$\hat{\phi}, \partial_i \hat{\phi}$和$\pi$的多项式,
我们可以使用\eqref{eq:symmetry-commutator}和\eqref{eq:symmetry-partial-mu-commutator}
将形如$\hat{\pi}\hat{\phi}$、$\hat{\pi}\partial_i \hat{\phi}$这样的式子改写为形如$\hat{\phi}\hat{\pi}$、$\hat{\partial_i \hat{\phi}}\hat{\pi}$这样的式子。这称为取\textbf{正规积序}:我们总是可以使用对易关系把一个算符多项式转化为一个与之恒等的多项式,后者中的每一项中的算符排列顺序都满足一定的要求。
因此不失一般性地,我们认为$\hat{\mathcal{H}}$的多项式表达式中的每一项都形如$\hat{\phi}^l (\partial_i \hat{\phi})^m \pi^n$。
我们有(因为时间$t$都一样,以下略去$t$变量)%
\footnote{注意下面的$\delta(\vb*{x} - \vb*{x}')$是系数,因此可以自由地移动;写在它们左边的算符不会作用在它们上面!}
\[
    \begin{aligned}
        &\quad\comm{\hat{\phi}(\vb*{x}')}{\int \dd[3]{x} \hat{\phi}^l(\vb*{x}) (\partial_i \hat{\phi})^m (\vb*{x}) \pi^n(\vb*{x})} \\
        &= \int \dd[3]{x} \comm{\hat{\phi}(\vb*{x}')}{\hat{\phi}^l (\vb*{x}) (\partial_i \hat{\phi})^m (\vb*{x}) \pi^n (\vb*{x})} \\
        &= \int \dd[3]{x} \left( \hat{\phi}^l (\vb*{x}) \comm{\hat{\phi}(\vb*{x}')}{(\partial_i \hat{\phi})^m (\vb*{x}) \pi^n (\vb*{x})} + [\hat{\phi}(\vb*{x}'), \hat{\phi}^l (\vb*{x})] \partial_i \hat{\phi}^m (\vb*{x}) \pi^n(\vb*{x}) \right) \\
        &= \int \dd[3]{x} \hat{\phi}^l (\vb*{x}) \left( (\partial_i \hat{\phi})^m (\vb*{x}) \comm{\hat{\phi} (\vb*{x}')}{\hat{\pi}^n (\vb*{x})} + \comm{\hat{\phi} (\vb*{x}')}{(\partial_i \hat{\phi})^m (\vb*{x})} \pi^n (\vb*{x}) \right) \\
        &= \int \dd[3]{x} \hat{\phi}^l (\vb*{x}) (\partial_i \hat{\phi})^m (\vb*{x}) \comm{\hat{\phi} (\vb*{x}')}{\hat{\pi}^n (\vb*{x})} \\
        &= \int \dd[3]{x} \hat{\phi}^l (\vb*{x}) (\partial_i \hat{\phi})^m (\vb*{x}) \ii \hbar \delta^3(\vb*{x'} - \vb*{x}) n \hat{\pi}^{n-1}(\vb*{x}) \\
        &= \ii \hbar n \hat{\phi}^l (\vb*{x}') (\partial_i \hat{\phi})^m (\vb*{x}') \hat{\pi}^{n-1}(\vb*{x}'), 
    \end{aligned}
\]
这正是求导公式。当$\hbar$接近零的时候$\phi$和$\pi$可以交换,于是$\mathcal{H}$可以写成普通的、字母顺序无关紧要的函数,此时
我们有
\[
    \begin{aligned}
        \dv{\hat{\phi}}{t} = \frac{1}{\ii \hbar} [\hat{\phi}(\vb*{x}), \hat{H}] &= \frac{1}{\ii \hbar} \sum_\text{terms} \comm{\hat{\phi}(\vb*{x}')}{\int \dd[3]{x} \hat{\phi}^l(\vb*{x}) (\partial_i \hat{\phi})^m (\vb*{x}) \pi^n(\vb*{x})} \\
        &= \frac{1}{\ii \hbar} \sum_\text{terms} \ii \hbar n \hat{\phi}^l (\vb*{x}') (\partial_i \hat{\phi})^m (\vb*{x}') \hat{\pi}^{n-1}(\vb*{x}') \\
        &= \pdv{\hat{\mathcal{H}}}{\hat{\pi}} = \fdv{\hat{H}}{\hat{\pi}},
    \end{aligned}
\]
这就意味着$\hbar\to 0$时关于$\hat{\phi}$的方程能够回退到经典版本。
同样也有
\[
    \begin{aligned}
        &\quad \comm{\hat{\pi}(\vb*{x}')}{\int \dd[3]{x} \hat{\phi}^l(\vb*{x}) (\partial_i \hat{\phi})^m (\vb*{x}) \pi^n(\vb*{x})} \\ 
        &= \int \dd[3]{x} \comm{\hat{\pi}(\vb*{x}')}{\hat{\phi}^l(\vb*{x}) (\partial_i \hat{\phi})^m (\vb*{x}) \pi^n(\vb*{x})} \\
        &= \int \dd[3]{x} \left( \hat{\phi}^l (\vb*{x}) (\partial_i \hat{\phi} )^m (\vb*{x}) \comm{\hat{\pi}(\vb*{x}')}{\hat{\pi}^n(\vb*{x})} + \hat{\phi}^l (\vb*{x}) \comm{\hat{\pi}(\vb*{x}')}{(\partial_i \hat{\phi})^m (\vb*{x})} \hat{\pi}^n (\vb*{x}) + \comm{\hat{\pi}(\vb*{x}')}{\hat{\phi}^l (\vb*{x})} (\partial_i \hat{\phi})^m (\vb*{x}) \hat{\pi}^n (\vb*{x}) \right) \\
        &= \int \dd[3]{x} \left( - \hat{\phi}^l (\vb*{x}) \ii \hbar \grad{\delta^3 (\vb*{x} - \vb*{x}')} m (\partial_i \hat{\phi})^{m-1} (\vb*{x}) \hat{\pi}^n (\vb*{x}) - \ii \hbar \delta^3(\vb*{x} - \vb*{x}') l \hat{\phi}^{l-1} (\vb*{x}) (\partial_i \hat{\phi})^m (\vb*{x}) \hat{\pi}^n (\vb*{x}) \right) \\
        &= \ii \hbar \partial_i \left(m (\partial_i \hat{\phi})^{m-1} (\vb*{x}') \hat{\pi}^n (\vb*{x}')\right) - \ii \hbar l \hat{\phi}^{l-1} (\vb*{x}') (\partial_i \hat{\phi})^m (\vb*{x}') \hat{\pi}^n (\vb*{x}').
    \end{aligned}
\]
当$\hbar \to 0$时
\[
    \begin{aligned}
        \dv{\hat{\pi}}{t} &= \frac{1}{\ii \hbar} [\hat{\pi}(\vb*{x}), H] \\
        &= \frac{1}{\ii \hbar} \sum_\text{terms} \left(\ii \hbar \partial_i \left(m (\partial_i \hat{\phi})^{m-1} (\vb*{x}') \hat{\pi}^n (\vb*{x}')\right) - \ii \hbar l \hat{\phi}^{l-1} (\vb*{x}') (\partial_i \hat{\phi})^m (\vb*{x}') \hat{\pi}^n (\vb*{x}')\right) \\
        &= - \sum_\text{terms} \left(l \hat{\phi}^{l-1} (\vb*{x}') (\partial_i \hat{\phi})^m (\vb*{x}') \hat{\pi}^n (\vb*{x}') - \partial_i \left(m (\partial_i \hat{\phi})^{m-1} (\vb*{x}') \hat{\pi}^n (\vb*{x}')\right) \right) \\
        &= - \left( \pdv{\hat{\mathcal{H}}}{\hat{\phi}} - \partial_i \pdv{\hat{\mathcal{H}}}{\partial_i \hat{\phi}} \right) = - \fdv{\hat{H}}{\hat{\phi}},
    \end{aligned}
\]
因此关于$\pi$的方程也回退到经典情况。
这表明\eqref{eq:symmetry-commutator}是一个可行的量子化方案。

第二个方案是指定反对易子——而不是对易子——为%
\footnote{彼此无关的场,无论它们自己服从\eqref{eq:symmetry-commutator}还是\eqref{eq:antisymmetry-commutator},相互之间总是对易的。在\eqref{eq:symmetry-commutator}中这是显然的,因为可以将无关的场看成某个多分量场的分量,然后因为它们是不同的分量,它们自然对易。但在\eqref{eq:antisymmetry-commutator}方案下需要额外增加一个规定:
\[
    \comm*{\hat{\phi}(\vb*{x}, t)}{\hat{\psi}(\vb*{y}, t)} = 0.
\]
}
\begin{equation}
    \{\hat{\phi}(\vb*{x}, t), \hat{\pi}(\vb*{y}, t)\} = \ii \hbar \delta^3(\vb*{x} - \vb*{y}), \quad \{\hat{\phi}(\vb*{x}, t), \hat{\phi}(\vb*{y}, t)\} = \{\hat{\pi}(\vb*{x}, t), \hat{\pi}(\vb*{y}, t)\} = 0.
    \label{eq:antisymmetry-commutator}
\end{equation}
同样,我们可以将哈密顿量写成若干个$\hat{\phi}^l (\partial_i \hat{\phi})^m \pi^n$形式的项的和。
需要注意的是\eqref{eq:antisymmetry-commutator}直接导出
\[
    \hat{\phi}(\vb*{x})^2 = 0, \quad \hat{\pi}(\vb*{x})^2 = 0, \quad (\partial_i \hat{\phi})^2(\vb*{x}) = 0,
\]
因此哈密顿量中$l, m, n \leq 1$。
这意味着这个量子化方案并不适用于所有的场,而是只适用于能够保证在任何情况下哈密顿量中的每一项都满足$l, m, n \leq 1$的场。
对于正常的实数/复数值场,这是一个不可能的事情。
事实上,设$\hat{\phi}$在$\vb*{x}$处的值为$\int \dd \phi(\vb*{x}) \dyad{\phi}$,
在$\vb*{y}$处的值为$\int \dd \phi(\vb*{y}) \dyad{\phi}$,则通过反对易关系能够得到
\[
    \phi(\vb*{x}) \phi(\vb*{y}) = - \phi(\vb*{y}) \phi(\vb*{x}).
\]
因此,反对易子意味着对应的场算符的本征值——也就是其经典极限——实际上并不是实数,甚至也不是复数,而是格拉斯曼数。
在复数域中满足反对易关系的场算符不能被对角化。
在路径积分量子化中,格拉斯曼数非常重要,因为路径积分量子化会分析经典场值的演化路径。
在正则量子化中只需要把这些格拉斯曼数看成算符(准确地说,是产生算符)就可以了——我们并不会用到它的微积分,因此也无需将它们看成数。

为了看出反对易方案的不同寻常,我们指出如下事实:一个通过\eqref{eq:antisymmetry-commutator}量子化的场不可能是厄米的。
我们有
\[
    \hat{\phi}(\vb*{x}, t) \hat{\pi} (\vb*{y}, t) + \hat{\pi} (\vb*{y}, t) \hat{\phi} (\vb*{x}, t) = \ii \hbar \delta(\vb*{x} - \vb*{y}),
\]
于是
\[
    \left(\hat{\phi}(\vb*{x}, t) \hat{\pi} (\vb*{y}, t) + \hat{\pi} (\vb*{y}, t) \hat{\phi} (\vb*{x}, t)\right)^\dagger = - \ii \hbar \delta(\vb*{x} - \vb*{y}),
\]
如果场是厄米的,那么就有
\[
    \hat{\pi} (\vb*{y}, t) \hat{\phi} (\vb*{x}, t) + \hat{\phi}(\vb*{x}, t) \hat{\pi} (\vb*{y}, t) = - \ii \hbar \delta(\vb*{x} - \vb*{y}).
\]
于是我们得到了一个矛盾。因此,使用\eqref{eq:antisymmetry-commutator}量子化的场应该分解成非零的厄米和反厄米部分,即
\begin{equation}
    \hat{\phi} = \hat{\phi}_1 + \ii \hat{\phi}_2,
\end{equation}
其中$\hat{\phi}_1$和$\hat{\phi}_2$分别是两个厄米算符。
对应的,描述它的拉氏量当中的场有实部和虚部,需要把它们——或者它们的线性组合——看成两个独立的场来列写\eqref{eq:el-eq}。

此外,$\mathcal{H}$中各项阶数的限制还意味着由此导出的运动方程在时间上只能是一阶的。
从而,$\pi$和$\phi$不是彼此独立的。这样,在哈密顿量只关于$\phi$和$\pi$时,我们总是可以适当地调节拉氏量和哈密顿量,或者对$\phi$和$\pi$做一些线性变换,使得$\pi$和$\phi$之间有线性关系。
这个关系显然不能是“乘以某个倍数”,否则将不能够区分这两个变量。
因此两者之间的关系涉及复共轭。通常取
\begin{equation}
    \pi = \ii \phi^\dagger.
\end{equation}
这也表明了取$\phi$为复场的重要性——否则将不能够区分$\phi$和$\pi$,从而难以建立哈密顿动力学。
% TODO:这段还是有问题

需要注意的是,无论是\eqref{eq:symmetry-commutator}还是\eqref{eq:antisymmetry-commutator},实际上都假定了$\phi^i$和$\pi_i$在时空变换下是协变的。
在场具有某些附加结构——例如,有某些外加约束以消除非物理的自由度——的时候,如果我们直接把独立的自由度拿出来写成$\phi^i$,就不能保证它们的协变性(虽然把原来的场恢复出来之后它仍是协变的),此时不能直接套用\eqref{eq:symmetry-commutator}或\eqref{eq:antisymmetry-commutator},而需要使用带约束的场论的有关知识。

此外,虽然本节通过表明指定对易子或者反对易子能够得到经典哈密顿动力学来论证量子化方案\eqref{eq:symmetry-commutator}和\eqref{eq:antisymmetry-commutator}的合理性,但是实际上这两个方案在本文展示的经典哈密顿动力学以外仍然适用。例如,如果哈氏量中出现了广义动量的导数,那么\eqref{eq:hamitonian-eq}需要做出修正,但是\eqref{eq:hamitonian-eq-quantum}仍然适用。换而言之,本节展示的量子动力学实际上才是最根本的理论。
% TODO:真的吗?

从正则量子化得到的算符运动方程就是经典的场运动方程算符化的结果,而后者又等价于通过最小作用量原理求出的运动方程。
这就产生了一个问题:路径积分量子化告诉我们,最小作用量原理只是路径积分的最速下降近似而已,
为什么在正则量子化中精确的运动方程却可以从最小作用量原理求出?
其原因在于,算符在演化过程中不同的本征态会混在一起(一个经典情况下不可能出现的现象),正是这一点构成了量子和经典的区别,
正则量子化中的本征态混合正好对应于路径积分量子化中非经典的路径。

\subsubsection{时间演化和绘景}\label{sec:time-evolution}
% TODO: 表征了相同的物理状态的态矢量之间只差了一个复数常数。好量子数就是守恒量(这是一个算符!)的值,它可以用来标记态。
在\autoref{sec:canonical-quantization}中我们仅仅将态矢量当成一个可以让场算符作用上去的对象。
但实际上如果我们想要的话,也可以让态矢量动起来而对算符做对应的修改,使得算符的谱结构始终不变(本征矢量重数一一对应、彼此对应的本征矢的内积相同),并且本征值不变。
只要算符的谱结构不变、对应的各个本征值不变,算符就正确地描述了系统。
两个算符的谱结构一致、对应的本征值相同的充要条件是它们酉相似(相似矩阵可以随时间变化)。
需要注意的是两个描述了同一个系统的算符会给出不同的基矢量,所以切换绘景的时候还需要改变态矢量。
综上,绘景变换公式为
\begin{equation}
    \hat{A}' = \hat{Q} \hat{A} \hat{Q}^\dagger, \quad \ket{\psi'} = \hat{Q} \ket{\psi},
    \label{eq:picture-trans}
\end{equation}
其中$\hat{Q}$为一个幺正算符,它可以显含时间。
对易子在绘景变换之下会发生下面的改变:
\begin{equation}
    \comm{\hat{A}}{\hat{B}} \longrightarrow \hat{Q} \comm{A}{B} \hat{Q}^\dagger = \comm{\hat{A}'}{\hat{B}'}.
\end{equation}

在\autoref{sec:canonical-quantization}中我们已经讨论了态矢量固定不动时怎么确定系统的动力学。
这种让态矢量固定、算符变动的方案称为\textbf{海森堡绘景}。以下我们使用上标$H$代表海森堡绘景下的量。
我们要证明的第一件事是,不同时间点上的同一个可观察量的值彼此酉相似。
要看清楚这是为什么,我们将酉相似的方程
\begin{equation}
    \hat{A}^H (t) = \hat{U}^H(t, t_0) \hat{A}^H (t_0) (\hat{U}^H)^\dagger(t, t_0)
    \label{eq:quantum-evolution-hes-u-operator}
\end{equation}
做一个等价变换,看看它等价于什么。%
\eqref{eq:quantum-evolution-hes-u-operator}中的$U$在$t=t_0$时必定为恒等变换,因为此时$\hat{A}^H (t) = \hat{A}^H (t_0)$;同时容易看出$\hat{U}^H(\tau)$实际上构成一个李群。这样我们就能够写出其生成元,记之为$\hat{G}(t)$:
\[
    \hat{U}^H(t+\dd{t}, t) = \hat{I} + \frac{\ii}{\hbar} \hat{G}(t) \dd{t}.
\]
$\hat{U}^H$是幺正的等价于$\hat{G}$是厄米的。
于是就能够写出\eqref{eq:quantum-evolution-hes-u-operator}的无穷小等价形式:
\[
    \hat{A}^H (t_0) + \dd{\hat{A}^H}(t_0) = \left( \hat{I} + \frac{\ii}{\hbar} \hat{G}(t) \dd{t} \right) \hat{A}^H (t_0) \left( \hat{I} - \frac{\ii}{\hbar} \hat{G}(t) \right) = \hat{A}^H (t_0) + \frac{\dd{t}}{\ii \hbar} \comm{\hat{A}^H}{\hat{G}(t)}.
\]
我们发现这就是\eqref{eq:quantum-evolution},只需要把$\hat{G}(t)$换成$\hat{H}(t)$;并且正则量子化的时候已经要求$\hat{H}$是厄米的了,因此$\hat{G}$的确是厄米的,从而$\hat{U}^H$是幺正的。
于是我们得出结论:海森堡绘景中的算符演化实际上是在做幺正变换,或者等价地说,海森堡绘景中各算符的本征态在做幺正变换。算符的变换式为\eqref{eq:quantum-evolution-hes-u-operator},相应的,本征态的变换式为
\begin{equation}
    \ket{a(t)} = \hat{U}^H(t, t_0) \ket{a(t_0)}.
\end{equation}
于是我们称$\hat{U}^H$为海森堡绘景下的时间演化算符。
$\hat{U}^H$可以写出显式表达式
\begin{equation}
    \hat{U}^H(t, t_0) = T \exp \left( \frac{\ii}{\hbar} \int_{t_0}^t \dd{t} \hat{H}^H (t) \right).
\end{equation}
注意\eqref{eq:quantum-evolution-hes-u-operator}保证了,一个可观察量在经过时间演化之后仍然是可观察量。

现在我们尝试使用\eqref{eq:picture-trans}来把时间演化完全转移到态矢量上面。
因此,我们希望在新的绘景中,$\hat{A}$始终不变。我们称这新的绘景为\textbf{薛定谔绘景}。
按照\eqref{eq:quantum-evolution-hes-u-operator},有
\[
    \hat{A}^H(t) = \hat{U}^H(t, t_0) \hat{A}^H (t_0) (\hat{U}^H)^\dagger(t, t_0) = \hat{U}^H(t, t_0) \hat{A}^S( \hat{U}^H)^\dagger(t, t_0),
\]
不失一般性地我们取$t=0$时的$\hat{A}^H$为$\hat{A}^S$,那么我们有
\[
    \hat{A}^H (t) = \hat{U}^H(t, 0) \hat{A}^S( \hat{U}^H)^\dagger(t, 0).
\]
将这个方程和\eqref{eq:picture-trans}对比可以看出
\[
    \hat{Q} = (\hat{U}^H)^\dagger(t, 0),
\]
于是得到薛定谔绘景下的态矢量演化公式
\[
    \ket{\psi^S(t)} = \hat{Q} \ket{\psi^H} = (\hat{U}^H)^\dagger (t, 0) \ket{\text{a constant}},
\]
考虑到$t=0$时$\hat{U}^H (t, 0)$就是恒等算符,上式又等价于
\[
    \ket{\psi^S(t)} = (\hat{U}^H)^\dagger (t, 0) \ket{\psi^S (0)},
\]
也即,薛定谔绘景下的时间演化算符和海森堡绘景下的时间演化算符互为逆。
这个方程还告诉我们,
\[
    \ket{\psi^H} = \ket{\psi^S(t_0)}.
\]
现在推导时间演化方程的微分形式。我们有
\[
    \begin{aligned}
        \ket{\psi^S (t + \dd{t})} &= \left( \hat{U}^H (t + \dd{t}, t) \hat{U}^H (t, 0)  \right)^\dagger \ket{\psi^S(0)} \\
        &= \left( (\hat{I} + \frac{\ii}{\hbar} \hat{H}(t) \dd{t})   \hat{U}^H (t, 0) \right)^\dagger \ket{\psi^S (0)} \\
        &= (\hat{U}^H)^\dagger (t, 0) \ket{\psi^S (0)} + \frac{\dd{t}}{\ii \hbar} (\hat{U}^H)^\dagger (t, 0) \hat{H}(t) \ket{\psi^S (0)} \\
        &= \ket{\psi^S (t)} + \frac{\dd{t}}{\ii \hbar} (\hat{U}^H)^\dagger (t, 0) \hat{H}(t) \hat{U}^H (t, 0) \ket{\psi^S (t)},
    \end{aligned}
\]
从而
\[
    \ii \hbar \dv{t} \ket{\psi^S (t)} = (\hat{U}^H)^\dagger (t, 0) \hat{H}(t) \hat{U}^H (t, 0) \ket{\psi^S (t)}.
\]
为了方便区分,我们将海森堡绘景中的$\hat{H}$记作$\hat{H}^H$,则它对应的薛定谔绘景中的算符为
\[
    \hat{H}^S = \hat{Q} \hat{H}^H \hat{Q}^\dagger = (\hat{U}^H)^\dagger (t, 0) \hat{H}^H(t) \hat{U}^H (t, 0), 
\]
这正是薛定谔绘景中态矢量的运动方程中出现的那个量,因此就获得了薛定谔绘景中的运动方程:
\begin{equation}
    \ii \hbar \dv{t} \ket{\psi^S(t)} = \hat{H}^S (t) \ket{\psi^S(t)}.
\end{equation}
% TODO:证明$\hat{H}^I_i$确实是$\hat{H}_i^H$在相互作用绘景下的
设$\hat{U}^S(t, t_0)$是薛定谔绘景下的时间演化算符,则容易证明$\hat{H}^S$是它的生成元,既然$\hat{H}^H$是厄米的,$\hat{H}^S$也是厄米的,从而$\hat{U}^S$是幺正的。%
\footnote{注意$\hat{H}^H$是$\hat{U}^H$的生成元而$\hat{H}^S$是$(\hat{U}^H)^\dagger$的生成元;由于$\hat{H}^H$可能含时,一般情况下
\[
    T \exp(\int \hat{H}^H (t) \dd{t})^\dagger \neq T \exp(- \int \hat{H}^H (t) \dd{t}),
\]
也就是说$\hat{H}^H$和$\hat{H}^S$之间没有简单的关系,而必须使用绘景变换公式联系两者。
}
因此薛定谔绘景中时间演化始终保持态矢量的幺正性。
时间演化算符的显式表达式为
\begin{equation}
    \hat{U}^S(t, t_0) = T \exp \left( - \frac{\ii}{\hbar} \int_{t_0}^t \dd{t} \hat{H}^S(t) \right),
\end{equation}
其中$T$为编时算符。

为了明显起见,我们将薛定谔绘景和海森堡绘景中哈密顿量相互换算的关系重复如下:
\begin{equation}
    \begin{aligned}
        \hat{H}^H(t) = T \exp \left( \frac{\ii}{\hbar} \int_{t_0}^t \dd{t} \hat{H}^H(t) \right) \hat{H}^S(t_0) \left(T \exp \left( \frac{\ii}{\hbar} \int_{t_0}^t \dd{t} \hat{H}^H(t) \right)\right)^\dagger, \\
        \hat{H}^S(t_0) = T \exp \left( - \frac{\ii}{\hbar} \int_{t_0}^t \dd{t} \hat{H}^S \right) \hat{H}^H(t) \left( T \exp \left( - \frac{\ii}{\hbar} \int_{t_0}^t \dd{t} \hat{H}^S \right)\right)^\dagger,
    \end{aligned}
\end{equation}
在$\hat{H}^H$在各个时间点的值彼此对易时,$\hat{U}^H$无非是$\hat{H}^H$的级数,因此它们对易,从而$\hat{H}^S$和$\hat{H}^H$相等。
这也等价于$\hat{H}^S$在各个时间点的值彼此对易。

事实上,虽然我们是从海森堡绘景出发建立我们的理论框架的,但\autoref{sec:back-to-classical}告诉我们,和经典力学中的系统状态直接对应的实际上就是态矢量,而不是算符,因此很多文献是从薛定谔绘景出发建立理论的。

% TODO:这里有些地方写得是有问题的。标准的相互作用绘景应该是薛定谔绘景的推论。但这个也奇怪得很:量子场论中的微扰论难道是使用薛定谔绘景的吗??
现在我们已经讨论了“让可观察量变动”和让基矢量变动“两种方案的不同了。我们还可以把哈密顿算符分解成一个比较简单的不含时部分和一个含时的部分,并要求这两者均为厄米算符,然后分别用两者让算符和态矢量都动起来。这样的方案称为\textbf{相互作用绘景}。
为方便起见,考虑从薛定谔绘景到相互作用绘景的变换。当然也可以从海森堡绘景出发推导相互作用绘景,但实际上这样会很不自然。对薛定谔绘景下的哈密顿量做分解
\begin{equation}
    \hat{H}^S = \hat{H}_0^S + \hat{H}_i^S,
\end{equation}
称前者为\textbf{自由哈密顿量}(通常我们要求它不显含时间),后者为\textbf{相互作用哈密顿量},并指定
\begin{equation}
    \ket{\psi^I(t)} = \hat{U}_0^\dagger(t,t_0) \ket{\psi^S(t)},
\end{equation}
其中
\begin{equation}
    \hat{U}_0 = T \exp \left( - \frac{\ii}{\hbar} \int_{t_0}^t \dd{t} \hat{H}_0^S(t) \right).
\end{equation}
于是可观察量的绘景变换为
\begin{equation}
    \hat{A}^I(t) = \hat{U}_0^\dagger(t,t_0) \hat{A}^S(t) \hat{U}_0(t,t_0).
    \label{eq:operator-from-schodinger-to-interaction}
\end{equation}
通过求导,分别可以计算出态矢量和可观察量的时间演化方程为
\begin{equation}
    \ii \hbar \dv{t} \ket{\psi^I(t)} = \hat{H}^I_i(t) \ket{\psi^I(t)},
    \label{eq:time-evolution-in-interation-picture}
\end{equation}
以及
\begin{equation}
    \dv{t} \hat{A}^I(t) = \frac{1}{\ii \hbar} \comm*{\hat{A}^I(t)}{\hat{H}_0^I}.
\end{equation}
其中$\hat{H}_0^I$和$\hat{H}_i^I$正是对$\hat{H}_0^S$和$\hat{H}_i^S$做绘景变换\eqref{eq:operator-from-schodinger-to-interaction}得到的结果。
这样我们就成功地让时间演化分别由态矢量和可观察量各自承担一部分。

如果我们在海森堡绘景中工作,要怎么样切换到相互作用绘景中呢?最一般的公式非常复杂。
但是,实际上,如果哈密顿量含时,通常直接在薛定谔绘景中工作;如果哈密顿量不含时,那么薛定谔绘景和海森堡绘景下的哈密顿量是一样的,那么只需要选择一个较简单的可观察量$\hat{H}_0$,指定它为$\hat{H}_0^S$,就可以切换到相互作用绘景。需注意整个过程并没有用到$\hat{H}_0^H$,一般来说,它和$\hat{H}_0^S$可能会有区别,但是我们从来不关注这个区别。

相互作用绘景在微扰量子场论计算中起到了非常重要的作用,因为通过对称性分析可以直接得到自由场的哈密顿量密度和演化方程,因此我们可以将相互作用项——也就是不同场之间的耦合——独立考虑,从而大大简化计算。
更加重要的是,此时相互作用绘景可以为我们提供有关量子场的态空间的结构的信息。如果假定态空间中有一个唯一的真空态——也就是所有场都是零的态——那么量子场的态空间就是多粒子态福克空间,在此基础上我们可以很自然地处理粒子创生和湮灭的过程。这就为我们展示了量子理论的另一面:波动看起来就像粒子一样。%
\footnote{需要注意的是,在处理相对论性量子场论的时候其实并不能完全放心地使用相互作用绘景。如果我们取$\hat{H}_i=0$,那么相互作用绘景就退化为了自由场的海森堡绘景;这样我们就看到了$\hat{H}_i$项的作用:它把带相互作用的场的态(也就是$\ket{\psi^I(t)}$)和自由场的态($\ket{\psi^I(0)}$,因为如果$\hat{H}_i=0$那么态就不会变化)使用一个幺正算符联系了起来,而且这个幺正算符是唯一的。然而Haag定理说,含相互作用的场有无数个不等价的幺正表示,因此我们并不能唯一地将带相互作用的场的态和自由场的态使用一个唯一的幺正算符联系起来。特别的,由于我们要求自由场和相互作用场的态空间都满足一定的物理条件(如有稳定的真空态,等等),自由场的态空间和相互作用场的满足这些条件的态空间一般来说并不幺正等价。这意味着类似于$\int \dd{t} \hat{H}^H_i$之类的表达式实际上并不收敛,于是相互作用绘景就失效了。但是有很多手段可以绕过这个定理的限制——例如因为我们从来只讨论一定能标下的物理现象而不把相对论性量子场论当成终极理论,实际上我们可以把空间格点化,这样量子场论就变成了有限自由度的量子力学,于是就可以使用相互作用绘景了。}

此外容易验证,各种形式的时间演化算符都满足以下公式:
\begin{equation}
    \hat{U}(t_3,t_2) \hat{U}(t_2,t_1) = \hat{U}(t_3,t_1),
\end{equation}
以及
\begin{equation}
    \hat{U}^\dagger (t_2, t_1) = \hat{U} (t_1, t_2).
\end{equation}

\subsubsection{测量}\label{sec:measure}

\textbf{测量}指的是这样一个过程:两个系统(分别称为\textbf{待测系统}和\textbf{仪器})发生相对剧烈而时间短促的相互作用,相互作用后待测系统的态发生很大改变,而仪器的态则体现了相互作用前待测系统的某些信息。
采用相互作用绘景,设$\hat{q}$完全描述了仪器的态空间,$\hat{a}$是关于待测系统的某个算符,它和另一个算符$\hat{b}$共同描述了待测系统的态空间。(被测量的量$\hat{a}$未必能够完整描述待测系统。下文中需要将待测系统的态做展开,因此引入$\hat{b}$)
由于相互作用非常剧烈而时间短促,仪器和待测系统的相互作用哈密顿量可以写成
\begin{equation}
    H_\text{int} = - \gamma(t-t_0) \hat{a} \otimes \hat{p},
\end{equation}
% 为什么偏偏就是这个形式?为什么所有量都是一次项?
其中$\hat{p}$是$\hat{q}$对应的共轭动量,也就是说
\[
    \comm*{\hat{q}}{\hat{p}} = \ii \hbar,
\]
$\gamma$是一个函数,它是一个$t_0$附近的尖峰。
极限情况下,$\gamma(t) = g \delta(t)$,这称为\textbf{冯诺依曼测量}或者\textbf{标准量子测量},
我们在相互作用绘景下分析问题。系统初态为
\[
    \ket{i} = \ket{\psi_i} \ket{D} = \int \dd{q} \sum_{k, n} \braket{q}{D} \braket{a_k, b_n}{\psi_i} \ket{a_k} \ket{b_n} \ket{q},
\]
其中$\ket{\psi_i}$和$\ket{D}$分别为待测系统和仪器的初态,本征态$\ket{a}_k$,$\ket{b}_n$和$\ket{q}$是$t_0$时刻对应算符的本征态(下同)。%
\footnote{提醒:算符本征态反映的是算符的代数结构,它们的时间演化是由自由哈密顿量而不是相互作用哈密顿量指导的。}%
我们要求$\hat{q}$是连续谱,而$\hat{a}$和$\hat{b}$可以是离散谱也可以是连续谱。要求$\hat{q}$是连续谱的原因很快就可以看到。
系统的末态为
\[
    \begin{aligned}
        \ket{f} &= T \exp \left( - \frac{\ii}{\hbar} \int \dd{t} H_\text{int} \right) \ket{i} \\
        &= T \exp \left( \frac{\ii}{\hbar} g \int \dd{t} \delta(t-t_0) \hat{a}(t) \otimes \hat{p}(t) \right) \ket{i} \\
        &= \exp \left( \frac{\ii}{\hbar} g \hat{a}(t_0) \otimes \hat{p}(t_0) \right) \ket{i} \\
        &= \sum_{n=0}^\infty \frac{1}{n!} \left(\frac{\ii}{\hbar} g\right)^n \hat{a}(t_0)^n \hat{p}(t_0)^n \int \dd{q} \sum_{k, l} \braket{q}{D} \braket{a_k, b_l}{\psi_i} \ket{a_k} \ket{b_l} \ket{q} \\
        &= \int \dd{q} \sum_{k, l} \braket{q}{D} \braket{a_k, b_l}{\psi_i} \sum_{n=0}^\infty \frac{1}{n!} \left(\frac{\ii}{\hbar} g\right)^n \hat{a}(t_0)^n \ket{a_k} \ket{b_l} \hat{p}(t_0)^n \ket{q} \\
        &= \int \dd{q} \sum_{k, l} \braket{q}{D} \braket{a_k, b_l}{\psi_i} \sum_{n=0}^\infty \frac{1}{n!} \left(\frac{\ii}{\hbar} g\right)^n a_k^n \ket{a_k} \ket{b_l} \hat{p}(t_0)^n \ket{q} \\
        &= \int \dd{q} \sum_{k, l} \braket{q}{D} \braket{a_k, b_l}{\psi_i} \ket{a_k} \ket{b_l} \sum_{n=0}^\infty \frac{1}{n!} \left(\frac{\ii}{\hbar} g a_k \hat{p}(t_0) \right)^n \ket{q} \\
        &= \int \dd{q} \sum_{k, l} \braket{q}{D} \braket{a_k, b_l}{\psi_i} \ket{a_k} \ket{b_l} \exp \left( \frac{\ii}{\hbar} g a_k \hat{p}(t_0) \right) \ket{q} \\
        &= \int \dd{q} \sum_{k, l} \braket{q}{D} \braket{a_k, b_l}{\psi_i} \ket{a_k} \ket{b_l} \ket{q + g a_k} \\
        &= \int \dd{q} \sum_{k, l} \braket{q - g a_k}{D} \braket{a_k, b_l}{\psi_i} \ket{a_k} \ket{b_l} \ket{q} .
    \end{aligned}
\]
总之我们得到经典测量前后态的变化公式
\begin{equation}
    \ket{f} = \int \dd{q} \sum_{k, l} \braket{q - g a_k}{D} \braket{a_k, b_l}{\psi_i} \ket{a_k} \ket{b_l} \ket{q}.
    \label{eq:standard-measurement}
\end{equation}
需要注意的是由于我们采取的是相互作用绘景,算符$\hat{a}$和$\hat{b}$一直会发生变化。
然而,由于自由哈密顿量不显含时间,\eqref{eq:standard-measurement}中$\ket{a_k, b_l}$的时间演化和$\bra{a_k, b_l}$的时间演化抵消了,等等,从而$\ket{f}$在相互作用结束后没有时间演化——正如我们预期的那样,因为相互作用结束之后相互作用哈密顿量就是零。

\eqref{eq:standard-measurement}看起来仍然十分复杂。
然而,在很多情况下(具体是什么情况我们很快会看到)仪器的初始态非常接近$\hat{q}$的本征态,也就是说$\braket{q}{D}$只有在$q$和某一个$q_0$非常接近的时候才有较大的值,其余时候都接近零,因此实际上是一个$\delta$函数。
这样的情况称为\textbf{理想测量}。我们现在可以看到为什么要求$\hat{q}$具有连续谱了,因为要实施一次理想测量必须允许仪器有连续分布的状态。此时\eqref{eq:standard-measurement}近似为
\begin{equation}
    \ket{f} = \sum_{k, l} \braket{a_k, b_l}{\psi_i} \ket{a_k} \ket{b_l} \ket{q = q_0 + g a_k}.
    \label{eq:ideal-measurement}
\end{equation}
我们这样就得到了一个典型的纠缠态,其中每一个分量中,仪器和待测系统在测量之后都处于完全对应的状态。
总之,如果待测系统和仪器组成的系统和外界毫无相互作用,那么测量就是如下所示的过程:
\[
    \ket{i} = \left(\sum_{k, l} \braket{a_k, b_l}{\psi_i} \ket{a_k} \ket{b_l} \right) \ket{D} \longrightarrow \ket{f} = \sum_{k, l} \braket{a_k, b_l}{\psi_i} \ket{a_k} \ket{b_l} \ket{q = q_0 + g a_k},
\]
也就是待测系统将其信息复制到了仪器当中。
然而,假如仪器足够大,那么待测系统和仪器组成的系统和外界将会有大量的相互作用。
例如,仪器可能被放置在灯光下来方便我们读取其示数,这就意味着它要不停地受到四面八方的光子的轰击。
这就意味着\eqref{eq:ideal-measurement}会很快发生退相干,最后终结于$\hat{a} \otimes \hat{b} \otimes \hat{q}$的某个本征态上,因此最后仪器停留在某个$q=q_0 + g a_k$附近,且待测系统的态也转化为$\ket{a_k}$。
将待测系统和仪器组成的系统以及所有可能的环境变量放在一起就得到了一个系综;系综中,待测系统和仪器组成的系统在退相干之后停留在本征态$\ket{a_k} \ket{b_l} \ket{q = q_0 + g a_k}$的概率正是$\abs{\braket{a_k, b_l}{\psi_i}}^2$,
也就是说,在时刻$t$测量$\hat{a}$得到$a_k$(同时将待测系统的态转化为$\ket{a_k}$)的概率就是
\begin{equation}
    P_t(a_k) = \sum_l \abs{\braket{a_k, b_l(t)}{\psi_i}}^2,
    \label{eq:probablity-of-measurement}
\end{equation}
由\eqref{eq:probablity-of-measurement}出发容易证明,待测系统为$\ket{\psi_i}$态时做测量,测量值的期望为
\begin{equation}
    \expval{\hat{a}}(t) = \mel{\psi_i}{\hat{a}(t)}{\psi_i}.
\end{equation}

实际上,我们可以把四面八方的光子或者空气分子或者这一类的干扰看成是一个巨型仪器:它和待测系统的相互作用使待测系统和它的态按照\eqref{eq:ideal-measurement}纠缠在一起,而由于这是开放体系,退相干快速发生,这就意味着在充满干扰的环境中实际上很难真的展示出待测系统的量子特性:待测系统几乎总是出现在其偏好本征态附近,因为它没完没了地受到测量。
这也是理想测量很容易就能够实现的原因:真的会用来做测量的仪器总是被做得很大,因此它们自身可以看成不停地被空气、杂散光或者别的什么东西不断测量的系统,因此它们的态总是出现在其偏好本征态附近。

在$\hat{a}$本身是待测系统的一个CSCO,从而不需要$\hat{b}$的情况下,测量$\hat{a}$得到$a_k$的概率为
\begin{equation}
    P(a_k) = \abs{\braket{a_k}{\psi_i}}^2.
\end{equation}
这表明,假如我们有一个正交归一化基$\{\ket{a_k}\}_k$,就可以使用一组不同的实数$a_k$构造算符
\[
    \hat{a} = \sum_k a_k \dyad{a_k},
\]
使用这个算符对系统做测量,则测量结束之后系统位于态$\ket{a_k}$的概率就是
\begin{equation}
    P(\ket{a_k}) = \abs{\braket{a_k}{\psi_i}}^2.
\end{equation}
注意到这个表达式只和$\ket{a_k}$有关。因此,对态矢量为$\ket{\psi}$的系统做一次测量,发现系统测量后处于态$\ket{\phi}$的概率为
\begin{equation}
    P(\ket{\phi}) = \abs{\braket{\phi}{\psi}}^2,
\end{equation}
于是我们称$\braket{\phi}{\psi}$为\textbf{概率振幅}。

需注意以上讨论建立在几个关键假设上:其一,仪器和待测系统的相互作用非常强而短促;其二,仪器和环境有杂乱无章的相互作用。
这意味着合理地构造不怎么受外界干扰而又不会严重地扰动待测系统的仪器,我们就能够得到关于待测系统状态的不完整信息而与此同时不让待测系统的态塌缩到某个本征态上。
这称为\textbf{弱测量}。

\subsubsection{有效哈密顿量}

有时,一个物理系统的哈密顿量涉及大量复杂的过程,而特定的初始条件意味着这个系统的态基本上只会出现在态空间的一小部分当中。
但这并不意味着态空间的其它部分就不会对系统的动力学造成影响。
例如,设想一个三能级系统,其中一个能级的能量远远高于另外两个能级,这意味着系统基本上不可能出现在这个能级上,但如果其余两个能级和这个高能量能级有耦合,那么这个高能量能级就可能成为另外两个能级相互转换的渠道。

设投影算符$\hat{P}$选择出了我们关注的那部分态空间,而且这部分态空间的定义不随时间变化而变化;设$\hat{Q}$是与之互补的投影算符,则
\[
    \hat{P} + \hat{Q} = 1, \quad \hat{P}^2 = \hat{P}, \quad \hat{Q}^2 = \hat{Q}.
\]
考虑薛定谔绘景,运动方程为
\[
    \hat{H} \ket{\psi} = \ii \hbar \dv{t} \ket{\psi},
\]
将投影算符作用于其上得到
\[
    \begin{aligned}
        \hat{P} \hat{H}(\hat{P}+\hat{Q}) \ket{\psi} = \ii \hbar \dv{t} \hat{P} \ket{\psi}, \\
        \hat{Q} \hat{H}(\hat{P}+\hat{Q}) \ket{\psi} = \ii \hbar \dv{t} \hat{Q} \ket{\psi}.
    \end{aligned}
\]
哈密顿量可以分成四部分,一部分完全位于$\hat{P}$筛选出来的空间中,一部分完全位于$\hat{Q}$筛选出来的空间中,另外两部分从其中一个空间跳跃到另一个空间,这四部分分别是
\[
    \hat{H}_{PP} = \hat{P} \hat{H} \hat{P}, \quad \hat{H}_{QQ} = \hat{Q} \hat{H} \hat{Q}, \quad \hat{H}_{PQ} = \hat{P} \hat{H} \hat{Q}, \quad \hat{H}_{QP} = \hat{Q} \hat{H} \hat{P}.
\]
使用投影算符的性质可以写出
\[
    \begin{aligned}
        \hat{H}_{PP} \hat{P} \ket{\psi} + \hat{H}_{PQ} \hat{Q} \ket{\psi} = \ii \hbar \dv{t} \hat{P} \ket{\psi}, \\
        \hat{H}_{QP} \hat{P} \ket{\psi} + \hat{H}_{QQ} \hat{Q} \ket{\psi} = \ii \hbar \dv{t} \hat{Q} \ket{\psi},
    \end{aligned}
\]
从后一个方程可以解出
\[
    \hat{Q} \ket{\psi} = \frac{1}{\ii \hbar \dv{t} - \hat{H}_{QQ}} \hat{H}_{QP} \hat{P} \ket{\psi},
\]
代入前一个方程就得到
\[
    \ii \hbar \dv{t} \hat{P} \ket{\psi} = \left(\hat{H}_{PP} + \hat{H}_{PQ} \frac{1}{\ii \hbar \dv{t} - \hat{H}_{QQ}} \hat{H}_{QP}\right) \hat{P} \ket{\psi}.
\]
因此我们发现,我们关注的那一部分态的时间演化由等效哈密顿量
\begin{equation}
    \hat{H}_\text{eff} = \hat{H}_{PP} + \hat{H}_{PQ} \frac{1}{\ii \hbar \dv{t} - \hat{H}_{QQ}} \hat{H}_{QP}
    \label{eq:effective-hamiltonian-original}
\end{equation}
指导,而且由$\hat{H}_{PP}, \hat{H}_{PQ}, \hat{H}_{QP}$的定义,该等效哈密顿量是$\hat{P}$筛选出的空间中的算符。
\eqref{eq:effective-hamiltonian-original}非常符合我们的直觉:时间演化可以仅仅涉及$\hat{H}_{PP}$,也可以以$\hat{H}_{QQ}$为中介。

一种特殊的情况是,态空间可以写成两个空间(记为$\mathcal{H}_1$和$\mathcal{H}_2$)的直积,系统的初始条件决定了大部分有意义的过程都发生在$\mathcal{H}_1$中,但由于耦合,不能简单地将$\mathcal{H}_2$排除掉。
这时可以构造算符$\hat{P}$使之筛选出只在$\mathcal{H}_1$中有显著活动的态,计算出有效哈密顿量;$\hat{P}$筛选出的态均形如$\ket{\psi}_1 \otimes \ket{0}_2$,由于有效哈密顿量仅涉及$\ket{\psi}_1$,不会出现两个空间之间的耦合,于是可以直接将$\mathcal{H}_2$去掉,使用$\mathcal{H}_1$和$\hat{H}_\text{eff}$来描述系统。

然而,\eqref{eq:effective-hamiltonian-original}显含一个时间求导算符的倒数,这意味着$\hat{H}_\text{eff}$实际上显含时间,而且还显含关于时间的算符,也即我们实际上是手动把关于$\mathcal{H}_2$的时间演化放进了有效哈密顿量当中,这是不便计算的。
为了让有效哈密顿量看起来像一个正常的哈密顿量,设我们考虑的过程的能量近似在$E_r$水平上,则对$\mathcal{H}_1$空间中的态,近似有
\[
    \ii \hbar \dv{t} \sim E_r,
\]
于是
\[
    \hat{H}_\text{eff} \sim \hat{H}_{PP} + \hat{H}_{PQ} \frac{1}{E_r - \hat{H}_{QQ}} \hat{H}_{QP}.
\]
对$\mathcal{H}_1$中$\hat{H}$的本征态而言,上式严格成立,我们得到自洽方程
\begin{equation}
    \left( \hat{H}_{PP} + \hat{H}_{PQ} \frac{1}{E - \hat{H}_{QQ}} \hat{H}_{QP} \right) \ket{\psi} = E \ket{\psi}.
\end{equation}
从这个方程求解出$E$,我们就得到了$\hat{H}_\text{eff}$在$\mathcal{H}_1$的一组基上的作用结果,于是也就完全确定下了$\hat{H}_\text{eff}$。
换而言之,完全精确求解的有效哈密顿量保留了原哈密顿量在我们关注的空间上的全部能谱。

然而,即使上述自洽方程也难以求解。为此通常使用微扰展开的方法。
设原哈密顿量中$\mathcal{H}_1$与$\mathcal{H}_2$没有耦合的部分为$\hat{H}_0$,其余部分为$\hat{H}'$,也即,以$\mathcal{H}_1$和$\mathcal{H}_2$为子空间将算符做分块,则$\hat{H}_0$包含对角部分,$\hat{H}'$包含非对角部分,则
\[
    \hat{H}_\text{eff} \sim \hat{H}_{0} + \hat{H}'_{PQ} \frac{1}{E_r - \hat{H}_{QQ}} \hat{H}'_{QP}.
\]
如果$\hat{H}'$让

在高能自由度和低能自由度的耦合并不明显时,高能自由度的存在与否对$\mathcal{H}_1$中的能量本征态只有不大的影响,这时可以以原哈密顿量中仅包含低能自由度的部分的本征值和本征态为起点,以低能自由度和高能自由度的耦合以及高能自由度的哈密顿量为微扰,求解出$\hat{H}$在$\mathcal{H}_1$中的本征态$\{\ket{n}\}$和本征值$\{E_n\}$。由于是本征态,它们和高能自由度没有耦合,于是低能自由度的运动完全由
\begin{equation}
    \hat{H}_\text{eff} = \sum_{\ket{n} \in \mathcal{H}_1} E_n \dyad{n}
\end{equation}
确定,我们也就得到了有效哈密顿量。

\begin{equation}
    \mel{m}{\hat{H}_\text{eff}}{n} = E_m \delta_{mn} + \mel{m}{\hat{H}'}{n} + \frac{1}{2} \sum_{\text{$l$ in $\mathcal{H}_2$}} \left( \frac{\mel{m}{\hat{H}'}{l} \mel{l}{\hat{H}'}{n}}{E_m - E_l} + \frac{\mel{m}{\hat{H}'}{l} \mel{l}{\hat{H}'}{n}}{E_n - E_l} \right) + \cdots.
\end{equation}

另一种做实际计算的方法是,考虑$\hat{H}$的本征态$\ket{\psi_n}$,其能量为$E_n$,则我们有
\[
    \hat{H} \ket{\psi_n} = E_n \ket{\psi_n}.
\]
有效哈密顿量只需要指导$\mathcal{H}_1$中的态的运行即可,因此它需要满足
\[
    \hat{H}_\text{eff} \ket{\psi_n} = E_n \ket{\psi_n}, \quad \text{for $\ket{\psi_n} \in \mathcal{H}_1$}.
\]
由于$\hat{H}_\text{eff}$是$\mathcal{H}_1$中的算符,以上方程的展开式为
\begin{equation}
    \sum_{\text{span}\{\ket{m}\} = \mathcal{H}_1} (\mel{l}{\hat{H}_\text{eff}}{m} - E_n \delta_{lm}) \braket{m}{\psi_n} = 0.
    \label{eq:effective-hamiltonian-eq-unfolded}
\end{equation}
这里我们使用一组任意的正交归一化基底$\{\ket{m}\}$,它们未必就是哈密顿量的本征态。相应的,$\hat{H}$满足
\begin{equation}
    \sum_m (\mel{l}{\hat{H}}{m} - E_n \delta_{lm}) \braket{m}{\psi_n} = 0, \quad \text{for $\ket{l} \in \mathcal{H}_1$}.
    \label{eq:hamiltonian-eq-unfolded}
\end{equation}
显然,\eqref{eq:hamiltonian-eq-unfolded}必须能够推导出\eqref{eq:effective-hamiltonian-eq-unfolded},线性代数上的结论告诉我们,记那些张成$\mathcal{H}_1$的基矢量的编号组成的集合为$W$,其余基矢量的编号组成的集合为$U$,则有
\[
    \mel{l}{\hat{H}_\text{eff}}{m} = \mel{l}{\hat{H}}{m} - \sum_{\alpha \in U} \frac{\mel{l}{H}{\alpha} \mel{\alpha}{H}{m}} {D_\alpha^W} + \sum_{\alpha \neq \beta \in U} \frac{\mel{l}{H}{\alpha} \mel{\alpha}{H}{\beta} \mel{\beta}{H}{m}}{D^W_{\alpha \beta}} + \cdots,
\]
其中,记$S$为一系列编号组成的集合,单脚标的$D$函数定义为
\[
    D_\alpha^S = H_{\alpha \alpha} - E_n - \sum_{\beta \in U, \beta \notin S} \frac{\mel{\alpha}{H}{\beta} \mel{\beta}{H}{\alpha}}{D_\alpha^S} + \sum_{\beta \neq \gamma \in U, \; \beta, \gamma \notin S} \frac{\mel{l}{H}{\alpha} \mel{\alpha}{H}{\beta} \mel{\beta}{H}{m}}{D_{\alpha \beta}^S} + \cdots,
\]
多脚标的$D$函数递归定义为
\[
    D_{\alpha \beta}^S = D_\alpha^S D_{\beta}^{S, \alpha}, \quad D_{\alpha, \beta, \gamma}^S = D_{\alpha}^S D_{\beta}^{S, \alpha} D_{\gamma}^{S, \alpha, \beta}, \ldots
\]
% TODO:和通常使用的微扰论有何关系???

\subsection{关于单位制的注记}

到现在为止我们的理论还带有一些常数。用以标记我们的理论多大程度上偏离了经典情况的$\hbar$是一个重要的常数,同时标记了时间和空间的换算关系的光速$c$是另外一个。
通过做变换
\[
    t \longrightarrow t' = ct,
\]
我们可以让光速$c$从所有的公式中消失。相应的,时间导数算符发生了
\[
    \partial_t \longrightarrow \partial_{t'} = \frac{1}{c} \partial_t
\]
的变换。
$\hbar$在计算对易子的时候出现。做变换
\[
    \pi \longrightarrow \pi' = \frac{\pi}{\hbar}
\]
也可以完全消去这个常数。由于$\pi$是通过$\mathcal{L}$对$\partial_0 \phi$求导计算出来的,这个变换实际上就是对拉氏量做了变换
\[
    \mathcal{L} \longrightarrow \mathcal{L}' = \frac{\mathcal{L}}{\hbar},
\]
而这当然不影响实际的物理。事实上它改变的是能量和动量的单位。

从本节开始,在本文的剩余部分我们将使用自然单位制,那就是说,取消时间和空间的单位差异,并且取$\hbar = 1$。
从自然单位制恢复到国际单位制就是把上面的变换反过来,也就是做变换
\[
    \begin{aligned}
        \mathcal{L}_\text{nat} &\longrightarrow \mathcal{L}_\text{int} = \hbar \mathcal{L}_\text{nat}, \\
        E_\text{nat} &\longrightarrow E_\text{int} = \hbar E_\text{nat} , \\
        \vb*{p}_\text{nat} &\longrightarrow \vb*{p}_\text{int} = \hbar \vb*{p}_\text{nat}, \\
        t_\text{nat} &\longrightarrow t_\text{int} = c t_\text{nat}.
    \end{aligned}
\]
与此同时保持各个公式的形式不变。

\subsection{单粒子情况}

在已经知道了3+1维场论的理论之后,单粒子情况实际上就是一个退化情况,因为它实际上是0+1维场论。
在单粒子情况下底流形就是时间轴,其上定义有各种物理量$\hat{A}(t)$。单粒子情况下几乎不需要使用反对易量子化方案\eqref{eq:antisymmetry-commutator},物理量和它的共轭动量之间的关系可以全部取
\begin{equation}
    \comm*{\hat{x}}{\hat{p}} = \ii.
\end{equation}
下面推导$\hat{x}, \hat{p}$和任意物理量的对易关系。
设能够将物理量$\hat{F}$展开为$\hat{x}, \hat{p}$的多项式$\hat{F} = F(\hat{x}, \hat{p})$。
对其中的每一项,都可以使用对易关系
\[
    \comm*{\hat{x}}{\hat{p}} = \ii
\]
把$\hat{x}$挪到最前面而把$\hat{p}$挪到后面,
因此展开式最后就可以写成若干个$a \hat{x}^m \hat{p}^n$形式的项之和。
现在分析其中的一项:
\[
    [\hat{x}, \hat{x}^m \hat{p}^n] = \hat{x}^m [\hat{x}, \hat{p}^n] + [\hat{x}, \hat{x}^m] \hat{p}^n = \hat{x}^m [\hat{x}, \hat{p}^n],
\]
而
\[
    [\hat{x}, \hat{p}^n] = [\hat{x}, \hat{p} \hat{p}^{n-1}] = 
    \hat{p} [\hat{x}, \hat{p}^{n-1}] + [\hat{x}, \hat{p}] \hat{p}^{n-1} = \hat{p} [\hat{x}, \hat{p}^{n-1}] + \ii \hat{p}^{n-1}
\]
于是递推得到
\[
    [\hat{x}, \hat{p}^n] = \ii n \hat{p}^{n-1},
\]
因此
\[
    [\hat{x}, \hat{x}^m \hat{p}^n] = \ii n \hat{x}^m \hat{p}^{n-1}.
\]
这样就可以写出
\begin{equation}
    [\hat{x}, \hat{F}(\hat{x}, \hat{p})] = \ii \pdv{p} \hat{F}(\hat{x}, \hat{p}),
\end{equation}
在作用偏微分符号之前需要先把$F$中的每一项都变形成$\hat{x}$在前$\hat{p}$在后的形式。
使用同样的方法还可以导出
\begin{equation}
    [\hat{p}, \hat{F}(\hat{x}, \hat{p})] = - \ii \pdv{x} \hat{F}(\hat{x}, \hat{p}),
\end{equation}
同样,作用偏微分符号之前需要先把$F$中的每一项都变形成$\hat{x}$在前$\hat{p}$在后的形式。

在海森堡绘景下
\[
    \dv{\hat{A}}{t} = \frac{1}{\ii} [\hat{A}, H] + \pdv{\hat{A}}{t},
\]
于是
\[
    \dv{\hat{x}}{t} = \frac{1}{\ii} [\hat{x}, H] = \pdv{p} \hat{H}(\hat{x}, \hat{p}), \quad
    \dv{\hat{p}}{t} = \frac{1}{\ii} [\hat{p}, H] = -\pdv{x} \hat{H}(\hat{x}, \hat{p})
\]
当$\hbar \to 0$时,上式仍然成立,而此时$\hat{x}$和$\hat{p}$已经是对易的了,因此它们退化为了可以直接使用实数表示的情况,我们也就过渡到了经典力学。

\section{准经典理论}

\subsection{退化到经典情况}\label{sec:back-to-classical}

前面提到,$\hbar \to 0$时,海森堡绘景下的量子时间演化方程\eqref{eq:quantum-evolution}退化为经典的时间演化方程\eqref{eq:evolution-of-any-quantity}。
但需要注意的是,在$\hbar\to 0$时由\eqref{eq:quantum-evolution}退化得到的方程仍然是一个算符方程。
要获得通常的关于物理量的方程,还需要做一些操作。$\hbar\to 0$时得到的演化方程是
\[
    \dv{\hat{A}}{t} = \pdv{\hat{A}}{\hat{\phi}} \dv{\hat{\phi}}{t} + \pdv{\hat{A}}{\partial_i \hat{\phi}} \partial_i \dv{\hat{\phi}}{t} + \pdv{\hat{A}}{\hat{\pi}} \dv{\hat{\pi}}{t},
\]
这个方程仅在海森堡绘景下成立。记系统的态矢量为$\ket{\psi}$,我们就得到
\[
    \dv{t} \mel{\psi}{\hat{A}}{\psi} =  \mel{\psi}{\pdv{\hat{A}}{\hat{\phi}} \dv{\hat{\phi}}{t}}{\psi} + \mel{\psi}{\pdv{\hat{A}}{\partial_i \hat{\phi}} \partial_i \dv{\hat{\phi}}{t}}{\psi} + \mel{\psi}{\pdv{\hat{A}}{\hat{\pi}} \dv{\hat{\pi}}{t}}{\psi}.
\]
在$\hbar\to 0$时,所有算符都近似是对易的,从而它们全部可以在同一组基下对角化。设这一组基为$\{\ket{n}\}$,则
% TODO:似乎$\ket{\psi}$总是几乎是这组基中的一个,为什么?
\[
    \begin{aligned}
        \mel{\psi}{\hat{A}\hat{B}}{\psi} &= \sum_{m,n} \braket{\psi}{m} \mel{m}{\hat{A}\hat{B}}{n} \braket{n}{\psi} \\
        &= \sum_{n} \braket{\psi}{n} \mel{n}{\hat{A}\hat{B}}{n} \braket{n}{\psi} \\
        &= 
    \end{aligned}
\]
% TODO
% 总之核心思想是,算符在$\hbar\to 0$时也不是实数物理量,真正的实数物理量的表达式必定会牵扯到态矢量。这也就是场算符的傅里叶分量看起来似乎是固定的值一样的原因,因为场算符本身包含了所有可能的经典场的取值,在$\hbar\to 0$时经典场的取值是多少不是场算符决定的而是态矢量决定的。

正则对易关系
\[
    [q_i, p_j] = \delta_{ij}
\]
实际上是非常自然的,因为使用这个关系推导出来的方程和使用对应的拉氏量和E-L方程推导出来的运动方程是一样的。
任何两个物理量的对易子$[A,B]$最后都可以写成一系列形如$\gamma_1 \gamma_2 \cdots [\gamma, \gamma] \cdots$这样的项的
叠加,其中每一个$\gamma$都是一个基本算符(坐标、动量、自旋等等),如果我们已知$[p, q] = \ii \hbar \cdot \text{something}$
而运动方程为
\[
    \dv{A}{t} = \frac{1}{\ii \hbar} [A, H]
\]
那么在最后得到的运动方程中$\ii \hbar$就被消去了。
现在让$\hbar \to 0$,我们会发现运动方程的形式没有发生变化(因为它根本就和$\hbar$无关),但是此时所有的物理量都是对易的了。
重新定义
\[
    \{A, B\} = \frac{1}{\ii \hbar}[A, B],
\]
它$\hbar \to 0$时仍然收敛于有限值。然后使用对易关系可以推导出它就是所谓的泊松括号。

使用不随时间变化的态矢量$\ket{\psi}$表述系统。
可观察量算符$A$随着时间的演化为
\begin{equation}
    \dv{A}{t} = \frac{1}{\ii \hbar} [A, H] + \pdv{A}{t}
    \label{eq:canonical-time-evolution}
\end{equation}

形式上这个式子和经典力学中的式子差了一个系数$\ii \hbar$。表面上看这正是量子力学和经典力学不同的地方(引入了常数$\hbar$),但实际上并非如此,因为在量子力学中有
\begin{equation}
    [x_i, p_j] = \ii \hbar \delta_{ij}
\end{equation}
一来一去,系数$\ii \hbar$就约掉了,实际上,完全可以定义
\[
    [x_i, p_j] = \delta_{ij}
\]
而此时的演化方程就变成
\[
    \dv{A}{t} = [A, H] + \pdv{A}{t}
\]
形式上和经典情况完全一致。那么量子力学和经典力学到底相差在哪里?
最关键的差别实际上是,量子力学中的$x, p$等量都是算符,因此有可能
\[
    AB - BA \neq 0
\]
而经典情况下上式恒为零。并且,这个不对易性直接和$[\cdot, \cdot]$的定义有关:
\[
    [A, B] = AB - BA
\]
在经典力学中$AB-BA$也是一个反对称的运算,但是它恒为零,因此和系统的演化无关——经典力学中和系统演化有关的那种$[\cdot, \cdot]$完全由
\[
    [A, B] = \sum_i \left( \pdv{A}{q_i} \pdv{B}{p_i} - \pdv{A}{p_i} \pdv{B}{q_i} \right)
\]
定义,上式又等价于两个假设:乘法交换律,以及
\[
    [q_i, q_j] = 0, [p_i, p_j] = 0, [q_i, p_j] = \delta_{ij}
\];
而在量子力学中,我们假定$xp-px=\ii \hbar$,并且认为
\[
    [A, B] = AB - BA
\]

总之,在括号$[\cdot, \cdot]$的性质、坐标和动量之间的括号的取值上,经典力学和量子力学之间完全没有差异。两者的差异在于,经典力学假定所有物理量都是可交换的实数,此时我们可以推导出泊松括号的表达式;量子力学假定$[\cdot, \cdot]$就代表两个物理量(现在是算符了!)的交换子。

因此在经典力学中使用“对易”一词可能引起误解:它可能指“两个量的乘积是不是可以交换”,此时的回答一概是“是”;它也可能指“两个量的泊松括号是不是零”。这两种理解之间完全没有联系。而在量子力学中这两种理解实际上是等价的。

可观察量经过时间演化之后还应该是可观察量。但是这个怎么证明呢?

\[
    \left(\dv{A}{t}\right)^\dagger = - \frac{1}{\ii \hbar} [A, H]^\dagger + \left(\pdv{A}{t}\right)^\dagger = \frac{1}{\ii \hbar} [A^\dagger, H^\dagger] + \left(\pdv{A}{t}\right)^\dagger
\]
如果在某一时刻$A$是观察算符,则下一刻它仍然是观察算符的充要条件就是
\[
    \frac{1}{\ii \hbar} [A, H^\dagger] + \left(\pdv{A}{t}\right)^\dagger = \frac{1}{\ii \hbar} [A, H] + \pdv{A}{t}
\]
所以什么情况下确凿无疑的有$H$是厄米算符呢?

直和:一个参数本来只能取这些值,现在可以取另一些值了(加入了基矢量),则两个空间要做直和。

直积:本来只需要考虑这个参数,现在需要考虑别的参数了。

TODO:正则对易关系与运动方程。好像如果不使用正则对易关系,那么算符演化方程就和通过对应的拉氏量写出的运动方程不一致。

可以使用傅里叶变换把哈密顿量中的$\nabla \phi$之类的项弄掉。
然后得到的哈密顿量做对角化(大部分情况下已经对角化好了),就得到了一系列谐振子哈密顿量的叠加:
\[
    H = \int \dd x^3 a a^\dagger + \text{something}
\]

拉氏量的耦合对应着态空间的耦合?混合态、直积还有一系列神奇的东西。直和其实是增加了基矢量。
也就是说一个算符的各个不变子空间的直和构成全空间。
\[
    \delta(\vb*{r} - \vb*{r}_0) = \delta(x - x_0) \delta(y - y_0) \delta(z - z_0)
\]
所以三维态矢量其实是一维态矢量的直积。

把问题规范一下:现在我们已知系统的演化可以完全由一组算符$\hat{O}_1, \hat{O}_2, \ldots, \hat{O}_n$描述,也就是说能够写出哈密顿算符$\hat{H}$来描述它们的演化。此外,这些算符的对易关系全部给定,从而$[\hat{O}_i, \hat{H}]$也确定了。
现在的问题是,态矢量应该怎么取?或者说,对应的希尔伯特空间应该是怎样的结构?
实际上在量子场论中这似乎并不是一个问题,因为很少用到态矢量。这是因为只有算符是重要的,态矢量实际上只是算符对应的李代数的幺正表示而已。

设算符$\hat{O}_1, \hat{O}_2$分别是希尔伯特空间$H_1$、$H_2$的CSCO,且它们组成的集合是$H$的CSCO,那么$H = H_1 \otimes H_2$,并且$\ket{x_1, x_2} = \ket{x_1} \otimes \ket{x_2}$,其中$\ket{x_1} \in H_1, \ket{x_2} \in H_2$,$\ket{x_1, x_2} \in H$。
顺便抨击一下常见的量子力学教材:一上来就讲态矢量在概念上真的很不清楚!

我好像有点反应过来了。CSCO就是用来做这个的!
设$\hat{O}_1, \hat{O}_2, \ldots, \hat{O}_n$组成了一个空间$H$上的CSCO,且与$\hat{O}_1$对应的

作用在一个算符上的元算符如果不改变它所作用的那个算符的定义域,那么将这个元算符作用在另一个算符上就相当于将第三个算符和第二个算符复合。

一些问题:是不是任何一个幺正算符都对应着某个物理过程?(从初态到末态的映射)

粒子数表象。

\section{二次量子化}

\section{微扰散射理论}
% TODO:T矩阵,S矩阵,似乎前者是关于“跃迁率”的,即每秒事件发生的可能性,后者是关于无限长时间的散射

任何一个幺正过程都对应一个散射过程。

关于升降算符:设算符$\hat{x}$组成希尔伯特空间$\mathcal{H}$上的CSCO,其本征态为$\ket{x_1}, \ket{x_2}, \ldots$。
由于需要且只需要给定基矢量的像就能够确定一个算符,必定存在这样一个算符$\hat{a}$,它能够将$\ket{x_1}$映射为$\ket{x_2}$的某个非零倍数,将$\ket{x_2}$映射为$\ket{x_3}$的某个非零倍数,等等。这个算符称为升算符;升算符的逆就是降算符。显然,只需要一个本征态和升降算符就能够完全把态空间确定下来。另:如果本征值有上界,那么升算符作用在最大的本征值对应的本征态后得到$0$;同理,如果本征值有下界,那么降算符作用在最小的本征值对应的本征态之后得到$0$。
将$\ket{x}$提升到$\ket{x+c}$(可能差一个常数)的算符$\hat{a}$满足
\[
    [\hat{x}, \hat{a}] = c \hat{a}.
\]
特别的,若$\hat{x}$是厄米算符,且$\hat{a}$让本征态提升了$c$,那么$\hat{a}^\dagger$就会让本征态下降$c$,也就是说升降算符互为共轭转置。

现在的问题是怎样构造出升降算符。当然,任何情况下升降算符都应该满足对易关系$[\hat{x}, \hat{a}] = c \hat{a}$,但是这是不是足够了?
实际上还是能够构造出反例的,但是这些反例都是基于具体的分析构造,而物理上应该仅仅关心有关的代数结构。

能量、能级:能级实际上只对二次型哈密顿量比较好处理,此时总是可以把哈密顿量写成
\[
    \hat{H} = \sum_i E_i \hat{a}^\dagger_i \hat{a}_i,
\]
然后可以讨论某个能级上有几个粒子,等等。

有相互作用不方便处理能级。

\subsection{跃迁}

\subsubsection{费米黄金法则}

设系统的自由哈密顿量为$\hat{H}_0$,它的一组基矢量为$\{\ket{m}\}$,本征值记为$\{E_m\}$,相互作用哈密顿量为$\hat{H}'$。
假定相互作用哈密顿量不显含时间。设系统初态为$\ket{m}$,态随时间的演化为
\[
    \ket{\psi(t)} = \sum_n a_n(t) \ket{n} \ee^{- \ii E_n t / \hbar},
\]
显然$t=0$时除了$a_m=1$以外其它$a$均为零。使用Dyson级数并只计算到一阶,有
\[
    \ii \hbar a^{(1)}_k(t) = \sum_n \int \dd{t'} \mel{k}{\hat{H}'}{n} \ee^{\ii \omega_{kn} t'} a_n^{(0)},
\]
其中
\[
    \omega_{mn} = \frac{E_m - E_n}{\hbar}.
\]
$a_n^{(0)}$只在$n=m$时有非零值,且时间演化是从$t'=0$演化到$t'=t$,于是
\[
    \begin{aligned}
        \ii \hbar a^{(1)}_k(t) &= \sum_n \int \dd{t'} \mel{k}{\hat{H}'}{n} \ee^{\ii \omega_{kn} t'} a_n^{(0)} \\
        &= \mel{k}{\hat{H}'}{m} \frac{\sin \omega_{km} t / 2}{\omega_{km} / 2} \ee^{\ii \omega_{km} t / 2}. 
    \end{aligned}
\]
注意到$m \neq k$时$a_k = a_k^{(1)}$,而$a_k(t)$的模长平方正是$t'=0$时系统状态为$\ket{m}$而$t'=t$时系统经过观测状态为$\ket{k}$的概率,此概率就是所谓的\textbf{跃迁概率},于是跃迁概率的表达式就是
\begin{equation}
    P_k(t) = \frac{4 \abs*{\mel{k}{\hat{H}'}{m}}^2}{\hbar^2} \frac{\sin^2 \omega_{km} t / 2}{\omega_{km}^2}.
\end{equation}
现在假如系统实际上是一个开放系统,且诸$\ket{m}$构成一组偏好基,则可以使用一个经典马尔可夫过程来描述系统的演化,而使用经典的态(各个态出现的概率就是$\ket{m}$的振幅的模长平方)描述系统的状态。
系统每个时刻都有一定概率跃迁,也有一定概率不跃迁而等待下一时刻,此时跃迁速率为% TODO:从量子转向经典地一般描述,还有类似于“经典电磁场和量子粒子耦合”这样有经典有量子的系统
\begin{equation}
    \Gamma_k(t) = \dv{P_k}{t} = \frac{2 \abs*{\mel{k}{\hat{H}'}{m}^2}}{\hbar^2} \frac{\sin \omega_{km} t}{\omega_{km}}.
\end{equation}
如果我们要计算系统跃迁到一系列态上的概率,那么总跃迁速率为
\[
    \Gamma(t) = \sum_k \Gamma_k(t) = \sum_{E_k} \frac{2 \abs*{\mel{k}{\hat{H}'}{m}^2}}{\hbar^2} \frac{\sin \omega_{km} t}{\omega_{km}},
\]
让$\omega$连续取值,并引入态密度来表示哪些态是允许的,有
\begin{equation}
    \begin{aligned}
        \Gamma(t) &= \int \dd{E} \rho(E) \frac{2 \abs*{\mel{k}{\hat{H}'}{m}^2}}{\hbar^2} \frac{\sin \omega t}{\omega} \\
        &= \int \dd{\omega} \rho(E) \frac{2 \abs*{\mel{k}{\hat{H}'}{m}^2}}{\hbar} \frac{\sin \omega t}{\omega}.
    \end{aligned}
\end{equation}
其中$E = \hbar \omega + E_m$。如果能够保证能量守恒,并且可以持续很长时间都有跃迁,则$t$很大,于是$\sin \omega t / \omega$是高度振荡的函数,则可以把态密度$\rho(E)$提到积分号外面,而积分号变成跃迁矩阵元的平均乘以$\sin \omega t / \omega$的积分值,最后计算得到
\begin{equation}
    \Gamma(t) = \frac{2\pi}{\hbar} \rho(E) \expval{\abs*{\mel{k}{\hat{H}'}{m}^2}}_k.
\end{equation}
这样,如果我们有数目巨大的一系列完全相同的系统,总数为$N$,它们之间相互影响很小,那么在$\dd{t}$时间内,发生跃迁的系统的数目几乎确定为$N \Gamma(t) \dd{t}$。
因此可以列出某个状态的系统的数目服从的微分方程。

\subsubsection{周期性策动}

再考虑一个略有不同的情况——外界作用周期性地施加在
% TODO

\section{自由度解耦}

设有哈密顿量$\hat{H}$,如果某个物理量$\hat{A}$满足
\[
    \comm*{\hat{H}}{\hat{A}} = \comm*{\hat{H}'(A)}{\hat{A}},
\]
其中$\hat{H}'$是只和$\hat{A}$有关的物理量,那么在只关心$\hat{A}$时可以以$\hat{H}'$为等效哈密顿量。

其实也可以通过路径积分的观点看这个问题:与$\hat{A}$对易的那些自由度可以直接积掉而只留下一个因子。

% TODO:什么时候可以将算符用其期望替代,因为这会改变对易关系。。

\section{浸染近似和几何相位}

% TODO:希尔伯特空间分裂成几个分支
设总的希尔伯特空间为$H$,由于守恒量、拓扑性质等,满足某个特定约束条件的态组成的空间为$H'$,则$H / H'$标记了不同的守恒量、拓扑性质等的参数。
例如对称性自发破缺意味着有序参量出现。
% 衡量哈密顿量的大小需要通过其造成的能级分裂的大小,而不是绝对大小,因为绝对大小可以通过加上或者减去一个任意的常数来轻易改变。
有些时候,一个理论中的两个不同的态可以看成是等价的态,这当然是因为系统中有冗余的、非物理的自由度。
这种情况下通常不说这是“对称性”,而是说这是“冗余的自由度”。这种对称性只是拉氏量的对称;态空间中的这种“对称”已经被模掉了。
% TODO:拉氏量的对称性和态空间的对称性

\end{document}

\bibliographystyle{plain}
\bibliography{optics} 

\end{document}