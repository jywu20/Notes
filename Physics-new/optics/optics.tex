\documentclass[hyperref, UTF8, a4paper, oneside]{ctexbook}

\usepackage{geometry}
\usepackage{titling}
\usepackage{titlesec}
\usepackage{paralist}
\usepackage{footnote}
\usepackage{enumerate}
\usepackage{amsmath, amssymb, amsthm}
\usepackage{mathtools}
\usepackage{cite}
\usepackage{booktabs}
\usepackage{multirow}
\usepackage{graphicx}
\usepackage{subfigure}
\usepackage{physics}
\usepackage{siunitx}
\usepackage[compat=1.1.0]{tikz-feynhand}
\usepackage{xr-hyper}
\usepackage[colorlinks, linkcolor=black, anchorcolor=black, citecolor=black, filecolor=black]{hyperref}
\usepackage[most]{tcolorbox}
\usepackage{caption}
\usepackage{prettyref}

\externaldocument[qft-]{../relativistic-qft/relativistic-qft}[relativistic-qft.pdf]
\externaldocument[solid-]{../solid/solid}[solid.pdf]

\geometry{left=3.28cm,right=3.28cm,top=2.54cm,bottom=2.54cm}
\titlespacing{\paragraph}{0pt}{1pt}{10pt}[20pt]
\setlength{\droptitle}{-5em}
\preauthor{\vspace{-10pt}\begin{center}}
\postauthor{\par\end{center}}

\newcommand*{\ee}{\mathrm{e}}
\newcommand*{\ii}{\mathrm{i}}
\newcommand*{\const}{\mathrm{const}}
\newcommand*{\natnums}{\mathbb{N}}
\newcommand*{\reals}{\mathbb{R}}
\newcommand*{\complexes}{\mathbb{C}}
\DeclareMathOperator{\timeorder}{T}
\newcommand*{\ogroup}[1]{\mathrm{O}(#1)}
\newcommand*{\sogroup}[1]{\mathrm{SO}(#1)}
\DeclareMathOperator{\legpoly}{P}

\newrefformat{sec}{第\ref{#1}节}
\newrefformat{note}{注\ref{#1}}
\newrefformat{fig}{图\ref{#1}}
\newrefformat{part}{第\ref{#1}部分}
\newrefformat{chap}{第\ref{#1}章}
\newrefformat{tbl}{表\ref{#1}}
\newrefformat{back}{背景知识\ref{#1}}
\newrefformat{info}{资料框\ref{#1}}
\newrefformat{warn}{注意事项\ref{#1}}
\renewcommand{\autoref}{\prettyref}

\newcommand{\concept}[1]{\underline{#1}}
\renewcommand{\emph}{\textbf}

% Support for tensor double arrows.
\newcommand{\tensor}[1]{ \stackrel{\leftrightarrow}{\vb*{#1}}}

\usetikzlibrary{arrows,shapes,positioning}
\usetikzlibrary{arrows.meta}
\usetikzlibrary{decorations.markings}
\tikzstyle arrowstyle=[scale=1]
\tikzstyle directed=[postaction={decorate,decoration={markings,
    mark=at position .5 with {\arrow[arrowstyle]{stealth}}}}]
\tikzstyle ray=[directed, thick]

\tikzfeynhandset{
    every boldphoton@@/.style={
    /tikz/draw=none,
    /tikz/postaction={
            /tikz/draw,
            /tikz/double,
            /tikz/line width = \feynhandlinesize,
            /tikz/decoration={
                complete sines,
                amplitude=3\feynhandlinesize,
                segment length=7.5\feynhandlinesize,
            },
            /tikz/decorate=true,
        },
    },
    every boldphoton/.style={/tikzfeynhand/every boldphoton@@/.append style={#1}},
    boldphoton/.style={
    /tikzfeynhand/every boldphoton@@,
    },
}

\tikzfeynhandset{
    every extphoton@@/.style={
        /tikz/draw=none,
        /tikz/decoration={name=none},
        /tikz/postaction={
          /tikz/draw,
          /tikz/line width = \feynhandlinesize,
          /tikzfeynhand/with arrow=0.9,
        }
    },
    every extphoton/.style={/tikzfeynhand/every extphoton@@/.append style={#1}},
    extphoton/.style={
    /tikzfeynhand/every extphoton@@,
    }
}

\tikzfeynhandset{
    every outphoton@@/.style={
    /tikz/draw=none,
    /tikzfeynhand/with arrow=0.9,
    /tikz/postaction={
            /tikz/draw,
            /tikz/line width = \feynhandlinesize,
            /tikz/decoration={
                complete sines,
                amplitude=3\feynhandlinesize,
                segment length=7.5\feynhandlinesize,
            },
            /tikz/decorate=true,
        },
    },
    every outphoton/.style={/tikzfeynhand/every outphoton@@/.append style={#1}},
    outphoton/.style={
    /tikzfeynhand/every outphoton@@,
    },
}

\tcbuselibrary{skins, breakable, theorems}

\newtcbtheorem[number within=chapter]{back}{背景知识}%
  {colback=blue!5,colframe=blue!65,fonttitle=\bfseries, breakable}{back}
\newtcbtheorem[number within=chapter]{info}{资料框}%
  {colback=blue!5,colframe=blue!65,fonttitle=\bfseries, breakable}{info}
\newtcbtheorem[number within=chapter]{warning}{注意事项}%
  {colback=orange!5,colframe=orange!65,fonttitle=\bfseries, breakable}{warn}

\newenvironment{bigcase}{\left\{\quad\begin{aligned}}{\end{aligned}\right.}

\numberwithin{equation}{chapter}

\newcommand{\qftdoc}{\href{../relativistic-qft/relativistic-qft.pdf}{相对论性量子场论笔记}}
\newcommand{\soliddoc}{\href{../solid/solid.pdf}{固体物理笔记}}

\title{电磁场,电磁波,光学}
\author{吴晋渊}

\begin{document}

\maketitle

\part{介质的电动力学性能}

\chapter{QED的低能近似}

关于QED的低能极限下物质是如何和电磁场耦合的,以及物质之间的等效库伦相互作用,\qftdoc中的第\ref{qft-sec:electron-photon-interaction}节和第\ref{qft-sec:qed-tree-level-channels-low-speed}节已有系统的介绍。
这里我们稍微深入讨论一下凝聚态介质中常常用到的一些情况。

\section{电磁场本身}

\subsection{关于电场和磁场的麦克斯韦方程}

\qftdoc中\eqref{qft-eq:four-maxwell}和\eqref{qft-eq:bianchi-identity}重述如下:
\begin{equation}
    \partial_\mu F^{\mu \nu} = J^\nu,
    \label{eq:four-maxwell}
\end{equation}
\begin{equation}
    \partial_\mu F_{\nu \rho} + \partial_\nu F_{\rho \mu} + \partial_\rho F_{\mu \nu} = 0,
    \label{eq:bianchi-identity}
\end{equation}

我们可以直接选择将电磁张量的分量写下来。我们首先观察电磁张量在\emph{空间}坐标变换(不涉及时间)下的变换。
首先电磁张量的对角元都是零,非对角元一共有$6$个独立变量,我们不妨设它们为
\[
    F^{\mu \nu} = \pmqty{
        0 & -E_x & -E_y & -E_z \\
        E_x & 0 & -B_z & B_y \\
        E_y & B_z & 0 & -B_x \\
        E_z & -B_y & B_x & 0
    }.
\]
一个不涉及时间的坐标变换一定形如
\[
    \pmqty{\dmat{1, \xmat*{a}{3}{3} }},
\]
我们将它作用在$F^{\mu \nu}$上(注意二阶张量的两个指标都要作用一遍),会发现$(E_x, E_y, E_z)$在坐标变换矩阵$\{a_{ij}\}$下发生了坐标变换,而
\[
    \pmqty{
        0 & -B_z & B_y \\
        B_z & 0 & -B_x \\
        -B_y & B_x & 0
    }
\]
在坐标变换矩阵$\{a_{ij}\}$下作为张量发生了坐标变换。
这就意味着$(E_x, E_y, E_z)$构成一个三维矢量的分量,而含有$B_i$的那部分矩阵构成一个三维二阶张量的分量。我们称前者为\concept{电场},它是一个极矢量。
至于后者,它是某个三元组和一个三指标反对称张量缩并而成的,反对称张量在连续的坐标变换下确实按照张量的方式变换,但是在反射下会变号,因此三元组$(B_x, B_y, B_z)$构成一个轴矢量的分量,称为\textbf{磁场}。

电场和磁场满足什么样的动力学方程?首先我们考虑\eqref{eq:four-maxwell},它给出两个方程(注意每一项的正负号,特别是梯度算符;$\partial_i$和$\partial^i$都是\emph{欧氏空间下的},简单地表示对$x_i$(也就是$x^i$)求偏导):
\[
    \div{\vb*{E}} = J^0, \quad - \partial_t E^i - \partial_j \epsilon_{jik} B^k = J^i.
\]
设$J^\mu$对应的守恒荷密度为$\rho$,对应的输运流为$\vb*{j}$,即
\begin{equation}
    J^\mu = (\rho, \vb*{j}), \quad \partial_\mu J^\mu = \pdv{\rho}{t} + \div{\vb*{j}} = 0,
\end{equation}
则这两个方程就是
\begin{equation}
    \div{\vb*{E}} = \rho, \quad \curl{\vb*{B}} = \pdv{\vb*{E}}{t} + \vb*{j}.
    \label{eq:maxwell-first-pair}
\end{equation}

光靠\eqref{eq:maxwell-first-pair}显然不能定解,其原因在于$\vb*{E}$和$\vb*{B}$依照定义是$F^{\mu \nu}$的独立分量,但$F^{\mu \nu}$并不是一个任意的反对称张量,它是矢量场$A^\mu$的梯度反对称化之后的产物。
例如,$F^{\mu \nu}$还需要满足\eqref{eq:bianchi-identity}。
分别考虑\eqref{eq:bianchi-identity}中$\mu, \nu, \rho$完全取空间维度的情况以及三个指标有一个取时间维度另外两个取空间维度的情况,得到
\begin{equation}
    \div{\vb*{B}} = 0, \quad \pdv{\vb*{B}}{t} + \curl{\vb*{E}} = 0.
    \label{eq:maxwell-second-pair}
\end{equation}
\eqref{eq:maxwell-first-pair}和\eqref{eq:maxwell-second-pair}放在一起就给出了著名的\concept{麦克斯韦方程组}:有电场、有磁场,有\concept{电荷密度}$\rho$,有\concept{电流密度}$\vb*{j}$。
整个麦克斯韦方程写出来就是
\begin{equation}
    \left\{
        \begin{aligned}
            \div{\vb*{E}} &= \rho, \\
            \curl{\vb*{E}} &= - \pdv{\vb*{B}}{t}, \\
            \div{\vb*{B}} &= 0, \\
            \curl{\vb*{B}} &= \pdv{\vb*{E}}{t} + \vb*{j}.
        \end{aligned}
    \right.
    \label{eq:maxwell-eq}
\end{equation}
可以看到,\eqref{eq:maxwell-eq}在时间上是一阶的,且电场和磁场的一阶导数都已经确定了,则给定适当的初始条件和边界条件应当能够定解。

在推导\eqref{eq:maxwell-eq}时我们还用到了一个条件,就是$J^\mu$是一个四维守恒流。但实际上这个条件也可以从\eqref{eq:maxwell-eq}中推导出来,只需要分别对第一式做对时间的偏导数,对第四式做散度即可得到连续性条件
\begin{equation}
    \pdv{\rho}{t} + \div{\vb*{j}} = 0.
    \label{eq:charge-transportation}
\end{equation}
这当然完全是预期之中的,因为正如我们在\qftdoc中的第\ref{qft-sec:four-eqs}节中看到的那样,电荷守恒可以从$A^\mu$的运动方程推导出来,而既然麦克斯韦方程完全描述了电磁场,自然也可以推导出电荷连续性方程。%
\footnote{一个可能的疑难是,\qftdoc中的\eqref{qft-eq:four-current}要求$(\rho, \vb*{j})$构成四维矢量的分量,而以上推导并未证明这一点。
但注意到$(\partial_t, \grad)$是四维梯度算符,而$0$是标量,那么坐标变化时必须保证$(\rho, \vb*{j})$是四维矢量才能够让连续性方程恒成立,因此$(\rho, \vb*{j})$确确实实构成四维矢量的分量,这是张量的商法则的一个例子。
}%

麦克斯韦方程本身没有给出电荷的动力学,但是它确实给出了一个守恒荷:如果电磁场中出现了让$\div{\vb*{E}}$非零的模式,那么这个模式一定和电荷一起出现;或者说,拉氏量或者哈密顿量中电荷-电磁场耦合项是将电荷和某种特殊的电场模式——让$\div{\vb*{E}}$非零的模式——粘在了一起。
在我们不关心携带电荷的物质场的细节时,所谓分析电荷实际上是在分析这种电场模式如何和其它模式耦合在一起。%
\footnote{
    这种将电荷视为规范场的特殊模式的思维方式在凝聚态物理中有应用。二维规范场会给其中的电荷赋予交换相位,从而产生任意子,我们要分析电荷的一般性质就需要引入一动不动的“测试电荷”,实际上就是在讨论规范场的局域的特殊构型的性质。
    \soliddoc的第\ref{solid-sec:gauge-charge-flux-z2}节是一个例子。
}%
然后我们可以利用这种分析结果,把携带电荷的物质场的动力学加回去:
\begin{enumerate}
    \item 给定边界条件和初始条件;
    \item 暂时将$\rho$和$\vb*{j}$看成已经给定的量;
    \item 依照以下三种方案中的其中一种前进,这三种方案都是等价的,提供了同样多的信息,在一个方案中增加别的方案的方程不会提供新的信息:
    \begin{itemize}
        \item 将$(\rho, \vb*{j})$代入麦克斯韦方程\eqref{eq:maxwell-eq}求解,解出$\vb*{E}, \vb*{B}$;
        \item 选定一个规范\footnote{可以看到,规范实际上就是让我们能够从电场、磁场唯一反推出电势和磁矢势的约束条件。},求解出$(\rho, \vb*{A})$,如在洛伦兹规范下求解\eqref{eq:wave-eq},然后根据\eqref{eq:e-b-from-a-phi}计算出$\vb*{E}, \vb*{B}$;
        \item 直接求解协变的\eqref{eq:four-maxwell},或是使用最小作用量原理等,然后根据\eqref{eq:e-b-from-a-phi}计算出$\vb*{E}, \vb*{B}$;
    \end{itemize}
    \item 如果必要的话,根据\qftdoc中的第\ref{qft-sec:gauge-def}节中提到的办法计算$A^\mu$,即$(\varphi, \vb*{A})$;
    \item 如果实际上$(\rho, \vb*{j})$没有给定,则将用$\rho, \vb*{j}$表示的$\vb*{E}, \vb*{B}$和物质的运动方程(见\autoref{sec:em-field-coupling-particle})联立求解。
\end{enumerate}

求解\eqref{eq:maxwell-eq}得到的只是$\vb*{E}$和$\vb*{B}$,或者说电磁张量,但正如\qftdoc中的第\ref{qft-sec:gauge-def}节中所说的那样,知道了电磁张量,就可以确定所有规范不变量,那么求解出$\vb*{E}$和$\vb*{B}$也就够了。
需要注意的是这\emph{不代表}电场和磁场就是本质上更基本的自由度,例如在一些情况下(如A-B效应等)直接使用$\vb*{A}$构造规范不变量(可以是一些非局域的量)更加方便。

总之,\eqref{eq:maxwell-eq}是一切经典电动力学现象关于电磁场的部分背后的机制,补充上电磁场以外的物质的定律就构成封闭方程组。
;实际上,在\qftdoc中我们没有解释为什么那里的\eqref{qft-eq:qed-lagrangian}能够被称为电动力学,但是既然它能够推导出\eqref{eq:maxwell-eq},说明这个拉氏量描写的确实是电动力学。
基于电场和磁场的表述和基于电磁张量的表述之间的转换关系就是
\begin{equation}
    F^{\mu \nu} = \pmqty{
        0 & -E_x & -E_y & -E_z \\
        E_x & 0 & -B_z & B_y \\
        E_y & B_z & 0 & -B_x \\
        E_z & -B_y & B_x & 0
    }, \quad F_{\mu \nu} = \pmqty{
        0 & E_x & E_y & E_z \\
        -E_x & 0 & -B_z & B_y \\
        -E_y & B_z & 0 & -B_x \\
        -E_z & -B_y & B_x & 0
    }.
\end{equation}

自然单位制下的关于电场和磁场的麦克斯韦方程已经导出。
在\concept{国际单位制}下,我们有
\begin{equation}
    \left\{
        \begin{aligned}
            \div{\vb*{E}} &= \frac{\rho}{\epsilon_0}, \\
            \curl{\vb*{E}} &= - \pdv{\vb*{B}}{t}, \\
            \div{\vb*{B}} &= 0, \\
            \curl{\vb*{B}} &= \mu_0 \epsilon_0 \pdv{\vb*{E}}{t} + \mu_0 \vb*{j}.
        \end{aligned}
    \right.
    \label{eq:maxwell-eq-si}
\end{equation}
以下如无特殊说明,均使用国际单位制。

从以上方程可以推导出场满足的波动方程
\begin{equation}
    \frac{1}{c^2} \pdv[2]{\vb*{E}}{t} - \laplacian{\vb*{E}} = - \frac{1}{\epsilon_0} \grad{\rho} - \mu_0 \pdv{\vb*{j}}{t} , \quad \frac{1}{c^2} \pdv[2]{\vb*{B}}{t} - \laplacian{\vb*{B}} = \mu_0 \curl{\vb*{j}},
    \label{eq:wave-eq-general}
\end{equation}
其中
\begin{equation}
    \frac{1}{c^2} = \epsilon_0 \mu_0
\end{equation}
为\concept{光速},实际上就是电磁波传播的速度。

看起来,麦克斯韦方程并不难求解。可以将它化成外加载荷为电流和电荷密度的二阶线性波动方程,我们马上可以在不同的边界条件下求解其本征函数,写出其格林函数(从而得到“通解”),等等。
但是事实证明,这样并不能提供太多有用的信息。例如,在介质存在的情况下,从头求解真空中的麦克斯韦方程\eqref{eq:maxwell-eq-si}基本上是不现实的,讨论材料引入的边界条件、响应等是必要的。
在讨论静电学和静磁学问题时,我们并不关心一个一般的体系如何收敛到静止情况,从而也无需求解完整的时间相关的麦克斯韦方程。
在讨论电磁场和电流的相互作用时,仅仅根据格林函数写出“电荷如何影响电磁场”是不够的,因为还需要考虑电荷受到的反作用;并且\eqref{eq:wave-eq-general}的格林函数实际上非常复杂,基本上没法直接使用。
对每一种情形——静电学,静磁学,电磁波传播,辐射——我们都需要采取不同的方法(可能还有适当的近似)求解麦克斯韦方程,而不能指望可以从一个通解出发从头计算出一切。

\subsection{亥姆霍兹分解与常见的电磁场形态}

在电动力学中我们基本上只需要使用散度和旋度。关于这件事有著名的\concept{亥姆霍兹分解}:任意一个矢量场$\vb*{X}$只要在无穷远处衰减得足够快(至少比$1 / r$快),则可以做如下分解:
\begin{equation}
    \vb*{X} = - \grad{U} + \curl{\vb*{W}},
    \label{eq:ht-decomp}
\end{equation}
其中$U$和$\vb*{X}$可以分别表示为
\begin{equation}
    U = \frac{1}{4\pi} \int \dd[3]{\vb*{r}'} \frac{\grad' \cdot \vb*{X}(\vb*{r}')}{\abs{\vb*{r} - \vb*{r}'}} - \frac{1}{4\pi} \oint_S \dd{S} \vu*{n} \cdot \frac{\vb*{X}(\vb*{r}')}{\abs{\vb*{r} - \vb*{r}'}},
    \label{eq:ht-decomp-u-def}
\end{equation}
以及
\begin{equation}
    \vb*{W} = \frac{1}{4\pi} \int \dd[3]{\vb*{r}'} \frac{\grad' \times \vb*{X}(\vb*{r}')}{\abs{\vb*{r} - \vb*{r}'}} - \frac{1}{4\pi} \oint_S \dd{S} \vu*{n} \times \frac{\vb*{X}(\vb*{r}')}{\abs{\vb*{r} - \vb*{r}'}}.
    \label{eq:ht-decomp-w-def}
\end{equation}
证明是相对简单的,因为\eqref{eq:ht-decomp-u-def}和\eqref{eq:ht-decomp-w-def}是完全构造性的,我们只需要验证它们的确满足\eqref{eq:ht-decomp}即可,这就证明了分解\eqref{eq:ht-decomp}总是可行的。
应注意场的衰减条件还是重要的,因为在一些情况中(如静电学问题中)我们的确会在无穷远处放置一些源(比如说一块其上有感应电荷的金属板),那么场可能衰减得没有那么快。

对电场和磁场作用亥姆霍兹分解,能够得到
\begin{equation}
    \begin{aligned}
        \vb*{E} &= - 
    \end{aligned}
\end{equation}
从这个分解中能够看到常见的几种电磁波模式。如果磁场不重要,那么我们就得到静电学,其中
静磁学
将$c \to \infty$,得到准静态近似
最后是电磁波

直观地看,准静态近似实际上是在描写一个天线:

亥姆霍兹分解本身无助于求解麦克斯韦方程,它更多用于在已知结果后诠释它。

\subsection{能量和动量}

电磁场中的粒子的运动方程为
\begin{equation}
    \dv{\vb*{p}}{t} = q \vb*{E} + q \vb*{v} \times \vb*{B},
\end{equation}

一个区域内部所有粒子的能量由于电磁场而发生的变化为
\begin{equation}
    \dv{E}{t} = \sum q \vb*{v} \cdot \vb*{E} = \int \dd[3]{\vb*{r}} \vb*{j} \cdot \vb*{E},
\end{equation}
而
\[
    \begin{aligned}
        \vb*{j} \cdot \vb*{E} &= \left( \frac{1}{\mu_0} \curl{\vb*{B}} - \epsilon_0 \pdv{\vb*{E}}{t} \right) \cdot \vb*{E} \\
        &= \frac{1}{\mu_0} \div{(\vb*{B} \times \vb*{E})} - \frac{\epsilon_0}{2} \pdv{\vb*{E}^2}{t} + \frac{1}{\mu_0} (\curl{\vb*{E}}) \cdot \vb*{B} \\
        &= \frac{1}{\mu_0} \div{(\vb*{B} \times \vb*{E})} - \frac{\epsilon_0}{2} \pdv{\vb*{E}^2}{t} - \frac{1}{\mu_0} \pdv{\vb*{B}}{t} \cdot \vb*{B} \\
        &= \frac{1}{\mu_0} \div{(\vb*{B} \times \vb*{E})} - \pdv{t} \left( \frac{\epsilon_0}{2} \vb*{E}^2 + \frac{1}{2 \mu_0} \vb*{B}^2 \right),
    \end{aligned}
\]
因此
\[
    \dv{E}{t} = - \int \dd[3]{\vb*{r}} \left( \frac{\epsilon_0}{2} \vb*{E}^2 + \frac{1}{2 \mu_0} \vb*{B}^2 \right) - \frac{1}{\mu_0} \int \dd{\vb*{S}} \cdot (\vb*{E} \times \vb*{B}).
\]
设电磁场能量密度为$u$,能流密度为$\vb*{S}$,则
\[
    \dv{E}{t} + \int \dd[3]{\vb*{r}} u = - \int \dd{\vb*{S}} \cdot \vb*{S},
\]
于是
\begin{equation}
    \int \dd[3]{\vb*{r}} u + \int \dd{\vb*{S}} \cdot \vb*{S} = \int \dd[3]{\vb*{r}} \left( \frac{1}{2} \epsilon_0 \vb*{E}^2 + \frac{1}{2\mu_0} \vb*{B}^2 \right) + \int \dd{\vb*{S}} \cdot \frac{1}{\mu_0} \vb*{E} \times \vb*{B},
    \label{eq:energy-flow-original}
\end{equation}
从而可以取
\begin{equation}
    u = \frac{1}{2} \epsilon_0 \vb*{E}^2 + \frac{1}{2\mu_0} \vb*{B}^2, \quad \vb*{S} = \frac{1}{\mu_0} \vb*{E} \times \vb*{B}.
    \label{eq:energy-flow}
\end{equation}
当然,实际上从\eqref{eq:energy-flow-original}不能唯一地确定能量密度和能流,因为在变换
\[
    \rho' = \rho + \div{\vb*{c}}, \quad \vb*{j}' = \vb*{j} - \pdv{\vb*{c}}{t}
\]
下输运方程保持成立。这也是可以预期的,因为可以看出\eqref{eq:energy-flow-original}是麦克斯韦方程能够给出的关于能量的全部结论,而通常从运动学方程出发并不能唯一地定义能量。
使用\eqref{eq:energy-flow}定义的$\vb*{S}$称为\concept{坡印廷矢量}。

使用类似的方法,设$\vb*{P}$为一个区域内的粒子总动量(不包括电磁场的动量),则
\[
    \dv{\vb*{P}}{t} = \sum (q \vb*{E} + q \vb*{v} \times \vb*{B}) = \int \dd[3]{\vb*{r}} \left( \rho \vb*{E} + \vb*{j} \times \vb*{B} \right),
\]
代入$\rho$和$\vb*{j}$,得到
\[
    \begin{aligned}
        \dv{\vb*{P}}{t} &= \int \dd[3]{\vb*{r}} \left(\epsilon_0 (\div{\vb*{E}}) \vb*{E} + \left( \frac{1}{\mu_0} \curl{\vb*{B}} - \epsilon_0 \pdv{\vb*{E}}{t} \right) \times \vb*{B} \right) \\
        &= \int \dd[3]{\vb*{r}} \epsilon_0 (\div{\vb*{E}}) \vb*{E} + \int \dd[3]{\vb*{r}} \frac{1}{\mu_0} (\curl{\vb*{B}}) \times \vb*{B} - \epsilon_0 \int \dd[3]{\vb*{r}} \pdv{t} (\vb*{E} \times \vb*{B}) + \epsilon_0 \int \dd[3]{\vb*{r}} \vb*{E} \times \pdv{\vb*{B}}{t} \\
        &= \epsilon_0 \int \dd[3]{\vb*{r}} ((\div{\vb*{E}}) \vb*{E} + (\curl{\vb*{E}}) \times \vb*{E}) + \frac{1}{\mu_0} \int \dd[3]{\vb*{r}} (\curl{\vb*{B}}) \times \vb*{B} - \epsilon_0 \int \dd[3]{\vb*{r}} \pdv{t} (\vb*{E} \times \vb*{B}),
    \end{aligned}
\]
而注意到
\[
    \begin{aligned}
        (\div{\vb*{E}}) \vb*{E} + (\curl{\vb*{E}}) \times \vb*{E}) &= \div{(\vb*{E} \vb*{E})} - (\vb*{E} \cdot \grad) \vb*{E} - \vb*{E} \times (\curl{\vb*{E}}) \\
        &= \div{(\vb*{E} \vb*{E})} - \frac{1}{2} \grad{\vb*{E}^2} \\
        &= \div{(\vb*{E} \vb*{E})} - \frac{1}{2} \div{(\vb*{E}^2 \vb*{I})},
    \end{aligned}
\]
且类似的可以得到
\[
    \begin{aligned}
        \underbrace{(\div{\vb*{B}}) \vb*{B}}_{=0} + (\curl{\vb*{B}}) \times \vb*{B}) &= \div{(\vb*{B} \vb*{B})} - (\vb*{B} \cdot \grad) \vb*{B} - \vb*{B} \times (\curl{\vb*{B}}) \\
        &= \div{(\vb*{B} \vb*{B})} - \frac{1}{2} \grad{\vb*{B}^2} \\
        &= \div{(\vb*{B} \vb*{B})} - \frac{1}{2} \div{(\vb*{B}^2 \vb*{I})},
    \end{aligned}
\]
于是就有
\[
    \begin{aligned}
        \dv{\vb*{P}}{t} &= - \epsilon_0 \int \dd[3]{\vb*{r}} \pdv{t} (\vb*{E} \times \vb*{B}) - \int \dd{\vb*{S}} \cdot \left( \frac{1}{2} \left( \epsilon_0 \vb*{E}^2 + \frac{1}{\mu_0} \vb*{B}^2 \right) \vb*{I} - \epsilon_0 \vb*{E} \vb*{E} - \frac{1}{\mu_0} \vb*{B} \vb*{B} \right) \\
        &= - \int \dd[3]{\vb*{r}} \vb*{g} - \int \dd{S_i} T_{ij},
    \end{aligned}
\]
其中$\vb*{g}$是动量密度而$T_{ij}$是动量流密度(一个二阶张量)。同样,只是知道上式不能够唯一确定动量密度和动量流密度,但是以下的选择是最简单的:
\begin{equation}
    T_{ij} = u \delta_{ij} - \epsilon_0 E_i E_j - \frac{1}{\mu_0} B_i B_j, \quad \vb*{g} = \frac{1}{c^2} \vb*{S}.
\end{equation}
张量$T_{ij}$称为\concept{麦克斯韦张量},它可以看成电磁场提供的应力。
电磁场对实物粒子的动量转移速率,也即力,就是
\begin{equation}
    \vb*{F} = - \int \dd[3]{\vb*{r}} \vb*{g} - \int \dd{S_i} T_{ij}.
\end{equation}

\subsection{电磁场的变换性质}\label{sec:e-m-field-transformation-feature-vacuum}

\eqref{eq:maxwell-eq}中出现的各个量——$\vb*{E}, \rho, \vb*{B}, \vb*{j}$——在不同的变换下的性质由\autoref{tbl:sym-eb}给出。
实际上据此我们会发现麦克斯韦方程组\eqref{eq:maxwell-eq}可以通过这些性质启发式地“推导”出来。

\begin{table}
    \centering
    \caption{各个物理量在不同的变换下的性质}
    \label{tbl:sym-eb}
    \begin{tabular}[c]{ccc}
        \toprule
        电磁场中的物理量 & 张量分析 & 时间反演 \\
        \midrule
        $\rho$ & 标量 & 偶 \\
        $\vb*{E}$ & 矢量 & 偶 \\
        $\vb*{j}$ & 矢量 & 奇 \\
        $\vb*{B}$ & 赝矢量 & 奇 \\
        \bottomrule
    \end{tabular}
\end{table}

首先,$\div{\vb*{E}}$是一个普通标量,它只能够是$\rho, \abs{\vb*{E}}^2, \abs{\vb*{j}}^2, \abs{\vb*{B}}^2$的标量函数。
如果我们要求外源对电磁场的激励是线性的,只能有
\[
    \div{\vb*{E}} = a_1 \rho.
\]
$\pdv*{\vb*{E}}{t}$是一个普通矢量。\autoref{tbl:sym-eb}中最低阶的普通矢量是$\vb*{j}, \vb*{E}, \curl{\vb*{B}}$。
我们不希望电场自行衰减,所以只能
\[
    \pdv{\vb*{E}}{t} = a_2 \vb*{j} + a_3 \curl{\vb*{B}}.
\]
$\pdv*{\vb*{B}}{t}$是一个赝矢量。\autoref{tbl:sym-eb}中最低阶的赝矢量就是$\curl{\vb*{E}}$一个,于是
\[
    \pdv{\vb*{B}}{t} = a_4 \curl{\vb*{E}}.
\]
$\div{\vb*{B}}$是一个赝标量,因为它是普通矢量$\grad$和赝矢量$\vb*{B}$的点乘,然而\autoref{tbl:sym-eb}中的物理量没有一个赝标量;要产生赝标量,只能够通过诸如$\vb*{B}$和$\vb*{E}$或是$\vb*{j}$点乘,均会引入非线性效应。
所以我们只能让
\[
    \div{\vb*{B}} = 0.
\]
因此我们就形式上得到了麦克斯韦方程;四个参数都可以通过适当的方法吸收进空间、时间、电场、磁场的单位中。

\section{非相对论性粒子和光场的耦合}\label{sec:em-field-coupling-particle}

\subsection{非相对论性粒子的哈密顿量}\label{sec:particle-hamiltonian}

考虑与电磁场发生相互作用的粒子,我们通常将这些粒子称为物质而将电磁场称为光场或是辐射,虽然严格说起来辐射也算是一种物质。
我们假定粒子做低速运动,从而不需要使用相对论性的理论描述粒子。
粒子轨道部分的哈密顿量是以下保证局部$U(1)$规范对称性的极小耦合:
\begin{equation}
    {H}_\text{orbit} = \frac{1}{2m} ({\vb*{p}} - q \vb*{A})^2 + q \phi,
    \label{eq:minimal-coupling}
\end{equation}
自旋-磁场相互作用还会引入以下哈密顿量:
\begin{equation}
    {H}_\text{spin} = - \frac{q}{m} {\vb*{S}} \cdot \vb*{B} = - \vb*{\mu} \cdot \vb*{B},
\end{equation}
而场的哈密顿量是
\begin{equation}
    {H}_\text{field} = \frac{\epsilon_0}{2} \int \dd[3]{\vb*{r}} (\vb*{E}^2 + c^2 \vb*{B}^2),
\end{equation}
则体系的总哈密顿量
\begin{equation}
    {H} = \sum_i \left( \frac{1}{2m_i} ({\vb*{p}_i} - q_i \vb*{A})^2 + q_i \varphi - \frac{q_i}{m_i} \vb*{S}_i \cdot \vb*{B} \right) + {H}_\text{field} + {H}_\text{int} + {H}_\text{ext},
    \label{eq:whole-system-hamiltonian}
\end{equation}
其中${H}_\text{int}$和${H}_\text{ext}$分别表示粒子间相互作用和外加势场。
粒子部分——包括轨道和自旋——的拉氏量也可以写成
\begin{equation}
    L = \sum_i \left( \frac{1}{2} m_i \vb*{v}_i^2 - q_i \varphi + q_i \vb*{v}_i \cdot \vb*{A} + \vb*{\mu}_i \cdot \vb*{B} \right).
\end{equation}
具体什么是粒子-粒子相互作用其实有一定人为因素,比如说凝聚态场论中默认电子之间的相互作用是库伦相互作用,但是库伦相互作用其实也是交换光子导致的,实际上是近场辐射的一个无时间延迟近似。
同样,“外加势场”也有人为因素。
不过,由于本文将要讨论光学,实际上可以以一种比较前后一致的方式确定哪些电磁场模式被粒子-粒子等效相互作用替代,哪些被纳入考虑。
我们总是可以将电磁波模式分解成无源有旋的和无旋有源的。通过简单的QED计算可以发现,全体电磁波模式造成的粒子间散射几乎压倒性地来自一个电子发射、一个电子接受的纵波$\varphi$模式,切换到电场中基本上就是库伦场,这是有源无旋的;另一方面,介质中的电磁波宏观地看都满足横波条件$\div{\vb*{E}}=0$。%
\footnote{
    我们称它为横波条件是因为在无穷大空间中这等价于$\vb*{k} \cdot \vb*{E} = 0$,但是这\emph{并不}意味着任何能够称为波矢的$\vb*{k}$都满足$\vb*{k} \cdot \vb*{E} = 0$。
    波导就是一个典型的例子。
}%
因此我们可以只在$\varphi$和$\vb*{A}$中保留满足横波条件、看起来就像真空中电磁波的电磁波模式,这些模式本身就不易被积掉;剩下的不满足横波条件,同时的确很容易积掉的模式,或者说“虚光子”(如库伦场)就归入粒子-粒子等效相互作用。
至于外加势场,它或者就是库伦场,或者是外加磁场,后者同样是一个容易被积掉,并且和电磁波非常不相似的模式。
这也解释了为什么在\eqref{eq:whole-system-hamiltonian}中我们将电子间的库伦排斥能和辐射场引入的能量简单相加,因为这的确是两种不同的过程:库伦排斥涉及的光子实际上是不满足横场条件的虚光子,而辐射场中的光子都是可以出现在实际的物理态中的光子。

\eqref{eq:minimal-coupling}中的$\vb*{p}$是正则动量,而不是机械动量。
然而,这反倒有好处:我们要讨论的是“向一个物理系统入射光会得到怎样的出射光”,根本不需要去测量机械动量。
这种情况下我们完全没有必要关注$\vb*{p}$是正则动量这回事:完全可以打开括号$(\vb*{p} - q \vb*{A})^2$,然后求解束缚态问题
\begin{equation}
    H = \sum_i \frac{\vb*{p}_i^2}{2m_i} + H_\text{ext} + H_\text{int},
    \label{eq:levels-ham}
\end{equation}
具体求解时可以直接援引将$\vb*{p}$当成机械动量而得到的现成的解,得到能谱之后引入电子-电磁波耦合项
\begin{equation}
    H_\text{couple} = q \varphi - \frac{q}{2m} (\vb*{p} \cdot \vb*{A} + \vb*{A} \cdot \vb*{p}) + \frac{q^2}{2m} \vb*{A}^2,
    \label{eq:couple-ham}
\end{equation}
计算物质和光场的耦合。($\vb*{A}^2$项中含有粒子的位置,因此也是耦合项)

总之,关于普通的凝聚态系统中的电磁场,\eqref{eq:couple-ham}原则上给出了所有值得关注的信息。

\subsection{非相对论性粒子的运动方程}\label{sec:non-relativity-particles-eom}

在电子约束在原子势场中小幅运动时,可以认为
\begin{equation}
    m \ddot{\vb*{r}} = - m \omega_0^2 \vb*{r} - m \gamma \dot{\vb*{r}} + e \vb*{E},
\end{equation}
即回复力取线性近似。这就是\concept{洛伦兹模型}。

需要注意的是洛伦兹模型中的$m$和$\vb*{E}$的定义有些微妙之处。我们可以将$m$理解成电子的“裸”质量,而$\vb*{E}$理解为$\vb*{r}$点的总电场。
然而注意到运动电子自身会辐射出电场,$\vb*{E}$其实包括“外加场”和“自场”。
将自场剥离出来并写成$\vb*{r}$的函数,我们就需要将$m$理解成电子的“辐射质量”而将$\vb*{E}$理解成外加场。

这个微妙之处会实实在在带来可观察的效应,因为宏观上我们会测量的基本上是\autoref{sec:continuous-epsilon-mu-calc}中的那种移除了特征长度特别小或者说波数特别大的那部分傅里叶分量的电磁场,这部分电磁场是不包含电子的自场的;然而微观下,电子自场的的确确会带来一些效应。
因此此时系统中其实存在两部分电磁场:一部分是宏观电磁场,一部分是宏观下不可观测,需要以某种方式积掉的电子自场。
这个现象有时称为\concept{局域场增强}。这种效应的一种处理方式见\autoref{eq:local-electric-field-enhancement}。

\section{规范选取}

\subsection{标势、矢势和常见规范}

电场、磁场是直接可观测的量,实际上在相对论协变的场论中标势和矢势才是基本的场自由度,但是它们是不唯一确定的。
这一节我们演示如何在一开始\emph{不知道}标势和矢势才是场论中的自由度时,从麦克斯韦方程中得到标势和矢势。

实际上,我们也可以从\eqref{eq:maxwell-eq}出发,重构出一个关于$A^\mu$的理论。
\eqref{eq:maxwell-eq}的第三式告诉我们,存在一个矢量场$\vb*{A}$使得
\[
    \vb*{B} = \curl{\vb*{A}}.
\]
上式代入\eqref{eq:maxwell-eq}第二式,得到
\[
    \curl{\vb*{E} + \pdv{\vb*{A}}{t}} = 0,
\]
于是存在标量场$\varphi$使得
\[
    \vb*{E} = - \pdv{\vb*{A}}{t} - \grad{\varphi}.
\]
于是我们可以用$(\varphi, \vb*{A})$完全将电场和磁场表示出来,具体说就是
\begin{equation}
    \vb*{B} = \curl{\vb*{A}}, \quad \vb*{E} = - \pdv{\vb*{A}}{t} - \grad{\varphi}.
    \label{eq:e-b-from-a-phi}
\end{equation}
当然,$\varphi$就是我们熟悉的\concept{电势}而$\vb*{A}$就是\concept{磁矢势}。
这样一来\eqref{eq:maxwell-eq}的第一、四式就是
\[
    - \pdv{t} \div{\vb*{A}} - \laplacian \varphi = \rho, \quad \pdv[2]{\vb*{A}}{t} - \laplacian \vb*{A} + \grad(\div{\vb*{A}}) + \pdv{t} \grad{\varphi} = \vb*{j}.
\]
这两个式子看起来毫无规律,但是如果我们假定$(\rho, \vb*{A})$是某个四维矢量$A^\mu$的分量%
\footnote{这里可能会有一个问题:为什么我们确定$(\rho, \vb*{A})$能够构成四维矢量的分量?实际上,正确的思路是先设$(\rho, \vb*{A})$能够构成某个四分量对象的分量,推导出\qftdoc中的\eqref{qft-eq:movement-eq-2},然后注意到\qftdoc中的\eqref{qft-eq:movement-eq-2}如果总是成立,那么$A^\mu$一定要是矢量分量。
\label{note:vector-component}}%
,那么立刻可以发现这两个式子就是\qftdoc中的\eqref{qft-eq:movement-eq-2}。
因此实际上麦克斯韦方程和\qftdoc中的\eqref{qft-eq:movement-eq-2}是等价的。
如果使用洛伦兹规范,还可以将以上两个方程写得形式漂亮一些,也就是
\begin{equation}
    \pdv[2]{\varphi}{t} - \laplacian \varphi = \rho, \quad \pdv[2]{\vb*{A}}{t} - \laplacian \vb*{A} = \vb*{j}.
    \label{eq:wave-eq}
\end{equation}

回顾从拉氏量推导\qftdoc中的\eqref{qft-eq:movement-eq-2}的过程,我们会发现它只用到了\qftdoc中的\eqref{qft-eq:vec-lagrangian}以及$A^\mu$和$\psi$的耦合项中$A^\mu$是线性的这一事实,因此我们得出结论:\eqref{eq:maxwell-eq}描述了四维闵可夫斯基时空中一个单一无质量矢量场受到线性策动后的动力学——这个矢量场其实可以不是电动力学中的电磁场的。
如果需要将电动力学中的经验移植到比如说QCD中,那么求解\eqref{eq:maxwell-eq}是可以原封不动地搬运到其它无质量矢量场上的!

我们有
\begin{equation}
    \vb*{B} = \curl{\vb*{A}}, \quad \vb*{E} = - \pdv{\vb*{A}}{t} - \grad{\varphi},
\end{equation}
并且可以证明,不失一般性地,总是可以取以下规范:
\begin{equation}
    \div{\vb*{A}} + \frac{1}{c^2} \pdv{\varphi}{t} = 0,
\end{equation}
称为\concept{洛伦兹规范}。在洛伦兹规范下麦克斯韦方程成为
\begin{equation}
    \begin{bigcase}
        \laplacian{\varphi} - \frac{1}{c^2} \pdv[2]{\varphi}{t} &= - \frac{\rho(\vb*{r})}{\epsilon_0}, \\
        \laplacian{\vb*{A}} - \frac{1}{c^2} \pdv[2]{\vb*{A}}{t} &= - \mu_0 \vb*{j}(\vb*{r}),
    \end{bigcase}
\end{equation}
此时标势和矢势非常清晰地组成一个四维矢量,它们遵从四维的波动方程,因此也是“洛伦兹”一词的来源。

另一种常见的规范选择是\concept{库伦规范},为
\begin{equation}
    \div{\vb*{A}} = 0.
\end{equation}
在库伦规范之下麦克斯韦方程成为
\begin{equation}
    \begin{bigcase}
        \laplacian{\varphi} &= - \frac{\rho(\vb*{r})}{\epsilon_0}, \\
        \laplacian{\vb*{A}} - \frac{1}{c^2} \pdv[2]{\vb*{A}}{t} &= - \mu_0 \vb*{j}(\vb*{r}) + \frac{1}{c^2} \pdv{t} \grad{\varphi}.
    \end{bigcase}
\end{equation}
换而言之,标势的行为和静电场下完全一样,但是标势会对矢势有影响。
在电场和磁场随时间不变时库伦规范和麦克斯韦方程是一样的。
库伦规范有时也称为\concept{辐射规范},因为一种处理辐射的方法是以$\vb*{A}$为基本自由度,此时库伦规范就是横波条件。

\subsection{辐射场中的辐射规范}

,并且在远离介质粒子(场源)的地方,即在辐射场中,我们可以取横波条件$\div{\vb*{E}}=0$。
这可以让我们施加一个比一般的情况更加严格的规范。根据$\div{\vb*{E}}=0$我们有
\[
    \laplacian{\varphi} + \pdv{t} \div{\vb*{A}} = 0,
\]
此时我们没有加入任何限制。我们总是可以取$\varphi=0$,此时
\[
    \pdv{t} \div{\vb*{A}} = 0,
\]
即$\div{\vb*{A}}$是不随时间变化的。那么,总是可以找到一个不随着时间变化的标量场$\chi$,使得
\[
    \div{(\vb*{A} + \grad{\chi})} = 0,
\]
因为这个条件等价于调和方程
\[
    \laplacian{\chi} = - \div{\vb*{A}}.
\]
因此,我们可以做规范变换
\[
    \varphi' = \varphi - \pdv{\chi}{t} = \varphi, \quad \vb*{A}' = \vb*{A} + \grad{\chi},
\]
变换后就有$\varphi=0$和$\div{\vb*{A}}=0$同时成立。
因此,对辐射场可以同时施加以下两个规范:
\begin{equation}
    \varphi = 0, \quad \div{\vb*{A}} = 0,
\end{equation}
而不用担心产生冲突。这也可以称为\concept{辐射规范}。辐射规范下$\div{\vb*{A}}=0$这一条件保证了$\vb*{p}$和$\vb*{A}$是可交换的。

\section{偶极辐射近似}\label{sec:dipole-radiation}

\subsection{束缚态系统,微扰论和多极矩展开}\label{sec:multipole}

将\eqref{eq:couple-ham}当成微扰做微扰论的适用条件是$H_\text{couple}$相对于\eqref{eq:levels-ham}来说很小。
如果微扰论适用,那么显然$q \vb*{A} \ll \vb*{p}$,从而$\vb*{A}^2$项相较于$\vb*{p} \cdot \vb*{A}$项总是非常小的。%
\footnote{
    一个可以抬杠的地方是$\vb*{p}$很小时,似乎$\vb*{p} \cdot \vb*{A}$项远小于$\vb*{A}^2$项。
    然而,由能量守恒,$\vb*{A}^2$项相比于动能加上势能的\eqref{eq:levels-ham}总是很小的。
    如果我们只要求$\vb*{A}^2$级别的精度,那么在$\vb*{p}$大时显然$\vb*{p} \cdot \vb*{A}$项比$\vb*{A}^2$项重要,而$\vb*{p}$小时$\vb*{A}^2$项小于我们的精度要求。
    无论如何,$\vb*{A}^2$项都不如$\vb*{p} \cdot \vb*{A}$重要——后者重要时前者不重要,后者不重要时前者也没有重要到哪儿去。
}%
$q \vb*{A} \ll \vb*{p}$的条件实际上是不那么平凡的。
对散射态系统,机械动量估计为
\[
    m v \sim m \omega x,
\]
而
\[
    q E = m \ddot{x} \sim m \omega^2 x,
\]
最后有
\[
    E \sim - \pdv{A}{t} \sim \omega A,
\]
于是我们会发现$mv$和$eA$实际上是同个量级的。反之,对束缚态系统,$\vb*{p}$的最大值或者说振幅可以估计为
\[
    m \omega^2 x \sim q \grad{V_\text{ext}},
\]
而
\[
    mv \sim m \omega x,
\]
于是$p \gg eA$,等价于$mv \ll eA$,就等价于
\[
    mv \sim \frac{q}{\omega} \grad{V_\text{ext}} \gg q A,
\]
即等价于
\begin{equation}
    \grad{V_\text{ext}} \gg \omega A \sim E_\text{light},
\end{equation}
即束缚电场远大于光场。这应该是能够保证的,否则就不是束缚态了,此时介质就被打穿为等离子体了,并且,这种情况下,将光场撤去,介质也未必会恢复为原状,即出现了光学损伤。

在知道了能将\eqref{eq:couple-ham}当成微扰的系统中的带电粒子高度定域之后,我们立刻想到,由于这些带电粒子的位置高度有界,可以做多极矩展开。
实际上我们看到,多极矩展开合法、带电粒子位置高度定域(这意味着带电粒子)、$e \vb*{A} \ll \vb*{p}$这几个条件是等价的。
应该说\eqref{eq:couple-ham}是很不直观的,因为它是关于$\vb*{A}$的而不是$\vb*{E}$和$\vb*{B}$的,做完多极矩展开之后我们就可以讨论“某个过程在电偶极矩跃迁下可以发生,另一个过程需要电四极矩跃迁,从而很弱”,等等。
以下我们用$0$作为带点粒子位置的“原点”,$\vb*{r}$不会偏离$0$太远。

下面我们尝试使用多种近似手段。我们很快会发现,这些方法都指向同一个事实:对普通凝聚态介质中的电磁场,基本上只有电偶极子相互作用是重要的。

直接丢弃$\vb*{A}^2$项。此时如果采取辐射规范,根据$\vb*{p}$和$\vb*{A}$的可交换性,我们就有
\begin{equation}
    H_\text{couple} = - \frac{q}{m} \vb*{A} \cdot \vb*{p}.
    \label{eq:velocity-gauge}
\end{equation}
这称为\concept{速度规范}下的哈密顿量。如果我们进一步,假定$\vb*{A}$在电子运动的区域内没有明显的空间变化,则在一个规范变换之下我们可以得到
\begin{equation}
    H_\text{couple} = - q \vb*{r} \cdot \vb*{E} = - \vb*{d} \cdot \vb*{E}.
    \label{eq:electric-dipole}
\end{equation}
或者,由于$\vb*{A}$在电子运动的区域内没有明显的空间变化,我们根据\eqref{eq:velocity-gauge}可以写出(这里我们假装$\vb*{p}$就是机械动量,但是因为$\vb*{p}$的实际物理意义在做了近似\eqref{eq:velocity-gauge}不再影响哈密顿量的形式,这是可以的)
\[
    \begin{aligned}
        S &= \int \dd{t} \left( \frac{1}{2} m \vb*{v}^2 + \frac{q}{m} \vb*{A} \cdot (m \vb*{v}) \right) \\
        &= \int \dd{t} \left( \frac{1}{2} m \vb*{v}^2 - q \dv{\vb*{A}}{t} \cdot \vb*{r} \right) \\
        &= \int \dd{t} \left( \frac{1}{2} m \vb*{v}^2 + q \vb*{r} \cdot \vb*{E} \right),
    \end{aligned} 
\]
第二个等号用到了分部积分法。再做勒让德变换,就得到\eqref{eq:electric-dipole}。
实际上,我们会注意到以上构造拉氏量以后用分布积分法的方法只用到了一个条件,就是$\vb*{A}$的空间变化不大(从而它对时间的全导数就是它对时间的偏导数,就是电场的相反数),因此只需要“$\vb*{A}$的空间变化不大”就足够推导出\eqref{eq:electric-dipole}。
我们称\eqref{eq:electric-dipole}为\concept{长度规范}下的哈密顿量。

我们现在考虑$\vb*{A}^2$项能够丢弃,但是$\vb*{A}$尚有比较大的空间变化的情况;当然,这是为了将磁场和轨道自由度做耦合。
乍一看,我们可以使用磁标势方法来得到磁场,但是这是行不通的:我们在处理的并非静磁学问题,位移电流项是到处都在的,从而如果要用磁标势方法,磁壳必须取在我们讨论的电子的周围,从而让磁标势毫无用处。
我们会发现取
\begin{equation}
    \vb*{A} = \frac{1}{2} \vb*{B} \times \vb*{r} - \int_0^t \dd{t'} \grad(\vb*{r} \cdot \vb*{E}(\vb*{r}, t'))
    \label{eq:a-containing-e-and-b}
\end{equation}
能够提供足够好的近似。直接计算就会发现
\[
    \curl{\vb*{A}} = \vb*{B},
\]
而
\[
    \begin{aligned}
        \pdv{\vb*{A}}{t} &= \frac{1}{2} \pdv{\vb*{B}}{t} \times \vb*{r} - \grad{(\vb*{r} \cdot \vb*{E})} = \frac{1}{2} \vb*{r} \times (\curl{\vb*{E}}) - \grad{(\vb*{r} \cdot \vb*{E})} \\
        &= \frac{1}{2} \left( \grad{(\vb*{r} \cdot \vb*{E})} - (\vb*{r} \cdot \grad) \vb*{E} - (\vb*{E} \cdot \grad) \vb*{r} - \vb*{E} \times (\curl{\vb*{r}}) \right) - \grad{(\vb*{r} \cdot \vb*{E})} \\
        &= \frac{1}{2} \left( \grad{(\vb*{r} \cdot \vb*{E})} - (\vb*{r} \cdot \grad) \vb*{E} - \vb*{E} \right)- \grad{(\vb*{r} \cdot \vb*{E})} .
    \end{aligned}
\]
如果假定$\vb*{E}$和$\vb*{B}$在空间上没有什么变化,那么就有
\[
    \pdv{\vb*{A}}{t} = \frac{1}{2} (\vb*{E} - \vb*{E}) - \vb*{E} = - \vb*{E}.
\]
因此,在电场和磁场在我们关心的区域基本均匀的情况下,\eqref{eq:a-containing-e-and-b}近似是辐射规范下的矢势。
现在我们再做一个规范变换:
\[
    \vb*{A} \longrightarrow \vb*{A} + \grad{\chi}, \quad \varphi \longrightarrow \varphi - \pdv{\chi}{t}, \quad \chi = \int_0^t \dd{t'} \vb*{r} \cdot \vb*{E}(\vb*{r}, t'),
\]
就有
\[
    \begin{aligned}
        H_\text{couple} &= q \varphi - \frac{q}{m} \vb*{A} \cdot \vb*{p} \\
        &= - q \vb*{r} \cdot \vb*{E} - \frac{q}{m} \frac{1}{2} (\vb*{B} \times \vb*{r}) \cdot \vb*{p} \\
        &= - \vb*{d} \cdot \vb*{E} - \frac{q}{2m} \vb*{B} \cdot (\vb*{r} \times \vb*{p}),
    \end{aligned}
\]
从而
\begin{equation}
    H_\text{couple} = - \vb*{d} \cdot \vb*{E} - \frac{q}{2m} \vb*{L} \cdot \vb*{B}.
\end{equation}
这里多出来了一项,即磁场和轨道角动量的耦合。

实际上,在磁偶极耦合比较重要的情况下,电四极耦合可能也是重要的,因为此时$\vb*{B}$不会很小,从而$\curl{\vb*{E}} = - \pdv*{\vb*{B}}{t}$也不会很小,这意味着电场有能够观察到的空间变化。
于是我们不应该使用$-q \vb*{r} \cdot \vb*{E}$代替$q \varphi$,而应该把它展开到电四极矩耦合项。

现在我们有了三种相互作用通道,有电偶极跃迁
\begin{equation}
    H_1 = - \vb*{d} \cdot \vb*{E},
\end{equation}
有自旋取向作用
\begin{equation}
    H_2 = - \frac{q}{m} \vb*{S} \cdot \vb*{B},
\end{equation}
还有轨道角动量取向作用
\begin{equation}
    H_3 = - \frac{q}{2m} \vb*{L} \cdot \vb*{B}.
\end{equation}
实际上,磁场对自旋的取向作用${H}_2$是很弱的。设电磁波波长的尺度为$\lambda$,则
\[
    \vb*{B} = \curl{\vb*{A}} \sim \frac{A}{\lambda},
\]
电子的活动范围的尺度和原子半径$a_0$同阶,由不确定性关系,
\[
    p a_0 \sim \hbar.
\]
于是
\[
    \frac{H_2}{H_1} \sim \frac{\hbar \frac{A}{\lambda}}{\frac{\hbar}{a_0} A} = \frac{a_0}{\lambda}.
\]
波长通常在几百纳米级别,而原子半径在纳米级别以下,从而${H}_1$远大于${H}_2$。
类似的可以证明$H_1 \gg H_3$。
不过,如果$H_1$对两个在没有电磁场的情况下是简并态的态的能量修正相同,可是$H_{2,3}$对它们的能量修正却不相同,那么后两者就会造成很重要的物理影响。

更加系统的多极矩展开可以在Daniel A. Steck的讲义Quantum and Atom Optics的9.5节(Atomic Interaction with the Quantized Field - Multipole Interactions)中找到。
此外应注意这些相互作用能在计算体系总能量(电磁场加粒子)时实际上不应该单独计算;这些“相互作用能”实际上最好还是看成拉氏量中的项,而不是哈密顿量中的项;见\prettyref{sec:medium-energy-lagrangian}。

\chapter{连续介质概述}

虽然介质对电磁波的行为的修正原则上只是电子-光子散射,但是由于介质本身的物态是繁多的,电磁波的行为的修正也是非常复杂的。
介质会引入对麦克斯韦方程的线性修正——此时介质中的电磁波遵循的方程仍然是经过某种修正的线性麦克斯韦方程,相应的介质就是\concept{线性介质}——也可能是非线性修正。
微扰量子场论中总是先讨论自由理论再加入相互作用,因此对电磁波,首先考虑线性介质中的电磁波,然后再考虑非线性效应——它给出光子的散射——也是通常的做法。

\section{连续介质中的电磁波}

\subsection{连续介质中的麦克斯韦方程的一个经典推导}\label{sec:classical-continuum}

在\autoref{sec:long-wavelength-photon-maxwell-general}中我们将要从对称性的角度说明,介质对麦克斯韦方程的修正可以归结为极化、磁化以及电流本构关系,并且这种修正在量子光学中同样适用。
本节则将介质暂且看成完全经典的东西,即将物质场抽象为电荷和电流,试图建立一套和真空中的麦克斯韦方程足够相似的介质中的麦克斯韦方程,并为其中各项提供直观的、经典的意义。

真空中的麦克斯韦方程组为我们熟知的形式:
\begin{equation}
    \begin{bigcase}
        \div{\vb*{E}} &= \frac{\rho}{\epsilon_0} \\
        \curl{\vb*{E}} &= - \pdv{\vb*{B}}{t} \\
        \div{\vb*{B}} &= 0 \\
        \curl{\vb*{B}} &= \mu_0 \vb*{j} + \mu_0 \epsilon_0 \pdv{\vb*{E}}{t}
    \end{bigcase}
    \label{eq:original-maxwell}
\end{equation}
介质的存在事实上在微观层面不会改变\eqref{eq:original-maxwell}的形式。
介质起作用的方式是,其内部已经有一个电荷分布,当外加电场的时候电荷重新排列、发生运动,在此过程中产生额外的电流、电场、磁场。
于是假定电荷和电流可以做以下分解:
\[
    \begin{bigcase}
        &\vb*{j} = \vb*{j}_\text{f} + \vb*{j}_\text{r}, \quad \rho = \rho_\text{f} + \rho_\text{r}, \\
        &\pdv{\rho_\text{f}}{t} + \div{\vb*{j}_\text{f}} = 0, \\
        &\pdv{\rho_\text{r}}{t} + \div{\vb*{j}_\text{r}} = 0
    \end{bigcase}
\]
其中$\vb*{j}_\text{f}$是所谓的自由电流,而$\vb*{j}_\text{r}$是介质的响应,如果它不能自由流动,就称为束缚电流。但是这种二分法实际上很大程度上是任意的。
例如,金属能导电,因为其内部含有大量几乎是自由的电子——那么,外加电场产生的金属中的电流就应该是自由电流了;
但是分析金属的光学属性的时候,这些由于外加电场产生的电流又无疑是介质的响应。
因此$\vb*{j}_\text{f}$和$\vb*{j}_\text{r}$只是辅助量,没有特殊的物理含义。
不过,我们通常要求$\vb*{j}_\text{r}$\emph{不能}被除了(我们关系的那部分——因为操纵电荷的几乎总是电磁波)电磁波以外的手段操纵,否则不能保证它们的状态可以完全被$\vb*{E}$和$\vb*{B}$确定。
使用费曼图的语言,就是我们希望$\vb*{j}_\text{r}$总是出现在内线而不能出现在外线。(经典理论中是可以有费曼图的,见\cite{hellingsolving})

为了能够将$\vb*{j}_\text{f}$和$\vb*{j}_\text{r}$整合进两个形式上和电场和磁感应强度很像的辅助量,
从而在形式上让\eqref{eq:original-maxwell}变成一个只和自由电荷和自由电流有关的方程组,我们进一步做下面的分解:
\[
    \vb*{j}_\text{r} = \vb*{j}_\text{s} + \vb*{j}_\text{c}
\]
且$\vb*{j}_\text{c}$是一个有旋无源场。光有这个条件不足以在给定$\vb*{j}_\text{r}$时唯一地确定下$\vb*{j}_\text{s}$和$\vb*{j}_\text{c}$,
因此还可以引入一个假设而不至于让$\vb*{j}_\text{s}$和$\vb*{j}_\text{c}$无解。
为了让\eqref{eq:original-maxwell}中第一式的右边只剩下自由电荷,假定
\begin{equation}
    \rho_\text{r} = - \div{\vb*{P}}.
    \label{eq:polarization-vector-def}
\end{equation}
这个假设\concept{没有}缩小$\vb*{j}_\text{s}$和$\vb*{j}_\text{c}$的选择范围,因为任意给定性质足够良好的$\rho_\text{r}$,相对应的$\vb*{P}$总是存在的(而且显然不唯一)。
同时由于$\vb*{j}_\text{c}$是一个有旋无源场,可以再引进一个辅助量$\vb*{M}$使
\[
    \vb*{j}_\text{c} = \curl{\vb*{M}}
\]
此时$\rho_\text{r}$的输运方程成为
\[
    \pdv{\rho_\text{r}}{t} + \div{\vb*{j}_\text{s}} = 0
\]
因为$\curl{\vb*{j}_\text{c}}$的散度为零。这个式子又可以写成
\[
    \div{\left(\vb*{j}_\text{s}-\pdv{\vb*{P}}{t}\right)} = 0
\]
受到这个式子的启发,我们\concept{假设}(不是推出,因为光有上式不能定解,而先前我们只对$\vb*{j}_\text{c}$做过假设而没有对$\vb*{j}_\text{s}$做过假设,因此后者的取值仍然是任意的)有
\[
    \vb*{j}_\text{s} = \pdv{\vb*{P}}{t}
\]
这个假设不会让$\vb*{j}_\text{s}$和$\vb*{j}_\text{c}$无解。

将以上引入的所有物理量代入\eqref{eq:original-maxwell},得到
\[
    \begin{bigcase}
        \epsilon_0 \div{\vb*{E}} &= \rho_\text{f} - \div{\vb*{P}}, \\
        \curl{\vb*{E}} &= - \pdv{\vb*{B}}{t}, \\
        \div{\vb*{B}} &= 0, \\
        \curl{\frac{\vb*{B}}{\mu_0}} &= \vb*{j}_\text{f} + \curl{\vb*{M}} + \pdv{\vb*{P}}{t} + \epsilon_0 \pdv{\vb*{E}}{t}
    \end{bigcase}
\]
引入辅助量
\[
    \vb*{D} = \epsilon_0 \vb*{E} + \vb*{P}, \quad \vb*{H} = \frac{\vb*{B}}{\mu_0} - \vb*{M}
\]
就得到了
\begin{equation}
    \begin{bigcase}
        \div{\vb*{D}} &= \rho_\text{f}, \\
        \curl{\vb*{E}} &= - \pdv{\vb*{B}}{t}, \\
        \div{\vb*{B}} &= 0, \\
        \curl{\vb*{H}} &= \vb*{j}_\text{f} + \pdv{\vb*{D}}{t}
    \end{bigcase}
    \label{eq:maxwell-material}
\end{equation}

方程组\eqref{eq:maxwell-material}除去了\eqref{eq:original-maxwell}中由于介质产生的电荷密度和电流密度,形式上更加简洁,
但是即使在自由电荷密度和电流密度已经给定的情况下,只靠\eqref{eq:maxwell-material}本身也没有办法定解,因为未知数太多了。
考虑到从$\vb*{E}, \vb*{B}$到$\vb*{D}, \vb*{H}$的变换是线性的,
这就意味着\eqref{eq:original-maxwell}在自由电荷密度和电流密度已经给定的情况下其实也不能定解。
这是理所当然的。

下面的问题是,在自由电荷密度和电流密度已经给定的情况下,增加什么方程能够让\eqref{eq:maxwell-material}定解?
当然,只要知道了从$\vb*{E}, \vb*{B}$到$\vb*{D}, \vb*{H}$的变换的具体计算式(而不是显含$\vb*{j}_\text{r}$的定义式)
就能够定解。
更进一步,在什么都不知道,只有初始条件和边界条件的情况下,怎样能够让\eqref{eq:maxwell-material}定解?
只需要增补$\vb*{j}_\text{f}$和$\vb*{E}$的显式关系,以及输运方程
\begin{equation}
    \pdv{\rho_\text{f}}{t} + \div{\vb*{j}_\text{f}} = 0
    \label{eq:transportation}
\end{equation}
就能够定解。

因此要求解出介质中的电磁场变化情况,首先需要\concept{物理方程}\eqref{eq:maxwell-material},
然后是\concept{本构关系}也就是$\vb*{D}$,$\vb*{H}$,$\vb*{j}_\text{f}$关于其他量的表达式,最后是\concept{几何关系}\eqref{eq:transportation},
再加上适当的\concept{边界条件}和\concept{初始条件},就能够定解。

关于本构关系实际上有一个问题,就是从$\vb*{E}$,$\vb*{B}$,$\vb*{j}_\text{f}$到$\vb*{D}$和$\vb*{H}$是不是真的有一个函数关系。
如果相同的$\vb*{E}$,$\vb*{B}$,$\vb*{j}_\text{f}$实际上对应着不同的系统状态,那就糟糕了。
但是在经典电动力学中$\vb*{E}$,$\vb*{B}$是仅有的场,它们完全标记了系统状态,
而如果对$\vb*{j}_\text{s}$和$\vb*{j}_\text{c}$加上足够的限制,总是可以使用$\vb*{j}_\text{f}$确定下整个$\vb*{j}$的分布,从而$\rho$的分布,
因此$\vb*{E}$,$\vb*{B}$,$\vb*{j}_\text{f}$能够完全确定系统状态,从而本构关系总是可以写出来的。
然而如果$\vb*{D}$和$\vb*{E}$之间的关系非常复杂——实际上,考虑到介质微观来看是非常\emph{不均匀}的,两者的关系就是应该非常复杂——那么写下它们的本构关系是完全不现实的。
在我们能够找到比较简单的、特征尺度远大于介质的微观结构(原子间距$\sim \SI{1}{\angstrom}$)的本构关系时,我们说,\emph{连续介质近似}适用。
之后,要讨论介质性能,就只需要考虑$\vb*{D}$和$\vb*{H}$——或者说$\vb*{P}$和$\vb*{M}$——即可。
我们将$\vb*{P}$称为\concept{极化矢量},将$\vb*{M}$称为\concept{磁化矢量},因为直观地看,$\vb*{P}$看起来和磁场没有直接关系,$\vb*{P}$的出现可以理解为来自外加电场让介质中出现电荷的重新分布,实际上就是让原本电中性的介质分子变成电偶极子乃至其它电多极子,而$\vb*{P}$就体现了这种“极化”的宏观效应;同理,$\vb*{M}$看起来主要和磁场有关,其出现可以理解为外加磁场让电介质中出现环状(“有旋”)电流,从而产生磁场,于是$\vb*{M}$体现了“磁化”的宏观效应。
我们将在\autoref{sec:continuous-epsilon-mu-calc}中更加详细地讨论这一点。

连续介质近似成立的线性介质的本构关系总是可以写成
\begin{equation}
    \vb*{D} = \vb*{\epsilon} \cdot \vb*{E}, \quad \vb*{B} = \vb*{\mu} \cdot \vb*{H},
\end{equation}
其中$\vb*{\epsilon}$和$\vb*{\mu}$是张量算符。非线性连续介质的$\vb*{P}$就不再简单地能够写成电磁场的线性函数了。

\subsection{连续介质的电磁性能的粗粒化导出}\label{sec:continuous-epsilon-mu-calc}

以上论证——\autoref{sec:classical-continuum}和\autoref{sec:quantum-continuum}——都是非常唯象的。
它们根本没有告诉我们要怎么获得$\vb*{\epsilon}$和$\vb*{\mu}$。
例如说它们没有告诉我们为什么$\vb*{P}$一般来说就是各个分子的电偶极矩加起来除以体积;只是看着\eqref{eq:polarization-vector-def}是什么也看不出来的。
本节做一些半经典的计算,来直观地展示连续介质近似忽略了什么。

在连续介质近似适用的问题中我们关心的“介质中电场”一般都是做了某种空间平均,抹去了非常小的空间尺度的变化的电场。
电动力学教科书上通常会泛泛地说这是“做了宏观平均的电场”,实际上这就是动量截断。
要看出这是为什么,注意到“做了宏观平均的电场”总是可以通过一个体积为$V$的移动平均给出:
\[
    \bar{\vb*{E}}(\vb*{r}) = \frac{1}{V} \int \dd[3]{\vb*{r}'} \vb*{E}(\vb*{r}'),
\]
做傅里叶展开,有
\[
    \begin{aligned}
        \bar{\vb*{E}}(\vb*{r}) &= \frac{1}{V} \int \dd[3]{\vb*{r}'} \int \frac{\dd[3]{\vb*{k}}}{(2\pi)^3} \vb*{E}(\vb*{k}) \ee^{\ii \vb*{k} \cdot \vb*{r}'} \\
        &= \int \frac{\dd[3]{\vb*{k}}}{(2\pi)^3} \vb*{E}(\vb*{k}) \frac{1}{V} \int \dd[3]{\vb*{r}'} \ee^{\ii \vb*{k} \cdot \vb*{r}'},
    \end{aligned}
\]
显然对那些较大的$\vb*{k}$,对$\vb*{r}'$的积分会由于快速振荡而变成零,于是我们就得到了一个动量截断。
我们将做了动量截断的$\vb*{E}$还是记作$\vb*{E}$,只是要记住此时的$\vb*{E}$并没有涵盖电磁场能够有的全部模式;那些高波数、小特征长度的电场分量会有重要的物理效应,正如我们在\autoref{sec:non-relativity-particles-eom}中讨论的那样。

既然对电磁场做了粗粒化我们也要对介质性能做粗粒化。
设有一个“过滤器”$\mathcal{F}[\cdot]$滤除一个场的高动量成分,即,滤除其微观细节。
我们在\eqref{eq:maxwell-material}的四个方程两边作用$\mathcal{F}$,就能够得到一个“平均化”的介质中麦克斯韦方程,其中所有的场都没有特征长度特别小的起伏,从而表现得和理想的连续介质中的场完全一样。
我们为了简便起见将$\mathcal{F}[\cdot]$记作$\overline{\cdots}$。

我们将全体束缚电荷选为$\rho_\text{r}$,完整考虑所有微观细节的束缚电荷密度为
\[
    \rho_\text{r}(\vb*{r}) = \sum_i q_i \delta(\vb*{r} - \vb*{r}_i).
\]
我们对它做傅里叶变换,有
\[
    \begin{aligned}
        \rho_\text{r}(\vb*{k}) &= \sum_i q_i \int \dd[3]{\vb*{r}} \ee^{- \ii \vb*{k} \cdot \vb*{r}} \delta(\vb*{r} - \vb*{r}_i) \\
        &= \sum_i q_i \int \dd[3]{\vb*{r}} \ee^{- \ii \vb*{k} \cdot \vb*{r}_i} \\
        &= \sum_i q_i \int \dd[3]{\vb*{r}} \ee^{- \ii \vb*{k} \cdot \vb*{r}} \ee^{- \ii \vb*{k} \cdot (\vb*{r}_i - \vb*{r})} \\
        &= \sum_i q_i \int \dd[3]{\vb*{r}} \ee^{- \ii \vb*{k} \cdot \vb*{r}} (1 - \ii \vb*{k} \cdot (\vb*{r}_i - \vb*{r}) + \cdots).
    \end{aligned}
\]
由于电介质整体上是电中性的,并且我们正在研究束缚电荷,总是可以找到几个$\vb*{r}_i$接近于$\vb*{r}$的$q_i$使得它们的代数和为零,我们将这些电荷的集合记作$C$。
进一步,由于只考虑小的$\vb*{k}$,略去上式中的高阶项,则有
\[
    \begin{aligned}
        \overline{\rho_\text{r}(\vb*{k})} &= \sum_i q_i \int \dd[3]{\vb*{r}} \ee^{- \ii \vb*{k} \cdot \vb*{r}} (1 - \ii \vb*{k} \cdot (\vb*{r}_i - \vb*{r}) ) \\
        &= \int \dd[3]{\vb*{r}} \ee^{- \ii \vb*{k} \cdot \vb*{r}} \sum_{\text{cluster $C$}} \sum_{i \in C} q_i (1 - \ii \vb*{k} \cdot (\vb*{r}_i - \vb*{r})) \\
        &= \int \dd[3]{\vb*{r}} \ee^{- \ii \vb*{k} \cdot \vb*{r}} \sum_{\text{cluster $C$}} \sum_{i \in C} q_i (- \ii \vb*{k} \cdot \vb*{r}_i) \\
        &= \int \dd[3]{\vb*{r}} \ee^{- \ii \vb*{k} \cdot \vb*{r}} \sum_{\text{$C$ near $\vb*{r}$}} (- \ii \vb*{k}) \cdot \vb*{p}_C,
    \end{aligned}
\]
其中$\vb*{p}$是$C$的电偶极矩。做傅里叶逆变换,有
\[
    \begin{aligned}
        \overline{\rho_\text{r}(\vb*{r})} &= \int \frac{\dd[3]{\vb*{k}}}{(2\pi)^3} \ee^{\ii \vb*{k} \cdot \vb*{r}} \int \dd[3]{\vb*{r}'} \ee^{- \ii \vb*{k} \cdot \vb*{r}'} \sum_{\text{$C$ near $\vb*{r}'$}} (- \ii \vb*{k}) \cdot \vb*{p}_C \\
        &= - \div \int \frac{\dd[3]{\vb*{k}}}{(2\pi)^3} \ee^{\ii \vb*{k} \cdot \vb*{r}} \int \dd[3]{\vb*{r}'} \ee^{- \ii \vb*{k} \cdot \vb*{r}'} \sum_{\text{$C$ near $\vb*{r}'$}} \vb*{p}_C \\
        &= - \div{\sum_{\text{$C$ near $\vb*{r}$}} \vb*{p}_C}.
    \end{aligned}
\]
如果我们只关心介质中长波电磁波的行为,则可以将所有物理量都替换成它们经过$\mathcal{F}$滤波之后的值,从而可以取
\begin{equation}
    \vb*{P} = \sum_{\text{$C$ near $\vb*{r}$}} \vb*{p}_C.
\end{equation}

电流的计算要复杂一些,思路是类似的。电流密度为
\[
    \vb*{j}_\text{r}(\vb*{r}) = \sum_i q_i \vb*{v}_i \delta(\vb*{r} - \vb*{r}_i),
\]
于是
磁偶极矩都是反对称化来的。

\subsection{连续介质模型在量子光学中的适用性}\label{sec:quantum-continuum}

一个直接的问题是前一节中给出的连续介质近似是不是适用于电磁场的量子特性显示出来的情形。
直接将QED和介质耦合起来并在零温场论中积掉介质虽然的确是正确的,在实际计算时却会产生一些理论上的问题。
例如,我们知道,介质通常出于热态,因此,一个光子和介质发生相互作用之后就处于混合态了,似乎不能写出一个场论来描述介质中光子;从介质吸收光子到发射光子会有时间延迟;介质微观上是非常不均匀的,从而平面波进入介质后波阵面将面目全非。
总之,完全在零温场论中从头计算介质影响实际上意味着我们要“积掉剩下的宇宙”;
需要用到非平衡态场论才能够严格解决以上疑难。

一种比较粗略,但是现实中确实有用的做法是形式上认为我们\emph{已经}将介质积掉了,并根据对称性等修正真空中麦克斯韦方程,而用介质中的电磁场-电磁场关联函数体现介质的存在。
我们随后通过第一性原理或者别的什么办法计算一些情景,和用连续介质模型计算出的情形做比较,从而拟合出介质中麦克斯韦方程中的参数。
如果介质不会让光热化,那么这么做就是正确的,通过介质中的电磁场-电磁场关联函数能够体现介质引入的等效过程;如果介质实际上会让光热化,这么做就失效了。
介质的线性效应体现为电磁场的作用量的二次型部分出现一个修正,非线性效应体现为电磁场的自相互作用,非幺正的部分体现为以上修正中的虚部。
后两个问题可以采用和经典电动力学类似的方法解决,即我们只处理“经过空间平均”的电磁场,这相当于做了一个动量截断,只讨论波长足够长的那部分电磁波模式,则介质中发生的过程相比于我们讨论的过程来说是非常快、且空间细节不甚清楚的,从而,介质导致的电磁场关联函数的修正可以认为没有时间上的延迟效应或是空间上的非局域效应。

在确实只需要考虑波长远大于介质的微观不均匀性的空间尺度的光子时,形式上,我们可以直接将介质中的麦克斯韦方程做正则量子化。要看出这是为什么,首先考虑线性部分,描述光场的宏观的线性介质中的麦克斯韦方程是
\[
    \begin{aligned}
        &\div{\vb*{D}} = 0, \quad \curl{\vb*{E}} = - \pdv{\vb*{B}}{t}, \\
        &\div{\vb*{B}} = 0, \quad \curl{\vb*{H}} = \pdv{\vb*{D}}{t} + \vb*{j},
    \end{aligned}
\]
这里我们保留了传导电流,这是为了提示系统哈密顿量中外加激励项$\vb*{j} \cdot \vb*{A}$的存在。取规范$\varphi=0$,并切换到频域,我们会发现以上方程等价于辐射规范加上
\begin{equation}
    \curl{(\mu^{-1} \cdot \curl{\vb*{A}})} - \omega^2 \epsilon \cdot \vb*{A} = \vb*{j}.
    \label{eq:photon-in-material}
\end{equation}
如果介质修正后的电磁场关联函数实际上就是上式的格林函数,我们就可以直接将线性介质中的麦克斯韦方程中的电场和磁场提升为算符,完成正则量子化。

对称性分析告诉我们,在长波光子条件成立时,破缺空间平移对称性和空间各向同性,但保留局域性,则\eqref{eq:photon-in-material}是最一般的方程。
可以在整个方程左边再乘上一个张量,但是我们随即可以将这个张量吸收到$\vb*{j}$的定义中;$\curl{\vb*{A}}$的形式不能改变,因为无论如何,从$\vb*{A}$出发能够得到的局域的规范不变矢量除了$\pdv*{\vb*{A}}{t}$——在频域下就正比于$\vb*{A}$——以外就只有它了。
因此,的确,对波长远大于介质微观不均匀性(晶格常数等)的光子(大部分能够称为“光学”的问题都是这样的,因为晶格常数差不多几百皮米,已经对应X射线的波长了),至少线性介质中的麦克斯韦方程可以被理解为海森堡绘景下的方程,虽然我们导出它用的似乎是完全经典的论证。

非线性项可以如法炮制。我们就是需要用赝矢量$\curl{\vb*{A}}$和矢量$\pdv*{\vb*{A}}{t}$构造出一个方程(见\autoref{sec:e-m-field-transformation-feature-vacuum})。
由于$\curl{\vb*{A}}$整体上是一个规范不变量,我们不能将它修正为$\curl{\vb*{T} \cdot \vb*{A}}$之类的东西;然而它是一个赝矢量,而$\pdv*{\vb*{A}}{t}$和$\vb*{j}$都是普通矢量,于是我们必须将$\curl{\vb*{A}}$乘上某个张量以后得到的赝矢量再乘上一个赝矢量,才能够得到普通矢量。
于是,最后我们能有的方程的形式还是\eqref{eq:photon-in-material}。
于是,光子相互作用就通过显含$\vb*{E}$的$\vb*{\epsilon}$引入了,实际上我们就是依靠\emph{量子化的极化矢量和磁化矢量}引入了光子相互作用。

从哈密顿量的角度出发可以更加容易地看出为什么线性麦克斯韦方程\eqref{eq:photon-in-material}可以直接量子化。
破缺空间平移对称性和空间各向同性之后,\eqref{eq:e-and-b-hamiltonian}能够有的修正方式是非常有限的:如果保持哈密顿量为二次型,我们只能够让$\vb*{E}^2$项和$\vb*{B}^2$项变得各向异性,即让它变成
\begin{equation}
    H = \int \dd[3]{\vb*{r}} \left( \frac{1}{2} \vb*{E} \cdot \vb*{\epsilon} \cdot \vb*{E} + \frac{1}{2} \vb*{B} \cdot (\vb*{\mu}^{-1}) \cdot \vb*{B} \right) = \frac{1}{2} \int \dd[3]{\vb*{r}} (\vb*{D} \cdot \vb*{E} + \vb*{B} \cdot \vb*{H}).
    \label{eq:material-hamiltonian}
\end{equation}
它和\eqref{eq:photon-in-material}是等价的。哈密顿量被修正在物理上对应着积掉介质,如果只考虑长波光子,那么这个过程应该给出在时间上和空间上都是局域的等效光子相互作用。
原则上可以产生$\vb*{E}$和$\vb*{B}$的任意次方项,只保留两项就得到\eqref{eq:material-hamiltonian},保留更多项就得到非线性光学效应(例如见\cite{drummond_hillery_2014}中对非线性光学的哈密顿量的讨论)。

最后,注意到\autoref{sec:continuous-epsilon-mu-calc}中的讨论实际上并没有用到多少物理量可交换之类的条件,也没有同时需要确定坐标和动量,而只是用到了傅里叶变换,因此实际上那里的讨论是可以直接搬到量子情况下的。
因此在连续介质中连续介质那一套的确是可以直接移植到量子情况下的。

我们要指出的是,在常见的凝聚态体系中,的确会经常出现上述近似失效的情况;
例如由于诸如声子散射和自发发射等过程,$\epsilon$中会出现损耗,从而\eqref{eq:material-hamiltonian}不是厄米的;显然,由于幺正性的要求,这些损耗必然会以噪声的形式释放出来;在自发发射比较重要的情况下体系会发出荧光,荧光有一个经典概率分布(用量子的术语说,永远是混合态的而不是纯态的),显然不能用确定性的$\epsilon$刻画(虽然使用涨落耗散定理一类的理论还是可以从$\epsilon$中提取荧光谱的信息)。
即使暂时不考虑耗散,由于实际材料的响应根本不可能足够快,$\epsilon$肯定是有时间延迟的,从而其频域版本会有$\omega$依赖——此时我们说体系有\concept{色散}——于是\eqref{eq:material-hamiltonian}必须改写成某种有推迟的形式;而一旦我们考虑了$\epsilon$的频率依赖,由于Kronig-Kramers关系,耗散就必须被考虑。%
\footnote{
    $\epsilon(\omega)$有较强频率依赖的常见例子是,光场和某个离散谱的能级$\omega_0$有强烈的耦合,于是$\epsilon(\omega)$在$\omega = \omega_0$处有一个峰值。
    但是如果体系中没有耗散,在$\omega = \omega_0$处$\epsilon(\omega)$必定发散,因为强度不变的频率为$\omega_0$的电磁波只会让$\omega_0$频率的自由度的强度越来越高。
    为了保证$\epsilon(\omega)$的平滑性,必须在$\omega = \omega_0$附近考虑耗散。
}

总之,最为精确的计算需要在理论中同时保留光场和介质,并且显式地根据光场和介质的电偶极子耦合做微扰计算;而在连续介质近似中,我们首先计算介质中的电磁场关联函数,然后根据\eqref{eq:photon-in-material}得到$\epsilon$和$\mu$,代入算符版本的介质中麦克斯韦方程,得到一个只关于光场的理论。
后一种近似实际上是非常粗糙的,但是波长明显长于介质微观结构尺度(如晶格常数)的光子确定适用,因此适用范围还是很大的。
本文仅仅说明介质的光学性质可以使用什么样的量表征,而并不具体计算这些量;对这些物理量的计算在\soliddoc中进行。

\subsection{连续介质近似的失效}

最后我们讨论连续介质近似何时失效。如前所述,对长波光子才有连续介质近似,那么如果光子波长到了介质的微观结构的尺度,连续介质近似就失效了。此时必须使用完整的第一性原理做计算。
硬X射线波长在\SI{100}{pm}量级,因此全体硬X射线乃至$\gamma$射线和普通的晶体的互动都肯定不能使用连续介质近似:它们甚至能够看到原子内部结构了。
波长在\SI{1}{\angstrom}量级的软X射线当然也不行,它们虽然看不到原子结构,但能够精确地看到晶格结构。
波长在\SI{1}{nm}附加的软X射线的行为则非常有趣:实验上它们根本看不到介质微观结构的细节,但是使用更长波的电磁波测出的$\epsilon$和$\mu$描述它们肯定会出错的。
这个波段的光需要使用一些连续介质近似的修正理论来描述。
波长再长的光就可以完美使用连续介质近似描述了。可以看到,连续介质近似在非常宽的频谱上都是非常有用的。

连续介质近似还要求有一部分电荷能够被划分为“束缚电荷”,它们不是系统的低能自由度,从而可以积掉。
如果实际上所有的电荷都能够自由移动、被外场操纵,总之,“出现在费曼图的外线中”,那么“束缚电荷”的概念本身就是缺乏意义的。
在强场下就会出现这种情况:此时单光子能量未必很大(光子波长未必很短),然而有如此多的光子,以至于场强很强,介质内的电子可以短时间内积累很大能量然后挣脱束缚。
这种场景的典型的单位面积功率是$\sim\SI{e13}{W/cm^2}$。作为对比,激光笔的单位面积功率是$\sim \SI{e5}{W/cm^2}$。

\section{介质电磁性能的计算}

\subsection{格林函数和电磁线性响应}\label{sec:green-and-linear-response}

计算介质的电磁性能时通常使用格林函数方法,即外加一个驱动,计算$\vb*{P}$的响应。不过这里需要注意一个地方,就是“外加驱动”\emph{不是}$\vb*{E}$。可以这么考虑这个问题:设想系统正处在某个本征模式上,则外加驱动为零时$\vb*{P}$不为零,但是$\vb*{P}$可以直接激发出电场,因此$\vb*{E}$不是零。
因此我们可以看出,$\vb*{E}$绝对不是“外加驱动”。实际上,根据$\vb*{D}$的定义,我们会发现$\vb*{D} / \epsilon_0$才是宏观的外加驱动电场。
我们可以看出,由于
\[
    \vb*{D} = \tensor{\vb*{\epsilon}} \vb*{E}, \quad \vb*{D} = \epsilon_0 \vb*{E} + \vb*{P},
\]
我们有
\begin{equation}
    \vb*{P} = \tensor{\vb*{\epsilon}}^{-1} (\tensor{\vb*{\epsilon}} - \epsilon_0) \vb*{D}.
\end{equation}
因此,在一个本征模式上,外加驱动到$\vb*{P}$的响应函数发散,而相应的,应有$\tensor{\vb*{\epsilon}} = 0$,宏观地看就是在没有外加驱动时仍然有非零的$\vb*{E}$(就是介质内部的模式产生的)。

因此,在通过介质的电磁响应分析介质内的模式时需要非常小心:通过格林函数得到的\emph{不是}$\vb*{p} = \alpha \vb*{E}$中的那个$\alpha$!确定介质中的模式的色散关系时,应取$\epsilon = 0$,或者与之等价的$P / D = \infty$。

\subsection{Clausius-Mossotti关系}

Clausius-Mossotti关系是局域场强增强的一个例子。在测定气体静态介电常数与浓度的关系时,人们发现并不能看到完美的$\epsilon_\text{r} \propto N$关系,而是
\begin{equation}
    \frac{\epsilon_\text{r} - 1}{\epsilon_\text{r} + 2} = \frac{N \alpha}{3 \epsilon_0}.
    \label{eq:clausius-mossotti-relation}
\end{equation}
后来在测定介质中的非零频介电常数时也观察到了这个关系。

这个关系的推导具有非常明显(并且惊人地粗糙)的经典意义。设一个原子位于球腔中,腔体外是极化矢量为$\vb*{P}$的连续介质,则原子会感受到的电场——就是介质的自场——为
\[
    E_\text{self} = - \frac{\vb*{P}}{3 \epsilon_0}.
\]
于是我们有
\[
    \vb*{P} = N e \vb*{r} = N \alpha \left( \vb*{E}_\text{ext} + \frac{\vb*{P}}{3 \epsilon_0} \right),
\]

根据\eqref{eq:clausius-mossotti-relation}我们会发现介质的共振频率为
\begin{equation}
    \tilde{\omega}_0^2 = \omega_0^2 - \frac{1}{3} \omega_\text{p}^2, \quad \omega_\text{p}^2 = \frac{N e^2}{m \epsilon_0}.
\end{equation}
因此自场带来的修正会让介质的共振频率红移。

\subsection{关于介质中能量的一些讨论}\label{sec:medium-energy-lagrangian}

计算介质中的能量的时候会有一些微妙之处。例如,在处理物质部分时,假如偶极辐射近似成立,我们一般会使用如下的哈密顿量:
\begin{equation}
    H_{\text{particle under $\vb*{E}$}} = \underbrace{
        \sum_{i} \left(
            \frac{\vb*{p}_i^2}{2m_i} + V(\vb*{r}_i)
        \right) + \text{interaction between particles}
    }_{H_{\text{particle}}}
    \underbrace{
        - \sum_i \vb*{\mu}_i \cdot \vb*{E}
    }_{\text{dipole interaction}}.
\end{equation}
当然,在计算整个体系(包括介质和电磁场)中的能量的时候,还需要计入“纯粹属于电磁场”的能量或者说真空中电磁场的能量,即
\begin{equation}
    H_{\text{EM}} = \int \dd[3]{\vb*{r}} \left(
        \frac{1}{2} \epsilon_0 \vb*{E}^2 
        + \frac{1}{2 \mu_0} \vb*{B}^2
    \right).
\end{equation}
这似乎是说,在计算材料中的总能量的时候,我们不仅需要计算真空中电磁场的能量以及粒子的动能和势能(当然,这部分势能通常也来自电磁场,不过是来自已经被积掉,从而已经被归入“介质能量”的电磁场模式),
还需要计入偶极相互作用能量。然而实际上,我们常常基于某种类似于热力学的论证,声称体系内的总能量(包括纯电磁场能量或者说真空中电磁场能量加介质内的能量)的变化应当由下式(即变化外加电磁场时对整个体系做的功)给出:
\begin{equation}
    \var{E} = \vb*{E} \cdot \var{\vb*{D}}
\end{equation}
而将这个方程对$\vb*{D}$做积分,实际上计算出的介质能量是不包括相互作用能的。
这可以从以下的论证看出。我们知道$\vb*{D} = \epsilon_0 \vb*{E} + \vb*{P}$,我们有
\begin{equation}
    \Delta E = \int \var{\vb*{D}} \cdot \vb*{E} = \var \left(\frac{1}{2} \epsilon_0 \vb*{E}^2\right) 
    + \int \var{\vb*{P}} \cdot \vb*{E} ,
    \label{eq:delta-electric-energy-def}
\end{equation}
这个方程的第一项就是$H_{\text{EM}}$的变化,而这其中的第二项实际上是电场对粒子系统做的功,即粒子系统的能量的变化,这个变化是不包括偶极相互作用能的!
这可以从经典力学的直觉中看出,也可以通过如下的适用于量子力学的推导中看出:
\begin{equation}
    \begin{aligned}
        &\quad \dv{H_{\text{particle}}}{t} = \frac{1}{\ii \hbar} \comm*{H_{\text{particle}}}{H_{\text{particle}} + H_{\text{dipole}}}
        = \frac{1}{\ii \hbar} \sum_i \frac{1}{2 m_i} \comm*{\vb*{p}_i^2}{- \vb*{\mu}_i \cdot \vb*{E}}
        = \sum_i q \vb*{E} \cdot \frac{\vb*{p}_i}{m_i} \\
        &\Rightarrow \var{H_{\text{particle}}} = \var{\vb*{P}} \cdot \vb*{E}.
    \end{aligned}
\end{equation}
应当注意此处取微元的时候我们暂时假装$\vb*{E}$是常数;类似的在\eqref{eq:delta-electric-energy-def}中我们做积分时也忽略$\vb*{E}$的时间演化而把它当成一个单纯的符号。
于是,我们发现实际上\eqref{eq:delta-electric-energy-def}意味着
\begin{equation}
    \var{E} = \var{H_{\text{pure EM}}} + \var{H_{\text{particle}}},
\end{equation}
于是总能量似乎是真空中电磁场的能量加上粒子的能量,并不包括偶极相互作用能。

实际上,这是正确的。我们先考虑偶极相互作用的情况。
如果我们检查一下运动方程的话,就会发现如果我们选取辐射规范,从而$\vb*{E} = - \dot{\vb*{A}}$,如下的拉氏量总是正确的(暂时忽略粒子间相互作用):
\begin{equation}
    L = \int \dd[3]{\vb*{r}} \left(
        \frac{1}{2} \epsilon_0 \dot{\vb*{A}}^2 
        - \frac{1}{2 \mu_0} (\curl{\vb*{A}})^2 
    \right)
    + \sum_{i} \left(
        \frac{1}{2} m_i \dot{\vb*{x}}_i^2 - (V(\vb*{r}_i) + q \vb*{r}_i \cdot \dot{\vb*{A}})
    \right)
\end{equation}
那么我们会发现如果我们将电磁场看成给定的、“外来的”场,从而计算$\sum_i \vb*{p}_i \cdot \dot{\vb*{r}}_i - L$,我们就得到了\eqref{eq:delta-electric-energy-def},而如果我们将所有的自由度都考虑在内,从而计算$\sum_i \vb*{p}_i \cdot \dot{\vb*{r}}_i + \int \dd[3]{\vb*{r}} \vb*{p}_{\vb*{A}} \cdot \dot{\vb*{A}} - L$,我们得到了$H_{\text{particle}} + H_{\text{EM}}$,而粒子自由度和电磁场的耦合是通过对易关系得到的。
实际上,这后一种情况正是哈密顿量
\begin{equation}
    H = \underbrace{
        \sum_{i} \left(
        \frac{(\vb*{p}_i - q \vb*{A})^2}{2 m_i} + V(\vb*{r}_i)
        \right)
    }_{H_{\text{particle}}} + H_{\text{EM}}
\end{equation}
的意思:这里$\vb*{p}$是正则动量,它减去$q \vb*{A}$给出机械动量,从而哈密顿量中没有出现任何相互作用能。
有趣的是,如果我们反过来把粒子自由度当成一个给定的、不受$\vb*{E}$影响的“外场”,那么电磁场的哈密顿量中也会出现“相互作用能”;但总之正确的总能量确实就是电磁场能量加上粒子能量。

如果拉氏量中的耦合不是$\vb*{\mu} \cdot \vb*{E}$这么简单,经过前述计算会发现体系总能量除了真空中电磁场能量加上粒子能量,可能还是会多出来一些项;例如考虑$x E^2$这样的耦合,它如果出现在了拉氏量中,就会在哈密顿量中引入一个新的项。
在这种非常复杂的情况下,不容易凭借直觉就写出满足电动力学中要求的各种关系的$\vb*{P}$。不过如果我们已经成功地积掉了介质自由度,并且\emph{定义}$\vb*{D}$为
\begin{equation}
    \vb*{D} = \fdv{L}{\vb*{E}} = - \fdv{L}{\dot{\vb*{A}}},
\end{equation}
那么虽然$\var{H_{\text{particle}}} = \var{\vb*{P}} \cdot \vb*{E}$不再具有明确的物理意义,但是其推论
\begin{equation}
    \var{H_{\text{total}}} = \vb*{E} \cdot \var{\vb*{D}}
\end{equation}
却依照定义是成立的:在介质自由度被积掉后,由于
\begin{equation}
    H_{\text{total}} = \fdv{L}{\dot{\vb*{A}}} \cdot \dot{\vb*{A}} - L 
    = \underbrace{\fdv{L}{\vb*{E}}}_{\vb*{D}} \cdot \vb*{E} - L,
\end{equation}
如果我们小幅变动$\vb*{E}$,则有
\begin{equation}
    \var{H} = \var{\vb*{D}} \cdot \vb*{E} + \vb*{D} \cdot \var{\vb*{E}} - \var{L}
    = \var{\vb*{D}} \cdot \vb*{E} + \vb*{D} \cdot \var{\vb*{E}} - \underbrace{\pdv{\vb*{L}}{\vb*{E}}}_{\vb*{D}} \cdot \var{\vb*{E}}
    = \var{\vb*{D}} \cdot \vb*{E} .
\end{equation}
在偶极辐射近似确实成立而非线性性主要来自介质的情况下,上述公式甚至可以反过来,用来从$\vb*{D}$的形式推算出总能量$H$。
这样,在非线性系统中,如果我们想要看到(下面暂时忽略偏振,因为它和我们将要处理的哈密顿量中每一项前面的$a/b$形式的因子是无关的)
\begin{equation}
    D = \chi^{(n)} E^n,
\end{equation}
那么
\begin{equation}
    H^{(n)} = \int E \cdot \var{D} = 
    \int E \cdot n \chi^{(n)} E^{n-1}  \dd{E}
    = \frac{n}{n+1} \chi^{(n)} E^{n+1},
\end{equation}
于是做完勒让德变换(再次注意,$\vb*{E}$是$\vb*{A}$的导数)后,拉氏量中会有
\begin{equation}
    L^{(n)} = - \frac{1}{n+1} \chi^{(n)} E^{n+1}
    \label{eq:nonlinear-in-lagrangian}
\end{equation}
这样的一项。

我们看到,使用拉氏量处理非线性光学似乎比使用哈密顿量更加方便;在需要量子化的时候也是这样:从基于$x, p$的量子力学中获得的直觉出发,可能根本无法推导出来正确的哈密顿量(正如前面指出的),推导出来了,对易关系也很容易写错;\cite{hillery2009introduction}的第九章提到了这件事。

最后我们要指出,上面所谓“哈密顿量”或者“拉氏量”云云的意义在$\chi^{(n)}$没有频率依赖的情况下才是完全清楚的。
如果$\chi^{(n)}$这些量在我们关心的范围内有明显的频率依赖,那么如何积掉介质自由度——或者等价地说,如何建立一个只关于介质中的光子的量子光学理论而不显式讨论介质自由度——就不再是容易的了。
这里至少要考虑下面几个问题。
首先,怎样处理有频率依赖的有效理论:这个问题在凝聚态物理中建立Hubbard模型时也会存在,一种解决方案是引入一个玻色型辅助场。
其次,$\chi^{(n)}$有频率依赖则它必然有耗散,但还需要注意,这里所谓的耗散无非是光子转化为了介质中的激发,而介质中的激发最后又会因为自发发射而转化为光子而溢出样品(所谓耗散,总是出现在一个模式和比它多得多的模式发生耦合的情况下;但是介质相比于它周围的空间总是“小”的)。
我们分析整个过程会发现这个过程可以用光子的散射矩阵的描述,并且散射末态中光子不存在和介质的纠缠,从而在没有其他机制时,整个过程应该能够被一个\emph{纯态}理论描述。
这正是激光如何被导出激光器的描述:激光器的腔体存在“耗散”,于是光子溢出腔体,但是溢出腔体的光子是相干的。
另一方面,前述“耗散”在介质中存在退相干机制(如热声子)时可能变为通常意义上的耗散,从而要求我们用Lindbladian描述光子的动力学。
总之应当说介质中的量子光学仍然是一个有很多地方不清楚的问题。

\part{静态和准静态情况}

\chapter{静电学}

\section{静电系统的基本方程}

本节讨论仅含有导体和线性电介质的静电系统。所谓\concept{静电}指的是系统中没有任何电流的情况,此时我们有$\div{\vb*{j}}=0$,从而电荷密度分布没有变化。
注意我们没有使用“电荷密度没有变化”作为定义,因为恒稳电路也具有这样的性质,但是与此同时的确有电荷流动。
在静电的条件下我们有
\[
    \curl{\vb*{E}} = - \pdv{\vb*{B}}{t}, \quad \curl{\vb*{B}} = \mu_0 \epsilon_0 \pdv{\vb*{E}}{t},
\]
然后我们会发现磁场$\vb*{B}$满足一个波动方程。看起来这非常奇怪,因为在根本没有电流的时候磁场怎么会存在呢?
这实际上是来自边界条件的不清晰。在处理电磁波时我们并不要求无穷远处场强衰减至零,因为我们实际上是认为电磁波由非常远的地方的一个源产生的,没完没了地传播到其它地方,从而处理电磁波时“无穷远处”有源是完全可以的。
在静电学中我们要求电荷约束在一个有限的范围内,从而无穷远处场强快速衰减,那么磁场满足的波动方程如果要有非平凡解,只能取类似球面波的形式,但是这样一来$\div{\vb*{B}}$会不为零,和磁场无源的条件违背。
总之,在静电学情况下$\vb*{B}=0$。于是我们就有静电学方程
\begin{equation}
    \div{\vb*{E}} = \frac{\rho}{\epsilon_0}, \quad \curl{\vb*{E}} = 0,
    \label{eq:static-e-field}
\end{equation}
或者
\begin{equation}
    \vb*{E} = - \grad{\varphi}, \quad \laplacian{\varphi} = - \frac{\rho_0}{\epsilon_0}.
    \label{eq:static-phi-field}
\end{equation}
这是拉普拉斯方程,有现成的通解,即
\begin{equation}
    \varphi(\vb*{r}) = \int \dd[3]{\vb*{r}'} \frac{1}{4\pi \epsilon_0} \frac{\rho(\vb*{r}')}{\abs*{\vb*{r} - \vb*{r}'}}.
    \label{eq:from-q-to-phi}
\end{equation}

静电学中能量可以看成是电荷携带的而不是电场携带的。这是因为一个区域内的能量为
\[
    E = \int \dd[3]{\vb*{r}} \frac{1}{2} \epsilon_0 \vb*{E}^2 = \frac{1}{2} \epsilon_0 \int \dd[3]{\vb*{r}} (\grad{\varphi})^2,
\]
做分部积分,并使用无穷远处场强为零这一条件,得到
\begin{equation}
    E = - \frac{\epsilon_0}{2} \int \dd[3]{\vb*{r}} \varphi \laplacian{\varphi} = \int \dd[3]{\vb*{r}} \rho \varphi.
\end{equation}
因此在静电学的情况下能量可以认为是定域在电荷周围的。

总之,求解
\begin{equation}
    \varphi|_\text{surface} = \const,
\end{equation}
\begin{equation}
    \pdv{\varphi}{\vb*{n}} = - \frac{\sigma}{\epsilon}
\end{equation}

真空中或者均匀线性电介质中不能有电势极大值或者极小值,因为在这样的区域内$\varphi$是调和函数,而调和函数在它调和的区域内部不能有极大值、极小值。
物理上这很好理解,如极大值出现意味着从这一点向它周围的各个方向都有电场,因此这一点上应该有电荷,矛盾。

在计算静电系统中导体的受力时,不能简单地将无导体的空间内的电场外推到导体表面,然后使用$\vb*{f}=\sigma\vb*{E}$,因为导体表面电场是不连续的。
更加物理地看,这是因为导体表面实际上是非常复杂的一个系统:电场在微观层面快速衰减,表面上的电荷之间有相互作用力,数量级估计可以发现这些电荷之间的相互作用力和电荷受到的电场力是同阶的,因此简单的$\vb*{f}=\sigma\vb*{E}$会漏掉一部分作用力。
最为可靠的方法是使用麦克斯韦张量来计算,因为动量守恒是总是成立的,则在静电情况下动量流是连续的,所以直接计算导体外的麦克斯韦张量然后外推到导体表面即可。%
\footnote{
    这里还有一个可能的疑难:麦克斯韦张量计算的是电场对自由电荷的作用力,但是首先导体上的电荷并不是自由的,其次我们要计算的也是导体受到的作用力。
    但是,电磁场本身对导体并没有任何作用,而由受力平衡,导体对电荷施加的作用力应该和电场对电荷施加的作用力平衡,于是电场对电荷的作用力就传递给了导体。
}%

\section{唯一性定理}

唯一性定理成立的条件是$\vb*{D}$和$\vb*{E}$之间的关系应该是一一对应的。
反之,在两者之间的关系实际上不一一对应的时候,唯一性定理就被破坏了。例如,如果$\vb*{D}-\vb*{E}$关系实际上构成了一条电滞回线,那就没有唯一性定理。

% TODO: 有限大小的体系内电荷总量应该为零:这是高斯定理的推论

\[
    \int \dd[3]{\vb*{r}} (\varphi_2 \laplacian{\varphi_1} - \varphi_1 \laplacian{\varphi_2}) = \int \dd{\vb*{S}} (\varphi_2 \grad{\varphi_1} - \varphi_1 \grad{\varphi_2}),
\]
右边是
\begin{equation}
    \int \dd[3]{\vb*{r}} \varphi_1 \rho_2 = \int \dd[3]{\vb*{r}} \varphi_2 \rho_1.
\end{equation}
在导体系统中电荷仅仅分布在导体表面上,而且同一个导体表面电势处处相同,于是
\begin{equation}
    \sum_i \varphi_i^{(1)} q_i^{(2)} = \sum_i \varphi_i^{(2)} q_i^{(1)}.
\end{equation}

\section{电多极子}

设空间中的电荷密度为$\rho$,可能还要算上面密度,电势为
\[
    \varphi(\vb*{r}) = \frac{1}{4\pi \epsilon_0} \int \dd[3]{\vb*{r}'} \frac{1}{\abs*{\vb*{r} - \vb*{r}'}} \rho(\vb*{r}'),
\]
对$1/\abs*{\vb*{r}-\vb*{r}'}$做多极展开:
\[
    \frac{1}{\abs*{\vb*{r}-\vb*{r}'}} = \frac{1}{\abs*{\vb*{r}}} - \vb*{r}' \cdot \grad{\frac{1}{\abs*{\vb*{r}}}} + \frac{1}{2} \vb*{r}' \vb*{r}' : \grad{\grad{\frac{1}{\abs*{\vb*{r}}}}} + \cdots,
\]
就得到一个$\varphi$的展开式,即所谓\concept{多极展开},其中
\begin{equation}
    \varphi^{(0)}(\vb*{r}) = \frac{1}{4\pi \epsilon_0} \frac{1}{\abs*{\vb*{r}}} \underbrace{\int \dd[3]{\vb*{r}'} \rho(\vb*{r}')}_{Q}
\end{equation}
就是将整个体系当成一个点电荷计算得到的电势,
\begin{equation}
    \begin{aligned}
        \varphi^{(1)}(\vb*{r}) &= - \frac{1}{4\pi \epsilon_0} \grad{\frac{1}{\abs*{\vb*{r}}}} \cdot \int \dd[3]{\vb*{r}'} \rho(\vb*{r}') \vb*{r}' \\
        &= \frac{1}{4\pi \epsilon_0} \frac{\vb*{r}}{\abs*{\vb*{r}}^3} \cdot \underbrace{\int \dd[3]{\vb*{r}'} \rho(\vb*{r}') \vb*{r}'}_{\vb*{p}}
    \end{aligned}
\end{equation}
是电偶极电势,电四极矩是
\begin{equation}
    \varphi^{(2)}(\vb*{r}) = \frac{1}{4 \pi \epsilon_0} \frac{1}{6} \grad{\grad{\frac{1}{\abs*{\vb*{r}}}}} : \underbrace{3 \int \dd[3]{\vb*{r}'} \rho(\vb*{r}') \vb*{r}' \vb*{r}' }_{\vb*{D}}.
\end{equation}
在电荷分布相对于坐标系原点空间反演对称时,电偶极矩是零,而当电荷分布相对于坐标系原点空间反演反对称时,电四极矩是零。

容易看出,电四极矩$D_{ij}$是对称的,因此有6个独立分量。实际上这些独立分量并不都是有用的,注意到$\vb*{r} \neq 0$时
\[
    \laplacian{\frac{1}{\abs*{\vb*{r}}}} = 0,
\]
我们发现
\[
    \grad{\grad{\frac{1}{\abs*{\vb*{r}}}}} : \vb*{I} = 0,
\]
即我们可以任意地在$\vb*{D}$中加上单位张量的倍数,而不改变电势分布。因此我们可以手动加入一个约束:定义\concept{约化电四极矩}
\begin{equation}
    \tilde{\vb*{D}} = \vb*{D} - \frac{1}{3} \trace(\vb*{D}) \vb*{I} = \int \dd[3]{\vb*{r}'} (3 \vb*{r}' \vb*{r}' - \abs*{\vb*{r}'}^2 \vb*{I}) \rho(\vb*{r}') ,
\end{equation}
将$\vb*{D}$的迹消除掉,然后用$\tilde{\vb*{D}}$代替$\vb*{D}$同样可以得到正确的电四极矩;$\tilde{\vb*{D}}$独立的分量有5个,因为读多了一个无迹的条件。

电四极矩造成的电势衰减得比电偶极矩造成的电势快,电偶极矩造成的电势的衰减又比点电荷快。随着场点越来越接近源点,越来越复杂的电多极矩结构开始展现出来。

在$\vb*{r}$很大,即电场源离我们很远,其长度尺度趋于零时,我们需要将电场源替换成某种“点源”。我们不能朴素地将$\rho(\vb*{r})$替换成一个$\delta$函数,因为这样无法拿到电多极子的信息。正确的做法是
\begin{equation}
    \rho(\vb*{r}) = \delta(\vb*{r} - \vb*{r}_0) - \vb*{p} \cdot \grad \delta(\vb*{r} - \vb*{r}_0) + \cdots,
\end{equation}
即通过$\delta$函数的导数来引入“电场源的内部不均匀分布”的信息;通过分部积分法可以将导数转移到$1 / \abs*{\vb*{r}}$上,我们就能够拿到电多极子了。

\chapter{静磁学}

和静电学类似,我们可以考虑恒定电流的情况,即虽然有电流但是没有任何电荷变化,电流强度也不变的情况,则由输运方程有$\div{\vb*{j}}=0$。
我们可以直接引用\eqref{eq:wave-eq-general},得到
\[
    \frac{1}{c^2} \pdv[2]{\vb*{E}}{t} - \laplacian{\vb*{E}} = - \frac{1}{\epsilon_0} \grad{\rho} , \quad \frac{1}{c^2} \pdv[2]{\vb*{B}}{t} - \laplacian{\vb*{B}} = \mu_0 \curl{\vb*{j}},
\]
由于$\rho$和$\vb*{j}$都不随时间变化,如果我们像在静电学中一样,要求无穷远处场强衰减足够快,那么以上两式可以直接化为静态的拉普拉斯方程
\[
    \laplacian{\vb*{E}} = \frac{1}{\epsilon_0} \grad{\rho}, \quad \laplacian{\vb*{B}} = - \mu_0 \curl{\vb*{j}}.
\]
由于$\vb*{B}$不会变化,我们直接得到\eqref{eq:static-e-field},于是就可以求解出电场。
至于$\vb*{B}$,引入磁矢势,就得到
\[
    \curl{\laplacian{\vb*{B}}} = - \mu_0 \curl{\vb*{j}},
\]
那么只需要取库伦规范$\div{A}=0$就可以有
\[
    \div{\laplacian{\vb*{B}}} = - \mu_0 \div{\vb*{j}},
\]
于是就得到
\begin{equation}
    \vb*{B} = \curl{\vb*{A}}, \quad \laplacian{\vb*{A}} = - \mu_0 \vb*{j}.
\end{equation}
这个方程的形式和\eqref{eq:static-phi-field}非常相似,也是拉普拉斯方程,从而直接可以写出
\begin{equation}
    \vb*{A}(\vb*{r}) = \int \dd[3]{\vb*{r}'} \frac{\mu_0}{4\pi} \frac{\vb*{j}(\vb*{r}')}{\abs*{\vb*{r} - \vb*{r}'}}.
    \label{eq:from-j-to-a}
\end{equation}

总之,在电荷分布不变、电流分布不变的情况下,电场可以用$\rho$表示出来,并且是无旋场;磁场可以用$\vb*{j}$表示出来。在存在$\vb*{j}$的情况下,电荷分布不变,电场的形式和静电场完全一样,但磁场的存在会导致和静电学不同的一些物理现象,因此此时的电场可以称为\concept{恒定电场}。

静磁学指的是存在电荷流动,但是各个物理量的分布都恒稳的情况。要保持电流存在必须有一个外部的驱动力(\concept{非静电力}),这意味着此时的电磁场%
\footnote{
    当然,这个外部驱动力通常归根到底也是电磁力;但是我们将与它有关的那部分场自由度积掉了。
}%
不再是一个孤立体系。这可能让一些使用能量做的推导不再成立。%
\footnote{
    一种可能的诘难是,维持静电场的稳定也需要外部力(恩肖定理),为什么我们从来将静电场当成孤立系统看待?
    原因是,单纯从理论上说,要维持静电场稳定我们只需要将各个导体、电荷的动力学“关掉”即可(如认为点电荷受力不运动),等价的,维持静电场稳定的外力并不做功。
    另一方面,我们不能对电流做同样的事情:我们必须引入电流和电场之间的本构关系,从而自然地产生一个能量耗散项。
    静磁学理论中不可能不考虑这个能量耗散项。
}%

\begin{equation}
    \vb*{A}(\vb*{r}) = \frac{\mu_0}{4\pi} \frac{I \vb*{S} \times \vb*{r}}{\abs*{\vb*{r}}^3} = \frac{\mu_0}{4\pi} \frac{\vb*{m} \times \vb*{r}}{\abs*{\vb*{r}}^3}.
\end{equation}

\begin{equation}
    \vb*{B}(\vb*{r}) = - \frac{\mu_0}{4\pi} \left( \frac{\vb*{m} - 3 (\vb*{m} \cdot \vb*{e}_r) \vb*{e}_r}{r^3} \right).
\end{equation}

由于在边界上$\vb*{B}$有限大,应有
\begin{equation}
    \vb*{n} \times (\vb*{A}_2 - \vb*{A}_1) = 0.
\end{equation}
对库伦规范,

讨论静磁学系统的能量时需要把电源考虑进去,因为系统构型的小的变化会带来一个感生电动势,从而改变一些分布?

磁场的多极展开从$1$开始编号。

\chapter{似稳场和电路}

\section{基本方程}

\subsection{似稳条件}

本节开始我们讨论随时间发生变化的系统。
介质中的麦克斯韦方程看起来是时间平移不变的,当然,本构关系可以显含时间,因此它也许并不真的是时间平移不变的,但这种情况非常少见。
为方便起见我们经常求解\concept{时谐场},即假定$\vb*{E} \propto \ee^{- \ii \omega t}$,这相当于将场的时间部分切换到频域。
傅里叶变换意味着这当然不会丢失任何一般性。

当然,完整地解麦克斯韦方程是最精确的,但是很多情况下我们发现这类系统并没有特别明显的电磁辐射。
只要电场生磁场、磁场生电场,就可以有电磁辐射,因此电磁辐射不明显的系统中要么基本上没有电场产生的磁场,要么没有磁场产生的电场。 % TODO:,要么两者都有但是可以在感生电场和感生磁场之间建立直接的关系从而简化
这就是\concept{似稳场}或者\concept{准静态场}。体系中的电流被束缚在一些体积相对于电磁波波长不大的导体中的情况,也\concept{电路},经常可以用似稳场处理。
本节将主要讨论没有电场产生的磁场的情况,即忽略了位移电流的情况。
如果特殊需求,假定系统中的各个本构关系都是线性的。
此时麦克斯韦方程为
\begin{equation}
    \begin{bigcase}
        \div{\vb*{D}} &= \rho, \\
        \curl{\vb*{E}} &= - \pdv{\vb*{B}}{t}, \\
        \div{\vb*{B}} &= 0, \\
        \curl{\vb*{H}} &= \vb*{j}.
    \end{bigcase}
    \label{eq:quasi-stable-field}
\end{equation}

何时能够使用似稳场近似?对良导体,位移电流肯定要充分小,即
\[
    \pdv*{\vb*{D}}{t} \ll \sigma \vb*{E},
\]
即
\begin{equation}
    \omega \ll \omega_\sigma = \frac{\sigma}{\epsilon}.
    \label{eq:quasi-stable-field-cond-1}
\end{equation}
表面上看有这个条件就够了,但实际上这里有一个微妙的地方。在电场频率很大时,很多材料中电子在外场作用下不断“折返跑”,不会有宏观上的定向移动。此时的电流更像是束缚电流而不是传导电流,有
\[
    \curl{\vb*{H}} = \vb*{j} \sim \pdv{\vb*{E}}{t},
\]
从而又有了位移电流,因此电场实际的行为更像电磁波而不是似稳场。在频域上看,在$\omega$增大时,$\sigma$会有较大的、随着$\omega$变化的虚部,从而\eqref{eq:quasi-stable-field}的解和我们马上要看到的场的扩散方程(在其中$\sigma$就是一个常数)非常不同。
因此如果我们将直流电阻代入\eqref{eq:quasi-stable-field-cond-1}那这个判据太弱了。

对绝缘体,即在导电区域以外,显然只有$\omega=0$即完全静态的情况下才有\eqref{eq:quasi-stable-field-cond-1}成立。
然而,这仅仅意味着我们没有全空间的似稳场近似,并不意味着在系统的空间尺度较小时没有似稳场近似。
在绝缘体条件下,以$\epsilon$和$\mu$代替真空中的李纳-维谢尔势中的$\epsilon_0$和$\mu_0$,于是
\[
    \begin{aligned}
        \vb*{B}(\vb*{r}, t) &\approx \frac{\mu}{4\pi} \int \dd[3]{\vb*{r}'} \frac{\vb*{j}(\vb*{r}', t - R / c) \times \vb*{R}}{R^3} \\
        &= \frac{\mu}{4\pi} \int \dd[3]{\vb*{r}'} \frac{\vb*{j}(\vb*{r}', t) \ee^{- \ii \omega (t - R / c)} \times \vb*{R}}{R^3},
    \end{aligned}
\]
如果某一点的电流变化要瞬间传递到系统的各处,应有
\begin{equation}
    R \ll \frac{c}{\omega}.
\end{equation}
这当然是非常合理的:扰动基本上以光速传递,因此如果系统足够小,那么系统内一点的扰动总是可以快速传遍整个系统。

如果一个系统能够用似稳场分析,这通常意味着我们可以把系统看成某种电路:磁场可以写成电流的函数,电场可以分成两部分,一部分是磁场的变化率的函数,一部分是电荷的函数,而电流又正比于总电场,因此可以写出一个类似于“电流乘以电导率=由磁场变化导致的电场+电荷导致的电场+外加电场”这样的方程,这正好是电路的方程,其中考虑了四种效应:电阻、电感、电容、电源。
反之,则需要将系统看成某种传导电磁波的介质。

\subsection{场的扩散方程}

% TODO:似乎涉及电荷的重新分布等问题的情况不能用似稳场近似,因为在似稳场情况下电流散度为零

在系统中各处的本构关系都是空间均匀的情况下,经过大约为$\epsilon/\sigma$量级的时间,电荷密度为零(请注意这个结论和是否有外加场、外加场是否变化无关),因此大部分时候我们只需要求解
\[
    \begin{bigcase}
        \div{\vb*{D}} &= 0, \\
        \curl{\vb*{E}} &= - \mu \pdv{\vb*{H}}{t}, \\
        \div{\vb*{H}} &= 0, \\
        \curl{\vb*{H}} &= \sigma \vb*{E},
    \end{bigcase}
\]
从而
\begin{equation}
    \laplacian{\vb*{E}} = \mu \sigma \pdv{\vb*{E}}{t}, \quad \laplacian{\vb*{H}} = \mu \sigma \pdv{\vb*{H}}{t}.
\end{equation}
这就是\concept{场的扩散方程}。可以发现,对良好的导体,场的扩散反而是非常慢的,这是正确的,因为静电场中导体内部不应该有电场,因此在似稳场下导体内部的电场应该很弱,正好说明场的扩散很差。

在频域下,我们有
\begin{equation}
    \laplacian{\vb*{E}} = - \ii \omega \mu \sigma \vb*{E}, \quad \laplacian{\vb*{H}} = - \ii \omega \mu \sigma \vb*{H}.
    \label{eq:semi-stable-omega}
\end{equation}
如果$\sigma$有很大的虚部,以上方程的行为看起来就更像亥姆霍兹方程,从而在时域给出传递的波动。
在$\omega$很大时$\sigma$通常会有很大的虚部,因此此时似稳场不适用。

在似稳场确实适用的情况下,导体内部基本上没有场强分布,即出现\concept{趋肤效应}。
这可以通过在导体表面求解\eqref{eq:semi-stable-omega}看出。在一个无穷大平面边界上,设
\begin{equation}
    \vb*{E} = \vb*{E}_0 \ee^{-\alpha z},
    \label{eq:damping-surface-field}
\end{equation}
% TODO
趋肤深度为
\begin{equation}
    \delta = \sqrt{\frac{2}{\mu \omega \sigma}}.
\end{equation}
对理想导体,$\sigma \to \infty$,因此任何频率下电场都不会进入导体内部。
对实际导体,频率越高,趋肤效应越明显,但是当$\omega$继续增大以至于似稳场不再适用时,趋肤效应就消失了,此时的导体是透明的。

这里有一个看起来的佯谬:对有限大小的电导率,$\omega$很小时似乎有$\delta \to \infty$,也即,静电场可以直接穿透导体!
但是应当注意到一点:似稳场近似不仅包括了静电场的那些模式,也包括了\concept{恒定电场}即导体有稳定的、不随时间变化的电流输入的那些模式。
后者的确不存在任何趋肤效应:很容易验证,$\vb*{E}$在导体内处处均匀分布,指向同一个方向的模式是存在的,这里没有任何趋肤效应。
实际上从麦克斯韦方程可以看出,要产生场的扩散方程,感生电场是必须的(注意$\mu$出现在了场的扩散方程中),因此趋肤效应实际上是因为导体内部感应出的涡旋电场抵消了外加电场。这个机制和静电屏蔽是不一样的。

静电屏蔽实际上来自边界条件。设电场从绝缘体被打到导体上,用1标记绝缘体,2标记导体,有
\[
    \vb*{n} \cdot (\vb*{D}_2 - \vb*{D}_1) = \sigma, \quad \pdv{\rho}{t} = \vb*{n} \cdot \vb*{j},
\]
在频域下就有
\[
    \vb*{n} \cdot (\vb*{D}_2 - \vb*{D}_1) = \sigma, \quad - \ii \omega \sigma = \vb*{n} \cdot \vb*{j},
\]
于是
\[
    \vb*{n} \cdot \vb*{E}_2 = \frac{\vb*{n} \cdot \vb*{D}_1}{\epsilon-0 + \frac{\sigma}{\ii \omega}},
\]
在$\omega \to 0$时$\vb*{n} \cdot \vb*{E}_2$趋于零,即使趋肤深度很大,电场也不能进入导体。因此与静电屏蔽相关的电场衰减的尺度是微观尺度上的:在导体和绝缘体的交接层上电场已经衰减了;另一方面,趋肤效应的尺度虽然很小,但仍然是宏观的,它发生在导体体块内部,而不是导体和绝缘体的交接层上。
我们在这里只讨论了垂直于表面的电场,因为平行于表面的电场不会出现在静电场中,但是可以出现在恒定电场中;如果导体表面附近有平行于表面的电场,那么由$\vb*{n} \times (\vb*{E}_2 - \vb*{E}_1)=0$导体内部肯定也有平行于表面的电场,因此会有电流,就和“静电场”的条件冲突了。
总之,无论外界的静电场如何分布,导体内部根本不会有静电场,因此导体内部$\omega=0$的电场全部都是恒定电场。

还有另一个可能造成疑难的地方:我们知道导体可以远距离传输电流,从而必然可以远距离传输电场,但是上面的论证似乎是说,导体中的电场一定会快速衰减!
我们来分析一下远距离传输的电场可能出现在哪里,也即,持续的、远距离的稳定电流可能出现在哪里。
趋肤效应的推导是非常一般的,因此这样的电流只能出现在导体边界附近,并且在没有外加电流流入的地方一定平行于导体边界。
这些电流一定最后会撞上另一个导体边界,因为电流不可能在缺乏非静电力的导体内部环流。
因此后一个导体边界的边界条件中,电流主要分布在这个边界的边缘上,如对柱状导体,主要分布在柱子的上下底面的棱上。
这样的场构型让\eqref{eq:damping-surface-field}在这些表面上失效,因为此时电磁场在$z$方向和$x, y$方向上都有很大变化。
在静电学中只有静电力,我们不会产生源源不断的电流,也不会将这样的电流从某处导入导体中,因此从来不会激发电流——从而电场——远距离传输的模式。
在导体和绝缘体交界的界面上根本无所谓电流输入,同样不会激发电流和电场远距离传播的模式;在导体和绝缘体交界的界面上,边界条件\eqref{eq:damping-surface-field}是很好的近似,从而在这些界面附近总是能够看到趋肤效应。

\chapter{辐射}\label{chap:radiation}

我们已经讨论了电磁波的传播,本节则讨论电磁波是怎么产生的。

\section{单粒子的辐射}

\subsection{李纳-维谢尔势}

本节求解空间中有单个运动电荷时的电势和矢势分布情况,所得结果称为\concept{李纳-维谢尔势}。
实际上这就是在求解真空中电磁场的格林函数,但是只有在讨论辐射时这才有意义,因为只有此时我们需要确切地知道“电荷运动方式给定后电磁场的分布”。
其它时候,或是根本不需要让电荷动起来(静电学),或是电荷运动恒稳,磁场可以直接计算出来(静磁学),或是电荷的存在根本可以归入有效介电常数中(电磁波的传播)。
当然由于李纳-维谢尔势就是麦克斯韦方程的通解,对它做近似和多麦克斯韦方程做近似是一样的,所以我们其实也可以通过将李纳-维谢尔势做不同的近似,恢复出电磁波传播、似稳场近似、静电学等。

在取洛伦兹规范之后,我们需要求解
\begin{equation}
    \begin{bigcase}
        \laplacian{\varphi} - \frac{1}{c^2} \pdv[2]{\varphi}{t} &= - \frac{\rho(\vb*{r})}{\epsilon_0}, \\
        \laplacian{\vb*{A}} - \frac{1}{c^2} \pdv[2]{\vb*{A}}{t} &= - \mu_0 \vb*{j}(\vb*{r}).
    \end{bigcase}
\end{equation}
使用格林函数法,有(暂时先不引入无穷小虚部)
\[
    \begin{aligned}
        \varphi(\vb*{r}, t) &= \int \frac{\dd{\omega}}{2\pi} \int \frac{\dd[3]{\vb*{k}}}{(2\pi)^3} \frac{\ee^{\ii (\vb*{k} \cdot \vb*{r} - \omega t)}}{- \vb*{k}^2 + \omega^2/c^2} \left( - \frac{\rho(\vb*{k}, \omega)}{\epsilon_0} \right) \\
        &= \frac{1}{\epsilon_0} \int \frac{\dd{\omega}}{2\pi} \int \frac{\dd[3]{\vb*{k}}}{(2\pi)^3} \frac{\ee^{\ii (\vb*{k} \cdot \vb*{r} - \omega t)}}{\vb*{k}^2 - \omega^2/c^2} \int \dd[3]{\vb*{r}'} \int \dd{t'} \ee^{\ii (\omega t' - \vb*{k} \cdot \vb*{r}')} \rho(\vb*{r}', t') \\
        &= \frac{1}{\epsilon_0} \int \dd[3]{\vb*{r}'} \int \dd{t'} \rho(\vb*{r}', t') \int \frac{\dd{\omega}}{2\pi} \int \frac{\dd[3]{\vb*{k}}}{(2\pi)^3} \frac{\ee^{\ii (\vb*{k} \cdot (\vb*{r} - \vb*{r}') - \omega (t - t'))}}{\vb*{k}^2 - \omega^2/c^2}.
    \end{aligned}
\]
首先计算$\vb*{k}$部分的积分,有
\[
    \begin{aligned}
        \int \dd[3]{\vb*{k}} \frac{\ee^{\ii \vb*{k} \cdot \vb*{R}}}{\vb*{k}^2 - \omega^2/c^2} &= \int k^2 \sin \theta \dd{k} \dd{\theta} \dd{\varphi} \frac{\ee^{\ii k R \cos \theta}}{k^2 - \omega^2 / c^2} \\
        &= 2\pi \int_0^\infty \frac{k^2 \dd{k}}{k^2 - \omega^2 / c^2} \frac{1}{\ii k R} (\ee^{\ii k R} - \ee^{-\ii k R}) \\
        &= \frac{\pi}{\ii R} \int_{-\infty}^\infty \frac{k \dd{k}}{k^2 - \omega^2 / c^2} (\ee^{\ii k R} - \ee^{-\ii k R}) \\
        &= \frac{\pi}{2 \ii R} \int_{-\infty}^\infty \dd{k} \left( \frac{1}{k + \omega / c} + \frac{1}{k - \omega / c} \right) (\ee^{\ii k R} - \ee^{-\ii k R}).
    \end{aligned}
\]
此时必须在分母上加入无穷小虚部。按照关于$\omega$的零点必须在下半平面以保证因果性的原则,我们将$\omega$替换为$\omega + \ii 0^+$,并使用留数定理(注意$\ee^{\ii k R}$项应取上半平面极点而$\ee^{- \ii k R}$项应取下半平面极点)就得到
\[
    \int \dd[3]{\vb*{k}} \frac{\ee^{\ii \vb*{k} \cdot \vb*{R}}}{\vb*{k}^2 - \omega^2/c^2} = \frac{2 \pi^2}{R} \ee^{\ii \omega R / c}, 
\]
于是
\begin{equation}
    \begin{aligned}
        \varphi(\vb*{r}, t) &= \int \dd[3]{\vb*{r}'} \int \dd{t'} \int \frac{\dd{\omega}}{2\pi} \ee^{-\ii \omega (t-t')} \rho(\vb*{r}', t') \frac{1}{4\pi \epsilon_0} \frac{\ee^{\ii \omega R / c}}{R} \\
        &= \int \dd[3]{\vb*{r}'} \int \frac{\dd{\omega}}{2\pi} \ee^{- \ii \omega t} \rho(\vb*{r}', \omega) \frac{1}{4\pi \epsilon_0} \frac{\ee^{\ii \omega R / c}}{R}.
    \end{aligned}
\end{equation}
这个结果展示了一个出射波:从$\rho(\vb*{r}', t')$出发的向外传播的球面波,而不是向内聚集的波。
现在我们再做关于$\omega$的积分,会直接得到一个$\delta$函数:
\[
    \begin{aligned}
        \varphi(\vb*{r}, t) &= \int \dd[3]{\vb*{r}'} \int \dd{t'} \int \frac{\dd{\omega}}{2\pi} \ee^{-\ii \omega (t-t')} \rho(\vb*{r}', t') \frac{1}{4\pi \epsilon_0} \frac{\ee^{\ii \omega R / c}}{R} \\
        &= \int \dd[3]{\vb*{r}'} \int \dd{t'} \delta(R/c + t' - t) \rho(\vb*{r}', t') \frac{1}{4\pi \epsilon_0 R} \\
        &= \int \dd[3]{\vb*{r}'} \frac{1}{4\pi \epsilon_0} \frac{\rho(\vb*{r}', t - R / c)}{R}.
    \end{aligned}
\]
同样的操作也可以对$\vb*{A}$和$\vb*{j}$做,最终得到
\begin{equation}
    \begin{bigcase}
        \varphi(\vb*{r}, t) &= \int \dd[3]{\vb*{r}'} \frac{1}{4\pi \epsilon_0} \frac{\rho(\vb*{r}', t - R / c)}{R}, \\
        \vb*{A}(\vb*{r}, t) &= \int \dd[3]{\vb*{r}'} \frac{\mu_0}{4\pi} \frac{\vb*{j}(\vb*{r}', t - R / c)}{R}.
    \end{bigcase}
    \label{eq:general-solution-wave}
\end{equation}
标势的形式和静电场一致,矢势的形式和静磁场一致,只不过出现了一个时间推迟。
我们经常把这样有时间推迟的量放在中括号里,即
\[
    \rho(\vb*{r}, t) = \int \dd[3]{\vb*{r}'} \frac{1}{4\pi \epsilon_0} \frac{[\rho]}{R},
\]
等等。

当空间中只有一个电荷时,有
\[
    \rho(\vb*{r}, t) = q \delta(\vb*{r} - \vb*{r}_0(t)), \quad \vb*{j}(\vb*{r}, t) = q \dot{\vb*{r}}_0(t) \delta(\vb*{r} - \vb*{r}_0(t)),
\]
其中$\vb*{r}_0 = \vb*{r}_0(t)$是该电荷的运动轨迹。代入\eqref{eq:general-solution-wave},有
\[
    \varphi(\vb*{r}, t) = \int \dd[3]{\vb*{r}'} \frac{1}{4\pi \epsilon_0} \frac{q \delta(\vb*{r}' - \vb*{r}_0(t - R / c))}{R},
\]
因此只有满足
\begin{equation}
    \vb*{r}' = \vb*{r}_0(t - R/c)
    \label{eq:retarded-position-original}
\end{equation}
的部分才有贡献。但是要注意,$\vb*{r}'$同时也出现在$R$中,因此积分时不能仅仅将$\vb*{r}'$替换为$\vb*{r}_0(t-R/c)$,还需要做一个积分测度的变换。
我们有
\[
    \grad_{\vb*{r}'} {(\vb*{r}' - \vb*{r}_0(t - R/c))} = \vb*{I} - \frac{\vb*{R}}{cR} \dot{\vb*{r}_0}(t-R/c) ,
\]
于是
\[
    \det(\grad_{\vb*{r}'} {(\vb*{r}' - \vb*{r}_0(t - R/c))}) = 1 - \frac{\vb*{R}}{cR} \cdot \dot{\vb*{r}}_0(t-R/c),
\]
从而
\[
    \begin{aligned}
        \varphi(\vb*{r}, t) &= \int \dd[3]{\vb*{r}'} \frac{1}{4\pi \epsilon_0} \frac{q \delta(\vb*{r}' - \vb*{r}_0(t - R / c))}{R} \\
        &= \frac{1}{4\pi \epsilon_0} \eval{\frac{1}{\det(\grad_{\vb*{r}'} {(\vb*{r}' - \vb*{r}_0(t - R/c))})} \frac{q}{R}}_{\vb*{r}' = \vb*{r}_0(t - R/c)} \\
        &= \frac{1}{4\pi \epsilon_0} \eval{\frac{q}{R - \frac{\vb*{R} \cdot \dot{\vb*{r}}_0(t-R/c)}{c}}}_{\vb*{r}' = \vb*{r}_0(t - R/c)}.
    \end{aligned}
\]
用$\vb*{v}$表示粒子的运动速度,就有
\begin{equation}
    \varphi(\vb*{r}, t) = \frac{1}{4\pi \epsilon_0} \frac{q}{R' - \frac{\vb*{R}' \cdot \vb*{v}'}{c}},
    \label{eq:retarded-phi}
\end{equation}
类似的
\begin{equation}
    \vb*{A}(\vb*{r}, t) = \frac{\mu_0}{4\pi} \frac{q \vb*{v}'}{R' - \frac{\vb*{R}' \cdot \vb*{v}'}{c}},
    \label{eq:retarded-a}
\end{equation}
其中$R'$和$\vb*{v}'$均为$t'$时刻的$R$和$\vb*{v}$而$t'$由
\begin{equation}
    R(t') = \abs*{\vb*{r} - \vb*{r}_0(t')} = c(t-t')
    \label{eq:retarded-time}
\end{equation}
确定。这个方程看起来非常合理,我们将$\rho$有速度地出现在某个地方当成一个事件,它传递到$\vb*{r}$必然存在时间延迟,事件传播的速度就是光速,在$t$时刻,$\vb*{r}$点看到的$\vb*{r}_0$处的情况是$t'$时刻的,两者之差为
\[
    t - t' = \frac{\abs*{\vb*{r} - \vb*{r}_0(t')}}{c},
\]
就得到\eqref{eq:retarded-time}。

前面$\delta$函数的积分改变了积分测度,让它比通常的要大一些。这看起来似乎有些奇怪,因为狭义相对论中似乎应该有尺缩效应,积分测度应该缩小。
这里的关键在于运动电荷对某一点的电场的贡献涉及的空间积分应该体现的是“在这一点看到的运动物体的长度”(在静止参考系看到的物体两端传来的信号可能来自不同时刻)而不是“试图在静止参考系中测量得到的运动物体的长度”(测量时物体两端到达观察点的用时是一样的)。

\subsection{辐射的多极展开}

\eqref{eq:general-solution-wave}可以做多极展开,所得结果和静电场、静磁场完全一致,仅有的区别在于$\varphi$和$\vb*{j}$是推迟的。

\subsubsection{电偶极辐射}\label{sec:electric-dipole-radiation}

设系统中的电荷分布主要体现为偶极子,对$\varphi$,零阶项为
\[
    \varphi^{(0)}(\vb*{r}, t) = \frac{1}{4\pi \epsilon_0} \frac{1}{\abs*{\vb*{r}}} \int \dd[3]{\vb*{r}'} \rho(\vb*{r}', t - R / c) = \frac{1}{4\pi \epsilon_0} \frac{Q}{r},
\]
不随时间变化,没有辐射。一阶项
\[
    \begin{aligned}
        \varphi^{(1)}(\vb*{r}, t) &= - \frac{1}{4\pi \epsilon_0} \grad{\frac{1}{\abs*{\vb*{r}}}} \cdot \int \dd[3]{\vb*{r}'} \rho(\vb*{r}') \vb*{r}' \\
        &= - \frac{1}{4\pi \epsilon_0} \div{\int \dd[3]{\vb*{r}'} \frac{\rho(\vb*{r}', t-R/c) \vb*{r}'}{\abs*{\vb*{r}}}},
    \end{aligned}
\]
设$\vb*{p}$为总偶极矩,则
\begin{equation}
    \varphi^{(1)}(\vb*{r}, t) = - \div{\frac{[\vb*{p}]}{4\pi \epsilon_0 r}}. 
\end{equation}
对磁矢势,有
\[
    \begin{aligned}
        \vb*{A}^{(1)}(\vb*{r}, t) &= \frac{\mu_0}{4\pi} \int \dd[3]{\vb*{r}'} \frac{\vb*{j}(\vb*{r}', t - R/c)}{\abs*{\vb*{r}}} \\
        &= \frac{\mu_0}{4\pi} \int \dd[3]{\vb*{r}'} \frac{[\rho][\vb*{v}]}{\abs*{\vb*{r}}} \\
        &= \frac{\mu_0}{4\pi} \dv{(t-R/c)} \int \dd[3]{\vb*{r}'} \frac{[\rho][\vb*{r}']}{\abs*{\vb*{r}}}.
    \end{aligned}
\]
最后一个等号需要解释一下。我们可以将电荷分布离散化,从而
\[
    \int \dd[3]{\vb*{r}'} \rho(\vb*{r}') \vb*{v}(\vb*{r}') = \sum_i e \vb*{v}_i = \dv{t} \sum_i e \vb*{r}_i = \dv{t} \int \dd[3]{\vb*{r}'} \rho(\vb*{r}') \vb*{r}',
\]
然后将$t$换成$t-R/c$即可。实际上,设$\phi$为某个守恒量的密度,那么一定有
\[
    \dv{t} \int \dd[3]{\vb*{r}} \phi \psi = \int \dd[3]{\vb*{r}} \phi \dv{\psi}{t},
\]
这个方程可以直接从连续性方程推得。于是我们有
\begin{equation}
    \vb*{A}^{(1)}(\vb*{r}, t) = \frac{\mu_0}{4\pi} \frac{[\dot{\vb*{p}}]}{r}.
\end{equation}
这里$[\dot{f}]$表示先让$f$对$t$求导,然后用$t-R/c$代替$t$,或者说让$f(t-R/c)$对$t-R/c$求导。

现在我们还是可以一如既往地开始讨论时谐场,此时只需要认为电偶极子在做周期性振动即可。
这种振动当然会消耗能量,但是我们暂时先假定有某些外加能量输入让电偶极子持续振荡。

先计算出磁场,然后计算出电场比较方便。
\begin{equation}
    \vb*{B} = \frac{\mu_0 \omega^2}{4\pi c r} \vb*{e}_r \times [\vb*{p}].
\end{equation}

\section{电路的辐射}


\part{线性介质中光的传播}

\chapter{线性极化介质与经典麦克斯韦方程}

本章考虑无源的、线性极化的介质中电磁波如何传播。

\section{线性介质假设到光学方程}\label{sec:linear-material-to-optical-eq}

\subsection{线性本构关系的一般形式}

现在我们认为不存在自由电流和自由电荷,从而\eqref{eq:maxwell-material}成为
\begin{equation}
    \begin{bigcase}
        \div{\vb*{D}} &= 0, \\
        \curl{\vb*{E}} &= - \pdv{\vb*{B}}{t}, \\
        \div{\vb*{B}} &= 0, \\
        \curl{\vb*{H}} &= \pdv{\vb*{D}}{t}
    \end{bigcase}
    \label{eq:no-free-charge}
\end{equation}
并且假定所有本构关系都是线性的(这里由于自由电荷被认为是零,不再需要考虑关于$\vb*{j}_\text{f}$的本构关系)。现在我们考虑一个比较一般的本构关系
\begin{equation}
    \pmqty{\vb*{D} \\ \vb*{B}} = \pmqty{\epsilon & \xi \\ \eta & \mu} \pmqty{\vb*{E} \\ \vb*{H}} \eqqcolon K \pmqty{\vb*{E} \\ \vb*{H}}.
\end{equation}
其中的$K$矩阵的每个分量实际上都是张量——实际上是张量算符,作用在“单光子波函数”上的张量算符——它们没有时间演化但是可以有空间变化。
我们总是把$K$表示成
\begin{equation}
    \pmqty{\vb*{D} \\ \vb*{B}} (\vb*{r}, t) = \int \dd[3]{\vb*{r}'} \int \dd{t'} K(\vb*{r}, t; \vb*{r}', t') \pmqty{\vb*{E} \\ \vb*{H}}(\vb*{r}' , t').
\end{equation}
很多体系中,电场和磁场不会通过本构关系耦合,即此时$\xi = \eta = 0$。

大部分情况下我们可以要求本构关系是空间、时间平移不变的,即
\begin{equation}
    \pmqty{\vb*{D} \\ \vb*{B}} (\vb*{r}, t) = \int \dd[3]{\vb*{r}'} \int \dd{t'} K(\vb*{r} - \vb*{r}', t - t') \pmqty{\vb*{E} \\ \vb*{H}} (\vb*{r}', t').
\end{equation}
此时做傅里叶变换切换到四维动量空间中,能够写出非常简洁的
\begin{equation}
    \pmqty{\vb*{D} \\ \vb*{B}} (\vb*{k}, \omega) = K(\vb*{k}, \omega) \pmqty{\vb*{E} \\ \vb*{H}} (\vb*{k}, \omega).
\end{equation}
这是正确的,因为“单光子理论”具有空间平移不变性意味着单光子动量不变,“单光子理论”具有时间平移不变性意味着单光子能量不变,或者换句话说就是波数和频率都不变。
空间平移不变性的破缺可能是因为介质本身就是不均匀的,而时间平移不变性无法被介质本身的性质破缺。
不过也有一种可能,就是介质中存在非常强的、作为背景的泵浦光,它的周期性起伏可以同时破缺空间平移不变性和时间平移不变性。
此时我们有
\begin{equation}
    \pmqty{\vb*{D} \\ \vb*{B}} (\vb*{k}, \omega) = \int \dd[3]{\vb*{k}'} \int \dd{t} K(\vb*{k}, \omega; \vb*{k}', \omega') \pmqty{\vb*{E} \\ \vb*{H}} (\vb*{k}', \omega'),
\end{equation}
由于将泵浦光和我们正在观察的那部分电磁波作为一个整体,空间和时间平移不变性还是能够保持的,上式左边的非零分量满足
\[
    (\omega, \vb*{k}) = (\omega', \vb*{k}') \pm (\omega_\text{pump}, \vb*{k}_\text{pump}), 
\]
直观地说就是泵浦光向我们正在观察的电磁波注入了动量和能量,或是我们正在观察的电磁波向泵浦光注入了动量和能量。
关于泵浦光具体是怎样和其它模式的光耦合的,\autoref{sec:input-pump-seed-2nd}是一个典型的例子。

另一种常见的对本构关系的约束是,本构关系在空间或是时间上是\emph{局域的},也即,我们有
\begin{equation}
    \pmqty{\vb*{D} \\ \vb*{B}} (\vb*{r}, t) = K(\vb*{r}, t) \pmqty{\vb*{E} \\ \vb*{H}}(\vb*{r} , t).
\end{equation}
这个限制体现在四维动量空间上,就是
\begin{equation}
    \pmqty{\vb*{D} \\ \vb*{B}} (\vb*{k}, \omega) = \int \dd[3]{\vb*{k}'} \int \dd{\omega'} K(\vb*{k} - \vb*{k}', \omega - \omega') \pmqty{\vb*{E} \\ \vb*{H}}(\vb*{k}' , \omega').
\end{equation}

如果本构关系同时是平移不变的和局域的,那么它就只能够是一个在空间和时间上均不变动的常数。
如果它在时间上平移不变且局域但是空间上未必,那么它就可以有空间依赖但是不能有时间依赖,也即,可以有动量依赖但是不能有频率依赖。
因此,在已知本构关系在某个坐标上平移不变时,如果本构关系在这个坐标的动量空间上不是常数,那么该本构关系在这个坐标上就不是局域的。
因此在空间平移不变性保留时,我们可以将$K$对$\vb*{k}$的依赖称为它的\emph{空间非局域性},类似的有时间非局域性。
这个术语有一些迷惑性,因为也有一些文献将泵浦光引入的空间平移不变性的违反称为非局域性。

\subsection{对称性}

设电场$\vb*{E}$的时间反演为$\vb*{E}'$,即
\begin{equation}
    \vb*{E}'(t) = \vb*{E}(-t),
\end{equation}
那么我们会发现
\[
    \begin{aligned}
        \vb*{E}'(\omega) &= \int \dd{t} \ee^{\ii \omega t} \vb*{E}'(t) = \int \dd{t} \ee^{\ii \omega t} \vb*{E}(-t) \\
        &=  \int_{-\infty}^\infty \dd{t} \ee^{\ii \omega t} \vb*{E}(-t) \\
        &= \int_{-\infty}^\infty \dd{t} \ee^{- \ii \omega t} \vb*{E}(t)  = \vb*{E}(- \omega) = \vb*{E}^*(\omega).
    \end{aligned}
\]
因此,在时间反演下,
\begin{equation}
    \vb*{E}(\omega) \longrightarrow T \vb*{E}(\omega) = \vb*{E}'(\omega) = \vb*{E}(- \omega) = \vb*{E}^*(\omega),
\end{equation}
即时间反演就是取复共轭。

对磁场做时间反演是类似的,不过磁场在时间反演变换下会变号,% TODO
因此我们有
\begin{equation}
    \vb*{B}(\omega) \longrightarrow T \vb*{B}(\omega) = - \vb*{B}'(\omega) = -\vb*{B}(- \omega) = - \vb*{B}^*(\omega),
\end{equation}

在时间反演变换下我们发现
\begin{equation}
    T \epsilon(\omega) = \exp^*(\omega), \quad T \mu(\omega) = \mu^*(\omega), \quad T \xi(\omega) = - \xi^*(\omega), \quad T \eta(\omega) = - \eta^*(\omega).
\end{equation}

在时间演化下$\epsilon$要取复共轭这件事实际上有很明显的物理意义:如\autoref{sec:light-propagate}中我们会看到的那样,$\epsilon$的虚部意味着光在介质中存在衰减;实际上在\eqref{eq:ohm-dissipation-maxwell}中我们看到对欧姆定律成立的介质,$\epsilon$的虚部和来自电场激发出的电流产生的发热耗散掉的能量。
如果一个过程中光受到\emph{阻尼},其时间反演中光应该受到\emph{增益}。
这就是$\epsilon$取复共轭的意义。

\subsection{频域方程}

\begin{equation}
    \begin{aligned}
        \vb*{D}(\vb*{r}, t) &= \int K_E(\vb*{r} - \vb*{r}', t - t') \vb*{E}(\vb*{r}', t') \dd^3 \vb*{r}' \dd t', \\
        \vb*{H}(\vb*{r}, t) &= \int K_B(\vb*{r} - \vb*{r}', t - t') \vb*{B}(\vb*{r}', t') \dd^3 \vb*{r}' \dd t'.
    \end{aligned}
\end{equation}

大部分情况下我们可以要求本构关系是局部的,则应有
\[
    \begin{aligned}
        \vb*{D}(\vb*{r}, t) &= \int K_E(\vb*{r}, t-t') \vb*{E}(\vb*{r}, t') \dd t', \\
        \vb*{H}(\vb*{r}, t) &= \int K_B(\vb*{r}, t-t') \vb*{B}(\vb*{r}, t') \dd t',
    \end{aligned}
\]
这是一个时间上的卷积运算。由傅里叶变换,我们只需要讨论平面波的本构关系就可以了,因为其它所有形式的场都能够写成平面波的叠加,既然\eqref{eq:no-free-charge}是线性齐次的(这就是假定没有自由电荷的好处!)。于是,不失一般性的,假设所有的场都取
\[
    \vb*{A}(\vb*{r}, \omega) \ee^{-\ii \omega t}
\]
的形式,其中$\vb*{A}(\vb*{r})$可以有虚部,但是一定能够写成某个实矢量乘以$\ee^{\ii \phi}$的形式。
在这个条件下,可能的$\vb*{A}$并不能覆盖整个$\reals^3 \to \complexes^3$,但这是正确的,因为允许$\vb*{A}$有虚部是为了表示$\vb*{A}_0 \cos (\omega t + \phi)$这样的函数,而这一点在限制$\vb*{A}(\vb*{r})$取某个实矢量乘以$\ee^{\ii \phi}$的形式时足以满足。归根到底$\vb*{E}$等都是实场,其傅里叶分量相比一个复场肯定是高度受限的。(这件事我们在\qftdoc中会看到,在那里,这个事实体现为实场量子化后产生一种粒子,而复场量子化后产生两种互为反粒子的粒子)
于是,本构关系就是
\begin{equation}
    \vb*{D}(\vb*{r}) = K_E(\vb*{r}, \omega) \vb*{E}(\vb*{r}), \quad \vb*{H}(\vb*{r}) = K_B(\vb*{r}, \omega) \vb*{B}(\vb*{r})
    \label{eq:linear-constitutive}
\end{equation}
的本构关系就可以了。\eqref{eq:linear-constitutive}中的所有变量可以取它们完整的表达式,也可以取它们的振幅。注意这些量可以是复数(用以表示相位)。
此时的麦克斯韦方程组成为
\begin{equation}
    \begin{bigcase}
        \div{\vb*{D}} &= 0, \\
        \curl{\vb*{E}} &= \ii \omega \vb*{B}, \\
        \div{\vb*{B}} &= 0, \\
        \curl{\vb*{H}} &= - \ii \omega \vb*{D}
    \end{bigcase}
    \label{eq:sin-wave-eqs}
\end{equation}

如果有自由电流和自由电荷,但是自由电流正比于电场,即服从\concept{欧姆定律},那么我们可以将它们吸收进$\vb*{D}$和$\vb*{H}$中;这件事在\autoref{sec:classical-continuum}中我们提到过,不过下面还是详细地展示具体怎么做计算。
同样不失一般性地设所有的场都可以写成一个复振幅乘以$\ee^{- \ii \omega t}$的形式。
使用导出$\vb*{D}$和$\vb*{E}$的关系、$\vb*{H}$和$\vb*{B}$的关系同样的方法,有
\begin{equation}
    \vb*{j}_\text{f}(\vb*{r}) = \sigma(\vb*{r}, \omega) \vb*{E}(\vb*{r})
    \label{eq:ohm-law-sin-wave}
\end{equation}
那么从\eqref{eq:maxwell-material}和\eqref{eq:transportation}可以得到
\[
    \begin{bigcase}
        \div{\vb*{D}} &= \rho_\text{f}, \\
        \curl{\vb*{E}} &= \ii \omega \vb*{B}, \\
        \div{\vb*{B}} &= 0, \\
        \curl{\vb*{H}} &= \vb*{j}_\text{f} - \ii \omega \vb*{D}, \\
        - \ii \omega \rho_\text{f} + \div{\vb*{j}_\text{f}} &= 0,
    \end{bigcase}
\]
我们希望能够消去$\rho_\text{f}$和$\vb*{j}_\text{f}$,为此反复使用输运方程得到
\[
    \begin{bigcase}
        \div{\left( \vb*{D} - \frac{\vb*{j}_\text{f}}{\ii \omega} \right)   } &= 0, \\
        \curl{\vb*{E}} &= \ii \omega \vb*{B}, \\
        \curl{\vb*{H}} &=  - \ii \omega \left( \vb*{D} - \frac{\vb*{j}_\text{f}}{\ii \omega} \right)
    \end{bigcase}
\]
代入本构关系\eqref{eq:ohm-law-sin-wave}得到
\begin{equation}
    \begin{bigcase}
        \div{\left( \vb*{D} - \frac{\sigma \vb*{E}}{\ii \omega} \right)   } &= 0, \\
        \curl{\vb*{E}} &= \ii \omega \vb*{B}, \\
        \curl{\vb*{H}} &=  - \ii \omega \left( \vb*{D} - \frac{\sigma \vb*{E}}{\ii \omega} \right)
    \end{bigcase}
    \label{eq:ohm-dissipation-maxwell}
\end{equation}
注意到$\vb*{D}$和$\vb*{E}$、$\vb*{H}$和$\vb*{B}$之间的关系(见\eqref{eq:linear-constitutive}),做下面的替换
\[
    \vb*{D} - \frac{\sigma \vb*{E}}{\ii \omega} \longrightarrow \vb*{D}, \quad K_E - \frac{\sigma}{\ii \omega} \longrightarrow K_E
\]
得到的方程和\eqref{eq:sin-wave-eqs}形式完全一致,本构关系的形式也还是\eqref{eq:linear-constitutive}。
唯一的不同是,$K_E$和$K_B$都是复数,并且
\begin{equation}
    \Im K_E = \frac{\sigma}{\omega}.
    \label{eq:linear-conductor-imaginary-part}
\end{equation}

\subsection{各向同性介质中的亥姆霍兹方程}

我们特别感兴趣的是\eqref{eq:linear-constitutive}中$K_E$和$K_B$都是标量(不一定是实数)的情况,
也就是说,\eqref{eq:linear-constitutive}中给出的响应是\concept{各向同性}的。
此时有
\begin{equation}
    \vb*{D}(\vb*{r}) = \epsilon(\vb*{r}, \omega) \vb*{E}(\vb*{r}), \quad \vb*{H}(\vb*{r}) = \frac{1}{\mu(\vb*{r}, \omega)} \vb*{B}(\vb*{r})
    \label{eq:scalar-constitutive}
\end{equation}
第二个公式的比例系数特意被放到了分母中,以达到和\eqref{eq:original-maxwell}相同的形式。
将\eqref{eq:scalar-constitutive}代入\eqref{eq:sin-wave-eqs}中,得到
\begin{equation}
    \begin{bigcase}
        \div{(\epsilon \vb*{E})} &= 0, \\
        \curl{\vb*{E}} &= \ii \omega \vb*{B}, \\
        \div{\vb*{B}} &= 0, \\
        \curl{\left(\frac{\vb*{B}}{\mu}\right)} &= - \ii \omega \epsilon \vb*{E}
    \end{bigcase}
    \label{eq:scalar-cons-maxwell-e-and-b}
\end{equation}
上式中第三式可以通过第二式推导而来;将第二式表示出的$\vb*{B}$代入第四式可以消去$\vb*{B}$。于是将\eqref{eq:scalar-cons-maxwell-e-and-b}简化为
\begin{equation}
    \begin{bigcase}
        \div{(\epsilon \vb*{E})} &= 0, \\
        \curl{ \left(\frac{1}{\mu} \curl{\vb*{E}} \right) } &= \omega^2 \epsilon \vb*{E}, \\
        \curl{\vb*{E}} &= \ii \omega \vb*{B}
    \end{bigcase}
    \label{eq:e-in-material}
\end{equation}
这个方程关于电场的部分等价于
\begin{equation}
    \begin{bigcase}
        \div{(\epsilon \vb*{E})} &= 0, \\
        \laplacian \vb*{E} + \mu \epsilon \omega^2 \vb*{E} &= - \frac{1}{\epsilon} \grad{(\vb*{E} \cdot \grad{\epsilon})} - \frac{1}{\mu} \grad{\mu} \cross (\curl{\vb*{E}})
    \end{bigcase}
    \label{eq:e-in-material-e-only}
\end{equation}
以上方程是基于$\vb*{E}$的,也可以写出基于$\vb*{H}$的类似的方程,遵从同样的步骤,可以得到
\begin{equation}
    \begin{bigcase}
        \div{\mu \vb*{H}} &= 0, \\
        \curl{\left( \frac{1}{\epsilon} \curl{\vb*{H}} \right)} &=  \mu \omega^2 \vb*{H}, \\
        \curl{\vb*{E}} &= \ii \omega \mu \vb*{H}.
    \end{bigcase}
    \label{eq:h-in-material}
\end{equation}
\eqref{eq:e-in-material}或\eqref{eq:h-in-material}称为\concept{主方程}。
\eqref{eq:e-in-material-e-only}的第二条方程有三组解,两组横波解,一组纵波解。
由于我们讨论的介质中没有自由电荷和自由电流,纵波解是不应该出现的,它被\eqref{eq:e-in-material-e-only}的第一条方程禁止。
纵波解是必然要出现的,因为涉及$\vb*{E}$的单个波动方程有三个偏振方向;横波条件需要另一个方程引入。

特别的,在$\epsilon, \mu$在空间中处处相等或者变化得比较缓慢时,有
\begin{equation}
    \laplacian \vb*{E} + \frac{\omega^2}{c^2} \vb*{E} = 0, \quad c^2 = \frac{1}{\epsilon \mu}
    \label{eq:halmholtz-eq}
\end{equation}
\eqref{eq:halmholtz-eq}就是所谓的\concept{亥姆霍兹方程},求解出它就求解出了整个\eqref{eq:e-in-material}。
需要注意的是并不是所有\eqref{eq:halmholtz-eq}的解都是\eqref{eq:e-in-material}的解,因为\eqref{eq:halmholtz-eq}没有包含\eqref{eq:e-in-material}的第一式。当然这个信息可以在求解\eqref{eq:halmholtz-eq}的时候使用一些边界条件加上去。
这里占用了$c$表示介质中光速,相对应地,设真空中光速为$c_0$,
其值为$1/\sqrt{\epsilon_0 \mu_0}$,因为简单地令$\epsilon=\epsilon_0$,$\mu = \mu_0$,
\eqref{eq:scalar-cons-maxwell-e-and-b}就退化到了真空的情况。

需要注意的是由于$\mu$和$\epsilon$可能有虚部(对应着吸收等情况),$c^2$不一定是实数。这就产生了一个问题:开根号在复平面上是多值的,那么$c$应该取哪一个值呢?

当$c$在某区域中基本上可以看成常的实数时,它就对应着平面波传播的速度。\eqref{eq:halmholtz-eq}有下面的平面波解
\begin{equation}
    \vb*{E} = \vb*{E}_0 \ee^{\ii(\vb*{k} \cdot \vb*{r} - \omega t)}, \quad k = \frac{\omega}{c}
    \label{eq:plane-wave}
\end{equation}
其中$\vb*{k}$是实矢量。
由空间中的傅里叶变换,\eqref{eq:halmholtz-eq}的所有解都可以写成不同$\vb*{k}$的平面波的叠加。

以上完成了关于介质中的控制方程的探讨。下面考虑界面上的衔接条件。最自然的想法,由\eqref{eq:scalar-cons-maxwell-e-and-b},可以写出自然边界条件
\[
    \begin{bigcase}
        \vb*{n} \cdot (\epsilon_i \vb*{E}_1 - \epsilon_t \vb*{E}_2) &= 0, \\
        \vb*{n} \times (\vb*{E}_1 - \vb*{E}_2) &= 0, \\
        \vb*{n} \cdot (\vb*{B}_1 - \vb*{B}_2) &= 0, \\
        \vb*{n} \times \left( \frac{\vb*{B}_1}{\mu_1} - \frac{\vb*{B}_2}{\mu_2} \right) &= 0
    \end{bigcase}
\]
但是正如\eqref{eq:scalar-cons-maxwell-e-and-b}实际上有冗余一样,上式也有方程是多余的。实际上,由\eqref{eq:e-in-material}可以得出的关于电场的相互独立的边界条件为
\begin{equation}
    \left\{\quad
        \begin{aligned}
            \epsilon_1 \vb*{n} \cdot \vb*{E}_1 = \epsilon_2 \vb*{n} \cdot \vb*{E}_2, \\
            \vb*{n} \times \vb*{E}_1 = \vb*{n} \times \vb*{E}_2, \\
            \vb*{n} \times \left( \frac{1}{\mu_1} \curl{\vb*{E}_1} \right) = \vb*{n} \times \left(\frac{1}{\mu_2} \curl{\vb*{E}_2} \right)
        \end{aligned}
    \right.
    \label{eq:e-bound-condition}
\end{equation}
只需要这些边界条件结合\eqref{eq:e-in-material}就能够定解。

\subsection{电各向异性光学介质}

现在考虑一种稍为推广的情况。很多介质——比如晶体——都具有空间上的各向异性,这是因为从不同的方向施加电场可以导致不同强度的极化。
在几乎所有常见的情况中,各向异性仅限于$\vb*{D}$和$\vb*{E}$的关系中,于是\eqref{eq:scalar-constitutive}修正为
\begin{equation}
    \vb*{D}(\vb*{r}) = \vb*{\epsilon}(\vb*{r}, \omega) \vb*{E}(\vb*{r}), \quad \vb*{H}(\vb*{r}) = \frac{1}{\mu(\vb*{r}, \omega)} \vb*{B}(\vb*{r})
    \label{eq:e-tensor-constitutive}
\end{equation}
相应的,\eqref{eq:e-in-material}修改为
\begin{equation}
    \begin{bigcase}
        \div{(\vb*{\epsilon} \cdot \vb*{E})} &= 0, \\
        \curl{ \left(\frac{1}{\mu} \curl{\vb*{E}} \right) } &= \omega^2 \vb*{\epsilon} \cdot \vb*{E}, \\
        \curl{\vb*{E}} &= \ii \omega \vb*{B}
    \end{bigcase}
    \label{eq:e-in-tensor-material}
\end{equation}

我们要研究$\mu$和$\vb*{\epsilon}$变化不大时\eqref{eq:e-in-tensor-material}的平面波解。
此时关于电场的全部方程为
\[
    \begin{aligned}
        \div{\vb*{\epsilon} \cdot \vb*{E}} = 0, \\
        \grad{(\div{\vb*{E}})} - \laplacian{\vb*{E}} = \omega^2 \mu \vb*{\epsilon} \cdot \vb*{E}
    \end{aligned}
\]
不过不难注意到第一个方程可以从第二个方程推导出来,因此方程
\begin{equation}
    \grad{(\div{\vb*{E}})} - \laplacian{\vb*{E}} = \omega^2 \mu \vb*{\epsilon} \cdot \vb*{E}
    \label{eq:anisotropy}
\end{equation}
如果$\vb*{E}$是一个波矢为$\vb*{k}$,频率为$\omega$的单色波,那么这两个方程就推出
\[
    \begin{aligned}
        - (\vb*{k} \cdot \vb*{E}) \vb*{k} + k^2 \vb*{E} = \omega^2 \mu \vb*{\epsilon} \cdot \vb*{E}, \\
    (\vb*{k} \cdot \vb*{\epsilon}) \cdot \vb*{E} = 0
    \end{aligned}
\]
但是这两个方程不是相互独立的——在第一个方程两边点乘$\vb*{k}$就可以推出第二个。
顺带提一句:第二个方程表明,在介质是电各向异性的情况下,波矢$\vb*{k}$并不垂直于电场,而是垂直于$\vb*{D}$。

那么电各向异性光学介质中的平面波就完全地被下式描写:
\[
    k^2 \vb*{E} - (\vb*{k} \cdot \vb*{E}) \vb*{k} - \omega^2 \mu \vb*{\epsilon} \cdot \vb*{E} = 0
\]
也就是说
\begin{equation}
    \left( k^2 \vb*{\delta} - \vb*{k} \vb*{k} - \omega^2 \mu \vb*{\epsilon} \right) \cdot \vb*{E} = 0
    \label{eq:plain-wave-in-anistrophy}
\end{equation}
其中$\vb*{\delta}$为单位张量。

\eqref{eq:plain-wave-in-anistrophy}将电场的方向和波矢的方向联系了起来。要观察波矢自己要满足什么条件,只需要求解
\begin{equation}
    \det \left( k^2 \vb*{\delta} - \vb*{k} \vb*{k} - \omega^2 \mu \vb*{\epsilon} \right) = 0
    \label{eq:k-det}
\end{equation}
即可。

直接处理矢量形式的方程会比较棘手。但是注意到由于能量守恒的原因,$\vb*{D}$不可能和$\vb*{E}$反向,也就是
\[
    \vb*{D} \cdot \vb*{E} = \vb*{E} \cdot \vb*{\epsilon} \cdot \vb*{E} > 0
\]
那么我们可以确定$\vb*{\epsilon}$是正定的(TODO:有虚部时怎么办?)另一方面,晶体的空间对称性意味着$\vb*{\epsilon}$一定可以对角化,因此我们总是可以找到一个坐标系(未必是正交坐标系),在其中$\vb*{\epsilon}$被对角化,且对角元均为正数。
这个坐标系就是所谓的\concept{主轴系}。

\section{介质中的能流}

为了讨论能量,这一节我们再次引入自由电荷的概念,用于讨论场中有实物粒子的情况。(如果实物粒子不带电荷,那么它就不参与电磁相互作用,此时不方便从牛顿力学出发讨论能量)
我们只追踪自由电荷的能量,而不追踪所有电荷的能量。
于是,使用\eqref{eq:maxwell-material}以及洛伦兹力公式
\[
    \vb*{f} = q \vb*{E} + q \vb*{v} \times \vb*{B}
\]
可以得到
\[
    \vb*{f} \cdot \vb*{v} = - \div{(\vb*{E} \times \vb*{H})} - \vb*{H} \cdot \pdv{\vb*{B}}{t} - \vb*{E} \cdot \pdv{\vb*{D}}{t}
\]
则可以选取最简单的电磁能量密度和电磁能流为
\begin{equation}
    \begin{bigcase}
        \vb*{S} &= \vb*{E} \times \vb*{H}, \\
        \pdv{w}{t} &= \vb*{E} \cdot \pdv{\vb*{D}}{t} + \vb*{H} \cdot \pdv{\vb*{B}}{t}
    \end{bigcase}
    \label{eq:energy-in-material}
\end{equation}
其中的$\vb*{S}$就是\concept{坡印廷矢量}。公式\eqref{eq:energy-in-material}的导出没有使用线性介质假设。
在已知$\vb*{D}$和$\vb*{E}$的关系线性、$\vb*{H}$和$\vb*{B}$的关系也是线性的情况下,有
\begin{equation}
    w = \frac{1}{2} (\vb*{E} \cdot \vb*{D} + \vb*{H} \cdot \vb*{B})
    \label{eq:energy-density-linear-material}
\end{equation}
需要注意的是\eqref{eq:energy-in-material}和\eqref{eq:energy-density-linear-material}本身都是二次型,因此在这些公式中的场必须是完整的电场和磁场,而不能够是(可能带有虚部的)傅里叶分量或者诸如此类的东西。

在稳态时各个场做正弦变化,这也是本文关注的主要场景之一。首先讨论介质中只有一个单色波的情况。不失一般性地认为$\vb*{E}$的相位零点在$t=0$处,则有
\[
    \vb*{S} = \vb*{E}_0 \times \vb*{H}_0 \cos(\omega t) \cos(\omega t + \phi)
\]
其中$\phi$是两者的相位差,$\vb*{E}_0, \vb*{H}_0$分别是振幅,这三者均只有实部。我们更加关心一段时间内的平均能流,则有
\[
    \expval*{\vb*{S}} = \frac{1}{2} \vb*{E}_0 \times \vb*{H}_0 \cos \phi
\]
现在考虑含有虚部的$\vb*{E}, \vb*{H}$表达式
\[
    \vb*{E} = \vb*{E}_0 \ee^{\ii \omega t}, \vb*{H} = \vb*{H}_0 \ee^{\ii \phi} \ee^{\ii \omega t}
\]
我们会发现一个有趣的结果:
\[
    \Re \vb*{E}^* \times \vb*{H} = \expval*{\vb*{S}}
\]
直观地看,可以将$\vb*{E}^* \times \vb*{H}$的实部看成“有功分量”,虚部看作“无功分量”。
这样可以定义\concept{实数版本的辐照度}
\begin{equation}
    \vb*{I} = \expval*{\vb*{S}} = \frac{1}{2} \vb*{E}_0 \times \vb*{H}_0 \cos \phi
    \label{eq:radiation-real}
\end{equation}
也可以定义\concept{复数版本的辐照度}
\begin{equation}
    \vb*{I} = \frac{1}{2} \Re \vb*{E}^* \times \vb*{H}
    \label{eq:radiation-complex}
\end{equation}
\eqref{eq:radiation-complex}的实部就是\eqref{eq:radiation-real}。

接着再讨论平面波的能量密度。由于
% \[
%     \vb*{k} \cdot \vb*{B} = 0, \quad \vb*{D} = - \frac{\vb*{k}}{\omega} \cross \vb*{H},
% \]
% 我们有
% \[
%     D^2 = \frac{k^2}{\mu^2 \omega^2} B^2
% \]
% 则
% \[
%     \frac{1}{2} \vb*{H} \cdot \vb*{B} = \frac{\mu \omega^2}{2 k^2} D^2, 
% \]
% \[
%     \frac{1}{2} \vb*{E} \cdot \vb*{D} = \frac{1}{2} \vb*{E} \cdot \vb*{\epsilon} \cdot \vb*{E}
% \]
% 于是
% \begin{equation}
%     w = w_E + w_B = \frac{1}{2} \vb*{E} \cdot \vb*{\epsilon} \cdot \vb*{E} + \frac{\mu \omega^2}{2 k^2} D^2
%     \label{eq:energy-density}
% \end{equation}
\[
    w = w_E + w_B,
\]
分别化简两项,有
\[
    w_E = \frac{1}{2} \vb*{E} \cdot \vb*{D} 
        = \frac{1}{2} \vb*{E} \cdot \left( - \frac{\vb*{k}}{\omega} \times \vb*{H} \right) 
        = \frac{1}{2} \frac{\vb*{k}}{\omega} \cdot (\vb*{E} \times \vb*{H}),
\]
且
\[
    w_B = \frac{1}{2} \vb*{H} \cdot \vb*{B} 
        = \frac{1}{2} \vb*{H} \cdot \left( \frac{\vb*{k}}{\omega} \cdot \vb*{E} \right) 
        = \frac{1}{2} \frac{\vb*{k}}{\omega} \cdot (\vb*{E} \times \vb*{H}),
\]
因此电场能和磁场能各占总能量的一半,且
\begin{equation}
    w = \frac{\vb*{k}}{\omega} \cdot (\vb*{E} \times \vb*{H}) = \frac{k S}{\omega} \cos \alpha
    \label{eq:energy-density-and-s}
\end{equation}
其中$\alpha$是$\vb*{k}$和$\vb*{S}$的夹角。
相应的,能量传递的速度为
\begin{equation}
    \vb*{v} = \frac{\vb*{S}}{w} = \frac{\omega}{k}  \cos \alpha \vu*{S}
\end{equation}


\section{线性介质的光学性能的微观模型}

\subsection{经典谐振子模型}

\subsubsection{洛伦兹模型}

我们现在转而考虑存在阻尼的谐振子,有
\begin{equation}
    m \ddot{\vb*{r}} = - m \gamma \dot {\vb*{r}} + q (\vb*{E} + \vb*{v} \times \vb*{B}),
\end{equation}
以$\vb*{B}$的指向为$z$轴,则$\vb*{E}$一定在$xy$平面内,则上式成为% TODO
\begin{equation}
    v_x = \frac{q}{m} \tau E_x + \omega_\text{L} \tau v_y,
\end{equation}
其中
\begin{equation}
    \omega_\text{L} = \frac{qB}{m}
\end{equation}
为\concept{拉莫频率},而
\begin{equation}
    \tau = \frac{1}{\gamma}.
\end{equation}

\subsection{能带理论的线性响应}

关于能带理论的详细介绍应当参考\soliddoc中的有关章节。

\chapter{均匀线性介质中的电磁波}\label{chap:linear-matter-no-scattering}

\section{各向同性线性介质中的平面波}\label{sec:light-propagate}

光无非是成束的电磁场,因此接下来可以通过求解\eqref{eq:e-in-material}获得光的传播情况。
在光学中常见的介质或者是性能变化比较均匀的,或者是性能变化非常剧烈的(也就是介质界面附近,例如水和空间交界处)。
前者中的

\subsection{均匀介质内部光的传播}\label{sec:in-interior-uniform}

在均匀介质内部光的传播情况由\eqref{eq:halmholtz-eq}控制。
当介质内部的$\epsilon$和$\mu$都是实数时可以直接使用平面波解\eqref{eq:plane-wave}。而当$\epsilon$和$\mu$含有虚部——也就是说,介质含有吸收等性质——那么根据解析延拓的原理,\eqref{eq:plane-wave}的形式应该得到保持,但是$\vb*{k}$需要有虚部。%
\footnote{从积分变换的角度来看,$\epsilon\mu$含有虚部意味着有关的方程难以直接通过傅里叶展开化简,因为傅里叶展开在实函数上比较简单。为此我们将$\vb*{k}$推广到复数的情况,实际上就是从空间傅里叶变换推广到了空间拉普拉斯变换。}
实际上,即使是$\epsilon$和$\mu$都取实数值的时候,\eqref{eq:plane-wave}中的$\vb*{k}$也可以有虚部。当然,此时的电场不再是一个基本解了,因为它可以使用若干个实数$\vb*{k}$的真正的平面波叠加出来。

在最一般的、$\epsilon$和$\mu$是否有虚部不知道的情况下,将\eqref{eq:plane-wave}代入\eqref{eq:halmholtz-eq}得到
\[
    - (\Re \vb*{k} + \Im \vb*{k})^2 + \omega^2 \epsilon\mu = 0
\]
化简,由于交叉项是纯虚数而其他项都是纯实数,可得
\[
    \begin{bigcase}
        (\Re \vb*{k})^2 - (\Im \vb*{k})^2 &= \omega^2\Re \epsilon\mu, \\
        2 (\Re \vb*{k}) \cdot (\Im \vb*{\vb*{k}}) &= \omega^2\Im \epsilon\mu.
    \end{bigcase}
\]
因此,在一个$\epsilon$和$\mu$都没有虚部的介质中,有
\[
    (\Re \vb*{k}) \cdot (\Im \vb*{\vb*{k}}) = 0
\]
还有\eqref{eq:e-in-material}的第一式没有使用。它意味着
\[
    \vb*{k} \cdot \vb*{E} = (\Re \vb*{k} + \ii \Im \vb*{k}) \cdot \vb*{E} = 0
\]

这样一来我们得出结论:在均匀的、可能有吸收等因素,从而使$c^2$有虚部的介质中,有形如下式的解:
\begin{equation}
    \left\{
        \begin{aligned}
            &\vb*{E} = \vb*{E}_0 \ee^{\ii (\vb*{k} \cdot \vb*{r} - \omega t)}, \\
            &(\Re \vb*{k})^2 - (\Im \vb*{k})^2 = \omega^2 \Re \epsilon\mu, \\
            &2 (\Re \vb*{k}) \cdot (\Im \vb*{\vb*{k}}) = \omega^2 \Im \epsilon\mu, \\
            &\vb*{k} \cdot \vb*{E} = 0
        \end{aligned}
    \right.
    \label{eq:uniform-wave}
\end{equation}

在$\epsilon\mu$完全就是实数的时候\eqref{eq:uniform-wave}意味着$\vb*{k}$的实部、虚部相互正交(电场的方向和$\vb*{k}$的实部、虚部没有特别明显的关系)。
这种情况只有可能发生在界面附近,并且要求$\Re \vb*{k}$平行于界面而$\Im \vb*{k}$要垂直于界面且方向从界面指向介质内部,
否则电场会在远处发散到无穷大,不是物理解。%
\footnote{关于物理解有必要说明这一点:实际上,平面波本身也不是真正具有物理意义的波,因为不可能让电场充满整个空间,且同时只具有一个频率。
然而,实际的光的分布满足的特定边界条件意味着我们确实没有必要讨论所有可能的平面波——只需要讨论满足这些边界条件的平面波即可。
例如,光通常是从一个光源打出来的,这就暗示了两个边界条件:
首先,某个平面(也就是光源所在的平面)上的场强是给定的,需要计算的是这个平面某一侧的场强,另一侧的场强没有意义(因为它在光源之后);
其次,光源两侧无穷远处的场强应趋于零。
这就意味着,组合成这个光的平面波在垂直于$\Re \vb*{k}$的方向上不能发散,在与$\Re\vb*{k}$相反的方向上可以发散,在与$\Re\vb*{k}$同向的方向上不能发散。
这就是我们使用的“物理解”的条件。
}
我们将在\ref{sec:total-reflect}节看到这种解的一个例子。
而当$\epsilon\mu$含有虚部时,可以做下面的分解:
\[
    \Im \vb*{k} = \vb*{k}_\parallel + \vb*{k}_\bot
\]
使$\vb*{k}_\parallel$平行于$\Re\vb*{k}$,$\vb*{k}_\bot$垂直于$\Re\vb*{k}$。
这两个分量当然都会让电场在远处发散到无穷大,但是物理解的要求意味着$\vb*{k}_\parallel$与$\Re \vb*{k}$同向,
而如果$\vb*{k}_\bot$非零,那么必定有$\Re \vb*{k}$(从而$\vb*{k}_\parallel$)平行于界面而$\vb*{k}_\bot$要垂直于界面且方向从界面指向介质内部。

既然$\vb*{k}_\parallel$和$\Re \vb*{k}$总是同向,我们有理由认为这两者可以看成同一个对象的实部和虚部。
因此,在\ref{sec:light-propagate}节的剩下部分,我们用$\vb*{k}$代替原本的$\Re \vb*{k} + \ii \vb*{k}_\parallel$,
而使用$\vb*{\beta}$代替原本的$\vb*{k}_\bot$,这样\eqref{eq:uniform-wave}就需要改写为
\begin{equation}
    \left\{\quad
        \begin{aligned}
            \vb*{E} = \vb*{E}_0 \ee^{- \vb*{\beta} \cdot \vb*{r}} \ee^{\ii (\vb*{k} \cdot \vb*{r} - \omega t)}, \\
            \vb*{k}^2 - \vb*{\beta}^2 = \mu\epsilon\omega^2, \quad \vb*{k} \cdot \vb*{\beta} = 0, \\
            (\vb*{k} + \ii \vb*{\beta}) \cdot \vb*{E} = 0
        \end{aligned}
    \right.
    \label{eq:beta-k-uniform-wave}
\end{equation}
其中$\vb*{k}$与波前传播方向同向。$\vb*{k}$,$\vb*{\beta}$,$\vb*{E}$三者相互垂直,$\vb*{k}$可以有虚部而$\vb*{\beta}$没有。
无论$\vb*{k}$有没有虚部,它都可以写成一个复数乘以一个实矢量的形式,因此可以非常良好地定义$\vb*{k}$方向上的单位矢量$\hat{\vb*{k}}$。
反之,在\eqref{eq:uniform-wave}中,由于没有将$\vb*{k}$做适当的分解,可能难以良定义一个单位矢量,因为此时不同基向量上的分量可能具有不成比例的实部和虚部。

同样,为了让解是物理解,应当有$\vb*{\beta}$垂直于某个界面且方向从界面指向介质内部,否则解会发散。

特别的,在$\vb*{\beta}$为零的时候,我们有
\begin{equation}
    \left\{
        \begin{aligned}
            \vb*{E} = \vb*{E}_0 \ee^{\ii(\vb*{k} \cdot \vb*{r} - \omega t)}, \\
            \vb*{k}^2 = \mu\epsilon \omega^2, \\
            \vb*{k} \cdot \vb*{E} = 0.
        \end{aligned}
    \right.
    \label{eq:beta-zero-uniform-wave}
\end{equation}

关于$\vb*{k}$的取值需要额外的注记。我们设$\vb*{k}=k \hat{\vb*{k}}$,其中$\hat{\vb*{k}}$正是\eqref{eq:uniform-wave}中的那个$\Re\vb*{k}$(和\eqref{eq:beta-k-uniform-wave}中的$\vb*{k}$同向但是不相同)的单位向量。
那么从\eqref{eq:beta-k-uniform-wave}就可以得出
\begin{equation}
    k^2 - \beta^2 = \mu \epsilon \omega^2
\end{equation}
其中$\beta$的取值通常和边界条件有关,确定了$\beta$就能够得到$k$。
但是还有一个额外的问题:$\mu \epsilon$含有虚部,因此下面的表达式
\[
    k = \sqrt{\beta^2 + \mu \epsilon \omega^2}
\]
就是多值的。具体取哪一个值需要根据物理条件确定。在能够确定具体取那个值的时候,我们规定\concept{折射率}为
\begin{equation}
    n(\omega) = c_0 \sqrt{\epsilon(\omega) \mu(\omega)}, 
    \label{eq:refractivity}
\end{equation}
这时就有
\begin{equation}
    \quad \frac{\omega}{k} = \frac{c_0}{n} \equiv c
    \label{eq:k-and-omega-and-n}
\end{equation}
在$n$没有虚部时$c$就是介质中\eqref{eq:beta-k-uniform-wave}的波前传播速度。

\subsection{两种透明均匀介质界面上的折射和反射}\label{sec:two-isotrophy-surface}

所谓透明介质指的是折射率完全为实数的介质。在这种介质中,
如图\ref{fig:ray-onto-flat-surface}所示,我们假定有两个几乎无限大的介质以一个完全平坦的界面隔开。
\begin{figure}
    \centering
    \begin{tikzpicture}
        % 两种介质
        \node[above] at (-4, 2) {$n_i$};
        \node[above] at (-4, -3) {$n_t$};
        % 介质界面
        \draw (-4,0) -- (4,0);
        \node[above] at (-3.5,0) {\small 介质界面};
        % 法线
        \draw[dash pattern=on5pt off3pt] (0,4) -- (0,-4);
        % 入射光线
        \draw[ray] (130:5.2) -- (0,0);
        \node[] at (-2,3) {$\vb*{k}_i$};
        \draw (0,1) arc (90:130:1);
        \node[] at (110:1.4) {$\theta_i$};
    \end{tikzpicture}
    \caption{光入射平整表面}
    \label{fig:ray-onto-flat-surface}
\end{figure}
现在让一束光照射到这个界面上。为了方便起见,假定在我们感兴趣的尺度内光强\concept{处处相同},也就是入射光形式为
\[
    \vb*{E}_i = \vb*{E}_{i0} \ee^{\ii(\vb*{k}_i \cdot \vb*{r} - \omega t)}
\]
(图\ref{fig:ray-onto-flat-surface}中的实线只是代表波的传播方向,不代表光束。)

\begin{figure}
    \centering
    \begin{tikzpicture}
        \draw [-{Stealth}] (-1, 1) -- (1, -1);
        \node[above] at (0,-1) {$\vb*{k}$};
        \draw [-{Stealth}, thick] (0.5,0.5) -- (1.5,1.5);
        \node at (0.4,1) {$\vb*{E}_p$};
        \node at (1.5, 0) {$\odot$};
        \node at (2, 0) {$\vb*{E}_s$};
    \end{tikzpicture}
    \caption{电场的分解}
    \label{fig:decomposition-of-e}
\end{figure}

同时为了将矢量方程\eqref{eq:e-bound-condition}标量化,将电场按照图\ref{fig:decomposition-of-e}的方式做分解。
我们没有给出平行于$\vb*{k}$的电场分量,因为按照\eqref{eq:beta-k-uniform-wave},这个电场分量不存在。

\subsubsection{有透射光的情况}

\begin{figure}
    \centering
    \begin{tikzpicture}
        % 两种介质
        \node[above] at (-4, 2) {$n_i$};
        \node[above] at (-4, -3) {$n_t$};
        % 介质界面
        \draw (-4,0) -- (4,0);
        \node[above] at (-3.5,0) {\small 介质界面};
        % 法线
        \draw[dash pattern=on5pt off3pt] (0,4) -- (0,-4);
        \draw [-{Stealth}, thick] (0,0) -- (0,2);
        \node [above] at (0.3, 2.1) {$\vb*{n}$};
        % 入射光线
        \draw[ray] (130:5.2) -- (0,0);
        \node[] at (-2,3) {$\vb*{k}_i$};
        \draw (0,1) arc (90:130:1);
        \node[] at (110:1.4) {$\theta_i$};
        % 反射光线
        \draw[ray] (0,0) -- (50:5.2) ;
        \node[] at (2,3) {$\vb*{k}_r$};
        \draw (0,1.2) arc (90:50:1.2);
        \node[] at (70:1.4) {$\theta_r$};
        % 折射光线
        \draw[ray] (0,0) -- (-70:4.5);
        \node[] at (1.5,-3) {$\vb*{k}_t$};
        \draw (0,-1.1) arc (-90:-70:1.1);
        \node[] at (-80:1.4) {$\theta_t$};
    \end{tikzpicture}
    \caption{光入射平整表面之后发生反射和折射}
    \label{fig:ray-refraction}
\end{figure}

凭借经验,我们会认为入射波$\vb*{E}_i$会导致两个平面波,分别称为\concept{反射波}和\concept{折射波},图示为图\ref{fig:ray-refraction},形式如下:
\[
    \begin{aligned}
        \vb*{E}_r &= \vb*{E}_{r0} \ee^{\ii(\vb*{k}_r \cdot \vb*{r} - \omega t)}, \\
        \vb*{E}_t &= \vb*{E}_{t0} \ee^{\ii(\vb*{k}_t \cdot \vb*{r} - \omega t)}
    \end{aligned}
\]

这样在介质1中和在介质2中分别有
\[
    \begin{aligned}
        \vb*{E}_1 = \vb*{E}_{i0} \ee^{\ii(\vb*{k}_i \cdot \vb*{r} - \omega t)} +  \vb*{E}_{r0} \ee^{\ii(\vb*{k}_r \cdot \vb*{r} - \omega t)}, \\
        \vb*{E}_2 = \vb*{E}_{t0} \ee^{\ii(\vb*{k}_t \cdot \vb*{r} - \omega t)}
    \end{aligned}
\]
使用\eqref{eq:e-bound-condition},可以得到形如下式的结果:
\[
    \text{something } \ee^{\ii(\vb*{k}_i \cdot \vb*{r} - \omega t)} + \text{something } \ee^{\ii(\vb*{k}_r \cdot \vb*{r} - \omega t)} = \text{something } \ee^{\ii(\vb*{k}_t \cdot \vb*{r} - \omega t)}
\]
从而必须有
\[
    (\vb*{k}_i - \vb*{k}_r) \cdot \vb*{r} = \const, \quad (\vb*{k}_i - \vb*{k}_t) \cdot \vb*{r} = \const
\]
考虑到这些方程描写了一个平面,而它们都在界面(一个平面)上恒成立,有%
\footnote{能够导出这个结果是因为$\ee^{\ii \vb*{k} \cdot \vb*{r}}$是函数基。上面的方程在界面上恒成立,因此在界面上$\vb*{k}_i$,$\vb*{k}_r$,$\vb*{k}_t$的投影,也就是$\vb*{n} \times \vb*{k}$都一样。}
\begin{equation}
    \vb*{n} \times (\vb*{k}_i - \vb*{k}_r) = \vb*{n} \times (\vb*{k}_i - \vb*{k}_t) = 0
    \label{eq:k-and-n}
\end{equation}

假设入射波、反射波、折射波\concept{它们自己}都能够成为\eqref{eq:e-in-material}的解。
考虑到介质被认为是均匀的,假设这三个波都能够写成\eqref{eq:uniform-wave}的形式。
进一步,假设这三个波均没有\eqref{eq:beta-k-uniform-wave}中的$\vb*{\beta}$。
这些假设是太多了,我们将发现它们只能够在一部分场景下成立。

在这种情况下,三个波的$\vb*{k}$都可以写成一个复数乘以一个实矢量的形式,因此可以良好地定义“波前前进的方向”。
于是就有
\[
    \vb*{k}_i = k_i \hat{\vb*{k}}_i, \; \vb*{k}_r = k_r \hat{\vb*{k}}_r, \; \vb*{k}_t = k_t \hat{\vb*{k}}_t
\]
由于入射波和反射波都位于介质1中,我们确定$k_i=k_r$。

按照图\ref{fig:ray-refraction}中展示的方式标记角度。
(注意$\vb*{k}_i$和法向量的夹角为$\pi-\theta_i$)
\eqref{eq:k-and-n}可以写作
\[
    k_i \vb*{n} \times \vu*{k}_i = k_r \vb*{n} \times \vu*{k}_r, \quad k_i \vb*{n} \times \vu*{k}_i = k_t \vb*{n} \times \vu*{k}_t
\]

\begin{equation}
    \theta_i = \theta_r, \quad \frac{\sin \theta_i}{\sin \theta_t} = \frac{n_t}{n_i}
    \label{eq:snell}
\end{equation}

最终得到\concept{菲涅尔公式}
\begin{equation}
    \begin{bigcase}
        r_\bot &= \frac{E_{rs}}{E_{is}} = 
        \frac{\frac{n_i}{\mu_i} \cos \theta_i - \frac{n_t}{\mu_t} \cos \theta_t}{\frac{n_i}{\mu_i} \cos \theta_i + \frac{n_t}{\mu_t} \cos \theta_t}, \\
        t_\bot &= \frac{E_{ts}}{E_{is}} = 
        \frac{2 \frac{n_i}{\mu_i} \cos \theta_i}{\frac{n_i}{\mu_i} \cos \theta_i + \frac{n_t}{\mu_t} \cos \theta_t}, \\
        r_\parallel &= \frac{E_{rp}}{E_{ip}} = 
        \frac{\frac{n_t}{\mu_t} \cos \theta_i - \frac{n_i}{\mu_i} \cos \theta_t}{\frac{n_i}{\mu_i} \cos \theta_t + \frac{n_t}{\mu_t} \cos \theta_i}, \\
        t_\parallel &= \frac{E_{tp}}{E_{ip}} =
        \frac{2 \frac{n_i}{\mu_i} \cos \theta_i}{\frac{n_i}{\mu_i} \cos \theta_t + \frac{n_t}{\mu_t} \cos \theta_i}
    \end{bigcase}
    \label{eq:fresnel-formulas}
\end{equation}

当$\theta_i \to 0$,也就是入射光垂直于界面时,我们有
\begin{equation}
    r_\bot = \frac{n_i / \mu_i - n_t / \mu_t}{n_i / \mu_i + n_t / \mu_t}, \quad 
    r_\parallel = \frac{n_t / \mu_t - n_i / \mu_i}{n_i / \mu_i + n_t / \mu_t}
\end{equation}
看起来这样很奇怪,因为此时不能区分p光和s光,而两者的反射系数却差了一个负号。
但实际上这是错觉。回顾图\ref{fig:decomposition-of-e},我们会发现入射和反射的p光的基矢量在$\theta \to 0$时是反向的,而入射和反射s光的基矢量在$\theta \to 0$时同向,因此反射系数就应该差一个负号——最后无论将入射光当成p光还是当成s光,都能够得到同样的反射光矢量,于是可以写出
\begin{equation}
    \vb*{E}_r = \frac{n_i / \mu_i - n_t / \mu_t}{n_i / \mu_i + n_t / \mu_t} \vb*{E}_i
\end{equation}

关于能量有下面的等式:
\begin{equation}
    T = \left( \frac{n_t \cos \theta_t / \mu_t}{n_i \cos \theta_i / \mu_i} \right) t^2, \quad R = r^2
\end{equation}

\subsubsection{全反射}\label{sec:total-reflect}

TODO:隐逝波的方向不完全

当$n_i > n_t$而
\[
    \theta_i > \arcsin \frac{n_t}{n_i}
\]
时,先前做的“入射波产生反射波和折射波,这三个波都是平面波”的假设就失效了,因为这个假设导致方程组\eqref{eq:e-bound-condition}无解。
为了在这种情况下求解\eqref{eq:e-bound-condition},尝试放松一个假设。
实验上,在全反射发生的时候没有观察到折射波,因此假定折射波在边界处快速衰减了。因此此时我们假定反射波、折射波采取下面的形式:
\[
    \begin{aligned}
        \vb*{E}_r &= \vb*{E}_{r0} \ee^{\ii (\vb*{k}_r \cdot \vb*{r} - \omega t)}, \\
        \vb*{E}_t &= \vb*{E}_{t0} \ee^{-\vb*{\beta} \cdot \vb*{r}} \ee^{\ii (\vb*{k}_t \cdot \vb*{r} - \omega t)}
    \end{aligned}
\]
其中$\vb*{\beta}$垂直于界面。所有的参量都是实的。

此时在边界上会有类似于这样的表达式:
\[
    \mathrm{something} \; \ee^{\ii (\vb*{k}_r \cdot \vb*{r} - \omega t)} + \mathrm{something} \; \ee^{\ii (\vb*{k}_r \cdot \vb*{r} - \omega t)} = \mathrm{something} \; \ee^{-\vb*{\beta} \cdot \vb*{r}} \ee^{\ii (\vb*{k}_t \cdot \vb*{r} - \omega t)}
\]

\subsubsection{公式形式的统一处理}

\subsection{各向同性介质中平面波的能量}

由于在各向同性介质中

\section{各向异性线性介质中的平面波}

以上讨论的都是各向同性介质中光的传播,也就是说,$\vb*{D}$和$\vb*{E}$、$\vb*{H}$和$\vb*{B}$之间的联系都是标量倍数。
现在稍微放松这个假设,假定$\vb*{E}$和之间的联系是一个张量,也就是说,
\[
    \vb*{D} = \vb*{\epsilon} \cdot \vb*{E}
\]
其中$\vb*{\epsilon}$是一个张量,或者也可以写成
\[
    D_i = \epsilon_{ij} E_j
\]

\subsection{单光轴透明介质内部的平面波}\label{sec:one-axis-transparent}

首先讨论一种最简单的情况。此时介质有一个对称轴,绕着这个对称轴有旋转不变性。
这就意味着$\vb*{\epsilon}$有两个特征值,其中一个对应着一个唯一的特征向量,另一个对应着两个特征向量,
且后面两个特征向量垂直于前一个特征向量(否则不能保证旋转不变性)。
% TODO:群论
此时可以将$\vb*{\epsilon}$正交对角化。
于是,可以找到一个直角坐标系$x,y,z$,使$z$方向对应着前一个特征向量,$x,y$方向对应着后两个特征向量(需要对它们做正交化)
此时没有必要区分逆变协变,可以直接写出矩阵形式
\begin{equation}
    [\mu \epsilon_{ij}]_{ij} = \frac{1}{c_0^2} \bmqty{\dmat{n_o^2,n_o^2,n_e^2}}
    \label{eq:one-axis-matrix}
\end{equation}
其中$n_e, n_o > 0$。我们下这个断言是因为介质对外加电场产生的响应不可能使总电场和外加电场方向相反,也就是说,如果$\vb*{E}$取某个适当的方向使$\vb*{E}$和$\vb*{D}$之间只差一个标量倍数,那么这个标量倍数一定大于零,因此这个倍数一定有正的平方根,从而,$\vb*{\epsilon}$的特征值一定大于零。

此时\eqref{eq:k-det}等价于
\[
    \mdet{
        k_y^2 + k_z^2 - \frac{\omega^2}{c_0^2} n_o^2 & -k_x k_y & - k_x k_z \\
        - k_x k_y & k_x^2 + k_z^2 - \frac{\omega^2}{c_0^2} n_o^2 & -k_y k_z \\
        -k_x k_z & -k_y k_z & k_x^2 + k_y^2 - \frac{\omega^2}{c_0^2} n_e^2
    } = 0
\]
化简得到
\begin{equation}
    \left( \frac{k_x^2}{n_o^2} + \frac{k_y^2}{n_o^2} + \frac{k_z^2}{n_o^2} - \frac{\omega^2}{c_0^2} \right) \left( \frac{k_x^2}{n_e^2} + \frac{k_y^2}{n_e^2} + \frac{k_z^2}{n_o^2} - \frac{\omega^2}{c_0^2} \right) = 0
    \label{eq:uniaxial-crystal}
\end{equation}
因此,$\vb*{k}$只需要让其中的一个因式为零,就是可能的解。

下面讨论\eqref{eq:uniaxial-crystal}的两个解。第一种情况是
\[
    \frac{k_x^2}{n_o^2} + \frac{k_y^2}{n_o^2} + \frac{k_z^2}{n_o^2} - \frac{\omega^2}{c_0^2} = 0
\]
我们称此时的波为\concept{o光}。o光的波矢需要且只需要满足
\[
    \abs{\vb*{k}_o} = \frac{n_o \omega}{c_0}
\]
o光的电场方向需要满足什么条件?可以将\eqref{eq:plain-wave-in-anistrophy}写成
\[
    \bmqty{
        k_y^2 + k_z^2 - \frac{\omega^2}{c_0^2} n_e^2 & -k_x k_y & - k_x k_z \\
        - k_x k_y & k_x^2 + k_z^2 - \frac{\omega^2}{c_0^2} n_e^2 & -k_y k_z \\
        -k_x k_z & -k_y k_z & k_x^2 + k_y^2 - \frac{\omega^2}{c_0^2} n_o^2
    }
    \bmqty{
        E_x \\ E_y \\ E_z
    } = 0
\]
在球坐标系中写出
\[
    k_x = \frac{n_o \omega}{c_0} \sin \varphi \cos \theta, 
    \quad k_y = \frac{n_o \omega}{c_0} \sin \varphi \sin \theta, \quad k_z = \frac{n_o \omega}{c_0} \cos \varphi
\]
然后代入上面的矩阵表达式,经过一系列初等变换得到
\[
    \bmqty{
        \sin \varphi \cos \theta & \sin \varphi \sin \theta & \cos \varphi \\
        0 & 0 & 1
    }
    \bmqty{E_x \\ E_y \\ E_z} = 0
\]
于是得到了o光需要(且只需要)满足的条件:
\begin{equation}
    \abs{\vb*{k}_o} = \frac{n_o \omega}{c_0}, \quad\vb*{e}_z \cdot \vb*{E}_o = \vb*{k}_o \cdot \vb*{E}_o = 0
    \label{eq:o-light}
\end{equation}
因此,除了o光永远不会有平行于光轴的分量以外,o光在介质中的传播方式和各向同性介质中的光完全一样:$\vb*{k}$和$\vb*{E}, \vb*{D}$均垂直(并且可以验证$\vb*{D}$和$\vb*{E}$平行),且$k$和$\omega$之间的关系就是普通的折射率确定的关系。

第二种情况是
\[
    \frac{k_x^2}{n_e^2} + \frac{k_y^2}{n_e^2} + \frac{k_z^2}{n_o^2} - \frac{\omega^2}{c_0^2} = 0
\]
称此时的波为\concept{e光}。使用和上面相同的方法,写出椭球坐标之下的$\vb*{k}_e$表达式
\[
    k_x = \frac{n_e \omega}{c_0} \sin \theta \cos \varphi, \quad k_y = \frac{n_e \omega}{c_0} \sin \theta \sin \varphi, \quad k_z = \frac{n_o \omega}{c_0} \cos \theta
\]
将\eqref{eq:plain-wave-in-anistrophy}写成矩阵形式之后代入椭球坐标下的$\vb*{k}_e$表达式,然后做初等变换,得到
\[
    \bmqty{
        n_0 \sin \theta \cos \varphi & n_0 \sin \theta \sin \varphi & n_e \cos \theta \\
        -\sin \varphi & \cos \varphi & 0
    }
    \bmqty{E_x \\ E_y \\ E_z} = 0
\]
这是一个有两个方程组成的方程组。
容易验证,在本坐标系中$\bmqty{n_0 \sin \theta \cos \varphi & n_0 \sin \theta \sin \varphi & n_e \cos \theta}$
和$\vb*{k} \cdot \vb*{\epsilon}$共线,$\bmqty{-\sin \varphi & \cos \varphi & 0}$与$\vb*{e}_z \times \vb*{k}$共线。
因此e光需要且只需要满足的方程为
\begin{equation}
    \frac{k_{ex}^2}{n_e^2} + \frac{k_{ey}^2}{n_e^2} + \frac{k_{ez}^2}{n_o^2} =
    \frac{\omega^2}{c_0^2}, \quad \vb*{k}_e \cdot \vb*{\epsilon} \cdot \vb*{E}_e = 0, \quad (\vb*{e}_z \times \vb*{k}_e) \cdot \vb*{E}_e = 0
    \label{eq:e-light}
\end{equation}
因此,e光的振动方向被限制在了光轴和波矢确定的平面上。
此时通过$k$和$\omega$之间的关系仍然可以定义等效的折射率,它是
\begin{equation}
    n = \frac{c_0 \abs*{\vb*{k}_e}}{\omega} , \quad \frac{1}{n^2} = \frac{\sin^2 \theta}{n_e^2} + \frac{\cos^2 \theta}{n_o^2}.
    \label{eq:e-light-effective-index}
\end{equation}

e光被认为是“反常的”,因为它具有许多各向同性介质中的光完全不显示的性质。

注意到\eqref{eq:o-light}和\eqref{eq:e-light}中关于$\vb*{k}$的方程无论$\vb*{k}$的方向是什么样都是有解的,而且有唯一解。
因此,一旦$\vb*{k}$的方向确定了,e光和o光的$\vb*{k}$以及可能的振动方向也就完全确定了。

总之,各向异性线性介质中的光一般来说是满足\eqref{eq:o-light}的o光和满足\eqref{eq:e-light}的e光的叠加。

% TODO:走移角

\subsection{双光轴透明介质内部的平面波}

\subsubsection{菲涅尔法线方程}

当我们试图将\ref{sec:one-axis-transparent}节中的方法原封不动地推广到一般的各向异性介质中时,会遇到一个严重的困难:无法得到像\eqref{eq:uniaxial-crystal}这样的因式分解好了的关于$\vb*{k}$的方程。

设介质中有一列平面波,记\concept{这一列波的折射率}为
\begin{equation}
    n = \frac{k c_0}{\omega} = \frac{k}{\omega} \frac{1}{\sqrt{\mu_0 \epsilon_0}}
\end{equation}
那么就有
\[
    \vb*{k} = n \omega \sqrt{\mu_o \epsilon_0} \vu*{k}
\]
由\eqref{eq:k-det},可以得到
\[
    \det \left( n^2 \omega^2 \epsilon_0 \mu_0 (\vb*{\delta} - \vu*{k} \vu*{k}) - \omega^2 \mu \vb*{\epsilon} \right) = 0
\]
\[
    \det \left( \vb*{\delta} - \vu*{k} \vu*{k} - \frac{\mu_r \vb*{\epsilon}_r}{n^2} \right) = 0
\]
然后求解这个方程。由于此时$\vu*{k}$已经是给定的了,我们将使用$\vu*{k}, \mu_r, \vb*{\epsilon}_r$表示出$n$。
在主轴坐标系中表示$\vb*{k}$和其它矢量,并且设对角化之后%
\footnote{
    所谓对角化,在这里指的是写成
    \[
        \vb*{\epsilon}_r = \epsilon_{r \underline{i}} \vb*{g}_i \vb*{g}^i
    \]
    的形式。
}%
的$\vb*{\epsilon}_r$的三个元素为$\epsilon_{rx}, \epsilon_{ry}, \epsilon_{rz}$
则有
\begin{equation}
    \frac{k_x k^x}{\frac{1}{n^2} - \frac{1}{\mu_r \epsilon_{rx}}} + \frac{k_y k^y}{\frac{1}{n^2} - \frac{1}{\mu_r \epsilon_{ry}}} + \frac{k_z k^z}{\frac{1}{n^2} - \frac{1}{\mu_r \epsilon_{rz}}} = 0
    \label{eq:fresnel-k-n}
\end{equation}
将\eqref{eq:fresnel-k-n}展开为$1/n$的多项式之后会发现这是一个关于$1/n^2$的二次多项式,且在实数域内有解,因此\eqref{eq:fresnel-k-n}有两个正根两个负根。
仅考虑物理解,能够得到两个正根。
这表明了任意的各向异性介质的特点:给定一个$\vb*{k}$,可以有两种相位传播速度不同的波。

若设$\vb*{\epsilon}$在主轴系中被对角化为
\begin{equation}
    c_0^2 [\mu \epsilon_{ij}]_{ij} = [\mu_r \epsilon_{r\;ij}]_{ij} = \bmqty{\dmat{n_x^2, n_y^2, n_z^2}}
    \label{eq:diag-two-axis}
\end{equation}
则有
\begin{equation}
    \frac{k_x k^x}{\frac{1}{n^2} - \frac{1}{n_x^2}} + \frac{k_y k^y}{\frac{1}{n^2} - \frac{1}{n_y^2}} + \frac{k_z k^z}{\frac{1}{n^2} - \frac{1}{n_z^2}} = 0
\end{equation}
这个方程称为\concept{菲涅尔法线方程}。虽然我们使用$\vb*{k}$的各个分量写出了它,由于其齐次性,完全可以将所有$\vb*{k}$的分量替换为$\vu*{k}$的分量。

需要注意的是对应于$n$的两个根的波的振动方向并不是任意的。使用本节的记号,可以将\eqref{eq:plain-wave-in-anistrophy}写成
\begin{equation}
    \left(\vb*{\delta} - \vu*{k}\vu*{k} - \frac{\mu_r \vb*{\epsilon}_r}{n^2} \right) \cdot \vb*{E} = 0
\end{equation}
它意味着:首先,$n$应当被适当地选定,让方程左边的张量的行列式为零,这等价于\eqref{eq:fresnel-k-n};其次,$n$被确定后,$\vb*{E}$可能的方向也被确定了下来。$\vb*{E}$可能的取值就是$\vb*{\delta} - \vu*{k}\vu*{k} - \mu_r \vb*{\epsilon}_r / n^2$的零空间。

一个可能的问题:在已经选定了$\vu*{k}$之后,我们能够得到两个$n$,从而两个$\vb*{E}$振动的方向,那么为什么不是三个方向?
原因在于我们有约束$\vb*{k} \cdot \vb*{D} = \vb*{k} \cdot \vb*{\epsilon} \cdot \vb*{E} = 0$,因此实际能够取的$\vb*{E}$(或者$\vb*{D}$)分布在一个二维的空间中,而不是三维的空间。

此外,注意到相速度$v_\text{p}$就是$c_0 / n$,因此在波矢方向已经给定的情况下从\eqref{eq:fresnel-k-n}可以解出两个相速度。当然,这就是对应于两个$n$的平面波的传播速度。

\subsubsection{折射率椭球}

% TODO:各向异性是不是还是电场能和磁场能各占一半?
现在考虑区域内电场能量密度
\[
    w_E = \frac{1}{2} \vb*{E} \cdot \vb*{D} = \frac{1}{2} (E^x D_x + E^y D_y + E^z D_z),
\]
同样在主轴坐标系当中工作,由于
\[
    D^x = \epsilon_0 \epsilon_{rx} E^x, \quad D^y = \epsilon_0 \epsilon_{ry} E^y, \quad D^z = \epsilon_0 \epsilon_{rz} E^z
\]
我们有
\begin{equation}
    \frac{2 \epsilon_0 w_E}{\mu_r} = \frac{D_x D^x}{n_x^2} + \frac{D_y D^y}{n_y^2} + \frac{D_z D^z}{n_z^2}
\end{equation}
因此如果固定电磁能密度不变,那么这一点的$\vb*{D}$扫过一个椭球面。这个椭球面称为\concept{折射率椭球}。

\subsubsection{各矢量方向的分析}

首先,下面三个方程还是成立的,正如在各向同性介质中一样:
\[
    \vb*{k} \cdot \vb*{D} = 0, \quad \vb*{k} \cdot \vb*{H} = 0, \quad \vb*{k} \times \vb*{H} = - \omega \vb*{D}
\]
这意味着$\vb*{D}, \vb*{H}, \vb*{k}$构成一组右手系。$\vb*{B}$的方向和$\vb*{H}$完全一致,因此无需单独讨论其方向。

反之,由于$\vb*{\epsilon}$的各向异性,$\vb*{E}$的方向需要特别注意。
由于
\[
    \vb*{k} \times \vb*{E} = \omega \vb*{B}, \quad \vb*{k} \times \vb*{H} = - \omega \vb*{D}
\]
可以导出下面的方程
\begin{equation}
    \vb*{D} = \mu_r \epsilon_0 n^2 (\vb*{E} - (\vu*{k} \cdot \vb*{E}) \vu*{k}) = \mu_r \epsilon_0 n^2 \vb*{E}_{\bot}
    \label{eq:first-crystal-eq}
\end{equation}
即所谓\concept{晶体光学第一方程},其中$\bot$表示在垂直于$\vb*{k}$的方向上做投影。实际上,通过将$\vb*{D} = \vb*{\epsilon} \cdot \vb*{E}$代入上式,也能够导出菲涅尔法线方程\eqref{eq:fresnel-k-n}。

% TODO:画图
\eqref{eq:first-crystal-eq}使用$\vb*{E}$表示了$\vb*{D}$;我们也可以反过来尝试使用$\vb*{D}$表示$\vb*{E}$。注意到$\vb*{E}$虽然和$\vb*{D}$未必重合,但是它一定落在垂直于$\vb*{H}$的平面内;而$\vu*{k} \propto \vb*{D} \times \vb*{H}$,$\vu*{S} \propto \vb*{E} \times \vb*{H}$,于是几何观察告诉我们,将$\vu*{D},\vu*{k}$做一个旋转角为$\alpha$(它正是\eqref{eq:energy-density-and-s}中的那个$\alpha$),在垂直于$\vb*{H}$的平面上的旋转就得到了$\vu*{E}, \vu*{S}$。
于是记
\begin{equation}
    n_r = n \cos \alpha,
\end{equation}
就得到了
\begin{equation}
    \vb*{E} = \frac{1}{\mu_r \epsilon_0 n_r^2} (\vb*{D} - (\vu*{S} \cdot \vb*{D}) \vu*{S})
    \label{eq:second-crystal-eq}
\end{equation}
即所谓的\concept{晶体光学第二方程}。

联立这两个方程\eqref{eq:first-crystal-eq}和\eqref{eq:second-crystal-eq}中的其中一个和本构关系$\vb*{D} = \vb*{\epsilon} \vb*{E}$,
可以得到仅仅关于$\vb*{E}$或仅仅关于$\vb*{D}$的方程。
仅仅关于$\vb*{E}$的方程已经被建立了,它就是\eqref{eq:plain-wave-in-anistrophy},它有非零解的条件就是\eqref{eq:fresnel-k-n}。
联立晶体光学第二方程\eqref{eq:second-crystal-eq}和本构关系,尝试得到仅仅关于$\vb*{D}$的方程。
在主轴系下进行计算,此时本构关系为\eqref{eq:diag-two-axis},
就得到
\begin{equation}
    \frac{S_x S^x}{n_x^2 - \mu_r n_r^2} + \frac{S_y S^y}{n_y^2 - \mu_r n_r^2} + \frac{S_z S^z}{n_z^2 - \mu_r n_r^2} = 0
\end{equation}
它是使用$\vb*{S}$和$n_r$表示的\eqref{eq:fresnel-k-n}的对应物。
定义

\subsubsection{能量和能流}



\chapter{几何光学}

在已经知道了光的物理机制的时候,我们希望分析它作为波的性质,如传递方式等,并且严格处理直觉上看起来合理的“光线”、“波前”等概念。

\section{各向同性标量理论的几何光学}\label{sec:isotropic-scalar}

\subsection{各向同性标量理论的微分方程}

实际的波动方程都是矢量方程。然而,由于其线性性,只要边界条件没有将电磁场的不同分量混合起来,我们总是可以将我们要研究的光分解成三个方向上的偏振光,然后将标量理论使用在其中每一个方向上,然后将结果再综合起来。因此以下讨论标量波动方程
\[
    c^2 \laplacian u - \pdv[2]{u}{t} = - \rho(\vb*{r}, t),
\]
并且通常讨论远离源的区域。这种明确求解波动方程的做法称为\concept{波动光学},而后面关于光线等的讨论则是\concept{几何光学}。
设波源集中在某一个空间区域内,则此空间区域外成立有齐次波动方程。
考虑圆频率为$\omega$的单色波,指定$k = \omega / c$,则在无波源区域有
\[
    \laplacian u + k^2 u = 0
\]
它的解满足
\[
    u(\vb*{r}) = 
    \frac{1}{4\pi} \int_{\Sigma} u(\vb*{r}_0) \grad_0{\frac{\ee^{\ii k R}}{R}} \cdot \dd \vb*{S}_0 
    - \frac{1}{4\pi} \int_{\Sigma} \frac{\ee^{\ii k R}}{R} \grad_0{u(\vb*{r}_0)} \cdot \dd{\vb*{S}_0}
\]
其中$\Sigma$是能够包围所有波源的一个曲面,其中$R$为$\abs{\vb*{r} - \vb*{r}'}$,$\vb*{R}$为$\vb*{r} - \vb*{r}'$。
若讨论的问题的尺度远大于$1/k$%
\footnote{这意味着,比如说,我们感兴趣的区域到波源的距离远大于波长,两种介质的界面远大于波长且在波长的尺度上没有很大的起伏,障碍物相对波长很大,等等。}
,且$\Sigma$为等相位面(见后一节),那么就有
\begin{equation}
    E(\vb*{r}) = - \frac{\ii k}{4\pi} \int_\Sigma E(\vb*{r}_0) \frac{\ee^{\ii k R}}{R} \left( \vu*{R} \cdot \vb*{n}_0 + 1 \right) \dd{S_0}
    \label{eq:kirchhoff-formula}
\end{equation}

\subsection{空间中的相位和光程}\label{sec:phrase-and-l}

空间中每一点都在某一个等相位面上,各点对应的等相位面的法向量$\vb*{n}$就构成了一个矢量场,并且可以证明$\vb*{n}$和$\grad{E}$平行。
相位为
\[
    \phi = \frac{1}{\ii} \int \dd{\vb*{r}} \cdot \frac{\grad{E}}{E},
\]

很多时候,$E(\vb*{r})$可以写成
\begin{equation}
    E(\vb*{r}) = E_0(\vb*{r})\ee^{\ii \omega L(\vb*{r}) / c_0}
    \label{eq:generalized-plain-wave}
\end{equation}
这样的形式,其中$E_0(\vb*{r})$和$L(\vb*{r})$都是实数。此时
\[
    \grad{E} = E \left(\frac{\ii \omega}{c_0} \grad{L} + \frac{\grad{E_0}}{E_0}\right)
\]
\[
    \phi = \int \dd{\vb*{r}} \cdot \left( \frac{\omega}{c_0} \grad{L} + \frac{1}{\ii} \frac{\grad{E_0}}{E_0} \right) 
\]
表面上看这会引起矛盾,因为相位应该是一个实数。但是在我们讨论的问题的尺度远大于波长(指介质中平面波的波长,这是一个介质性质)时(大部分光学问题都满足这个条件,不满足这个条件的场景很少被称为光学),我们有$k \gg 1/R$%
\footnote{注意此处的$k$是一个介质性质,它和解\eqref{eq:generalized-plain-wave}没有直接联系——解\eqref{eq:generalized-plain-wave},到现在为止i,尚不能定义一个波矢。以下在不能定义一个确定无疑的波矢时所谓的$k$、$\lambda$等量,均指在同样的介质中通入平面波得到的值,它们由介质本身的性质,也就是$\epsilon$和$\mu$确定。}
,此时上式中第二项相比于第一项很小,于是
\begin{equation}
    \phi = \frac{\omega}{c_0} \int \dd{\vb*{r}} \cdot \grad{L}
    \label{eq:phrase-change}
\end{equation}
注意到被积函数是一个函数的梯度,因此能够在空间中每一点单值地定义$\phi$。
于是将$L$称为\concept{光程},它度量了变化给定的相位,光需要在真空中走过的距离。
从这个方程也可以得到
\[
    \grad{\phi} = \frac{\omega}{c_0} \grad{L}
\]
因此相位变化最快的方向——从而波阵面传播的方向——就沿着$\grad{L}$。并且我们也看到了光程和相位的换算关系
\begin{equation}
    \phi = \frac{\omega}{c_0} L
\end{equation}
实际上两者本来应该差一个常数因子,但是因为两者加上或者减去一个常数都不会影响实际的物理量,我们就附加一个条件,认为两者同时取到零值。

顺带我们也获得了一个结论:在$k \gg 1/R$时,或者说波长相对于空间尺度很短时,我们有
\begin{equation}
    \grad{E} = \frac{\ii \omega}{c_0} E \grad{L}
    \label{eq:e-change-l}
\end{equation}
也就是说,$E$的变化主要来自$L(\vb*{r})$的快速周期性变化,而和它前面的系数$E_0$的变化关系不大。需要注意的是这个公式实际上比\eqref{eq:kirchhoff-formula}更加粗糙,因此实际计算光强时通常使用\eqref{eq:kirchhoff-formula}而不是\eqref{eq:e-change-l}。

总之,本节内容意味着,可以将形如\eqref{eq:generalized-plain-wave}的解看成平面波,并且可以做替换
\begin{equation}
    \frac{\omega}{c_0} \grad{L} \longrightarrow \vb*{k}
    \label{eq:generalized-k}
\end{equation}
只要
% TODO:证明介质中的k和$\omega \grad{L} / c_0$同一量级

\subsection{边界对光程和相位的影响}

接下来我们转而观察边界上的现象,也就是说,讨论反射和折射中的相位变化。
在本节中我们同样假定空间尺度远大于波长,也就是$kr \gg 1$,这样做的一个好处就是,如果介质界面不是平面,我们总可以将其上每一点处的反射和折射都看成无穷大平面的反射和折射,并用这一点处的切平面作为这个假想的无穷大平面。
反射和折射的规律由\ref{sec:two-isotrophy-surface}节给出,但需要注意这一节中讨论的都是平面波的反射和折射,而我们却要讨论任意一种能够良好定义相位的波的反射和折射。
因此现在需要探讨\ref{sec:two-isotrophy-surface}节中的公式能不能适用。

\ref{sec:two-isotrophy-surface}节实际上是在一个无限大水平面上应用了边界条件\eqref{eq:e-bound-condition}。在本节中也可以把边界当成水平的。

同样还是只考虑方向$\vu*{E}$方向上的线偏光$\vb*{E} = E \vu*{E}$。
两侧介质中的电场分别设为
\[
    E_1 = E_{i0}(\vb*{r}) \ee^{\ii L_i(\vb*{r}) \omega / c_0} + E_{r0}(\vb*{r}) \ee^{\ii L_r(\vb*{r}) \omega / c_0}, \quad E_2 = E_{t0} \ee^{\ii L_t(\vb*{r}) \omega / c_0}
\]
并假设两侧介质中电场振动方向一致。同样由解的唯一性,如果做了以上假设之后有解,那么它们就是正确的假设。(要注意指数上都是$c_0$,因为光程是假定光束在真空中运行而得出的,它实际上是相位的一个代表,和具体介质无关)

由于假设波长远小于空间尺度,有
\[
    \begin{aligned}
        \curl{\vb*{E}_1} &= \curl{(E_1 \vu*{E})} = \grad{E_1} \times \vb*{E} \\
        &= \frac{\ii \omega}{c_0} (E_{i} \grad{L_i} + E_{r} \grad{L_r}) \times \vu*{E}, \\
        \curl{\vb*{E}_2} &= \curl{(E_2 \vu*{E})} = \grad{E_2} \times \vb*{E} \\
        &= \frac{\ii \omega}{c_0} E_{t} \grad{L_t} \times \vu*{E}
    \end{aligned}
\]
代入\eqref{eq:e-bound-condition}得到
\[
    \begin{aligned}
        \epsilon_1 E_{10} \vb*{n} \cdot \vu*{E} = \epsilon_2 E_{20} \vb*{n} \cdot \vu*{E}, \\
        E_{10} \vb*{n} \times \vu*{E} = E_{20} \vb*{n} \times \vu*{E}, \\
        \frac{1}{\mu_1} E_i \vb*{n} \times (\grad{L_i}) + \frac{1}{\mu_1} E_r \vb*{n} \times (\grad{L_r}) = \frac{1}{\mu_2} E_t \vb*{n} \times (\grad{L_t})
    \end{aligned}
\]
如果做替换\eqref{eq:generalized-k},以上三个方程的形式就和平面波反射完全一致。因此只需要在菲涅尔公式\eqref{eq:fresnel-formulas}中做这样的替换,就得到了形式和平面波折反射公式完全一致的公式。%
\footnote{一个更加物理的观点:我们可以在界面上找到这样一小块面积,在其上$E_0$也没有明显变化,因此在这块面积上可以将入射波看成一个平面波;另一方面由于假定$kR \gg 1$,把尺度拉到波长水平,这块面积看起来就是一个无限大的平面。于是在这块宏观小微观大的面积上就可以使用标准的平面波反射、折射理论。}
因此\eqref{eq:generalized-k}提供了波矢定义的推广。
于是公式\eqref{eq:phrase-change}可以简单地写成
\begin{equation}
    \phi = \int \vb*{k} \cdot \dd{\vb*{r}}
    \label{eq:phrase-change-k}
\end{equation}

要注意一个地方:正如入射波往往伴随着一个反射波一样,一块均匀介质中可能有多个形如\eqref{eq:generalized-plain-wave}的波,使用\eqref{eq:phrase-change}(从而\eqref{eq:phrase-change-k})可以给它们中的每一个都确定一个相位,但是却未必能够给它们的和确定一个相位。例如,似乎很难给一个驻波(可以看成传播方向相反的两路波的叠加)

由于\eqref{eq:phrase-change}是在没有介质性质突然改变的均匀空间中推导出来的,我们要问:当积分路径通过介质界面时,是不是还能够良定义一个相位,并且有\eqref{eq:phrase-change-k}成立?
也就是说我们要写出相位满足的边界条件。我们有
\[
    E_{t0} \ee^{\ii \omega L_t / c_0} = t E_{i0} \ee^{\ii \omega L_i / c_0}, \quad E_{r0} \ee^{\ii \omega L_r / c_0} = r E_{i0} \ee^{\ii \omega L_i / c_0}
\]
因此在边界上有
\begin{equation}
    \phi_t = \arg t + \phi_i, \quad \phi_r = \arg r + \phi_i
    \label{eq:phrase-on-surface}
\end{equation}
在通常情况下$t$是实数,因此在边界上透射波的相位和入射波保持一致;但是在入射介质的折射率大于出射介质时,$r$可能有复数部分,此时反射波的相位会发生一个突变。垂直入射或者掠入射会导致$r$近似有一个$\pi$的辐角,这就是所谓的半波损失。

当入射和出射介质都是各向同性的透明介质时,透射波的相位和和入射波的相位在边界上完全一样,因此习惯上将透射波当成入射波的延续,因为可以将透射波和入射波的相位拼接成一个没有突变的单值的标量场。

\subsection{几何光学与光线}

考虑介质中$\epsilon, \mu$变化得不是非常剧烈的情况,此时\eqref{eq:halmholtz-eq}成立。
需注意此处的$c=c(\vb*{r})$。
取形如$\vb*{E} = \vb*{A}(\vb*{r}) \exp (\ii \omega L(\vb*{r}) / c_0)$的解形式,其中$L$为实函数;我们将尽可能多的相位变化归入$\exp(\ii \omega L / c_0)$中,即认为$\vb*{A}(\vb*{r})$基本上是实的。
容易看出改变$L$的形式可以使用上面的解的形式叠加出任何函数。
代入\eqref{eq:halmholtz-eq}中,得到
\begin{equation}
    \ee^{\ii \frac{\omega}{c_0} L} \laplacian \vb*{A} 
    + 2 \frac{\ii \omega}{c_0} \ee^{\ii \frac{\omega}{c_0} L} \grad{\vb*{A}} \cdot \grad{L} 
    + \frac{\ii \omega}{c_0} \ee^{\ii \frac{\omega}{c_0} L} \vb*{A} 
    - \frac{\omega^2}{c_0^2} \ee^{\ii \frac{\omega}{c_0} L} \vb*{A} (\grad{L})^2 
    + \frac{\omega^2}{c^2} \ee^{\ii \frac{\omega}{c_0} L} \vb*{A} = 0.
    \label{eq:expanded-wave-eq}
\end{equation}
当$\omega$相对于我们讨论的空间尺度非常大时(或者说,波长充分短,直观地看就是不容易发生衍射等现象),就有
\begin{equation}
    (\grad{L})^2 = n^2 = \frac{c_0^2}{c^2}
    \label{eq:optical-distance}
\end{equation}
或者如果$n>0$,可以写成
\[
    \abs{\grad{L}} = n
\]
称其中的$n$为\concept{折射率}。\eqref{eq:light-ray}称为\concept{程函方程},$L$就是光程。

实际上,在已知$1/k \ll r$的时候,可以直接使用$\omega\grad{L}/c_0$在关于平面波的方程中代替$k$,因为此时$L$的变化是场强变化的主要因素。也就是说,我们把形如$\vb*{E}_0 \ee^{\ii L(r)}$的波在一个小区域内近似看成了平面波。

从这个方程我们还可以得到一个结论。注意到
\[
    \dd{\phi} = \dd{\vb*{r}} \cdot \vb*{\phi} = \frac{\omega}{c_0} \dd{\vb*{r}} \cdot \grad{L},
\]
因此如果取$\dd{\vb*{r}}$的方向和$\grad{L}$一致,方程右边就变成$\omega / c = k$,因此从一个相位为$\phi$的等相位面出发做位移
\begin{equation}
    \dd{\vb*{r}} = \frac{\dd{\phi}}{k}\vu*{k}
    \label{eq:phrase-movement}
\end{equation}
就得到相位为$\phi + \dd{\phi}$的等相位面。

从$\exp (\ii \omega L(\vb*{r}) / c_0)$的形式可以看出波动的前进方向正是沿着$\grad{L}$,也就是沿着$\vb*{k}$。
因此我们尝试计算$\grad{L}$的向量线,它指明了光大致的传播路径。设$\vb*{r}(s)$($s$是弧长)为一条这样路径(称为\concept{光线})。
考虑到它沿着$\grad{L}$,且
\[
    \abs{\dv{\vb*{r}}{s}} = 1
\],
有
\[
    \dv{\vb*{r}}{s} = \frac{\grad{L}}{\abs{\grad{L}}} = \frac{\grad{L}}{n},
\]
从而
\[
    \begin{aligned}
        \dv{s}\left(n \dv{\vb*{r}}{s}\right) &= \dv{\grad{L}}{s} \\
        &= \frac{\grad{L}(\vb*{r} + \dd \vb*{r}) - \grad{L}}{\dd s} \\
        &= \frac{\grad{(\dd \vb*{r} \cdot \grad{L})}}{\dd s} \\
        &= \grad{\left( \dv{\vb*{r}}{s} \cdot n \dv{\vb*{r}}{s} \right)}
    \end{aligned}
\]
于是得到\concept{光线方程}
\begin{equation}
    \dv{s} \left( n \dv{\vb*{r}}{s} \right) = \grad{n}
    \label{eq:light-ray}
\end{equation}

在求出光线方程之后,可以写出各点光程和相位。注意到
\[
    \abs{\vb*{k}} = \frac{\omega}{c_0} \abs{\grad{L}} = \frac{\omega}{c_0} n = \frac{\omega}{c}
\]
也就是说,即使波不是平面波,其波矢大小仍然和相同的介质中的平面波一样。而$\vb*{k}$的方向和$\dd{\vb*{r}} / \dd s$一致,因此我们发现
\begin{equation}
    \begin{aligned}
        \Delta \phi = \int \vb*{k} \cdot \dd{\vb*{r}} = \int_\text{light ray} k \dd{s}, \\
        \Delta L = \int_\text{light ray} n \dd s
    \end{aligned}
    \label{eq:light-ray-and-phrase}
\end{equation}

以上都是在讨论介质内部,现在考虑边界。由于光线的定义依赖于光程——从而依赖于相位——实际上也要分别给入射波、反射波、透射波定义一套光线。但是既然入射波和透射波共享相位,实际上只需要求出反射波的光线和入射波-透射波的光线即可。
反射波的光线的初始条件可以使用反射定律
\[
    \theta_i = \theta_t,
\]
透射波的光线的初始条件可以使用折射定律
\[
    \frac{\sin \theta_i}{\sin \theta_t} = \frac{n_t}{n_i}.
\]
描述。

\section{费马原理,以及它的量子化和波动光学}

\subsection{费马原理}

一个有趣的事实是,\eqref{eq:light-ray}正是变分原理
\begin{equation}
    \var \int n \dd s = 0
    \label{eq:fermet-principle}
\end{equation}
的解。这个变分原理称为\concept{费马原理},它表明,光线在折射率变化不太大的区域内总是让光程取极值——通常是最小值,但也有例外。
从\eqref{eq:fermet-principle}求微分方程时不建议使用弧长$s$为参数,因为做变分时微分测度会发生变化。比较好的方法是采用时间为参数,即根据
\[
    \var \int \dd{t} n \sqrt{\dot{\vb*{r}}^2} = 0,
\]
写出欧拉-拉格朗日方程
\[
    \dv{t}\left( n \frac{\dot{\vb*{r}}}{\abs*{\dot{\vb*{r}}}} \right) = \abs*{\dot{\vb*{r}}} \grad{n},
\]
注意到
\[
    \dd{s} = \abs*{\dot{\vb*{r}}} \dd{t},
\]
就推导出了\eqref{eq:light-ray}。

实际上,计算可以发现,正是将费马原理\eqref{eq:fermet-principle}应用在边界上得到的结果。
这就意味着空间中所有的光线都可以使用费马原理描述。

\subsection{费马原理的量子化}

\eqref{eq:fermet-principle}的这种广泛适用性显得很有意思。
\eqref{eq:fermet-principle}看起来似乎是一个“最小作用量”式的理论,并且能够导出几何光学,那么我们无疑要问,将它路径积分量子化之后能够得到什么。
我们自始至终没有做光场的量子化,因此肯定是拿不到“光是一份一份的”之类的物理的,自然也拿不到纠缠光之类的需要直积结构才能有的物理;然而,本文也不关心这些现象,因为我们一直在解麦克斯韦方程,相当于是在处理“单光子波函数”。
\eqref{eq:fermet-principle}是几何光学的代表,几何光学可以看成光子作为粒子的经典理论。因此我们推测\eqref{eq:fermet-principle}做量子化之后得到波动光学。
如果确实如此,那么这就解释了为什么\eqref{eq:fermet-principle}是如此普适,因为波动光学直接能够推导出\eqref{eq:fermet-principle},从而,的确,任何几何光学适用的地方\eqref{eq:fermet-principle}也适用。

不过,在开始尝试对\eqref{eq:fermet-principle}做路径积分量子化之前,需要先弄清楚我们是在做什么。
本文在做的实际上就是一件事:求解非时变的、无源(从而横场条件成立)的电动力学系统中的电磁波模式。
因此,所有的时间演化都是时谐的,从而都是平凡的;在实际的计算中从来没有出现真正的时间演化,也没有时间变量。
的确,$\grad{L}$大体上指出了光的前进方向,如果我们制备一个波包,其运动方向的确会由$\grad{L}$指导,因此似乎可以说几何光学可以看成光子作为粒子的经典理论。
但是,几何光学在处理的实际上仍然是电磁场的频率本征态,因此其中仍然没有“时间”的概念;我们是利用几何光学求解出光线轨迹之后,才手动再次引入时间的概念,让光的波包动起来的。
这里的情况和拉格朗日力学与莫陪都原理之间的关系很像:前者给出完整的时间演化,而后者只是给出轨迹。
几何光学更像后者而不是前者。实际上,根据莫陪都原理中简约作用量的定义,我们能够直接得到费马原理:
\begin{equation}
    S = \int p \dd{q} = \int \vb*{k} \cdot \dd{\vb*{r}} = \frac{\omega}{c} \int n \dd{s}.
    \label{eq:abbr-action-ray}
\end{equation}
上式中出现了$\omega$,这在拉格朗日力学中是匪夷所思的。
莫陪都原理似乎没有什么用,因为它不能给出有意义的量子理论。直接将\eqref{eq:fermet-principle}当作作用量而定义路径积分看起来是毫无意义的——其实不是这样,不过我们首先还是从正则量子化的角度分析如何量子化费马原理。

\eqref{eq:fermet-principle}在某种意义上\emph{可以}当成拉格朗日动力学中的作用量,只要我们把$(x, y, z)$中的其中一个坐标——通常是$z$——当成“时间”即可。
此时按照\eqref{eq:abbr-action-ray},设
\begin{equation}
    L = \sqrt{1 + \left(\pdv{x}{z}\right)^2 + \left(\pdv{y}{z}\right)^2},
\end{equation}
做勒让德变换,得到光线的经典哈密顿动力学:正则动量为
\begin{equation}
    p_x = \pdv{L}{x'} = \frac{n x'}{\sqrt{1 + x'^2 + y'^2}}, \quad p_y = \pdv{L}{y'} = \frac{n y'}{\sqrt{1 + x'^2 + y'^2}},
\end{equation}
哈密顿量为
\begin{equation}
    H = x' p_x + y' p_y - L = - \sqrt{n^2 - p_x^2 - p_y^2} ,
\end{equation}
这里我们已经用$f'$表示$\partial_z f$了。
再次提醒,这个哈密顿动力学并没有直接描述波包的运动情况。我们自始至终都是在分析频域的麦克斯韦方程,或者说在分析麦克斯韦方程的本征态。这段分析结束\emph{之后}我们才根据计算结果去讨论波包的动力学;波包的动力学中的时间不会出现在任何实际的光学计算中。
实际上,这个哈密顿动力学的时空是$2+1$维的。
然后我们做正则量子化。我们还需要指定一个和$\hbar$地位相同的量。
在波长很长,光的波动性很强时几何光学失效,正如$\hbar$的大小相对于系统作用量很大时经典力学失效一样,因此我们尝试取
\begin{equation}
    \hbar \longrightarrow \frac{\lambda}{2 \pi} = \frac{c}{\omega},
\end{equation}
则在坐标表象下(由于是$2+1$维时空,只有$x, y$两个坐标)有
\begin{equation}
    p_x \longrightarrow - \ii \frac{c}{\omega} \pdv{x}, \quad p_y \longrightarrow - \ii \frac{c}{\omega} \pdv{y},
\end{equation}
就得到坐标表象下的演化方程
\[
    \ii \frac{c}{\omega} \pdv{z} \psi = - \sqrt{n^2 + \frac{c^2}{\omega^2} \pdv[2]{x} + \frac{c^2}{\omega^2} \pdv[2]{y}},
\]
平方就得到正确的亥姆霍兹方程
\[
    \left( \pdv[2]{x} + \pdv[2]{y} + \pdv[2]{z} + \frac{n^2 \omega^2}{c^2} \right) \psi = 0.
\]
因此,的确,把$z$当成$t$,把$c/\omega$当成$\hbar$而做正则量子化,确实能够得到波动光学,不过这里的波动光学是亥姆霍兹方程,其中没有真正的时间演化,$\omega$已经确定,时间由$z$充当的,且“波函数”$\psi$实际上是我们关系的那个偏振方向上的电场分量$u$。

相应的,做路径积分量子化,则传播子由
\begin{equation}
    \mel{\vb*{r}(z_2)}{S}{\vb*{r}(z_1)} = \frac{1}{Z} \int \mathcal{D} {\vb*{r}(z)} \exp(\ii \frac{\omega}{c} \int_1^2 n \dd{s})
\end{equation}
给出。或者,用在波动力学中更加常见的写法,我们有
\begin{equation}
    u(\vb*{r}) = \int \dd{S'} u(\vb*{r}') \alpha(\vb*{r}, \vb*{r}') \int \mathcal{D} {\vb*{r}(t)} \exp(\ii \frac{\omega}{c} \int_{\vb*{r}'}^{\vb*{r}} n \dd{s})
    \label{eq:field-propagating}
\end{equation}
上式中出现的$t$是一个任意的参数,并非实际时间(虽然完全可以将它取为光运动的时间,但是这个时间的地位和拉格朗日力学或是哈密顿力学中的时间是不同的);$\vb*{r}$和$\vb*{r}'$是曲线$\vb*{r}(t)$的末端和初端,$\vb*{r}'$遍及所有按照光线指向在$\vb*{r}$“之前”的。
\eqref{eq:field-propagating}给人强烈的感觉,似乎振幅沿着光线从前往后传播一样,不过其中还是没有显式出现时间:我们还是只是在分析亥姆霍兹方程的本征态而已。
一些书(如费曼的科普书QED: The Strange Theory of Light and Matter)会说“随着时间演进,光的相位如何如何变化”,但是这里的“时间演进”实际上只是用时间充当了$t$而已。
$\alpha$因子给出适当的归一化。我们在此不计算其具体形式,\cite{Gitin_2013}详细计算了它。

\subsection{费马原理的量子化为何能够工作}

以上事实——将费马原理或者说几何光学量子化能够得到亥姆霍兹方程——其实是有些奇怪的,因为没有任何理由让我们相信费马原理\emph{应该}正则量子化。

实际上,一种更加清晰易懂的关于费马原理的量子化的论述是这样的。路径积分中虽然允许能量不守恒的路径,但是这些路径由于权重彼此抵消,对最终结果的贡献并不大,或者有时候严格为零。
因此,在计算路径积分时,做WKB近似——认定路径上能量守恒,将作用量写成简约作用量
\begin{equation}
    S' = \int p(E, q) \dd{q}
\end{equation}
这种做法——很多时候是能够得到很好的效果的。我们马上看到对单光子这就是\eqref{eq:abbr-action-ray}。
原来如此——\eqref{eq:abbr-action-ray}确实是能够出现在路径积分的$\ee$指数上的!

然后,注意到由于$z$方向动量算符的形式$- \ii \partial_z$和时间演化的$\ii \partial_t$形式完全相同,将$z$坐标选择为一个(虚设的)“时间”在形式上是合理的。
从严格的电动力学理论退化得到的路径积分理论
\[
    Z = \int \mathcal{D}\vb*{r}(z) \exp(\ii \frac{\omega}{c} \int n \dd{s})
\]
中什么是时间取决于我们想怎样计算关联函数——这不会改变理论的形式。因此我们将$z$当成时间,那么该路径积分理论——费马原理的路径积分量子化——显然应该等价于以$z$为时间,以光程关于时间$z$的勒让德变换为哈密顿量的一个正则量子化理论。
这就解释了对费马原理做正则量子化的正确性。

以上步骤对单电子量子力学也是完全适用的。实际上,在那里,只需要做替换
\[
    \frac{\omega}{c} n \longrightarrow \sqrt{2m(E-V)}
\]
即可——我们据此立刻可以得到定态薛定谔方程,它之于非定态薛定谔方程,正如亥姆霍兹方程之于标量波动方程。

以上步骤最终给出定态波动方程并不是没有来由的。做了WKB近似的路径积分描述的大体上就是一个频率基本确定(可以有一定展宽)的波包如何运动。
这样的一个波包的运动方式当然也可以通过定态波动方程描述,因为数学上可以证明,偏微分方程的定态的波阵面的传播方式就是偏微分方程中的扰动的传播方式,即扰动导致的不连续面遵循程函近似给出的光线方程。

总之,几何光学虽然通常通过对标量波动方程做程函近似得到,但是它至少提供了以下信息:
\begin{itemize}
    \item 电磁场模式的波阵面形式,其中光线是波阵面的切矢量的矢量线;
    \item 有外源情况下的稳态电磁场模式的波阵面形式,或者说外源产生的介质中的稳定光束的轨迹;
    \item 考虑“光线的量子涨落”(在$\omega$固定,以$z$为时间的意义下,光线是作为2+1维“量子理论”的亥姆霍兹方程对应的“经典理论”),得到“二维空间上的光线波函数随着时间(也就是$z$坐标)的演化”,即电磁场模式;
    \item 考虑“光线的量子涨落”(在$\omega$固定,以$z$为时间的意义下),得到有外源情况下的稳态电磁场模式;
    \item 光场中扰动的传播方式。(从而光线可以用于绘制波包的运动轨迹——见\cite{born2013principles}关于电磁场中的间断的讨论)
\end{itemize}
这些信息是彼此兼容的。例如,既然几何光学能够描述外源下稳态光束的分布,而光束可以看成一系列波包的组合,每个波包可以认为是一个脉冲产生的,既然波包彼此没有相互作用,因此每个波包的运动轨迹一定是在光束内部的,从而,单独一个波包的运行轨迹也一定是沿着光线的。
例如,这样我们就知道,光线是可以不止一条的,如果介质中折射率在变化,那么会有反射,从而一个脉冲也能够被反射。
这也是很多教材在通过程函近似得到几何光学后,直接用它处理外源引入的光束的轨迹、光场中脉冲的传播而不加说明的原因:这确实是正确的。

\section{耗散}

还留有一个尚未解决的问题:遮挡物。所谓遮挡物就是完全吸收射到其上的光,不留下
TODO:例如金属表面/粗糙表面

\subsubsection{波动光学的不同精度的求解}

至此我们已经获得了一套在几何尺度远大于波长的情况下不求解麦克斯韦方程而能够获得相当精确的结果的方法。总结如下:
\begin{enumerate}
    \item 将电场分解成三个方向上的线偏光;
    \item 找到波源,在波源外找到一个等相位面,通常可以使用对称性分析或者渐进性质确定,例如很远处的星光可以看成是平面波,柱状灯管外的光看成柱面波,等等;如果波源形状复杂,也可以将它分解成许多简单的小波源,然后观察其中一个产生的光的等相位面;
    \item 计算这个等相位面上各点光强;在光源对称性好的时候可以直接通过能量守恒和对称性算出来;
    \item 计算出空间中各点的相位,这可以通过下面几种方式完成:
    \begin{itemize}
        \item 求解程函方程\eqref{eq:light-ray},然后计算积分\eqref{eq:light-ray-and-phrase};
        \item 通过费马原理\eqref{eq:fermet-principle}求出光线然后计算积分\eqref{eq:light-ray-and-phrase};
        \item 首先求出一族波阵面或者说等相位面,按照\eqref{eq:phrase-movement}获得一系列相位标定好的等相位面,然后据此在空间中各点作出$\vb*{k}$,画出光线;
    \end{itemize}
    这一步说明了为什么即使是在求解波动光学的问题时,仍然可以使用几何光学中“光路”、“光线”的概念%
    \footnote{有人可能会说几何光学解释不了小孔衍射之类的现象;
    实际上,如果将障碍物中的折射率视为无限大,那么通过费马原理可以很明确地导出“光直线射向小孔,拐了个弯射向目标点”,同样可以导出衍射。
    几何光学解释不了衍射指的是不能明确地写出不同位置的场强,从而不能写出衍射条带,但这并没有什么奇怪的,因为几何光学根本就不涉及场强!}
    ;
    \item 在划分好等相位面之后使用\eqref{eq:kirchhoff-formula}式求出各点场强
    (注意这样计算出来的就是带有相位信息的$\vb*{E}$,而不是\eqref{eq:generalized-plain-wave}中的$\vb*{E}_0$);
    \item 将所有分解之后的场叠加回去。
\end{enumerate}

还留有一个尚未解决的问题:遮挡物。所谓遮挡物就是完全吸收射到其上的光,不留下
TODO:例如金属表面/粗糙表面

色散会导致波包展宽等,这些又是几何光学无法描述的现象了。

\section{光强和成像}

\subsection{几何光学中的光强传输}

实际上,我们也可以将\eqref{eq:optical-distance}当成$L$的\emph{定义},此时无论几何光学是不是适用,都能够定义光线等。在此基础上,\eqref{eq:expanded-wave-eq}化为
\begin{equation}
    \laplacian \vb*{A} + 2 \frac{\ii \omega}{c_0} \grad{\vb*{A}} \cdot \grad{L} + \frac{\ii \omega}{c_0} \vb*{A} = 0.
\end{equation}
如果我们做\concept{慢变振幅近似},则上式可以写成
\begin{equation}
    \div{(\vb*{A}^2 \grad{L})} = 0.
\end{equation}
因此实际上$\vb*{A}^2 \grad{L}$可以看成一个静态的“光强传输的流”,即光线实际上给出了光强传输的流管。
物理地看,由于是各向同性介质,$\vb*{S}$和$\vb*{k}$平行,这意味着$\vu*{k}$的方向连缀而成的“光线”实际上就是$\vb*{S}$的流量线,因此若干光线包围出的“管路”实际上就是能量流动的流管。

当两束光相交时,光线可以交叉,因此“流管”的概念实际上并没有良好定义——实际上,这正是衍射、干涉等现象的起源。% TODO

因此,在慢变振幅近似成立时,使用几何光学确定光线,并以光线为振幅传输的“流线”就足够给出可靠的结果了。
我们通常称\emph{此时几何光学适用}。理论上我们对任何体系都可以通过光线方程计算光线,但是最终我们关注的是空间中各点的亮度分布,因此如果光线的概念无助于计算亮度分布,则几何光学没有什么意义,这就好像对任何一个量子理论我们都可以计算其经典版本,但是有时候计算经典版本并不能提供什么信息。

\section{各向异性理论}

各向异性介质中的光学完全由\eqref{eq:anisotropy}确定。但是这个方程过于复杂了,很难直接分析,而且我们通常更愿意处理标量理论。

同样还是考虑形如\eqref{eq:generalized-plain-wave}的解。
由于\ref{sec:phrase-and-l}节实际上并不依赖经典的标量波动方程,在各向异性理论中它对形如\eqref{eq:generalized-plain-wave}的解还是适用的。

\subsection{单光轴晶体的标量理论}

首先从比较简单的单光轴晶体出发。
由于同一个$\vu*{k}$对应着两个不同振动方向的平面波,它当然也对应两个不同振动方向的\eqref{eq:generalized-plain-wave}形式的波。
我们尝试分析每一个振动方向上的波。由于振动方向是已经确定的,可以将矢量理论化归为标量理论。
也就是说我们有
\[
    \vb*{E} = E \vu*{E} = E_0(\vb*{r}) \ee^{\ii \omega L(\vb*{r}) / c_0} \vu*{E}
\]
为了简明起见我们尽可能尝试消去$\grad{L}$。

首先考虑o光。在\eqref{eq:o-light}中做替换\eqref{eq:generalized-k},就得到
\[
    \vu*{E} \cdot \grad{L} = 0, \quad \vu*{E} \cdot \vb*{e}_z = 0, \quad \abs{\grad{L}} = n_0,
\]
另一方面,各向异性波动方程\eqref{eq:anisotropy}在主轴系下为
\[
    \grad{(\vu*{E} \cdot \grad{E})} - \laplacian E \vu*{E} = \omega^2 E \mu \vb*{\epsilon} \cdot \vu*{E}
\]
考虑到\eqref{eq:one-axis-matrix},我们有
\[
    \mu \vb*{\epsilon} \cdot \vu*{E} = \frac{1}{c_0} n_o^2 \vu*{E},
\]
而
\[
    \grad{E} = \frac{\ii \omega}{c_0} E \grad{L}
\]
综合以上各式,得到
\begin{equation}
    \laplacian E + \frac{\omega^2 n_o^2}{c_0^2} E = 0.
    \label{eq:o-light-wave-eq}
\end{equation}
因此o光的行为就好像一个以$n_o$为折射率的介质中的波。因此\ref{sec:isotropic-scalar}节中的所有结论都适用于o光。
相应的,我们知道o光的波阵面为球形,其方程就是\eqref{eq:o-light}中做变量代换\eqref{eq:generalized-k}之后得到的
\begin{equation}
    \abs{\grad{L}} = n_0.
\end{equation}

下面再看e光。实际上由于我们有\eqref{eq:e-light},可以直接写出e光的波阵面方程为
\begin{equation}
    \frac{1}{n_e^2} \left( \pdv{L}{x} \right)^2 + \frac{1}{n_e^2} \left( \pdv{L}{y} \right)^2 + \frac{1}{n_o^2} \left( \pdv{L}{z} \right)^2 = 1
\end{equation}
也就是说e光的波阵面是一个椭球(当然,它的退化情况就是普通的平面波,正如\eqref{eq:e-light}所示)。

e光中能量传播的速度满足
\begin{equation}
    \frac{v_x^2}{v_e^2} + \frac{v_y^2}{v_e^2} + \frac{v_z^2}{v_o^2} = 1
\end{equation}

TODO:e光的惠更斯原理:椭球面的切点对应着$\vb*{S}$,从而光线,的方向。

\chapter{干涉,衍射和经典相干理论}

假定光学系统在每个瞬时都可以认为是处在某个稳态上,从而,虽然我们在分析动态问题,沿用亥姆霍兹方程的解足以给出精确的结果。
然而,假定光学系统中的光源会以一种随机的方式发射电磁波,从而空间中某一点的光强实际上是一系列具有不同概率权重的亥姆霍兹方程的解在这一点给出的光强的期望值。
写成公式,设某一点的电场包含从各个方向传来的电场(注意此时我们已经切换到了亥姆霍兹方程下,所谓“电场传播”实际上是电场在空间上的联系,虽然它和时域下波包的传播是直接相关的)之和
\begin{equation}
    \vb*{E}(\vb*{r}) = \sum_i \vb*{E}_i \ee^{\ii k_i \abs*{\vb*{r} - \vb*{r}_i}},
\end{equation}
展开电场平方的期望值,有
\begin{equation}
    \begin{aligned}
        \expval*{\vb*{E}(\vb*{r})^2} &= \sum_P P(\vb*{E}_1, \vb*{E}_2, \cdots, \vb*{E}_n) \sum_{i, j} \vb*{E}_i \ee^{\ii k_i \abs*{\vb*{r} - \vb*{r}_i}} \vb*{E}_j^* \ee^{- \ii k_j \abs*{\vb*{r} - \vb*{r}_j}}  \\
        &= \sum_{i, j} \expval*{\vb*{E}_i \vb*{E}_j^*} \ee^{\ii (k_i \abs*{\vb*{r} - \vb*{r}_i} - k_j \abs*{\vb*{r} - \vb*{r}_j})}.
    \end{aligned}
\end{equation}
在组成$\vb*{E}$的各个组分极度非相干的情况下,$\expval*{\vb*{E}_i \vb*{E}_i^*}$项占据压倒性优势,从而
\begin{equation}
    \expval*{\vb*{E}^2} = \sum_i \expval*{\vb*{E}_i^2}.
\end{equation}
这就是说,对高度非相干的情况,不同来源的光强可以直接相加,而无需考虑衍射等问题。
这实际上说明对高度非相干的光几何光学通常都是适用的。但这并不是说以高度非相干的光为光源就产生不了干涉和衍射,例如我们将一束非相干光分束,得到的两束光中,一束光的一个分量和另一束光中的一个分量相干,从而仍然可能产生干涉。
必须每两个$\vb*{E}_i$和$\vb*{E}_j$之间都不相干才能够保证几何光学总是成立。

由于几何光学成立时光线实际上就是光功率的流线,如果在一个闭合表面上光功率通量正比于光线条数,那么做光线追踪后会发现任何一个面上的光通量也都正比于这个面上的光线条数。

何时衍射不明显:波长非常短的时候肯定不明显;相干性差的时候也是;这和“量子效应什么时候不明显”是类似的



\chapter{光学谐振腔}

前几章的电磁波传播都是没有任何边界条件约束的,此时如果介质均匀,那么电磁波可以以平面波的形式稳定传播。
本章则讨论束缚在有限区域内的电磁波模式。

原则上当然可以通过求解关于标势和矢势的方程来分析谐振腔的行为,例如解\eqref{eq:wave-eq}即可。
通常谐振腔内部没有源,于是\eqref{eq:wave-eq}中的各个方程都是齐次的,且各个矢势分量之间没有关系,从而似乎求解标量亥姆霍兹方程即可获知谐振腔的行为。
然而,边界条件实际上还是会让各个矢势分量和标势等都产生关系。
因此使用标势和矢势并不能简化问题。于是,本章还是直接求解\eqref{eq:e-in-tensor-material}。
特别的,各向同性体态中有
\begin{equation}
    \laplacian \vb*{E} + k_0^2 \vb*{E} = 0, \quad \div{\vb*{E}} = 0, \quad k_0 = \frac{\omega}{c},
    \label{eq:isotropic-cavity-problem-origin}
\end{equation}
这里$\vb*{E} = \vb*{E}(\vb*{r}, \omega)$。
通常我们会定义
\begin{equation}
    \tilde{\vb*{E}} = \sqrt{\epsilon} \vb*{E}(\vb*{r}, \omega), \quad \tilde{\vb*{B}} = \frac{\ii}{\sqrt{\mu}} \vb*{B}(\vb*{r}, \omega)
\end{equation}
来让$\vb*{E}$和$\vb*{B}$相差不要太大,此时
\begin{equation}
    \tilde{\vb*{B}} = \frac{1}{k_0} \curl{\tilde{\vb*{E}}}.
    \label{eq:tilde-b-tilde-e-cavity}
\end{equation}
将\eqref{eq:isotropic-cavity-problem-origin}用$\tilde{\vb*{E}}$和$\tilde{\vb*{B}}$写出来,就是
\begin{equation}
    \begin{bigcase}
        &\laplacian \tilde{\vb*{E}} + k_0^2 \tilde{\vb*{E}} = 0, \quad \laplacian \tilde{\vb*{B}} + k_0^2 \tilde{\vb*{B}} = 0, \\
        &\tilde{\vb*{B}} = \frac{1}{k_0} \curl{\tilde{\vb*{E}}}, \quad \tilde{\vb*{E}} = \frac{1}{k_0} \curl{\tilde{\vb*{B}}}.
    \end{bigcase}
    \label{eq:isotropic-cavity-problem}
\end{equation}
最后一个方程只需要在倒数第二个方程两边作用散度,通过一些计算即可看出。

\section{方形谐振腔}

\subsection{立方体谐振腔}

\subsection{方形波导}

\concept{波导}是指一个长条状的、电磁波可以在其中传递的装置。我们讨论一个柱状的波导,它是一个形状任意的闭合曲线沿着垂直于截面的方向平移而形成的直导管,其壁为导体,内部填充了某种均匀介质。
基本上,能够称为电磁波的电磁场构型都有很强的趋肤效应,因此接下来如无特殊说明我们认为波导的壁是理想导体,即认为导体内部没有任何场分布,即认为边界条件为
\begin{equation}
    \vb*{n} \cdot \vb*{B} = 0, \quad \vb*{n} \times \vb*{E} = 0.
\end{equation}
表面上,从边界条件$\vb*{n} \cdot (\vb*{D}_1 - \vb*{D}_2) = 0$出发,并利用导体内部没有电场分布这一条件,似乎可以得到$\vb*{n} \cdot \vb*{D}_1=0$,但这是错误的:如果我们要求$\vb*{n} \cdot (\vb*{D}_1 - \vb*{D}_2) = 0$成立,即将导体内部的电流归入$\epsilon$中,即给$\epsilon$一个虚部,那么随着导体电导率的上升,导体内部的$\vb*{E}$的确会下降,但是$\epsilon$会上升,最后在边界上会留下一个不为零的$\vb*{n} \cdot \vb*{D}_2$。
而如果我们不将电流归入$\epsilon$,那么就有$\vb*{n} \cdot (\vb*{D}_2 - \vb*{D}_1) = \sigma$,而$\vb*{D}_2$正常地衰减,于是边界条件就是$\vb*{n} \cdot \vb*{D}_1 = \sigma$。
我们并不知道$\sigma$到底是什么,因此这个边界条件实际上是用来在$\vb*{E}$已知后返回来求解$\sigma$的。
$\vb*{n} \times (\vb*{H}_2 - \vb*{H}_1) = \vb*{j}$同理。

考虑时谐场。由于$z$方向上的平移不变性,我们可以认为
\[
    \vb*{E}, \vb*{B} \propto \ee^{\ii (k_z z - \omega t)},
\]
在导管内部,波动方程为
\begin{equation}
    \left( \pdv[2]{x} + \pdv[2]{y} + k_0^2 - k_z^2 \right) \pmqty{\vb*{E} \\ \vb*{B}} = 0, \quad k_0 = \frac{\omega}{c}.
\end{equation}
以上方程并不能定解,但是实际上通过使用方程$\curl{\vb*{E}}=-\partial_t \vb*{B}$以及$\curl{\vb*{H}} = \partial_t \vb*{E}$,$x, y$方向上的场可以写成$z$方向上的场及其导数的线性函数,因此我们只需要求解
\begin{equation}
    \left( \pdv[2]{x} + \pdv[2]{y} + k_\text{c}^2 \right) \pmqty{E_z \\ B_z} = 0, \quad k_\text{c}^2 = k_0^2 - k_z^2.
\end{equation}
我们不能指望$E_z$和$B_z$都是零,因为此时没有非平庸解,即波导的约束意味着严格的横波是不可能的。
可能的偏振模式可以分成$B_z=0$的\concept{横磁波}(TM)和$E_z=0$的\concept{横电波}(TE)两类。

简单的计算表明对横电波我们有(本段中所有的$\grad$都是二维平面上的,我们暂时忽略电磁场在$z$方向上的周期性波动)
\begin{equation}
    \vb*{B}_\text{t} = \frac{\ii k_z}{k_\text{c}^2} \grad{B_z}, \quad \vb*{E}_\text{t} = - \ii \frac{c k_0}{k_\text{c}^2} \vb*{e}_z \times \grad{B_z},
\end{equation}
于是从$\vb*{n} \cdot \vb*{B} = 0$得到$B_z$满足的边界条件
\begin{equation}
    \pdv{B_z}{n} = 0,
\end{equation}
并且这个条件也能够让$\vb*{n} \times \vb*{E} = 0$成立,于是据此条件求解$B_z$满足的亥姆霍兹方程就确定了一切。对横磁波类似的有
\begin{equation}
    \vb*{E}_\text{t} = \frac{\ii k_z}{k_\text{c}^2} \grad{E_z}, \quad \vb*{B}_\text{t} = \ii \frac{k_0}{ck_\text{c}^2} \vb*{e}_z \times \grad{E_z},
\end{equation}
边界条件为
\begin{equation}
    E_z = 0.
\end{equation}
这个边界条件是$\vb*{n} \times \vb*{E}=0$的直接推论,但是由于它让$\grad{E_z}$在边界上一定沿着$\vb*{n}$,可以验证$\vb*{n} \cdot \vb*{B}=0$也是成立的。

在$xy$平面上求解可能的TE或是TM模式,得到的是离散谱,而电磁场在$z$方向的传播却是散射态,即$\omega$和$k_z$都可以连续取值,于是波导内的模式的能谱形如
\begin{equation}
    \omega = c \sqrt{k_z^2 + k_{\text{c}, mn}^2},
\end{equation}
其中$m, n$为标记$xy$平面上的模式的整数编号。可以看到这个能谱是有能隙的,能量低于
\begin{equation}
    \omega_\text{c} = \min (c k_{\text{c}, mn})
\end{equation}
的电磁波入射波导之后会快速衰减。

\section{导引矢量法}\label{sec:guiding-vector}

注意到问题\eqref{eq:isotropic-cavity-problem}中$\tilde{\vb*{E}}$和$\tilde{\vb*{B}}$的形式高度对称,我们可以尝试通过一个特殊的构造产生它的解。

由于\eqref{eq:isotropic-cavity-problem}中的电场和磁场均满足横波条件,它们总是可以写成某个东西的旋度。
设某个矢量场$\vb*{M}$是某个东西的旋度,即满足
\begin{equation}
    {\vb*{M}} = \curl{(\psi \vb*{c})},
    \label{eq:guiding-vector-construction}
\end{equation}
这样关于$\vb*{M}$的亥姆霍兹方程就变为
\begin{equation}
    0 = \laplacian {\vb*{M}} + k_0^2 {\vb*{M}} = \curl{((\laplacian \psi + k_0^2 \psi) \vb*{c} + \psi \laplacian \vb*{c})}.
    \label{eq:m-helmholtz-original}
\end{equation}
注意此处$\vb*{c}$和$\psi$的定义不唯一。我们总是能够找到(虽然一般都不容易解析地找到)一个$\vb*{c}$满足
\begin{equation}
    \div{\vb*{c}} = \text{const}, \quad \curl{\vb*{c}} = 0.
\end{equation}
这是因为,设
\[
    \vb*{M} = \curl{\vb*{M}'},
\]
我们注意到关于某个标量场$\lambda$的方程
\[
    \div{(\lambda \vb*{M}')} = \text{const}, \quad \curl{(\lambda \vb*{M}')} = 0
\]
一定有解,因为第二个方程的三个分量方程实际上只有两个独立。
因此,我们设
\[
    \vb*{M}' = \psi \vb*{c}, \quad \psi = \frac{1}{\lambda}
\]
即可得到\eqref{eq:guiding-vector-construction},即\eqref{eq:guiding-vector-construction}中的$\vb*{c}$和$\psi$总是可以构造出来的。
我们称$\vb*{c}$为\concept{导引矢量},因为它大体上描绘了$\vb*{M}$的“指向”。

对$\vb*{c}$我们有
\[
    \laplacian \vb*{c} = \grad(\div{\vb*{c}}) - \curl{(\curl{\vb*{c}})} = 0,
\]
于是
\begin{equation}
    \vb*{M} = \grad{\psi} \times \vb*{c}, \quad \vb*{M} \bot \vb*{c},
\end{equation}
关于$\vb*{M}$的亥姆霍兹方程\eqref{eq:m-helmholtz-original}等价于
\begin{equation}
    \laplacian \psi + k_0^2 \psi = 0.
    \label{eq:scalar-cavity-eq}
\end{equation}
在解出$\vb*{M}$之后我们会注意到矢量场
\begin{equation}
    \vb*{N} = \frac{1}{k_0} \curl{\vb*{M}}
    \label{eq:cavity-n-def}
\end{equation}
满足
\begin{equation}
    \vb*{M} = \frac{1}{k_0} \curl{\vb*{N}}, 
\end{equation}
并且它也满足和$\vb*{M}$满足的亥姆霍兹方程完全一样的方程
\begin{equation}
    \laplacian \vb*{N} + k_0^2 \vb*{N} = 0.
\end{equation}

对比$\vb*{M}$和$\vb*{N}$矢量场满足的各个方程和\eqref{eq:isotropic-cavity-problem},我们发现可以将$\vb*{M}$看成$\tilde{\vb*{E}}$,将$\vb*{N}$看成$\tilde{\vb*{B}}$,也可以反过来将$\vb*{M}$看成$\tilde{\vb*{B}}$,将$\vb*{N}$看成$\tilde{\vb*{E}}$。
因此我们得到了一种原则上一般的求解光学谐振腔中的模式的方法:求解标量方程\eqref{eq:scalar-cavity-eq},然后将各个模式代入\eqref{eq:guiding-vector-construction}和\eqref{eq:cavity-n-def},这样就得到了全部的电磁波模式。

\subsection{平凡的例子:平面波}

平面波的经验给出了一种挑选$\vb*{c}$的方法:尽可能让$\vb*{c}$垂直于主要的界面方向。

\subsection{圆柱波导中的柱面波}

在圆柱波导中可以验证将$\vb*{c}$选择为
\begin{equation}
    \vb*{c} = \vb*{e}_\rho
\end{equation}
是可行的,此时需要求解
\begin{equation}
    \frac{1}{\rho} \pdv{\rho} \left( \rho \pdv{\psi}{\rho} \right) + \frac{1}{\rho^2} \pdv[2]{\psi}{\phi} + \pdv[2]{\psi}{z} + k_0^2 \psi = 0.
\end{equation}
这个方程的求解是已知的:它最终转化为柱贝塞尔方程的求解。


我们分析几种极限情况。$k \rho \to 0$的情况对应于在我们关心的距离尺度内电磁波的传播速度可以忽略的情况(或者$\rho$很小,或者$c$很大以至于$k$很小),而$x \to 0$时 % TODO:贝塞尔函数的渐近行为
此时的解就是静电势的通解。

圆柱波导中的电磁波模式本质上还是标量的。

\subsection{球面波}

选取
\begin{equation}
    \vb*{c} = r \vb*{e}_r.
\end{equation}
关于$\psi$的亥姆霍兹方程为
\begin{equation}
    \frac{1}{r^2} \pdv{r}\left( r^2 \pdv{\psi}{r} \right) + \frac{1}{r^2 \sin \theta} \pdv{\theta} \left( \sin \theta \pdv{\psi}{\theta} \right) + \frac{1}{r^2 \sin \theta} \pdv[2]{\psi}{\phi} + k_0^2 \psi = 0.
\end{equation}

以上求解过程说明三维球腔中不存在s波。
从数学上看这来自所谓\emph{毛球定理}:$S^1$上可以有一个连续而处处不为零的切向量场,但是$S^2$上不可能有这样的切向量场。

我们将球面波推导出的$\vb*{N}$和$\vb*{M}$称为\concept{球波函数}。
球波函数显然可以用于做多极矩展开。实际上,它比我们前面通过泰勒级数得到的多极矩展开更加优越,因为后者只在远场情况下能够毫无疑难地定义,在近场时会有一定的模糊性。
一个重要的例子就是\concept{环形磁偶极矩}。

\chapter{非均匀折射率导致的散射}

\part{光源}\label{part:source}

\chapter{激光}

本节首先介绍强光如何能够产生。

如果我们能够有一个长期保持粒子数反转的系统(显然需要持续的能量输入),那么向这个系统入射一束光将会产生更强的出射光,因为会有受激发射,且受激发射出的光和入射光是非常相干的。
因此我们向粒子数反转的系统注入的能量可以用于增强入射光,并且如果入射光相干性非常好,那么出射光也保持非常好的相干性,并且比入射光更亮,这就是\concept{激光}。

最简单的方案——使用一个二能级系统,直接通过一次入射来得到激光——是不现实的,因为此时大量的能量会消耗在保持粒子数反转上,而为了 TODO
两个可能的改进:将粒子数反转的系统放在一个四壁强反射的腔体内,从而光束可以来回走,不断被增强,并且使用一个三能级系统,其中间能级是一个亚稳态,从而很容易制造粒子数反转。
可以在腔体的一个地方“开洞”——比如说让反射率稍微低一些——让一些光子泄露出来,激光就从这里被导出。

多光子过程:可以避免光学屏障,以及排除荧光本底

\part{非线性光学}

\chapter{非线性极化}

极化矢量和电场之间的关系当然不完全是线性的。我们

\section{非线性光学过程的经典模型}\label{sec:classical-models}

\subsection{非线性谐振子模型}\label{sec:classical-oscillator}

本节将材料当成一系列谐振子的组合,并且暂时不考虑谐振子之间的相互作用。
这是相对合理的,因为能够长距离传输光的介质一般不是金属,从而电子是相对定域的。
然而,这并不意味着我们的理论是自由的。
我们知道一个标准的经典谐振子可以用
\begin{equation}
    m \dv[2]{x}{t} + m \gamma \dv{x}{t} + m \omega_0^2 x = q E
\end{equation}
来描述,而如果我们加上诸如$x^3$这样的项,即让谐振子的回复力为非线性的,就可以造成谐振子模式发生自相互作用。
我们讨论的问题的能量都并不高,谐振子运动不会特别快,因此可以认为谐振子只产生电场,并且其方式为“电偶极子产生库伦场”。
用作用量表示,就是% TODO:有误,这里的关键在于qxE可以给出电场对电荷的作用,但是是否能够给出电荷对电场的作用??或者说,电场和电荷的相互作用拉氏量或是哈密顿量要怎么写??
\[
    S = \int \dd{t} \left( \frac{1}{2} m \dot{x}^2 - \frac{1}{2} k x^2 + \text{higher order $x^n$} + qxE \right).
\]
现在积掉谐振子,就能够得到非线性的光子-光子过程,即多个光子和一或是一个光子分裂为多个光子。

\subsubsection{二阶非线性极化}

我们现在在谐振子能量中加入一个三次项,即在运动方程中加入一个二次回复力:
\begin{equation}
    m \dv[2]{x}{t} + m \gamma \dv{x}{t} + m \omega_0^2 x + m a x^2 = q E.
    \label{eq:x3-eq}
\end{equation}
这相当于在能量中加入了一个$\frac{1}{3} m a x^3$项。这个项破坏了系统的中心反演对称性。
我们实际上是要从电场计算$x$(从$x$计算响应电场的公式是显然的,就是大量偶极子加总为$\vb*{P}$然后从$\vb*{P}$出发算电场)。

对这个问题的标准的处理是微扰求解微分方程,但是实际上可以使用费曼图分析这个问题。由于只考虑经典情况,无需计算圈图。
“经典情况”到底指的是什么需要进一步说明:在经典情况下我们没有二次量子化,粒子图景的经典理论和场的图景的经典理论还不一样。
粒子图景下运动方程是关于各个粒子的位置和动量的,入射和出射外线没有任何限制。
场的图景下,我们求解系统的基本自由度(在这里是电场和谐振子坐标)的运动方程,得到场变量随时间的变化情况,即实际上在求解$\expval*{\phi}$,因此只能有一条出射外线,入射外线应当被当成外源。
在经典极限下这两种图景不会造成太大差别:同一张图的外线数目是固定的,在场的图景下,出射外线多了外源就少,由于我们要求场强满足$\phi / \hbar \ll 1$(但与此同时能标又没有高到多顶角图非常重要,从而圈图修正有必要计算),外源较少的过程是非常不重要的。

我们采用后一种图景,因为我们实际上就是在微扰求解\eqref{eq:x3-eq}。
将电磁场和带有$x^3$形势能的非线性振子耦合,则费曼图中应该有\autoref{fig:x3-vertex}和\autoref{fig:light-osci-couple}两种基本元件。
应当注意这里的短直线代表的是$x$的某个频率的分量,如果做量子化,就是一个谐振子模式。
这里的传播子并不代表谐振子的状态本身。这也就是\autoref{fig:light-osci-couple}顶角中只有一条短直线而不是两条的原因:它代表一个入射光子激发出一个谐振子模式,而不是谐振子整体吸收一个光子之后变成另一个状态。
由于电场和谐振子的耦合是完全线性的(并且由于谐振子是一个没有空间分布的点,实际上耦合项就是电偶极子能量),且我们关心的是“非线性介质中有哪些光学过程”,可以将电场暂时当成背景场,于是\autoref{fig:light-osci-couple}应该被\autoref{fig:external-field}取代。
例如,我们只需要分析一阶过程\autoref{fig:first-order-x3-external}就能知道\autoref{fig:first-order-x3-photon}的来源——如果$E$让$x$产生非线性响应,那就有光子分裂和合并的过程。

费曼规则可以很容易地写出:(我们认为频率为$\omega$的成分携带$\ee^{- \ii \omega t}$因子)
\begin{itemize}
    \item 传播子为
    \[
        \begin{tikzpicture}
            \begin{feynhand}
                \vertex (a) at (0, 0);
                \vertex (b) at (1, 0);
                \propag [plain, mom={$\omega$}] (a) to (b); 
            \end{feynhand}
        \end{tikzpicture} = \frac{\ii}{m (\omega^2 + \ii \gamma \omega - \omega_0^2)}.
    \]
    \item 顶角为
    \[
        \begin{tikzpicture}
            \begin{feynhand}
                \vertex (a) at (-1,-1); \vertex (b) at (1,-1); \vertex (c) at (0,1);
                \vertex (o) at (0,0); 
                \propag [plain] (a) to (o);
                \propag [plain] (b) to (o); 
                \propag [plain] (c) to (o);    
            \end{feynhand}
        \end{tikzpicture} = - \ii 2 m a \cdot 2\pi \delta(\sum \omega).
    \]
    注意正常情况下$x^3$相互作用要配一个$1/3!$的因子但是这里只有$1/3$,因此顶角实际上是$2ma$而不是$ma$。
    \item 外源为
    \[
        \begin{tikzpicture}
            \begin{feynhand}
                \vertex [crossdot] (a) at (0, 0){};
                \vertex (b) at (1, 0);
                \propag [plain, mom={$\omega$}] (a) to (b); 
            \end{feynhand}
        \end{tikzpicture} = \ii q E(\omega).
    \]
    请注意这里没有负号,而$x^3$是负号的,这是因为均匀电场会倾向于把谐振子拉向无穷远处而回复力则会将谐振子拉回来。
    本节采取的傅里叶变换约定为
    \[
        E(t) = \int \dd{\omega} E(\omega) \ee^{- \ii \omega t},
    \]
    没有加入$2\pi$是因为很多时候入射光并不是连续谱,而是离散的几个频域分量加起来。
\end{itemize}

\begin{figure}
    \centering
    \subfigure[$x^3$自相互作用顶角]{
        \begin{tikzpicture}
            \begin{feynhand}
                \vertex (a) at (-1,-1); \vertex (b) at (1,-1); \vertex (c) at (0,1);
                \vertex [dot] (o) at (0,0) {}; 
                \propag [plain] (a) to (o);
                \propag [plain] (b) to (o); 
                \propag [plain] (c) to (o);    
            \end{feynhand}
        \end{tikzpicture}
        \label{fig:x3-vertex}
    }
    \subfigure[光子激发出一个谐振子模式]{
        \begin{tikzpicture}
            \begin{feynhand}
                \vertex (a) at (-1, 1.5);
                \vertex (b) at (0, 1.5);
                \vertex (c) at (1, 1.5);
                \propag [photon] (a) to (b);
                \propag [plain] (b) to (c);
            \end{feynhand}
        \end{tikzpicture}
        \label{fig:light-osci-couple}
    }
    \subfigure[外源驱动谐振子,即\autoref{fig:light-osci-couple}中的光子被当成无动力学的外场后得到的图形]{
        \begin{tikzpicture}
            \begin{feynhand}
                \vertex [crossdot] (a) at (0, 0) {};
                \vertex (b) at (1, 0);
                \propag [plain] (a) to (b);
            \end{feynhand}
        \end{tikzpicture}
        \label{fig:external-field}
    }
    \caption{加入$\frac{1}{3} m a x^3$势能之后的费曼图元件}
\end{figure}

\begin{figure}
    \centering
    \subfigure[外场导致的响应的一阶近似]{
        \begin{tikzpicture}
            \begin{feynhand}
                \vertex [crossdot] (a) at (-1,-1) {};
                \vertex [crossdot] (b) at (1,-1) {}; 
                \vertex (c) at (0,1);
                \vertex (o) at (0,0) ; 
                \propag [plain, mom={$\omega_1$}] (a) to (o);
                \propag [plain, mom={$\omega_2$}] (b) to (o); 
                \propag [plain, mom={$\omega_1 + \omega_2$}] (o) to (c);
            \end{feynhand}
        \end{tikzpicture}
        \label{fig:first-order-x3-external}
    }
    \subfigure[\autoref{fig:first-order-x3-external}导致的非线性光学过程]{
        \begin{tikzpicture}
            \begin{feynhand}
                \vertex (a0) at (-1.5, -1.5);
                \vertex (a) at (-0.5,-0.5);
                \vertex (b0) at (1.5, -1.5);
                \vertex (b) at (0.5,-0.5); 
                \vertex (c) at (0,0.5);
                \vertex (c0) at (0, 1.5);
                \vertex (o) at (0,0) ; 
                \propag [photon, mom={$\omega_1$}] (a0) to (a);
                \propag [plain] (a) to (o);
                \propag [photon, mom={$\omega_2$}] (b0) to (b);
                \propag [plain] (b) to (o); 
                \propag [plain] (o) to (c);
                \propag [photon, mom={$\omega_1 + \omega_2$}] (c) to (c0);
            \end{feynhand}
        \end{tikzpicture}
        \label{fig:first-order-x3-photon}
    }
    \caption{一阶过程}
    \label{fig:x3-first-order}
\end{figure}

据此,线性响应(零阶,没有发生任何非线性效应)为
\begin{equation}
    x_1(\omega) = \ii q E(\omega) \frac{\ii}{m (\omega^2 + \ii \gamma \omega - \omega_0^2)} = \frac{q / m}{\omega_0^2 - \omega^2 - \ii \gamma \omega} E(\omega).
\end{equation}
一阶过程(即\autoref{fig:first-order-x3-external})给出如下修正:
\begin{equation}
    \begin{aligned}
        x_2(\omega) &= \frac{1}{2} \int \dd{\omega_1} \int \dd{\omega_2} (\ii q E(\omega_1)) (\ii q E(\omega_2)) \frac{\ii}{m(\omega_1^2 + \ii \gamma \omega_1 - \omega_0^2)} \frac{\ii}{m(\omega_2^2 + \ii \gamma \omega_2 - \omega_0^2)} \\ 
        &\quad \quad \times (-\ii 2 m a) \frac{\ii}{(m(\omega_1 + \omega_2)^2 + \ii \gamma (\omega_1 + \omega_2) - \omega_0^2)} 2\pi \delta(\omega_1 + \omega_2 - \omega) \\
        &= \int \dd{\omega_1} \int \dd{\omega_2} \frac{a (q / m)^2}{(\omega_1^2 + \ii \gamma \omega_1 - \omega_0^2) (\omega_2^2 + \ii \gamma \omega_2 - \omega_0^2) (\omega^2 + \ii \gamma \omega - \omega_0^2)}  \\
        &\quad \quad \times 2\pi \delta(\omega_1 + \omega_2 - \omega) \times E(\omega_1) E(\omega_2).
    \end{aligned} 
    \label{eq:continuous-x3-first-order}
\end{equation}
这里需要注意一点:\autoref{fig:first-order-x3-external}中外场出现了两次,而
\[
    E(t)^2 = \int \dd{\omega_1} \int \dd{\omega_2} E(\omega_1) E(\omega_2) \ee^{-\ii (\omega_1 + \omega_2) t},
\]
如果$\omega_1 \neq \omega_2$那么$E(\omega_1) E(\omega_2)$项实际上会被求和两次;同样,此时\eqref{eq:continuous-x3-first-order}中的$E(\omega_1) E(\omega_2)$项也会被求和两次。
直观地看,外场是给定的而不能随意交换,所以\autoref{fig:first-order-x3-external}中的两个外场从左到右为$\omega_1$和$\omega_2$的图和从左到右为$\omega_2$和$\omega_1$的图虽然给出一样的结果,但是是两张图,不能看成一张图,它们加起来会导致因子$2$出现。
如果我们令$E(t)$实际上只有两个频率分量,这一点会显得尤其明显。
$\omega_1 = \omega_2$的情况中没有因子$2$,我们常将这样的过程称为\emph{简并的}。

现在我们采取更加常规的,微扰求解微分方程的做法。费曼图计算已经告诉我们主要的光学过程来自\autoref{fig:first-order-x3-external}。
因此,我们将输入的$E$设置为
\begin{equation}
    E = E_1 (\ee^{\ii \omega_1 t} + \ee^{-\ii \omega_1 t}) + E_2 (\ee^{\ii \omega_2 t} + \ee^{-\ii \omega_2 t}), 
    \label{eq:input-two-freq-e}
\end{equation}
做展开
\begin{equation}
    x = x_1 + x_2 + \ldots, \quad x_n \sim E^{n },
\end{equation}
并记
\begin{equation}
    x_i = \sum_n x_i(\omega_n) + \text{c.c.}.
\end{equation}
线性项$x_1$由
\[
    m \ddot{x}_1 + m \gamma \dot{x_1} +  m \omega_0^2 x_1 = q E
\]
给出,为
\begin{equation}
    x_1(\omega_1) = \frac{(q/m) E_i}{\omega_0^2 } \ee^{-\ii \omega_n t},
\end{equation}
二阶项由
% TODO:懒得写了
这些项分别称为:
\begin{itemize}
    \item \concept{和频(SFG, sum frequency generation)},
    \item \concept{差频(DFG, difference frequency generation)},
    \item \concept{倍频(SHG, second harmonic generation)},
    \item \concept{光学整流(OR, optic rectification)}(因为输入交流波而得到直流波,那当然是整流了)。
\end{itemize}
倍频是和频的特殊情况,光学整流是差频的特殊情况。当然,直接将\eqref{eq:input-two-freq-e}代入\eqref{eq:continuous-x3-first-order}也能够得到这些过程。

乍一看,费曼图方法不仅能够得到两个光子合并为一个光子的过程,也能够得到一个光子分裂成两个光子的过程(所谓的\concept{SPDC过程(Spontaneous parametric down-conversion)},也称为\concept{OPG过程(Optical parametric generation)}),但是我们后面将看到,微分方程方法似乎只能给出两个光子合并为一个光子的过程——如果我们在$E$中放入只有一个频率$\omega_0$的波,那么非线性效应似乎只会给出$\omega=0$和$\omega=2\omega_0$两种波。
但是其实这里并没有矛盾:SPDC过程需要两条出射外线;如前所述,我们采用场的图景,由于我们采用微分方程的写法,即从$E$求解$x$(其实是$\expval*{x}$),然后用“谐振子位置的偏离导致极化电场产生”计算总电场的变化,对$x$也要采用场的图景,所以的确只应该考虑只有一条外线的费曼图。
这暗示着SPDC过程实际上是非常弱的(本该如此,和频过程的振幅正比于$E^2$而SPDC过程的振幅正比于$E$),因此在经典图景下这个过程根本就不会出现。%
\footnote{
    从这里也可以看出光学中量子理论的重要性,即使我们讨论的能标自始至终都没有高到让只有QED才有的过程(如四光子等效相互作用)出现。
    经典理论对电磁波的描述是非常粗糙的:如果我们要描述一个物理状态中有两种不同频率的光子,应该怎么做?
    在经典理论中只有一种方法:设
    \[
        \vb*{E} = \vb*{E}_1 \ee^{\ii \omega_1 t} + \vb*{E}_2 \ee^{\ii \omega_2 t} + \text{c.c.}.
    \]
    现在如果要将从一个单频波到以上状态的过程画成费曼图,由于只能画一条外线的图,势必只能画出$\omega_0 \to \omega_1$和$\omega_0 \to \omega_2$两个图,然后能量就不守恒了。
}%

这并不是说SPDC过程——或者说OPG过程——在适用经典近似的体系中完全看不到,因为我们可以在OPG过程后面再放一个DFG过程。
DFG过程也可以称为\concept{OPA过程(Optical parametric amplification)},因为它让入射的两束光的一束变弱而另一束变强。
例如,设我们希望将一束频率为$\omega_1$的光分裂成两束光,频率分别是$\omega_2$和$\omega_3$。
我们可以将一个有二阶非线性极化的光学晶体放在一个内壁对频率为$\omega_2$和$\omega_3$的反射率很高的谐振腔中。
按照后面会提到的\eqref{eq:sfg-intensity},如果有相位匹配条件成立,那么$\omega_1 \to \omega_2 + \omega_3$的OPG过程转化效率很高(在那里是SFG过程效率很高,这里就是OPG过程转化效率很高),于是产生足够强的$\omega_2$光束和$\omega_3$光束,这些光束被谐振腔反射回来,回到非线性晶体内部,于是发生很强的OPA过程。
因此,我们仅仅通过一束单频入射光就得到了两束不同频率的出射光。
% TODO:怎么定量算?
在经典理论中OPA过程是允许的,因此时间反演对称性并没有丧失:的的确确可以有光子的分裂。
但是,经典理论中所有电磁波模式上的光子都是足够多的,因此从“完全没有光子”到“有一个光子”的过程在经典理论中无法被描述。这就是OPG过程看不到的原因。
换而言之,经典理论中的光子分裂,即OPA,不仅需要入射的泵浦光,还需要一个(直观上看,引导泵浦光分裂成哪些频率的光的)\concept{种子光}。
一旦种子光入射了,随着光的传播它会增强。

\begin{figure}
    \centering
    \subfigure[外场导致的响应的二阶近似]{
        \begin{tikzpicture}
            \begin{feynhand}
                \vertex [crossdot] (a) at (-1,-1) {};
                \vertex [crossdot] (b) at (1,-1) {}; 
                \vertex (c) at (0,1);
                \vertex (o) at (0,0) ;
                \vertex [crossdot] (e) at (1, 2) {};
                \vertex  (f) at (-1, 2) ; 
                \propag [plain, mom={$\omega_1$}] (a) to (o);
                \propag [plain, mom={$\omega_2$}] (b) to (o); 
                \propag [plain, mom={$\omega_1 + \omega_2$}] (o) to (c);
                \propag [plain, mom={$\omega_3$}] (e) to (c);
                \propag [plain, mom={$\omega_1 + \omega_2 + \omega_3$}] (c) to (f);
            \end{feynhand}
        \end{tikzpicture}
        \label{fig:second-order-x3-external}
    }
    \subfigure[\autoref{fig:second-order-x3-external}导致的非线性光学过程]{
        \begin{tikzpicture}
            \begin{feynhand}
                \vertex (a0) at (-1.5, -1.5);
                \vertex (a) at (-0.5,-0.5);
                \vertex (b0) at (1.5, -1.5);
                \vertex (b) at (0.5,-0.5); 
                \vertex (c) at (0,0.5);
                \vertex (o) at (0,0) ; 
                \vertex (d) at (0.5, 1);
                \vertex (e) at (-0.5, 1);
                \vertex (d0) at (1.5, 2);
                \vertex (e0) at (-1.5, 2);
                \propag [photon, mom={$\omega_1$}] (a0) to (a);
                \propag [plain] (a) to (o);
                \propag [photon, mom={$\omega_2$}] (b0) to (b);
                \propag [plain] (b) to (o); 
                \propag [plain] (o) to (c);
                \propag [plain] (c) to (e);
                \propag [plain] (c) to (d);
                \propag [photon, mom={$\omega_3$}] (d0) to (d);
                \propag [photon, mom={$\omega_1 + \omega_2 + \omega_3$}] (e) to (e0);
            \end{feynhand}
        \end{tikzpicture}
        \label{fig:second-order-x3-photon}
    }
    \caption{二阶过程}
    \label{fig:x3-second-order}
\end{figure}

还可以进一步往上计算微扰。例如,二阶微扰将给出\autoref{fig:x3-second-order}。
这里给出的四光子相互作用和\autoref{fig:first-order-x3-photon}产生的等效四光子相互作用不同,后者需要两个光子先合并,产生的光子传播一会,然后再和另一个光子合并。
\autoref{fig:second-order-x3-photon}给出的四光子相互作用是直截了当的。

我们来对各阶微扰的量级做一个估计。如果$\omega$和$\omega_0$比较接近,那么微扰论根本就不适用:此时共振发生,$x$是非常大的,可能高阶修正比低阶修正还大。
此时需要从头做光和物质耦合的计算而不能使用加入微弱非线性因素的振子模型。
如果$\omega$远大于$\omega_0$,我们将得到等离子体,此时彼此无关的、振幅不大的振子的图像更加失效了,可能晶格都已经被破坏了,电子的运动状况主要受电场控制。
在这两种情况下本节给出的非线性振子模型都不适用。(等离子体情况下有另一个非线性来源,即协变导数的输运项;见后文)
对$\omega \ll \omega_0$的情况,线性响应的振幅的量级为
\[
    x_1 \sim \frac{(q/m) E}{\omega_0^2},
\]
而
\[
    x_2 \sim \frac{a (q/m)^2 E^2}{\omega_0^6},
\]
因此
\begin{equation}
    \frac{x_2}{x_1} \sim \frac{a q E}{m \omega_0^4}.
\end{equation}
设原子对电子的束缚电场的量级为$E_\text{atom}$,则总位移$x$的振幅可以估计为
\[
    q E_\text{atom} \sim m \omega_0^2 x .
\]
$x$的量级具体有多大是不确定的,它包括没有外加电场时由$q E_\text{atom}$做回复力的内禀振荡,线性响应$x_1$和非线性响应$x_2$。
我们不妨采取一个非常极端的假设,认为线性回复力和非线性回复力已经一样大了(如果非线性回复力很小,那么当然只需要计算一阶图),此时
\[
    m \omega_0^2 x \sim m a x^2,
\]
于是
\[
    q E_\text{atom} \sim m \omega_0^2 \frac{\omega_0^2}{a},
\]
最后
\begin{equation}
    \frac{x_2}{x_1} \sim \frac{E}{E_\text{atom}}.
\end{equation}
通常原子内部电场的数量级为\SI{3e8}{V/m},因此即使认为非线性回复力和线性回复力一样大,一般来说$x_2$也远小于$x_1$,即此时非线性极化相对于线性极化来说还是不大的。
介质的光学性能由极化给出,和回复力没有直接关系,因此非线性极化一般来说总是比非线性极化小得多的。
类似地实际上可以证明
\begin{equation}
    \frac{x_{n+1}}{x_n} \sim \frac{E}{E_\text{atom}}.
\end{equation}

\subsubsection{三阶非线性极化}

\subsection{自由电子气的输运项}

本节将自由电子气视为带电荷的连续介质。非线性效应来自$(\vb*{v} \cdot \grad) \vb*{v}$。
我们写下电荷的运动方程:
\begin{equation}
    \pdv{\vb*{v}}{t} + (\vb*{v} \cdot \grad) \vb*{v} = - e (\vb*{E} + \vb*{v} \times \vb*{B}),
\end{equation}
这里会导致非线性效应的包括输运项$(\vb*{v} \cdot \grad) \vb*{v}$和洛伦兹力项。

我们现在加载单频周期性外场。设频率为$\omega$的电场分量为
\begin{equation}
    \vb*{E}(\omega) = A \ee^{- \ii \omega t} + A^* \ee^{\ii \omega t},
\end{equation}
在忽略所有微扰的情况下显然
\begin{equation}
    \vb*{v}_1 = \frac{e \vb*{E}}{\ii m \omega}.
\end{equation}
于是
\[
    \pdv{\vb*{v}_2}{t} + (\vb*{v}_1 \cdot \grad) \vb*{v}_1 = - e \vb*{v}_1 \times \vb*{B},
\]
最终计算得到
\begin{equation}
    \vb*{j}(2\omega) = \rho_1 \vb*{v}_1 + \rho_0 \vb*{v}_2 = \frac{\epsilon_0 e}{\ii m \omega} (\div{\vb*{E}(\omega)}) \vb*{E}(\omega) + \rho_0 \frac{\ii}{4\omega} \div(\vb*{E}(\omega) \cdot \vb*{E}(\omega)).
    \label{eq:double-freq-current}
\end{equation}

\subsection{金属表面效应}

金属表面也能够产生二次谐波。将金属中的电子视为上一节讨论的带负电荷的连续介质。
金属表面发生了一个突变:内侧是带负电荷的连续介质而外侧则什么也没有。
由于金属外侧没有任何电荷分布,以下提到电流密度等时都是在金属内部讨论。

在金属表面附近,平行于表面的电流密度分量只有\eqref{eq:double-freq-current}的第一项,于是
\begin{equation}
    \vb*{j}_\parallel(2\omega) = \frac{\epsilon_0 e}{\ii m \omega} \pdv{E_\bot(\omega)}{z} \vb*{E}_\parallel(\omega) , 
\end{equation}
对它求积分,得到
\[
    \int_{0^-}^{0^+} \dd{z} \vb*{j}_\parallel(2\omega) = \frac{\epsilon_0 e}{\ii m \omega} \vb*{E}_{\parallel} (\omega) E_\bot(\omega) (1 - \epsilon),
\]
而垂直分量则是
\begin{equation}
    \vb*{j}_\bot(2\omega) = \frac{\epsilon_0 e}{\ii m \omega} \pdv{E_z}{z} E_z + \rho \frac{\ii}{4\omega} \left(\frac{e}{m\omega}\right)^2 \pdv{z} (E_z^2),
\end{equation}

\section{非线性极化的量子理论}

\subsection{单电子系统与经典光场耦合}

绝缘体中,电子-电子库伦散射一般来说是不重要的(除了在一些比较特殊的点由于库仑相互作用打开能隙等)。
从而,对透明晶体——一般来说不导电——以及普通的不导电的气体,的光学性能的计算通常可以使用束缚态单电子模型,即电子的本征态由能级编号标记,然后可以分析光如何让电子在能级之间跃迁。
\autoref{sec:classical-oscillator}中我们以光子和经典谐振子的振动模式为基本自由度,通过为谐振子引入一个非二次型的势来得到非线性效应。
这种做法在量子理论中当然也是成立的,并且此时谐振子的振动模式真的就像一个个粒子一样。
然而,应当注意,这种“光子和谐振子振动模式相互作用,谐振子的振动模式通过非简谐的势能相互作用”的理论并不是最方便的,因为无论为振子——在这里实际上就是原子中的电子——引入怎样的非线性势,由于是束缚态,电子自身的能谱都可以被一系列能级完整描述,并且这些能级是比较容易算出来的。
光子与电子碰撞会让电子从一个能级跳到另一个能级,并且能量守恒条件——其中电子的能量由已经经过非二次型势能修正的能级给出——必须成立。
这意味着,首先考虑非简谐的势能的作用,计算出电子能级,然后考虑光子让电子在这些能级之间跃迁,是更加方便的。
这个图景在经典理论中无法使用,因为此时的光子吸收相互作用顶角有一条入射线,两条出射线,从而一个有光子出射的过程一定有多条出射线(至少一条光子线,以及一条雷打不动的电子出射线),从而无法在经典理论中表达。

在基于电子能级的图景中,非线性光学效应来自高阶微扰论,因为相互作用顶角上连接了两条电子线和一条光子线,从而,一张费曼图中可以有数量任意的入射和出射光子线,另一方面,电子线除了和光子相互作用以外,没有别的相互作用。
这和基于非线性振子的图景非常不同,在后者中一个光子只能连接到一条代表振子振动模式(而不是电子本身)的内线上,即光子到振子振动模式的转换始终是线性的,但是振子振动模式之间可以碰撞,从而有非线性过程。
遮去电子线,将电子的非线性响应表示为宏观的“极化”,以上两种图景统一变成了\autoref{sec:non-linear-maxwell}中的图景。
电子与光场通过电偶极跃迁耦合,在耦合哈密顿量中电场是线性的,从而,电子电偶极矩的$n$次方就对应一个$n$光子顶角。

\subsubsection{电子状态的含时微扰论}

本节考虑一个受到经典电磁场扰动的单电子系统。设系统一开始位于某个态$\ket*{g}$上,电磁场扰动会让它在各个瞬时的状态变得不确定起来。
我们用$m, n$等标记电子能级,用$p, q$等标记光子模式,并做傅里叶分解
\begin{equation}
    \vb*{E}(t) = \sum_p \vb*{E}(\omega_p) \ee^{- \ii \omega_p t}.
\end{equation}

我们首先计算电场扰动下的电子波函数$\ket*{\psi}$。单光子吸收过程$\braket*{m}{\psi^{(1)}}$为
\begin{equation}
    \begin{aligned}
        \begin{gathered}
            \begin{tikzpicture}
                \begin{feynhand}
                    \vertex (a) at (-1.3, 0) {$g$};
                    \vertex (o) at (0, 0);
                    \vertex (b) at (1.3, 0) {$m$};
                    \vertex (c) at (-0.5, 0.87) ;
                    
                    \propag[fermion] (a) to (o);
                    \propag[fermion] (o) to (b);
                    \propag[photon, mom={$p$}] (c) to (o);
                \end{feynhand}
            \end{tikzpicture}
        \end{gathered} &= \frac{1}{\hbar} \sum_p \frac{1}{\omega_g + \omega_p - \omega_m} \mel{m}{- \vb*{d} \cdot \vb*{E}(\omega_p)}{g} \ee^{- \ii (\omega_g + \omega_p - \omega_m) t} \\
        &= \frac{1}{\hbar} \sum_p \frac{\vb*{d}_{mg} \cdot \vb*{E}(\omega_p)}{\omega_{mg} - \omega_p} \ee^{\ii (\omega_{mg} - \omega_p) t},
    \end{aligned}
\end{equation}
双光子吸收过程$\braket*{n}{\psi^{(2)}}$为%
\footnote{
    这里使用的实际上是time ordered perturbation theory, 但是它的各阶微扰和covariant perturbation theory基本上是一样的。
    有两种方式看出这一点。我们可以将开头的$\ket{\text{g}}$也看成一个外线,从而第一个传播子中的$\omega_g + \omega_p$可以看成通过能量守恒定律计算出的能量,而$\omega_m$可以看成从能谱中读出的$m$模式的能量,整个传播子和$\omega - \vb*{p}^2 / 2m$是差不多的。
    这样末端的$l$线可以看成“有外场存在时的自发极化”,然后从此出发即可计算出$\chi$。
    我们也可以在covariant perturbation theory中积分掉各种中间变量来得到推导出此处的分母。
}%
\begin{equation}
    \begin{aligned}
        \begin{gathered}
            \begin{tikzpicture}
                \begin{feynhand}
                    \vertex (a) at (-1.3, 0) {$g$};
                    \vertex (o1) at (0, 0);
                    \vertex (o2) at (1.7, 0);
                    \vertex (b) at (3.0, 0) {$n$};
                    \vertex (c) at (-0.5, 0.87) ;
                    \vertex (d) at (1.2, 0.87);
                    
                    \propag[fermion] (a) to (o1);
                    \propag[fermion] (o1) to[edge label={$m$}] (o2);
                    \propag[fermion] (o2) to (b);
                    \propag[photon, mom={$p$}] (c) to (o1);
                    \propag[photon, mom={$q$}] (d) to (o2);
                \end{feynhand}
            \end{tikzpicture}
        \end{gathered} &= \frac{1}{\hbar^2} \sum_{p, q} \sum_m \frac{1}{\omega_g + \omega_p + \omega_q - \omega_n} \mel{n}{- \vb*{d} \cdot \vb*{E}(\omega_p)}{m} \\ 
        &\quad \quad \times \frac{1}{\omega_g + \omega_p - \omega_m} \mel{m}{- \vb*{d} \cdot \vb*{E}(\omega_p)}{g} \ee^{- \ii (\omega_g + \omega_p + \omega_q - \omega_n) t} \\
        &= \frac{1}{\hbar^2} \sum_{p, q} \sum_m \frac{(\vb*{d}_{nm} \cdot \vb*{E}(\omega_q)) (\vb*{d}_{mg} \cdot \vb*{E}(\omega_p))}{(\omega_{ng} - \omega_q - \omega_p) (\omega_{mg} - \omega_p)} \ee^{\ii (\omega_{ng} - \omega_p - \omega_q) t},
    \end{aligned}
\end{equation}
同理还能够得到三光子吸收过程$\braket*{l}{\psi}$为
\begin{equation}
    \begin{aligned}
        &\quad \begin{gathered}
            \begin{tikzpicture}
                \begin{feynhand}
                    \vertex (a) at (-1.3, 0) {$g$};
                    \vertex (o1) at (0, 0);
                    \vertex (o2) at (1.7, 0);
                    \vertex (o3) at (3.4, 0);
                    \vertex (b) at (4.7, 0) {$l$};
                    \vertex (c) at (-0.5, 0.87) ;
                    \vertex (d) at (1.2, 0.87);
                    \vertex (e) at (2.9, 0.87);
                    
                    \propag[fermion] (a) to (o1);
                    \propag[fermion] (o1) to[edge label={$m$}] (o2);
                    \propag[fermion] (o2) to[edge label={$n$}] (o3);
                    \propag[fermion] (o3) to (b);
                    \propag[photon, mom={$p$}] (c) to (o1);
                    \propag[photon, mom={$q$}] (d) to (o2);
                    \propag[photon, mom={$r$}] (e) to (o3);
                \end{feynhand}
            \end{tikzpicture}
        \end{gathered} \\
        &= \frac{1}{\hbar^3} \sum_{p, q, r} \sum_{m, n} \frac{(\vb*{d}_{ln} \cdot \vb*{E}(\omega_r)) (\vb*{d}_{nm} \cdot \vb*{E}(\omega_q)) (\vb*{d}_{mg} \cdot \vb*{E}(\omega_p))}{(\omega_{lg} - \omega_p - \omega_q - \omega_r) (\omega_{ng} - \omega_q - \omega_p) (\omega_{mg} - \omega_p)} \ee^{\ii (\omega_{lg} - \omega_p - \omega_q - \omega_r) t}.
    \end{aligned}
\end{equation}

以上三个过程都是严格按照电偶极辐射哈密顿量计算出来的;实际的系统中除了电偶极辐射以外,还有各种各样的噪声扰动。
我们假定噪声的主要效果是让系统倾向于自动地弛豫到基态,于是唯象地向传播子中加入有限大小的虚部$\ii \gamma_{mn}$以产生某种阻尼,为了简化书写,令
\begin{equation}
    \omega_{mn} = \omega_{m} - \omega_{n} - \ii \gamma_{mn},
\end{equation}
从而以上三个过程的表达式仍然是正确的,但是此时$\omega_{mn}$具有虚部,其复共轭不等于它本身。
这在计算期望值$\mel*{\psi}{\cdot}{\psi}$时非常重要。
我们还假定
\begin{equation}
    \gamma_{mn} = \gamma_{nm},
\end{equation}
这个假设的合理性需要在\autoref{sec:electron-density-matrix}中看到。

我们在这里用电子波函数描述电子,而用经典的电磁波描述光,一方面没有做光场量子化就能够得到非线性响应,一方面确保了非线性响应的量子本质能够被体现出来。
在经典理论中是画不出以上三个过程的,原因是显然的:在经典理论中光和电子的耦合方式就是\autoref{fig:light-osci-couple},只有这个顶角的话,在经典的费曼图(严格区分“先发生”和“后发生”的过程,从而禁止相当一大类圈图)是画不出来电子吸收多个光子的过程的。
我们此处用实线表示的实际上是\emph{电子场}而不是\emph{电子坐标},或者说不是(像\autoref{fig:light-osci-couple}那样的)电子的振动模式。
二次量子化之后电场和电子的耦合实际上形如$A_\mu \bar{\psi} \psi$,因此的确有两条电子线。
虽然电场和电子的耦合中电场是线性的,电子线却有两条,因此仅仅用电子-光子相互作用顶角就能够构造出多光子的图。

\subsubsection{双侧费曼图方法}\label{sec:pure-double-sided-feynman}

耦合项$- \vb*{d} \cdot \vb*{E}$会导致电子受到电场影响,自然也会导致电场被电子激发出来。
本节讨论经典电磁场,从而不能真的用光子入射散射等概念计算等效光子-光子顶角。
经典电磁场中介质极化是新的波源,而极化矢量为
\begin{equation}
    \vb*{P} = N \expval*{\vb*{d}} = N \mel*{\psi}{\vb*{d}}{\psi},
\end{equation}
将$\vb*{P}$代入介质中的麦克斯韦方程,即可得到介质中光的行为。
我们这里直接计算$\vb*{d}$的期望值,以得到极化矢量,这个做法的合理性在于我们本质上还是在积掉电子,即在配分函数中保留电磁场不动,积掉电子场,计算(两边是基态的)关联函数。

这样,极化矢量对电场的一阶响应为
\begin{equation}
   \begin{aligned}
    &\quad \mel*{\psi^{(0)}}{\vb*{d}}{\psi^{(1)}} + \mel*{\psi^{(1)}}{\vb*{d}}{\psi^{(0)}} \\
    &= \sum_m \ee^{- \ii \omega_m t} \vb*{d}_{gm} \ee^{\ii \omega_g t} \frac{1}{\hbar} \sum_p \frac{\vb*{d}_{mg} \cdot \vb*{E}(\omega_p)}{\omega_{mg} - \omega_p} \ee^{\ii (\omega_{mg} - \omega_p) t} + \text{h.c.} \\
    &= \frac{1}{\hbar} \sum_{m, p} \left( \frac{ \vb*{d}_{gm} (\vb*{d}_{mg} \cdot \vb*{E}(\omega_p))}{\omega_{mg} - \omega_p} \ee^{- \ii \omega_p t} + \frac{(\vb*{d}_{gm} \cdot \vb*{E}(\omega_p) )^* \vb*{d}_{mg}}{\omega_{mg}^* - \omega_p} \ee^{\ii \omega_p t} \right) \\
    &= \frac{1}{\hbar} \sum_{m, p} \left( \frac{ \vb*{d}_{gm} (\vb*{d}_{mg} \cdot \vb*{E}(\omega_p))}{\omega_{mg} - \omega_p} \ee^{- \ii \omega_p t} + \frac{(\vb*{d}_{gm} \cdot \vb*{E}(\omega_p) ) \vb*{d}_{mg}}{\omega_{mg}^* + \omega_p} \ee^{- \ii \omega_p t} \right), 
   \end{aligned} 
   \label{eq:dipole-first-perturbation}
\end{equation}
其中第三个等号将第二项中的$\omega_p$换成了$-\omega_p$。更高阶的响应也可以用类似的方式获得。
在计算更高阶的响应时,直接展开计算是非常繁琐的,例如,$\vb*{d}$的期望值中电场的二次项为
\begin{equation}
    \begin{aligned}
        &\quad \mel*{\psi^{(0)}}{\vb*{d}}{\psi^{(2)}} + \mel*{\psi^{(2)}}{\vb*{d}}{\psi^{(0)}} + \mel*{\psi^{(1)}}{\vb*{d}}{\psi^{(1)}} \\
        &= \frac{1}{\hbar^2} \sum_{p, q} \sum_{m, n} \frac{\vb*{d}_{gn} (\vb*{d}_{nm} \cdot \vb*{E}(\omega_q)) (\vb*{d}_{mg} \cdot \vb*{E}(\omega_p))}{(\omega_{ng} - \omega_p - \omega_q) (\omega_{mg} - \omega_p)} \ee^{-\ii (\omega_p + \omega_q) t} \\
        &\quad + \frac{1}{\hbar^2} \sum_{p, q} \sum_{m, n} \frac{(\vb*{d}_{ng} \cdot \vb*{E}(\omega_q))^* \vb*{d}_{nm} (\vb*{d}_{mg} \cdot \vb*{E}(\omega_p))}{(\omega_{ng}^* - \omega_q) (\omega_{mg} - \omega_p)} \ee^{-\ii (\omega_p - \omega_q) t} \\
        &\quad + \frac{1}{\hbar^2} \sum_{p, q} \sum_{m, n} \frac{ (\vb*{d}_{ng} \cdot \vb*{E}(\omega_q))^* (\vb*{d}_{mn} \cdot \vb*{E}(\omega_p))^* \vb*{d}_{mg} }{(\omega_{ng}^* - \omega_q) (\omega_{mg}^* - \omega_p - \omega_q)} \ee^{\ii (\omega_p + \omega_q) t},
    \end{aligned}
\end{equation}
或者,通过调整求和变量让$\ee$指数完全成为$\ee^{- \ii (\omega_p + \omega_q) t}$,上式就是
\begin{equation}
    \begin{aligned}
        &\quad \mel*{\psi^{(0)}}{\vb*{d}}{\psi^{(2)}} + \mel*{\psi^{(2)}}{\vb*{d}}{\psi^{(0)}} + \mel*{\psi^{(1)}}{\vb*{d}}{\psi^{(1)}} \\
        &= \frac{1}{\hbar^2} \sum_{p, q} \sum_{m, n} \frac{\vb*{d}_{gn} (\vb*{d}_{nm} \cdot \vb*{E}(\omega_q)) (\vb*{d}_{mg} \cdot \vb*{E}(\omega_p))}{(\omega_{ng} - \omega_p - \omega_q) (\omega_{mg} - \omega_p)} \ee^{-\ii (\omega_p + \omega_q) t} \\
        &\quad + \frac{1}{\hbar^2} \sum_{p, q} \sum_{m, n} \frac{(\vb*{d}_{gn} \cdot \vb*{E}(\omega_q)) \vb*{d}_{nm} (\vb*{d}_{mg} \cdot \vb*{E}(\omega_p))}{(\omega_{ng}^* + \omega_q) (\omega_{mg} - \omega_p)} \ee^{-\ii (\omega_p + \omega_q) t} \\
        &\quad + \frac{1}{\hbar^2} \sum_{p, q} \sum_{m, n} \frac{ (\vb*{d}_{gn} \cdot \vb*{E}(\omega_q)) (\vb*{d}_{nm} \cdot \vb*{E}(\omega_p)) \vb*{d}_{mg} }{(\omega_{ng}^* + \omega_q) (\omega_{mg}^* + \omega_p + \omega_q)} \ee^{- \ii (\omega_p + \omega_q) t},
    \end{aligned}
\end{equation}
而三阶项就更加繁琐了。实际上,我们可以将这些计算总结为一套费曼图,其规则如下:
\begin{itemize}
    \item 电子线包括从下而上的左侧线和从上而下的右侧线,即所谓\emph{双边费曼图(double sided Feynman diagram)};
    \item 在左侧线最顶端放置$\vb*{d}$算符,使用一根波浪线代表它产生电磁场;
    \item 由于顶角总是有两条电子线,传播子和顶角可以合并成一个组件。在左侧线上,从下到上第$i$个顶角给出
    \begin{equation}
        \begin{gathered}
            \begin{tikzpicture}
                \begin{feynhand}
                    \vertex (g) at (0, 0);
                    \vertex (t) at (0, 2);
                    \vertex (l) at (-1, 0.5) {$\omega_i$};
                    \vertex (v) at (0, 1);
                    
                    \propag[plain] (g) to[edge label={$m$}] (v) ;
                    \propag[plain] (v) to[edge label={$n$}] (t);
                    \propag[extphoton] (l) to (v);
                \end{feynhand}
            \end{tikzpicture}
        \end{gathered} = \frac{\vb*{d}_{nm} \cdot \vb*{E}(\omega_p) }{\omega_{ng} - \sum_{j=1}^i \omega_j},
        \label{eq:feynman-diagram-left}
    \end{equation}
    其中$\omega_k$表示左侧线从下到上第$j$个光子线的频率。
    而在右侧线上,从下到上第$i$个顶角给出
    \begin{equation}
        \begin{gathered}
            \begin{tikzpicture}
                \begin{feynhand}
                    \vertex (g) at (0, 0);
                    \vertex (t) at (0, 2);
                    \vertex (l) at (1, 0.5) {$\omega_i$};
                    \vertex (v) at (0, 1);
                    
                    \propag[plain] (g) to[edge label={$m$}] (v) ;
                    \propag[plain] (v) to[edge label={$n$}] (t);
                    \propag[extphoton] (l) to (v);
                \end{feynhand}
            \end{tikzpicture}
        \end{gathered} = \frac{\vb*{d}_{mn} \cdot \vb*{E}(\omega_p) }{\omega_{ng}^* + \sum_{j=1}^i \omega_j}.
        \label{eq:feynman-diagram-right}
    \end{equation}
    其中$\omega_k$表示右侧线从下到上第$j$个光子线的频率。为了区分外场和$\mel*{\psi}{\vb*{d}}{\psi}$,我们用直线表示前者而用波浪线代表后者,虽然在凝聚态场论中我们通常用带有$\otimes$的波浪线代表前者。
\end{itemize}
例如,\eqref{eq:dipole-first-perturbation}可以用费曼图表示如下:
\begin{equation}
    \begin{gathered}
        \begin{tikzpicture}
            \begin{feynhand}
                \vertex (g1) at (-0.25, 0) {$g$};
                \vertex (g2) at (0.25, 0) {$g$};
                \vertex (t1) at (-0.25, 2.5);
                \vertex (t2) at (0.25, 2.5);
                \propag[plain] (t1) to[out=90, in=90] (t2);

                \vertex (v1) at (-0.25, 1) ;
                \vertex (l1) at (-1.55, 0.5) {$\omega_p$};
                \propag[extphoton] (l1) to (v1);
                \propag[plain] (g1) to (v1) ;

                \vertex (o) at (-0.25, 2);
                \vertex (e) at (-1.55, 2.5) {$\omega_p$};
                \propag[outphoton] (o) to (e);
                \propag[plain] (v1) to[edge label={$m$}] (o);

                \propag[plain] (o) to (t1);

                \propag[plain] (t2) to (g2);
            \end{feynhand}
        \end{tikzpicture}
    \end{gathered} = \frac{1}{\hbar} \sum_{m, p} \frac{ \vb*{d}_{gm} (\vb*{d}_{mg} \cdot \vb*{E}(\omega_p))}{\omega_{mg} - \omega_p} \ee^{- \ii \omega_p t} ,
    \label{eq:left-in-one-order-perturbation}
\end{equation}
以及
\begin{equation}
    \begin{gathered}
        \begin{tikzpicture}
            \begin{feynhand}
                \vertex (g1) at (-0.25, 0) {$g$};
                \vertex (g2) at (0.25, 0) {$g$};
                \vertex (t1) at (-0.25, 2.5);
                \vertex (t2) at (0.25, 2.5);
                \propag[plain] (t1) to[out=90, in=90] (t2);

                \vertex (v1) at (0.25, 1) ;
                \vertex (l1) at (1.55, 0.5) {$\omega_p$};
                \propag[extphoton] (l1) to (v1);
                \propag[plain] (g2) to (v1) ;

                \propag[plain] (v1) to[edge label'={$m$}] (t2);
                \propag[plain] (t1) to (o);

                \vertex (o) at (-0.25, 2);
                \vertex (e) at (-1.55, 2.5) {$\omega_p$};
                \propag[outphoton] (o) to (e);

                \propag[plain] (o) to (g1);
            \end{feynhand}
        \end{tikzpicture}
    \end{gathered} = \frac{1}{\hbar} \sum_{m, p} \frac{ \vb*{d}_{mg} (\vb*{d}_{gm} \cdot \vb*{E}(\omega_p) )}{\omega_{mg}^* + \omega_p} \ee^{- \ii \omega_p t}.
    \label{eq:right-in-one-order-perturbation}
\end{equation}
更高阶的响应也可以用类似的方式用费曼图计算,如二阶响应对应如下三个图:
\begin{equation}
    \begin{gathered}
        \begin{tikzpicture}
            \begin{feynhand}
                \vertex (g1) at (-0.25, 0) {$g$};
                \vertex (g2) at (0.25, 0) {$g$};
                \vertex (t1) at (-0.25, 3.5);
                \vertex (t2) at (0.25, 3.5);
                \propag[plain] (t1) to[out=90, in=90] (t2);

                \vertex (v1) at (-0.25, 1) ;
                \vertex (l1) at (-1.55, 0.5) {$\omega_p$};
                \propag[extphoton] (l1) to (v1);

                \vertex (v2) at (-0.25, 2);
                \vertex (l2) at (-1.55, 1.5) {$\omega_q$};
                \propag[extphoton] (l2) to (v2);

                \propag[plain] (g1) to (v1) ;
                \propag[plain] (v1) to[edge label={$m$}] (v2);

                \vertex (o) at (-0.25, 3);
                \vertex (e) at (-1.55, 3.5) {$\omega_p + \omega_q$};
                \propag[outphoton] (o) to (e);
                \propag[plain] (v2) to[edge label={$n$}] (o);

                \propag[plain] (o) to (t1);

                \propag[plain] (t2) to (g2);
            \end{feynhand}
        \end{tikzpicture}
    \end{gathered} = \frac{1}{\hbar^2} \sum_{p, q} \sum_{m, n} \frac{\vb*{d}_{gn} (\vb*{d}_{nm} \cdot \vb*{E}(\omega_q)) (\vb*{d}_{mg} \cdot \vb*{E}(\omega_p))}{(\omega_{ng} - \omega_p - \omega_q) (\omega_{mg} - \omega_p)} \ee^{-\ii (\omega_p + \omega_q) t},
    \label{eq:second-response-1}
\end{equation}
\begin{equation}
    \begin{gathered}
        \begin{tikzpicture}
            \begin{feynhand}
                \vertex (g1) at (-0.25, 0) {$g$};
                \vertex (g2) at (0.25, 0) {$g$};
                \vertex (t1) at (-0.25, 3.5);
                \vertex (t2) at (0.25, 3.5);
                \propag[plain] (t1) to[out=90, in=90] (t2);

                \vertex (v1) at (0.25, 1) ;
                \vertex (l1) at (1.55, 0.5) {$\omega_q$};
                \propag[extphoton] (l1) to (v1);
                \propag[plain] (g2) to (v1) ;

                \propag[plain] (v1) to[edge label'={$n$}] (t2);
                \propag[plain] (t1) to (o);

                \vertex (o) at (-0.25, 3);
                \vertex (e) at (-1.55, 3.5) {$\omega_p + \omega_q$};
                \propag[outphoton] (o) to (e);

                \propag[plain] (g1) to (v2);
                \vertex (v2) at (-0.25, 2);
                \vertex (l2) at (-1.55, 1.5) {$\omega_p$};
                \propag[extphoton] (l2) to (v2);
                \propag[plain] (v2) to[edge label={$m$}] (o);
            \end{feynhand}
        \end{tikzpicture}
    \end{gathered} = \frac{1}{\hbar^2} \sum_{p, q} \sum_{m, n} \frac{(\vb*{d}_{gn} \cdot \vb*{E}(\omega_q)) \vb*{d}_{nm} (\vb*{d}_{mg} \cdot \vb*{E}(\omega_p))}{(\omega_{ng}^* + \omega_q) (\omega_{mg} - \omega_p)} \ee^{-\ii (\omega_p + \omega_q) t} ,
\end{equation}
以及
\begin{equation}
    \begin{gathered}
        \begin{tikzpicture}
            \begin{feynhand}
                \vertex (g1) at (-0.25, 0) {$g$};
                \vertex (g2) at (0.25, 0) {$g$};
                \vertex (t1) at (-0.25, 3.5);
                \vertex (t2) at (0.25, 3.5);
                \propag[plain] (t1) to[out=90, in=90] (t2);

                \vertex (v1) at (0.25, 1) ;
                \vertex (l1) at (1.55, 0.5) {$\omega_q$};
                \propag[extphoton] (l1) to (v1);
                \propag[plain] (g2) to (v1) ;

                \vertex (v2) at (0.25, 2);
                \vertex (l2) at (1.55, 1.5) {$\omega_p$};
                \propag[extphoton] (l2) to (v2);
                \propag[plain] (v2) to[edge label={$n$}] (v1);

                \propag[plain] (v2) to[edge label'={$m$}] (t2);
                \propag[plain] (t1) to (o);

                \vertex (o) at (-0.25, 3);
                \vertex (e) at (-1.55, 3.5) {$\omega_p + \omega_q$};
                \propag[outphoton] (o) to (e);

                \propag[plain] (o) to (g1);
            \end{feynhand}
        \end{tikzpicture}
    \end{gathered} = \frac{1}{\hbar^2} \sum_{p, q} \sum_{m, n} \frac{ (\vb*{d}_{gn} \cdot \vb*{E}(\omega_q)) (\vb*{d}_{nm} \cdot \vb*{E}(\omega_p)) \vb*{d}_{mg} }{(\omega_{ng}^* + \omega_q) (\omega_{mg}^* + \omega_p + \omega_q)} \ee^{- \ii (\omega_p + \omega_q) t}.
\end{equation}

\subsubsection{极化矢量和极化率}

现在我们获得了$\vb*{d}$,于是可以计算出$\vb*{P}$,于是进一步可以计算出$\chi^{(1)}$, $\chi^{(2)}$等等。
在假定介质稀薄,或者说假定光只会被散射单次,并且介质中所有电子吸收入射光的概率均相同时,我们有
\[
    \vb*{P} = N \expval*{\vb*{d}} = \epsilon_0 \chi^{(1)}_{ij} E^j + \epsilon_0 \chi^{(2)}_{ij} E^i E^j + \cdots,
\]
即可得到
\begin{equation}
    \chi^{(1)}_{ij}(\omega_p) = \frac{N}{\epsilon_0 \hbar} \sum_{m} \left( \frac{ d_{gm}^i {d}_{mg}^j}{\omega_{mg} - \omega_p} + \frac{{d}_{mg}^i d_{gm}^j}{\omega_{mg}^* + \omega_p} \right)  ,
    \label{eq:linear-chi-pure}
\end{equation}
以及
\begin{equation}
    \begin{aligned}
        \chi^{(2)}_{ijk}(\omega_p + \omega_q, \omega_p, \omega_q) &= \frac{N}{\epsilon_0 \hbar^2} \mathcal{P}_\text{I} \sum_{m, n} \Bigg( \frac{d_{gn}^i d_{nm}^j d_{mg}^k }{(\omega_{ng} - \omega_p - \omega_q) (\omega_{mg} - \omega_p)} \\
        &\quad + \frac{d_{gn}^j d_{nm}^i d_{mg}^k }{(\omega_{ng}^* + \omega_q) (\omega_{mg} - \omega_p)} \\
        &\quad + \frac{ d_{gn}^j d_{nm}^k d_{mg}^i }{(\omega_{ng}^* + \omega_q) (\omega_{mg}^* + \omega_p + \omega_q)} \Bigg) .
    \end{aligned}
    \label{eq:second-chi-pure}
\end{equation}
等式坐标括号内的第一个频率是转换产生的光的频率,后面的频率是输入光的频率。
这里元算符$\mathcal{P}_\text{I}$将其后的表达式中的$\omega_p$和$\omega_q$交换,将$j$与$k$交换,并且求和所有可能的交换情况。
它的出现是因为以上所有求和式中$\sum_{p, q}$均未固定各个频率出现的顺序,从而,例如,如果输入光有两个频率分量$\omega_1, \omega_2$,那么$\omega_p = \omega_1, \omega_q = \omega_2$和$\omega_p = \omega_2, \omega_q = \omega_1$这两种情况都必须考虑进去;由于指标$j$和$\vb*{E}(\omega_q)$缩并,指标$k$和$\vb*{E}(\omega_p)$缩并,交换$\omega_p$和$\omega_q$也要求交换$j$和$k$。
因此,\eqref{eq:second-chi-pure}在去掉$\mathcal{P}_\text{I}$之后其实有六项。

从\eqref{eq:linear-chi-pure}和\eqref{eq:second-chi-pure}出发可以证明关于极化率的一些性质。
首先是熟知的二阶非线性极化对应空间反演对称性破缺。
在空间反演下,正比于$\vb*{r}$的电偶极矩$\vb*{d}$变号,此时$\chi^{(1)}$不变而$\chi^{(2)}$变号。
对具有空间反演不变性的系统,空间反演下张量不应该有变动,因此对具有空间反演不变性的系统,二阶非线性极化为零。
类似的其它从晶体的对称性在宏观层面分析得到的非线性极化率的性质也可以通过观察这些对称性对$\vb*{d}$的矩阵元的约束而获得。

在远离共振的频段,阻尼不会造成特别大的影响,从而\eqref{eq:second-chi-pure}中所有的$\omega_{mn}$都是实数。
此时\eqref{eq:second-chi-pure}可以进一步化简。
设
\begin{equation}
    \omega_s = \omega_p + \omega_q,
\end{equation}
我们会发现,如果我们定义元算符$\mathcal{P}_\text{F}$为将其后的表达式中的$(\omega_q, j)$,$(\omega_p, k)$和$(- \omega_s, i)$做所有可能的置换并求和,则可以验证
\begin{equation}
    \chi^{(2)}_{ijk}(\omega_s, \omega_p, \omega_q) = \frac{N}{\epsilon_0 \hbar^2} \mathcal{P}_\text{F} \sum_{m, n} \frac{d_{gn}^i d_{nm}^j d_{mg}^k }{(\omega_{ng} - \omega_s) (\omega_{mg} - \omega_p)}.
\end{equation}
这个表达式说明了另一个性质:在无阻尼的情况下,二阶非线性极化率中的$\omega_p, \omega_q, -\omega_s$是等价的:把它们连通$i, j, k$一起轮换不改变$\chi^{(2)}_{ijk}(\omega_s, \omega_p, \omega_q)$的值。
这是因为在没有阻尼时,二阶非线性极化可以通过在哈密顿量中加入一个等效相互作用
\begin{equation}
    {H}_\text{int} = \int \dd[3]{\vb*{r}} \frac{1}{3} \epsilon_0 \chi^{(2)} : \vb*{E} \vb*{E} \vb*{E}
\end{equation}
得到,从而$\chi^{(2)}$的$i, j, k$指标是等价的,切换到频域下后,三个电场分别用$\omega_p, \omega_q, \omega_s$标记,则$\omega_p, \omega_q, \omega_s$也应该是等价的。
轮换三个电场相当于同时轮换$i, j, k$和$\omega_p, \omega_q, \omega_s$,这对系统的行为不造成任何改变,这就解释了前述的$\chi^{(2)}$的性质。
$\omega_s$在轮换时带有负号可以用下图理解:
\begin{equation}
    \begin{gathered}
        \begin{tikzpicture}
            \begin{feynhand}
                \vertex (o) at (0, 0);
                \vertex (a) at (1.5, 0);
                \vertex (b) at (-0.75, 1.3);
                \vertex (c) at (-0.75, -1.3);
                
                \propag[photon, mom={$\omega_s$}] (o) to (a);
                \propag[photon, mom={$\omega_p$}] (b) to (o);
                \propag[photon, mom={$\omega_q$}] (c) to (o);
            \end{feynhand}
        \end{tikzpicture}
    \end{gathered} = \begin{gathered}
        \begin{tikzpicture}
            \begin{feynhand}
                \vertex (o) at (0, 0);
                \vertex (a) at (1.5, 0);
                \vertex (b) at (-0.75, 1.3);
                \vertex (c) at (-0.75, -1.3);
                
                \propag[photon, mom={$- \omega_s$}] (a) to (o);
                \propag[photon, mom={$\omega_p$}] (b) to (o);
                \propag[photon, mom={$\omega_q$}] (c) to (o);
            \end{feynhand}
        \end{tikzpicture}
    \end{gathered}
\end{equation}
需注意我们这里实际上还是在处理经典电磁场,从而需要严格区分一个顶角上的入射线和出射线,因此出射的$\omega_s$和入射的$\omega_q$和$\omega_p$在推导$\chi^{(2)}$时的地位看起来不同,但是在最终结果中一定有$\omega_p, \omega_q, -\omega_s$以及附带的$i, j, k$的轮换对称性成立。
下面给出这种轮换性的一些显式表达式,这里我们为了强调“转化”,将$\chi^{(2)}(\omega_s, \omega_q, \omega_p)$写成$\chi^{(2)}(\omega_s = \omega_q + \omega_p)$:
\begin{equation}
    \chi^{(2)}_{ijk}(\omega_s = \omega_q + \omega_p) = \chi^{(2)}_{jik}(- \omega_q = - \omega_s + \omega_p) = \chi^{(2)}_{jik}(\omega_q = \omega_s - \omega_p)^*, 
\end{equation}
\begin{equation}
    \chi^{(2)}_{ijk}(\omega_s = \omega_q + \omega_p) = \chi^{(2)}_{ikj}(\omega_s = \omega_p + \omega_q).
\end{equation}
容易看出更一般的轮换关系对任意阶的远离共振(从而阻尼可以忽略)的非线性极化率都成立,且当$\chi^{(n)}(\omega_s = \sum_i \omega_i)$括号内等式两边的频率从一侧移到另一侧时需要加复共轭。
例如对线性极化就有熟知的
\begin{equation}
    \chi^{(1)}_{ij}(\omega) = \chi^{(1)}_{ji}(-\omega) = \chi^{(1)}_{ji}(\omega)^* = \chi^{(1)}_{ij}(-\omega).
\end{equation}

总之,在量子力学的框架下,产生非线性极化并不需要介质内有非线性势或者类似的非线性相互作用。
在量子力学的框架下电子和光子的相互作用方式——一个电子吸收或者放出一个光子——自然地允许出现\eqref{eq:second-response-1}这样的过程;而另一方面,经典理论禁止多条外线的出现,从而也无法在介质内部没有非线性相互作用的情况下得到非线性响应。

我们此处做的计算积掉了电子自由度。电子在电磁波的作用下没有处在某个明确的能级上,虽然费曼图中我们说电子可能处在$m$或$n$号能级,但这都是离壳的。
光学上,我们说此时电子处在“虚能级”上。后文中当光和介质中实实在在的模式(如声子)耦合时,将会涉及“实能级”,因为声子的发射和吸收让系统从一个明确的能级跃迁到另一个明确的能级上。

实际上,虽然本节在一开始似乎假定了系统处在束缚态,但以上推导完全没有用到不同能级之间有有限的能隙这个条件。(这和诸如拓扑序的波函数重整化之类的东西是不同的,在那些话题中,能隙是至关重要的,因为要保证小的激发不会让系统跑到激发态上去的概率变得很大;本节所述的过程都是求和了所有可能的跃迁,所以有没有能隙不重要)
因此,对那些电子可以长距离运动的系统,如果能够适用能带理论,即单电子近似仍然成立,我们以上的推导在考虑了诸如电子的激发在费米面附近等费米统计带来的特殊效应后仍然是适用的。
一旦通过各种方法——比如DFT计算——得到系统能谱,就能够计算出非线性极化率。

最后我们需要指出,我们目前做的所有计算都依赖两个假设:一个是电子之间的库仑相互作用可以忽略,即我们无需真的去分析一个多体系统;另一个假设是介质充分稀薄。
这第二个假设不成立时我们就有局域场修正,即介质外部的电磁场进入到介质分子或原子附近时会因为邻近的原子的响应而受到“修饰”或者说“增强”。

\subsubsection{与凝聚态场论方法的一致性}

双边费曼图的规则看起来非常奇特,但是实际上可以将它理解为凝聚态场论中的图。
它和一般的凝聚态场论的不同之处在于,本节讨论的带有弛豫的模型是非幺正的,时间反演不变性破缺,从而传播子的书写有一些需要注意的地方。
大体上说我们是要计算
\[
    \expval*{{c}^\dagger_m(t) \vb*{d}_{mn} c_n(t) }, \quad \ket*{\Omega} = c^\dagger_g(-\infty) \ket*{0}.
\]
我们有两种做法,其中之一是直接计算$\mel*{0}{c_g(-\infty) {c}^\dagger_m(t) \vb*{d}_{mn} c_n(t) c^\dagger_g(-\infty)}{0}$,对应的费曼图形如
\begin{equation}
    \sum_{m, n} \vb*{d}_{mn} \begin{gathered}
        \begin{tikzpicture}
            \begin{feynhand}
                \vertex [grayblob] (o) at (0, 0) {};
                \vertex (a) at (-1, 1) {$g, -\infty$};
                \vertex (b) at (-1, -1) {$g, -\infty$};
                \vertex (c) at (1, 1) {$m, t$};
                \vertex (d) at (1, -1) {$n, t$};
    
                \propag[fermion] (a) to (o);
                \propag[fermion] (o) to (b);
                \propag[fermion] (o) to (c);
                \propag[fermion] (d) to (o);
            \end{feynhand}
        \end{tikzpicture}
    \end{gathered},
    \label{eq:double-sided-feynman-diagram-in-condensed}
\end{equation}
其中$m, n$两端均可以不在壳,即有对应的传播子。这是可以理解的,因为如果电磁场也是量子化的,上图实际上就是
\[
    \begin{gathered}
        \begin{tikzpicture}
            \begin{feynhand}
                \vertex [grayblob] (o) at (0, 0) {};
                \vertex (a) at (-1, 1) {$g, -\infty$};
                \vertex (b) at (-1, -1) {$g, -\infty$};
                \vertex (e) at (1.75, 0);
                \vertex (f) at (2.75, 0);
    
                \propag[fermion] (a) to (o);
                \propag[fermion] (o) to (b);
                \propag[fermion] (o) to[out=45, in=135] (e);
                \propag[fermion] (e) to[in=315, out=225] (o);
                \propag[photon] (e) to (f);
            \end{feynhand}
        \end{tikzpicture}
    \end{gathered}.
\]
在本节中我们无需计算上图最右边的产生光子的顶角,光子产生是在非线性麦克斯韦方程中通过极化矢量引入的。

\eqref{eq:double-sided-feynman-diagram-in-condensed}中的$g$外线实际上是系统基态的一部分,因此我们不能将两个$g$外线连接起来,否则得到的实际上是真空气泡图。因此,我们只需要考虑下图
\begin{equation}
    \sum_{m, n} \vb*{d}_{mn} \ee^{- \ii (\omega_m - \omega_n) t} \begin{gathered}
        \begin{tikzpicture}
            \begin{feynhand}
                \vertex (a) at (-2, 1) {$g, -\infty$};
                \vertex (b) at (-2, 0) {$g, -\infty$};
                \vertex (c) at (2, 1) {$m, t$};
                \vertex (d) at (2, 0) {$n, t$};
                \vertex[grayblob] (o1) at (0, 1) {};
                \vertex[grayblob] (o2) at (0, 0) {};
                
                \propag[fermion] (a) to (o1);
                \propag[fermion] (o1) to (c);
                \propag[fermion] (o2) to (b);
                \propag[fermion] (d) to (o2);
            \end{feynhand}
        \end{tikzpicture}
    \end{gathered}
    \label{eq:general-form-condensed}
\end{equation}
即可,其中圆圈内是一个或多个光子吸收或发射过程。由于本节的系统时间反演对称性破缺,因为存在弛豫项,图
\[
    \begin{gathered}
        \begin{tikzpicture}
            \begin{feynhand}
                \vertex (a) at (-2, 1) {$g, -\infty$};
                \vertex (c) at (2, 1) {$m, t$};
                \vertex[grayblob] (o1) at (0, 1) {};
                
                \propag[fermion] (a) to (o1);
                \propag[fermion] (o1) to (c);
            \end{feynhand}
        \end{tikzpicture}
    \end{gathered}
\]
中的频域电子传播子是
\[
    \int \ee^{\ii \omega t} \dd{t} T\expval*{\psi_m(t) \bar{\psi}_g} = \frac{\ii}{\omega - \omega_m + \ii \gamma},
\]
而
\[
    \begin{gathered}
        \begin{tikzpicture}
            \begin{feynhand}
                \vertex (a) at (-2, 1) {$g, -\infty$};
                \vertex (c) at (2, 1) {$m, t$};
                \vertex[grayblob] (o1) at (0, 1) {};
                
                \propag[fermion] (o1) to (a);
                \propag[fermion] (c) to (o1);
            \end{feynhand}
        \end{tikzpicture}
    \end{gathered}
\]
中的频域电子传播子是
\[
    \int \ee^{\ii \omega t} \dd{t} T\expval*{{\psi}_g \bar{\psi}_m(t)} = \frac{\ii}{\omega - \omega_m - \ii \gamma}.
\]
$\mel{\psi}{\vb*{d}}{\psi}$中有一系列从$-\infty$演化到我们要计算的时间点的粒子线,也有一系列从我们要计算的时间点演化回$-\infty$的粒子线,它们的电子传播子是不同的:在时间反演变换下$\gamma$需要加上一个负号。
从\eqref{eq:dipole-first-perturbation}出发,我们也可以将$\gamma$的负号理解为是$\bra{\psi}$相比于$\ket*{\psi}$取了复共轭而产生的。
对$\gamma$的负号的两种理解是一致的,因为时间反演和复共轭紧密相关。
\eqref{eq:general-form-condensed}中的$-\infty$演化到我们要计算的时间点的一系列电子线实际上就是双边费曼图的左侧电子线,而\eqref{eq:general-form-condensed}中从我们要计算的时间点演化到$-\infty$的一系列电子线实际上就是双边费曼图中的右侧电子线。
右边的传播子中的$\gamma$要多一个负号。

基于以上考虑,把这些费曼图当成凝聚态场论的图理解,可以写出
\begin{equation}
    \begin{aligned}
        &\quad \sum_{m, n} \vb*{d}_{mn} \ee^{-\ii (\omega_m - \omega_n) t} \times \left( \begin{gathered}
            \begin{tikzpicture}
                \begin{feynhand}
                    \vertex[crossdot] (l) at (0, 2) {};
                    \vertex (a) at (-2, 1) {$g, -\infty$};
                    \vertex (b) at (-2, 0) {$g, -\infty$};
                    \vertex (c) at (2, 1) {$m, t$};
                    \vertex (d) at (2, 0) {$n, t$};
                    \vertex (v) at (0, 1);

                    \propag[photon, mom={$\omega_p$}] (l) to (v);
                    \propag[fermion] (a) to (v);
                    \propag[fermion] (v) to (c);
                    \propag[fermion] (d) to (b);
                \end{feynhand}
            \end{tikzpicture}
        \end{gathered} \right) \\
        &= \sum_p \vb*{d}_{gm} \ee^{-\ii \omega_p t} \sum_{m} \frac{\ii}{\omega_p + \omega_g - \omega_m + \ii \gamma_{mg}} \frac{(\ii \vb*{d}_{mg} \cdot \vb*{E}(\omega_p))}{\hbar},
    \end{aligned}
\end{equation}
计算结果和\eqref{eq:left-in-one-order-perturbation}是一致的。同样我们使用凝聚态场论的理解方式计算入射光子在右侧电子线的图,有
\begin{equation}
    \begin{aligned}
        &\quad \sum_{m, n} \vb*{d}_{mn} \ee^{-\ii (\omega_m - \omega_n) t} \times \left( \begin{gathered}
            \begin{tikzpicture}
                \begin{feynhand}
                    \vertex[crossdot] (l) at (0, -1) {};
                    \vertex (a) at (-2, 1) {$g, -\infty$};
                    \vertex (b) at (-2, 0) {$g, -\infty$};
                    \vertex (c) at (2, 1) {$m, t$};
                    \vertex (d) at (2, 0) {$n, t$};
                    \vertex (v) at (0, 0);

                    \propag[photon, mom={$\omega_p$}] (l) to (v);
                    \propag[fermion] (a) to (c);
                    \propag[fermion] (d) to (v);
                    \propag[fermion] (v) to (b);
                \end{feynhand}
            \end{tikzpicture}
        \end{gathered} \right) \\
        &= \sum_p \vb*{d}_{mg} \ee^{- \ii \omega_p t} \sum_m \frac{\ii \vb*{d}_{gm} \cdot \vb*{E}(\omega_p)}{\hbar} \frac{\ii}{\omega_g - \omega_p - \omega_m + \ii \gamma_{mg}} ,
    \end{aligned}
\end{equation}
和\eqref{eq:right-in-one-order-perturbation}一致。
类似的可以验证,\eqref{eq:feynman-diagram-left}和\eqref{eq:feynman-diagram-right}两个费曼规则也是可以通过凝聚态场论导出的,其中诸如$\sum \omega_i$这样奇怪的表达式实际上是能量守恒条件的结果。

\subsubsection{密度矩阵表述}\label{sec:electron-density-matrix}

以上所有的计算中的非幺正效应都是手动添加在传播子中的,并且我们到最后实际上又忽略了它们。
本节给出对电子弛豫到基态这件事的稍微严格一些的处理。
我们使用密度矩阵描述电子状态。实际系统中总是存在各种各样的过程让处于激发态的电子回到基态,如自发辐射、碰撞等。唯象地引入参数$\gamma_{mn}$,写出密度矩阵的时间演化方程
\begin{equation}
    \dot{\rho}_{mn} = - \frac{\ii}{\hbar} \comm*{H}{\rho}_{mn} - \gamma_{mn} (\rho_{mn} - \rho^\text{eq}_{mn}),
    \label{eq:electron-density-matrix-evolve}
\end{equation}
通常可以假定电子处在热态,而由于$\{\ket*{m}\}$是能量本征态,有
\begin{equation}
    \rho^\text{eq}_{mn} = \rho^\text{eq}_{mm} \delta_{mn},
\end{equation}
其中$\rho^\text{eq}_{mn}$给出热平衡态分布。

方程\eqref{eq:electron-density-matrix-evolve}

实际上,\eqref{eq:electron-density-matrix-evolve}的解也可以画成双边费曼图,在每个时间点,有\emph{两个}状态,分别对应密度矩阵的左矢和右矢,因此电子线包括左右两部分。
但此时的双边费曼图和\autoref{sec:pure-double-sided-feynman}中有所差别,主要体现在传播子的表达式不同,以及会多出来几个图。
例如,密度矩阵的双边费曼图同时包括下面的两张图:
\[
    \begin{gathered}
        \begin{tikzpicture}
            \begin{feynhand}
                \vertex (g1) at (-0.25, 0) {$\ket*{g}$};
                \vertex (g2) at (0.25, 0) {$\bra*{g}$};
                \vertex (t1) at (-0.25, 4) {$\ket*{n}$};
                \vertex (t2) at (0.25, 4) {$\bra*{n}$};

                \vertex (v1) at (0.25, 1) ;
                \vertex (l1) at (1.55, 0.5) {$\omega_q$};
                \propag[extphoton] (l1) to (v1);
                \propag[plain] (g2) to (v1) ;
                
                \propag[plain] (v1) to (t2);

                \vertex (v2) at (-0.25, 2);
                \vertex (l2) at (-1.55, 1.5) {$\omega_p$};
                \propag[extphoton] (l2) to (v2);
                \propag[plain] (g1) to (v2);

                \vertex (o) at (-0.25, 3);
                \vertex (e) at (-1.55, 3.5) {$\omega_p + \omega_q$};
                \propag[outphoton] (o) to (e);
                \propag[plain] (v2) to[edge label={$\ket*{m}$}] (o);
                \propag[plain] (o) to (t1);
            \end{feynhand}
        \end{tikzpicture}
    \end{gathered} , \quad \begin{gathered}
        \begin{tikzpicture}
            \begin{feynhand}
                \vertex (g1) at (-0.25, 0) {$\ket*{g}$};
                \vertex (g2) at (0.25, 0) {$\bra*{g}$};
                \vertex (t1) at (-0.25, 4) {$\ket*{n}$};
                \vertex (t2) at (0.25, 4) {$\bra*{n}$};

                \vertex (v1) at (0.25, 2) ;
                \vertex (l1) at (1.55, 1.5) {$\omega_q$};
                \propag[extphoton] (l1) to (v1);
                \propag[plain] (g2) to (v1) ;
                
                \propag[plain] (v1) to (t2);

                \vertex (v2) at (-0.25, 1);
                \vertex (l2) at (-1.55, 0.5) {$\omega_p$};
                \propag[extphoton] (l2) to (v2);
                \propag[plain] (g1) to (v2);

                \vertex (o) at (-0.25, 3);
                \vertex (e) at (-1.55, 3.5) {$\omega_p + \omega_q$};
                \propag[outphoton] (o) to (e);
                \propag[plain] (v2) to[edge label={$\ket*{m}$}] (o);
                \propag[plain] (o) to (t1);
            \end{feynhand}
        \end{tikzpicture}
    \end{gathered} ,
\]
第一张图中电子状态的演化路径是
\[
    \dyad{g} \longrightarrow \dyad*{g}{n} \longrightarrow \dyad*{m}{n} \longrightarrow \dyad*{n}{n},
\]
第二张图中电子状态的演化路径为
\[
    \dyad*{g} \longrightarrow \dyad*{m}{g} \longrightarrow \dyad*{m}{n} \longrightarrow \dyad*{n}{n},
\]
两者不同。因此的确,左右两边的入射光子线如果上下位置不同,则产生不同的图。
实际上,此处的密度矩阵的双边费曼图实际上是Keldysh场论的一个特例(见\cite{Hansen_2012})。
最终可以证明,在远离共振的情况下两者是一致的。

\subsection{空间相位}

TODO:多个分子之间的间距如果恰到好处地让它们的辐射相位正好差了$\pi$,有可能在远处观察不到辐射



\chapter{均匀介质中的经典非线性光学过程}

\section{非线性极化的麦克斯韦方程}\label{sec:non-linear-maxwell}

本节将从一个比较唯象的角度描述经典非线性光学——我们忽略所有微观细节,只是去解麦克斯韦方程组,不过在这里,$\vb*{P}$和$\vb*{E}$之间的关系不再是线性的。
一般来说$E \sim \SI{1}{kV/cm}$时非线性效应变得重要起来。

\subsection{非线性波动方程的建立和微扰求解}

\subsubsection{均匀介质中的传播}

本节讨论均匀非线性介质中的光的传播。在大部分情况下介质对电磁波中的磁场的响应都是可以忽略的,因此介质对光的影响可以概括为极化。
以下我们用短直线代表极化$P$,用波浪线代表电场;后者的意义一目了然,前者和前面几节的意义其实也是一样的,因为无非有
\[
    \vb*{P} = N q \vb*{x},
\]
其中$N$为单位体积的谐振子数目。我们知道极化强度矢量可以产生一个电场。我们有波动方程
\[
    \curl{(\curl{\vb*{E}})} + \frac{1}{\epsilon_0 c^2} \pdv[2]{\vb*{D}}{t} = 0,
\]
代入$\vb*{D}$的定义,就得到
\begin{equation}
    \curl{(\curl{\vb*{E}})} + \frac{1}{c^2} \pdv[2]{\vb*{E}}{t} = - \frac{1}{\epsilon_0 c^2} \pdv[2]{\vb*{P}}{t}.
\end{equation}
实际上,这就是“电流作为一个天线,发出电磁波”,只不过此时天线是由总电场驱动的,从而上式左边和右边都有$\vb*{E}$,并且由于$\vb*{P}$和$\vb*{E}$之间的关心是非线性的,上式关于$\vb*{E}$也是非线性的。
不过,从$\vb*{P}$到$\vb*{E}$的关系的确是完全线性的,因此可以使用
\begin{equation}
    \begin{gathered}
        \begin{tikzpicture}
            \begin{feynhand}
                \vertex (a) at (0, 0);
                \vertex (b) at (1, 0);
                \vertex (c) at (2, 0);

                \propag[plain] (a) to (b);
                \propag[photon] (b) to (c);
            \end{feynhand}
        \end{tikzpicture}
    \end{gathered} = - \left( \curl{\curl{}} + \frac{1}{c^2} \pdv[2]{t} \right)^{-1} \frac{1}{\epsilon_0 c^2} \pdv[2]{\vb*{P}}{t}
\end{equation}
表示。这是$\vb*{P}$消灭而光子产生的过程,看起来非常直观:一个天线上的激发态消失了,取而代之的是天线发出来的光。
本节讨论的是均匀介质,我们可以认为$\grad{\rho_\text{p}}=0$,从而$\grad{\div{\vb*{P}}}=0$,从而就得到$\div{\vb*{E}} = 0$。这样一来我们要求解的就是
\begin{equation}
    - \laplacian \vb*{E} + \frac{1}{c^2} \pdv[2]{\vb*{E}}{t} = -\frac{1}{\epsilon_0 c^2} \pdv[2]{\vb*{P}}{t},
\end{equation}
而
\begin{equation}
    \begin{gathered}
        \begin{tikzpicture}
            \begin{feynhand}
                \vertex (a) at (0, 0);
                \vertex (b) at (1, 0);
                \vertex (c) at (2, 0);

                \propag[plain] (a) to (b);
                \propag[photon] (b) to (c);
            \end{feynhand}
        \end{tikzpicture}
    \end{gathered} = \left( \laplacian - \frac{1}{c^2} \pdv[2]{t} \right)^{-1} \frac{1}{\epsilon_0 c^2} \pdv[2]{\vb*{P}}{t}.
\end{equation}

此外我们还有任意的形如
\begin{equation}
    \begin{gathered}
        \begin{tikzpicture}
            \begin{feynhand}
                \vertex (a) at (-0.8, 0.25) ;
                \vertex (b) at (-0.8, -0.25) ;
                \vertex (d) at (0, -1) {$\ldots$};
                \vertex (c) at (0, 0);
                \vertex (e) at (1, 0);

                \propag[photon] (a) to (c);
                \propag[photon] (b) to (c);
                \propag[photon] (d) to (c);
                \propag[plain] (c) to (e);
            \end{feynhand}
        \end{tikzpicture}
    \end{gathered} = \epsilon_0 \chi^{(n)}_{i_1 i_2 \ldots i_n}
\end{equation}
的产生$\vb*{P}$的顶角,代表$n$阶非线性极化(一阶极化是线性的),使得
\[
    \begin{gathered}
        \begin{tikzpicture}
            \begin{feynhand}
                \vertex[crossdot] (a) at (-0.8, 0.25) {};
                \vertex[crossdot] (b) at (-0.8, -0.25) {};
                \vertex (d) at (0, -1) {$\ldots$};
                \vertex (c) at (0, 0);
                \vertex (e) at (1, 0);

                \propag[photon] (a) to (c);
                \propag[photon] (b) to (c);
                \propag[photon] (d) to (c);
                \propag[plain] (c) to (e);
            \end{feynhand}
        \end{tikzpicture}
    \end{gathered} = \epsilon_0 \vb*{\chi}^{(n)} : \vb*{E} \vb*{E} \cdots \vb*{E}.
\]
线性光学中只有一条入射线,非线性光学中可以有多条。

这就是非线性波动方程中的所有顶角了。我们马上可以发现一件事,就是虽然看起来$\vb*{P}$场能够有自能修正
\[
    \begin{gathered}
        \begin{tikzpicture}
            \begin{feynhand}
                \vertex (a) at (0, 0);
                \vertex (b) at (1, 0);
                \propag[double] (a) to (b);
            \end{feynhand}
        \end{tikzpicture}
    \end{gathered} = \begin{gathered}
        \begin{tikzpicture}
            \begin{feynhand}
                \vertex (a) at (0, 0);
                \vertex (b) at (1, 0);
                \propag[plain] (a) to (b);
            \end{feynhand}
        \end{tikzpicture}
    \end{gathered} + 
    \begin{gathered}
        \begin{tikzpicture}
            \begin{feynhand}
                \vertex (a) at (0, 0);
                \vertex (b) at (1, 0);
                \vertex (c) at (2, 0);
                \vertex (d) at (3, 0);
                \propag[plain] (a) to (b);
                \propag[photon] (b) to (c);
                \propag[plain] (c) to (d);
            \end{feynhand}
        \end{tikzpicture}
    \end{gathered} + \cdots,
\]
但是,因为我们讨论的所有过程都以光子开始以光子结束,加上入射和出射线之后上式实际上是光的自能修正。因此实际上$\vb*{P}$只是一个辅助量,我们可以将
\[
    \begin{gathered}
        \begin{tikzpicture}
            \begin{feynhand}
                \vertex (a) at (0, 0);
                \vertex (b) at (1, 0);
                \vertex (c) at (2, 0);
                \vertex (d) at (3, 0);
                \propag[photon] (a) to (b);
                \propag[plain] (b) to (c);
                \propag[photon] (c) to (d);
            \end{feynhand}
        \end{tikzpicture}
    \end{gathered}
\]
看成一个整体。当然,本该如此。

我们要求解外加电场下的总场。这就是说,要计算
\[
    \begin{aligned}
        \begin{gathered}
            \begin{tikzpicture}
                \begin{feynhand}
                    \vertex[grayblob] (a) at (0, 0) {};
                    \vertex (b) at (1.25, 0);
                    \propag[boldphoton] (a) to (b);
                \end{feynhand}
            \end{tikzpicture}
        \end{gathered} &= 
        \begin{gathered}
            \begin{tikzpicture}
                \begin{feynhand}
                    \vertex[crossdot] (a) at (0, 0) {};
                    \vertex (b) at (1, 0);
                    \propag[photon] (a) to (b);
                \end{feynhand}
            \end{tikzpicture} 
        \end{gathered} + 
        \begin{gathered}
            \begin{tikzpicture}
                \begin{feynhand}
                    \vertex[crossdot] (a) at (0, 0) {};
                    \vertex (b) at (1, 0);
                    \vertex (c) at (2, 0);
                    \vertex (d) at (3, 0);
                    \propag[photon] (a) to (b);
                    \propag[plain] (b) to (c);
                    \propag[photon] (c) to (d);
                \end{feynhand}
            \end{tikzpicture}
        \end{gathered} + \cdots \\
        &+ \begin{gathered}
            \begin{tikzpicture}
                \begin{feynhand}
                    \vertex[crossdot] (a) at (-0.8, 0.25) {};
                    \vertex[crossdot] (b) at (-0.8, -0.25) {};
                    \vertex (d) at (0, -1) {$\ldots$};
                    \vertex (c) at (0, 0);
                    \vertex (e) at (1, 0);
                    \vertex (f) at (2, 0);
    
                    \propag[photon] (a) to (c);
                    \propag[photon] (b) to (c);
                    \propag[photon] (d) to (c);
                    \propag[plain] (c) to (e);
                    \propag[photon] (e) to (f);
                \end{feynhand}
            \end{tikzpicture}
        \end{gathered}
    \end{aligned}
\]
请注意上式左边的源不是$\otimes$而是一个灰色圆圈,因为非线性效应的存在让我们不知道外源被用到了多少次。
上式直接计算当然是非常复杂的,我们做重求和,根据最后一个顶角是什么,可以将上式写成
\begin{equation}
    \begin{aligned}
        \begin{gathered}
            \begin{tikzpicture}
                \begin{feynhand}
                    \vertex[grayblob] (a) at (0, 0) {};
                    \vertex (b) at (1.25, 0);
                    \propag[boldphoton] (a) to (b);
                \end{feynhand}
            \end{tikzpicture}
        \end{gathered} &= \begin{gathered}
            \begin{tikzpicture}
                \begin{feynhand}
                    \vertex[crossdot] (a) at (0, 0) {};
                    \vertex (b) at (1, 0);
                    \propag[photon] (a) to (b);
                \end{feynhand}
            \end{tikzpicture} 
        \end{gathered} +
        \begin{gathered}
            \begin{tikzpicture}
                \begin{feynhand}
                    \vertex[grayblob] (a) at (0, 0) {};
                    \vertex (b) at (1, 0);
                    \vertex (c) at (1.6, 0);
                    \vertex (d) at (2.2, 0);
                    \propag[boldphoton] (a) to (b);
                    \propag[plain] (b) to (c);
                    \propag[photon] (c) to (d);
                \end{feynhand}
            \end{tikzpicture}
        \end{gathered} +
        \sum \begin{gathered}
            \begin{tikzpicture}
                \begin{feynhand}
                    \vertex[grayblob] (a) at (-1.2, 0.5) {};
                    \vertex[grayblob] (b) at (-1.2, -0.5) {};
                    \vertex (f) at (0, -1) {$\ldots$};
                    \vertex (c) at (0, 0);
                    \vertex (d) at (0.75, 0);
                    \vertex (e) at (1.5, 0);
    
                    \propag[boldphoton] (a) to (c);
                    \propag[boldphoton] (b) to (c);
                    \propag[boldphoton] (f) to (c);
                    \propag[plain] (c) to (d);
                    \propag[photon] (d) to (e);
                \end{feynhand}
            \end{tikzpicture}
        \end{gathered} \\
        &= \begin{gathered}
            \begin{tikzpicture}
                \begin{feynhand}
                    \vertex[crossdot] (a) at (0, 0) {};
                    \vertex (b) at (1, 0);
                    \propag[photon] (a) to (b);
                \end{feynhand}
            \end{tikzpicture} 
        \end{gathered} +
        \begin{gathered}
            \begin{tikzpicture}
                \begin{feynhand}
                    \vertex[grayblob] (a) at (0, 0) {};
                    \vertex (b) at (1, 0);
                    \vertex (c) at (2, 0);
                    \propag[plain] (a) to (b);
                    \propag[photon] (b) to (c);
                \end{feynhand}
            \end{tikzpicture}
        \end{gathered} \ .
    \end{aligned}
\end{equation}
当然,这就是
\[
    \vb*{E} = \vb*{E}_\text{ext} + \left( \laplacian - \frac{1}{c^2} \pdv[2]{t} \right)^{-1} \frac{1}{\epsilon_0 c^2} \pdv[2]{\vb*{P}}{t} ,
\]
稍加变形就会发现实际上这就是
\[
    \left( \laplacian - \frac{1}{c^2} \pdv[2]{t} \right) (\vb*{E} - \vb*{E}_\text{ext}) = \frac{1}{\epsilon_0 c^2} \pdv[2]{\vb*{P}}{t}.
\]
求解出上式中的$\vb*{E}$就完全确定了外加电场下的总场。“外加电场”当然是一个不良定义的概念,但这是因为我们将$\otimes$的来源省去了;它可能来自介质中的电荷或是介质两边的极板(归根到底,也是一些电荷),无论如何不来自$\vb*{P}$,即它是“没有经过相互作用修正的”,从而
\[
    \left( \laplacian - \frac{1}{c^2} \pdv[2]{t} \right) \vb*{E}_\text{ext} = 0.
\]
如果将理论量子化,它就是入射的近乎自由的粒子。我们经常称外加电场为\concept{泵浦光},因为它将材料中的电子激发。%
\footnote{
    只要满足这个形式的电场分量都有机会被称为泵浦光。因此这个概念的确有定义模糊的地方。它可以指代从空气射向介质的入射光,可以指代将空气-介质界面当成线性介质界面而获得的折射光,甚至我们不能排除它指代经过局域场修正之后的折射光。
    本节讨论均匀介质内部的宏观问题,因此泵浦光不是从空气到介质的入射光也不需要做局域场修正。
}%

不过,具体什么是“相互作用修正”仍然有选择的余地,如我们可以将线性极化(即$\vb*{P}$关于$\vb*{E}$是线性的那部分;从$\vb*{P}$到$\vb*{E}$的转换则完全是线性的)部分作为自能修正重求和,然后将经过线性极化修正的光束(也就是服从关于不等于$\epsilon_0$的$\epsilon$的线性麦克斯韦方程组的光束)重新定义为\begin{tikzpicture}
    \begin{feynhand}
        \vertex (a) at (0, 0);
        \vertex (b) at (1, 0);
        \propag[photon] (a) to (b);
    \end{feynhand}
\end{tikzpicture},此时\begin{tikzpicture}
    \begin{feynhand}
        \vertex (a) at (0, 0);
        \vertex (b) at (1, 0);
        \vertex (c) at (2, 0);

        \propag[plain] (a) to (b);
        \propag[photon] (b) to (c);
    \end{feynhand}
\end{tikzpicture}也需要调整(原本的意思是总的$\vb*{P}$产生真空中电场的响应函数,现在的意思是(非线性的)$\vb*{P}$产生线性介质中电场的响应函数),引入$\epsilon_{ij}$张量,然后顶角只剩下一种,就有
\begin{equation}
    \begin{gathered}
        \begin{tikzpicture}
            \begin{feynhand}
                \vertex[grayblob] (a) at (0, 0) {};
                \vertex (b) at (1.25, 0);
                \propag[boldphoton] (a) to (b);
            \end{feynhand}
        \end{tikzpicture}
    \end{gathered} = \begin{gathered}
        \begin{tikzpicture}
            \begin{feynhand}
                \vertex[crossdot] (a) at (0, 0) {};
                \vertex (b) at (1, 0);
                \propag[photon] (a) to (b);
            \end{feynhand}
        \end{tikzpicture} 
    \end{gathered} +
    \sum \begin{gathered}
        \begin{tikzpicture}
            \begin{feynhand}
                \vertex[grayblob] (a) at (-1.2, 0.5) {};
                \vertex[grayblob] (b) at (-1.2, -0.5) {};
                \vertex (f) at (0, -1) {$\ldots$};
                \vertex (c) at (0, 0);
                \vertex (d) at (0.75, 0);
                \vertex (e) at (1.5, 0);

                \propag[boldphoton] (a) to (c);
                \propag[boldphoton] (b) to (c);
                \propag[boldphoton] (f) to (c);
                \propag[plain] (c) to (d);
                \propag[photon] (d) to (e);
            \end{feynhand}
        \end{tikzpicture}
    \end{gathered} \ ,
\end{equation}
即
\begin{equation}
    \left( \laplacian - \frac{\vb*{\epsilon}_\text{r} \cdot}{c^2} \pdv[2]{t} \right) (\vb*{E} - \vb*{E}_\text{ext}) = \left( \laplacian - \frac{\vb*{\epsilon}_\text{r} \cdot}{c^2} \pdv[2]{t} \right) \vb*{E} = \frac{1}{\epsilon_0 c^2} \pdv[2]{\vb*{P}_\text{NL}}{t},
    \label{eq:nonlinear-maxwell-eq}
\end{equation}
其中NL表示非线性。这样做的好处一目了然:线性极化相当于对光速做了一个修正,是可以严格处理的,从而可以将主要精力集中在非线性极化上。

一个在线性光学中也会出现的现象是\concept{局域场强化}。这个现象来自于,从$\vb*{P}$产生的电场在空间上可能有比较大的变化,我们关心的是空间平均过之后的电场,即只是从$\vb*{P}$产生的电场的低频傅里叶分量。
因此我们做分解
\begin{equation}
    \begin{gathered}
        \begin{tikzpicture}
            \begin{feynhand}
                \vertex (a) at (0, 0);
                \vertex (b) at (1, 0);
                \vertex (c) at (2, 0);

                \propag[plain] (a) to (b);
                \propag[photon] (b) to (c);
            \end{feynhand}
        \end{tikzpicture}
    \end{gathered} = 
    \begin{gathered}
        \begin{tikzpicture}
            \begin{feynhand}
                \vertex (a) at (0, 0);
                \vertex (b) at (1, 0);
                \vertex (c) at (2, 0);

                \propag[plain] (a) to (b);
                \propag[photon, mom={$\text{low $k$}$}] (b) to (c);
            \end{feynhand}
        \end{tikzpicture}
    \end{gathered} + 
    \begin{gathered}
        \begin{tikzpicture}
            \begin{feynhand}
                \vertex (a) at (0, 0);
                \vertex (b) at (1, 0);
                \vertex (c) at (2, 0);

                \propag[plain] (a) to (b);
                \propag[photon, mom={$\text{high $k$}$}] (b) to (c);
            \end{feynhand}
        \end{tikzpicture}
    \end{gathered},
\end{equation}
那么就有
\begin{equation}
    \begin{aligned}
        \begin{gathered}
            \begin{tikzpicture}
                \begin{feynhand}
                    \vertex[grayblob] (a) at (0, 0) {};
                    \vertex (b) at (1.25, 0);
                    \propag[boldphoton, mom={$\text{low $k$}$}] (a) to (b);
                \end{feynhand}
            \end{tikzpicture}
        \end{gathered} &= \begin{gathered}
            \begin{tikzpicture}
                \begin{feynhand}
                    \vertex[crossdot] (a) at (0, 0) {};
                    \vertex (b) at (1, 0);
                    \propag[photon, mom={$\text{low $k$}$}] (a) to (b);
                \end{feynhand}
            \end{tikzpicture} 
        \end{gathered} +
        \begin{gathered}
            \begin{tikzpicture}
                \begin{feynhand}
                    \vertex[grayblob] (a) at (0, 0) {};
                    \vertex (b) at (1, 0);
                    \vertex (c) at (1.6, 0);
                    \vertex (d) at (2.2, 0);
                    \propag[boldphoton] (a) to (b);
                    \propag[plain] (b) to (c);
                    \propag[photon, mom={$\text{low $k$}$}] (c) to (d);
                \end{feynhand}
            \end{tikzpicture}
        \end{gathered} +
        \sum \begin{gathered}
            \begin{tikzpicture}
                \begin{feynhand}
                    \vertex[grayblob] (a) at (-1.2, 0.5) {};
                    \vertex[grayblob] (b) at (-1.2, -0.5) {};
                    \vertex (f) at (0, -1) {$\ldots$};
                    \vertex (c) at (0, 0);
                    \vertex (d) at (0.75, 0);
                    \vertex (e) at (1.5, 0);
    
                    \propag[boldphoton] (a) to (c);
                    \propag[boldphoton] (b) to (c);
                    \propag[boldphoton] (f) to (c);
                    \propag[plain] (c) to (d);
                    \propag[photon, mom={$\text{low $k$}$}] (d) to (e);
                \end{feynhand}
            \end{tikzpicture}
        \end{gathered} \\
        &= \begin{gathered}
            \begin{tikzpicture}
                \begin{feynhand}
                    \vertex[crossdot] (a) at (0, 0) {};
                    \vertex (b) at (1, 0);
                    \propag[photon, mom={$\text{low $k$}$}] (a) to (b);
                \end{feynhand}
            \end{tikzpicture} 
        \end{gathered} +
        \begin{gathered}
            \begin{tikzpicture}
                \begin{feynhand}
                    \vertex[grayblob] (a) at (0, 0) {};
                    \vertex (b) at (1, 0);
                    \vertex (c) at (2, 0);
                    \propag[plain] (a) to (b);
                    \propag[photon, mom={$\text{low $k$}$}] (b) to (c);
                \end{feynhand}
            \end{tikzpicture}
        \end{gathered} \ .
    \end{aligned}
    \label{eq:non-linear-smooth-p-e}
\end{equation}
然而,在上式中与$\vb*{P}$有关的图中的入射电场却没有做这个动量截断。
其结果是我们无法单纯从上式计算经过动量截断的出射电场,即我们需要手动提供没做动量截断的入射电场。
由于入射电场通常只具有低频分量,即
\[
    \begin{gathered}
        \begin{tikzpicture}
            \begin{feynhand}
                \vertex[crossdot] (a) at (0, 0) {};
                \vertex (b) at (1, 0);
                \propag[photon, mom={$\text{low $k$}$}] (a) to (b);
            \end{feynhand}
        \end{tikzpicture} 
    \end{gathered} = \begin{gathered}
        \begin{tikzpicture}
            \begin{feynhand}
                \vertex[crossdot] (a) at (0, 0) {};
                \vertex (b) at (1, 0);
                \propag[photon] (a) to (b);
            \end{feynhand}
        \end{tikzpicture} 
    \end{gathered},
\]
我们有
\[
    \begin{aligned}
        \begin{gathered}
            \begin{tikzpicture}
                \begin{feynhand}
                    \vertex[grayblob] (a) at (0, 0) {};
                    \vertex (b) at (1.25, 0);
                    \propag[boldphoton, mom={$\text{high $k$}$}] (a) to (b);
                \end{feynhand}
            \end{tikzpicture}
        \end{gathered} &= 
        \begin{gathered}
            \begin{tikzpicture}
                \begin{feynhand}
                    \vertex[grayblob] (a) at (0, 0) {};
                    \vertex (b) at (1, 0);
                    \vertex (c) at (1.6, 0);
                    \vertex (d) at (2.5, 0);
                    \propag[boldphoton] (a) to (b);
                    \propag[plain] (b) to (c);
                    \propag[photon, mom={$\text{high $k$}$}] (c) to (d);
                \end{feynhand}
            \end{tikzpicture}
        \end{gathered} +
        \sum \begin{gathered}
            \begin{tikzpicture}
                \begin{feynhand}
                    \vertex[grayblob] (a) at (-1.2, 0.5) {};
                    \vertex[grayblob] (b) at (-1.2, -0.5) {};
                    \vertex (f) at (0, -1) {$\ldots$};
                    \vertex (c) at (0, 0);
                    \vertex (d) at (0.75, 0);
                    \vertex (e) at (1.5, 0);
    
                    \propag[boldphoton] (a) to (c);
                    \propag[boldphoton] (b) to (c);
                    \propag[boldphoton] (f) to (c);
                    \propag[plain] (c) to (d);
                    \propag[photon, mom={$\text{high $k$}$}] (d) to (e);
                \end{feynhand}
            \end{tikzpicture}
        \end{gathered} \\
        &= \begin{gathered}
            \begin{tikzpicture}
                \begin{feynhand}
                    \vertex[grayblob] (a) at (0, 0) {};
                    \vertex (b) at (1, 0);
                    \vertex (c) at (2, 0);
                    \propag[plain] (a) to (b);
                    \propag[photon, mom={$\text{high $k$}$}] (b) to (c);
                \end{feynhand}
            \end{tikzpicture}
        \end{gathered} \ ,
    \end{aligned}
\]
于是就有
\begin{equation}
    \begin{gathered}
        \begin{tikzpicture}
            \begin{feynhand}
                \vertex[grayblob] (a) at (0, 0) {};
                \vertex (b) at (1.25, 0);
                \propag[boldphoton] (a) to (b);
            \end{feynhand}
        \end{tikzpicture}
    \end{gathered} = \begin{gathered}
        \begin{tikzpicture}
            \begin{feynhand}
                \vertex[grayblob] (a) at (0, 0) {};
                \vertex (b) at (1.25, 0);
                \propag[boldphoton, mom={$\text{low $k$}$}] (a) to (b);
            \end{feynhand}
        \end{tikzpicture}
    \end{gathered} + 
    \begin{gathered}
        \begin{tikzpicture}
            \begin{feynhand}
                \vertex[grayblob] (a) at (0, 0) {};
                \vertex (b) at (1, 0);
                \vertex (c) at (2, 0);
                \propag[plain] (a) to (b);
                \propag[photon, mom={$\text{high $k$}$}] (b) to (c);
            \end{feynhand}
        \end{tikzpicture}
    \end{gathered}.
    \label{eq:local-field-enhancement-diagram}
\end{equation}
联立计算\eqref{eq:non-linear-smooth-p-e}和\eqref{eq:local-field-enhancement-diagram}(求解过程中很容易看到,虽然$\vb*{P}$是辅助量但是确实能够简化书写),就得到外场作用下的做了动量截断的$\vb*{E}$——实际上,由于宏观下我们讨论的“介质中电场”一般都是均一化的电场,我们将做了动量截断的$\vb*{E}$就称为$\vb*{E}$。%
\footnote{
    电动力学教科书上通常会泛泛地说这是“做了宏观平均的电场”,实际上这就是动量截断。
    要看出这是为什么,注意到“做了宏观平均的电场”总是可以通过一个体积为$V$的移动平均给出:
    \[
        \bar{\vb*{E}}(\vb*{r}) = \frac{1}{V} \int \dd[3]{\vb*{r}'} \vb*{E}(\vb*{r}'),
    \]
    做傅里叶展开,有
    \[
        \begin{aligned}
            \bar{\vb*{E}}(\vb*{r}) &= \frac{1}{V} \int \dd[3]{\vb*{r}'} \int \frac{\dd[3]{\vb*{k}}}{(2\pi)^3} \vb*{E}(\vb*{k}) \ee^{\ii \vb*{k} \cdot \vb*{r}'} \\
            &= \int \frac{\dd[3]{\vb*{k}}}{(2\pi)^3} \vb*{E}(\vb*{k}) \frac{1}{V} \int \dd[3]{\vb*{r}'} \ee^{\ii \vb*{k} \cdot \vb*{r}'},
        \end{aligned}
    \]
    显然对那些较大的$\vb*{k}$,对$\vb*{r}'$的积分会由于快速振荡而变成零,于是我们就得到了一个动量截断。
}%
此时\eqref{eq:local-field-enhancement-diagram}应当理解为“靠近介质粒子时电场的局域增强”。
在这种视角下,有
\begin{equation}
    \begin{aligned}
        \begin{gathered}
            \begin{tikzpicture}
                \begin{feynhand}
                    \vertex[grayblob] (a) at (0, 0) {};
                    \vertex (b) at (1.25, 0);
                    \propag[boldphoton] (a) to (b);
                \end{feynhand}
            \end{tikzpicture}
        \end{gathered} &= \begin{gathered}
            \begin{tikzpicture}
                \begin{feynhand}
                    \vertex[crossdot] (a) at (0, 0) {};
                    \vertex (b) at (1, 0);
                    \propag[photon] (a) to (b);
                \end{feynhand}
            \end{tikzpicture} 
        \end{gathered} +
        \begin{gathered}
            \begin{tikzpicture}
                \begin{feynhand}
                    \vertex[grayblob] (a) at (0, 0) {};
                    \vertex (b) at (1, 0);
                    \vertex (c) at (1.6, 0);
                    \vertex (d) at (2.2, 0);
                    \propag[boldphoton, mom={$\text{enhanced}$}] (a) to (b);
                    \propag[plain] (b) to (c);
                    \propag[photon] (c) to (d);
                \end{feynhand}
            \end{tikzpicture}
        \end{gathered} +
        \sum \begin{gathered}
            \begin{tikzpicture}
                \begin{feynhand}
                    \vertex[grayblob] (a) at (-1.2, 0.5) {};
                    \vertex[grayblob] (b) at (-1.2, -0.5) {};
                    \vertex (f) at (0, -1) {$\ldots$};
                    \vertex (c) at (0, 0);
                    \vertex (d) at (0.75, 0);
                    \vertex (e) at (1.5, 0);
    
                    \propag[boldphoton, mom={$\text{enhanced}$}] (a) to (c);
                    \propag[boldphoton, mom={$\text{enhanced}$}] (b) to (c);
                    \propag[boldphoton, mom={$\text{enhanced}$}] (f) to (c);
                    \propag[plain] (c) to (d);
                    \propag[photon] (d) to (e);
                \end{feynhand}
            \end{tikzpicture}
        \end{gathered} \\
        &= \begin{gathered}
            \begin{tikzpicture}
                \begin{feynhand}
                    \vertex[crossdot] (a) at (0, 0) {};
                    \vertex (b) at (1, 0);
                    \propag[photon] (a) to (b);
                \end{feynhand}
            \end{tikzpicture} 
        \end{gathered} +
        \begin{gathered}
            \begin{tikzpicture}
                \begin{feynhand}
                    \vertex[grayblob] (a) at (0, 0) {};
                    \vertex (b) at (1, 0);
                    \vertex (c) at (2, 0);
                    \propag[plain] (a) to (b);
                    \propag[photon] (b) to (c);
                \end{feynhand}
            \end{tikzpicture}
        \end{gathered} , \\
        \begin{gathered}
            \begin{tikzpicture}
                \begin{feynhand}
                    \vertex[grayblob] (a) at (0, 0) {};
                    \vertex (b) at (1.25, 0);
                    \propag[boldphoton, mom={$\text{enhanced}$}] (a) to (b);
                \end{feynhand}
            \end{tikzpicture}
        \end{gathered} &= \begin{gathered}
            \begin{tikzpicture}
                \begin{feynhand}
                    \vertex[grayblob] (a) at (0, 0) {};
                    \vertex (b) at (1.25, 0);
                    \propag[boldphoton] (a) to (b);
                \end{feynhand}
            \end{tikzpicture}
        \end{gathered} + 
        \begin{gathered}
            \begin{tikzpicture}
                \begin{feynhand}
                    \vertex[grayblob] (a) at (0, 0) {};
                    \vertex (b) at (1, 0);
                    \vertex (c) at (2, 0);
                    \propag[plain] (a) to (b);
                    \propag[photon, mom={$\text{high $k$}$}] (b) to (c);
                \end{feynhand}
            \end{tikzpicture}
        \end{gathered}.
    \end{aligned}
\end{equation}

\subsubsection{边界条件}

边界,反射,折射:在线性的折射、反射上引入三个相互作用,一个是介质内部的光子-光子和频和差频,一个是入射端,介质内部两个光子打在边界上然后返回形成一个光子,一个是出射端,两个光子打在边界上然后一起出射。

\subsubsection{哪些解是重要的?}

非线性光学和线性光学非常不同的地方是,在线性光学中,给定一个时谐的(并且通常是平面波的,因为介质一般都很大,近似空间平移不变)泵浦光,计算入射导致的响应以及总电场,我们就得到了一个\emph{稳定的}模式;一个任意的入射波传入介质后的行为可以直接根据入射波在各个模式上的分量计算出来。
这意味着在线性光学中我们无需为任意的泵浦光求解麦克斯韦方程:只需要得到线性波动方程的所有振动模式就可以,各个振动模式的形状决定了这个光学系统的全部信息;场论上,这是自由场的情况。
线性性也意味着一个光束可以被分割成许多波包,计算每个波包的时间演化,加起来就得到光束的时间演化——实际上,我们经常反过来用线性性,即先求解一个系统中的电磁波模式(其时间演化是平凡的)然后诠释说“这意味着光先走到这里再走到那里”。
非线性光学则不同,输出和激励并没有简单的线性关系。不过实际上,这并不意味着我们真的需要为所有的泵浦场独立地从头求解非线性波动方程。

首先,我们看到\eqref{eq:nonlinear-maxwell-eq}相对于时间是线性的。这意味着我们可以在时间的频域求解\eqref{eq:nonlinear-maxwell-eq}。
也就是说,\eqref{eq:nonlinear-maxwell-eq}的全体时谐解已经包含了诸如“两个略有错开的波包打进介质将会如何反应”之类的瞬态问题的答案——只要入射波包不过于极端,以至于非线性麦克斯韦方程不再适用(如非常快导致介质来不及弛豫或是过强导致介质损坏等)。
在线性光学中这是常用的手段:例如,我们求解了时谐场在金属中的衰减,就可以不重新求解麦克斯韦方程就知道一个波包在金属中如何衰减,做一个傅里叶变换就行了:稳态问题提供了瞬态问题的答案,而前者往往更加容易做具体计算。
非线性光学中这个手段也是适用的。
因此,我们会在频域求解\eqref{eq:nonlinear-maxwell-eq},然后看着光强随着光束前进的\emph{距离}发生的变化,评论说“波包随着传播\emph{时间}发生的变化”,因为两者是可以互相转化的。

其次,似乎存在这样一个疑难:在后文中我们将会求解诸如“泵浦光含有两个频率成分,它们都是平面波”这样的问题,但是实际上我们能够制备的光都是含有很多混杂的成分的,那么有什么理由保证后者发生的现象和前者一致?
例如,设泵浦光为
\[
    \vb*{E}^\text{ext} = \vb*{E}_1 \ee^{-\ii \omega_1 t} + \vb*{E}_2 \ee^{-\ii \omega_2 t} + \text{c.c.},
\]
微扰求解得到输出光为
\[
    \vb*{E} = \alpha^{(1)}(\omega_1) \vb*{E}_1 \ee^{-\ii \omega_1 t} + \alpha^{(1)}(\omega_2) \vb*{E}_2 \ee^{-\ii \omega_2 t} + \alpha^{(2)}(\omega_3 = \omega_1 + \omega_2) : \vb*{E}_1 \vb*{E}_2 \ee^{-\ii (\omega_1 + \omega_2) t} + \cdots + \text{c.c.}.
\]
现在如果我们要将泵浦光取为
\[
    \vb*{E}^\text{ext}(t) = \int \dd{\omega} \ee^{-\ii \omega t} \vb*{E}^\text{ext}(\omega),
\]
那么输出光$E(t)$不能通过简单地做一个傅里叶变换获得。
但其实我们可以效仿量子场论中的做法来解决这个问题。
我们要做的事情是通过计算\eqref{eq:nonlinear-maxwell-eq}的一些特解来获得通用的“看到泵浦光-写出输出光”的方法,这相当于我们在积掉整个光学系统,直接得到泵浦光的各个频率分量和输出的非线性谐波之间的关系。
所以其实可以采用量子场论中计算等效相互作用顶角的标准方法,即只计算连通图:设泵浦光包含$n$个(频率不同的)单频成分,微扰求解\eqref{eq:nonlinear-maxwell-eq},只计算那些完全连通且正好包含$n$个\begin{tikzpicture}
    \begin{feynhand}
        \vertex[crossdot] (a) at (0, 0) {};
        \vertex (b) at (1, 0);
        \propag[photon] (a) to (b);
    \end{feynhand}
\end{tikzpicture}的图(后一个条件是为了保证我们计算的是$n$阶非线性过程,前一个条件是为了保证低于$n$阶的非线性过程没有被重复计数),这样可以得到形如$\alpha^{(n)} \vb*{E} \vb*{E} \cdots \vb*{E}$的响应,输出光就是
\[
    \vb*{E} = \int \dd{\omega} \alpha^{(1)}(\omega) \vb*{E}^\text{ext}(\omega) \ee^{-\ii \omega t} + \int \dd{\omega_1} \int \dd{\omega_2} \alpha^{(2)}(\omega_1 + \omega_2) : \vb*{E}^\text{ext}(\omega_1) \vb*{E}^\text{ext}(\omega_2) \ee^{-\ii (\omega_1 + \omega_2) t} + \cdots.
\]
请注意由于不存在圈图,每个$\alpha^{(n)}$中的图的数目是有限的,从而可以精确计算而不需要对圈数做截断等。

总之,虽然非线性光学问题比线性光学复杂上不少,我们仍然可以只计算泵浦光为若干个单频光叠加的情况,并且计算到给定的微扰阶数,这样足够提取到足够多的关于非线性光学体系的信息。

\subsection{近似求解方法}

由前述讨论,我们只要拿到非线性极化$\vb*{P}_\text{NL}$,就能够完全确定一个介质的非线性光学性能,只需要计算向介质中输入若干束单频光得到的输出就够了。
此时对\eqref{eq:nonlinear-maxwell-eq}做时域傅里叶变换,就得到不同频率的电场的空间分布方程,$\vb*{P}_\text{NL}$中如果有电场的$n-1$次方,这就是\concept{$n$波混频方程}。

求解$n$波混频方程时会用到一系列近似,列举如下:
\begin{itemize}
    \item \concept{泵浦光无衰减近似},在非线性介质非常厚,或者转换效率非常高以至于泵浦光衰减很明显时失效。
    \item \concept{平面波近似},即认为介质中电磁波的所有频率分量都可以看成(振幅可能会变化的)平面波。
    \item \concept{慢变振幅近似},和几何光学中的那种类似。在分析超短脉冲时失效。
\end{itemize}

我们仍然可以定义折射率的概念。
TODO:$n^2 = 1 + \chi$

\subsection{非线性极化率和对称性}

通过对称性可以减少非线性极化率的独立分量个数。

\subsubsection{置换对称性}


\section{二阶非线性极化的波动光学}\label{sec:chi-2-wave}

之前已经讨论过一个问题:具有两个频率分量的泵浦光被打入具有二阶非线性极化的材料中,计算和频光的强度。本节用非线性波动方程重新计算它。
二阶非线性极化意味着材料的中心反演对称性破缺。

本节中如无说明,$\omega>0$,能量守恒条件形如$\omega_3 = \omega_1 + \omega_2$,我们总是使用$\omega_3$表示频率最大的模式。

\subsection{SFG过程的一阶微扰}

\subsubsection{输入光为两束泵浦光}

我们考虑一个最为简单的情况:各向同性介质,泵浦光足够强,以至于在介质中泵浦光几乎没有衰减,即做了\concept{泵浦波无衰减近似}。
假定泵浦光中有两个频率分量的光,它们的偏振方向均相同。取其波矢指向为$z$轴。
我们用$E_i$指代第$i$种傅里叶分量,用$A_i$表示其大小,即
\[
    E_i = \ee^{\ii k_i z - \ii \omega_i t} A_i + \ee^{- \ii k_i z + \ii \omega_i t} A_i^*.
\]
在线性光学中只取正频率或是负频率的分量不会造成任何问题,并且通常会让问题更加容易,但是我们现在在讨论非线性过程,因此$E$必须是实际的入射光,显然必须是实数。
这样,非线性极化导致的一阶微扰中%
\footnote{
    “阶”在本文中有两个意思,一个是非线性光学过程的阶数,$n$阶非线性光学过程就是$n$束光变成一束光;还有一个意思是微扰计算的阶数。
    前者描述单个顶角的粒子线数目,后者描述一张图中顶角的总数。
}%
,SFG过程给出修正(我们为何只考虑SFG过程,而且只考虑$\omega_2$和$\omega_3$指定的SFG过程马上就可以看到;相位匹配条件决定了SFG过程和DFG过程等不太可能同时很重要)
\[
    \left( \pdv[2]{z} - \frac{\epsilon_\text{r}}{c^2} \pdv[2]{t} \right) E_3 = \frac{1}{\epsilon_0 c^2} \pdv[2]{t} \left( 2 \epsilon_0  \chi^{(2)} A_1 A_2 \ee^{\ii (k_1 + k_2) z} \ee^{-\ii \omega_3 t} \right) + \text{h.c.},
\]
其中我们设$\omega_3 = \omega_1 + \omega_2$,$\chi^{(2)}$指的是$\chi^{(2)}(\omega_3=\omega_1+\omega_2)$。
上式右边的因子$2$来自$\chi^{(2)} \vb*{E} \vb*{E}$展开之后——实际上,也可以用一种更绕的办法得出这个因子:设拉氏量中产生了$\chi^{(2)} \vb*{E} \vb*{E}$项的项的耦合常数为$\alpha$,则大体上
\[
    \dv{\vb*{E}} (\alpha \vb*{E} \vb*{E} \vb*{E}) = \chi^{(2)} \vb*{E} \vb*{E},
\]
有
\[
    \alpha = \frac{1}{3} \chi^{(2)},
\]
由于$\alpha$对应一个三条粒子线的顶角,它需要乘上对称性因子$3!$,于是最后就有因子$3! / 3 = 2$。
请注意没有什么能够保证线性极化是频率无关的——没有理由认为$\omega=\omega(k)$是线性的。
$E_3$的频率当然还是$\omega_3$。我们假定$E_3$大体上仍然是一个平面波(即做了\concept{平面波近似}),但是随着$z$增大,$A_3$会有变化。这样就有
\[
    \ee^{\ii k_3 z - \ii \omega_3 t} \left( \pdv[2]{z} + 2 \ii k_3 \pdv{z} - k_3^2 + \frac{\epsilon_\text{r}}{c^2} \omega^2 \right) A_3 = - \frac{\omega_3^2}{\epsilon_0 c^2} 2 \epsilon_0  \chi^{(2)} A_1 A_2 \ee^{\ii (k_1 + k_2) z} \ee^{-\ii \omega_3 t} .
\]
由于色散关系可以不是线性的,从而$k_3$和$k_1 + k_2$完全可以不同。
上式左边括号中最后两项抵消了。再做\emph{慢变振幅近似}(几何光学中就做了这一近似,也可以称为旁轴近似),就有
\[
    \pdv{z} A_3 = \frac{\ii \omega_3^2}{k_3 c^2} \chi^{(2)} A_1 A_2 \ee^{\ii (k_1 + k_2 - k_3) z}.
\]
求解这一方程,考虑到泵浦光中没有$E_3$分量,从而$z=0$处$A_3=0$,以及晶体厚度为$L$,就得到
\begin{equation}
    A_3(L) = \frac{\ii \chi^{(2)} \omega_3^2 A_1 A_2}{k_3 c^2} \left( \frac{\ee^{\ii \Delta k L} - 1}{\ii \Delta k} \right),
    \label{eq:two-pump-a3}
\end{equation}
其中
\begin{equation}
    \Delta k = k_1 + k_2 - k_3.
\end{equation}
$E_3$分量的强度是
\[
    I_3 = \expval*{\epsilon_0 n_3 c E_3^2} = 2 \epsilon_0 n_3 c \abs*{A_3}^2,
\]
于是就得到
\begin{equation}
    I_3 = \frac{8 \epsilon_0 (\chi^{(2)})^2 \abs*{A_1}^2 \abs*{A_2}^2 \omega_3^2}{n_3 c} \frac{\sin^2(\Delta k L / 2)}{(\Delta k)^2} = \frac{(\chi^{(2)})^2 \omega_3^2 I_1 I_2}{2 \epsilon_0 c^3 n_1 n_2 n_3} \mathrm{sinc}^2 \left( \frac{\Delta k L}{2} \right) L^2.
    \label{eq:sfg-intensity}
\end{equation}

如果\concept{相位匹配条件}——即$k_3 = k_1 + k_2$——的确成立,那么$I_3$就稳定地随着$L^2$增大而线性增大,此时的SFG转换效率是非常高的。
当然,$L$大到一定程度时,$I_1$和$I_2$就开始随着$L$增大而下降,并且吸收也开始明显了。
大部分情况下相位匹配条件是没法成立的,因为色散非线性。此时随着$L$的增大,能量先是从$E_1$和$E_2$移向$E_3$模式,然后再从$E_3$模式移向$E_1$和$E_2$模式。
大约需要
\begin{equation}
    L_\text{coh} = \frac{1}{\Delta k}
\end{equation}
的距离可以看到明显的能量转移,这个长度称为\concept{相干长度}。

相位匹配条件叫做这个名字并非没有原因。在某个位置$z$处新产生的$E_3$光大体上形如
\[
    E_3 \sim E_1 E_2 \sim A_1 A_2 \ee^{\ii (k_1 + k_2) z - \ii (\omega_1 + \omega_2) t},
\]
如果$\Delta k \neq 0$,那么新产生的$E_3$光的波矢并不是$k_3$,从而它和原来已有的$E_3$光就会有一个随着$z$增大不断变化的相位差,在一些地方它们会发生相消干涉。
这就是$\Delta k \neq 0$时$E_3$光强度发生振荡的原因。

这里要说明一点:表面上相位匹配条件似乎应该总是成立,因为波矢实际上就是动量而我们讨论的系统动量守恒。
然而,我们并没有真的在动量空间下工作:$A_i$是可以有空间起伏的,而在动量空间中它们只能是定值。
不确定性原理告诉我们,此时无法严格定义$k$,或者等价地说用慢变振幅近似定义出来的$k$并不是真正的动量,从而也不必严格遵循动量守恒定律。
只有相位匹配时才有较高的转换效率这件事实际上就是一个弱化版本的动量守恒条件。
实际上,在分析SFG以外的过程时,不必做任何计算,只要系统具有空间平移对称性,我们就知道,相位匹配条件成立时转换效率最高,因为这是弱化版本的动量守恒条件。

相位匹配时才有较高效的能量转换这一事实实际上大大简化了我们的分析。
原则上,我们将频率为$\omega_1$和$\omega_2$的光输入一个非线性晶体,所有过程——和频,差频,倍频,光学整流——都会发生,因此我们\emph{不能}只取一个过程并按照它写出振幅分布方程。
例如,种子光可以因为DFG过程而指数上升,也可以因为SFG过程而振荡,这两个效应按理说应该叠加,然后反过来影响SFG和DFG产生的光的强度。
然而,很多时候,入射光给定之后SFG和DFG过程或者其它一些过程不能同时满足相位匹配条件,从而\eqref{eq:sfg-intensity}分母中的$\Delta k$会压低$I_3$能够达到的最大值。
因此,其实可以分开计算每个过程。
在有广谱的输入光——如制作光谱仪时——这个近似可能不再适用,此时需要做完整的三波混频方程求解。

\subsubsection{输入光为泵浦光和种子光}

本节考虑一个有些不同的情况:输入光还是有两个频率分量,但是其中一个频率分量是很弱的,从而SFG过程产生的光的幅度振荡的同时,那个比较弱的光的幅度也会振荡。
用1表示泵浦光,2表示那个频率分量较弱的输入光(即所谓种子光),3表示SFG过程产生的光。
此时的非线性麦克斯韦方程为
\[
    \begin{aligned}
        \left( \pdv[2]{z} - \frac{\epsilon_\text{r}}{c^2} \pdv[2]{t} \right) E_3 &= \frac{1}{\epsilon_0 c^2} \pdv[2]{t} \left( 2 \epsilon_0  \chi^{(2)} A_1 A_2 \ee^{\ii (k_1 + k_2) z} \ee^{-\ii \omega_3 t} \right) + \text{h.c.}, \\
        \left( \pdv[2]{z} - \frac{\epsilon_\text{r}}{c^2} \pdv[2]{t} \right) E_2 &= \frac{1}{\epsilon_0 c^2} \pdv[2]{t} \left( 2 \epsilon_0  \chi^{(2)} A_1^* A_3 \ee^{\ii (- k_1 + k_3) z} \ee^{-\ii \omega_2 t} \right) + \text{h.c.}.
    \end{aligned}
\]
使用和上一节类似的慢变振幅近似,我们有
\begin{equation}
    \begin{aligned}
        \pdv{A_3}{z} &= \frac{\ii \omega_3^2}{k_3 c^2} \chi^{(2)} A_1 A_2 \ee^{\ii \Delta k z}, \\
        \pdv{A_2}{z} &= \frac{\ii \omega_2^2}{k_2 c^2} \chi^{(2)} A_1^* A_3 \ee^{- \ii \Delta k z}.
    \end{aligned}
    \label{eq:pump-and-seed-sfg}
\end{equation}
我们需要尝试将$A_2$和$A_3$的方程解耦。在相位匹配条件成立的情况下,只需要让以上两个方程两边各自对$z$再求一次导数,即可得到分别关于$A_2$和$A_3$的两个二阶微分方程
\[
    \pdv[2]{A_3}{z} = - \frac{\omega_3^2 \omega_2^2}{k_2 k_3 c^4} \abs*{A_1}^2 (\chi^{(2)})^2 A_3, \quad \pdv[2]{A_2}{z} = - \frac{\omega_3^2 \omega_2^2}{k_2 k_3 c^4} \abs*{A_1}^2 (\chi^{(2)})^2 A_2.
\]
在相位匹配条件不成立的时候,对\eqref{eq:pump-and-seed-sfg}中每个方程求导时等式右边会多出来一个$A_2$或$A_3$项,让求解变得困难。
然而,根据之前“相位不匹配会导致新产生的光和已有的光在一些地方发生相消干涉”的物理图像,我们仍然可以确信,相位不匹配会导致能量转化效率下降。

在相位匹配条件成立时,代入边界条件$A_3(0) = 0$,解得
\begin{equation}
    A_2(z) = A_2(0) \cos(k z), \quad A_3(z) = \ii \sqrt{\frac{\omega_3 n_2}{\omega_2 n_3}} \frac{A_1}{\abs*{A_1}} A_2(0) \sin(k z),
\end{equation}
其中
\begin{equation}
    k^2 = \frac{\omega_3^2 \omega_2^2}{k_2 k_3 c^4} \abs*{A_1}^2 (\chi^{(2)})^2. 
\end{equation}
我们发现一开始,能量从$\omega_2$模式转向$\omega_3$模式,但是一段时间之后能量又从$\omega_3$模式转向$\omega_2$模式。
两个耦合的模式如此来回振荡。存在能量从$E_3$模式流向$E_2$模式的过程意味着在完全经典的计算中,SFG过程的逆过程也是可行的,虽然由于没有量子涨落,必须有种子光这个过程才能发生。
在本节计算的例子中,泵浦光$E_1$起到了$\omega_3$转化为$\omega_1$和$\omega_2$的种子光的作用。

\subsubsection{三束光均很强的情况}



\subsection{DFG过程的一阶微扰}

我们现在转而考虑DFG过程。前面提到过,经典理论中DFG过程不能从一束光凭空产生两束光。
于是我们考虑一个这样的过程:$E_3$是基本上不衰减的泵浦光,它将要产生$E_1$和$E_2$两束光;$E_1$光也是输入光,即$E_1$模式上有种子光;$E_2$在$z=0$处为零。
使用慢变振幅近似,有以下方程:
\begin{equation}
    \begin{aligned}
        \pdv{A_1}{z} &= \frac{\ii \chi^{(2)}(\omega_1 = \omega_3 - \omega_2) \omega_1^2}{k_1 c^2} A_3 A_2^* \ee^{-\ii \Delta k z}, \\
        \pdv{A_2}{z} &= \frac{\ii \chi^{(2)}(\omega_2 = \omega_3 - \omega_1) \omega_2^2}{k_2 c^2} A_3 A_1^* \ee^{-\ii \Delta k z}.
    \end{aligned}
    \label{eq:dfg-amplitude-eq}
\end{equation}
显然我们可以指定
\begin{equation}
    \chi^{(2)}(\omega_1 = \omega_3 - \omega_2) = \chi^{(2)}(\omega_2 = \omega_3 - \omega_1) = \chi^{(2)}.
\end{equation}
求解以上方程,得到
\begin{equation}
    A_1(z) = A_1(0) \cosh(\kappa z), \quad A_2(z) = \ii \sqrt{\frac{n_1 \omega_2}{n_2 \omega_1}} \frac{A_3}{\abs*{A_3}} A_1^*(0) \sinh(\kappa z),
    \label{eq:dfg-amplitude}
\end{equation}
其中
\begin{equation}
    \kappa^2 = \frac{\omega_1^2 \omega_2^2}{k_1 k_2 c^4} \abs*{A_3}^2 (\chi^{(2)})^2 .
\end{equation}
这里,$\omega_1$和$\omega_2$光都指数增长。这里的关键点在于关于$A_1$的方程右边的$A_2$取了复共轭。
物理图像上,单泵浦光的DFG过程中,$\omega_1$光子的出现能够刺激$\omega_2$光子的出现,反之亦然(请注意虽然在经典情况下需要有种子光$\omega_1$,但是生成$\omega_2$时\emph{不会}湮灭$\omega_1$光子——这可以从\eqref{eq:dfg-amplitude-eq}中看出,关于$A_2$的方程右边是表示产生的$A_1^*$而不是表示湮灭的$A_1$;$\omega_1$光只需要在那里即可,并不会被消耗),因此一旦有$\omega_1$光和泵浦光同时出现,$\omega_1$光和$\omega_2$光就会不断扩增,从而指数增长。
反之,在单泵浦光输入的SFG过程中,产生$\omega_3$光子会消耗$\omega_2$光子,但是$\omega_3$光子不能诱发更多$\omega_1$光子转化为$\omega_2$光子,从而$\omega_2$光子和$\omega_3$光子存在竞争关系。

从以上解可以看到一个有趣的现象,就是种子光$\omega_1$的相位其实是自行决定的,而$\omega_2$光的相位同时由泵浦光$\omega_3$和和种子光$\omega_1$的相位决定。
这意味着非线性晶体不仅可以用来从一个频率的光源产生另一个频率的光源,还可以用于产生相位特定的新光源。

\subsection{相位匹配条件的实现}

如前所述,只有相位匹配时SFG过程或是DFG过程才足够明显。本节讨论给定三个任意频率的光,如何让它们能够满足相位匹配条件。
由于介质色散的存在,相位匹配条件不总是能够完成的,因为联立方程
\[
    \omega_3 = \omega_1 + \omega_2, \quad \omega_3 n(\omega_3) = \omega_2 n(\omega_2) + \omega_1 n(\omega_1)
\]
未必有解。实际上,在所谓的正常折射率的情况下——即在$n$随着$\omega$增大而增大的情况下——这个方程就是无解的,因为显然
\[
    \omega_3 > \omega_1, \quad \omega_3 > \omega_2,
\]
从而
\[
    \omega_3 n(\omega_3) > \omega_1 n(\omega_3) + \omega_2 n(\omega_3) > \omega_1 n(\omega_1) + \omega_2 n(\omega_2).
\]

\subsubsection{双折射}

一种获得相位匹配的方式是使用双折射晶体或者说单轴晶体。设我们有一块同时展现非线性光学效应和双折射的晶体——这样的晶体并不难找,因为大部分晶体的折射率都具有各向异性。
大体上,我们有两种方法可以得到较强的二阶非线性过程:一种是所谓的Type I相位匹配,其中$\omega_1$光和$\omega_2$光的偏振方向一致。对负单光轴晶体我们要设法让
\begin{equation}
    n^\text{e}_3 \omega_3 = n^\text{o}_1 \omega_1 + n^\text{o}_2 \omega_2
\end{equation}
成立,而对正光轴晶体我们则设法让
\begin{equation}
    n^\text{o}_3 \omega_3 = n^\text{e}_1 \omega_1 + n^\text{e}_2 \omega_2
\end{equation}
成立。
还有一种是所谓的Type II相位匹配,其中$\omega_1$光和$\omega_2$光的偏振方向不同。对负光轴晶体我们要设法让
\begin{equation}
    n^\text{e}_3 \omega_3 = n^\text{o}_1 \omega_1 + n^\text{e}_2 \omega_2
\end{equation}
成立,对正光轴晶体我们要设法让
\begin{equation}
    n^\text{o}_3 \omega_3 = n^\text{o}_1 \omega_1 + n^\text{e}_2 \omega_2
\end{equation}
成立。
如果二阶非线性过程的产出在o光偏振方向或是e光偏振方向上有足够大的分量(既然晶体已经确定是各向异性的,输入o光大抵是能够产生e光的,反之亦然),我们就可以获得高效率的二阶非线性转化了。

让这些式子之一成立似乎还是很难,但应注意到这里所谓的“$n^\text{e}$”\emph{可以不是}\eqref{eq:one-axis-matrix}中定义的那个——以任何一个方向为波矢方向的模式都可以是o光模式也可以是e光模式,如果它是e光模式,它的等效折射率由\eqref{eq:e-light-effective-index}给出。
因此我们可以通过调整入射光波矢和晶体光轴方向之间的夹角来调整以上相位匹配条件中的$n^\text{e}$。
为了避免讨厌的反射,我们可以让入射光垂直入射,此时所谓“调整入射光波矢和晶体光轴方向之间的夹角”就是调整光轴和非线性晶体表面的夹角。
这就是所谓的\concept{角度调节}。

使用二阶非线性双折射晶体有一个好处:可以通过偏振滤除输出光中的一些我们不需要的成分。
例如,设我们要用两束泵浦光产生和频光子,并且使用Type I相位匹配,那就可以通过适当放置偏振片,滤掉未发生转化的泵浦光。
无需使用不同频率的滤光片——偏振片就够了。

角度调节的问题是,这样的光路中一般会存在走移角,这是宏观的一个量,从而可能会让我们希望能够重叠的光束无法重叠。
例如,如果使用Type II相位匹配做SFG,那么$\omega_1$光和$\omega_2$光很快会因为走移角而分开,从而无法发生非线性相互作用。
在Type I相位匹配中非线性相互作用可以持续发生,但是最终产生的光束在空间上会被展宽很多(从而单点强度就弱了)。
作为替代,我们可以选择一些光学性能和温度关系很大的非线性晶体,固定$\theta = \SI{90}{\degree}$,此时没有走移角,并使用温度来调节$n^\text{o}$和$n^\text{e}$的相对关系,即所谓\concept{温度调节}。

温度调节的好处是,这种方案中晶体可以做得比较厚,而不必担心走移角。
温度调节也存在自己的不足:温度变化是很缓慢的,而此时晶体又通常比较厚,于是有可能晶体内外性变不一致,晶体内部会产生应力,甚至可能爆裂。

\subsubsection{准相位匹配}

在$\Delta k$非零但不大——即所谓\concept{准相位匹配}——时,可以通过这样的方法获得高效率的SFG转换:将一系列$\chi^{(2)}$指向周期性倒转的二阶非线性晶体贴在一起,让第一块晶体的厚度是$\pi L_\text{coh} / 2$,后面所有的晶体的厚度都是$\pi L_\text{coh}$。
这样,根据\eqref{eq:sfg-intensity},走过第一块晶体时,$I_3$强度打到最大值,此时能和$E_3$光发生相长干涉的光的相位是$\pi$;随后进入第二块晶体,$\chi^{(2)}$倒转,根据\eqref{eq:two-pump-a3},这意味着新产生的$E_3$光获得一个$\pi$的相位,于是第二块晶体内仍然发生了相长干涉;在第二块晶体和第三块晶体的交界处,能和$E_3$发生相长干涉的光的相位是$0$,$\chi^{(2)}$再次倒转,根据\eqref{eq:two-pump-a3},在第三块晶体中产生的$E_3$光的相位是$0$,于是还是相长干涉……
如此重复即可获得持续增加的$\omega_3$光,虽然其增速不如相位完全匹配时的情况。

准相位匹配已经是成熟的技术。PPLN装置等铁电体阵列是实现准相位匹配的常用装置。

准相位匹配实际上说明了一点,就是晶体的性质有空间起伏时,之前的诸如相位匹配条件的东西都是不能直接适用的。
这就提示我们,晶体内部的元激发——声子或者别的什么——可以借此和光耦合。
我们将在\autoref{sec:general-four-wave}中讨论一些这样的过程。

\subsubsection{反常折射}

还有一种最容易想到的办法:为了实现相位匹配,我们可以去寻找一种特殊的材料,它在$\omega_3$附近有反常折射率,从而让相位匹配条件能够成立。
如果我们可以调控非线性材料中的每个原子,做到这点其实是相对容易的,但是对任意的频率找对应的现成晶体还是比较难的。

使用这种方法需要小心避开强吸收的频段(谐振子模型告诉我们,通常反常折射率就在这样的频段附近)。

\subsection{高阶过程}

SFG过程和DFG过程可以连续地发生,从而产生含有多个顶角的图,也产生大量频率不同的出射光,它们相对于一阶SFG和DFG过程当然很弱,但是是可以观测到的。
在测量样品的光学响应时常有疏忽大意的实验者用一块二阶非线性晶体产生一束特定频率的光,结果发现用这束光通入样品之后出现了意料之外的一些频率的出射光,而实际上这些“有趣的新现象”只不过是二阶非线性晶体中的高阶过程产生的光而已。

高阶过程产生的光并非全然无用。一些时候我们需要极高能量的光子(如ARPES需要紫外光),此时可以使用二阶非线性晶体的二阶谐波产生,三阶谐波产生甚至四阶谐波产生。

\subsection{二阶非线性效应的各种应用}

\subsubsection{应力测量}

应力导致空间反演对称性破缺,从而可以产生二阶非线性光学过程。

\subsubsection{波长调节}

设我们手头上只有一个频率的光——比如说\SI{800}{nm}的光——而需要得到一束频率低一些的光——比如说\SI{1500}{nm}的光。没有别的光源可用。
在光纤通信中经常会遇到这样的任务,如需要做这样的转换来避开容易受到干扰或是吸收的频段。
这时候可以这么做:首先将\SI{800}{nm}光尽可能聚焦到一个材料中,让supercontinuum generation发生,从而得到一个非常宽的频谱(并且很弱),然后将纯净的\SI{800}{nm}光和这个宽谱光入射到一个二阶非线性晶体当中,这样这两者就分别起到了泵浦光和种子光的作用。
当然,DFG过程产生的光的频率也是宽谱的,但是请注意只有满足相位匹配条件的过程才是最可能发生的。
因此我们可以通过选用适当的非线性晶体,调整温度、入射角度等,让\SI{800}{nm}光变成\SI{1500}{nm}光的过程正好符合相位匹配条件,于是就得到了相当纯净的\SI{1500}{nm}输出光。

\subsubsection{光学参数共振器(OPO)}\label{sec:opo}

\concept{光学参数共振器(OPO)}是一个内部放置了一块二阶非线性晶体的光学共振腔。
一束强泵浦光$\vb*{E}_3$始终穿过这块非线性晶体,其强度衰减不明显。
如果在一定条件下只有DFG过程$\omega_3 = \omega_1 + \omega_2$发生,由于泵浦光恒定不变,$\omega_1$光和$\omega_2$光之间实际上只有线性耦合。
如果从两个不同的方向向非线性晶体输入光,则方向1的$\omega_2$光不会和方向2的$\omega_2$光耦合,$\omega_3$光同理。
方向1的$\omega_2$光和方向2的$\omega_3$光也不会有相互作用,因为此时相位匹配条件无法满足。
因此,只有方向1上的$\omega_2$光和$\omega_3$光会有耦合,方向2上的$\omega_2$光和$\omega_3$光会有耦合,并且都是线性耦合。
这种线性耦合服从\eqref{eq:dfg-amplitude},即这两束光会指数增长,能够定义一个增益$g$。

现在假定泵浦光已经输入了腔体。腔体本身有确定的、非常密集的离散谱,其量子涨落会让这些模式上的光子随机地产生,这就是DFG过程需要的种子光。
相位匹配条件决定了只有频率在特定的$\omega_1$和$\omega_2$频率附近的光能够放大。
设一个频率大体上是$\omega_1$的光子产生,它来回穿过非线性晶体,从而被一次次指数放大。
这个过程当然不会永远持续下去,稳定时,在腔体内来回走一趟的增益和损耗(包括吸收和溢出腔体)彼此抵消。
能够建立这样一个平衡,那么泵浦光就能够持续、高效率地转化为频率大体上是$\omega_1$和$\omega_2$、是腔体的某个本征频率的光。
否则,这样的光传播几次就衰减到零,不能形成稳定的输出。

形成稳定模式的条件的一个粗糙估计是
\begin{equation}
    \ee^{2 g L} = 1 - R^2,
\end{equation}
等式左边给出光在晶体中走两次得到的增益(如前所述,前一次穿过晶体的光和反射回来的光之间不存在耦合,因为相位匹配条件不满足),等式右边给出光的反射损失。
由于增益很小而$R$很接近1,在上式两边做泰勒展开,我们有
\begin{equation}
    g L = 1 - R.
\end{equation}


\subsubsection{荧光信号的时间分辨}

荧光持续的时间很长,但是很弱,我们可以在需要仔细分析的时间段制备一个脉冲,将荧光和这个脉冲同时输入一个非线性晶体,就可以把我们需要仔细观察的那一段分离出来。

\subsubsection{光谱学}

\section{三阶非线性极化的波动光学}\label{sec:chi-3-wave}

受激拉曼效应,受激光栅(让光自己产生干涉,然后非线性效应让折射率发生周期性变化,再来一束光,就发生了衍射),CARS

将一个波包压缩成一个阿秒级别的脉冲

三阶非线性极化允许这样的过程发生:$\omega = \omega - \omega + \omega$,即一束光可以同时提供三个光子,产生同一频率的一个光子。
也即,三阶非线性极化允许单频光自相互作用,或者说允许只涉及一个频率的简并混频。在二阶非线性极化中没有这种现象。
仅仅这个过程就足够导致很多新奇的现象出现。

\subsection{Kerr效应和相关的自相互作用}

\subsubsection{Kerr效应和自聚焦}

波列很长的波进入一个各向同性的、中心反演对称的非线性晶体,我们会发现它的折射率大体上是
\begin{equation}
    n = n_0 + \bar{n}_2 \expval*{\vb*{E}(t)^2} = n_0 + 2 \bar{n}_2(\omega) \abs*{\vb*{E}(\omega)}^2,
    \label{eq:ref-index-change}
\end{equation}
其中$n_0$是弱光的折射率而$n_2$是一个二阶折射系数。这种折射率因为入射光而发生变化的效应称为\concept{Kerr效应}。

Kerr效应可能是三阶非线性极化的结果。如果我们只考虑三阶非线性极化,就有
\[
    \vb*{P}_\text{NL} = 3 \epsilon_0 \chi^{(3)}(\omega=\omega + \omega - \omega) : \vb*{E}(\omega) \vb*{E}(\omega) \vb*{E}(\omega)^*,
\]
因子$3$可以直接通过展开$\chi^{(3)} \vb*{E} \vb*{E} \vb*{E}$获得,也可以再一次通过费曼图对称性因子获得:设拉氏量中产生了$\chi^{(3)} \vb*{E} \vb*{E} \vb*{E}$的那一项的耦合常数为$\alpha$,则
\[
    \dv{\vb*{E}} (\alpha \vb*{E} \vb*{E} \vb*{E} \vb*{E}) = \chi^{(2)} \vb*{E} \vb*{E} \vb*{E},
\]
于是
\[
    \alpha = \frac{1}{4} \chi^{(2)},
\]
而由于$\alpha \vb*{E} \vb*{E} \vb*{E} \vb*{E}$项有四条外线,含有它的图需要乘以一个对称性因子$4!$。
在本节的自相互作用的情况下,由于有两条输入线的频率相同,需要除以一个因子$2$。(频率为$-\omega$的输入线和频率为$\omega$的输出线不等价,因为前者是外场线而后者不是)
于是最终的因子是$4! / (4 \cdot 2) = 3$。
在介质各向同性的情况下可以将$\chi^{(3)}$用一个标量代替,它后面的三个电场的乘积和单位张量缩并,于是
\[
    \begin{aligned}
        \vb*{P} &= \epsilon_0 \chi^{(1)}(\omega) \vb*{E}(\omega) +  3 \chi^{(3)}(\omega=\omega + \omega - \omega) : \vb*{E}(\omega) \vb*{E}(\omega) \vb*{E}(\omega)^* \\
        &= \epsilon_0 (\chi^{(1)}(\omega) + 3 \chi^{(3)}(\omega=\omega+\omega-\omega) \abs*{\vb*{E}(\omega)}^2) \vb*{E}(\omega).
    \end{aligned}
\]
折射率的定义为
\[
    n^2 = 1 + \chi,
\]
取小量近似,就得到
\begin{equation}
    n_0(\omega)^2 = 1 + \chi^{(1)}(\omega), \quad \bar{n}_2(\omega) = \frac{3}{4n_0} \chi^{(3)}(\omega=\omega-\omega+\omega).
\end{equation}

我们还可以将折射率写成光的强度
\begin{equation}
    I = \frac{\epsilon_0 c n_0}{2} \abs*{\vb*{E}(\omega)}^2
\end{equation}
的函数,即
\begin{equation}
    n = n_0 + n_2 I,
    \label{eq:reflective-index-indensity-dependence}
\end{equation}
其中
\begin{equation}
    n_2 = \frac{4}{\epsilon_0 c n_0} \bar{n}_2.
\end{equation}

折射率会随着入射光强而变化这件事意味着进入三阶非线性晶体的光会\concept{自聚焦}。
如果$\bar{n}_2 > 0$,那么光束中心的光被偏折得更厉害,光束相当于经过了一个凸面镜;反之光束相当于经过了一个凹面镜。
这个等效“透镜”的行为和频率相关,因此自聚焦可以用来设计一个频率筛选装置,即可以用于锁频。
例如,可以在一个谐振腔内部放置两个自聚焦晶体,则只有频率适当的光能够在谐振腔中稳定地来回传播,频率不适当的光经过多次成像,会散得越来越开。

\subsubsection{自相位调节}

光波经过介质之后(相比另一束没有经过介质的光)会有相位变化,而由于$\bar{n}_2$的存在,光波经过一个三阶非线性晶体之后会有一个额外的相位变化
\begin{equation}
    \phi_\text{NL}(t) = \frac{\omega}{c} L n_2 I(t).
    \label{eq:self-phase-adjustement}
\end{equation}
仪器把一束光大体上当成单频光而测它的频率(即所谓\concept{即时频率},对单色光它就是光的频率,对有多频率光它大概是波包的中心频率),就得到
\[
    \omega_\text{temp} = - \pdv{\phi}{t},
\]
于是从三阶非线性晶体出来的光的即时频率会因为非线性效应而变化
\begin{equation}
    \Delta \omega_\text{temp} = - \frac{n_2 \omega L}{c} \pdv{I}{t},
    \label{eq:temp-phase-change}
\end{equation}
即一束多频率的光经过时前面的会看起来更红,后面的会看起来更蓝。
把这束光做频谱分析当然会发现频谱变宽了。频谱变宽了,时域的波包尺度就会变窄——因此可以用三阶非线性晶体做一个超快激光。
例如可以用三阶非线性晶体做一系列薄片,波包每经过一个薄片,在时域的展宽就窄一些。
我们这里不用一个非常厚的晶体,这是为了避免光强增强过快,导致光丝出现而损坏材料。
如果使用多个薄片,光束被聚焦之后出射后又会散焦,从而可以比较容易地调控聚焦程度。
同样,表面上,虽然弱光的$I$并不大,但我们可以增大$L$去调节它的相位,但其实这是不现实的。
通过这种方式可以不需要什么苛刻的条件就产生超快脉冲。相比之下,使用固体激光器直接产生脉冲需要苛刻的温度条件。

设脉冲在时间上持续了$\tau$,则频谱宽度为
\[
    \Delta \omega \sim \frac{2\pi}{\tau},
\]
而即时频率的变化为$\Delta \phi_\text{NL} / \tau$,因此为了让频率展宽足够明显,应当有
\begin{equation}
    \Delta \phi_\text{NL} \sim 2\pi.
\end{equation}

\subsubsection{光学孤子}

自相位调节其实提醒我们一点:可能可以使用一些特殊的非线性效应反过来补偿色散导致的不同频率的光的相位差,从而让介质中能够产生孤子,这是一个波包,它能够稳定地在介质中传播,而不发生波包展宽。
在光纤中这已经有了应用,一些时候可以制造一个展宽为微米级的孤子。

我们设有一个波包
\begin{equation}
    E(z, t) = A(z, t) \ee^{\ii (k_0 - \omega_0 t)},
    \label{eq:wave-package}
\end{equation}
其中$k_0$和$\omega_0$是中心波矢和频率,$A(z, t)$的时间和频率依赖给出波包的(随时间变化的)包络线。我们通常认为中心波矢和频率之间的关系是遵从线性折射率的,即
\begin{equation}
    k_0 = n_0(\omega) \frac{\omega_0}{c}.
\end{equation}
我们不区分线性和非线性效应,统一地求解非线性波动方程
\[
    \pdv[2]{E}{z} - \frac{1}{\epsilon_0 c^2} \pdv[2]{D}{t} = 0.
\]
做傅里叶变换
\[
    E(z, t) = \int \frac{\dd{\omega}}{2\pi} \ee^{- \ii \omega t} E(z, \omega), \quad D(z, t) = \int \frac{\dd{\omega}}{2\pi} \ee^{- \ii \omega t} D(z, \omega),
\]
并且
\[
    D(z, \omega) = \epsilon(\omega) E(z, \omega),
\]
其中$\epsilon(\omega)$是(带有非线性效应,依赖于$E$的)介电常数。
我们于是得到非线性版本的亥姆霍兹方程
\begin{equation}
    \pdv[2]{E(z, \omega)}{z} + \epsilon(\omega) \frac{\omega^2}{c^2} E(z, \omega) = 0.
    \label{eq:nonlinear-freq-domain-eq}
\end{equation}
我们能有幸得到形式这么好的方程当然归功于材料本身没有时间演化,否则在频域中$D$和$E$之间的关系就不是简单的“乘以一个系数”了。
我们根据\eqref{eq:wave-package}以及$k_0$和$\omega_0$之间的关系是线性色散关系,得到
\begin{equation}
    E(z, \omega) = A(z, \omega - \omega_0) \ee^{\ii k_0 z } + \text{c.c.}.
\end{equation}

我们考虑慢变振幅近似。需要注意的是,此时这个近似是可能会失效的,几百飞秒的波包仍然满足这个近似,再小一些可能就失效了。
我们进一步假定波包的频谱展宽相比于$\omega_0$是很小的。
这样,\eqref{eq:nonlinear-freq-domain-eq}就变成
\begin{equation}
    2 k_0 \pdv{A(z, \omega - \omega_0)}{z} + (k^2 - k_0^2) A(z, \omega - \omega_0) = 0.
    \label{eq:wave-package-amplitude}
\end{equation}
做泰勒展开
\[
    k = k_0 + \Delta k_\text{NL} + k_1 (\omega - \omega_0) + \frac{1}{2} k_2 (\omega - \omega_0)^2 + \cdots,
\]
其中$k_\text{NL}$为非线性效应导致的自相位调节。这样,\eqref{eq:wave-package-amplitude}就成为
\begin{equation}
    \pdv{A}{z} - \ii \Delta k_\text{NL} A - \ii k_1 (\omega - \omega_0) A - \frac{1}{2} \ii k_2 (\omega - \omega_0)^2 A = 0.
\end{equation}
做傅里叶反变换
\[
    A(z, t) = \int \frac{\dd{\omega}}{2\pi} A(z, \omega - \omega_0) \ee^{- \ii (\omega - \omega_0) t},
\]
得到
\begin{equation}
    \pdv{A}{z} + k_1 \pdv{A}{t} + \frac{1}{2} \ii k_2 \pdv[2]{A}{t} = \ii \Delta k_\text{NL} A.
    \label{eq:wave-package-evolve}
\end{equation}
这其中,$k_1$和$k_2$是线性色散的一阶和二阶泰勒展开系数,它们是
\begin{equation}
    k_1 = \left(\pdv{k}{\omega}\right)_{\omega = \omega_0} = \left( \frac{1}{v_\text{g}} \right)_{\omega = \omega_0},
\end{equation}
以及
\begin{equation}
    k_2 = \left( \pdv[2]{k}{\omega} \right)_{\omega = \omega_0} = - \left( \frac{1}{v_\text{g}} \dv{v_\text{g}}{\omega} \right)_{\omega = \omega_0}.
\end{equation}
自相位调节为(把\eqref{eq:self-phase-adjustement}右边除以$L$就得到)
\begin{equation}
    \Delta k_\text{NL} = n_2 I \frac{\omega_0}{c}.
\end{equation}

如果\eqref{eq:wave-package-evolve}中完全没有非线性光学效应,并且$k$和$\omega$之间的关系不是线性的,那么就会出现波包展宽,因为此时$k_2 \neq 0$,做代换
\[
    \tau = t - \frac{z}{v_\text{g}},
\]
得到% TODO:衰减
同理,自相位调节也会导致波包展宽。如果我们要求
\begin{equation}
    \frac{1}{2} k_2 \pdv[2]{A}{t} = \Delta k_\text{NL} A,
\end{equation}
那么就不会有任何波包展宽。一个例子是
\begin{equation}
    A(z, \tau) = A_0 \sech(\tau / \tau_0) \ee^{\ii k z},
\end{equation}
虽然以上求解过程似乎要求波包要具有特定的形状,这样才能形成孤子,但是实际上,一些形状不那么好的波包输入材料之后其实也能形成孤子,因为不符合波包形状要求的那些频率成分由于色散都各自跑远了,只留下一个孤子波包。

\subsubsection{自陷}

想象一束已经被聚焦过了,然后被输入一个三阶非线性晶体。例如,可以将三阶非线性晶体放在一个高斯光的光腰上。
自聚焦现象如期发生,让光变得更强,然后自聚焦进一步增强……如果几何光学总是适用,那么最终光束将终结到一个点上。
当然,在此之前衍射已经变得明显了。
这里发生的事情就好像“外压和量子涨落的竞争”(我们会看到“量子涨落”并不只是比喻),最终形成一个光束尺寸相对稳定的\concept{光丝}。这就是所谓的\concept{自陷}。

光丝并不是一个非常稳定的状态,因为如果介质中有什么东西散射了一下光丝,它的直径就会增大,光强变小,于是自聚焦的逆过程开始发生,最后光束又四散开去。
但如果光丝足够强,它可能已经将介质内部打出一个等离子体通道了,这个时候支配光丝所在区域的光学性质的不是三阶非线性极化,而是等离子体的光学,光丝也就这样一直传播下去了。

我们来估算什么时候自聚焦会发生。假定自聚焦区域内的折射率大体上是均一的(从而自聚焦区域内的$I$是常数,一旦出了自聚焦区域,就快速衰减为零),则自聚焦区域的边界上的临界角为
\[
    \cos \theta_0 = \frac{n_0}{n_0 + \var{n}}, \quad \var{n} = n_2 I,
\]
射向边界而入射角大于这个角的光将被反射回去,即不会溢出光丝。
另一方面,光丝会有衍射,即在偏离入射光的波矢的地方仍然有光传播,本质上这是因为长得像平面波的光束不可能具有有限直径——光丝内部类似于平面波,但是它有有限大小的直径,我们在偏离入射光的波矢的方向上计算总电场,是能够得到非零结果的。
我们也可以说这是位置和动量的不确定性:光丝的位置是比较确定的,从而“传播方向”是不完全确定的。
我们借用孔径衍射的公式(因为这可以算是一个孔径衍射),衍射角为
\[
    \theta_d = \frac{0.61 \lambda_0}{n_0 d}.
\]
如果很多衍射光的衍射角小于全反射临界角,那么衍射会破坏光丝,而反之光丝可以进一步聚焦。
因此,平衡时,$\theta_0 \sim \theta_d$。做近似
\[
    \cos\theta = 1 - \frac{\theta^2}{2},
\]
并且注意到$n_2 I$无论如何相比$n_0$都是非常小的,我们能够得到
\begin{equation}
    d \sim \frac{0.61 \lambda_0}{\sqrt{2 n_0 n_2 I}} .
\end{equation}
这给出了指定波长、线性折射率和自聚焦效应之后,形成稳定光丝的直径。
我们马上可以,这对应一个功率
\begin{equation}
    P_\text{cr} = \frac{\pi d^2}{4} I = \frac{\pi 0.61^2 \lambda_0^2}{8 n_0 n_2}.
\end{equation}
这是一个完全确定的功率值,不多也不少。如果入射光束的总功率大于$P_\text{cr}$,那么光丝实际上只使用了入射光束的一部分,而如果入射光束的总功率小于$P_\text{cr}$则无法形成光丝。

入射光仍然需要走过一段距离才能够形成光丝。

\subsection{频率相同的多束光}

\subsubsection{光学相位共轭}

四波混频还能够导致另一个神奇的现象:\concept{光学相位共轭},即在被一束入射光激励之后能够产生另一束波矢方向和入射光完全相反而频率不变的的出射光。
设想我们向一块三阶非线性晶体输入两束波矢完全相反的泵浦光,用1和2标记它们,然后再输入一束随便什么光$\vb*{E}_3$。这样我们就有
\begin{equation}
    \begin{aligned}
        \left( \laplacian - \frac{\epsilon_\text{r}}{c^2} \pdv[2]{t} \right) E_4 &= \frac{1}{\epsilon_0 c^2} \pdv[2]{t} (6 \epsilon_0 \chi^{(3)} A_1 A_2 A_3^* \ee^{\ii (\vb*{k}_1 + \vb*{k}_2 - \vb*{k}_3)} \cdot \vb*{r}) \ee^{- \ii \omega_4 t} + \text{h.c.} \\
        &= \frac{1}{\epsilon_0 c^2} \pdv[2]{t} (6 \epsilon_0 \chi^{(3)} A_1 A_2 A_3^* \ee^{- \ii \vb*{k}_3} \cdot \vb*{r}) \ee^{- \ii \omega_4 t} + \text{h.c.} ,
    \end{aligned}
    \label{eq:four-wave-omega-4}
\end{equation}
其中
\begin{equation}
    \omega_4 = \omega_1 + \omega_2 - \omega_3.
\end{equation}
因此我们已经得到了一个波矢刚好反过来的$\omega_4$光。为了让$\omega_4$光和$\omega_3$光光学共轭,我们需要让
\[
    \omega_4 = \omega_1 + \omega_2 - \omega_3 = \omega_3,
\]
于是
\begin{equation}
    \omega_1 = \omega_2 = \omega_3 = \omega_4 = \omega.
\end{equation}
此时我们实际上还是在处理单频光;但是这不再是自相互作用了,因为$\vb*{k}_1$和$\vb*{k}_3$可以不平行和不相反:我们只需要
\begin{equation}
    \vb*{k}_1 = - \vb*{k}_2, \quad \vb*{k}_3 = - \vb*{k}_4,
\end{equation}
而$\vb*{k}_1$和$\vb*{k}_3$之间可以没有任何关系。上式满足之后相位匹配条件自然成立,因此转化效率是非常高的。
这个系统在输入一束$\vb*{E}_3$光之后,会产生一束相位完全相反(虽然振幅未必一样的)的$\vb*{E}_4$光。
% TODO:三阶晶体的相位匹配
由于是单频光,\eqref{eq:four-wave-omega-4}中的因子$6$有时候也需要调整。

% TODO:解方程

干涉条纹的空间频率是波矢差;干涉条纹因为自相位调节导致折射率光栅

\subsubsection{光学双稳}

设我们在一个法布里-波洛腔中放置了一个三阶非线性晶体。本节考虑一个最简单的情况,即三阶非线性晶体完全充满整个谐振腔。
设外界入射光振幅为$A_1$,晶体内的折射光振幅为$A_2$,从谐振腔的另一侧透出去的光的振幅为$A_3$,从谐振腔的另一侧反射回来的光的振幅为$A_3'$,与入射光方向相反,从入射侧返回的光的振幅为$A_1'$。
暂时忽略介质吸收,则
\begin{equation}
    A_2' = r A_2 \ee^{2 \ii k L}, \quad A_2 = t A_1 + r A_2'
\end{equation}
其中$r$和$t$是谐振腔壁的反射和透射系数。从以上方程我们解出
\begin{equation}
    A_2 = \frac{\tau A_1}{1 - \rho^2 \ee^{2 \ii k L}} = \frac{\tau A_1}{1 - R \ee^{2 \ii \phi + 2 \ii k L}} = \frac{\tau A_1}{1 - R \ee^{\ii \delta}},
    \label{eq:a2-from-a1-cavity}
\end{equation}
其中$\phi$是$r$的辐角(需要考虑这一点,因为腔壁可能有金属镀层)。
需要注意$k$由于三阶非线性极化会导致折射率依赖于$I$。
这里不能直接用\eqref{eq:reflective-index-indensity-dependence},因为\eqref{eq:reflective-index-indensity-dependence}给出的是单束光的自相互作用,而在本节中谐振腔中有方向相反的两束光$A_2$和$A_2'$。
正比于$A_2 A_2^*$的那部分非线性极化和正比于$A_2' (A_2')^*$的那部分非线性极化都会修正折射率,但是正比于$A_2 A_2'$的那部分非线性极化给出的电场的频率是$3 \omega$,而正比于$A_2^* A_2'$的那部分非线性极化不满足相位匹配条件,因此实际上我们有
\begin{equation}
    \delta = 2 \phi + 2 n_0 \frac{\omega}{c} L + 2 n_2 I_2 \frac{\omega}{c} L + 2 n_2 I_{2}' \frac{\omega}{c} L = 2 \phi + 2 \frac{\omega}{c} L (n_0 + \underbrace{(1+R) I_2}_{\approx 2 I_2}),
    \label{eq:phase-cavity}
\end{equation}
而不是简单地将$\abs*{A_2}^2 + \abs*{A_2'}^2$代入\eqref{eq:reflective-index-indensity-dependence}。
将\eqref{eq:a2-from-a1-cavity}取模长,就得到
\begin{equation}
    \begin{aligned}
        I_2 &= \frac{T I_1}{(1 - R \ee^{\ii \delta}) (1 - R \ee^{- \ii \delta})} = \frac{T I_1}{(1 - R)^2 + 4 R \sin^2 \frac{\delta}{2}} \\
        &= \frac{T I_1}{T^2 + 4 R \sin^2 \frac{\delta}{2}} = \frac{I_1 / T}{1 + (4 R / T^2) \sin^2 \frac{\delta}{2}}.
    \end{aligned}
    \label{eq:relation-i1-i2-cavity}
\end{equation}
联立\eqref{eq:phase-cavity}和\eqref{eq:relation-i1-i2-cavity},即求解
\begin{equation}
    I_1 = T I_2 \left( 1 + \frac{4R}{T^2} \sin^2\left( \phi + \frac{\omega}{c} L (n_0 + (1+R) n_2 I_2) \right) \right),
    \label{eq:cavity-bistability}
\end{equation}
即可得到谐振腔中的光强分布情况。

显然,\eqref{eq:cavity-bistability}给出的$I_2$有可能不止一个。这意味着给定$I_1$实际上是\emph{不能}唯一确定$I_2$的:$I_2$和过去的$I_1$是有关系的。
在存在两个$I_2$时我们说腔体存在\concept{光学双稳},存在更多$I_2$时称为光学多稳。

光开关:见Boyd 7.3.3,大体上说就是光强可以调控折射率从而可以调控光的干涉,从而控制不同方向上的光强。

无损光强测量

\chapter{均匀介质中的非参数过程}

\section{拉曼散射和布里渊散射}\label{sec:general-four-wave}

\autoref{sec:chi-2-wave}和\autoref{sec:chi-3-wave}给出了纯粹的光子-光子等效相互作用。
介质中能被称为“波”的东西当然不只有光。本节讨论光和介质中的各种模式(或者说元激发)的耦合,即广义的、不局限于电磁波的多波混频。
我们将介质初末态一样的过程称为\concept{参数过程},如前面所说的三波混频和四波混频;本节要讨论的光和介质中模式的耦合则是\concept{非参数过程},因为介质初末态是不同的。
最为常见的两种广义多波混频是\concept{布里渊散射}和\concept{拉曼散射}。
前者指的是光和晶体中的通常能够使用某种动量$\vb*{k}$标记的长距离集体模式——通常是声学声子——之间的散射,后者指的是光和介质中可以比较局域的模式——包括单个分子内部化学键的伸缩模式和光学声子——之间的散射。
拉曼散射涉及的光子的频率变化显然要明显高于布里渊散射。

\autoref{sec:chi-2-wave}和\autoref{sec:chi-3-wave}给出的可以说是光子的弹性散射,而本节讨论的则是光子的非弹性散射,因为有能量被放进了介质中的元激发,或是介质中的元激发将能量转移给了光子。
我们将前者称为\concept{斯托克斯散射},而将后者称为\concept{反斯托克斯散射}。

\subsection{自发Raman效应的经典描述}

\subsubsection{介质性能的涨落}

自发Raman效应指的是光子在介质中吸收和放出某种元激发(以下我们为了叙述方便,统一称为“声子”,虽然这个激发完全可以是介质中分子的化学键的伸缩等无法定义波矢的激发),导致频率变化的现象。
本节用一些经典图景分析此过程。
设介质中出现了声子,这当然会导致介质性能出现振荡,于是将$\alpha$泰勒展开到一阶,得到
\begin{equation}
    \vb*{P} = \alpha \vb*{E}, \quad \alpha(t) = \alpha_0 + \pdv{\alpha}{Q} Q(t).
\end{equation}
由于我们在分析Raman散射,介质中的模式未必有良定义的波矢,因此以下我们只将时间切换到频域。
设介质中谐振子正在以$\omega_q$振荡,泵浦光频率为$\omega_l$,则
\begin{equation}
    \vb*{P}(t) = \alpha_0 \vb*{E}_0 \cos(\omega_l t) + \frac{1}{2} \pdv{\alpha}{Q} \vb*{E}_0 Q_0 (\cos(\omega_l + \omega_q) t + \cos(\omega_l - \omega_q)t).
\end{equation}
可以看到我们有三个过程:一个是普通的$\omega_l$光的传播,对应\concept{瑞利散射},一个是\concept{Stokes过程},即输出光频率为$\omega_l - \omega_q$,散射光子的能量低于入射光子,还有一个是\concept{反Stokes过程},即输出频率为$\omega_l + \omega_q$,散射光子的能量高于入射光子。
Stokes过程和反Stokes过程各自给出一个有两条光子线和一条介质中元激发线的相互作用顶角,构成广义的三波混频。

\subsubsection{能级}

以上经典图景无法给出一些需要“能级布居数”才能解释的东西。例如,基本上低能级上的电子数目要远大于高能级,因此Stokes过程总是比反Stokes过程容易发生。
实际上这个原理可以用来测定一个已知能谱的系统的温度,因为通过比较Stokes过程和反Stokes过程的发生几率来确定能级布居数,从而推算出温度。
例如,通过声子和光子的耦合,我们可以测量出固体晶格的温度。
分析Raman过程中的$\omega_q$也可以用于确定系统中的各个能级的相对能量差。

\subsection{受激Raman效应}

自发Raman效应在实验上价值不大。它首先过于弱了,其次由于Stokes过程占据主导而它会产生长波光子,这有可能会淹没在荧光信号中。
此外,由于自发Raman效应要和已有的、在热态中的声子散射,它产生的出射光必然是非相干的。
\concept{受激Raman效应}在实验中有着更多的用途。所谓受激Raman效应发生在入射光中存在Stokes光子和反Stokes光子的模式上的种子光的情况下。

我们使用量子力学计算Raman效应的发生几率。由于并没有对光场做量子化,只需要做含时微扰即可。
\[
    \dv{W}{\omega} = \frac{2\pi}{\hbar} g(\omega) \abs{\sum_n \left(
        \frac{\mel*{f}{e \vb*{r} \cdot \vb*{E}}{n} \mel*{n}{e \vb*{r} \cdot \vb*{E}}{g}}{\omega_l - }
    \right)}^2
\]
我们在这里遇到了之前遇到过的类似的问题:如果没有频率为$\omega_l + \omega_q$的种子光,似乎Stokes光不能产生。
换而言之,将光场当成经典场,将
当然,这是因为我们没有考虑电磁场的量子涨落。

\subsection{量子化光场的非参数过程}

我们现在考虑一个量子化的光场和一个束缚态单电子系统的相互作用。系统状态包括两部分:光子占据数和电子能级。
分别用希腊字母表示一个多光子态,用拉丁字母表示一个电子状态。

我们考虑如下过程:电子从初态$\ket*{i}$吸收一个频率为$\omega_1$的光子,放出一个频率为$\omega_2$的光子,到达末态$\ket*{f}$。
首先电子可以先放出$\omega_2$光子再吸收$\omega_1$光子,也可以先吸收$\omega_1$光子再放出$\omega_2$光子,并且吸收光子或者放出光子后会先到达一个虚能级。
我们积掉虚能级就能够得到一个关于$\ket*{i}, \ket*{f}$和光子的有效哈密顿量。
积掉虚能级给出的最低阶修正为
\begin{equation}
    \begin{aligned}
        \mel*{\alpha_f, f}{H_\text{eff}}{\alpha_i, i} = &\mel*{\alpha_f}{\mel{f}{e \vb*{r} \cdot \vb*{E}_2^\dagger}{n} \frac{1}{\omega_1 + \omega_i - \omega_n} \mel*{n}{e \vb*{r} \cdot \vb*{E}_1}{i}}{\alpha_i} \\
        + &\mel*{\alpha_f}{\mel{f}{e \vb*{r} \cdot \vb*{E}_1}{n} \frac{1}{- \omega_2 + \omega_i - \omega_n} \mel*{n}{e \vb*{r} \cdot \vb*{E}_2^\dagger}{i}}{\alpha_i},
    \end{aligned}
\end{equation}
其中$\vb*{E}_1$代表正比于$a_1$的那个电场傅里叶分量,$\vb*{E}_2^\dagger$代表正比于$a_2^\dagger$的那个电场傅里叶分量。注意此时电场是算符。
在体积为$V$的腔体中,有
\begin{equation}
    \vb*{E}_1 = \sqrt{\frac{\hbar \omega_1}{2 \epsilon_1 V}} \vu*{e}_1 \ee^{\ii \vb*{k}_1 \cdot \vb*{r} - \ii \omega_1 t} a_1, \quad \vb*{E}_2^\dagger = \sqrt{\frac{\hbar \omega_2}{2 \epsilon_2 V}} \vu*{e}_2^* \ee^{- \ii \vb*{k}_2 \cdot \vb*{r} + \ii \omega_2 t} a^\dagger_2,
\end{equation}
费米黄金法则为
\[
    \dv{W}{\omega} = 
\]
因此这个过程的跃迁率为
\begin{equation}
    \dv{W}{\omega} = \frac{8\pi^3 N \omega_1 \omega_2}{\epsilon_1 \epsilon_2 V} \abs*{\mel{f}{M}{i}}^2 g(\omega) \abs*{\mel*{\alpha_f}{a_2^\dagger a_1}{\alpha_i}}^2
\end{equation}

如果$\alpha_i$中没有$\omega_2$光子,那么
\begin{equation}
    \mel*{n_1 = m_1 - 1, n_2 = 1, \cdots}{a_2^\dagger a_1}{n_1 = m_1, n_2 = 0, \cdots} = \sqrt{m_{1}},
\end{equation}
从而$\omega_1$光子被吸收而产生$\omega_2$光子的概率正比于$\omega_1$光的强度。
换而言之,自发Raman效应的发生几率正比于泵浦光的强度。
另一方面,如果$\omega_2$光存在,则
\begin{equation}
    \mel*{n_1 = m_1 - 1, n_2 = m_2 + 1, \cdots}{a^\dagger_2 a_1}{n_1 = m_1, n_2 = m_2, \cdots} = \sqrt{m_1} \sqrt{m_2 + 1},
\end{equation}
因此在经典极限下,受激Raman效应发生的几率同时正比于泵浦光和种子光的强度。

Raman效应可以看成是泵浦光、介质中元激发和Stokes/反Stokes光三个模式之间的耦合,即可以看成一个等效三波混频过程。
由于介质中的元激发和Stokes(或是反Stokes)光都是泵浦光转化而来的,这个过程是\emph{自动相位匹配}的。这里的情况和\autoref{sec:opo}很像,虽然我们可以想象出一些相位匹配条件不满足的情况,但是最终的产出一定是相位匹配的。

% 好像SRS指的又不是这个
\subsection{(非相干的)受激Raman效应}

全量子理论描述了一个双光子过程,看起来似乎对电磁波是线性的。然而,它涉及介质中原激发的产生,而我们并不希望讨论光学问题时显式处理介质中的元激发。
因此从本节开始我们讨论它导致的等效光子-光子过程。

最为简单的等效光子-光子过程均要求一定的非幺正过程。
Stokes过程将产生一个声子,我们需要这个声子被弛豫掉;而反Stokes过程需要介质的热涨落提供一个声子。
之后所谓“受激拉曼散射”——或者简写为\concept{SRS过程}——指的就是这两个过程。
因此,即使泵浦光是高度相干的,通过SRS过程得到的Stokes光子之间也不相干,同样的,反Stokes光子之间也不相干,因为产生Stokes光子和反Stokes光子时均有环境扰动。

按照全量子的理论,受激Raman散射发生的几率同时正比于泵浦光和种子光的强度,而我们知道受激Raman散射产生的光的功率又正比于受激Raman散射发生的几率。
如果我们假定介质中元激发的寿命足够短,就能够写下这样一个(完全关于电磁波的)方程:
\begin{equation}
    \dv{I_2}{z} \propto I_1 I_2,
\end{equation}
或者如果用电场表示,$\omega_2$光会受到一个等效三阶非线性极化(涉及两个$\omega_1$光子和一个$\omega_2$光子)的补充。

SRS过程无需考虑相位匹配条件,因为我们对介质中的元激发没有任何要求。

在计算完单次光子-材料中模式的Raman散射之后,我们可以把有关计算结果用于确定宏观参数。我们有
\begin{equation}
    \abs{\vb*{E}_s}^2 = \ee^{G_R z - \alpha_s z} \abs{\vb*{E}_s(0)}^2,
\end{equation}
其中
\begin{equation}
    G_R = \gamma m_l \propto \frac{N}{V} \dv{\sigma}{\Omega} \frac{m_l}{\Gamma},
\end{equation}
其中
\[
    \frac{N}{V} \dv{\sigma}{\Omega} \sim \SI{10^{-8}}{cm^{-1}}, \quad \Gamma \sim \SI{1}{cm^{-1}}, \quad \gamma \sim \SI{10^{-3}}{cm/MW}.
\]

\subsection{CARS过程和等效四波混频}\label{sec:stimulated-raman-effective-four-wave}

现在我们转而考虑另一个情况,即系统中基本上没有非幺正过程,从而可以使用零温场论积掉介质中的元激发。
最简单的过程显然是将一个Stokes过程和一个反Stokes过程结合起来,让它们共用一个声子。
这就给出了一个等效四波混频过程,% TODO
这就是\concept{相干反Stokes拉曼散射(CARS)}。

\subsubsection{光谱展宽}

受激Raman效应可以用于展宽光谱:$\omega_1$光转化为频率为$\omega_2 = \omega_1 - \omega_{fi}$的Stokes光和频率为$\omega_2' = \omega_1 + \omega_{fi}$的反Stokes光,这个过程可以不断重复,而产生频率间隔为$\omega_{fi}$的梳子状的频谱。

\subsubsection{Raman激光}

既然受激Raman效应的发生几率正比于种子光,受激Raman效应中Stokes光似乎可以以$\ee$指数增加。这个机制称为\concept{Stokes激光},是一种产生激光的机制。

\subsection{受激Raman效应的光-物质波耦合描述}

下面我们讨论什么时候“积掉介质中的元激发”不可行。在\autoref{sec:stimulated-raman-effective-four-wave}中我们相当于近似认为介质中的元激发在被一个二光子过程产生之后立刻被另一个二光子过程消耗。
在介质中元激发产生不多,或者说光子频率和元激发频率相距很大时,这个近似是合理的。
直观地看,光跑得比介质中的元激发快得多,所以在一个二光子过程中产生的元激发如果不马上被另一个二光子过程消耗,那很快它和产生它的光脉冲就已经离得很远了;

然而,在光子频率和元激发频率非常接近时,泵浦光会持续不断地产生介质中的元激发,从而材料中会有大片的相干的元激发。因此,介质中的元激发(光学声子、自旋波等)、泵浦光和Stokes光能够持续耦合,“滚着往前跑”。
此时

\subsubsection{探测和灵敏度估计}

强度变强了,荧光变弱了,灵敏度提高了

使用拉曼散射还可以做高精度测量。其测定精度

拉曼散射


传播相位和天线相位不同导致相位差。

\subsection{应用}

\subsubsection{生物成像}

生物体中不同组分的吸收波长的重合不大。
自发拉曼散射足够用于做静态成像。如果要做动态成像,自发拉曼散射发生的速率就太慢了,必须要通过受激拉曼散射完成测定。

设介质中有声波,则介质性能出现空间起伏,有
\begin{equation}
    \chi(\vb*{r}) = \chi_0 \cos(\vb*{G} \cdot (\vb*{r} - \vb*{v}_{\vb*{G}} t)), \quad \omega_\text{S} = \vb*{G} \cdot \vb*{v}_{\vb*{G}}.
\end{equation}
毫无疑问这会导致一定的光学效应:我们想象一系列折射率不同的介质被贴在一起,一束光被打入其中,则每一层界面都会有微弱的反射,在条件适当的情况下这些反射光相长干涉,产生明显的总的反射。%
\footnote{
    实际上,这样可以获得非常好的反射镜,比镀银的好得多。通过这种方法可以获得5到6个9的反射率。
}%
$\chi$出现周期性起伏的介质就是这样一种系统。
另一个值得注意的地方是反射光会有小的频率变化,因为反射面在动,会有多普勒效应。

在有了以上直觉性的考虑之后我们开始解方程。做通常的拟设

\subsubsection{光纤通信的噪声及其消除方式}

受激布里渊散射会限制光线通信的激光功率。光纤存在热涨落,时不时就会自发出现声子,

\subsubsection{激光冷却}

由于斯托克斯过程和反斯托克斯过程完全是等价的,可以设想,既然能够将激光照在介质上向介质提供能量,当然也可以设法将介质的能量转移到激光中。
实际上,我们可以制备一个腔共振线正好落在反斯托克斯过程上的谐振腔,然后将一些分子放在腔中,并入射激光。
其结果是反斯托克斯过程受到激励,分子不仅没有从激光吸收能量,反而被冷却了。

\section{超快脉冲}

\subsection{离子化}

direct ionization, multiphoton ionization, tunnel ionization

\part{光的量子性质}

\chapter{量子化的光场}

\section{线性介质中的光场量子化}

电磁场足够强以至于难以看到单光子效应,而又足够弱以至于能量不至于强到需要考虑量子电动力学的圈图修正,这样就可以使用经典电动力学描述整个系统。
为了讨论电磁波的量子涨落(在分析诸如腔内辐射场,或是非线性光学中的DFG过程时非常重要,在做高精度测量时有时也要考虑),即使没有圈图效应,我们也要做光场的量子化。
这个做法的必要性将在后续的章节中多次体现出来,我们这里只是讨论量子化技术本身,暂时不考虑光的量子性在哪些情境下最为明显。

\subsection{真空}\label{sec:quantization-in-vacuum}

我们首先考虑真空中的光场的量子化,此时我们无非是在重复QED中的运算,实际上是在重复无质量矢量场的量子化(见\qftdoc中的\ref{qft-sec:massless-vector-quantize}节)。
QED中矢量场展开为
\begin{equation}
    A_\mu(\vb*{x}, t) = (\frac{\varphi}{c}, - \vb*{A}) = \int \frac{\dd[3]{\vb*{k}}}{(2\pi)^3} \sqrt{\frac{\hbar}{2\omega_{\vb*{k}} \epsilon_0}} \sum_{\sigma=1}^2 \left( a_{\vb*{k} \sigma} \epsilon_\mu^\sigma(\vb*{k}) \ee^{\ii \vb*{k} \cdot \vb*{x} - \ii \omega_{\vb*{k}} t} + a_{\vb*{k} \sigma}^\dagger \epsilon_\mu^\sigma(\vb*{k})^* \ee^{- \ii \vb*{k} \cdot \vb*{x} + \ii \omega_{\vb*{k}} t} \right),
    \label{eq:vector-field-components}
\end{equation}
其中电磁场模式为平面波。
取费曼规范,做一些分部积分并去掉表面项,得到
\begin{equation}
    \mathcal{L} = - \frac{1}{2 \mu_0} \partial_\mu A_\nu \partial^\mu A^\nu,
\end{equation}
从而正则动量为
\begin{equation}
    \pi^\mu = \pdv{\mathcal{L}}{\partial_0 A_\mu} = - \partial^0 A^\mu,
\end{equation}
可以据此写出正则量子化条件,即时间相同时,$A^\mu$同$A^\nu$对易,而
\begin{equation}
    [A^\mu(\vb*{x}, t), \pi^\mu(\vb*{y}, t)] = \ii \eta^{\mu \nu} \delta^{(3)}(\vb*{x} - \vb*{y}).
\end{equation}
哈密顿量为
\[
    \begin{aligned}
        H &= \int \dd[3]{\vb*{r}} (\pi^\mu \partial_0 A_\mu - \mathcal{L}) \\
        &= \int \dd[3]{\vb*{r}} \left( - \frac{1}{c^2} (\partial_t A^\mu)^2 + \frac{1}{2} \partial_\mu A_\nu \partial^\mu A^\nu \right),
    \end{aligned}
\]
这里要注意$x^0 = c t$。代入$A_\mu$的展开式计算得到
\begin{equation}
    H = \sum_{\sigma=1}^2 \int \frac{\dd[2]{\vb*{k}}}{(2\pi)^3} \hbar \omega_{\vb*{k}} \left( a^\dagger_{\vb*{k} \sigma} a_{\vb*{k} \sigma} + \frac{1}{2} \right), \quad \omega_{\vb*{k}} = c \abs*{\vb*{k}}.
\end{equation}
这就得到了量子化的能量。
在量子化过程中我们已经通过限制$\sigma$而施加了规范,不过这个规范并不是辐射规范,而是洛伦兹规范。横波条件通过$\epsilon$矢量和波矢垂直而保证。
不过,既然我们只关心电偶极辐射而有关的相互作用哈密顿量可以完全写成$\vb*{E}$,这也不重要。

从四维矢量计算电场,得到
\begin{equation}
    \vb*{E}(\vb*{r}, t) = \int \frac{\dd[3]{\vb*{k}}}{(2\pi)^3} \sqrt{\frac{\hbar}{2\omega_{\vb*{k}} \epsilon_0}} \sum_{\sigma=1}^2 \left( (- \ii \vb*{k} \epsilon_0^\sigma(\vb*{k}) + \ii \omega_{\vb*{k}} \vb*{\epsilon}^\sigma(\vb*{k})) a_{\vb*{k} \sigma} \ee^{\ii \vb*{k} \cdot \vb*{r} - \ii \omega_{\vb*{k}} t} + \text{h.c.} \right),
\end{equation}
以及
\begin{equation}
    \vb*{B}(\vb*{r}, t) = \int \frac{\dd[3]{\vb*{k}}}{(2\pi)^3} \sqrt{\frac{\hbar}{2\omega_{\vb*{k}} \epsilon_0}} \sum_{\sigma=1}^2 \left( \ii \vb*{k} \times \vb*{\epsilon}_\sigma a_{\vb*{k} \sigma} \ee^{\ii \vb*{k} \cdot \vb*{r} - \ii \omega_{\vb*{k}} t } + \text{h.c.} \right).
\end{equation}
其中
\begin{equation}
    \epsilon^\mu_\sigma = (\epsilon_\sigma^0, \vb*{\epsilon}_\sigma), \quad \frac{\omega_{\vb*{k}}}{c} \epsilon_0^\sigma - \vb*{k} \cdot \vb*{\epsilon}_\sigma = 0, \quad \abs*{\epsilon_\sigma^0}^2 - \abs*{\vb*{\epsilon}_\sigma}^2 = 1.
\end{equation}
通过以上公式,能够验证以下哈密顿量形式:
\begin{equation}
    H = \int \dd[3]{\vb*{r}} \left( \frac{\epsilon_0}{2} \vb*{E}^2 + \frac{1}{2\mu_0} \vb*{B}^2 \right).
    \label{eq:e-and-b-hamiltonian}
\end{equation}
这正是电动力学中常见的形式。因此实际上我们也可以直接用\eqref{eq:vector-field-components}写出$\vb*{E}$和$\vb*{B}$并代入\eqref{eq:e-and-b-hamiltonian}。

在本文讨论的光学问题中,我们可以使用一种对具体计算更加友好的形式,即采用\concept{辐射规范}。
在辐射规范之下,我们有
\begin{equation}
    \begin{aligned}
        \mathcal{L} &= - \frac{1}{4 \mu_0} (\partial_\mu A_\nu - \partial_\nu A_\mu) (\partial^\mu A^\nu - \partial^\nu A^\mu) \\
        &= \frac{1}{2 \mu_0} \frac{1}{c^2} (\partial_t \vb*{A})^2 - \frac{1}{4\mu_0} (\partial_i A_j - \partial_j A_i) (\partial^i A^j - \partial^j A^i) \\
        &= \frac{\epsilon_0}{2} ((\dot{\vb*{A}})^2 - c^2 (\curl{\vb*{A}})^2).
    \end{aligned}
\end{equation}
以这个拉氏量为出发点做正则量子化。做展开
\begin{equation}
    \vb*{A}(\vb*{r}, t) = \int \frac{\dd[3]{\vb*{k}}}{(2\pi)^3} \sqrt{\frac{\hbar}{2 \omega_{\vb*{k}} \epsilon_0}} \sum_{\sigma=1}^2 (a_{\vb*{k} \sigma} \vu*{e}^\sigma \ee^{\ii \vb*{k} \cdot \vb*{r} - \ii \omega_{\vb*{k}} t} + a^\dagger_{\vb*{k} \sigma} (\vu*{e}^\sigma)^* \ee^{- \ii \vb*{k} \cdot \vb*{r} + \ii \omega_{\vb*{k}} t}),
\end{equation}
从而电场和磁场分别为
\begin{equation}
    \vb*{E}(\vb*{r}, t) = \ii \int \frac{\dd[3]{\vb*{k}}}{(2\pi)^3} \sqrt{\frac{\hbar \omega_{\vb*{k}}}{2 \epsilon_0}} \sum_{\sigma=1}^2 (a_{\vb*{k} \sigma} \vu*{e}^\sigma \ee^{\ii \vb*{k} \cdot \vb*{r} - \ii \omega_{\vb*{k}} t} - a^\dagger_{\vb*{k} \sigma} (\vu*{e}^\sigma)^* \ee^{- \ii \vb*{k} \cdot \vb*{r} + \ii \omega_{\vb*{k}} t})
    \label{eq:vacuum-e-field}
\end{equation}
和
\begin{equation}
    \vb*{B}(\vb*{r}, t) = \ii \int \frac{\dd[3]{\vb*{k}}}{(2\pi)^3} \sqrt{\frac{\hbar}{2 \omega_{\vb*{k}} \epsilon_0}} \sum_{\sigma=1}^2 (a_{\vb*{k} \sigma} \vb*{k} \times \vu*{e}_\sigma \ee^{\ii \vb*{k} \cdot \vb*{r} - \ii \omega_{\vb*{k}} t} - a^\dagger_{\vb*{k} \sigma} \vb*{k} \times \vu*{e}_\sigma^* \ee^{- \ii \vb*{k} \cdot \vb*{r} + \ii \omega_{\vb*{k}} t}).
    \label{eq:vacuum-b-field}
\end{equation}
正则动量为
\begin{equation}
    \vb*{\pi} = \epsilon_0 \dot{\vb*{A}},
\end{equation}
施加正则对易关系,会得到正确的
\begin{equation}
    \comm*{a_{\vb*{k} \sigma}}{a_{\vb*{k}' \sigma'}} = (2\pi)^3 \delta(\vb*{k} - \vb*{k}') \delta_{\sigma \sigma'},
\end{equation}
而哈密顿量为
\begin{equation}
    \begin{aligned}
        H &= \int \dd[3]{\vb*{r}} \left(\vb*{\pi} \cdot \pdv{\vb*{A}}{t} - \mathcal{L} \right) = \int \dd[3]{\vb*{r}} \left( \frac{\epsilon_0}{2} \left(\pdv{\vb*{A}}{t}\right)^2 + \frac{\epsilon_0}{2} c^2 (\curl{\vb*{A}})^2 \right) \\
        &= \int \dd[3]{\vb*{r}} \left( \frac{\epsilon_0}{2} \vb*{E}^2 + \frac{1}{2\mu_0} \vb*{B}^2 \right) \\
        &= \sum_{\sigma=1}^2 \int \frac{\dd[3]{\vb*{k}}}{(2\pi)^3} \hbar \omega_{\vb*{k}} \left(a^\dagger_{\vb*{k} \sigma} a_{\vb*{k} \sigma} + \frac{1}{2} \right).
    \end{aligned}
\end{equation}
因此,辐射规范给出的结果和完整的QED计算是完全一致的。
在辐射规范中我们还可以证明一个在横场条件成立时也成立,并且在一般的QED中很难计算的公式:
\begin{equation}
    \comm*{E^i(\vb*{r}, t)}{B^j(\vb*{r}', t)} = - \frac{\ii \hbar}{\epsilon_0} \pdv{x^k} \delta(\vb*{r} - \vb*{r}')
\end{equation}
其中$i, j, k$是$x, y, z$的轮换排列;其它情况下对易子为零。
还能够发现电场和自己的对易子始终为零,磁场亦然。因此电场的三个分量可以同时确定地被测量,磁场亦然。
但是不能同时准确测出$\vb*{E}$和$\vb*{B}$。
由于$(\vb*{E}, \vb*{B})$,$\vb*{A}$和$a_{\vb*{k} \sigma}$之间的关系是线性的,$a$的产生湮灭算符对易关系、$\vb*{A}$和$\vb*{\pi}$的正则对易关系以及$\vb*{E}$和$\vb*{B}$的对易关系是彼此等价的。

\begin{back}{粒子数表象和升降算符}{ladder-operator-particle-number}
    如果
    \begin{equation}
        n = a^\dagger a,
    \end{equation}
    且
    \begin{equation}
        \comm*{a}{a^\dagger} = 1,
    \end{equation}
    那么
    \begin{equation}
        a = \sum_{n} \sqrt{n} \ket*{n-1} \bra{n}, \quad a^\dagger = \sum_{n} \sqrt{n} \ket*{n} \bra*{n-1},
    \end{equation}
    或者说
\end{back}

\subsection{长波光子和介质中的麦克斯韦方程}\label{sec:long-wavelength-photon-maxwell-general}

下面我们讨论和\eqref{eq:material-hamiltonian}匹配的对易关系,以及它对角化之后将给出什么样的能谱。
应当指出,此时真空中的那些对易关系——$\vb*{A}$和$\epsilon_0 \dot{\vb*{A}}$之间的对易关系,$\vb*{E}$和$\vb*{B}$之间的对易关系——可能不能够直接适用。
这是因为正则量子化中,积掉自由度会导致哈密顿量的本征态的意义发生变化,从而算符的意义会发生变化(例如重整化后产生一个介质中的光子实际上意味着产生了如下的态:单真空光子态,叠加上单真空光子加电子空穴对的态,叠加上别的一些东西)。在高能物理中这导致场强重整化,在本文讨论的量子光学中这意味着$\vb*{E}$的各个分量之间的正则对易关系会发生变化(当然,在适当的定义下,单光子产生湮灭算符之间的对易关系仍然不变)。
同样这也会让横波条件的形式发生变化——介质中横波条件是$\div{\vb*{\epsilon} \cdot \vb*{E}} = 0$。

我们需要直接从\eqref{eq:material-hamiltonian}计算正则动量。同样取辐射规范,以$\vb*{A}$为基本自由度,则\eqref{eq:material-hamiltonian}就是
\begin{equation}
    H = \int \dd[3]{\vb*{r}} \left( \frac{1}{2} \dot{\vb*{A}} \cdot \vb*{\epsilon} \cdot \dot{\vb*{A}} + \frac{1}{2} (\curl{\vb*{A}}) \cdot (\vb*{\mu}^{-1}) \cdot (\curl{\vb*{A}}) \right),
\end{equation}
于是正则动量为
\begin{equation}
    \vb*{\pi} = \vb*{\epsilon} \cdot \dot{\vb*{\vb*{A}}}.
\end{equation}
注意到此时的正则动量的形式和真空中是不同的。
我们现在需要展开$\vb*{A}$。此时空间平移不变性不能保持,我们不能使用动量来标记电场的振动模式,
我们将\eqref{eq:photon-in-material}右边的$\vb*{j}$取为零——我们此处在对线性介质做正则量子化,暂时不考虑电流——那就得到了一个广义本征值问题。
这就意味着,我们可以求解出一整套本征函数,它们由下式
\begin{equation}
    \curl{(\mu^{-1} \cdot \curl{\vb*{u}_n})} - \omega_n^2 \epsilon \cdot \vb*{u}_n = 0
\end{equation}
确定,其中$\omega_n$对应着能够在系统中稳定传播的电磁波模式的频率,且有正交归一关系
\begin{equation}
    \int_V \dd[3]{\vb*{r}} \vb*{u}^*_m \cdot \vb*{\epsilon} \cdot \vb*{u}_n = \delta_{mn}, 
\end{equation}
请注意由于$\vb*{A}$的厄米性,$\vb*{u}_m^*$一般是另一个$\vb*{u}_n$。
正交归一关系又意味着
\begin{equation}
    \omega_n^2 \delta_{mn} = \int \dd[3]{\vb*{r}} (\curl{\vb*{u}_m^*}) \cdot (\vb*{\mu}^{-1}) \cdot (\curl{\vb*{u}_n}) + \int \dd{\vb*{S}} \cdot (\vb*{u}_m^* \times ((\vb*{\mu}^{-1}) \cdot (\curl{\vb*{u}_n}))),
\end{equation}
在自由空间中等式右边第二项可以略去,在一个反射性能尚可的反射腔体(如果我们只讨论有限空间中的问题,那么基本上这个问题需要放在一个腔体中,否则无法忽视外界影响)中可以把第二项当成微扰。
本节仅仅给出最为简单的理论,暂时不考虑第二项。
用这组基底$\{\vb*{u}_n\}$做展开
\begin{equation}
    \vb*{A}(\vb*{r}, t) = \sum_n \ii \sqrt{\frac{\hbar}{2\omega_{n}}} \vb*{u}_n(\vb*{r}) a_n \ee^{- \ii \omega_n t} + \text{h.c.},
\end{equation}
得到
\begin{equation}
    - \vb*{E} = \dot{\vb*{A}} = \sum_n \sqrt{\frac{\hbar \omega_n}{2}} \vb*{u}_n(\vb*{r}) a_n \ee^{-\ii \omega_n t} + \text{h.c.},
\end{equation}
以及
\begin{equation}
    \vb*{B} = \curl{\vb*{A}} = \sum_n \ii \sqrt{\frac{\hbar}{2\omega_n}} \curl{\vb*{u}_n(\vb*{r})} a_n \ee^{-\ii \omega_n t} + \text{h.c.}.
\end{equation}
施加正则对易关系
\begin{equation}
    \comm*{A^i(\vb*{r}, t)}{\pi^j(\vb*{r}', t)} = \ii \hbar \delta(\vb*{r} - \vb*{r}') \delta^{ij},
\end{equation}
我们发现我们能够得到我们想要的产生湮灭算符对易关系
\begin{equation}
    \comm*{a_n}{a_m^\dagger} = \delta_{mn}.
\end{equation}
然后,可以计算出哈密顿量为
\begin{equation}
    H = \sum_n \hbar \omega_n \left( a^\dagger_n a_n + \frac{1}{2} \right).
\end{equation}
这个哈密顿量的形式和真空中完全一样,不同的地方在于$\omega_{\vb*{k}}$被$\omega_n$取代,色散关系可能变得非常不一样。

既然$\epsilon$和$\mu$的概念对长波光子在量子情况下仍然适用,反射、折射等概念对长波光子仍然有意义,且和经典情况非常类似。
特别的,光场可能被约束在一个四面都是反射镜的腔体中,此时的光场被所谓的\concept{cavity QED}或者简写为\concept{cQED}描述。

一个介质系统中的量子化光场的自由哈密顿量就是普通的谐振子哈密顿量。
使用本质上是经典的方程\eqref{eq:photon-in-material},得到一系列振动模式,其频率即为这个介质系统中的量子化光场中的模式的频率,振动模式的场强分布就是\eqref{eq:photon-in-material}给出的本征模式。

\subsection{归一化和单光子电场}

现在设想我们在一个有限大小的空间中做光场量子化,不过该空间中还是能够良定义波矢。
这样我们只需要在\autoref{sec:quantization-in-vacuum}中做代换
\[
    \int \frac{\dd[3]{\vb*{k}}}{(2\pi)^3} \longrightarrow \frac{1}{V} \sum_{\vb*{k}},
\]
于是电场和磁场分别为
\begin{equation}
    \vb*{E}(\vb*{r}, t) = \frac{\ii}{V} \sum_{\vb*{k}} \sqrt{\frac{\hbar \omega_{\vb*{k}}}{2 \epsilon_0}} \sum_{\sigma=1}^2 (a_{\vb*{k} \sigma} \vu*{e}^\sigma \ee^{\ii \vb*{k} \cdot \vb*{r} - \ii \omega_{\vb*{k}} t} - a^\dagger_{\vb*{k} \sigma} (\vu*{e}^\sigma)^* \ee^{- \ii \vb*{k} \cdot \vb*{r} + \ii \omega_{\vb*{k}} t})
    \label{eq:vacuum-e-field-cavity-origin}
\end{equation}
和
\begin{equation}
    \vb*{B}(\vb*{r}, t) = \frac{\ii}{V} \sum_{\vb*{k}} \sqrt{\frac{\hbar}{2 \omega_{\vb*{k}} \epsilon_0}} \sum_{\sigma=1}^2 (a_{\vb*{k} \sigma} \vb*{k} \times \vu*{e}_\sigma \ee^{\ii \vb*{k} \cdot \vb*{r} - \ii \omega_{\vb*{k}} t} - a^\dagger_{\vb*{k} \sigma} \vb*{k} \times \vu*{e}_\sigma^* \ee^{- \ii \vb*{k} \cdot \vb*{r} + \ii \omega_{\vb*{k}} t}).
    \label{eq:vacuum-b-field-cavity-origin}
\end{equation}
不过我们注意到,此时哈密顿量为
\[
    H = \frac{1}{V} \sum_{\vb*{k}, \sigma} \hbar \omega_{\vb*{k}} \left( a^\dagger_{\vb*{k} \sigma} a_{\vb*{k} \sigma} + \frac{1}{2} \right),
\]
多出来一个不太美观的因子$V$;此外对易关系也是
\[
    \comm*{a_{\vb*{k} \sigma}}{a_{\vb*{k}' \sigma'}^\dagger} = (2\pi)^3 \delta(\vb*{k} - \vb*{k}') \delta_{\sigma \sigma'} \longrightarrow \frac{1}{V} \delta_{\vb*{k} \vb*{k}'} \delta_{\sigma \sigma'}.
\]
为此我们如下重新定义产生湮灭算符:
\begin{equation}
    a_{\vb*{k} \sigma} \longrightarrow \sqrt{V} a_{\vb*{k} \sigma},
\end{equation}
于是电场和磁场分别为
\begin{equation}
    \vb*{E}(\vb*{r}, t) = \ii \sum_{\vb*{k}} \sqrt{\frac{\hbar \omega_{\vb*{k}}}{2 \epsilon_0 V}} \sum_{\sigma=1}^2 (a_{\vb*{k} \sigma} \vu*{e}^\sigma \ee^{\ii \vb*{k} \cdot \vb*{r} - \ii \omega_{\vb*{k}} t} - a^\dagger_{\vb*{k} \sigma} (\vu*{e}^\sigma)^* \ee^{- \ii \vb*{k} \cdot \vb*{r} + \ii \omega_{\vb*{k}} t})
    \label{eq:vacuum-e-field-cavity-1}
\end{equation}
和
\begin{equation}
    \vb*{B}(\vb*{r}, t) = \ii \sum_{\vb*{k}} \sqrt{\frac{\hbar}{2 \omega_{\vb*{k}} \epsilon_0 V}} \sum_{\sigma=1}^2 (a_{\vb*{k} \sigma} \vb*{k} \times \vu*{e}_\sigma \ee^{\ii \vb*{k} \cdot \vb*{r} - \ii \omega_{\vb*{k}} t} - a^\dagger_{\vb*{k} \sigma} \vb*{k} \times \vu*{e}_\sigma^* \ee^{- \ii \vb*{k} \cdot \vb*{r} + \ii \omega_{\vb*{k}} t}),
    \label{eq:vacuum-b-field-cavity-1}
\end{equation}
哈密顿量为
\begin{equation}
    H = \sum_{\vb*{k}, \sigma} \hbar \omega_{\vb*{k}} \left( a^\dagger_{\vb*{k} \sigma} a_{\vb*{k} \sigma} + \frac{1}{2} \right),
\end{equation}
对易关系为
\begin{equation}
    \comm*{a_{\vb*{k} \sigma}}{a_{\vb*{k}' \sigma'}^\dagger} = \delta_{\vb*{k} \vb*{k}'} \delta_{\sigma \sigma'}.
\end{equation}
我们定义
\begin{equation}
    \mathcal{E}_{\vb*{k} \sigma} = \sqrt{\frac{\hbar \omega_{\vb*{k}}}{2 \epsilon_0 V}},
\end{equation}
则
\begin{equation}
    \vb*{E} = \sum_{\vb*{k}, \sigma} \mathcal{E}_{\vb*{k} \sigma} \vb*{f}_{\vb*{k} \sigma} a_{\vb*{k} \sigma} \ee^{- \ii \omega_{\vb*{k}} t} + \text{h.c.},
\end{equation}
其中
\begin{equation}
    \vb*{f}_{\vb*{k} \sigma} = \ii \ee^{\ii \vb*{k} \cdot \vb*{r}} \vu*{e}^\sigma,
\end{equation}
满足
\begin{equation}
    \frac{1}{V} \int \dd[3]{\vb*{r}} \vb*{f}^*_{\vb*{k} \sigma} \cdot \vb*{f}_{\vb*{k}' \sigma'} = \delta_{\vb*{k} \vb*{k}'} \delta_{\sigma \sigma'}.
\end{equation}
我们称$\mathcal{E}_{\vb*{k} \sigma}$为\concept{单光子电场},因为它给出了增多一个光子,电场大体上增多的幅度,而$\vb*{f}$则称为\concept{模式函数},它给出了光场的稳定振动方式,且不显含体积(体积出现在了积分前的归一化常数中)。

对一个一般的体系,我们会定义
\begin{equation}
    \mathcal{E}_{n} = \sqrt{\frac{\hbar \omega_n}{2 \epsilon_0 V}},
\end{equation}
从而电场为
\begin{equation}
    \vb*{E} = \sum_n \mathcal{E}_n \vb*{f}_{n} a_n \ee^{- \ii \omega_n t} + \text{h.c.},
    \label{eq:general-optical-field}
\end{equation}
其中
\begin{equation}
    \vb*{f}_n = \sqrt{\epsilon_0 V} \vb*{u}_n,
\end{equation}
且
\begin{equation}
    \frac{1}{V} \int \dd[3]{\vb*{r}} \vb*{f}_m^* \cdot \frac{\vb*{\epsilon}}{\epsilon_0} \cdot \vb*{f}_n = \delta_{mn}.
\end{equation}

实际上从这里我们可以看出,经典的麦克斯韦方程本身已经是一个具有一定量子特性的理论了——“单光子波函数”(虽然没有良定义的单光子量子力学,但是我们不妨这么指代$\mel*{0}{A^i(\vb*{r})}{\psi}$)服从的方程就是麦克斯韦方程。
也可以从另一个角度看这件事:在麦克斯韦方程两边乘上$\hbar$,由于$E \sim \hbar \omega$,得到的理论看上去就是一个量子理论。
将光场量子化引入的新物理只有两件事:光束由分立的光子构成,以及存在光子数的量子涨落,但是,在没有非线性光学效应的情况下,光子数目守恒,第一件事完全可以通过手动引入“光子”的概念并指派其波函数为(经过适当归一化的)经典电磁场来做到。
光的量子性只有在下面的地方才会变得重要:
\begin{itemize}
    \item 光子生灭明显,一些光子模式上原本没有电子而一段时间后有光子产生时,即处理非线性光学时,因为此时会有一些原本完全没有光子分布的模式上出现了光子。经典处理只能在有种子光的时候处理光子的产生——并不奇怪,因为光子从零到一产生的过程涉及一个极为弱的场强,弱到经典场论不再使用。
    \item 光子的“颗粒感”较为明显时。如即使多光子Fock态$\ket*{n}$中的$n$很大,由于其光子数目完全确定,我们有$\expval*{\vb*{E}}{n} = 0$而$\expval*{\vb*{E}^2}{n} \neq 0$,因为经过$\vb*{E}$中的产生湮灭算符作用后,光子数目发生变动,从而和原来的态完全正交。
    \item 纠缠重要时。经典电动力学面对“光子增多了”的描述方法是更大的场强,而没有直积的希尔伯特空间这样的概念,从而无法捕捉到纠缠。
\end{itemize}
可以看到这些光的量子性变得明显的情况都涉及多光子Fock态。
我们其实可以在这里看到一个相当有趣的情况:当所研究的问题中涉及非常弱的光场(如特定频率的光子一开始没有,但是一段时间后被产生)时,经典电动力学就失效了,然而经典电动力学的形式却又很像是在处理“单光子波函数”。
当然,这两者并没有矛盾,本质上是因为经典电动力学无法正确处理“多光子形成的多体波函数”:“单光子波函数”不涉及多体波函数,它给出的所有物理就是一个麦克斯韦方程,正好和经典电动力学一致;
相干态下电场的标准差相比于电场期望值本身比较小,从而电场能够被看成经典的量,同样可以被经典电动力学处理。
经典电动力学无法正确处理“多光子形成的多体波函数”,因为没有Fock空间;但是这并不是说经典电动力学就缺乏(相比于经典质点动力学的)一切量子性,如坐标和动量的不确定性等。
实际上,对缺乏纠缠、缺乏粒子生灭和碰撞、粒子数大的有质量粒子系统,单粒子波函数乘以适当的因子也可以诠释为“粒子数的平方根”。在这个意义上它和电磁场的地位是类似的。
当然,有质量粒子系统中有大量的碰撞,其宏观理论通常是动理学方程,且“经典费米场”不是一个物理意义特别明确的东西,因此我们很少看到“电子的宏观场”。
然而在电动力学以外的系统中,由于系统非常接近相干态,量子场能够被看成经典场的例子也是有的,如超流和超导中的序参量。

从本节的计算中也可以看到为什么很多时候经典电动力学已经够用了,因为一般的偶极辐射产生的就是相干光。要产生明显偏离相干光的辐射场实际上是很不容易的。

在本文中“电磁场”可能代表量子化的场算符,也可能代表经典电磁场,也可能代表“单光子波函数”。在不考虑非线性效应时这三者的时间演化是相同的。
在$\vb*{E}$被认为是经典场时,$\vb*{E}^2$——从而$I$——在形式上对应于“光子出现的概率”。
光场中被传输的不是$I$,电磁场的相位信息是很重要的,正如量子力学中叠加的不是概率而是概率振幅一样。
对非相干光(后文将讨论),可以直接将$I$相加,正如高度混合态的系统可以直接使用经典概率论处理一样。

\section{相干态和Wigner波函数}

如果我们直接计算电场在光子的Fock态下的期望值,无疑会得到零,因为$\vb*{E}$算符正比于单个产生算符和湮灭算符之和,从而,设$\ket*{\psi}$是一个光子Fock态,$\mel{\psi}{E_n}{\psi}$中,$E_n$作用到右边后模式$n$上的光子数目发生变化,因此电场期望值为零。

这看起来真是匪夷所思,因为这似乎说明Fock态中没有光子,没有能量,而这当然不正确。
例如,计算电场平方的期望值却又会得到非零结果。电场平方正比于该点的“电场能”,因此Fock态是有能量的。

问题的核心在于Fock态\emph{不是}电场的本征态。我们现在要找到一个和经典的电场接近的量子态,并且考虑如何用一种经典意义明显的方式表示一个任意的量子光学中的光场波函数。

\subsection{相干态}

\begin{back}{相干态和相干态路径积分}{coherent-state}
    设动力学变量$x$和它的正则动量满足正则对易关系
    \begin{equation}
        \comm{x}{p} = \ii,
    \end{equation}
    引入产生湮灭算符
    \begin{equation}
        a = \frac{1}{\sqrt{2}} (x + \ii p), \quad a^\dagger = \frac{1}{\sqrt{2}} (x - \ii p),
    \end{equation}
    则有正确的对易关系
    \begin{equation}
        \comm{a}{a^\dagger} = 1.
    \end{equation}
    我们现在考虑怎么将一个密度矩阵写成某种“准概率分布”的形式,即能够将它和一个函数$W(x, p)$或者$W(a, a^*)$建立线性关系。

    考虑用复参数定义的相干态
    \begin{equation}
        \ket*{\alpha} = \ee^{- \abs*{\alpha}^2 / 2} \sum_{n = 0}^\infty \frac{\alpha^n}{\sqrt{n!}} \ket*{0},
    \end{equation}
    其中$\ket*{n}$是谐振子哈密顿量
    \begin{equation}
        H = \sum_{n \geq 0} \hbar \omega \left( a^\dagger a + \frac{1}{2} \right) 
    \end{equation}
    的第$n$激发态。
    能够证明完备条件
    \begin{equation}
        \frac{1}{\pi} \int \dd[2]{\alpha} \dyad{\alpha} = 1
    \end{equation}
    成立,这里积分测度为
    \begin{equation}
        \dd[2]{\alpha} = \frac{\dd{\alpha^*} \wedge \dd{\alpha}}{2 \ii} = R \dd{R} \wedge \dd{\theta},
    \end{equation}
    其中$R$是$\alpha$的长度而$\theta$是相角,$\dd[2]{\alpha}$是复平面上的积分测度。
    由于不同$\alpha$的相干态彼此不正交——实际上,我们有
    \begin{equation}
        \braket*{\alpha}{\beta} = \ee^{- (\abs{\alpha}^2 + \abs{\beta}^2) / 2 + \alpha^* \beta},
    \end{equation}
    全体相干态实际上是超完备的。

\end{back}

我们在构造路径积分时遇到过形式最为一般的相干态。在量子光学中相干态还有特殊的意义。

设想空间中有一个振动频率给定的电偶极子
\begin{equation}
    \vb*{d}(t) = \vb*{d} \ee^{- \ii \omega t } + \text{h.c.},
\end{equation}
我们考虑空间中的光场的演化情况。设光场波函数为$\ket*{\psi}$,取偶极辐射近似。
在相互作用绘景中有
\begin{equation}
    \ii \pdv{t} \ket*{\psi} = - \vb*{d}(t) \cdot \vb*{E} \ket*{\psi}.
\end{equation}
假定$t=0$时没有任何光子,因此,等价地我们可以认为从$t=0$开始电偶极子才开始振动,在此之前系统中一直没有任何光子。
我们有形式解
\[
    \begin{aligned}
        \ket*{\psi(t)} &= \exp(\frac{\ii}{\hbar} \int_0^t \dd{t'} \sum_n (\vb*{d} \ee^{- \ii \omega t } + \text{h.c.}) \cdot (\mathcal{E}_n \vb*{f}_n a_n \ee^{- \ii \omega_n t'} + \text{h.c.}) ) \ket*{0} \\
        &= \exp(\sum_n (\alpha_n a_n^\dagger - \alpha_n^* a_n) ) \ket*{0},
    \end{aligned}
\]
这里我们定义
\begin{equation}
    \alpha_n \coloneqq \frac{\ii}{\hbar} \int_0^t \dd{t'} \mathcal{E}_n \vb*{f}_n^* \ee^{\ii \omega_n t'} \cdot (\vb*{d} \ee^{- \ii \omega t} + \text{h.c.}).
\end{equation}
由于
\[
    \comm*{a_n}{\comm*{a_n}{a^\dagger_n}} = \comm*{a^\dagger_n}{\comm*{a_n}{a^\dagger_n}} = 0,
\]
我们有
\[
    \begin{aligned}
        \exp(\alpha_n a_n^\dagger - \alpha_n^* a_n) &= \ee^{\alpha_n a_n^\dagger} \ee^{- \alpha_n^* a_n} \ee^{\frac{1}{2} \comm*{\alpha_n a_n^\dagger}{\alpha_n^* a_n}} \\
        &= \ee^{\alpha_n a_n^\dagger} \ee^{- \alpha_n^* a_n} \ee^{\frac{1}{2} \abs{\alpha_n}},
    \end{aligned}
\]
定义其为\concept{位移算符}
\begin{equation}
    D(\{\alpha_n\}) \coloneqq \prod_n \ee^{\alpha_n a_n^\dagger} \ee^{- \alpha_n^* a_n} \ee^{\frac{1}{2} \abs{\alpha_n}},
    \label{eq:displacement-operator}
\end{equation}
它作用在真空态上给出
\begin{equation}
    \ket*{\{\alpha_n\}} \coloneqq D(\{\alpha_n\}) \ket*{0} = \exp(\sum_n (\alpha_n a_n^\dagger - \alpha_n^* a_n) ) \ket*{0} = \prod_n \ee^{- \frac{\abs{\alpha_n}^2}{2}} \ee^{\alpha_n a^\dagger_n} \ket*{0_n}.
\end{equation}
将上式中的$\ee^{\alpha_n a^\dagger_n}$展开,我们发现
\begin{equation}
    \ket*{\{\alpha_n\}} = \prod_n \ee^{- \frac{\abs{\alpha_n}^2}{2}} \sum_{i_n=0}^\infty \frac{\alpha_n^{i_n}}{\sqrt{i_n!}} \ket*{0_n},
\end{equation}
因此它的确就是路径积分量子化中定义的相干态。
位移算符这个说法是比较直观的:它给出了$\alpha$的位移;此外注意到电偶极相互作用哈密顿量$- \vb*{d} \cdot \vb*{E}$就是“一个外场乘在动力学变量上”,在谐振子的物理图像中,相当于一个谐振子从某时刻起被放置在一个匀强外场中,形象地说,就是电梯中放置了一个谐振子,然后电梯突然启动了。

现在我们回来看看$\{\alpha_n\}$到底是什么,或者说相干态到底是什么。
实际上在路径积分量子化中我们已经知道系统处在相干态上意味着系统大体上服从经典场论了,不过还是做一些具体计算来更清楚地展示这一点。

为了更加简便我们考虑一个\concept{单模光场},即只考虑\eqref{eq:general-optical-field}中的一个模式。
这是合理的,因为一个线性体系中的所有模式之间不存在任何关系。
对某个特定的模式对应的产生湮灭算符$(a, a^\dagger)$定义
\begin{equation}
    X_1 = \frac{1}{2} (a + a^\dagger), \quad X_2 = \frac{1}{2\ii} (a - a^\dagger),
\end{equation}
则
\begin{equation}
    \vb*{E} = 2 \mathcal{E} (\vb*{f} X_1 \cos(\omega t) + \vb*{f}^* X_2 \sin(\omega t)).
\end{equation}
经典电动力学中系统总是在相干态附近,将$a$替换成$\alpha$就能够从量子光学过渡到经典光学。
在做了这个过渡之后,$X_1$大体上是$\Re{\alpha}$而$X_2$大体上就是$\Im{\alpha}$。
于是我们就看到了相干态$\ket*{\alpha}$的经典意义:一个相干态$\ket*{\alpha}$对应一个正弦振动的经典电场,其幅度为
\begin{equation}
    A = 2 \mathcal{E} \abs{\vb*{f}} \sqrt{X_1^2 + X_2^2} = 2 \mathcal{E} \abs{\vb*{f}} \abs{\alpha},
\end{equation}
其相位则是
\begin{equation}
    \varphi = \arg \alpha = \arctan \frac{X_2}{X_1}.
\end{equation}

$X_1$和$X_2$当然是不对易的,因此即使是相干态下(当然也包括其它相干态下),实际上电场
\begin{equation}
    E = A \ee^{\ii \varphi} \ee^{- \omega t} + \text{h.c.}
\end{equation}
是不能够完全确定的。或者也可以说,$A$(实际上就是粒子数开平方)和$\varphi$是不能同时完全确定的。

相干态下的光子数服从泊松分布。
因此,相干态的平均值和经典电场类似,但是存在涨落,其涨落行为和真空类似。

\subsection{准概率分布函数}

\begin{back}{量子力学中的相空间和准概率分布函数}{phase-wigner}
    虽然在量子力学中坐标和动量不能同时确定,从而看起来似乎不能够有良定义的相空间,不过注意到,算符$O$的矩阵元一般来说形如$\mel{x}{O}{x'}$,即需要两个标签——$x$和$x'$——标记一个矩阵元,那么我们将$x$和$x'$线性组合一下,将其中一个坐标做傅里叶变换切换到动量空间,似乎还是能够写出$O(x, p)$这样的式子,从而有一个等效的相空间。
    当然,这样得到的“相空间中的分布函数”,即满足下述条件的函数$f(x, p)$:
    \[
        \expval*{O} = \int \dd{x} \dd{p} O(x, p) f(x, p)
    \]
    未必能够赋予概率的意义,因为它可以取负值甚至虚数值。

    我们现在考虑几种$f(x, p)$的定义。最为知名的可能是Wigner函数,它基本上是经典力学中的粒子分布函数的量子推广。
    设$\rho$是一个密度矩阵,\concept{Wigner函数}定义为
    \begin{equation}
        W(x, p) = \frac{1}{2\pi \hbar} \int \dd{y} \mel{x - \frac{y}{2}}{\rho}{x + \frac{y}{2}} \ee^{\ii p y / \hbar}.
    \end{equation}
    当然也可以用$\alpha$和$\alpha^*$——或者说,用二维相平面上的复数$\alpha$——做Wigner函数的宗量。
    将算符$O$写成关于$a$和$a^\dagger$的\emph{对称排序}$O_\text{S}$之后,我们有
    \begin{equation}
        \expval*{O} = \int \dd[2]{\alpha} W(\alpha, \alpha^*) O_\text{S}(\alpha, \alpha^*).
    \end{equation}
    例如,
    \begin{equation}
        \frac{1}{2} \expval*{a a^\dagger + a^\dagger a} = \int \dd[2]{\alpha} W(\alpha, \alpha^*) \alpha \alpha^*.
    \end{equation}
    
    Wigner分布函数可以验证是实数,不过有正有负。能够证明Wigner函数取负值的区域不会太大——在几个$\hbar$以内——这直观地展示了从量子过渡到经典的过程。
    与路径积分类似,Wigner函数也并非只能在坐标-动量构成的相空间中定义——任何有广义坐标和广义动量的能够用哈密顿动力学描述的系统中的密度矩阵都能够用Wigner函数等价地给出。
    因此,Wigner函数实际上可以构成量子力学的另一种形式理论:所谓量子力学,就是将普通的概率分布拓展为Wigner函数的物理理论。

    我们当然也可以寻找一种函数$P(\alpha, \alpha^*)$使得
    \begin{equation}
        \expval*{O} = \int \dd[2]{\alpha} P(\alpha, \alpha^*) O_\text{N}(\alpha, \alpha^*),
    \end{equation}
    这里$O_\text{N}$表示算符$O$的\emph{正则排序},即将湮灭算符排在右边,产生算符排在左边。
    我们有
    \[
        \begin{aligned}
            \expval*{O} &= \sum_{m, n} c_{mn} \trace((a^\dagger)^m a^n \rho) \\
            &= \sum_{m, n} c_{mn} \int \dd[2]{\alpha} \trace((\alpha^*)^m \alpha^n \rho \delta(\alpha^* - a^\dagger) \delta(\alpha - a)) \\
            &= \int \dd[2]{\alpha} \sum_{m,n} c_{mn} (\alpha^*)^m \alpha^n \trace(\rho \delta(\alpha^* - a^\dagger) \delta(\alpha - a)),
        \end{aligned}
    \]
    上式最后一行的形式正是$O_\text{N}$乘以某个分布函数,于是我们得到
    \begin{equation}
        P(\alpha, \alpha^*) = \trace(\rho \delta(\alpha^* - a^\dagger) \delta(\alpha - a)).
    \end{equation}
    简单地验证会发现
    \begin{equation}
        \rho=\int \dd^{2} \alpha P\left(\alpha, \alpha^{*}\right)|\alpha\rangle\langle\alpha| .
    \end{equation}

    我们当然还可以让寻找一种函数$Q(\alpha, \alpha^*)$使得
    \begin{equation}
        \expval*{O} = \int \dd[2]{\alpha} Q(\alpha, \alpha^*) O_\text{A}(\alpha, \alpha^*),
    \end{equation}
    这里$O_\text{A}$表示算符$O$的\emph{反正则排序},即湮灭算符排在左边而产生算符排在右边。和$P$函数一样如法炮制,能够得到
    \begin{equation}
        Q\left(\alpha, \alpha^{*}\right)=\operatorname{tr}\left[\rho \delta(\alpha-a) \delta\left(\alpha^{*}-a^{\dagger}\right)\right].
    \end{equation}
    由于
    \[
        \begin{aligned}
            Q\left(\alpha, \alpha^{*}\right) &=\frac{1}{\pi} \operatorname{tr} \int d^{2} \alpha^{\prime}\left[\rho \delta(\alpha-a)\left|\alpha^{\prime}\right\rangle\left\langle\alpha^{\prime}\right| \delta\left(\alpha^{*}-a^{\dagger}\right)\right] \\
            &=\frac{1}{\pi} \operatorname{tr} \int d^{2} \alpha^{\prime}\left\{\rho \delta\left(\alpha-\alpha^{\prime}\right)\left|\alpha^{\prime}\right\rangle\left\langle\alpha^{\prime}\right| \delta\left[\alpha^{*}-\left(\alpha^{\prime}\right)^{*}\right]\right\} \\
            &=\frac{1}{\pi} \operatorname{tr}(\rho|\alpha\rangle\langle\alpha|) \\
            &=\frac{1}{\pi}\langle\alpha|\rho| \alpha\rangle,
            \end{aligned}
    \]
    我们有
    \begin{equation}
        Q(\alpha, \alpha^*) = \frac{1}{\pi} \mel{\alpha}{\rho}{\alpha},
    \end{equation}

    $P$函数,$W$函数,$Q$函数都能够写成某些算符的期望值的某种积分变换。
    我们有
    \begin{equation}
        \delta\left(\alpha^{*}-a^{\dagger}\right) \delta(\alpha-a) =\frac{1}{\pi^{2}} \int \dd^{2} \beta \exp \left[-\beta\left(\alpha^{*}-a^{\dagger}\right)\right] \exp \left[\beta^{*}(\alpha-a)\right],
    \end{equation}
    于是
    \begin{equation}
        P\left(\alpha, \alpha^{*}\right)=\frac{1}{\pi^{2}} \int \dd^{2} \beta \ee^{- \ii \beta \alpha^{*}- \ii \beta^{*} \alpha} C_\text{N}\left(\beta, \beta^{*}\right),
    \end{equation}
    其中
    \begin{equation}
        C_\text{N}\left(\beta, \beta^{*}\right)=\operatorname{tr}\left(\ee^{\ii \beta a^{\dagger}} \ee^{\ii \beta^{*} a} \rho\right).
    \end{equation}
    类似的
    \begin{equation}
        Q\left(\alpha, \alpha^{*}\right)=\frac{1}{\pi^{2}} \int \dd^{2} \beta \ee^{- \ii \beta \alpha^{*}- \ii \beta^{\cdot} \alpha} C_\text{A}\left(\beta, \beta^{*}\right),
    \end{equation}
    其中
    \begin{equation}
        C_\text{A}\left(\beta, \beta^{*}\right)=\operatorname{tr}\left(\ee^{\ii \beta^{*} a} \ee^{\ii \beta a^{\dagger}} \rho\right).
    \end{equation}
    对Wigner函数也有类似的公式:
    \begin{equation}
        W\left(\alpha, \alpha^{*}\right)=\frac{1}{\pi^{2}} \int \dd^{2} \beta \ee^{-\ii \beta \alpha^{*}- \ii \beta^{*} \alpha} C_\text{S}\left(\beta, \beta^{*}\right),
    \end{equation}
    其中
    \begin{equation}
        C_\text{S}\left(\beta, \beta^{*}\right)=\operatorname{tr}\left(\ee^{\ii \beta a^{\dagger}+ \ii \beta^{*} a} \rho\right).
    \end{equation}
\end{back}

\section{光场的多光子概率分布和相干性}

\subsection{关联函数}\label{sec:correlation-function-photon}

实用的光探测器基本上都是根据偶极相互作用哈密顿量$- \vb*{d} \cdot \vb*{E}$工作的,即一个凝聚态系统——这里是光传感器——吸收了一个光子,将电子激发出来,然后探测由此产生的光电流。
由于光传感器中电子的能量通常很低,不会激发出多少光子,实验中置于光场中的探测器探测到光信号的概率几乎完全由“光场中一个光子湮灭,光场掉落到一个能量更低的状态”这样的过程贡献。
因此,一个可观察量的期望值基本上就是
\[
    \expval*{O} = \sum_\text{final state $f$} \abs{\mel{f}{\text{photon annihilation}}{\psi}}^2 \times \text{a function of $f$},
\]
容易看出,只要将关于末态$\ket*{f}$的函数写成$a$和$a^\dagger$的正规序(注意我们将电子的波函数看成背景的,$\ket*{f}$只是光场的波函数),这样的一个期望值可以展开成一系列形如
\[
    \mel{\psi}{a^\dagger_{\vb*{q}_1} a^\dagger_{\vb*{q}_2} \cdots a_{\vb*{p}_2} a_{\vb*{p}_1}}{\psi}
\]
的正规序关联函数之和;考虑到“关于末态$\ket*{f}$的函数”很多时候就是光子数的函数,实际上大部分可观察量的期望值\emph{就是}一个正规序关联函数。
考虑到光场本身可能热化,更加一般的形式是
\[
    \trace(\rho a^\dagger_{\vb*{q}_1} a^\dagger_{\vb*{q}_2} \cdots a_{\vb*{p}_2 a_{\vb*{p}_2}}) \eqqcolon \expval*{a^\dagger_{\vb*{q}_1} a^\dagger_{\vb*{q}_2} \cdots a_{\vb*{p}_2 a_{\vb*{p}_2}}},
\]
其中$\rho$是光场的密度矩阵。同样,关于电场的有明确物理意义的关联函数均形如
\begin{equation}
    \expval*{\vb*{E}^-_{1'} \vb*{E}^-_{2'} \cdots \vb*{E}^+_2 \vb*{E}^+_1},
    \label{eq:correlation-function-optical-field}
\end{equation}
其中
\begin{equation}
    \vb*{E}^{+} (\vb*{r}, t) = \sum_n \mathcal{E}_n \vb*{f}_n(\vb*{r}) a_n \ee^{-\ii \omega_n t} , \quad \vb*{E}^{-} (\vb*{r}, t) = \sum_n \mathcal{E}_n \vb*{f}^*_n(\vb*{r}) a^\dagger_n \ee^{\ii \omega_n t}, \quad \vb*{E} = \vb*{E}^+ + \vb*{E}^-,
\end{equation}
符号$\vb*{E}^{\pm}_n$表示时空坐标$\vb*{r}_n, t_n$处的$\vb*{E}^\pm$场;为了保证正规序——实际上是为了保证因果性——我们要求对$\vb*{E}^+$算符的乘积序列有$t_1 < t_2 < \cdots$。
如果我们用光子激发出的电偶极矩作为观测量,那么\eqref{eq:correlation-function-optical-field}有直接的物理意义。

我们考虑在空间位置$\vb*{r}$探测到光子的概率,或者等价地说,在$\vb*{r}$处的基于电偶极相互作用的一个探测器附近湮灭了一个光子而有电子首受到激发。
根据费米黄金法则,大体上有
\[
    P_f \propto \mel{f}{- \vb*{d} \cdot \vb*{E}^+(\vb*{r})}{\psi},
\]
如果我们不关心激发出了一个怎样的电子(也即,不关心被湮灭的的是一个怎样的光子,相当于是除了光子粒子数变化了以外什么也不想知道),那么可以忽略末态$f$的细节而将所有$f$的概率直接非相干求和,得到
\[
    P \propto \sum_f \mel{f}{- \vb*{d} \cdot \vb*{E}^+(\vb*{r})}{\psi}.
\]
由于末态的完备性,我们就得到
\begin{equation}
    P(\vb*{r}) = \eta \expval{\vb*{E}^-(\vb*{r}) \cdot \vb*{E}^+(\vb*{r})},
\end{equation}
其中$\eta$是一个(一般来说难以确定的)关于探测器细节,可能还含有诸如$2\pi / \hbar$之类的费米黄金法则引入的因子的常数。
点积$\cdot$出现,是因为$\vb*{d}$可以指向各个方向,于是求和后它正比于单位张量,从而最后$(\vb*{E}^- \cdot \vb*{d}^\dagger) (\vb*{d} \cdot \vb*{E}^+)$就变成$\vb*{E}^- \cdot \vb*{E}^+$。
% TODO:完整的费米黄金法则计算

类似的,在$\vb*{r}_1, \vb*{r}_2, \ldots, \vb*{r}_N$同时探测到光子的概率为
\[
    P \propto \sum_{f} \prod_i (- \vb*{d} \cdot \vb*{E}^-(\vb*{r}_i)),
\]
经过完全一样的化简过程,我们发现最一般的$N$光子联合概率分布函数为
\begin{equation}
    P^{(N)}(\vb*{r}_1, \ldots, \vb*{r}_N) = \eta^N \expval*{\vb*{E}^-(\vb*{r}_1) \cdots \vb*{E}^-(\vb*{r}_N) \cdot \vb*{E}^+(\vb*{r}_N) \cdots \vb*{E}^+(\vb*{r}_1)},
    \label{eq:general-n-photon-correlation}
\end{equation}
其中$\eta$定义如前,点乘$\cdot$还是表示将$\vb*{E}^-(\vb*{r}_N)$和$\vb*{E}^+(\vb*{r}_N)$缩并,将$\vb*{E}^-(\vb*{r}_{N-1})$和$\vb*{E}^+(\vb*{r}_{N-1})$缩并等等。

以上推导都是针对纯态的,然而由于对末态$f$的求和是非相干求和,\eqref{eq:general-n-photon-correlation}对热态也适用。

简单的计算告诉我们,$N$粒子Fock态的

我们还可以定义\concept{相干度},即将$N$点关联函数归一化后得到的结果。
我们有
\begin{equation}
    g^{(N)}(\vb*{r}_1, \vb*{r}_2) = \frac{\expval*{\vb*{E}^-(\vb*{r}_1) \cdots \vb*{E}^-(\vb*{r}_N) \cdot \vb*{E}^+(\vb*{r}_N) \cdots \vb*{E}^+(\vb*{r}_1)}}{\expval*{\vb*{E}^-(\vb*{r}_1) \cdot \vb*{E}^+(\vb*{r}_1)} \cdots \expval*{\vb*{E}^-(\vb*{r}_N) \cdot \vb*{E}^+(\vb*{r}_N)}},
\end{equation}
它给出一个无量纲的,衡量光场中多点相干性的量。
我们对两点相干度或者说一阶相干度比较熟悉。用狄拉克的话说,这是“光子自己和自己干涉”:这里最重要的是计算$\expval*{\vb*{E}^+(\vb*{r}_1) \vb*{E}^-(\vb*{r}_2)}$,它体现的是单光子波函数的空间分布。

能够证明相干态的任意阶相干度都是1。

对Fock态的相干度,我们有
\begin{equation}
    g^{(2)}(\vb*{r}) = 1 - \frac{1}{n},
\end{equation}
其中$n$为光子数。这是非常合理的:例如,$n=1$时$g^{(2)} = 0$,因为系统中确定只有一个光子,这个光子在一个地方被湮灭了以后无论如何不可能出现在另一个地方。

对单模光场的热态
\begin{equation}
    \rho = \sum_n \frac{\bar{n}^n}{(1 + \bar{n})^{n+1}} \dyad{n},
\end{equation}
我们有
\[
    \begin{aligned}
        &\quad \expval{\vb*{E}^-(\vb*{r}_1) \cdots \vb*{E}^-(\vb*{r}_N) \cdot \vb*{E}^+(\vb*{r}_N) \cdots \vb*{E}^+(\vb*{r}_1)} \\
        &= \sum_n 
    \end{aligned}
\]
例如,
\begin{equation}
    g^{(2)} = 2.
\end{equation}
相当奇怪的,在热态中,观察到一个光子会让观察到下一个光子的概率\emph{增加}。

\subsection{Hanbury Brown和Twiss效应}

\subsubsection{测量天狼星的张角}

\begin{figure}
    \centering
    

\tikzset{every picture/.style={line width=0.75pt}} %set default line width to 0.75pt        

\begin{tikzpicture}[x=0.75pt,y=0.75pt,yscale=-1,xscale=1]
%uncomment if require: \path (0,300); %set diagram left start at 0, and has height of 300

%Shape: Star [id:dp7993730556190444] 
\draw  [fill={rgb, 255:red, 144; green, 19; blue, 254 }  ,fill opacity=0.07 ] (115,22.33) -- (117.35,27.3) -- (122.61,28.09) -- (118.8,31.95) -- (119.7,37.41) -- (115,34.83) -- (110.3,37.41) -- (111.2,31.95) -- (107.39,28.09) -- (112.65,27.3) -- cycle ;
%Straight Lines [id:da6010958544087177] 
\draw    (107.39,28.09) -- (119,208.33) ;
\draw [shift={(113.2,118.21)}, rotate = 266.31] [fill={rgb, 255:red, 0; green, 0; blue, 0 }  ][line width=0.08]  [draw opacity=0] (12,-3) -- (0,0) -- (12,3) -- cycle    ;
%Straight Lines [id:da5792212007108453] 
\draw    (122.61,28.09) -- (119,208.33) ;
\draw [shift={(120.8,118.21)}, rotate = 271.15] [fill={rgb, 255:red, 0; green, 0; blue, 0 }  ][line width=0.08]  [draw opacity=0] (12,-3) -- (0,0) -- (12,3) -- cycle    ;
%Shape: Arc [id:dp02266441256288565] 
\draw  [draw opacity=0] (88.36,213.69) .. controls (97.39,210.04) and (108.63,208.04) .. (120.75,208.37) .. controls (130.27,208.63) and (139.16,210.3) .. (146.81,213) -- (119.93,238.36) -- cycle ; \draw   (88.36,213.69) .. controls (97.39,210.04) and (108.63,208.04) .. (120.75,208.37) .. controls (130.27,208.63) and (139.16,210.3) .. (146.81,213) ;
%Curve Lines [id:da2800220853905422] 
\draw    (110,76.33) .. controls (118,77.33) and (118,75.33) .. (122,77) ;
%Straight Lines [id:da7563818948167416] 
\draw    (107.39,28.09) -- (96,209.33) ;
\draw [shift={(101.7,118.71)}, rotate = 273.6] [fill={rgb, 255:red, 0; green, 0; blue, 0 }  ][line width=0.08]  [draw opacity=0] (12,-3) -- (0,0) -- (12,3) -- cycle    ;
%Straight Lines [id:da6535572732710992] 
\draw    (122.61,29.09) -- (96,210.33) ;
\draw [shift={(109.3,119.71)}, rotate = 278.35] [fill={rgb, 255:red, 0; green, 0; blue, 0 }  ][line width=0.08]  [draw opacity=0] (12,-3) -- (0,0) -- (12,3) -- cycle    ;

% Text Node
\draw (126,65.4) node [anchor=north west][inner sep=0.75pt]    {$\theta $};
% Text Node
\draw (82,214.4) node [anchor=north west][inner sep=0.75pt]    {$I_{1}$};
% Text Node
\draw (116,209.4) node [anchor=north west][inner sep=0.75pt]    {$I_{2}$};


\end{tikzpicture}

    \caption{HBT实验测量天狼星的张角}
    \label{fig:hbt-initial-experiment}
\end{figure}

1954年两位电气工程师Robert Hanbury Brown%
\footnote{
    Hanbury Brown是一个复姓。
}%
和Richard Q. Twiss试图想出一种方法测量天狼星的张角。如\autoref{fig:hbt-initial-experiment}所示,我们在地球上安装两个相隔一定距离的探测器,用它们接受天狼星两端的光。
与通常的干涉实验不同,Hanbury Brown和Twiss测量\emph{二阶相干度}

这个说法——遥远星体上相隔很远的两点产生的光不知道怎么回事具有很好的相干性——不出意外地引起了轩然大波。

\subsubsection{经典电动力学解释}

\subsubsection{两个激光器产生的光束的干涉}

两个激光器产生的光似乎是相干的?

实际上激光器产生的是相干态光而不是光子数确定的多光子玻色波函数。

\subsubsection{玻色-爱因斯坦凝聚态中的干涉}

\chapter{线性量子光学过程}

\section{线性过程}

我们这一节要做的事情和\autoref{chap:linear-matter-no-scattering}中没有什么本质上的区别:我们讨论线性过程——或者说没有光子生灭的过程。
对这样的一个过程,我们选择海森堡绘景,设$\{a_k\}$是某组模式,$\{b_l \}$是它经过一个线性过程演化之后的模式,即\begin{equation}
    b_l^\dagger = S_{lk} a^\dagger_k,
\end{equation}
且根据幺正性要求我们有
\begin{equation}
    a_k^\dagger = S_{kl}^* b^\dagger_l,
\end{equation}
则使用$a$算符表述的系统状态是
\begin{equation}
    \ket*{\psi}_\text{in} = f(\{ a_k^\dagger \}) \ket*{0},
\end{equation}
而使用$b$算符表述的系统状态是
\begin{equation}
    \ket*{\psi}_\text{out} = f(\{ S_{kl}^* b_l^\dagger \}) \ket*{0}.
\end{equation}
例如,对相干态我们有
\begin{equation}
    \ket*{\alpha} = 
\end{equation}
因此我们有
\begin{equation}
    \beta_l = S_{lk}^* \alpha_k ,
\end{equation}
即相干态的标签$\alpha$的时间演化方式和$a$算符完全一样,正好是预期的结果。

应当注意这里的$S$矩阵\emph{不是}薛定谔绘景中经过元件前后的光场的量子态的演化算符,而是它的共轭转置。
这个微妙之处在量子场论中就出现过:在一个自由理论中,薛定谔绘景下单粒子波函数的运动方程和海森堡绘景中\emph{湮灭}算符的时间演化方程保持一致,而不是和产生算符。
这里的关键点在于在海森堡绘景中态矢量根本不应该发生任何变化,海森堡绘景中如果某个矢量发生了时间演化,那是因为用于定义这个矢量的算符发生了时间演化,我们自然没有理由要求这种所谓的“矢量的时间演化”遵循薛定谔绘景中的演化方程。

我们还应当注意$S$矩阵实际上有两重地位:它一方面是一个时间演化算符,另一方面其实也是一个类似于边界条件的东西:线性光学元件入射和反射端口处的模式\emph{必定}满足关系式$b^\dagger = S a^\dagger$。
这两重身份是兼容的:前者用高能物理中的话说是在约束$t = -\infty$的入射态和$t = \infty$的出射态之间的关系,后者是在描写散射定态,这两者当然是有关的。
类似的,我们在讨论非线性晶体时是在求解非线性晶体内部的\emph{稳态}电场分布,计算得到的是各种模式上的能量分布和透射深度的关系,但是由于散射定态决定了$t = -\infty$的入射态和$t = \infty$的出射态之间的关系,实际上稳态电场分布给出的电场随着透射深度的变化就反映了在非线性晶体的一侧输入一个脉冲后输出端的脉冲的成分。

\subsection{分束器}

\begin{figure}
    \centering
    \subfigure[]{
        

\tikzset{every picture/.style={line width=0.75pt}} %set default line width to 0.75pt        

\begin{tikzpicture}[x=0.75pt,y=0.75pt,yscale=-1,xscale=1]
%uncomment if require: \path (0,300); %set diagram left start at 0, and has height of 300

%Shape: Square [id:dp7292498374829528] 
\draw   (235,140) -- (261,140) -- (261,166) -- (235,166) -- cycle ;
%Straight Lines [id:da010139268745176677] 
\draw    (235,166) -- (261,140) ;

%Straight Lines [id:da5573883310885233] 
\draw    (168,153) -- (248,153) ;
\draw [shift={(208,153)}, rotate = 180] [fill={rgb, 255:red, 0; green, 0; blue, 0 }  ][line width=0.08]  [draw opacity=0] (12,-3) -- (0,0) -- (12,3) -- cycle    ;
%Straight Lines [id:da26712641506196255] 
\draw    (248,153) -- (353.71,153) ;
\draw [shift={(300.85,153)}, rotate = 180] [fill={rgb, 255:red, 0; green, 0; blue, 0 }  ][line width=0.08]  [draw opacity=0] (12,-3) -- (0,0) -- (12,3) -- cycle    ;
%Straight Lines [id:da6641395542938842] 
\draw    (248,153) -- (248,221.67) ;
\draw [shift={(248,187.33)}, rotate = 270] [fill={rgb, 255:red, 0; green, 0; blue, 0 }  ][line width=0.08]  [draw opacity=0] (12,-3) -- (0,0) -- (12,3) -- cycle    ;
%Straight Lines [id:da6178054140461928] 
\draw    (248,85.33) -- (248,153) ;
\draw [shift={(248,119.17)}, rotate = 270] [fill={rgb, 255:red, 0; green, 0; blue, 0 }  ][line width=0.08]  [draw opacity=0] (12,-3) -- (0,0) -- (12,3) -- cycle    ;
%Straight Lines [id:da6569709591508739] 
\draw    (248,221.67) -- (353.71,221.67) ;
%Shape: Chord [id:dp4222004696727377] 
\draw   (353.82,137.75) .. controls (363.01,137.77) and (370.52,144.37) .. (370.69,152.67) .. controls (370.85,161.09) and (363.38,168.06) .. (354,168.25) -- cycle ;
%Shape: Chord [id:dp5038597778456795] 
\draw   (353.82,206.42) .. controls (363.01,206.43) and (370.52,213.04) .. (370.69,221.33) .. controls (370.85,229.76) and (363.38,236.73) .. (354,236.91) -- cycle ;

% Text Node
\draw (375.71,153) node [anchor=west] [inner sep=0.75pt]    {$n_{1}$};
% Text Node
\draw (373.71,221) node [anchor=west] [inner sep=0.75pt]    {$n_{2}$};
% Text Node
\draw (253,74.4) node [anchor=north west][inner sep=0.75pt]    {$\ket{0}$};


\end{tikzpicture}

        \label{fig:beam-splitter-1}
    }
    \subfigure[]{
        

\tikzset{every picture/.style={line width=0.75pt}} %set default line width to 0.75pt        

\begin{tikzpicture}[x=0.75pt,y=0.75pt,yscale=-1,xscale=1]
%uncomment if require: \path (0,300); %set diagram left start at 0, and has height of 300

%Shape: Square [id:dp6835515765563589] 
\draw   (434,169) -- (460,169) -- (460,195) -- (434,195) -- cycle ;
%Straight Lines [id:da46226803039130826] 
\draw    (434,195) -- (460,169) ;

%Straight Lines [id:da1670024331738862] 
\draw    (367,182) -- (447,182) ;
\draw [shift={(407,182)}, rotate = 180] [fill={rgb, 255:red, 0; green, 0; blue, 0 }  ][line width=0.08]  [draw opacity=0] (12,-3) -- (0,0) -- (12,3) -- cycle    ;
%Straight Lines [id:da9323888654594039] 
\draw    (447,182) -- (521,182) ;
\draw [shift={(484,182)}, rotate = 180] [fill={rgb, 255:red, 0; green, 0; blue, 0 }  ][line width=0.08]  [draw opacity=0] (12,-3) -- (0,0) -- (12,3) -- cycle    ;
%Straight Lines [id:da9013055724976109] 
\draw    (447,182) -- (447,250.67) ;
\draw [shift={(447,216.33)}, rotate = 90] [fill={rgb, 255:red, 0; green, 0; blue, 0 }  ][line width=0.08]  [draw opacity=0] (12,-3) -- (0,0) -- (12,3) -- cycle    ;
%Straight Lines [id:da04689085671987714] 
\draw    (447,114.33) -- (447,182) ;
\draw [shift={(447,148.17)}, rotate = 90] [fill={rgb, 255:red, 0; green, 0; blue, 0 }  ][line width=0.08]  [draw opacity=0] (12,-3) -- (0,0) -- (12,3) -- cycle    ;




\end{tikzpicture}

        \label{fig:beam-splitter-2}
    }
    \caption{分束器}
    \label{fig:beam-splitter}
\end{figure}

一个分束器是一个形如\autoref{fig:beam-splitter}的装置,最简单的分束器是一个有特殊镀膜的玻璃片(\autoref{fig:beam-splitter-1}),也可以是如今实验上更加常见的两个等腰直角三角形夹着特殊镀膜的结构(\autoref{fig:beam-splitter-2})。
它将两束光变换为另外两束光。分束器的变换矩阵为
\begin{equation}
    \pmqty{\tilde{\mathcal{E}}_1 \\ \tilde{\mathcal{E}}_2} = \pmqty{t & r \\ - r^* & t} \pmqty{\mathcal{E}_1 \\ \mathcal{E}_2}.
\end{equation}
这里,$-r^*$的负号来自幺正性的要求;实际上这就是经典电动力学中的半波损失。
例如,我们有\SI{50}{\percent}分束器,其变换矩阵为
\begin{equation}
    S = \frac{1}{\sqrt{2}} \pmqty{1 & 1 \\ -1 & 1}.
\end{equation}

\subsection{反射镜}

\section{线性光路}

\subsection{Aspect实验}

要证明单光子具有量子性,最好的办法是使用一些这样的实验:它的一种版本能够证明光子的粒子性,它的另一个仅仅做了少许修正的版本(比如说探测器被移动到别的位置)能够证明光子的波动性。
两个版本区别很小这件事能够排除实验装置和光的复杂相互作用显著地改变了光的行为这样的说法,而粒子性和波动性同时出现则强烈暗示需要量子理论描述光。
1986年的Aspect实验是这种实验的一个典范。

\begin{equation}
    \begin{aligned}
        S &= \frac{1}{\sqrt{2}} \pmqty{1 & 1 \\ -1 & 1} \pmqty{ \dmat{\ee^{\ii \varphi / 2}, \ee^{- \ii \varphi / 2}} } \frac{1}{\sqrt{2}} \pmqty{1 & 1 \\ -1 & 1}  \\
        &= \pmqty{\cos\frac{\varphi}{2} & \sin\frac{\varphi}{2} \\ - \sin\frac{\varphi}{2} & \cos\frac{\varphi}{2}}.
    \end{aligned}
\end{equation}

\subsection{Zeilinger实验}

A.Zeilinger
一种更加简明的实验是这样的:同样使用分束器和反射镜,构造这样的光路:

\subsection{Hong-Ou-Mondel效应}

\section{线性单光子量子计算}

\subsection{单光子态产生和条件量子态}

\begin{figure}
    \centering
    

\tikzset{every picture/.style={line width=0.75pt}} %set default line width to 0.75pt        

\begin{tikzpicture}[x=0.75pt,y=0.75pt,yscale=-1,xscale=1]
%uncomment if require: \path (0,300); %set diagram left start at 0, and has height of 300

%Shape: Rectangle [id:dp9242602312444221] 
\draw   (193,119) -- (263,119) -- (263,159) -- (193,159) -- cycle ;
%Straight Lines [id:da7075004076272056] 
\draw    (127,140) -- (193,140) ;
\draw [shift={(160,140)}, rotate = 180] [fill={rgb, 255:red, 0; green, 0; blue, 0 }  ][line width=0.08]  [draw opacity=0] (12,-3) -- (0,0) -- (12,3) -- cycle    ;
%Straight Lines [id:da42996678105947117] 
\draw    (263,140) -- (348,79) ;
\draw [shift={(305.5,109.5)}, rotate = 504.33] [fill={rgb, 255:red, 0; green, 0; blue, 0 }  ][line width=0.08]  [draw opacity=0] (12,-3) -- (0,0) -- (12,3) -- cycle    ;
%Straight Lines [id:da7812145894988733] 
\draw    (263,140) -- (348,201) ;
\draw [shift={(305.5,170.5)}, rotate = 215.67000000000002] [fill={rgb, 255:red, 0; green, 0; blue, 0 }  ][line width=0.08]  [draw opacity=0] (12,-3) -- (0,0) -- (12,3) -- cycle    ;
%Straight Lines [id:da29099064424770593] 
\draw    (348,201) -- (426,201) ;
\draw [shift={(387,201)}, rotate = 180] [fill={rgb, 255:red, 0; green, 0; blue, 0 }  ][line width=0.08]  [draw opacity=0] (12,-3) -- (0,0) -- (12,3) -- cycle    ;
%Shape: Chord [id:dp5153738151961664] 
\draw   (425.67,179) .. controls (438.92,179.02) and (449.76,188.55) .. (450,200.52) .. controls (450.23,212.67) and (439.46,222.73) .. (425.93,223) -- cycle ;
%Straight Lines [id:da7719196407492437] 
\draw  [dash pattern={on 4.5pt off 4.5pt}]  (450,201) -- (466,201) ;
%Straight Lines [id:da27041501657550526] 
\draw    (348,79) -- (460,79) ;
\draw [shift={(404,79)}, rotate = 180] [fill={rgb, 255:red, 0; green, 0; blue, 0 }  ][line width=0.08]  [draw opacity=0] (12,-3) -- (0,0) -- (12,3) -- cycle    ;
%Straight Lines [id:da20605124142789344] 
\draw  [dash pattern={on 4.5pt off 4.5pt}]  (466,105.33) -- (466,201) ;
%Shape: Rectangle [id:dp3038430486229844] 
\draw   (471,63) -- (459.92,63) -- (459.92,105.33) -- (471,105.33) -- cycle ;
%Straight Lines [id:da18864355508447694] 
\draw    (471,79) -- (526,79) ;
\draw [shift={(498.5,79)}, rotate = 180] [fill={rgb, 255:red, 0; green, 0; blue, 0 }  ][line width=0.08]  [draw opacity=0] (12,-3) -- (0,0) -- (12,3) -- cycle    ;

% Text Node
\draw (228,139) node   [align=left] {SHG};
% Text Node
\draw (118,115.4) node [anchor=north west][inner sep=0.75pt]    {$E_{\text{p}}$};
% Text Node
\draw (350,54.4) node [anchor=north west][inner sep=0.75pt]    {$E_{1}$};
% Text Node
\draw (346,177.4) node [anchor=north west][inner sep=0.75pt]    {$E_{2}$};


\end{tikzpicture}

    \caption{产生条件量子态的装置}
    \label{fig:condition-photon}
\end{figure}

考虑\autoref{sec:chi-2-wave}的量子版本,它会引入相互作用哈密顿量
\begin{equation}
    V = \int \dd[3]{\vb*{r}} \chi^{(2)} : \vb*{E}_\text{p}^\dagger \vb*{E}_1 \vb*{E}_2 + \text{h.c.},
\end{equation}
其中$\vb*{E}_\text{p}$为泵浦光;通常泵浦光是强激光,从而可以认为处在相干态上。
出射态可以相当好地被下式近似:% TODO
\begin{equation}
    \ket{\psi_\text{out}} = \exp(\beta a_1^\dagger a_2^\dagger - \text{h.c.}) \ket{0}.
\end{equation}

现在我们制造一个非幺正的装置,展示如\autoref{fig:condition-photon}:它首先测量$E_2$模式上的光子数,如果发现光子数不是$1$,那么启动一个挡板,将$E_1$模式上的光子全部吸收掉;反之,如果发现光子数真的是$1$,那么让$E_1$模式上的光子通过。
最终我们会得到一个混合态:它有一定的概率是$E_1$上的单光子态,而剩下的可能性则是什么也没有。
并且,我们能够通过$E_2$处的探测器的探测结果,明确地知道我们是否成功制备了一个单光子态。
我们得到的$E_1$态是\concept{条件量子态}的一个例子。

\subsection{量子门}

条件量子态的产生后我们需要做量子控制,即用一个光子控制另一个光子;这可以称为\concept{单光子非线性性},因为对应的哈密顿量不再能够写成电磁场的二次型,而是必须含有非线性光学过程。
因此,使用前面列出的纯粹线性的操作,是肯定不足以实现通用量子计算的。

最为直接的想法是,我们能够找到一个足够强的二阶非线性晶体,那么就能够实现:这意味着我们可以。
例如,就SHG过程而言,如果它足够强,使得两个不同模式的光子共同存在时就能够聚合为一个另一个模式的光子,那么这就是一个与门。
可惜的是目前在普通的非线性晶体中无法实现这种过程:这些非线性光学过程都实在是太弱了。冷原子体系能够产生强烈的光子-光子耦合,这是一个可能的方向。
我们这里则致力于通过一些巧妙的含有测量的构造来实现\emph{等效}的量子门。

\begin{figure}
    \centering
    

\tikzset{every picture/.style={line width=0.75pt}} %set default line width to 0.75pt        

\begin{tikzpicture}[x=0.75pt,y=0.75pt,yscale=-1,xscale=1]
%uncomment if require: \path (0,300); %set diagram left start at 0, and has height of 300

%Shape: Square [id:dp5812640844481787] 
\draw   (206,108) -- (232,108) -- (232,134) -- (206,134) -- cycle ;
%Straight Lines [id:da23406010312221892] 
\draw    (206,134) -- (232,108) ;

%Straight Lines [id:da12470247976338489] 
\draw    (105,121) -- (219,121) ;
\draw [shift={(162,121)}, rotate = 180] [fill={rgb, 255:red, 0; green, 0; blue, 0 }  ][line width=0.08]  [draw opacity=0] (12,-3) -- (0,0) -- (12,3) -- cycle    ;
%Straight Lines [id:da8231064877854108] 
\draw    (219,121) -- (333,121) ;
%Straight Lines [id:da03344738041998285] 
\draw    (219,121) -- (219,211.33) ;
\draw [shift={(219,166.17)}, rotate = 90] [fill={rgb, 255:red, 0; green, 0; blue, 0 }  ][line width=0.08]  [draw opacity=0] (12,-3) -- (0,0) -- (12,3) -- cycle    ;
%Straight Lines [id:da3041250062025471] 
\draw    (219,53.33) -- (219,121) ;
\draw [shift={(219,87.17)}, rotate = 90] [fill={rgb, 255:red, 0; green, 0; blue, 0 }  ][line width=0.08]  [draw opacity=0] (12,-3) -- (0,0) -- (12,3) -- cycle    ;
%Shape: Square [id:dp9371797224374869] 
\draw   (320,108) -- (346,108) -- (346,134) -- (320,134) -- cycle ;
%Straight Lines [id:da05948010816759908] 
\draw    (320,134) -- (346,108) ;

%Straight Lines [id:da4093214516440322] 
\draw    (333,121) -- (447,121) ;
\draw [shift={(390,121)}, rotate = 180] [fill={rgb, 255:red, 0; green, 0; blue, 0 }  ][line width=0.08]  [draw opacity=0] (12,-3) -- (0,0) -- (12,3) -- cycle    ;
%Straight Lines [id:da3614482552806686] 
\draw    (333,121) -- (333,211.33) ;
\draw [shift={(333,166.17)}, rotate = 90] [fill={rgb, 255:red, 0; green, 0; blue, 0 }  ][line width=0.08]  [draw opacity=0] (12,-3) -- (0,0) -- (12,3) -- cycle    ;
%Straight Lines [id:da7606039605325201] 
\draw    (333,53.33) -- (333,121) ;
\draw [shift={(333,87.17)}, rotate = 90] [fill={rgb, 255:red, 0; green, 0; blue, 0 }  ][line width=0.08]  [draw opacity=0] (12,-3) -- (0,0) -- (12,3) -- cycle    ;
%Straight Lines [id:da4133297014948891] 
\draw    (219,53.33) -- (274,53.33) ;
%Straight Lines [id:da6242316516385615] 
\draw    (333,53.33) -- (388,53.33) ;
%Shape: Chord [id:dp8695097730726187] 
\draw   (274.11,38.08) .. controls (283.31,38.1) and (290.82,44.7) .. (290.98,53) .. controls (291.14,61.42) and (283.68,68.4) .. (274.3,68.58) -- cycle ;
%Shape: Chord [id:dp8381933152038832] 
\draw   (388.11,38.08) .. controls (397.31,38.1) and (404.82,44.7) .. (404.98,53) .. controls (405.14,61.42) and (397.68,68.4) .. (388.3,68.58) -- cycle ;

% Text Node
\draw (14,98.4) node [anchor=north west][inner sep=0.75pt]    {$\alpha \ket{0} +\beta \ket{1} +\gamma \ket{2}$};
% Text Node
\draw (204,214.4) node [anchor=north west][inner sep=0.75pt]    {$\ket{1}$};
% Text Node
\draw (322,212.4) node [anchor=north west][inner sep=0.75pt]    {$\ket{0}$};


\end{tikzpicture}

    \caption{通过观察实现受控$\pi$相位门(或者说受控符号门)}
    \label{fig:controled-phase-gate-by-obs}
\end{figure}

受控相位门可以通过观察以及条件量子态实现。光路见\autoref{fig:controled-phase-gate-by-obs},计算发现末态为
\begin{equation}
    \ket*{\psi_\text{c}'} = \alpha \ket*{0} + \beta \ket*{1} - \gamma \ket*{2}
\end{equation}

为了更加清楚地展示光路中的各种操作的时间顺序,我们通常会将\autoref{fig:controled-phase-gate-by-obs}这样的光路图进一步抽象,将它画成

\chapter{量子光学测量}

本章我们讨论怎样通过测量一个量子光学态来重建关于它的性质。大部分情况下我们只能够测量光子数期望,然而量子光学中的信息远不止光子数期望——例如,形如
\[
    \ket{n_1} + \ee^{\ii \theta} \ket{n_2}
\]
的态就含有有物理意义的相位信息,而通过测量光子数是测不出来这个相位的。

\section{相位的零差探测}

Homodyne

\begin{figure}
    \centering
    

\tikzset{every picture/.style={line width=0.75pt}} %set default line width to 0.75pt        

\begin{tikzpicture}[x=0.75pt,y=0.75pt,yscale=-1,xscale=1]
%uncomment if require: \path (0,300); %set diagram left start at 0, and has height of 300

%Straight Lines [id:da828543766930488] 
\draw    (228,144) -- (322,144) ;
\draw [shift={(275,144)}, rotate = 180] [fill={rgb, 255:red, 0; green, 0; blue, 0 }  ][line width=0.08]  [draw opacity=0] (12,-3) -- (0,0) -- (12,3) -- cycle    ;
%Shape: Square [id:dp6034028374707234] 
\draw   (335,131) -- (309,131) -- (309,157) -- (335,157) -- cycle ;
%Straight Lines [id:da08340572984807304] 
\draw    (335,157) -- (309,131) ;

%Straight Lines [id:da08559705327208378] 
\draw    (322,58) -- (322,144) ;
\draw [shift={(322,101)}, rotate = 270] [fill={rgb, 255:red, 0; green, 0; blue, 0 }  ][line width=0.08]  [draw opacity=0] (12,-3) -- (0,0) -- (12,3) -- cycle    ;
%Straight Lines [id:da571324279939923] 
\draw    (322,144) -- (427,144) ;
\draw [shift={(374.5,144)}, rotate = 180] [fill={rgb, 255:red, 0; green, 0; blue, 0 }  ][line width=0.08]  [draw opacity=0] (12,-3) -- (0,0) -- (12,3) -- cycle    ;
%Shape: Chord [id:dp5477809121009893] 
\draw   (427.11,128.08) .. controls (436.31,128.1) and (443.82,134.7) .. (443.98,143) .. controls (444.14,151.42) and (436.68,158.4) .. (427.3,158.58) -- cycle ;
%Shape: Rectangle [id:dp5820030129707479] 
\draw   (129,123) -- (228,123) -- (228,163) -- (129,163) -- cycle ;
%Straight Lines [id:da3527058340560272] 
\draw    (322,144) -- (322,230) ;
\draw [shift={(322,187)}, rotate = 270] [fill={rgb, 255:red, 0; green, 0; blue, 0 }  ][line width=0.08]  [draw opacity=0] (12,-3) -- (0,0) -- (12,3) -- cycle    ;

% Text Node
\draw (138,123) node [anchor=north west][inner sep=0.75pt]   [align=left] {Source to be \\evaluated};
% Text Node
\draw (322,54.6) node [anchor=south] [inner sep=0.75pt]    {$\ket{\beta _{0}}$};
% Text Node
\draw (259,120.4) node [anchor=north west][inner sep=0.75pt]    {$b_{1}$};
% Text Node
\draw (324,80.4) node [anchor=north west][inner sep=0.75pt]    {$b_{2}$};
% Text Node
\draw (323,206.4) node [anchor=north west][inner sep=0.75pt]    {$a_{2}$};
% Text Node
\draw (369,120.4) node [anchor=north west][inner sep=0.75pt]    {$a_{1}$};
% Text Node
\draw (448,140.7) node [anchor=west] [inner sep=0.75pt]    {$\langle a^{\dagger } a\rangle $};


\end{tikzpicture}

    \caption{相位测量装置1}
    \label{fig:phase-detection-1}
\end{figure}

\autoref{fig:phase-detection-1}是一种可能的探测方式。我们将待测量的光源的态记作
\begin{equation}
    \ket{\text{source}} = f(b_1^\dagger) \ket{0},
\end{equation}
我们将它和一个相干态$\ket{\beta_0}$通过一个分束器,该分束器的变换矩阵为
\begin{equation}
    S = \pmqty{ t & r \\ -r^* & t }.
\end{equation}
使用海森堡绘景,我们将分束器的入射态
\[
    \ket{\Psi} = \ee^{\beta_0 b_2^\dagger - \beta_0^* b_2} f(b_1^\dagger) \ket{0}
\] 
用时间演化之后的$a$算符重写,得到
\begin{equation}
    \begin{aligned}
        \ket{\Psi} &= \\
        &\eqqcolon \ee^{\beta_0 t a_2^\dagger - \beta_0^* a_2} \ee^{\beta a_1^\dagger - \beta^* a_1} f(b_1^\dagger) \ket{0},
    \end{aligned}
\end{equation}
这里我们定义
\begin{equation}
    \beta = - \beta_0 r.
\end{equation}
如果分束器和$\beta_0$被适当选择,使得
\begin{equation}
    \abs*{\beta} \gg 1, \quad t \ll 1,
\end{equation}
就有
\begin{equation}
    \ket{\Psi} \approx \ee^{\beta a_1^\dagger - \beta^* a_1} f( a_1^\dagger) \ket{0} = D(\beta) f()
\end{equation}

\begin{figure}
    

\tikzset{every picture/.style={line width=0.75pt}} %set default line width to 0.75pt        

\begin{tikzpicture}[x=0.75pt,y=0.75pt,yscale=-1,xscale=1]
%uncomment if require: \path (0,300); %set diagram left start at 0, and has height of 300

%Shape: Square [id:dp48465100337411604] 
\draw   (286,119) -- (260,119) -- (260,145) -- (286,145) -- cycle ;
%Straight Lines [id:da009730264904362462] 
\draw    (286,145) -- (260,119) ;

%Straight Lines [id:da841426330546053] 
\draw    (179,132) -- (273,132) ;
\draw [shift={(226,132)}, rotate = 180] [fill={rgb, 255:red, 0; green, 0; blue, 0 }  ][line width=0.08]  [draw opacity=0] (12,-3) -- (0,0) -- (12,3) -- cycle    ;
%Straight Lines [id:da3980567150516523] 
\draw    (273,46) -- (273,132) ;
\draw [shift={(273,89)}, rotate = 270] [fill={rgb, 255:red, 0; green, 0; blue, 0 }  ][line width=0.08]  [draw opacity=0] (12,-3) -- (0,0) -- (12,3) -- cycle    ;
%Straight Lines [id:da7111057972380808] 
\draw    (273,132) -- (273,218) ;
\draw [shift={(273,175)}, rotate = 270] [fill={rgb, 255:red, 0; green, 0; blue, 0 }  ][line width=0.08]  [draw opacity=0] (12,-3) -- (0,0) -- (12,3) -- cycle    ;
%Straight Lines [id:da12521956138792478] 
\draw    (273,132) -- (392.71,132.95) ;
\draw [shift={(332.85,132.48)}, rotate = 180.46] [fill={rgb, 255:red, 0; green, 0; blue, 0 }  ][line width=0.08]  [draw opacity=0] (12,-3) -- (0,0) -- (12,3) -- cycle    ;
%Shape: Chord [id:dp14603723174788152] 
\draw   (393.11,117.08) .. controls (402.31,117.1) and (409.82,123.7) .. (409.98,132) .. controls (410.14,140.42) and (402.68,147.4) .. (393.3,147.58) -- cycle ;
%Straight Lines [id:da09107409379680376] 
\draw    (273,218) -- (392.71,218.95) ;
\draw [shift={(332.85,218.48)}, rotate = 180.46] [fill={rgb, 255:red, 0; green, 0; blue, 0 }  ][line width=0.08]  [draw opacity=0] (12,-3) -- (0,0) -- (12,3) -- cycle    ;
%Shape: Chord [id:dp158377588748835] 
\draw   (393.11,203.08) .. controls (402.31,203.1) and (409.82,209.7) .. (409.98,218) .. controls (410.14,226.42) and (402.68,233.4) .. (393.3,233.58) -- cycle ;
%Straight Lines [id:da39545292367915774] 
\draw    (409.71,132.95) -- (449.71,132.95) ;
%Straight Lines [id:da29050927155087436] 
\draw    (409.71,218.95) -- (449.71,218.95) ;
%Straight Lines [id:da4259702929343543] 
\draw    (449.71,132.95) -- (449.68,162.95) ;
%Straight Lines [id:da376644991684453] 
\draw    (449.68,191.95) -- (449.71,218.95) ;
%Shape: Circle [id:dp47795417904392656] 
\draw   (435.18,177.45) .. controls (435.18,169.45) and (441.67,162.95) .. (449.68,162.95) .. controls (457.69,162.95) and (464.18,169.45) .. (464.18,177.45) .. controls (464.18,185.46) and (457.69,191.95) .. (449.68,191.95) .. controls (441.67,191.95) and (435.18,185.46) .. (435.18,177.45) -- cycle ;
%Straight Lines [id:da9614966812921062] 
\draw    (464.18,177.45) -- (494.71,177.45) ;
%Shape: Rectangle [id:dp9974767092600316] 
\draw   (80,113) -- (179,113) -- (179,153) -- (80,153) -- cycle ;

% Text Node
\draw (279,32.4) node [anchor=north west][inner sep=0.75pt]    {$\mathrm{e}^{\alpha_{2} a^\dagger_2 - \alpha^*_2 a_2}$};
% Text Node
\draw (324.5,103.4) node [anchor=north west][inner sep=0.75pt]    {$b_{1}^{\dagger }$};
% Text Node
\draw (324.5,188.4) node [anchor=north west][inner sep=0.75pt]    {$b_{2}^{\dagger }$};
% Text Node
\draw (449.68,177.45) node    {$-$};
% Text Node
\draw (388,92.4) node [anchor=north west][inner sep=0.75pt]    {$\expval{b_1^\dagger b_1}$};
% Text Node
\draw (389,235.4) node [anchor=north west][inner sep=0.75pt]    {$\expval{b_2^\dagger b_2}$};
% Text Node
\draw (89,113) node [anchor=north west][inner sep=0.75pt]   [align=left] {Source to be \\evaluated};
% Text Node
\draw (219,102.4) node [anchor=north west][inner sep=0.75pt]    {$a_{1}^{\dagger }$};


\end{tikzpicture}

    \caption{相位测量装置2}
\end{figure}

例如,如果入射态是一个相干态$\ket{\alpha}$,计算得到
\begin{equation}
    \bar{n}_{12} = \abs*{\alpha \alpha_0} \cos \Delta \varphi.
\end{equation}

我们现在分析误差、噪声等引入的不确定度。

\begin{equation}
    \var{\Delta\varphi} = \frac{\var{n_{12}}}{\abs{\pdv{n_{12}}{\Delta\varphi}}} = \frac{\abs*{\alpha}}{\abs*{\alpha} \abs*{\alpha_0} \sin \Delta \varphi} = \frac{1}{\abs*{\alpha_0} \sin \Delta \Delta \varphi},
\end{equation}
因此我们有
\begin{equation}
    \var{\Delta \varphi} \geq \frac{1}{\abs*{\alpha_0}} = \frac{1}{\sqrt{\bar{n}}}.
\end{equation}
这个结论称为\concept{标准量子极限}。

Heterodyne

\section{量子层析}

从不同角度测量一个Wigner函数,然后用这些信息“粘”出完整的Winger函数。
因此,只要我们能够确定某个光源总是产生一样的量子光学态,那么我们是可以通过多次测定来(差一个相位地)确定它产生地量子光学态。
这个过程和CT很类似,从而称为\concept{量子层析}。

\bibliographystyle{plain}
\bibliography{optics,../formalism/classical-feyn,quantum-state} 

\end{document}