\documentclass[hyperref, a4paper]{article}

\usepackage{geometry}
\usepackage{titling}
\usepackage{titlesec}
% No longer needed, since we will use enumitem package
% \usepackage{paralist}
\usepackage{enumitem}
\usepackage{footnote}
\usepackage{enumerate}
\usepackage{amsmath, amssymb, amsthm}
\usepackage{mathtools}
\usepackage{bbm}
\usepackage{cite}
\usepackage{graphicx}
\usepackage{subcaption}
\usepackage{physics}
\usepackage{tensor}
\usepackage{siunitx}
\usepackage[version=4]{mhchem}
\usepackage{tikz}
\usepackage{xcolor}
\usepackage{listings}
\usepackage{autobreak}
\usepackage[ruled, vlined, linesnumbered]{algorithm2e}
\usepackage{nameref,zref-xr}
\zxrsetup{toltxlabel}
\zexternaldocument*[last-]{12-2}[12-2.pdf]
\usepackage[colorlinks,unicode]{hyperref} % , linkcolor=black, anchorcolor=black, citecolor=black, urlcolor=black, filecolor=black
\usepackage[most]{tcolorbox}
\usepackage{prettyref}

% Page style
\geometry{left=3.18cm,right=3.18cm,top=2.54cm,bottom=2.54cm}
\titlespacing{\paragraph}{0pt}{1pt}{10pt}[20pt]
\setlength{\droptitle}{-5em}
\preauthor{\vspace{-10pt}\begin{center}}
\postauthor{\par\end{center}}

% More compact lists 
\setlist[itemize]{
    itemindent=17pt, 
    leftmargin=1pt,
    listparindent=\parindent,
    parsep=0pt,
}

% Math operators
\DeclareMathOperator{\timeorder}{\mathcal{T}}
\DeclareMathOperator{\diag}{diag}
\DeclareMathOperator{\legpoly}{P}
\DeclareMathOperator{\primevalue}{P}
\DeclareMathOperator{\sgn}{sgn}
\newcommand*{\ii}{\mathrm{i}}
\newcommand*{\ee}{\mathrm{e}}
\newcommand*{\const}{\mathrm{const}}
\newcommand*{\suchthat}{\quad \text{s.t.} \quad}
\newcommand*{\argmin}{\arg\min}
\newcommand*{\argmax}{\arg\max}
\newcommand*{\normalorder}[1]{: #1 :}
\newcommand*{\pair}[1]{\langle #1 \rangle}
\newcommand*{\fd}[1]{\mathcal{D} #1}
\DeclareMathOperator{\bigO}{\mathcal{O}}

% TikZ setting
\usetikzlibrary{arrows,shapes,positioning}
\usetikzlibrary{arrows.meta}
\usetikzlibrary{decorations.markings}
\tikzstyle arrowstyle=[scale=1]
\tikzstyle directed=[postaction={decorate,decoration={markings,
    mark=at position .5 with {\arrow[arrowstyle]{stealth}}}}]
\tikzstyle ray=[directed, thick]
\tikzstyle dot=[anchor=base,fill,circle,inner sep=1pt]

% Algorithm setting
% Julia-style code
\SetKwIF{If}{ElseIf}{Else}{if}{}{elseif}{else}{end}
\SetKwFor{For}{for}{}{end}
\SetKwFor{While}{while}{}{end}
\SetKwProg{Function}{function}{}{end}
\SetArgSty{textnormal}

\newcommand*{\concept}[1]{{\textbf{#1}}}

% Embedded codes
\lstset{basicstyle=\ttfamily,
  showstringspaces=false,
  commentstyle=\color{gray},
  keywordstyle=\color{blue}
}

% Reference formatting
\newrefformat{fig}{Figure~\ref{#1} on page~\pageref{#1}}

% Color boxes
\tcbuselibrary{skins, breakable, theorems}
\newtcbtheorem[number within=section]{warning}{Warning}%
  {colback=orange!5,colframe=orange!65,fonttitle=\bfseries, breakable}{warn}
\newtcbtheorem[number within=section]{note}{Note}%
  {colback=green!5,colframe=green!65,fonttitle=\bfseries, breakable}{note}

\newcommand{\lastnote}{\href{12-9.pdf}{the last lecture}}

\title{Quantum Optics by Prof. Saijun Wu}
\author{Jinyuan Wu}

\begin{document}

\maketitle

\section{Master equation}

We consider a open quantum system, whose Hamiltonian is $H$, and at each time step, there is a probability 
of quantum jump to several given states. Suppose the probability of jumping to state $\ket*{i}$ is $\Gamma_i$,
we define 
\begin{equation}
    C_i = \sqrt{\Gamma_i} \dyad{i},
\end{equation} 
and the system can be described using a stochastic wave function method shown in previous lectures.
If we use a density matrix formalism, we find 
\[
    \rho(t + \Delta t) = \sum_i \Gamma_i \Delta t \dyad{i} + (1 - \sum_i \Gamma_i) \dyad{\psi_\text{s}(t + \Delta t)},
\]
where $\ket*{\psi_\text{s}}$ evolves according to $H$. We therefore find 
\begin{equation}
    \dot{\rho} = \frac{1}{\ii \hbar} \comm*{H_\text{eff}}{\rho} + \sum_i C_i \rho C_i^\dagger,
    \label{eq:master-eq}
\end{equation}
where 
\begin{equation}
    H_\text{eff} = H - \frac{\ii \hbar}{2} \sum_i C_i^\dagger C_i.
\end{equation}
\eqref{eq:master-eq} is called \concept{the master equation in Lindblad form}.

We want to check the unitarity of \eqref{eq:master-eq}. We have 
\begin{equation}
    \trace \dot{\rho} = - \sum_j \trace \left( \frac{1}{2} C_i^\dagger C_i \rho + \frac{1}{2} \rho C_i^\dagger C_i - C_i \rho C_i^\dagger \right).
\end{equation}
The last term is called the \concept{recycling term}, which makes the total probability increase, while the first 
two terms make the total probability decrease. With trace cyclic property, we find the total probability 
is conserved.

We consider a light-atom interacting system with RWA, where 
\begin{equation}
    H = \frac{\hbar}{2} \vb*{\Omega} \cdot \vb*{\sigma}, \quad C = \sqrt{\Gamma} \dyad{g}{e},
\end{equation} 
and we have 
\begin{equation}
    \begin{aligned}
        \dot{\rho}_{gg} &= \frac{\ii \Omega}{2} \rho_{ge} - \frac{\ii \Omega^*}{2} \rho_{eg} + \Gamma \rho_{ee}, \\
        \dot{\rho}_{ee} &= - \dot{\rho}_{gg}, \\
        \dot{\rho}_{ge} &= \left( - \frac{\Gamma}{2} + \ii \Delta \right) \rho_{ge} - \frac{\ii \Omega}{2} (\rho_{ee} - \rho_{gg}),
    \end{aligned}
\end{equation}
where 
\begin{equation}
    \rho_{ee} + \rho_{gg} = 1, \quad \rho_{eg} = \rho_{ge}^*.
\end{equation}
We define 
\begin{equation}
    \expval*{\sigma^x} = \Re \rho_{eg} \eqqcolon u, \quad \expval*{\sigma^y} = \Im \rho_{eg} \eqqcolon v,
    \expval*{\sigma^z} = \rho_{ee} - \rho_{gg} \eqqcolon w,
\end{equation}
and 
\begin{equation}
    \vb*{n} = (\expval*{\sigma^x}, \expval*{\sigma^y}, \expval*{\sigma^z}),
\end{equation}
and we have 
\begin{equation}
    \dot{\vb*{n}} = \vb*{\Omega} \times \vb*{n} - \pmqty{ \gamma_\text{T} u \\ \gamma_\text{T} v \\ \gamma_\text{L} (w+1) },
    \label{eq:optical-bloch}
\end{equation}
where 
\begin{equation}
    \gamma_\text{T} = \frac{\Gamma}{2}
\end{equation}
is called the \concept{transverse damping rate} and 
\begin{equation}
    \gamma_\text{L} = \Gamma
\end{equation}
is called the \concept{longitude relaxing rate}. \eqref{eq:optical-bloch} is called the 
\concept{optical Bloch equation}.

Now we try to find a stable solution of \eqref{eq:optical-bloch}. It is 
\begin{equation}
    \rho_{ee}^\text{stable} = \frac{(\Omega / \Gamma)^2 }{1 + 2 (\Omega / \Gamma)^2 + 4 (\Delta / \Gamma)^2},
\end{equation}
and 
\begin{equation}
    \rho_{ge}^\text{stable} = - \frac{\Omega / 2}{\Delta - \ii \Gamma / 2} (2 \rho_{ee} - 1).
\end{equation}
We define 
\begin{equation}
    S = 2 \left(\frac{\Omega}{\Gamma}\right)^2.
\end{equation}
We can also evaluate the response of the electric dipole. We have 
\begin{equation}
    \expval*{d} = \rho_{eg} d_{ge} + \text{h.c.} = \alpha E + \text{c.c.},
    \quad \alpha = \frac{1}{1 + S + 4 (\Delta / \Gamma)^2 } \left( \frac{2 \Delta}{\Gamma} + \ii \right) 
    \frac{3 \lambda^3}{4 \pi^2} \epsilon_0 \eqqcolon \alpha(I).
\end{equation}

saturated absorption, Saturated absorption spectroscopy 
\concept{Lamb dip} 

\section{Rate equation} 

We choose a \emph{adiabatic} basis, which are dressed states of $H_\text{eff}$. In this basis, assuming that 
the non-diagonal elements of the density matrix damp quickly enough, we have 
\begin{equation}
    \dot{\rho}_{nn} = - \gamma_{n} \rho_{nn} + \sum_{m \neq n} \gamma_{nm} \rho_{mm}, \quad \rho_{mn}|_{m \neq n} = 0,
\end{equation} 
which is called the \concept{rate equation}. 

Again for a two-level system where RWA works, we have 
\begin{equation}
    \gamma_{\tilde{g}} = \Gamma \sin^2 \theta , \quad \gamma_{\tilde{e}} = \Gamma \cos^2 \theta,
    \gamma_{\tilde{g} \tilde{e}} = \Gamma \sin^4 \theta , \gamma_{\tilde{e} \tilde{g}} = \Gamma \cos^4 \theta,
\end{equation}
and the rate equation is 
\begin{equation}
    \dot{\rho}_{\tilde{g} \tilde{g}} = (- \sin^4 \theta \rho_{\tilde{g} \tilde{g}} + \cos^4 \theta \rho_{\tilde{e} \tilde{e}})  \Gamma.
\end{equation}
The stable solution is 
\begin{equation}
    \rho^\text{stable}_{\tilde{g} \tilde{g}} = \frac{\cos^4 \theta}{\cos^4 \theta + \sin^4 \theta}, \quad 
    \rho^\text{stable}_{\tilde{e} \tilde{e}} = \frac{\sin^4 \theta}{\cos^4 \theta + \sin^4 \theta}.
\end{equation}

\section{How atoms move in the space}

All previous discussions were based on the assumption that atoms are somehow ``fixed'' or ``trapped'' at a 
given point. This is of course possible (using laser trap or something), but a more interesting case is when 
atoms are not that constrained. In this case, we need to take the spacial motion of atoms into account.

Consider a stationary mode in a cavity:
\begin{equation}
    E = E_0 \cos(kx) \ee^{\ii \omega t} + \text{c.c.},
\end{equation}
and at each point the energies of the ground state and the excited state of a two-level atom are different.
Since 
\begin{equation}
    H = \frac{\hbar}{2} \vb*{\Omega} \cdot \vb*{\sigma} \propto E,
\end{equation}
we have 
\begin{equation}
    m \expval*{\ddot{\vb*{r}}} = - {\grad \expval*{H}} \propto - \grad{E}.
\end{equation}

Sisyphus cooling

Note: there are some subtleties here. 
\begin{equation}
    \expval*{F} = \Re \frac{\grad \Omega}{\Omega} \alpha_r \abs*{E}^2 + 
    \Im \frac{\grad \Omega}{\Omega} \alpha_i \abs*{E}^2 +
\end{equation}
We find that the conservative force causes cooling, while the scattering force causes heating.

Doppler cooling is another cooling approach
We will find 
\begin{equation}
    \Delta \to \Delta - k v, 
\end{equation}
\begin{equation}
    F = - \beta v
\end{equation}
What Doppler cooling fails to take into account is the Sisyphus cooling or heating mechanism above.

\section{A four-level atom model, and when polarization is important}

\begin{equation}
    H = \hbar \Delta (\dyad{e+} + \dyad{e-}) 
    + \frac{\hbar \Omega_\text{T}}{2} (\dyad*{e+}{g+} + \dyad*{e-}{g-}) + \text{h.c.}
    + \frac{\hbar \Omega_{\sigma-}}{2} (\dyad*{e-}{g+} + \text{h.c.}),
\end{equation}
and the dressed states are 
\begin{equation}
    \ket*{\tilde{g}+} = \ket*{g+} + \frac{\Omega_{\sigma-} / 2}{\Delta} \ket*{e-}, \quad
    \ket*{\tilde{g}-} = \ket*{g-}.
\end{equation} 
We find $\ket*{g-}$ actually has no coupling with the optical field and therefore is not dressed at all.
States like this are often called \concept{dark states}, because if a atom falls on such a state, no radiation
will be seen. It does not mean these states are completely irrelevant, because quantum hopping may drive 
an atom to such a state.

In such a system we can realize \concept{sub-Doppler cooling}, again by Sisyphus cooling. This was first 
found by Steven Chu, who found his cooling device mysteriously worked better than the estimation of Doppler 
cooling.

\end{document}