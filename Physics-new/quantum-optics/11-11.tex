\documentclass[hyperref, a4paper]{article}

\usepackage{geometry}
\usepackage{titling}
\usepackage{titlesec}
% No longer needed, since we will use enumitem package
% \usepackage{paralist}
\usepackage{enumitem}
\usepackage{footnote}
\usepackage{enumerate}
\usepackage{amsmath, amssymb, amsthm}
\usepackage{mathtools}
\usepackage{bbm}
\usepackage{cite}
\usepackage{graphicx}
\usepackage{subfigure}
\usepackage{physics}
\usepackage{tensor}
\usepackage{siunitx}
\usepackage[version=4]{mhchem}
\usepackage{tikz}
\usepackage{xcolor}
\usepackage{listings}
\usepackage{autobreak}
\usepackage[ruled, vlined, linesnumbered]{algorithm2e}
\usepackage{nameref,zref-xr}
\zxrsetup{toltxlabel}
\zexternaldocument*[optics-]{../optics/optics}[optics.pdf]
\zexternaldocument*[solid-]{../solid/solid}[solid.pdf]
\zexternaldocument*[info-]{../information/quantum-circuit}[quantum-circuit.pdf]
\usepackage[colorlinks,unicode]{hyperref} % , linkcolor=black, anchorcolor=black, citecolor=black, urlcolor=black, filecolor=black
\usepackage{prettyref}

% Page style
\geometry{left=3.18cm,right=3.18cm,top=2.54cm,bottom=2.54cm}
\titlespacing{\paragraph}{0pt}{1pt}{10pt}[20pt]
\setlength{\droptitle}{-5em}
\preauthor{\vspace{-10pt}\begin{center}}
\postauthor{\par\end{center}}

% More compact lists 
\setlist[itemize]{
    itemindent=17pt, 
    leftmargin=1pt,
    listparindent=\parindent,
    parsep=0pt,
}

% Math operators
\DeclareMathOperator{\timeorder}{\mathcal{T}}
\DeclareMathOperator{\diag}{diag}
\DeclareMathOperator{\legpoly}{P}
\DeclareMathOperator{\primevalue}{P}
\DeclareMathOperator{\sgn}{sgn}
\newcommand*{\ii}{\mathrm{i}}
\newcommand*{\ee}{\mathrm{e}}
\newcommand*{\const}{\mathrm{const}}
\newcommand*{\suchthat}{\quad \text{s.t.} \quad}
\newcommand*{\argmin}{\arg\min}
\newcommand*{\argmax}{\arg\max}
\newcommand*{\normalorder}[1]{: #1 :}
\newcommand*{\pair}[1]{\langle #1 \rangle}
\newcommand*{\fd}[1]{\mathcal{D} #1}
\DeclareMathOperator{\bigO}{\mathcal{O}}

% TikZ setting
\usetikzlibrary{arrows,shapes,positioning}
\usetikzlibrary{arrows.meta}
\usetikzlibrary{decorations.markings}
\tikzstyle arrowstyle=[scale=1]
\tikzstyle directed=[postaction={decorate,decoration={markings,
    mark=at position .5 with {\arrow[arrowstyle]{stealth}}}}]
\tikzstyle ray=[directed, thick]
\tikzstyle dot=[anchor=base,fill,circle,inner sep=1pt]

% Algorithm setting
% Julia-style code
\SetKwIF{If}{ElseIf}{Else}{if}{}{elseif}{else}{end}
\SetKwFor{For}{for}{}{end}
\SetKwFor{While}{while}{}{end}
\SetKwProg{Function}{function}{}{end}
\SetArgSty{textnormal}

\newcommand*{\concept}[1]{{\textbf{#1}}}

% Embedded codes
\lstset{basicstyle=\ttfamily,
  showstringspaces=false,
  commentstyle=\color{gray},
  keywordstyle=\color{blue}
}

\newcommand{\opticsdoc}{\href{../optics/optics}{the optics note}}
\newcommand{\soliddoc}{\href{../solid/solid}{the solid state physics note}}
\newcommand{\infodoc}{\href{../information/quantum-circuit}{the quantum information note}}

\newrefformat{fig}{Figure~\ref{#1} on page~\pageref{#1}}
\newrefformat{sec}{Section~\ref{#1}}

\title{Quantum Optics by Prof. Saijun Wu}
\author{Jinyuan Wu}
\date{November 11, 2021}

\begin{document}

\maketitle

\section{Review of the quantization of the optical field}

The electric field is 
\begin{equation}
    \vb*{E}(\vb*{r}, t) = \vb*{E}^+(\vb*{r}, t) + \vb*{E}^-(\vb*{r}, t),
\end{equation}
and we make the mode expansion 
\begin{equation}
    \vb*{E}^+ = \sum_{k} \mathcal{E}_k \vb*{f}_k(\vb*{r}) \ee^{- \ii \omega_k t} a_k, 
    \quad \vb*{E}^- = \sum_{k} \mathcal{E}_k \vb*{f}^*_k(\vb*{r}) \ee^{\ii \omega_k t} a^\dagger_k,
\end{equation}
where 
\begin{equation}
    \mathcal{E} = \sqrt{\frac{\hbar \omega_k}{2 \epsilon_0 V}}.
\end{equation}
It can be verified that the following quantization scheme 
\begin{equation}
    \comm*{a_k}{a^\dagger_{k'}} = \delta_{k k'}, \quad \text{other commutators of $a, a^\dagger$} = 0,
\end{equation}
agrees with QED. The normalization condition for $\vb*{f}_k$ is 
\begin{equation}
    \int \dd[3]{\vb*{r}} \vb*{f}^*_k \cdot \frac{\vb*{\epsilon}}{\epsilon_0} \cdot \vb*{f}_k = V \delta_{k k'}.
\end{equation}

It can then be verified that for a single-mode multi-photon state $\ket*{n}$, we have 
\begin{equation}
    \expval*{\vb*{E}} = 0, \quad \expval*{\vb*{E}(\vb*{r}, t)^2}
\end{equation}
A thermal state is given by
\begin{equation}
    \rho_\text{thermal} = \prod_k \rho^k_\text{thermal}, \quad 
    \rho^k_\text{thermal} = Z(k) \ee^{- \beta \hbar \omega_k} a^\dagger_k a_k = \sum_n \frac{\bar{n}_k^n}{(1 + \bar{n}_k)^{n+1}}, 
    \quad \bar{n}_k = \frac{1}{\ee^{\beta \hbar \omega_k} - 1}.
\end{equation}

Thermal states are hard to manipulate and what are often used in laboratories are the ``most classical'' state in quantum optics - coherent states.
They also appear in path integrals, which is quite reasonable.
The definition of a coherent state $\ket*{\alpha}$ is 
\begin{equation}
    \ket*{\alpha} = \ee^{\alpha a^\dagger - \alpha^* a} \ket*{0} .
\end{equation}
They can be verified to be eigenstates of the annihilation operator and 
\begin{equation}
    a \ket*{\alpha} = \alpha \ket*{\alpha}.
\end{equation}
We have 
\begin{equation}
    \ket*{\alpha} = \ee^{- \abs*{\alpha}^2 / 2} \frac{\alpha^n}{n!} (a^\dagger)^n \ket*{0} 
    = \ee^{- \abs*{\alpha} / 2} \frac{\alpha^n}{\sqrt{n!}} \ket*{n},
\end{equation}
and therefore 
\begin{equation}
    \abs*{\braket*{n}{\alpha}}^2 = \ee^{- \abs*{\alpha}^2} \frac{\abs*{\alpha}^{2n}}{n!}.
\end{equation}

If we have a normal-ordered function $f(a, a^\dagger)$, i.e. all $a^\dagger$'s are placed left to all $a$'s, 
we have 
\begin{equation}
    \expval*{f(a, a^\dagger)}{\alpha} = f(\alpha, \alpha^*),
\end{equation}
which seems quite classical. Note, however, that we still have 
\[
    \expval*{f^2} \neq \expval*{f}^2,
\]
and that to obtain a normal ordered function we still need to calculate all the commutators.

Since coherent states are so intuitive we may want to connect it to every state of the optical field.
We introduce 
\begin{equation}
    X = \frac{1}{\sqrt{2}} (a + a^\dagger), \quad P = \frac{1}{\sqrt{2} \ii} (a - a^\dagger)
\end{equation}
and define 
\begin{equation}
    W(X, P) = \int \frac{\dd{x'} \dd{p'}}{2\pi} \expval*{\ee^{\ii p' (\hat{X} - X) - \ii x'(\hat{P} - P)}}. 
\end{equation}

Now we discuss observation - or ``detection'' - in quantum optics. Ideally according to Fermi golden rule we have 
\[
    P^{(n)}(\vb*{r}_1 t_1, \vb*{r}_2 t_2, \ldots, \vb*{r}_n t_n) 
    \sim \expval*{E^-(\vb*{r}_1, t_1) E^-(\vb*{r}_2, t_2) \cdots E^-(\vb*{r}_n, t_n) E^+(\vb*{r}_n, t_n) \cdots E^+(\vb*{r}_1, t_1)}.
\]
TODO: whether observation effects are taken into account 

For a Fock state, we have 
\begin{equation}
    g^{(2)} = 1 - \frac{1}{N_\text{photon}},
    \label{eq:fock-2-coherence}
\end{equation}
and for a coherent state, we have 
\begin{equation}
    g^{(2)} = 1,
    \label{eq:coherent-2-coherence}
\end{equation}
while for a thermal state, we have 
\begin{equation}
    g^{(2)} = 2.
    \label{eq:thermal-2-coherence}
\end{equation}
The last equation is quite confusing. It is understandable that photons are nonrenewable resource in a Fock state so we have \eqref{eq:fock-2-coherence}, and that the optical field is very ``classical'' so we have \eqref{eq:coherent-2-coherence}, 
but it is weird that observing one photon at a place \emph{increases} the probability to obtain another photon at the same site.
\eqref{eq:thermal-2-coherence} actually comes from the statistical properties of bosons: Bosons tend to bunch.

We can also perform a time-dependent measurement. 

\section{Linear quantum optical circuit}

We usually use Heisenberg picture in linear quantum optics. A process can then be described by 
\begin{equation}
    b_l^\dagger = S_{lk} a^\dagger_k,
    \label{eq:linear-transformation}
\end{equation}
where $a^\dagger_k$'s are creation operators of the input modes and $b^\dagger_l$'s are creation operators of the output modes.
$S$ is called the transformation matrix.
\eqref{eq:linear-transformation} can also be seen as the boundary condition around the optical component.
Multiplication of transformation matrices can be seen as multiplication of two time evolution operators as well as obtaining an effective transformation matrix like the case in electric circuit analysis.

\begin{figure}
    \centering
    \begin{tikzpicture}[x=0.75pt,y=0.75pt,yscale=-1,xscale=1]
    %uncomment if require: \path (0,300); %set diagram left start at 0, and has height of 300
    
    %Shape: Square [id:dp735452643736866] 
    \draw   (209,145) -- (183,145) -- (183,171) -- (209,171) -- cycle ;
    %Straight Lines [id:da4347060971974699] 
    \draw    (209,171) -- (183,145) ;
    
    %Straight Lines [id:da969366212911055] 
    \draw    (106,158) -- (196,158) ;
    \draw [shift={(151,158)}, rotate = 180] [fill={rgb, 255:red, 0; green, 0; blue, 0 }  ][line width=0.08]  [draw opacity=0] (12,-3) -- (0,0) -- (12,3) -- cycle    ;
    %Straight Lines [id:da04015524552837446] 
    \draw    (196,158) -- (286,158) ;
    \draw [shift={(241,158)}, rotate = 180] [fill={rgb, 255:red, 0; green, 0; blue, 0 }  ][line width=0.08]  [draw opacity=0] (12,-3) -- (0,0) -- (12,3) -- cycle    ;
    %Straight Lines [id:da36430760434590326] 
    \draw    (196,82.98) -- (196,158) ;
    \draw [shift={(196,120.49)}, rotate = 270] [fill={rgb, 255:red, 0; green, 0; blue, 0 }  ][line width=0.08]  [draw opacity=0] (12,-3) -- (0,0) -- (12,3) -- cycle    ;
    %Straight Lines [id:da2657274096314084] 
    \draw    (196,158) -- (196,233.02) ;
    \draw [shift={(196,195.51)}, rotate = 270] [fill={rgb, 255:red, 0; green, 0; blue, 0 }  ][line width=0.08]  [draw opacity=0] (12,-3) -- (0,0) -- (12,3) -- cycle    ;
    
    % Text Node
    \draw (104,158) node [anchor=east] [inner sep=0.75pt]    {$f\left( a_{1}^{\dagger }\right)$};
    % Text Node
    \draw (196,79.58) node [anchor=south] [inner sep=0.75pt]    {$D_{2}( \alpha )$};
    % Text Node
    \draw (116,132.4) node [anchor=north west][inner sep=0.75pt]    {$a_{1}^{\dagger }$};
    % Text Node
    \draw (204,96.4) node [anchor=north west][inner sep=0.75pt]    {$a_{2}^{\dagger }$};
    % Text Node
    \draw (242,132.4) node [anchor=north west][inner sep=0.75pt]    {$b_{1}^{\dagger }$};
    % Text Node
    \draw (206,201.4) node [anchor=north west][inner sep=0.75pt]    {$b_{2}^{\dagger }$};
    
    
    \end{tikzpicture}    
    \caption{Optical circuit transforming one coherent state into another}
    \label{fig:coherent-move}
\end{figure}

Consider the linear optical circuit shown in \prettyref{fig:coherent-move}.
The transformation matrix is 
\begin{equation}
    \pmqty{r}
\end{equation}
\[
    \ket*{\psi} = D_2(\alpha) f(a_1^\dagger) \ket*{0},
\]

We define 
\begin{equation}
    X_\beta = \frac{1}{\abs*{\beta}} (\beta^* a + \beta a^\dagger),
\end{equation}
and a beam splitter with a very small reflective coefficient 

\end{document}