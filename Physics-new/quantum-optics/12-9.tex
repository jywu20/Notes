\documentclass[hyperref, a4paper]{article}

\usepackage{geometry}
\usepackage{titling}
\usepackage{titlesec}
% No longer needed, since we will use enumitem package
% \usepackage{paralist}
\usepackage{enumitem}
\usepackage{footnote}
\usepackage{enumerate}
\usepackage{amsmath, amssymb, amsthm}
\usepackage{mathtools}
\usepackage{bbm}
\usepackage{cite}
\usepackage{graphicx}
\usepackage{subcaption}
\usepackage{physics}
\usepackage{tensor}
\usepackage{siunitx}
\usepackage[version=4]{mhchem}
\usepackage{tikz}
\usepackage{xcolor}
\usepackage{listings}
\usepackage{autobreak}
\usepackage[ruled, vlined, linesnumbered]{algorithm2e}
\usepackage{nameref,zref-xr}
\zxrsetup{toltxlabel}
\zexternaldocument*[last-]{12-2}[12-2.pdf]
\usepackage[colorlinks,unicode]{hyperref} % , linkcolor=black, anchorcolor=black, citecolor=black, urlcolor=black, filecolor=black
\usepackage[most]{tcolorbox}
\usepackage{prettyref}

% Page style
\geometry{left=3.18cm,right=3.18cm,top=2.54cm,bottom=2.54cm}
\titlespacing{\paragraph}{0pt}{1pt}{10pt}[20pt]
\setlength{\droptitle}{-5em}
\preauthor{\vspace{-10pt}\begin{center}}
\postauthor{\par\end{center}}

% More compact lists 
\setlist[itemize]{
    itemindent=17pt, 
    leftmargin=1pt,
    listparindent=\parindent,
    parsep=0pt,
}

% Math operators
\DeclareMathOperator{\timeorder}{\mathcal{T}}
\DeclareMathOperator{\diag}{diag}
\DeclareMathOperator{\legpoly}{P}
\DeclareMathOperator{\primevalue}{P}
\DeclareMathOperator{\sgn}{sgn}
\newcommand*{\ii}{\mathrm{i}}
\newcommand*{\ee}{\mathrm{e}}
\newcommand*{\const}{\mathrm{const}}
\newcommand*{\suchthat}{\quad \text{s.t.} \quad}
\newcommand*{\argmin}{\arg\min}
\newcommand*{\argmax}{\arg\max}
\newcommand*{\normalorder}[1]{: #1 :}
\newcommand*{\pair}[1]{\langle #1 \rangle}
\newcommand*{\fd}[1]{\mathcal{D} #1}
\DeclareMathOperator{\bigO}{\mathcal{O}}

% TikZ setting
\usetikzlibrary{arrows,shapes,positioning}
\usetikzlibrary{arrows.meta}
\usetikzlibrary{decorations.markings}
\tikzstyle arrowstyle=[scale=1]
\tikzstyle directed=[postaction={decorate,decoration={markings,
    mark=at position .5 with {\arrow[arrowstyle]{stealth}}}}]
\tikzstyle ray=[directed, thick]
\tikzstyle dot=[anchor=base,fill,circle,inner sep=1pt]

% Algorithm setting
% Julia-style code
\SetKwIF{If}{ElseIf}{Else}{if}{}{elseif}{else}{end}
\SetKwFor{For}{for}{}{end}
\SetKwFor{While}{while}{}{end}
\SetKwProg{Function}{function}{}{end}
\SetArgSty{textnormal}

\newcommand*{\concept}[1]{{\textbf{#1}}}

% Embedded codes
\lstset{basicstyle=\ttfamily,
  showstringspaces=false,
  commentstyle=\color{gray},
  keywordstyle=\color{blue}
}

% Reference formatting
\newrefformat{fig}{Figure~\ref{#1} on page~\pageref{#1}}

% Color boxes
\tcbuselibrary{skins, breakable, theorems}
\newtcbtheorem[number within=section]{warning}{Warning}%
  {colback=orange!5,colframe=orange!65,fonttitle=\bfseries, breakable}{warn}
\newtcbtheorem[number within=section]{note}{Note}%
  {colback=green!5,colframe=green!65,fonttitle=\bfseries, breakable}{note}

\newcommand{\lastnote}{\href{12-2.pdf}{the last lecture}}

\title{Quantum Optics by Prof. Saijun Wu}
\author{Jinyuan Wu}

\begin{document}

\maketitle

We consider the following non-Hermitian effective Hamiltonian:
\begin{equation}
    \tilde{H}_\text{eff} = \left( H_\text{SE} - \sum_e \ii \hbar \frac{\Gamma_e}{2} \dyad{e}  \right) \dyad{0}
    + \hbar \sum_{e} g^*_\text{eg,k} \ee^{- \ii \Delta_k t} a_k^\dagger \dyad{g, 0}{e, 0} .
\end{equation}
We find 

\section{Random wave function approach to a three-level system}

\begin{figure}
    \centering
    \tikzset{every picture/.style={line width=0.75pt}} %set default line width to 0.75pt        

\begin{tikzpicture}[x=0.75pt,y=0.75pt,yscale=-1,xscale=1]
%uncomment if require: \path (0,300); %set diagram left start at 0, and has height of 300

%Straight Lines [id:da44377087514798186] 
\draw    (252,210) -- (328.02,210) ;
%Straight Lines [id:da5143457632085762] 
\draw    (186,116) -- (262.02,116) ;
%Straight Lines [id:da8426810677907874] 
\draw    (137,186) -- (213.02,186) ;
%Straight Lines [id:da7715478621282204] 
\draw    (224.01,116) .. controls (226.33,116.41) and (227.29,117.77) .. (226.88,120.09) .. controls (226.48,122.41) and (227.44,123.77) .. (229.76,124.18) .. controls (232.08,124.59) and (233.04,125.96) .. (232.63,128.28) .. controls (232.22,130.6) and (233.18,131.96) .. (235.5,132.37) .. controls (237.82,132.78) and (238.78,134.14) .. (238.37,136.46) .. controls (237.97,138.78) and (238.93,140.14) .. (241.25,140.55) .. controls (243.57,140.96) and (244.53,142.32) .. (244.12,144.64) .. controls (243.71,146.96) and (244.67,148.33) .. (246.99,148.74) .. controls (249.31,149.15) and (250.27,150.51) .. (249.87,152.83) .. controls (249.46,155.15) and (250.42,156.51) .. (252.74,156.92) .. controls (255.06,157.33) and (256.02,158.69) .. (255.61,161.01) .. controls (255.21,163.33) and (256.17,164.69) .. (258.49,165.1) .. controls (260.81,165.51) and (261.77,166.88) .. (261.36,169.2) .. controls (260.95,171.52) and (261.91,172.88) .. (264.23,173.29) .. controls (266.55,173.7) and (267.51,175.06) .. (267.11,177.38) .. controls (266.7,179.7) and (267.66,181.06) .. (269.98,181.47) .. controls (272.3,181.88) and (273.26,183.25) .. (272.85,185.57) .. controls (272.45,187.89) and (273.41,189.25) .. (275.73,189.66) .. controls (278.05,190.07) and (279.01,191.43) .. (278.6,193.75) .. controls (278.19,196.07) and (279.15,197.43) .. (281.47,197.84) -- (284.26,201.82) -- (288.86,208.36) ;
\draw [shift={(290.01,210)}, rotate = 234.93] [fill={rgb, 255:red, 0; green, 0; blue, 0 }  ][line width=0.08]  [draw opacity=0] (12,-3) -- (0,0) -- (12,3) -- cycle    ;
%Straight Lines [id:da685105054346147] 
\draw    (224.01,116) .. controls (224.42,118.32) and (223.46,119.69) .. (221.14,120.1) .. controls (218.82,120.51) and (217.86,121.87) .. (218.27,124.19) .. controls (218.68,126.51) and (217.73,127.88) .. (215.41,128.29) .. controls (213.09,128.7) and (212.13,130.06) .. (212.54,132.38) .. controls (212.95,134.7) and (211.99,136.07) .. (209.67,136.48) .. controls (207.35,136.89) and (206.4,138.26) .. (206.81,140.58) .. controls (207.22,142.9) and (206.26,144.26) .. (203.94,144.67) .. controls (201.62,145.08) and (200.66,146.45) .. (201.07,148.77) .. controls (201.48,151.09) and (200.52,152.46) .. (198.2,152.87) .. controls (195.88,153.28) and (194.93,154.64) .. (195.34,156.96) .. controls (195.75,159.28) and (194.79,160.65) .. (192.47,161.06) .. controls (190.15,161.47) and (189.19,162.83) .. (189.6,165.15) .. controls (190.01,167.47) and (189.05,168.84) .. (186.73,169.25) .. controls (184.41,169.66) and (183.46,171.03) .. (183.87,173.35) .. controls (184.28,175.67) and (183.32,177.03) .. (181,177.44) -- (180.74,177.81) -- (176.16,184.36) ;
\draw [shift={(175.01,186)}, rotate = 304.99] [fill={rgb, 255:red, 0; green, 0; blue, 0 }  ][line width=0.08]  [draw opacity=0] (12,-3) -- (0,0) -- (12,3) -- cycle    ;

% Text Node
\draw (330.02,210) node [anchor=west] [inner sep=0.75pt]    {$a$};
% Text Node
\draw (264.02,116) node [anchor=west] [inner sep=0.75pt]    {$e$};
% Text Node
\draw (215.02,186) node [anchor=west] [inner sep=0.75pt]    {$g$};
% Text Node
\draw (259.01,159.6) node [anchor=south west] [inner sep=0.75pt]    {$k$};
% Text Node
\draw (197.51,147.6) node [anchor=south east] [inner sep=0.75pt]    {$q$};


\end{tikzpicture}

    \caption{The energy level diagram of a $\Lambda$-type three-level system}
    \label{fig:lambda-system}
\end{figure}

Now we consider three-level systems.
We try to use the technique in Section~\ref{last-sec:stochastic-two-level} in \lastnote{} to describe 
these systems. It should be noted that in Section~\ref{last-sec:stochastic-two-level} in \lastnote{} we
can guess the form of the collapse operator since there are only one damping channel, but in a three-level 
system, we need to determine the collapse operators according to how emitted photons are observed. 
For example, for a $V$-type system, two excited states may jump to one ground state. If we assume that 
observers in the environment can only detect the photon number, then there is only one collapse operator:
\begin{equation}
    C = \sqrt{\Gamma_{eg}} \dyad{g}{e} + \sqrt{\Gamma_{ag}} \dyad{g}{a}.
    \label{eq:only-c}
\end{equation}
If the environment can also see the color of the emitted photon, then we have two collapse operators
\begin{equation}
    C_1 = \sqrt{\Gamma_{eg}} \dyad{g}{e}, \quad C_2 = \sqrt{\Gamma_{ag}} \dyad{g}{a}.
    \label{eq:c1-and-c2}
\end{equation}
\eqref{eq:only-c} and \eqref{eq:c1-and-c2} give different damping Hamiltonian. \eqref{eq:c1-and-c2} gives
\begin{equation}
    H_\text{damping} = - \frac{\ii \hbar}{2} (\Gamma_{eg} \dyad{e} + \Gamma_{ag} \dyad{g}),
\end{equation} 
while \eqref{eq:only-c} adds two cross terms.

A good demonstration of the physical consequence of this fact can be seen as follows.
Suppose there is no quantum jump before $t = t_0$, and we have 
\begin{equation}
    \ket*{\psi_\text{s}(t)} = \frac{1}{C} (\ee^{-\ii \omega_e t - \Gamma_e t / 2} \ket*{e} + \ee^{- \ii \omega_a t - \Gamma_a t/ 2} \ket*{a}), 
    \quad {\abs*{C}^2} = \ee^{- \Gamma_e t } + \ee^{- \Gamma_a t},
\end{equation}
and the probability of quantum jump (i.e. spontaneous radiation) is 
\begin{equation}
    \gamma(t) = \dv{P_\text{emission}}{t} = \frac{1}{\ee^{- \Gamma_e t } + \ee^{- \Gamma_a t}} (\Gamma_e \ee^{- \Gamma_e t} + \Gamma_a \ee^{- \Gamma_a t} + 2 \sqrt{\Gamma_e \Gamma_a} \ee^{- (\Gamma_e + \Gamma_a) t / 2} \cos(\omega_e - \omega_a) t ).
\end{equation}
In the specific case of $\Gamma_a = \Gamma_e = \Gamma$, we have 
\begin{equation}
    \gamma(t) = 1 + \ee^{- \Gamma t} \cos(\Delta \omega t).
\end{equation}
The oscillating form of $\gamma$ is called \concept{quantum beat}. It comes from the fact that the quantum jump
$a \to g$ and $e \to g$ are indistinguishable.

We can also calculate the survival ratio until $t=t_0$, i.e. 
\begin{equation}
    \prod_{j=1}^N (1 - \var{P}_\text{j}(t_j)) \to \ee^{- \int_0^t \gamma(\tau) \dd{\tau}}.
\end{equation}

\section{Linear response in a two-level system}

We consider a two level system with an external field, the Hamiltonian of which is  
\begin{equation}
    H = H_0 + \frac{\hbar}{2} \Omega \dyad{e}{g} + \text{h.c.},
\end{equation}
and we take spontaneous radiation into account, so we work with 
\begin{equation}
    H_\text{eff} = H_0 - \frac{\ii \Gamma_e}{2} \dyad{e} + \frac{\hbar \Omega}{2} \dyad{e}{g} + \text{h.c.}.
\end{equation} 
We again use the random wave function approach, and we have 
\begin{equation}
    \ii \dot{c}_e = \left( \Delta - \frac{\ii \Gamma}{2} \right) c_e + \frac{\Omega}{2} c_g, \quad
    \ii \dot{c}_g = \frac{\Omega^*}{2} c_e.
\end{equation}
We find a stable solution 
\begin{equation}
    c_g \approx 1, \quad c_e = - \frac{\Omega / 2}{\Delta - \ii \Gamma / 2}.
\end{equation}
We therefore find the decay rate is  
\begin{equation}
    \gamma = \mel*{\psi_\text{s}}{C^\dagger C}{\psi_\text{s}} = \frac{\abs*{\Omega}^2}{4 \Delta^2 + \Gamma^2} \Gamma.
\end{equation}
The response of the electric dipole can now be found to be 
\begin{equation}
    \expval*{\vb*{d}}{\psi_\text{s}} = \frac{\vb*{d}_{eg} E \ee^{- \ii \omega t}}{\hbar (\Delta - \ii \Gamma /2)} + \text{h.c.}.
\end{equation}
From the definition of 
\begin{equation}
    E_\text{ext} = E_\text{in} \ee^{- \rho \sigma L / 2},
\end{equation}
we have 
\begin{equation}
    \sigma = k \alpha = \frac{3 \lambda^2}{2 \pi} \frac{\Gamma / 2}{\Delta - \ii \Gamma / 2}.
\end{equation}
It can be found that $\Im \alpha$ is just the scattering rate, which is a result of the optical theorem.

\section{Spontaneous radiation as energy correction}

\begin{figure}
    \centering
    

\tikzset{every picture/.style={line width=0.75pt}} %set default line width to 0.75pt        

\begin{tikzpicture}[x=0.75pt,y=0.75pt,yscale=-1,xscale=1]
%uncomment if require: \path (0,300); %set diagram left start at 0, and has height of 300

%Straight Lines [id:da24445385188714508] 
\draw    (144,27) -- (220.02,27) ;
%Straight Lines [id:da8341755791762298] 
\draw    (187,168) -- (263.02,168) ;
%Straight Lines [id:da3267094757089666] 
\draw    (138,238) -- (214.02,238) ;
%Straight Lines [id:da558948044302779] 
\draw    (182.01,27) .. controls (184.09,28.11) and (184.58,29.7) .. (183.47,31.78) .. controls (182.36,33.86) and (182.85,35.46) .. (184.93,36.57) .. controls (187.01,37.68) and (187.49,39.27) .. (186.38,41.35) .. controls (185.27,43.43) and (185.76,45.02) .. (187.84,46.13) .. controls (189.92,47.24) and (190.41,48.83) .. (189.3,50.91) .. controls (188.19,52.99) and (188.68,54.59) .. (190.76,55.7) .. controls (192.84,56.81) and (193.33,58.4) .. (192.22,60.48) .. controls (191.11,62.56) and (191.6,64.15) .. (193.68,65.26) .. controls (195.76,66.37) and (196.25,67.96) .. (195.14,70.04) .. controls (194.03,72.12) and (194.51,73.72) .. (196.59,74.83) .. controls (198.67,75.94) and (199.16,77.53) .. (198.05,79.61) .. controls (196.94,81.69) and (197.43,83.28) .. (199.51,84.39) .. controls (201.59,85.5) and (202.08,87.09) .. (200.97,89.17) .. controls (199.86,91.25) and (200.35,92.85) .. (202.43,93.96) .. controls (204.51,95.07) and (205,96.66) .. (203.89,98.74) .. controls (202.78,100.82) and (203.27,102.41) .. (205.35,103.52) .. controls (207.43,104.63) and (207.91,106.22) .. (206.8,108.3) .. controls (205.69,110.38) and (206.18,111.98) .. (208.26,113.09) .. controls (210.34,114.2) and (210.83,115.79) .. (209.72,117.87) .. controls (208.61,119.95) and (209.1,121.54) .. (211.18,122.65) .. controls (213.26,123.76) and (213.75,125.35) .. (212.64,127.43) .. controls (211.53,129.51) and (212.02,131.11) .. (214.1,132.22) .. controls (216.18,133.33) and (216.66,134.92) .. (215.55,137) .. controls (214.44,139.08) and (214.93,140.67) .. (217.01,141.78) .. controls (219.09,142.89) and (219.58,144.48) .. (218.47,146.56) .. controls (217.36,148.64) and (217.85,150.24) .. (219.93,151.35) .. controls (222.01,152.46) and (222.5,154.05) .. (221.39,156.13) -- (222.09,158.43) -- (224.43,166.09) ;
\draw [shift={(225.01,168)}, rotate = 253.04] [fill={rgb, 255:red, 0; green, 0; blue, 0 }  ][line width=0.08]  [draw opacity=0] (12,-3) -- (0,0) -- (12,3) -- cycle    ;
%Straight Lines [id:da327675170888881] 
\draw    (225.01,168) .. controls (225.42,170.32) and (224.46,171.69) .. (222.14,172.1) .. controls (219.82,172.51) and (218.86,173.87) .. (219.27,176.19) .. controls (219.68,178.51) and (218.73,179.88) .. (216.41,180.29) .. controls (214.09,180.7) and (213.13,182.06) .. (213.54,184.38) .. controls (213.95,186.7) and (212.99,188.07) .. (210.67,188.48) .. controls (208.35,188.89) and (207.4,190.26) .. (207.81,192.58) .. controls (208.22,194.9) and (207.26,196.26) .. (204.94,196.67) .. controls (202.62,197.08) and (201.66,198.45) .. (202.07,200.77) .. controls (202.48,203.09) and (201.52,204.46) .. (199.2,204.87) .. controls (196.88,205.28) and (195.93,206.64) .. (196.34,208.96) .. controls (196.75,211.28) and (195.79,212.65) .. (193.47,213.06) .. controls (191.15,213.47) and (190.19,214.83) .. (190.6,217.15) .. controls (191.01,219.47) and (190.05,220.84) .. (187.73,221.25) .. controls (185.41,221.66) and (184.46,223.03) .. (184.87,225.35) .. controls (185.28,227.67) and (184.32,229.03) .. (182,229.44) -- (181.74,229.81) -- (177.16,236.36) ;
\draw [shift={(176.01,238)}, rotate = 304.99] [fill={rgb, 255:red, 0; green, 0; blue, 0 }  ][line width=0.08]  [draw opacity=0] (12,-3) -- (0,0) -- (12,3) -- cycle    ;
%Straight Lines [id:da3179271411319158] 
\draw    (83.01,238) -- (145.85,148.04) ;
\draw [shift={(147,146.4)}, rotate = 124.94] [fill={rgb, 255:red, 0; green, 0; blue, 0 }  ][line width=0.08]  [draw opacity=0] (12,-3) -- (0,0) -- (12,3) -- cycle    ;
%Straight Lines [id:da9756847113133453] 
\draw    (147,146.4) -- (116.57,46.7) ;
\draw [shift={(115.99,44.79)}, rotate = 73.03] [fill={rgb, 255:red, 0; green, 0; blue, 0 }  ][line width=0.08]  [draw opacity=0] (12,-3) -- (0,0) -- (12,3) -- cycle    ;

% Text Node
\draw (222.02,27) node [anchor=west] [inner sep=0.75pt]    {$a$};
% Text Node
\draw (265.02,168) node [anchor=west] [inner sep=0.75pt]    {$e$};
% Text Node
\draw (216.02,238) node [anchor=west] [inner sep=0.75pt]    {$g$};
% Text Node
\draw (205.51,94.1) node [anchor=south west] [inner sep=0.75pt]    {$\omega _{2}$};
% Text Node
\draw (198.51,199.6) node [anchor=south east] [inner sep=0.75pt]    {$\omega _{1}$};
% Text Node
\draw (113,188.8) node [anchor=south east] [inner sep=0.75pt]    {$\Omega $};
% Text Node
\draw (129.5,98.99) node [anchor=north east] [inner sep=0.75pt]    {$\Omega _{a}$};


\end{tikzpicture}

    \caption{A ladder-type three-level system}
    \label{fig:ladder-system}
\end{figure}

We already know that a random wave function theory is fully described by both the Hamiltonian and a set 
of collapse operators. Sometimes, for example \eqref{last-eq:polarization} in \lastnote, wave function collapse
caused by spontaneous radiation and the subsequent observation of emitted photon is weak, and the main 
effect of spontaneous radiation is a correction of energy levels.

In this section we consider a ladder-type three-level system shown in \prettyref{fig:ladder-system}.
The Hamiltonian is 

\begin{equation}
    \dot{\rho}(t) = \frac{1}{\ii \hbar} H_\text{eff} \rho - \frac{1}{\ii \hbar} \rho H_\text{eff}^\dagger + \sum_j C_j \rho C_j^\dagger.
\end{equation}

\end{document}