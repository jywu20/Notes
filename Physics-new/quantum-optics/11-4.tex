\documentclass[hyperref, a4paper]{article}

\usepackage{geometry}
\usepackage{titling}
\usepackage{titlesec}
% No longer needed, since we will use enumitem package
% \usepackage{paralist}
\usepackage{enumitem}
\usepackage{footnote}
\usepackage{enumerate}
\usepackage{amsmath, amssymb, amsthm}
\usepackage{mathtools}
\usepackage{bbm}
\usepackage{cite}
\usepackage{graphicx}
\usepackage{subfigure}
\usepackage{physics}
\usepackage{tensor}
\usepackage{siunitx}
\usepackage[version=4]{mhchem}
\usepackage{tikz}
\usepackage{xcolor}
\usepackage{listings}
\usepackage{autobreak}
\usepackage[ruled, vlined, linesnumbered]{algorithm2e}
\usepackage{nameref,zref-xr}
\zxrsetup{toltxlabel}
\zexternaldocument*[optics-]{../optics/optics}[optics.pdf]
\zexternaldocument*[solid-]{../solid/solid}[solid.pdf]
\zexternaldocument*[info-]{../information/quantum-circuit}[quantum-circuit.pdf]
\usepackage[colorlinks,unicode]{hyperref} % , linkcolor=black, anchorcolor=black, citecolor=black, urlcolor=black, filecolor=black
\usepackage{prettyref}

% Page style
\geometry{left=3.18cm,right=3.18cm,top=2.54cm,bottom=2.54cm}
\titlespacing{\paragraph}{0pt}{1pt}{10pt}[20pt]
\setlength{\droptitle}{-5em}
\preauthor{\vspace{-10pt}\begin{center}}
\postauthor{\par\end{center}}

% More compact lists 
\setlist[itemize]{
    itemindent=17pt, 
    leftmargin=1pt,
    listparindent=\parindent,
    parsep=0pt,
}

% Math operators
\DeclareMathOperator{\timeorder}{\mathcal{T}}
\DeclareMathOperator{\diag}{diag}
\DeclareMathOperator{\legpoly}{P}
\DeclareMathOperator{\primevalue}{P}
\DeclareMathOperator{\sgn}{sgn}
\newcommand*{\ii}{\mathrm{i}}
\newcommand*{\ee}{\mathrm{e}}
\newcommand*{\const}{\mathrm{const}}
\newcommand*{\suchthat}{\quad \text{s.t.} \quad}
\newcommand*{\argmin}{\arg\min}
\newcommand*{\argmax}{\arg\max}
\newcommand*{\normalorder}[1]{: #1 :}
\newcommand*{\pair}[1]{\langle #1 \rangle}
\newcommand*{\fd}[1]{\mathcal{D} #1}
\DeclareMathOperator{\bigO}{\mathcal{O}}

% TikZ setting
\usetikzlibrary{arrows,shapes,positioning}
\usetikzlibrary{arrows.meta}
\usetikzlibrary{decorations.markings}
\tikzstyle arrowstyle=[scale=1]
\tikzstyle directed=[postaction={decorate,decoration={markings,
    mark=at position .5 with {\arrow[arrowstyle]{stealth}}}}]
\tikzstyle ray=[directed, thick]
\tikzstyle dot=[anchor=base,fill,circle,inner sep=1pt]

% Algorithm setting
% Julia-style code
\SetKwIF{If}{ElseIf}{Else}{if}{}{elseif}{else}{end}
\SetKwFor{For}{for}{}{end}
\SetKwFor{While}{while}{}{end}
\SetKwProg{Function}{function}{}{end}
\SetArgSty{textnormal}

\newcommand*{\concept}[1]{{\textbf{#1}}}

% Embedded codes
\lstset{basicstyle=\ttfamily,
  showstringspaces=false,
  commentstyle=\color{gray},
  keywordstyle=\color{blue}
}

\newcommand{\opticsdoc}{\href{../optics/optics}{the optics note}}
\newcommand{\soliddoc}{\href{../solid/solid}{the solid state physics note}}
\newcommand{\infodoc}{\href{../information/quantum-circuit}{the quantum information note}}

\newrefformat{fig}{Figure~\ref{#1} on page~\pageref{#1}}
\newrefformat{sec}{Section~\ref{#1}}

\title{Quantum Optics by Prof. Saijun Wu}
\author{Jinyuan Wu}
\date{November 4, 2021}

\begin{document}

\maketitle

\section{The rotational wave approximation and the Bloch sphere}

\begin{figure}
    \centering
    

\tikzset{every picture/.style={line width=0.75pt}} %set default line width to 0.75pt        

\begin{tikzpicture}[x=0.75pt,y=0.75pt,yscale=-1,xscale=1]
%uncomment if require: \path (0,300); %set diagram left start at 0, and has height of 300

%Straight Lines [id:da18819191009761904] 
\draw    (359,113) -- (450,113) ;
%Straight Lines [id:da1376948784583416] 
\draw    (359,200) -- (450,200) ;
%Straight Lines [id:da21818736070050204] 
\draw [color={rgb, 255:red, 248; green, 231; blue, 28 }  ,draw opacity=1 ]   (414,146) -- (414,154) .. controls (415.67,155.67) and (415.67,157.33) .. (414,159) .. controls (412.33,160.67) and (412.33,162.33) .. (414,164) .. controls (415.67,165.67) and (415.67,167.33) .. (414,169) .. controls (412.33,170.67) and (412.33,172.33) .. (414,174) .. controls (415.67,175.67) and (415.67,177.33) .. (414,179) .. controls (412.33,180.67) and (412.33,182.33) .. (414,184) .. controls (415.67,185.67) and (415.67,187.33) .. (414,189) .. controls (412.33,190.67) and (412.33,192.33) .. (414,194) .. controls (415.67,195.67) and (415.67,197.33) .. (414,199) -- (414,200) -- (414,200) ;
\draw [shift={(414,144)}, rotate = 90] [fill={rgb, 255:red, 248; green, 231; blue, 28 }  ,fill opacity=1 ][line width=0.08]  [draw opacity=0] (12,-3) -- (0,0) -- (12,3) -- cycle    ;
%Straight Lines [id:da6990718420713145] 
\draw  [dash pattern={on 4.5pt off 4.5pt}]  (378,144) -- (450,144) ;
%Straight Lines [id:da6995865544982842] 
\draw    (392,115) -- (392,142) ;
\draw [shift={(392,144)}, rotate = 270] [fill={rgb, 255:red, 0; green, 0; blue, 0 }  ][line width=0.08]  [draw opacity=0] (12,-3) -- (0,0) -- (12,3) -- cycle    ;
\draw [shift={(392,113)}, rotate = 90] [fill={rgb, 255:red, 0; green, 0; blue, 0 }  ][line width=0.08]  [draw opacity=0] (12,-3) -- (0,0) -- (12,3) -- cycle    ;
%Straight Lines [id:da8678920245446373] 
\draw [color={rgb, 255:red, 74; green, 144; blue, 226 }  ,draw opacity=1 ]   (366,115) -- (366,200) ;
\draw [shift={(366,113)}, rotate = 90] [fill={rgb, 255:red, 74; green, 144; blue, 226 }  ,fill opacity=1 ][line width=0.08]  [draw opacity=0] (12,-3) -- (0,0) -- (12,3) -- cycle    ;
%Straight Lines [id:da46707020520155873] 
\draw    (147,113) -- (238,113) ;
%Straight Lines [id:da15216197072886306] 
\draw    (147,200) -- (238,200) ;
%Straight Lines [id:da7329038078465273] 
\draw [color={rgb, 255:red, 248; green, 231; blue, 28 }  ,draw opacity=1 ]   (202,85) -- (202,93) .. controls (203.67,94.67) and (203.67,96.33) .. (202,98) .. controls (200.33,99.67) and (200.33,101.33) .. (202,103) .. controls (203.67,104.67) and (203.67,106.33) .. (202,108) .. controls (200.33,109.67) and (200.33,111.33) .. (202,113) .. controls (203.67,114.67) and (203.67,116.33) .. (202,118) .. controls (200.33,119.67) and (200.33,121.33) .. (202,123) .. controls (203.67,124.67) and (203.67,126.33) .. (202,128) .. controls (200.33,129.67) and (200.33,131.33) .. (202,133) .. controls (203.67,134.67) and (203.67,136.33) .. (202,138) .. controls (200.33,139.67) and (200.33,141.33) .. (202,143) .. controls (203.67,144.67) and (203.67,146.33) .. (202,148) .. controls (200.33,149.67) and (200.33,151.33) .. (202,153) .. controls (203.67,154.67) and (203.67,156.33) .. (202,158) .. controls (200.33,159.67) and (200.33,161.33) .. (202,163) .. controls (203.67,164.67) and (203.67,166.33) .. (202,168) .. controls (200.33,169.67) and (200.33,171.33) .. (202,173) .. controls (203.67,174.67) and (203.67,176.33) .. (202,178) .. controls (200.33,179.67) and (200.33,181.33) .. (202,183) .. controls (203.67,184.67) and (203.67,186.33) .. (202,188) .. controls (200.33,189.67) and (200.33,191.33) .. (202,193) .. controls (203.67,194.67) and (203.67,196.33) .. (202,198) -- (202,200) -- (202,200) ;
\draw [shift={(202,83)}, rotate = 90] [fill={rgb, 255:red, 248; green, 231; blue, 28 }  ,fill opacity=1 ][line width=0.08]  [draw opacity=0] (12,-3) -- (0,0) -- (12,3) -- cycle    ;
%Straight Lines [id:da5384879208606175] 
\draw  [dash pattern={on 4.5pt off 4.5pt}]  (166,83) -- (238,83) ;
%Straight Lines [id:da9870657692472646] 
\draw    (180,85) -- (180,112) ;
\draw [shift={(180,114)}, rotate = 270] [fill={rgb, 255:red, 0; green, 0; blue, 0 }  ][line width=0.08]  [draw opacity=0] (12,-3) -- (0,0) -- (12,3) -- cycle    ;
\draw [shift={(180,83)}, rotate = 90] [fill={rgb, 255:red, 0; green, 0; blue, 0 }  ][line width=0.08]  [draw opacity=0] (12,-3) -- (0,0) -- (12,3) -- cycle    ;
%Straight Lines [id:da34754409527991026] 
\draw [color={rgb, 255:red, 74; green, 144; blue, 226 }  ,draw opacity=1 ]   (154,115) -- (154,200) ;
\draw [shift={(154,113)}, rotate = 90] [fill={rgb, 255:red, 74; green, 144; blue, 226 }  ,fill opacity=1 ][line width=0.08]  [draw opacity=0] (12,-3) -- (0,0) -- (12,3) -- cycle    ;

% Text Node
\draw (452,200) node [anchor=west] [inner sep=0.75pt]    {$\omega _{g}$};
% Text Node
\draw (452,113) node [anchor=west] [inner sep=0.75pt]    {$\omega _{e}$};
% Text Node
\draw (400,128.5) node [anchor=west] [inner sep=0.75pt]    {$|\Delta |$};
% Text Node
\draw (411,172) node [anchor=east] [inner sep=0.75pt]    {$\omega $};
% Text Node
\draw (364,156.5) node [anchor=east] [inner sep=0.75pt]    {$\Omega $};
% Text Node
\draw (240,200) node [anchor=west] [inner sep=0.75pt]    {$\omega _{g}$};
% Text Node
\draw (240,113) node [anchor=west] [inner sep=0.75pt]    {$\omega _{e}$};
% Text Node
\draw (150,97.5) node [anchor=west] [inner sep=0.75pt]    {$|\Delta |$};
% Text Node
\draw (199,172) node [anchor=east] [inner sep=0.75pt]    {$\omega $};
% Text Node
\draw (152,156.5) node [anchor=east] [inner sep=0.75pt]    {$\Omega $};
% Text Node
\draw (184,234) node [anchor=north west][inner sep=0.75pt]   [align=left] {(a)};
% Text Node
\draw (393,234) node [anchor=north west][inner sep=0.75pt]   [align=left] {(b)};


\end{tikzpicture}

    \caption{The energy diagram of a two-level system in an external optical field}
\end{figure}

Under the rotational wave approximation and the corresponding RWA transformation, the Hamiltonian of a two-level system is 
\begin{equation}
    H = \frac{\hbar}{2} \vb*{\Omega} \cdot \vb*{\sigma},
\end{equation}
where 
\begin{equation}
    \abs*{\vb*{\Omega}} = \sqrt{\Omega^2 + \Delta^2},
\end{equation}
\begin{equation}
    \Omega = \frac{\vb*{E}_0 \cdot \vb*{d}_{eg}}{\hbar}
\end{equation}
is the \concept{Rabi frequency} and $\Delta$ the \concept{detuning}.
The wave function is always in the form of 
\begin{equation}
    \ket*{\psi} = \cos \frac{\theta}{2} \ee^{\ii \varphi /2 } \ket*{g} + \sin \frac{\theta}{2} \ee^{- \ii \varphi / 2} \ket*{e}, 
\end{equation}
and the density matrix is 
\begin{equation}
    \rho = \frac{1}{2} (1 + \vb*{n} \cdot \vb*{\sigma}),
\end{equation}
where 
\begin{equation}
    \vb*{n} = (\sin \theta \cos \varphi, \sin \theta \sin \varphi, \cos \theta).
\end{equation}
It is natural to put $\vb*{n}$ on a sphere.
The constructions are standard for a qubit and can be found in Section~\ref{info-sec:single-qubit} in \infodoc.
The equation of motion is 
\[
    \frac{\ii \hbar}{2} \vb*{n} \cdot \vb*{\sigma} = \ii \hbar \rho = \comm*{H}{\rho} = \comm*{\vb*{\Omega} \cdot \vb*{\sigma}}{\frac{1}{2} \vb*{n} \cdot \vb*{\sigma}} = \frac{1}{2} (\vb*{\Omega} \times \vb*{n}) \cdot \vb*{\sigma},
\]
and therefore we have 
\begin{equation}
    \dot{\vb*{n}} = \vb*{\Omega} \times \vb*{n}. 
\end{equation}

\end{document}