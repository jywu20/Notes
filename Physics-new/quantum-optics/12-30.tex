\documentclass[hyperref, a4paper]{article}

\usepackage{geometry}
\usepackage{titling}
\usepackage{titlesec}
% No longer needed, since we will use enumitem package
% \usepackage{paralist}
\usepackage{enumitem}
\usepackage{footnote}
\usepackage{enumerate}
\usepackage{amsmath, amssymb, amsthm}
\usepackage{mathtools}
\usepackage{bbm}
\usepackage{cite}
\usepackage{graphicx}
\usepackage{subcaption}
\usepackage{physics}
\usepackage{tensor}
\usepackage{siunitx}
\usepackage[version=4]{mhchem}
\usepackage{tikz}
\usepackage{xcolor}
\usepackage{listings}
\usepackage{autobreak}
\usepackage[ruled, vlined, linesnumbered]{algorithm2e}
\usepackage{nameref,zref-xr}
\zxrsetup{toltxlabel}
% \zexternaldocument*[last-]{12-2}[12-2.pdf]
\usepackage[colorlinks,unicode]{hyperref} % , linkcolor=black, anchorcolor=black, citecolor=black, urlcolor=black, filecolor=black
\usepackage[most]{tcolorbox}
\usepackage{prettyref}

% Page style
\geometry{left=3.18cm,right=3.18cm,top=2.54cm,bottom=2.54cm}
\titlespacing{\paragraph}{0pt}{1pt}{10pt}[20pt]
\setlength{\droptitle}{-5em}
\preauthor{\vspace{-10pt}\begin{center}}
\postauthor{\par\end{center}}

% More compact lists 
\setlist[itemize]{
    itemindent=17pt, 
    leftmargin=1pt,
    listparindent=\parindent,
    parsep=0pt,
}

% Math operators
\DeclareMathOperator{\timeorder}{\mathcal{T}}
\DeclareMathOperator{\diag}{diag}
\DeclareMathOperator{\legpoly}{P}
\DeclareMathOperator{\primevalue}{P}
\DeclareMathOperator{\sgn}{sgn}
\newcommand*{\ii}{\mathrm{i}}
\newcommand*{\ee}{\mathrm{e}}
\newcommand*{\const}{\mathrm{const}}
\newcommand*{\suchthat}{\quad \text{s.t.} \quad}
\newcommand*{\argmin}{\arg\min}
\newcommand*{\argmax}{\arg\max}
\newcommand*{\normalorder}[1]{: #1 :}
\newcommand*{\pair}[1]{\langle #1 \rangle}
\newcommand*{\fd}[1]{\mathcal{D} #1}
\DeclareMathOperator{\bigO}{\mathcal{O}}

% TikZ setting
\usetikzlibrary{arrows,shapes,positioning}
\usetikzlibrary{arrows.meta}
\usetikzlibrary{decorations.markings}
\tikzstyle arrowstyle=[scale=1]
\tikzstyle directed=[postaction={decorate,decoration={markings,
    mark=at position .5 with {\arrow[arrowstyle]{stealth}}}}]
\tikzstyle ray=[directed, thick]
\tikzstyle dot=[anchor=base,fill,circle,inner sep=1pt]

% Algorithm setting
% Julia-style code
\SetKwIF{If}{ElseIf}{Else}{if}{}{elseif}{else}{end}
\SetKwFor{For}{for}{}{end}
\SetKwFor{While}{while}{}{end}
\SetKwProg{Function}{function}{}{end}
\SetArgSty{textnormal}

\newcommand*{\concept}[1]{{\textbf{#1}}}

% Embedded codes
\lstset{basicstyle=\ttfamily,
  showstringspaces=false,
  commentstyle=\color{gray},
  keywordstyle=\color{blue}
}

% Reference formatting
\newrefformat{fig}{Figure~\ref{#1} on page~\pageref{#1}}

% Color boxes
\tcbuselibrary{skins, breakable, theorems}
\newtcbtheorem[number within=section]{warning}{Warning}%
  {colback=orange!5,colframe=orange!65,fonttitle=\bfseries, breakable}{warn}
\newtcbtheorem[number within=section]{note}{Note}%
  {colback=green!5,colframe=green!65,fonttitle=\bfseries, breakable}{note}

\newcommand{\lastnote}{\href{12-9.pdf}{the last lecture}}

\title{Quantum Optics by Prof. Saijun Wu}
\author{Jinyuan Wu}

\begin{document}

\maketitle

\section{A brief overview of cavity QED}

For a system where RWA and RWT work but photons emitted by the atom does not necessarily fly to infinity, 
the Hamiltonian is 
\begin{equation}
    H_\text{eff} = \hbar (\Delta_c - \ii \kappa / 2) a^\dagger a + \hbar (\Delta - \ii \Gamma / 2) \dyad{e} + \hbar (g_{ac} a + \Omega / 2) \dyad{e}{g} + \text{h.c.},
    \label{eq:photon-atom-coupling}
\end{equation}
where 
\begin{equation}
    g_{ac} = \frac{\vb*{\mathcal{E}}_{c}\cdot \vb*{d}_{eg}}{\hbar}, \quad 
    C_1 = \sqrt{\Gamma} \dyad{g}{e}, \quad 
    C_2 = \sqrt{\kappa} a.
\end{equation}
The difference between \eqref{eq:photon-atom-coupling} and previous models is that neither the optical field nor 
the atom can be integrated out, leaving a Markovian evolution. \eqref{eq:photon-atom-coupling} is a famous model 
in \concept{cavity QED}, where a \emph{local} optical field interacts with a localized atom. This is called 
\emph{QED} because we are still studying both electrons and photons. When ignoring all decaying, we get 
\begin{equation}
    H_\text{eff} = \hbar \Delta_c a^\dagger a + \hbar \Delta \dyad{e} + \hbar (g_{ac} a + \Omega / 2) \dyad{e}{g} + \text{h.c.}.
\end{equation}
This is \concept{Jaynes–Cummings model}. Jaynes–Cummings model is much harder to implement in laboratories 
because no one can create a cavity without any decaying, but anyway, someone has implemented it.

Now we discuss when we should calculate with the complete \eqref{eq:photon-atom-coupling}, instead of 
Markovian approximations. Intuitively speaking, we need very, very strong coupling between the atom and 
the optical field in the cavity so that if the atom radiates a photon in the cavity, before it runs away,
it already takes part in another process. This is the \concept{strong coupling} condition. Note that 
\[
    g \sim \frac{\vb*{\mathcal{E}} \cdot \vb*{d}_{eg}}{\hbar}, \quad 
    \Gamma \sim \frac{\omega^3 \vb*{d}_{eg}^2}{3 \pi \epsilon_0 c^3},
\]
what we need to do is to increase the intensity at the position of the atom, A schematic illustration of 
this is \prettyref{fig:focus}.

\begin{figure}
    \centering
    

\tikzset{every picture/.style={line width=0.75pt}} %set default line width to 0.75pt        

\begin{tikzpicture}[x=0.75pt,y=0.75pt,yscale=-1,xscale=1]
%uncomment if require: \path (0,300); %set diagram left start at 0, and has height of 300

%Curve Lines [id:da7513389130911845] 
\draw    (100,124) .. controls (114.5,132.06) and (140.5,144.06) .. (201.5,144.06) .. controls (262.5,144.06) and (294.5,134.06) .. (308.5,124) ;
%Curve Lines [id:da18714732754068852] 
\draw    (99,181.06) .. controls (113.5,173) and (140.5,161) .. (201.5,161) .. controls (262.5,161) and (294.5,171) .. (308.5,181.06) ;
%Shape: Circle [id:dp26153395911676935] 
\draw   (197.75,152.81) .. controls (197.75,150.19) and (199.88,148.06) .. (202.5,148.06) .. controls (205.12,148.06) and (207.25,150.19) .. (207.25,152.81) .. controls (207.25,155.44) and (205.12,157.56) .. (202.5,157.56) .. controls (199.88,157.56) and (197.75,155.44) .. (197.75,152.81) -- cycle ;
%Straight Lines [id:da08522771807029916] 
\draw    (142.5,97.06) -- (196.34,151.39) ;
\draw [shift={(197.75,152.81)}, rotate = 225.26] [fill={rgb, 255:red, 0; green, 0; blue, 0 }  ][line width=0.08]  [draw opacity=0] (12,-3) -- (0,0) -- (12,3) -- cycle    ;
%Straight Lines [id:da1494587986591045] 
\draw    (233.5,75.06) -- (287.34,129.39) ;
\draw [shift={(288.75,130.81)}, rotate = 225.26] [fill={rgb, 255:red, 0; green, 0; blue, 0 }  ][line width=0.08]  [draw opacity=0] (12,-3) -- (0,0) -- (12,3) -- cycle    ;
%Curve Lines [id:da33008860680925944] 
\draw    (106.5,195.4) .. controls (92.5,177.4) and (93.5,136.4) .. (104.5,112.4) ;
%Curve Lines [id:da260988740926406] 
\draw    (89.5,204.4) .. controls (75.5,186.4) and (76.5,132.4) .. (87.5,108.4) ;
%Straight Lines [id:da8582315637857714] 
\draw    (87.5,108.4) -- (104.5,112.4) ;
%Straight Lines [id:da14105437835741919] 
\draw    (89.5,204.4) -- (106.5,195.4) ;
%Curve Lines [id:da5818058260413539] 
\draw    (302.12,194.4) .. controls (316.12,176.4) and (315.12,135.4) .. (304.12,111.4) ;
%Curve Lines [id:da3164075576512748] 
\draw    (319.12,203.4) .. controls (333.12,185.4) and (332.12,131.4) .. (321.12,107.4) ;
%Straight Lines [id:da3830114193192764] 
\draw    (321.12,107.4) -- (304.12,111.4) ;
%Straight Lines [id:da13330150450389766] 
\draw    (319.12,203.4) -- (302.12,194.4) ;

% Text Node
\draw (123,76) node [anchor=north west][inner sep=0.75pt]   [align=left] {atom};
% Text Node
\draw (193,55) node [anchor=north west][inner sep=0.75pt]   [align=left] {light beam};


\end{tikzpicture}

    \caption{A cavity which focuses the light beam, and the atom is placed at the brightest position}
    \label{fig:focus}
\end{figure}

\section{Jaynes–Cummings model} 

\begin{figure}
    \centering
    

\tikzset{every picture/.style={line width=0.75pt}} %set default line width to 0.75pt        

\begin{tikzpicture}[x=0.75pt,y=0.75pt,yscale=-1,xscale=1]
%uncomment if require: \path (0,300); %set diagram left start at 0, and has height of 300

%Straight Lines [id:da0992307669277781] 
\draw    (122,257.06) -- (223.5,257.06) ;
%Straight Lines [id:da7121816296099286] 
\draw    (122,156.06) -- (223.5,156.06) ;
%Straight Lines [id:da8033743672162073] 
\draw    (122,131.06) -- (223.5,131.06) ;
%Straight Lines [id:da8227639834389773] 
\draw    (150,158.06) -- (150,257.06) ;
\draw [shift={(150,156.06)}, rotate = 90] [fill={rgb, 255:red, 0; green, 0; blue, 0 }  ][line width=0.08]  [draw opacity=0] (12,-3) -- (0,0) -- (12,3) -- cycle    ;
%Straight Lines [id:da3025880663175864] 
\draw    (132,133.06) -- (132,154.06) ;
\draw [shift={(132,156.06)}, rotate = 270] [fill={rgb, 255:red, 0; green, 0; blue, 0 }  ][line width=0.08]  [draw opacity=0] (12,-3) -- (0,0) -- (12,3) -- cycle    ;
\draw [shift={(132,131.06)}, rotate = 90] [fill={rgb, 255:red, 0; green, 0; blue, 0 }  ][line width=0.08]  [draw opacity=0] (12,-3) -- (0,0) -- (12,3) -- cycle    ;
%Shape: Rectangle [id:dp7214653784779961] 
\draw  [draw opacity=0][fill={rgb, 255:red, 0; green, 0; blue, 0 }  ,fill opacity=0.2 ] (89,116) -- (294.5,116) -- (294.5,172.06) -- (89,172.06) -- cycle ;

% Text Node
\draw (225.5,257.06) node [anchor=west] [inner sep=0.75pt]    {$g,n-1$};
% Text Node
\draw (225.5,156.06) node [anchor=west] [inner sep=0.75pt]    {$g,n$};
% Text Node
\draw (225.5,131.06) node [anchor=west] [inner sep=0.75pt]    {$e,n-1$};
% Text Node
\draw (134,197.4) node [anchor=north west][inner sep=0.75pt]    {$\omega $};
% Text Node
\draw (128,143.56) node [anchor=east] [inner sep=0.75pt]    {$\Delta $};
% Text Node
\draw (307,134) node [anchor=north west][inner sep=0.75pt]   [align=left] {the subspace};


\end{tikzpicture}

    \caption{The energy level diagram of \eqref{eq:jc-subspace}}
    \label{fig:jc-subspace}
\end{figure}

Now we discuss the Jaynes–Cummings model. From the RWA Hamiltonian
\begin{equation}
    H = \hbar g \ee^{- \ii \Delta t} a \dyad{e}{g} + \text{h.c.},
\end{equation}
after RWT, we have  
\begin{equation}
    H_\text{JC} = \hbar \Delta_c a^\dagger a + (\hbar g a \dyad{e}{g} + \text{h.c.}). 
\end{equation}
If we just consider the subspace spanned by $\ket*{g, n}$ and $\ket*{e, n-1}$, the Hamiltonian is 
\begin{equation}
    H = \pmqty{n \Delta & \sqrt{n} g \\ \sqrt{n} g & (n-1) \Delta}.
    \label{eq:jc-subspace}
\end{equation}
We find two \emph{dressed} states, the energies of them being 
\begin{equation}
    \hbar \tilde{\omega}_{g,n} = \left(n - \frac{1}{2}\right) \hbar \Delta - \frac{1}{2} (\sqrt{\Delta^2 + 4 n g^2} - \Delta), 
\end{equation}
and 
\begin{equation}
    \hbar \tilde{\omega}_{e, n-1} = \left(n - \frac{1}{2}\right) \hbar \Delta + \frac{1}{2} (\sqrt{\Delta^2 + 4 n g^2} - \Delta).
\end{equation}

Now the Jaynes–Cummings model where the optical field state is a single-photon state coupled with only 
one atom - or in other words, \concept{single-photon Rabi oscillation} - appears to be solved completely. 
Things are not that simple, since the parameters $g$ and $\Delta$ sometimes are time-dependent. 

When the input state involves a non-Fock state, the time evolution can also be non-trivial, since we no longer 
work in \prettyref{fig:jc-subspace}. Suppose the input state is 
\begin{equation}
    \ket*{\psi(0)} = \ket*{e, \alpha} = \ket*{e} \otimes \sum_{n} \ee^{- \abs*{\alpha} / 2} \frac{\alpha^n}{\sqrt{n!}} \ket*{n},
\end{equation}
then we have 
We can observe how the 

This is a vivid demonstration of when the semi-classical approximation of light-atom coupling works and does not 
work. We can see that the time scale of the semi-classical regions is $1 / g$. When $t \gg 1 / g$, the state of 
the optical field becomes a cat state. In ordinary systems, $g$ is small, and therefore $1/g$ is large, and long 
before a cat state is formed, spontaneous radiation or leaking or other quantum jumps happens. However, 
in Jaynes–Cummings model, there is no quantum jump and $g$ is strong, and we can just see a cat state occurring. 

\section{Heisenberg picture of a leaky cavity}

Now we consider the Heisenberg picture of a leaky cavity:
\begin{equation}
    H_\text{eff} = \hbar (\omega - \ii \kappa / 2) a^\dagger a, \quad \kappa = \frac{2\abs*{t}^2 c}{L}
    \label{eq:leaky-cavity}
\end{equation}
We already know that the master equation has a term $\rho H_\text{eff} - H_\text{eff}^\dagger \rho$, 
the $^\dagger$ symbol coming from the non-Hermitian property of \eqref{eq:leaky-cavity}. 
Now we naively consider the Heisenberg picture of \eqref{eq:leaky-cavity}, \emph{without} considering 
the non-Hermitian property. We have 
\[
    \ii \dot{a} = \comm*{a}{H} = (\omega_c - \ii \kappa / 2) a, 
\]
and similarly 
\[
    \ii \dot{a}^\dagger = (- \omega_c - \ii \kappa / 2) a^\dagger.
\]
These equations correctly show the damping behavior of the system, but now the commutation relation is 
\[
    \comm*{a}{a^\dagger} = \ee^{- \kappa t},
\]
which is obviously wrong, since we expect the creation and annihilation operators to always satisfy 
$[a, a^\dagger] = 1$. 

It does not mean that a Heisenberg picture of \eqref{eq:leaky-cavity} is impossible. To see clearly how, 
let us go back to the original Hamiltonian 
\begin{equation}
    H = \underbrace{\hbar \omega_c a^\dagger a}_{H_S} + \underbrace{\sum_k \hbar \omega_k a^\dagger_k a_k}_{H_R} + \underbrace{\sum_k \hbar g_k a a_k^\dagger + \text{h.c.}}_{H_{SR}}.
\end{equation}
From the Heisenberg EOMs we have 
\[
    \ii \dot{a} = \sum_k g_k \ee^{- \ii \Delta_k t} a_k, \quad \ii \dot{a}_k = g_k^* \ee^{\ii \Delta_k t} a,
\]
and therefore 
\[
    \ii a = - \ii \int_0^t \dd{\tau} a(\tau) \sum_k \abs*{g}_k^2 \ee^{\ii \Delta_k (t - \tau)}
    + \sum_k g_k \ee^{\ii \Delta_k t} a(k).
\]
Now we coarse-grain the time. When the time scale we are interested is much larger than $1 / \Delta \omega$,
we have 
\begin{equation}
    \ii \dot{\tilde{a}} = (\Delta_c - \ii \kappa / 2) \tilde{a} + \tilde{F}(t),
    \label{eq:quantum-langevin}
\end{equation}
and it can be verified that 
\begin{equation}
    \expval{F(t)} = 0, 
\end{equation}
and 
\begin{equation}
    \begin{aligned}
        \expval{F(t) F^\dagger(t')} &= \sum_{k, k'} \ee^{\ii \Delta_k t - \ii \Delta_{k'} t} \expval*{a_k a^\dagger_{k'}} \\
        &= \sum_k \abs*{g_k}^2 \ee^{\ii \Delta_k (t - t')} = \kappa \delta(t - t').
    \end{aligned}
\end{equation}
Therefore, we can see \eqref{eq:quantum-langevin} is actually a \emph{quantum} Langevin equation.
This is the counterpart of the wave function formalism in the Schrodinger picture.
By a picture transformation we can eliminate the $\Delta_c$ term, and in this case we have 
\begin{equation}
    a(t) = a(0) \ee^{- \kappa t / 2} - \ii \int_0^t \tilde{F}(\tau) \ee^{\kappa (\tau - t) / 2} \dd{\tau},
\end{equation}
and since the fluctuation term has zero expectation, we have  
\begin{equation}
    \expval{a(t)} = \ee^{- \kappa t / 2} \expval{a(0)}.
\end{equation}

\section{Experimental realization of cavity QED}

To realize a cavity QED model, we need large $g$ and small $\kappa$ and $\Gamma$. For a small $\Gamma$, we need 
to excite atoms to a Rydberg state. For % TODO: cavity size 
We therefore find ordinary optical cavities are not handy. We know superconductors are impenetrable for
electromagnetic waves, and therefore they can be used to create a cavity with \SI{100}{\percent} reflection. 

With quantum non-deconstructive measurement, we can even \emph{measure} the photon number in the cavity and 
how quantum jump happens. These quantum jumps are not caused by measurement. Instead, they are caused by 
leakage of the cavity.

\section{Some unexpected inspiration of the stochastic wave function method}

Consider we use two cavities to 

Quantum localization can also be used to \emph{localize} the position of mirrors in a quantum optical system.
We know the position of mirrors are actually inherently quantum, but since they are ``observed'' constantly
by the photons, if the position of a mirror is observed to be $\vb*{r}_1$, then subsequently it will more 
likely to appear at $\vb*{r}_1$. This also explains why we do not feel quantum uncertainly in everyday life.

\end{document}