\documentclass[hyperref, a4paper]{article}

\usepackage{geometry}
\usepackage{titling}
\usepackage{titlesec}
% No longer needed, since we will use enumitem package
% \usepackage{paralist}
\usepackage{enumitem}
\usepackage{footnote}
\usepackage{enumerate}
\usepackage{amsmath, amssymb, amsthm}
\usepackage{mathtools}
\usepackage{bbm}
\usepackage{cite}
\usepackage{graphicx}
\usepackage{subcaption}
\usepackage{physics}
\usepackage{tensor}
\usepackage{siunitx}
\usepackage[version=4]{mhchem}
\usepackage{tikz}
\usepackage{xcolor}
\usepackage{listings}
\usepackage{autobreak}
\usepackage[ruled, vlined, linesnumbered]{algorithm2e}
\usepackage{nameref,zref-xr}
\zxrsetup{toltxlabel}
\zexternaldocument*[optics-]{../optics/optics}[optics.pdf]
\zexternaldocument*[solid-]{../solid/solid}[solid.pdf]
\zexternaldocument*[prev-]{11-18}[11-18.pdf]
\usepackage[colorlinks,unicode]{hyperref} % , linkcolor=black, anchorcolor=black, citecolor=black, urlcolor=black, filecolor=black
\usepackage[most]{tcolorbox}
\usepackage{prettyref}

% Page style
\geometry{left=3.18cm,right=3.18cm,top=2.54cm,bottom=2.54cm}
\titlespacing{\paragraph}{0pt}{1pt}{10pt}[20pt]
\setlength{\droptitle}{-5em}
\preauthor{\vspace{-10pt}\begin{center}}
\postauthor{\par\end{center}}

% More compact lists 
\setlist[itemize]{
    itemindent=17pt, 
    leftmargin=1pt,
    listparindent=\parindent,
    parsep=0pt,
}

% Math operators
\DeclareMathOperator{\timeorder}{\mathcal{T}}
\DeclareMathOperator{\diag}{diag}
\DeclareMathOperator{\legpoly}{P}
\DeclareMathOperator{\primevalue}{P}
\DeclareMathOperator{\sgn}{sgn}
\newcommand*{\ii}{\mathrm{i}}
\newcommand*{\ee}{\mathrm{e}}
\newcommand*{\const}{\mathrm{const}}
\newcommand*{\suchthat}{\quad \text{s.t.} \quad}
\newcommand*{\argmin}{\arg\min}
\newcommand*{\argmax}{\arg\max}
\newcommand*{\normalorder}[1]{: #1 :}
\newcommand*{\pair}[1]{\langle #1 \rangle}
\newcommand*{\fd}[1]{\mathcal{D} #1}
\DeclareMathOperator{\bigO}{\mathcal{O}}

% TikZ setting
\usetikzlibrary{arrows,shapes,positioning}
\usetikzlibrary{arrows.meta}
\usetikzlibrary{decorations.markings}
\tikzstyle arrowstyle=[scale=1]
\tikzstyle directed=[postaction={decorate,decoration={markings,
    mark=at position .5 with {\arrow[arrowstyle]{stealth}}}}]
\tikzstyle ray=[directed, thick]
\tikzstyle dot=[anchor=base,fill,circle,inner sep=1pt]

% Algorithm setting
% Julia-style code
\SetKwIF{If}{ElseIf}{Else}{if}{}{elseif}{else}{end}
\SetKwFor{For}{for}{}{end}
\SetKwFor{While}{while}{}{end}
\SetKwProg{Function}{function}{}{end}
\SetArgSty{textnormal}

\newcommand*{\concept}[1]{{\textbf{#1}}}

% Embedded codes
\lstset{basicstyle=\ttfamily,
  showstringspaces=false,
  commentstyle=\color{gray},
  keywordstyle=\color{blue}
}

% Reference formatting
\newrefformat{fig}{Figure~\ref{#1} on page~\pageref{#1}}

% Color boxes
\tcbuselibrary{skins, breakable, theorems}
\newtcbtheorem[number within=section]{warning}{Warning}%
  {colback=orange!5,colframe=orange!65,fonttitle=\bfseries, breakable}{warn}
\newtcbtheorem[number within=section]{note}{Note}%
  {colback=green!5,colframe=green!65,fonttitle=\bfseries, breakable}{note}

\newcommand{\prevlecture}{\href{11-18.pdf}{the previous lecture}}

\title{Quantum Optics by Prof. Saijun Wu}
\author{Jinyuan Wu}

\begin{document}

\maketitle

\section{Spontaneous radiation of a two-level atom}\label{sec:spon-rad}

We continue the discussion in \href{11-18.pdf}{the last lecture}.
For an atom coupled with an optical field (with its own dynamics), we have 
\[
    \ii \dot{c}_\text{e} = \sum_k g_k \ee^{- \ii \Delta_k t} c_k, \quad 
    \ii \dot{c}_k = g_k^* \ee^{\ii \Delta_k t} c_\text{e},
\]
so we have 
\begin{equation}
    \dot{c}_\text{e}(t) = - \int_0^t \dd{\tau} K(t - \tau) c_\text{e}(\tau),
    \label{eq:one-component-eom}
\end{equation}
where 
\begin{equation}
    K(t) = \sum_k \abs*{g_k}^2 \ee^{- \ii \Delta_k t}.
\end{equation}
We make a Markov approximation, i.e. we assume $K(t - \tau)$ is significant only for $\tau$ close enough to $t$,
and therefore we have approximately
\[
    \dot{c}_\text{e} = - c_\text{e}(t) \int_0^t \dd{\tau} K(t - \tau).
\]
Formally, we define 
\begin{equation}
    \int_0^t \dd{\tau} K(t - \tau) \eqqcolon \frac{\Gamma}{2} + \ii \delta_\text{L},
\end{equation}
so the time evolution equation is 
\begin{equation}
    \dot{c}_\text{e} = - \left( \frac{\Gamma}{2} + \ii \delta_\text{L} \right) c_\text{e}(t) .
    \label{eq:markov-eq}
\end{equation}
\begin{note*}{}
    The definition of $\Gamma$ has a $1/2$ factor. This is to ensure that $\Gamma$ is the damping rate of 
    the probability that the atom is on the excited state, because if the time evolution of $c_\text{e}$ is 
    \[
        c_\text{e} \to (1 - \Gamma \Delta t / 2) c_\text{e},
    \]
    then the time evolution of $\abs*{c_\text{e}}^2$ is 
    \[
        \abs*{c_\text{e}}^2 \to (1 - \Gamma \Delta t / 2)^2 \abs*{c_\text{e}}^2 = (1 - \Gamma \Delta t) \abs*{c_\text{e}}^2.
    \]
    We immediately find that the discussion in \prevlecture{} around \eqref{prev-eq:original-spontaneous-emission}
    and \eqref{prev-eq:fermi-golden-rule} is a specific case of the above derivation, where 
    $\abs*{c_\text{e}(0)}^2$. In other words, the two definitions of $\Gamma$ in \prevlecture{} and in this 
    lecture are actually the same. In the language of QFT, we can call $\Gamma$ the \emph{scattering rate}.
\end{note*}

Now we need to evaluate $\Gamma$ and $\delta_\text{L}$. We have 
\[
    \begin{aligned}
        \int_0^t \dd{\tau} K(t - \tau) &= \sum_k \abs*{g_k}^2 \int_0^t \dd{\tau} \ee^{- \ii \Delta_k (t - \tau)} \\
        &= \sum_k \abs*{g_k}^2 \int_0^t \dd{\tau} \ee^{- \ii (\omega_k - \omega_{eg}) (t - \tau)} \\
        &= \sum_k \abs*{g_k}^2 \int_0^t \dd{\tau'} \ee^{- \ii (\omega_k - \omega_{eg}) \tau' },
    \end{aligned}
\]
and since we are working in the region where $t \gg 1 / \omega_{eg}$, we push the integral to the $t \to \infty$ 
limit, i.e. making the approximation that 
\[
    \begin{aligned}
        \int_0^t \dd{\tau} K(t - \tau) &\approx \sum_k \abs*{g_k}^2 \int_0^\infty \dd{\tau'} \ee^{- \ii (\omega_k - \omega_{eg}) \tau' - 0^+ \tau'} \\
        &= \sum_k \abs*{g_k}^2 \frac{-1}{- \ii (\omega_k - \omega_{eg})  - 0^+ } \\
        &= \sum_k \abs*{g_k}^2 \frac{\ii}{\omega_{eg} - \omega_k + \ii 0^+} \\
        &= \sum_k \abs*{g_k}^2 \left( \primevalue \frac{\ii}{\omega_{eg} - \omega_k } + \pi \delta(\omega_{eg} - \omega_k) \right),
    \end{aligned}
\]
The final results are
\begin{equation}
    \Gamma = 2 \pi \sum_k \abs*{g_k}^2 \delta(\omega_k - \omega_\text{eg}),
    \label{eq:scattering-rate}
\end{equation}
and 
\begin{equation}
    \delta_\text{L} = \primevalue \sum_k \abs*{g_k}^2 \rho(k) \frac{1}{\omega_\text{eg} - \omega_k }.
    \label{eq:phase-shift}
\end{equation}
We can see that Fermi golden rule can be derived from the Markov approximation. We also see that the approach 
we used in the last lecture fail to capture an imaginary factor. The imaginary part $\delta_\text{L}$ is a part 
of Lamb shift, which arises from the vacuum fluctuation of the optical field. The two parts can be obtained 
from the following Feynman diagram 
\begin{equation}
    \begin{gathered}
        \begin{tikzpicture}[x=0.75pt,y=0.75pt,yscale=-1,xscale=1]
            %uncomment if require: \path (0,300); %set diagram left start at 0, and has height of 300
            
            %Straight Lines [id:da01124014261520312] 
            \draw    (79,179) -- (153,179) ;
            %Straight Lines [id:da5714521171594549] 
            \draw    (118,179) -- (121,179) ;
            \draw [shift={(123,179)}, rotate = 180] [fill={rgb, 255:red, 0; green, 0; blue, 0 }  ][line width=0.08]  [draw opacity=0] (12,-3) -- (0,0) -- (12,3) -- cycle    ;
            %Straight Lines [id:da7252034130949252] 
            \draw    (153,179) -- (227,179) ;
            %Straight Lines [id:da483809936593951] 
            \draw    (192,179) -- (195,179) ;
            \draw [shift={(197,179)}, rotate = 180] [fill={rgb, 255:red, 0; green, 0; blue, 0 }  ][line width=0.08]  [draw opacity=0] (12,-3) -- (0,0) -- (12,3) -- cycle    ;
            %Curve Lines [id:da12282081273981671] 
            \draw    (153,179) .. controls (151.65,176.93) and (151.97,175.2) .. (153.97,173.82) .. controls (156.01,172.68) and (156.49,171.09) .. (155.42,169.06) .. controls (154.49,166.91) and (155.2,165.3) .. (157.57,164.21) .. controls (159.74,163.75) and (160.52,162.46) .. (159.9,160.35) .. controls (159.69,157.84) and (160.83,156.44) .. (163.3,156.13) .. controls (165.45,156.34) and (166.6,155.26) .. (166.77,152.9) .. controls (167.19,150.49) and (168.6,149.51) .. (170.99,149.95) .. controls (173.22,150.62) and (174.74,149.87) .. (175.53,147.68) .. controls (176.58,145.51) and (178.01,145.02) .. (179.82,146.21) .. controls (181.78,147.46) and (183.42,147.13) .. (184.74,145.22) .. controls (186.31,143.39) and (187.98,143.29) .. (189.75,144.91) .. controls (191.3,146.64) and (192.98,146.76) .. (194.77,145.28) .. controls (196.78,143.95) and (198.42,144.3) .. (199.71,146.33) .. controls (200.76,148.42) and (202.35,149.01) .. (204.5,148.08) .. controls (206.83,147.36) and (208.35,148.17) .. (209.07,150.51) .. controls (209.26,152.64) and (210.54,153.57) .. (212.91,153.29) .. controls (215.4,153.28) and (216.57,154.39) .. (216.44,156.63) .. controls (216.34,159.07) and (217.39,160.37) .. (219.6,160.53) .. controls (222.08,161.28) and (222.99,162.77) .. (222.34,165) .. controls (221.53,167.14) and (222.21,168.62) .. (224.37,169.45) .. controls (226.52,170.52) and (227.06,172.16) .. (225.97,174.35) .. controls (224.76,176.44) and (225.11,177.99) .. (227,179) -- (227,179) ;
            %Straight Lines [id:da2752757330155595] 
            \draw    (227,179) -- (301,179) ;
            %Straight Lines [id:da34684664641318586] 
            \draw    (265,179) -- (268,179) ;
            \draw [shift={(270,179)}, rotate = 180] [fill={rgb, 255:red, 0; green, 0; blue, 0 }  ][line width=0.08]  [draw opacity=0] (12,-3) -- (0,0) -- (12,3) -- cycle    ;
            
            % Text Node
            \draw (118,182.4) node [anchor=north] [inner sep=0.75pt]    {$e$};
            % Text Node
            \draw (192,182.4) node [anchor=north] [inner sep=0.75pt]    {$g$};
            % Text Node
            \draw (264,182.4) node [anchor=north] [inner sep=0.75pt]    {$e$};
            \end{tikzpicture}
    \end{gathered},
    \label{eq:one-loop}
\end{equation}
where $\Gamma$ comes from half of the diagram. 
Note that the diagram actually has $\omega$ dependence 
in the frequency domain (and therefore belongs to off-shell formalism),
and that's expected since the above time domain
form of $K(t - \tau)$ considers the retardation effect; 
it's the Markovian approximation that removes the frequency dependence. 
The above procedure -- working in a subspace, regarding the 
optical field as in a large box and then applying the Markovian approximation and calculating $\Gamma$ and 
$\delta_\text{L}$ -- is called \concept{Wigner-Weisskopf theory}.

\begin{note*}{}
    \eqref{eq:scattering-rate} and \eqref{eq:phase-shift} are just two parts of a propagator in the form of $1 / (E - H + \ii 0^+)$.
    What we are actually doing here is to ``integrate out'' the external optical field and get a non-Hermitian
    theory of the atom. We are assuming 
    \begin{itemize}
        \item that the emitted photon leaves the atom quickly enough, so the Markovian approximation (the state 
        of the system can be labeled completely by $c_e$ and $c_g$), without any need to fear that the emitted 
        photon may come back (i.e. the optical degrees of freedom can effectively be seen as a \emph{bath}), and
        \item that the time scale we are interested is large enough so that we are able to take the $t \to \infty$
        limit.
    \end{itemize} 
    Formally, the first and the second approximation means the concepts in perturbative QFT work here, i.e.
    we have concepts like propagators and Feynman diagrams and we can just interpret the imaginary 
    part of a propagator as decay. Only one diagram - \eqref{eq:one-loop} - is taken into account, because 
    actually we are doing the self-energy correction to the atom, and \eqref{eq:one-loop} is the irreducible
    self energy. Therefore, the Wigner-Weisskopf theory is actually a non-perturbative theory.
\end{note*}

However, we will soon find a huge problem in the Wigner-Weisskopf theory: the total probability is not conserved.
What is happening is that the emitted photon gets observed by the environment from time to time, and therefore 
the wave function gets collapsed to certain states (for a two-level atom, the ground state) from time to time.
We will see a detailed calculation in \prettyref{sec:stochastic-two-level}.

Solving \eqref{eq:markov-eq} and redefining the basis to absorb the Lamb shift, we get 
\begin{equation}
    c_{e}(t) = \ee^{- \frac{\Gamma}{2} t}, \quad 
    c_{k}(t) = \ii g_k^* \frac{1 - \ee^{- \ii (\omega_\text{eg} - \ii \Gamma / 2 - \omega_k) t}}{\omega_\text{eg} - \ii \Gamma / 2 - \omega_k} .
\end{equation}
Now we can calculate the optical correlation functions in the space. Since we are working in the single-photon 
subspace, we have 
\begin{equation}
    P^{(1)}(\vb*{r}, t) = \eta \abs*{\mathcal{E}(\vb*{r}, t)}^2,
\end{equation}
where 
\begin{equation}
    \mathcal{E}(\vb*{r}, t) = \mel{0}{E^+(\vb*{r}, t) \sum_k c_k }{1_k}.
\end{equation}
Again we consider the time region where $t \gg 1 / \Gamma$ so spontaneous radiation is likely to start, while the 
time is not too large so that Markov approximation still works. We have 
\[
    \begin{aligned}
        \mathcal{E} &= \frac{\ii}{(2\pi)^3} \int_0^\infty k^2 \dd{k} \int_0^\pi \sin \theta \dd{\theta}
        \int_0^{2\pi} \dd{\varphi} \frac{\hbar \omega_k \vb*{d}_\text{eg}^2}{2 \epsilon_0} 
        \frac{\ee^{- \ii (\omega_k t - k r \cos \theta)}}{ \omega_\text{eg} - \frac{\ii \Gamma}{2} - \omega_k} \\
        &= \int_0^\infty \dd{k} \frac{k^2 d^2}{\epsilon_0 r} \frac{\ee^{- \ii \omega_k (t - r / c)}}{\omega_\text{eg} - \frac{\ii \Gamma}{2} - \omega_k} - \text{$(r + t/c)$ terms} \\
        &\approx - \int_{-\infty}^\infty \frac{\omega_k^2}{c^2} ,
    \end{aligned}
\]
so finally we get the single photon wave function of the spontaneous radiation:
\begin{equation}
    \mathcal{E}(\vb*{r}, t) = \frac{\omega_\text{eg}^2 d_\text{eg}^2}{\epsilon_0 r c^2} 
    \ee^{- \ii \omega_\text{eg} (t - r / c)} \Theta(t - r/c).
\end{equation}

\section{Three-level atoms}

\begin{figure}
    \centering
    

\tikzset{every picture/.style={line width=0.75pt}} %set default line width to 0.75pt        

\begin{tikzpicture}[x=0.75pt,y=0.75pt,yscale=-1,xscale=1]
%uncomment if require: \path (0,300); %set diagram left start at 0, and has height of 300

%Straight Lines [id:da46501807278883933] 
\draw    (45,65) -- (121.02,65) ;
%Straight Lines [id:da9309002085456306] 
\draw    (95,145) -- (171.02,145) ;
%Straight Lines [id:da9554603087472111] 
\draw    (46,215) -- (122.02,215) ;
%Straight Lines [id:da6350132577991978] 
\draw    (83.01,65) .. controls (85.3,65.53) and (86.19,66.94) .. (85.66,69.24) .. controls (85.13,71.53) and (86.02,72.95) .. (88.31,73.48) .. controls (90.6,74.01) and (91.49,75.43) .. (90.96,77.72) .. controls (90.43,80.01) and (91.32,81.43) .. (93.61,81.96) .. controls (95.9,82.49) and (96.79,83.91) .. (96.26,86.2) .. controls (95.73,88.49) and (96.62,89.91) .. (98.91,90.44) .. controls (101.2,90.97) and (102.09,92.39) .. (101.56,94.68) .. controls (101.03,96.97) and (101.92,98.39) .. (104.21,98.92) .. controls (106.5,99.45) and (107.39,100.87) .. (106.86,103.16) .. controls (106.33,105.45) and (107.22,106.87) .. (109.51,107.4) .. controls (111.8,107.93) and (112.69,109.35) .. (112.16,111.64) .. controls (111.63,113.93) and (112.52,115.35) .. (114.81,115.88) .. controls (117.1,116.41) and (117.99,117.83) .. (117.46,120.12) .. controls (116.93,122.41) and (117.82,123.83) .. (120.11,124.36) .. controls (122.4,124.89) and (123.29,126.31) .. (122.76,128.6) .. controls (122.23,130.89) and (123.12,132.31) .. (125.41,132.84) -- (127.71,136.52) -- (131.95,143.3) ;
\draw [shift={(133.01,145)}, rotate = 237.99] [fill={rgb, 255:red, 0; green, 0; blue, 0 }  ][line width=0.08]  [draw opacity=0] (12,-3) -- (0,0) -- (12,3) -- cycle    ;
%Straight Lines [id:da9514514154304783] 
\draw    (133.01,145) .. controls (133.42,147.32) and (132.46,148.69) .. (130.14,149.1) .. controls (127.82,149.51) and (126.86,150.87) .. (127.27,153.19) .. controls (127.68,155.51) and (126.73,156.88) .. (124.41,157.29) .. controls (122.09,157.7) and (121.13,159.06) .. (121.54,161.38) .. controls (121.95,163.7) and (120.99,165.07) .. (118.67,165.48) .. controls (116.35,165.89) and (115.4,167.26) .. (115.81,169.58) .. controls (116.22,171.9) and (115.26,173.26) .. (112.94,173.67) .. controls (110.62,174.08) and (109.66,175.45) .. (110.07,177.77) .. controls (110.48,180.09) and (109.52,181.46) .. (107.2,181.87) .. controls (104.88,182.28) and (103.93,183.64) .. (104.34,185.96) .. controls (104.75,188.28) and (103.79,189.65) .. (101.47,190.06) .. controls (99.15,190.47) and (98.19,191.83) .. (98.6,194.15) .. controls (99.01,196.47) and (98.05,197.84) .. (95.73,198.25) .. controls (93.41,198.66) and (92.46,200.03) .. (92.87,202.35) .. controls (93.28,204.67) and (92.32,206.03) .. (90,206.44) -- (89.74,206.81) -- (85.16,213.36) ;
\draw [shift={(84.01,215)}, rotate = 304.99] [fill={rgb, 255:red, 0; green, 0; blue, 0 }  ][line width=0.08]  [draw opacity=0] (12,-3) -- (0,0) -- (12,3) -- cycle    ;

% Text Node
\draw (123.02,65) node [anchor=west] [inner sep=0.75pt]    {$a$};
% Text Node
\draw (173.02,145) node [anchor=west] [inner sep=0.75pt]    {$b$};
% Text Node
\draw (124.02,215) node [anchor=west] [inner sep=0.75pt]    {$c$};
% Text Node
\draw (110.01,101.6) node [anchor=south west] [inner sep=0.75pt]    {$k$};
% Text Node
\draw (110.51,183.4) node [anchor=north west][inner sep=0.75pt]    {$q$};


\end{tikzpicture}

    \caption{Energy levels of a three-level atom}
    \label{fig:three-level-atom}
\end{figure}

Now we consider a three-level atom shown in \prettyref{fig:three-level-atom}. We work in the following 
subspace:
\begin{equation}
    \ket*{\psi} = c_a \ket*{a, 0} + \sum_k c_{b_k} \ket*{b, 1_k} + \sum_{k, q} c_{c, k, g} \ket*{c, 1_k, 1_q}. 
\end{equation}
The Schrödinger equation now reads 
\begin{equation}
    \begin{aligned}
        \ii \dot{c}_a &= \sum_k g_{a, k}^* \ee^{- \ii \Delta t} c_{b, k}, \\
        \ii \dot{c}_{b, k} &= g_{a,k} \ee^{\ii \Delta_k t} c_a 
        + \sum_q g^*_{b,q} \ee^{- \ii \Delta_q t} c_{c, k, q}, \\
        \ii \dot{c}_{c, k, q} &= g_{b,q} \ee^{- \ii \Delta_q t} c_{b, k},
    \end{aligned}
\end{equation} 
and we can just repeat the procedure in the last section.

\begin{note*}{}{}
    We should note that the so-called ``wave function in a subspace'' does not capture the high frequency 
    details of the real wave function. What we are actually doing is \emph{integrating out} high energy 
    states and working with an effective theory. The renormalization of constants in the theory about 
    the subspace can be calculated via path integral or adiabatic elimination or other approaches.
    The loss of high-frequency components of the wave function usually does not matter, though, 
    because usually no detection means can be used to find the high-frequency behavior, and an effective
    theory is just enough.
\end{note*}

\section{Random wave function description of the two-level atom}\label{sec:stochastic-two-level}

We want to write down a theory on the two-level atom itself and ignore the optical field. 
This is usually hard because photons radiated may come back and interact with the atom again.
If, however, we investigate the circumstance in a typical laboratory, where there are plenty 
of things that may absorb or ``observe'' the photon, we can assume that as soon as a photon 
is emitted, it gets observed, and so does the atom. If we observe a emitted photon then we are sure 
that the atom is currently on the ground state.

\concept{Random wave function} is a formalism that is frequently used in quantum optics.
We write down a non-Hermitian effective theory describing the atom, which is 
\begin{equation}
    H = H_0 - \frac{\ii}{2} C^\dagger C,
\end{equation}
where $H_0$ is the RWA Hamiltonian in \href{10-28.pdf}{this lecture}, and
\begin{equation}
    C = \sqrt{\Gamma} \dyad{\text{g}}{\text{e}}
\end{equation}
is called the \concept{collapse operator}.
The effective Hamiltonian does not keep the total probability. The correct way to use it is the following 
\emph{random} Schrödinger evolution. For $\ket*{\psi(t)}$, we have 
\begin{equation}
    P(\ket*{\psi(t + \Delta t)} = \text{normalized $C \ket*{\psi(t)}$}) = \expval*{C^\dagger C}{\psi(t)} \Delta t,
\end{equation}
and 
\begin{equation}
    P(\ket*{\psi(t + \Delta t)} = \text{normalized $(1 + \Delta H_\text{eff} / \ii \hbar) \ket*{\psi(t)}$}) 
    = 1 - \expval*{C^\dagger C}{\psi(t)} \Delta t.
\end{equation}
The former possibility is called a \concept{quantum jump}: a photon is emitted and observed, and now we are 
sure that the atom is on the ground state.

\begin{figure}
    \centering
    

\tikzset{every picture/.style={line width=0.75pt}} %set default line width to 0.75pt        

\begin{tikzpicture}[x=0.75pt,y=0.75pt,yscale=-1,xscale=1]
%uncomment if require: \path (0,300); %set diagram left start at 0, and has height of 300

%Straight Lines [id:da6247332483497259] 
\draw    (116,261) -- (455.02,261) ;
\draw [shift={(457.02,261)}, rotate = 180] [fill={rgb, 255:red, 0; green, 0; blue, 0 }  ][line width=0.08]  [draw opacity=0] (12,-3) -- (0,0) -- (12,3) -- cycle    ;
%Straight Lines [id:da2835118791893678] 
\draw [color={rgb, 255:red, 208; green, 2; blue, 27 }  ,draw opacity=1 ]   (118,168.84) -- (199.02,168.84) ;
%Straight Lines [id:da007828056910233583] 
\draw [color={rgb, 255:red, 208; green, 2; blue, 27 }  ,draw opacity=1 ]   (199.02,168.84) -- (199.02,260.69) ;
%Straight Lines [id:da7024111687244319] 
\draw [color={rgb, 255:red, 74; green, 144; blue, 226 }  ,draw opacity=1 ]   (118,168.84) -- (247.02,168.84) ;
%Straight Lines [id:da5129694666706073] 
\draw [color={rgb, 255:red, 74; green, 144; blue, 226 }  ,draw opacity=1 ]   (247.02,168.84) -- (247.02,260.69) ;
%Straight Lines [id:da5606569788219309] 
\draw [color={rgb, 255:red, 245; green, 166; blue, 35 }  ,draw opacity=1 ]   (115.51,168.84) -- (282.53,168.84) ;
%Straight Lines [id:da9152358870457014] 
\draw [color={rgb, 255:red, 245; green, 166; blue, 35 }  ,draw opacity=1 ]   (282.53,168.84) -- (282.53,260.69) ;
%Straight Lines [id:da7239267784946593] 
\draw [color={rgb, 255:red, 184; green, 233; blue, 134 }  ,draw opacity=1 ]   (118,168.84) -- (358.51,168.84) ;
%Straight Lines [id:da704319248689653] 
\draw [color={rgb, 255:red, 184; green, 233; blue, 134 }  ,draw opacity=1 ]   (358.51,168.84) -- (358.51,260.69) ;
%Curve Lines [id:da28503514308489986] 
\draw  [dash pattern={on 4.5pt off 4.5pt}]  (116,168.84) .. controls (185.02,208.69) and (355.02,254.69) .. (437.02,256.69) ;
%Straight Lines [id:da6548127862421653] 
\draw    (116,261) -- (116,78.69) ;
\draw [shift={(116,76.69)}, rotate = 90] [fill={rgb, 255:red, 0; green, 0; blue, 0 }  ][line width=0.08]  [draw opacity=0] (12,-3) -- (0,0) -- (12,3) -- cycle    ;

% Text Node
\draw (459.02,261) node [anchor=west] [inner sep=0.75pt]    {$t$};
% Text Node
\draw (116,73.29) node [anchor=south] [inner sep=0.75pt]    {$|c_{\text{e}} |^{2}$};


\end{tikzpicture}

    \caption{Random wave function calculating of a two-level atom. The lines with color are possible 
    evolution path of the wave function, and after averaging over them we get the dashed exponential decay
    line.}
    \label{fig:excited-decay}
\end{figure}

Random wave function is an intuitive approach. If, for example, we start from a excited state, 
what we will get is \prettyref{fig:excited-decay}. 

We consider another example. Suppose from $0$ to $t$ there is no quantum jump, and we want to know 
what is the probability that a quantum jump happens in the time duration $t$ to $t + \Delta t$.
We assume $\Omega \ll \Gamma$, and we have 
\begin{equation}
    c_\text{g} \approx 1, \quad c_\text{e} = \frac{\Omega}{2} 
    \frac{1 - \ee^{- \ii \Delta t - \Omega t / 2}}{\Delta - \ii \Gamma / 2}.
\end{equation}
We find that 
\begin{equation}
    P(\text{jump during $t$ to $t + \Delta t$}) = \gamma \Delta t,
\end{equation}
where 
\begin{equation}
    \gamma = \frac{\abs{\Omega}^2}{4 \Delta^2 + \Gamma^2 } \Gamma.
\end{equation}

We can also use the random wave function method to calculate polarization in materials.
We find the wave function
\begin{equation}
    \ket*{\psi(t)} \approx \frac{1}{\Delta - \ii \Gamma / 2} \frac{\Omega}{2} \ket*{\text{e}} + \ket*{\text{g}}
\end{equation}
is a meta-stable solution of the random evolution, and we have 
\begin{equation}
    \expval*{d} = \frac{d_\text{eg} \Omega / 2}{\Delta - \ii \Gamma / 2} \ee^{- \ii \omega t} + \text{h.c.}.
\end{equation}
We therefore have 
\begin{equation}
    \alpha = \frac{\abs{d_\text{eg}}^2}{\omega_\text{eg} - \ii \Gamma / 2 - \omega}.
    \label{eq:polarization}
\end{equation}
The refractive index is now given by $n = \rho \alpha$. For glass, $\omega_\text{eg}$ is usually in the UV 
spectrum, so for the visible spectrum, $\omega_\text{eg} - \omega$ is large, and therefore absorption 
(or in other words random scattering) is not that important.

\begin{figure}
    \centering
    

\tikzset{every picture/.style={line width=0.75pt}} %set default line width to 0.75pt        

\begin{tikzpicture}[x=0.75pt,y=0.75pt,yscale=-1,xscale=1]
%uncomment if require: \path (0,300); %set diagram left start at 0, and has height of 300

%Straight Lines [id:da9652880271919453] 
\draw    (136,260) -- (475.02,260) ;
\draw [shift={(477.02,260)}, rotate = 180] [fill={rgb, 255:red, 0; green, 0; blue, 0 }  ][line width=0.08]  [draw opacity=0] (12,-3) -- (0,0) -- (12,3) -- cycle    ;
%Straight Lines [id:da6511106676108389] 
\draw    (136,260) -- (136,77.69) ;
\draw [shift={(136,75.69)}, rotate = 90] [fill={rgb, 255:red, 0; green, 0; blue, 0 }  ][line width=0.08]  [draw opacity=0] (12,-3) -- (0,0) -- (12,3) -- cycle    ;
%Shape: Wave [id:dp021604223743891593] 
\draw   (324,217.06) .. controls (319.92,195.03) and (316.02,174.06) .. (311.5,174.06) .. controls (306.98,174.06) and (303.08,195.03) .. (299,217.06) .. controls (294.92,239.09) and (291.02,260.06) .. (286.5,260.06) .. controls (281.98,260.06) and (278.08,239.09) .. (274,217.06) .. controls (269.92,195.03) and (266.02,174.06) .. (261.5,174.06) .. controls (256.98,174.06) and (253.08,195.03) .. (249,217.06) .. controls (244.92,239.09) and (241.02,260.06) .. (236.5,260.06) .. controls (231.98,260.06) and (228.08,239.09) .. (224,217.06) .. controls (219.92,195.03) and (216.02,174.06) .. (211.5,174.06) .. controls (206.98,174.06) and (203.08,195.03) .. (199,217.06) .. controls (194.92,239.09) and (191.02,260.06) .. (186.5,260.06) .. controls (181.98,260.06) and (178.08,239.09) .. (174,217.06) .. controls (169.92,195.03) and (166.02,174.06) .. (161.5,174.06) .. controls (156.98,174.06) and (153.08,195.03) .. (149,217.06) .. controls (144.92,239.09) and (141.02,260.06) .. (136.5,260.06) .. controls (136.33,260.06) and (136.17,260.03) .. (136,259.98) ;
%Straight Lines [id:da04371660203791228] 
\draw    (324,217.06) -- (324,260.06) ;
%Shape: Wave [id:dp5437918145745864] 
\draw   (412,217.06) .. controls (407.92,195.03) and (404.02,174.06) .. (399.5,174.06) .. controls (394.98,174.06) and (391.08,195.03) .. (387,217.06) .. controls (382.92,239.09) and (379.02,260.06) .. (374.5,260.06) .. controls (369.98,260.06) and (366.08,239.09) .. (362,217.06) .. controls (357.92,195.03) and (354.02,174.06) .. (349.5,174.06) .. controls (344.98,174.06) and (341.08,195.03) .. (337,217.06) .. controls (332.92,239.09) and (329.02,260.06) .. (324.5,260.06) .. controls (324.33,260.06) and (324.17,260.03) .. (324,259.98) ;
%Straight Lines [id:da9781524694295942] 
\draw    (346,143.06) -- (325.63,204.17) ;
\draw [shift={(325,206.06)}, rotate = 288.43] [fill={rgb, 255:red, 0; green, 0; blue, 0 }  ][line width=0.08]  [draw opacity=0] (12,-3) -- (0,0) -- (12,3) -- cycle    ;

% Text Node
\draw (479.02,260) node [anchor=west] [inner sep=0.75pt]    {$t$};
% Text Node
\draw (136,72.29) node [anchor=south] [inner sep=0.75pt]    {$|c_{\text{e}} |^{2}$};
% Text Node
\draw (317,117.06) node [anchor=north west][inner sep=0.75pt]   [align=left] {phase shift};


\end{tikzpicture}

    \caption{Time evolution in the strong coupling limit.}
\end{figure}

We now consider the $\Omega \gg \Gamma$ case. The wave function 
\begin{equation}
    \ket*{\psi(t)} = \cos \frac{\Omega}{2} t \ket*{\text{g}} - \ii \sin \frac{\Omega}{2} t \ket*{\text{e}}
\end{equation}
is a meta-stable state, but as time goes by, the probability of a quantum jump accumulates. 
A single time evolution trajectory is like 

\eqref{eq:markov-eq} is a quantum master equation. We can, actually, write down a non-Hermitian Hamiltonian
\begin{equation}
    H_\text{eff} = - \frac{\ii \hbar \Gamma}{2} \dyad{\text{e}} 
    + \hbar \sum_{k} g_k^* \ee^{ \ii \Delta_k t} a_k^\dagger \dyad{g}{e} + \text{h.c.}.
\end{equation}
which reproduces \eqref{eq:markov-eq}. We can replace the second and the third term using rotational 
wave approximation in \href{10-28.pdf}{this lecture}, and after averaging over all paths we get a 
damped oscillating curve.

\end{document}