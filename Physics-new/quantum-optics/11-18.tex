\documentclass[hyperref, a4paper]{article}

\usepackage{geometry}
\usepackage{titling}
\usepackage{titlesec}
% No longer needed, since we will use enumitem package
% \usepackage{paralist}
\usepackage{enumitem}
\usepackage{footnote}
\usepackage{enumerate}
\usepackage{amsmath, amssymb, amsthm}
\usepackage{mathtools}
\usepackage{bbm}
\usepackage{cite}
\usepackage{graphicx}
\usepackage{subfigure}
\usepackage{physics}
\usepackage{tensor}
\usepackage{siunitx}
\usepackage[version=4]{mhchem}
\usepackage{tikz}
\usepackage{xcolor}
\usepackage{listings}
\usepackage{autobreak}
\usepackage[ruled, vlined, linesnumbered]{algorithm2e}
\usepackage{nameref,zref-xr}
\zxrsetup{toltxlabel}
\zexternaldocument*[optics-]{../optics/optics}[optics.pdf]
\zexternaldocument*[solid-]{../solid/solid}[solid.pdf]
\zexternaldocument*[info-]{../information/quantum-circuit}[quantum-circuit.pdf]
\zexternaldocument*[1028-]{./10-28}[10-28.pdf]
\usepackage[colorlinks,unicode]{hyperref} % , linkcolor=black, anchorcolor=black, citecolor=black, urlcolor=black, filecolor=black
\usepackage{prettyref}

% Page style
\geometry{left=3.18cm,right=3.18cm,top=2.54cm,bottom=2.54cm}
\titlespacing{\paragraph}{0pt}{1pt}{10pt}[20pt]
\setlength{\droptitle}{-5em}
\preauthor{\vspace{-10pt}\begin{center}}
\postauthor{\par\end{center}}

% More compact lists 
\setlist[itemize]{
    itemindent=17pt, 
    leftmargin=1pt,
    listparindent=\parindent,
    parsep=0pt,
}

% Math operators
\DeclareMathOperator{\timeorder}{\mathcal{T}}
\DeclareMathOperator{\diag}{diag}
\DeclareMathOperator{\legpoly}{P}
\DeclareMathOperator{\primevalue}{P}
\DeclareMathOperator{\sgn}{sgn}
\newcommand*{\ii}{\mathrm{i}}
\newcommand*{\ee}{\mathrm{e}}
\newcommand*{\const}{\mathrm{const}}
\newcommand*{\suchthat}{\quad \text{s.t.} \quad}
\newcommand*{\argmin}{\arg\min}
\newcommand*{\argmax}{\arg\max}
\newcommand*{\normalorder}[1]{: #1 :}
\newcommand*{\pair}[1]{\langle #1 \rangle}
\newcommand*{\fd}[1]{\mathcal{D} #1}
\DeclareMathOperator{\bigO}{\mathcal{O}}

% TikZ setting
\usetikzlibrary{arrows,shapes,positioning}
\usetikzlibrary{arrows.meta}
\usetikzlibrary{decorations.markings}
\tikzstyle arrowstyle=[scale=1]
\tikzstyle directed=[postaction={decorate,decoration={markings,
    mark=at position .5 with {\arrow[arrowstyle]{stealth}}}}]
\tikzstyle ray=[directed, thick]
\tikzstyle dot=[anchor=base,fill,circle,inner sep=1pt]

% Algorithm setting
% Julia-style code
\SetKwIF{If}{ElseIf}{Else}{if}{}{elseif}{else}{end}
\SetKwFor{For}{for}{}{end}
\SetKwFor{While}{while}{}{end}
\SetKwProg{Function}{function}{}{end}
\SetArgSty{textnormal}

\newcommand*{\concept}[1]{{\textbf{#1}}}

% Embedded codes
\lstset{basicstyle=\ttfamily,
  showstringspaces=false,
  commentstyle=\color{gray},
  keywordstyle=\color{blue}
}

\newcommand{\opticsdoc}{\href{../optics/optics}{the optics note}}
\newcommand{\soliddoc}{\href{../solid/solid}{the solid state physics note}}
\newcommand{\infodoc}{\href{../information/quantum-circuit}{the quantum information note}}
\newcommand{\atomicintrodoc}{\href{./10-28}{this note}}

\newrefformat{fig}{Figure~\ref{#1} on page~\pageref{#1}}
\newrefformat{sec}{Section~\ref{#1}}

\title{Quantum Optics by Prof. Saijun Wu}
\author{Jinyuan Wu}
\date{November 18, 2021}

\begin{document}

\maketitle

\section{Existence of back-reaction from the atom to the optical field}

In the semi-classical approximation made in Section~\ref{1028-sec:semi-classical} in \atomicintrodoc,
we assume the optical field does not feel the back-reaction of the atom, which is certainly impossible.
In this section we discuss how to include the back-interaction.

The Hamiltonian is again 
\begin{equation}
    H = \frac{\hbar}{2} \vb*{\Omega} \cdot \vb*{\sigma},
\end{equation}
where 
\begin{equation}
    \vb*{\Omega} = (\Omega, 0, 0), \quad \Omega = \frac{\vb*{\mathcal{E}} \cdot \vb*{d}_{eg}}{\hbar}.
\end{equation}
The initial state is $\ket*{\psi(0)} = \ket*{g}$, and we have 
\begin{equation}
    \ket*{\psi(t)} = \cos \frac{\Omega t}{2} \ket*{g} + \ii \sin \frac{\Omega t}{2} \ket*{e}.
\end{equation}

Now we calculate the expectation of the electric dipole.
Under the interaction picture we picked up in \atomicintrodoc{} (see \eqref{1028-eq:dipole-interaction-picture})
we have 
\begin{equation}
    \vb*{d} = \vb*{d}_{eg} \ee^{\ii \omega_{eg} t } \dyad*{e}{g} + \text{h.c.},
\end{equation}
and thus we have 
\begin{equation}
    \expval*{\vb*{d}} = \vb*{d}_{eg} \sin \Omega t \sin \omega_{eg} t .
\end{equation}
We can see the expectation of the electric dipole has four frequency components, which are $\pm (\Omega \pm \omega_{eg})$,
and therefore the atom will radiate electromagnetic wave in these frequencies.
This is quite reasonable because the atom is receiving constant energy input, and we can expect somehow there is energy loss.
This example of a two-level atom in a vibrating external field means we have \concept{stimulated emission}.

Let us see another example. If we set 
\begin{equation}
    \Omega = \begin{cases}
        \frac{\pi}{2 T}, &\quad 0 < t < T, \\
        0,               &\quad \text{otherwise}.
    \end{cases}
\end{equation}
After $t = T$ we have 
\begin{equation}
    \ket*{\psi} = \frac{1}{\sqrt{2}} (\ket*{g} + \ii \ket*{e}).
\end{equation}
Again we calculate the expectation of the electric dipole, and we have 
\begin{equation}
    \expval*{\vb*{d}} = \vb*{d}_{eg} \sin \omega_{eg} t,
\end{equation}
and again we expect radiation from the atom with frequency $\omega_{eg}$.
This example hints that an excited atom may radiate and this is \concept{spontaneous emission}.

\section{Spontaneous emission of a two-level atom}

Now we consider the joint (quantum) system of the atom and the optical field.
We make the approximation that the optical field may either be on the vacuum state or on the one-photon state, 
and intuitively, we assume that when the atom is on the ground state there is a photon and when the atom is 
on the excited state there is no photon, so the joint wave function is
\begin{equation}
    \ket*{\Phi(t)} = c_e \ket*{e, 0} + \sum_{k} c_k \ket*{g, 1_k}.
\end{equation}
The interaction Hamiltonian is 
\begin{equation}
    H_\text{I} = - \sum_k g_k \ee^{\ii \Delta_k t} a_k \dyad*{e}{g} + \text{h.c.},
\end{equation}
where 
\begin{equation}
    \Delta_k = \omega_k - \omega_{eg},
\end{equation}
and
\begin{equation}
    g_k = \sqrt{\frac{\hbar \omega_k}{2 \epsilon_0 V}} \frac{\vb*{f}_k \cdot \vb*{d}_{eg}}{\hbar}.
\end{equation}
The time evolution is given by
\begin{equation}
    \ii \dot{c}_e = \sum_k g_k \ee^{\ii \Delta_k t} c_k, \quad \ii \dot{c}_k = g_k \ee^{- \ii \Delta_k t} c_e.
\end{equation}

We consider the example of spontaneous emission. We set 
\begin{equation}
    \ket*{\Phi(0)} = \ket*{e, 0},
\end{equation}
and further we assume that 
\begin{equation}
    \sum_k \abs*{c_k(t)}^2 \ll 1.
\end{equation}
The first order perturbation is 
\[
    c^{(1)}_k(T) = - \ii \int_0^T \dd{t} \ee^{- \ii \Delta_k t} c^{(0)}_e = - g_k \frac{\ee^{- \ii \Delta_k t} - 1}{\Delta_k},
\]
and thus we have 
\[
    \abs*{c_k(t)}^2 = \abs*{g_k}^2 \frac{t^2 \sin^2 \Delta_k t / 2}{(\Delta_k t / 2)^2} .
\]
If we measure the system at $t$, then the probability to find a photon (i.e. spontaneous emission happens) is 
\[
    P_\text{emission}(t) = \sum_k \abs*{c_k(t)}^2 = \sum_k \abs*{g_k}^2 \frac{ \sin^2 \Delta_k t / 2}{(\Delta_k / 2)^2} .
\]
The transition rate - which is defined by thinking \eqref{eq:original-spontaneous-emission} as a classical random process - is
\begin{equation}
    \begin{aligned}
        \Gamma &= \dv{P_\text{emission}}{t} = 
        \sum_k \abs*{g_k}^2 \frac{(\Delta_k / 2) \times 2 \sin \Delta_k t / 2 \cos \Delta_k t / 2 }{(\Delta_k / 2)^2} \\
        &= \sum_k \abs*{g_k}^2 \frac{2 \sin \Delta_k t  }{\Delta_k}.
    \end{aligned}
    \label{eq:original-spontaneous-emission}
\end{equation}

We try to evaluate \eqref{eq:original-spontaneous-emission} in the free space, where $k = (\vb*{k}, \sigma)$.
We have 
\[
    \sum_k \abs*{g_k}^2 \frac{2 \sin \Delta_k t  }{\Delta_k} 
    = \int \dd{\omega} g(\omega) |g_k|^2 \frac{2 \sin \Delta_k t}{\Delta_k},
\]
where the $g_k$ on the RHS is rephrase into a function of $\omega$ and some additional labels that distinguish degenerate states.
We see that when $t$ is large compared to $\omega_{eg}$, 
the RHS is proportion to $t$ when $\Delta_k = 0$ and is almost zero otherwise, 
and therefore we may think ${2 \sin \Delta_k t} / {\Delta_k}$ as an approximation of the Dirac $\delta$-function.
Note that 
\[
    \int_{-\infty}^\infty \dd{\omega} \frac{2 \sin \Delta_k t}{\Delta_k} = 2 \pi \delta(\Delta_k),
\]
we get 
\begin{equation}
    \Gamma = \int \dd{\omega} g(\omega) \abs*{g_k}^2 2\pi \delta(\omega - \omega_{eg}).
    \label{eq:fermi-golden-rule}
\end{equation}
It can be easily found that we are just proving Fermi golden rule for this specific case.
The final result is 
\begin{equation}
    \Gamma = \frac{\omega_{eg}^3 \abs*{\vb*{d}_{eg}}^2 }{3 \pi \epsilon_0 c^3 \hbar} 
    = \frac{4 \pi}{3} \omega_{eg} \frac{a_\text{Bohr}^2}{\lambda_{eg}^2} \alpha^3.
    \label{eq:spontaneous-emission-free-space}
\end{equation}

Similar procedures can be applied to other atomic models. For example, for a Rydberg atom, we have 
\begin{equation}
    \vb*{d}_{eg} \propto n^2 e \gamma_\text{Bohr}, 
    \quad \omega_{eg} \propto \frac{1}{n^2} - \frac{1}{(n-1)^2} \propto \frac{1}{n^3},
\end{equation}
so finally we have 
\begin{equation}
    \Gamma \sim \Gamma_{2 \text{P}} \frac{1}{n^5}.
\end{equation}
This means that high excited states are actually \emph{more} stable, with a long lifetime and a narrow line width.

Note that the above derivation has strict requirement on $t$: it must be small enough to make the first order perturbation work,
while it must be large enough compared to $\omega_{eg}$.
For a system with strong atom-light coupling, the two conditions contradict with each other, and we need to take the interaction 
between the atom and the optical field more seriously instead of using a first order perturbative result.
This is often dubbed as \concept{cavity QED}.

The fact that when $t$ is small compared to $\omega_{eg}$ \eqref{eq:spontaneous-emission-free-space} does not work, on the other hand,  
gives rise to \concept{quantum Zeno effect}, which is a thought experiment where the atom is surrounded by a lot of observers
and the duration between two measurements is smaller than $1 / \omega_{eg}$. In such a system the atom stays at the excited state
and there is no spontaneous emission.

Spontaneous emission of two atoms may affect each other. Suppose we have two two-level atoms, and we let 
\begin{equation}
    \ket*{\psi_+(t=0)} = \frac{1}{\sqrt{2}} (\ket*{g_1, e_2, 0} + \ket*{g_2, e_1, 0}), \quad 
    \ket*{\psi_-(t=0)} = \frac{1}{\sqrt{2}} (\ket*{g_1, e_2, 0} - \ket*{g_2, e_1, 0}),
\end{equation}
and we find that 
\begin{equation}
    \Gamma_{+ \to g} = 2 \Gamma_e, \quad \Gamma_{- \to 0} = 0.
\end{equation}

\section{Markov description of spontaneous emission}

After obtain \eqref{eq:spontaneous-emission-free-space} we can describe 
\begin{equation}
    \dot{c}_e = - c_e 
\end{equation}

\end{document}