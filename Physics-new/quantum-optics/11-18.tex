\documentclass[hyperref, a4paper]{article}

\usepackage{geometry}
\usepackage{titling}
\usepackage{titlesec}
% No longer needed, since we will use enumitem package
% \usepackage{paralist}
\usepackage{enumitem}
\usepackage{footnote}
\usepackage{enumerate}
\usepackage{amsmath, amssymb, amsthm}
\usepackage{mathtools}
\usepackage{bbm}
\usepackage{cite}
\usepackage{graphicx}
\usepackage{subfigure}
\usepackage{physics}
\usepackage{tensor}
\usepackage{siunitx}
\usepackage[version=4]{mhchem}
\usepackage{tikz}
\usepackage{xcolor}
\usepackage{listings}
\usepackage{autobreak}
\usepackage[ruled, vlined, linesnumbered]{algorithm2e}
\usepackage{nameref,zref-xr}
\zxrsetup{toltxlabel}
\zexternaldocument*[optics-]{../optics/optics}[optics.pdf]
\zexternaldocument*[solid-]{../solid/solid}[solid.pdf]
\zexternaldocument*[info-]{../information/quantum-circuit}[quantum-circuit.pdf]
\zexternaldocument*[1028-]{./10-28}[10-28.pdf]
\zexternaldocument*[122-]{./12-2}[12-2.pdf]
\zexternaldocument*[1110-]{../advanced-electrodynamics/lecture-11-10}[lecture-11-10.pdf]
\usepackage[colorlinks,unicode]{hyperref} % , linkcolor=black, anchorcolor=black, citecolor=black, urlcolor=black, filecolor=black
\usepackage{prettyref}

% Page style
\geometry{left=3.18cm,right=3.18cm,top=2.54cm,bottom=2.54cm}
\titlespacing{\paragraph}{0pt}{1pt}{10pt}[20pt]
\setlength{\droptitle}{-5em}
\preauthor{\vspace{-10pt}\begin{center}}
\postauthor{\par\end{center}}

% More compact lists 
\setlist[itemize]{
    itemindent=17pt, 
    leftmargin=1pt,
    listparindent=\parindent,
    parsep=0pt,
}

% Math operators
\DeclareMathOperator{\timeorder}{\mathcal{T}}
\DeclareMathOperator{\diag}{diag}
\DeclareMathOperator{\legpoly}{P}
\DeclareMathOperator{\primevalue}{P}
\DeclareMathOperator{\sgn}{sgn}
\newcommand*{\ii}{\mathrm{i}}
\newcommand*{\ee}{\mathrm{e}}
\newcommand*{\const}{\mathrm{const}}
\newcommand*{\suchthat}{\quad \text{s.t.} \quad}
\newcommand*{\argmin}{\arg\min}
\newcommand*{\argmax}{\arg\max}
\newcommand*{\normalorder}[1]{: #1 :}
\newcommand*{\pair}[1]{\langle #1 \rangle}
\newcommand*{\fd}[1]{\mathcal{D} #1}
\DeclareMathOperator{\bigO}{\mathcal{O}}

% TikZ setting
\usetikzlibrary{arrows,shapes,positioning}
\usetikzlibrary{arrows.meta}
\usetikzlibrary{decorations.markings}
\tikzstyle arrowstyle=[scale=1]
\tikzstyle directed=[postaction={decorate,decoration={markings,
    mark=at position .5 with {\arrow[arrowstyle]{stealth}}}}]
\tikzstyle ray=[directed, thick]
\tikzstyle dot=[anchor=base,fill,circle,inner sep=1pt]

% Algorithm setting
% Julia-style code
\SetKwIF{If}{ElseIf}{Else}{if}{}{elseif}{else}{end}
\SetKwFor{For}{for}{}{end}
\SetKwFor{While}{while}{}{end}
\SetKwProg{Function}{function}{}{end}
\SetArgSty{textnormal}

\newcommand*{\concept}[1]{{\textbf{#1}}}

% Embedded codes
\lstset{basicstyle=\ttfamily,
  showstringspaces=false,
  commentstyle=\color{gray},
  keywordstyle=\color{blue}
}

\newcommand{\opticsdoc}{\href{../optics/optics}{the optics note}}
\newcommand{\soliddoc}{\href{../solid/solid}{the solid state physics note}}
\newcommand{\infodoc}{\href{../information/quantum-circuit}{the quantum information note}}
\newcommand{\atomicintrodoc}{\href{./10-28}{this note}}

\newrefformat{fig}{Figure~\ref{#1} on page~\pageref{#1}}
\newrefformat{sec}{Section~\ref{#1}}

\title{Quantum Optics by Prof. Saijun Wu}
\author{Jinyuan Wu}
\date{November 18, 2021}

\begin{document}

\maketitle

\section{Existence of back-reaction from the atom to the optical field}

In the semi-classical approximation made in Section~\ref{1028-sec:semi-classical} in \atomicintrodoc,
we assume the optical field does not feel the back-reaction of the atom, which is certainly impossible.
In this section we discuss how to include the back-interaction.

The Hamiltonian is again 
\begin{equation}
    H = \frac{\hbar}{2} \vb*{\Omega} \cdot \vb*{\sigma}, \quad \vb*{\Omega} = (\Omega, 0, 0), \quad \Omega = \frac{\vb*{\mathcal{E}} \cdot \vb*{d}_{eg}}{\hbar}.
\end{equation}
The initial state is $\ket*{\psi(0)} = \ket*{g}$, and we have 
\begin{equation}
    \ket*{\psi(t)} = \cos \frac{\Omega t}{2} \ket*{g} + \ii \sin \frac{\Omega t}{2} \ket*{e}.
\end{equation}

Now we calculate the expectation of the electric dipole.
Under the interaction picture we picked up in \atomicintrodoc{} (see \eqref{1028-eq:dipole-interaction-picture})
we have 
\begin{equation}
    \vb*{d} = \vb*{d}_{eg} \ee^{\ii \omega_{eg} t } \dyad*{e}{g} + \text{h.c.},
\end{equation}
and thus we have 
\begin{equation}
    \expval*{\vb*{d}} = \vb*{d}_{eg} \sin \Omega t \sin \omega_{eg} t .
\end{equation}
We can see the expectation of the electric dipole has four frequency components, which are $\pm (\Omega \pm \omega_{eg})$,
and therefore the atom will radiate electromagnetic wave in these frequencies.
This is quite reasonable because the atom is receiving constant energy input, and we can expect somehow there is energy loss.
This example of a two-level atom in a vibrating external field means we have \concept{stimulated emission}.

Let us see another example. If we set 
\begin{equation}
    \Omega = \begin{cases}
        \frac{\pi}{2 T}, &\quad 0 < t < T, \\
        0,               &\quad \text{otherwise}.
    \end{cases}
\end{equation}
After $t = T$ we have 
\begin{equation}
    \ket*{\psi} = \frac{1}{\sqrt{2}} (\ket*{g} + \ii \ket*{e}).
\end{equation}
Again we calculate the expectation of the electric dipole, and we have 
\begin{equation}
    \expval*{\vb*{d}} = \vb*{d}_{eg} \sin \omega_{eg} t,
\end{equation}
and again we expect radiation from the atom with frequency $\omega_{eg}$.
This example hints that an excited atom may radiate and this is \concept{spontaneous emission}.

\section{Spontaneous emission of a two-level atom}

Now we consider the joint (quantum) system of the atom and the optical field.
We make the approximation that the optical field may either be on the vacuum state or on the one-photon state, 
and intuitively, we assume that when the atom is on the ground state there is a photon and when the atom is 
on the excited state there is no photon, so the joint wave function is
\begin{equation}
    \ket*{\Phi(t)} = c_e \ket*{e, 0} + \sum_{k} c_k \ket*{g, 1_k}.
\end{equation}
The interaction Hamiltonian is 
\begin{equation}
    H_\text{I} = \sum_k \hbar g_k \ee^{- \ii \Delta_k t} a_k \dyad*{e}{g} + \text{h.c.},
\end{equation}
where 
\begin{equation}
    \Delta_k = \omega_k - \omega_{eg},
\end{equation}
and
\begin{equation}
    g_k = \sqrt{\frac{\hbar \omega_k}{2 \epsilon_0 V}} \frac{\vb*{f}_k \cdot \vb*{d}_{eg}}{\hbar}.
\end{equation}
The time evolution is given by
\begin{equation}
    \ii \dot{c}_e = \sum_k g_k \ee^{- \ii \Delta_k t} c_k, \quad \ii \dot{c}_k = g_k \ee^{\ii \Delta_k t} c_e.
\end{equation}

We consider the example of spontaneous emission. We set 
\begin{equation}
    \ket*{\Phi(0)} = \ket*{e, 0},
\end{equation}
and further we assume that 
\begin{equation}
    \sum_k \abs*{c_k(t)}^2 \ll 1.
\end{equation}
The first order perturbation is 
\[
    c^{(1)}_k(t) = - \ii g_k \int_0^t \dd{\tau} \ee^{\ii \Delta_k \tau} c^{(0)}_e = - g_k \frac{\ee^{\ii \Delta_k t} - 1}{\Delta_k},
\]
and thus we have 
\[
    \abs*{c_k(t)}^2 = \abs*{g_k}^2 \frac{t^2 \sin^2 \Delta_k t / 2}{(\Delta_k t / 2)^2} .
\]
If we measure the system at $t$, then the probability to find a photon (i.e. spontaneous emission happens) is 
\[
    P_\text{emission}(t) = \sum_k \abs*{c_k(t)}^2 = \sum_k \abs*{g_k}^2 \frac{ \sin^2 \Delta_k t / 2}{(\Delta_k / 2)^2} .
\]
The transition rate - which is defined by thinking \eqref{eq:original-spontaneous-emission} as a classical random process - is
\begin{equation}
    \begin{aligned}
        \Gamma &= \dv{P_\text{emission}}{t} = 
        \sum_k \abs*{g_k}^2 \frac{(\Delta_k / 2) \times 2 \sin \Delta_k t / 2 \cos \Delta_k t / 2 }{(\Delta_k / 2)^2} \\
        &= \sum_k \abs*{g_k}^2 \frac{2 \sin \Delta_k t  }{\Delta_k}.
    \end{aligned}
    \label{eq:original-spontaneous-emission}
\end{equation}

We try to evaluate \eqref{eq:original-spontaneous-emission} in the free space or in a large box, 
where $k = (\vb*{k}, \sigma)$. We have 
\[
    \begin{aligned}
        \sum_k \abs*{g_k}^2 \frac{2 \sin \Delta_k t  }{\Delta_k} 
        &\to \int \dd[3]{\vb*{k}} \frac{V}{(2\pi)^3} \sum_\text{spins} \abs*{g_k}^2 \frac{2 \sin \Delta_k t  }{\Delta_k}  \\
        &= V \int \dd{\omega} g(\omega) \expval{|g_k|^2 \frac{2 \sin \Delta_k t}{\Delta_k}}_{\omega_{\vb*{k} \sigma} = \omega},
    \end{aligned}
\]
where $V$ is the volume of the box, $g(\omega)$ is the density of states, and the $k$ dependence on the RHS 
is averaged on the $\omega_{\vb*{k} \sigma} = \omega$ surface. We see that when $t$ is large compared to 
$1 / \omega_{eg}$, the RHS is proportion to $t$ when $\Delta_k = 0$ (but still small so that the first order 
perturbation theory works - see \prettyref{fig:pe-time-evo}) and is almost zero otherwise, 
and therefore we may think ${2 \sin \Delta_k t} / {\Delta_k}$ as an approximation of the Dirac $\delta$-function.
Note that 
\[
    \int_{-\infty}^\infty \dd{\omega} \frac{2 \sin \Delta_k t}{\Delta_k} = 2 \pi \delta(\Delta_k),
\]
we get 
\begin{equation}
    \Gamma = V \int \dd{\omega} g(\omega) 2\pi \delta(\omega - \omega_{eg}) \expval*{\abs*{g_k}^2}_{\omega = \omega_{\vb*{k} \sigma}}.
    \label{eq:fermi-golden-rule}
\end{equation}
It can be easily found that we are just proving Fermi golden rule for this specific case.
Actually, we can generalized the derivation above to the case where $\abs*{c_e}^2 < 1$, and in this case 
we will find that 
\[
    \dv{P_e}{t} = - \Gamma P_e,
\]
where $\Gamma$ is given by \eqref{eq:fermi-golden-rule}. Therefore the occupation of the excited state 
damps exponentially. We will redo the proof in a more systematic way with the Markovian approximation in 
Section~\ref{122-sec:spon-rad} in \href{12-2.pdf}{the next lecture}. 

\begin{figure}
    \centering
    

\tikzset{every picture/.style={line width=0.75pt}} %set default line width to 0.75pt        

\begin{tikzpicture}[x=0.75pt,y=0.75pt,yscale=-1,xscale=1]
%uncomment if require: \path (0,300); %set diagram left start at 0, and has height of 300

%Straight Lines [id:da3136645111128993] 
\draw    (175.35,229.69) -- (526,229.69) ;
\draw [shift={(528,229.69)}, rotate = 180] [fill={rgb, 255:red, 0; green, 0; blue, 0 }  ][line width=0.08]  [draw opacity=0] (12,-3) -- (0,0) -- (12,3) -- cycle    ;
%Straight Lines [id:da9561030204147447] 
\draw    (175.35,229.69) -- (175.35,77.34) ;
\draw [shift={(175.35,75.34)}, rotate = 90] [fill={rgb, 255:red, 0; green, 0; blue, 0 }  ][line width=0.08]  [draw opacity=0] (12,-3) -- (0,0) -- (12,3) -- cycle    ;
%Curve Lines [id:da24141871701343098] 
\draw    (175.35,122.52) .. controls (186,121.04) and (195,125.04) .. (200,134.04) ;
%Curve Lines [id:da6858896034833271] 
\draw    (200,134.04) .. controls (212.94,152.44) and (229.19,168.04) .. (247.72,180.89) .. controls (293.64,212.73) and (353.58,227.75) .. (412,227.04) ;
%Straight Lines [id:da841804291945832] 
\draw  [dash pattern={on 4.5pt off 4.5pt}]  (198,95) -- (198,229.04) ;
%Straight Lines [id:da39165318886202516] 
\draw  [dash pattern={on 4.5pt off 4.5pt}]  (405,95) -- (405,229.04) ;
%Curve Lines [id:da8947188003962334] 
\draw    (412,227.04) .. controls (431,226.04) and (454,226.04) .. (491,211.04) ;

% Text Node
\draw (175.35,71.94) node [anchor=south] [inner sep=0.75pt]    {$P_{e}$};
% Text Node
\draw (530,229.69) node [anchor=west] [inner sep=0.75pt]    {$t$};
% Text Node
\draw (198,232.44) node [anchor=north] [inner sep=0.75pt]    {$\sim \frac{1}{\omega _{eg}}$};
% Text Node
\draw (405,232.44) node [anchor=north] [inner sep=0.75pt]    {$\sim \frac{1}{g}$};
% Text Node
\draw (268,133) node [anchor=north west][inner sep=0.75pt]   [align=left] {Markovian};
% Text Node
\draw (449,133) node [anchor=north west][inner sep=0.75pt]   [align=left] {recurrence};


\end{tikzpicture}

    \caption{The time evolution of the occupation of the excited state. The Fermi golden rule only works for the 
    Markovian period. }
    \label{fig:pe-time-evo}
\end{figure}

Now it is time to explicitly complete the integral in \eqref{eq:fermi-golden-rule}. 
The density of states is given by \eqref{1110-eq:density-of-states-light}  
\href{../advanced-electrodynamics/lecture-11-10.pdf}{here}, which is 
\[
    \begin{aligned}
        \rho(\omega) &= 2 \times  \dv{k^2(\omega)}{\omega} \int \frac{\dd[3]{\vb*{q}}}{(2\pi)^3} \delta(k^2(\omega) - \abs*{\vb*{q}}^2) \\ 
        &= 2 \times \dv{k^2(\omega)}{\omega} \int \frac{\dd{S}}{(2\pi)^3} \frac{1}{\abs*{\grad_{\vb*{k}} \vb*{k}^2 }} \\
        &= 2 \times \frac{2 \omega}{c^2} \frac{1}{(2\pi)^3} \frac{4 \pi \vb*{k}^2}{2 \abs*{\vb*{k}}} \\
        &= \frac{\omega^2}{\pi^2 c^3} .
    \end{aligned}
\]
Note that the $2$ factor comes from the fact that a momentum mode has two possible polarization.
The square of the coupling constant is 
\begin{equation}
    \abs*{g_{\vb*{k} \sigma}}^2 = \frac{\omega_{\vb*{k} \sigma}}{2 \epsilon_0 V \hbar} \abs*{\vu*{\epsilon}_{\sigma} \cdot \vb*{d}_{eg}}^2,
\end{equation}
and since on the $\omega_{\vb*{k} \sigma} = \omega$ surface, $\vb*{k}$ distributes uniformly, we can also 
find that 
\[
    \expval*{(\epsilon_{\vb*{k} \sigma})_x^2} = \expval*{(\epsilon_{\vb*{k} \sigma})_y^2} = \expval*{(\epsilon_{\vb*{k} \sigma})_z^2} = \frac{1}{3}, \quad 
    \expval*{(\epsilon_{\vb*{k} \sigma})_x (\epsilon_{\vb*{k} \sigma})_y} = 
    \expval*{(\epsilon_{\vb*{k} \sigma})_y (\epsilon_{\vb*{k} \sigma})_z} = 
    \expval*{(\epsilon_{\vb*{k} \sigma})_z (\epsilon_{\vb*{k} \sigma})_x} = 0,
\]
so 
\[
    \expval*{\abs*{g_{\vb*{k} \sigma}}^2} = \frac{\omega_{\vb*{k} \sigma}}{2 \epsilon_0 V \hbar} \frac{1}{3} \abs*{\vb*{d}_{eg}}^2.
\]
Putting these factors into \eqref{eq:fermi-golden-rule}, we get the famous spontaneous radiation rate
\begin{equation}
    \Gamma = \frac{\omega_{eg}^3 \abs*{\vb*{d}_{eg}}^2 }{3 \pi \epsilon_0 c^3 \hbar} 
    = \frac{4 \pi}{3} \omega_{eg} \frac{a_\text{Bohr}^2}{\lambda_{eg}^2} \alpha^3.
    \label{eq:spontaneous-emission-free-space}
\end{equation}

The only fact about the two-level approximation used in the above derivation is the value of $\vb*{d}_{eg}$, 
and we can just apply \eqref{eq:spontaneous-emission-free-space} to other atomic models as long as we replace 
$\vb*{d}_{eg}$ by the dipole matrix elements between two energy levels of interest. For example, for a 
Rydberg atom, we have 
\begin{equation}
    \vb*{d}_{eg} \propto n^2 e \gamma_\text{Bohr}, 
    \quad \omega_{eg} \propto \frac{1}{n^2} - \frac{1}{(n-1)^2} \propto \frac{1}{n^3},
\end{equation}
so finally we have 
\begin{equation}
    \Gamma \sim \Gamma_{2 \text{P}} \frac{1}{n^5}.
\end{equation}
This means that high excited states are actually \emph{more} stable, with a long lifetime and a narrow line width.

Note that the above derivation has strict requirement on $t$: it must be small enough to make the first order perturbation work,
while it must be large enough compared to $\omega_{eg}$.
For a system with strong atom-light coupling, the two conditions contradict with each other, and we need to take the interaction 
between the atom and the optical field more seriously instead of using a first order perturbative result.
This is often dubbed \concept{cavity QED}.

The fact that when $t$ is small compared to $\omega_{eg}$ \eqref{eq:spontaneous-emission-free-space} does not work, on the other hand,  
gives rise to \concept{quantum Zeno effect}, which is a thought experiment where the atom is surrounded by a lot of observers
and the duration between two measurements is smaller than $1 / \omega_{eg}$. In such a system the atom stays at the excited state
and there is no spontaneous emission.

\begin{figure}
    \centering
    

% Gradient Info
  
\tikzset {_wou27m9t1/.code = {\pgfsetadditionalshadetransform{ \pgftransformshift{\pgfpoint{0 bp } { 0 bp }  }  \pgftransformrotate{-90 }  \pgftransformscale{2 }  }}}
\pgfdeclarehorizontalshading{_mx03xpnva}{150bp}{rgb(0bp)=(1,1,1);
rgb(37.5bp)=(1,1,1);
rgb(50.17857142857143bp)=(0,0,0);
rgb(62.5bp)=(1,1,1);
rgb(100bp)=(1,1,1)}
\tikzset{every picture/.style={line width=0.75pt}} %set default line width to 0.75pt        

\begin{tikzpicture}[x=0.75pt,y=0.75pt,yscale=-1,xscale=1]
%uncomment if require: \path (0,300); %set diagram left start at 0, and has height of 300

%Straight Lines [id:da30984289381495134] 
\draw    (243,235) -- (339,235) ;
%Straight Lines [id:da9971381482409483] 
\draw    (243,104) -- (339,104) ;
%Straight Lines [id:da8148668338099976] 
\draw    (178,170) -- (274,170) ;
%Straight Lines [id:da11538291197175354] 
\draw    (310,170) -- (406,170) ;
%Shape: Square [id:dp6882035746519533] 
\draw  [draw opacity=0][shading=_mx03xpnva,_wou27m9t1] (96,147) -- (142,147) -- (142,193) -- (96,193) -- cycle ;

% Text Node
\draw (291,100.6) node [anchor=south] [inner sep=0.75pt]    {$\ket{e_{1} ,e_{2} ,0}$};
% Text Node
\draw (291,231.6) node [anchor=south] [inner sep=0.75pt]    {$\ket{\psi_+}$};
% Text Node
\draw (226,166.6) node [anchor=south] [inner sep=0.75pt]    {$\ket{\psi_-}$};
% Text Node
\draw (358,166.6) node [anchor=south] [inner sep=0.75pt]    {$\ket{e_{1} ,g_{2} ,0}$};
% Text Node
\draw (121.53,141.1) node [anchor=south] [inner sep=0.75pt]    {$\ket{g_{1} ,g_{2} ,1_{k}}$};


\end{tikzpicture}

    \caption{The energy level configuration of two two-level atoms slightly coupled}
    \label{fig:energy-level-config}
\end{figure}

Suppose we have two two-level atoms which are  very close in position, and therefore the electric fields they 
feel are the same. The interaction Hamiltonian is 
\begin{equation}
    H_\text{int} = - \vb*{E} \cdot (\vb*{d}_1 + \vb*{d}_2).
\end{equation}
When there is no interaction between two atoms, the spontaneous radiation here has no difference with 
the single atom case. However, usually two very close atoms are coupled, and the energy eigenstates 
$\ket*{g_1, e_2, 0}$ and $\ket*{e_1, g_2, 0}$ are mixed into
\begin{equation}
    \ket*{\psi_+(t=0)} = \frac{1}{\sqrt{2}} (\ket*{g_1, e_2, 0} + \ket*{e_1, g_2, 0}), \quad 
    \ket*{\psi_-(t=0)} = \frac{1}{\sqrt{2}} (\ket*{g_1, e_2, 0} - \ket*{e_1, g_2, 0}),
\end{equation}
and we find that 
\[
    \mel{g_1, g_2, 0}{\vb*{d}_1 + \vb*{d}_2}{\psi_+} = \frac{1}{\sqrt{2}} (\vb*{d}_{eg} + \vb*{d}_{eg}), \quad
    \mel{g_1, g_2, 0}{\vb*{d}_1 + \vb*{d}_2}{ \psi_+} = \frac{1}{\sqrt{2}} (\vb*{d}_{eg} - \vb*{d}_{eg}),
\]
and therefore we have
\begin{equation}
    \Gamma_{+ \to g} = 2 \Gamma_e, \quad \Gamma_{- \to g} = 0.
\end{equation}
This is called \concept{Dicke superradiance}, where one state has a stronger tendency to spontaneously emit,
while another state almost has no spontaneous radiation.

\end{document}