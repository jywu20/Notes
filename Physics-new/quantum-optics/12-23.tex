\documentclass[hyperref, a4paper]{article}

\usepackage{geometry}
\usepackage{titling}
\usepackage{titlesec}
% No longer needed, since we will use enumitem package
% \usepackage{paralist}
\usepackage{enumitem}
\usepackage{footnote}
\usepackage{enumerate}
\usepackage{amsmath, amssymb, amsthm}
\usepackage{mathtools}
\usepackage{bbm}
\usepackage{cite}
\usepackage{graphicx}
\usepackage{subcaption}
\usepackage{physics}
\usepackage{tensor}
\usepackage{siunitx}
\usepackage[version=4]{mhchem}
\usepackage{tikz}
\usepackage{xcolor}
\usepackage{listings}
\usepackage{autobreak}
\usepackage[ruled, vlined, linesnumbered]{algorithm2e}
\usepackage{nameref,zref-xr}
\zxrsetup{toltxlabel}
% \zexternaldocument*[last-]{12-2}[12-2.pdf]
\usepackage[colorlinks,unicode]{hyperref} % , linkcolor=black, anchorcolor=black, citecolor=black, urlcolor=black, filecolor=black
\usepackage[most]{tcolorbox}
\usepackage{prettyref}

% Page style
\geometry{left=3.18cm,right=3.18cm,top=2.54cm,bottom=2.54cm}
\titlespacing{\paragraph}{0pt}{1pt}{10pt}[20pt]
\setlength{\droptitle}{-5em}
\preauthor{\vspace{-10pt}\begin{center}}
\postauthor{\par\end{center}}

% More compact lists 
\setlist[itemize]{
    itemindent=17pt, 
    leftmargin=1pt,
    listparindent=\parindent,
    parsep=0pt,
}

% Math operators
\DeclareMathOperator{\timeorder}{\mathcal{T}}
\DeclareMathOperator{\diag}{diag}
\DeclareMathOperator{\legpoly}{P}
\DeclareMathOperator{\primevalue}{P}
\DeclareMathOperator{\sgn}{sgn}
\newcommand*{\ii}{\mathrm{i}}
\newcommand*{\ee}{\mathrm{e}}
\newcommand*{\const}{\mathrm{const}}
\newcommand*{\suchthat}{\quad \text{s.t.} \quad}
\newcommand*{\argmin}{\arg\min}
\newcommand*{\argmax}{\arg\max}
\newcommand*{\normalorder}[1]{: #1 :}
\newcommand*{\pair}[1]{\langle #1 \rangle}
\newcommand*{\fd}[1]{\mathcal{D} #1}
\DeclareMathOperator{\bigO}{\mathcal{O}}

% TikZ setting
\usetikzlibrary{arrows,shapes,positioning}
\usetikzlibrary{arrows.meta}
\usetikzlibrary{decorations.markings}
\tikzstyle arrowstyle=[scale=1]
\tikzstyle directed=[postaction={decorate,decoration={markings,
    mark=at position .5 with {\arrow[arrowstyle]{stealth}}}}]
\tikzstyle ray=[directed, thick]
\tikzstyle dot=[anchor=base,fill,circle,inner sep=1pt]

% Algorithm setting
% Julia-style code
\SetKwIF{If}{ElseIf}{Else}{if}{}{elseif}{else}{end}
\SetKwFor{For}{for}{}{end}
\SetKwFor{While}{while}{}{end}
\SetKwProg{Function}{function}{}{end}
\SetArgSty{textnormal}

\newcommand*{\concept}[1]{{\textbf{#1}}}

% Embedded codes
\lstset{basicstyle=\ttfamily,
  showstringspaces=false,
  commentstyle=\color{gray},
  keywordstyle=\color{blue}
}

% Reference formatting
\newrefformat{fig}{Figure~\ref{#1} on page~\pageref{#1}}

% Color boxes
\tcbuselibrary{skins, breakable, theorems}
\newtcbtheorem[number within=section]{warning}{Warning}%
  {colback=orange!5,colframe=orange!65,fonttitle=\bfseries, breakable}{warn}
\newtcbtheorem[number within=section]{note}{Note}%
  {colback=green!5,colframe=green!65,fonttitle=\bfseries, breakable}{note}

\newcommand{\lastnote}{\href{12-9.pdf}{the last lecture}}

\title{Quantum Optics by Prof. Saijun Wu}
\author{Jinyuan Wu}

\begin{document}

\maketitle

\section{Optical mode in a cavity with a nanosphere}

In previous lectures we are discussing atoms in a given optical field. Now we turn to discuss the optical field 
itself in a cavity. 

We consider such a system: We have a cavity, in which there is a small sphere (yet still large enough so that quantum fluctuation is not that important), and only one optical mode we are interested in is coupled to 
the sphere. There are other optical modes coupled to the sphere, but we are not that interested in them, 
and therefore they effectively create a damping channel for the mode we are interested in. 
A beam of light is shed on the sphere.

The Hamiltonian of photons can be obtained by solving the optical modes of the cavity,
and the electric field component of mode $c$ - the mode we are interested in - is 
\begin{equation}
    \vb*{E}_c(\vb*{r}, t) = \sqrt{\frac{\hbar \omega_c}{2 \epsilon_0 V}} f(\vb*{r}) \ee^{- \ii \omega_c t} + \text{c.c}.
\end{equation}
For example, for a Fabry-Perot cavity, we have 
\begin{equation}
    \vb*{E}_c(\vb*{r}, t) = \sqrt{\frac{\hbar \omega_c}{2 \epsilon_0 V}} f(x, y) \cos(k_z z) \ee^{ - \ii \omega_c t} + \text{c.c}.
\end{equation}
We take a inverse approach of what has been used in previous lectures, trying not to explicitly include the sphere. 
Note, however, that the sphere is not a many-body system and the inner states of it cannot be simply ignored, 
and therefore the sphere is not a fixed external deriving field. We assume a large detuning, and the sphere 
just ``move according to the external field'', and in this way 
\begin{equation}
    \expval{\vb*{d}} \approx \alpha \vb*{E}_k.
\end{equation} 
In this way we do not need to rigorously integrate out the sphere. Therefore the interaction Hamiltonian 
is effectively 
\begin{equation}
    V = \sum_k \alpha \vb*{\mathcal{E}}(\vb*{r}_0)^* \vb*{\mathcal{E}}_k(\vb*{r_0}) a_k a^\dagger + \text{h.c.},
\end{equation}
where $a_k$ is the annihilation operator of a mode we are not interested in, and we have assume only 
\emph{one} mode in the cavity is strongly coupled to the sphere. After RWA, the interaction Hamiltonian is 
\begin{equation}
    H_\text{int} = \sum_k g_k \ee^{- \ii \Delta_k} a^\dagger a_k + \text{h.c.}, \quad \Delta_k = \omega - \omega_k, \quad g_k = \frac{\alpha}{\hbar} \vb*{\mathcal{E}}^*(\vb*{r}_0) \vb*{\mathcal{E}}_k(\vb*{r}_0).
\end{equation}

Now we solve the Hamiltonian. We work in the subspace where the photon number change is limited to 1, i.e.
\begin{equation}
    \ket*{\psi(t)} = c_n \ket*{n, 0} + \sum_k c_k \ket*{n-1, 1_k}.
\end{equation}
Unlike the case in the two-level atom system, this approximation may fail when the time is sufficiently long,
but we can assume that the environment is noisy enough so that an observation has already taken place before 
a second photon transition occurs. The time evolution equation turns into
\begin{equation}
    \ii \dot{c}_n = \sqrt{n} \sum_k g_k \ee^{\ii \Delta_k t} c_k, \quad 
    \ii \dot{c}_k = \sqrt{n} g_k^* \ee^{- \ii \Delta_k t} c_n. 
\end{equation} 
Again we can make a Markovian approximation. We have 
\[
    \begin{aligned}
        \ii \dot{c}_n &= \int_0^t \dd{\tau} n \sum_k \abs*{g_k}^2 \ee^{\ii \Delta_k (t - \tau)} c_n(\tau) \\
        &\eqqcolon n \int_0^t \dd{\tau } K(t - \tau) c(\tau),
    \end{aligned}
\]
and by the Markovian approximation, we have 
\begin{equation}
    \dot{c}_n(t) = n (\delta_c - \ii \kappa / 2) c_n(t),
\end{equation}
where 
\begin{equation}
    \kappa = 2 \pi \sum_k \abs*{g_k}^2 \delta(\omega - \omega_k) \sim \frac{\alpha^2 k_c^3 \abs*{\vb*{\mathcal{E}}_c}}{\hbar \epsilon_0},
\end{equation}
and 
\begin{equation}
    \delta_c \sim \primevalue \int_{-\infty}^\infty \rho(\omega) \abs*{g_k}^2 \frac{1}{\Delta_k} \dd{k}.
\end{equation}
Therefore we get an effective damping Hamiltonian 
\begin{equation}
    H_\text{eff} = - \frac{\ii n \kappa}{2} \dyad{n}, \quad C = \sqrt{n \kappa} \dyad*{n-1}{n}.
\end{equation}
Now we generalize to the second quantized formulation without explicit dependence on the photon number, and we have 
\begin{equation}
    H_\text{eff} = - \frac{\ii \kappa}{2} a^\dagger a, \quad C = \sqrt{\kappa} a.
\end{equation}
The probability of a quantum jump is 
\begin{equation}
    \var{P}_\text{jump} = \mel{\psi_\text{s}(t)}{C^\dagger C}{\psi_\text{s}(t)} \var{t} = \expval{n} \kappa \var{t} \eqqcolon \gamma \var{t}.
\end{equation}
The master equation is therefore 
\begin{equation}
    \dot{\rho} = - \frac{\kappa}{2} \left( a^\dagger a \rho + \rho a^\dagger a - 2 a \rho a^\dagger \right).
\end{equation}
If we turn the external driving field on, the Hamiltonian is now 
\begin{equation}
    H_\text{eff} = - \frac{\ii \kappa}{2} a^\dagger a + \hbar g \alpha \ee^{\ii \Delta_c t} + \text{h.c.}.
    \label{eq:damping-and-driving}
\end{equation}

Plotting the time evolution according to \eqref{eq:damping-and-driving} gives us a quite intuitive picture.
We see the last two terms correspond to a displacement operator, and the first term drags a coherent state 
to the vacuum. Note that even there is no quantum jump, since high photon number states damp faster than 
low photon number state, we can still see some explicit ``damping'' effector. For example, if the input 
state is a coherent state, we will see its amplitude damping.
If the input state is highly non-classical, we can see the quantum jumps destroy quantum interference.
For example, for an input cat state 
\[
    \ket*{\psi(0)} \approx \frac{1}{\sqrt{2}} (\ket*{\alpha \ee^{- \kappa t / 2}} + \ket*{- \alpha \ee^{- \kappa t / 2}}),
\]
and after a quantum jump we have 
\[
    \ket*{\psi(t)} \approx \frac{1}{\sqrt{2}} (\ket*{\alpha \ee^{- \kappa t / 2}} - \ket*{- \alpha \ee^{- \kappa t / 2}}).
\]
After several quantum jumps, the pure input state becomes a mixed state where the non-trivial relative phase factor 
between the two components of the wave functions vanish, and a cat state damps into a classical $1/2$ - $1/2$ 
probability distribution.

\section{Mode coupling and leaky cavities}

The last section is about energy loss induced by a nanosphere. Another way how energy leaks from the cavity is 
via the boundary of the cavity. The boundaries cannot be \SI{100}{\percent} reflective, and therefore the 
optical modes in the cavity are inevitably coupled to the external modes.

\section{Optical components in a cavity}

Now we consider an optical 

\section{Cavity QED}

When the small sphere is very, very small, the approximation $\vb*{d} \approx \expval{\vb*{d}} = \alpha \vb*{E}$
no longer works. The atom and the optical field must be quantized together. 


\end{document}