\documentclass[a4paper]{article}

\usepackage{geometry}
\usepackage{caption}
\usepackage{subcaption}
\usepackage{abstract}
\usepackage{paralist}
\usepackage{ulem}
\usepackage{amsmath, amssymb}
\usepackage{qtree}
\usepackage{gb4e}
\usepackage[colorlinks, urlcolor=cyan]{hyperref}
\usepackage{prettyref}

\geometry{left=3.18cm,right=3.18cm,top=2.54cm,bottom=2.54cm}

\title{Proclamation of the Birth of Christ}
\author{Jinyuan Wu}

\begin{document}

\maketitle

\section{The Text}

The Proclamation of the Birth of Christ is a chant sung before the midnight mass for christmas \cite{wiki}.
The chant is traditionally sung in Cantus Gregorianus melodies, some instances of which can be found in \cite{2016christmas, 2019christmas}. 
Some versions with English lyrics and Gregorian melodies also exist, for example \cite{eng-ver}.
There are several Latin versions of the chant and the current version is given as below: 

\begin{quotation}
    Octavo Kalendas ianuarii. Luna decima nona.
    Innumeris transactis saeculis a creatione mundi, quando in principio Deus creavit caelum et terram et hominem formavit ad imaginem suam;
    permultis etiam saeculis, ex quo post diluvium Altissimus in nubibus arcum posuerat, signum fœderis et pacis;
    a migratione Abrahae, patris nostri in fide, de Ur Chaldaeorum saeculo vigesimo primo;
    ab egressu populi Israël de Aegypto, Moyse duce, saeculo decimo tertio;
    ab unctione David in regem, anno circiter millesimo;
    hebdomada sexagesima quinta, iuxta Danielis prophetiam;
    Olympiade centesima nonagesima quarta;
    ab Urbe condita anno septingentesimo quinquagesimo secundo;
    anno imperii Caesaris Octaviani Augusti quadragesimo secundo;
    toto Orbe in pace composito, Iesus Christus, aeternus Deus aeternique Patris Filius, mundum volens adventu suo piissimo consecrare, de Spiritu Sancto conceptus, novemque post conceptionem decursis mensibus, in Bethlehem Iudae nascitur ex Maria Virgine factus homo:
    Nativitas Domini nostri Iesu Christi secundum carnem.
\end{quotation}

\section{Translation}

\subsection{Parsing of the First Several Sentences}

\begin{exe}
    \sn
    \gll (in) Octavo (ante) Kalendas ianuarii. Luna decima nona.  \\
    (on) 8th (before) the.first.day January.DAT Luna decima nona\\
    \glt  `On the 8th day before the first day of January' 

    \sn
    \gll Innumeris transactis saeculis a creatione mundi,  \\
     This is an example.\\
    \trans `Here's where the translation would be'
\end{exe}


\subsection{Translation of the rest}



\bibliographystyle{plain}
\bibliography{christmas-proclaim} 

\end{document}