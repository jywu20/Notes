\documentclass[a4paper, 12pt]{article}

\usepackage{libertinus} % global font
\usepackage{geometry}   % page layout
\usepackage{float}      % floating control
\usepackage{caption}
\usepackage{subcaption}
\usepackage{booktabs}
\usepackage{amsmath}
\usepackage{qtree}
\usepackage{gb4e}
\usepackage[colorlinks, urlcolor=cyan]{hyperref}
\usepackage{prettyref}

\geometry{left=3.18cm,right=3.18cm,top=2.54cm,bottom=2.54cm}

\newcommand*{\concept}[1]{\textbf{#1}}
\newcommand*{\term}[1]{\emph{#1}}
\newcommand{\form}[1]{\emph{#1}}
\newcommand*{\category}[1]{\textsc{#1}}
\newcommand{\translate}[1]{`#1'}
\newcommand*{\changeto}{$>$}
\newcommand*{\changefrom}{$<$}

\title{\form{Summa}, 1, 1}
\author{Jinyuan Wu}

\begin{document}

\maketitle

The glossing abbreviations used in this note 
follows what are used in Wikipedia.

\section{Adverbs and conjunctions appearing frequently}

\begin{table}[H]
    \caption{Frequent adverbs and conjunctions}
    \centering
    \begin{tabular}{ll}
        \toprule
        word               & meaning \\
        \midrule
        \form{autem}       & \translate{but, while, moreover, on the other hand}    \\
        \form{enim}        & \translate{because} \\
        \form{igitur}      & \translate{therefore (postpositive)}                   \\
        \form{sicut}       & \translate{just as}                                    \\
        \form{tantummodo}  & \translate{merely}                                     \\
        \form{quia}        & \translate{because} \\
        \form{nam}         & \translate{because} \\
        \form{rursus}      & \translate{again} \\
        \form{num}         & \translate{whether} \\
        \form{imm\={o}}    & \translate{on the contrary} \\
        \form{iam}         & \translate{now}  \\
        \form{itaque}      & \translate{in this way}  \\
        \form{dum}         & \translate{when} \\
        \form{aut}         & \translate{or} \\
        \form{quod}        & \translate{because, until} (also used as interrogative pronoun) \\
        \form{atque}       & \translate{and} \\
        \form{quid}        & \translate{why, why for} \\
        \form{qc}          & = \form{atque} \\
        \form{vel}         & \translate{even} \\
        \bottomrule
    \end{tabular}
\end{table}



\section{Frequent correlatives}

\form{quandam}, \form{aliam}

\section{Collection of information packaging constructions}

\form{huiusmodi similitudinibus uti non est conveniens huic scientiae}
\translate{in this way it's not fitting for this science to use similitudes}.
Note that this is an impersonal construction.



\section{Question 1.1, article 1}

\begin{exe}
    \ex \gll Praeterea, doctrina non potest esse nisi de ente, 
    nihil enim scitur nisi verum, 
    quod cum ente convertitur. \\
    besides doctrine-\category{sg.nom} not can-\category{3sg.act} be-\category{pres.inf}
    if.not about be-\category{prp}-\category{sg.abl}
    nothing.\category{nom} because know-\category{prf}-\category{pass.3sg}
    if.not true-\category{sg.nom.n} 
    \category{rel}.\category{sg.nom.n} with be-\category{pres.inf} convert-\category{prf}-\category{pass.3sg} \\
    \translate{Besides, a doctrine cannot exist if not about the being, 
    because nothing is known if not true, 
    which (??? TODO) with the being.}
    
    \ex \gll Sed de omnibus entibus 
    tractatur in philosophicis disciplinis, 
    et etiam de Deo, 
    unde quaedam pars philosophiae dicitur theologia, 
    sive scientia divina, 
    ut patet per philosophum in VI Metaphys. \\
    but on all-\category{pl.abl} be-\category{prp}-\category{pl.abl}  
    handle-\category{sbj.prs}-\category{pass.3sg} in philosophy-\category{adj}-\category{pl.abl} discipline-\category{pl.abs} 
    and also on God-\category{sg.abl}
    from.where something-\category{sg.nom.f} part-\category{sg.nom} philosophy-\category{sg.gen} say-\category{ind.prf}-\category{pass.3sg} theology-\category{sg.TODO} 
    or science-\category{sg.nom} divine-\category{sg.nom.f},
    as reveal-\category{act.3sg} through philosopher-\category{sg.acc} in  \\
    \glt \translate{But on all beings, and also on God, 
    it would be handled in the philosophical disciplines, 
    where something that is a part of philosophy is called theology,
    }
\end{exe}

\begin{itemize}
    \item \form{de \dots tractatur in \dots} -- TODO: special verb frame?
    \item Verb frame of \form{dicitur}: \translate{call something X} - what's the case of X?
\end{itemize}

\begin{exe}
    \ex Non fuit igitur \\
    \translate{Therefore, besides philosophical disciplines, 
    it has not been necessary for another doctrine to be possessed.}
\end{exe}

\begin{itemize}
    \item \form{igitur} is a conjunction; it seems \form{non fuit} is fronted as a whole; 
    this is similar to \form{on the other hand, however, \dots}; 
    what is fronted is the core VP 
    with all the arguments moved out, i.e. Dixon's verb phrase.
    \item 
\end{itemize}

\begin{exe}
    \ex Sed contra est quod dicitur II ad Tim. III, omnis Scriputura divinitus utilis est \\
    \translate{But contrarily is what is said }
\end{exe}

\begin{itemize}
    \item Note here \form{omnis} is in nominative case; 
    \form{-is} is not a \category{sg.gen} or \category{pl.abl/dat} ending;
    instead, \form{-s} is a third declension \category{sg.nom} ending. 
    \item \form{autem} shows \form{however}-like syntax, i.e. the postpositive position.
    \item \form{divinitus} is \form{div-in-tus} \translate{god-\category{adj}-\category{adv}}; 
    the citation forms of the half-way derivations are 
    \form{divus}, \form{divinus}, \form{divinitus}.
    This is not an morphological adjective. 
    \item \form{utilis ad \dots}: adjective frame?
    \item \form{quae sunt secundum rationem humanam inventae}: note that the main verb is the last one; 
    it's not a part of the \form{secundum} phrase.
    \item \form{utile} is nominative, not vocative or ablative.
\end{itemize}

\begin{exe}
    \ex Respondeo dicendum quod \dots \\ 
    \translate{I respond to what will be said, that \dots}
\end{exe}

\begin{itemize}
    \item \form{salutem} is singular accusative; the nominative form is \form{salus}, 
    but it's a third declension noun.
    \item \form{praeter philosophicas disciplinas quae ratione humana investigantur} 
        \translate{besides philosophical disciplines, which are investigated by human ration}
\end{itemize}

\begin{exe}
    \ex Primo quidem quia \dots
\end{exe}

This reason clause stands as an independent piece of utterance, 
explaining why \translate{it has been necessary for human salvation for another doctrine besides\dots} 

\begin{exe}
    \ex Finem autem oportet esse praecognitum hominibus, qui suas intentiones et actiones debent ordinare in finem. \\
    \translate{But it is necessary for the end to be foreseen by people \dots}
\end{exe}

\begin{itemize}
    \item Here \form{finem} is actually the accusative subject of the 
        \form{esse praecognitum hominibus} infinitive,
        which serves as the only argument of the impersonal verb \form{oportet}.
    \item The \form{qui suas \dots} clause modifies \form{hominibus} before it;
        in this clause \form{suas intentiones} is the subject of the 
        infinitive \form{ordinare in finem}; 
        the whole \form{qui} clause translates as 
        \translate{who are bound to direct their intentions and actions to that end.}
\end{itemize}

\begin{exe}
    \ex Unde necessarium fuit homini ad salutem, quod ei nota fierent quaedam per revelationem divinam, quae rationem humanam excedunt. 
\end{exe}

\begin{itemize}
    \item \form{necessarium fuit homini ad salutem, quod \dots} 
    \translate{it has been necessarium for the human race regarding salvation, that \dots}
    \item \form{ei nota fierent quaedam} 
    \translate{some things should be made known to humankind} 
    \item Here the singular and plural forms of \form{homo}
    are used alternatively.
\end{itemize}

\begin{exe}
    \ex Ad ea etiam quae de Deo ratione humana investigari possunt, necessarium fuit hominem instrui revelatione divina.
\end{exe}

\begin{itemize}
    \item \emph{etiam}, again, is inserted after the first several words;  
    \item The first part of the sentence is the \form{ad ea \dots} phrase,
    which is modified by the \form{quae} clause following it:
    \translate{even for those that \dots}
    \item The \form{quae} pronoun is \category{3pl.nom}: 
    it agrees with \form{possunt} in the clause: 
    \translate{that are about Gods and can be investigated by human ration}.
    \item In the second part, \form{instrui} is an infinitive: 
    don't forget the verb is in third conjugation.
    \item The second part of the sentence means 
    \translate{it has been necessary for humankind to be instructed by divine revelation.}
\end{itemize}

\begin{exe}
    \ex Quia veritas de Deo, per rationem investigata, a paucis, et per longum tempus, et cum admixtione multorum errorum, homini proveniret, a cuius tamen veritatis cognitione dependet tota hominis salus, quae in Deo est. 
\end{exe}

\begin{itemize}
    \item The core of the sentence is 
    \form{quia veritas de Deo \dots homini proveniret}, 
    \translate{because truth about God would come to humankind \dots}
    \item The prepositional expressions between the subject 
    and the verb are all manner expressions:
    \translate{through ration's investigating, by a few (people), 
    and through long time, and with the admixture of many errors} .
    These expressions are used to emphasize how hard it is 
    for the knowledge of God to come to the humankind.
    \item \form{per rationem investigata}: 
    \form{investigata} modifies the subject, 
    and \form{per rationem} modifies \form{investigata}.
    Note that this is not a gerund clause construction.
    \item The final part of the sentence, 
    \form{a cuius tamen \dots quae in Deo est}, 
    can be puzzling.
    The \form{cuius} clause, which is a fused-head relative clause, 
    has the same reference as that of \form{paucis} in \form{a paucis}: 
    it refers to the bunch of people 
    who indeed know something about God by human ration.
    The clause translates as 
    \translate{whose cognition of the truth, however, is depended upon 
    for humankind's whole salvation, which is in God.}
    Here \form{cuius} is a constituent in the \emph{object}, not the subject; 
    English doesn't have this construction 
    so the passive voice is used in the translation, 
    while the Latin text is in active voice.
    \item The final literal translation of the sentence is 
    \translate{Because truth about God investigated by ration 
    would come to humankind through a long time, 
    and with admixture of multiple errors,
    (and only) by a few (people),
    whose cognition of the truth however is depended on by the whole salvation, 
    which is in God, of humankind.}
    \item TODO: why is \form{proveniret} in \category{past} tense?
\end{itemize}

I let ChatGPT to analyze this sentence; 
it gave me a perfect answer, 
but somehow insisting on treating \form{a cuius \dots}
as a modifier of the subject, not the verb.
This doesn't sound plausible.
The dependency tree it gives is also nonsense
(although local relations like the relation between the preposition and the noun is correct).

\begin{exe}
    \ex Ut igitur salus hominibus et convenientius et certius proveniat, necessarium fuit quod de divinis per divinam revelationem instruantur.
\end{exe}

\begin{itemize}
    \item \form{ut}, followed by a subjunctive clause, means \translate{in order to \dots}
    \item \form{et convenientius et certius} is a manner adverbial phrase;  
    \form{convenientius} is not a noun, but a comparative adverb \translate{more convenient}.
    The overall structure of the adverbial is the \form{et \dots et \dots} construction.
\end{itemize}

\begin{exe}
    \ex Ad primum ergo dicendum quod, licet ea quae sunt altiora hominis cognitione, non sint ab homine per rationem inquirenda, sunt tamen, a Deo revelata, suscipienda per fidem. 
\end{exe}

\begin{itemize}
    \item Here \form{ergo} means \translate{subsequently, also};
    the paragraph can be seen as a continuation of 
    the \form{respondeo dicendum quod \dots} construction 
    appearing at the answer section: 
    \translate{I respond to what will be said that \dots, 
    and subsequently to the first that will be said that \dots}.
    Thus the part following \form{respondeo} in a Question in \form{Summa} 
    can be seen as a giant sentence, 
    with only one verb \form{respondeo}.
    
    This fact actually can be used to ``prove'' that Latin lacks complement clause constructions:
    clearly \form{ad primum ergo dicemdum quod} can be seen as 
    a piece of independent utterance;
    but then the \form{quod} clause after \form{respondeo}
    can be seen as a piece of independent utterance as well.
    Following Everette's mindset, 
    we may claim that \form{quod} is a marker of new information 
    (and of course it is, because a clause is almost always a piece of new information),
    and that Latin lacks complement clause construction.
    \item \form{altiora hominis cognitione} \translate{higher than cognition of humankind}
\end{itemize}

\begin{exe}
    \ex Unde et ibidem subditur, plurima supra sensum hominum ostensa sunt tibi.
\end{exe}

\begin{itemize}
    \item The second part of the sentence is actually a claus.
    The translation is straightforward: 
    \translate{many above human's sense have been revealed to you}.
    \item \form{subditur} is the main verb of the whole sentence:
    \translate{that many above \dots has been placed \dots}.
    \item \form{unde et ibidem} seems to mean \translate{from where, and at exactly this place}; 
    TODO: it modifies \form{fidem} in the last sentence?
\end{itemize}

\begin{exe}
    \ex Et in huiusmodi sacra doctrina consistit.
\end{exe}

\begin{itemize}
    \item TODO: \form{in huiusmodi}???
\end{itemize}

\begin{exe}
    \ex unde nihil prohibet de eisdem rebus, de quibus 
\end{exe}

\begin{itemize}
    \item TODO: \form{prohibeo de \dots} what's this? Why \form{de}?
    \item Tentative reading: \translate{
        Hence (lit. from where) nothing prohibits the same things (see below), 
        regarding which the philosophical disciplines deal with 
        what are learnable by the light of natural ration, 
        and (regarding which) another science deal with what are recognized by the light of divine revelation. 
    } 
    \item The problem is what \form{eisdem rebus} refers to; 
    it refers to the phenomenon that the same conclusion can be shown by two ways, 
    or divine knowledges (and thus \form{eisdem} does not refer to something in the previous context).
    \item \form{aliam scientiam tractare\dots}: why infinitive? TODO
\end{itemize}

\begin{exe}
    \ex Unde theologia quae ad sacram doctrinam pertinet, differt secundum genus ab illa theologia quae pars philosophiae ponitur.
\end{exe}

\begin{itemize}
    \item Tentative translation: \translate{The theology that is related to the sacred doctrine differs 
    from the kind of that theology that is placed as a part of philosophies.}
    \item TODO: verb frame of \form{ponitur}: why \form{pars philosophiae} is nominative?
    \item TODO: valency of \form{genus}: \form{genus ab\dots}?
\end{itemize}

\begin{exe}
    \ex naturalis autem per medium circa materiam consideratum 
\end{exe}

\begin{itemize}
    \item \form{circa materiam consideratum} \translate{about the matter that is being considered}
\end{itemize}

\section{Question 1.1, article 2}

\begin{exe}
    \ex huic scientiae attribuitur illud tantummodo quo fides saluberrima gignitur, nutritur, defenditur, roboratur. 
\end{exe}

\begin{itemize}
    \item The overall translation is 
    \translate{To this science is attributed that (thing) by which the most beneficial faith is created, nourished, defended, and protected alone.}
    \item \form{illud} seems to be the subject, modified by the \form{quo} clause \translate{by which the faith is created \dots}
    \item 
\end{itemize}

\begin{itemize}
    \item \form{ex principiis notis lumine naturali intellectus} \translate{from principles known by the nature light of intelligence}.
\end{itemize}

\begin{exe}
    \ex \form{diversa ratio cognoscibilis diversitatem scientiarum inducit}: 
    \translate{diverse cognitive method leads to diversity of sciences.}
    The clause is actually quite straightforward.
\end{exe}

\begin{exe}
    \ex Eandem enim conclusionem\dots
\end{exe}

\begin{itemize}
    \item \form{eandem enim conclusionem}: 
    note that here \form{enim} breaks a noun phrase.
    \item \form{naturalis}: note that \form{-lis} is not a \category{sg.gen} or \category{pl.abl/dat} ending.
    \item \translate{Because the astronomer and the physicist demonstrate the same conclusion}
\end{itemize}

\begin{exe}
    \ex puta quod terra est rotunda
\end{exe}

\begin{itemize}
    \item \form{puta} modifies the following complement clause, 
    which is a appositive of \form{eandem conclusionem}.
    \item \form{rotundus} is not a future passive participle.
\end{itemize}

\begin{exe}
    \ex sed astrologus per medium mathematicum, idest a materia abstractum 
\end{exe}

\begin{itemize}
    \item \form{a materia abstractum}: 
    here \form{materia} is singular ablative; 
    the meaning is \translate{towards abstracting from matter}; 
    TODO: the role of \form{a}.
    \item The whole sentence omits the verb.
\end{itemize}


\section{Question 1.1, article 3}

\begin{exe}
    \ex creator autem et creatura, de quibus in sacra doctrina tractatur, 
\end{exe}

\begin{itemize}
    \item \form{de quibus in sacra doctrina tractatur}: TODO: special verb frame? 
    \item \translate{However, the creator and the creature, which are treated in the sacred doctrine, 
    are not placed under one kind of subjects.}
    \item \form{subiecti}: why it's singular?
\end{itemize}

\begin{exe}
    \ex Praeterea, in sacra doctrina tractatur de Angelis, de creaturis corporalibus
\end{exe}

\begin{itemize}
    \item \translate{in the sacred doctrin, angels, corporeal creatures, and moralities of humans}
    \item The weird usage of \form{tractare} appears again.
\end{itemize}

\begin{exe}
    \ex huiusmodi autem ad diversas 
\end{exe}

\begin{itemize}
    \item \translate{As such, however, they belong to diverse philosophical sciences.}
\end{itemize}

\begin{exe}
    \ex sacra Scriputura de ea loquitur sicut de una scientia
\end{exe}

\begin{itemize}
    \item \translate{talk about it as if (it's talking about) one science}
    \item The construction of \form{sicut} is interesting, it's so similar to English\dots
    \item \form{dedit illi scientiam sanctorum}: \translate{(somebody) gave that person the science of the holy (things)}.
\end{itemize}

\begin{exe}
    \ex est enim unitas potentiae et habitus consideranda 
\end{exe}

\begin{itemize}
    \item \form{consideranda} is a future passive participle, 
    which, I guess, is used here; the copula carrying TAM information 
    is for some reason fronted. 
    \item \form{secundum obiectum, non quidem materialiter}:
    \translate{accourding to the object, not indeed materially.} 
    The structure is weird: what does \translate{not indeed} mean?
    \item \form{sed secundum rationem formalem obiecti}
    \translate{but according to the formal ration of the object}
\end{itemize}

\begin{exe}
    \ex puta homo, asinus et lapis conveniunt in una formali ratione colorati
\end{exe}

\begin{itemize}
    \item \translate{For example, a human, a donkey, and a stone meet in one formal ration of being colored.}
    \item \form{colorati}: genitive case of \form{coloratus} in singular; 
    the literal translation would be \translate{of what are being colored}?
    \item \form{visus} is not a participle, but a fourth declension noun: \translate{of vision}.
\end{itemize}

\begin{exe}
    \ex quia igitur sacra Scriptura\dots
\end{exe}

\begin{itemize}
    \item \form{quia igitur} actually is \translate{therefore, because}:
    \form{igitur} is \translate{because} and is \emph{postpositive}.
    \item \form{considerat aliqua secundum\dots}: here \form{aliqua} is not an adverb.
    \item \translate{Therefore, because the holy scripture considers things about what are divinely revealed, 
    about what is said, 
    whatever are divinely revealable communicate in one formal ration 
    of this science.
    }
    \item Here \form{communicant} simply means ``connect'', 
    and therefore ``sharing the same property'' here.
\end{itemize}

\begin{exe}
    \ex et ideo comprehenduntur sub sacra doctrina sicut sub scientia una
    \translate{And therefore, (these things) are placed under sacred doctrine as under one science.}
\end{exe}

\begin{itemize}
    \item Here \form{sicut} \translate{as} connecting two prepositional phrases 
    which are oblique arguments of one verb appears again.
\end{itemize}

\begin{exe}
    \ex sacra doctrina non determinat de Deo et de creaturis ex aequo
\end{exe}

\begin{exe}
    \ex Here \form{ad Deum, ut ad principium vel finem} 
    \translate{to God, as to the beginning or the end}: 
    here \form{principium} and \form{finem} are appositives of \form{Deum}.
\end{exe}

\section{Question 1, article 5}

\begin{exe}
    \ex Certitudo enim pertinet ad dignitatem scientiae \\
    \translate{Because certainty is related to the dignity of a science.}
\end{exe}

\begin{exe}
    \ex de quarum principiis dubitari non potest \\ 
    \translate{whose principles it is not possible to be doubted on}
\end{exe}

The fact that \form{de} and \form{dubitari} appear together is strange:
if the structure is \translate{it's not possible to doubt on whose principles}, 
then the active \form{dubitare} should be used instead.
If the structure is \translate{whose principles can't be doubted}, 
then \form{de} is mysterious.
This, together with the mysterious \form{ad }

\section{Question 1, article 9}



\end{document}