\documentclass[a4paper, 12pt]{article}

\usepackage{libertinus}
\usepackage{geometry}
\usepackage{caption}
\usepackage{subcaption}
\usepackage{abstract}
\usepackage{amsmath, amssymb}
\usepackage{qtree}
\usepackage{gb4e}
\usepackage[colorlinks, urlcolor=cyan]{hyperref}
\usepackage{prettyref}

\geometry{left=3.18cm,right=3.18cm,top=2.54cm,bottom=2.54cm}

\newcommand*{\concept}[1]{\textbf{#1}}
\newcommand*{\term}[1]{\emph{#1}}
\newcommand{\form}[1]{\emph{#1}}
\newcommand*{\category}[1]{\textsc{#1}}
\newcommand{\translate}[1]{`#1'}
\newcommand*{\changeto}{$>$}
\newcommand*{\changefrom}{$<$}

\title{Psalm 136: translation and analysis}
\author{Jinyuan Wu}

\begin{document}

\maketitle

\section{Overall structure}

\subsection{Verses and sentences}

The whole chapter can be divided into four sentences:
verse 1, verse 2, verses 3-25, and verse 26.
Verses 4-22 are all \form{qui} relative clauses modifying 
the object \form{Domino} in the matrix clause in verse 3:
in all of these relative clauses, 
the relative pronoun is the subject, and hence the form \form{qui}.
Verses 23-24 are clauses of cause explaining 
God's intention in all these events.

\subsection{The initial verb \form{confitemini}}

All the four sentences start with \form{confitemini} 
\translate{you all please praise};
the meaning of \translate{praise} is uniquely found in Ecclesiastical Latin;
the Classical meaning is just \translate{confess} or \translate{admit}.

The ending \form{-mini} is used because the verb is deponent
and the passive form has to be used.
The fact that \form{confiteor} is deponent might be 
historically related to it being unaccusative (which agrees with the ``admitting'' meaning);
in Ecclesiastical usage however this no longer is the case, 
but the passive feature seems to have already merged into 
the realizational morphological profile of the verb. 

The complements of the verb \form{confitemini} in all the four sentences 
are all dative; because the psalm asks readers to 
\translate{praise to God (\changefrom confess \emph{to} God)}.
In Nicene Creed however we have direct objects selected by \form{confiteor}
(\ref{ex:nc-ex-1}),
where the reader confesses the baptism (to others).

\begin{exe}
    \ex\label{ex:nc-ex-1} Confíteor unum baptísma in remissiónem peccatórum.
\end{exe}

\subsection{The ``response'' part of each verse}

After each verse there is a reason clause \form{quoniam in aetenum \dots};
the copula is emitted in the clause.
The conjunction \form{quoniam}, 
which is the univerbation of 
the conjunction \form{quom} (Old Latin; \changeto Classical Latin conj. \form{cum})
and the adverb \form{iam} \translate{already, now}.

\begin{exe}
    \ex\label{ex:verse-2-ex-1} confitemini Deo deorum [quoniam in aetenum misericordia eius]  \\ 
    \translate{lit. (You all please) praise God in gods, since his mercy is into the eternal.}
\end{exe}

\section{Verse 1}

\section{Verses 7-9: the creation of light}

Verses 7-9 constitute one relative clause modifying \form{Domino} in verse 3.
An interesting way is used to express 
that the purpose of the sun and the moon and stars 
is to govern the day and the night: 
\emph{to govern the day and the night}:
the prepositional phrase \form{in potestatem diei/noctis}
(\ref{ex:verse-7-9-ex-1}; note that \form{potestatem} is in the accusative case).


\begin{exe}
    \ex\label{ex:verse-7-9-ex-1} qui fecit \dots solem in potestatem diei \\ 
    \translate{lit. who has made the sun into the power of the day}
\end{exe}

\section{Verses 10-13: summary of the Exodus from Egypt}

Verses 10-13 constitute one relative clause modifying \form{Domino} in verse 3.


\begin{exe}
    \ex qui percussit Aegyptum cum primitivis suis \\
    \translate{who has struck Egypt with their primitives}

    \ex et eduxit Israhel de medio eorum \\
    \translate{and has led Israel from the middle of them} \\

\end{exe}

\section{Verses 13-15: details of Exodus}


\section{Verses 17-20: defeat of Sihon and Og}

\section{Verses 21-22: the promised land}

The valency of the verb \form{dedit} deviates from the standard dative construction:
the dative target argument is replaced by 
the prepositional phrase \form{in hereditatem};
the dative target noun phrase however is still present 
(\form{servo tuo}).
TODO: I don't know what licenses the dative NP.

The literal meaning is 
\translate{who has given the land of them into inheritance, 
inheritance of Israel, }

\section{Verses 23-24}

Verses 23-24 are two \form{quia} clauses; 
they describe the reason for the aforementioned stories.


\translate{because he has been mindful of us in our humuliation}

Note that \form{nostra} is a determiner 
with adjective morphological behaviors,
and thus it agrees with the \category{sg.abl.f} categories of \form{humilitate}
and takes the \form{-a} ending.

\end{document}