\documentclass[a4paper, 12pt]{article}

\usepackage{libertinus}
\usepackage{geometry}
\usepackage{caption}
\usepackage{subcaption}
\usepackage{abstract}
\usepackage{amsmath, amssymb}
\usepackage{qtree}
\usepackage{gb4e}
\usepackage[colorlinks, urlcolor=cyan]{hyperref}
\usepackage{prettyref}

\geometry{left=3.18cm,right=3.18cm,top=2.54cm,bottom=2.54cm}

\newcommand*{\concept}[1]{\textbf{#1}}
\newcommand*{\term}[1]{\emph{#1}}
\newcommand{\form}[1]{\emph{#1}}
\newcommand*{\category}[1]{\textsc{#1}}
\newcommand{\translate}[1]{`#1'}

\title{Psalm 136: translation and analysis}
\author{Jinyuan Wu}

\begin{document}

\maketitle

\section{Verses 7-9}

Verses 7-9 constitute one relative clause modifying \form{Domino} in verse 3.

\begin{exe}
    \ex solem in potestatem diei 
\end{exe}

The meaning of \form{in potestatem diei} is interesting \dots
\translate{Sun in the power of days}

\end{document}