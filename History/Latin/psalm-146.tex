\documentclass[a4paper, 12pt]{article}

\usepackage{libertinus}
\usepackage{geometry}
\usepackage{caption}
\usepackage{subcaption}
\usepackage{abstract}
\usepackage{amsmath, amssymb}
\usepackage{qtree}
\usepackage{gb4e}
\usepackage[colorlinks, urlcolor=cyan]{hyperref}
\usepackage{prettyref}

\geometry{left=3.18cm,right=3.18cm,top=2.54cm,bottom=2.54cm}

\newcommand*{\concept}[1]{\textbf{#1}}
\newcommand*{\term}[1]{\emph{#1}}
\newcommand{\form}[1]{\emph{#1}}
\newcommand*{\category}[1]{\textsc{#1}}
\newcommand{\translate}[1]{`#1'}
\newcommand*{\changeto}{$>$}
\newcommand*{\changefrom}{$<$}

\title{Psalm 146}
\author{Jinyuan Wu}

\begin{document}

\maketitle

\begin{exe}
    \ex laudate Dominum [quoniam bonum est caticum Dei nostri] [quoniam decorum est pulchra laudatio] 
\end{exe}

\begin{exe}
    \ex qui sanat contritos corde
\end{exe}

The literal translation of this sentence seems to be \translate{who heals the contrite by the hearts}:
yet another example of external possession
(c.f. \form{qui percussit Aegyptum cum primogenitis eorum}).

Also, it seems a V2 constituent order can be observed in this psalm.

\begin{exe}
    \ex omnes nomine suo vocat \\
    \translate{calls all things with (instr.) your name}
\end{exe}

\begin{exe}
    \ex multus fortitudine prudentiae eius non est numerus
\end{exe}
Why \form{fortitudine} is ablative?

\end{document}