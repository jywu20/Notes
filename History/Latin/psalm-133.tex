\documentclass[a4paper, 12pt]{article}

\usepackage{libertinus}
\usepackage{geometry}
\usepackage{caption}
\usepackage{subcaption}
\usepackage{abstract}
\usepackage{amsmath, amssymb}
\usepackage{qtree}
\usepackage{gb4e}
\usepackage[colorlinks, urlcolor=cyan]{hyperref}
\usepackage{prettyref}

\geometry{left=3.18cm,right=3.18cm,top=2.54cm,bottom=2.54cm}

\newcommand*{\concept}[1]{\textbf{#1}}
\newcommand*{\term}[1]{\emph{#1}}
\newcommand{\form}[1]{\emph{#1}}
\newcommand*{\category}[1]{\textsc{#1}}
\newcommand{\translate}[1]{`#1'}
\newcommand*{\changeto}{$>$}
\newcommand*{\changefrom}{$<$}

\title{The short ``Songs of the Ascents''}
\author{Jinyuan Wu}

\begin{document}

\maketitle

\section{The phrase \form{canticum graduum}}

The structure of the noun \form{canticum} is 
\form{cant-ic-um}.
\begin{enumerate}
    \item The root \form{cant-} is the \category{supine} stem, 
    which is given by the fourth principal part \form{cantus} 
    of the verb \form{cant\={o}}.
    \item The morpheme \form{-ic-} is usually refer to 
    as the \form{-icus} suffix (singular masculine nominative);
    the suffix is an adjectivizer.
    \item The ending \form{-um} is the singular neutral nominative ending. 
\end{enumerate}
The whole word \form{canticum} was recognized at a certain stage 
as a noun, and its gender was fixed to neutral.

The noun \form{graduum} has ending \form{-uum},
the fourth declension plural genitive.
The root \form{gradu-} further forms the verb 
\form{gradu\={o}} \translate{(Medieval Latin) graduate} 
by the verbalizer \form{-\={o}}.
The suffix \form{-\={o}} is in the indicative present first singular positive form; 
it creates first conjugation verbs, 
and the corresponding passive perfect participle suffix is \form{-ātus}.
Thus we have \form{graduātus},
which eventually was borrowed into English as \form{graduate}.

The phrase \form{canticum graduum} means \translate{song of the graduations/ascents}, 
i.e. songs sung when ascending the road to Jerusalem.

\section{Psalm 133} 

The second verse is kind of unusual: 
\form{ecce} is followed by an imperative clause:
\translate{all servants of Lord, who stands in Lord's home, 
please praise Lord}, 
instead of a noun phrase.
Also note that the division of the verses seems in contradictory with 
English translations: 
is it \translate{all servants \dots who stand in Lord's home; 
in the nights, raise your hands to the sanctuary and praise Lord}, 
or \translate{all servants \dots who stand in Lord's home in the nights; 
raise your hands to \dots}?

The verb \form{benedīcō} takes a dative complement;
this again is because of historical reasons:
\form{benedīcō} is \form{bene-dic-\={o}} \translate{well-speak},
and the direct object of \form{dic\={o}} \translate{say}
is the utterance being said, 
while the indirect object is the person addressed to; 
after the direct object is suppressed 
the indirect object remains in the dative case, 
and hence the verb frame of \form{benedīcō}
which has since gained the specific meaning of \translate{(religious) praise}.

\section{Psalm 132}

In the first verse, \form{bonum} and \form{decorum} modifies 
the infinitive \form{habitare fratres in uno}; 
note that since this is an infinitive, 
\form{fratres} is in accusative, not nominative.



\end{document}