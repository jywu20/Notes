\documentclass[a4paper]{article}

\usepackage{geometry}
\usepackage{caption}
\usepackage{subcaption}
\usepackage{abstract}
% \usepackage{paralist}
\usepackage{amsmath, amssymb}
\usepackage{qtree}
\usepackage{gb4e}
\usepackage[colorlinks, urlcolor=cyan]{hyperref}
\usepackage{prettyref}

\geometry{left=3.18cm,right=3.18cm,top=2.54cm,bottom=2.54cm}

% \renewenvironment{itemize}{\begin{compactitem}}{\end{compactitem}}
% \renewenvironment{enumerate}{\begin{compactenum}}{\end{compactenum}}


\title{Translation of Anima Christi}
\author{Jinyuan Wu}

\begin{document}

\maketitle

\section{The Text}

\emph{Anima Christi} is a medieval hymn of unknown origin \cite{wiki}. It was once believed to be written by Saint Ignatius of Loyola, but it can actually be dated to earlier periods.

Several versions of the hymn, in different melodies and different languages, can be found online \cite{youtube2,youtube3,youtube1}.

The Latin lyrics is shown below:
\begin{quotation}
    Anima Christi, sanctifica me. 

    Corpus Christi, salva me.
    
    Sanguis Christi, inebria me.
    
    Aqua lateris Christi, lava me.
    
    Passio Christi, conforta me.
    
    O bone Jesu, exaudi me.
    
    Intra tua vulnera absconde me.
    
    Ne permittas me separari a te.
    
    Ab hoste maligno defende me.
    
    In hora mortis meae voca me.
    
    Et jube me venire ad te,
    
    Ut cum Sanctis tuis laudem te,
    
    In saecula saeculorum.
    
    Amen.
\end{quotation}

\section{Translation}

The Latin lyrics can be glossed as below. Etymologies mentioned here come from Wiktionary by default.

\begin{exe}

\sn
\gll Anima        Christi,       sanctifica                me.  \\
     soul-SG.VOC  Christ-SG.GEN  sanctify-IMP.PRS.ACT.2SG  1SG.ACC \\
\glt  ``O soul of Christ, sanctify me.''

\sn
\gll Corpus      Christi,       salva                  me. \\
     body-SG.VOC Christ-SG.GEN  save-IMP.PRS.ACT.2SG   1SG.ACC \\
\glt ``O the body of Christ, save me.''
\end{exe}

\paragraph{Note} \begin{itemize}
    \item \emph{Sanctifica} is the second person singular active present imperative of \emph{sānctificō}, which is from \emph{sanctus} (``holy'') + \emph{-fico} (``make'').
    \item \emph{Corpus} is a third declension noun.
\end{itemize}

\begin{exe}

\sn
\gll Sanguis       Christi,       inebria                   me. \\
     blood-SG.VOC  Christ-SG.GEN  inebriate-IMP.PRS.ACT.2SG 1SG.ACC \\
\glt ``O blood of Christ, inebriate me.''

\gll Aqua           lateris      Christi,        lava                   me. \\
     water-SG.VOC   side-SG.GEN  Christ-SG.GEN   wash-IMP.PRS.ACT.2SG   1SG.GEN \\
\glt ``O water of Christ's side, wash me.''
\end{exe}

\paragraph{Note} \begin{itemize}
    \item \emph{Sanguis} is a third declension noun.
\end{itemize}

\begin{exe}

\sn
\gll Passio          Christi,        conforta                       me. \\
     passion-SG.VOC  Christ-SG.GEN   strengthen-IMP.PRS.ACT.2SG     1SG.GEN \\
\glt ``Passion of Christ, strengthen me.''

\sn    
\gll O  bone            Jesu,           exaudi           me.  \\
     O  good-2SG.VOC.M  Jesus.SG.VOC    hear-IMP.PRS.ACT.2SG 1SG.GEN \\
\glt ``O good Jesus, hear me.''
\end{exe}

\paragraph{Note} \begin{itemize}
    \item \emph{Cōnfortā} comes from \emph{con-} ( \textless \emph{cum}, ``with'') + \emph{fortis} (``power'') + \emph{-ō} (verbalize suffix, first conjugation). The word is also the origin of \emph{comfort} in English.
    \item \emph{Jesus} is a highly irregular word in Latin. Its nominative and accusative forms are \emph{Jesus} and \emph{Jesum}, respectively, and all other cases are \emph{Jesu}.
    \item \emph{Exaudī} is the second person singular active present imperative of \emph{exaudiō}, which is from \emph{ex-} + \emph{audiō }.
\end{itemize}

\begin{exe}
    
\sn
\gll Intra tua            vulnera        absconde             me. \\
     in    your-PL.ACC.N  wound-PL.ACC   hide-IMP.PRS.ACT.2SG 1SG.ACC  . \\
\glt ``Hide me in your wounds.''

\sn
\gll Ne   permittas               me       separari                a     te. \\
     not  permit-SBJ.PRS.ACT.2SG  1SG.ACC  separate-PST.PASS.INF   from  2SG.ABL \\
\glt ``You should not permit me to be separated from you.''

\sn
\gll Ab    hoste          maligno        defende                 me. \\
     from  enemy-SG.ABL   bad-SG.ABL.M   defend-IMP.PRS.ACT.2SG  1SG.ACC \\
\glt ``Defend me from the malign enemy.''

\end{exe}

\paragraph{Note} \begin{itemize}
    \item \emph{Vulnera} is a third declension noun, and here \emph{tua} (adjective) is not SG.NOM.F. 
    \item \emph{Abscondo} is from \emph{abs-} (``away from'', alternative forms \emph{a-} and \emph{ab-}) + \emph{condo} (``hide'').
    \item \emph{Hostis} - the nominative form of \emph{hoste} - can be both masculine and feminine.
\end{itemize}

\begin{exe}
\sn 
\gll  In hora        mortis        meae        voca                  me. \\
      in hour-SG.ABL death-SG.GEN  my-SG.GEN.F call-IMP.PRS.ACT.2SG  1SG.ACC \\
\glt ``Call me in the hour of my death.''

\sn
\gll Et   jube                       me        venire             ad   te, \\
     and  authorize-IMP.PRS.ACT.2SG  1SG.ACC   come-PST.ACT.INF   to   2SG.ACC \\
\glt ``And let me come to you,''

\sn
\gll Ut        cum   Sanctis       tuis          laudem                  te, \\
     so.that   with  saint-PL.ABL  your-PL.ABL   praise-SBJ.PRS.ACT.1SG  2SG.ACC \\
\glt ``so that with your saints I can praise you,''

\paragraph{Note} \begin{itemize}
    \item Recall that \emph{ut}'s meaning differs according to the mood of the clause following it. By the context we can infer that in the above sentence \emph{ut} is likely to mean ``so that'', so \emph{laudem} is likely to be subjunctive, and then we will find everything fits well.
\end{itemize}

\sn
\gll In   saecula      saeculorum. \\
     in   time-PL.ACC  time-PL.GEN \\
\glt ``into times of times (= for ever and ever).''
\end{exe}

\bibliographystyle{plain}
\bibliography{anima-christi} 

\end{document}